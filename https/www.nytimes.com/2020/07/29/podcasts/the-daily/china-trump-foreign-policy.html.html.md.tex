Sections

SEARCH

\protect\hyperlink{site-content}{Skip to
content}\protect\hyperlink{site-index}{Skip to site index}

\href{https://www.nytimes.com/podcasts/the-daily}{The Daily}

\href{https://myaccount.nytimes.com/auth/login?response_type=cookie\&client_id=vi}{}

\href{https://www.nytimes.com/section/todayspaper}{Today's Paper}

\href{/podcasts/the-daily}{The Daily}\textbar{}Confronting China

\url{https://nyti.ms/313FOmC}

\begin{itemize}
\item
\item
\item
\item
\item
\item
\end{itemize}

Advertisement

\protect\hyperlink{after-top}{Continue reading the main story}

transcript

Back to The Daily

bars

0:00/28:40

-28:40

transcript

\hypertarget{confronting-china}{%
\subsection{Confronting China}\label{confronting-china}}

\hypertarget{hosted-by-michael-barbaro-produced-by-asthaa-chaturvedi-and-austin-mitchell-with-help-from-neena-pathak-and-sydney-harper-and-edited-by-lisa-chow-and-lisa-tobin}{%
\subsubsection{Hosted by Michael Barbaro; produced by Asthaa Chaturvedi
and Austin Mitchell; with help from Neena Pathak and Sydney Harper; and
edited by Lisa Chow and Lisa
Tobin}\label{hosted-by-michael-barbaro-produced-by-asthaa-chaturvedi-and-austin-mitchell-with-help-from-neena-pathak-and-sydney-harper-and-edited-by-lisa-chow-and-lisa-tobin}}

\hypertarget{some-members-of-the-trump-administration-believe-the-superpower-country-poses-an-existential-threat-to-the-us--one-they-are-working-to-address-now}{%
\paragraph{Some members of the Trump administration believe the
superpower country poses an existential threat to the U.S. --- one they
are working to address
now.}\label{some-members-of-the-trump-administration-believe-the-superpower-country-poses-an-existential-threat-to-the-us--one-they-are-working-to-address-now}}

Wednesday, July 29th, 2020

\begin{itemize}
\item
  michael barbaro\\
  From The New York Times, I'm Michael Barbaro. This is ``The Daily.''
\item
  {[}music{]}\\
  Today: A cooperative relationship with China has been a pillar of the
  United States' foreign policy for more than half a century. Edward
  Wong on why the Trump administration believes it's time for a change.

  It's Wednesday, July 29.

  Edward, can you tell me what happened in Houston last week?
\item
  edward wong\\
  Sure. We first got a tip that something was up with the Chinese
  consulate in Houston around Tuesday afternoon or so --- that the
  Chinese ambassador to the U.S. had been told by American officials
  that he had three days to shut down the consulate, and that the
  employees here had 30 days to then leave the country. And a colleague
  and I started chasing this tip, but we couldn't quite nail it down to
  publish a story.
\item
  archived recording\\
  Right now at 10, breaking news ---
\end{itemize}

edward wong

And then ---

\begin{itemize}
\item
  archived recording 1\\
  Houston firefighters and police responding to the Chinese consulate in
  Montrose after reports of a fire.
\item
  archived recording 2\\
  Crews were called to the building off Montrose and Herald about 8:20
  tonight.
\end{itemize}

edward wong

In the evening, I started seeing these videos of people burning things
in metal barrels, in open metal barrels. And there was video of fire
trucks and police cars surrounding the consulate with their lights on,
so it's quite a dramatic scene.

\begin{itemize}
\tightlist
\item
  archived recording\\
  And local media were reporting that documents appeared to be being
  burned in the courtyard of that building.
\end{itemize}

edward wong

You know, for people in the national security world and the foreign
policy world, when you see people burning lots of documents or papers at
a diplomatic mission, the assumption is that they're about to clear out
of the mission. So when I saw those videos, I realized that the tip we
had gotten that the Chinese ambassador had been told to shut down the
Houston consulate within three days was indeed true.

\begin{itemize}
\tightlist
\item
  archived recording (wang wenbin)\\
  On July the 21st, the U.S. abruptly asked China to close its consulate
  in Houston.
\end{itemize}

edward wong

And within hours, the Chinese foreign ministry confirmed that in
Beijing.

\begin{itemize}
\tightlist
\item
  archived recording (wang wenbin)\\
  We urge the U.S. to reverse this incorrect decision immediately.
  Otherwise, China will definitely take necessary legitimate actions.
\end{itemize}

michael barbaro

And why would the U.S. take this pretty significant move of kicking
these Chinese diplomats out of this consulate in Texas?

edward wong

Officially, people in American government told us that they targeted the
Houston consulate because it was a hub of economic espionage and trade
secrets espionage in the U.S. But American officials haven't given us
detailed evidence on the activities undertaken by the Chinese diplomats.
And it's not clear to us how much farther these activities go beyond the
types of covert or espionage activities that take place at missions
around the world, including ones run by Americans. But in the bigger
picture, a main goal of some American officials in the Trump
administration is to unwind a range of diplomatic and economic ties that
have built up between the U.S. and China over the decades ever since
President Nixon started the opening of China back in 1971.

michael barbaro

So there's a version of this where the U.S. was looking for a reason to
unwind this relationship, and espionage --- real or not --- was that
reason.

edward wong

Right.

michael barbaro

And why would the Trump administration want to unwind its relationship
with China? I mean, it's our single biggest trading partner. It's a
global superpower. It's a nuclear power, so that's a pretty significant
decision.

edward wong

It is significant, and there are some senior officials in the
administration who are against this. Throughout the last three and a
half years, we've seen, broadly speaking, two factions of advisers on
China competing against each other for Trump's ear. And that helps
explain some of the contradictory impulses and policies that we've seen
coming out of the administration on China during this period.

michael barbaro

What do you mean?

edward wong

On one side, you had the ones wanting to confront China, in part over
trade, and also in part over national security matters. You had Peter
Navarro, who's a White House trade adviser.

\begin{itemize}
\tightlist
\item
  archived recording (peter navarro)\\
  How do you work with a country that lies through its teeth?
\end{itemize}

edward wong

Who wrote a book called ``Death by China,'' and then you also had
Secretary of State Mike Pompeo.

\begin{itemize}
\tightlist
\item
  archived recording (mike pompeo)\\
  They very much want to undermine our Western values, all the things
  that we hold most dear.
\end{itemize}

edward wong

And those people saw China as a threat to America. And then on the more
cooperative side, you have, for example, Treasury Secretary Steve
Mnuchin.

\begin{itemize}
\tightlist
\item
  archived recording (steve mnuchin)\\
  We need to work together to maximize the benefit for both sides.
\end{itemize}

edward wong

People who still clung to the classic notions of free trade and thought
that the traditional relationship with China was a stabilizing force in
the world. And that this had helped American companies get wealthy over
the years, as well as had benefited American consumers.

michael barbaro

And Edward, when comes to those who want to confront China, when it
comes to the Peter Navarros and the Mike Pompeos, what is their case for
why China is such a threat to the U.S. and should be reined in?

edward wong

Well, they argue that China presents a range of strategic threats to the
U.S. For example, they say that China's attempts to export its 5G
technology, its next generation communications technology, around the
world presents a security threat. They say that China's recent military
expansionism in the South China Sea, and its vast maritime claims in
that sea, are also a security threat, and they would impede American
military dominance in the Asia-Pacific. They point to attempts at
economic espionage by China and a vast range of cyber attacks that have
targeted the American government and other important institutions around
the world.

michael barbaro

Am I right to think that, from the start of his presidency, the
confrontation camp more or less prevailed?

edward wong

Well, it's complicated. The first big blow to the U.S.-China
relationship under the Trump administration was in mid-2018.

\begin{itemize}
\tightlist
\item
  archived recording (donald trump)\\
  We're going to have some incredible things. We're just announcing very
  big tariffs today on China, because China has been ---
\end{itemize}

edward wong

When President Trump started putting tariffs on billions of dollars of
goods made in China, China retaliated by doing the same on American
goods.

\begin{itemize}
\tightlist
\item
  archived recording\\
  So here's what they would do. They target farm products such as
  soybean cars, seafood ---
\end{itemize}

edward wong

And then just spiral downward from there.

\begin{itemize}
\item
  archived recording 1\\
  Medical equipment, energy products, that would start a little bit ---
\item
  archived recording 2\\
  As the U.S.-China trade war escalates, business leaders have been
  speaking up. FedEx C.E.O. ---
\end{itemize}

edward wong

So the trade war had this huge impact on companies, both in the U.S. and
outside of the U.S. And it created a lot of instability in their
thinking about how to do business.

\begin{itemize}
\tightlist
\item
  archived recording\\
  The escalating trade battle between the U.S. and China is rocking
  investors around the world.
\end{itemize}

edward wong

It created a lot of instability in the stock markets, which Trump
watches closely. And ---

\begin{itemize}
\tightlist
\item
  archived recording\\
  Some farmers in the U.S.A., the disruption of normal trade with China
  has forced many of them to go bankrupt.
\end{itemize}

edward wong

Important groups of voters who had supported Trump, for example, farmers
in the Midwest, were starting to suffer.

\begin{itemize}
\tightlist
\item
  archived recording\\
  I was a Trump voter. I voted for the president, certainly. But he
  certainly hasn't come through. He's lost on trade. He's lost on trade,
  and certainly ---
\end{itemize}

edward wong

They saw agricultural products like corn and soybeans piling up in the
Midwest, because China had imposed tariffs on their end to strike back
at Trump.

\begin{itemize}
\tightlist
\item
  archived recording\\
  So I won't be voting for the president again.
\end{itemize}

edward wong

So Trump and some of his economic advisers, especially those who were
preaching more cooperation with China, started to get nervous about
these economic signs that they were seeing, as well as about the
anxieties of these midwestern farmers and potential Trump voters there.

michael barbaro

And so what do they do, these cooperation camp folks who are not happy
with this trade war?

edward wong

Well, as they go through negotiations for a potential truce to the trade
war, President Trump talks with President Xi of China several times. And
they have these, like, sort of one-on-one conversations that Trump likes
to do with leaders. And in each of these, Trump sort of cozies up to Xi,
and it's clear he's willing to sort of brush aside a lot of sort of the
most egregious behaviors of China in the pursuit of this trade deal.

michael barbaro

Like what?

edward wong

In one conversation, according to John Bolton --- the former national
security adviser --- Trump encouraged Xi to actually continue building
internment camps for Muslims in the northwest of China and sort of
signaled that this wasn't a big issue for him.

michael barbaro

These are the Uighurs?

edward wong

Right, these are the Uighurs, exactly. The ones a million or more held
over recent years in internment camps. And, for example, we've seen
these, during this period, these pro-democracy protests arise in Hong
Kong. And while Trump's national security aides are supportive of them,
Trump himself tells Xi privately in a phone call that Xi should just
handle those in whatever way he wants to deal with those, and that Trump
himself will not say anything about those, and he'll tell his aides not
to say anything vocally among those protests either.

michael barbaro

So in this trade war that's supposed to represent confrontation with
China, there's actually a fair amount of cooperation going on, most of
it behind the scenes.

edward wong

Right, exactly. And ultimately, in December, they reach a tentative
agreement, and then they signed that in January of this year. And I
think that brought a big sense of relief to the people in the
cooperation camp. I think they were relieved to see a sense of stability
returned to this key economic relationship. Now, the confrontation
people in the White House and in other agencies were generally
disappointed, I think, by the outcome of the deal. They felt that Trump
had sidelined a lot of the hardline policies they had pushed for in the
first half of the administration for the sake of just trying to get a
marginal increase in agricultural purchases. And also, there is a sense
of outrage among some of them.

And this was in John Bolton's recent book, that Trump was also aiming
for this negotiated truce purely for re-election purposes. That he
pleaded with Xi in a conversation that Xi should get help him get
re-elected, should help him win, and that the best way to do this was to
reach some sort of truce or deal in the trade war that he could then
bring back to his constituents. And so certain national security people
were outraged by this, saying that Trump was focused purely on personal
politics and was not looking after the national security interests of
the United States.

michael barbaro

Edward, what you're describing so far, especially this trade deal, does
not seem like a relationship that is about to be fundamentally unwound
and blown up. So what happens to get us from that truce to now, into the
shutdown of this consulate in Houston?

edward wong

Well, what changes things is this pandemic that starts in central China
and spreads across the globe.

That sets the two powers on a much more confrontational course.

{[}music{]}

michael barbaro

We'll be right back.

\begin{itemize}
\tightlist
\item
  archived recording (donald trump)\\
  I spoke with President Xi. We had a great talk. He's working very
  hard. I have to say he's working very, very hard.
\end{itemize}

edward wong

So in the first weeks after the virus started spreading around the
globe, Trump was still praising Xi publicly.

\begin{itemize}
\tightlist
\item
  archived recording (donald trump)\\
  If you know anything about him, I think he'll be in pretty good shape.
  They've had a rough patch, and I think right now, they have it --- it
  looks like they're getting it under control more and more.
\end{itemize}

edward wong

This was in January and February right after they had signed the trade
agreement, so Trump was still in this mode where he wanted intensely to
preserve that negotiated truce. But by the spring ---

\begin{itemize}
\tightlist
\item
  archived recording (donald trump)\\
  We got hit by the virus that came from China.
\end{itemize}

edward wong

Trump was laying into China publicly for what had happened. You know,
the pandemic had spread into all corners of the U.S. The economy was in
shambles. Trump was seeing his re-election chances starting to go down
the drain.

\begin{itemize}
\item
  archived recording (donald trump)\\
  And we continue our relentless effort to defeat the Chinese virus.
\item
  archived recording\\
  Why do you keep using this? A lot of people say it's racist.
\item
  archived recording (donald trump)\\
  Because it comes from China. It's not racist at all, no. Not at all.
  It comes from China. That's why. It comes from China. I want to be
  accurate.
\end{itemize}

edward wong

And so his campaign strategists came up with this idea that they can try
and shift the conversation to China, rather than having people focused
on Trump's failures on the pandemic. And that by blaming China for all
of this, they could win back some of the votes that Trump's starting to
lose. Some of his top advisers. started speculating whether the virus
might have started from a lab accident ---

\begin{itemize}
\tightlist
\item
  archived recording\\
  I can tell you that there is a significant amount of evidence that
  this came from that laboratory in Wuhan.
\end{itemize}

edward wong

--- even though there was no evidence for that.

\begin{itemize}
\item
  archived recording\\
  Have you seen anything at this point that gives you a high degree of
  confidence that the Wuhan Institute of Virology was the origin of this
  virus?
\item
  archived recording (donald trump)\\
  Yes, I have. Yes, I have.
\end{itemize}

edward wong

So you have this very concerted effort by Trump to really cast China as
the person or the entity to blame for all of this.

\begin{itemize}
\tightlist
\item
  archived recording (donald trump)\\
  China's cover-up of the Wuhan virus allowed the disease to spread all
  over the world, instigating a global pandemic.
\end{itemize}

michael barbaro

And where does the pandemic fit into the kind of now familiar outlines
that you have described of the confrontation camp versus the cooperation
camp? I have to imagine it kind of challenges both.

edward wong

The pandemic really empowers the hawks in the administration to say, we
really have to go after China. Look at how their misgovernance, how
their political system led us to this point --- led America into an
economic crisis that's been the worst since the Great Depression. And
even the people in the cooperation camp are starting to change their
minds a bit. It's hard to tell the world that we should prioritize this
trade agreement that just rests on some agriculture purchases when
you've got this global crisis enveloping everything, and when American
citizens are anxious about their future.

michael barbaro

And how does China respond to these attacks from Trump and from his
advisers?

edward wong

So what we're hearing this spring is Chinese officials denouncing the
U.S. for all of these attacks. And they also point out that the Chinese
system actually has handled the virus a lot better than the American
system. They say even though there might have been this outbreak in
central China, look at how we controlled it through the measures we
took, and look at how the virus is running rampant in the U.S. And China
also then starts to try and send out aid to other countries. It starts
sending shipments of, for example, medical supplies, medical equipment,
facemasks, to other countries around the world, and even to parts of the
U.S., to try and sort of mask over its own responsibility for how the
outbreak began in its country. So the relationship between the two
powers was bad, and then it got worse.

\begin{itemize}
\tightlist
\item
  archived recording\\
  And we have some breaking news coming in. China's annual parliamentary
  meeting has been officially opened in Beijing, and it's expected that
  national security legislation for Hong Kong will be discussed during
  the seven-day session.
\end{itemize}

edward wong

In the late spring, Chinese officials start talking about this new
national security law that they want to impose on Hong Kong.

\begin{itemize}
\item
  archived recording 1\\
  Well, the specifics of the news has sent shockwaves across the city.
\item
  archived recording 2\\
  It says Beijing will set up a new National Security Bureau in Hong
  Kong, supervised by the central government to crack down on dissent in
  the city.
\item
  archived recording\\
  The legislation has faced sharp criticism from governments all around
  the world and sparked new protests in Hong Kong.
\end{itemize}

edward wong

And so this continues the downward spiral that U.S.-China relations have
been on.

michael barbaro

Right, and I'm imagining that that security law was especially upsetting
to those who want confrontation with China. That seems to be exactly the
kind of thing that they find so objectionable.

edward wong

That's right. As you recall, they were very upset at Trump for putting
the Hong Kong issue on the backburner in his aim to try and reach some
sort of trade truce with China. And now they were intent on pushing
forward on policies and actions that would make the Communist Party pay
a price, not only for what they would do in Hong Kong, but for their
actions in other parts of the world and for their role in the pandemic.
So they started announcing a series of actions against China that really
brought the relationship to a new low. They said that Hong Kong was no
longer an autonomous entity, and that the U.S. would break off its
special relationship with Hong Kong.

michael barbaro

Wow.

edward wong

They imposed visa restrictions on a category of students who were
associated with military institutions in China. They said that these
students can no longer come to the U.S. to do research or study because
of suspicions of potential economic espionage. They've even floated a
proposal internally to block all 92 million members of the Communist
Party from traveling to the U.S., as well as their family members, which
could encompass hundreds of millions of people. It's really felt like a
moment where the gloves have been taken off in this relationship, and
where the people in the administration who want to fundamentally
reorient the relationship with China have the upper hand right now.

michael barbaro

Edward, is it possible that, at the end of the day, what you're
describing here and the events of the past couple of weeks, it's the
right strategy for the U.S.? Because China is behaving in ways that
fundamentally violate American values, especially in Hong Kong,
especially with the Uighurs. And so no matter what motivates Donald
Trump to begin confronting China, is that potentially a good thing for
the United States?

edward wong

Well, the people who are supportive of the more confrontational approach
say that this type of strategy on China is long overdue. Now it's time
to really push back against China on all these fronts, especially at a
time when China hasn't overtaken the U.S. yet as the world's largest
economy and it's still a rising power. And this is a moment when we have
this opening to really reframe the conversation on China, not only U.S.,
but globally, and sort of rally countries to really confront China on a
whole range of issues.

michael barbaro

Right. So basically, this is our last chance?

edward wong

Right. They see it as time running out. Then you've got people on the
other camp who say, we don't know where this will end. This starts this
downward spiral in relations that starts to erode all the diplomatic
ties, economic ties, the people-to-people ties that have kept the
relationship firm over the decades, a relationship that's an unlikely
one. You've got this close relationship between a Western democracy and
an authoritarian state. And somehow, they've managed to avoid open
conflict. They've managed to avoid war. And where could we end up, where
could the world end up if we start breaking off those ties now?

michael barbaro

Right. It could end up in a pretty dangerous place.

edward wong

Right.

michael barbaro

So I want to return to where we started this conversation, Edward, which
is with the U.S. kicking China out of this consulate in Houston, because
it very much seems like this is the capstone to this approach. And I
wonder what the response has been from China, and what that tells us
about what this dynamic of confrontation is going to start to look like
over the next coming months and maybe even years.

edward wong

Well last Friday, we saw China announce that it was going to force the
U.S. to shut down its consulate in Chengdu, which is the only diplomatic
mission that the U.S. has in Western China. It's a very critical mission
for the U.S., because it allows American officials to observe what's
going on in the vast reaches of that part of the country, including in
Tibet, which is a very important issue for the U.S.

The people in Beijing couch this as a reciprocal action. And some people
still say that they could have taken a more escalatory step, but that
they appear to be willing to hold back and see whether there might be
some reset of the relationship if Trump loses the election in November.
But even if that were the case, I'm not sure that their orientation of
the relationship would change. There might be a temporary halt to the
tit-for-tat cycle that we're seeing. But it feels like because of where
the U.S. and China are now in the world, and the entrenched ideological
systems in both countries, we might be on course for a long-term
confrontation.

\begin{itemize}
\tightlist
\item
  archived recording (mike pompeo)\\
  Thank you. Thank you all.
\end{itemize}

edward wong

And you could hear that a few days ago in this very dark speech that
Secretary of State Mike Pompeo gave at the Nixon Library.

\begin{itemize}
\tightlist
\item
  archived recording (mike pompeo)\\
  We, the freedom-loving nations of the world, must induce China to
  change in more creative and assertive ways, because Beijing's actions
  threaten our people and our prosperity.
\end{itemize}

edward wong

He laid out a vision of a potential cold war with China, and said that
China was the most challenging foe to the United States.

\begin{itemize}
\tightlist
\item
  archived recording (mike pompeo)\\
  Now, people of good faith can debate why free agents allowed these bad
  things to happen for all these years. Perhaps we were naive about
  China's virulent strain of communism, or triumphalist after our
  victory in the Cold War. Or cravenly capitalist, or hoodwinked by
  Beijing's talk of a peaceful rise. Whatever the reason, whatever the
  reason, today, China is increasingly authoritarian at home and more
  aggressive in its hostility to freedom everywhere else. And President
  Trump has said enough.
\end{itemize}

{[}music{]}

michael barbaro

Edward, thank you very much.

edward wong

Thanks a lot, Michael. It's been great being on the show.

michael barbaro

We'll be right back.

Here's what else you need to know today.

\begin{itemize}
\item
  archived recording\\
  Mr. Barr, you may begin.
\item
  archived recording (william barr)\\
  Good morning, Mr. Chairman, Ranking Member Jordan. I'm pleased to be
  here this morning. On behalf of the Department of Justice, I want to
  pay my respects ---
\end{itemize}

michael barbaro

During his first appearance before the House since Democrats took
control in 2018, Attorney General Bill Barr was repeatedly challenged
over his response to everything from the Russia investigation to
nationwide protests over policing.

\begin{itemize}
\item
  archived recording (david cicilline)\\
  Is it ever appropriate, sir, for the president to solicit or accept
  foreign assistance in an election?
\item
  archived recording (william barr)\\
  It depends what kind of assistance.
\item
  archived recording (david cicilline)\\
  Is it ever appropriate for the president or presidential candidate to
  accept or solicit foreign assistance of any kind in his or her
  election?
\item
  archived recording (william barr)\\
  No, it's not appropriate.
\item
  archived recording (david cicilline)\\
  OK. Sorry you had to struggle with that one, Mr. Attorney General. Now
  let's turn to ---
\end{itemize}

michael barbaro

Several Democratic lawmakers, including Representative Pramila Jayapal
of Washington State, demanded to know why Barr had deployed federal
agents to Oregon to monitor Black Lives Matter protests, but not to
Michigan, where conservatives protested a coronavirus lockdown order.

\begin{itemize}
\item
  archived recording (pramila jayapal)\\
  There is a real discrepancy in how you react as the attorney general,
  the top cop in this country. When white men with swastikas storm a
  government building with guns, there is no need for the president to
  quote, ``activate you,'' because they're getting the president's
  personal agenda done. But when black people and people of color
  protest police brutality, systemic racism and the president's very own
  lack of response to those critical issues, then you forcibly remove
  them with armed federal officers, pepper bombs, because they are
  considered terrorists by the president. Did I get it right, Mr. Barr?
\item
  archived recording (william barr)\\
  I have responsibility for the federal government, and the White House
  is the seat of the ---
\item
  archived recording (pramila jayapal)\\
  Mr. Barr, let me just make it clear ---
\end{itemize}

michael barbaro

And on Tuesday, the nation's second-largest teachers' unit, the American
Federation of Teachers, announced that it would support members if they
choose to go on strike over unsafe school reopenings. The union said
that strikes should be a last resort, but the announcement gives local
teachers greater leverage in negotiations over the kinds of protections
that teachers should have in reopened schools.

{[}music{]}

That's it for ``The Daily.'' I'm Michael Barbaro. See you tomorrow.

\href{https://www.nytimes.com/column/the-daily}{\includegraphics{https://static01.nyt.com/images/2017/01/29/podcasts/the-daily-album-art/the-daily-album-art-square320-v4.png}The
Daily}Subscribe:

\begin{itemize}
\tightlist
\item
  \href{https://itunes.apple.com/us/podcast/id1200361736}{Apple
  Podcasts}
\item
  \href{https://www.google.com/podcasts?feed=aHR0cHM6Ly9yc3MuYXJ0MTkuY29tL3RoZS1kYWlseQ\%3D\%3D}{Google
  Podcasts}
\end{itemize}

\hypertarget{confronting-china-1}{%
\section{Confronting China}\label{confronting-china-1}}

\hypertarget{some-members-of-the-trump-administration-believe-the-superpower-country-poses-an-existential-threat-to-the-us--one-they-are-working-to-address-now-1}{%
\subsection{Some members of the Trump administration believe the
superpower country poses an existential threat to the U.S. --- one they
are working to address
now.}\label{some-members-of-the-trump-administration-believe-the-superpower-country-poses-an-existential-threat-to-the-us--one-they-are-working-to-address-now-1}}

Hosted by Michael Barbaro; produced by Asthaa Chaturvedi and Austin
Mitchell; with help from Neena Pathak and Sydney Harper; and edited by
Lisa Chow and Lisa Tobin

Transcript

transcript

Back to The Daily

bars

0:00/28:40

-0:00

transcript

\hypertarget{confronting-china-2}{%
\subsection{Confronting China}\label{confronting-china-2}}

\hypertarget{hosted-by-michael-barbaro-produced-by-asthaa-chaturvedi-and-austin-mitchell-with-help-from-neena-pathak-and-sydney-harper-and-edited-by-lisa-chow-and-lisa-tobin-1}{%
\subsubsection{Hosted by Michael Barbaro; produced by Asthaa Chaturvedi
and Austin Mitchell; with help from Neena Pathak and Sydney Harper; and
edited by Lisa Chow and Lisa
Tobin}\label{hosted-by-michael-barbaro-produced-by-asthaa-chaturvedi-and-austin-mitchell-with-help-from-neena-pathak-and-sydney-harper-and-edited-by-lisa-chow-and-lisa-tobin-1}}

\hypertarget{some-members-of-the-trump-administration-believe-the-superpower-country-poses-an-existential-threat-to-the-us--one-they-are-working-to-address-now-2}{%
\paragraph{Some members of the Trump administration believe the
superpower country poses an existential threat to the U.S. --- one they
are working to address
now.}\label{some-members-of-the-trump-administration-believe-the-superpower-country-poses-an-existential-threat-to-the-us--one-they-are-working-to-address-now-2}}

Wednesday, July 29th, 2020

\begin{itemize}
\item
  michael barbaro\\
  From The New York Times, I'm Michael Barbaro. This is ``The Daily.''
\item
  {[}music{]}\\
  Today: A cooperative relationship with China has been a pillar of the
  United States' foreign policy for more than half a century. Edward
  Wong on why the Trump administration believes it's time for a change.

  It's Wednesday, July 29.

  Edward, can you tell me what happened in Houston last week?
\item
  edward wong\\
  Sure. We first got a tip that something was up with the Chinese
  consulate in Houston around Tuesday afternoon or so --- that the
  Chinese ambassador to the U.S. had been told by American officials
  that he had three days to shut down the consulate, and that the
  employees here had 30 days to then leave the country. And a colleague
  and I started chasing this tip, but we couldn't quite nail it down to
  publish a story.
\item
  archived recording\\
  Right now at 10, breaking news ---
\end{itemize}

edward wong

And then ---

\begin{itemize}
\item
  archived recording 1\\
  Houston firefighters and police responding to the Chinese consulate in
  Montrose after reports of a fire.
\item
  archived recording 2\\
  Crews were called to the building off Montrose and Herald about 8:20
  tonight.
\end{itemize}

edward wong

In the evening, I started seeing these videos of people burning things
in metal barrels, in open metal barrels. And there was video of fire
trucks and police cars surrounding the consulate with their lights on,
so it's quite a dramatic scene.

\begin{itemize}
\tightlist
\item
  archived recording\\
  And local media were reporting that documents appeared to be being
  burned in the courtyard of that building.
\end{itemize}

edward wong

You know, for people in the national security world and the foreign
policy world, when you see people burning lots of documents or papers at
a diplomatic mission, the assumption is that they're about to clear out
of the mission. So when I saw those videos, I realized that the tip we
had gotten that the Chinese ambassador had been told to shut down the
Houston consulate within three days was indeed true.

\begin{itemize}
\tightlist
\item
  archived recording (wang wenbin)\\
  On July the 21st, the U.S. abruptly asked China to close its consulate
  in Houston.
\end{itemize}

edward wong

And within hours, the Chinese foreign ministry confirmed that in
Beijing.

\begin{itemize}
\tightlist
\item
  archived recording (wang wenbin)\\
  We urge the U.S. to reverse this incorrect decision immediately.
  Otherwise, China will definitely take necessary legitimate actions.
\end{itemize}

michael barbaro

And why would the U.S. take this pretty significant move of kicking
these Chinese diplomats out of this consulate in Texas?

edward wong

Officially, people in American government told us that they targeted the
Houston consulate because it was a hub of economic espionage and trade
secrets espionage in the U.S. But American officials haven't given us
detailed evidence on the activities undertaken by the Chinese diplomats.
And it's not clear to us how much farther these activities go beyond the
types of covert or espionage activities that take place at missions
around the world, including ones run by Americans. But in the bigger
picture, a main goal of some American officials in the Trump
administration is to unwind a range of diplomatic and economic ties that
have built up between the U.S. and China over the decades ever since
President Nixon started the opening of China back in 1971.

michael barbaro

So there's a version of this where the U.S. was looking for a reason to
unwind this relationship, and espionage --- real or not --- was that
reason.

edward wong

Right.

michael barbaro

And why would the Trump administration want to unwind its relationship
with China? I mean, it's our single biggest trading partner. It's a
global superpower. It's a nuclear power, so that's a pretty significant
decision.

edward wong

It is significant, and there are some senior officials in the
administration who are against this. Throughout the last three and a
half years, we've seen, broadly speaking, two factions of advisers on
China competing against each other for Trump's ear. And that helps
explain some of the contradictory impulses and policies that we've seen
coming out of the administration on China during this period.

michael barbaro

What do you mean?

edward wong

On one side, you had the ones wanting to confront China, in part over
trade, and also in part over national security matters. You had Peter
Navarro, who's a White House trade adviser.

\begin{itemize}
\tightlist
\item
  archived recording (peter navarro)\\
  How do you work with a country that lies through its teeth?
\end{itemize}

edward wong

Who wrote a book called ``Death by China,'' and then you also had
Secretary of State Mike Pompeo.

\begin{itemize}
\tightlist
\item
  archived recording (mike pompeo)\\
  They very much want to undermine our Western values, all the things
  that we hold most dear.
\end{itemize}

edward wong

And those people saw China as a threat to America. And then on the more
cooperative side, you have, for example, Treasury Secretary Steve
Mnuchin.

\begin{itemize}
\tightlist
\item
  archived recording (steve mnuchin)\\
  We need to work together to maximize the benefit for both sides.
\end{itemize}

edward wong

People who still clung to the classic notions of free trade and thought
that the traditional relationship with China was a stabilizing force in
the world. And that this had helped American companies get wealthy over
the years, as well as had benefited American consumers.

michael barbaro

And Edward, when comes to those who want to confront China, when it
comes to the Peter Navarros and the Mike Pompeos, what is their case for
why China is such a threat to the U.S. and should be reined in?

edward wong

Well, they argue that China presents a range of strategic threats to the
U.S. For example, they say that China's attempts to export its 5G
technology, its next generation communications technology, around the
world presents a security threat. They say that China's recent military
expansionism in the South China Sea, and its vast maritime claims in
that sea, are also a security threat, and they would impede American
military dominance in the Asia-Pacific. They point to attempts at
economic espionage by China and a vast range of cyber attacks that have
targeted the American government and other important institutions around
the world.

michael barbaro

Am I right to think that, from the start of his presidency, the
confrontation camp more or less prevailed?

edward wong

Well, it's complicated. The first big blow to the U.S.-China
relationship under the Trump administration was in mid-2018.

\begin{itemize}
\tightlist
\item
  archived recording (donald trump)\\
  We're going to have some incredible things. We're just announcing very
  big tariffs today on China, because China has been ---
\end{itemize}

edward wong

When President Trump started putting tariffs on billions of dollars of
goods made in China, China retaliated by doing the same on American
goods.

\begin{itemize}
\tightlist
\item
  archived recording\\
  So here's what they would do. They target farm products such as
  soybean cars, seafood ---
\end{itemize}

edward wong

And then just spiral downward from there.

\begin{itemize}
\item
  archived recording 1\\
  Medical equipment, energy products, that would start a little bit ---
\item
  archived recording 2\\
  As the U.S.-China trade war escalates, business leaders have been
  speaking up. FedEx C.E.O. ---
\end{itemize}

edward wong

So the trade war had this huge impact on companies, both in the U.S. and
outside of the U.S. And it created a lot of instability in their
thinking about how to do business.

\begin{itemize}
\tightlist
\item
  archived recording\\
  The escalating trade battle between the U.S. and China is rocking
  investors around the world.
\end{itemize}

edward wong

It created a lot of instability in the stock markets, which Trump
watches closely. And ---

\begin{itemize}
\tightlist
\item
  archived recording\\
  Some farmers in the U.S.A., the disruption of normal trade with China
  has forced many of them to go bankrupt.
\end{itemize}

edward wong

Important groups of voters who had supported Trump, for example, farmers
in the Midwest, were starting to suffer.

\begin{itemize}
\tightlist
\item
  archived recording\\
  I was a Trump voter. I voted for the president, certainly. But he
  certainly hasn't come through. He's lost on trade. He's lost on trade,
  and certainly ---
\end{itemize}

edward wong

They saw agricultural products like corn and soybeans piling up in the
Midwest, because China had imposed tariffs on their end to strike back
at Trump.

\begin{itemize}
\tightlist
\item
  archived recording\\
  So I won't be voting for the president again.
\end{itemize}

edward wong

So Trump and some of his economic advisers, especially those who were
preaching more cooperation with China, started to get nervous about
these economic signs that they were seeing, as well as about the
anxieties of these midwestern farmers and potential Trump voters there.

michael barbaro

And so what do they do, these cooperation camp folks who are not happy
with this trade war?

edward wong

Well, as they go through negotiations for a potential truce to the trade
war, President Trump talks with President Xi of China several times. And
they have these, like, sort of one-on-one conversations that Trump likes
to do with leaders. And in each of these, Trump sort of cozies up to Xi,
and it's clear he's willing to sort of brush aside a lot of sort of the
most egregious behaviors of China in the pursuit of this trade deal.

michael barbaro

Like what?

edward wong

In one conversation, according to John Bolton --- the former national
security adviser --- Trump encouraged Xi to actually continue building
internment camps for Muslims in the northwest of China and sort of
signaled that this wasn't a big issue for him.

michael barbaro

These are the Uighurs?

edward wong

Right, these are the Uighurs, exactly. The ones a million or more held
over recent years in internment camps. And, for example, we've seen
these, during this period, these pro-democracy protests arise in Hong
Kong. And while Trump's national security aides are supportive of them,
Trump himself tells Xi privately in a phone call that Xi should just
handle those in whatever way he wants to deal with those, and that Trump
himself will not say anything about those, and he'll tell his aides not
to say anything vocally among those protests either.

michael barbaro

So in this trade war that's supposed to represent confrontation with
China, there's actually a fair amount of cooperation going on, most of
it behind the scenes.

edward wong

Right, exactly. And ultimately, in December, they reach a tentative
agreement, and then they signed that in January of this year. And I
think that brought a big sense of relief to the people in the
cooperation camp. I think they were relieved to see a sense of stability
returned to this key economic relationship. Now, the confrontation
people in the White House and in other agencies were generally
disappointed, I think, by the outcome of the deal. They felt that Trump
had sidelined a lot of the hardline policies they had pushed for in the
first half of the administration for the sake of just trying to get a
marginal increase in agricultural purchases. And also, there is a sense
of outrage among some of them.

And this was in John Bolton's recent book, that Trump was also aiming
for this negotiated truce purely for re-election purposes. That he
pleaded with Xi in a conversation that Xi should get help him get
re-elected, should help him win, and that the best way to do this was to
reach some sort of truce or deal in the trade war that he could then
bring back to his constituents. And so certain national security people
were outraged by this, saying that Trump was focused purely on personal
politics and was not looking after the national security interests of
the United States.

michael barbaro

Edward, what you're describing so far, especially this trade deal, does
not seem like a relationship that is about to be fundamentally unwound
and blown up. So what happens to get us from that truce to now, into the
shutdown of this consulate in Houston?

edward wong

Well, what changes things is this pandemic that starts in central China
and spreads across the globe.

That sets the two powers on a much more confrontational course.

{[}music{]}

michael barbaro

We'll be right back.

\begin{itemize}
\tightlist
\item
  archived recording (donald trump)\\
  I spoke with President Xi. We had a great talk. He's working very
  hard. I have to say he's working very, very hard.
\end{itemize}

edward wong

So in the first weeks after the virus started spreading around the
globe, Trump was still praising Xi publicly.

\begin{itemize}
\tightlist
\item
  archived recording (donald trump)\\
  If you know anything about him, I think he'll be in pretty good shape.
  They've had a rough patch, and I think right now, they have it --- it
  looks like they're getting it under control more and more.
\end{itemize}

edward wong

This was in January and February right after they had signed the trade
agreement, so Trump was still in this mode where he wanted intensely to
preserve that negotiated truce. But by the spring ---

\begin{itemize}
\tightlist
\item
  archived recording (donald trump)\\
  We got hit by the virus that came from China.
\end{itemize}

edward wong

Trump was laying into China publicly for what had happened. You know,
the pandemic had spread into all corners of the U.S. The economy was in
shambles. Trump was seeing his re-election chances starting to go down
the drain.

\begin{itemize}
\item
  archived recording (donald trump)\\
  And we continue our relentless effort to defeat the Chinese virus.
\item
  archived recording\\
  Why do you keep using this? A lot of people say it's racist.
\item
  archived recording (donald trump)\\
  Because it comes from China. It's not racist at all, no. Not at all.
  It comes from China. That's why. It comes from China. I want to be
  accurate.
\end{itemize}

edward wong

And so his campaign strategists came up with this idea that they can try
and shift the conversation to China, rather than having people focused
on Trump's failures on the pandemic. And that by blaming China for all
of this, they could win back some of the votes that Trump's starting to
lose. Some of his top advisers. started speculating whether the virus
might have started from a lab accident ---

\begin{itemize}
\tightlist
\item
  archived recording\\
  I can tell you that there is a significant amount of evidence that
  this came from that laboratory in Wuhan.
\end{itemize}

edward wong

--- even though there was no evidence for that.

\begin{itemize}
\item
  archived recording\\
  Have you seen anything at this point that gives you a high degree of
  confidence that the Wuhan Institute of Virology was the origin of this
  virus?
\item
  archived recording (donald trump)\\
  Yes, I have. Yes, I have.
\end{itemize}

edward wong

So you have this very concerted effort by Trump to really cast China as
the person or the entity to blame for all of this.

\begin{itemize}
\tightlist
\item
  archived recording (donald trump)\\
  China's cover-up of the Wuhan virus allowed the disease to spread all
  over the world, instigating a global pandemic.
\end{itemize}

michael barbaro

And where does the pandemic fit into the kind of now familiar outlines
that you have described of the confrontation camp versus the cooperation
camp? I have to imagine it kind of challenges both.

edward wong

The pandemic really empowers the hawks in the administration to say, we
really have to go after China. Look at how their misgovernance, how
their political system led us to this point --- led America into an
economic crisis that's been the worst since the Great Depression. And
even the people in the cooperation camp are starting to change their
minds a bit. It's hard to tell the world that we should prioritize this
trade agreement that just rests on some agriculture purchases when
you've got this global crisis enveloping everything, and when American
citizens are anxious about their future.

michael barbaro

And how does China respond to these attacks from Trump and from his
advisers?

edward wong

So what we're hearing this spring is Chinese officials denouncing the
U.S. for all of these attacks. And they also point out that the Chinese
system actually has handled the virus a lot better than the American
system. They say even though there might have been this outbreak in
central China, look at how we controlled it through the measures we
took, and look at how the virus is running rampant in the U.S. And China
also then starts to try and send out aid to other countries. It starts
sending shipments of, for example, medical supplies, medical equipment,
facemasks, to other countries around the world, and even to parts of the
U.S., to try and sort of mask over its own responsibility for how the
outbreak began in its country. So the relationship between the two
powers was bad, and then it got worse.

\begin{itemize}
\tightlist
\item
  archived recording\\
  And we have some breaking news coming in. China's annual parliamentary
  meeting has been officially opened in Beijing, and it's expected that
  national security legislation for Hong Kong will be discussed during
  the seven-day session.
\end{itemize}

edward wong

In the late spring, Chinese officials start talking about this new
national security law that they want to impose on Hong Kong.

\begin{itemize}
\item
  archived recording 1\\
  Well, the specifics of the news has sent shockwaves across the city.
\item
  archived recording 2\\
  It says Beijing will set up a new National Security Bureau in Hong
  Kong, supervised by the central government to crack down on dissent in
  the city.
\item
  archived recording\\
  The legislation has faced sharp criticism from governments all around
  the world and sparked new protests in Hong Kong.
\end{itemize}

edward wong

And so this continues the downward spiral that U.S.-China relations have
been on.

michael barbaro

Right, and I'm imagining that that security law was especially upsetting
to those who want confrontation with China. That seems to be exactly the
kind of thing that they find so objectionable.

edward wong

That's right. As you recall, they were very upset at Trump for putting
the Hong Kong issue on the backburner in his aim to try and reach some
sort of trade truce with China. And now they were intent on pushing
forward on policies and actions that would make the Communist Party pay
a price, not only for what they would do in Hong Kong, but for their
actions in other parts of the world and for their role in the pandemic.
So they started announcing a series of actions against China that really
brought the relationship to a new low. They said that Hong Kong was no
longer an autonomous entity, and that the U.S. would break off its
special relationship with Hong Kong.

michael barbaro

Wow.

edward wong

They imposed visa restrictions on a category of students who were
associated with military institutions in China. They said that these
students can no longer come to the U.S. to do research or study because
of suspicions of potential economic espionage. They've even floated a
proposal internally to block all 92 million members of the Communist
Party from traveling to the U.S., as well as their family members, which
could encompass hundreds of millions of people. It's really felt like a
moment where the gloves have been taken off in this relationship, and
where the people in the administration who want to fundamentally
reorient the relationship with China have the upper hand right now.

michael barbaro

Edward, is it possible that, at the end of the day, what you're
describing here and the events of the past couple of weeks, it's the
right strategy for the U.S.? Because China is behaving in ways that
fundamentally violate American values, especially in Hong Kong,
especially with the Uighurs. And so no matter what motivates Donald
Trump to begin confronting China, is that potentially a good thing for
the United States?

edward wong

Well, the people who are supportive of the more confrontational approach
say that this type of strategy on China is long overdue. Now it's time
to really push back against China on all these fronts, especially at a
time when China hasn't overtaken the U.S. yet as the world's largest
economy and it's still a rising power. And this is a moment when we have
this opening to really reframe the conversation on China, not only U.S.,
but globally, and sort of rally countries to really confront China on a
whole range of issues.

michael barbaro

Right. So basically, this is our last chance?

edward wong

Right. They see it as time running out. Then you've got people on the
other camp who say, we don't know where this will end. This starts this
downward spiral in relations that starts to erode all the diplomatic
ties, economic ties, the people-to-people ties that have kept the
relationship firm over the decades, a relationship that's an unlikely
one. You've got this close relationship between a Western democracy and
an authoritarian state. And somehow, they've managed to avoid open
conflict. They've managed to avoid war. And where could we end up, where
could the world end up if we start breaking off those ties now?

michael barbaro

Right. It could end up in a pretty dangerous place.

edward wong

Right.

michael barbaro

So I want to return to where we started this conversation, Edward, which
is with the U.S. kicking China out of this consulate in Houston, because
it very much seems like this is the capstone to this approach. And I
wonder what the response has been from China, and what that tells us
about what this dynamic of confrontation is going to start to look like
over the next coming months and maybe even years.

edward wong

Well last Friday, we saw China announce that it was going to force the
U.S. to shut down its consulate in Chengdu, which is the only diplomatic
mission that the U.S. has in Western China. It's a very critical mission
for the U.S., because it allows American officials to observe what's
going on in the vast reaches of that part of the country, including in
Tibet, which is a very important issue for the U.S.

The people in Beijing couch this as a reciprocal action. And some people
still say that they could have taken a more escalatory step, but that
they appear to be willing to hold back and see whether there might be
some reset of the relationship if Trump loses the election in November.
But even if that were the case, I'm not sure that their orientation of
the relationship would change. There might be a temporary halt to the
tit-for-tat cycle that we're seeing. But it feels like because of where
the U.S. and China are now in the world, and the entrenched ideological
systems in both countries, we might be on course for a long-term
confrontation.

\begin{itemize}
\tightlist
\item
  archived recording (mike pompeo)\\
  Thank you. Thank you all.
\end{itemize}

edward wong

And you could hear that a few days ago in this very dark speech that
Secretary of State Mike Pompeo gave at the Nixon Library.

\begin{itemize}
\tightlist
\item
  archived recording (mike pompeo)\\
  We, the freedom-loving nations of the world, must induce China to
  change in more creative and assertive ways, because Beijing's actions
  threaten our people and our prosperity.
\end{itemize}

edward wong

He laid out a vision of a potential cold war with China, and said that
China was the most challenging foe to the United States.

\begin{itemize}
\tightlist
\item
  archived recording (mike pompeo)\\
  Now, people of good faith can debate why free agents allowed these bad
  things to happen for all these years. Perhaps we were naive about
  China's virulent strain of communism, or triumphalist after our
  victory in the Cold War. Or cravenly capitalist, or hoodwinked by
  Beijing's talk of a peaceful rise. Whatever the reason, whatever the
  reason, today, China is increasingly authoritarian at home and more
  aggressive in its hostility to freedom everywhere else. And President
  Trump has said enough.
\end{itemize}

{[}music{]}

michael barbaro

Edward, thank you very much.

edward wong

Thanks a lot, Michael. It's been great being on the show.

michael barbaro

We'll be right back.

Here's what else you need to know today.

\begin{itemize}
\item
  archived recording\\
  Mr. Barr, you may begin.
\item
  archived recording (william barr)\\
  Good morning, Mr. Chairman, Ranking Member Jordan. I'm pleased to be
  here this morning. On behalf of the Department of Justice, I want to
  pay my respects ---
\end{itemize}

michael barbaro

During his first appearance before the House since Democrats took
control in 2018, Attorney General Bill Barr was repeatedly challenged
over his response to everything from the Russia investigation to
nationwide protests over policing.

\begin{itemize}
\item
  archived recording (david cicilline)\\
  Is it ever appropriate, sir, for the president to solicit or accept
  foreign assistance in an election?
\item
  archived recording (william barr)\\
  It depends what kind of assistance.
\item
  archived recording (david cicilline)\\
  Is it ever appropriate for the president or presidential candidate to
  accept or solicit foreign assistance of any kind in his or her
  election?
\item
  archived recording (william barr)\\
  No, it's not appropriate.
\item
  archived recording (david cicilline)\\
  OK. Sorry you had to struggle with that one, Mr. Attorney General. Now
  let's turn to ---
\end{itemize}

michael barbaro

Several Democratic lawmakers, including Representative Pramila Jayapal
of Washington State, demanded to know why Barr had deployed federal
agents to Oregon to monitor Black Lives Matter protests, but not to
Michigan, where conservatives protested a coronavirus lockdown order.

\begin{itemize}
\item
  archived recording (pramila jayapal)\\
  There is a real discrepancy in how you react as the attorney general,
  the top cop in this country. When white men with swastikas storm a
  government building with guns, there is no need for the president to
  quote, ``activate you,'' because they're getting the president's
  personal agenda done. But when black people and people of color
  protest police brutality, systemic racism and the president's very own
  lack of response to those critical issues, then you forcibly remove
  them with armed federal officers, pepper bombs, because they are
  considered terrorists by the president. Did I get it right, Mr. Barr?
\item
  archived recording (william barr)\\
  I have responsibility for the federal government, and the White House
  is the seat of the ---
\item
  archived recording (pramila jayapal)\\
  Mr. Barr, let me just make it clear ---
\end{itemize}

michael barbaro

And on Tuesday, the nation's second-largest teachers' unit, the American
Federation of Teachers, announced that it would support members if they
choose to go on strike over unsafe school reopenings. The union said
that strikes should be a last resort, but the announcement gives local
teachers greater leverage in negotiations over the kinds of protections
that teachers should have in reopened schools.

{[}music{]}

That's it for ``The Daily.'' I'm Michael Barbaro. See you tomorrow.

Previous

More episodes ofThe Daily

\href{https://www.nytimes.com/2020/08/03/podcasts/the-daily/algorithmic-justice-racism.html?action=click\&module=audio-series-bar\&region=header\&pgtype=Article}{\includegraphics{https://static01.nyt.com/images/2020/06/24/business/03daily/24michigan-arrest1-thumbLarge.jpg}}

August 3, 2020~~•~ 28:13Wrongfully Accused by an Algorithm

\href{https://www.nytimes.com/2020/08/02/podcasts/the-daily/on-female-rage.html?action=click\&module=audio-series-bar\&region=header\&pgtype=Article}{\includegraphics{https://static01.nyt.com/images/2018/01/21/magazine/21mag-femaleanger1-copy/21mag-femaleanger1-thumbLarge.jpg}}

August 2, 2020The Sunday Read: `On Female Rage'

\href{https://www.nytimes.com/2020/07/31/podcasts/the-daily/vanessa-guillen-military-metoo.html?action=click\&module=audio-series-bar\&region=header\&pgtype=Article}{\includegraphics{https://static01.nyt.com/images/2020/07/12/us/politics/31daily/00dc-army-metoo-thumbLarge.jpg}}

July 31, 2020A \#MeToo Moment in the Military

\href{https://www.nytimes.com/2020/07/30/podcasts/the-daily/congress-facebook-amazon-google-apple.html?action=click\&module=audio-series-bar\&region=header\&pgtype=Article}{\includegraphics{https://static01.nyt.com/images/2020/07/30/reader-center/30daily/merlin_175077825_5ebc931b-baa1-489a-960c-34e4d845e997-thumbLarge.jpg}}

July 30, 2020~~•~ 35:19The Big Tech Hearing

\href{https://www.nytimes.com/2020/07/29/podcasts/the-daily/china-trump-foreign-policy.html?action=click\&module=audio-series-bar\&region=header\&pgtype=Article}{\includegraphics{https://static01.nyt.com/images/2020/07/26/world/29daily/00china-us-clash1-thumbLarge.jpg}}

July 29, 2020~~•~ 28:40Confronting China

\href{https://www.nytimes.com/2020/07/28/podcasts/the-daily/unemployment-benefits-coronavirus.html?action=click\&module=audio-series-bar\&region=header\&pgtype=Article}{\includegraphics{https://static01.nyt.com/images/2020/07/23/business/28daily/23virus-uiexplain1-thumbLarge.jpg}}

July 28, 2020~~•~ 26:13Why \$600 Checks Are Tearing Republicans Apart

\href{https://www.nytimes.com/2020/07/27/podcasts/the-daily/new-york-hospitals-covid.html?action=click\&module=audio-series-bar\&region=header\&pgtype=Article}{\includegraphics{https://static01.nyt.com/images/2020/07/27/world/27daily-hospitals/27daily-hospitals-thumbLarge.jpg}}

July 27, 2020~~•~ 33:28The Mistakes New York Made

\href{https://www.nytimes.com/2020/07/26/podcasts/the-daily/the-accusation-the-sunday-read.html?action=click\&module=audio-series-bar\&region=header\&pgtype=Article}{\includegraphics{https://static01.nyt.com/images/2020/03/22/magazine/26audm-2/22mag-titleix-thumbLarge.jpg}}

July 26, 2020The Sunday Read: `The Accusation'

\href{https://www.nytimes.com/2020/07/24/podcasts/the-daily/mlb-baseball-season-coronavirus.html?action=click\&module=audio-series-bar\&region=header\&pgtype=Article}{\includegraphics{https://static01.nyt.com/images/2020/07/22/sports/24daily/22mlb-previewlede1-thumbLarge.jpg}}

July 24, 2020~~•~ 45:34The Battle for a Baseball Season

\href{https://www.nytimes.com/2020/07/23/podcasts/the-daily/portland-protests.html?action=click\&module=audio-series-bar\&region=header\&pgtype=Article}{\includegraphics{https://static01.nyt.com/images/2020/07/22/us/23daily-image/22portland-tactics02-thumbLarge.jpg}}

July 23, 2020~~•~ 30:04The Showdown in Portland

\href{https://www.nytimes.com/2020/07/22/podcasts/the-daily/school-reopenings-coronavirus.html?action=click\&module=audio-series-bar\&region=header\&pgtype=Article}{\includegraphics{https://static01.nyt.com/images/2020/07/12/science/22daily/00virus-schools-reopen01-thumbLarge.jpg}}

July 22, 2020~~•~ 27:24The Science of School Reopenings

\href{https://www.nytimes.com/2020/07/21/podcasts/the-daily/coronavirus-vaccine.html?action=click\&module=audio-series-bar\&region=header\&pgtype=Article}{\includegraphics{https://static01.nyt.com/images/2020/07/19/science/21daily/00VIRUS-VAX-DOUBTS1-thumbLarge.jpg}}

July 21, 2020~~•~ 29:14The Vaccine Trust Problem

\href{https://www.nytimes.com/column/the-daily}{See All Episodes ofThe
Daily}

Next

July 29, 2020

\begin{itemize}
\item
\item
\item
\item
\item
\item
\end{itemize}

\emph{\textbf{Listen and subscribe to our podcast from your mobile
device:}}\\
\textbf{\href{https://itunes.apple.com/us/podcast/the-daily/id1200361736?mt=2}{\emph{Via
Apple Podcasts}}} \emph{\textbf{\textbar{}}}
\textbf{\href{https://open.spotify.com/show/3IM0lmZxpFAY7CwMuv9H4g?si=SfuMSC55R1qprFsRZU3_zw}{\emph{Via
Spotify}}} \emph{\textbf{\textbar{}}}
\textbf{\href{http://www.stitcher.com/podcast/the-new-york-times/the-daily-10}{\emph{Via
Stitcher}}}

A cooperative relationship with China has been a pillar of U.S. foreign
policy for more than half a century. So why does the Trump
administration think it's time for a change?

\textbf{On today's episode:}

\begin{itemize}
\tightlist
\item
  \href{https://www.nytimes.com/by/edward-wong}{Edward Wong}, a
  diplomatic correspondent for The New York Times.
\end{itemize}

\includegraphics{https://static01.nyt.com/images/2020/07/26/world/29daily/merlin_157181268_478b9364-1e98-4d34-a4af-7e959f4ae9a8-articleLarge.jpg?quality=75\&auto=webp\&disable=upscale}

\textbf{Background reading:}

\begin{itemize}
\tightlist
\item
  Why top aides to President Trump want to leave a
  \href{https://www.nytimes.com/2020/07/25/world/asia/us-china-trump-xi.html}{lasting
  legacy of ruptured ties} between China and the United States.
\end{itemize}

\emph{Tune in, and tell us what you think. Email us at}
\href{mailto:thedaily@nytimes.com}{\emph{thedaily@nytimes.com}}\emph{.
Follow Michael Barbaro on Twitter:}
\href{https://twitter.com/mikiebarb}{\emph{@mikiebarb}}\emph{. And if
you're interested in advertising with ``The Daily,'' write to us at}
\href{mailto:thedaily-ads@nytimes.com}{\emph{thedaily-ads@nytimes.com}}\emph{.}

Edward Wong contributed reporting.

``The Daily'' is made by Theo Balcomb, Andy Mills, Lisa Tobin, Rachel
Quester, Lynsea Garrison, Annie Brown, Clare Toeniskoetter, Paige
Cowett, Michael Simon Johnson, Brad Fisher, Larissa Anderson, Wendy
Dorr, Chris Wood, Jessica Cheung, Stella Tan, Alexandra Leigh Young,
Jonathan Wolfe, Lisa Chow, Eric Krupke, Marc Georges, Luke Vander Ploeg,
Adizah Eghan, Kelly Prime, Julia Longoria, Sindhu Gnanasambandan, M.J.
Davis Lin, Austin Mitchell, Sayre Quevedo, Neena Pathak, Dan Powell,
Dave Shaw, Sydney Harper, Daniel Guillemette, Hans Buetow, Robert
Jimison, Mike Benoist, Bianca Giaever and Asthaa Chaturvedi. Our theme
music is by Jim Brunberg and Ben Landsverk of Wonderly. Special thanks
to Sam Dolnick, Mikayla Bouchard, Lauren Jackson, Julia Simon, Mahima
Chablani and Nora Keller.

Advertisement

\protect\hyperlink{after-bottom}{Continue reading the main story}

\hypertarget{site-index}{%
\subsection{Site Index}\label{site-index}}

\hypertarget{site-information-navigation}{%
\subsection{Site Information
Navigation}\label{site-information-navigation}}

\begin{itemize}
\tightlist
\item
  \href{https://help.nytimes.com/hc/en-us/articles/115014792127-Copyright-notice}{©~2020~The
  New York Times Company}
\end{itemize}

\begin{itemize}
\tightlist
\item
  \href{https://www.nytco.com/}{NYTCo}
\item
  \href{https://help.nytimes.com/hc/en-us/articles/115015385887-Contact-Us}{Contact
  Us}
\item
  \href{https://www.nytco.com/careers/}{Work with us}
\item
  \href{https://nytmediakit.com/}{Advertise}
\item
  \href{http://www.tbrandstudio.com/}{T Brand Studio}
\item
  \href{https://www.nytimes.com/privacy/cookie-policy\#how-do-i-manage-trackers}{Your
  Ad Choices}
\item
  \href{https://www.nytimes.com/privacy}{Privacy}
\item
  \href{https://help.nytimes.com/hc/en-us/articles/115014893428-Terms-of-service}{Terms
  of Service}
\item
  \href{https://help.nytimes.com/hc/en-us/articles/115014893968-Terms-of-sale}{Terms
  of Sale}
\item
  \href{https://spiderbites.nytimes.com}{Site Map}
\item
  \href{https://help.nytimes.com/hc/en-us}{Help}
\item
  \href{https://www.nytimes.com/subscription?campaignId=37WXW}{Subscriptions}
\end{itemize}
