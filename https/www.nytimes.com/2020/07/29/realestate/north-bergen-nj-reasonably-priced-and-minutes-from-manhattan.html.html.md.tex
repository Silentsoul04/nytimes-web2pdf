Sections

SEARCH

\protect\hyperlink{site-content}{Skip to
content}\protect\hyperlink{site-index}{Skip to site index}

\href{https://www.nytimes.com/section/realestate}{Real Estate}

\href{https://myaccount.nytimes.com/auth/login?response_type=cookie\&client_id=vi}{}

\href{https://www.nytimes.com/section/todayspaper}{Today's Paper}

\href{/section/realestate}{Real Estate}\textbar{}North Bergen, N.J.:
Reasonably Priced and Minutes From Manhattan

\begin{itemize}
\item
\item
\item
\item
\item
\item
\end{itemize}

Advertisement

\protect\hyperlink{after-top}{Continue reading the main story}

Supported by

\protect\hyperlink{after-sponsor}{Continue reading the main story}

Living in

\hypertarget{north-bergen-nj-reasonably-priced-and-minutes-from-manhattan}{%
\section{North Bergen, N.J.: Reasonably Priced and Minutes From
Manhattan}\label{north-bergen-nj-reasonably-priced-and-minutes-from-manhattan}}

Supporters praise the community's young, culturally diverse population
and its (relatively) affordable housing. And then there are the views.

\href{https://www.nytimes.com/slideshow/2020/07/29/realestate/living-in-north-bergen-nj.html}{}

\hypertarget{living-in--north-bergen-nj}{%
\subsection{Living In ... North Bergen,
N.J.}\label{living-in--north-bergen-nj}}

15 Photos

View Slide Show ›

\includegraphics{https://static01.nyt.com/images/2020/07/29/realestate/29LIVING-NORTHBERGEN-slide-MXJF/29LIVING-NORTHBERGEN-slide-MXJF-articleLarge.jpg?quality=75\&auto=webp\&disable=upscale}

Chang W. Lee/The New York Times

By Julie Lasky

\begin{itemize}
\item
  Published July 29, 2020Updated July 31, 2020
\item
  \begin{itemize}
  \item
  \item
  \item
  \item
  \item
  \item
  \end{itemize}
\end{itemize}

In 1975, Calvin Trillin wrote in The New Yorker magazine about ``the
half-dozen cities in northern Hudson County that are, except for some
arbitrary and invisible lines drawn every mile or so, one mass of
blue-collar sprawl running along the Hudson, just across from midtown
Manhattan.''

North Bergen, N.J., sits at the top of that urban cluster, but times
have changed. Today, Mr. Trillin's ``sprawl'' is better known as the New
Jersey Gold Coast: the riverfront communities that extend from Bayonne
north to the George Washington Bridge, sprouting expensive luxury
buildings that eat the Manhattan views with a spoon.

Cliffside

Park

New JERSEY

N.J. TURNPIKE

North Bergen

Free Public

Library

North

Bergen

HUDSON

COUNTY

James J.

Braddock

Park

Hackensack

Meridian Health

Palisades

Medical Center

Secaucus

N.J. TRANSIT

Hudson

River

N.J. TURNPIKE

N.Y.

PA.

North

Bergen

Union

City

New

York

City

HUDSON

N.J.

1/2 mile

By The New York Times

North Bergen could be accused of lacking definition. The
five-square-mile township with about 62,000 residents has irregular
boundaries that bend around neighbors; it is sometimes hard to know
exactly when you are there. But its supporters single it out as a
reasonably priced community minutes from Manhattan, with an increasingly
young, culturally diverse population.

Lakshay Bhatia, 38, a technology consultant, discovered North Bergen
eight years ago when he and his wife, Neetika, set off from their
Englewood, N.J., rental on the Fourth of July to see the fireworks in
Hoboken. The roads were blocked, and the couple found themselves on John
F. Kennedy Boulevard East, known locally as Boulevard East, with
startling views of the New York skyline.

``Oh, wow,'' Mr. Bhatia recalled thinking.

The Bhatias, who are from India, decided that Boulevard East was their
future. They eventually paid \$470,000 for a two-family house three
blocks away and live with their 3-year-old daughter on the lower floor,
renting the upstairs duplex.

Commuting to Manhattan before the pandemic, Ms. Bhatia, 38, a textile
designer, boarded the buses that streamed down Boulevard East each
morning. Now the Bhatias work from home and de-stress in their backyard,
or in James J. Braddock Park, a 167-acre sward three blocks away with a
16-acre lake and 45 athletic facilities (tracks, courts, swings, et
cetera).

\includegraphics{https://static01.nyt.com/images/2020/07/29/realestate/29LIVING-NORTHBERGEN-slide-J7BF/29LIVING-NORTHBERGEN-slide-J7BF-articleLarge.jpg?quality=75\&auto=webp\&disable=upscale}

``Affordability is a key for North Bergen,'' said Frances Rosado, a real
estate agent with Keller Williams, who grew up in Hudson County. Born in
Cuba, she was part of the diaspora that turned the area into what was
once described as ``Havana on the Hudson.'' The Hispanic population,
which makes up 71 percent of North Bergen, has since diversified.

If you're priced out of Jersey City, Hoboken or Weehawken, you can still
buy something in the township, Ms. Rosado said. Single-family houses
start in the low \$300,000s.

There are luxury options, too, like the Duchess, a three-year-old
apartment complex near the Hudson with 320 units in a trio of
interconnected 12-story towers. The amenities include a heated outdoor
pool that was recently reopened, a fitness area and a dog park.

Narcis Versteeg, 30, an Afghani-born lactation nurse, and her husband,
Erik Versteeg, 33, a Dutch-born investment banker, moved to the Duchess
in September. They chose it, Ms. Versteeg said, because the units are
much bigger than their previous residence in Long Island City (before
that, they lived in Amsterdam), ``and there is more nature around.''

Their rental costs \$4,500 a month for two bedrooms, two and a half
bathrooms, two walk-in closets and Manhattan views. Outdoor parking for
the couple's new car is \$125 a month, and the pet fee for their
bernedoodle puppy, Ola, is \$50 a month. Michael Pestronk, the Duchess's
developer, said the building stakes out a middle ground between urban
convenience and suburban expansiveness, and is close to 90 percent
occupied. The website currently offers two free months' rent for new
leases on select apartments.

Image

8818 CHURCHILL ROAD \textbar{} A townhouse with four bedrooms, four full
bathrooms, two half bathrooms and river views,~built in 2017 on a
1,516-square-foot lot, listed for \$2.25 million.
973-945-0362Credit...Chang W. Lee/The New York Times

\hypertarget{what-youll-find}{%
\subsection{What You'll Find}\label{what-youll-find}}

North Bergen is the northernmost municipality in Hudson County. Shaped
like a letter ``L'' that has been turned upside down, it has a
proportionally short section of riverfront, much of it taken up by the
Hackensack Meridian Health Palisades medical complex. (It also shares a
1.5-acre waterfront park with neighboring Guttenberg.) The stem of the
``L'' is sandwiched between Guttenberg, West New York and Union City to
the east, and the Meadowlands, Secaucus and Kearny to the west. Bergen
County is north and Jersey City south.

What North Bergen lacks in shoreline it makes up for in dramatic
topography. Built on the Palisades clifftops, it has many east-west
streets with precarious slopes and single-family houses tightly packed
along the sides.

North-south arteries --- principally Tonnelle Avenue, John F. Kennedy
Boulevard, Bergenline Avenue and River Road --- bind North Bergen to its
neighbors and broadcast the region's ethnic character.

Bergenline Avenue, the main commercial stretch, is dominated by
businesses run by Mexicans, Puerto Ricans, Dominicans, Ecuadoreans,
Salvadorans, Colombians, Peruvians, Hondurans, Cubans and their progeny.
Recent reconfigurations have provided more room on the avenue for
parking.

But pizzerias, a vestige of the decades when North Bergen was largely
Italian, give taquerias a run for their money here. Popular examples
include Roma on John F. Kennedy Boulevard and Gandolfo on Bergenline.

Neighborhoods are as diverse as the landscape. In addition to the high-
and low-rise condos and apartment buildings near the waterfront, there
is the Racetrack district, between Bergenline Avenue and John F. Kennedy
Boulevard, named for a notorious 19th-century gambling attraction that
evolved into an amusement park. This area borders the west side of
Braddock Park and takes in the high school, the public library and the
vintage White Castle on Kennedy Boulevard.

Woodcliff, on the south side of the park, between Boulevard East and
Bergenline, is prized for its access to recreation and public
transportation.

Bergenwood lies between John F. Kennedy Boulevard and Tonnelle Avenue
and has especially steep grades.

Image

2002 45th STREET \textbar{} A two-family house in the New Durham
neighborhood with a total of six bedrooms and five bathrooms, built in
1916 on a 6,255-square-foot lot, listed for \$849,995.
888-501-6953Credit...Chang W. Lee/The New York Times

\hypertarget{what-youll-pay}{%
\subsection{What You'll Pay}\label{what-youll-pay}}

Real estate agents say the housing supply is tight for home buyers but
abundant for renters, who compose about 60 percent of the market.
Ricardo Garcia, an agent with Re/Max Villa in North Bergen, said that
many commuters who rent but have no certain timeline for returning to
their offices are moving to less dense regions to work from home. (As of
July 27,
\href{https://hudson-county-coronavirus-resources-hudsoncogis.hub.arcgis.com/}{Hudson
County} had 19,928 reported Covid-19 cases and 1,380 deaths.) According
to Zillow, the median rent in North Bergen is \$1,990.

Data from the real estate brokerage Redfin shows that the average home
sale price in June declined 15.3 percent over the previous year, to
\$360,000. The average time on the market was 94 days.

As of July 28, 159 residential properties were listed for sale, ranging
from a 435-square-foot studio co-op on Boulevard East for \$115,000,
with a monthly homeowner's fee of \$694, including taxes, to a
four-family building on Grand Avenue for \$1.2 million, with taxes of
\$14,286.

Image

416 78th STREET \textbar{} A four-bedroom, two-and-a-half-bathroom house
near Bergenline Avenue, built in 1916 on a 3,798-square-foot lot, listed
for \$579,000. 201-693-8158Credit...Chang W. Lee/The New York Times

\hypertarget{the-vibe}{%
\subsection{The Vibe}\label{the-vibe}}

Steven Barrera, 40, a chauffeur-relations manager at EmpireCLS, has been
living in North Bergen since he was 7. What makes the township special,
he said, is its diversity and sense of community.

Oh yes, and the setting. ``We're on Boulevard East, facing New York
City,'' he said of his two-bedroom apartment, for which he pays \$2,200
a month. ``The view that I have, I wish I could show you.''

Image

8012 Fourth Avenue \textbar{} A three-bedroom, one-and-a-half-bathroom
house in the Racetrack section, built in 1926 on a 3,598-square-foot
lot, listed for \$465,450. 732-874-3999Credit...Chang W. Lee/The New
York Times

\hypertarget{the-schools}{%
\subsection{The Schools}\label{the-schools}}

The North Bergen School District has six elementary schools that extend
from prekindergarten, kindergarten or first grade through eighth grade,
and one high school. Together, they enroll about 7,660 students.

On 2019 state tests, 24 percent of the students met standards in math,
versus 41 percent statewide; 44 percent met standards in English
language arts, versus 53 percent statewide.

The average 2019 SAT scores at North Bergen High School, which serves
about 2,260 students, were 482 in math and 495 in reading and writing.

In 2018, High Tech High School, a magnet school in North Bergen, moved
to Secaucus, N.J. There are plans to develop the old campus into a
middle school for seventh through ninth grades and to renovate North
Bergen High School's overcrowded building for grades 10 through 12. As
of February, the \$65 million project was scheduled to be completed in
September 2022.

Image

The Palisades, seen from River Road.Credit...Chang W. Lee/The New York
Times

\hypertarget{the-commute}{%
\subsection{The Commute}\label{the-commute}}

New Jersey Transit buses to Port Authority run along several major
north-south corridors; travel time is 10 minutes to an hour, depending
on traffic. Jitney commuter buses to Port Authority and the George
Washington Bridge bus terminal operate along Bergenline Avenue.
Hudson-Bergen Light Rail provides service to Bayonne, Jersey City,
Hoboken, Weehawken and Union City from its station on Tonnelle Avenue.
Ferry service to Midtown Manhattan is available at the Edgewater Ferry
Landing, at 989 River Road.

\hypertarget{the-history}{%
\subsection{The History}\label{the-history}}

In 1949, a 760-foot steel transmission tower was erected in the
residential neighborhood of Woodcliff for WWOR-TV, channel 9 ---~at the
time, It was one of the tallest man-made structures in the world.
Residents were not happy, particularly when it began to shed ice ``that
hurtled to the street for blocks in the area,'' according to The Jersey
Journal. After a twin-engine plane hit the tower in 1956, resulting in
six deaths, it was dismantled.

For weekly email updates on residential real estate news,
\href{http://www.nytimes.com/newsletters/realestate/}{sign up here}.
Follow us on Twitter:
\href{https://twitter.com/nytrealestate}{@nytrealestate}.

Advertisement

\protect\hyperlink{after-bottom}{Continue reading the main story}

\hypertarget{site-index}{%
\subsection{Site Index}\label{site-index}}

\hypertarget{site-information-navigation}{%
\subsection{Site Information
Navigation}\label{site-information-navigation}}

\begin{itemize}
\tightlist
\item
  \href{https://help.nytimes.com/hc/en-us/articles/115014792127-Copyright-notice}{©~2020~The
  New York Times Company}
\end{itemize}

\begin{itemize}
\tightlist
\item
  \href{https://www.nytco.com/}{NYTCo}
\item
  \href{https://help.nytimes.com/hc/en-us/articles/115015385887-Contact-Us}{Contact
  Us}
\item
  \href{https://www.nytco.com/careers/}{Work with us}
\item
  \href{https://nytmediakit.com/}{Advertise}
\item
  \href{http://www.tbrandstudio.com/}{T Brand Studio}
\item
  \href{https://www.nytimes.com/privacy/cookie-policy\#how-do-i-manage-trackers}{Your
  Ad Choices}
\item
  \href{https://www.nytimes.com/privacy}{Privacy}
\item
  \href{https://help.nytimes.com/hc/en-us/articles/115014893428-Terms-of-service}{Terms
  of Service}
\item
  \href{https://help.nytimes.com/hc/en-us/articles/115014893968-Terms-of-sale}{Terms
  of Sale}
\item
  \href{https://spiderbites.nytimes.com}{Site Map}
\item
  \href{https://help.nytimes.com/hc/en-us}{Help}
\item
  \href{https://www.nytimes.com/subscription?campaignId=37WXW}{Subscriptions}
\end{itemize}
