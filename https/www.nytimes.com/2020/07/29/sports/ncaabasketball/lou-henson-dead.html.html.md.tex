Sections

SEARCH

\protect\hyperlink{site-content}{Skip to
content}\protect\hyperlink{site-index}{Skip to site index}

\href{https://www.nytimes.com/section/sports/ncaabasketball}{College
Basketball}

\href{https://myaccount.nytimes.com/auth/login?response_type=cookie\&client_id=vi}{}

\href{https://www.nytimes.com/section/todayspaper}{Today's Paper}

\href{/section/sports/ncaabasketball}{College Basketball}\textbar{}Lou
Henson, Final Four Coach With Two Schools, Dies at 88

\url{https://nyti.ms/2Pdc8O1}

\begin{itemize}
\item
\item
\item
\item
\item
\end{itemize}

Advertisement

\protect\hyperlink{after-top}{Continue reading the main story}

Supported by

\protect\hyperlink{after-sponsor}{Continue reading the main story}

\hypertarget{lou-henson-final-four-coach-with-two-schools-dies-at-88}{%
\section{Lou Henson, Final Four Coach With Two Schools, Dies at
88}\label{lou-henson-final-four-coach-with-two-schools-dies-at-88}}

A genial ``old school'' leader, he took New Mexico State and then
Illinois deep into the N.C.A.A. tournament. In a 42-year career, he won
close to 800 games.

\includegraphics{https://static01.nyt.com/images/2020/07/30/obituaries/Henson-print/merlin_175074606_52a72aba-b669-44bc-b659-9ec43a7148bf-articleLarge.jpg?quality=75\&auto=webp\&disable=upscale}

By \href{https://www.nytimes.com/by/frank-litsky}{Frank Litsky}

\begin{itemize}
\item
  July 29, 2020
\item
  \begin{itemize}
  \item
  \item
  \item
  \item
  \item
  \end{itemize}
\end{itemize}

Lou Henson, who led New Mexico State and the University of Illinois to
the Final Four in a 42-year head-coaching career in which he piled up
almost 800 victories, died on Saturday at his home in Champaign, Ill. He
was 88.

Kent Brown, the associate director of athletics and media relations for
the Illinois athletic department, confirmed the death on Wednesday, the
same day the university said Henson was buried in a private ceremony.

Henson learned he had non-Hodgkin's lymphoma in 2003 but continued to
coach New Mexico State in the 2003-4 season. In September 2004 he was
hospitalized with viral encephalitis, which left his right leg
paralyzed, requiring him to use a wheelchair. In January 2005, two days
before he was scheduled to return to lead the Aggies, he was
hospitalized with pneumonia. Two weeks later, he retired.

``I have always been a very demanding coach,'' he said in announcing his
retirement. ``I expect my players to give 100 percent or they come out
of the game. I can expect no less of myself, so because I am physically
unable to give my all, I am taking myself out of the game.''

\includegraphics{https://static01.nyt.com/images/2020/07/29/obituaries/Henson2/merlin_11508898_988e2270-d06d-479f-84c1-a279bcb3dc30-articleLarge.jpg?quality=75\&auto=webp\&disable=upscale}

Henson coached four seasons (1962-66) at Hardin-Simmons University in
Texas, nine (1966-75) at New Mexico State and 21 (1975-96) at Illinois,
before returning to Las Cruces, N.M., to coach the Aggies for eight more
seasons (1997-2005).

He had retired from the Illinois job, in Champaign, when he returned to
New Mexico State to help rescue the basketball program. At the time,
only one of the team's players had graduated in five years, and the
N.C.A.A. had put the team on probation.

Henson took 19 teams to the N.C.A.A. tournament and four to the National
Invitation Tournament. He reached the N.C.A.A.'s Final Four with New
Mexico State in 1970 and with Illinois in 1989. A year after Illinois
got there, it received severe penalties from the N.C.A.A. for recruiting
violations.

\href{https://www.thechampaignroom.com/2018/10/25/18006802/university-of-illinois-fighting-illini-basketball-flyin-illini-30-year-anniversary-lou-henson}{The
Illinois team}, led by the future professionals Kenny Battle, Kendall
Gill, Nick Anderson and Stephen Bardo, racked up 31 wins in the 1988-89
season before losing to Michigan by two points in a Final Four semifinal
game at the Kingdome in Seattle. (Michigan went on to win the title,
beating Seton Hall.)

Henson's career coaching record was 779-413. When he retired, only four
coaches (Dean Smith, Adolph Rupp, Bob Knight and Jim Phelan) had won 800
games in Division I, and only Knight was still coaching. Today, Henson
is
\href{https://www.ncaa.com/news/basketball-men/article/2019-07-29/mens-di-college-basketball-coaches-most-wins\#:~:text=MORE\%20CBB\%20RECORDS\%3A\%20The\%209,Jim\%20Boeheim\%20is\%20946-385.}{ranked
15th} among coaches with the most career wins, 12 of whom have notched
more than 800 victories. (Mike Krzyzewski, who is still active at Duke,
has the most, 1,132.)

Henson was named to the
\href{https://collegebasketballexperience.com/members/lou-henson/}{National
Collegiate Basketball Hall of Fame} in St. Louis in 2015, the same year
the basketball court at the State Farm Center in Champaign was dedicated
in his honor.

Away from the court, Henson was genial and charming. On the court, he
was all business, stressing defense, two-handed passes and conservative
ball handling.

``He's an old-school guy, and he insisted on doing things soundly,''
Derek Harper, an Illinois player who became an outstanding professional,
told The Chicago Tribune in 1996. Eddie Johnson, another former player,
said of Henson: ``He never wavered in his demeanor. I think that's what
made him a great coach, because you always knew what to expect.''

Image

Henson during an Illinois practice in 1995. He coached at Illinois for
21 seasons and took the team to the Final Four in 1989.Credit...Mark
Cowan/Associated Press

Louis Raymond Henson was born Jan. 10, 1932, in the small town of Okay,
Okla., to Lora (Falconer) and Joe Henson, who were sharecroppers. One of
eight children, he grew up in a house with no indoor plumbing.

At New Mexico State, he earned a bachelor's degree in secondary
education in 1955 and a master's in educational administration in 1956.
A year later, he started his coaching career as the junior varsity coach
at Las Cruces High School. He married Mary Brantner in 1954.

His survivors include his wife and three daughters, Lori Henson, Lisa
Rutter and Leigh Anne Edison; a sister, Rosemary Yates; a brother, Ken;
12 grandchildren; and seven great-grandchildren. A son, Lou Jr., was the
basketball coach at Parkland College, a community college in Champaign,
when he was killed in an automobile accident in 1992.

After his retirement, Henson split his time between Las Cruces and
Champaign. A passionate bridge player, he participated in many local
tournaments. In his later years, as he dealt with his lymphoma, he
golfed and swam regularly.

Henson was proud of his players' post-collegiate achievements. He once
said, ``I have doctors, lawyers and business people all over the
country.''

He acknowledged the pressures of his job. ``It's not a normal
profession,'' he said. ``If you don't learn to live with it, you don't
coach.''

Through all the late-hour demands, he found time for his growing
daughters. Lori, the eldest, told The Chicago Tribune in 1995, ``We
lived our father's career.'' She added: ``It's part of your life. It
surrounds you.''

Leigh Anne, the youngest daughter, said that as a child she would draw
basketball plays and leave them on her father's desk.

``After the game,'' she said, ``he would say, `Did you see that play of
yours we used?' And I'd say yes. I believed it for the longest time.''

Frank Litsky, a longtime sportswriter for The Times, died in 2018. Julia
Carmel contributed reporting.

Advertisement

\protect\hyperlink{after-bottom}{Continue reading the main story}

\hypertarget{site-index}{%
\subsection{Site Index}\label{site-index}}

\hypertarget{site-information-navigation}{%
\subsection{Site Information
Navigation}\label{site-information-navigation}}

\begin{itemize}
\tightlist
\item
  \href{https://help.nytimes.com/hc/en-us/articles/115014792127-Copyright-notice}{©~2020~The
  New York Times Company}
\end{itemize}

\begin{itemize}
\tightlist
\item
  \href{https://www.nytco.com/}{NYTCo}
\item
  \href{https://help.nytimes.com/hc/en-us/articles/115015385887-Contact-Us}{Contact
  Us}
\item
  \href{https://www.nytco.com/careers/}{Work with us}
\item
  \href{https://nytmediakit.com/}{Advertise}
\item
  \href{http://www.tbrandstudio.com/}{T Brand Studio}
\item
  \href{https://www.nytimes.com/privacy/cookie-policy\#how-do-i-manage-trackers}{Your
  Ad Choices}
\item
  \href{https://www.nytimes.com/privacy}{Privacy}
\item
  \href{https://help.nytimes.com/hc/en-us/articles/115014893428-Terms-of-service}{Terms
  of Service}
\item
  \href{https://help.nytimes.com/hc/en-us/articles/115014893968-Terms-of-sale}{Terms
  of Sale}
\item
  \href{https://spiderbites.nytimes.com}{Site Map}
\item
  \href{https://help.nytimes.com/hc/en-us}{Help}
\item
  \href{https://www.nytimes.com/subscription?campaignId=37WXW}{Subscriptions}
\end{itemize}
