Sections

SEARCH

\protect\hyperlink{site-content}{Skip to
content}\protect\hyperlink{site-index}{Skip to site index}

\href{https://myaccount.nytimes.com/auth/login?response_type=cookie\&client_id=vi}{}

\href{https://www.nytimes.com/section/todayspaper}{Today's Paper}

A Food Writer's Sicilian Pasta Dish --- and Tips for Sharing It

\url{https://nyti.ms/2X8XADx}

\begin{itemize}
\item
\item
\item
\item
\item
\end{itemize}

\href{https://www.nytimes.com/spotlight/at-home?action=click\&pgtype=Article\&state=default\&region=TOP_BANNER\&context=at_home_menu}{At
Home}

\begin{itemize}
\tightlist
\item
  \href{https://www.nytimes.com/2020/07/28/books/time-for-a-literary-road-trip.html?action=click\&pgtype=Article\&state=default\&region=TOP_BANNER\&context=at_home_menu}{Take:
  A Literary Road Trip}
\item
  \href{https://www.nytimes.com/2020/07/29/magazine/bored-with-your-home-cooking-some-smoky-eggplant-will-fix-that.html?action=click\&pgtype=Article\&state=default\&region=TOP_BANNER\&context=at_home_menu}{Cook:
  Smoky Eggplant}
\item
  \href{https://www.nytimes.com/2020/07/27/travel/moose-michigan-isle-royale.html?action=click\&pgtype=Article\&state=default\&region=TOP_BANNER\&context=at_home_menu}{Look
  Out: For Moose}
\item
  \href{https://www.nytimes.com/interactive/2020/at-home/even-more-reporters-editors-diaries-lists-recommendations.html?action=click\&pgtype=Article\&state=default\&region=TOP_BANNER\&context=at_home_menu}{Explore:
  Reporters' Obsessions}
\end{itemize}

Advertisement

\protect\hyperlink{after-top}{Continue reading the main story}

Supported by

\protect\hyperlink{after-sponsor}{Continue reading the main story}

One Good Meal

\hypertarget{a-food-writers-sicilian-pasta-dish--and-tips-for-sharing-it}{%
\section{A Food Writer's Sicilian Pasta Dish --- and Tips for Sharing
It}\label{a-food-writers-sicilian-pasta-dish--and-tips-for-sharing-it}}

Skye McAlpine has made a name for herself serving bountiful meals to
large groups of friends. During lockdown, she's discovered the joy of
cooking for just one or two.

\includegraphics{https://static01.nyt.com/images/2020/07/28/t-magazine/27tmag-mcalpine-slide-7NGZ/27tmag-mcalpine-slide-7NGZ-articleLarge.jpg?quality=75\&auto=webp\&disable=upscale}

By \href{https://www.nytimes.com/by/isabel-wilkinson}{Isabel Wilkinson}

\begin{itemize}
\item
  July 29, 2020
\item
  \begin{itemize}
  \item
  \item
  \item
  \item
  \item
  \end{itemize}
\end{itemize}

\emph{In
``}\href{https://www.nytimes.com/column/one-good-meal}{\emph{One Good
Meal}}\emph{,'' we ask cooking-inclined creative people to share the
story behind a favorite dish they actually make and eat at home on a
regular basis --- and not just when they're trying to impress.}

Over the last few years, the British food writer and chef
\href{https://www.nytimes.com/2018/07/02/dining/table-in-venice-book-skye-mcalpine.html}{Skye
McAlpine} has built a loyal following with her unfussy dishes, inspired
by her upbringing in both England and Italy, which she serves in big,
mismatched platters at lively gatherings of friends. Or, as she puts it
in her new book, ``A Table for Friends,'' ``The kind of food you can
plonk down in the center of the table for everyone to tuck into,
towering platefuls of it.''

But then the pandemic hit and McAlpine found herself in quarantine in
London with **** far fewer people to cook for. While she wasn't
entertaining, though, making and presenting food remained a reliable
source of solace. ``Feeding people is such a great way of showing love
and care and putting happy energy out in the world,'' says McAlpine, who
still had her husband and two young sons for company. ``And it's
obviously great to be able to do that for 20, but it's equally great to
do that for supper for two. And, particularly in this period of
lockdown, it's even more important to show love and care for yourself.''
With more time to prepare meals, she tried to give each one a sense of
occasion, setting out ``proper napkins'' (as she describes any made from
cloth) and pulling out the eccentric china that she has collected over
the years from vintage stores, flea markets and eBay.

\includegraphics{https://static01.nyt.com/images/2020/07/28/t-magazine/27tmag-mcalpine-slide-GUB5/27tmag-mcalpine-slide-GUB5-articleLarge.jpg?quality=75\&auto=webp\&disable=upscale}

Among the dishes she's cooked most often is pasta chi vruocculi
arriminati, which a Sicilian friend had claimed for years was the ``best
pasta dish'' --- but which she had never tried herself until she and her
husband made it last year. ``We haven't turned back,'' she says with a
laugh. ``The trick is to use the same pan to cook both your cauliflower
and your pasta,'' McAlpine says, ``which imbues the pasta with extra
flavor and also saves on time washing up.'' And while you can make it
with romanesco instead of cauliflower or use a different pasta in place
of linguine, ``My one insistence,'' she says, ``is that you not skip the
bread crumbs at the end --- deliciously crisp, salty and golden, they're
just what the almost-sweet sauce needs.'' Below is McAlpine's version of
the recipe, as well as her tips for styling and presenting your food ---
even if you're sharing it with friends on Instagram, rather than in real
life.

Image

The ingredients for McAlpine's pasta chi vruocculi arriminati, clockwise
from right: olive oil, linguine, anchovy fillets, an~onion, pine nuts,
saffron strands, raisins, a cauliflower, and some stale bread, to make
bread crumbs.~Credit...Skye McAlpine

\hypertarget{skye-mcalpines-pasta-chi-vruocculi-arriminati}{%
\subsubsection{\texorpdfstring{\textbf{Skye McAlpine's} Pasta chi
Vruocculi
Arriminati}{Skye McAlpine's Pasta chi Vruocculi Arriminati}}\label{skye-mcalpines-pasta-chi-vruocculi-arriminati}}

\emph{Serves 4}

\begin{itemize}
\tightlist
\item
  1 whole cauliflower (roughly chopped into florets)
\end{itemize}

\begin{itemize}
\tightlist
\item
  2 ½ ounces pine nuts
\end{itemize}

\begin{itemize}
\tightlist
\item
  3 ounces stale bread
\end{itemize}

\begin{itemize}
\tightlist
\item
  3 tablespoons olive oil, plus extra to serve
\end{itemize}

\begin{itemize}
\tightlist
\item
  1 onion, chopped
\end{itemize}

\begin{itemize}
\tightlist
\item
  8 anchovy fillets
\end{itemize}

\begin{itemize}
\tightlist
\item
  2 ½ ounces raisins
\end{itemize}

\begin{itemize}
\tightlist
\item
  1 teaspoon saffron strands
\end{itemize}

\begin{itemize}
\tightlist
\item
  14 ounces linguine
\end{itemize}

1. Bring a large saucepan of generously salted water to boil. Add the
cauliflower florets to the water and turn the heat down to a gentle
simmer. Cook for 15-20 minutes, until the cauliflower can easily be cut
through with a butter knife.

2. While the cauliflower is cooking, toast the pine nuts in a
medium-size frying pan for 2-3 minutes over medium heat, giving the pan
an occasional shake, until they are golden brown. Set aside.

3. Tear the bread into chunks and blend in a food processor to make
coarse crumbs. Using the same pan you cooked the pine nuts in, heat 1
tablespoon of olive oil over medium heat and add the bread crumbs. Fry
gently, shaking the pan occasionally, for 4-5 minutes until they turn
crisp and golden, then take off the heat and set aside.

4. In a second, large frying pan, heat 2 tablespoons of olive oil over
medium heat, add the onion and a generous pinch of salt. Cook for 3-5
minutes, until the onion becomes soft and translucent. Add the anchovies
to the pan, and fry gently until they melt into the onions. Then add the
raisins and the toasted pine nuts. Stir and turn the heat to a simmer.

5. Use a pestle and mortar to grind the saffron and a pinch of salt into
a fine red powder. Scoop out a splash (roughly 1-2 tablespoons) of the
cooking water into a small cup; add the powdered saffron and set to one
side to infuse for a few minutes.

6. When the cauliflower is cooked, use a slotted spoon to scoop the
florets out of the water and toss them into the pan with the onion mix.
Save the cooking water. Pour the saffron-infused liquid over the
cauliflower, and stir, breaking up any large pieces of cauliflower with
a wooden spoon. Season with salt to taste.

7. Cook the pasta in the same water as the cauliflower (top it up with
fresh water if needed) until al dente*,* as per the instructions on the
packet.

8. When the pasta is cooked, scoop out half a cup of the cooking water
and set aside. Drain the pasta and toss it into the pan with the sauce
and the reserved cooking water, and stir together so the pasta is coated
in sauce.

9. Spoon the pasta chi vruocculi arriminati onto a large serving dish,
add a generous drizzle of olive oil and sprinkle the bread crumbs on
top. Eat immediately.

Image

McAlpine's new book about cooking and entertaining, ``A Table for
Friends,'' is out this week.~Credit...Skye McAlpine

\hypertarget{styling-tips}{%
\subsubsection{\texorpdfstring{\textbf{Styling
Tips}}{Styling Tips}}\label{styling-tips}}

\emph{The pandemic has inspired even the most reluctant among us to
become home cooks --- and document our efforts on Instagram. McAlpine,
who photographed all the images in her new book, offers tips to make
your food look camera-ready.}

\textbf{Don't Be Afraid of Portrait Mode}

To take a strong food photograph, McAlpine suggests either a colorful
tablecloth (like the
\href{https://theedition94.com/collections/just-in/products/canvas-mimi-vichy-table-cloth-large-145-x-350}{checked
linen canvas style shown above}) or a clean wooden or stone surface as a
backdrop. ``I love to use portrait mode on my iPhone (and ignore it when
it says I'm too far or too close to the subject),'' she says, which
creates a sharper, more professional look. She also advises that you
take the photo with your phone **** held parallel or at a 45-degree
angle to the table. And she's not afraid to stand on a chair to capture
a bird's-eye view.

\textbf{Go Wild With Plates}

``A pretty plate goes a long way toward making even the plainest food
look beautiful,'' McAlpine says. ``Painted, colored, plain, vintage
\ldots{} what works best on the table is really only a matter of
taste.'' (As if to prove the point, she recently released a
\href{https://www.anthropologie.com/brands/skye-mcalpine}{collection of
tableware} with Anthropologie that looks like the kind of well-loved
stuff you grandmother might have passed down to you.) Try using platters
and serving bowls in mismatched colors and patterns and, if you have
one, a cake stand can be surprisingly versatile (use it for sweets but
also quiches and tarts). The key, she says, is to ``mix heights, shapes
and textures wherever you can to create a bustling and abundant table
--- and have fun with it.''

Image

``The trick is to use the same pan to cook both your cauliflower and
your pasta,'' McAlpine says. ``This imbues the pasta with extra flavor
--- and also saves on time~washing up.''Credit...Skye McAlpine

\textbf{Think of Your Plate as a Canvas}

When considering what to serve or photograph, McAlpine always takes the
palette of her food **** into account. ``Color and texture, along with
taste, \emph{create} flavor,'' she writes in ``A Table for Friends.''
``However comforting and brown a meal might be --- and brown food tends
to be the most comforting of all --- it will always taste (and look)
best when paired with a pop of something fresh.'' She recommends
offsetting the warm yellows of pasta chi vruocculi arriminati, for
example, with a crisp green salad in the summer or a side of striking
purple radicchio in the fall. And she tries to avoid serving similarly
colored dishes together. Roast pork with a red tomato salad, she warns,
``feels a bit clashy.''

\textbf{Find a Window}

``Take photos in natural light,'' McAlpine advises. ``Food just looks
better that way. Otherwise it can take on a slightly yellow tinge.''
Once you're near a window or other natural light source, the most
important thing is not to overthink it. ``Keep things relaxed, simple
and genuine,'' says McAlpine. ``If it's a beautiful moment in real life,
that will shine through on camera, too.''

Advertisement

\protect\hyperlink{after-bottom}{Continue reading the main story}

\hypertarget{site-index}{%
\subsection{Site Index}\label{site-index}}

\hypertarget{site-information-navigation}{%
\subsection{Site Information
Navigation}\label{site-information-navigation}}

\begin{itemize}
\tightlist
\item
  \href{https://help.nytimes.com/hc/en-us/articles/115014792127-Copyright-notice}{©~2020~The
  New York Times Company}
\end{itemize}

\begin{itemize}
\tightlist
\item
  \href{https://www.nytco.com/}{NYTCo}
\item
  \href{https://help.nytimes.com/hc/en-us/articles/115015385887-Contact-Us}{Contact
  Us}
\item
  \href{https://www.nytco.com/careers/}{Work with us}
\item
  \href{https://nytmediakit.com/}{Advertise}
\item
  \href{http://www.tbrandstudio.com/}{T Brand Studio}
\item
  \href{https://www.nytimes.com/privacy/cookie-policy\#how-do-i-manage-trackers}{Your
  Ad Choices}
\item
  \href{https://www.nytimes.com/privacy}{Privacy}
\item
  \href{https://help.nytimes.com/hc/en-us/articles/115014893428-Terms-of-service}{Terms
  of Service}
\item
  \href{https://help.nytimes.com/hc/en-us/articles/115014893968-Terms-of-sale}{Terms
  of Sale}
\item
  \href{https://spiderbites.nytimes.com}{Site Map}
\item
  \href{https://help.nytimes.com/hc/en-us}{Help}
\item
  \href{https://www.nytimes.com/subscription?campaignId=37WXW}{Subscriptions}
\end{itemize}
