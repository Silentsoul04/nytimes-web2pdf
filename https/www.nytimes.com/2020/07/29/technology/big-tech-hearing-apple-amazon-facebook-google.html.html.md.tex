Sections

SEARCH

\protect\hyperlink{site-content}{Skip to
content}\protect\hyperlink{site-index}{Skip to site index}

\href{https://www.nytimes.com/section/technology}{Technology}

\href{https://myaccount.nytimes.com/auth/login?response_type=cookie\&client_id=vi}{}

\href{https://www.nytimes.com/section/todayspaper}{Today's Paper}

\href{/section/technology}{Technology}\textbar{}Lawmakers, United in
Their Ire, Lash Out at Big Tech's Leaders

\url{https://nyti.ms/3hNC8vL}

\begin{itemize}
\item
\item
\item
\item
\item
\end{itemize}

Advertisement

\protect\hyperlink{after-top}{Continue reading the main story}

Supported by

\protect\hyperlink{after-sponsor}{Continue reading the main story}

\hypertarget{lawmakers-united-in-their-ire-lash-out-at-big-techs-leaders}{%
\section{Lawmakers, United in Their Ire, Lash Out at Big Tech's
Leaders}\label{lawmakers-united-in-their-ire-lash-out-at-big-techs-leaders}}

The chiefs of Amazon, Apple, Google and Facebook faced withering
questions from Democrats about anti-competitive practices and from
Republicans about anti-conservative bias.

\includegraphics{https://static01.nyt.com/images/2020/07/30/reader-center/29TECHHEARING-A1/merlin_175077825_5ebc931b-baa1-489a-960c-34e4d845e997-articleLarge.jpg?quality=75\&auto=webp\&disable=upscale}

By \href{https://www.nytimes.com/by/cecilia-kang}{Cecilia Kang} and
\href{https://www.nytimes.com/by/david-mccabe}{David McCabe}

\begin{itemize}
\item
  Published July 29, 2020Updated July 31, 2020
\item
  \begin{itemize}
  \item
  \item
  \item
  \item
  \item
  \end{itemize}
\end{itemize}

WASHINGTON --- The chief executives of
\href{https://www.nytimes.com/2020/07/30/podcasts/the-daily/congress-facebook-amazon-google-apple.html}{Amazon,
Apple, Google and Facebook}, four tech giants worth nearly \$5 trillion
combined,
\href{https://www.nytimes.com/live/2020/07/29/technology/tech-ceos-hearing-testimony}{faced
withering questions}from Republican and Democratic lawmakers alike on
Wednesday for the tactics and market dominance that had made their
enterprises successful.

For more than five hours, the 15 members of an antitrust panel in the
House lobbed questions and repeatedly interrupted and talked over Jeff
Bezos of Amazon, Tim Cook of Apple, Mark Zuckerberg of Facebook and
Sundar Pichai of Google.

It was the first congressional hearing for some time where Democrats and
Republicans acted as if they had a common foe, though for different
reasons. Democratic lawmakers criticized the tech companies for buying
start-ups to stifle them and for unfairly using their data hoards to
clone and kill off competitors, while Republicans questioned whether the
platforms had muzzled conservative viewpoints and were unpatriotic.

``As gatekeepers to the digital economy, these platforms enjoy the power
to pick winners and losers, shake down small businesses and enrich
themselves while choking off competitors,'' said Representative David
Cicilline, Democrat of Rhode Island and chairman of the House Judiciary
Committee's antitrust subcommittee. ``Our founders would not bow before
a king. Nor should we bow before the emperors of the online economy.''

In response, Mr. Pichai, Mr. Zuckerberg, Mr. Cook and Mr. Bezos, who
testified via videoconference because of the coronavirus pandemic, were
forced to strike a more humble chord. They presented themselves as
participants in enormously competitive and fast-changing digital
marketplaces, and they evaded questions about the decisions that turned
their companies into giants.

``We approach this process with respect and humility, but we make no
concession on the facts,'' said Mr. Cook at the outset of his testimony.

Not since
\href{https://www.nytimes.com/1999/02/27/business/microsoft-rests-its-case-ending-on-a-misstep.html}{Microsoft
stood trial in the late 1990s} for antitrust charges have tech chief
executives been under such a microscope for the power of their
businesses. With echoes of the trustbusting of U.S. Steel and Standard
Oil more than a century ago and AT\&T in 1984, the hearing underlined
the government's recognition that this cohort of tech companies ---
which wield immense control over commerce, communications and public
discourse --- had become the new trusts of the internet age.

President Trump also used the event to rail against tech power. In a
\href{https://twitter.com/realDonaldTrump/status/1288506554585505793}{post
on Twitter} before the hearing began, he said that he would issue
executive orders to rein in the companies if Congress did not.

From its conception, the House antitrust hearing was set to be a
spectacle, lining up four of the world's most powerful executives ---
with two of them among the planet's richest individuals --- to answer
largely hostile questions together. While the joint appearance limited
sustained questioning of any one executive, it created a side-by-side
image that recalled the
\href{https://www.nytimes.com/1994/04/15/us/tobacco-chiefs-say-cigarettes-aren-t-addictive.html}{1994
congressional}hearing of top American tobacco executives, who said they
did not believe that cigarettes were addictive.

House lawmakers, who had opened an investigation into the tech companies
in June 2019, made the most of it. Representative Jerry Nadler, Democrat
of New York, confronted Mr. Zuckerberg with the C.E.O.'s own emails,
saying they showed a plot to take out a young competitor. Representative
Jim Jordan, Republican of Ohio, said Google was biased and asked Mr.
Pichai whether the company would change its products to help elect
Joseph R. Biden for president.

In one of the sharpest exchanges, Representative Pramila Jayapal, a
Washington Democrat, confronted Mr. Bezos on accusations that an Amazon
lawyer had lied to the committee about how the company develops its own
products. She asked him to answer whether it misused data with a yes or
no.

``I can't answer that question yes or no,'' said Mr. Bezos, appearing
rattled.

Yet while the hearing was ripe with theater, any impact will be limited
by antitrust laws that were created a century ago and that are
\href{https://www.nytimes.com/2018/09/07/technology/monopoly-antitrust-lina-khan-amazon.html}{imperfect
for corralling internet firms}. Since the 1980s, enforcement officials
have used the notion of consumer welfare as the predominant test for
antitrust violations --- generally meaning that if prices are not going
up, the markets are most likely competitive enough. The tech giants have
generally not driven up prices of digital services or consumer goods;
many do not charge at all for services like Google Maps or Instagram.

While Democrats at the hearing indicated they were more inclined to
change antitrust law, Representative Jim **** Sensenbrenner, Republican
of Wisconsin, said he did not think the laws needed to change. ****
Ultimately, Congress doesn't have the power to break up the companies.

Still, the proceedings provided fuel to other investigations of the tech
companies by the Justice Department, the Federal Trade Commission and
state attorneys general. The Justice Department is expected to
\href{https://www.nytimes.com/2020/06/25/technology/barr-google-investigation.html}{soon
announce charges against Google}accusing it of abusing its dominance in
online advertising, people with knowledge of the investigation have
said. The
\href{https://www.nytimes.com/2020/07/17/technology/ftc-facebook-investigation.html}{F.T.C.
is preparing to question Mr. Zuckerberg} under oath in its investigation
of Facebook's grip over social networking and acquisitions of nascent
rivals.

``This is a critical juncture in how the Washington policy clash with
the titans of Silicon Valley unfolds,'' said Gene Kimmelman, a former
Justice Department antitrust official and a special adviser to the
consumer advocacy group Public Knowledge.

Regulators around the world are also moving to limit the power of the
tech giants. Europe has led the charge with antitrust investigations and
Margrethe Vestager, the region's top trustbuster,
\href{https://www.nytimes.com/2019/11/19/technology/tech-regulator-europe.html}{recently
vowed to take a harder line} on the companies. On Wednesday,
\href{https://www.nytimes.com/2020/07/29/world/europe/turkey-social-media-control.html}{Turkey
passed legislation} giving its government sweeping new powers to
regulate social media content.

The hearing on Wednesday was a turnabout from just a few years ago, when
Facebook, Google, Amazon and Apple were emblems of national pride for
their innovation and growth. But the expanding reach of the four ---
which are involved in everything from smartphones to e-commerce to
digital payments --- and their stumbles
in\href{https://www.nytimes.com/2020/07/28/technology/virus-video-trump.html}{misinformation},
privacy,
\href{https://www.nytimes.com/2018/02/17/technology/indictment-russian-tech-facebook.html}{election
interference} and labor issues have increasingly raised hackles.

Even so, the companies have continued growing as more people live their
lives online, with all of them expected to post solid financial
performances when they report quarterly earnings on Thursday.

The hearing was made more bizarre by Mr. Bezos, Mr. Cook, Mr. Pichai and
Mr. Zuckerberg dialing in remotely using Cisco's Webex videoconferencing
service. Lawmakers --- who mostly appeared in person wearing masks in a
House hearing room --- faced empty chairs and a jumbo screen with the
faces of the executives, who looked soberly into their cameras.

Lawmakers nonetheless drilled down on key moments when the companies had
gained power or allegedly squeezed consumers, competitors and small
businesses. They directed most of their questions to Mr. Zuckerberg and
Mr. Pichai, then to Mr. Bezos, according
to\href{https://www.nytimes.com/live/2020/07/29/technology/tech-ceos-hearing-testimony/heres-which-tech-ceo-is-getting-asked-the-most-questions-by-lawmakers}{a
tally} by The New York Times. Mr. Cook was asked the fewest questions.

The tone of the hearing was set with Mr. Cicilline's very first
question, directed at Mr. Pichai. ``Why does Google steal content from
honest businesses?'' Mr. Cicilline asked. Mr. Pichai replied: ``Mr.
Chairman, with respect, I disagree with that characterization.''

Mr. Pichai was repeatedly asked about Google's dominance in search and
how the company was potentially trying to keep users within ``a walled
garden.'' He said Google had many competitors for specific categories of
search, such as shopping.

Mr. Zuckerberg was asked about Facebook emails where executives
discussed the company's 2012 acquisition of Instagram as a possible
strategy to take out a nascent competitor. Mr. Zuckerberg said that, in
fact, Instagram's success had never been guaranteed and was the result
of Facebook's investment in the product.

When lawmakers asked Mr. Bezos if Amazon had bullied small merchants, he
said that it was ``not how we operate the business'' --- before being
confronted by an audio recording of a bookseller begging him directly
for relief.

In response to questions about whether Apple favored some app developers
over others, Mr. Cook said there were ``open and transparent rules''
that applied ``evenly to everyone.''

David Heinemeier Hansson, the co-founder of Basecamp, a
project-management company that has battled with both Google and Apple
over their market power, said the hearing would be irrelevant if the
government did not act to rein in the tech giants.

``What we ultimately need is relief. We don't just need a historic
moment. We need this to lead to legislation and regulation and
enforcement,'' he said.

But, Mr. Heinemeier Hansson added, ``thankfully I've never been more
optimistic for that than I am right now.''

\includegraphics{https://static01.nyt.com/images/2017/01/29/podcasts/the-daily-album-art/the-daily-album-art-articleInline-v2.jpg?quality=75\&auto=webp\&disable=upscale}

\hypertarget{listen-to-the-daily-the-big-tech-hearing}{%
\subsubsection{Listen to `The Daily': The Big Tech
Hearing}\label{listen-to-the-daily-the-big-tech-hearing}}

A grilling on the power of digital giants in the internet age.

Reporting was contributed by Jack Nicas, Mike Isaac, Daisuke
Wakabayashi, Karen Weise and Kellen Browning.

Advertisement

\protect\hyperlink{after-bottom}{Continue reading the main story}

\hypertarget{site-index}{%
\subsection{Site Index}\label{site-index}}

\hypertarget{site-information-navigation}{%
\subsection{Site Information
Navigation}\label{site-information-navigation}}

\begin{itemize}
\tightlist
\item
  \href{https://help.nytimes.com/hc/en-us/articles/115014792127-Copyright-notice}{©~2020~The
  New York Times Company}
\end{itemize}

\begin{itemize}
\tightlist
\item
  \href{https://www.nytco.com/}{NYTCo}
\item
  \href{https://help.nytimes.com/hc/en-us/articles/115015385887-Contact-Us}{Contact
  Us}
\item
  \href{https://www.nytco.com/careers/}{Work with us}
\item
  \href{https://nytmediakit.com/}{Advertise}
\item
  \href{http://www.tbrandstudio.com/}{T Brand Studio}
\item
  \href{https://www.nytimes.com/privacy/cookie-policy\#how-do-i-manage-trackers}{Your
  Ad Choices}
\item
  \href{https://www.nytimes.com/privacy}{Privacy}
\item
  \href{https://help.nytimes.com/hc/en-us/articles/115014893428-Terms-of-service}{Terms
  of Service}
\item
  \href{https://help.nytimes.com/hc/en-us/articles/115014893968-Terms-of-sale}{Terms
  of Sale}
\item
  \href{https://spiderbites.nytimes.com}{Site Map}
\item
  \href{https://help.nytimes.com/hc/en-us}{Help}
\item
  \href{https://www.nytimes.com/subscription?campaignId=37WXW}{Subscriptions}
\end{itemize}
