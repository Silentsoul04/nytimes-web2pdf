Sections

SEARCH

\protect\hyperlink{site-content}{Skip to
content}\protect\hyperlink{site-index}{Skip to site index}

\href{https://www.nytimes.com/section/technology/personaltech}{Personal
Tech}

\href{https://myaccount.nytimes.com/auth/login?response_type=cookie\&client_id=vi}{}

\href{https://www.nytimes.com/section/todayspaper}{Today's Paper}

\href{/section/technology/personaltech}{Personal Tech}\textbar{}How to
Fight Against Big Tech's Power

\url{https://nyti.ms/3jQDGah}

\begin{itemize}
\item
\item
\item
\item
\item
\item
\end{itemize}

Advertisement

\protect\hyperlink{after-top}{Continue reading the main story}

Supported by

\protect\hyperlink{after-sponsor}{Continue reading the main story}

tech fix

\hypertarget{how-to-fight-against-big-techs-power}{%
\section{How to Fight Against Big Tech's
Power}\label{how-to-fight-against-big-techs-power}}

We are beholden to a few Big Tech overlords for much of our digital
lives. We can be more conscientious about it.

\includegraphics{https://static01.nyt.com/images/2020/07/30/business/29Techfix-illo/30Techfix-illo-articleLarge.gif?quality=75\&auto=webp\&disable=upscale}

\href{https://www.nytimes.com/by/brian-x-chen}{\includegraphics{https://static01.nyt.com/images/2018/02/16/multimedia/author-brian-x-chen/author-brian-x-chen-thumbLarge.jpg}}

By \href{https://www.nytimes.com/by/brian-x-chen}{Brian X. Chen}

\begin{itemize}
\item
  July 29, 2020
\item
  \begin{itemize}
  \item
  \item
  \item
  \item
  \item
  \item
  \end{itemize}
\end{itemize}

In the morning, you check email. At noon, you browse social media and
message friends. In the evening, you listen to music while shopping
online. Around bedtime, you curl up with an e-book.

For all of those activities, you probably used a product made or sold by
Google, Amazon, Apple or Facebook. There's no simple way to avoid those
Big Four. Even if you subscribed to Spotify, you would probably still be
using a Google Android phone, an Amazon speaker or an Apple iPhone to
stream the music. Even if you deleted Facebook, you might still be using
the Facebook-owned Instagram or WhatsApp.

Being beholden to a small set of companies that touch every corner of
our digital lives is precisely why lawmakers have summoned the chief
executives of
\href{https://www.nytimes.com/2020/07/28/technology/amazon-apple-facebook-google-antitrust-hearing.html}{Amazon,
Google, Facebook and Apple to testify in an antitrust hearing} on
Wednesday. Expect the tech titans to be grilled over whether their
companies have become so powerful and far-reaching that they harm rivals
and all of us, too.

So what can we do if we want to break out of the stranglehold of Big
Tech?

At first glance, there may not seem like much we can do to escape.
``It's not like you can start shopping at local bookstores and put
Amazon out of business,'' said Jason Fried, the founder of Basecamp, a
Chicago-based company that offers productivity apps.

But the more I thought about this, the more I realized that there were
some steps that we could take to better support tech's little guys, too.
We would do ourselves and smaller businesses a favor by staying informed
on alternatives, for one. We could change our consumption patterns so
that we were not just buying new products from the tech giants. And we
could show our support for indie developers who make the apps we love.

As Mr. Fried put it, ``We can do things to change our own conscience.''
Here's how.

\hypertarget{when-possible-find-alternatives}{%
\subsection{When possible, find
alternatives}\label{when-possible-find-alternatives}}

Step One to becoming a more conscientious consumer is doing some
research.

While Google Chrome may be the most popular web browser, there are
alternatives that collect less data about us. And while all of our
friends are on Facebook, there are also smaller apps or methods we can
use to stay connected with them. The key is to read news sites and tech
blogs to learn about options.

``You have to read and be informed,'' said Don Heider, chief executive
of the Markkula Center for Applied Ethics at Santa Clara University.
``Otherwise, you're not going to have a clue of where to go and what to
pick and what the impact is.''

Mr. Heider pointed to a few examples: Instead of Google Chrome, people
can download great browsers, including
\href{https://www.nytimes.com/2019/07/15/technology/duckduckgo-private-search.html}{DuckDuckGo},
\href{https://brave.com/}{Brave} and
\href{https://www.opera.com/}{Opera}, which focus on stronger privacy
and security protections. Instead of Facebook, we can tell our friends
to hang out with us on social media apps like
\href{https://vero.co/}{Vero} and
\href{https://joinmastodon.org/}{Mastodon}, which are both ad-free, he
said.

The same goes for Amazon. Instead of ordering paper towels and hand
sanitizer on Amazon, consider picking up those items at a local store.
Instead of ordering a new dog collar on Amazon, consider buying a
custom-made one from an independent merchant on Etsy.

Mr. Fried says he rarely shops on Amazon, takes cabs instead of Ubers
and finds books via \href{https://www.indiebound.org/}{IndieBound}, a
resource for buying titles from local bookstores. ``When the default is
just Amazon, Amazon, Amazon, you're just feeding the flame,'' he said.

\hypertarget{why-buy-new-buy-used}{%
\subsection{Why buy new? Buy used}\label{why-buy-new-buy-used}}

Speaking of alternatives, there's a different way to buy tech hardware
altogether: Purchase gadgets used or refurbished.

When you buy a new phone or computer, your dollars go directly to the
tech giants who created the products. But when you buy used, you are
supporting a broader community of small businesses that repair and
resell equipment.

Many of us generally shy away from used electronics because we fear the
products may be in shoddy condition. The reality is that resellers work
with technicians who restore products to their former glory before
putting them up on sale --- and the gadgets are often backed by a
warranty.
\href{https://www.nytimes.com/2016/04/28/technology/personaltech/taking-the-stigma-out-of-buying-usedelectronics.html}{Reputable
vendors of used goods} include GameStop and Gazelle.

Buying used also contributes to a broader mission: the so-called
\href{https://www.ifixit.com/Right-to-Repair/Intro}{right to repair
movement}.

Unlike car mechanics, small electronics repair shops have limited access
to the parts and instructions that they need to service our smartphones,
tablets and computers. Public advocacy groups and the repair community
have
\href{https://uspirg.org/blogs/blog/usp/right-repair-wraps-big-year}{pushed
to pass legislation} that would require electronics manufacturers to
share all of the components and information needed to fix our gadgets.

If more people opt to buy used or refurbished goods, that will show that
there is demand for repaired products. That, in turn, puts pressure on
manufacturers to make repair more accessible to independent technicians
and consumers, said Carole Mars, the director of technical development
and innovation at
the\href{https://www.sustainabilityconsortium.org/}{Sustainability
Consortium}, which studies the sustainability of consumer goods.

``It comes down to accepting refurbished and demanding refurbished,''
Dr. Mars said. ``That will lead you to ask, `Why can't I get this
product used or fixed?' It's because the company locked it down.''

So try to make this a habit: Whenever you are shopping for an electronic
online, check if there is a used or refurbished option. If there is one
in good condition, go for it and save some bucks.

\hypertarget{support-indie-developers}{%
\subsection{Support indie developers}\label{support-indie-developers}}

A lot of what we do with our devices is made possible by smaller
companies that produce our apps and games. One way to show our support
to David rather than Goliath is to have some patience and empathy for
the indie developers.

People often get frustrated when an app or game they love gets a big
software update and charges another \$3 to \$10 for the new version, for
example. Try not to get irritated --- these are small outfits trying to
survive, not big corporations trying to milk you --- and be willing to
pay. It's the same amount of money as a cup of coffee or a sandwich, and
you're polishing a piece of software that you love.

``If you can pay for software that you like,'' said Brianna Wu, a game
developer, ``you probably have an ethical responsibility to do so in the
same way that you'd have the ethical responsibility to tip a waitress.
The reality is that most of the time when you play an indie video game,
that group of people have bet their entire company's future on you
paying for it.''

Keep in mind also that small app developers lack the huge marketing
budgets of our tech overlords. They rely largely on all of us to do
grass-roots marketing in the form of written reviews or word of mouth,
said David Barnard, founder of the app studio Contrast. So when you love
an app, tell your friends about it.

I'll close with an example: My favorite piece of indie software for the
Mac is Fantastical, a calendar app, which does a better, more reliable
job organizing my online calendars than Apple's calendar app.

It was an expensive calendar app --- \$50 --- but it's kept me punctual,
which makes it worth every penny.

Advertisement

\protect\hyperlink{after-bottom}{Continue reading the main story}

\hypertarget{site-index}{%
\subsection{Site Index}\label{site-index}}

\hypertarget{site-information-navigation}{%
\subsection{Site Information
Navigation}\label{site-information-navigation}}

\begin{itemize}
\tightlist
\item
  \href{https://help.nytimes.com/hc/en-us/articles/115014792127-Copyright-notice}{©~2020~The
  New York Times Company}
\end{itemize}

\begin{itemize}
\tightlist
\item
  \href{https://www.nytco.com/}{NYTCo}
\item
  \href{https://help.nytimes.com/hc/en-us/articles/115015385887-Contact-Us}{Contact
  Us}
\item
  \href{https://www.nytco.com/careers/}{Work with us}
\item
  \href{https://nytmediakit.com/}{Advertise}
\item
  \href{http://www.tbrandstudio.com/}{T Brand Studio}
\item
  \href{https://www.nytimes.com/privacy/cookie-policy\#how-do-i-manage-trackers}{Your
  Ad Choices}
\item
  \href{https://www.nytimes.com/privacy}{Privacy}
\item
  \href{https://help.nytimes.com/hc/en-us/articles/115014893428-Terms-of-service}{Terms
  of Service}
\item
  \href{https://help.nytimes.com/hc/en-us/articles/115014893968-Terms-of-sale}{Terms
  of Sale}
\item
  \href{https://spiderbites.nytimes.com}{Site Map}
\item
  \href{https://help.nytimes.com/hc/en-us}{Help}
\item
  \href{https://www.nytimes.com/subscription?campaignId=37WXW}{Subscriptions}
\end{itemize}
