Sections

SEARCH

\protect\hyperlink{site-content}{Skip to
content}\protect\hyperlink{site-index}{Skip to site index}

\href{https://www.nytimes.com/section/well/move}{Move}

\href{https://myaccount.nytimes.com/auth/login?response_type=cookie\&client_id=vi}{}

\href{https://www.nytimes.com/section/todayspaper}{Today's Paper}

\href{/section/well/move}{Move}\textbar{}Is Your Blood Sugar Undermining
Your Workouts?

\url{https://nyti.ms/3fhnUBX}

\begin{itemize}
\item
\item
\item
\item
\item
\item
\end{itemize}

Advertisement

\protect\hyperlink{after-top}{Continue reading the main story}

Supported by

\protect\hyperlink{after-sponsor}{Continue reading the main story}

Phys Ed

\hypertarget{is-your-blood-sugar-undermining-your-workouts}{%
\section{Is Your Blood Sugar Undermining Your
Workouts?}\label{is-your-blood-sugar-undermining-your-workouts}}

Eating a diet high in sugar and processed foods could dent our long-term
health in part by changing how well our bodies respond to exercise.

\includegraphics{https://static01.nyt.com/images/2020/08/04/well/physed-runner/merlin_170719374_ca7d8b99-4c72-4542-aa89-5fa3012f069e-articleLarge.jpg?quality=75\&auto=webp\&disable=upscale}

\href{https://www.nytimes.com/by/gretchen-reynolds}{\includegraphics{https://static01.nyt.com/images/2019/03/18/multimedia/author-gretchen-reynolds/author-gretchen-reynolds-thumbLarge.png}}

By \href{https://www.nytimes.com/by/gretchen-reynolds}{Gretchen
Reynolds}

\begin{itemize}
\item
  July 29, 2020
\item
  \begin{itemize}
  \item
  \item
  \item
  \item
  \item
  \item
  \end{itemize}
\end{itemize}

People with consistently high levels of blood sugar could get less
benefit from exercise than those whose blood sugar levels are normal,
according to a cautionary new study of nutrition, blood sugar and
exercise. The study, which involved rodents and people, suggests that
eating a diet high in sugar and processed foods, which may set the stage
for poor blood sugar control, could dent our long-term health in part by
changing how well our bodies respond to a workout.

We already have plenty of evidence, of course, that elevated blood sugar
is unhealthy. People with hyperglycemia tend to be overweight and face
greater long-term risks for heart disease and Type 2 diabetes, even if,
in the early stages, their condition does not meet the criteria for
those diseases.

They also tend to be out of shape. In epidemiological studies, people
with elevated blood sugar often also have low aerobic fitness, while, in
animal studies, rats bred with low endurance from birth show early
blood-sugar problems, as well. This interrelationship between blood
sugar and fitness is consequential in part because low aerobic fitness
is closely linked to a high risk of premature death.

But most past studies of blood sugar and fitness have been
epidemiological, meaning they have identified links between the two
conditions but not their sequence or mechanisms. They have not clarified
whether hyperglycemia usually precedes and leads to low fitness, or the
other way around, or how either condition manages to influence the
other.

So, for the \href{https://www.nature.com/articles/s42255-020-0240-7}{new
study, which was published this month in Nature Metabolism,} researchers
at the Joslin Diabetes Center in Boston and other institutions decided
to raise blood sugar levels in mice and see what happened when they
exercised.

They started with adult mice, switching some from normal chow to a diet
high in sugar and saturated fat, similar to what many of us in the
developed world eat nowadays. These mice rapidly gained weight and
developed habitually high blood sugar.

They injected other mice with a substance that reduces their ability to
produce insulin, a hormone that helps to control blood sugar, similar to
when people have certain forms of diabetes. Those animals did not get
fatter, but their blood sugar levels rose to the same extent as among
the mice in the sugary diet group.

Other animals remained on their normal chow, as a control group.

After four months, the scientists checked each mouse's fitness by
measuring how long it could run on a treadmill before exhaustion. They
then put a running wheel in each animal's cage and let them jog at will
for the next six weeks, which they did. On average, each mouse ran about
300 miles during that month and a half.

But they did not all gain the same level of fitness. The control group
now ran for a much longer period of time on the treadmill before
exhaustion; they were much fitter. But the animals with high blood sugar
showed little improvement. Their aerobic fitness had barely budged.

Interestingly, their exercise resistance was the same, whether their
blood sugar problems stemmed from poor diet or lack of insulin, and
whether they were overweight or slimmer. If they had high blood sugar,
they resisted the benefits of exercise.

To better understand why, the scientists next looked inside muscles. And
conditions there were telling. The muscles of the control animals teemed
with healthy, new muscle fibers and a network of new blood vessels
ferrying extra oxygen and fuel to them. But the muscle tissues of the
animals with high blood sugar displayed mostly new deposits of collagen,
a rigid substance that seems to have crowded out new blood vessels and
prevented the muscles from adapting to the exercise and contributing to
better fitness.

Finally, since rodents are not people, the scientists checked blood
sugar levels and endurance in a group of 24 young adults. None had
diabetes, although some had blood-sugar levels that could be considered
prediabetic. During treadmill fitness testing, those volunteers with the
worst blood-sugar control also had the lowest endurance, and when the
scientists later microscopically examined their muscle tissues after the
exercise, they found high activation of proteins that can inhibit
improvements to aerobic fitness.

Taken as a whole, these results in mice and people suggest that
``constantly bathing your tissues in sugar is just not a good idea'' and
could undercut any subsequent benefits from exercise, says Sarah
Lessard, an assistant professor at the Joslin Diabetes Center and
Harvard Medical School, who oversaw the new study.

In practical terms, the findings suggest that, for those of us whose
blood-sugar levels depend on our diets, we might want to ``cut back on
sugar'' and the highly processed, fatty foods that also can raise blood
sugar and blunt exercise effects, she says. (The control mice ate a
high-carbohydrate chow, so carbohydrates, per se, are not necessarily
the issue, she says; diet quality is.)

More fundamentally, the study intimates that ``diet and exercise should
be considered together'' when we start thinking about how to improve our
health, Dr. Lessard says. They affect each other and they influence how
each affects us more than we might expect, she says.

But perhaps most important, the study contains some encouraging data,
Dr. Lessard points out. The hyperglycemic mice gained little endurance
from their weeks of working out, but they were beginning to show early
signs of better blood-sugar control, she says. So, it might require time
and gritty determination, but exercise eventually could help people with
hyperglycemia to stabilize their blood sugar, she says, and then start
feeling their fitness rise.

Advertisement

\protect\hyperlink{after-bottom}{Continue reading the main story}

\hypertarget{site-index}{%
\subsection{Site Index}\label{site-index}}

\hypertarget{site-information-navigation}{%
\subsection{Site Information
Navigation}\label{site-information-navigation}}

\begin{itemize}
\tightlist
\item
  \href{https://help.nytimes.com/hc/en-us/articles/115014792127-Copyright-notice}{©~2020~The
  New York Times Company}
\end{itemize}

\begin{itemize}
\tightlist
\item
  \href{https://www.nytco.com/}{NYTCo}
\item
  \href{https://help.nytimes.com/hc/en-us/articles/115015385887-Contact-Us}{Contact
  Us}
\item
  \href{https://www.nytco.com/careers/}{Work with us}
\item
  \href{https://nytmediakit.com/}{Advertise}
\item
  \href{http://www.tbrandstudio.com/}{T Brand Studio}
\item
  \href{https://www.nytimes.com/privacy/cookie-policy\#how-do-i-manage-trackers}{Your
  Ad Choices}
\item
  \href{https://www.nytimes.com/privacy}{Privacy}
\item
  \href{https://help.nytimes.com/hc/en-us/articles/115014893428-Terms-of-service}{Terms
  of Service}
\item
  \href{https://help.nytimes.com/hc/en-us/articles/115014893968-Terms-of-sale}{Terms
  of Sale}
\item
  \href{https://spiderbites.nytimes.com}{Site Map}
\item
  \href{https://help.nytimes.com/hc/en-us}{Help}
\item
  \href{https://www.nytimes.com/subscription?campaignId=37WXW}{Subscriptions}
\end{itemize}
