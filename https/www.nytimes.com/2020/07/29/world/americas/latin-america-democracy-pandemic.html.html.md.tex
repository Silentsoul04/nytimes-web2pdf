Sections

SEARCH

\protect\hyperlink{site-content}{Skip to
content}\protect\hyperlink{site-index}{Skip to site index}

\href{https://www.nytimes.com/section/world/americas}{Americas}

\href{https://myaccount.nytimes.com/auth/login?response_type=cookie\&client_id=vi}{}

\href{https://www.nytimes.com/section/todayspaper}{Today's Paper}

\href{/section/world/americas}{Americas}\textbar{}Latin America Is
Facing a `Decline of Democracy' Under the Pandemic

\url{https://nyti.ms/2P1jbtk}

\begin{itemize}
\item
\item
\item
\item
\item
\item
\end{itemize}

\href{https://www.nytimes.com/news-event/coronavirus?action=click\&pgtype=Article\&state=default\&region=TOP_BANNER\&context=storylines_menu}{The
Coronavirus Outbreak}

\begin{itemize}
\tightlist
\item
  live\href{https://www.nytimes.com/2020/08/01/world/coronavirus-covid-19.html?action=click\&pgtype=Article\&state=default\&region=TOP_BANNER\&context=storylines_menu}{Latest
  Updates}
\item
  \href{https://www.nytimes.com/interactive/2020/us/coronavirus-us-cases.html?action=click\&pgtype=Article\&state=default\&region=TOP_BANNER\&context=storylines_menu}{Maps
  and Cases}
\item
  \href{https://www.nytimes.com/interactive/2020/science/coronavirus-vaccine-tracker.html?action=click\&pgtype=Article\&state=default\&region=TOP_BANNER\&context=storylines_menu}{Vaccine
  Tracker}
\item
  \href{https://www.nytimes.com/interactive/2020/07/29/us/schools-reopening-coronavirus.html?action=click\&pgtype=Article\&state=default\&region=TOP_BANNER\&context=storylines_menu}{What
  School May Look Like}
\item
  \href{https://www.nytimes.com/live/2020/07/31/business/stock-market-today-coronavirus?action=click\&pgtype=Article\&state=default\&region=TOP_BANNER\&context=storylines_menu}{Economy}
\end{itemize}

Advertisement

\protect\hyperlink{after-top}{Continue reading the main story}

Supported by

\protect\hyperlink{after-sponsor}{Continue reading the main story}

\hypertarget{latin-america-is-facing-a-decline-of-democracy-under-the-pandemic}{%
\section{Latin America Is Facing a `Decline of Democracy' Under the
Pandemic}\label{latin-america-is-facing-a-decline-of-democracy-under-the-pandemic}}

The coronavirus is battering Latin American health systems and
economies. It is also threatening the region's fragile political
freedoms.

\includegraphics{https://static01.nyt.com/images/2020/07/16/world/00latam-top/00latam-top-articleLarge-v2.jpg?quality=75\&auto=webp\&disable=upscale}

By Anatoly Kurmanaev

\begin{itemize}
\item
  July 29, 2020
\item
  \begin{itemize}
  \item
  \item
  \item
  \item
  \item
  \item
  \end{itemize}
\end{itemize}

\href{https://www.nytimes.com/es/2020/07/29/espanol/america-latina/democracia-america-latina-pandemia.html}{Leer
en español}

CARACAS, Venezuela --- Postponed elections. Sidelined courts. A
persecuted opposition.

As the coronavirus pandemic tears through Latin America and the
Caribbean, killing more than 180,000 and destroying the livelihoods of
tens of millions in the region, it is also undermining democratic norms
that were already under strain.

Leaders ranging from the center-right to the far left have used the
crisis as justification to extend their time in office, weaken oversight
of government actions and silence critics --- actions that under
different circumstances would be described as authoritarian and
antidemocratic but that now are being billed as lifesaving measures to
curb the spread of the disease.

The gradual undermining of democratic rules during an economic crisis
and public health catastrophe could leave Latin America primed for
slower growth and an increase in corruption and human rights abuses,
experts warned. This is particularly true in places where political
rights and accountability were already in steep decline.

``It's not a matter of left or right, it's a general decline of
democracy across the region,'' said Alessandra Pinna, a Latin America
researcher at Freedom House, an independent Washington-based research
organization that measures global political liberties.

There are now five Latin American and Caribbean nations with recent
democratic histories --- Venezuela, Nicaragua, Guyana, Bolivia and Haiti
--- where governments weren't chosen in free and fair elections or have
overstayed their time in office. It's the highest number since the late
1980s, when the Cold War receded and several countries in the grips of
civil war or military dictatorships transitioned to peace and democracy.

\includegraphics{https://static01.nyt.com/images/2020/07/16/world/00latam-bolivia1/merlin_164809362_1399db40-307c-4b40-a421-c071e0f7a183-articleLarge.jpg?quality=75\&auto=webp\&disable=upscale}

Most of these leaders were already bending the rules of democracy to
stay in power before the pandemic, but seized on emergency conditions
created by the spread of the virus to strengthen their position.

President Nicolás Maduro of Venezuela has detained or raided the homes
of dozens of journalists, social activists and opposition leaders for
questioning the government's dubious coronavirus figures.

In Nicaragua, President Daniel Ortega released thousands of inmates
because of the threat posed by the virus, but
\href{https://www.barrons.com/news/nicaragua-excludes-political-prisoners-from-mass-release-01586430304}{kept
political prisoners} behind bars, while in Guyana, a lockdown prevented
protests against the government's attempt to stay in power despite
losing an election.

\hypertarget{latest-updates-global-coronavirus-outbreak}{%
\section{\texorpdfstring{\href{https://www.nytimes.com/2020/08/01/world/coronavirus-covid-19.html?action=click\&pgtype=Article\&state=default\&region=MAIN_CONTENT_1\&context=storylines_live_updates}{Latest
Updates: Global Coronavirus
Outbreak}}{Latest Updates: Global Coronavirus Outbreak}}\label{latest-updates-global-coronavirus-outbreak}}

Updated 2020-08-02T06:58:18.835Z

\begin{itemize}
\tightlist
\item
  \href{https://www.nytimes.com/2020/08/01/world/coronavirus-covid-19.html?action=click\&pgtype=Article\&state=default\&region=MAIN_CONTENT_1\&context=storylines_live_updates\#link-34047410}{The
  U.S. reels as July cases more than double the total of any other
  month.}
\item
  \href{https://www.nytimes.com/2020/08/01/world/coronavirus-covid-19.html?action=click\&pgtype=Article\&state=default\&region=MAIN_CONTENT_1\&context=storylines_live_updates\#link-780ec966}{Top
  U.S. officials work to break an impasse over the federal jobless
  benefit.}
\item
  \href{https://www.nytimes.com/2020/08/01/world/coronavirus-covid-19.html?action=click\&pgtype=Article\&state=default\&region=MAIN_CONTENT_1\&context=storylines_live_updates\#link-2bc8948}{Its
  outbreak untamed, Melbourne goes into even greater lockdown.}
\end{itemize}

\href{https://www.nytimes.com/2020/08/01/world/coronavirus-covid-19.html?action=click\&pgtype=Article\&state=default\&region=MAIN_CONTENT_1\&context=storylines_live_updates}{See
more updates}

More live coverage:
\href{https://www.nytimes.com/live/2020/07/31/business/stock-market-today-coronavirus?action=click\&pgtype=Article\&state=default\&region=MAIN_CONTENT_1\&context=storylines_live_updates}{Markets}

In Bolivia, a caretaker government has used the pandemic to postpone
elections, tap into emergency aid to bolster its electoral campaign and
threaten to ban the main opposition candidate from running.

And in the islands of St. Kitts and Nevis, the government imposed a
strict lockdown on its 50,000 people during the campaign for general
elections in June, hampering opposition efforts to meet voters while
also keeping international election observers from traveling to the
country.

It was the first time that the Organization of American States, a
regional group that promotes democracy, had its invitation to observe
elections withdrawn by a host country in recent history.

Image

Ms. Añez's government has postponed elections in Bolivia because of the
pandemic.Credit...Federico Rios for The New York Times

The loss of public trust in Latin America is not new, but the erosion of
democratic norms in the pandemic arrived at a time when the region's
economic growth and social progress were already unraveling, leaving
many uncertain about the ability of democratic leaders to solve
entrenched problems such as inequality, crime and corruption.

By 2018, only one in four Latin Americans said they were satisfied with
democracy --- the lowest number since Latinobarómetro, a regional
polling company, began asking that question 25 years ago.

Discontent with the political establishment led to a wave of populist
victories in recent years, including President Jair Bolsonaro of Brazil,
who is on the far right, and President Andrés Manuel López Obrador of
Mexico, who is on the left. It also led to mass street protests in
several Latin American countries last year.

The pandemic, hitting during this time of political upheaval, has
plunged the region into the deepest recession in its history,
exacerbating weaknesses in health and welfare systems and highlighting
the ways in which many leaders are unable to meet public demands.

``All the things that Latin Americans have already been clamoring for
--- greater equality, better services --- have been dramatically
worsened by the pandemic,'' said Cynthia Arnson, Latin America program
director at the Wilson Center, a think-tank in Washington. ``The
economic pain is dramatic, and it's putting additional strain on the
already-weak institutions.''

Image

Masks for sale on the outskirts of Guatemala City. The pandemic has
ripped through the region, leaving more than 180,000
dead.Credit...Daniele Volpe for The New York Times

It has also put a strain on the region's struggling health care systems.
Latin America has become a global hot spot for the virus, with Brazil,
Mexico and Peru among the 10 nations with the highest number of deaths.
And according to the United Nations, about 16 million Latin Americans
are expected to fall into extreme poverty this year, reversing nearly
all the gains made by the region this century.

\href{https://www.nytimes.com/news-event/coronavirus?action=click\&pgtype=Article\&state=default\&region=MAIN_CONTENT_3\&context=storylines_faq}{}

\hypertarget{the-coronavirus-outbreak-}{%
\subsubsection{The Coronavirus Outbreak
›}\label{the-coronavirus-outbreak-}}

\hypertarget{frequently-asked-questions}{%
\paragraph{Frequently Asked
Questions}\label{frequently-asked-questions}}

Updated July 27, 2020

\begin{itemize}
\item ~
  \hypertarget{should-i-refinance-my-mortgage}{%
  \paragraph{Should I refinance my
  mortgage?}\label{should-i-refinance-my-mortgage}}

  \begin{itemize}
  \tightlist
  \item
    \href{https://www.nytimes.com/article/coronavirus-money-unemployment.html?action=click\&pgtype=Article\&state=default\&region=MAIN_CONTENT_3\&context=storylines_faq}{It
    could be a good idea,} because mortgage rates have
    \href{https://www.nytimes.com/2020/07/16/business/mortgage-rates-below-3-percent.html?action=click\&pgtype=Article\&state=default\&region=MAIN_CONTENT_3\&context=storylines_faq}{never
    been lower.} Refinancing requests have pushed mortgage applications
    to some of the highest levels since 2008, so be prepared to get in
    line. But defaults are also up, so if you're thinking about buying a
    home, be aware that some lenders have tightened their standards.
  \end{itemize}
\item ~
  \hypertarget{what-is-school-going-to-look-like-in-september}{%
  \paragraph{What is school going to look like in
  September?}\label{what-is-school-going-to-look-like-in-september}}

  \begin{itemize}
  \tightlist
  \item
    It is unlikely that many schools will return to a normal schedule
    this fall, requiring the grind of
    \href{https://www.nytimes.com/2020/06/05/us/coronavirus-education-lost-learning.html?action=click\&pgtype=Article\&state=default\&region=MAIN_CONTENT_3\&context=storylines_faq}{online
    learning},
    \href{https://www.nytimes.com/2020/05/29/us/coronavirus-child-care-centers.html?action=click\&pgtype=Article\&state=default\&region=MAIN_CONTENT_3\&context=storylines_faq}{makeshift
    child care} and
    \href{https://www.nytimes.com/2020/06/03/business/economy/coronavirus-working-women.html?action=click\&pgtype=Article\&state=default\&region=MAIN_CONTENT_3\&context=storylines_faq}{stunted
    workdays} to continue. California's two largest public school
    districts --- Los Angeles and San Diego --- said on July 13, that
    \href{https://www.nytimes.com/2020/07/13/us/lausd-san-diego-school-reopening.html?action=click\&pgtype=Article\&state=default\&region=MAIN_CONTENT_3\&context=storylines_faq}{instruction
    will be remote-only in the fall}, citing concerns that surging
    coronavirus infections in their areas pose too dire a risk for
    students and teachers. Together, the two districts enroll some
    825,000 students. They are the largest in the country so far to
    abandon plans for even a partial physical return to classrooms when
    they reopen in August. For other districts, the solution won't be an
    all-or-nothing approach.
    \href{https://bioethics.jhu.edu/research-and-outreach/projects/eschool-initiative/school-policy-tracker/}{Many
    systems}, including the nation's largest, New York City, are
    devising
    \href{https://www.nytimes.com/2020/06/26/us/coronavirus-schools-reopen-fall.html?action=click\&pgtype=Article\&state=default\&region=MAIN_CONTENT_3\&context=storylines_faq}{hybrid
    plans} that involve spending some days in classrooms and other days
    online. There's no national policy on this yet, so check with your
    municipal school system regularly to see what is happening in your
    community.
  \end{itemize}
\item ~
  \hypertarget{is-the-coronavirus-airborne}{%
  \paragraph{Is the coronavirus
  airborne?}\label{is-the-coronavirus-airborne}}

  \begin{itemize}
  \tightlist
  \item
    The coronavirus
    \href{https://www.nytimes.com/2020/07/04/health/239-experts-with-one-big-claim-the-coronavirus-is-airborne.html?action=click\&pgtype=Article\&state=default\&region=MAIN_CONTENT_3\&context=storylines_faq}{can
    stay aloft for hours in tiny droplets in stagnant air}, infecting
    people as they inhale, mounting scientific evidence suggests. This
    risk is highest in crowded indoor spaces with poor ventilation, and
    may help explain super-spreading events reported in meatpacking
    plants, churches and restaurants.
    \href{https://www.nytimes.com/2020/07/06/health/coronavirus-airborne-aerosols.html?action=click\&pgtype=Article\&state=default\&region=MAIN_CONTENT_3\&context=storylines_faq}{It's
    unclear how often the virus is spread} via these tiny droplets, or
    aerosols, compared with larger droplets that are expelled when a
    sick person coughs or sneezes, or transmitted through contact with
    contaminated surfaces, said Linsey Marr, an aerosol expert at
    Virginia Tech. Aerosols are released even when a person without
    symptoms exhales, talks or sings, according to Dr. Marr and more
    than 200 other experts, who
    \href{https://academic.oup.com/cid/article/doi/10.1093/cid/ciaa939/5867798}{have
    outlined the evidence in an open letter to the World Health
    Organization}.
  \end{itemize}
\item ~
  \hypertarget{what-are-the-symptoms-of-coronavirus}{%
  \paragraph{What are the symptoms of
  coronavirus?}\label{what-are-the-symptoms-of-coronavirus}}

  \begin{itemize}
  \tightlist
  \item
    Common symptoms
    \href{https://www.nytimes.com/article/symptoms-coronavirus.html?action=click\&pgtype=Article\&state=default\&region=MAIN_CONTENT_3\&context=storylines_faq}{include
    fever, a dry cough, fatigue and difficulty breathing or shortness of
    breath.} Some of these symptoms overlap with those of the flu,
    making detection difficult, but runny noses and stuffy sinuses are
    less common.
    \href{https://www.nytimes.com/2020/04/27/health/coronavirus-symptoms-cdc.html?action=click\&pgtype=Article\&state=default\&region=MAIN_CONTENT_3\&context=storylines_faq}{The
    C.D.C. has also} added chills, muscle pain, sore throat, headache
    and a new loss of the sense of taste or smell as symptoms to look
    out for. Most people fall ill five to seven days after exposure, but
    symptoms may appear in as few as two days or as many as 14 days.
  \end{itemize}
\item ~
  \hypertarget{does-asymptomatic-transmission-of-covid-19-happen}{%
  \paragraph{Does asymptomatic transmission of Covid-19
  happen?}\label{does-asymptomatic-transmission-of-covid-19-happen}}

  \begin{itemize}
  \tightlist
  \item
    So far, the evidence seems to show it does. A widely cited
    \href{https://www.nature.com/articles/s41591-020-0869-5}{paper}
    published in April suggests that people are most infectious about
    two days before the onset of coronavirus symptoms and estimated that
    44 percent of new infections were a result of transmission from
    people who were not yet showing symptoms. Recently, a top expert at
    the World Health Organization stated that transmission of the
    coronavirus by people who did not have symptoms was ``very rare,''
    \href{https://www.nytimes.com/2020/06/09/world/coronavirus-updates.html?action=click\&pgtype=Article\&state=default\&region=MAIN_CONTENT_3\&context=storylines_faq\#link-1f302e21}{but
    she later walked back that statement.}
  \end{itemize}
\end{itemize}

Adding to these challenges, democracy in Latin America has also lost a
champion in the United States, which had played an important role in
promoting democracy after the end of the Cold War by financing good
governance programs and calling out authoritarian abuses.

Under President Trump, the United States has mostly focused regional
policy on opposing left-wing autocrats in Venezuela and Cuba and curbing
immigration, making aid to Central American nations, among the region's
poorest, contingent on cooperating with the administration on
immigration.

The Trump administration also refrained from commenting when Nayib
Bukele, the president of El Salvador, ignored Supreme Court rulings and
used the military to crack down on quarantine violators during the
pandemic.

American support for democracy initiatives in Latin America fell by
almost half last year to \$326 million, according to preliminary figures
compiled by the United States Agency for International Development.

``In the last few years, we have not only abandoned our role as a
democratizing force in Latin America and the world, but we have promoted
negative forces,'' said Orlando Pérez, a political scientist at the
University of North Texas. ``Our policy is now: `You're on your own ---
America first.'''

Image

Supporters of President David Granger of Guyana celebrating after an
election in March. Mr. Granger lost but refused to step
down.Credit...Adriana Loureiro Fernandez for The New York Times

In the few democratic strongholds in Latin America, such as Uruguay and
Costa Rica, leaders responded to the pandemic with efficiency and
transparency, boosting public trust in the government. In the Dominican
Republic and Suriname, incumbent presidents recently bowed out of power
after losing elections that were held despite the pandemic.

In many instances, judges and civil servants have resisted the attacks
on democratic institutions during the pandemic, said Javier Corrales, a
professor of Latin American studies at Amherst College in Massachusetts.
``The defenders of liberal democracy in Latin America are not
defeated,'' said Mr. Corrales. ``It's not an open terrain for would-be
authoritarians.''

Yet in most Latin American nations, the coronavirus accelerated a
pre-existing democratic decline by exposing the weakness and corruption
of governments in the face of the catastrophe.

``When confronted with an existential threat, countries that did not
already have deep democratic systems are choosing tactics that help
leaders consolidate their power,'' said John Polga-Hecimovich, a
political scientist at the United States Naval Academy in Maryland.

The political tensions gripping the region in the pandemic could be just
the beginning of a longer wave of unrest and authoritarianism, said
Thomas Carothers, a fellow at the Carnegie Endowment for International
Peace. ``It will drag the region down into poorer economic
performance,'' he said. ``It also means poorer treatment of human
beings, their dignity and rights.''

Image

In the islands of St. Kitts and Nevis, the government prevented some
electoral observers from traveling to the country during a recent
electoral campaign because of precautions to prevent the spread of the
virus.Credit...Dennis M. Rivera Pichardo for The New York Times

Natalie Kitroeff contributed reporting from Mexico City.

Advertisement

\protect\hyperlink{after-bottom}{Continue reading the main story}

\hypertarget{site-index}{%
\subsection{Site Index}\label{site-index}}

\hypertarget{site-information-navigation}{%
\subsection{Site Information
Navigation}\label{site-information-navigation}}

\begin{itemize}
\tightlist
\item
  \href{https://help.nytimes.com/hc/en-us/articles/115014792127-Copyright-notice}{©~2020~The
  New York Times Company}
\end{itemize}

\begin{itemize}
\tightlist
\item
  \href{https://www.nytco.com/}{NYTCo}
\item
  \href{https://help.nytimes.com/hc/en-us/articles/115015385887-Contact-Us}{Contact
  Us}
\item
  \href{https://www.nytco.com/careers/}{Work with us}
\item
  \href{https://nytmediakit.com/}{Advertise}
\item
  \href{http://www.tbrandstudio.com/}{T Brand Studio}
\item
  \href{https://www.nytimes.com/privacy/cookie-policy\#how-do-i-manage-trackers}{Your
  Ad Choices}
\item
  \href{https://www.nytimes.com/privacy}{Privacy}
\item
  \href{https://help.nytimes.com/hc/en-us/articles/115014893428-Terms-of-service}{Terms
  of Service}
\item
  \href{https://help.nytimes.com/hc/en-us/articles/115014893968-Terms-of-sale}{Terms
  of Sale}
\item
  \href{https://spiderbites.nytimes.com}{Site Map}
\item
  \href{https://help.nytimes.com/hc/en-us}{Help}
\item
  \href{https://www.nytimes.com/subscription?campaignId=37WXW}{Subscriptions}
\end{itemize}
