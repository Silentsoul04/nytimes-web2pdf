Sections

SEARCH

\protect\hyperlink{site-content}{Skip to
content}\protect\hyperlink{site-index}{Skip to site index}

\href{https://www.nytimes.com/section/arts}{Arts}

\href{https://myaccount.nytimes.com/auth/login?response_type=cookie\&client_id=vi}{}

\href{https://www.nytimes.com/section/todayspaper}{Today's Paper}

\href{/section/arts}{Arts}\textbar{}Martha Nierenberg, Entrepreneur Who
Sought Art's Return, Dies at 96

\url{https://nyti.ms/3hUQxXk}

\begin{itemize}
\item
\item
\item
\item
\item
\end{itemize}

Advertisement

\protect\hyperlink{after-top}{Continue reading the main story}

Supported by

\protect\hyperlink{after-sponsor}{Continue reading the main story}

\hypertarget{martha-nierenberg-entrepreneur-who-sought-arts-return-dies-at-96}{%
\section{Martha Nierenberg, Entrepreneur Who Sought Art's Return, Dies
at
96}\label{martha-nierenberg-entrepreneur-who-sought-arts-return-dies-at-96}}

A co-founder of Dansk Designs, she fought to recover family masterpieces
stolen by the Nazis when they invaded Hungary. The case, mired in a U.S.
court, goes on.

\includegraphics{https://static01.nyt.com/images/2020/08/02/obituaries/02Nierenberg-obit1/Nierenberg-1-articleLarge.jpg?quality=75\&auto=webp\&disable=upscale}

By \href{https://www.nytimes.com/by/tom-mashberg}{Tom Mashberg}

\begin{itemize}
\item
  July 30, 2020
\item
  \begin{itemize}
  \item
  \item
  \item
  \item
  \item
  \end{itemize}
\end{itemize}

Martha Nierenberg had barely turned 20 when she was bundled off a train
in central Hungary and hidden by nuns in a Roman Catholic hospital. It
was March 1944, and a Nazi occupying force that included Adolf Eichmann
was marching into her home city, Budapest. Eichmann, a principal
engineer of the Holocaust, would immediately embark on the annihilation
of 500,000 Hungarian Jews.

Mrs. Nierenberg, who was born into one of Hungary's wealthiest families,
evaded capture for two months before friends assured her that she could
venture home. There she learned that she would be among 42 family
members and close associates who were to be driven by the Germans to the
Austrian border and, several weeks later, allowed to escape to
Switzerland or Portugal.

The cost of life was high. The Nazis strong-armed the family into
signing away their estates, including some 2,500 pieces of precious art.
And Mrs. Nierenberg's father, Alfons Weiss de Csepel, was among five
relatives who were forced to stay behind as hostages of the Third Reich.

A trained biochemist who spoke six languages, Mrs. Nierenberg made it to
the United States with her mother in 1945. She set off on a career as a
scientist and researcher at M.I.T. and the Rockefeller Institute for
Medical Research in New York, and then as an entrepreneur. She achieved
major success when, in 1954, she and her husband, Theodore Nierenberg,
founded the
\href{https://designresearch.sva.edu/research/dansk-designs-reinventing-the-american-tabletop-1954-1985/}{Dansk
Designs housewares company}, which reimagined and invigorated the
American tabletop.

Mrs. Nierenberg died in her sleep on June 27 at a senior living facility
in Rye, N.Y., according to her family. She was 96.

At her death she was a lead plaintiff in a 30-year Holocaust art
restitution battle with Hungary that counts as one of the highest-value
cases ever pursued by a single family. Among the 40 paintings Hungary
has refused to return are four by El Greco and others by Corot,
Velázquez and Courbet. Her trustee, her granddaughter Robin Bunevich,
estimates that the collection is worth \$100 million. She said the
family would continue to press the case.

\includegraphics{https://static01.nyt.com/images/2020/07/27/obituaries/Nierenberg/merlin_21470445_f722669f-8c01-46c0-8cf9-d7fcb8f1c6df-articleLarge.jpg?quality=75\&auto=webp\&disable=upscale}

In a 2019 interview for her memoirs, Mrs. Nierenberg spoke of growing up
surrounded by the
\href{http://hungarylootedart.com/?page_id=32}{confiscated masterpieces}
and other fine art objects, ancient sculptures and pre-Renaissance
furniture and rugs, many of which had been expropriated by Eichmann
himself.

``We love these paintings,'' she said. ``We would dearly like to have
something back.''

Martha Weiss de Csepel was born in Budapest on March 12, 1924, a
granddaughter of Baron Mor Lipot Herzog, one of Europe's premier art and
antiquities collectors. Her paternal grandfather, Manfred Weiss de
Csepel, had built the Manfred Weiss Steel and Metal Works, Hungary's
largest machine factory, which employed 40,000 people and pumped out
trucks, washing machines and other items, including munitions. Once the
Nazis seized control, they converted the works entirely into a factory
for weapons and war machinery.

Martha's mother, Erzsebet Herzog Weiss de Csepel, held a medical degree
and had studied psychiatry in Vienna. She saw to it that Martha and her
two brothers and a sister received advanced educations. Jewish by birth,
Martha was nonetheless sent to a Calvinist school, where she could focus
on science and math. After graduation she enrolled in a science college
in Budapest.

Despite enduring frightening moments, Mrs. Nierenberg was well aware
that her family's wealth and prestige had insulated them from the worst
horrors of the Nazi genocide.

``Our family was really less touched by what was happening,'' she
recalled. ``But my friends were in trouble,'' she added. ``I had a
number of friends who were poured into labor camps. Everybody was in
hiding.''

Her mother and two of her mother's brothers had inherited the Herzog art
collection and chateau upon Baron Herzog's death in 1934. During the
\href{https://memoirs.azrielifoundation.org/articles-and-excerpts/suddenly-the-shadow-fell-the-german-invasion-of-hungary}{upheaval
of 1944}, they hid as much of the works as possible in bomb shelters,
salt mines and the basement of the Weiss factory.

Image

Credit...Herzog Family Archive

Image

Credit...Herzog Family Archive

Most of the masterworks were ferreted out by collaborators and delivered
to Eichmann's headquarters at the Majestic Hotel in Budapest. He
earmarked a few dozen paintings for Berlin and handed the rest to the
Hungarian National Gallery and Budapest Museum of Fine Arts, which hold
them to this day. Hundreds of items remain unaccounted for.

Mrs. Nierenberg and her mother never lost sight of recovering their
artistic birthright, but they were also intent on establishing their new
lives in New York. Martha set out to complete her science degree, moving
for a time to Cambridge, Mass., to study at Radcliffe and the
Massachusetts Institute of Technology.

She met Ted Nierenberg, a Manhattanite and the owner of a metal
finishing company, at the Broadway premiere of ``Guys and Dolls'' in
November 1950. They married the following year, moved to Great Neck, on
Long Island, and had four children, Lisa, Karin, Peter and Al, all of
whom survive their mother.
\href{https://www.nytimes.com/2009/08/04/arts/design/04nierenberg.html}{Mr.
Nierenberg died in 2009 at 86.}

Eager to start a new business, the Nierenbergs toured Europe in 1954 to
seek out industrial items for the American market. In Copenhagen, they
discovered the work of the Danish designer
\href{https://www.nytimes.com/2008/02/02/arts/design/02quistgaard.html}{Jens
Quistgaard}, who was well-known in Europe for his sleek, elegant
everyday flatware.

Enraptured by the Scandinavian modern style, Mr. Nierenberg barged into
Mr. Quistgaard's studio that very day and proposed that they go into
business together. As Mrs. Nierenberg recalled, ``Ted was often
impulsive, and I had to go along with his antics.''

So began a 30-year partnership that saw Dansk extend well beyond cutlery
into silverware and tableware; saucepans and casserole dishes made of
enamel-coated steel; glazed stoneware; wine glasses; and pitchers, bowls
and pepper mills made from exotic woods. With Mr. Quistgaard as chief
designer and Mr. Nierenberg as head of marketing, the brand achieved
international success by taking aim at high-end buyers with slogans like
``Expensive \ldots{} by Design.''

The Nierenbergs, by then living in an expansive glass and timber home
surrounded by woodlands in Armonk, N.Y., retired in 1985 and sold the
company and its 31 retail stores to their employees. Dansk is now owned
by Lenox China.

With the opening of Hungary to the West in 1989, Mrs. Nierenberg and the
other Herzog heirs, spearheaded by David L. de Csepel of Los Angeles, a
grandson of Erzsebet Herzog Weiss de Csepel, reached out to the
authorities in their homeland hoping for an amicable agreement on the
art. But despite help from Senators Hillary Rodham Clinton of New York,
Edward M. Kennedy of Massachusetts and Frank R. Lautenberg of New
Jersey, they were rebuffed.

The family took their case to the Hungarian court system in the
mid-2000s, but after several years of rulings they found no relief. So
in 2010, with funding from the billionaire philanthropist Ronald S.
Lauder, they
\href{http://www.hungarylootedart.com/wp-content/uploads/2010/06/Herzog_Background.pdf}{filed
suit} in United States District Court for the District of Columbia,
where the case has faced a welter of procedural and technical claims
ever since.

``It would be so simple for the government to make this right, but our
struggle goes on,'' Mrs. Nierenberg said in 2019. Ruefully, she added,
``I guess they're hoping they can wait me out.''

Advertisement

\protect\hyperlink{after-bottom}{Continue reading the main story}

\hypertarget{site-index}{%
\subsection{Site Index}\label{site-index}}

\hypertarget{site-information-navigation}{%
\subsection{Site Information
Navigation}\label{site-information-navigation}}

\begin{itemize}
\tightlist
\item
  \href{https://help.nytimes.com/hc/en-us/articles/115014792127-Copyright-notice}{©~2020~The
  New York Times Company}
\end{itemize}

\begin{itemize}
\tightlist
\item
  \href{https://www.nytco.com/}{NYTCo}
\item
  \href{https://help.nytimes.com/hc/en-us/articles/115015385887-Contact-Us}{Contact
  Us}
\item
  \href{https://www.nytco.com/careers/}{Work with us}
\item
  \href{https://nytmediakit.com/}{Advertise}
\item
  \href{http://www.tbrandstudio.com/}{T Brand Studio}
\item
  \href{https://www.nytimes.com/privacy/cookie-policy\#how-do-i-manage-trackers}{Your
  Ad Choices}
\item
  \href{https://www.nytimes.com/privacy}{Privacy}
\item
  \href{https://help.nytimes.com/hc/en-us/articles/115014893428-Terms-of-service}{Terms
  of Service}
\item
  \href{https://help.nytimes.com/hc/en-us/articles/115014893968-Terms-of-sale}{Terms
  of Sale}
\item
  \href{https://spiderbites.nytimes.com}{Site Map}
\item
  \href{https://help.nytimes.com/hc/en-us}{Help}
\item
  \href{https://www.nytimes.com/subscription?campaignId=37WXW}{Subscriptions}
\end{itemize}
