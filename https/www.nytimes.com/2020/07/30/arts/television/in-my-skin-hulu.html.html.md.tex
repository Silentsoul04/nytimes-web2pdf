Sections

SEARCH

\protect\hyperlink{site-content}{Skip to
content}\protect\hyperlink{site-index}{Skip to site index}

\href{https://www.nytimes.com/section/arts/television}{Television}

\href{https://myaccount.nytimes.com/auth/login?response_type=cookie\&client_id=vi}{}

\href{https://www.nytimes.com/section/todayspaper}{Today's Paper}

\href{/section/arts/television}{Television}\textbar{}With American TV on
Pause, Here Are 5 British Series to Watch

\url{https://nyti.ms/2PawDLm}

\begin{itemize}
\item
\item
\item
\item
\item
\end{itemize}

Advertisement

\protect\hyperlink{after-top}{Continue reading the main story}

Supported by

\protect\hyperlink{after-sponsor}{Continue reading the main story}

critic's notebook

\hypertarget{with-american-tv-on-pause-here-are-5-british-series-to-watch}{%
\section{With American TV on Pause, Here Are 5 British Series to
Watch}\label{with-american-tv-on-pause-here-are-5-british-series-to-watch}}

On outlets from Hulu to Peacock to PBS, it's the summer of the
trans-Atlantic import.

\includegraphics{https://static01.nyt.com/images/2020/07/30/arts/30british-series/merlin_172859073_50d8470f-744c-4299-a4b4-fe38eea4228e-articleLarge.jpg?quality=75\&auto=webp\&disable=upscale}

\href{https://www.nytimes.com/by/mike-hale}{\includegraphics{https://static01.nyt.com/images/2018/02/16/multimedia/author-mike-hale/author-mike-hale-thumbLarge.jpg}}

By \href{https://www.nytimes.com/by/mike-hale}{Mike Hale}

\begin{itemize}
\item
  July 30, 2020
\item
  \begin{itemize}
  \item
  \item
  \item
  \item
  \item
  \end{itemize}
\end{itemize}

Among the things the Covid-19 pandemic has taken away from us, at least
temporarily, new American-made television series are not the most
important. But for those who keep track of these things, the paucity of
domestic scripted shows premiering in the next 10 days or so is
striking. There's a ``Star Trek'' cartoon, a DreamWorks cartoon, a new
season of ``Umbrella Academy'' and --- that's about it.

And yet there are plenty of fresh comedies and dramas arriving during
that time, more than 20 of them, scoured from countries around the globe
where they were made before the virus struck. The majority are British,
continuing a trend that began as a small stream with
\href{https://www.nytimes.com/2020/07/15/arts/television/capture-intelligence-review-peacock.html}{the
launch lineups of HBO Max and Peacock} and is turning into a
cross-Atlantic tsunami as summer progresses. Here are some highlights of
this latest batch of British imports, in chronological order.

\hypertarget{in-my-skin}{%
\subsection{`In My Skin'}\label{in-my-skin}}

\href{https://www.hulu.com/series/in-my-skin-91be18aa-30c6-40bf-b191-74503708b305}{\emph{Hulu}}

Bethan (Gabrielle Creevy), the Welsh teenager at the center of this
gently barbed coming-of-age story, is a full-time fabulist. She feeds
her friends and teachers a steady diet of haute-bourgeois lies --- one
of her more inspired ad-libs when a friend wants to come over is, ``I
can't, we're having a conservatory built'' --- because she's mortified
by the sad, even dangerous reality of life with her bipolar mom and
drunk, deadbeat dad.

It's part of her larger artistic impulse: While she's spinning her
vision of a stable, prosperous home environment as a smoke screen for
those around her, she's writing derivative proletarian verses for her
high school literary anthology. (The show frequently cuts away from the
action to show us flashes of what's going on inside Bethan's head; her
poetry is accompanied by heroic black-and-white images of Welsh coal
miners.)

The lies begin to catch up with her, of course, partly because she's
powerfully distracted by a popular female classmate (Zadeiah
Campbell-Davies). But across the five episodes of the initial season ---
written by Kayleigh Llewellyn and directed by Lucy Forbes, who directed
half of the second season of
``\href{https://www.nytimes.com/2019/11/04/arts/television/review-the-end-of-the-world-netflix.html}{The
End of the \_\_\_\_ World},'' and shown on BBC Three in March ---
happily smutty dark humor and light melancholy mostly win out over
maudlin life lessons. The distinctively British mix of winsome-glum
kitchen-sink drama and sitcom beats works in this case, helped by the
loose, run-and-gun style of Forbes and her cinematographer, Benedict
Spence, and Creevy's alert, understated performance.

\includegraphics{https://static01.nyt.com/images/2020/07/30/arts/30british-2/30british-2-articleLarge.jpg?quality=75\&auto=webp\&disable=upscale}

\hypertarget{hitmen}{%
\subsection{`Hitmen'}\label{hitmen}}

\emph{Peacock, Aug. 6}

Imagine Laverne and Shirley as a pair of working-class contract killers
and you've pretty much got the idea of this comedy, whose six-episode
first season ran in March on Sky. Mel Giedroyc and Sue Perkins, best
known as the original hosts of
``\href{https://www.nytimes.com/2016/09/14/business/media/great-british-bake-off-will-leave-bbc-for-rival.html}{The
Great British Bake Off},'' play Jamie and Fran, who approach their
violent occupation with the enthusiasm and professionalism of
shelf-stockers at a big-box store. (Joe Markham and Joe Parham, the
show's creators, previously worked together on the nutty animated series
``The Amazing World of Gumball.'')

The broad humor, largely of the restless-middle-age variety, often takes
place while the hit women sit in their van with a trussed-up victim,
waiting for instructions from their unseen employer, Mr. K. Much of the
fun comes from the actors playing the testy, garrulous targets,
including Jason Watkins of ``The Crown'' as a crooked lawyer and Sian
Clifford of ``Fleabag'' as a disloyal accountant.

Image

PBS's ``Endeavour,'' with Roger Allam, left, and Shaun Evans, is
returning for its seventh season.Credit...Jonathan Ford and Mammoth for
ITV and Masterpiece

\hypertarget{endeavour}{%
\subsection{`Endeavour'}\label{endeavour}}

\emph{PBS, Aug. 9}

This prequel series, a fixture of PBS's ``Masterpiece,'' is creeping
closer in time to ``Inspector Morse,'' the popular British mystery from
which it was spun off: The seventh season of
``\href{https://www.nytimes.com/2012/06/30/arts/television/inspector-morse-returns-in-endeavour-a-prequel.html?searchResultPosition=2}{Endeavour}''
is set in 1970, within hailing distance of the 1987 advent of ``Morse.''
And as the shows converge, the notion that the stern young detective
Endeavour Morse played by Shaun Evans in the current series is going to
age into the paunchy, sardonic, thoroughly modern misanthrope played by
John Thaw in the original is becoming increasingly hard to entertain.

Evans's formal, diffident, awkward Morse is fine in its own right,
though, and ITV's ``Endeavour'' shares the original's pensive, almost
mournful atmosphere. The new three-episode season (it premiered in
February) carries on story lines from Season 6 that find Morse
increasingly at odds with his boss and mentor, Fred Thursday (Roger
Allam), as the case of the killer haunting the towpaths of Oxford's
canals refuses to stay solved. The racism and sexism of the time figure
into other homicides, and the indignities of aging and Morse's latest
disastrous love affair contribute to the generally downbeat tone. As
always, the dolorous goings-on are exquisitely enacted by Evans, Allam
and, as their superintendent, Anton Lesser.

Image

Babou Ceesay, left, Eve Myles play mismatched partners in ``We Hunt
Together.''Credit...Ludovic Robert/BBC Studios/UKTV

\hypertarget{we-hunt-together}{%
\subsection{`We Hunt Together'}\label{we-hunt-together}}

\emph{Showtime, Aug. 9}

At the far end of the British mystery spectrum from ``Endeavour,'' this
rare original series from Alibi --- a channel that exists primarily to
show reruns of other channel's crime shows --- is firmly within the camp
of lurid melodrama. Everyone is damaged, from the former child soldier
to the brainy phone-sex worker to the frighteningly rigid cop.

\href{https://www.nytimes.com/2019/08/30/arts/television/keeping-faith-eve-myles.html?searchResultPosition=2}{Eve
Myles} (``Torchwood'') and Babou Ceesay (``Into the Badlands'') play the
latest variation on mismatched partners --- her the all-business
sergeant, him the jolly, empathetic, higher-ranking detective just
brought in from internal affairs. Myles and Ceesay make the familiar
byplay fairly engaging, but they're only half the story: Equal time, and
nearly equal sympathy, is given across the six episodes (which debuted
in Britain in May) to the Bonnie-and-Clyde killers played by Hermione
Corfield and Dipo Ola. The murder-for-love plotline may not hold water,
but everyone involved is fun to watch.

Image

``The Other One,'' on Acorn, stars Lauren Socha, left, and Ellie White
as half sisters who discover each other as adults.Credit...AcornTV

\hypertarget{the-other-one}{%
\subsection{`The Other One'}\label{the-other-one}}

\emph{Acorn TV, Aug. 10}

This series about two half sisters who discover each other when their
father dies belongs to a genre, the life-force comedy, that isn't my
favorite. (It often involves weddings, as in ``Muriel's'' and ``My Big
Fat Greek.'') But the show's creator, Holly Walsh (``Motherland''),
deftly undercuts the inherent sentimentalities of her story, even as the
supremely uptight Cathy (Ellie White) and the raucous, free-spirited Cat
(Lauren Socha) predictably overcome their differences and form a new
family blended from emotional openness and cheap white wine run through
a SodaStream.

White, who plays the dire Princess Beatrice in ``The Windsors,'' is
entirely convincing as the anxious and controlling but big-hearted
Cathy, and she's ably supported in the first season's seven episodes
(shown on BBC beginning in June) by Socha and a pair of scene-stealing
veterans, Rebecca Front and Siobhan Finneran, as the dead man's furious
wife and his dizzy, agoraphobic mistress. Perhaps most important in
setting the show's tone is a classic-pop soundtrack centered in the
missing father's late-70s sweet spot: Supertramp, Orleans, Hall and
Oates, ``The Piña Colada Song.''

\textbf{More recent and coming British, Australian and Canadian series
premieres:} ``Maxxx,'' Hulu; ``Ladhood,'' Hulu; ``Frayed,'' HBO Max;
``Brassic,'' Hulu (Friday); ``Get Even,'' Netflix (Friday); ``Wild
Bill,'' BritBox (Tuesday); ``Coroner,'' the CW (Wednesday); ``Upright,''
Sundance Now (Aug. 6); ``Being Reuben,'' the CW (Aug. 7); ``Five
Bedrooms,'' Peacock (Aug. 13).

Advertisement

\protect\hyperlink{after-bottom}{Continue reading the main story}

\hypertarget{site-index}{%
\subsection{Site Index}\label{site-index}}

\hypertarget{site-information-navigation}{%
\subsection{Site Information
Navigation}\label{site-information-navigation}}

\begin{itemize}
\tightlist
\item
  \href{https://help.nytimes.com/hc/en-us/articles/115014792127-Copyright-notice}{©~2020~The
  New York Times Company}
\end{itemize}

\begin{itemize}
\tightlist
\item
  \href{https://www.nytco.com/}{NYTCo}
\item
  \href{https://help.nytimes.com/hc/en-us/articles/115015385887-Contact-Us}{Contact
  Us}
\item
  \href{https://www.nytco.com/careers/}{Work with us}
\item
  \href{https://nytmediakit.com/}{Advertise}
\item
  \href{http://www.tbrandstudio.com/}{T Brand Studio}
\item
  \href{https://www.nytimes.com/privacy/cookie-policy\#how-do-i-manage-trackers}{Your
  Ad Choices}
\item
  \href{https://www.nytimes.com/privacy}{Privacy}
\item
  \href{https://help.nytimes.com/hc/en-us/articles/115014893428-Terms-of-service}{Terms
  of Service}
\item
  \href{https://help.nytimes.com/hc/en-us/articles/115014893968-Terms-of-sale}{Terms
  of Sale}
\item
  \href{https://spiderbites.nytimes.com}{Site Map}
\item
  \href{https://help.nytimes.com/hc/en-us}{Help}
\item
  \href{https://www.nytimes.com/subscription?campaignId=37WXW}{Subscriptions}
\end{itemize}
