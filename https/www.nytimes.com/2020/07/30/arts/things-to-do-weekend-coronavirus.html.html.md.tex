Sections

SEARCH

\protect\hyperlink{site-content}{Skip to
content}\protect\hyperlink{site-index}{Skip to site index}

\href{https://www.nytimes.com/section/arts}{Arts}

\href{https://myaccount.nytimes.com/auth/login?response_type=cookie\&client_id=vi}{}

\href{https://www.nytimes.com/section/todayspaper}{Today's Paper}

\href{/section/arts}{Arts}\textbar{}7 Things to Do This Weekend

\url{https://nyti.ms/3ffPijF}

\begin{itemize}
\item
\item
\item
\item
\item
\end{itemize}

\href{https://www.nytimes.com/spotlight/at-home?action=click\&pgtype=Article\&state=default\&region=TOP_BANNER\&context=at_home_menu}{At
Home}

\begin{itemize}
\tightlist
\item
  \href{https://www.nytimes.com/2020/07/28/books/time-for-a-literary-road-trip.html?action=click\&pgtype=Article\&state=default\&region=TOP_BANNER\&context=at_home_menu}{Take:
  A Literary Road Trip}
\item
  \href{https://www.nytimes.com/2020/07/29/magazine/bored-with-your-home-cooking-some-smoky-eggplant-will-fix-that.html?action=click\&pgtype=Article\&state=default\&region=TOP_BANNER\&context=at_home_menu}{Cook:
  Smoky Eggplant}
\item
  \href{https://www.nytimes.com/2020/07/27/travel/moose-michigan-isle-royale.html?action=click\&pgtype=Article\&state=default\&region=TOP_BANNER\&context=at_home_menu}{Look
  Out: For Moose}
\item
  \href{https://www.nytimes.com/interactive/2020/at-home/even-more-reporters-editors-diaries-lists-recommendations.html?action=click\&pgtype=Article\&state=default\&region=TOP_BANNER\&context=at_home_menu}{Explore:
  Reporters' Obsessions}
\end{itemize}

Advertisement

\protect\hyperlink{after-top}{Continue reading the main story}

Supported by

\protect\hyperlink{after-sponsor}{Continue reading the main story}

Weekend Roundup

\hypertarget{7-things-to-do-this-weekend}{%
\section{7 Things to Do This
Weekend}\label{7-things-to-do-this-weekend}}

How can you get your cultural fix when many arts institutions remain
closed? Our writers offer suggestions for what to listen to and watch.

July 30, 2020

\begin{itemize}
\item
\item
\item
\item
\item
\end{itemize}

Pop MUSIC

\hypertarget{queen-bey-unveils-black-is-king}{%
\subsection{Queen Bey Unveils `Black Is
King'}\label{queen-bey-unveils-black-is-king}}

\includegraphics{https://static01.nyt.com/images/2020/07/30/arts/30wkd-arts-roundup-pop/merlin_175066431_37817126-f8c9-4f9b-a5ee-e748d85b1972-articleLarge.jpg?quality=75\&auto=webp\&disable=upscale}

Madonna beat Beyoncé to the title ``Queen of Pop,'' but Beyoncé's more
open-ended honorific, ``Queen Bey,'' turned out to be quite fitting:
These days, she hardly limits herself to just the one medium. Since
2013's self-titled LP, her albums have been cinematic feats as much as
musical events. Her latest opus, due out on Friday, is
\href{https://disneyplusoriginals.disney.com/movie/black-is-king}{``Black
Is King,''} a visual corollary to last year's album,
\href{https://www.youtube.com/watch?v=XnDdyeOaXy0\&list=OLAK5uy_nkj8U4dj1uSMUhZsAp0R3TpYH0xWvbcPc}{``The
Lion King: The Gift''} --- itself a companion to Disney's
\href{https://www.nytimes.com/2019/07/11/movies/the-lion-king-review.html}{blockbuster
remake} of the 1994 animated classic.

A new chapter in her ongoing project of foregrounding Black experience
in her work, the film represents the collaborative efforts of more than
a dozen co-directors, including the Ghanaian filmmaker Blitz Bazawule
(\href{https://www.nytimes.com/2019/03/28/movies/the-burial-of-kojo-review.html}{``The
Burial of Kojo''}) and Ibra Ake, Donald Glover's longtime creative
collaborator. And like the album that inspired it, ``Black Is King''
boasts an all-star cast that includes Naomi Campbell, Lupita Nyong'o,
Kelly Rowland, Pharrell Williams and Jay-Z.

``Black Is King'' premieres on Disney+ as part of an exclusive
distribution deal that will bring the film to many countries in Africa.
The combined might of three cultural juggernauts --- Beyoncé, Disney and
its leonine (and most profitable) franchise --- should make for a truly
spectacular global celebration of ``the breadth and beauty of Black
ancestry,'' to borrow Beyoncé's
\href{https://www.instagram.com/p/CCAMxfrHjAL/}{words}.\\
\emph{OLIVIA HORN}

Art \& Museums

\hypertarget{quilts-with-stories-to-tell}{%
\subsection{Quilts With Stories to
Tell}\label{quilts-with-stories-to-tell}}

Image

Bisa Butler's ``Broom Jumpers'' (2019), on view at the Katonah Museum of
Art until Oct. 4.Credit...Mount Holyoke College Art Museum

Bisa Butler's work originates from the idea of absence. The subjects for
her quilts tend to be anonymous, sometimes given little more than a
designation of ``Negro,'' which is the search term she plugged into one
of the Library of Congress's photographic databases to find some of her
source material.

Fusing figuration with collage for the pieces in her current exhibition
at the Katonah Museum of Art,
\href{http://www.katonahmuseum.org/exhibitions/}{``Bisa Butler:
Portraits,''} she used vividly patterned African fabrics to create
large-scale images of Black people, reconstructing their stories and
seemingly riffing on a Black tradition of oral histories that take shape
through their retelling. Her work evokes the poignant, generations-old
legacy of quilting in the Black community, made famous by the
\href{https://www.soulsgrowndeep.org/gees-bend-quiltmakers}{women of
Gee's Bend}.

The \href{http://www.katonahmuseum.org/}{Katonah Museum of Art}, about
45 miles north of New York City, has reopened, so you can see the show
in person through Oct. 4, or you can visit the museum's website, whose
offerings include a virtual walk-through of the galleries.
\href{http://www.katonahmuseum.org/programs-and-events/BisaButler-Virtual-BisaButler-LiveInConversation/}{On
Sunday at 4 p.m.} Eastern time, Butler will talk with the museum's
executive director, Michael Gitlitz, in a Zoom session available to the
public for \$5; the proceeds will benefit Black Lives Matter. The
discussion will focus on her works and their influences --- those who
have names, and the many others who do not.\\
\emph{MELISSA SMITH}

Theater

\hypertarget{soundwalking-through-the-graveyard}{%
\subsection{Soundwalking Through the
Graveyard}\label{soundwalking-through-the-graveyard}}

Image

Gelsey Bell and Joseph White's immersive audio project ``Cairns'' takes
you on a tour of Green-Wood Cemetery.Credit...Sarah Blesener for The New
York Times

For many Brooklynites, Green-Wood Cemetery has emerged as a welcome
oasis over the past few months; the setting, free of exercise
enthusiasts, offers breathing space and quiet. Now, Gelsey Bell and
Joseph White's immersive audio project ``Cairns'' will take visitors on
a self-guided tour that not only respects the cemetery's tranquillity
but also preserves social distancing.

Bell wrote and narrated the track, and composed the music with White.
Her involvement makes ``Cairns'' particularly intriguing: In recent
years, she has emerged as one of New York's most adventurous musicians,
leading visitors through the Museum of Modern Art's Fluxus sound
collection one day and appearing in the Dave Malloy musicals ``Natasha,
Pierre \& the Great Comet of 1812'' and ``Ghost Quartet'' the next.

You can download ``Cairns'' (available for \$7 starting Friday) from
\href{https://gelseybell.bandcamp.com/album/cairns}{Bell's Bandcamp
page} and the website of the performing arts center
\href{http://here.org/events/}{HERE}, which commissioned the piece. Then
head to Green-Wood's Sunset Park entrance, on Fourth Avenue and 35th
Street, and amble along as directed. Expect to drop by some of
Green-Wood's notable, if undersung, views and burial sites, including
those of the 19th-century Native American performer Do-Hum-Me and Susan
S. McKinney Steward, New York's first Black female doctor.

Not near Green-Wood? You can listen from home and be transported.\\
\emph{ELISABETH VINCENTELLI}

Dance

\hypertarget{candid-talk-on-cunninghams-company}{%
\subsection{Candid Talk on Cunningham's
Company}\label{candid-talk-on-cunninghams-company}}

Image

Rashaun Mitchell performing Merce Cunningham's ``Antic Meet'' in
2011.~Credit...Andrea Mohin/The New York Times

The weekly podcast \href{https://www.danceandstuff.com/}{``Dance and
Stuff,''} hosted by the artists
\href{https://www.nytimes.com/2020/05/05/arts/dance/reid-bartelme-jack-ferver-podcast.html}{Jack
Ferver and Reid Bartelme}, is full of spirited conversation with
performers, choreographers and others working in dance. But a recent
pair of episodes struck a deeper chord.

A few weeks ago, Ferver and Bartelme released a two-part interview with
three of the four Black dancers ever to join the Merce Cunningham Dance
Company in that institution's nearly 60 years: Gus Solomons Jr., Michael
Cole and Rashaun Mitchell. (The fourth,
\href{https://charlierose.com/videos/15779}{Ulysses Dove}, died in
1996.)

In
\href{https://anchor.fm/danceandstuff/episodes/Episode-159-With-Gus-Solomons--Jr---Michael-Cole--and-Rashaun-Mitchell-eg322s}{the
first part}, the dancers share their personal stories of discovering and
pursuing Cunningham's work. In
\href{https://anchor.fm/danceandstuff/episodes/Episode-160-Dancing-for-Merce-Cunningham-eg7ed4}{the
second}, they speak in greater depth about being the only Black company
member at a given time (their tenures never overlapped) and the broader
implications of the company's whiteness.

The candid, cross-generational dialogue sheds light on dimensions of
Cunningham's legacy too rarely discussed on the record. And it's easy to
complement these podcasts with videos. A few places to start: Learn more
about Solomons in the web series ``Mondays With Merce'' (he's featured
in \href{https://www.youtube.com/watch?v=_rRKRX0U6NA}{Episode 14}); see
Cole in
\href{https://dancecapsules.mercecunningham.org/overview.cfm?capid=46030}{``Beach
Birds for Camera,''} accessible through the Dance Capsules section of
mercecunningam.org; and watch
\href{https://www.ontheboards.tv/performances/tesseract-o}{``Tesseract,''}
Mitchell's collaboration with Silas Riener and Charles Atlas, at
OntheBoards.tv.\\
\emph{SIOBHAN BURKE}

KIDS

\hypertarget{standing-up-to-prejudice}{%
\subsection{Standing Up to Prejudice}\label{standing-up-to-prejudice}}

Image

Davied Morales, center, as Jelani in ``A Kids Play About Racism,'' with,
clockwise from bottom left, Isaiah Christopher-Lord Harris, Regan Sims,
Moses Goods, Jessenia Ingram, Rapheal Hamilton and Angel
Adedokun.Credit...Bay Area Children's Theater

Of all the difficult subjects to explain to children, racism is one of
the hardest and most relevant.

Last year, Jelani Memory, a biracial author and father, took on the task
with
\href{https://akidsbookabout.com/products/a-kids-book-about-racism}{``A
Kids Book About Racism,''} which incorporates his own experiences. Now
\href{https://khaliadavis.com/}{Khalia Davis} has adapted
\href{https://www.youtube.com/watch?v=LnaltG5N8nE}{his text} into a
half-hour virtual theater production,
\href{https://www.akidsplayabout.org/}{``A Kids Play About Racism,''}
which will be free all weekend on
\href{https://www.broadwayondemand.com/series/teJ66dfuOEak-a-kids-play-about-racism}{Broadway
on Demand}.

``We cast an actor who is also biracial to play Jelani at 10 years
old,'' said Davis, who directed the show as well. That performer is
\href{https://www.youtube.com/watch?v=gWPOrq7qt_g}{Davied Morales}, who
wrote the raps it includes.
(\href{http://www.justinellington.com/}{Justin Ellington} composed the
music.) ``I wanted to expand the world of the book, so he had someone to
respond to,'' Davis said of the Jelani character, who is surrounded by
players enacting his memories and emotions.

Produced by 41 companies in the organization
\href{http://www.tyausa.org/}{Theater for Young Audiences/USA}, the show
and accompanying educational videos will be streamable from midnight on
Friday to midnight on Sunday Eastern time. (Davis hopes to make the
presentation permanently available online.) Families can also register
for related \href{https://www.akidsplayabout.org/more}{Zoom theater
workshops} on Saturday and Sunday at 1 and 3 p.m.

The play, Davis added, helps children of any background understand not
only racism, but also how to ``do something about it.''\\
\emph{LAUREL GRAEBER}

Classical Music

\hypertarget{minimalism-meets-dream-pop}{%
\subsection{Minimalism Meets Dream
Pop}\label{minimalism-meets-dream-pop}}

Image

Molly Joyce released her debut full-length solo album, ``Breaking and
Entering,'' in June.Credit...Shervin Lainez

After a car accident nearly resulted in the amputation of her left hand
at age 7, Molly Joyce spent years in search of an instrument that would
fit her body.

When the composer, who has written for
\href{https://vickychow.bandcamp.com/track/rave-composed-by-molly-joyce}{virtuosos
like Vicky Chow}, started working with vintage toy organs, she quickly
perceived the opportunities they offered her as a performer. (The
buttons on a toy organ's left side permit a musician to play a chord
with one finger while navigating traditional keys with another hand on
its right side.)

In \href{https://www.youtube.com/watch?v=HcIavUYjRzg}{a 2017 TEDx Talk},
Joyce described how composing on this instrument allowed for a creative
process that could move beyond the binary of ability and disability.
Proof of her breakthrough is abundant throughout
\href{https://mollyjoyce.bandcamp.com/album/breaking-and-entering}{``Breaking
and Entering,''} the musician's debut full-length solo album. In a phone
interview before the recording's release in June, Joyce cited not only
early minimalists like Steve Reich and Philip Glass as stylistic
touchstones, but also artists like the Cocteau Twins, Beach House and My
Brightest Diamond.

Aside from her appreciation for ``less vibrato, very on-pitch'' singing,
Joyce noted her taste for enveloping production styles that come across
as a ``wash'' of sound. All those affections can be heard on the album's
opening track,
\href{https://mollyjoyce.bandcamp.com/track/body-and-being}{``Body and
Being,''} in which sustained chords, MIDI tones and her dream-pop vocals
work together to produce an airy, liberating sensation.\\
\emph{SETH COLTER WALLS}

Comedy

\hypertarget{a-special-that-should-have-been-a-contender}{%
\subsection{A Special That Should Have Been a
Contender}\label{a-special-that-should-have-been-a-contender}}

Image

Gary Gulman in his 2019 HBO special, ``The Great
Depresh.''Credit...Craig Blankenhorn/HBO

Far be it for me to quibble with Emmy voters, but quibble I shall,
because \href{https://garygulman.com/}{Gary Gulman}, perhaps the best
comedy writer in America, put out a special in the past year that's both
heartfelt and hilarious, with inimitable diction holding it together,
and yet the show failed to receive a nomination.

``Quibble'' is one of many words Gulman employs with such unequivocal
specificity in his 2019 HBO special,
\href{https://www.hbo.com/specials/gary-gulman-the-great-depresh}{``The
Great Depresh,''} which features his stand-up at Roulette in Brooklyn,
along with his conversations with stand-up colleagues at the Comedy
Cellar and sessions with his psychiatrist and his wife, Sadé, at Weill
Cornell Medicine. Cameras even follow Gulman back to his mother's house
outside of Boston to revisit his childhood. Over the course of 70-plus
minutes, Gulman demonstrates that comedians can struggle with depression
without becoming sad clowns, and that if he could find help, so can you.

That he manages to do so while accentuating his punch lines with precise
vocabulary sets him apart. In one of the special's early bits, he
describes his experience at drinking fountains in elementary school as
``fraught'' and ``perilous'' for a ``precocious'' kid trying to get his
full ``quench'' from the ``iron spout'' without a smack from ``the
cretin'' behind him.

You can relish Gulman's wordplay in ``The Great Depresh'' on HBO Max.\\
\emph{SEAN L. McCARTHY}

Advertisement

\protect\hyperlink{after-bottom}{Continue reading the main story}

\hypertarget{site-index}{%
\subsection{Site Index}\label{site-index}}

\hypertarget{site-information-navigation}{%
\subsection{Site Information
Navigation}\label{site-information-navigation}}

\begin{itemize}
\tightlist
\item
  \href{https://help.nytimes.com/hc/en-us/articles/115014792127-Copyright-notice}{©~2020~The
  New York Times Company}
\end{itemize}

\begin{itemize}
\tightlist
\item
  \href{https://www.nytco.com/}{NYTCo}
\item
  \href{https://help.nytimes.com/hc/en-us/articles/115015385887-Contact-Us}{Contact
  Us}
\item
  \href{https://www.nytco.com/careers/}{Work with us}
\item
  \href{https://nytmediakit.com/}{Advertise}
\item
  \href{http://www.tbrandstudio.com/}{T Brand Studio}
\item
  \href{https://www.nytimes.com/privacy/cookie-policy\#how-do-i-manage-trackers}{Your
  Ad Choices}
\item
  \href{https://www.nytimes.com/privacy}{Privacy}
\item
  \href{https://help.nytimes.com/hc/en-us/articles/115014893428-Terms-of-service}{Terms
  of Service}
\item
  \href{https://help.nytimes.com/hc/en-us/articles/115014893968-Terms-of-sale}{Terms
  of Sale}
\item
  \href{https://spiderbites.nytimes.com}{Site Map}
\item
  \href{https://help.nytimes.com/hc/en-us}{Help}
\item
  \href{https://www.nytimes.com/subscription?campaignId=37WXW}{Subscriptions}
\end{itemize}
