Sections

SEARCH

\protect\hyperlink{site-content}{Skip to
content}\protect\hyperlink{site-index}{Skip to site index}

\href{https://www.nytimes.com/section/books}{Books}

\href{https://myaccount.nytimes.com/auth/login?response_type=cookie\&client_id=vi}{}

\href{https://www.nytimes.com/section/todayspaper}{Today's Paper}

\href{/section/books}{Books}\textbar{}13 Books to Watch For in August

\url{https://nyti.ms/311HlJF}

\begin{itemize}
\item
\item
\item
\item
\item
\end{itemize}

Advertisement

\protect\hyperlink{after-top}{Continue reading the main story}

Supported by

\protect\hyperlink{after-sponsor}{Continue reading the main story}

\hypertarget{13-books-to-watch-for-in-august}{%
\section{13 Books to Watch For in
August}\label{13-books-to-watch-for-in-august}}

Stephenie Meyer's retelling of ``Twilight,'' Isabel Wilkerson's
examination of American racism, a biography of the drug kingpin El
Chapo, and plenty more.

\includegraphics{https://static01.nyt.com/images/2020/07/31/books/00AUGUST-BOOKS-COMBO/00AUGUST-BOOKS-COMBO-articleLarge.jpg?quality=75\&auto=webp\&disable=upscale}

\href{https://nytimes.com/by/joumana-khatib}{\includegraphics{https://static01.nyt.com/images/2018/09/13/multimedia/author-joumana-khatib/author-joumana-khatib-thumbLarge.png}}

By \href{https://nytimes.com/by/joumana-khatib}{Joumana Khatib}

\begin{itemize}
\item
  Published July 30, 2020Updated Aug. 3, 2020
\item
  \begin{itemize}
  \item
  \item
  \item
  \item
  \item
  \end{itemize}
\end{itemize}

August is often a quiet month for publishers, but this year there's a
lot to look forward to: new books from Akwaeke Emezi and Daisy Johnson;
a timely re-examination of William Faulkner, with a special focus on how
he wrote about race; and an astrophysicist's (surprisingly soothing)
guide to the end of the universe as we know it.

Image

\hypertarget{caste-the-origins-of-our-discontents-by-isabel-wilkerson-random-house-aug-4}{%
\subsubsection{\texorpdfstring{`\href{https://www.penguinrandomhouse.com/books/653196/caste-by-isabel-wilkerson/}{Caste:
The Origins of Our Discontents},' by Isabel Wilkerson (Random House,
Aug.
4)}{`Caste: The Origins of Our Discontents,' by Isabel Wilkerson (Random House, Aug. 4)}}\label{caste-the-origins-of-our-discontents-by-isabel-wilkerson-random-house-aug-4}}

Wilkerson, the Pulitzer Prize-winning journalist and author of
``\href{https://www.nytimes.com/2010/09/05/books/review/Oshinsky-t.html}{The
Warmth of Other Suns},'' places America's racism in a global context,
linking it to the caste system in India as well as Nazi ideology. She
identifies eight cornerstones of caste systems throughout history and
across the world, and uses vignettes from real-life people to illustrate
how inequality acts as an invisible, but deeply felt, blueprint for
their lives.

\emph{{[}}
\href{https://www.nytimes.com/2020/07/31/books/review-caste-isabel-wilkerson-origins-of-our-discontents.html}{\emph{Read
our review}}\emph{. {]}}

Image

\hypertarget{the-death-of-vivek-oji-by-akwaeke-emezi-riverhead-aug-4}{%
\subsubsection{\texorpdfstring{`\href{https://www.penguinrandomhouse.com/books/604152/the-death-of-vivek-oji-by-akwaeke-emezi/}{The
Death of Vivek Oji},' by Akwaeke Emezi (Riverhead, Aug.
4)}{`The Death of Vivek Oji,' by Akwaeke Emezi (Riverhead, Aug. 4)}}\label{the-death-of-vivek-oji-by-akwaeke-emezi-riverhead-aug-4}}

A family in southeastern Nigeria confronts what little it knew of ---
and was willing to accept about --- its son after his body is delivered
to his mother's doorstep. This mystery, by the author of
``\href{https://www.nytimes.com/2018/02/26/books/review/freshwater-akwaeke-emezi.html}{Freshwater}''
and
``\href{https://www.nytimes.com/2019/09/30/books/review/pet-akwaeke-emezi.html}{Pet},''
raises an unsettling question: How does a family mourn a young man who
was forced to hide his true self?

\emph{{[}}
\href{https://www.nytimes.com/2019/09/09/books/akwaeke-emezi-pet-freshwater.html}{\emph{Read
our profile of Emezi}}\emph{. \textbar{}}
\href{https://www.nytimes.com/2020/07/28/books/death-vivek-oji-akwaeke-emezi-group-text.html}{\emph{Read
our review}}\emph{. {]}}

Image

\hypertarget{el-jefe-the-stalking-of-chapo-guzmuxe1n-by-alan-feuer-flatiron-aug-25}{%
\subsubsection{\texorpdfstring{`\href{https://us.macmillan.com/books/9781250254528}{El
Jefe: The Stalking of Chapo Guzmán},' by Alan Feuer (Flatiron, Aug.
25)}{`El Jefe: The Stalking of Chapo Guzmán,' by Alan Feuer (Flatiron, Aug. 25)}}\label{el-jefe-the-stalking-of-chapo-guzmuxe1n-by-alan-feuer-flatiron-aug-25}}

El Chapo, the most famous drug trafficker of his generation,
\href{https://www.nytimes.com/2019/07/17/nyregion/el-chapo-sentencing.html}{received
a life sentence} last year. He had evaded the Mexican authorities for
years, smuggled hundreds of tons of drugs and became notorious for his
violence and corruption. Feuer, a Metro reporter for The New York Times
who covered El Chapo's trial, gives a brisk, rich account of the
kingpin's rise to power and his downfall.

Image

\hypertarget{the-end-of-everything-astrophysically-speaking-by-katie-mack-scribner-aug-4}{%
\subsubsection{\texorpdfstring{`\href{https://www.simonandschuster.com/books/The-End-of-Everything/Katie-Mack/9781982103545}{The
End of Everything (Astrophysically Speaking)},' by Katie Mack (Scribner,
Aug.
4)}{`The End of Everything (Astrophysically Speaking),' by Katie Mack (Scribner, Aug. 4)}}\label{the-end-of-everything-astrophysically-speaking-by-katie-mack-scribner-aug-4}}

``In about five billion years, the Sun will swell to its red giant
phase, engulf the orbit of Mercury and perhaps Venus, and leave the
Earth a charred, lifeless, magma-covered rock.'' That's how Mack, a
theoretical astrophysicist, begins her engrossing, elegant timeline of
the cosmos. Despite the book's sobering title, she sprinkles in
delightful esoterica along the way, while providing a guide to some of
the most plausible scenarios about the end of the universe.

Image

\hypertarget{evil-geniuses-the-unmaking-of-america-by-kurt-andersen-random-house-aug-11}{%
\subsubsection{\texorpdfstring{`\href{https://www.penguinrandomhouse.com/books/594493/evil-geniuses-by-kurt-andersen/}{Evil
Geniuses: The Unmaking of America},' by Kurt Andersen (Random House,
Aug.
11)}{`Evil Geniuses: The Unmaking of America,' by Kurt Andersen (Random House, Aug. 11)}}\label{evil-geniuses-the-unmaking-of-america-by-kurt-andersen-random-house-aug-11}}

Starting in the 1970s, according to Andersen, a ``cultural U-turn''
caused the nation to abandon the middle class, instead rewarding
corporate interests and capitalist greed. The United States, he writes,
might be ``the first large modern society to go from fully developed to
failing.'' But for all his grim assessments, he offers solutions
(including stronger unions and a universal basic income), and believes
change is possible --- so long as the left adopts tactics the right used
decades ago.

Image

\hypertarget{life-of-a-klansman-a-family-history-in-white-supremacy-by-edward-ball-farrar-straus--giroux-aug-4}{%
\subsubsection{\texorpdfstring{`\href{https://us.macmillan.com/books/9780374186326}{Life
of a Klansman: A Family History in White Supremacy},' by Edward Ball
(Farrar, Straus \& Giroux, Aug.
4)}{`Life of a Klansman: A Family History in White Supremacy,' by Edward Ball (Farrar, Straus \& Giroux, Aug. 4)}}\label{life-of-a-klansman-a-family-history-in-white-supremacy-by-edward-ball-farrar-straus--giroux-aug-4}}

In his National Book Award-winning book
``\href{https://archive.nytimes.com/www.nytimes.com/books/98/03/01/reviews/980301.01faustt.html}{Slaves
in the Family},'' Ball tracked down descendants of the people that his
ancestors, plantation owners in South Carolina, had enslaved. Now, he
returns again to his family, focusing on one of his
great-great-grandfathers and his association with the Ku Klux Klan.

Image

\hypertarget{luster-by-raven-leilani-farrar-straus--giroux-aug-4}{%
\subsubsection{\texorpdfstring{`\href{https://us.macmillan.com/books/9780374910334}{Luster},'
by Raven Leilani (Farrar, Straus \& Giroux, Aug.
4)}{`Luster,' by Raven Leilani (Farrar, Straus \& Giroux, Aug. 4)}}\label{luster-by-raven-leilani-farrar-straus--giroux-aug-4}}

Edie is a Black woman in her 20s, an artist in Bushwick who is
unfulfilled by virtually every part of her life. When she begins dating
a white man in an open marriage, she becomes entangled in his family's
life --- emotionally, physically and even economically.

\emph{{[}}
\href{https://www.nytimes.com/2020/07/31/books/raven-leilani-luster.html}{\emph{Read
our profile of Leilani}}\emph{. {]}}

Image

\hypertarget{midnight-sun-by-stephenie-meyer-little-brown-aug-4}{%
\subsubsection{\texorpdfstring{`\href{https://www.hachettebookgroup.com/titles/stephenie-meyer/midnight-sun/9780316592253/}{Midnight
Sun},' by Stephenie Meyer (Little, Brown, Aug.
4)}{`Midnight Sun,' by Stephenie Meyer (Little, Brown, Aug. 4)}}\label{midnight-sun-by-stephenie-meyer-little-brown-aug-4}}

It's been almost 15 years since Meyer published ``Twilight,'' the
best-selling young-adult vampire novel that sparked a worldwide interest
in paranormal romance. Now she returns to the story of Edward Cullen and
Bella Swan, but this time, she tells it from his point of view.

\emph{{[}}
\href{https://www.nytimes.com/2020/08/03/books/midnight-sun-stephenie-meyer-twilight.html}{\emph{Read
our Q.\&A. with Meyer}}\emph{. {]}}

Image

\hypertarget{reaganland-americas-right-turn-1976-1980-by-rick-perlstein-simon--schuster-aug-18}{%
\subsubsection{\texorpdfstring{`\href{https://www.simonandschuster.com/books/Reaganland/Rick-Perlstein/9781476793054}{Reaganland:
America's Right Turn 1976-1980},' by Rick Perlstein (Simon \& Schuster,
Aug.
18)}{`Reaganland: America's Right Turn 1976-1980,' by Rick Perlstein (Simon \& Schuster, Aug. 18)}}\label{reaganland-americas-right-turn-1976-1980-by-rick-perlstein-simon--schuster-aug-18}}

By 1976, Ronald Reagan's political career appeared to be over. In
Perlstein's new book, the final volume of his series charting the
ascendancy of the right in America, he traces Reagan's political
comeback and how he reinvigorated the Republican Party's base with his
pledge to ``Make America Great Again.'' Perlstein, an engaging
storyteller, offers a clear guide to the intellectual and ideological
debates of the time.

Image

\hypertarget{the-saddest-words-william-faulkners-civil-war-by-michael-gorra-liveright-aug-25}{%
\subsubsection{\texorpdfstring{`\href{https://wwnorton.com/books/9781631491702}{The
Saddest Words: William Faulkner's Civil War},' by Michael Gorra
(Liveright, Aug.
25)}{`The Saddest Words: William Faulkner's Civil War,' by Michael Gorra (Liveright, Aug. 25)}}\label{the-saddest-words-william-faulkners-civil-war-by-michael-gorra-liveright-aug-25}}

Faulkner's enduring, ubiquitous quote that ``the past is never dead''
might be a fitting epitaph for this new book. In this timely
re-examination, Gorra considers how Faulkner should be read in the 21st
century, with a focus on the depiction of Black people and racism in his
fiction.

Image

\hypertarget{sisters-by-daisy-johnson-riverhead-aug-25}{%
\subsubsection{\texorpdfstring{`\href{https://www.penguinrandomhouse.com/books/624960/sisters-by-daisy-johnson/}{Sisters},'
by Daisy Johnson (Riverhead, Aug.
25)}{`Sisters,' by Daisy Johnson (Riverhead, Aug. 25)}}\label{sisters-by-daisy-johnson-riverhead-aug-25}}

With her debut novel
``\href{https://www.nytimes.com/2018/11/20/books/review/daisy-johnson-everything-under.html}{Everything
Under},'' Johnson became the youngest author shortlisted for the Booker
Prize. Her new book focuses on two teenage sisters, July and September,
who arrive with their mother at a desolate house on the eastern coast of
England after leaving school for reasons that aren't entirely clear. The
sisters are fiercely close, less than a year apart in age, but their
relationship takes on a sinister tone over time. As their circumstances
and past come into focus, Johnson delivers a shocking twist.

Image

\hypertarget{summer-by-ali-smith-pantheon-aug-25}{%
\subsubsection{\texorpdfstring{`\href{https://www.penguinrandomhouse.com/books/259057/summer-by-ali-smith/}{Summer},'
by Ali Smith (Pantheon, Aug.
25)}{`Summer,' by Ali Smith (Pantheon, Aug. 25)}}\label{summer-by-ali-smith-pantheon-aug-25}}

The final volume in Smith's seasonal quartet, set during the coronavirus
pandemic, centers on Sacha and Robert, two siblings grappling with the
awakening of their political and cultural consciousness. As she did in
the series' earlier books, Smith balances timely real-life issues
(Brexit, the refugee crisis, Trump) with her characters' inner lives ---
and characters from the previous books reappear, too.

Image

\hypertarget{vesper-flights-by-helen-macdonald-grove-atlantic-aug-25}{%
\subsubsection{\texorpdfstring{`\href{https://groveatlantic.com/book/vesper-flights/}{Vesper
Flights},' by Helen Macdonald (Grove Atlantic, Aug.
25)}{`Vesper Flights,' by Helen Macdonald (Grove Atlantic, Aug. 25)}}\label{vesper-flights-by-helen-macdonald-grove-atlantic-aug-25}}

Macdonald's debut,
``\href{https://www.nytimes.com/2015/02/22/books/review/helen-macdonalds-h-is-for-hawk.html}{H
Is for Hawk}'' --- about grappling with the death of her beloved father
by training a goshawk --- was one of the
\href{https://www.nytimes.com/interactive/2015/12/02/books/review/best-books-of-2015.html}{Book
Review's 10 best books of 2015}. Now, Macdonald returns with a
collection of essays, new and previously published, about the natural
world.

\emph{Follow New York Times Books on}
\href{https://www.facebook.com/nytbooks/}{\emph{Facebook}}\emph{,}
\href{https://twitter.com/nytimesbooks}{\emph{Twitter}} \emph{and}
\href{https://www.instagram.com/nytbooks/}{\emph{Instagram}}\emph{, sign
up for}
\href{https://www.nytimes.com/newsletters/books-review}{\emph{our
newsletter}} \emph{or}
\href{https://www.nytimes.com/interactive/2017/books/books-calendar.html}{\emph{our
literary calendar}}\emph{. And listen to us on the}
\href{https://www.nytimes.com/column/book-review-podcast}{\emph{Book
Review podcast}}\emph{.}

Advertisement

\protect\hyperlink{after-bottom}{Continue reading the main story}

\hypertarget{site-index}{%
\subsection{Site Index}\label{site-index}}

\hypertarget{site-information-navigation}{%
\subsection{Site Information
Navigation}\label{site-information-navigation}}

\begin{itemize}
\tightlist
\item
  \href{https://help.nytimes.com/hc/en-us/articles/115014792127-Copyright-notice}{©~2020~The
  New York Times Company}
\end{itemize}

\begin{itemize}
\tightlist
\item
  \href{https://www.nytco.com/}{NYTCo}
\item
  \href{https://help.nytimes.com/hc/en-us/articles/115015385887-Contact-Us}{Contact
  Us}
\item
  \href{https://www.nytco.com/careers/}{Work with us}
\item
  \href{https://nytmediakit.com/}{Advertise}
\item
  \href{http://www.tbrandstudio.com/}{T Brand Studio}
\item
  \href{https://www.nytimes.com/privacy/cookie-policy\#how-do-i-manage-trackers}{Your
  Ad Choices}
\item
  \href{https://www.nytimes.com/privacy}{Privacy}
\item
  \href{https://help.nytimes.com/hc/en-us/articles/115014893428-Terms-of-service}{Terms
  of Service}
\item
  \href{https://help.nytimes.com/hc/en-us/articles/115014893968-Terms-of-sale}{Terms
  of Sale}
\item
  \href{https://spiderbites.nytimes.com}{Site Map}
\item
  \href{https://help.nytimes.com/hc/en-us}{Help}
\item
  \href{https://www.nytimes.com/subscription?campaignId=37WXW}{Subscriptions}
\end{itemize}
