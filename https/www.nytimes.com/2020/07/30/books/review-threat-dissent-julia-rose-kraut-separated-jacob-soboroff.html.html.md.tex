Sections

SEARCH

\protect\hyperlink{site-content}{Skip to
content}\protect\hyperlink{site-index}{Skip to site index}

\href{https://www.nytimes.com/section/books}{Books}

\href{https://myaccount.nytimes.com/auth/login?response_type=cookie\&client_id=vi}{}

\href{https://www.nytimes.com/section/todayspaper}{Today's Paper}

\href{/section/books}{Books}\textbar{}New Looks at the Fate of
Foreigners in America, From the Privileged to the Most Vulnerable

\url{https://nyti.ms/2P6WHHj}

\begin{itemize}
\item
\item
\item
\item
\item
\end{itemize}

Advertisement

\protect\hyperlink{after-top}{Continue reading the main story}

Supported by

\protect\hyperlink{after-sponsor}{Continue reading the main story}

\href{/column/books-of-the-times}{Books of The Times}

\hypertarget{new-looks-at-the-fate-of-foreigners-in-america-from-the-privileged-to-the-most-vulnerable}{%
\section{New Looks at the Fate of Foreigners in America, From the
Privileged to the Most
Vulnerable}\label{new-looks-at-the-fate-of-foreigners-in-america-from-the-privileged-to-the-most-vulnerable}}

By \href{https://www.nytimes.com/by/jennifer-szalai}{Jennifer Szalai}

\begin{itemize}
\item
  July 30, 2020
\item
  \begin{itemize}
  \item
  \item
  \item
  \item
  \item
  \end{itemize}
\end{itemize}

\includegraphics{https://static01.nyt.com/images/2020/08/05/books/05BOOKSOBOROFF-KRAUT2/05BOOKSOBOROFF-KRAUT2-articleLarge.png?quality=75\&auto=webp\&disable=upscale}

Amid all the fickle reversals of the last four years, there's one area
where the Trump administration has demonstrated a steady and unrelenting
focus: restricting immigration.

There is, of course, the wall --- whose mythos looms large, even if the
actual structure is less impressive (and less effective) than the
president lets on. But if you think of the wall as not so much a
physical deterrent to migrants as a symbolic monument to nativist
ambitions, President Trump's implacable devotion to it begins to make
sense. The one discernible principle that seems to animate his
policymaking has been an unwavering determination to keep foreigners
out. Barely two weeks after his inauguration in 2017, he famously
announced his travel ban --- an executive order
\href{https://www.nytimes.com/2017/02/05/us/politics/donald-trump-mike-pence-travel-ban-judge.html}{so
hastily put together} that it would undergo numerous challenges and
iterations for more than a year until a version of it was upheld with
\href{https://www.nytimes.com/2018/06/26/us/politics/supreme-court-trump-travel-ban.html}{a
5-4 ruling by the Supreme Court}.

Image

Credit....

Visitors aren't immigrants, but their treatment is connected. As the
lawyer and historian Julia Rose Kraut recounts in her new book, ``Threat
of Dissent,'' there's a long history of foreigners in the United States
being subjected to the vicissitudes of the government's discretionary
powers. Another recent book, ``Separated,'' by the MSNBC and NBC News
correspondent
\href{https://www.nytimes.com/2020/07/23/books/review/separated-jacob-soboroff.html}{Jacob
Soboroff}, shows how the Trump administration implemented a policy that
amounted to a humanitarian catastrophe: systematically taking children
from their migrant parents at the border. Reading these two books
together will give you a sense of how the United States, a country that
prides itself on its constitutional protections, also possesses a body
of immigration laws that can be weaponized by an executive branch
willing to do it.

In ``Threat of Dissent,'' Kraut writes about what she calls
``ideological exclusion'' --- the effort to block and even deport
noncitizens because of their ideas and beliefs. Suspicion of foreigners
goes back to the earliest days of the republic. The Alien Friends Act of
1798 allowed the president to detain and deport any noncitizen deemed
``dangerous to the peace and safety of the United States,'' which at the
time was in an undeclared naval war with France.

President John Adams used the Alien Friends Act as an opportunity to
refuse entry to a visiting delegation of scholars (``We have had too
many French philosophers already''), and to draw up a list of Frenchmen
to be deported. One of them had fled the Reign of Terror years before
and settled in Philadelphia, opening a bookshop whose customers included
Adams himself. Asked why this bookseller was on the list, Adams replied:
``Nothing in particular, but he's too French.''

Image

Julia Rose Kraut, author of ``Threat of Dissent.''Credit...Britney Young

Kraut traces how different ideologies would be considered intolerably
dangerous according to the dominant fears of a given era. Anarchism gave
way to communism; communism gave way to Islamic radicalism. Foreigners
suspected of unacceptable anti-Americanism included Charlie Chaplin and
Graham Greene. (Chaplin was so incensed by the ritual humiliations he
had to endure at the hands of immigration authorities that after leaving
the United States for a European tour he decided not to return.) Even
citizenship didn't always ensure protection. The anarchist Emma Goldman
was denaturalized in 1909, and shipped to Revolutionary Russia a decade
later.

More recently, in 2019, a 17-year-old Palestinian from Lebanon who was
about to begin his freshman year at Harvard was denied entry at the
Boston airport; border patrol agents searched his phone and laptop and
told him he was ``inadmissible'' because of social media posts --- not
by him, but by his friends.

The foreigners in Kraut's book generally constitute a privileged class
--- scholars, writers and artists whose ideas (or mere proximity to
ideas) have been used against them. The foreigners in Soboroff's book,
by stark contrast, are among the most vulnerable, persecuted for their
presence alone. In ``Separated,'' he describes traveling along the
southern border during the early part of the Trump presidency to report
on tightening immigration enforcement. All the while, a more horrifying
story was starting to take shape.

Image

Jacob Soboroff, author of ``Separated: Inside an American
Tragedy.''Credit...Art Streiber for MSNBC

By the time the homeland security secretary Kirstjen Nielsen put her
signature to an official policy of family separation in May 2018, border
agents had already been separating asylum seekers from their children
\href{https://www.nytimes.com/2018/04/20/us/immigrant-children-separation-ice.html}{since
the previous year}. This punishment of migrant families was compounded
by a process that Soboroff describes as either willfully cruel or
cruelly negligent. Record-keeping was so shoddy and inadequate that
authorities didn't properly keep track of which child belonged to whom,
\href{https://www.newyorker.com/news/news-desk/a-new-report-on-family-separations-shows-the-depths-of-trumps-negligence}{making
reunification for some families exceedingly complicated, if not
impossible}.

The statistics that do exist are startling: Since the summer of 2017,
Soboroff writes, at least 5,556 children were taken from their parents
--- the true number is still unknown. The head of the American Academy
of Pediatrics called family separation ``government-sanctioned child
abuse''; the nonprofit Physicians for Human Rights called it
``torture.'' Even though the policy was officially ended after a public
outcry in the summer of 2018,
\href{https://www.nytimes.com/2019/07/30/us/migrant-family-separations.html}{separations
continued}. Migrant parents who are detained at the border with their
children
\href{https://www.nbcnews.com/politics/immigration/despite-judge-s-order-migrant-children-remain-detained-amid-covid-n1234705}{have
again been presented with an impossible choice}: consent to their
children being released without them, or stay together in indefinite
detention.

``Separated'' is structured chronologically, with the narrative of
Soboroff's own discovery of what was happening presented incrementally,
highlighting the secrecy and ``extraordinary confusion'' of the process
--- and how removed even a journalist like Soboroff was from what was
happening on the ground. He also recounts the story of Juan and José, a
father and son who fled narco-traffickers in Guatemala in the summer of
2018. José, 14 at the time, was taken from his father at the border.
Juan and José would endure 124 days of separation before they were
reunited. There was no information about José in his father's case file,
and it would take a social worker to track the father down to a
detention facility located 1,500 miles from where the son was being
held.

``Nobody warned of the impact on children,'' one anonymous official told
Soboroff. Given that family separation was adopted as a merciless form
of deterrence, this excuse makes no sense. The entire policy was
predicated on how traumatic that ``impact'' promised to be. The subtitle
of ``Separated'' is ``Inside an American Tragedy,'' but what Soboroff
memorably depicts isn't just tragic but brutal. Any soaring rhetoric
about yearning to breathe free has been traded in for the crudest of
threats: If you try to come here, just look at what we're willing to do.

Advertisement

\protect\hyperlink{after-bottom}{Continue reading the main story}

\hypertarget{site-index}{%
\subsection{Site Index}\label{site-index}}

\hypertarget{site-information-navigation}{%
\subsection{Site Information
Navigation}\label{site-information-navigation}}

\begin{itemize}
\tightlist
\item
  \href{https://help.nytimes.com/hc/en-us/articles/115014792127-Copyright-notice}{©~2020~The
  New York Times Company}
\end{itemize}

\begin{itemize}
\tightlist
\item
  \href{https://www.nytco.com/}{NYTCo}
\item
  \href{https://help.nytimes.com/hc/en-us/articles/115015385887-Contact-Us}{Contact
  Us}
\item
  \href{https://www.nytco.com/careers/}{Work with us}
\item
  \href{https://nytmediakit.com/}{Advertise}
\item
  \href{http://www.tbrandstudio.com/}{T Brand Studio}
\item
  \href{https://www.nytimes.com/privacy/cookie-policy\#how-do-i-manage-trackers}{Your
  Ad Choices}
\item
  \href{https://www.nytimes.com/privacy}{Privacy}
\item
  \href{https://help.nytimes.com/hc/en-us/articles/115014893428-Terms-of-service}{Terms
  of Service}
\item
  \href{https://help.nytimes.com/hc/en-us/articles/115014893968-Terms-of-sale}{Terms
  of Sale}
\item
  \href{https://spiderbites.nytimes.com}{Site Map}
\item
  \href{https://help.nytimes.com/hc/en-us}{Help}
\item
  \href{https://www.nytimes.com/subscription?campaignId=37WXW}{Subscriptions}
\end{itemize}
