\href{/section/health}{Health}\textbar{}A Covid Patient Goes Home After
a Rare Double Lung Transplant

\url{https://nyti.ms/2CSlZ9O}

\begin{itemize}
\item
\item
\item
\item
\item
\item
\end{itemize}

\href{https://www.nytimes.com/news-event/coronavirus?action=click\&pgtype=Article\&state=default\&region=TOP_BANNER\&context=storylines_menu}{The
Coronavirus Outbreak}

\begin{itemize}
\tightlist
\item
  live\href{https://www.nytimes.com/2020/08/01/world/coronavirus-covid-19.html?action=click\&pgtype=Article\&state=default\&region=TOP_BANNER\&context=storylines_menu}{Latest
  Updates}
\item
  \href{https://www.nytimes.com/interactive/2020/us/coronavirus-us-cases.html?action=click\&pgtype=Article\&state=default\&region=TOP_BANNER\&context=storylines_menu}{Maps
  and Cases}
\item
  \href{https://www.nytimes.com/interactive/2020/science/coronavirus-vaccine-tracker.html?action=click\&pgtype=Article\&state=default\&region=TOP_BANNER\&context=storylines_menu}{Vaccine
  Tracker}
\item
  \href{https://www.nytimes.com/interactive/2020/07/29/us/schools-reopening-coronavirus.html?action=click\&pgtype=Article\&state=default\&region=TOP_BANNER\&context=storylines_menu}{What
  School May Look Like}
\item
  \href{https://www.nytimes.com/live/2020/07/31/business/stock-market-today-coronavirus?action=click\&pgtype=Article\&state=default\&region=TOP_BANNER\&context=storylines_menu}{Economy}
\end{itemize}

\includegraphics{https://static01.nyt.com/images/2020/08/02/science/02Virus-Lungtransplant1/merlin_175086177_1077d762-30a6-4bd9-84be-71fea068258a-articleLarge.jpg?quality=75\&auto=webp\&disable=upscale}

Sections

\protect\hyperlink{site-content}{Skip to
content}\protect\hyperlink{site-index}{Skip to site index}

\hypertarget{a-covid-patient-goes-home-after-a-rare-double-lung-transplant}{%
\section{A Covid Patient Goes Home After a Rare Double Lung
Transplant}\label{a-covid-patient-goes-home-after-a-rare-double-lung-transplant}}

Mayra Ramirez was the first of a small but growing number of people
whose only hope of surviving the coronavirus was to replace their lungs.

Mayra Ramirez of Chicago, recipient of the first double-lung transplant.
``I want you to show the scars,'' she said. ``I want people to see what
this virus does to a person.''Credit...Sebastian Hidalgo for The New
York Times

Supported by

\protect\hyperlink{after-sponsor}{Continue reading the main story}

By \href{https://www.nytimes.com/by/denise-grady}{Denise Grady}

\begin{itemize}
\item
  July 30, 2020
\item
  \begin{itemize}
  \item
  \item
  \item
  \item
  \item
  \item
  \end{itemize}
\end{itemize}

The last thing that Mayra Ramirez remembers from the emergency room at
Northwestern Memorial Hospital in Chicago is calling her family to say
she had Covid, was about to be put on a ventilator and needed her mother
to make medical decisions for her.

Ms. Ramirez, 28, did not wake up for more than six weeks. And then she
learned that on June 5,
\href{https://www.nytimes.com/2020/06/11/health/coronavirus-lung-transplant.html?searchResultPosition=2}{she
had become the first Covid patient} in the United States to receive a
double-lung transplant.

On Wednesday, she went home from the hospital.

Ms. Ramirez is one of a small but growing number of patients whose lungs
have been destroyed by the coronavirus, and whose only hope of survival
is a lung transplant.

``I'm pretty sure that if I had been at another center, they would have
just ended care and let me die,'' she said in an interview on Wednesday.

The surgery is considered a desperate measure reserved for people with
fatal, irreversible lung damage. Doctors do not want to remove a
person's lungs if there is any chance they will heal. Over all, only
\href{https://unos.org/data/transplant-trends/}{about 2,700 lung
transplants} were performed in the United States last year.

Patients must be sick enough to need a transplant, and yet also strong
enough to survive the operation, recover and get back on their feet.
With a new disease like Covid-19, doctors are still learning how to
strike that balance.

``It's such a paradigm change,'' said Ms. Ramirez's surgeon, Dr. Ankit
Bharat. ``Lung transplant has not been considered a treatment option for
an infectious disease, so people need to get a little bit more of a
comfort level with it.''

On July 5, he performed a similar operation on a second Covid patient,
Brian Kuhns, 62, from Lake Zurich, Ill.

Mr. Kuhns spent 100 days on life-support machines before receiving the
transplant. Before becoming ill, he had thought Covid was a hoax, his
wife, Nancy Kuhns, said, in a statement issued by the hospital.

Mr. Kuhns said, ``If my story can teach you one thing, it's that
Covid-19 isn't a joke.''

Two more patients at Northwestern are awaiting transplants, one from
Chicago and one from Washington, D.C., said Dr. Bharat, who is the chief
of thoracic surgery and surgical director of the lung transplant
program.

\hypertarget{latest-updates-global-coronavirus-outbreak}{%
\section{\texorpdfstring{\href{https://www.nytimes.com/2020/08/01/world/coronavirus-covid-19.html?action=click\&pgtype=Article\&state=default\&region=MAIN_CONTENT_1\&context=storylines_live_updates}{Latest
Updates: Global Coronavirus
Outbreak}}{Latest Updates: Global Coronavirus Outbreak}}\label{latest-updates-global-coronavirus-outbreak}}

Updated 2020-08-02T06:51:12.403Z

\begin{itemize}
\tightlist
\item
  \href{https://www.nytimes.com/2020/08/01/world/coronavirus-covid-19.html?action=click\&pgtype=Article\&state=default\&region=MAIN_CONTENT_1\&context=storylines_live_updates\#link-34047410}{The
  U.S. reels as July cases more than double the total of any other
  month.}
\item
  \href{https://www.nytimes.com/2020/08/01/world/coronavirus-covid-19.html?action=click\&pgtype=Article\&state=default\&region=MAIN_CONTENT_1\&context=storylines_live_updates\#link-780ec966}{Top
  U.S. officials work to break an impasse over the federal jobless
  benefit.}
\item
  \href{https://www.nytimes.com/2020/08/01/world/coronavirus-covid-19.html?action=click\&pgtype=Article\&state=default\&region=MAIN_CONTENT_1\&context=storylines_live_updates\#link-2bc8948}{Its
  outbreak untamed, Melbourne goes into even greater lockdown.}
\end{itemize}

\href{https://www.nytimes.com/2020/08/01/world/coronavirus-covid-19.html?action=click\&pgtype=Article\&state=default\&region=MAIN_CONTENT_1\&context=storylines_live_updates}{See
more updates}

More live coverage:
\href{https://www.nytimes.com/live/2020/07/31/business/stock-market-today-coronavirus?action=click\&pgtype=Article\&state=default\&region=MAIN_CONTENT_1\&context=storylines_live_updates}{Markets}

A patient is to be flown in from Seattle next week, and the Northwestern
team is consulting on still another case with a medical group in
Washington, D.C. Other transplant centers are considering similar
surgeries, Dr. Bharat said.

Last Friday, a Covid patient transferred from another state underwent a
double lung transplant at the University of Florida Health Shands
Hospital in Gainesville, Dr. Tiago Machuca said.

While other centers have also sought to refer cases, most of the
patients had other serious medical problems that ruled them out, he
said.

\includegraphics{https://static01.nyt.com/images/2020/07/30/science/30VIRUS-LUNGTRANSPLANT2/merlin_175086201_3fc4747b-11e7-47c2-a10c-fc2d47cda571-articleLarge.jpg?quality=75\&auto=webp\&disable=upscale}

In some cases, Dr. Bharat said, hospitals appeared to have waited too
long to recommend a transplant. One patient being referred to his center
seemed like a good candidate but then had major bleeding into the lungs
as well as kidney failure, and the surgery was no longer feasible.

``I think people need to recognize this option earlier and just start at
least talking about it before it gets to that point,'' Dr. Bharat said.

In some cases, he said, insurers' reluctance to cover the surgery or to
pay for travel to transfer patients has led to delays.

``This is so new to our field,'' Dr. Machuca said. ``It will be a
challenge for physicians to determine which patients truly are
candidates and what's the timing. We don't want to do it too early when
the patient still can recover from Covid lung disease and resume with
good quality of life, but also you don't want to miss the boat and have
a patient where it's futile, the patient is too sick.''

He said that, in some cases, extensive rehabilitation has brought about
recovery in Covid patients who were being considered as possible
transplant candidates.

Because the extensive lung damage in Covid patients makes transplant
surgery especially difficult, most patients would be referred to major
transplant centers that are best equipped to perform the risky
operations and provide the intensive aftercare that patients need, the
surgeons said. Mr. Kuhns was transferred to Northwestern from another
health system.

Image

Ms. Ramirez wasn't told for several days after her surgery that she had
had a double lung transplant.

Image

Ms. Ramirez with her dog, Molly Monster, working from home on April 10,
shortly before she got ill.

Before she became ill, Ms. Ramirez, a paralegal for a law firm
specializing in immigration, was working from home and having her
groceries delivered. She was in good health, but had an autoimmune
condition, neuromyelitis optica, and took medication that suppressed her
immune system and might have made her more vulnerable to the coronavirus
infection.

She was ill for about two weeks, and consulted with a Covid hotline
about her symptoms. At one point, she headed to the hospital but then
turned back without going in. She dreaded the idea of being admitted,
and told herself she would recover.

But on April 26, her temperature reached 105 degrees Fahrenheit, and she
was so weak that she fell when she tried to walk. A friend drove her to
the hospital. When doctors told her that she needed a ventilator, she
had no idea what they meant. She thought it meant some kind of fan, like
the word in Spanish.

``I thought I'd just be there for a couple of days, max, and get back to
my normal life,'' she said.

But she spent six weeks on the ventilator, and also needed a machine to
provide oxygen directly into her bloodstream.

\href{https://www.nytimes.com/news-event/coronavirus?action=click\&pgtype=Article\&state=default\&region=MAIN_CONTENT_3\&context=storylines_faq}{}

\hypertarget{the-coronavirus-outbreak-}{%
\subsubsection{The Coronavirus Outbreak
›}\label{the-coronavirus-outbreak-}}

\hypertarget{frequently-asked-questions}{%
\paragraph{Frequently Asked
Questions}\label{frequently-asked-questions}}

Updated July 27, 2020

\begin{itemize}
\item ~
  \hypertarget{should-i-refinance-my-mortgage}{%
  \paragraph{Should I refinance my
  mortgage?}\label{should-i-refinance-my-mortgage}}

  \begin{itemize}
  \tightlist
  \item
    \href{https://www.nytimes.com/article/coronavirus-money-unemployment.html?action=click\&pgtype=Article\&state=default\&region=MAIN_CONTENT_3\&context=storylines_faq}{It
    could be a good idea,} because mortgage rates have
    \href{https://www.nytimes.com/2020/07/16/business/mortgage-rates-below-3-percent.html?action=click\&pgtype=Article\&state=default\&region=MAIN_CONTENT_3\&context=storylines_faq}{never
    been lower.} Refinancing requests have pushed mortgage applications
    to some of the highest levels since 2008, so be prepared to get in
    line. But defaults are also up, so if you're thinking about buying a
    home, be aware that some lenders have tightened their standards.
  \end{itemize}
\item ~
  \hypertarget{what-is-school-going-to-look-like-in-september}{%
  \paragraph{What is school going to look like in
  September?}\label{what-is-school-going-to-look-like-in-september}}

  \begin{itemize}
  \tightlist
  \item
    It is unlikely that many schools will return to a normal schedule
    this fall, requiring the grind of
    \href{https://www.nytimes.com/2020/06/05/us/coronavirus-education-lost-learning.html?action=click\&pgtype=Article\&state=default\&region=MAIN_CONTENT_3\&context=storylines_faq}{online
    learning},
    \href{https://www.nytimes.com/2020/05/29/us/coronavirus-child-care-centers.html?action=click\&pgtype=Article\&state=default\&region=MAIN_CONTENT_3\&context=storylines_faq}{makeshift
    child care} and
    \href{https://www.nytimes.com/2020/06/03/business/economy/coronavirus-working-women.html?action=click\&pgtype=Article\&state=default\&region=MAIN_CONTENT_3\&context=storylines_faq}{stunted
    workdays} to continue. California's two largest public school
    districts --- Los Angeles and San Diego --- said on July 13, that
    \href{https://www.nytimes.com/2020/07/13/us/lausd-san-diego-school-reopening.html?action=click\&pgtype=Article\&state=default\&region=MAIN_CONTENT_3\&context=storylines_faq}{instruction
    will be remote-only in the fall}, citing concerns that surging
    coronavirus infections in their areas pose too dire a risk for
    students and teachers. Together, the two districts enroll some
    825,000 students. They are the largest in the country so far to
    abandon plans for even a partial physical return to classrooms when
    they reopen in August. For other districts, the solution won't be an
    all-or-nothing approach.
    \href{https://bioethics.jhu.edu/research-and-outreach/projects/eschool-initiative/school-policy-tracker/}{Many
    systems}, including the nation's largest, New York City, are
    devising
    \href{https://www.nytimes.com/2020/06/26/us/coronavirus-schools-reopen-fall.html?action=click\&pgtype=Article\&state=default\&region=MAIN_CONTENT_3\&context=storylines_faq}{hybrid
    plans} that involve spending some days in classrooms and other days
    online. There's no national policy on this yet, so check with your
    municipal school system regularly to see what is happening in your
    community.
  \end{itemize}
\item ~
  \hypertarget{is-the-coronavirus-airborne}{%
  \paragraph{Is the coronavirus
  airborne?}\label{is-the-coronavirus-airborne}}

  \begin{itemize}
  \tightlist
  \item
    The coronavirus
    \href{https://www.nytimes.com/2020/07/04/health/239-experts-with-one-big-claim-the-coronavirus-is-airborne.html?action=click\&pgtype=Article\&state=default\&region=MAIN_CONTENT_3\&context=storylines_faq}{can
    stay aloft for hours in tiny droplets in stagnant air}, infecting
    people as they inhale, mounting scientific evidence suggests. This
    risk is highest in crowded indoor spaces with poor ventilation, and
    may help explain super-spreading events reported in meatpacking
    plants, churches and restaurants.
    \href{https://www.nytimes.com/2020/07/06/health/coronavirus-airborne-aerosols.html?action=click\&pgtype=Article\&state=default\&region=MAIN_CONTENT_3\&context=storylines_faq}{It's
    unclear how often the virus is spread} via these tiny droplets, or
    aerosols, compared with larger droplets that are expelled when a
    sick person coughs or sneezes, or transmitted through contact with
    contaminated surfaces, said Linsey Marr, an aerosol expert at
    Virginia Tech. Aerosols are released even when a person without
    symptoms exhales, talks or sings, according to Dr. Marr and more
    than 200 other experts, who
    \href{https://academic.oup.com/cid/article/doi/10.1093/cid/ciaa939/5867798}{have
    outlined the evidence in an open letter to the World Health
    Organization}.
  \end{itemize}
\item ~
  \hypertarget{what-are-the-symptoms-of-coronavirus}{%
  \paragraph{What are the symptoms of
  coronavirus?}\label{what-are-the-symptoms-of-coronavirus}}

  \begin{itemize}
  \tightlist
  \item
    Common symptoms
    \href{https://www.nytimes.com/article/symptoms-coronavirus.html?action=click\&pgtype=Article\&state=default\&region=MAIN_CONTENT_3\&context=storylines_faq}{include
    fever, a dry cough, fatigue and difficulty breathing or shortness of
    breath.} Some of these symptoms overlap with those of the flu,
    making detection difficult, but runny noses and stuffy sinuses are
    less common.
    \href{https://www.nytimes.com/2020/04/27/health/coronavirus-symptoms-cdc.html?action=click\&pgtype=Article\&state=default\&region=MAIN_CONTENT_3\&context=storylines_faq}{The
    C.D.C. has also} added chills, muscle pain, sore throat, headache
    and a new loss of the sense of taste or smell as symptoms to look
    out for. Most people fall ill five to seven days after exposure, but
    symptoms may appear in as few as two days or as many as 14 days.
  \end{itemize}
\item ~
  \hypertarget{does-asymptomatic-transmission-of-covid-19-happen}{%
  \paragraph{Does asymptomatic transmission of Covid-19
  happen?}\label{does-asymptomatic-transmission-of-covid-19-happen}}

  \begin{itemize}
  \tightlist
  \item
    So far, the evidence seems to show it does. A widely cited
    \href{https://www.nature.com/articles/s41591-020-0869-5}{paper}
    published in April suggests that people are most infectious about
    two days before the onset of coronavirus symptoms and estimated that
    44 percent of new infections were a result of transmission from
    people who were not yet showing symptoms. Recently, a top expert at
    the World Health Organization stated that transmission of the
    coronavirus by people who did not have symptoms was ``very rare,''
    \href{https://www.nytimes.com/2020/06/09/world/coronavirus-updates.html?action=click\&pgtype=Article\&state=default\&region=MAIN_CONTENT_3\&context=storylines_faq\#link-1f302e21}{but
    she later walked back that statement.}
  \end{itemize}
\end{itemize}

``The entire time, I had nightmares,'' she said.

Many of the nightmares involved drowning, her family saying goodbye, the
doctors telling her she was going to die.

The disease was relentless. Bacterial infections set in, scarring her
lungs and eating holes in them. The lung damage caused circulatory
problems that began to take a toll on her liver and heart.

The doctors told her family in North Carolina that it might be time to
come to Chicago to say goodbye, and her mother and two sisters made the
trip.

But Ms. Ramirez held on, cleared the coronavirus from her body and was
placed on the transplant list. Two days later, on June 5, she underwent
a grueling, 10-hour operation.

She woke scarred, bruised, desperately thirsty and unable to speak,
``with all these tubes coming out of me, and I just couldn't recognize
my own body.''

The nurses asked if she knew the date. She guessed early May. It was the
middle of June.

She was not told she'd had a lung transplant until several days after
she woke up.

``I couldn't process it,'' she said. ``I was just struggling to breathe
and I was thirsty. It wasn't until weeks later that I could be grateful,
and think there was a family out there who had lost someone.''

Image

Ms. Ramirez in the coronavirus I.C.U. at Northwestern Medicine in
May.Credit...Northwestern Medicine

Because of concerns about infection, her family could not visit after
the surgery. At a news conference on Thursday, Ms. Ramirez said, ``The
hardest part was going through this alone.''

She suffered from anxiety and panic attacks, she said. Eventually, the
rules were relaxed, and her mother could visit. But it was wrenching to
say goodbye each day.

Before her illness, she worked full-time and enjoyed running and playing
with her two small, scrappy dogs. Now, she still feels short of breath,
can walk only a short distance and needs help to shower and stand up
from a chair. The dogs were overjoyed at her homecoming, but their
energy was a bit much. Her mother, who lives in North Carolina, took
time away from her job at a meatpacking plant and traveled to Chicago to
help her recover.

Ms. Ramirez said she was learning to use her new lungs and getting
stronger every day.

She is looking forward to getting back to work, but she still has a way
to go. Her family is assisting her, and a friend started a
\href{https://www.gofundme.com/f/covid19-lung-transplant?utm_source=customer\&utm_campaign=p_cp+share-sheet\&utm_medium=copy_link-tip}{GoFundMe
page} to help pay the bills.

``I definitely feel like I have a purpose,'' Ms. Ramirez said. ``It may
be to help other people going through the same situation that I am,
maybe even just sharing my story and helping young people realize that
if this happened to me it could happen to them, and to protect
themselves and protect others around them who are more vulnerable. And
to motivate and help other centers around the world to realize that lung
transplantation is an option for terminally ill Covid patients.''

The outlook for Ms. Ramirez is good, Dr. Bharat said, because she is
young and healthy. She will be on anti-rejection medicines for the rest
of her life. Transplanted lungs can still be rejected, he said, but he
has seen some last 20 years. And patients may be able to receive a
second transplant.

``I think from now on she'll continue to get stronger and stronger,'' he
said. ``She asked if she could go skydiving. We'll probably get her
there in a few months.''

Advertisement

\protect\hyperlink{after-bottom}{Continue reading the main story}

\hypertarget{site-index}{%
\subsection{Site Index}\label{site-index}}

\hypertarget{site-information-navigation}{%
\subsection{Site Information
Navigation}\label{site-information-navigation}}

\begin{itemize}
\tightlist
\item
  \href{https://help.nytimes.com/hc/en-us/articles/115014792127-Copyright-notice}{©~2020~The
  New York Times Company}
\end{itemize}

\begin{itemize}
\tightlist
\item
  \href{https://www.nytco.com/}{NYTCo}
\item
  \href{https://help.nytimes.com/hc/en-us/articles/115015385887-Contact-Us}{Contact
  Us}
\item
  \href{https://www.nytco.com/careers/}{Work with us}
\item
  \href{https://nytmediakit.com/}{Advertise}
\item
  \href{http://www.tbrandstudio.com/}{T Brand Studio}
\item
  \href{https://www.nytimes.com/privacy/cookie-policy\#how-do-i-manage-trackers}{Your
  Ad Choices}
\item
  \href{https://www.nytimes.com/privacy}{Privacy}
\item
  \href{https://help.nytimes.com/hc/en-us/articles/115014893428-Terms-of-service}{Terms
  of Service}
\item
  \href{https://help.nytimes.com/hc/en-us/articles/115014893968-Terms-of-sale}{Terms
  of Sale}
\item
  \href{https://spiderbites.nytimes.com}{Site Map}
\item
  \href{https://help.nytimes.com/hc/en-us}{Help}
\item
  \href{https://www.nytimes.com/subscription?campaignId=37WXW}{Subscriptions}
\end{itemize}
