Sections

SEARCH

\protect\hyperlink{site-content}{Skip to
content}\protect\hyperlink{site-index}{Skip to site index}

\href{https://www.nytimes.com/section/nyregion}{New York}

\href{https://myaccount.nytimes.com/auth/login?response_type=cookie\&client_id=vi}{}

\href{https://www.nytimes.com/section/todayspaper}{Today's Paper}

\href{/section/nyregion}{New York}\textbar{}About 20\% of N.J. Prisoners
Could Be Freed to Avoid Virus

\url{https://nyti.ms/2XdJh0K}

\begin{itemize}
\item
\item
\item
\item
\item
\item
\end{itemize}

\href{https://www.nytimes.com/news-event/coronavirus?action=click\&pgtype=Article\&state=default\&region=TOP_BANNER\&context=storylines_menu}{The
Coronavirus Outbreak}

\begin{itemize}
\tightlist
\item
  live\href{https://www.nytimes.com/2020/08/01/world/coronavirus-covid-19.html?action=click\&pgtype=Article\&state=default\&region=TOP_BANNER\&context=storylines_menu}{Latest
  Updates}
\item
  \href{https://www.nytimes.com/interactive/2020/us/coronavirus-us-cases.html?action=click\&pgtype=Article\&state=default\&region=TOP_BANNER\&context=storylines_menu}{Maps
  and Cases}
\item
  \href{https://www.nytimes.com/interactive/2020/science/coronavirus-vaccine-tracker.html?action=click\&pgtype=Article\&state=default\&region=TOP_BANNER\&context=storylines_menu}{Vaccine
  Tracker}
\item
  \href{https://www.nytimes.com/interactive/2020/07/29/us/schools-reopening-coronavirus.html?action=click\&pgtype=Article\&state=default\&region=TOP_BANNER\&context=storylines_menu}{What
  School May Look Like}
\item
  \href{https://www.nytimes.com/live/2020/07/31/business/stock-market-today-coronavirus?action=click\&pgtype=Article\&state=default\&region=TOP_BANNER\&context=storylines_menu}{Economy}
\end{itemize}

Advertisement

\protect\hyperlink{after-top}{Continue reading the main story}

Supported by

\protect\hyperlink{after-sponsor}{Continue reading the main story}

\hypertarget{about-20-of-nj-prisoners-could-be-freed-to-avoid-virus}{%
\section{About 20\% of N.J. Prisoners Could Be Freed to Avoid
Virus}\label{about-20-of-nj-prisoners-could-be-freed-to-avoid-virus}}

Believed to be the first Covid-19 bill of its kind, the legislation
could free more than 3,000 New Jersey prisoners who are within a year of
release.

\includegraphics{https://static01.nyt.com/images/2020/07/30/nyregion/30njprisons1/merlin_175066893_6527394e-4370-4096-977d-14e2c9e363f7-articleLarge.jpg?quality=75\&auto=webp\&disable=upscale}

By \href{https://www.nytimes.com/by/tracey-tully}{Tracey Tully}

\begin{itemize}
\item
  Published July 30, 2020Updated July 31, 2020
\item
  \begin{itemize}
  \item
  \item
  \item
  \item
  \item
  \item
  \end{itemize}
\end{itemize}

New Jersey lawmakers are expected to approve legislation that could free
more than 3,000 prisoners --- about 20 percent of the state's prison
population --- months before their release dates in response to the
extraordinary threat posed by the coronavirus in tightly packed
correctional facilities.

Inmates who are within a year of completing their state prison sentences
would be eligible to be released up to eight months early based on
credits awarded for time served during the pandemic.

The bill, which the American Civil Liberties Union believes to be the
first legislative initiative of its kind in the United States, would not
permit the release of most sex offenders, but would apply to inmates
sentenced for other violent crimes, including murder.

``There are people who were sentenced to long prison terms, but they
weren't sentenced to die in prison,'' said Amol Sinha, executive
director of the A.C.L.U. of New Jersey, one of the groups urging passage
of the legislation.

The Legislature was prepared to take a final vote on the bill on
Thursday but it was delayed when several last-minute amendments were
introduced.

The proposal comes amid nationwide efforts to reduce state and federal
prison populations to protect inmates and employees from the virus,
which
\href{https://www.nytimes.com/2020/06/16/us/coronavirus-inmates-prisons-jails.html}{continues
to spread rapidly} through some prisons. The
\href{https://www.nytimes.com/interactive/2020/us/coronavirus-us-cases.html\#clusters}{five
largest known clusters} of the virus in the United States are now linked
to correctional facilities.

New Jersey's prison death rate is the highest in the nation, with 27
deaths per 10,000 prisoners, according to
\href{https://www.themarshallproject.org/2020/05/01/a-state-by-state-look-at-coronavirus-in-prisons}{data
collected} by The Marshall Project and The Associated Press.

In California, the governor ordered the
\href{https://www.latimes.com/california/story/2020-07-10/california-release-8000-prisoners-coronavirus-crisis-newsom}{release
of up to 8,000 nonviolent offenders} by the end of August.
\href{https://portal.ct.gov/DOC/Common-Elements/Common-Elements/Press-Release-Related-COVID19}{Connecticut's
prison population} has dropped by 16 percent since March to the lowest
levels in 29 years, in part because of coronavirus releases.

New Jersey has already released about 800 inmates early from its prison
system under an
\href{https://www.nj.gov/governor/news/news/562020/20200410d.shtml}{April
executive order} and freed nearly 700 people from its
\href{https://www.nytimes.com/2020/03/23/nyregion/coronavirus-nj-inmates-release.html}{county
jails} after a legal challenge. But those releases have occurred largely
on a case-by-case basis and did not involve legislative action.

Rory Price Jr., a 39-year-old inmate who was serving a
three-and-a-half-year sentence for drug and weapons convictions, was
weeks away from being released from a prison halfway house when he
developed a deep cough that sapped the color from his face.

\hypertarget{latest-updates-global-coronavirus-outbreak}{%
\section{\texorpdfstring{\href{https://www.nytimes.com/2020/08/01/world/coronavirus-covid-19.html?action=click\&pgtype=Article\&state=default\&region=MAIN_CONTENT_1\&context=storylines_live_updates}{Latest
Updates: Global Coronavirus
Outbreak}}{Latest Updates: Global Coronavirus Outbreak}}\label{latest-updates-global-coronavirus-outbreak}}

Updated 2020-08-02T07:13:03.337Z

\begin{itemize}
\tightlist
\item
  \href{https://www.nytimes.com/2020/08/01/world/coronavirus-covid-19.html?action=click\&pgtype=Article\&state=default\&region=MAIN_CONTENT_1\&context=storylines_live_updates\#link-34047410}{The
  U.S. reels as July cases more than double the total of any other
  month.}
\item
  \href{https://www.nytimes.com/2020/08/01/world/coronavirus-covid-19.html?action=click\&pgtype=Article\&state=default\&region=MAIN_CONTENT_1\&context=storylines_live_updates\#link-780ec966}{Top
  U.S. officials work to break an impasse over the federal jobless
  benefit.}
\item
  \href{https://www.nytimes.com/2020/08/01/world/coronavirus-covid-19.html?action=click\&pgtype=Article\&state=default\&region=MAIN_CONTENT_1\&context=storylines_live_updates\#link-2bc8948}{Its
  outbreak untamed, Melbourne goes into even greater lockdown.}
\end{itemize}

\href{https://www.nytimes.com/2020/08/01/world/coronavirus-covid-19.html?action=click\&pgtype=Article\&state=default\&region=MAIN_CONTENT_1\&context=storylines_live_updates}{See
more updates}

More live coverage:
\href{https://www.nytimes.com/live/2020/07/31/business/stock-market-today-coronavirus?action=click\&pgtype=Article\&state=default\&region=MAIN_CONTENT_1\&context=storylines_live_updates}{Markets}

``He was, like, gray,'' said Art Devlin, 55, who shared a dormitory room
with Mr. Price for a year at the house in Bridgeton, N.J. ``He started
with the coughing, the hacking. I told them, `This guy's sick.'''

But Mr. Price continued to work in the kitchen, sleep in a 12-man
dormitory and dream of the party his family was planning when he was
freed in May, Mr. Devlin said.

He never got home. Mr. Price died May 1 of the coronavirus in a
Vineland, N.J., hospital 21 days before his release date.

``It has been a living nightmare,'' said his mother, Bernice Ferguson,
54. ``He was doing his time. He did not go there for a death sentence.''

\includegraphics{https://static01.nyt.com/images/2020/07/30/nyregion/30njprisons2/30njprisons2-articleLarge.jpg?quality=75\&auto=webp\&disable=upscale}

At least 48 other prison inmates in New Jersey have
\href{https://www.njdoc.gov/pages/COVID19Updates.shtml}{died from the
virus}.

If approved, the bill could free more than 3,000 inmates --- about
one-fifth of the 16,704 people serving state criminal sentences in New
Jersey. The prison population in New Jersey has declined by about 12
percent since the start of the pandemic in March based on a combination
of factors, including the early releases and a virus-related slowdown
within the court system.

The proposed legislation earned unanimous bipartisan support during a
Senate committee hearing, and has won backing from key lawmakers,
leaving supporters hopeful it will be approved. Lawmakers voted to
approve several final amendments on Thursday, including the addition of
a 45-day implementation window.

A final vote is expected in late August. If signed into law by Gov.
Philip D. Murphy, the release of inmates could begin by the middle of
September, according to a sponsor of the bill in the Senate, Nellie Pou.

``We have to realize that we're dealing with an emergency here,'' said
Ms. Pou, who added that she is confident the bill will be approved in
the Assembly and Senate.

A spokesman for Mr. Murphy would not comment on whether he would sign
the bill into law. A spokesman for the state attorney general, Gurbir S.
Grewal, who oversees the State Police and county prosecutors, also said
he would not comment on pending legislation.

Senator Gerald Cardinale, a Republican who represents parts of Bergen
and Passaic Counties and describes himself as a law-and-order
conservative, said his support for the legislation stemmed from ``basic,
simple justice.''

He said the state Department of Corrections had failed to keep inmates
in its custody safe.

``We are not doing very well at all in terms of protecting people,''
said Mr. Cardinale, who voted to approve the bill in a Senate committee.
``They're prisoners, but they are human beings.''

Since March,
\href{https://www.njdoc.gov/pages/COVID19Updates.shtml}{2,892 inmates}
--- about 17 percent of the population --- and 781 employees have tested
positive for the virus at New Jersey correctional facilities. In
addition to the 49 inmates who have died, several employee deaths have
been linked to Covid-19.

At one point since March, there were 800 active cases of Covid-19 in
state correctional facilities. Prisons have since successfully slowed
the spread. The state has begun its second phase of universal testing,
and there are now fewer than 30 coronavirus cases linked to state
correctional facilities, according to the commissioner of the Department
of Corrections, Marcus O. Hicks.

Still, employees continue to pose a risk of importing new cases of the
virus, and social distancing in crowded facilities remains virtually
impossible.

\href{https://www.nytimes.com/news-event/coronavirus?action=click\&pgtype=Article\&state=default\&region=MAIN_CONTENT_3\&context=storylines_faq}{}

\hypertarget{the-coronavirus-outbreak-}{%
\subsubsection{The Coronavirus Outbreak
›}\label{the-coronavirus-outbreak-}}

\hypertarget{frequently-asked-questions}{%
\paragraph{Frequently Asked
Questions}\label{frequently-asked-questions}}

Updated July 27, 2020

\begin{itemize}
\item ~
  \hypertarget{should-i-refinance-my-mortgage}{%
  \paragraph{Should I refinance my
  mortgage?}\label{should-i-refinance-my-mortgage}}

  \begin{itemize}
  \tightlist
  \item
    \href{https://www.nytimes.com/article/coronavirus-money-unemployment.html?action=click\&pgtype=Article\&state=default\&region=MAIN_CONTENT_3\&context=storylines_faq}{It
    could be a good idea,} because mortgage rates have
    \href{https://www.nytimes.com/2020/07/16/business/mortgage-rates-below-3-percent.html?action=click\&pgtype=Article\&state=default\&region=MAIN_CONTENT_3\&context=storylines_faq}{never
    been lower.} Refinancing requests have pushed mortgage applications
    to some of the highest levels since 2008, so be prepared to get in
    line. But defaults are also up, so if you're thinking about buying a
    home, be aware that some lenders have tightened their standards.
  \end{itemize}
\item ~
  \hypertarget{what-is-school-going-to-look-like-in-september}{%
  \paragraph{What is school going to look like in
  September?}\label{what-is-school-going-to-look-like-in-september}}

  \begin{itemize}
  \tightlist
  \item
    It is unlikely that many schools will return to a normal schedule
    this fall, requiring the grind of
    \href{https://www.nytimes.com/2020/06/05/us/coronavirus-education-lost-learning.html?action=click\&pgtype=Article\&state=default\&region=MAIN_CONTENT_3\&context=storylines_faq}{online
    learning},
    \href{https://www.nytimes.com/2020/05/29/us/coronavirus-child-care-centers.html?action=click\&pgtype=Article\&state=default\&region=MAIN_CONTENT_3\&context=storylines_faq}{makeshift
    child care} and
    \href{https://www.nytimes.com/2020/06/03/business/economy/coronavirus-working-women.html?action=click\&pgtype=Article\&state=default\&region=MAIN_CONTENT_3\&context=storylines_faq}{stunted
    workdays} to continue. California's two largest public school
    districts --- Los Angeles and San Diego --- said on July 13, that
    \href{https://www.nytimes.com/2020/07/13/us/lausd-san-diego-school-reopening.html?action=click\&pgtype=Article\&state=default\&region=MAIN_CONTENT_3\&context=storylines_faq}{instruction
    will be remote-only in the fall}, citing concerns that surging
    coronavirus infections in their areas pose too dire a risk for
    students and teachers. Together, the two districts enroll some
    825,000 students. They are the largest in the country so far to
    abandon plans for even a partial physical return to classrooms when
    they reopen in August. For other districts, the solution won't be an
    all-or-nothing approach.
    \href{https://bioethics.jhu.edu/research-and-outreach/projects/eschool-initiative/school-policy-tracker/}{Many
    systems}, including the nation's largest, New York City, are
    devising
    \href{https://www.nytimes.com/2020/06/26/us/coronavirus-schools-reopen-fall.html?action=click\&pgtype=Article\&state=default\&region=MAIN_CONTENT_3\&context=storylines_faq}{hybrid
    plans} that involve spending some days in classrooms and other days
    online. There's no national policy on this yet, so check with your
    municipal school system regularly to see what is happening in your
    community.
  \end{itemize}
\item ~
  \hypertarget{is-the-coronavirus-airborne}{%
  \paragraph{Is the coronavirus
  airborne?}\label{is-the-coronavirus-airborne}}

  \begin{itemize}
  \tightlist
  \item
    The coronavirus
    \href{https://www.nytimes.com/2020/07/04/health/239-experts-with-one-big-claim-the-coronavirus-is-airborne.html?action=click\&pgtype=Article\&state=default\&region=MAIN_CONTENT_3\&context=storylines_faq}{can
    stay aloft for hours in tiny droplets in stagnant air}, infecting
    people as they inhale, mounting scientific evidence suggests. This
    risk is highest in crowded indoor spaces with poor ventilation, and
    may help explain super-spreading events reported in meatpacking
    plants, churches and restaurants.
    \href{https://www.nytimes.com/2020/07/06/health/coronavirus-airborne-aerosols.html?action=click\&pgtype=Article\&state=default\&region=MAIN_CONTENT_3\&context=storylines_faq}{It's
    unclear how often the virus is spread} via these tiny droplets, or
    aerosols, compared with larger droplets that are expelled when a
    sick person coughs or sneezes, or transmitted through contact with
    contaminated surfaces, said Linsey Marr, an aerosol expert at
    Virginia Tech. Aerosols are released even when a person without
    symptoms exhales, talks or sings, according to Dr. Marr and more
    than 200 other experts, who
    \href{https://academic.oup.com/cid/article/doi/10.1093/cid/ciaa939/5867798}{have
    outlined the evidence in an open letter to the World Health
    Organization}.
  \end{itemize}
\item ~
  \hypertarget{what-are-the-symptoms-of-coronavirus}{%
  \paragraph{What are the symptoms of
  coronavirus?}\label{what-are-the-symptoms-of-coronavirus}}

  \begin{itemize}
  \tightlist
  \item
    Common symptoms
    \href{https://www.nytimes.com/article/symptoms-coronavirus.html?action=click\&pgtype=Article\&state=default\&region=MAIN_CONTENT_3\&context=storylines_faq}{include
    fever, a dry cough, fatigue and difficulty breathing or shortness of
    breath.} Some of these symptoms overlap with those of the flu,
    making detection difficult, but runny noses and stuffy sinuses are
    less common.
    \href{https://www.nytimes.com/2020/04/27/health/coronavirus-symptoms-cdc.html?action=click\&pgtype=Article\&state=default\&region=MAIN_CONTENT_3\&context=storylines_faq}{The
    C.D.C. has also} added chills, muscle pain, sore throat, headache
    and a new loss of the sense of taste or smell as symptoms to look
    out for. Most people fall ill five to seven days after exposure, but
    symptoms may appear in as few as two days or as many as 14 days.
  \end{itemize}
\item ~
  \hypertarget{does-asymptomatic-transmission-of-covid-19-happen}{%
  \paragraph{Does asymptomatic transmission of Covid-19
  happen?}\label{does-asymptomatic-transmission-of-covid-19-happen}}

  \begin{itemize}
  \tightlist
  \item
    So far, the evidence seems to show it does. A widely cited
    \href{https://www.nature.com/articles/s41591-020-0869-5}{paper}
    published in April suggests that people are most infectious about
    two days before the onset of coronavirus symptoms and estimated that
    44 percent of new infections were a result of transmission from
    people who were not yet showing symptoms. Recently, a top expert at
    the World Health Organization stated that transmission of the
    coronavirus by people who did not have symptoms was ``very rare,''
    \href{https://www.nytimes.com/2020/06/09/world/coronavirus-updates.html?action=click\&pgtype=Article\&state=default\&region=MAIN_CONTENT_3\&context=storylines_faq\#link-1f302e21}{but
    she later walked back that statement.}
  \end{itemize}
\end{itemize}

Scott Clements said he feared that his brother Brian, who has diabetes
and heart disease and will be 59 next week, may not make it until his
February release date.

Image

Brian Clements, left, with his brother, Scott. Credit...via Scott
Clements

Brian Clements is serving a seven-and-a-half-year sentence at South
Woods State Prison for his role in a fatal car accident, but does not
fall into any of the four categories eligible to apply for release under
the governor's April executive
\href{https://www.nj.gov/governor/news/news/562020/20200410d.shtml}{order}.

``If he was to contract the virus, it's going to be as close to a death
sentence as anyone could imagine,'' said Scott Clements, 48. ``I want to
bring my brother home in my Toyota Camry, not in a hearse.''

Image

``I want to bring my brother home in my Toyota Camry, not in a hearse,''
said Scott Clements, whose brother is scheduled to be released from
prison in February.Credit...Hannah Yoon for The New York Times

Nicole D. Porter, the director of advocacy for the Sentencing Project, a
national nonprofit that advocates for sentencing reform and the
elimination of racial disparities in the criminal justice system, said
that while the proposal is a step it the right direction, it is still
not enough.

``What the United States has done pales in comparison to other
countries,'' she said, citing the release of
\href{https://www.nytimes.com/2020/04/26/world/americas/coronavirus-brazil-prisons.html}{85,000
prisoners in Iran} and
\href{https://www.reuters.com/article/us-health-coronavirus-indonesia-prisons/indonesia-to-release-30000-prisoners-early-amid-virus-concerns-idUSKBN21I11Z}{30,000
in Indonesia.}

``Every day is urgent for somebody in prison,'' Ms. Porter said. ``It's
just now the entire world is going through an urgent experience
together.''

A Republican who voted against the bill in an Assembly committee,
Christopher P. DePhillips, said he was sympathetic to the significant
challenges facing prisoners in the middle of a pandemic.

But, Mr. DePhillips said, he was concerned about allowing the early
release of violent offenders, and would have preferred for the
Legislature to have had a role in devising ways to keep inmates safe ---
before being asked to free them.

Mr. Devlin, however, said it was impossible to remain apart in the
prison housing units, where masks were not regularly worn. Officials
from the halfway house, run by the Kintock Group, did not return calls.

``I was scared for my life --- absolutely scared for my life,'' said Mr.
Devlin, who completed his sentence in late May but still worries about
those left behind. ``Now I'm scared for their lives.''

Advertisement

\protect\hyperlink{after-bottom}{Continue reading the main story}

\hypertarget{site-index}{%
\subsection{Site Index}\label{site-index}}

\hypertarget{site-information-navigation}{%
\subsection{Site Information
Navigation}\label{site-information-navigation}}

\begin{itemize}
\tightlist
\item
  \href{https://help.nytimes.com/hc/en-us/articles/115014792127-Copyright-notice}{©~2020~The
  New York Times Company}
\end{itemize}

\begin{itemize}
\tightlist
\item
  \href{https://www.nytco.com/}{NYTCo}
\item
  \href{https://help.nytimes.com/hc/en-us/articles/115015385887-Contact-Us}{Contact
  Us}
\item
  \href{https://www.nytco.com/careers/}{Work with us}
\item
  \href{https://nytmediakit.com/}{Advertise}
\item
  \href{http://www.tbrandstudio.com/}{T Brand Studio}
\item
  \href{https://www.nytimes.com/privacy/cookie-policy\#how-do-i-manage-trackers}{Your
  Ad Choices}
\item
  \href{https://www.nytimes.com/privacy}{Privacy}
\item
  \href{https://help.nytimes.com/hc/en-us/articles/115014893428-Terms-of-service}{Terms
  of Service}
\item
  \href{https://help.nytimes.com/hc/en-us/articles/115014893968-Terms-of-sale}{Terms
  of Sale}
\item
  \href{https://spiderbites.nytimes.com}{Site Map}
\item
  \href{https://help.nytimes.com/hc/en-us}{Help}
\item
  \href{https://www.nytimes.com/subscription?campaignId=37WXW}{Subscriptions}
\end{itemize}
