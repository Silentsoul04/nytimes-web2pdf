Sections

SEARCH

\protect\hyperlink{site-content}{Skip to
content}\protect\hyperlink{site-index}{Skip to site index}

\href{https://www.nytimes.com/section/nyregion}{New York}

\href{https://myaccount.nytimes.com/auth/login?response_type=cookie\&client_id=vi}{}

\href{https://www.nytimes.com/section/todayspaper}{Today's Paper}

\href{/section/nyregion}{New York}\textbar{}New York Love Story: The
Submarine Officer and the Beatles Cover Band

\url{https://nyti.ms/3jNGG7n}

\begin{itemize}
\item
\item
\item
\item
\item
\item
\end{itemize}

\href{https://www.nytimes.com/news-event/coronavirus?action=click\&pgtype=Article\&state=default\&region=TOP_BANNER\&context=storylines_menu}{The
Coronavirus Outbreak}

\begin{itemize}
\tightlist
\item
  live\href{https://www.nytimes.com/2020/08/01/world/coronavirus-covid-19.html?action=click\&pgtype=Article\&state=default\&region=TOP_BANNER\&context=storylines_menu}{Latest
  Updates}
\item
  \href{https://www.nytimes.com/interactive/2020/us/coronavirus-us-cases.html?action=click\&pgtype=Article\&state=default\&region=TOP_BANNER\&context=storylines_menu}{Maps
  and Cases}
\item
  \href{https://www.nytimes.com/interactive/2020/science/coronavirus-vaccine-tracker.html?action=click\&pgtype=Article\&state=default\&region=TOP_BANNER\&context=storylines_menu}{Vaccine
  Tracker}
\item
  \href{https://www.nytimes.com/interactive/2020/07/29/us/schools-reopening-coronavirus.html?action=click\&pgtype=Article\&state=default\&region=TOP_BANNER\&context=storylines_menu}{What
  School May Look Like}
\item
  \href{https://www.nytimes.com/live/2020/07/31/business/stock-market-today-coronavirus?action=click\&pgtype=Article\&state=default\&region=TOP_BANNER\&context=storylines_menu}{Economy}
\end{itemize}

Advertisement

\protect\hyperlink{after-top}{Continue reading the main story}

Supported by

\protect\hyperlink{after-sponsor}{Continue reading the main story}

\hypertarget{new-york-love-story-the-submarine-officer-and-the-beatles-cover-band}{%
\section{New York Love Story: The Submarine Officer and the Beatles
Cover
Band}\label{new-york-love-story-the-submarine-officer-and-the-beatles-cover-band}}

A Columbia grad student, new to the city, lost his lease. So he
organized the perfect send-off.

\includegraphics{https://static01.nyt.com/images/2020/08/01/nyregion/01nyrooftop-02/01nyrooftop-02-articleLarge.jpg?quality=75\&auto=webp\&disable=upscale}

By \href{https://www.nytimes.com/by/alex-vadukul}{Alex Vadukul}

\begin{itemize}
\item
  July 30, 2020
\item
  \begin{itemize}
  \item
  \item
  \item
  \item
  \item
  \item
  \end{itemize}
\end{itemize}

On an otherwise quiet pandemic Sunday, the unmistakable songs of the
Beatles started blaring from the roof of a building on the Upper West
Side. The band was belting out faithful renditions from the
\href{https://www.rollingstone.com/music/music-news/beatles-famous-rooftop-concert-15-things-you-didnt-know-58342/}{1969
rooftop concert} in London --- ``Get Back,'' ``Don't Let Me Down,''
``I've Got a Feeling'' --- and people stepped out onto their balconies
and fire escapes to listen.

When the band finished, and Upper West Siders shut their windows and
headed back into their apartments, a 28-year-old Columbia University
physics grad student named Ben Markham stood on the roof savoring the
moment with a joint and beer.

``I was worried the cops might come,'' he said. ``If John was watching,
I hope he liked it.''

Mr. Markham moved to New York a year ago, and a week before he was
forced to move out of his apartment, he'd found the cover band,
\href{http://www.meetles.com/}{the Meetles}, on Craigslist, hiring them
to play on his building's rooftop. The show was Mr. Markham's unusual
way of saying goodbye to his first New York neighborhood.

In a few days, he would become one of the last tenants to vacate a small
apartment building on West 74th Street that has been steadily emptying
out since the pandemic began. Last month, his landlord didn't renew his
lease, because he's apparently considering selling the building,
according to Mr. Markham, and so he was getting the boot. (He stressed
that his landlord worked with him to accommodate a smooth exit.) Mr.
Markham decided to commemorate his first strange year in New York City
with an absurdist send-off to himself.

\includegraphics{https://static01.nyt.com/images/2020/08/01/nyregion/01nyrooftop-04/01nyrooftop-04-articleLarge.jpg?quality=75\&auto=webp\&disable=upscale}

But as Mr. Markham crushed more beers on his rooftop, he said that he
was also just trying to make life in the city a little bit more
interesting. He wanted to reclaim some of the crazy and colorful New
York that was promised but stolen away by the pandemic.

``When I got here a year ago, I realized that what everyone says about
New York is true,'' said Mr. Markham. ``The hype is real. It was like
being on line for a roller coaster. You wait and wait and wait. And then
it started raining really hard.''

\hypertarget{latest-updates-global-coronavirus-outbreak}{%
\section{\texorpdfstring{\href{https://www.nytimes.com/2020/08/01/world/coronavirus-covid-19.html?action=click\&pgtype=Article\&state=default\&region=MAIN_CONTENT_1\&context=storylines_live_updates}{Latest
Updates: Global Coronavirus
Outbreak}}{Latest Updates: Global Coronavirus Outbreak}}\label{latest-updates-global-coronavirus-outbreak}}

Updated 2020-08-02T06:58:18.835Z

\begin{itemize}
\tightlist
\item
  \href{https://www.nytimes.com/2020/08/01/world/coronavirus-covid-19.html?action=click\&pgtype=Article\&state=default\&region=MAIN_CONTENT_1\&context=storylines_live_updates\#link-34047410}{The
  U.S. reels as July cases more than double the total of any other
  month.}
\item
  \href{https://www.nytimes.com/2020/08/01/world/coronavirus-covid-19.html?action=click\&pgtype=Article\&state=default\&region=MAIN_CONTENT_1\&context=storylines_live_updates\#link-780ec966}{Top
  U.S. officials work to break an impasse over the federal jobless
  benefit.}
\item
  \href{https://www.nytimes.com/2020/08/01/world/coronavirus-covid-19.html?action=click\&pgtype=Article\&state=default\&region=MAIN_CONTENT_1\&context=storylines_live_updates\#link-2bc8948}{Its
  outbreak untamed, Melbourne goes into even greater lockdown.}
\end{itemize}

\href{https://www.nytimes.com/2020/08/01/world/coronavirus-covid-19.html?action=click\&pgtype=Article\&state=default\&region=MAIN_CONTENT_1\&context=storylines_live_updates}{See
more updates}

More live coverage:
\href{https://www.nytimes.com/live/2020/07/31/business/stock-market-today-coronavirus?action=click\&pgtype=Article\&state=default\&region=MAIN_CONTENT_1\&context=storylines_live_updates}{Markets}

The rain, so to speak, began with the first signs of lockdown, and one
after the other, his roommates gradually all left to return home. Then
it was just him and another guy in the building. Mr. Markham says he's
become only more resolved to stay here, and he's now convinced of his
destiny to become a New Yorker.

``They all went back to mommy and daddy,'' he said. ``But New York is my
home now. The pandemic won't stop me. I think I've maybe always been a
New Yorker, I just had to get here first.''

Mr. Markham, who is from Jupiter, Fla., arrived in Manhattan last August
after serving in the Navy as a nuclear submarine officer aboard the
\href{https://www.csp.navy.mil/pasadena/About/}{U.S.S. Pasadena} in San
Diego. He'd been accepted into Columbia University's engineering school,
and soon after getting here, he immersed himself in the city's colorful
tumult.

By spring, Mr. Markham was smitten with New York, seeing a kind of
mathematical poetry in it. ``Physics is beautiful because it is
unpredictable and random,'' he said. ``I see that same beautiful chaos
in New York.''

Then, the pandemic gripped the city, and the New York he was getting
intimate with became bleak and barren overnight.

Couches got tossed on his street as people moved out. On Tinder,
conversations ended swiftly: ``I'd try to meet up and they'd say, `Oh,
I'm with my parents in Connecticut. I'll be here until it settles down.'
I'd reply: `Uh, OK. Talk soon then, I guess.''' Soon, Mr. Markham felt
like he had been left alone on a ghostly Upper West Side.

To his surprise, he began appreciating the city anew. At night, he biked
through the empty canyons of Midtown. He set up a projector on his roof
and watched ``\href{https://www.youtube.com/watch?v=vefJAtG-ZKI}{Yellow
Submarine}'' on the wall of the building next door. One day, he walked
into a nice neighborhood he'd never visited before, and discovered it
had its own
\href{https://gothamist.com/news/how-gramercy-park-became-a-private-playground-for-nycs-elite}{private
garden}.

``It's called Gramercy Park,'' he said. ``They have this special little
garden just to themselves. You need a key to get in, right? No, you
don't. You just need a pair of legs and to be able to jump a fence. And
guess what? It's just like any other park.''

\href{https://www.nytimes.com/news-event/coronavirus?action=click\&pgtype=Article\&state=default\&region=MAIN_CONTENT_3\&context=storylines_faq}{}

\hypertarget{the-coronavirus-outbreak-}{%
\subsubsection{The Coronavirus Outbreak
›}\label{the-coronavirus-outbreak-}}

\hypertarget{frequently-asked-questions}{%
\paragraph{Frequently Asked
Questions}\label{frequently-asked-questions}}

Updated July 27, 2020

\begin{itemize}
\item ~
  \hypertarget{should-i-refinance-my-mortgage}{%
  \paragraph{Should I refinance my
  mortgage?}\label{should-i-refinance-my-mortgage}}

  \begin{itemize}
  \tightlist
  \item
    \href{https://www.nytimes.com/article/coronavirus-money-unemployment.html?action=click\&pgtype=Article\&state=default\&region=MAIN_CONTENT_3\&context=storylines_faq}{It
    could be a good idea,} because mortgage rates have
    \href{https://www.nytimes.com/2020/07/16/business/mortgage-rates-below-3-percent.html?action=click\&pgtype=Article\&state=default\&region=MAIN_CONTENT_3\&context=storylines_faq}{never
    been lower.} Refinancing requests have pushed mortgage applications
    to some of the highest levels since 2008, so be prepared to get in
    line. But defaults are also up, so if you're thinking about buying a
    home, be aware that some lenders have tightened their standards.
  \end{itemize}
\item ~
  \hypertarget{what-is-school-going-to-look-like-in-september}{%
  \paragraph{What is school going to look like in
  September?}\label{what-is-school-going-to-look-like-in-september}}

  \begin{itemize}
  \tightlist
  \item
    It is unlikely that many schools will return to a normal schedule
    this fall, requiring the grind of
    \href{https://www.nytimes.com/2020/06/05/us/coronavirus-education-lost-learning.html?action=click\&pgtype=Article\&state=default\&region=MAIN_CONTENT_3\&context=storylines_faq}{online
    learning},
    \href{https://www.nytimes.com/2020/05/29/us/coronavirus-child-care-centers.html?action=click\&pgtype=Article\&state=default\&region=MAIN_CONTENT_3\&context=storylines_faq}{makeshift
    child care} and
    \href{https://www.nytimes.com/2020/06/03/business/economy/coronavirus-working-women.html?action=click\&pgtype=Article\&state=default\&region=MAIN_CONTENT_3\&context=storylines_faq}{stunted
    workdays} to continue. California's two largest public school
    districts --- Los Angeles and San Diego --- said on July 13, that
    \href{https://www.nytimes.com/2020/07/13/us/lausd-san-diego-school-reopening.html?action=click\&pgtype=Article\&state=default\&region=MAIN_CONTENT_3\&context=storylines_faq}{instruction
    will be remote-only in the fall}, citing concerns that surging
    coronavirus infections in their areas pose too dire a risk for
    students and teachers. Together, the two districts enroll some
    825,000 students. They are the largest in the country so far to
    abandon plans for even a partial physical return to classrooms when
    they reopen in August. For other districts, the solution won't be an
    all-or-nothing approach.
    \href{https://bioethics.jhu.edu/research-and-outreach/projects/eschool-initiative/school-policy-tracker/}{Many
    systems}, including the nation's largest, New York City, are
    devising
    \href{https://www.nytimes.com/2020/06/26/us/coronavirus-schools-reopen-fall.html?action=click\&pgtype=Article\&state=default\&region=MAIN_CONTENT_3\&context=storylines_faq}{hybrid
    plans} that involve spending some days in classrooms and other days
    online. There's no national policy on this yet, so check with your
    municipal school system regularly to see what is happening in your
    community.
  \end{itemize}
\item ~
  \hypertarget{is-the-coronavirus-airborne}{%
  \paragraph{Is the coronavirus
  airborne?}\label{is-the-coronavirus-airborne}}

  \begin{itemize}
  \tightlist
  \item
    The coronavirus
    \href{https://www.nytimes.com/2020/07/04/health/239-experts-with-one-big-claim-the-coronavirus-is-airborne.html?action=click\&pgtype=Article\&state=default\&region=MAIN_CONTENT_3\&context=storylines_faq}{can
    stay aloft for hours in tiny droplets in stagnant air}, infecting
    people as they inhale, mounting scientific evidence suggests. This
    risk is highest in crowded indoor spaces with poor ventilation, and
    may help explain super-spreading events reported in meatpacking
    plants, churches and restaurants.
    \href{https://www.nytimes.com/2020/07/06/health/coronavirus-airborne-aerosols.html?action=click\&pgtype=Article\&state=default\&region=MAIN_CONTENT_3\&context=storylines_faq}{It's
    unclear how often the virus is spread} via these tiny droplets, or
    aerosols, compared with larger droplets that are expelled when a
    sick person coughs or sneezes, or transmitted through contact with
    contaminated surfaces, said Linsey Marr, an aerosol expert at
    Virginia Tech. Aerosols are released even when a person without
    symptoms exhales, talks or sings, according to Dr. Marr and more
    than 200 other experts, who
    \href{https://academic.oup.com/cid/article/doi/10.1093/cid/ciaa939/5867798}{have
    outlined the evidence in an open letter to the World Health
    Organization}.
  \end{itemize}
\item ~
  \hypertarget{what-are-the-symptoms-of-coronavirus}{%
  \paragraph{What are the symptoms of
  coronavirus?}\label{what-are-the-symptoms-of-coronavirus}}

  \begin{itemize}
  \tightlist
  \item
    Common symptoms
    \href{https://www.nytimes.com/article/symptoms-coronavirus.html?action=click\&pgtype=Article\&state=default\&region=MAIN_CONTENT_3\&context=storylines_faq}{include
    fever, a dry cough, fatigue and difficulty breathing or shortness of
    breath.} Some of these symptoms overlap with those of the flu,
    making detection difficult, but runny noses and stuffy sinuses are
    less common.
    \href{https://www.nytimes.com/2020/04/27/health/coronavirus-symptoms-cdc.html?action=click\&pgtype=Article\&state=default\&region=MAIN_CONTENT_3\&context=storylines_faq}{The
    C.D.C. has also} added chills, muscle pain, sore throat, headache
    and a new loss of the sense of taste or smell as symptoms to look
    out for. Most people fall ill five to seven days after exposure, but
    symptoms may appear in as few as two days or as many as 14 days.
  \end{itemize}
\item ~
  \hypertarget{does-asymptomatic-transmission-of-covid-19-happen}{%
  \paragraph{Does asymptomatic transmission of Covid-19
  happen?}\label{does-asymptomatic-transmission-of-covid-19-happen}}

  \begin{itemize}
  \tightlist
  \item
    So far, the evidence seems to show it does. A widely cited
    \href{https://www.nature.com/articles/s41591-020-0869-5}{paper}
    published in April suggests that people are most infectious about
    two days before the onset of coronavirus symptoms and estimated that
    44 percent of new infections were a result of transmission from
    people who were not yet showing symptoms. Recently, a top expert at
    the World Health Organization stated that transmission of the
    coronavirus by people who did not have symptoms was ``very rare,''
    \href{https://www.nytimes.com/2020/06/09/world/coronavirus-updates.html?action=click\&pgtype=Article\&state=default\&region=MAIN_CONTENT_3\&context=storylines_faq\#link-1f302e21}{but
    she later walked back that statement.}
  \end{itemize}
\end{itemize}

``Now everything is strange, but I like strange,'' he added. ``I'm
learning more about the city now than I was before.''

And then there's his Beatles obsession.

Mr. Markham has had a morning ritual: a quarantine sanity stroll from
his apartment to the
\href{https://www.centralparknyc.org/attractions/strawberry-fields}{Strawberry
Fields memorial} in Central Park, where he'd sit on a bench and strum
Beatles songs on his guitar. Sometimes, he'd stand outside the Dakota
and its flickering gas lanterns, and wonder what Lennon's life was like
in New York. Once, he said, he played the riff to ``Day Tripper'' on the
very spot Lennon was murdered.

``I've been a Beatles fan my whole life,'' he said. ``I didn't even
realize the Dakota was a couple blocks from me until after I moved in.
The fact that John died near me, and I had the opportunity to pay homage
to him in some way, with everything going on, it just made sense.''

Image

Neighbors, however briefly, enjoying the music.Credit...Sara Naomi
Lewkowicz for The New York Times

Well, it made sense to Mr. Markham, anyway.

At the rooftop concert, he invited a friend he made in the neighborhood,
Richard Wooley, 55, to watch the ``Let It Be'' show with him. Mr. Wooley
considered the newcomer's choice to commit to New York.

``Does anyone get the New York they are promised?'' Mr. Wooley asked.
``I praise people like Ben who have stuck it out, because this is how
New York will survive. He just got here, but maybe people like him are
the real New Yorkers.''

Last week, Mr. Markham finally settled into his new apartment, a sublet
in the East Village he's sharing with three roommates. Tompkins Square
Park is just down the block and he's already become fond of its resident
musicians and eccentrics and he said that the neighborhood has better
bars than the Upper West Side.

As it happens, August will mark Mr. Markham's first full year in New
York.

``I think I'm still standing on line for that roller coaster,'' he said.
``It's still raining and I'm waiting for the next chapter. But the wait
has been beautiful in its own way, even in the rain.''

Advertisement

\protect\hyperlink{after-bottom}{Continue reading the main story}

\hypertarget{site-index}{%
\subsection{Site Index}\label{site-index}}

\hypertarget{site-information-navigation}{%
\subsection{Site Information
Navigation}\label{site-information-navigation}}

\begin{itemize}
\tightlist
\item
  \href{https://help.nytimes.com/hc/en-us/articles/115014792127-Copyright-notice}{©~2020~The
  New York Times Company}
\end{itemize}

\begin{itemize}
\tightlist
\item
  \href{https://www.nytco.com/}{NYTCo}
\item
  \href{https://help.nytimes.com/hc/en-us/articles/115015385887-Contact-Us}{Contact
  Us}
\item
  \href{https://www.nytco.com/careers/}{Work with us}
\item
  \href{https://nytmediakit.com/}{Advertise}
\item
  \href{http://www.tbrandstudio.com/}{T Brand Studio}
\item
  \href{https://www.nytimes.com/privacy/cookie-policy\#how-do-i-manage-trackers}{Your
  Ad Choices}
\item
  \href{https://www.nytimes.com/privacy}{Privacy}
\item
  \href{https://help.nytimes.com/hc/en-us/articles/115014893428-Terms-of-service}{Terms
  of Service}
\item
  \href{https://help.nytimes.com/hc/en-us/articles/115014893968-Terms-of-sale}{Terms
  of Sale}
\item
  \href{https://spiderbites.nytimes.com}{Site Map}
\item
  \href{https://help.nytimes.com/hc/en-us}{Help}
\item
  \href{https://www.nytimes.com/subscription?campaignId=37WXW}{Subscriptions}
\end{itemize}
