Sections

SEARCH

\protect\hyperlink{site-content}{Skip to
content}\protect\hyperlink{site-index}{Skip to site index}

\href{https://www.nytimes.com/section/nyregion}{New York}

\href{https://myaccount.nytimes.com/auth/login?response_type=cookie\&client_id=vi}{}

\href{https://www.nytimes.com/section/todayspaper}{Today's Paper}

\href{/section/nyregion}{New York}\textbar{}Food Tourists Trickle In to
New York's Pandemic Dining Scene

\url{https://nyti.ms/30TWqwI}

\begin{itemize}
\item
\item
\item
\item
\item
\item
\end{itemize}

\href{https://www.nytimes.com/news-event/coronavirus?action=click\&pgtype=Article\&state=default\&region=TOP_BANNER\&context=storylines_menu}{The
Coronavirus Outbreak}

\begin{itemize}
\tightlist
\item
  live\href{https://www.nytimes.com/2020/08/04/world/coronavirus-cases.html?action=click\&pgtype=Article\&state=default\&region=TOP_BANNER\&context=storylines_menu}{Latest
  Updates}
\item
  \href{https://www.nytimes.com/interactive/2020/us/coronavirus-us-cases.html?action=click\&pgtype=Article\&state=default\&region=TOP_BANNER\&context=storylines_menu}{Maps
  and Cases}
\item
  \href{https://www.nytimes.com/interactive/2020/science/coronavirus-vaccine-tracker.html?action=click\&pgtype=Article\&state=default\&region=TOP_BANNER\&context=storylines_menu}{Vaccine
  Tracker}
\item
  \href{https://www.nytimes.com/2020/08/02/us/covid-college-reopening.html?action=click\&pgtype=Article\&state=default\&region=TOP_BANNER\&context=storylines_menu}{College
  Reopening}
\item
  \href{https://www.nytimes.com/live/2020/08/04/business/stock-market-today-coronavirus?action=click\&pgtype=Article\&state=default\&region=TOP_BANNER\&context=storylines_menu}{Economy}
\end{itemize}

Advertisement

\protect\hyperlink{after-top}{Continue reading the main story}

Supported by

\protect\hyperlink{after-sponsor}{Continue reading the main story}

\hypertarget{food-tourists-trickle-in-to-new-yorks-pandemic-dining-scene}{%
\section{Food Tourists Trickle In to New York's Pandemic Dining
Scene}\label{food-tourists-trickle-in-to-new-yorks-pandemic-dining-scene}}

Visitors --- as long as they don't come from a quarantine state --- are
enjoying the city's rooftop bars and sidewalk restaurants.

\includegraphics{https://static01.nyt.com/images/2020/08/02/nyregion/02REFER2/00nyvirus-touristfood-articleLarge.jpg?quality=75\&auto=webp\&disable=upscale}

By Alyson Krueger

\begin{itemize}
\item
  July 30, 2020
\item
  \begin{itemize}
  \item
  \item
  \item
  \item
  \item
  \item
  \end{itemize}
\end{itemize}

In early July, Bruna Borelli, along with her husband and two other
couples, drove a minivan for 12 hours from Sterling Heights, Mich., to
New York City for a vacation.

Before they left, the friends watched the news carefully, making sure
that Michigan hadn't been added to the long and growing list of states
whose residents or visitors, heading to New York City, would be asked to
quarantine for 14 days upon arrival. When the coast was clear, they hit
the road.

The stress of visiting the city during the pandemic had its benefits.
The group got rooms at
\href{https://www.marriott.com/hotels/travel/nycox-moxy-nyc-times-square/?scid=bb6e16b5-1692-44e9-9459-428cef21e75b\&ppc=ppc\&pId=ustbppc\&nst=paid\&gclid=EAIaIQobChMIzfqa1uvu6gIVCYvICh1WVAGtEAAYASAAEgLHAfD_BwE\&gclsrc=aw.ds}{Moxy},
a chic hotel in Times Square with a miniature golf course, featuring
statues of animals in flirty poses, as part of its rooftop bar, for \$75
a night. ``That's a really good price, yeah?'' said Ms. Borelli, a
30-year-old lawyer.

The group of friends strolled down Fifth Avenue, had a picnic in Central
Park, walked over the Brooklyn Bridge, and stopped for a photograph at
the ``Friends'' building in the West Village. They went to a different
rooftop bar every night. For lunch one day they picked up pastrami
sandwiches from \href{https://katzsdelicatessen.com/}{Katz's
Delicatessen} and ate them on the sidewalk. ``That was my first time
trying that,'' she said. ``I loved it.''

Although official tourism numbers
\href{https://www.nytimes.com/2020/07/24/nyregion/nyc-tourism-coronavirus.html}{are
dismal} (Krikor Daglian, owner of
\href{http://www.truetalesnyc.com/}{True Tales of NYC Walking Tours},
reopened his business on July 10 but has had no bookings from
out-of-towners, he said), some intrepid visitors are trickling in to
enjoy the great outdoors of New York City.

Broadway and the museums might be closed, but there is a booming
sidewalk dining scene, open streets and verdant parks, exciting
architecture on almost every block, and bridges, ferries and bike share
systems that connect the boroughs in the fresh air. If you can get here
safely (or have a few weeks to quarantine before venturing out), a
breezy, somewhat affordable vacation in New York is indeed possible
these days.

For some, the city is providing a stand-in for European destinations
where Americans are currently not welcome. ``We went to Rome and
Barcelona on our honeymoon, and eating outdoors and people watching was
our favorite part,'' said David Zavac, a government employee from
Toledo, who plans to bring his young family here once Ohio is taken off
the quarantine list. ``Now that New York City has that, it's really
appealing.''

\hypertarget{latest-updates-global-coronavirus-outbreak}{%
\section{\texorpdfstring{\href{https://www.nytimes.com/2020/08/04/world/coronavirus-cases.html?action=click\&pgtype=Article\&state=default\&region=MAIN_CONTENT_1\&context=storylines_live_updates}{Latest
Updates: Global Coronavirus
Outbreak}}{Latest Updates: Global Coronavirus Outbreak}}\label{latest-updates-global-coronavirus-outbreak}}

Updated 2020-08-05T07:58:24.076Z

\begin{itemize}
\tightlist
\item
  \href{https://www.nytimes.com/2020/08/04/world/coronavirus-cases.html?action=click\&pgtype=Article\&state=default\&region=MAIN_CONTENT_1\&context=storylines_live_updates\#link-762df92}{As
  talks drag on, McConnell signals openness to jobless aid extension,
  and negotiators agree on a deadline.}
\item
  \href{https://www.nytimes.com/2020/08/04/world/coronavirus-cases.html?action=click\&pgtype=Article\&state=default\&region=MAIN_CONTENT_1\&context=storylines_live_updates\#link-1228a480}{Novavax
  sees encouraging results from two studies of its experimental
  vaccine.}
\item
  \href{https://www.nytimes.com/2020/08/04/world/coronavirus-cases.html?action=click\&pgtype=Article\&state=default\&region=MAIN_CONTENT_1\&context=storylines_live_updates\#link-794484ed}{Mississippians
  must now wear masks in public, governor says.}
\end{itemize}

\href{https://www.nytimes.com/2020/08/04/world/coronavirus-cases.html?action=click\&pgtype=Article\&state=default\&region=MAIN_CONTENT_1\&context=storylines_live_updates}{See
more updates}

More live coverage:
\href{https://www.nytimes.com/live/2020/08/04/business/stock-market-today-coronavirus?action=click\&pgtype=Article\&state=default\&region=MAIN_CONTENT_1\&context=storylines_live_updates}{Markets}

Ms. Borelli's group even managed to experience some New York interiors
(wearing masks, of course). ``We visited Grand Central Terminal and
found all the places that are in so many movies and television shows,''
Ms. Borelli said. They took the subway to get around, something
\href{https://www.nytimes.com/2020/07/17/nyregion/coronavirus-subways-spread-nyc.html}{many
locals haven't even} started to do. ``I loved it because I am from a
small city in Brazil, so I don't have the subway there, and in Michigan
we also don't,'' she said. ``It's so cheap and easy and nice.''

But New York City during the pandemic is not a destination for the meek.
There are closures, mask requirements, shifting quarantine guidelines,
and the classic directness of New Yorkers who now, more than ever, want
their city to remain as safe as possible. But all of this can be part of
the experience.

``In Times Square we took our masks off to take a picture and someone
told us to put it back on,'' Ms. Borelli said. ``I thought it was nice.
I wish people would do that in Michigan.''

Helen Nunes, an au pair stationed in Stroudsburg, Pa., was less
enchanted with her visit. She drove to the city July 4 weekend to visit
a friend in Brooklyn. Friday night they went to an outdoor bar where
they drank caipirinhas and ate empanadas. Saturday they took the NYC
Ferry to Manhattan and walked around Times Square.

But Ms. Nunes, 28, never felt like she could relax. ``You need to wear
masks and cover your nose, so it's hard to walk,'' she said. ``I kept
thinking I would get coronavirus if I touched something. I wanted to
sanitize my hands all the time.''

Those who live in neighboring states and towns who had to cancel more
far-flung travel plans are also visiting the city, curious to see how it
has changed.

Stephanie Pirhala and John Lanning were supposed to visit family in the
Carolinas this summer, but decided not to after it was announced that
they would have to quarantine for 14 days upon their return to Sound
Beach, Long Island, where they live. So they decided to visit New York
City instead.

They started their getaway at the TWA Hotel, part of John F. Kennedy
International Airport, where Mr. Lanning, 39, a court officer, proposed
to Ms. Pirhala, 32, a bank analyst, at sunset on the roof overlooking
the Manhattan skyline on one side and planes taking off and landing on
the other. They were upgraded to the presidential suite because of the
special occasion.

\includegraphics{https://static01.nyt.com/images/2020/08/02/nyregion/02nyvirus-touristfood2/02nyvirus-touristfood2-articleLarge.jpg?quality=75\&auto=webp\&disable=upscale}

The couple then drove into the city where they had brunch outside and
walked around to see how New York had adjusted to the pandemic.

``We had been to the city for a concert right before everything shut
down, so we were there when things ended and we wanted to be there when
it opened again,'' Ms. Pirhala said. ``Just seeing everyone in masks, it
feels like a different world.''

\href{https://www.nytimes.com/news-event/coronavirus?action=click\&pgtype=Article\&state=default\&region=MAIN_CONTENT_3\&context=storylines_faq}{}

\hypertarget{the-coronavirus-outbreak-}{%
\subsubsection{The Coronavirus Outbreak
›}\label{the-coronavirus-outbreak-}}

\hypertarget{frequently-asked-questions}{%
\paragraph{Frequently Asked
Questions}\label{frequently-asked-questions}}

Updated August 4, 2020

\begin{itemize}
\item ~
  \hypertarget{i-have-antibodies-am-i-now-immune}{%
  \paragraph{I have antibodies. Am I now
  immune?}\label{i-have-antibodies-am-i-now-immune}}

  \begin{itemize}
  \tightlist
  \item
    As of right
    now,\href{https://www.nytimes.com/2020/07/22/health/covid-antibodies-herd-immunity.html?action=click\&pgtype=Article\&state=default\&region=MAIN_CONTENT_3\&context=storylines_faq}{that
    seems likely, for at least several months.} There have been
    frightening accounts of people suffering what seems to be a second
    bout of Covid-19. But experts say these patients may have a
    drawn-out course of infection, with the virus taking a slow toll
    weeks to months after initial exposure. People infected with the
    coronavirus typically
    \href{https://www.nature.com/articles/s41586-020-2456-9}{produce}
    immune molecules called antibodies, which are
    \href{https://www.nytimes.com/2020/05/07/health/coronavirus-antibody-prevalence.html?action=click\&pgtype=Article\&state=default\&region=MAIN_CONTENT_3\&context=storylines_faq}{protective
    proteins made in response to an
    infection}\href{https://www.nytimes.com/2020/05/07/health/coronavirus-antibody-prevalence.html?action=click\&pgtype=Article\&state=default\&region=MAIN_CONTENT_3\&context=storylines_faq}{.
    These antibodies may} last in the body
    \href{https://www.nature.com/articles/s41591-020-0965-6}{only two to
    three months}, which may seem worrisome, but that's perfectly normal
    after an acute infection subsides, said Dr. Michael Mina, an
    immunologist at Harvard University. It may be possible to get the
    coronavirus again, but it's highly unlikely that it would be
    possible in a short window of time from initial infection or make
    people sicker the second time.
  \end{itemize}
\item ~
  \hypertarget{im-a-small-business-owner-can-i-get-relief}{%
  \paragraph{I'm a small-business owner. Can I get
  relief?}\label{im-a-small-business-owner-can-i-get-relief}}

  \begin{itemize}
  \tightlist
  \item
    The
    \href{https://www.nytimes.com/article/small-business-loans-stimulus-grants-freelancers-coronavirus.html?action=click\&pgtype=Article\&state=default\&region=MAIN_CONTENT_3\&context=storylines_faq}{stimulus
    bills enacted in March} offer help for the millions of American
    small businesses. Those eligible for aid are businesses and
    nonprofit organizations with fewer than 500 workers, including sole
    proprietorships, independent contractors and freelancers. Some
    larger companies in some industries are also eligible. The help
    being offered, which is being managed by the Small Business
    Administration, includes the Paycheck Protection Program and the
    Economic Injury Disaster Loan program. But lots of folks have
    \href{https://www.nytimes.com/interactive/2020/05/07/business/small-business-loans-coronavirus.html?action=click\&pgtype=Article\&state=default\&region=MAIN_CONTENT_3\&context=storylines_faq}{not
    yet seen payouts.} Even those who have received help are confused:
    The rules are draconian, and some are stuck sitting on
    \href{https://www.nytimes.com/2020/05/02/business/economy/loans-coronavirus-small-business.html?action=click\&pgtype=Article\&state=default\&region=MAIN_CONTENT_3\&context=storylines_faq}{money
    they don't know how to use.} Many small-business owners are getting
    less than they expected or
    \href{https://www.nytimes.com/2020/06/10/business/Small-business-loans-ppp.html?action=click\&pgtype=Article\&state=default\&region=MAIN_CONTENT_3\&context=storylines_faq}{not
    hearing anything at all.}
  \end{itemize}
\item ~
  \hypertarget{what-are-my-rights-if-i-am-worried-about-going-back-to-work}{%
  \paragraph{What are my rights if I am worried about going back to
  work?}\label{what-are-my-rights-if-i-am-worried-about-going-back-to-work}}

  \begin{itemize}
  \tightlist
  \item
    Employers have to provide
    \href{https://www.osha.gov/SLTC/covid-19/standards.html}{a safe
    workplace} with policies that protect everyone equally.
    \href{https://www.nytimes.com/article/coronavirus-money-unemployment.html?action=click\&pgtype=Article\&state=default\&region=MAIN_CONTENT_3\&context=storylines_faq}{And
    if one of your co-workers tests positive for the coronavirus, the
    C.D.C.} has said that
    \href{https://www.cdc.gov/coronavirus/2019-ncov/community/guidance-business-response.html}{employers
    should tell their employees} -\/- without giving you the sick
    employee's name -\/- that they may have been exposed to the virus.
  \end{itemize}
\item ~
  \hypertarget{should-i-refinance-my-mortgage}{%
  \paragraph{Should I refinance my
  mortgage?}\label{should-i-refinance-my-mortgage}}

  \begin{itemize}
  \tightlist
  \item
    \href{https://www.nytimes.com/article/coronavirus-money-unemployment.html?action=click\&pgtype=Article\&state=default\&region=MAIN_CONTENT_3\&context=storylines_faq}{It
    could be a good idea,} because mortgage rates have
    \href{https://www.nytimes.com/2020/07/16/business/mortgage-rates-below-3-percent.html?action=click\&pgtype=Article\&state=default\&region=MAIN_CONTENT_3\&context=storylines_faq}{never
    been lower.} Refinancing requests have pushed mortgage applications
    to some of the highest levels since 2008, so be prepared to get in
    line. But defaults are also up, so if you're thinking about buying a
    home, be aware that some lenders have tightened their standards.
  \end{itemize}
\item ~
  \hypertarget{what-is-school-going-to-look-like-in-september}{%
  \paragraph{What is school going to look like in
  September?}\label{what-is-school-going-to-look-like-in-september}}

  \begin{itemize}
  \tightlist
  \item
    It is unlikely that many schools will return to a normal schedule
    this fall, requiring the grind of
    \href{https://www.nytimes.com/2020/06/05/us/coronavirus-education-lost-learning.html?action=click\&pgtype=Article\&state=default\&region=MAIN_CONTENT_3\&context=storylines_faq}{online
    learning},
    \href{https://www.nytimes.com/2020/05/29/us/coronavirus-child-care-centers.html?action=click\&pgtype=Article\&state=default\&region=MAIN_CONTENT_3\&context=storylines_faq}{makeshift
    child care} and
    \href{https://www.nytimes.com/2020/06/03/business/economy/coronavirus-working-women.html?action=click\&pgtype=Article\&state=default\&region=MAIN_CONTENT_3\&context=storylines_faq}{stunted
    workdays} to continue. California's two largest public school
    districts --- Los Angeles and San Diego --- said on July 13, that
    \href{https://www.nytimes.com/2020/07/13/us/lausd-san-diego-school-reopening.html?action=click\&pgtype=Article\&state=default\&region=MAIN_CONTENT_3\&context=storylines_faq}{instruction
    will be remote-only in the fall}, citing concerns that surging
    coronavirus infections in their areas pose too dire a risk for
    students and teachers. Together, the two districts enroll some
    825,000 students. They are the largest in the country so far to
    abandon plans for even a partial physical return to classrooms when
    they reopen in August. For other districts, the solution won't be an
    all-or-nothing approach.
    \href{https://bioethics.jhu.edu/research-and-outreach/projects/eschool-initiative/school-policy-tracker/}{Many
    systems}, including the nation's largest, New York City, are
    devising
    \href{https://www.nytimes.com/2020/06/26/us/coronavirus-schools-reopen-fall.html?action=click\&pgtype=Article\&state=default\&region=MAIN_CONTENT_3\&context=storylines_faq}{hybrid
    plans} that involve spending some days in classrooms and other days
    online. There's no national policy on this yet, so check with your
    municipal school system regularly to see what is happening in your
    community.
  \end{itemize}
\end{itemize}

Mr. Lanning said one of the best parts of the trip was supporting local
businesses. ``The restaurant couldn't have customers this whole time,''
he said, referring to the city's shutdown last spring. ``You don't want
to see places close up after all the suffering.''

This is especially true when your favorite restaurants are involved.

In non-pandemic times, Charles King, a photographer and videographer
from Luray, Va., visits New York City at least 30 times a year to shoot
weddings and other events. When he's here, he always makes a point of
visiting some of his favorite places, including Shake Shack. Since
March, however, Mr. King's New York gigs have dried up. So in June, when
he was asked to livestream a virtual graduation on Long Island and
passed through La Guardia Airport, he made a beeline for the Shake Shack
there.

``I ordered the Smoke Shack, the one with bacon on it,'' he said. ``It
was my first Shake Shack in months. I had a moment during the pandemic
when all I wanted were those crinkle fries.''

When Elle Andrews Patt drove 12 hours from Knoxville, Tenn., to help
move her daughter out of her Brooklyn apartment, she kept a low profile,
sleeping in her daughter's apartment and leaving almost as soon as she'd
arrived. ``This was a very different trip,'' she said. ``We didn't want
to expose anyone or get exposed by them.''

But the mother and daughter did manage to pick up slices from
\href{https://originaltonysofbushwick.com/}{Tony's Pizza} and bagels
from \href{http://www.kbbagel.com/}{Knickerbocker Bagel}, both in
Bushwick. ``They were open and friendly and recognized us,'' Ms. Andrews
said. ``I am glad they are still there.''

Then there are those who were planning to visit New York, but who have
stayed away because of their new turista non grata status.

Jamie Miller of Miami Beach planned to celebrate her anniversary here
this summer. The 34-year-old attorney and her husband had booked a suite
for two nights at
\href{https://www.themarkhotel.com/?gclid=EAIaIQobChMIrKP4jvDu6gIVFaSzCh3SdAsCEAAYASAAEgJM-vD_BwE}{The
Mark}, a boutique luxury hotel on the Upper East Side. They were looking
forward to dining at scenic open-air restaurants like
\href{https://grupogitano.com/}{Gitano Garden of Love} in SoHo or
\href{https://bryantparkgrillnyc.com/}{Bryant Park Grill} in Midtown.

But then, coronavirus cases in Florida skyrocketed and the state was put
on the quarantine list. The couple canceled the trip. ``It's so sad,''
Ms. Miller said. ``It's hot, and Covid is everywhere.''

Advertisement

\protect\hyperlink{after-bottom}{Continue reading the main story}

\hypertarget{site-index}{%
\subsection{Site Index}\label{site-index}}

\hypertarget{site-information-navigation}{%
\subsection{Site Information
Navigation}\label{site-information-navigation}}

\begin{itemize}
\tightlist
\item
  \href{https://help.nytimes.com/hc/en-us/articles/115014792127-Copyright-notice}{©~2020~The
  New York Times Company}
\end{itemize}

\begin{itemize}
\tightlist
\item
  \href{https://www.nytco.com/}{NYTCo}
\item
  \href{https://help.nytimes.com/hc/en-us/articles/115015385887-Contact-Us}{Contact
  Us}
\item
  \href{https://www.nytco.com/careers/}{Work with us}
\item
  \href{https://nytmediakit.com/}{Advertise}
\item
  \href{http://www.tbrandstudio.com/}{T Brand Studio}
\item
  \href{https://www.nytimes.com/privacy/cookie-policy\#how-do-i-manage-trackers}{Your
  Ad Choices}
\item
  \href{https://www.nytimes.com/privacy}{Privacy}
\item
  \href{https://help.nytimes.com/hc/en-us/articles/115014893428-Terms-of-service}{Terms
  of Service}
\item
  \href{https://help.nytimes.com/hc/en-us/articles/115014893968-Terms-of-sale}{Terms
  of Sale}
\item
  \href{https://spiderbites.nytimes.com}{Site Map}
\item
  \href{https://help.nytimes.com/hc/en-us}{Help}
\item
  \href{https://www.nytimes.com/subscription?campaignId=37WXW}{Subscriptions}
\end{itemize}
