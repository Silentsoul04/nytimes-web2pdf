Sections

SEARCH

\protect\hyperlink{site-content}{Skip to
content}\protect\hyperlink{site-index}{Skip to site index}

\href{https://www.nytimes.com/section/nyregion}{New York}

\href{https://myaccount.nytimes.com/auth/login?response_type=cookie\&client_id=vi}{}

\href{https://www.nytimes.com/section/todayspaper}{Today's Paper}

\href{/section/nyregion}{New York}\textbar{}A Protester, an Unmarked
N.Y.P.D. Van and a Viral Video

\url{https://nyti.ms/30Ze7v3}

\begin{itemize}
\item
\item
\item
\item
\item
\item
\end{itemize}

\href{https://www.nytimes.com/news-event/george-floyd-protests-minneapolis-new-york-los-angeles?action=click\&pgtype=Article\&state=default\&region=TOP_BANNER\&context=storylines_menu}{Race
and America}

\begin{itemize}
\tightlist
\item
  \href{https://www.nytimes.com/2020/07/26/us/protests-portland-seattle-trump.html?action=click\&pgtype=Article\&state=default\&region=TOP_BANNER\&context=storylines_menu}{Protesters
  Return to Other Cities}
\item
  \href{https://www.nytimes.com/2020/07/24/us/portland-oregon-protests-white-race.html?action=click\&pgtype=Article\&state=default\&region=TOP_BANNER\&context=storylines_menu}{Portland
  at the Center}
\item
  \href{https://www.nytimes.com/2020/07/23/podcasts/the-daily/portland-protests.html?action=click\&pgtype=Article\&state=default\&region=TOP_BANNER\&context=storylines_menu}{Podcast:
  Showdown in Portland}
\item
  \href{https://www.nytimes.com/interactive/2020/07/16/us/black-lives-matter-protests-louisville-breonna-taylor.html?action=click\&pgtype=Article\&state=default\&region=TOP_BANNER\&context=storylines_menu}{45
  Days in Louisville}
\end{itemize}

Advertisement

\protect\hyperlink{after-top}{Continue reading the main story}

Supported by

\protect\hyperlink{after-sponsor}{Continue reading the main story}

New York Today

\hypertarget{a-protester-an-unmarked-nypd-van-and-a-viral-video}{%
\section{A Protester, an Unmarked N.Y.P.D. Van and a Viral
Video}\label{a-protester-an-unmarked-nypd-van-and-a-viral-video}}

By Juliana Kim

\begin{itemize}
\item
  July 30, 2020
\item
  \begin{itemize}
  \item
  \item
  \item
  \item
  \item
  \item
  \end{itemize}
\end{itemize}

\emph{{[}Want to get New York Today by email?}
\href{https://www.nytimes.com/newsletters/newyorktoday}{\emph{Here's the
sign-up}}\emph{.{]}}

\textbf{It's Thursday.}

\textbf{Weather:} ** Mostly sunny, with a high in the low 90s. Chance of
an afternoon thunderstorm. **

\textbf{Alternate-side parking}: Suspended through Sunday.

\begin{center}\rule{0.5\linewidth}{\linethickness}\end{center}

The
\href{https://twitter.com/Naddleez/status/1288232816451432453}{videos
showing the arrest} quickly went viral: New York City police officers,
some in plain clothes, interrupted a peaceful march against police
brutality on Tuesday and pulled a protester into an unmarked minivan.

On social media, people shared the video thousands of times and
immediately compared the tactics they observed with those used by
federal agents at protests in Portland, Ore.

With the ``anxiety about what's happening in Portland, the N.Y.P.D.
deploying unmarked vans with plainclothes cops to make street arrests of
protesters feels more like provocation than public safety,''
\href{https://twitter.com/bradlander/status/1288289187880476672}{Councilman
Brad Lander of Brooklyn wrote on Twitter.}

The Police Department said the protester, Nikki Stone, was arrested in
connection with ``damaging police cameras during five separate criminal
incidents in and around City Hall Park.'' She was charged with criminal
mischief and vandalism.

Mayor Bill de Blasio on Wednesday suggested that the arrest was
justified, but he called the execution ``troubling.'' ``It was the wrong
time and the wrong place to effectuate that arrest,'' he said at a news
briefing.

Here's what you need to know.

\hypertarget{the-details}{%
\subsubsection{The details}\label{the-details}}

Ms. Stone, 18, was arrested at about 6 p.m. Tuesday at Second Avenue and
East 25th Street in the Kips Bay section of Manhattan as she
participated in a demonstration.

People at the scene and those who viewed video of the arrest seemed
shocked, and also confused about exactly what had occurred. Most videos
showed a silver van driving alongside the protesters when officers, some
wearing bulletproof vests and others in plain clothes, jumped out of the
vehicle to detain Ms. Stone. Almost immediately, officers on bicycles
blocked anyone from intervening.

Hours later, she was released from police custody with a desk appearance
ticket, requiring her to return to court at a later date.

On Wednesday, Chief of Detectives Rodney K. Harrison posted a
\href{https://twitter.com/NYPDDetectives/status/1288493685571821568}{video
montage} on Twitter that appeared to show a woman vandalizing police
cameras.

``The N.Y.P.D. welcomes peaceful protests,'' Chief Harrison wrote.
``However, damage to N.Y.P.D. technology that helps keep this city safe
will never be tolerated.''

\hypertarget{the-context}{%
\subsubsection{The context}\label{the-context}}

Ms. Stone was protesting the July 22 clearing of the Occupy City Hall
encampment at City Hall Park. Her sudden arrest drew parallels to
tactics used in Portland.

There, federal agents have been clashing with protesters on the streets,
with personnel without obvious markings
\href{https://www.nytimes.com/2020/07/17/us/portland-protests.html}{pulling
demonstrators into unmarked vans}.

\emph{{[}Read more about}
\href{https://www.nytimes.com/2020/07/28/nyregion/nypd-protester-van.html}{\emph{how
the video of the protester's arrest drew criticism}}\emph{.{]}}

The New York police said that warrant squads typically use unmarked
vehicles to look for people wanted in connection with crimes, and that
officers were following standard procedure.

\hypertarget{the-reaction}{%
\subsubsection{The reaction}\label{the-reaction}}

Several city and state officials joined Mr. Lander in criticizing the
incident.

Councilwoman Carlina Rivera, who represents the district where the
arrest took place, called the strategy ``a massive overstep.''
Comptroller Scott M. Stringer said he was ``deeply concerned,'' and
Council Speaker Corey Johnson called the arrest ``totally
unacceptable.''

At a news briefing on Wednesday, Gov. Andrew M. Cuomo described the
video as ``disturbing.''

``I'm surprised that, especially at this time, the N.Y.P.D. would take
such an obnoxious action,'' Mr. Cuomo said. ``It was wholly insensitive
to everything that has gone on.''

In a statement, Paul DiGiacomo, the president of the N.Y.P.D.
Detectives' Endowment Association, responded to the governor:
``Detectives did what the government asked of them. What's `obnoxious'
is your unjustified criticism of those men and women who are holding
this city together, and the only ones preventing its descent into
lawlessness.''

\begin{center}\rule{0.5\linewidth}{\linethickness}\end{center}

\hypertarget{from-the-times}{%
\subsection{From The Times}\label{from-the-times}}

\href{https://www.nytimes.com/2020/07/30/nyregion/seth-ducharme-us-attorney-brooklyn.html}{Why
Barr's Pick for Brooklyn Prosecutor Faces Scrutiny From All Sides}

\href{https://www.nytimes.com/2020/07/29/nyregion/lake-solitude-closed-racism.html}{`Hidden
Gem' Made Popular by TikTok Is Shut to Keep Out-of-Towners Away}

\href{https://www.nytimes.com/2020/07/28/arts/design/met-museum-wangechi-mutu.html}{Met
Museum Acquires Two Sculptures by Wangechi Mutu}

\href{https://www.nytimes.com/2020/07/29/arts/bronx-zoo-apology-racism.html}{Racist
Incident From Bronx Zoo's Past Draws Apology}

Want more news? \href{https://www.nytimes.com/section/nyregion}{Check
out our full coverage}.

\textbf{The Mini Crossword:} Here is
\href{https://www.nytimes.com/crosswords/game/mini}{today's puzzle}.

\begin{center}\rule{0.5\linewidth}{\linethickness}\end{center}

\hypertarget{what-were-reading}{%
\subsection{What we're reading}\label{what-were-reading}}

A nonbinary law student has sued to get \textbf{an ``X'' gender option}
on New York driver's licenses.
{[}\href{https://gothamist.com/news/nonbinary-nyu-law-student-sues-get-x-gender-option-ny-drivers-licenses}{Gothamist}{]}

The city's Department of Education handed out over \textbf{320,000
iPads} to students. Now it needs them back.
{[}\href{https://ny.chalkbeat.org/2020/7/29/21347043/remote-learning-devices-distribution-nyc}{Chalkbeat}{]}

People are \textbf{injecting drugs} in broad daylight in Midtown
Manhattan.
{[}\href{https://nypost.com/2020/07/28/apparent-junkies-turn-part-of-nycs-midtown-into-shooting-gallery/}{New
York Post}{]}

\begin{center}\rule{0.5\linewidth}{\linethickness}\end{center}

\hypertarget{and-finally-july-in-nyc-168-years-ago}{%
\subsection{And finally: July in N.Y.C., 168 years
ago}\label{and-finally-july-in-nyc-168-years-ago}}

For decades, The Times has been chronicling broiling days like the ones
we're experiencing now, but 168 years ago this week it published one of
the paper's earliest and most extraordinary weather stories.

Ben Weiser, who covers the Manhattan federal courts for The Times, was
browsing through the paper's voluminous electronic database in 2013 when
he stumbled upon that article, titled ``The Streets in Midsummer,''
which reported on New York City's stifling-hot summer of 1852.

``It dawned on me that this might be The Times's first weather story,''
he told me.

Mr. Weiser set out to learn more about the 1,500-word piece, published
in the first year of the newspaper's existence, and he discovered that
it was about much more than the weather.

It was ``full of meticulous detail, social commentary, references to art
and literature, and overwrought prose,''
\href{https://archive.nytimes.com/www.nytimes.com/interactive/2013/07/28/nyregion/heat-struck-july-1852.html}{Mr.
Weiser wrote in an article in 2013} about the original piece and the
window it offered into how the paper covered the city nearly a decade
before the Civil War.

Dust, for example, was everywhere. As the 1852 article noted, the city,
with 515,000 people, was coated with a mix of ``decomposed vegetable
matters; the filth left there by thousands of passing animals; all
conceivable sources of dirt, and all degrees of rottenness.''

In addition, the carcasses of dead horses and other animals often lay
unattended, as did discarded food waste. Piles of manure stood outside
stables and in the street. The city had no organized sanitation service.

``We have no idea how terrible the city smelled,'' Jon A. Peterson, a
professor emeritus of history at Queens College, said in Mr. Weiser's
article.

\emph{It's Thursday --- stay cool.}

\begin{center}\rule{0.5\linewidth}{\linethickness}\end{center}

\hypertarget{metropolitan-diary-big-birthday}{%
\subsection{Metropolitan Diary: Big
birthday}\label{metropolitan-diary-big-birthday}}

\includegraphics{https://static01.nyt.com/images/2020/07/26/nyregion/26diary-illos-03/26diary-illos-03-articleLarge.jpg?quality=75\&auto=webp\&disable=upscale}

Dear Diary:

It was March 2013, and I was having a big birthday. Because the date
fell on a weekday, my wife and I had delayed going out for a celebratory
dinner until Saturday. I was looking forward to that, but my actual
birthday just didn't feel memorable.

Normally, I walked the mile from my home to the office and back, but
that night, I decided to take the M102 bus instead. I boarded the bus at
42nd Street and Third Avenue.

``Whose birthday is it today?'' the driver said. ``It must be
somebody's.''

After thinking it over for a moment, I spoke up somewhat sheepishly to
say that it was my birthday. Another passenger did too.

After asking our names, the driver encouraged the bus full of strangers
to sing ``Happy Birthday'' to us.

Which they did, with gusto.

\emph{--- Richard Rubenstein}

\begin{center}\rule{0.5\linewidth}{\linethickness}\end{center}

\emph{New York Today is published weekdays around 6 a.m.}
\href{https://www.nytimes.com/newsletters/newyorktoday?module=inline}{\emph{Sign
up here}} \emph{to get it by email. You can also find it at}
\href{http://www.nytoday.com/}{\emph{nytoday.com}}\emph{.}

\emph{We're experimenting with the format of New York Today. What would
you like to see more (or less) of? Post a comment or email us:}
\href{mailto:nytoday@nytimes.com}{\emph{nytoday@nytimes.com}}\emph{.}

Advertisement

\protect\hyperlink{after-bottom}{Continue reading the main story}

\hypertarget{site-index}{%
\subsection{Site Index}\label{site-index}}

\hypertarget{site-information-navigation}{%
\subsection{Site Information
Navigation}\label{site-information-navigation}}

\begin{itemize}
\tightlist
\item
  \href{https://help.nytimes.com/hc/en-us/articles/115014792127-Copyright-notice}{©~2020~The
  New York Times Company}
\end{itemize}

\begin{itemize}
\tightlist
\item
  \href{https://www.nytco.com/}{NYTCo}
\item
  \href{https://help.nytimes.com/hc/en-us/articles/115015385887-Contact-Us}{Contact
  Us}
\item
  \href{https://www.nytco.com/careers/}{Work with us}
\item
  \href{https://nytmediakit.com/}{Advertise}
\item
  \href{http://www.tbrandstudio.com/}{T Brand Studio}
\item
  \href{https://www.nytimes.com/privacy/cookie-policy\#how-do-i-manage-trackers}{Your
  Ad Choices}
\item
  \href{https://www.nytimes.com/privacy}{Privacy}
\item
  \href{https://help.nytimes.com/hc/en-us/articles/115014893428-Terms-of-service}{Terms
  of Service}
\item
  \href{https://help.nytimes.com/hc/en-us/articles/115014893968-Terms-of-sale}{Terms
  of Sale}
\item
  \href{https://spiderbites.nytimes.com}{Site Map}
\item
  \href{https://help.nytimes.com/hc/en-us}{Help}
\item
  \href{https://www.nytimes.com/subscription?campaignId=37WXW}{Subscriptions}
\end{itemize}
