Sections

SEARCH

\protect\hyperlink{site-content}{Skip to
content}\protect\hyperlink{site-index}{Skip to site index}

\href{https://www.nytimes.com/section/nyregion}{New York}

\href{https://myaccount.nytimes.com/auth/login?response_type=cookie\&client_id=vi}{}

\href{https://www.nytimes.com/section/todayspaper}{Today's Paper}

\href{/section/nyregion}{New York}\textbar{}\$25,000 Pod Schools: How
Well-to-Do Children Will Weather the Pandemic

\url{https://nyti.ms/338TEGU}

\begin{itemize}
\item
\item
\item
\item
\item
\item
\end{itemize}

\href{https://www.nytimes.com/news-event/coronavirus?action=click\&pgtype=Article\&state=default\&region=TOP_BANNER\&context=storylines_menu}{The
Coronavirus Outbreak}

\begin{itemize}
\tightlist
\item
  live\href{https://www.nytimes.com/2020/08/01/world/coronavirus-covid-19.html?action=click\&pgtype=Article\&state=default\&region=TOP_BANNER\&context=storylines_menu}{Latest
  Updates}
\item
  \href{https://www.nytimes.com/interactive/2020/us/coronavirus-us-cases.html?action=click\&pgtype=Article\&state=default\&region=TOP_BANNER\&context=storylines_menu}{Maps
  and Cases}
\item
  \href{https://www.nytimes.com/interactive/2020/science/coronavirus-vaccine-tracker.html?action=click\&pgtype=Article\&state=default\&region=TOP_BANNER\&context=storylines_menu}{Vaccine
  Tracker}
\item
  \href{https://www.nytimes.com/interactive/2020/07/29/us/schools-reopening-coronavirus.html?action=click\&pgtype=Article\&state=default\&region=TOP_BANNER\&context=storylines_menu}{What
  School May Look Like}
\item
  \href{https://www.nytimes.com/live/2020/07/31/business/stock-market-today-coronavirus?action=click\&pgtype=Article\&state=default\&region=TOP_BANNER\&context=storylines_menu}{Economy}
\end{itemize}

Advertisement

\protect\hyperlink{after-top}{Continue reading the main story}

Supported by

\protect\hyperlink{after-sponsor}{Continue reading the main story}

\hypertarget{25000-pod-schools-how-well-to-do-children-will-weather-the-pandemic}{%
\section{\$25,000 Pod Schools: How Well-to-Do Children Will Weather the
Pandemic}\label{25000-pod-schools-how-well-to-do-children-will-weather-the-pandemic}}

``This is our emergency fund,'' said one parent considering a pod
school. ``And this is our emergency.''

\includegraphics{https://static01.nyt.com/images/2020/07/20/nyregion/nypodschools1/nypodschools1-articleLarge.jpg?quality=75\&auto=webp\&disable=upscale}

By David Zweig

\begin{itemize}
\item
  July 30, 2020
\item
  \begin{itemize}
  \item
  \item
  \item
  \item
  \item
  \item
  \end{itemize}
\end{itemize}

One Sunday afternoon in July, around a dozen parents gathered in a
backyard in Hastings-on-Hudson, a village about a half-hour north of
Midtown Manhattan. Another dozen joined them via Zoom. Folding chairs
had been placed at prudent distances, and masks were dutifully worn. An
Australian Labradoodle belonging to the home's owners strutted among the
guests.

Despite the breezy suburban tableau, the occasion was fraught. Over the
preceding months, the adults in the backyard had grown increasingly
concerned about the coming school year, which, it had become clear,
might put students in class, at best, only part-time. Parents were
determined to avoid having their children sit alone for hours each day,
withering in the gray light of a Chromebook.

A young woman named Cate Han, one of the founders of the
\href{https://www.hudsonlabschool.com/}{Hudson Lab School}, in Hastings,
along with partners from
\href{https://www.portfolio-school.com/}{Portfolio School}, in TriBeCa,
were in the yard to pitch a solution: Learning Pods. ``We looked at the
situation and asked, `What makes sense?''' Ms. Han said. ``A social
bubble, consistent schooling, and have it be with a teacher in person.''

Babur Habib, a Portfolio founder, took a more visionary tone. ``Yes, we
are in a pandemic,'' he said. ``But when it comes to education, we also
feel some good may even come out of this.''

One thing was certain. It was going to be very expensive --- maybe
prohibitively so --- but these parents wanted a solution. Many had
chosen to live in Hastings because of its outstanding public schools.
Now they were considering withdrawing their children and embracing a
novel teaching model that could have implications for public education
for years.

There might be no more potent symbol of inequality during the pandemic
than
\href{https://www.nytimes.com/2020/07/22/parenting/school-pods-coronavirus.html}{the
pod school}: A single semester in a Hudson Lab pod can cost more than
\$13,000.

This fall, a majority of
\href{https://nces.ed.gov/programs/digest/d18/tables/dt18_203.10.asp}{50
million American children} enrolled in public school are almost
certainly going to be confined within their homes for part or all of the
school day. The numerous harms of being kept out of school --- academic,
social, emotional, psychological, physical --- are felt by all children,
but a disproportionate weight will
be\href{https://www.nytimes.com/2020/06/05/us/coronavirus-education-lost-learning.html}{borne}
by those with the least resources. The wealthiest children will be
ensconced in private schools and catered to by tutors and nannies. For
most, there are few options.

But for a slice of enterprising American parents with resources,
so-called pod schools have arrived. Cropping up all over, these small
educational groups aim to offset the looming wreckage of a national
experiment in distance learning. Among the many options are
\href{https://www.getschoolhouse.com/}{School House}, based in New York
City, which is offering ``micro schools'' around the country, and
\href{https://www.whizkidztutoring.com/wk-pods}{Whiz Kidz}, based in
nearby Irvington.

The idea of learning pods, both formally organized by outfits like the
Hudson Lab, as well as more ad hoc parent-run variations, appears to be
speaking to parents who feel that government agencies and school
districts have failed to plan for their children.

\includegraphics{https://static01.nyt.com/images/2020/07/20/nyregion/nypodschools3/nypodschools3-articleLarge.jpg?quality=75\&auto=webp\&disable=upscale}

The program presented by Ms. Han and her colleagues that afternoon
sounded relatively simple.

Parents will form groups of three to 10 children, usually in the same
grade. The ``pods'' will meet each day from around 9 a.m. to 2 p.m. in a
host parent's basement or cordoned-off living room or perhaps somewhere
outdoors, to learn from a teacher provided by the Hudson Lab.

\hypertarget{latest-updates-global-coronavirus-outbreak}{%
\section{\texorpdfstring{\href{https://www.nytimes.com/2020/08/01/world/coronavirus-covid-19.html?action=click\&pgtype=Article\&state=default\&region=MAIN_CONTENT_1\&context=storylines_live_updates}{Latest
Updates: Global Coronavirus
Outbreak}}{Latest Updates: Global Coronavirus Outbreak}}\label{latest-updates-global-coronavirus-outbreak}}

Updated 2020-08-01T23:43:33.308Z

\begin{itemize}
\tightlist
\item
  \href{https://www.nytimes.com/2020/08/01/world/coronavirus-covid-19.html?action=click\&pgtype=Article\&state=default\&region=MAIN_CONTENT_1\&context=storylines_live_updates\#link-34047410}{The
  U.S. reels as July cases more than double the total of any other
  month.}
\item
  \href{https://www.nytimes.com/2020/08/01/world/coronavirus-covid-19.html?action=click\&pgtype=Article\&state=default\&region=MAIN_CONTENT_1\&context=storylines_live_updates\#link-3ac56579}{Top
  officials work to break impasse over jobless benefit.}
\item
  \href{https://www.nytimes.com/2020/08/01/world/coronavirus-covid-19.html?action=click\&pgtype=Article\&state=default\&region=MAIN_CONTENT_1\&context=storylines_live_updates\#link-25930521}{Thousands
  in Berlin protest Germany's coronavirus measures.}
\end{itemize}

\href{https://www.nytimes.com/2020/08/01/world/coronavirus-covid-19.html?action=click\&pgtype=Article\&state=default\&region=MAIN_CONTENT_1\&context=storylines_live_updates}{See
more updates}

More live coverage:
\href{https://www.nytimes.com/live/2020/07/31/business/stock-market-today-coronavirus?action=click\&pgtype=Article\&state=default\&region=MAIN_CONTENT_1\&context=storylines_live_updates}{Markets}

Each \href{https://www.learning-pods.com/}{pod} for grades K-4 will cost
\$125,000 for the academic year, or \$68,750 for a five-month
commitment. With five children in a pod, for example, the cost per
student will run \$13,750 per semester. The more children in the pod,
the lower the cost per student. There is different pricing and reduced
hours for preschool pods. (Both H.L.S. and School House are seeking
partnerships with foundations so they can offer scholarships.)

``My children didn't get an education this spring,'' said Ronit
Sukenick, who offered her backyard for the Hudson Lab presentation. She
has two boys, Jonah and David, who experienced the last three months of
second and sixth grade at their dining table.

``None of the things that were great about my sons' schools translated
to remote learning,'' she said. ``No science experiments or a teacher at
their sides. No interaction with friends.''

The loss of both the academic and social aspects of school was a refrain
from more than a dozen parents I spoke with regarding distance learning.

Ms. Han appealed to the palpable anxiety among the parents in the yard.
``We are social beings,'' she said as a few parents nodded in agreement.
``Our identities are formed based on our interactions with others.''
Learning for children is tied to being with peers, she said.

Why should a child suffer in isolation, struggling to learn how to read
through a computer, when they could thrive with a small group of peers
and a real teacher instead?

To enroll full-time in the Hudson Lab learning pods, parents withdraw
their children from the public school and register as home-schoolers, a
process that requires parents to submit academic plans to their district
for approval. Hudson Lab also offers an option that allows students to
remain enrolled in school remotely while using a learning pod as a
supplement.

Pod families also need to draw up agreements on the various facets of
distancing behavior each child and family will be obligated to. Is it OK
if someone gets a haircut in the city? What about play dates outside the
pod?

Aside from safety concerns, other issues will surely arise --- different
learning styles, teacher and parent goals. It all needs to be worked
out. But as part of its fee, the Hudson Lab School will help with the
paperwork, mediate parent interaction with the teacher, and align pod
curriculums to state standards.

Pod programs offer an inoculation against the possibility that public
schools might close. As long as the community isn't in complete
lockdown, the learning pods can keep going.

Lauren Lazarin is a single mother in Riverdale, in the Bronx, with a
limited income, but she's considering signing up for a Hudson Lab pod.
``If I have to ditch my salaried job as an early childhood educator and
work as a private tutor so my 5-year-old daughter can get her education
and I can keep my family safe,'' she said, ``then I might have to make
that choice.''

For Ms. Sukenick, who works as a physical therapist, the choice is
stark. She can either quit working while she looks after her sons when
they're home, or she can keep her very fulfilling job and deplete her
savings to pay for the pods. ``This is our emergency fund,'' she said,
noting they were thankful to have it. ``And this is our emergency.''

Erica Paris, a stay-at-home mom in Towaco, N.J., recently received her
daughters' fall schedule from the school district. Her 8-year-old,
Alexa, will attend elementary school four half-days per week; Emma, 12,
will attend middle school just two half-days per week. Ms. Paris said
she will have to hire someone to help with remote learning during their
copious time out of school. ``I respect educators. I don't know how to
take the curriculum and do it myself,'' she said. ``I can't be both a
teacher and a parent.''

Image

Parents attended the~backyard pod-school presentation by
Zoom.Credit...Gregg Vigliotti for The New York Times

She has reached out to other parents ``to see if their girls want to do
a small group and share a tutor.'' For Ms. Paris, the sharing of a tutor
is less about saving money than for the advantage of her girls being in
a group setting. Her daughters' schedules are also baffling and
frustrating.

``The reason they have half-days is because the teachers' contract
guarantees a duty-free lunch, which is understandable,'' she said. ``But
couldn't we figure out how to hire aides for an hour? Two half-days of
school a week is ridiculous.''

\href{https://www.nytimes.com/news-event/coronavirus?action=click\&pgtype=Article\&state=default\&region=MAIN_CONTENT_3\&context=storylines_faq}{}

\hypertarget{the-coronavirus-outbreak-}{%
\subsubsection{The Coronavirus Outbreak
›}\label{the-coronavirus-outbreak-}}

\hypertarget{frequently-asked-questions}{%
\paragraph{Frequently Asked
Questions}\label{frequently-asked-questions}}

Updated July 27, 2020

\begin{itemize}
\item ~
  \hypertarget{should-i-refinance-my-mortgage}{%
  \paragraph{Should I refinance my
  mortgage?}\label{should-i-refinance-my-mortgage}}

  \begin{itemize}
  \tightlist
  \item
    \href{https://www.nytimes.com/article/coronavirus-money-unemployment.html?action=click\&pgtype=Article\&state=default\&region=MAIN_CONTENT_3\&context=storylines_faq}{It
    could be a good idea,} because mortgage rates have
    \href{https://www.nytimes.com/2020/07/16/business/mortgage-rates-below-3-percent.html?action=click\&pgtype=Article\&state=default\&region=MAIN_CONTENT_3\&context=storylines_faq}{never
    been lower.} Refinancing requests have pushed mortgage applications
    to some of the highest levels since 2008, so be prepared to get in
    line. But defaults are also up, so if you're thinking about buying a
    home, be aware that some lenders have tightened their standards.
  \end{itemize}
\item ~
  \hypertarget{what-is-school-going-to-look-like-in-september}{%
  \paragraph{What is school going to look like in
  September?}\label{what-is-school-going-to-look-like-in-september}}

  \begin{itemize}
  \tightlist
  \item
    It is unlikely that many schools will return to a normal schedule
    this fall, requiring the grind of
    \href{https://www.nytimes.com/2020/06/05/us/coronavirus-education-lost-learning.html?action=click\&pgtype=Article\&state=default\&region=MAIN_CONTENT_3\&context=storylines_faq}{online
    learning},
    \href{https://www.nytimes.com/2020/05/29/us/coronavirus-child-care-centers.html?action=click\&pgtype=Article\&state=default\&region=MAIN_CONTENT_3\&context=storylines_faq}{makeshift
    child care} and
    \href{https://www.nytimes.com/2020/06/03/business/economy/coronavirus-working-women.html?action=click\&pgtype=Article\&state=default\&region=MAIN_CONTENT_3\&context=storylines_faq}{stunted
    workdays} to continue. California's two largest public school
    districts --- Los Angeles and San Diego --- said on July 13, that
    \href{https://www.nytimes.com/2020/07/13/us/lausd-san-diego-school-reopening.html?action=click\&pgtype=Article\&state=default\&region=MAIN_CONTENT_3\&context=storylines_faq}{instruction
    will be remote-only in the fall}, citing concerns that surging
    coronavirus infections in their areas pose too dire a risk for
    students and teachers. Together, the two districts enroll some
    825,000 students. They are the largest in the country so far to
    abandon plans for even a partial physical return to classrooms when
    they reopen in August. For other districts, the solution won't be an
    all-or-nothing approach.
    \href{https://bioethics.jhu.edu/research-and-outreach/projects/eschool-initiative/school-policy-tracker/}{Many
    systems}, including the nation's largest, New York City, are
    devising
    \href{https://www.nytimes.com/2020/06/26/us/coronavirus-schools-reopen-fall.html?action=click\&pgtype=Article\&state=default\&region=MAIN_CONTENT_3\&context=storylines_faq}{hybrid
    plans} that involve spending some days in classrooms and other days
    online. There's no national policy on this yet, so check with your
    municipal school system regularly to see what is happening in your
    community.
  \end{itemize}
\item ~
  \hypertarget{is-the-coronavirus-airborne}{%
  \paragraph{Is the coronavirus
  airborne?}\label{is-the-coronavirus-airborne}}

  \begin{itemize}
  \tightlist
  \item
    The coronavirus
    \href{https://www.nytimes.com/2020/07/04/health/239-experts-with-one-big-claim-the-coronavirus-is-airborne.html?action=click\&pgtype=Article\&state=default\&region=MAIN_CONTENT_3\&context=storylines_faq}{can
    stay aloft for hours in tiny droplets in stagnant air}, infecting
    people as they inhale, mounting scientific evidence suggests. This
    risk is highest in crowded indoor spaces with poor ventilation, and
    may help explain super-spreading events reported in meatpacking
    plants, churches and restaurants.
    \href{https://www.nytimes.com/2020/07/06/health/coronavirus-airborne-aerosols.html?action=click\&pgtype=Article\&state=default\&region=MAIN_CONTENT_3\&context=storylines_faq}{It's
    unclear how often the virus is spread} via these tiny droplets, or
    aerosols, compared with larger droplets that are expelled when a
    sick person coughs or sneezes, or transmitted through contact with
    contaminated surfaces, said Linsey Marr, an aerosol expert at
    Virginia Tech. Aerosols are released even when a person without
    symptoms exhales, talks or sings, according to Dr. Marr and more
    than 200 other experts, who
    \href{https://academic.oup.com/cid/article/doi/10.1093/cid/ciaa939/5867798}{have
    outlined the evidence in an open letter to the World Health
    Organization}.
  \end{itemize}
\item ~
  \hypertarget{what-are-the-symptoms-of-coronavirus}{%
  \paragraph{What are the symptoms of
  coronavirus?}\label{what-are-the-symptoms-of-coronavirus}}

  \begin{itemize}
  \tightlist
  \item
    Common symptoms
    \href{https://www.nytimes.com/article/symptoms-coronavirus.html?action=click\&pgtype=Article\&state=default\&region=MAIN_CONTENT_3\&context=storylines_faq}{include
    fever, a dry cough, fatigue and difficulty breathing or shortness of
    breath.} Some of these symptoms overlap with those of the flu,
    making detection difficult, but runny noses and stuffy sinuses are
    less common.
    \href{https://www.nytimes.com/2020/04/27/health/coronavirus-symptoms-cdc.html?action=click\&pgtype=Article\&state=default\&region=MAIN_CONTENT_3\&context=storylines_faq}{The
    C.D.C. has also} added chills, muscle pain, sore throat, headache
    and a new loss of the sense of taste or smell as symptoms to look
    out for. Most people fall ill five to seven days after exposure, but
    symptoms may appear in as few as two days or as many as 14 days.
  \end{itemize}
\item ~
  \hypertarget{does-asymptomatic-transmission-of-covid-19-happen}{%
  \paragraph{Does asymptomatic transmission of Covid-19
  happen?}\label{does-asymptomatic-transmission-of-covid-19-happen}}

  \begin{itemize}
  \tightlist
  \item
    So far, the evidence seems to show it does. A widely cited
    \href{https://www.nature.com/articles/s41591-020-0869-5}{paper}
    published in April suggests that people are most infectious about
    two days before the onset of coronavirus symptoms and estimated that
    44 percent of new infections were a result of transmission from
    people who were not yet showing symptoms. Recently, a top expert at
    the World Health Organization stated that transmission of the
    coronavirus by people who did not have symptoms was ``very rare,''
    \href{https://www.nytimes.com/2020/06/09/world/coronavirus-updates.html?action=click\&pgtype=Article\&state=default\&region=MAIN_CONTENT_3\&context=storylines_faq\#link-1f302e21}{but
    she later walked back that statement.}
  \end{itemize}
\end{itemize}

I've been similarly disheartened over the pending school policies in
Hastings. I was in the Sukenicks' back yard that afternoon because
remote learning this spring for my children, ages 9 and 11, had been an
abysmal affair.

My younger son would blast through the days' assignments, often
consisting of watching videos and checking boxes. By 9:30 each morning
he'd confront me, panicked: ``I'm done. Now what do I do?'' My daughter
was saddled with extensive projects and a detailed course load she often
needed help with.

Mainly it was a profoundly lonely time for them. The neighborhood kids
shut off from one another, hours were spent alone in their rooms,
dissolving into electronic screens.

And I'm aware my children are more fortunate than many others.

The many parents I've spoken with for this article have expressed
sadness and unease over the inequities that a remote-learning model
engenders. Yet allowing their children to suffer is not the remedy to a
systemic problem.

As a journalist I began scouring the research on children and Covid-19
when our schools shut down some five months ago. I haven't climbed out
of the data pit since.

The preponderance of evidence suggests that children, by and large, are
spared dangerous effects of the virus.

\href{https://www.folkhalsomyndigheten.se/contentassets/5e248b82cc284971a1c5fd922e7770f8/forekomst-covid-19-olika-yrkesgrupper.pdf}{Data}
from Sweden, where lower schools have been open for the entirety of the
pandemic, without specific distancing mandates or children in masks,
show that teachers were at no greater risk than other professionals.

Philosophically, I believe schools should be among the last institutions
to close and the very first to open. Many experts have argued similarly,
including in
\href{https://www.nytimes.com/2020/07/01/opinion/coronavirus-schools.html}{The
Times}. But I am aware plenty of people feel differently.

What I can say here, though, is what I know to be true for my family.
After months of solitude, where screen time essentially became time
itself, my kids have finally escaped. They've been in camp for the past
few weeks, playing and interacting with other children all day. The
transformation has been akin to a time-lapse video of the reanimation of
a long dormant creature. They finally seem like themselves again. I want
my children, and all children, to have the opportunity to be with their
peers, every day, in person.

That afternoon at the Sukenicks' I quickly realized that the Hudson Lab
pod program is well beyond my family's financial means. I don't know
what exactly we're going to do when school starts.

In my district, there has been an interesting overlap between the
send-them-back-to-schoolers and the keep-them-homers, in that most are
seeking any alternative they can think of to their children sitting
alone at a computer each day. A
\href{https://www.nytimes.com/2020/07/17/nyregion/coronavirus-nyc-schools-reopening-outdoors.html}{movement
for outdoor classes} seemed to bring many from opposing sides together.
Yet despite the enthusiasm of parents, the likelihood of that coming to
fruition seems dim. American can-do spirit somehow seems absent from
getting kids in school, even --- or especially --- if school is
outdoors.

Image

Students waiting to enter school in Belgium, when lockdown restrictions
were relaxed in May. Credit...Francisco Seco/Associated Press

Several weeks after the backyard presentation, Cate Han told me that
Hudson Lab has not finalized any pods yet but has around 60
applications. Many families are waiting to hear what the state and local
guidelines will be before making a decision.

Ms. Sukenick has not yet decided on plans for her boys. Whatever it is,
she hopes it will include an in-person experience every day.

Ms. Lazarin, the single mother in Riverdale, isn't only worried about
remote learning. She's also worried about the in-school environment,
with ``these unnatural experiences of eating lunch in cubicles, recess
alone in a square on the pavement.''

After pausing for a moment, thinking about the pod, she said, ``If I
have to sell my arm to make that happen, then I'll do it.''

Advertisement

\protect\hyperlink{after-bottom}{Continue reading the main story}

\hypertarget{site-index}{%
\subsection{Site Index}\label{site-index}}

\hypertarget{site-information-navigation}{%
\subsection{Site Information
Navigation}\label{site-information-navigation}}

\begin{itemize}
\tightlist
\item
  \href{https://help.nytimes.com/hc/en-us/articles/115014792127-Copyright-notice}{©~2020~The
  New York Times Company}
\end{itemize}

\begin{itemize}
\tightlist
\item
  \href{https://www.nytco.com/}{NYTCo}
\item
  \href{https://help.nytimes.com/hc/en-us/articles/115015385887-Contact-Us}{Contact
  Us}
\item
  \href{https://www.nytco.com/careers/}{Work with us}
\item
  \href{https://nytmediakit.com/}{Advertise}
\item
  \href{http://www.tbrandstudio.com/}{T Brand Studio}
\item
  \href{https://www.nytimes.com/privacy/cookie-policy\#how-do-i-manage-trackers}{Your
  Ad Choices}
\item
  \href{https://www.nytimes.com/privacy}{Privacy}
\item
  \href{https://help.nytimes.com/hc/en-us/articles/115014893428-Terms-of-service}{Terms
  of Service}
\item
  \href{https://help.nytimes.com/hc/en-us/articles/115014893968-Terms-of-sale}{Terms
  of Sale}
\item
  \href{https://spiderbites.nytimes.com}{Site Map}
\item
  \href{https://help.nytimes.com/hc/en-us}{Help}
\item
  \href{https://www.nytimes.com/subscription?campaignId=37WXW}{Subscriptions}
\end{itemize}
