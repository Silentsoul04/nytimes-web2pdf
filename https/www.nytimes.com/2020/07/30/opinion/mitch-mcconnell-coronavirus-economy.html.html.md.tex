Sections

SEARCH

\protect\hyperlink{site-content}{Skip to
content}\protect\hyperlink{site-index}{Skip to site index}

\href{https://myaccount.nytimes.com/auth/login?response_type=cookie\&client_id=vi}{}

\href{https://www.nytimes.com/section/todayspaper}{Today's Paper}

\href{/section/opinion}{Opinion}\textbar{}Mitch McConnell Could Rescue
Millions. What Is He Waiting For?

\url{https://nyti.ms/30e0qJz}

\begin{itemize}
\item
\item
\item
\item
\item
\item
\end{itemize}

Advertisement

\protect\hyperlink{after-top}{Continue reading the main story}

\href{/section/opinion}{Opinion}

Supported by

\protect\hyperlink{after-sponsor}{Continue reading the main story}

\hypertarget{mitch-mcconnell-could-rescue-millions-what-is-he-waiting-for}{%
\section{Mitch McConnell Could Rescue Millions. What Is He Waiting
For?}\label{mitch-mcconnell-could-rescue-millions-what-is-he-waiting-for}}

The economy is in dire shape. Millions of Americans are about to lose
federal aid. The Senate is going on vacation.

By
\href{https://www.nytimes.com/interactive/opinion/editorialboard.html}{The
Editorial Board}

The editorial board is a group of opinion journalists whose views are
informed by expertise, research, debate and certain longstanding ****
\href{https://www.nytimes.com/interactive/2018/opinion/editorialboard.html}{values}.
It is separate from the newsroom.

\begin{itemize}
\item
  July 30, 2020
\item
  \begin{itemize}
  \item
  \item
  \item
  \item
  \item
  \item
  \end{itemize}
\end{itemize}

\includegraphics{https://static01.nyt.com/images/2020/07/31/opinion/30econ/30econ-articleLarge-v3.jpg?quality=75\&auto=webp\&disable=upscale}

The worst economic news on Thursday was not the official announcement
that the American economy shrank at an annualized rate of
\href{https://www.nytimes.com/live/2020/07/30/business/stock-market-today-coronavirus?action=click\&module=Top\%20Stories\&pgtype=Homepage}{32.9
percent} in the second quarter of 2020 --- a grim quantification of the
pain caused by the coronavirus pandemic.

No, the really bad news was the lack of action on Capitol Hill.

Congress needs to
\href{https://www.nytimes.com/2020/03/27/us/politics/coronavirus-house-voting.html}{extend}
the emergency aid programs that were created in March to help Americans
endure a broad suspension of economic activity. Instead, even as the
pandemic rages on, Congress is allowing those aid programs to expire.

People who lost jobs during the pandemic have received \$600 a week from
the federal government on top of standard unemployment benefits. For
many, the money is all that has
\href{https://www.msn.com/en-us/finance/markets/almost-30-million-in-us-didn-e2-80-99t-have-enough-to-eat-last-week/ar-BB17ldBN}{kept
them from going hungry} and has allowed them to stay in their homes. It
has prevented a significant increase in the share of Americans
\href{https://www.nytimes.com/2020/06/21/us/politics/coronavirus-poverty.html}{living
in poverty}. But those payments end this week, even as unemployment
remains at a level last experienced during the Great Depression.

The federal government also is ending a moratorium on
\href{https://www.nytimes.com/2020/07/23/opinion/coronavirus-evictions-rent.html}{evictions},
as well as a program that provides aid to small businesses.

Among those pleading for aid that hasn't come: state and local
governments starved of tax revenue. School districts that need money for
safety equipment. Hospitals caring for the victims of the pandemic.
Elections officials bracing for November.

The abject failure to act is not the fault of Congress in a collective
sense. House Democrats passed a serviceable
\href{https://www.nytimes.com/2020/05/15/us/politics/house-simulus-vote.html}{aid
bill} more than two months ago. Responsibility for the current debacle
rests specifically and squarely on the shoulders of the Senate majority
leader, Mitch McConnell, Republican of Kentucky, and the other 52 Senate
Republicans.

From the moment Congress passed the last big coronavirus aid bill, in
March, it has been a matter of public record that the aid was going to
end in August.

For a time, there was reason to hope that the worst of the pandemic
could be over by now, too. But it has been clear for weeks that the
United States has failed to control the pandemic and that many Americans
still would need economic aid beyond July. Yet Mr. McConnell and his
caucus chose to spend the summer
\href{https://news.ballotpedia.org/2020/07/29/u-s-senate-confirms-hardy-to-u-s-district-court-for-the-western-district-of-pennsylvania/}{confirming}
federal judges rather than confronting the crisis.

Only in recent days have Republicans belatedly begun a frantic effort to
devise a coherent response to the crisis. Like students who wait until
the night before an assignment is due, they have pleaded for more time
and asked if they could submit a part of the work. The nation will
suffer the consequences.

Mr. McConnell put forward a proposal on Monday that included billions of
dollars for new F-35 jet fighters, but not a penny in aid for state and
local governments. In any event, it quickly became clear that many
Senate Republicans were not exactly on board. ``There's no consensus on
anything,''
\href{https://www.washingtonpost.com/us-policy/2020/07/29/trump-pushes-short-term-fix-unemployment-insurance-eviction-moratorium/}{said}
Mr. McConnell's deputy, Senator John Cornyn of Texas. Senator Josh
Hawley, Republican of Missouri,
\href{https://twitter.com/alexanderbolton/status/1288177098839003136}{called}
the proposal ``a mess.''

Lawmaking is laborious and rarely proceeds in a straight line. If the
calendar still said June, there would be less reason to worry about
these convolutions.

But behaving in late July as if it were still June is a recipe for
disaster.

Even with the infusion of trillions of dollars in federal aid since
March, many Americans are struggling to ride out the crisis. Almost 40
million people do not expect to be able to make their next rent or
mortgage payment. Almost 30 million Americans
\href{https://www.bloomberg.com/news/articles/2020-07-29/almost-30-million-in-u-s-didn-t-have-enough-to-eat-last-week\#:~:text=Food\%20insecurity\%20for\%20U.S.\%20households,seven\%20days\%20through\%20July\%2021.}{said}
they did not have enough to eat during the week ending July 21. Last
week, for the 19th straight week,
\href{https://www.washingtonpost.com/business/2020/07/09/another-13-million-people-applied-new-jobless-benefits-last-week-pandemics-toll-economy-continued/}{more
than} a million people filed fresh claims for unemployment benefits.

Grim as those numbers may be, the United States is on the verge of an
even deeper crisis.

Ernie Tedeschi, an economist at Evercore ISI, a financial research firm,
\href{https://twitter.com/ernietedeschi/status/1283834505627865088}{estimates}
that failing to resume the federal unemployment payments would cause a
drop in consumer spending large enough to eliminate about 1.7 million
jobs --- roughly the magnitude of job losses during the recessions of
the early 1990s and the early 2000s.

Britt Coundiff of Indianapolis is living on unemployment benefits after
losing her job at an art-house cinema. Without the federal payments,
she'll be left with a weekly state payment of \$193. She told
\href{https://www.nytimes.com/2020/07/22/opinion/sunday/unemployment-supplement-congress.html}{Talmon
Joseph Smith} of The Times, ``With two kids and rent and groceries, that
is not enough for us to survive.''

On Thursday, Senate Republicans proposed an inadequate stopgap: a narrow
extension of supplemental unemployment benefits. Instead of continuing
the \$600 weekly payments, however, Republicans proposed cutting the sum
to \$200 a week, through the end of the year. That would replace only a
portion of the income of the average unemployed worker, which is
reasonable in normal times; it encourages people to find jobs. But in
the midst of a pandemic, with few jobs available, the benefit cut is an
act of pointless cruelty.

Democrats refused to accept the proposal, and Republicans refused to do
anything more.

The result: More than 20 million unemployed Americans are about to lose
\$600 a week. They need the money. They can't find jobs. And the Senate
is leaving for vacation.

President Trump is not helping. The administration fought for the
inclusion of a
\href{https://www.nytimes.com/2020/07/23/business/payroll-tax-cut-trump-recession.html}{payroll}
tax cut --- a proposal so patently misguided that it forced Senate
Republicans into the rare position of opposing a tax cut. It asked
Republicans to include
\href{https://www.nytimes.com/2020/07/28/us/politics/republicans-trump-fbi-building-virus-relief-bill.html}{funding}
for a new F.B.I. headquarters. It backed the inclusion of billions of
dollars in new military spending, partly to replace money Mr. Trump
diverted for construction of his border wall.

One thing Mr. Trump has not fought for is aid for Americans in need.

But Mr. Trump is not a member of the Senate. He does not have the power
to prevent Senate Republicans from doing their jobs. That responsibility
is theirs alone.

\emph{The Times is committed to publishing}
\href{https://www.nytimes.com/2019/01/31/opinion/letters/letters-to-editor-new-york-times-women.html}{\emph{a
diversity of letters}} \emph{to the editor. We'd like to hear what you
think about this or any of our articles. Here are some}
\href{https://help.nytimes.com/hc/en-us/articles/115014925288-How-to-submit-a-letter-to-the-editor}{\emph{tips}}\emph{.
And here's our email:}
\href{mailto:letters@nytimes.com}{\emph{letters@nytimes.com}}\emph{.}

\emph{Follow The New York Times Opinion section on}
\href{https://www.facebook.com/nytopinion}{\emph{Facebook}}\emph{,}
\href{http://twitter.com/NYTOpinion}{\emph{Twitter (@NYTopinion)}}
\emph{and}
\href{https://www.instagram.com/nytopinion/}{\emph{Instagram}}\emph{.}

Advertisement

\protect\hyperlink{after-bottom}{Continue reading the main story}

\hypertarget{site-index}{%
\subsection{Site Index}\label{site-index}}

\hypertarget{site-information-navigation}{%
\subsection{Site Information
Navigation}\label{site-information-navigation}}

\begin{itemize}
\tightlist
\item
  \href{https://help.nytimes.com/hc/en-us/articles/115014792127-Copyright-notice}{©~2020~The
  New York Times Company}
\end{itemize}

\begin{itemize}
\tightlist
\item
  \href{https://www.nytco.com/}{NYTCo}
\item
  \href{https://help.nytimes.com/hc/en-us/articles/115015385887-Contact-Us}{Contact
  Us}
\item
  \href{https://www.nytco.com/careers/}{Work with us}
\item
  \href{https://nytmediakit.com/}{Advertise}
\item
  \href{http://www.tbrandstudio.com/}{T Brand Studio}
\item
  \href{https://www.nytimes.com/privacy/cookie-policy\#how-do-i-manage-trackers}{Your
  Ad Choices}
\item
  \href{https://www.nytimes.com/privacy}{Privacy}
\item
  \href{https://help.nytimes.com/hc/en-us/articles/115014893428-Terms-of-service}{Terms
  of Service}
\item
  \href{https://help.nytimes.com/hc/en-us/articles/115014893968-Terms-of-sale}{Terms
  of Sale}
\item
  \href{https://spiderbites.nytimes.com}{Site Map}
\item
  \href{https://help.nytimes.com/hc/en-us}{Help}
\item
  \href{https://www.nytimes.com/subscription?campaignId=37WXW}{Subscriptions}
\end{itemize}
