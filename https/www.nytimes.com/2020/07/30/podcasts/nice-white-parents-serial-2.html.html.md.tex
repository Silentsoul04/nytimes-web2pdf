Sections

SEARCH

\protect\hyperlink{site-content}{Skip to
content}\protect\hyperlink{site-index}{Skip to site index}

\href{https://www.nytimes.com/spotlight/podcasts}{Podcasts}

\href{https://myaccount.nytimes.com/auth/login?response_type=cookie\&client_id=vi}{}

\href{https://www.nytimes.com/section/todayspaper}{Today's Paper}

\href{/spotlight/podcasts}{Podcasts}\textbar{}Episode Two: `I Still
Believe in It'

\url{https://nyti.ms/3jU55bu}

\begin{itemize}
\item
\item
\item
\item
\item
\end{itemize}

Advertisement

\protect\hyperlink{after-top}{Continue reading the main story}

transcript

Back to Nice White Parents

bars

0:00/53:37

-53:37

transcript

\hypertarget{episode-two-i-still-believe-in-it}{%
\subsection{Episode Two: `I Still Believe in
It'}\label{episode-two-i-still-believe-in-it}}

\hypertarget{reported-by-chana-joffe-walt-produced-by-julie-snyder-edited-by-sarah-koenig-neil-drumming-and-ira-glass-editorial-consulting-by-eve-l-ewing-and-rachel-lissy-and-sound-mix-by-stowe-nelson}{%
\subsubsection{Reported by Chana Joffe-Walt; produced by Julie Snyder;
edited by Sarah Koenig, Neil Drumming and Ira Glass; editorial
consulting by Eve L. Ewing and Rachel Lissy; and sound mix by Stowe
Nelson}\label{reported-by-chana-joffe-walt-produced-by-julie-snyder-edited-by-sarah-koenig-neil-drumming-and-ira-glass-editorial-consulting-by-eve-l-ewing-and-rachel-lissy-and-sound-mix-by-stowe-nelson}}

\hypertarget{white-parents-in-the-1960s-fought-to-be-part-of-a-new-racially-integrated-school-in-brooklyn-so-why-did-their-children-never-attend}{%
\paragraph{White parents in the 1960s fought to be part of a new,
racially integrated school in Brooklyn. So why did their children never
attend?}\label{white-parents-in-the-1960s-fought-to-be-part-of-a-new-racially-integrated-school-in-brooklyn-so-why-did-their-children-never-attend}}

Thursday, July 30th, 2020

\begin{itemize}
\item
  ``Nice White Parents'' is brought to you by Serial Productions, a New
  York Times Company.
\item
  chana joffe-walt\\
  The New York City Board of Education has an archive of all of its
  records. Everything that goes into making thousands of schools run for
  years and years is sitting in boxes in the municipal building. I love
  the B.O.E. archive.
\item
  chana joffe-walt\\
  Good morning. How are you doing?
\end{itemize}

chana joffe-walt

First of all, to look through it, you have to go to a century-old
municipal building downtown. Arched doorways, lots of marble, an echo,
vaulted ceilings really makes a person feel like she's up to something
important. You sit at a table, and then a librarian rolls your boxes up
to you on a cart. Inside the boxes are all the dramas of a school
system. Big ones, tiny ones, bureaucratic, personal, it's all in there.
There's a union contract and then a zoning plan and special reports on
teacher credentialing, a weird personal note from a bureaucrat to his
assistant, a three-page single-spaced plea from Cindy's grandmother, who
would please like for her not to be held back in the second grade. An
historian friend once pulled a folder out of the archive and a note fell
out, something a teacher clearly made a kid write in the 1950s, that
read, quote, ``I am a lazy boy. Miss Fitzgerald says, when I go in the
army, I will be expendable. Expendable means that the country doesn't
care whether I get killed or not. I do not like to be expendable. I'm
going to do my work and improve.''

{[}music{]}

I came to the Board of Ed archive after I attended the gala thrown by
the French embassy, the fundraiser for SIS organized by the new
upper-class white families coming into the school. I felt like I'd just
watched an unveiling ceremony for a brand-new school, but I didn't
really know what it was replacing. Everyone was talking as if this was
the first time white parents were taking an interest in the School for
International Studies. But at the archive, I found out it wasn't the
first time. White parents had invested in the school before, way before,
at the very beginning of the school. Before the beginning. I found a
folder labeled I.S. 293, Intermediate School 293, the original name for
SIS. And this folder was filled with personal letters to the president
of the New York City Board of Education, a man named Max Rubin, pleading
with him to please make I.S. 293 an integrated school. ``Dear Mr. Rubin,
my husband and I were educated in public schools, and we very much want
for our children to have this experience. However, we also want them to
attend a school which will give them a good education, and today, that
is synonymous with an integrated school.'' ``Dear Mr. Rubin, as a
resident of Cobble Hill, a teacher and a parent, I want my child to
attend schools which are desegregated. I do not want her to be in a
situation in which she will be a member of a small, white, middle-income
clique.'' These are letters from parents --- largely white parents, as
far as I could tell --- written in 1963, just a few years before I.S.
293 was built. At issue was where the school was going to be built. The
Board of Education was proposing to build the school right next to some
housing projects. The school would be almost entirely Black and Puerto
Rican. These parents, white parents, came in and said, no, no, no, don't
build it there. Put it closer to the white neighborhood. That way, all
our kids can go to school together. These parents wanted the school
built in what was known as a fringe zone. This was a popular idea at the
time, fringe schools to promote school integration. Comes up in the
letters. ``Dear Mr. Rubin, this neighborhood is changing with the influx
of a middle-class group which is very interested in public education for
their children.'' ``Dear Mr. Rubin, if there is a possibility of
achieving some degree of integration, it is more likely if the Board of
Education's theory of fringe schools is applied.'' And from another
letter, ``it is apparent from the opinion of the neighborhood groups
involved that the situation is not at all hopeless.'' This lobbying
effort was so successful that the Board of Education did move the site
of the school. This is why SIS is located where it is today, on the
fringe, closer to the white side of town, so that it would be
integrated.

I tried to imagine who these people were --- young, idealistic white
parents living in Brooklyn in the 1960s, feeling good about the future.
They would have had their children around the time the Supreme Court
ruled on Brown versus Board of Education. They probably followed the
news of the Civil Rights Movement unfolding down South. Maybe they were
supporters or active in the movement themselves. These were white
parents saying, we understand we're at a turning point and we have a
choice to make right now, and we choose integration. One of my favorite
letters was from a couple who left the suburbs to come to New York City
for integration, the opposite of white flight. ``Dear Mr. Rubin, we have
recently moved into the home we purchased at the above address in Cobble
Hill. It was our hope in moving into the neighborhood that our children
would enjoy the advantages of mixing freely with children of other
classes and races, which we were not able to provide to them when we
lived in a Westchester suburb.''

\begin{itemize}
\item
  chana joffe-walt\\
  So this is the letter.
\item
  carol netzer\\
  This is the letter that I wrote? I can't believe it. OK.
\end{itemize}

chana joffe-walt

This is Carol Netzer. Most of the letter writers were not that hard to
find.

\begin{itemize}
\tightlist
\item
  carol netzer\\
  We had moved to Scarsdale for the children, because Scarsdale has the
  best --- it probably still does --- the best school system in the
  country, but we hated it. We found that we were bored to death with
  it. It was bland. It was just homogeneous. But living --- I don't know
  if you've ever lived in a suburb. It's just boring, tedious, you know?
  There's nothing going on.
\end{itemize}

chana joffe-walt

She didn't like the suburbs. So they moved to Brooklyn and wrote that
letter, which I showed her, her 37-year-old self writing about her hopes
for her young children, the choices she made back then.

\begin{itemize}
\tightlist
\item
  carol netzer\\
  But it sounds as though I was fairly impassioned about it. You know,
  that it meant something. But I --- actually, I can't think what it
  meant.
\end{itemize}

{[}music{]}

chana joffe-walt

I went through this box of letters and called as many parents as I
could. Most of them didn't remember writing these letters, which isn't
surprising, more than 50 years ago and all. What I did find surprising
is that, by the time 293 opened, five years later, none of them, not a
one, actually sent their kids to I.S. 293.

{[}music{]}

From Serial Productions, I'm Chana Joffe-Walt. This is ``Nice White
Parents,'' a series about the 60-year relationship between white parents
and the public school down the block, a relationship that began with a
commitment to integration. In the 1960s, much like today, white people
were surrounded by a movement for the civil rights of Black Americans.
White people were forced to contend with systemic racism. And here was a
group of white parents who supported the movement for school
integration, threw their weight behind it. What happened in those five
years between 1963, when these white parents planted an impassioned
pro-integration flag on the school, and 1968, when it came time to
enroll their children? Why didn't they show up?

These white parents who wanted an integrated I.S. 293, they didn't come
to that idea on their own. They were part of a bigger story unfolding
around them. I want to zoom out to that dramatic story because it takes
us right up to the moment these parents wrote their letters, and then
made the decision not to send their kids to the school. To begin, I'd
like to introduce you to our main character in this historical, tale,
the recipient of the parents' letters, the New York City Board of
Education. Back in the 1950s, the New York City Board of Ed was not one
of those boring bureaucracies that chugs along in the background,
keeping its head down. It had personality. It invested in self-image.
For instance, in 1954, when the Supreme Court found school segregation
unconstitutional, New York City didn't just say we support that ruling,
it celebrated the Brown v Board decision. And notably, it celebrated
itself, calling Brown, quote, ``a moral reaffirmation of our fundamental
educational principles.'' That same year, 1954, the New York City Board
of Ed made a film honoring multiculturalism in its schools. {[}CHILDREN
SINGING{]} The film opens with a multiracial choir of schoolchildren
singing ``Let Us Break Bread Together.'' Like I said, the Board of Ed
went the extra mile. The Schools Superintendent was a 66-year-old man
named Dr. William Jansen, a man that newspapers described as slow and
steady. And he definitely delivers on that promise here.

\begin{itemize}
\tightlist
\item
  archived recording (william jansen)\\
  The film you're about to see tells the story of how the schools and
  community are working together to build brotherhood.
\end{itemize}

chana joffe-walt

A teacher addresses her classroom, filled with children of all races and
ethnicities.

\begin{itemize}
\tightlist
\item
  archived recording\\
  Who among you can give some of the reasons why people left their
  native lands to come to the United States of America?
\end{itemize}

chana joffe-walt

The camera cuts to a white boy, maybe 9 or 10.

\begin{itemize}
\tightlist
\item
  archived recording\\
  Some came because they wanted to get away from the tyranny and cruelty
  of kings.
\end{itemize}

chana joffe-walt

Then a Black girl, around the same age.

\begin{itemize}
\tightlist
\item
  archived recording\\
  My people are free now. They are proud to be American. But the Negroes
  were brought here by wicked men who traded in slaves.
\end{itemize}

chana joffe-walt

This keeps going, kid to kid.

\begin{itemize}
\tightlist
\item
  archived recording\\
  We came a little while ago from Puerto Rico. My father wanted work. He
  wants to give me and my brother a good education. Japan is very
  overcrowded. The people have little land. So many Japanese came to
  this country because they wanted to farm.
\end{itemize}

chana joffe-walt

New York City was the biggest city in America, with the largest Black
population in America, and it was saying in films, press releases,
public speeches, Brown v Board, we agree. Separate but equal has no
place in the field of public education. No problem here. It was also
saying, you know who does have a problem? The South. New York City loved
comparing itself to the backward South. There are plenty of examples of
this in the Board archives, New Yorkers bragging about their superiority
to places like Georgia or Virginia or Louisiana. This was the story the
Board of Ed was telling. The South was ignorant and racist. New York
City was enlightened and integrated. But here is what it was actually
like to walk into a New York City school in a Black neighborhood at this
time.

\begin{itemize}
\tightlist
\item
  archived recording (mae mallory)\\
  The school had an awful smell. It was just --- oh, it smelled like
  this county abattoir.
\end{itemize}

chana joffe-walt

This is an archival recording of a woman named Mae Mallory. In the
1950s, Mallory's two Black children were students in Harlem. And when
Mallory walked into their school, she did not see children building
brotherhood in interracial classrooms. She saw an all-Black and Puerto
Rican school with terrible facilities, in disrepair.

\begin{itemize}
\tightlist
\item
  archived recording (mae mallory)\\
  So my kids told me, says, well, Mommy, this is what we've been trying
  to tell you all along, that this place is so dirty. And this is why we
  run home to the bathroom every night. So I went to the bathroom. And
  in 1957 in New York City, they had toilets that were worse than the
  toilets in the schools that I went to in Macon, Georgia in the heart
  of the South. The toilet was a thing that looked like horse stalls.
  And then it had one long board with holes cut in it. And then you'd
  have to go and use the toilet, but you couldn't flush it. The water
  would come down periodically and flush, you know, whatever's there.
  Now imagine what this is like, you know, dumping waste on top of waste
  that's sitting there waiting, you know, accumulating till the water
  comes. This was why this place smelled so bad.
\end{itemize}

chana joffe-walt

Mae Mallory says the school had two bathrooms for 1,600 children.
Mallory's family fled racial violence in the South, like millions of
other Black Americans, who headed to places like New York City, where
everyone was supposed to be equal. Instead of welcoming these new
students and spreading them out, creating interracial classrooms, the
Board of Education kept Black and Puerto Rican students segregated in
what were sometimes referred to as ghetto schools, schools that were
often just blocks away from white schools. White schools in New York
City had toilets that flushed. White children had classrooms with
experienced teachers and principals, people who lived in their
communities and looked like them. In Black and Puerto Rican schools,
half the teachers were not certified to teach by the Board of Education.
The buildings were in disrepair, and packed, sometimes more than 1,000
kids in a single hallway. The overcrowding got so bad the Board of
Education decided to send kids to school in shifts. And mind you, this
was not in the middle of a global pandemic. This was normal, non-crisis
school for Black and Puerto Rican kids. One group of children would go
to school in the morning until noon. The next group of kids would come
in at noon, and stay until 3:00. The Board was literally giving Black
kids half an education. In some schools in Harlem, they had triple
shifts. This made it harder to learn elementary skills. Reading, for
instance. Black parents complained that the schools were not teaching
their kids basic literacy, that their white teachers didn't care, that
the summer reading programs were only in white communities, that their
children were two years behind white children in reading. This at
exactly the same time the Board of Education was making a film promoting
the virtues of integration. It was effectively running a dual,
segregated and unequal school system.

{[}music{]}

For many Black families, the Board of Education was not to be trusted.
It did not care for Black children, and it didn't respect the voices and
concerns of Black parents. Mae Mallory says she visited her kids' school
that day because they'd come home the day before and told her a child
had died at school. He was playing in the street at recess. Mallory
hardly believed it, but she says when she visited the school, she
learned, yes, indeed, this child was playing the street because the
schoolyard was closed. He was hit by a beer truck. And she learned the
schoolyard was closed because pieces of steel from the side of the
building had fallen into the yard.

\begin{itemize}
\tightlist
\item
  archived recording (mae mallory)\\
  And when I found out that this was true, I went to the principal. So
  this principal told me that, well, Mrs. Mallory, you really don't have
  anything to worry about. You see, our sunshine club went to see the
  mother, and we took her a bag of canned goods. So actually, she's
  better off, because she had so many children to feed. And I couldn't
  believe that here a white man is going to tell a Black woman in Harlem
  that a can of peaches is better than your child. I just didn't know
  what to do or where to go. But I know you're supposed to do something.
\end{itemize}

chana joffe-walt

It was 1957, three years after the Supreme Court declared segregation by
law unconstitutional. New York City didn't have Jim Crow laws on the
books, but Mae Mallory would ask, the schools are segregated. What's the
difference? She didn't care whether that segregation was codified by law
or by convention. The harm was just as dire. And she wanted it
addressed.

\begin{itemize}
\tightlist
\item
  archived recording (mae mallory)\\
  This was nothing to do with wanting to sit next to white folks. But it
  was obvious that a whole pattern of Black retardation was the program
  of the Board of Education. So I filed a suit against the Board of
  Education. And I just fought back.
\end{itemize}

chana joffe-walt

Integration, Mae Mallory would say, was about, quote, demanding a fair
share of the pie. She said, our children want to learn, and they
certainly have the ability to learn. What they need is the opportunity.
The Board of Education had defined integration as a multiracial choir.
It was a virtue in and of itself. Mae Mallory saw integration as a
remedy, a way to get the same stuff everyone else had --- functioning
toilets, books, certified teachers, a full school day. Integration was a
means to an end.

{[}music{]}

Mae Mallory won her lawsuit. She and a few other parents were allowed to
transfer their kids out of segregated schools. As for the segregation in
the entire system, the judge in the lawsuit turned to the Board of Ed
and said, this segregation, it's your responsibility. Fix it.

Now, on the question of responsibility, the Board of Education was
cagey. And that caginess set the stage for the I.S. 293 parents when it
came time to send their kids to the school. Here's what happened. The
Schools Superintendent, William Jansen, decided school segregation was
not his problem. In fact, he rejected the idea that New York City had
segregated schools in the first place. After all, New York City was not
barring Black children from entering white schools. This wasn't the
South. Segregation, Jansen said, is such an unfortunate word. He
preferred the phrase racial imbalance or racial separation. The way he
saw it, racial imbalance in the schools was just a matter of housing.
Neighborhoods were segregated. Again, unfortunate, but that had nothing
to do with the schools. To make this argument, William Jansen had to
ignore the many powerful tools available to the Board of Education. The
Board of Education was responsible for where kids went to school. It
decided where to build new schools. It drew zoning lines. It decided
where experienced teachers teach. There were many ways the Board could
have made schools less segregated. I know this because of the Board's
own reports. Jansen did very little to break up school segregation, but
man, did he study it. He organized commissions that led to reports that
led to further study. You see a pattern emerge, starting in the late
1950s, that looks something like this. Black parents and civil rights
groups would pressure the Board to act on segregation. The Board would
invite its critics to join a commission to investigate the problem. The
commission would study the schools, discover extreme segregation, lay
out solutions. The Board of Ed would then take a tiny step toward
implementing some of the recommendations until white parents started to
complain about the changes, at which point the Board would back off and
say it needed more evidence. Another commission, another report. For
instance, there's the Report on the Committee on Integration, a Plan for
Integration, the City's Children and the Challenge of Racial
Discrimination, Redoubling Efforts on Integration, the Board Commission
on Integration, the Status of the Public School Education of Negro and
Puerto Rican Children in New York City, and, my favorite, a bound little
red book from 1960 called Toward Greater Opportunity, which summarizes
the previous investigations with this groundbreaking conclusion. Quote,
``we must integrate as much and as quickly as we can.'' I want to pause
for one second and step out of the past back into the world we all live
in, just to point out that, over the last few years in New York City,
we've been reliving this chapter of history. It's eerie. New York City
schools are segregated. There's a growing movement to do something about
that. And for the first five years of his administration, the city's
mayor, Mayor Bill de Blasio, responded in the following way. He refused
to say the word segregation, commissioned a number of reports on school
diversity. He's pointed a finger at housing problems as a way to say
this isn't our fault, and he's studying the problem deeply, which,
again, is not segregation, no matter how many times reporters would ask
the mayor at press conferences, why don't you use that word?

\begin{itemize}
\tightlist
\item
  archived recording (bill de blasio)\\
  I don't get lost in terminology. I think the notion of saying we have
  to diversify our schools is the best way to say it.
\end{itemize}

chana joffe-walt

I heard a live call-in show on WNYC, the public radio station. A young
integration advocate, an 11th grader named Tiffani Torres, asked the
mayor, how much longer until you do something?

\begin{itemize}
\item
  archived recording (tiffani torres)\\
  And how much more time do you need to study the issue? So to repeat my
  question, how much longer will it take?
\item
  archived recording (bill de blasio)\\
  Tiffani, with all due respect, I really think you're not hearing what
  we're saying to you, so I'll repeat it. There is a task force, an
  extraordinary task force, which I've met with. They are coming forward
  with their next report in a matter of weeks. So when that diversity
  task force comes out with their report, I think they're amazing. I
  think they've done fantastic work. And so far, there's a high level
  ---
\end{itemize}

chana joffe-walt

Mayor de Blasio likes to point out that this was a problem created by
people long before him, which is exactly what people long before him
said, too.

{[}music{]}

In the late 1950s, when Black parents and civil rights activists also
asked the Board of Ed, why is it taking so long, board members
complained about the, quote, extremists who wanted instant integration.
Superintendent Jansen said, ``some people want us to build Rome in one
day.'' While the Board of Education was building Rome in 1956, `57, `59,
and in 1960, 1962, `63, Black parents found each other on PTAs, in civil
rights organizations, pro-integration groups. They formed new groups,
organized sit-ins, boycotts, demanded the Board provide a timetable for
citywide integration. They joined forces with Puerto Rican parents, and
their numbers grew. These were volunteers, mothers mostly, who left
their jobs at the end of a workday and headed directly to a meeting
about how to get the Board to give their kids the education white
children were already receiving. Finally, in 1964, 10 years after Brown
versus Board, Black and Puerto Rican parents said, enough. They were
sick of waiting, sick of lawsuits, sick of asking for a remedy, sick of
being ignored. So they went big, spectacularly big. They shut down the
schools. They organized a civil rights demonstration that was the
largest in US history, larger than the March on Washington. It was
called Freedom Day, a massive school boycott.

\begin{itemize}
\tightlist
\item
  archived recording\\
  (CHANTING) Freedom now!
\end{itemize}

chana joffe-walt

On February 3rd, 1964, parents headed out to schools in the morning
before sunrise to spread the word about the boycott. It was freezing
cold that day. There's a brief TV news clip of a group of mothers
picketing outside their kids' school at the start of the school day.
They're holding up signs that say, ``we demand a real integration
timetable now,'' and ``integration means better schools for all.''
They're handing out leaflets to other parents about Freedom Day, looking
spirited and cold. A white NBC news reporter in a fedora walks up to one
of the women.

\begin{itemize}
\tightlist
\item
  archived recording\\
  Ma'am, it's a little after eight o'clock now. How successful has the
  boycott been so far? Very effective. So far, about 10 children have
  gone in, and there would be ordinarily 240 children. And 10 have gone
  into the morning session, which begins at eight o'clock. So you think
  you've already seen the result? Yes, I think so. The school is just
  empty. Does it surprise you? No, because we knew how effective --- We
  talked with the parents. We distributed leaflets. We've been working
  very hard. And we prayed that it would be effective.
\end{itemize}

chana joffe-walt

There were maps and charts and instructions with picket times and picket
captains for hundreds of schools. There were volunteer shifts to make
peanut butter and jelly sandwiches, to hand out thousands of leaflets
and stencil posters. The boycott wasn't just effective --- it was
extraordinarily effective. Half a million kids stayed home from school
that day. Half a million, close to half the school system. But the press
barely covered it. After searching every major TV network, I found only
one kid who was interviewed, a teenage boy, maybe around 16, on the
street with some friends, protesting. A white ABC News reporter doesn't
ask him why he's there. The only thing he asks him about is violence.
The kid responds.

\begin{itemize}
\tightlist
\item
  archived recording\\
  We're coming down here today for a peaceful --- peaceful--- No
  comment! No, we're not going to be violent. We're just teenagers and
  kids. And --- Do you expect violence here today? No, sir, not if ---
  look at the blue uniforms. You ask me do I expect violence.
\end{itemize}

chana joffe-walt

He gestures to the police on horseback.

\begin{itemize}
\tightlist
\item
  archived recording\\
  None of us have any weapons, horses. And all we want is equal
  education. That's all. Equal education. Thank you. You get all that?
\end{itemize}

chana joffe-walt

That was it. Every once in a while, I'll hear a politician or friend or
school administrator say, yeah, integration was a good idea, but there
was no political will to make it happen. 460,000 kids, half the school
system. The will was there. The majority wanted integration.

{[}music{]}

After Freedom Day, the Board of Education introduced some small-scale
integration plans, and white parents protested. {[}CHILDREN SINGING{]}
We love our children. Oh, yes, we do. We will not transfer ---

chana joffe-walt

With their own marches, they put on their own school boycott. The
flipside of freedom day, a white boycott. The white parents were far
fewer in number. But as far as I can tell, they got a thousand times
more press coverage.

\begin{itemize}
\item
  archived recording\\
  Mrs. Carcevski?
\item
  speaker\\
  Yes? Are you going to send Johnny back to school now? No. She belongs
  here, and I want to send my child here. So nobody is going to tell me
  where to send my kid.
\end{itemize}

chana joffe-walt

This protest worked. The Board of Ed backed off. And in the decades
since, the Board of Education has never proposed a city-wide integration
plan. The schools have never been integrated. I think the fact of white
moms in Queens in the 1960s yelling about zoning changes and busing,
it's not surprising they played a role in killing school integration
efforts. But there was another group of white parents who played a
quieter, but I'd argue more forceful, role in killing integration. The
white parents who said they supported it, parents like the ones who
wrote letters asking for an integrated I.S. 293. How did their vocal
support for integration turn lethal? That's after the break. In the
American South, schools were desegregated with court orders. Cities and
counties mandated desegregation, and the schools desegregated. By the
early 1970s, the South was the most integrated region in the country.
But New York City did not want to do it that way. No mandates. The New
York City Board of Education wanted to appeal to hearts and minds. They
wanted to sell white people on the virtues of integration. Have it all
happen, quote, ``naturally.'' Some white people were sold. The white
parents who wrote letters about I.S. 293. They believed in integration.
So I made a lot of calls to ask, why'd you bail? They had a lot of
different reasons. One couple got divorced, and moved. Another guy told
me he had political ambitions that pulled him out of the city.

\begin{itemize}
\tightlist
\item
  speaker\\
  We loved our brownstone, but I was involved in a political race. And
  we needed some money for that.
\end{itemize}

chana joffe-walt

So he sold the house and moved the family to the suburbs, where he
thought he'd have a better chance running against Republicans. Many
white people moved to the suburbs for jobs, for newly paved roads and
subsidized mortgages, leaving Brooklyn behind. I understood what
happened there. But some explanations made less sense. Like one guy I
called, he did stay in Brooklyn. On the phone, he was telling me why he
believed it was important that I.S. 293 be integrated. But then he said
his own kids went to Brooklyn Friends, a Quaker private school. I said,
oh, they didn't go to I.S. 293.

\begin{itemize}
\tightlist
\item
  speaker\\
  No. As I said, I'm a Quaker, and ---
\end{itemize}

chana joffe-walt

But you were a Quaker when you wrote this letter, asking for an
integrated 293.

\begin{itemize}
\tightlist
\item
  speaker\\
  I believed it. I believed in it, but ---
\end{itemize}

chana joffe-walt

You weren't planning to send your kid there?

\begin{itemize}
\tightlist
\item
  speaker\\
  No, no, no.
\end{itemize}

chana joffe-walt

What to make of that? When you get what you say you want and then, given
the opportunity, don't take it. Maybe you never really wanted it in the
first place. Then I spoke to Elaine Hencke. Of all the people I spoke
with, everything about Elaine indicated someone who did believe in
integration, someone who would send her kids to 293. And yet, she
didn't. Elaine was a public school teacher. She taught in an integrated
elementary school, until she had her own kids. She was looking forward
to sending them to an integrated 293. When her daughter was old enough
for junior high school, Elaine visited the school. She was the only
letter writer I spoke with who actually went into the building. If this
was going to work with anyone, it was going to be Elaine.

\begin{itemize}
\tightlist
\item
  elaine hencke\\
  I didn't know quite what to make of it because the school had a nice
  plant. Physically, it was a nice school. But it just seemed chaotic
  and noisy, and kids were disruptive. And kids --- {[}LAUGHS{]} ---
  kids were doing the wrong things, you know? And kids do. I mean, it
  wasn't that they were nasty kids or doing --- it was not drugs. It was
  not drugs. It was just --- it just seemed too chaotic to me at the
  time.
\end{itemize}

chana joffe-walt

Elaine and I talked for a long time I pushed her --- not to make her
feel bad, but to get to what felt like a more real answer. At the time
that you are visiting, was it majority Black and Hispanic kids?

\begin{itemize}
\tightlist
\item
  elaine hencke\\
  Yes, I'm sure it was.
\end{itemize}

chana joffe-walt

And did that have anything to do with the way that you saw the classroom
as disruptive and chaotic?

\begin{itemize}
\item
  elaine hencke\\
  I would hope not.

  I'm not --- I'm not sure how well educated they were, or --- you know,
  I don't know. I don't know why I'm going into this.
\end{itemize}

chana joffe-walt

Well, did you have reason to think that they weren't well educated?

\begin{itemize}
\tightlist
\item
  elaine hencke\\
  Before 293? Well, their reading levels were way down. You know.
\end{itemize}

chana joffe-walt

I'm just --- when you say chaos and disruptive, I'm trusting that what
you saw was chaotic and disruptive. But I also know that those are words
white people use --- we use to express our racial fears, to express real
racial fears. Do you think that's what was happening with you?

\begin{itemize}
\tightlist
\item
  elaine hencke\\
  I don't think I would admit to that. I don't think that was true. But
  what I may have thought was that these kids are not expected to do so
  well in school, all the way from the beginning of school. And here
  they are, really unprepared in some way, for junior high school or ---
  I mean, the reading levels were low.
\end{itemize}

chana joffe-walt

Elaine told me when she wrote that letter to the Board of Education, she
pictured her children becoming friends with Black kids, learning
side-by-side, learning that all children are equal. That's what
motivated her to write that letter. She wanted the picture of
integration the Board of Ed was promoting --- the picture of harmonious
integration. But when she visited I.S. 293, that didn't seem possible.
The reading levels were low. The kids were not entering the school on
equal grounds. Her white children had received years of high quality
teaching at well-resourced schools. The kids coming from segregated
elementary schools had not had that experience.

\begin{itemize}
\tightlist
\item
  elaine hencke\\
  I mean, one of the problems is that many of the white kids had higher
  sort of academic skills, or skills. They could read better. I think
  --- I mean, if the white kids knew how to read in first grade and ---
  and I guess there were Black kids who also could. But it just seemed
  as if most of the black kids didn't really learn --- learn to read.
\end{itemize}

chana joffe-walt

But part of the --- part of the vocal complaints of black parents at
this period of time was that their kids were not learning how to read
because schools were segregated, and their kids were kept in schools
that were inferior. And that was part of the argument for integration.

\begin{itemize}
\tightlist
\item
  elaine hencke\\
  Yes, yes.
\end{itemize}

chana joffe-walt

That their kids were not going to get the resources, and quality
teaching, and good facilities unless they were in the same buildings
with kids like yours.

\begin{itemize}
\item
  elaine hencke\\
  Right.

  I don't know what to say to that. I just --- I guess I just began to
  feel that things were really difficult for these kids. Schools were
  not made for them. If the schools were made for them, with their
  background, what would they be like?

  I think there was --- and that's another whole thing. I don't know
  about it. I think there was sort of anger in the black community at
  the white community. A lot of the teachers were white. There were more
  white teachers, I suppose. People said that that was racism. And of
  course, it was racism. But maybe the kids were a little angry at the
  school. I wouldn't --- I couldn't fault them for that. But on the
  other hand, then they don't get as much from the school. I don't know.
  I thought the problems were kind of enormous. And I guess I just, at
  one point, I just decided that my kids should go --- went to Brooklyn
  Friends. And we could afford to pay for it. It wasn't easy, you know.
  It was --- {[}LAUGHS{]} but ---
\end{itemize}

chana joffe-walt

Did your feelings about integration change? Did you believe in it less?

\begin{itemize}
\item
  elaine hencke\\
  Maybe.

  I think I would have said no, theoretically. But maybe they did. I
  guess I saw it as a more difficult project then. I sort of did back
  off from it. I just ---
\end{itemize}

chana joffe-walt

Yeah. It felt when you guys wrote these letters like, this is ---
integration is this exciting ideal, and we can be part of it, and it's
going to be a meaningful project that's also going to be kind of easy.

\begin{itemize}
\tightlist
\item
  elaine hencke\\
  I certainly didn't think it would be so difficult. But I --- I was, I
  was innocent, you know? I don't know. I still believe in it. I do.
\end{itemize}

{[}music{]}

chana joffe-walt

I think what Elaine actually meant was not that she was innocent, but
that she was naive. She was naive about the reality of segregation, the
harm of it. And naive about what it would take to undo it. She did not
know. And I think she didn't want to know. When Elaine said the word
innocent, I felt a jolt of recognition. I felt like Elaine had walked me
right up to the truth about her, and about me.

When my own kids were old enough, I sent them to our zoned public
school. It was racially mixed and economically mixed. I was excited
about that. And it was nice walking to school with neighbors, people I
likely never would have gotten to know otherwise. My kid's first day of
school was another boy's first week in the country. He'd just moved from
China, and his mom asked the neighbor where the school was. When she
said goodbye that first morning, I think he thought I was a teacher, and
he crawled into my lap. We had no words in common, so I just held him
while he screamed and cried. By the holiday show three months later, I
watched that same boy belt out ``This Pretty Planet'' on a stage with
his classmates. He was the star. He nailed the hand motions. Every other
kid up on stage was just following his lead, just trying to keep up. It
was such a sweet picture, all of them up there --- Black kids, and
Mexican kids, and Colombian, and Asian and white kids. And all of us
adults supporting all of them. It's moving, to me, this picture of
integration. It is also, I'm realizing right now, writing these words
down, the very same picture the Board of Education put forth in 1954 ---
a multiracial choir singing together, building brotherhood. And it's
dangerous, I think, this picture of integration. It seems perfectly
designed to preserve my innocence, to make me comfortable, not to remedy
inequality, but a way to bypass it entirely. I can sit in that assembly
and feel good about the gauzy display of integration without ever being
asked to think about the fact that much of the time, white kids in the
school building are having a different educational experience than kids
of color. A large share of the white students at the school are
clustered in a gifted program. They have separate classrooms and
separate teachers. We all blithely call these white children gifted and
talented, G\&T, starting at four years old. White children are
performing better at the school than black children and Latino children.
White families are the loudest and most powerful voices in the building.
The advantages white kids had back in the 1950s, they're still in place.
When Elaine said she was innocent, I thought about the things we say,
nice, white parents, to each other about why we won't send our kids to
segregated schools --- because they're too strict, or too chaotic, or
too disruptive. Because the test scores are bad, because we want more
play. We want fewer worksheets. Because we don't want to ride a bus. We
don't want uniforms. We don't want tests. We want innocence. We need it,
to protect us from the reality that we are the ones creating the
segregation, and we're not sure we're ready to give it up.

{[}music{]}

Elaine was not for segregation. But in the end, she wasn't really for
integration, either. All of the choices she made, choices she had the
luxury of making, were meant to advantage her own kids. And I understand
that. That's what parents do.

\begin{itemize}
\tightlist
\item
  elaine\\
  I remember thinking very clearly that OK, I believe in this. But I
  don't sort of want to sacrifice my children to it. I have to look at
  what they will learn, and what they will do. And for people who sent
  their kids to 293, it seemed to work out well. So that made me think,
  well, maybe I made a mistake. Maybe they should have gone there. I
  know at one point it was very clear to me that I had beliefs that I
  thought were kind of contrary to my own children's best interests. And
  I decided that I wasn't going to use them to sort of extend my own
  beliefs. But then I regretted that, because that wasn't really true.
\end{itemize}

chana joffe-walt

You regretted what?

\begin{itemize}
\tightlist
\item
  elaine\\
  Well, I kind of wish I had sent them to 293 because Joan's kids had a
  good experience there.
\end{itemize}

chana joffe-walt

Elaine's friend Joan, another white mom who did send her kids to I.S.
293. Elaine still feels bad about her choice. But not everyone felt bad.

\begin{itemize}
\tightlist
\item
  carol netzer\\
  We were not pious, kind of, oh, the kids have to go to public school.
  Not at all. I went to public schools, and there's nothing to write
  about.
\end{itemize}

chana joffe-walt

Carol is the woman who wrote the letter about how she'd come to New York
City from the suburbs for integration. I had a hard time reconciling her
lack of piety with her letter, which I read back to her, about wanting
her kids to mix freely with children of other classes and races.
{[}READING{]} --- which we were not able to provide for them when we
lived in the Westchester suburb.

\begin{itemize}
\tightlist
\item
  carol netzer\\
  That was all true. Yeah, yeah.
\end{itemize}

chana joffe-walt

You remember feeling that way?

\begin{itemize}
\tightlist
\item
  carol netzer\\
  Well, I don't really remember feeling that way. And I think that we
  say a lot of things that are politically correct, without even
  realizing that we are not telling exactly how we feel. So I can't
  really guarantee that it was 100\% the way I felt. I don't really
  remember. Probably close to it, but I mean, I'm a liberal, you know?
\end{itemize}

chana joffe-walt

As a parent, did you --- do you remember feeling like, I hope my kid has
experiences outside of just people like them?

\begin{itemize}
\tightlist
\item
  carol netzer\\
  Not especially. I mean, we rushed right away to send them to private
  school, right? So what was most important to us was that they get the
  best education. But one of the things that changed it was St. Anne's
  School, a sort of progressive school with this man, headmaster, who
  was brilliant. Opened up St. Anne's. And if you keep working on this,
  you'll hear a lot about St. Anne's.
\end{itemize}

chana joffe-walt

I'm not going to tell you a lot about St. Anne's, except to say this ---
it's one of the most prominent private schools in Brooklyn. Upscale
neighborhood, prime real estate, lots of heavy-hitters send their kids
to St. Anne's. I had heard of it. What I didn't know is that St. Anne's
opened at the very same time that Black parents were waging their
strongest fight for integration in New York City, in 1965. Right when a
lot of the letter writers would have been looking for schools. And it
wasn't just St. Anne's. New progressive private schools were opening and
expanding all over the city. Brooklyn Friends School expanded into a new
building, and would double its enrollment. They were opening private
schools in the South, too. But down there, it was all very explicit.
They became known as quote, unquote, ``segregation academies,'' schools
for white people who were wholeheartedly committed to avoiding
integration. In the North, private schools opened as if they were
completely disconnected from everything else that was happening at that
very moment. St. Anne's marketed itself as a pioneer, a community of
like-minded, gifted kids, no grades. Lots of talk about progressive,
child-centered education, the whole child. At one point in my
conversation with Carol Netzer I was talking about how integration was
happening around his time. And she surprised me by saying, no, not at
that time.

\begin{itemize}
\tightlist
\item
  carol netzer\\
  I think the --- I think that you may be off on the timing for me,
  because it was too early. They didn't start really any kind of crusade
  about integrating until well after I had left the neighborhood.
\end{itemize}

chana joffe-walt

No, they were integrating the schools in the `60s, though.

\begin{itemize}
\tightlist
\item
  carol netzer\\
  Oh. It didn't make much of a splash. We weren't against it. There was
  --- it wasn't a big item.
\end{itemize}

chana joffe-walt

That's how easy it was to walk away from integration in New York City.
You could do it without even knowing you'd thrown a bomb over your
shoulder on the way out.

{[}music{]}

Here is what I think happened over those five years between the writing
of the letters in 1963 and not sending their kids to the school in 1968.
Those five years were a battle between the Board of Education's
definition of integration and the actual integration that black parents
wanted. For black parents, integration was about safe schools for their
children, with qualified teachers and functioning toilets, a full day of
school. For them, integration was a remedy for injustice. The Board of
Ed, though, took that definition and retooled it. Integration wasn't a
means to an end. It was about racial harmony and diversity. The Board
spun integration into a virtue that white parents could feel good about.
And their side triumphed. That's the definition of integration that
stuck, that's still with us today. It's the version of integration that
was being celebrated 50 years later, at the French Cultural Services
Building at the Gala for SAS.

In some of my calls with the white letter writers, a few people
mentioned that yes, they wanted integration. But also, they wanted the
school closer to them. They weren't comfortable sending their kids over
to the other side of the neighborhood. Which brings me to one final
letter from the other side of the neighborhood. One I haven't told you
about, from the I.S. 293 folder in the archives. It's one of the only
letters, as far as I can tell, that is not from a white parent. It's
from the Tenants Association for the Gowanus Houses, a housing project,
home to mostly Black and Puerto Rican families. They also wanted a
school closer to them. The letter from the Tenants Association is formal
and straightforward. It says, please build the school on the original
site you proposed, right next to the projects. That way, they explained,
our kids won't have to cross many streets. We'll get recreational
facilities, which we desperately need. And it'll be close to the people
who will actually use it. The letter says they represent over 1,000
families. The white families, they numbered a couple dozen. Still, in
the name of integration, the white letter writers got what they wanted
--- a new building close to where they lived, that they did not attend.
Note the Black and Puerto Rican families we're not asking to share a
school with white people. They were not seeking integration. That's not
what their letter was about. They were asking for a school, period. The
school they got was three blocks further than they wanted. And from the
moment it opened, I.S. 293 was de facto segregated --- an overwhelmingly
Black and Puerto Rican school. What were those years like, once the
white parents pushing their priorities went away? Once there were no
more efforts at feel-good integration, and the community was finally
left alone? Was that better? That's next time, on ``Nice White
Parents.''

``Nice White Parents'' is produced by Julie Snyder and me, with editing
on this episode from Sarah Koenig, Nancy Updike and Ira Glass. Neil
Drumming is our Managing Editor. Eve Ewing and Rachel Lissy are our
editorial consultants. Fact-checking and research by Ben Phelan, with
additional research from Lilly Sullivan. Archival research by Rebecca
Kent. Music supervision and mixing by Stowe Nelson. Our Director of
Operations is Seth Lind. Julie Whitaker is our Digital Manager. Finance
management by Cassie Howley and production management by Frances
Swanson. The original music for Nice White Parents is by The Bad Plus,
with additional music written and performed by Matt McGinley. A thank
you to all the people and organizations who helped provide archival
sound for this episode, including the Moorland-Spingarn Research Center,
Andy Lanset at WNYC, Ruta Abolins and the Walter J. Brown Media Archives
at the University of Georgia and David Ment, Dwight Johnson and all the
other people at the Board of Education archives. Special thanks to
Francine Almash, Jeanne Theoharis, Matt Delmont, Paula Marie Seniors,
Ashley Farmer, Sherrilyn Ifill, Monifa Edwards, Charles Isaacs, Noliwe
Rooks, Jerald Podair and Judith Kafka.

``Nice White Parents'' is produced by Serial Productions, a New York
Times Company.

\href{https://www.nytimes.com/column/nice-white-parents}{\includegraphics{https://static01.nyt.com/images/2020/07/21/podcasts/nice-white-parents-album-art/nice-white-parents-album-art-square320.jpg}Nice
White Parents}

\hypertarget{episode-two-i-still-believe-in-it-1}{%
\section{Episode Two: `I Still Believe in
It'}\label{episode-two-i-still-believe-in-it-1}}

\hypertarget{white-parents-in-the-1960s-fought-to-be-part-of-a-new-racially-integrated-school-in-brooklyn-so-why-did-their-children-never-attend-1}{%
\subsection{White parents in the 1960s fought to be part of a new,
racially integrated school in Brooklyn. So why did their children never
attend?}\label{white-parents-in-the-1960s-fought-to-be-part-of-a-new-racially-integrated-school-in-brooklyn-so-why-did-their-children-never-attend-1}}

Reported by Chana Joffe-Walt; produced by Julie Snyder; edited by Sarah
Koenig, Neil Drumming and Ira Glass; editorial consulting by Eve L.
Ewing and Rachel Lissy; and sound mix by Stowe Nelson

Transcript

transcript

Back to Nice White Parents

bars

0:00/53:37

-0:00

transcript

\hypertarget{episode-two-i-still-believe-in-it-2}{%
\subsection{Episode Two: `I Still Believe in
It'}\label{episode-two-i-still-believe-in-it-2}}

\hypertarget{reported-by-chana-joffe-walt-produced-by-julie-snyder-edited-by-sarah-koenig-neil-drumming-and-ira-glass-editorial-consulting-by-eve-l-ewing-and-rachel-lissy-and-sound-mix-by-stowe-nelson-1}{%
\subsubsection{Reported by Chana Joffe-Walt; produced by Julie Snyder;
edited by Sarah Koenig, Neil Drumming and Ira Glass; editorial
consulting by Eve L. Ewing and Rachel Lissy; and sound mix by Stowe
Nelson}\label{reported-by-chana-joffe-walt-produced-by-julie-snyder-edited-by-sarah-koenig-neil-drumming-and-ira-glass-editorial-consulting-by-eve-l-ewing-and-rachel-lissy-and-sound-mix-by-stowe-nelson-1}}

\hypertarget{white-parents-in-the-1960s-fought-to-be-part-of-a-new-racially-integrated-school-in-brooklyn-so-why-did-their-children-never-attend-2}{%
\paragraph{White parents in the 1960s fought to be part of a new,
racially integrated school in Brooklyn. So why did their children never
attend?}\label{white-parents-in-the-1960s-fought-to-be-part-of-a-new-racially-integrated-school-in-brooklyn-so-why-did-their-children-never-attend-2}}

Thursday, July 30th, 2020

\begin{itemize}
\item
  ``Nice White Parents'' is brought to you by Serial Productions, a New
  York Times Company.
\item
  chana joffe-walt\\
  The New York City Board of Education has an archive of all of its
  records. Everything that goes into making thousands of schools run for
  years and years is sitting in boxes in the municipal building. I love
  the B.O.E. archive.
\item
  chana joffe-walt\\
  Good morning. How are you doing?
\end{itemize}

chana joffe-walt

First of all, to look through it, you have to go to a century-old
municipal building downtown. Arched doorways, lots of marble, an echo,
vaulted ceilings really makes a person feel like she's up to something
important. You sit at a table, and then a librarian rolls your boxes up
to you on a cart. Inside the boxes are all the dramas of a school
system. Big ones, tiny ones, bureaucratic, personal, it's all in there.
There's a union contract and then a zoning plan and special reports on
teacher credentialing, a weird personal note from a bureaucrat to his
assistant, a three-page single-spaced plea from Cindy's grandmother, who
would please like for her not to be held back in the second grade. An
historian friend once pulled a folder out of the archive and a note fell
out, something a teacher clearly made a kid write in the 1950s, that
read, quote, ``I am a lazy boy. Miss Fitzgerald says, when I go in the
army, I will be expendable. Expendable means that the country doesn't
care whether I get killed or not. I do not like to be expendable. I'm
going to do my work and improve.''

{[}music{]}

I came to the Board of Ed archive after I attended the gala thrown by
the French embassy, the fundraiser for SIS organized by the new
upper-class white families coming into the school. I felt like I'd just
watched an unveiling ceremony for a brand-new school, but I didn't
really know what it was replacing. Everyone was talking as if this was
the first time white parents were taking an interest in the School for
International Studies. But at the archive, I found out it wasn't the
first time. White parents had invested in the school before, way before,
at the very beginning of the school. Before the beginning. I found a
folder labeled I.S. 293, Intermediate School 293, the original name for
SIS. And this folder was filled with personal letters to the president
of the New York City Board of Education, a man named Max Rubin, pleading
with him to please make I.S. 293 an integrated school. ``Dear Mr. Rubin,
my husband and I were educated in public schools, and we very much want
for our children to have this experience. However, we also want them to
attend a school which will give them a good education, and today, that
is synonymous with an integrated school.'' ``Dear Mr. Rubin, as a
resident of Cobble Hill, a teacher and a parent, I want my child to
attend schools which are desegregated. I do not want her to be in a
situation in which she will be a member of a small, white, middle-income
clique.'' These are letters from parents --- largely white parents, as
far as I could tell --- written in 1963, just a few years before I.S.
293 was built. At issue was where the school was going to be built. The
Board of Education was proposing to build the school right next to some
housing projects. The school would be almost entirely Black and Puerto
Rican. These parents, white parents, came in and said, no, no, no, don't
build it there. Put it closer to the white neighborhood. That way, all
our kids can go to school together. These parents wanted the school
built in what was known as a fringe zone. This was a popular idea at the
time, fringe schools to promote school integration. Comes up in the
letters. ``Dear Mr. Rubin, this neighborhood is changing with the influx
of a middle-class group which is very interested in public education for
their children.'' ``Dear Mr. Rubin, if there is a possibility of
achieving some degree of integration, it is more likely if the Board of
Education's theory of fringe schools is applied.'' And from another
letter, ``it is apparent from the opinion of the neighborhood groups
involved that the situation is not at all hopeless.'' This lobbying
effort was so successful that the Board of Education did move the site
of the school. This is why SIS is located where it is today, on the
fringe, closer to the white side of town, so that it would be
integrated.

I tried to imagine who these people were --- young, idealistic white
parents living in Brooklyn in the 1960s, feeling good about the future.
They would have had their children around the time the Supreme Court
ruled on Brown versus Board of Education. They probably followed the
news of the Civil Rights Movement unfolding down South. Maybe they were
supporters or active in the movement themselves. These were white
parents saying, we understand we're at a turning point and we have a
choice to make right now, and we choose integration. One of my favorite
letters was from a couple who left the suburbs to come to New York City
for integration, the opposite of white flight. ``Dear Mr. Rubin, we have
recently moved into the home we purchased at the above address in Cobble
Hill. It was our hope in moving into the neighborhood that our children
would enjoy the advantages of mixing freely with children of other
classes and races, which we were not able to provide to them when we
lived in a Westchester suburb.''

\begin{itemize}
\item
  chana joffe-walt\\
  So this is the letter.
\item
  carol netzer\\
  This is the letter that I wrote? I can't believe it. OK.
\end{itemize}

chana joffe-walt

This is Carol Netzer. Most of the letter writers were not that hard to
find.

\begin{itemize}
\tightlist
\item
  carol netzer\\
  We had moved to Scarsdale for the children, because Scarsdale has the
  best --- it probably still does --- the best school system in the
  country, but we hated it. We found that we were bored to death with
  it. It was bland. It was just homogeneous. But living --- I don't know
  if you've ever lived in a suburb. It's just boring, tedious, you know?
  There's nothing going on.
\end{itemize}

chana joffe-walt

She didn't like the suburbs. So they moved to Brooklyn and wrote that
letter, which I showed her, her 37-year-old self writing about her hopes
for her young children, the choices she made back then.

\begin{itemize}
\tightlist
\item
  carol netzer\\
  But it sounds as though I was fairly impassioned about it. You know,
  that it meant something. But I --- actually, I can't think what it
  meant.
\end{itemize}

{[}music{]}

chana joffe-walt

I went through this box of letters and called as many parents as I
could. Most of them didn't remember writing these letters, which isn't
surprising, more than 50 years ago and all. What I did find surprising
is that, by the time 293 opened, five years later, none of them, not a
one, actually sent their kids to I.S. 293.

{[}music{]}

From Serial Productions, I'm Chana Joffe-Walt. This is ``Nice White
Parents,'' a series about the 60-year relationship between white parents
and the public school down the block, a relationship that began with a
commitment to integration. In the 1960s, much like today, white people
were surrounded by a movement for the civil rights of Black Americans.
White people were forced to contend with systemic racism. And here was a
group of white parents who supported the movement for school
integration, threw their weight behind it. What happened in those five
years between 1963, when these white parents planted an impassioned
pro-integration flag on the school, and 1968, when it came time to
enroll their children? Why didn't they show up?

These white parents who wanted an integrated I.S. 293, they didn't come
to that idea on their own. They were part of a bigger story unfolding
around them. I want to zoom out to that dramatic story because it takes
us right up to the moment these parents wrote their letters, and then
made the decision not to send their kids to the school. To begin, I'd
like to introduce you to our main character in this historical, tale,
the recipient of the parents' letters, the New York City Board of
Education. Back in the 1950s, the New York City Board of Ed was not one
of those boring bureaucracies that chugs along in the background,
keeping its head down. It had personality. It invested in self-image.
For instance, in 1954, when the Supreme Court found school segregation
unconstitutional, New York City didn't just say we support that ruling,
it celebrated the Brown v Board decision. And notably, it celebrated
itself, calling Brown, quote, ``a moral reaffirmation of our fundamental
educational principles.'' That same year, 1954, the New York City Board
of Ed made a film honoring multiculturalism in its schools. {[}CHILDREN
SINGING{]} The film opens with a multiracial choir of schoolchildren
singing ``Let Us Break Bread Together.'' Like I said, the Board of Ed
went the extra mile. The Schools Superintendent was a 66-year-old man
named Dr. William Jansen, a man that newspapers described as slow and
steady. And he definitely delivers on that promise here.

\begin{itemize}
\tightlist
\item
  archived recording (william jansen)\\
  The film you're about to see tells the story of how the schools and
  community are working together to build brotherhood.
\end{itemize}

chana joffe-walt

A teacher addresses her classroom, filled with children of all races and
ethnicities.

\begin{itemize}
\tightlist
\item
  archived recording\\
  Who among you can give some of the reasons why people left their
  native lands to come to the United States of America?
\end{itemize}

chana joffe-walt

The camera cuts to a white boy, maybe 9 or 10.

\begin{itemize}
\tightlist
\item
  archived recording\\
  Some came because they wanted to get away from the tyranny and cruelty
  of kings.
\end{itemize}

chana joffe-walt

Then a Black girl, around the same age.

\begin{itemize}
\tightlist
\item
  archived recording\\
  My people are free now. They are proud to be American. But the Negroes
  were brought here by wicked men who traded in slaves.
\end{itemize}

chana joffe-walt

This keeps going, kid to kid.

\begin{itemize}
\tightlist
\item
  archived recording\\
  We came a little while ago from Puerto Rico. My father wanted work. He
  wants to give me and my brother a good education. Japan is very
  overcrowded. The people have little land. So many Japanese came to
  this country because they wanted to farm.
\end{itemize}

chana joffe-walt

New York City was the biggest city in America, with the largest Black
population in America, and it was saying in films, press releases,
public speeches, Brown v Board, we agree. Separate but equal has no
place in the field of public education. No problem here. It was also
saying, you know who does have a problem? The South. New York City loved
comparing itself to the backward South. There are plenty of examples of
this in the Board archives, New Yorkers bragging about their superiority
to places like Georgia or Virginia or Louisiana. This was the story the
Board of Ed was telling. The South was ignorant and racist. New York
City was enlightened and integrated. But here is what it was actually
like to walk into a New York City school in a Black neighborhood at this
time.

\begin{itemize}
\tightlist
\item
  archived recording (mae mallory)\\
  The school had an awful smell. It was just --- oh, it smelled like
  this county abattoir.
\end{itemize}

chana joffe-walt

This is an archival recording of a woman named Mae Mallory. In the
1950s, Mallory's two Black children were students in Harlem. And when
Mallory walked into their school, she did not see children building
brotherhood in interracial classrooms. She saw an all-Black and Puerto
Rican school with terrible facilities, in disrepair.

\begin{itemize}
\tightlist
\item
  archived recording (mae mallory)\\
  So my kids told me, says, well, Mommy, this is what we've been trying
  to tell you all along, that this place is so dirty. And this is why we
  run home to the bathroom every night. So I went to the bathroom. And
  in 1957 in New York City, they had toilets that were worse than the
  toilets in the schools that I went to in Macon, Georgia in the heart
  of the South. The toilet was a thing that looked like horse stalls.
  And then it had one long board with holes cut in it. And then you'd
  have to go and use the toilet, but you couldn't flush it. The water
  would come down periodically and flush, you know, whatever's there.
  Now imagine what this is like, you know, dumping waste on top of waste
  that's sitting there waiting, you know, accumulating till the water
  comes. This was why this place smelled so bad.
\end{itemize}

chana joffe-walt

Mae Mallory says the school had two bathrooms for 1,600 children.
Mallory's family fled racial violence in the South, like millions of
other Black Americans, who headed to places like New York City, where
everyone was supposed to be equal. Instead of welcoming these new
students and spreading them out, creating interracial classrooms, the
Board of Education kept Black and Puerto Rican students segregated in
what were sometimes referred to as ghetto schools, schools that were
often just blocks away from white schools. White schools in New York
City had toilets that flushed. White children had classrooms with
experienced teachers and principals, people who lived in their
communities and looked like them. In Black and Puerto Rican schools,
half the teachers were not certified to teach by the Board of Education.
The buildings were in disrepair, and packed, sometimes more than 1,000
kids in a single hallway. The overcrowding got so bad the Board of
Education decided to send kids to school in shifts. And mind you, this
was not in the middle of a global pandemic. This was normal, non-crisis
school for Black and Puerto Rican kids. One group of children would go
to school in the morning until noon. The next group of kids would come
in at noon, and stay until 3:00. The Board was literally giving Black
kids half an education. In some schools in Harlem, they had triple
shifts. This made it harder to learn elementary skills. Reading, for
instance. Black parents complained that the schools were not teaching
their kids basic literacy, that their white teachers didn't care, that
the summer reading programs were only in white communities, that their
children were two years behind white children in reading. This at
exactly the same time the Board of Education was making a film promoting
the virtues of integration. It was effectively running a dual,
segregated and unequal school system.

{[}music{]}

For many Black families, the Board of Education was not to be trusted.
It did not care for Black children, and it didn't respect the voices and
concerns of Black parents. Mae Mallory says she visited her kids' school
that day because they'd come home the day before and told her a child
had died at school. He was playing in the street at recess. Mallory
hardly believed it, but she says when she visited the school, she
learned, yes, indeed, this child was playing the street because the
schoolyard was closed. He was hit by a beer truck. And she learned the
schoolyard was closed because pieces of steel from the side of the
building had fallen into the yard.

\begin{itemize}
\tightlist
\item
  archived recording (mae mallory)\\
  And when I found out that this was true, I went to the principal. So
  this principal told me that, well, Mrs. Mallory, you really don't have
  anything to worry about. You see, our sunshine club went to see the
  mother, and we took her a bag of canned goods. So actually, she's
  better off, because she had so many children to feed. And I couldn't
  believe that here a white man is going to tell a Black woman in Harlem
  that a can of peaches is better than your child. I just didn't know
  what to do or where to go. But I know you're supposed to do something.
\end{itemize}

chana joffe-walt

It was 1957, three years after the Supreme Court declared segregation by
law unconstitutional. New York City didn't have Jim Crow laws on the
books, but Mae Mallory would ask, the schools are segregated. What's the
difference? She didn't care whether that segregation was codified by law
or by convention. The harm was just as dire. And she wanted it
addressed.

\begin{itemize}
\tightlist
\item
  archived recording (mae mallory)\\
  This was nothing to do with wanting to sit next to white folks. But it
  was obvious that a whole pattern of Black retardation was the program
  of the Board of Education. So I filed a suit against the Board of
  Education. And I just fought back.
\end{itemize}

chana joffe-walt

Integration, Mae Mallory would say, was about, quote, demanding a fair
share of the pie. She said, our children want to learn, and they
certainly have the ability to learn. What they need is the opportunity.
The Board of Education had defined integration as a multiracial choir.
It was a virtue in and of itself. Mae Mallory saw integration as a
remedy, a way to get the same stuff everyone else had --- functioning
toilets, books, certified teachers, a full school day. Integration was a
means to an end.

{[}music{]}

Mae Mallory won her lawsuit. She and a few other parents were allowed to
transfer their kids out of segregated schools. As for the segregation in
the entire system, the judge in the lawsuit turned to the Board of Ed
and said, this segregation, it's your responsibility. Fix it.

Now, on the question of responsibility, the Board of Education was
cagey. And that caginess set the stage for the I.S. 293 parents when it
came time to send their kids to the school. Here's what happened. The
Schools Superintendent, William Jansen, decided school segregation was
not his problem. In fact, he rejected the idea that New York City had
segregated schools in the first place. After all, New York City was not
barring Black children from entering white schools. This wasn't the
South. Segregation, Jansen said, is such an unfortunate word. He
preferred the phrase racial imbalance or racial separation. The way he
saw it, racial imbalance in the schools was just a matter of housing.
Neighborhoods were segregated. Again, unfortunate, but that had nothing
to do with the schools. To make this argument, William Jansen had to
ignore the many powerful tools available to the Board of Education. The
Board of Education was responsible for where kids went to school. It
decided where to build new schools. It drew zoning lines. It decided
where experienced teachers teach. There were many ways the Board could
have made schools less segregated. I know this because of the Board's
own reports. Jansen did very little to break up school segregation, but
man, did he study it. He organized commissions that led to reports that
led to further study. You see a pattern emerge, starting in the late
1950s, that looks something like this. Black parents and civil rights
groups would pressure the Board to act on segregation. The Board would
invite its critics to join a commission to investigate the problem. The
commission would study the schools, discover extreme segregation, lay
out solutions. The Board of Ed would then take a tiny step toward
implementing some of the recommendations until white parents started to
complain about the changes, at which point the Board would back off and
say it needed more evidence. Another commission, another report. For
instance, there's the Report on the Committee on Integration, a Plan for
Integration, the City's Children and the Challenge of Racial
Discrimination, Redoubling Efforts on Integration, the Board Commission
on Integration, the Status of the Public School Education of Negro and
Puerto Rican Children in New York City, and, my favorite, a bound little
red book from 1960 called Toward Greater Opportunity, which summarizes
the previous investigations with this groundbreaking conclusion. Quote,
``we must integrate as much and as quickly as we can.'' I want to pause
for one second and step out of the past back into the world we all live
in, just to point out that, over the last few years in New York City,
we've been reliving this chapter of history. It's eerie. New York City
schools are segregated. There's a growing movement to do something about
that. And for the first five years of his administration, the city's
mayor, Mayor Bill de Blasio, responded in the following way. He refused
to say the word segregation, commissioned a number of reports on school
diversity. He's pointed a finger at housing problems as a way to say
this isn't our fault, and he's studying the problem deeply, which,
again, is not segregation, no matter how many times reporters would ask
the mayor at press conferences, why don't you use that word?

\begin{itemize}
\tightlist
\item
  archived recording (bill de blasio)\\
  I don't get lost in terminology. I think the notion of saying we have
  to diversify our schools is the best way to say it.
\end{itemize}

chana joffe-walt

I heard a live call-in show on WNYC, the public radio station. A young
integration advocate, an 11th grader named Tiffani Torres, asked the
mayor, how much longer until you do something?

\begin{itemize}
\item
  archived recording (tiffani torres)\\
  And how much more time do you need to study the issue? So to repeat my
  question, how much longer will it take?
\item
  archived recording (bill de blasio)\\
  Tiffani, with all due respect, I really think you're not hearing what
  we're saying to you, so I'll repeat it. There is a task force, an
  extraordinary task force, which I've met with. They are coming forward
  with their next report in a matter of weeks. So when that diversity
  task force comes out with their report, I think they're amazing. I
  think they've done fantastic work. And so far, there's a high level
  ---
\end{itemize}

chana joffe-walt

Mayor de Blasio likes to point out that this was a problem created by
people long before him, which is exactly what people long before him
said, too.

{[}music{]}

In the late 1950s, when Black parents and civil rights activists also
asked the Board of Ed, why is it taking so long, board members
complained about the, quote, extremists who wanted instant integration.
Superintendent Jansen said, ``some people want us to build Rome in one
day.'' While the Board of Education was building Rome in 1956, `57, `59,
and in 1960, 1962, `63, Black parents found each other on PTAs, in civil
rights organizations, pro-integration groups. They formed new groups,
organized sit-ins, boycotts, demanded the Board provide a timetable for
citywide integration. They joined forces with Puerto Rican parents, and
their numbers grew. These were volunteers, mothers mostly, who left
their jobs at the end of a workday and headed directly to a meeting
about how to get the Board to give their kids the education white
children were already receiving. Finally, in 1964, 10 years after Brown
versus Board, Black and Puerto Rican parents said, enough. They were
sick of waiting, sick of lawsuits, sick of asking for a remedy, sick of
being ignored. So they went big, spectacularly big. They shut down the
schools. They organized a civil rights demonstration that was the
largest in US history, larger than the March on Washington. It was
called Freedom Day, a massive school boycott.

\begin{itemize}
\tightlist
\item
  archived recording\\
  (CHANTING) Freedom now!
\end{itemize}

chana joffe-walt

On February 3rd, 1964, parents headed out to schools in the morning
before sunrise to spread the word about the boycott. It was freezing
cold that day. There's a brief TV news clip of a group of mothers
picketing outside their kids' school at the start of the school day.
They're holding up signs that say, ``we demand a real integration
timetable now,'' and ``integration means better schools for all.''
They're handing out leaflets to other parents about Freedom Day, looking
spirited and cold. A white NBC news reporter in a fedora walks up to one
of the women.

\begin{itemize}
\tightlist
\item
  archived recording\\
  Ma'am, it's a little after eight o'clock now. How successful has the
  boycott been so far? Very effective. So far, about 10 children have
  gone in, and there would be ordinarily 240 children. And 10 have gone
  into the morning session, which begins at eight o'clock. So you think
  you've already seen the result? Yes, I think so. The school is just
  empty. Does it surprise you? No, because we knew how effective --- We
  talked with the parents. We distributed leaflets. We've been working
  very hard. And we prayed that it would be effective.
\end{itemize}

chana joffe-walt

There were maps and charts and instructions with picket times and picket
captains for hundreds of schools. There were volunteer shifts to make
peanut butter and jelly sandwiches, to hand out thousands of leaflets
and stencil posters. The boycott wasn't just effective --- it was
extraordinarily effective. Half a million kids stayed home from school
that day. Half a million, close to half the school system. But the press
barely covered it. After searching every major TV network, I found only
one kid who was interviewed, a teenage boy, maybe around 16, on the
street with some friends, protesting. A white ABC News reporter doesn't
ask him why he's there. The only thing he asks him about is violence.
The kid responds.

\begin{itemize}
\tightlist
\item
  archived recording\\
  We're coming down here today for a peaceful --- peaceful--- No
  comment! No, we're not going to be violent. We're just teenagers and
  kids. And --- Do you expect violence here today? No, sir, not if ---
  look at the blue uniforms. You ask me do I expect violence.
\end{itemize}

chana joffe-walt

He gestures to the police on horseback.

\begin{itemize}
\tightlist
\item
  archived recording\\
  None of us have any weapons, horses. And all we want is equal
  education. That's all. Equal education. Thank you. You get all that?
\end{itemize}

chana joffe-walt

That was it. Every once in a while, I'll hear a politician or friend or
school administrator say, yeah, integration was a good idea, but there
was no political will to make it happen. 460,000 kids, half the school
system. The will was there. The majority wanted integration.

{[}music{]}

After Freedom Day, the Board of Education introduced some small-scale
integration plans, and white parents protested. {[}CHILDREN SINGING{]}
We love our children. Oh, yes, we do. We will not transfer ---

chana joffe-walt

With their own marches, they put on their own school boycott. The
flipside of freedom day, a white boycott. The white parents were far
fewer in number. But as far as I can tell, they got a thousand times
more press coverage.

\begin{itemize}
\item
  archived recording\\
  Mrs. Carcevski?
\item
  speaker\\
  Yes? Are you going to send Johnny back to school now? No. She belongs
  here, and I want to send my child here. So nobody is going to tell me
  where to send my kid.
\end{itemize}

chana joffe-walt

This protest worked. The Board of Ed backed off. And in the decades
since, the Board of Education has never proposed a city-wide integration
plan. The schools have never been integrated. I think the fact of white
moms in Queens in the 1960s yelling about zoning changes and busing,
it's not surprising they played a role in killing school integration
efforts. But there was another group of white parents who played a
quieter, but I'd argue more forceful, role in killing integration. The
white parents who said they supported it, parents like the ones who
wrote letters asking for an integrated I.S. 293. How did their vocal
support for integration turn lethal? That's after the break. In the
American South, schools were desegregated with court orders. Cities and
counties mandated desegregation, and the schools desegregated. By the
early 1970s, the South was the most integrated region in the country.
But New York City did not want to do it that way. No mandates. The New
York City Board of Education wanted to appeal to hearts and minds. They
wanted to sell white people on the virtues of integration. Have it all
happen, quote, ``naturally.'' Some white people were sold. The white
parents who wrote letters about I.S. 293. They believed in integration.
So I made a lot of calls to ask, why'd you bail? They had a lot of
different reasons. One couple got divorced, and moved. Another guy told
me he had political ambitions that pulled him out of the city.

\begin{itemize}
\tightlist
\item
  speaker\\
  We loved our brownstone, but I was involved in a political race. And
  we needed some money for that.
\end{itemize}

chana joffe-walt

So he sold the house and moved the family to the suburbs, where he
thought he'd have a better chance running against Republicans. Many
white people moved to the suburbs for jobs, for newly paved roads and
subsidized mortgages, leaving Brooklyn behind. I understood what
happened there. But some explanations made less sense. Like one guy I
called, he did stay in Brooklyn. On the phone, he was telling me why he
believed it was important that I.S. 293 be integrated. But then he said
his own kids went to Brooklyn Friends, a Quaker private school. I said,
oh, they didn't go to I.S. 293.

\begin{itemize}
\tightlist
\item
  speaker\\
  No. As I said, I'm a Quaker, and ---
\end{itemize}

chana joffe-walt

But you were a Quaker when you wrote this letter, asking for an
integrated 293.

\begin{itemize}
\tightlist
\item
  speaker\\
  I believed it. I believed in it, but ---
\end{itemize}

chana joffe-walt

You weren't planning to send your kid there?

\begin{itemize}
\tightlist
\item
  speaker\\
  No, no, no.
\end{itemize}

chana joffe-walt

What to make of that? When you get what you say you want and then, given
the opportunity, don't take it. Maybe you never really wanted it in the
first place. Then I spoke to Elaine Hencke. Of all the people I spoke
with, everything about Elaine indicated someone who did believe in
integration, someone who would send her kids to 293. And yet, she
didn't. Elaine was a public school teacher. She taught in an integrated
elementary school, until she had her own kids. She was looking forward
to sending them to an integrated 293. When her daughter was old enough
for junior high school, Elaine visited the school. She was the only
letter writer I spoke with who actually went into the building. If this
was going to work with anyone, it was going to be Elaine.

\begin{itemize}
\tightlist
\item
  elaine hencke\\
  I didn't know quite what to make of it because the school had a nice
  plant. Physically, it was a nice school. But it just seemed chaotic
  and noisy, and kids were disruptive. And kids --- {[}LAUGHS{]} ---
  kids were doing the wrong things, you know? And kids do. I mean, it
  wasn't that they were nasty kids or doing --- it was not drugs. It was
  not drugs. It was just --- it just seemed too chaotic to me at the
  time.
\end{itemize}

chana joffe-walt

Elaine and I talked for a long time I pushed her --- not to make her
feel bad, but to get to what felt like a more real answer. At the time
that you are visiting, was it majority Black and Hispanic kids?

\begin{itemize}
\tightlist
\item
  elaine hencke\\
  Yes, I'm sure it was.
\end{itemize}

chana joffe-walt

And did that have anything to do with the way that you saw the classroom
as disruptive and chaotic?

\begin{itemize}
\item
  elaine hencke\\
  I would hope not.

  I'm not --- I'm not sure how well educated they were, or --- you know,
  I don't know. I don't know why I'm going into this.
\end{itemize}

chana joffe-walt

Well, did you have reason to think that they weren't well educated?

\begin{itemize}
\tightlist
\item
  elaine hencke\\
  Before 293? Well, their reading levels were way down. You know.
\end{itemize}

chana joffe-walt

I'm just --- when you say chaos and disruptive, I'm trusting that what
you saw was chaotic and disruptive. But I also know that those are words
white people use --- we use to express our racial fears, to express real
racial fears. Do you think that's what was happening with you?

\begin{itemize}
\tightlist
\item
  elaine hencke\\
  I don't think I would admit to that. I don't think that was true. But
  what I may have thought was that these kids are not expected to do so
  well in school, all the way from the beginning of school. And here
  they are, really unprepared in some way, for junior high school or ---
  I mean, the reading levels were low.
\end{itemize}

chana joffe-walt

Elaine told me when she wrote that letter to the Board of Education, she
pictured her children becoming friends with Black kids, learning
side-by-side, learning that all children are equal. That's what
motivated her to write that letter. She wanted the picture of
integration the Board of Ed was promoting --- the picture of harmonious
integration. But when she visited I.S. 293, that didn't seem possible.
The reading levels were low. The kids were not entering the school on
equal grounds. Her white children had received years of high quality
teaching at well-resourced schools. The kids coming from segregated
elementary schools had not had that experience.

\begin{itemize}
\tightlist
\item
  elaine hencke\\
  I mean, one of the problems is that many of the white kids had higher
  sort of academic skills, or skills. They could read better. I think
  --- I mean, if the white kids knew how to read in first grade and ---
  and I guess there were Black kids who also could. But it just seemed
  as if most of the black kids didn't really learn --- learn to read.
\end{itemize}

chana joffe-walt

But part of the --- part of the vocal complaints of black parents at
this period of time was that their kids were not learning how to read
because schools were segregated, and their kids were kept in schools
that were inferior. And that was part of the argument for integration.

\begin{itemize}
\tightlist
\item
  elaine hencke\\
  Yes, yes.
\end{itemize}

chana joffe-walt

That their kids were not going to get the resources, and quality
teaching, and good facilities unless they were in the same buildings
with kids like yours.

\begin{itemize}
\item
  elaine hencke\\
  Right.

  I don't know what to say to that. I just --- I guess I just began to
  feel that things were really difficult for these kids. Schools were
  not made for them. If the schools were made for them, with their
  background, what would they be like?

  I think there was --- and that's another whole thing. I don't know
  about it. I think there was sort of anger in the black community at
  the white community. A lot of the teachers were white. There were more
  white teachers, I suppose. People said that that was racism. And of
  course, it was racism. But maybe the kids were a little angry at the
  school. I wouldn't --- I couldn't fault them for that. But on the
  other hand, then they don't get as much from the school. I don't know.
  I thought the problems were kind of enormous. And I guess I just, at
  one point, I just decided that my kids should go --- went to Brooklyn
  Friends. And we could afford to pay for it. It wasn't easy, you know.
  It was --- {[}LAUGHS{]} but ---
\end{itemize}

chana joffe-walt

Did your feelings about integration change? Did you believe in it less?

\begin{itemize}
\item
  elaine hencke\\
  Maybe.

  I think I would have said no, theoretically. But maybe they did. I
  guess I saw it as a more difficult project then. I sort of did back
  off from it. I just ---
\end{itemize}

chana joffe-walt

Yeah. It felt when you guys wrote these letters like, this is ---
integration is this exciting ideal, and we can be part of it, and it's
going to be a meaningful project that's also going to be kind of easy.

\begin{itemize}
\tightlist
\item
  elaine hencke\\
  I certainly didn't think it would be so difficult. But I --- I was, I
  was innocent, you know? I don't know. I still believe in it. I do.
\end{itemize}

{[}music{]}

chana joffe-walt

I think what Elaine actually meant was not that she was innocent, but
that she was naive. She was naive about the reality of segregation, the
harm of it. And naive about what it would take to undo it. She did not
know. And I think she didn't want to know. When Elaine said the word
innocent, I felt a jolt of recognition. I felt like Elaine had walked me
right up to the truth about her, and about me.

When my own kids were old enough, I sent them to our zoned public
school. It was racially mixed and economically mixed. I was excited
about that. And it was nice walking to school with neighbors, people I
likely never would have gotten to know otherwise. My kid's first day of
school was another boy's first week in the country. He'd just moved from
China, and his mom asked the neighbor where the school was. When she
said goodbye that first morning, I think he thought I was a teacher, and
he crawled into my lap. We had no words in common, so I just held him
while he screamed and cried. By the holiday show three months later, I
watched that same boy belt out ``This Pretty Planet'' on a stage with
his classmates. He was the star. He nailed the hand motions. Every other
kid up on stage was just following his lead, just trying to keep up. It
was such a sweet picture, all of them up there --- Black kids, and
Mexican kids, and Colombian, and Asian and white kids. And all of us
adults supporting all of them. It's moving, to me, this picture of
integration. It is also, I'm realizing right now, writing these words
down, the very same picture the Board of Education put forth in 1954 ---
a multiracial choir singing together, building brotherhood. And it's
dangerous, I think, this picture of integration. It seems perfectly
designed to preserve my innocence, to make me comfortable, not to remedy
inequality, but a way to bypass it entirely. I can sit in that assembly
and feel good about the gauzy display of integration without ever being
asked to think about the fact that much of the time, white kids in the
school building are having a different educational experience than kids
of color. A large share of the white students at the school are
clustered in a gifted program. They have separate classrooms and
separate teachers. We all blithely call these white children gifted and
talented, G\&T, starting at four years old. White children are
performing better at the school than black children and Latino children.
White families are the loudest and most powerful voices in the building.
The advantages white kids had back in the 1950s, they're still in place.
When Elaine said she was innocent, I thought about the things we say,
nice, white parents, to each other about why we won't send our kids to
segregated schools --- because they're too strict, or too chaotic, or
too disruptive. Because the test scores are bad, because we want more
play. We want fewer worksheets. Because we don't want to ride a bus. We
don't want uniforms. We don't want tests. We want innocence. We need it,
to protect us from the reality that we are the ones creating the
segregation, and we're not sure we're ready to give it up.

{[}music{]}

Elaine was not for segregation. But in the end, she wasn't really for
integration, either. All of the choices she made, choices she had the
luxury of making, were meant to advantage her own kids. And I understand
that. That's what parents do.

\begin{itemize}
\tightlist
\item
  elaine\\
  I remember thinking very clearly that OK, I believe in this. But I
  don't sort of want to sacrifice my children to it. I have to look at
  what they will learn, and what they will do. And for people who sent
  their kids to 293, it seemed to work out well. So that made me think,
  well, maybe I made a mistake. Maybe they should have gone there. I
  know at one point it was very clear to me that I had beliefs that I
  thought were kind of contrary to my own children's best interests. And
  I decided that I wasn't going to use them to sort of extend my own
  beliefs. But then I regretted that, because that wasn't really true.
\end{itemize}

chana joffe-walt

You regretted what?

\begin{itemize}
\tightlist
\item
  elaine\\
  Well, I kind of wish I had sent them to 293 because Joan's kids had a
  good experience there.
\end{itemize}

chana joffe-walt

Elaine's friend Joan, another white mom who did send her kids to I.S.
293. Elaine still feels bad about her choice. But not everyone felt bad.

\begin{itemize}
\tightlist
\item
  carol netzer\\
  We were not pious, kind of, oh, the kids have to go to public school.
  Not at all. I went to public schools, and there's nothing to write
  about.
\end{itemize}

chana joffe-walt

Carol is the woman who wrote the letter about how she'd come to New York
City from the suburbs for integration. I had a hard time reconciling her
lack of piety with her letter, which I read back to her, about wanting
her kids to mix freely with children of other classes and races.
{[}READING{]} --- which we were not able to provide for them when we
lived in the Westchester suburb.

\begin{itemize}
\tightlist
\item
  carol netzer\\
  That was all true. Yeah, yeah.
\end{itemize}

chana joffe-walt

You remember feeling that way?

\begin{itemize}
\tightlist
\item
  carol netzer\\
  Well, I don't really remember feeling that way. And I think that we
  say a lot of things that are politically correct, without even
  realizing that we are not telling exactly how we feel. So I can't
  really guarantee that it was 100\% the way I felt. I don't really
  remember. Probably close to it, but I mean, I'm a liberal, you know?
\end{itemize}

chana joffe-walt

As a parent, did you --- do you remember feeling like, I hope my kid has
experiences outside of just people like them?

\begin{itemize}
\tightlist
\item
  carol netzer\\
  Not especially. I mean, we rushed right away to send them to private
  school, right? So what was most important to us was that they get the
  best education. But one of the things that changed it was St. Anne's
  School, a sort of progressive school with this man, headmaster, who
  was brilliant. Opened up St. Anne's. And if you keep working on this,
  you'll hear a lot about St. Anne's.
\end{itemize}

chana joffe-walt

I'm not going to tell you a lot about St. Anne's, except to say this ---
it's one of the most prominent private schools in Brooklyn. Upscale
neighborhood, prime real estate, lots of heavy-hitters send their kids
to St. Anne's. I had heard of it. What I didn't know is that St. Anne's
opened at the very same time that Black parents were waging their
strongest fight for integration in New York City, in 1965. Right when a
lot of the letter writers would have been looking for schools. And it
wasn't just St. Anne's. New progressive private schools were opening and
expanding all over the city. Brooklyn Friends School expanded into a new
building, and would double its enrollment. They were opening private
schools in the South, too. But down there, it was all very explicit.
They became known as quote, unquote, ``segregation academies,'' schools
for white people who were wholeheartedly committed to avoiding
integration. In the North, private schools opened as if they were
completely disconnected from everything else that was happening at that
very moment. St. Anne's marketed itself as a pioneer, a community of
like-minded, gifted kids, no grades. Lots of talk about progressive,
child-centered education, the whole child. At one point in my
conversation with Carol Netzer I was talking about how integration was
happening around his time. And she surprised me by saying, no, not at
that time.

\begin{itemize}
\tightlist
\item
  carol netzer\\
  I think the --- I think that you may be off on the timing for me,
  because it was too early. They didn't start really any kind of crusade
  about integrating until well after I had left the neighborhood.
\end{itemize}

chana joffe-walt

No, they were integrating the schools in the `60s, though.

\begin{itemize}
\tightlist
\item
  carol netzer\\
  Oh. It didn't make much of a splash. We weren't against it. There was
  --- it wasn't a big item.
\end{itemize}

chana joffe-walt

That's how easy it was to walk away from integration in New York City.
You could do it without even knowing you'd thrown a bomb over your
shoulder on the way out.

{[}music{]}

Here is what I think happened over those five years between the writing
of the letters in 1963 and not sending their kids to the school in 1968.
Those five years were a battle between the Board of Education's
definition of integration and the actual integration that black parents
wanted. For black parents, integration was about safe schools for their
children, with qualified teachers and functioning toilets, a full day of
school. For them, integration was a remedy for injustice. The Board of
Ed, though, took that definition and retooled it. Integration wasn't a
means to an end. It was about racial harmony and diversity. The Board
spun integration into a virtue that white parents could feel good about.
And their side triumphed. That's the definition of integration that
stuck, that's still with us today. It's the version of integration that
was being celebrated 50 years later, at the French Cultural Services
Building at the Gala for SAS.

In some of my calls with the white letter writers, a few people
mentioned that yes, they wanted integration. But also, they wanted the
school closer to them. They weren't comfortable sending their kids over
to the other side of the neighborhood. Which brings me to one final
letter from the other side of the neighborhood. One I haven't told you
about, from the I.S. 293 folder in the archives. It's one of the only
letters, as far as I can tell, that is not from a white parent. It's
from the Tenants Association for the Gowanus Houses, a housing project,
home to mostly Black and Puerto Rican families. They also wanted a
school closer to them. The letter from the Tenants Association is formal
and straightforward. It says, please build the school on the original
site you proposed, right next to the projects. That way, they explained,
our kids won't have to cross many streets. We'll get recreational
facilities, which we desperately need. And it'll be close to the people
who will actually use it. The letter says they represent over 1,000
families. The white families, they numbered a couple dozen. Still, in
the name of integration, the white letter writers got what they wanted
--- a new building close to where they lived, that they did not attend.
Note the Black and Puerto Rican families we're not asking to share a
school with white people. They were not seeking integration. That's not
what their letter was about. They were asking for a school, period. The
school they got was three blocks further than they wanted. And from the
moment it opened, I.S. 293 was de facto segregated --- an overwhelmingly
Black and Puerto Rican school. What were those years like, once the
white parents pushing their priorities went away? Once there were no
more efforts at feel-good integration, and the community was finally
left alone? Was that better? That's next time, on ``Nice White
Parents.''

``Nice White Parents'' is produced by Julie Snyder and me, with editing
on this episode from Sarah Koenig, Nancy Updike and Ira Glass. Neil
Drumming is our Managing Editor. Eve Ewing and Rachel Lissy are our
editorial consultants. Fact-checking and research by Ben Phelan, with
additional research from Lilly Sullivan. Archival research by Rebecca
Kent. Music supervision and mixing by Stowe Nelson. Our Director of
Operations is Seth Lind. Julie Whitaker is our Digital Manager. Finance
management by Cassie Howley and production management by Frances
Swanson. The original music for Nice White Parents is by The Bad Plus,
with additional music written and performed by Matt McGinley. A thank
you to all the people and organizations who helped provide archival
sound for this episode, including the Moorland-Spingarn Research Center,
Andy Lanset at WNYC, Ruta Abolins and the Walter J. Brown Media Archives
at the University of Georgia and David Ment, Dwight Johnson and all the
other people at the Board of Education archives. Special thanks to
Francine Almash, Jeanne Theoharis, Matt Delmont, Paula Marie Seniors,
Ashley Farmer, Sherrilyn Ifill, Monifa Edwards, Charles Isaacs, Noliwe
Rooks, Jerald Podair and Judith Kafka.

``Nice White Parents'' is produced by Serial Productions, a New York
Times Company.

Previous

More episodes ofNice White Parents

\href{https://www.nytimes.com/2020/08/13/podcasts/nice-white-parents-school.html?action=click\&module=audio-series-bar\&region=header\&pgtype=Article}{\includegraphics{https://static01.nyt.com/images/2020/07/30/podcasts/30nwp-art/nice-white-parents-album-art-thumbLarge.jpg}}

August 13, 2020~~•~ 50:38Episode Four: `Here's Another Fun Thing You Can
Do'

\href{https://www.nytimes.com/2020/08/06/podcasts/episode-three-this-is-our-school-how-dare-you.html?action=click\&module=audio-series-bar\&region=header\&pgtype=Article}{\includegraphics{https://static01.nyt.com/images/2020/07/30/podcasts/30nwp-art/nice-white-parents-album-art-thumbLarge.jpg}}

August 6, 2020~~•~ 46:55Episode Three: `This Is Our School, How Dare
You?'

\href{https://www.nytimes.com/2020/07/30/podcasts/nice-white-parents-serial-2.html?action=click\&module=audio-series-bar\&region=header\&pgtype=Article}{\includegraphics{https://static01.nyt.com/images/2020/07/30/podcasts/30nwp-art/nice-white-parents-album-art-thumbLarge.jpg}}

July 30, 2020~~•~ 53:37Episode Two: `I Still Believe in It'

\href{https://www.nytimes.com/2020/07/30/podcasts/nice-white-parents-serial.html?action=click\&module=audio-series-bar\&region=header\&pgtype=Article}{\includegraphics{https://static01.nyt.com/images/2020/07/30/podcasts/30nwp-art/nice-white-parents-album-art-thumbLarge.jpg}}

July 30, 2020~~•~ 1:02:23Episode One: The Book of Statuses

\href{https://www.nytimes.com/2020/07/23/podcasts/nice-white-parents-serial.html?action=click\&module=audio-series-bar\&region=header\&pgtype=Article}{\includegraphics{https://static01.nyt.com/images/2020/07/21/podcasts/nice-white-parents-album-art/nice-white-parents-album-art-thumbLarge.jpg}}

July 23, 2020~~•~ 2:49Introducing: Nice White Parents

\href{https://www.nytimes.com/column/nice-white-parents}{See All
Episodes ofNice White Parents}

Next

Published July 30, 2020Updated Aug. 10, 2020

\begin{itemize}
\item
\item
\item
\item
\item
\end{itemize}

``Nice White Parents'' is a new podcast from Serial Productions, a New
York Times Company, about the 60-year relationship between white parents
and the public school down the block.

\textbf{Listen to the first two episodes now and keep an eye out for new
episodes each Thursday, available here and on your mobile device:}
\textbf{\href{https://podcasts.apple.com/us/podcast/nice-white-parents/id1524080195}{Via
Apple Podcasts}} \textbf{\textbar{}}
\textbf{\href{https://open.spotify.com/show/7oBSLCZFCgpdCaBjIG8mLV?si=YcEPLD3xT2ejXmpQz-tRpw}{Via
Spotify}} \textbf{\textbar{}}
\textbf{\href{https://podcasts.google.com/feed/aHR0cHM6Ly9yc3MuYXJ0MTkuY29tL25pY2Utd2hpdGUtcGFyZW50cw}{Via
Google}}

In this episode, Chana Joffe-Walt searches the New York City Board of
Education archives for more information about the School for
International Studies, which was originally called I.S. 293.

In the process, she finds a folder of letters written in 1963 by mostly
white families in Cobble Hill, Brooklyn. They are asking for the board
to change the proposed construction of the school to a site where it
would be more likely to be racially integrated.

It's less than a decade after Brown v. Board of Education, amid a
growing civil rights movement, and the white parents writing letters are
emphatic that they want an integrated school. They get their way and the
school site changes --- but after that, nothing else goes as planned.

\includegraphics{https://static01.nyt.com/images/2020/07/29/multimedia/mae-mallory--dear-white-parents-podcast/mae-mallory--dear-white-parents-podcast-articleLarge.jpg?quality=75\&auto=webp\&disable=upscale}

``Nice White Parents'' was reported by Chana Joffe-Walt; produced by
Julie Snyder; edited by Sarah Koenig, Neil Drumming and Ira Glass;
editorial consulting by Eve L. Ewing and Rachel Lissy; and sound mix by
Stowe Nelson.

The original score for ``Nice White Parents'' was written and performed
by the jazz group The Bad Plus. The band consists of bassist Reid
Anderson, pianist Orrin Evans and drummer Dave King. Additional music
from Matt McGinley.

Special thanks to Sam Dolnick, Julie Whitaker, Seth Lind, Julia Simon
and Lauren Jackson.

Advertisement

\protect\hyperlink{after-bottom}{Continue reading the main story}

\hypertarget{site-index}{%
\subsection{Site Index}\label{site-index}}

\hypertarget{site-information-navigation}{%
\subsection{Site Information
Navigation}\label{site-information-navigation}}

\begin{itemize}
\tightlist
\item
  \href{https://help.nytimes.com/hc/en-us/articles/115014792127-Copyright-notice}{©~2020~The
  New York Times Company}
\end{itemize}

\begin{itemize}
\tightlist
\item
  \href{https://www.nytco.com/}{NYTCo}
\item
  \href{https://help.nytimes.com/hc/en-us/articles/115015385887-Contact-Us}{Contact
  Us}
\item
  \href{https://www.nytco.com/careers/}{Work with us}
\item
  \href{https://nytmediakit.com/}{Advertise}
\item
  \href{http://www.tbrandstudio.com/}{T Brand Studio}
\item
  \href{https://www.nytimes.com/privacy/cookie-policy\#how-do-i-manage-trackers}{Your
  Ad Choices}
\item
  \href{https://www.nytimes.com/privacy}{Privacy}
\item
  \href{https://help.nytimes.com/hc/en-us/articles/115014893428-Terms-of-service}{Terms
  of Service}
\item
  \href{https://help.nytimes.com/hc/en-us/articles/115014893968-Terms-of-sale}{Terms
  of Sale}
\item
  \href{https://spiderbites.nytimes.com}{Site Map}
\item
  \href{https://help.nytimes.com/hc/en-us}{Help}
\item
  \href{https://www.nytimes.com/subscription?campaignId=37WXW}{Subscriptions}
\end{itemize}
