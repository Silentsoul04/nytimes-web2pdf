Sections

SEARCH

\protect\hyperlink{site-content}{Skip to
content}\protect\hyperlink{site-index}{Skip to site index}

\href{https://www.nytimes.com/spotlight/podcasts}{Podcasts}

\href{https://myaccount.nytimes.com/auth/login?response_type=cookie\&client_id=vi}{}

\href{https://www.nytimes.com/section/todayspaper}{Today's Paper}

\href{/spotlight/podcasts}{Podcasts}\textbar{}Episode One: The Book of
Statuses

\url{https://nyti.ms/2D1jZMl}

\begin{itemize}
\item
\item
\item
\item
\item
\item
\end{itemize}

Advertisement

\protect\hyperlink{after-top}{Continue reading the main story}

transcript

Back to Nice White Parents

bars

0:00/01:02:23

-01:02:23

transcript

\hypertarget{episode-one-the-book-of-statuses}{%
\subsection{Episode One: The Book of
Statuses}\label{episode-one-the-book-of-statuses}}

\hypertarget{reported-by-chana-joffe-walt-produced-by-julie-snyder-edited-by-sarah-koenig-neil-drumming-and-ira-glass-editorial-consulting-by-eve-l-ewing-and-rachel-lissy-and-sound-mix-by-stowe-nelson}{%
\subsubsection{Reported by Chana Joffe-Walt; produced by Julie Snyder;
edited by Sarah Koenig, Neil Drumming and Ira Glass; editorial
consulting by Eve L. Ewing and Rachel Lissy; and sound mix by Stowe
Nelson}\label{reported-by-chana-joffe-walt-produced-by-julie-snyder-edited-by-sarah-koenig-neil-drumming-and-ira-glass-editorial-consulting-by-eve-l-ewing-and-rachel-lissy-and-sound-mix-by-stowe-nelson}}

\hypertarget{a-group-of-parents-takes-one-big-step-together}{%
\paragraph{A group of parents takes one big step
together.}\label{a-group-of-parents-takes-one-big-step-together}}

Thursday, July 30th, 2020

\begin{itemize}
\item
  ``Nice White Parents'' is brought to you by Serial Productions, a New
  York Times Company.
\item
  chana joffe-walt\\
  I started reporting this story at the very same moment as I was trying
  to figure out my own relationship to the subject of this story, white
  parents in New York City public schools. I was about to be one of
  them. When my kid was old enough, I started learning about my options.
  I had many. There was our zoned public school in Brooklyn, or I could
  apply to a handful of specialty programs --- a gifted program, or a
  magnet school, or a language program. So I started to look around.
  This was five years ago now, but I vividly remember these tours. I'd
  show up in the lobby of the school at the time listed on the website,
  look around, and notice that all or almost all of the other parents
  who'd shown up for the 11:00 AM, middle-of-the-workday,
  early-in-the-shopping-season school tour were other white parents. As
  a group, we'd walk the halls, following a school administrator ---
  almost always a man or woman of color --- through a school full of
  black and brown kids. We'd peer into classroom windows, watch the kids
  sit in a circle on the rug, ask questions about the lunch menu,
  homework policy, discipline. Some of us would take notes. And the
  administrators would sell. The whole thing was essentially a pitch. We
  offer STEM. We have a partnership with Lincoln Center. We have a dance
  studio. They were pleading with us to please take part in this public
  school. I don't think I've ever felt my own consumer power more
  viscerally than I did shopping for a public school as a white parent.
  We were entering schools that people like us had ignored for decades.
  They were not our places, but we were being invited to make them ours.
  The whole thing was made so much more awkward by the fact that nobody
  on those tours ever acknowledged the obvious racial difference, that
  roughly 100\% of the parents in this group did not match, say, 90\% of
  the kids in this building. I remember one time being guided into a
  classroom and being told that this was the class for gifted kids, and
  noticing, oh, here's where all the white kids are. Everyone on our
  tour saw this, all of us parents, but nobody said anything, including
  me. We walked out into the hallway. A mom raised her hand and said, I
  do have one question I've been meaning to ask. And the group got
  quiet. I was thinking, OK, here it comes. But then she said, do the
  kids here play outside every day?
\item
  {[}music{]}
\item
  chana joffe-walt\\
  I knew the schools were segregated. I shouldn't have been surprised.
  By the time I was touring schools as a parent, I had spent a fair
  amount of time in schools as a reporter. I'd done stories on the stark
  inequality in public education. And I'd looked at some of the many
  programs and reforms we've tried to fix our schools. So many ideas.
  We've tried standardized tests and charter schools. We've tried
  smaller classes, longer school days, stricter discipline, looser
  discipline, tracking, differentiation. We've decided the problem is
  teachers, the problem is parents. What is true about almost all of
  these reforms is that when we look for what's broken, for how our
  schools are failing, we focus on who they're failing --- poor kids,
  black kids, and brown kids. We ask, why aren't they performing better?
  Why aren't they achieving more? Those are not the right questions.
  There is a powerful force that is shaping our public schools, arguably
  the most powerful force. It's there even when we pretend not to notice
  it, like on that school tour. If you want to understand why our
  schools aren't better, that's where you have to look. You have to look
  at white parents. From Serial Productions, I'm Channa Joffe-Walt. This
  is ``Nice White Parents,'' a series about the 60-year relationship
  between white parents and the public school down the block.

  {[}CHILDREN CHATTERING{]} I'm going to take you inside a public school
  building, an utterly ordinary, squat, three-story New York City public
  school building not far from where I live. This isn't one of the
  schools I've toured. And my own kids don't go here. They're too
  little. This is a middle and high school called the School for
  International Studies, SIS. The story I want to tell you spans decades
  in this one school building, but I'm going to begin when I first
  encountered SIS, in the spring of 2015, right before everything
  changed. In 2015, the students at SIS were black, Latino, and Middle
  Eastern kids, mostly from working class and poor families. That year,
  like the year before and the year before that, the school was
  shrinking. The principal, Jillian Juman, was worried.
\item
  jillian juman\\
  Yeah, so the last two years, we had 30 students in our sixth grade
  class. And so we really have room for 100. And so numbers, I think,
  are hard.
\end{itemize}

chana joffe-walt

Ms. Juman started to reach out to families from the neighborhood,
inviting them to please come take a look. Parents started showing up for
tours of SIS, mostly groups of white parents. Ms. Juman was thrilled and
relieved. She walked parents through the building, saying, stop me
anytime if anyone has any questions. Really, anything; I want you to
feel comfortable. And Ms. Juman says they did have questions, mostly
about the poor test scores. That was fair. Ms. Juman expected those
questions. She did not expect the other set of questions she got a lot
from parents.

\begin{itemize}
\item
  jillian juman\\
  Is there weapons? Is there --- you know, are you scanning? Are you a
  scanning school, because kids are dangerous and they have weapons?
  I've heard that there's ---
\item
  chana joffe-walt\\
  Scanning, like metal detectors?
\item
  jillian juman\\
  Right. I heard there's fights and those kinds of things. I don't know
  what school you're talking about. I have never heard of that incident
  ever happening, ever. So the fears of what this building is and what
  this building has represented has transcended itself. There's a
  different story of International Studies outside this building.
\item
  chana joffe-walt\\
  How much of that do you think is racism?
\item
  jillian juman\\
  I think our entire society is fearful of the unknown.
\end{itemize}

chana joffe-walt

Excellent principal answer. Principal Juman is black, by the way. She
needed these parents. Schools get money per student. A shrinking school
means a shrinking budget. Ms. Juman was worried if this continued, the
middle school could be in danger of being shut down by the city. SIS is
in Cobble Hill, Brooklyn. Leafy streets, brownstones, it's a wealthy,
white neighborhood that's gotten wealthier and whiter in the last
decade. But white families were not sending their kids to SIS. Ms. Juman
told these parents, choose SIS. We're turning things around. We're in
the process of bringing in a new, prestigious International
Baccalaureate curriculum, renovating the library. Here's the new,
gorgeous yard. It's an excellent school. The parents seemed interested,
but I believe that might have had just as much to do with what was
happening outside of the school as what they were seeing inside the
building.

\begin{itemize}
\tightlist
\item
  rob hansen\\
  Sure, so my name is Rob Hansen, and I'm a parent. So we were --- the
  middle school process is interesting.
\end{itemize}

chana joffe-walt

Rob lives nearby SAS, but he had never heard of the school.

\begin{itemize}
\tightlist
\item
  rob hansen\\
  You know, probably eight school ---
\end{itemize}

chana joffe-walt

In his district, Rob could choose from 11 middle schools. The majority
of white families sent their kids to the same three schools. Rob's
white. Those were the schools he'd heard of, and those were the schools
he toured.

\begin{itemize}
\tightlist
\item
  rob hansen\\
  And it also had space.
\end{itemize}

chana joffe-walt

But they were packed. There were too many wealthy, white families in the
district to continue cramming into just three schools.

\begin{itemize}
\tightlist
\item
  rob hansen\\
  There's a couple of citywide ones where we went and we stood on line
  for like an hour, an hour and a half, and then joined in an auditorium
  full of parents, and then had them announce that they were accepting
  15 students in the following, in the coming class.
\end{itemize}

chana joffe-walt

And they'd been running tours all day. Most cities have some amount of
school choice like this, tours and options. New York City, though, is an
amped-up version of what happens elsewhere. The level of competition,
the level of wealth, the diversity of people sorting into different
schools, everything is more intense. Rob found this process frustrating,
although Rob is very even-tempered even when he's frustrated. He's
Canadian. When he gets especially hot, he starts calling things
``interesting.'' And this whole middle school thing was very
interesting. He asked other parents on school tours, what are we going
to do? Someone said, have you guys heard of SIS, that building down the
block? Rob hadn't. The others hadn't. They decided to all go check it
out together.

\begin{itemize}
\tightlist
\item
  rob hansen\\
  I walked away, and lots of parents walked away from those tours
  thinking, wow, you know, people are jamming up into some schools, and
  you're leaving 60 or 70 seats empty, empty all year long. If you have
  30 kids, it doesn't --- that's, you spread them out around, and that's
  a big school, then all of a sudden, you're sort of like, wait a
  second. What's --- there's nobody here.
\end{itemize}

{[}music{]}

chana joffe-walt

As Rob toured SIS, he had an idea. That night, he emailed Principal
Juman, and he asked, would she be open to starting a dual-language
French program at SIS? They had one at the elementary school Rob's kids
went to, and everyone loved it. Sure, Principal Juman was open. So Rob
started spreading the word. SIS is starting a dual-language French
program. We should all go. Rob says there was interest, but a lot of
people he talked to had this question. Wait, are other people going?

\begin{itemize}
\item
  rob hansen\\
  And families have that kind of fear. Like, what if I'm --- if I look
  around, nobody else came with me. And I came for something that's not
  here, because nobody --- so it's a collective action problem.
\item
  chana joffe-walt\\
  Wait, why is it a collective action? Why do you need a collective ---
\item
  rob hansen\\
  Well, just on the --- just, I think, overall, there is a collective
  action issue. But if you're interested in this, in part, because of
  the French dual-language part of it, if you're the only one to show
  up, there's no French teacher for one student. But there's a program
  if 15 come, if 20 come. But we all have to, then, take one step
  forward at exactly the same time. The vision requires people to come.
  And what if nobody comes?
\end{itemize}

chana joffe-walt

When it came time to choose middle schools, parents are supposed to rank
their top choices. Right before they did, before everyone chose their
schools, Rob sent a survey out to the families he'd been talking to try
to ensure that a group of them would choose SIS together. It was a
simple SurveyMonkey. If enough people said Yes, they'd rank SIS as one
of their top picks, and they would be able to act as a collective.
People said Yes. The numbers were stunning. In 2014, there had been 30
sixth graders at SIS. In 2015, there would be 103. That 200\% increase
was almost entirely white kids.

\begin{itemize}
\item
  chana joffe-walt\\
  Did you think about yourself as integrators? Or did you think about
  ---
\item
  rob hansen\\
  No. {[}LAUGHS{]} My pause was because I was trying to think if that
  had gone through my mind. And no, no. The --- not integrators.
  Participants in a school that was going to, hopefully, be diverse, but
  yeah, not --- that's not a framing or a way of thinking about it that
  would have occurred to me at the time. {[}LAUGHS{]}
\end{itemize}

chana joffe-walt

Nobody I talked to from SIS characterized what was happening there as
``integration.'' But here's why integration was on my mind. The New York
City Department of Education was aware their schools were segregated. It
was also aware that desegregation is the most effective way to close the
gap in achievement between black and white students. But it did not want
to mandate racial integration through zoning or school placements. The
city was trying to make integration happen through choice, hoping to
lure white families into segregated schools. The school tours I went on
for my own kids, the sparkly programs and amenities, that was the new
approach to integration. But can this work? For white parents to opt in
to integration not because we have to, or because it's the right thing
to do, but because it's a selling point? Because we get a dance studio,
and STEM, and a school that was, hopefully, diverse? Integration,
without talking about race. {[}CHILDREN CHATTERING{]} The kids at SIS,
though, they did talk about race --- immediately. Fall 2015, the first
few weeks of school, a senior named Kristen leans over to her classmate,
Chris, and mumbles, there are a lot of white kids in the school. And
Chris says, oh, yeah, a teacher warned me about that over the summer.

\begin{itemize}
\tightlist
\item
  chris\\
  Like, he told me, like, oh, there's going to be a lot of white kids
  coming in, French white kids from upper economic statuses. So be
  prepared for that.
\end{itemize}

chana joffe-walt

Kristen nods. Yeah, I guess we were prepared. And then she turns to me
to say, I should have been ready for that. We saw the parents on the
tours last year.

\begin{itemize}
\tightlist
\item
  kristen\\
  It was like we would see them walk to the hall, but we never knew it
  was so serious that a whole group of Caucasians would come in, like it
  would be so diverse. But mm-hmm, it's such a big change. Like, not to
  be prejudiced or anything, but I noticed the big change. High
  schoolers are more Hispanics and Blacks, and with the few Caucasians,
  and then, the new group that came in were all Caucasians. They had
  tried to make it so diverse.
\end{itemize}

chana joffe-walt

``Diverse.'' This was a word I heard over and over in the first few
weeks of school --- ``diversity.''

\begin{itemize}
\tightlist
\item
  chris\\
  I love diversity. So it doesn't --- so when I did see other white
  kids, I'm like, so?
\end{itemize}

chana joffe-walt

``Diversity'' seemed to have two different definitions. White families
would talk about all the diversity at SIS, and they were talking about
Black and Hispanic kids. When kids of color noted the diversity, they
were referring to the new white kids. For a lot of kids of color, this
looked a lot like something they'd already seen happen in their
neighborhoods --- white families showing up in large numbers, taking
over stores, familiar spots. There's a word for that. It's
gentrification. But I noticed that no one was using that word about the
school. What was happening here was ``diversity.'' That's how the adults
talked about it. Diversity is a good thing, something you're supposed to
be OK with. For the most part, the kids were. It was different for the
parents. Some of them saw specific advantages to the diversity, like
Kenya Blount, the co-vice president of the PTA at SIS. He was excited.

\begin{itemize}
\tightlist
\item
  kenya blount\\
  Having the new parents coming in and the diversity that, in
  particular, maybe comes from the new --- as I'll call it --- the
  ``new'' neighborhood, the way that things are changing in the
  neighborhood, is that we have a gentleman who his profession is
  fundraising.
\end{itemize}

chana joffe-walt

Rob Hansen, the dad who started the SurveyMonkey. Rob raises money for
nonprofits and foundations for a living. Over the course of the year,
I'll here Rob Hanson referred to as Todd Hanson, Ted Manson, Mr.
Handsome. Kenya was the only one who went with ``the gentlemen whose
profession is fundraising.'' The most common was just ``the guy who gets
the money.'' Rob told the PTA he was eager to raise money for the
school. To Kenya, this meant more resources at his own kids' school. His
boys and all the kids could benefit.

\begin{itemize}
\item
  kenya blount\\
  He has brought on the challenge and taken it upon himself to raise
  \$50,000. So ---
\item
  chana joffe-walt\\
  5-0?
\item
  kenya blount\\
  5-0, with three zeros after that, yes --- \$50,000. Which, again, this
  again goes back to the whole, I'll say, diversity thing and new people
  who were thinking outside the box. As our PTA, I don't think that we
  were thinking that big.
\end{itemize}

chana joffe-walt

They were definitely not thinking that big. Because the PTA was run by
Imee Hernandez and her co-president, Susan Moesker. Imee is not a
gentleman who fundraises. She's a social worker. The first time I met
Imee, she was wearing a T-shirt that said, ``I'm not spoiled. My husband
just loves me.'' She's Puerto Rican, grew up in Brooklyn. Her husband
Maurice is Puerto Rican and Black and really does adore her. He grew up
in Brooklyn, too. They have one daughter, one pit bull, one Persian cat,
and one school.

\begin{itemize}
\tightlist
\item
  imee hernandez\\
  I make it my business to stick myself in her school. {[}LAUGHS{]}
\end{itemize}

chana joffe-walt

For Imee, the new diversity, it gave her pause.

\begin{itemize}
\item
  imee hernandez\\
  Like, what I saw in September the population that came in, I was like,
  oh, that's a little frightening. {[}LAUGHS{]} And even the socio ---
\item
  chana joffe-walt\\
  If you could describe it for people who are on the radio and don't
  know what you saw.
\item
  imee hernandez\\
  I saw a lot of white people with very high socioeconomic backgrounds.
  You know, they have money. And that's great, but money tends to scare
  people. And I'm one of the people it scares. {[}LAUGHS{]} I'm one of
  the people it scares, because it twists everything around, and I don't
  like that. I don't like that. I don't like that --- I'd rather have a
  dinner where people of different cultures bring their food and we
  share together than have somebody else cater it. Like, that's how I
  feel you build community. I'm a social worker. That's my background,
  and that's what I believe in.
\end{itemize}

chana joffe-walt

Imee was in her second year at the school. The year before, she put on
community events, teacher appreciation, a spring carnival with face
painting and hot dogs. They raised some money here and there, but Imee's
vision for the PTA wasn't really about fundraising. The new parents,
though, they wanted to be active in their new school, and they were
accustomed to supporting their kids' schools by fundraising. The two
approaches came face-to-face at a PTA meeting in October.

\begin{itemize}
\item
  imee hernandez\\
  Three more minutes.
\item
  speaker 1\\
  All right, all right.
\item
  imee hernandez\\
  And then it's up to everyone ---
\item
  speaker 1\\
  Yes, I'm very ---
\item
  imee hernandez\\
  Because y'all got your --- you got to go home.
\item
  speaker 1\\
  I got my ---
\item
  imee hernandez\\
  {[}LAUGHS{]} You got to go home.
\end{itemize}

chana joffe-walt

There are about a dozen grownups, sitting on small plastic chairs around
a classroom table, the PTA Executive Board. Principal Juman is here,
too. Imee's leading, and the principal jumps in. She says she wants a
minute to share how much the new fundraising committee had raised so
far. Imee looks confused. Principal Juman goes on to say, the new
fundraising committee has had a lot of success.

\begin{itemize}
\item
  jillian juman\\
  The total they have raised, according to Rob, about \$18,000.
\item
  imee hernandez\\
  Mm-hmm, OK.
\item
  jillian juman\\
  And then, we just had a donation from a family a couple weeks ago who
  wanted to be anonymous that they're going to give either 5 to 10 grand
  in December. So this is big money.
\end{itemize}

chana joffe-walt

People seem unclear what to do with their faces. This is good news,
right? But also, wait, what's the Fundraising Committee? Imee turns to
her husband, Maurice. A retired cop, Maurice is also the treasurer of
the PTA, because when he retired, his wife told him he couldn't just sit
around at home. Maurice shrugs at Imee, doesn't seem to know anything
about this new money. Imee turns back to Principal Juman. So can we use
that money?

\begin{itemize}
\item
  imee hernandez\\
  --- answered my question.
\item
  jillian juman\\
  Yeah.
\item
  imee hernandez\\
  That was the question, if the PTA can have access to this money.
  Because I know already ---
\item
  jillian juman\\
  But what is the PTA? So that's all part off the question that's going
  around.
\item
  imee hernandez\\
  Right, yeah.
\item
  jillian juman\\
  So this \$18,000 Rob has raised under the umbrella of PTA.
\end{itemize}

chana joffe-walt

That's principal Juman.

\begin{itemize}
\item
  imee hernandez\\
  OK.
\item
  jillian juman\\
  So --- I think.
\item
  imee hernandez\\
  But who's --- who's got it? And where's it going? Like, yeah.
\item
  speaker\\
  This PTA member don't know nothing about it, so you know? {[}LAUGHS{]}
\item
  maurice\\
  How can that be access for Mr. Negrone.
\end{itemize}

chana joffe-walt

Maurice asks, how can that money be accessed for Mr. Negrone, who wants
new gym uniforms, or Mr. Lowe, to get his microscopes? Imee nods

\begin{itemize}
\item
  imee hernandez\\
  I mean, God bless Rob, and more power to him.
\item
  jillian juman\\
  Yeah.
\item
  imee hernandez\\
  But he's not an official member.
\item
  jillian juman\\
  Right.
\item
  imee hernandez\\
  So I think that's what makes it confusing, at least for me. You know,
  he is a PTA member because he's a parent, but he's not part of the
  executive board. So I think that's what makes it ---
\item
  jillian juman\\
  That's probably true.
\item
  imee hernandez\\
  Yeah, it makes it tricky.
\item
  jillian juman\\
  Right.
\item
  imee hernandez\\
  I mean, and again, I'm not {[}INAUDIBLE{]}.
\item
  speaker\\
  And then, I mean, I'm not going to lie ---
\item
  imee hernandez\\
  I he could bring in the money, that's great, but you know ---
\end{itemize}

chana joffe-walt

Principal Juman nods, repeats that she wishes Rob had been able to make
it. She was hoping everyone could be here and get on the same page about
money. But Rob is chaperoning a sixth grade overnight trip. They're late
getting back. One mom, a white woman, who came in with a new group of
sixth graders, says, look, I know Rob. He means well.

\begin{itemize}
\item
  speaker 1\\
  I think Rob, he's a professional fundraiser.
\item
  speaker 2\\
  Yeah, he's great.
\item
  speaker 1\\
  And therefore, he took it as his initiative ---
\item
  speaker 3\\
  They need money.
\item
  speaker 1\\
  --- to do the fundraising.
\item
  speaker 3\\
  Yep.
\item
  speaker 1\\
  And I think that's great. But I don't --- he should communicate with
  the PTA. And my impression is, I don't think he's meaning to offend
  anybody.
\item
  all\\
  No.
\item
  speaker 1\\
  I think he's sort of so laser-focused. That he's not thinking about,
  like, well, maybe you might want to let somebody know what he's doing.
\item
  jillian juman\\
  And he's been amazing. He really has. Yeah, yeah.
\end{itemize}

chana joffe-walt

That's principal Juman. At this point, everyone seems to feel a little
weird about how long they've spent talking about a fellow parent who is
not present. And anyway, it's money for the school. We're all for that.
We just need better communication. Imee she says, yeah, it's just
usually, money raised by parents goes through the PTA, so we can all
talk about where to spend it.

\begin{itemize}
\item
  speaker 1\\
  Yeah, OK.
\item
  speaker 2\\
  Because then we have to decide who has the say, because if it's a
  collective PTA ---
\item
  speaker 1\\
  Hey!
\item
  speaker 3\\
  There's Rob.
\item
  speaker 1\\
  There he is.
\item
  speaker 4\\
  There he is!
\item
  rob hansen\\
  Sorry I'm late.
\item
  imee hernandez\\
  I can't believe you made, after everything ---
\end{itemize}

chana joffe-walt

Rob walks into the room. He just got back from the sixth grade trip. He
sits down, and they all start to talk. We need to sort out some
questions about money. Then, a mom from the fundraising committee says
she's worried about me recording and asks me to stop. So I do.

\begin{itemize}
\tightlist
\item
  maurice\\
  It's been requested, but we don't know ---
\end{itemize}

chana joffe-walt

They let me stay, though, and take notes. Rob apologizes and then
explains. A group of them have been meeting to raise money for the
school. The new dual-language French program is expensive, and they
promised the principal they'd help raise money to cover it. They were
just eager to help, Rob says, so they formed a committee. He's really
sorry. He should have communicated and coordinated better with the PTA.
But good news is, it's going great. Someone has a contact with the
French embassy, a guy at the Cultural Services Arm in New York, and he
says he wants to help cover the costs of new French teachers and books.
They've already kicked in around 10 grand. At this point in my notes, I
wrote, ``lots of looks, big money.'' Rob says, ``The embassy suggested
we do a fundraiser, an event. They can help.'' Here I wrote, ``Looks ---
confused, mad? Nobody really talking.'' Imee says, ``This fundraiser
will be at the school though, right? And free, for everyone?'' Rob says,
``Yes.'' ``Good, good.'' She asks one more time, ``Free? I just want to
make sure everyone can go.'' Lots of nods. Rob says, ``Totally. This is
a community event, for our community.'' After about 20 minutes, Imee
says, we're out of time, guys. I can't tell if this is out of a
professional commitment Imee has to stick to the schedule, or a personal
commitment to getting out of that room.

{[}music{]}

chana joffe-walt

Before I came to SIS, I never thought much about the role of PTAs ---
ever. At SIS, early on, I have this feeling of, oh, a PTA is actually
critical to the success of an integrated school. A PTA has a very simple
democratic structure. Every parent has an equal vote. Smart --- it's
like a built-in system to equalize power, to help them make a budget
together, make decisions, set priorities collectively --- or not.

\begin{itemize}
\tightlist
\item
  imee hernandez\\
  So we're lucky enough that we have Rob here, who has really taken over
  fundraising and tried to bring it to the next level here at our
  school. So ---
\end{itemize}

chana joffe-walt

It's another PTA meeting, and the whole collective thing is not really
happening. It seems like the new parents are still raising money
separate from the PTA, and the communication problems do not seem to be
resolved. And some of the new parents have an idea. They propose a
formal separation --- the PTA and the people doing fundraising. Rob says
this way, there'll be two organizations collecting money for SIS.

\begin{itemize}
\tightlist
\item
  rob hansen\\
  --- is there will be two sorts of ways dollars are raised. One would
  be a community raise --- bake sales, direct gifts.
\end{itemize}

chana joffe-walt

That would be the PTA side, the community fund. Then there would be a
separate organization that would go after grants and big donors. Up
until this point, there seemed to be tension bubbling under the surface
between the new parents and the old parents, but it wasn't really until
this moment that the unsaid started to get said --- mostly by Imee's
husband, Maurice.

\begin{itemize}
\item
  maurice\\
  I think a lot of us feel that there's two different groups. There's
  the fundraising group and the PTA, which is --- That's what it looks
  like. You guys have this goal of making \$50,000 that is going to the
  French program. Now, as you said, what about the rest of your school?
  Where's all this money going? We have no answer. We don't know.
\item
  maureen\\
  And it's very easy to feel steamrolled.
\end{itemize}

chana joffe-walt

That's Maureen, a white mom who's new. There are lots of nods.

\begin{itemize}
\tightlist
\item
  maureen\\
  And I just don't want ---
\end{itemize}

chana joffe-walt

Maurice is asking, is this new money you're talking about, is it just
for the new French dual-language program? Which is another way of
asking, is this money just for your kids, or is it for everybody? Rob
says emphatically, it's for everybody. Maurice says, really?

\begin{itemize}
\tightlist
\item
  maurice\\
  I mean, that's being naive. We think, OK, they're going to donate all
  this money through the French embassy, and we're going to, OK, well,
  we're going to buy new chalkboards. That's kind of --- that's being
  naive. Now, you're saying the 50,000 will be for the PTA community to
  decide where it's going to go. So I mean, I hear what you're saying,
  which sounds great. But again, maybe I'm still talking about last
  meeting, when Jillian said, OK, well, are we only get a percentage of
  that? So we still don't have an answer.
\end{itemize}

chana joffe-walt

Later, talking to Rob, I learned that the new separate fundraising arm
he's talking about is actually a foundation. They want to create a
school-based foundation at SIS. The plan is to call it the Brooklyn
World Project. I asked Rob, why do you need another way to raise money?

\begin{itemize}
\item
  chana joffe-walt\\
  There's a PTA. Most people have heard of a school PTA. Why do you need
  a separate organization that's not the PTA?
\item
  rob hansen\\
  Yeah, so probably the easiest way to explain it is to not think about
  it from the school side, but to think about it from the potential
  donor side. So basic idea that we're following is that the --- let's
  say the international says, we want to do extend today, and we want to
  do theater. And so we go and we find a donor who loves theater, and
  loves the French language, and loves the idea that kids who've never
  spoken French and had no exposure get the chance to go and compete,
  actually, against some of the most established schools in the city.
  And a donor just loves that. Like, I love it. I love giving that kind
  of opportunity to kids. I'm going to cover all of that, because I
  think it's that important. If that money goes to the PTA, you could
  have a situation that the PTA says, or members that PTA say, I don't
  know that we really like the theater program. I'm not sure I think
  that we should be using those dollars to do x, or y, or z. Now,
  normally, you'd be able to say, well, donor intent is what it is. You
  should probably use it towards what it was intended for.
\item
  chana joffe-walt\\
  You mean normally, in another fundraising context?
\item
  jillian juman\\
  Yeah, meaning in nonprofits. So there's a basic kind of morality of a
  nonprofit to say, if a donor gives you it to you to do something, you
  should try to do that. Donor intent is an important part of it. It's
  sort of a trust that's established.
\end{itemize}

chana joffe-walt

Rob says, because the PTA is a democracy, it makes things complicated.
The very thing I saw as a strength of a PTA --- one parent, one vote ---
to Rob, that's a problem for fundraising. Parents come and go and change
their minds about what's important. A private donor wants stability, and
Rob is trying to raise money for the kind of programming that was
available at his son's wealthy elementary school. At that school, Rob
was co-president of the PTA, and the previous year, his PTA pulled in
close to \$800,000. \$800,000 --- money that paid for after school
programming, and ballroom dancing, chess, art, music, a garden.
\$800,000 for a school that is 75\% white and serves a tiny fraction of
the poor kids in the district. There aren't enough wealthy parents at
SIS to raise that kind of money. That year, Rob helped raise \$800,000,
the SIS PTA raised \$2,000. So Rob was trying to be creative. A
foundation was a way for his new school to catch up. The school
leadership, the principal, was behind the idea. . Ms. Juman told me she
saw the foundation as a path to equity and access. More resources meant
they'd be able to provide all kids with opportunities --- like, say, a
school trip to France. But the parent leadership, they found it
annoying. Imee knew the new parents were trying to help the school. But
she already liked the school. She felt like she was being saved against
her will. Plus, they're new, she said. Shouldn't we be the ones helping
them? She was fine with them bringing in ideas, but she didn't
understand why they hadn't brought them to her first. They hadn't
thought to consult her. She to me, multiple times, why are they coming
up with all these private plans and meeting in secret committees?

\begin{itemize}
\item
  jillian juman\\
  You were pissed about that.
\item
  imee hernandez\\
  Totally.
\item
  chana joffe-walt\\
  Yeah.
\item
  imee hernandez\\
  Because I wasn't involved.
\item
  chana joffe-walt\\
  I mean, why were you angry about that?
\item
  imee hernandez\\
  Because here I am trying to build something with the school. Why
  didn't you just involve me? Why didn't you just tell me about it?
  Like, it felt like it was a secret. I don't know if it was or if it
  wasn't. I'm invested in the school. Clearly, I've proven to you I'm
  invested in the school. And you couldn't tell us that you wanted to
  fundraise in a different way?
\end{itemize}

chana joffe-walt

Rob and the new parents did tell the principal that they wanted to
fundraise in a different way, but Imee felt like, what about the rest of
us? She felt like the PTA was ignored. At that last meeting, Imee went
quiet. She told me she just felt enraged, and then embarrassed for
feeling so enraged.

\begin{itemize}
\item
  imee hernandez\\
  I guess I just threw a tantrum. {[}LAUGHS{]} And I just didn't want to
  be a part of it. Which is not right, but I think, again, in the
  moment, I just felt like --- I was hurt.
\item
  chana joffe-walt\\
  Do you usually throw --- was the tantrum the thing I saw? Because that
  did not seem like a tantrum.
\item
  imee hernandez\\
  No, that was not a tantrum. I could have been a lot worse, and I was
  really, really trying to restrain myself. Yeah, I really was. That was
  really under control --- really, really under control. It wasn't, but
  it was really, really under control.
\item
  chana joffe-walt\\
  I asked, was there another time?
\item
  imee hernandez\\
  Tantrum? Yeah, at home with my husband. {[}LAUGHS{]} That's when I
  threw my tantrum.
\end{itemize}

chana joffe-walt

So it was tense, among the parents. But this is a school for children.
Did it matter if the adults were not getting along, or who controlled
which pot of money? Yes. Yes it did. That's coming up after the break.
The school year went on. Rob's fundraising committee moved forward with
the French Embassy to plan a fundraiser. It was now being called a
``gala.'' The PTA moved forward with parent volunteers to plan a spring
carnival. It was being called ``the spring carnival.'' Quiet resentments
locked in place. On the phone one night, Imee's co-president on the PTA,
Susan Moesker, told me she worried the school was changing in ways that
were damaging to the community. Susan is white herself, but she didn't
come in with the new white parents. When she started, her son was one of
the only white kids in the school. And now she felt like they were all
being written into a narrative that wasn't true, that SIS was a bad
school before, and now that the new white families had arrived, it was
being turned around.

\begin{itemize}
\item
  susan moesker\\
  It is noticeable. I think it is something that even my child has
  picked up on --- just like a very different feeling among some of the
  students and some of the parents, this real sense again that here they
  come to save our poor struggling school that couldn't possibly make it
  on its own without their money and their vision. And we do not all
  feel that that is, necessarily, the case.
\item
  chana joffe-walt\\
  What do you feel?
\item
  susan moesker\\
  {[}SIGHS{]} {[}LAUGHS{]} Ah ---
\end{itemize}

chana joffe-walt

This was a long conversation. The upshot? She's not happy with the way
the new parents are behaving. It was true. A new narrative was taking
hold at SIS. It's not like the kids were talking about it all the time,
but it was in the air, and the kids were starting to pick up on who was
valued and why. In the cafeteria, I'd hear middle schoolers saying, the
French kids could kill someone, and they'd get away with it. Upstairs in
the high school, I'd hear kids complain, all the attention has shifted
to the new middle schoolers. We're being pushed aside. And down in the
library, I met three sixth grade boys, white boys new to SIS. They're
sweaty from playing soccer and looking very small against their huge
backpacks. These boys, even at 11 years old, they've absorbed the same
messages --- that SIS wasn't so good before. It was a bad school.

\begin{itemize}
\tightlist
\item
  boy\\
  The kids wouldn't pay attention, and they had all the --- zone out
  every little thing. And I bet they learned very little. And now, this
  generation, with us, I think we're doing a lot better, and I think
  that we're learning at a much faster pace.
\end{itemize}

chana joffe-walt

He and his friends, they've turned the school around. That's what he's
learning.

\begin{itemize}
\tightlist
\item
  boy\\
  It's going to be one of the top choices. Already, in the Brooklyn,
  when you're applying to middle schools, you get a book on statuses and
  stuff. And I think this school is actually really high up in the
  statuses.
\end{itemize}

chana joffe-walt

Nobody calls it ``the book on statuses.'' They call it a directory of
schools, with info like enrollment numbers for different schools, test
scores, and special programs. But I love the calls it, ``the book on
statuses,'' because this is what happened at SIS. The school had a bad
reputation among white families, and then, suddenly, it was in demand.
Its status had changed because of the white kids. A powerful draw for
white families into any school is other white families. Once you have a
critical mass of white kids, you pass what one city calls ``a bliss
point.'' This is a real thing researchers study --- how many white kids
are needed at a school to make other white families feel comfortable
choosing it. That number, the bliss point, is 26\%. That fall, white
families were crowding the school tours at SIS, not because the test
scores had improved --- the new scores hadn't even come out yet --- but
because the other white families made them feel blissfully comfortable.

{[}music{]}

chana joffe-walt

Of course, the thing that made the new white parents comfortable coming
to SIS in the first place was the promise of a French program. They
wanted French, and they got French. So now, all the sixth, seventh,
eighth, and ninth graders are learning French. It wasn't a true
dual-language program where kids learned in French for half the day or
whatever. That first year, most of the French was happening in the after
school program. You'd sign up for regular after school stuff like
culinary, or soccer, or drama, and it would be conducted in French.

\begin{itemize}
\item
  teacher\\
  Hey, hey! {[}KIDS CHATTERING IN FRENCH{]}

  Hey, everybody, you have to listen, OK?
\end{itemize}

chana joffe-walt

We're in the auditorium. And it's sweet. The kids are onstage rehearsing
this play they wrote in French. And it's like they're having fun. But I
couldn't help feeling like there's something off balance about this.
Most of the kids doing this drama program seem to be native French
speakers, but not all. A sixth grader named Maya is standing to the side
of the stage, script in hand, waiting for her line.

\begin{itemize}
\item
  imee hernandez\\
  For me, it's a bit weird, because I have no idea they're saying. Yeah.
\item
  chana joffe-walt\\
  Really? Even in the play that you've been practicing, you don't know
  what they're saying?
\item
  imee hernandez\\
  Yeah, I don't know what they're saying still.
\item
  maurice\\
  You have the translations of the script.
\item
  maya\\
  Yeah, but sometimes, when a teacher talks in French to the class, I
  don't understand.
\item
  chana joffe-walt\\
  And do you figure it out? Or is it confusing?
\item
  maya\\
  Confusing.
\end{itemize}

chana joffe-walt

Still, she's excited. She's grinning, watching the other kids on stage.
She's hanging out with her friend Constance. Maya gets up to deliver her
lines.

\begin{itemize}
\item
  maya\\
  {[}FRENCH{]}?
\item
  constance\\
  Oh you did a wrong line.
\item
  maya\\
  Yeah, what?
\item
  constance\\
  {[}FRENCH{]}.
\item
  maya\\
  That's confusing.
\item
  constance\\
  You just say {[}FRENCH{]}, and then {[}FRENCH{]} is {[}FRENCH{]}
  after.
\end{itemize}

chana joffe-walt

Constance, a native French speaker, tells Maya, you said the wrong
thing. Constance correct her, pronounces it for her.

\begin{itemize}
\item
  constance\\
  {[}FRENCH{]}.
\item
  maya\\
  {[}FRENCH{]}.
\item
  teacher\\
  But it's OK.
\item
  maya\\
  Yeah, because I can't do it that well.
\end{itemize}

chana joffe-walt

Maya says, I can't, and her friend says, I'll do it for you.

\begin{itemize}
\item
  constance\\
  OK, I'll just say it.
\item
  maya\\
  {[}FRENCH{]}.
\item
  constance\\
  {[}FRENCH{]}.
\end{itemize}

chana joffe-walt

Learning another language is not new to Maya.

\begin{itemize}
\item
  maya\\
  My dad speaks Arabic, and my mom's Turkish.
\item
  jillian juman\\
  Uh-huh, and now you're learning French.
\item
  maya\\
  Yes, it's so confusing. Three languages at the same time.
\end{itemize}

{[}music{]}

chana joffe-walt

When the new white parents asked for a dual-language French program at
SIS, Principal Juman said yes. SIS was supposedly an international
school, but she told me they didn't really have a lot of international
programming, so it seemed like a good idea to her. But there was no
school-wide debate about it, or consensus. The community didn't decide.
What if they had? More than 1/3 of the families at sis are Hispanic.
What if the dual-language program was Spanish, or Arabic? 10\% of the
students speak Arabic. If they had made a different choice, if SIS had a
dual-language Arabic program, Maya would be teaching Constance how to
read her lines. She'd be the one explaining the cultural references and
teasing her friend about her terrible accent. She'd be the one
translating the teacher's stage directions. There was money for a French
program, which meant that at SIS, French had value. Arabic didn't.
Spanish didn't. That's something Maya is learning at school, along with
her French script.

From the very beginning, Imee and the others had insisted on three
things from the new parents and the fundraising committee --- that the
gala fundraiser they were planning with the French Embassy, would,
number one, be open to everyone, number two, take place at the school,
and number three, be free. Then, four weeks before the gala, the PTA
asked for an update, and a parent named Deb showed up --- a mom to a new
sixth grader, part of Rob's fundraising committee.

\begin{itemize}
\item
  deb\\
  So I will start with the fact that I had a nice conversation with ---
\item
  fabrice\\
  Fabrice.
\item
  deb\\
  Fabrice. Is it Fabrice?
\item
  fabrice\\
  Fabrice, yeah.
\end{itemize}

chana joffe-walt

Deb volunteered early on to help organize the party and she tells
everyone, I met with our partner, Fabrice, at the French Embassy, and
the event can't be at the school. The embassy won't be able to draw
their supporters to Brooklyn. It'll be at the Cultural Services Building
on the Upper East Side, Manhattan, 45 minutes away.

\begin{itemize}
\item
  deb\\
  I apologize if I'm saying things you guys already know, but I didn't
  know some of this info, so it was good. But the event is really ---
  it's their event. It's not really our event.
\item
  susan moesker\\
  Oh!
\item
  deb\\
  It's their event.
\end{itemize}

chana joffe-walt

That's Susan with the ``Oh.'' Maurice leans forward, elbows on the
table. Imee is not here. She knew the meeting would be almost entirely
about fundraising, and she's sitting this one out. Maurice is now
concentrating on Rob, who turns to Deb and says, ``In what sense is that
their event?'' ``They make the rules, she says.''

\begin{itemize}
\tightlist
\item
  deb\\
  With our input, but there are certain things that are not flexible.
  The biggest thing is, nobody will be allowed in at the door. You have
  to be on a list. You have to RSVP. You have to be on the list. All
  names.
\end{itemize}

chana joffe-walt

Security. It's a government building after all.

\begin{itemize}
\item
  deb\\
  He sends out the invitation to 22,000 people on his mailing list. So
  now, making it a free event is a problem, because now we're inviting
  22,000 people for free to drink wine and eat food that may not have
  any interest in us. So we thought the best thing to do would be a
  suggested donation. Can't afford to go?
\item
  rob hansen\\
  Can I give a variant on that?
\item
  deb\\
  Yeah.
\end{itemize}

chana joffe-walt

That's Rob, asking to give a variant, which is, how about we have a
separate invitation for our people that doesn't ask for any money? Rob
seems to be picking up on the instant irritation in this room, and he's
adding many variants to Deb's report.

\begin{itemize}
\item
  rob hansen\\
  It's either a modified version, or just a clarity that everybody in
  this community ---
\item
  deb\\
  He won't. There'll be one invite. It will say the same thing. That's
  what I suggested. I suggested \$50 a head on the outside.
\item
  rob hansen\\
  Even if we simply put a cover note saying, no charge. We want you to
  come join us, our community.
\item
  deb\\
  Right, but on the invite, it will say, ``suggested donation.'' Then,
  if you want to --- however we want to forward it, we can say that. But
  they will only do one invite.
\end{itemize}

chana joffe-walt

Deb hasn't been able to make previous PTA meetings. So all Deb knows is
she got an email from the fundraising committee at her kid's new school,
which she assumed was part of the PTA. She's volunteering her time, a
ton of her time, to organize a huge event. She does not understand that
the email list she's on is for a separate fundraising committee that
just became even more unpopular with the official PTA leadership. I
think I stopped moving watching Deb. It's so tense. She's like a
porcupine who's just wandered into a balloon store.

\begin{itemize}
\item
  deb\\
  They're serving wine, water, and then French hors d'oeuvres. And as
  far as the auction, we have a couple of cleanses. We have restaurants.
  We have a soccer camp. We have a vacation rental in California. We've
  got a couple hair salons. Very few from the community here, and that's
  really when I wanted to talk about.
\item
  speaker\\
  Is he from the parent community or geographic community?
\item
  deb\\
  Parent community and geographic community.
\end{itemize}

chana joffe-walt

Deb says at her kid's elementary school, they got a lot more donated
items from parents. She tells the room, you can ask at the restaurants
you go to if they do gift certificates. The salon, your employers ---
you'd be surprised what people can offer. Just ask.

\begin{itemize}
\tightlist
\item
  deb\\
  Then that's what people like my friends --- and most of my friends,
  though, they're all in other schools. I'm just new here. I don't
  really know many people. So the only people I've been able to reach
  out to are the 36 on Rob's email list. {[}LAUGHTER{]} And then, a
  quarter of them gave, have donated something already, like found
  something. So I'm telling you, that house in Sonoma County is gorgeous
  --- four bedrooms, three baths, beautiful.
\end{itemize}

chana joffe-walt

I think about a PTA meeting a few months before where I watched Imee
gently explain to one of the new parents why it might be hard for some
families to throw in \$5 for classroom supplies, that even being asked
to donate can feel alienating. Some people in this room seem to be
experiencing this whole thing as a routine update about public school
volunteering. Others look like someone who's walked into the wrong room
and is now looking around to the friends they came with for affirmation,
we're in the wrong room, right? How do we get out?

\begin{itemize}
\item
  deb\\
  Usually, I get more tickets to shows, games, things like --- I've
  gotten Broadway tickets, but I haven't gotten anything in the ticket
  arena. Knicks?
\item
  speaker\\
  I have a contact at the Knicks. I'm willing to reach out.
\item
  deb\\
  Yeah, they always go. Everybody wants to go to a game. There's always
  somebody. And they also make great Christmas gifts. And that's the
  other thing we're lacking, is actual items. We used to have a parent
  --- well, we still have the parent, but she's not in my school ---
  that worked at Tiffany. And we always had some beautiful Tiffany
  pieces, or a Coach bag, some products --- makes it look nice.
\end{itemize}

{[}music{]}

chana joffe-walt

I spent a small chunk of that meeting occupied by an admittedly
sentimental thought. Just looking around, the room was kind of
incredible. People with homes in Sonoma and people who live in public
housing, sitting together at a long wooden table in the library of a
public school that they all share. That never happens. And I didn't want
them to mess it up. But of course, they are. This is not something we
have a lot of practice in. New York City has one of the most segregated
school systems in the country. White parents here have very little
practice sharing public schools. Maybe this is all to be expected. White
parents will charge ahead, will sometimes be careless, secretive, or
entitled. In response, parents of color will sometimes be cautious, or
distrustful, defensive. These are well-established patterns, repeated
over generations. It's easier for us to continue operating on separate
tracks, because it's what we already know how to do.

The guy from the French embassy apparently has a mailing list of 22,000
people in the New York area. 300 people RSVPed to the gala for SIS. I
couldn't believe it. And I couldn't believe that one of them was Imee.

\begin{itemize}
\item
  susan moesker\\
  Hello. Good evening!
\item
  chana joffe-walt\\
  You guys look lovely.
\item
  imee hernandez\\
  Hi, how are you? {[}LAUGHS{]}
\end{itemize}

chana joffe-walt

Imee, Maurice, and Susan carpooled together to the Upper East Side. It's
winter. Central Park is across the street. It's cold. Imee told me she
decided she needed to be a grown-up and come. They got stuck in traffic,
so they're rushing up the sidewalk.

\begin{itemize}
\item
  imee hernandez\\
  We're not that late, are we, Susan?
\item
  susan moesker\\
  It's not serious.
\item
  imee hernandez\\
  No, we're not that late.
\end{itemize}

chana joffe-walt

The Cultural Services Building is ivy-covered with columns. The doors
are wrought iron. The entryway is marble.

{[}music{]}

\begin{itemize}
\tightlist
\item
  susan moesker\\
  All right, you guys all came in together? All right.
\end{itemize}

chana joffe-walt

A huge marble staircase winds up the side of the room. Later, I look up
the architectural style --- Italian renaissance, Palazzo style. It's a
palace. There are people milling, sampling 17 different cheeses. I don't
recognize anyone else from the school. Who are these people who have
chosen to come out on a weekday evening for a fundraising event for a
not-prominent or well-known-at-all public school in Brooklyn?

\begin{itemize}
\tightlist
\item
  man\\
  I'm not involved with the school, but my wife is involved with the ---
\end{itemize}

chana joffe-walt

I started asking people how they heard about the event.

\begin{itemize}
\item
  jillian juman\\
  And what brought you here tonight?
\item
  barbara\\
  Actually, an invitation by my wonderful French professor.
\end{itemize}

chana joffe-walt

A lady named Barbara tells me she's never heard of SIS, like most people
here, but she loves French, and she loves Paris, and it sounded like a
fun night with other people who do too. She goes to France every year.

\begin{itemize}
\tightlist
\item
  barbara\\
  October is my saison preferee. Actually, I found this October too
  warm, but I like it when it's a nice fall, crisp, and you wear your
  scarf, your foulard.
\end{itemize}

chana joffe-walt

I enjoy a person who likes to talk, where you can just get on the ride
and sit back. Barbara is definitely that kind of person.

\begin{itemize}
\item
  barbara\\
  And my apartment in Paris, it's sort of --- I'm confused sometimes. I
  say, am I in Gramercy Park, or am I in Saint-Germain-des-Pres? It's
  got a similar ambiance of being a neighborhood. It's great. Have you
  been?
\item
  chana joffe-walt\\
  I've never been.
\end{itemize}

barbara

Oh my god, she hasn't been to Paris. Barbara's looking around for her
French teacher to tell her the news. Barbara's teacher, it turns out,
heard about this evening the same way most people here did. She was
invited by this man, Fabrice.

\begin{itemize}
\tightlist
\item
  fabrice\\
  For the School for International Studies, we are hoping we will raise
  \$100,000 each year for the next seven years.
\end{itemize}

chana joffe-walt

Fabrice Germain works for the Cultural Services arm of the French
Embassy. He tells me he's fundraising for dual-language programs in
public schools because his mission is to promote French language and
culture. He called it ``soft power,'' which I was kind of surprised he
said out loud, since I associate that with something we do in developing
countries, not something you're allowed to do in American public
schools. After Fabrice and I talked, I walked into the main room and
immediately saw Maurice. Maurice was so skeptical of this whole embassy
thing, but there he is at a table, selling raffle tickets next to Imee,
cheerfully raising money for a program neither of them ever wanted at
their school.

\begin{itemize}
\tightlist
\item
  maurice\\
  We are raffling off two airline tickets to France. Warm blue ticket's
  going to win. It could be yours!
\end{itemize}

chana joffe-walt

Maurice, amiable as ever, is trying --- and mostly failing --- to
convert ticket sales into social connections. He asks everyone, so if
you win, when are you thinking of going? Oh, you're going anyway? For
Easter? Oh, nice.

\begin{itemize}
\item
  maurice\\
  How's your there in Easter?
\item
  speaker\\
  Yes, I am from France.
\item
  maurice\\
  How's the weather there in Easter?
\item
  speaker\\
  Good. Fantastic. I'm --- before we do it, I have a question. Can I
  from here, from this ticket, buy something to go from Paris to
  Marseille, when I'm leaving?
\item
  maurice\\
  No.
\end{itemize}

chana joffe-walt

Barbara, from Gramercy Park, the woman who loves fall in Paris, wanders
across the marble floor toward the raffle table, beside where Imee is
sitting. And I thought, oh, no.

\begin{itemize}
\item
  imee hernandez\\
  It's a pleasure to meet you.
\item
  barbara\\
  Hi, how are you?
\item
  imee hernandez\\
  Good, thank you.
\item
  barbara\\
  You're one of the parents of a bilingual student?
\item
  imee hernandez\\
  She's not bilingual, but she does go to the school. {[}LAUGHS{]}
\item
  barbara\\
  She will be bilingual, eventually.
\item
  imee hernandez\\
  Oh, she will be, yes, eventually.
\item
  barbara\\
  What a wonderful thing. Are you pleased with the program?
\item
  imee hernandez\\
  Yes, I love the school.
\item
  barbara\\
  It's so important to learn another language. It opens the world for
  you. And what is your name?
\item
  chana joffe-walt\\
  Chana.
\item
  barbara\\
  Anna. I was just telling Anna, when I go to Paris, which I do every
  year ---
\item
  imee hernandez\\
  Cool!
\item
  barbara\\
  It is cool. And it's cooler because I can speak the language. And you
  have entree into their society. Not totally; one will never have total
  entree. But you can interact with your neighbors. You can interact in
  a restaurant. You can interact at the dry cleaner, at the supermarket.
  And they so appreciate an American who can speak French.
\item
  imee hernandez\\
  Yes, yes.
\item
  barbara\\
  And the language is beautiful.
\end{itemize}

chana joffe-walt

Imee starts looking around. Maurice moves closer and leans in to hear
why his wife is doing that nervous laugh, as Barbara explains Imee, a
Puerto Rican woman, that being bilingual makes a person more
sophisticated. Imee is exceedingly polite.

\begin{itemize}
\item
  barbara\\
  Paris is a lodestar, and if you really want to enjoy it, you've got to
  speak the language.
\item
  imee hernandez\\
  {[}LAUGHS{]} It was a pleasure. Thank you.
\end{itemize}

chana joffe-walt

That entire conversation, Imee never mentioned to Barbara that she does
speak another language --- Spanish. Later, I ask her why. She shrugs it
off.

\begin{itemize}
\item
  imee hernandez\\
  {[}LAUGHS{]} I was like, no, I'll just let her talk. It's OK. It was
  all right. So ---
\item
  chana joffe-walt\\
  Do you remember when you were telling me about the silent tantrum that
  you're having ---
\item
  imee hernandez\\
  Yes.
\item
  chana joffe-walt\\
  How would I know if that was happening?
\item
  imee hernandez\\
  You wouldn't. Only my husband would. {[}LAUGHS{]} If I'm throwing a
  silent tantrum. He would know if I'm throwing a silent tantrum. Is
  that happening right now?
\item
  maurice\\
  No, not right now.
\end{itemize}

chana joffe-walt

Imee turns her back to her husband, facing me, and behind her, Maurice
is looking right at me, nodding vigorously --- yes.

So here we were in our fancy clothes on the Upper East Side of
Manhattan, raising money for a French program at an utterly normal
Brooklyn public school. That was already weird. But the toasts --- the
toasts were when the cognitive dissonance of the evening really kicked
in for me. Fabrice steps up onto the marble staircase and clinks his
glass, announces it's time to celebrate what we've created and raise
some money.

\begin{itemize}
\tightlist
\item
  fabrice\\
  It takes a village. It takes a dedicated principal. She's here with
  us. It takes --- {[}CHEERS{]} Yes, yes, yes.
\end{itemize}

chana joffe-walt

Fabrice hands the make to Principal Juman.

\begin{itemize}
\item
  jillian juman\\
  Hi!
\item
  all\\
  Hi
\item
  jillian juman\\
  It's so nice to see all of you. And I think the number one thing is I
  think all of us standing here believe in public education and believe
  that all students --- {[}APPLAUSE{]} Yes.
\end{itemize}

chana joffe-walt

One by one, people are talking about equality, and diversity, and
community, and the meaning of public education. Here, at the Cultural
Services Palatial Palace, full of white people.

\begin{itemize}
\tightlist
\item
  rob hansen\\
  So I went on the sixth great trip, 11-year-olds going on an overnight
  trip up into the Catskills. I would not recommend doing that.
\end{itemize}

chana joffe-walt

Rob gives a toast about that time he went on the overnight trip. It
starts off OK, but then veers into strange and sort of cringe-y
territory. He's on a ropes course, 40 feet up, looking down.

\begin{itemize}
\tightlist
\item
  rob hansen\\
  Below me was that diverse group of kids. They were diverse kids
  belaying me, making sure that when I jumped, they would actually
  cushion my fall. That day, each of those kids was going to climb up
  that pole and was going to have the same opportunity and the same
  challenge. And it made me think that that's what this school is about.
  It's about the opportunity to do the International Baccalaureate, the
  challenge of it, so the opportunity to explore French, and the
  challenge of it, for all kids. And ---
\end{itemize}

chana joffe-walt

I agree with Rob. It's great to give kids equal access to opportunity.
But what they're being given access to are the opportunities that Rob
and the other white parents care about.

Downstairs, I find Susan, Imee's PTA co-president, on a bench by
herself. She's near the band, drinking wine, looking a little
dumbstruck. I ask if she's OK. This is something else, she says. And
then she adds, it's just hard to explain how this is a public school
fundraiser.

When the founder of American public education, Horace Mann, laid out his
vision for public schools back in the day, he rode his horse around
Massachusetts, podium to podium. And his pitch was common schools would
make democracy possible. They would bind us to one another, indoctrinate
us, give us the skills and tools we need for democratic living. Public
schools, he believed, would be the great equalizer. Rich and poor would
come together and develop what he called ``fellow-feeling,'' and, in
doing so, quote, ``obliterate factitious distinctions in society.'' For
that to happen, you need everyone in the same school together. At SIS,
they've gotten that far. Everyone was in the same school together. But
there was no equalizing. We can be in the same school together and not
be equal, just like we can be in the same country together. It's not
enough. What do we do about that? After the gala, money poured into SIS,
and more white families enrolled their kids at the school. But in the
years after that, there was a backlash, and SIS changed in ways that
made Rob question himself. He wondered if he'd made mistakes. He told me
he thought they all wanted the same thing for the kids. He just didn't
know. Not knowing? That happens a lot with white parents. I looked into
the history of this school, and I learned that this wasn't the first
time white parents showed up here. White parents have been involved all
along, all the way back to the very beginning of this school half a
century ago, doing the same kinds of things I'd just seen. It happens
again and again, white parents wielding their power without even
noticing, like a guy wandering through a crowded store with a huge
backpack, knocking things over every time he turns.

Horace Mann believed public schools would make us equal, but it doesn't
work. I'm not sure how to fix that, but I want to lay out the story, the
whole story of this one American public school, because what I am sure
of is that in order to address inequality in our public schools, we are
going to need a shared sense of reality. At the very least, it's a place
to start. That's next time on ``Nice White Parents.''

{[}music{]}

chana joffe-walt

``Nice White Parents'' is produced by Julie Snyder and me, with editing
on this episode from Sarah Koenig, Nancy Updike and Ira Glass. Neil
Drumming is our Managing Editor. Eve Ewing and Rachel Lissy are our
editorial consultants. Fact-checking and research by Ben Phelan, with
additional research from Lilly Sullivan. Music supervision and mixing by
Stowe Nelson. Our Director of Operations is Seth Lind. Julie Whitaker is
our Digital Manager. Finance management by Cassie Howley, and Production
management by Frances Swanson. Original music for Nice White Parents is
by The Bad Plus, with additional music written and performed by Matt
McGinley and Dan Reitz. Special thanks to Whitney Dangerfield, Rich
Orris, Amy Pedulla, Nikole Hannah-Jones, Scott Sargrad, Jackie Carrier,
Gene Demby, Charles Jones, Lenny Garcia, and Valero Doval. At The New
York Times, thanks to Sam Dolnick, Stephanie Preiss, Nina Lassam and
Julia Simon.

``Nice White Parents'' is produced by Serial Productions, a New York
Times Company.

\href{https://www.nytimes.com/column/nice-white-parents}{\includegraphics{https://static01.nyt.com/images/2020/07/21/podcasts/nice-white-parents-album-art/nice-white-parents-album-art-square320.jpg}Nice
White Parents}

\hypertarget{episode-one-the-book-of-statuses-1}{%
\section{Episode One: The Book of
Statuses}\label{episode-one-the-book-of-statuses-1}}

\hypertarget{a-group-of-parents-takes-one-big-step-together-1}{%
\subsection{A group of parents takes one big step
together.}\label{a-group-of-parents-takes-one-big-step-together-1}}

Reported by Chana Joffe-Walt; produced by Julie Snyder; edited by Sarah
Koenig, Neil Drumming and Ira Glass; editorial consulting by Eve L.
Ewing and Rachel Lissy; and sound mix by Stowe Nelson

Transcript

transcript

Back to Nice White Parents

bars

0:00/01:02:23

-0:00

transcript

\hypertarget{episode-one-the-book-of-statuses-2}{%
\subsection{Episode One: The Book of
Statuses}\label{episode-one-the-book-of-statuses-2}}

\hypertarget{reported-by-chana-joffe-walt-produced-by-julie-snyder-edited-by-sarah-koenig-neil-drumming-and-ira-glass-editorial-consulting-by-eve-l-ewing-and-rachel-lissy-and-sound-mix-by-stowe-nelson-1}{%
\subsubsection{Reported by Chana Joffe-Walt; produced by Julie Snyder;
edited by Sarah Koenig, Neil Drumming and Ira Glass; editorial
consulting by Eve L. Ewing and Rachel Lissy; and sound mix by Stowe
Nelson}\label{reported-by-chana-joffe-walt-produced-by-julie-snyder-edited-by-sarah-koenig-neil-drumming-and-ira-glass-editorial-consulting-by-eve-l-ewing-and-rachel-lissy-and-sound-mix-by-stowe-nelson-1}}

\hypertarget{a-group-of-parents-takes-one-big-step-together-2}{%
\paragraph{A group of parents takes one big step
together.}\label{a-group-of-parents-takes-one-big-step-together-2}}

Thursday, July 30th, 2020

\begin{itemize}
\item
  ``Nice White Parents'' is brought to you by Serial Productions, a New
  York Times Company.
\item
  chana joffe-walt\\
  I started reporting this story at the very same moment as I was trying
  to figure out my own relationship to the subject of this story, white
  parents in New York City public schools. I was about to be one of
  them. When my kid was old enough, I started learning about my options.
  I had many. There was our zoned public school in Brooklyn, or I could
  apply to a handful of specialty programs --- a gifted program, or a
  magnet school, or a language program. So I started to look around.
  This was five years ago now, but I vividly remember these tours. I'd
  show up in the lobby of the school at the time listed on the website,
  look around, and notice that all or almost all of the other parents
  who'd shown up for the 11:00 AM, middle-of-the-workday,
  early-in-the-shopping-season school tour were other white parents. As
  a group, we'd walk the halls, following a school administrator ---
  almost always a man or woman of color --- through a school full of
  black and brown kids. We'd peer into classroom windows, watch the kids
  sit in a circle on the rug, ask questions about the lunch menu,
  homework policy, discipline. Some of us would take notes. And the
  administrators would sell. The whole thing was essentially a pitch. We
  offer STEM. We have a partnership with Lincoln Center. We have a dance
  studio. They were pleading with us to please take part in this public
  school. I don't think I've ever felt my own consumer power more
  viscerally than I did shopping for a public school as a white parent.
  We were entering schools that people like us had ignored for decades.
  They were not our places, but we were being invited to make them ours.
  The whole thing was made so much more awkward by the fact that nobody
  on those tours ever acknowledged the obvious racial difference, that
  roughly 100\% of the parents in this group did not match, say, 90\% of
  the kids in this building. I remember one time being guided into a
  classroom and being told that this was the class for gifted kids, and
  noticing, oh, here's where all the white kids are. Everyone on our
  tour saw this, all of us parents, but nobody said anything, including
  me. We walked out into the hallway. A mom raised her hand and said, I
  do have one question I've been meaning to ask. And the group got
  quiet. I was thinking, OK, here it comes. But then she said, do the
  kids here play outside every day?
\item
  {[}music{]}
\item
  chana joffe-walt\\
  I knew the schools were segregated. I shouldn't have been surprised.
  By the time I was touring schools as a parent, I had spent a fair
  amount of time in schools as a reporter. I'd done stories on the stark
  inequality in public education. And I'd looked at some of the many
  programs and reforms we've tried to fix our schools. So many ideas.
  We've tried standardized tests and charter schools. We've tried
  smaller classes, longer school days, stricter discipline, looser
  discipline, tracking, differentiation. We've decided the problem is
  teachers, the problem is parents. What is true about almost all of
  these reforms is that when we look for what's broken, for how our
  schools are failing, we focus on who they're failing --- poor kids,
  black kids, and brown kids. We ask, why aren't they performing better?
  Why aren't they achieving more? Those are not the right questions.
  There is a powerful force that is shaping our public schools, arguably
  the most powerful force. It's there even when we pretend not to notice
  it, like on that school tour. If you want to understand why our
  schools aren't better, that's where you have to look. You have to look
  at white parents. From Serial Productions, I'm Channa Joffe-Walt. This
  is ``Nice White Parents,'' a series about the 60-year relationship
  between white parents and the public school down the block.

  {[}CHILDREN CHATTERING{]} I'm going to take you inside a public school
  building, an utterly ordinary, squat, three-story New York City public
  school building not far from where I live. This isn't one of the
  schools I've toured. And my own kids don't go here. They're too
  little. This is a middle and high school called the School for
  International Studies, SIS. The story I want to tell you spans decades
  in this one school building, but I'm going to begin when I first
  encountered SIS, in the spring of 2015, right before everything
  changed. In 2015, the students at SIS were black, Latino, and Middle
  Eastern kids, mostly from working class and poor families. That year,
  like the year before and the year before that, the school was
  shrinking. The principal, Jillian Juman, was worried.
\item
  jillian juman\\
  Yeah, so the last two years, we had 30 students in our sixth grade
  class. And so we really have room for 100. And so numbers, I think,
  are hard.
\end{itemize}

chana joffe-walt

Ms. Juman started to reach out to families from the neighborhood,
inviting them to please come take a look. Parents started showing up for
tours of SIS, mostly groups of white parents. Ms. Juman was thrilled and
relieved. She walked parents through the building, saying, stop me
anytime if anyone has any questions. Really, anything; I want you to
feel comfortable. And Ms. Juman says they did have questions, mostly
about the poor test scores. That was fair. Ms. Juman expected those
questions. She did not expect the other set of questions she got a lot
from parents.

\begin{itemize}
\item
  jillian juman\\
  Is there weapons? Is there --- you know, are you scanning? Are you a
  scanning school, because kids are dangerous and they have weapons?
  I've heard that there's ---
\item
  chana joffe-walt\\
  Scanning, like metal detectors?
\item
  jillian juman\\
  Right. I heard there's fights and those kinds of things. I don't know
  what school you're talking about. I have never heard of that incident
  ever happening, ever. So the fears of what this building is and what
  this building has represented has transcended itself. There's a
  different story of International Studies outside this building.
\item
  chana joffe-walt\\
  How much of that do you think is racism?
\item
  jillian juman\\
  I think our entire society is fearful of the unknown.
\end{itemize}

chana joffe-walt

Excellent principal answer. Principal Juman is black, by the way. She
needed these parents. Schools get money per student. A shrinking school
means a shrinking budget. Ms. Juman was worried if this continued, the
middle school could be in danger of being shut down by the city. SIS is
in Cobble Hill, Brooklyn. Leafy streets, brownstones, it's a wealthy,
white neighborhood that's gotten wealthier and whiter in the last
decade. But white families were not sending their kids to SIS. Ms. Juman
told these parents, choose SIS. We're turning things around. We're in
the process of bringing in a new, prestigious International
Baccalaureate curriculum, renovating the library. Here's the new,
gorgeous yard. It's an excellent school. The parents seemed interested,
but I believe that might have had just as much to do with what was
happening outside of the school as what they were seeing inside the
building.

\begin{itemize}
\tightlist
\item
  rob hansen\\
  Sure, so my name is Rob Hansen, and I'm a parent. So we were --- the
  middle school process is interesting.
\end{itemize}

chana joffe-walt

Rob lives nearby SAS, but he had never heard of the school.

\begin{itemize}
\tightlist
\item
  rob hansen\\
  You know, probably eight school ---
\end{itemize}

chana joffe-walt

In his district, Rob could choose from 11 middle schools. The majority
of white families sent their kids to the same three schools. Rob's
white. Those were the schools he'd heard of, and those were the schools
he toured.

\begin{itemize}
\tightlist
\item
  rob hansen\\
  And it also had space.
\end{itemize}

chana joffe-walt

But they were packed. There were too many wealthy, white families in the
district to continue cramming into just three schools.

\begin{itemize}
\tightlist
\item
  rob hansen\\
  There's a couple of citywide ones where we went and we stood on line
  for like an hour, an hour and a half, and then joined in an auditorium
  full of parents, and then had them announce that they were accepting
  15 students in the following, in the coming class.
\end{itemize}

chana joffe-walt

And they'd been running tours all day. Most cities have some amount of
school choice like this, tours and options. New York City, though, is an
amped-up version of what happens elsewhere. The level of competition,
the level of wealth, the diversity of people sorting into different
schools, everything is more intense. Rob found this process frustrating,
although Rob is very even-tempered even when he's frustrated. He's
Canadian. When he gets especially hot, he starts calling things
``interesting.'' And this whole middle school thing was very
interesting. He asked other parents on school tours, what are we going
to do? Someone said, have you guys heard of SIS, that building down the
block? Rob hadn't. The others hadn't. They decided to all go check it
out together.

\begin{itemize}
\tightlist
\item
  rob hansen\\
  I walked away, and lots of parents walked away from those tours
  thinking, wow, you know, people are jamming up into some schools, and
  you're leaving 60 or 70 seats empty, empty all year long. If you have
  30 kids, it doesn't --- that's, you spread them out around, and that's
  a big school, then all of a sudden, you're sort of like, wait a
  second. What's --- there's nobody here.
\end{itemize}

{[}music{]}

chana joffe-walt

As Rob toured SIS, he had an idea. That night, he emailed Principal
Juman, and he asked, would she be open to starting a dual-language
French program at SIS? They had one at the elementary school Rob's kids
went to, and everyone loved it. Sure, Principal Juman was open. So Rob
started spreading the word. SIS is starting a dual-language French
program. We should all go. Rob says there was interest, but a lot of
people he talked to had this question. Wait, are other people going?

\begin{itemize}
\item
  rob hansen\\
  And families have that kind of fear. Like, what if I'm --- if I look
  around, nobody else came with me. And I came for something that's not
  here, because nobody --- so it's a collective action problem.
\item
  chana joffe-walt\\
  Wait, why is it a collective action? Why do you need a collective ---
\item
  rob hansen\\
  Well, just on the --- just, I think, overall, there is a collective
  action issue. But if you're interested in this, in part, because of
  the French dual-language part of it, if you're the only one to show
  up, there's no French teacher for one student. But there's a program
  if 15 come, if 20 come. But we all have to, then, take one step
  forward at exactly the same time. The vision requires people to come.
  And what if nobody comes?
\end{itemize}

chana joffe-walt

When it came time to choose middle schools, parents are supposed to rank
their top choices. Right before they did, before everyone chose their
schools, Rob sent a survey out to the families he'd been talking to try
to ensure that a group of them would choose SIS together. It was a
simple SurveyMonkey. If enough people said Yes, they'd rank SIS as one
of their top picks, and they would be able to act as a collective.
People said Yes. The numbers were stunning. In 2014, there had been 30
sixth graders at SIS. In 2015, there would be 103. That 200\% increase
was almost entirely white kids.

\begin{itemize}
\item
  chana joffe-walt\\
  Did you think about yourself as integrators? Or did you think about
  ---
\item
  rob hansen\\
  No. {[}LAUGHS{]} My pause was because I was trying to think if that
  had gone through my mind. And no, no. The --- not integrators.
  Participants in a school that was going to, hopefully, be diverse, but
  yeah, not --- that's not a framing or a way of thinking about it that
  would have occurred to me at the time. {[}LAUGHS{]}
\end{itemize}

chana joffe-walt

Nobody I talked to from SIS characterized what was happening there as
``integration.'' But here's why integration was on my mind. The New York
City Department of Education was aware their schools were segregated. It
was also aware that desegregation is the most effective way to close the
gap in achievement between black and white students. But it did not want
to mandate racial integration through zoning or school placements. The
city was trying to make integration happen through choice, hoping to
lure white families into segregated schools. The school tours I went on
for my own kids, the sparkly programs and amenities, that was the new
approach to integration. But can this work? For white parents to opt in
to integration not because we have to, or because it's the right thing
to do, but because it's a selling point? Because we get a dance studio,
and STEM, and a school that was, hopefully, diverse? Integration,
without talking about race. {[}CHILDREN CHATTERING{]} The kids at SIS,
though, they did talk about race --- immediately. Fall 2015, the first
few weeks of school, a senior named Kristen leans over to her classmate,
Chris, and mumbles, there are a lot of white kids in the school. And
Chris says, oh, yeah, a teacher warned me about that over the summer.

\begin{itemize}
\tightlist
\item
  chris\\
  Like, he told me, like, oh, there's going to be a lot of white kids
  coming in, French white kids from upper economic statuses. So be
  prepared for that.
\end{itemize}

chana joffe-walt

Kristen nods. Yeah, I guess we were prepared. And then she turns to me
to say, I should have been ready for that. We saw the parents on the
tours last year.

\begin{itemize}
\tightlist
\item
  kristen\\
  It was like we would see them walk to the hall, but we never knew it
  was so serious that a whole group of Caucasians would come in, like it
  would be so diverse. But mm-hmm, it's such a big change. Like, not to
  be prejudiced or anything, but I noticed the big change. High
  schoolers are more Hispanics and Blacks, and with the few Caucasians,
  and then, the new group that came in were all Caucasians. They had
  tried to make it so diverse.
\end{itemize}

chana joffe-walt

``Diverse.'' This was a word I heard over and over in the first few
weeks of school --- ``diversity.''

\begin{itemize}
\tightlist
\item
  chris\\
  I love diversity. So it doesn't --- so when I did see other white
  kids, I'm like, so?
\end{itemize}

chana joffe-walt

``Diversity'' seemed to have two different definitions. White families
would talk about all the diversity at SIS, and they were talking about
Black and Hispanic kids. When kids of color noted the diversity, they
were referring to the new white kids. For a lot of kids of color, this
looked a lot like something they'd already seen happen in their
neighborhoods --- white families showing up in large numbers, taking
over stores, familiar spots. There's a word for that. It's
gentrification. But I noticed that no one was using that word about the
school. What was happening here was ``diversity.'' That's how the adults
talked about it. Diversity is a good thing, something you're supposed to
be OK with. For the most part, the kids were. It was different for the
parents. Some of them saw specific advantages to the diversity, like
Kenya Blount, the co-vice president of the PTA at SIS. He was excited.

\begin{itemize}
\tightlist
\item
  kenya blount\\
  Having the new parents coming in and the diversity that, in
  particular, maybe comes from the new --- as I'll call it --- the
  ``new'' neighborhood, the way that things are changing in the
  neighborhood, is that we have a gentleman who his profession is
  fundraising.
\end{itemize}

chana joffe-walt

Rob Hansen, the dad who started the SurveyMonkey. Rob raises money for
nonprofits and foundations for a living. Over the course of the year,
I'll here Rob Hanson referred to as Todd Hanson, Ted Manson, Mr.
Handsome. Kenya was the only one who went with ``the gentlemen whose
profession is fundraising.'' The most common was just ``the guy who gets
the money.'' Rob told the PTA he was eager to raise money for the
school. To Kenya, this meant more resources at his own kids' school. His
boys and all the kids could benefit.

\begin{itemize}
\item
  kenya blount\\
  He has brought on the challenge and taken it upon himself to raise
  \$50,000. So ---
\item
  chana joffe-walt\\
  5-0?
\item
  kenya blount\\
  5-0, with three zeros after that, yes --- \$50,000. Which, again, this
  again goes back to the whole, I'll say, diversity thing and new people
  who were thinking outside the box. As our PTA, I don't think that we
  were thinking that big.
\end{itemize}

chana joffe-walt

They were definitely not thinking that big. Because the PTA was run by
Imee Hernandez and her co-president, Susan Moesker. Imee is not a
gentleman who fundraises. She's a social worker. The first time I met
Imee, she was wearing a T-shirt that said, ``I'm not spoiled. My husband
just loves me.'' She's Puerto Rican, grew up in Brooklyn. Her husband
Maurice is Puerto Rican and Black and really does adore her. He grew up
in Brooklyn, too. They have one daughter, one pit bull, one Persian cat,
and one school.

\begin{itemize}
\tightlist
\item
  imee hernandez\\
  I make it my business to stick myself in her school. {[}LAUGHS{]}
\end{itemize}

chana joffe-walt

For Imee, the new diversity, it gave her pause.

\begin{itemize}
\item
  imee hernandez\\
  Like, what I saw in September the population that came in, I was like,
  oh, that's a little frightening. {[}LAUGHS{]} And even the socio ---
\item
  chana joffe-walt\\
  If you could describe it for people who are on the radio and don't
  know what you saw.
\item
  imee hernandez\\
  I saw a lot of white people with very high socioeconomic backgrounds.
  You know, they have money. And that's great, but money tends to scare
  people. And I'm one of the people it scares. {[}LAUGHS{]} I'm one of
  the people it scares, because it twists everything around, and I don't
  like that. I don't like that. I don't like that --- I'd rather have a
  dinner where people of different cultures bring their food and we
  share together than have somebody else cater it. Like, that's how I
  feel you build community. I'm a social worker. That's my background,
  and that's what I believe in.
\end{itemize}

chana joffe-walt

Imee was in her second year at the school. The year before, she put on
community events, teacher appreciation, a spring carnival with face
painting and hot dogs. They raised some money here and there, but Imee's
vision for the PTA wasn't really about fundraising. The new parents,
though, they wanted to be active in their new school, and they were
accustomed to supporting their kids' schools by fundraising. The two
approaches came face-to-face at a PTA meeting in October.

\begin{itemize}
\item
  imee hernandez\\
  Three more minutes.
\item
  speaker 1\\
  All right, all right.
\item
  imee hernandez\\
  And then it's up to everyone ---
\item
  speaker 1\\
  Yes, I'm very ---
\item
  imee hernandez\\
  Because y'all got your --- you got to go home.
\item
  speaker 1\\
  I got my ---
\item
  imee hernandez\\
  {[}LAUGHS{]} You got to go home.
\end{itemize}

chana joffe-walt

There are about a dozen grownups, sitting on small plastic chairs around
a classroom table, the PTA Executive Board. Principal Juman is here,
too. Imee's leading, and the principal jumps in. She says she wants a
minute to share how much the new fundraising committee had raised so
far. Imee looks confused. Principal Juman goes on to say, the new
fundraising committee has had a lot of success.

\begin{itemize}
\item
  jillian juman\\
  The total they have raised, according to Rob, about \$18,000.
\item
  imee hernandez\\
  Mm-hmm, OK.
\item
  jillian juman\\
  And then, we just had a donation from a family a couple weeks ago who
  wanted to be anonymous that they're going to give either 5 to 10 grand
  in December. So this is big money.
\end{itemize}

chana joffe-walt

People seem unclear what to do with their faces. This is good news,
right? But also, wait, what's the Fundraising Committee? Imee turns to
her husband, Maurice. A retired cop, Maurice is also the treasurer of
the PTA, because when he retired, his wife told him he couldn't just sit
around at home. Maurice shrugs at Imee, doesn't seem to know anything
about this new money. Imee turns back to Principal Juman. So can we use
that money?

\begin{itemize}
\item
  imee hernandez\\
  --- answered my question.
\item
  jillian juman\\
  Yeah.
\item
  imee hernandez\\
  That was the question, if the PTA can have access to this money.
  Because I know already ---
\item
  jillian juman\\
  But what is the PTA? So that's all part off the question that's going
  around.
\item
  imee hernandez\\
  Right, yeah.
\item
  jillian juman\\
  So this \$18,000 Rob has raised under the umbrella of PTA.
\end{itemize}

chana joffe-walt

That's principal Juman.

\begin{itemize}
\item
  imee hernandez\\
  OK.
\item
  jillian juman\\
  So --- I think.
\item
  imee hernandez\\
  But who's --- who's got it? And where's it going? Like, yeah.
\item
  speaker\\
  This PTA member don't know nothing about it, so you know? {[}LAUGHS{]}
\item
  maurice\\
  How can that be access for Mr. Negrone.
\end{itemize}

chana joffe-walt

Maurice asks, how can that money be accessed for Mr. Negrone, who wants
new gym uniforms, or Mr. Lowe, to get his microscopes? Imee nods

\begin{itemize}
\item
  imee hernandez\\
  I mean, God bless Rob, and more power to him.
\item
  jillian juman\\
  Yeah.
\item
  imee hernandez\\
  But he's not an official member.
\item
  jillian juman\\
  Right.
\item
  imee hernandez\\
  So I think that's what makes it confusing, at least for me. You know,
  he is a PTA member because he's a parent, but he's not part of the
  executive board. So I think that's what makes it ---
\item
  jillian juman\\
  That's probably true.
\item
  imee hernandez\\
  Yeah, it makes it tricky.
\item
  jillian juman\\
  Right.
\item
  imee hernandez\\
  I mean, and again, I'm not {[}INAUDIBLE{]}.
\item
  speaker\\
  And then, I mean, I'm not going to lie ---
\item
  imee hernandez\\
  I he could bring in the money, that's great, but you know ---
\end{itemize}

chana joffe-walt

Principal Juman nods, repeats that she wishes Rob had been able to make
it. She was hoping everyone could be here and get on the same page about
money. But Rob is chaperoning a sixth grade overnight trip. They're late
getting back. One mom, a white woman, who came in with a new group of
sixth graders, says, look, I know Rob. He means well.

\begin{itemize}
\item
  speaker 1\\
  I think Rob, he's a professional fundraiser.
\item
  speaker 2\\
  Yeah, he's great.
\item
  speaker 1\\
  And therefore, he took it as his initiative ---
\item
  speaker 3\\
  They need money.
\item
  speaker 1\\
  --- to do the fundraising.
\item
  speaker 3\\
  Yep.
\item
  speaker 1\\
  And I think that's great. But I don't --- he should communicate with
  the PTA. And my impression is, I don't think he's meaning to offend
  anybody.
\item
  all\\
  No.
\item
  speaker 1\\
  I think he's sort of so laser-focused. That he's not thinking about,
  like, well, maybe you might want to let somebody know what he's doing.
\item
  jillian juman\\
  And he's been amazing. He really has. Yeah, yeah.
\end{itemize}

chana joffe-walt

That's principal Juman. At this point, everyone seems to feel a little
weird about how long they've spent talking about a fellow parent who is
not present. And anyway, it's money for the school. We're all for that.
We just need better communication. Imee she says, yeah, it's just
usually, money raised by parents goes through the PTA, so we can all
talk about where to spend it.

\begin{itemize}
\item
  speaker 1\\
  Yeah, OK.
\item
  speaker 2\\
  Because then we have to decide who has the say, because if it's a
  collective PTA ---
\item
  speaker 1\\
  Hey!
\item
  speaker 3\\
  There's Rob.
\item
  speaker 1\\
  There he is.
\item
  speaker 4\\
  There he is!
\item
  rob hansen\\
  Sorry I'm late.
\item
  imee hernandez\\
  I can't believe you made, after everything ---
\end{itemize}

chana joffe-walt

Rob walks into the room. He just got back from the sixth grade trip. He
sits down, and they all start to talk. We need to sort out some
questions about money. Then, a mom from the fundraising committee says
she's worried about me recording and asks me to stop. So I do.

\begin{itemize}
\tightlist
\item
  maurice\\
  It's been requested, but we don't know ---
\end{itemize}

chana joffe-walt

They let me stay, though, and take notes. Rob apologizes and then
explains. A group of them have been meeting to raise money for the
school. The new dual-language French program is expensive, and they
promised the principal they'd help raise money to cover it. They were
just eager to help, Rob says, so they formed a committee. He's really
sorry. He should have communicated and coordinated better with the PTA.
But good news is, it's going great. Someone has a contact with the
French embassy, a guy at the Cultural Services Arm in New York, and he
says he wants to help cover the costs of new French teachers and books.
They've already kicked in around 10 grand. At this point in my notes, I
wrote, ``lots of looks, big money.'' Rob says, ``The embassy suggested
we do a fundraiser, an event. They can help.'' Here I wrote, ``Looks ---
confused, mad? Nobody really talking.'' Imee says, ``This fundraiser
will be at the school though, right? And free, for everyone?'' Rob says,
``Yes.'' ``Good, good.'' She asks one more time, ``Free? I just want to
make sure everyone can go.'' Lots of nods. Rob says, ``Totally. This is
a community event, for our community.'' After about 20 minutes, Imee
says, we're out of time, guys. I can't tell if this is out of a
professional commitment Imee has to stick to the schedule, or a personal
commitment to getting out of that room.

{[}music{]}

chana joffe-walt

Before I came to SIS, I never thought much about the role of PTAs ---
ever. At SIS, early on, I have this feeling of, oh, a PTA is actually
critical to the success of an integrated school. A PTA has a very simple
democratic structure. Every parent has an equal vote. Smart --- it's
like a built-in system to equalize power, to help them make a budget
together, make decisions, set priorities collectively --- or not.

\begin{itemize}
\tightlist
\item
  imee hernandez\\
  So we're lucky enough that we have Rob here, who has really taken over
  fundraising and tried to bring it to the next level here at our
  school. So ---
\end{itemize}

chana joffe-walt

It's another PTA meeting, and the whole collective thing is not really
happening. It seems like the new parents are still raising money
separate from the PTA, and the communication problems do not seem to be
resolved. And some of the new parents have an idea. They propose a
formal separation --- the PTA and the people doing fundraising. Rob says
this way, there'll be two organizations collecting money for SIS.

\begin{itemize}
\tightlist
\item
  rob hansen\\
  --- is there will be two sorts of ways dollars are raised. One would
  be a community raise --- bake sales, direct gifts.
\end{itemize}

chana joffe-walt

That would be the PTA side, the community fund. Then there would be a
separate organization that would go after grants and big donors. Up
until this point, there seemed to be tension bubbling under the surface
between the new parents and the old parents, but it wasn't really until
this moment that the unsaid started to get said --- mostly by Imee's
husband, Maurice.

\begin{itemize}
\item
  maurice\\
  I think a lot of us feel that there's two different groups. There's
  the fundraising group and the PTA, which is --- That's what it looks
  like. You guys have this goal of making \$50,000 that is going to the
  French program. Now, as you said, what about the rest of your school?
  Where's all this money going? We have no answer. We don't know.
\item
  maureen\\
  And it's very easy to feel steamrolled.
\end{itemize}

chana joffe-walt

That's Maureen, a white mom who's new. There are lots of nods.

\begin{itemize}
\tightlist
\item
  maureen\\
  And I just don't want ---
\end{itemize}

chana joffe-walt

Maurice is asking, is this new money you're talking about, is it just
for the new French dual-language program? Which is another way of
asking, is this money just for your kids, or is it for everybody? Rob
says emphatically, it's for everybody. Maurice says, really?

\begin{itemize}
\tightlist
\item
  maurice\\
  I mean, that's being naive. We think, OK, they're going to donate all
  this money through the French embassy, and we're going to, OK, well,
  we're going to buy new chalkboards. That's kind of --- that's being
  naive. Now, you're saying the 50,000 will be for the PTA community to
  decide where it's going to go. So I mean, I hear what you're saying,
  which sounds great. But again, maybe I'm still talking about last
  meeting, when Jillian said, OK, well, are we only get a percentage of
  that? So we still don't have an answer.
\end{itemize}

chana joffe-walt

Later, talking to Rob, I learned that the new separate fundraising arm
he's talking about is actually a foundation. They want to create a
school-based foundation at SIS. The plan is to call it the Brooklyn
World Project. I asked Rob, why do you need another way to raise money?

\begin{itemize}
\item
  chana joffe-walt\\
  There's a PTA. Most people have heard of a school PTA. Why do you need
  a separate organization that's not the PTA?
\item
  rob hansen\\
  Yeah, so probably the easiest way to explain it is to not think about
  it from the school side, but to think about it from the potential
  donor side. So basic idea that we're following is that the --- let's
  say the international says, we want to do extend today, and we want to
  do theater. And so we go and we find a donor who loves theater, and
  loves the French language, and loves the idea that kids who've never
  spoken French and had no exposure get the chance to go and compete,
  actually, against some of the most established schools in the city.
  And a donor just loves that. Like, I love it. I love giving that kind
  of opportunity to kids. I'm going to cover all of that, because I
  think it's that important. If that money goes to the PTA, you could
  have a situation that the PTA says, or members that PTA say, I don't
  know that we really like the theater program. I'm not sure I think
  that we should be using those dollars to do x, or y, or z. Now,
  normally, you'd be able to say, well, donor intent is what it is. You
  should probably use it towards what it was intended for.
\item
  chana joffe-walt\\
  You mean normally, in another fundraising context?
\item
  jillian juman\\
  Yeah, meaning in nonprofits. So there's a basic kind of morality of a
  nonprofit to say, if a donor gives you it to you to do something, you
  should try to do that. Donor intent is an important part of it. It's
  sort of a trust that's established.
\end{itemize}

chana joffe-walt

Rob says, because the PTA is a democracy, it makes things complicated.
The very thing I saw as a strength of a PTA --- one parent, one vote ---
to Rob, that's a problem for fundraising. Parents come and go and change
their minds about what's important. A private donor wants stability, and
Rob is trying to raise money for the kind of programming that was
available at his son's wealthy elementary school. At that school, Rob
was co-president of the PTA, and the previous year, his PTA pulled in
close to \$800,000. \$800,000 --- money that paid for after school
programming, and ballroom dancing, chess, art, music, a garden.
\$800,000 for a school that is 75\% white and serves a tiny fraction of
the poor kids in the district. There aren't enough wealthy parents at
SIS to raise that kind of money. That year, Rob helped raise \$800,000,
the SIS PTA raised \$2,000. So Rob was trying to be creative. A
foundation was a way for his new school to catch up. The school
leadership, the principal, was behind the idea. . Ms. Juman told me she
saw the foundation as a path to equity and access. More resources meant
they'd be able to provide all kids with opportunities --- like, say, a
school trip to France. But the parent leadership, they found it
annoying. Imee knew the new parents were trying to help the school. But
she already liked the school. She felt like she was being saved against
her will. Plus, they're new, she said. Shouldn't we be the ones helping
them? She was fine with them bringing in ideas, but she didn't
understand why they hadn't brought them to her first. They hadn't
thought to consult her. She to me, multiple times, why are they coming
up with all these private plans and meeting in secret committees?

\begin{itemize}
\item
  jillian juman\\
  You were pissed about that.
\item
  imee hernandez\\
  Totally.
\item
  chana joffe-walt\\
  Yeah.
\item
  imee hernandez\\
  Because I wasn't involved.
\item
  chana joffe-walt\\
  I mean, why were you angry about that?
\item
  imee hernandez\\
  Because here I am trying to build something with the school. Why
  didn't you just involve me? Why didn't you just tell me about it?
  Like, it felt like it was a secret. I don't know if it was or if it
  wasn't. I'm invested in the school. Clearly, I've proven to you I'm
  invested in the school. And you couldn't tell us that you wanted to
  fundraise in a different way?
\end{itemize}

chana joffe-walt

Rob and the new parents did tell the principal that they wanted to
fundraise in a different way, but Imee felt like, what about the rest of
us? She felt like the PTA was ignored. At that last meeting, Imee went
quiet. She told me she just felt enraged, and then embarrassed for
feeling so enraged.

\begin{itemize}
\item
  imee hernandez\\
  I guess I just threw a tantrum. {[}LAUGHS{]} And I just didn't want to
  be a part of it. Which is not right, but I think, again, in the
  moment, I just felt like --- I was hurt.
\item
  chana joffe-walt\\
  Do you usually throw --- was the tantrum the thing I saw? Because that
  did not seem like a tantrum.
\item
  imee hernandez\\
  No, that was not a tantrum. I could have been a lot worse, and I was
  really, really trying to restrain myself. Yeah, I really was. That was
  really under control --- really, really under control. It wasn't, but
  it was really, really under control.
\item
  chana joffe-walt\\
  I asked, was there another time?
\item
  imee hernandez\\
  Tantrum? Yeah, at home with my husband. {[}LAUGHS{]} That's when I
  threw my tantrum.
\end{itemize}

chana joffe-walt

So it was tense, among the parents. But this is a school for children.
Did it matter if the adults were not getting along, or who controlled
which pot of money? Yes. Yes it did. That's coming up after the break.
The school year went on. Rob's fundraising committee moved forward with
the French Embassy to plan a fundraiser. It was now being called a
``gala.'' The PTA moved forward with parent volunteers to plan a spring
carnival. It was being called ``the spring carnival.'' Quiet resentments
locked in place. On the phone one night, Imee's co-president on the PTA,
Susan Moesker, told me she worried the school was changing in ways that
were damaging to the community. Susan is white herself, but she didn't
come in with the new white parents. When she started, her son was one of
the only white kids in the school. And now she felt like they were all
being written into a narrative that wasn't true, that SIS was a bad
school before, and now that the new white families had arrived, it was
being turned around.

\begin{itemize}
\item
  susan moesker\\
  It is noticeable. I think it is something that even my child has
  picked up on --- just like a very different feeling among some of the
  students and some of the parents, this real sense again that here they
  come to save our poor struggling school that couldn't possibly make it
  on its own without their money and their vision. And we do not all
  feel that that is, necessarily, the case.
\item
  chana joffe-walt\\
  What do you feel?
\item
  susan moesker\\
  {[}SIGHS{]} {[}LAUGHS{]} Ah ---
\end{itemize}

chana joffe-walt

This was a long conversation. The upshot? She's not happy with the way
the new parents are behaving. It was true. A new narrative was taking
hold at SIS. It's not like the kids were talking about it all the time,
but it was in the air, and the kids were starting to pick up on who was
valued and why. In the cafeteria, I'd hear middle schoolers saying, the
French kids could kill someone, and they'd get away with it. Upstairs in
the high school, I'd hear kids complain, all the attention has shifted
to the new middle schoolers. We're being pushed aside. And down in the
library, I met three sixth grade boys, white boys new to SIS. They're
sweaty from playing soccer and looking very small against their huge
backpacks. These boys, even at 11 years old, they've absorbed the same
messages --- that SIS wasn't so good before. It was a bad school.

\begin{itemize}
\tightlist
\item
  boy\\
  The kids wouldn't pay attention, and they had all the --- zone out
  every little thing. And I bet they learned very little. And now, this
  generation, with us, I think we're doing a lot better, and I think
  that we're learning at a much faster pace.
\end{itemize}

chana joffe-walt

He and his friends, they've turned the school around. That's what he's
learning.

\begin{itemize}
\tightlist
\item
  boy\\
  It's going to be one of the top choices. Already, in the Brooklyn,
  when you're applying to middle schools, you get a book on statuses and
  stuff. And I think this school is actually really high up in the
  statuses.
\end{itemize}

chana joffe-walt

Nobody calls it ``the book on statuses.'' They call it a directory of
schools, with info like enrollment numbers for different schools, test
scores, and special programs. But I love the calls it, ``the book on
statuses,'' because this is what happened at SIS. The school had a bad
reputation among white families, and then, suddenly, it was in demand.
Its status had changed because of the white kids. A powerful draw for
white families into any school is other white families. Once you have a
critical mass of white kids, you pass what one city calls ``a bliss
point.'' This is a real thing researchers study --- how many white kids
are needed at a school to make other white families feel comfortable
choosing it. That number, the bliss point, is 26\%. That fall, white
families were crowding the school tours at SIS, not because the test
scores had improved --- the new scores hadn't even come out yet --- but
because the other white families made them feel blissfully comfortable.

{[}music{]}

chana joffe-walt

Of course, the thing that made the new white parents comfortable coming
to SIS in the first place was the promise of a French program. They
wanted French, and they got French. So now, all the sixth, seventh,
eighth, and ninth graders are learning French. It wasn't a true
dual-language program where kids learned in French for half the day or
whatever. That first year, most of the French was happening in the after
school program. You'd sign up for regular after school stuff like
culinary, or soccer, or drama, and it would be conducted in French.

\begin{itemize}
\item
  teacher\\
  Hey, hey! {[}KIDS CHATTERING IN FRENCH{]}

  Hey, everybody, you have to listen, OK?
\end{itemize}

chana joffe-walt

We're in the auditorium. And it's sweet. The kids are onstage rehearsing
this play they wrote in French. And it's like they're having fun. But I
couldn't help feeling like there's something off balance about this.
Most of the kids doing this drama program seem to be native French
speakers, but not all. A sixth grader named Maya is standing to the side
of the stage, script in hand, waiting for her line.

\begin{itemize}
\item
  imee hernandez\\
  For me, it's a bit weird, because I have no idea they're saying. Yeah.
\item
  chana joffe-walt\\
  Really? Even in the play that you've been practicing, you don't know
  what they're saying?
\item
  imee hernandez\\
  Yeah, I don't know what they're saying still.
\item
  maurice\\
  You have the translations of the script.
\item
  maya\\
  Yeah, but sometimes, when a teacher talks in French to the class, I
  don't understand.
\item
  chana joffe-walt\\
  And do you figure it out? Or is it confusing?
\item
  maya\\
  Confusing.
\end{itemize}

chana joffe-walt

Still, she's excited. She's grinning, watching the other kids on stage.
She's hanging out with her friend Constance. Maya gets up to deliver her
lines.

\begin{itemize}
\item
  maya\\
  {[}FRENCH{]}?
\item
  constance\\
  Oh you did a wrong line.
\item
  maya\\
  Yeah, what?
\item
  constance\\
  {[}FRENCH{]}.
\item
  maya\\
  That's confusing.
\item
  constance\\
  You just say {[}FRENCH{]}, and then {[}FRENCH{]} is {[}FRENCH{]}
  after.
\end{itemize}

chana joffe-walt

Constance, a native French speaker, tells Maya, you said the wrong
thing. Constance correct her, pronounces it for her.

\begin{itemize}
\item
  constance\\
  {[}FRENCH{]}.
\item
  maya\\
  {[}FRENCH{]}.
\item
  teacher\\
  But it's OK.
\item
  maya\\
  Yeah, because I can't do it that well.
\end{itemize}

chana joffe-walt

Maya says, I can't, and her friend says, I'll do it for you.

\begin{itemize}
\item
  constance\\
  OK, I'll just say it.
\item
  maya\\
  {[}FRENCH{]}.
\item
  constance\\
  {[}FRENCH{]}.
\end{itemize}

chana joffe-walt

Learning another language is not new to Maya.

\begin{itemize}
\item
  maya\\
  My dad speaks Arabic, and my mom's Turkish.
\item
  jillian juman\\
  Uh-huh, and now you're learning French.
\item
  maya\\
  Yes, it's so confusing. Three languages at the same time.
\end{itemize}

{[}music{]}

chana joffe-walt

When the new white parents asked for a dual-language French program at
SIS, Principal Juman said yes. SIS was supposedly an international
school, but she told me they didn't really have a lot of international
programming, so it seemed like a good idea to her. But there was no
school-wide debate about it, or consensus. The community didn't decide.
What if they had? More than 1/3 of the families at sis are Hispanic.
What if the dual-language program was Spanish, or Arabic? 10\% of the
students speak Arabic. If they had made a different choice, if SIS had a
dual-language Arabic program, Maya would be teaching Constance how to
read her lines. She'd be the one explaining the cultural references and
teasing her friend about her terrible accent. She'd be the one
translating the teacher's stage directions. There was money for a French
program, which meant that at SIS, French had value. Arabic didn't.
Spanish didn't. That's something Maya is learning at school, along with
her French script.

From the very beginning, Imee and the others had insisted on three
things from the new parents and the fundraising committee --- that the
gala fundraiser they were planning with the French Embassy, would,
number one, be open to everyone, number two, take place at the school,
and number three, be free. Then, four weeks before the gala, the PTA
asked for an update, and a parent named Deb showed up --- a mom to a new
sixth grader, part of Rob's fundraising committee.

\begin{itemize}
\item
  deb\\
  So I will start with the fact that I had a nice conversation with ---
\item
  fabrice\\
  Fabrice.
\item
  deb\\
  Fabrice. Is it Fabrice?
\item
  fabrice\\
  Fabrice, yeah.
\end{itemize}

chana joffe-walt

Deb volunteered early on to help organize the party and she tells
everyone, I met with our partner, Fabrice, at the French Embassy, and
the event can't be at the school. The embassy won't be able to draw
their supporters to Brooklyn. It'll be at the Cultural Services Building
on the Upper East Side, Manhattan, 45 minutes away.

\begin{itemize}
\item
  deb\\
  I apologize if I'm saying things you guys already know, but I didn't
  know some of this info, so it was good. But the event is really ---
  it's their event. It's not really our event.
\item
  susan moesker\\
  Oh!
\item
  deb\\
  It's their event.
\end{itemize}

chana joffe-walt

That's Susan with the ``Oh.'' Maurice leans forward, elbows on the
table. Imee is not here. She knew the meeting would be almost entirely
about fundraising, and she's sitting this one out. Maurice is now
concentrating on Rob, who turns to Deb and says, ``In what sense is that
their event?'' ``They make the rules, she says.''

\begin{itemize}
\tightlist
\item
  deb\\
  With our input, but there are certain things that are not flexible.
  The biggest thing is, nobody will be allowed in at the door. You have
  to be on a list. You have to RSVP. You have to be on the list. All
  names.
\end{itemize}

chana joffe-walt

Security. It's a government building after all.

\begin{itemize}
\item
  deb\\
  He sends out the invitation to 22,000 people on his mailing list. So
  now, making it a free event is a problem, because now we're inviting
  22,000 people for free to drink wine and eat food that may not have
  any interest in us. So we thought the best thing to do would be a
  suggested donation. Can't afford to go?
\item
  rob hansen\\
  Can I give a variant on that?
\item
  deb\\
  Yeah.
\end{itemize}

chana joffe-walt

That's Rob, asking to give a variant, which is, how about we have a
separate invitation for our people that doesn't ask for any money? Rob
seems to be picking up on the instant irritation in this room, and he's
adding many variants to Deb's report.

\begin{itemize}
\item
  rob hansen\\
  It's either a modified version, or just a clarity that everybody in
  this community ---
\item
  deb\\
  He won't. There'll be one invite. It will say the same thing. That's
  what I suggested. I suggested \$50 a head on the outside.
\item
  rob hansen\\
  Even if we simply put a cover note saying, no charge. We want you to
  come join us, our community.
\item
  deb\\
  Right, but on the invite, it will say, ``suggested donation.'' Then,
  if you want to --- however we want to forward it, we can say that. But
  they will only do one invite.
\end{itemize}

chana joffe-walt

Deb hasn't been able to make previous PTA meetings. So all Deb knows is
she got an email from the fundraising committee at her kid's new school,
which she assumed was part of the PTA. She's volunteering her time, a
ton of her time, to organize a huge event. She does not understand that
the email list she's on is for a separate fundraising committee that
just became even more unpopular with the official PTA leadership. I
think I stopped moving watching Deb. It's so tense. She's like a
porcupine who's just wandered into a balloon store.

\begin{itemize}
\item
  deb\\
  They're serving wine, water, and then French hors d'oeuvres. And as
  far as the auction, we have a couple of cleanses. We have restaurants.
  We have a soccer camp. We have a vacation rental in California. We've
  got a couple hair salons. Very few from the community here, and that's
  really when I wanted to talk about.
\item
  speaker\\
  Is he from the parent community or geographic community?
\item
  deb\\
  Parent community and geographic community.
\end{itemize}

chana joffe-walt

Deb says at her kid's elementary school, they got a lot more donated
items from parents. She tells the room, you can ask at the restaurants
you go to if they do gift certificates. The salon, your employers ---
you'd be surprised what people can offer. Just ask.

\begin{itemize}
\tightlist
\item
  deb\\
  Then that's what people like my friends --- and most of my friends,
  though, they're all in other schools. I'm just new here. I don't
  really know many people. So the only people I've been able to reach
  out to are the 36 on Rob's email list. {[}LAUGHTER{]} And then, a
  quarter of them gave, have donated something already, like found
  something. So I'm telling you, that house in Sonoma County is gorgeous
  --- four bedrooms, three baths, beautiful.
\end{itemize}

chana joffe-walt

I think about a PTA meeting a few months before where I watched Imee
gently explain to one of the new parents why it might be hard for some
families to throw in \$5 for classroom supplies, that even being asked
to donate can feel alienating. Some people in this room seem to be
experiencing this whole thing as a routine update about public school
volunteering. Others look like someone who's walked into the wrong room
and is now looking around to the friends they came with for affirmation,
we're in the wrong room, right? How do we get out?

\begin{itemize}
\item
  deb\\
  Usually, I get more tickets to shows, games, things like --- I've
  gotten Broadway tickets, but I haven't gotten anything in the ticket
  arena. Knicks?
\item
  speaker\\
  I have a contact at the Knicks. I'm willing to reach out.
\item
  deb\\
  Yeah, they always go. Everybody wants to go to a game. There's always
  somebody. And they also make great Christmas gifts. And that's the
  other thing we're lacking, is actual items. We used to have a parent
  --- well, we still have the parent, but she's not in my school ---
  that worked at Tiffany. And we always had some beautiful Tiffany
  pieces, or a Coach bag, some products --- makes it look nice.
\end{itemize}

{[}music{]}

chana joffe-walt

I spent a small chunk of that meeting occupied by an admittedly
sentimental thought. Just looking around, the room was kind of
incredible. People with homes in Sonoma and people who live in public
housing, sitting together at a long wooden table in the library of a
public school that they all share. That never happens. And I didn't want
them to mess it up. But of course, they are. This is not something we
have a lot of practice in. New York City has one of the most segregated
school systems in the country. White parents here have very little
practice sharing public schools. Maybe this is all to be expected. White
parents will charge ahead, will sometimes be careless, secretive, or
entitled. In response, parents of color will sometimes be cautious, or
distrustful, defensive. These are well-established patterns, repeated
over generations. It's easier for us to continue operating on separate
tracks, because it's what we already know how to do.

The guy from the French embassy apparently has a mailing list of 22,000
people in the New York area. 300 people RSVPed to the gala for SIS. I
couldn't believe it. And I couldn't believe that one of them was Imee.

\begin{itemize}
\item
  susan moesker\\
  Hello. Good evening!
\item
  chana joffe-walt\\
  You guys look lovely.
\item
  imee hernandez\\
  Hi, how are you? {[}LAUGHS{]}
\end{itemize}

chana joffe-walt

Imee, Maurice, and Susan carpooled together to the Upper East Side. It's
winter. Central Park is across the street. It's cold. Imee told me she
decided she needed to be a grown-up and come. They got stuck in traffic,
so they're rushing up the sidewalk.

\begin{itemize}
\item
  imee hernandez\\
  We're not that late, are we, Susan?
\item
  susan moesker\\
  It's not serious.
\item
  imee hernandez\\
  No, we're not that late.
\end{itemize}

chana joffe-walt

The Cultural Services Building is ivy-covered with columns. The doors
are wrought iron. The entryway is marble.

{[}music{]}

\begin{itemize}
\tightlist
\item
  susan moesker\\
  All right, you guys all came in together? All right.
\end{itemize}

chana joffe-walt

A huge marble staircase winds up the side of the room. Later, I look up
the architectural style --- Italian renaissance, Palazzo style. It's a
palace. There are people milling, sampling 17 different cheeses. I don't
recognize anyone else from the school. Who are these people who have
chosen to come out on a weekday evening for a fundraising event for a
not-prominent or well-known-at-all public school in Brooklyn?

\begin{itemize}
\tightlist
\item
  man\\
  I'm not involved with the school, but my wife is involved with the ---
\end{itemize}

chana joffe-walt

I started asking people how they heard about the event.

\begin{itemize}
\item
  jillian juman\\
  And what brought you here tonight?
\item
  barbara\\
  Actually, an invitation by my wonderful French professor.
\end{itemize}

chana joffe-walt

A lady named Barbara tells me she's never heard of SIS, like most people
here, but she loves French, and she loves Paris, and it sounded like a
fun night with other people who do too. She goes to France every year.

\begin{itemize}
\tightlist
\item
  barbara\\
  October is my saison preferee. Actually, I found this October too
  warm, but I like it when it's a nice fall, crisp, and you wear your
  scarf, your foulard.
\end{itemize}

chana joffe-walt

I enjoy a person who likes to talk, where you can just get on the ride
and sit back. Barbara is definitely that kind of person.

\begin{itemize}
\item
  barbara\\
  And my apartment in Paris, it's sort of --- I'm confused sometimes. I
  say, am I in Gramercy Park, or am I in Saint-Germain-des-Pres? It's
  got a similar ambiance of being a neighborhood. It's great. Have you
  been?
\item
  chana joffe-walt\\
  I've never been.
\end{itemize}

barbara

Oh my god, she hasn't been to Paris. Barbara's looking around for her
French teacher to tell her the news. Barbara's teacher, it turns out,
heard about this evening the same way most people here did. She was
invited by this man, Fabrice.

\begin{itemize}
\tightlist
\item
  fabrice\\
  For the School for International Studies, we are hoping we will raise
  \$100,000 each year for the next seven years.
\end{itemize}

chana joffe-walt

Fabrice Germain works for the Cultural Services arm of the French
Embassy. He tells me he's fundraising for dual-language programs in
public schools because his mission is to promote French language and
culture. He called it ``soft power,'' which I was kind of surprised he
said out loud, since I associate that with something we do in developing
countries, not something you're allowed to do in American public
schools. After Fabrice and I talked, I walked into the main room and
immediately saw Maurice. Maurice was so skeptical of this whole embassy
thing, but there he is at a table, selling raffle tickets next to Imee,
cheerfully raising money for a program neither of them ever wanted at
their school.

\begin{itemize}
\tightlist
\item
  maurice\\
  We are raffling off two airline tickets to France. Warm blue ticket's
  going to win. It could be yours!
\end{itemize}

chana joffe-walt

Maurice, amiable as ever, is trying --- and mostly failing --- to
convert ticket sales into social connections. He asks everyone, so if
you win, when are you thinking of going? Oh, you're going anyway? For
Easter? Oh, nice.

\begin{itemize}
\item
  maurice\\
  How's your there in Easter?
\item
  speaker\\
  Yes, I am from France.
\item
  maurice\\
  How's the weather there in Easter?
\item
  speaker\\
  Good. Fantastic. I'm --- before we do it, I have a question. Can I
  from here, from this ticket, buy something to go from Paris to
  Marseille, when I'm leaving?
\item
  maurice\\
  No.
\end{itemize}

chana joffe-walt

Barbara, from Gramercy Park, the woman who loves fall in Paris, wanders
across the marble floor toward the raffle table, beside where Imee is
sitting. And I thought, oh, no.

\begin{itemize}
\item
  imee hernandez\\
  It's a pleasure to meet you.
\item
  barbara\\
  Hi, how are you?
\item
  imee hernandez\\
  Good, thank you.
\item
  barbara\\
  You're one of the parents of a bilingual student?
\item
  imee hernandez\\
  She's not bilingual, but she does go to the school. {[}LAUGHS{]}
\item
  barbara\\
  She will be bilingual, eventually.
\item
  imee hernandez\\
  Oh, she will be, yes, eventually.
\item
  barbara\\
  What a wonderful thing. Are you pleased with the program?
\item
  imee hernandez\\
  Yes, I love the school.
\item
  barbara\\
  It's so important to learn another language. It opens the world for
  you. And what is your name?
\item
  chana joffe-walt\\
  Chana.
\item
  barbara\\
  Anna. I was just telling Anna, when I go to Paris, which I do every
  year ---
\item
  imee hernandez\\
  Cool!
\item
  barbara\\
  It is cool. And it's cooler because I can speak the language. And you
  have entree into their society. Not totally; one will never have total
  entree. But you can interact with your neighbors. You can interact in
  a restaurant. You can interact at the dry cleaner, at the supermarket.
  And they so appreciate an American who can speak French.
\item
  imee hernandez\\
  Yes, yes.
\item
  barbara\\
  And the language is beautiful.
\end{itemize}

chana joffe-walt

Imee starts looking around. Maurice moves closer and leans in to hear
why his wife is doing that nervous laugh, as Barbara explains Imee, a
Puerto Rican woman, that being bilingual makes a person more
sophisticated. Imee is exceedingly polite.

\begin{itemize}
\item
  barbara\\
  Paris is a lodestar, and if you really want to enjoy it, you've got to
  speak the language.
\item
  imee hernandez\\
  {[}LAUGHS{]} It was a pleasure. Thank you.
\end{itemize}

chana joffe-walt

That entire conversation, Imee never mentioned to Barbara that she does
speak another language --- Spanish. Later, I ask her why. She shrugs it
off.

\begin{itemize}
\item
  imee hernandez\\
  {[}LAUGHS{]} I was like, no, I'll just let her talk. It's OK. It was
  all right. So ---
\item
  chana joffe-walt\\
  Do you remember when you were telling me about the silent tantrum that
  you're having ---
\item
  imee hernandez\\
  Yes.
\item
  chana joffe-walt\\
  How would I know if that was happening?
\item
  imee hernandez\\
  You wouldn't. Only my husband would. {[}LAUGHS{]} If I'm throwing a
  silent tantrum. He would know if I'm throwing a silent tantrum. Is
  that happening right now?
\item
  maurice\\
  No, not right now.
\end{itemize}

chana joffe-walt

Imee turns her back to her husband, facing me, and behind her, Maurice
is looking right at me, nodding vigorously --- yes.

So here we were in our fancy clothes on the Upper East Side of
Manhattan, raising money for a French program at an utterly normal
Brooklyn public school. That was already weird. But the toasts --- the
toasts were when the cognitive dissonance of the evening really kicked
in for me. Fabrice steps up onto the marble staircase and clinks his
glass, announces it's time to celebrate what we've created and raise
some money.

\begin{itemize}
\tightlist
\item
  fabrice\\
  It takes a village. It takes a dedicated principal. She's here with
  us. It takes --- {[}CHEERS{]} Yes, yes, yes.
\end{itemize}

chana joffe-walt

Fabrice hands the make to Principal Juman.

\begin{itemize}
\item
  jillian juman\\
  Hi!
\item
  all\\
  Hi
\item
  jillian juman\\
  It's so nice to see all of you. And I think the number one thing is I
  think all of us standing here believe in public education and believe
  that all students --- {[}APPLAUSE{]} Yes.
\end{itemize}

chana joffe-walt

One by one, people are talking about equality, and diversity, and
community, and the meaning of public education. Here, at the Cultural
Services Palatial Palace, full of white people.

\begin{itemize}
\tightlist
\item
  rob hansen\\
  So I went on the sixth great trip, 11-year-olds going on an overnight
  trip up into the Catskills. I would not recommend doing that.
\end{itemize}

chana joffe-walt

Rob gives a toast about that time he went on the overnight trip. It
starts off OK, but then veers into strange and sort of cringe-y
territory. He's on a ropes course, 40 feet up, looking down.

\begin{itemize}
\tightlist
\item
  rob hansen\\
  Below me was that diverse group of kids. They were diverse kids
  belaying me, making sure that when I jumped, they would actually
  cushion my fall. That day, each of those kids was going to climb up
  that pole and was going to have the same opportunity and the same
  challenge. And it made me think that that's what this school is about.
  It's about the opportunity to do the International Baccalaureate, the
  challenge of it, so the opportunity to explore French, and the
  challenge of it, for all kids. And ---
\end{itemize}

chana joffe-walt

I agree with Rob. It's great to give kids equal access to opportunity.
But what they're being given access to are the opportunities that Rob
and the other white parents care about.

Downstairs, I find Susan, Imee's PTA co-president, on a bench by
herself. She's near the band, drinking wine, looking a little
dumbstruck. I ask if she's OK. This is something else, she says. And
then she adds, it's just hard to explain how this is a public school
fundraiser.

When the founder of American public education, Horace Mann, laid out his
vision for public schools back in the day, he rode his horse around
Massachusetts, podium to podium. And his pitch was common schools would
make democracy possible. They would bind us to one another, indoctrinate
us, give us the skills and tools we need for democratic living. Public
schools, he believed, would be the great equalizer. Rich and poor would
come together and develop what he called ``fellow-feeling,'' and, in
doing so, quote, ``obliterate factitious distinctions in society.'' For
that to happen, you need everyone in the same school together. At SIS,
they've gotten that far. Everyone was in the same school together. But
there was no equalizing. We can be in the same school together and not
be equal, just like we can be in the same country together. It's not
enough. What do we do about that? After the gala, money poured into SIS,
and more white families enrolled their kids at the school. But in the
years after that, there was a backlash, and SIS changed in ways that
made Rob question himself. He wondered if he'd made mistakes. He told me
he thought they all wanted the same thing for the kids. He just didn't
know. Not knowing? That happens a lot with white parents. I looked into
the history of this school, and I learned that this wasn't the first
time white parents showed up here. White parents have been involved all
along, all the way back to the very beginning of this school half a
century ago, doing the same kinds of things I'd just seen. It happens
again and again, white parents wielding their power without even
noticing, like a guy wandering through a crowded store with a huge
backpack, knocking things over every time he turns.

Horace Mann believed public schools would make us equal, but it doesn't
work. I'm not sure how to fix that, but I want to lay out the story, the
whole story of this one American public school, because what I am sure
of is that in order to address inequality in our public schools, we are
going to need a shared sense of reality. At the very least, it's a place
to start. That's next time on ``Nice White Parents.''

{[}music{]}

chana joffe-walt

``Nice White Parents'' is produced by Julie Snyder and me, with editing
on this episode from Sarah Koenig, Nancy Updike and Ira Glass. Neil
Drumming is our Managing Editor. Eve Ewing and Rachel Lissy are our
editorial consultants. Fact-checking and research by Ben Phelan, with
additional research from Lilly Sullivan. Music supervision and mixing by
Stowe Nelson. Our Director of Operations is Seth Lind. Julie Whitaker is
our Digital Manager. Finance management by Cassie Howley, and Production
management by Frances Swanson. Original music for Nice White Parents is
by The Bad Plus, with additional music written and performed by Matt
McGinley and Dan Reitz. Special thanks to Whitney Dangerfield, Rich
Orris, Amy Pedulla, Nikole Hannah-Jones, Scott Sargrad, Jackie Carrier,
Gene Demby, Charles Jones, Lenny Garcia, and Valero Doval. At The New
York Times, thanks to Sam Dolnick, Stephanie Preiss, Nina Lassam and
Julia Simon.

``Nice White Parents'' is produced by Serial Productions, a New York
Times Company.

Previous

More episodes ofNice White Parents

\href{https://www.nytimes.com/2020/08/13/podcasts/nice-white-parents-school.html?action=click\&module=audio-series-bar\&region=header\&pgtype=Article}{\includegraphics{https://static01.nyt.com/images/2020/07/30/podcasts/30nwp-art/nice-white-parents-album-art-thumbLarge.jpg}}

August 13, 2020~~•~ 50:38Episode Four: `Here's Another Fun Thing You Can
Do'

\href{https://www.nytimes.com/2020/08/06/podcasts/episode-three-this-is-our-school-how-dare-you.html?action=click\&module=audio-series-bar\&region=header\&pgtype=Article}{\includegraphics{https://static01.nyt.com/images/2020/07/30/podcasts/30nwp-art/nice-white-parents-album-art-thumbLarge.jpg}}

August 6, 2020~~•~ 46:55Episode Three: `This Is Our School, How Dare
You?'

\href{https://www.nytimes.com/2020/07/30/podcasts/nice-white-parents-serial-2.html?action=click\&module=audio-series-bar\&region=header\&pgtype=Article}{\includegraphics{https://static01.nyt.com/images/2020/07/30/podcasts/30nwp-art/nice-white-parents-album-art-thumbLarge.jpg}}

July 30, 2020~~•~ 53:37Episode Two: `I Still Believe in It'

\href{https://www.nytimes.com/2020/07/30/podcasts/nice-white-parents-serial.html?action=click\&module=audio-series-bar\&region=header\&pgtype=Article}{\includegraphics{https://static01.nyt.com/images/2020/07/30/podcasts/30nwp-art/nice-white-parents-album-art-thumbLarge.jpg}}

July 30, 2020~~•~ 1:02:23Episode One: The Book of Statuses

\href{https://www.nytimes.com/2020/07/23/podcasts/nice-white-parents-serial.html?action=click\&module=audio-series-bar\&region=header\&pgtype=Article}{\includegraphics{https://static01.nyt.com/images/2020/07/21/podcasts/nice-white-parents-album-art/nice-white-parents-album-art-thumbLarge.jpg}}

July 23, 2020~~•~ 2:49Introducing: Nice White Parents

\href{https://www.nytimes.com/column/nice-white-parents}{See All
Episodes ofNice White Parents}

Next

Published July 30, 2020Updated Aug. 7, 2020

\begin{itemize}
\item
\item
\item
\item
\item
\item
\end{itemize}

``Nice White Parents'' is a new podcast from Serial Productions, a New
York Times Company, about the 60-year relationship between white parents
and the public school down the block.

\textbf{Listen to the first two episodes now, and keep an eye out for
new episodes each Thursday, available here and on your mobile device:}
\textbf{\href{https://podcasts.apple.com/us/podcast/nice-white-parents/id1524080195}{Via
Apple Podcasts}} \textbf{\textbar{}}
\textbf{\href{https://open.spotify.com/show/7oBSLCZFCgpdCaBjIG8mLV?si=YcEPLD3xT2ejXmpQz-tRpw}{Via
Spotify}} \textbf{\textbar{}}
\textbf{\href{https://podcasts.google.com/feed/aHR0cHM6Ly9yc3MuYXJ0MTkuY29tL25pY2Utd2hpdGUtcGFyZW50cw}{Via
Google}}

It's 2015, and one Brooklyn middle school is about to receive a huge
influx of new students.

In this episode, Chana Joffe-Walt, a reporter, follows what happens when
the School of International Studies' 6th grade class swells from 30
mostly Latino, Black and Middle Eastern students, to 103~--- an influx
almost entirely driven by white families.

Everyone wants ``what's best for the school,'' but it becomes clear that
they don't share the same vision of what ``best'' means.

\textbf{Up next: In Episode 2, Chana digs into New York City's Municipal
Archives to trace the complicated history of integration at the school
--- and across the city.}

\includegraphics{https://static01.nyt.com/images/2020/07/21/podcasts/nice-white-parents-album-art/nice-white-parents-album-art-articleInline.jpg?quality=75\&auto=webp\&disable=upscale}

\hypertarget{episode-two-i-still-believe-in-it}{%
\subsubsection{Episode Two: `I Still Believe in
It'}\label{episode-two-i-still-believe-in-it}}

White parents in the 1960s fought to be part of a new, racially
integrated school in Brooklyn. So why did their children never attend?

transcript

Back to Nice White Parents

bars

0:00/53:37

-53:37

transcript

\hypertarget{episode-two-i-still-believe-in-it-1}{%
\subsection{Episode Two: `I Still Believe in
It'}\label{episode-two-i-still-believe-in-it-1}}

\hypertarget{reported-by-chana-joffe-walt-produced-by-julie-snyder-edited-by-sarah-koenig-neil-drumming-and-ira-glass-editorial-consulting-by-eve-l-ewing-and-rachel-lissy-and-sound-mix-by-stowe-nelson-2}{%
\subsubsection{Reported by Chana Joffe-Walt; produced by Julie Snyder;
edited by Sarah Koenig, Neil Drumming and Ira Glass; editorial
consulting by Eve L. Ewing and Rachel Lissy; and sound mix by Stowe
Nelson}\label{reported-by-chana-joffe-walt-produced-by-julie-snyder-edited-by-sarah-koenig-neil-drumming-and-ira-glass-editorial-consulting-by-eve-l-ewing-and-rachel-lissy-and-sound-mix-by-stowe-nelson-2}}

\hypertarget{white-parents-in-the-1960s-fought-to-be-part-of-a-new-racially-integrated-school-in-brooklyn-so-why-did-their-children-never-attend}{%
\paragraph{White parents in the 1960s fought to be part of a new,
racially integrated school in Brooklyn. So why did their children never
attend?}\label{white-parents-in-the-1960s-fought-to-be-part-of-a-new-racially-integrated-school-in-brooklyn-so-why-did-their-children-never-attend}}

\begin{itemize}
\item
  ``Nice White Parents'' is brought to you by Serial Productions, a New
  York Times Company.
\item
  chana joffe-walt\\
  The New York City Board of Education has an archive of all of its
  records. Everything that goes into making thousands of schools run for
  years and years is sitting in boxes in the municipal building. I love
  the B.O.E. archive.
\item
  chana joffe-walt\\
  Good morning. How are you doing?
\end{itemize}

chana joffe-walt

First of all, to look through it, you have to go to a century-old
municipal building downtown. Arched doorways, lots of marble, an echo,
vaulted ceilings really makes a person feel like she's up to something
important. You sit at a table, and then a librarian rolls your boxes up
to you on a cart. Inside the boxes are all the dramas of a school
system. Big ones, tiny ones, bureaucratic, personal, it's all in there.
There's a union contract and then a zoning plan and special reports on
teacher credentialing, a weird personal note from a bureaucrat to his
assistant, a three-page single-spaced plea from Cindy's grandmother, who
would please like for her not to be held back in the second grade. An
historian friend once pulled a folder out of the archive and a note fell
out, something a teacher clearly made a kid write in the 1950s, that
read, quote, ``I am a lazy boy. Miss Fitzgerald says, when I go in the
army, I will be expendable. Expendable means that the country doesn't
care whether I get killed or not. I do not like to be expendable. I'm
going to do my work and improve.''

{[}music{]}

I came to the Board of Ed archive after I attended the gala thrown by
the French embassy, the fundraiser for SIS organized by the new
upper-class white families coming into the school. I felt like I'd just
watched an unveiling ceremony for a brand-new school, but I didn't
really know what it was replacing. Everyone was talking as if this was
the first time white parents were taking an interest in the School for
International Studies. But at the archive, I found out it wasn't the
first time. White parents had invested in the school before, way before,
at the very beginning of the school. Before the beginning. I found a
folder labeled I.S. 293, Intermediate School 293, the original name for
SIS. And this folder was filled with personal letters to the president
of the New York City Board of Education, a man named Max Rubin, pleading
with him to please make I.S. 293 an integrated school. ``Dear Mr. Rubin,
my husband and I were educated in public schools, and we very much want
for our children to have this experience. However, we also want them to
attend a school which will give them a good education, and today, that
is synonymous with an integrated school.'' ``Dear Mr. Rubin, as a
resident of Cobble Hill, a teacher and a parent, I want my child to
attend schools which are desegregated. I do not want her to be in a
situation in which she will be a member of a small, white, middle-income
clique.'' These are letters from parents --- largely white parents, as
far as I could tell --- written in 1963, just a few years before I.S.
293 was built. At issue was where the school was going to be built. The
Board of Education was proposing to build the school right next to some
housing projects. The school would be almost entirely Black and Puerto
Rican. These parents, white parents, came in and said, no, no, no, don't
build it there. Put it closer to the white neighborhood. That way, all
our kids can go to school together. These parents wanted the school
built in what was known as a fringe zone. This was a popular idea at the
time, fringe schools to promote school integration. Comes up in the
letters. ``Dear Mr. Rubin, this neighborhood is changing with the influx
of a middle-class group which is very interested in public education for
their children.'' ``Dear Mr. Rubin, if there is a possibility of
achieving some degree of integration, it is more likely if the Board of
Education's theory of fringe schools is applied.'' And from another
letter, ``it is apparent from the opinion of the neighborhood groups
involved that the situation is not at all hopeless.'' This lobbying
effort was so successful that the Board of Education did move the site
of the school. This is why SIS is located where it is today, on the
fringe, closer to the white side of town, so that it would be
integrated.

I tried to imagine who these people were --- young, idealistic white
parents living in Brooklyn in the 1960s, feeling good about the future.
They would have had their children around the time the Supreme Court
ruled on Brown versus Board of Education. They probably followed the
news of the Civil Rights Movement unfolding down South. Maybe they were
supporters or active in the movement themselves. These were white
parents saying, we understand we're at a turning point and we have a
choice to make right now, and we choose integration. One of my favorite
letters was from a couple who left the suburbs to come to New York City
for integration, the opposite of white flight. ``Dear Mr. Rubin, we have
recently moved into the home we purchased at the above address in Cobble
Hill. It was our hope in moving into the neighborhood that our children
would enjoy the advantages of mixing freely with children of other
classes and races, which we were not able to provide to them when we
lived in a Westchester suburb.''

\begin{itemize}
\item
  chana joffe-walt\\
  So this is the letter.
\item
  carol netzer\\
  This is the letter that I wrote? I can't believe it. OK.
\end{itemize}

chana joffe-walt

This is Carol Netzer. Most of the letter writers were not that hard to
find.

\begin{itemize}
\tightlist
\item
  carol netzer\\
  We had moved to Scarsdale for the children, because Scarsdale has the
  best --- it probably still does --- the best school system in the
  country, but we hated it. We found that we were bored to death with
  it. It was bland. It was just homogeneous. But living --- I don't know
  if you've ever lived in a suburb. It's just boring, tedious, you know?
  There's nothing going on.
\end{itemize}

chana joffe-walt

She didn't like the suburbs. So they moved to Brooklyn and wrote that
letter, which I showed her, her 37-year-old self writing about her hopes
for her young children, the choices she made back then.

\begin{itemize}
\tightlist
\item
  carol netzer\\
  But it sounds as though I was fairly impassioned about it. You know,
  that it meant something. But I --- actually, I can't think what it
  meant.
\end{itemize}

{[}music{]}

chana joffe-walt

I went through this box of letters and called as many parents as I
could. Most of them didn't remember writing these letters, which isn't
surprising, more than 50 years ago and all. What I did find surprising
is that, by the time 293 opened, five years later, none of them, not a
one, actually sent their kids to I.S. 293.

{[}music{]}

From Serial Productions, I'm Chana Joffe-Walt. This is ``Nice White
Parents,'' a series about the 60-year relationship between white parents
and the public school down the block, a relationship that began with a
commitment to integration. In the 1960s, much like today, white people
were surrounded by a movement for the civil rights of Black Americans.
White people were forced to contend with systemic racism. And here was a
group of white parents who supported the movement for school
integration, threw their weight behind it. What happened in those five
years between 1963, when these white parents planted an impassioned
pro-integration flag on the school, and 1968, when it came time to
enroll their children? Why didn't they show up?

These white parents who wanted an integrated I.S. 293, they didn't come
to that idea on their own. They were part of a bigger story unfolding
around them. I want to zoom out to that dramatic story because it takes
us right up to the moment these parents wrote their letters, and then
made the decision not to send their kids to the school. To begin, I'd
like to introduce you to our main character in this historical, tale,
the recipient of the parents' letters, the New York City Board of
Education. Back in the 1950s, the New York City Board of Ed was not one
of those boring bureaucracies that chugs along in the background,
keeping its head down. It had personality. It invested in self-image.
For instance, in 1954, when the Supreme Court found school segregation
unconstitutional, New York City didn't just say we support that ruling,
it celebrated the Brown v Board decision. And notably, it celebrated
itself, calling Brown, quote, ``a moral reaffirmation of our fundamental
educational principles.'' That same year, 1954, the New York City Board
of Ed made a film honoring multiculturalism in its schools. {[}CHILDREN
SINGING{]} The film opens with a multiracial choir of schoolchildren
singing ``Let Us Break Bread Together.'' Like I said, the Board of Ed
went the extra mile. The Schools Superintendent was a 66-year-old man
named Dr. William Jansen, a man that newspapers described as slow and
steady. And he definitely delivers on that promise here.

\begin{itemize}
\tightlist
\item
  archived recording (william jansen)\\
  The film you're about to see tells the story of how the schools and
  community are working together to build brotherhood.
\end{itemize}

chana joffe-walt

A teacher addresses her classroom, filled with children of all races and
ethnicities.

\begin{itemize}
\tightlist
\item
  archived recording\\
  Who among you can give some of the reasons why people left their
  native lands to come to the United States of America?
\end{itemize}

chana joffe-walt

The camera cuts to a white boy, maybe 9 or 10.

\begin{itemize}
\tightlist
\item
  archived recording\\
  Some came because they wanted to get away from the tyranny and cruelty
  of kings.
\end{itemize}

chana joffe-walt

Then a Black girl, around the same age.

\begin{itemize}
\tightlist
\item
  archived recording\\
  My people are free now. They are proud to be American. But the Negroes
  were brought here by wicked men who traded in slaves.
\end{itemize}

chana joffe-walt

This keeps going, kid to kid.

\begin{itemize}
\tightlist
\item
  archived recording\\
  We came a little while ago from Puerto Rico. My father wanted work. He
  wants to give me and my brother a good education. Japan is very
  overcrowded. The people have little land. So many Japanese came to
  this country because they wanted to farm.
\end{itemize}

chana joffe-walt

New York City was the biggest city in America, with the largest Black
population in America, and it was saying in films, press releases,
public speeches, Brown v Board, we agree. Separate but equal has no
place in the field of public education. No problem here. It was also
saying, you know who does have a problem? The South. New York City loved
comparing itself to the backward South. There are plenty of examples of
this in the Board archives, New Yorkers bragging about their superiority
to places like Georgia or Virginia or Louisiana. This was the story the
Board of Ed was telling. The South was ignorant and racist. New York
City was enlightened and integrated. But here is what it was actually
like to walk into a New York City school in a Black neighborhood at this
time.

\begin{itemize}
\tightlist
\item
  archived recording (mae mallory)\\
  The school had an awful smell. It was just --- oh, it smelled like
  this county abattoir.
\end{itemize}

chana joffe-walt

This is an archival recording of a woman named Mae Mallory. In the
1950s, Mallory's two Black children were students in Harlem. And when
Mallory walked into their school, she did not see children building
brotherhood in interracial classrooms. She saw an all-Black and Puerto
Rican school with terrible facilities, in disrepair.

\begin{itemize}
\tightlist
\item
  archived recording (mae mallory)\\
  So my kids told me, says, well, Mommy, this is what we've been trying
  to tell you all along, that this place is so dirty. And this is why we
  run home to the bathroom every night. So I went to the bathroom. And
  in 1957 in New York City, they had toilets that were worse than the
  toilets in the schools that I went to in Macon, Georgia in the heart
  of the South. The toilet was a thing that looked like horse stalls.
  And then it had one long board with holes cut in it. And then you'd
  have to go and use the toilet, but you couldn't flush it. The water
  would come down periodically and flush, you know, whatever's there.
  Now imagine what this is like, you know, dumping waste on top of waste
  that's sitting there waiting, you know, accumulating till the water
  comes. This was why this place smelled so bad.
\end{itemize}

chana joffe-walt

Mae Mallory says the school had two bathrooms for 1,600 children.
Mallory's family fled racial violence in the South, like millions of
other Black Americans, who headed to places like New York City, where
everyone was supposed to be equal. Instead of welcoming these new
students and spreading them out, creating interracial classrooms, the
Board of Education kept Black and Puerto Rican students segregated in
what were sometimes referred to as ghetto schools, schools that were
often just blocks away from white schools. White schools in New York
City had toilets that flushed. White children had classrooms with
experienced teachers and principals, people who lived in their
communities and looked like them. In Black and Puerto Rican schools,
half the teachers were not certified to teach by the Board of Education.
The buildings were in disrepair, and packed, sometimes more than 1,000
kids in a single hallway. The overcrowding got so bad the Board of
Education decided to send kids to school in shifts. And mind you, this
was not in the middle of a global pandemic. This was normal, non-crisis
school for Black and Puerto Rican kids. One group of children would go
to school in the morning until noon. The next group of kids would come
in at noon, and stay until 3:00. The Board was literally giving Black
kids half an education. In some schools in Harlem, they had triple
shifts. This made it harder to learn elementary skills. Reading, for
instance. Black parents complained that the schools were not teaching
their kids basic literacy, that their white teachers didn't care, that
the summer reading programs were only in white communities, that their
children were two years behind white children in reading. This at
exactly the same time the Board of Education was making a film promoting
the virtues of integration. It was effectively running a dual,
segregated and unequal school system.

{[}music{]}

For many Black families, the Board of Education was not to be trusted.
It did not care for Black children, and it didn't respect the voices and
concerns of Black parents. Mae Mallory says she visited her kids' school
that day because they'd come home the day before and told her a child
had died at school. He was playing in the street at recess. Mallory
hardly believed it, but she says when she visited the school, she
learned, yes, indeed, this child was playing the street because the
schoolyard was closed. He was hit by a beer truck. And she learned the
schoolyard was closed because pieces of steel from the side of the
building had fallen into the yard.

\begin{itemize}
\tightlist
\item
  archived recording (mae mallory)\\
  And when I found out that this was true, I went to the principal. So
  this principal told me that, well, Mrs. Mallory, you really don't have
  anything to worry about. You see, our sunshine club went to see the
  mother, and we took her a bag of canned goods. So actually, she's
  better off, because she had so many children to feed. And I couldn't
  believe that here a white man is going to tell a Black woman in Harlem
  that a can of peaches is better than your child. I just didn't know
  what to do or where to go. But I know you're supposed to do something.
\end{itemize}

chana joffe-walt

It was 1957, three years after the Supreme Court declared segregation by
law unconstitutional. New York City didn't have Jim Crow laws on the
books, but Mae Mallory would ask, the schools are segregated. What's the
difference? She didn't care whether that segregation was codified by law
or by convention. The harm was just as dire. And she wanted it
addressed.

\begin{itemize}
\tightlist
\item
  archived recording (mae mallory)\\
  This was nothing to do with wanting to sit next to white folks. But it
  was obvious that a whole pattern of Black retardation was the program
  of the Board of Education. So I filed a suit against the Board of
  Education. And I just fought back.
\end{itemize}

chana joffe-walt

Integration, Mae Mallory would say, was about, quote, demanding a fair
share of the pie. She said, our children want to learn, and they
certainly have the ability to learn. What they need is the opportunity.
The Board of Education had defined integration as a multiracial choir.
It was a virtue in and of itself. Mae Mallory saw integration as a
remedy, a way to get the same stuff everyone else had --- functioning
toilets, books, certified teachers, a full school day. Integration was a
means to an end.

{[}music{]}

Mae Mallory won her lawsuit. She and a few other parents were allowed to
transfer their kids out of segregated schools. As for the segregation in
the entire system, the judge in the lawsuit turned to the Board of Ed
and said, this segregation, it's your responsibility. Fix it.

Now, on the question of responsibility, the Board of Education was
cagey. And that caginess set the stage for the I.S. 293 parents when it
came time to send their kids to the school. Here's what happened. The
Schools Superintendent, William Jansen, decided school segregation was
not his problem. In fact, he rejected the idea that New York City had
segregated schools in the first place. After all, New York City was not
barring Black children from entering white schools. This wasn't the
South. Segregation, Jansen said, is such an unfortunate word. He
preferred the phrase racial imbalance or racial separation. The way he
saw it, racial imbalance in the schools was just a matter of housing.
Neighborhoods were segregated. Again, unfortunate, but that had nothing
to do with the schools. To make this argument, William Jansen had to
ignore the many powerful tools available to the Board of Education. The
Board of Education was responsible for where kids went to school. It
decided where to build new schools. It drew zoning lines. It decided
where experienced teachers teach. There were many ways the Board could
have made schools less segregated. I know this because of the Board's
own reports. Jansen did very little to break up school segregation, but
man, did he study it. He organized commissions that led to reports that
led to further study. You see a pattern emerge, starting in the late
1950s, that looks something like this. Black parents and civil rights
groups would pressure the Board to act on segregation. The Board would
invite its critics to join a commission to investigate the problem. The
commission would study the schools, discover extreme segregation, lay
out solutions. The Board of Ed would then take a tiny step toward
implementing some of the recommendations until white parents started to
complain about the changes, at which point the Board would back off and
say it needed more evidence. Another commission, another report. For
instance, there's the Report on the Committee on Integration, a Plan for
Integration, the City's Children and the Challenge of Racial
Discrimination, Redoubling Efforts on Integration, the Board Commission
on Integration, the Status of the Public School Education of Negro and
Puerto Rican Children in New York City, and, my favorite, a bound little
red book from 1960 called Toward Greater Opportunity, which summarizes
the previous investigations with this groundbreaking conclusion. Quote,
``we must integrate as much and as quickly as we can.'' I want to pause
for one second and step out of the past back into the world we all live
in, just to point out that, over the last few years in New York City,
we've been reliving this chapter of history. It's eerie. New York City
schools are segregated. There's a growing movement to do something about
that. And for the first five years of his administration, the city's
mayor, Mayor Bill de Blasio, responded in the following way. He refused
to say the word segregation, commissioned a number of reports on school
diversity. He's pointed a finger at housing problems as a way to say
this isn't our fault, and he's studying the problem deeply, which,
again, is not segregation, no matter how many times reporters would ask
the mayor at press conferences, why don't you use that word?

\begin{itemize}
\tightlist
\item
  archived recording (bill de blasio)\\
  I don't get lost in terminology. I think the notion of saying we have
  to diversify our schools is the best way to say it.
\end{itemize}

chana joffe-walt

I heard a live call-in show on WNYC, the public radio station. A young
integration advocate, an 11th grader named Tiffani Torres, asked the
mayor, how much longer until you do something?

\begin{itemize}
\item
  archived recording (tiffani torres)\\
  And how much more time do you need to study the issue? So to repeat my
  question, how much longer will it take?
\item
  archived recording (bill de blasio)\\
  Tiffani, with all due respect, I really think you're not hearing what
  we're saying to you, so I'll repeat it. There is a task force, an
  extraordinary task force, which I've met with. They are coming forward
  with their next report in a matter of weeks. So when that diversity
  task force comes out with their report, I think they're amazing. I
  think they've done fantastic work. And so far, there's a high level
  ---
\end{itemize}

chana joffe-walt

Mayor de Blasio likes to point out that this was a problem created by
people long before him, which is exactly what people long before him
said, too.

{[}music{]}

In the late 1950s, when Black parents and civil rights activists also
asked the Board of Ed, why is it taking so long, board members
complained about the, quote, extremists who wanted instant integration.
Superintendent Jansen said, ``some people want us to build Rome in one
day.'' While the Board of Education was building Rome in 1956, `57, `59,
and in 1960, 1962, `63, Black parents found each other on PTAs, in civil
rights organizations, pro-integration groups. They formed new groups,
organized sit-ins, boycotts, demanded the Board provide a timetable for
citywide integration. They joined forces with Puerto Rican parents, and
their numbers grew. These were volunteers, mothers mostly, who left
their jobs at the end of a workday and headed directly to a meeting
about how to get the Board to give their kids the education white
children were already receiving. Finally, in 1964, 10 years after Brown
versus Board, Black and Puerto Rican parents said, enough. They were
sick of waiting, sick of lawsuits, sick of asking for a remedy, sick of
being ignored. So they went big, spectacularly big. They shut down the
schools. They organized a civil rights demonstration that was the
largest in US history, larger than the March on Washington. It was
called Freedom Day, a massive school boycott.

\begin{itemize}
\tightlist
\item
  archived recording\\
  (CHANTING) Freedom now!
\end{itemize}

chana joffe-walt

On February 3rd, 1964, parents headed out to schools in the morning
before sunrise to spread the word about the boycott. It was freezing
cold that day. There's a brief TV news clip of a group of mothers
picketing outside their kids' school at the start of the school day.
They're holding up signs that say, ``we demand a real integration
timetable now,'' and ``integration means better schools for all.''
They're handing out leaflets to other parents about Freedom Day, looking
spirited and cold. A white NBC news reporter in a fedora walks up to one
of the women.

\begin{itemize}
\tightlist
\item
  archived recording\\
  Ma'am, it's a little after eight o'clock now. How successful has the
  boycott been so far? Very effective. So far, about 10 children have
  gone in, and there would be ordinarily 240 children. And 10 have gone
  into the morning session, which begins at eight o'clock. So you think
  you've already seen the result? Yes, I think so. The school is just
  empty. Does it surprise you? No, because we knew how effective --- We
  talked with the parents. We distributed leaflets. We've been working
  very hard. And we prayed that it would be effective.
\end{itemize}

chana joffe-walt

There were maps and charts and instructions with picket times and picket
captains for hundreds of schools. There were volunteer shifts to make
peanut butter and jelly sandwiches, to hand out thousands of leaflets
and stencil posters. The boycott wasn't just effective --- it was
extraordinarily effective. Half a million kids stayed home from school
that day. Half a million, close to half the school system. But the press
barely covered it. After searching every major TV network, I found only
one kid who was interviewed, a teenage boy, maybe around 16, on the
street with some friends, protesting. A white ABC News reporter doesn't
ask him why he's there. The only thing he asks him about is violence.
The kid responds.

\begin{itemize}
\tightlist
\item
  archived recording\\
  We're coming down here today for a peaceful --- peaceful--- No
  comment! No, we're not going to be violent. We're just teenagers and
  kids. And --- Do you expect violence here today? No, sir, not if ---
  look at the blue uniforms. You ask me do I expect violence.
\end{itemize}

chana joffe-walt

He gestures to the police on horseback.

\begin{itemize}
\tightlist
\item
  archived recording\\
  None of us have any weapons, horses. And all we want is equal
  education. That's all. Equal education. Thank you. You get all that?
\end{itemize}

chana joffe-walt

That was it. Every once in a while, I'll hear a politician or friend or
school administrator say, yeah, integration was a good idea, but there
was no political will to make it happen. 460,000 kids, half the school
system. The will was there. The majority wanted integration.

{[}music{]}

After Freedom Day, the Board of Education introduced some small-scale
integration plans, and white parents protested. {[}CHILDREN SINGING{]}
We love our children. Oh, yes, we do. We will not transfer ---

chana joffe-walt

With their own marches, they put on their own school boycott. The
flipside of freedom day, a white boycott. The white parents were far
fewer in number. But as far as I can tell, they got a thousand times
more press coverage.

\begin{itemize}
\item
  archived recording\\
  Mrs. Carcevski?
\item
  speaker\\
  Yes? Are you going to send Johnny back to school now? No. She belongs
  here, and I want to send my child here. So nobody is going to tell me
  where to send my kid.
\end{itemize}

chana joffe-walt

This protest worked. The Board of Ed backed off. And in the decades
since, the Board of Education has never proposed a city-wide integration
plan. The schools have never been integrated. I think the fact of white
moms in Queens in the 1960s yelling about zoning changes and busing,
it's not surprising they played a role in killing school integration
efforts. But there was another group of white parents who played a
quieter, but I'd argue more forceful, role in killing integration. The
white parents who said they supported it, parents like the ones who
wrote letters asking for an integrated I.S. 293. How did their vocal
support for integration turn lethal? That's after the break. In the
American South, schools were desegregated with court orders. Cities and
counties mandated desegregation, and the schools desegregated. By the
early 1970s, the South was the most integrated region in the country.
But New York City did not want to do it that way. No mandates. The New
York City Board of Education wanted to appeal to hearts and minds. They
wanted to sell white people on the virtues of integration. Have it all
happen, quote, ``naturally.'' Some white people were sold. The white
parents who wrote letters about I.S. 293. They believed in integration.
So I made a lot of calls to ask, why'd you bail? They had a lot of
different reasons. One couple got divorced, and moved. Another guy told
me he had political ambitions that pulled him out of the city.

\begin{itemize}
\tightlist
\item
  speaker\\
  We loved our brownstone, but I was involved in a political race. And
  we needed some money for that.
\end{itemize}

chana joffe-walt

So he sold the house and moved the family to the suburbs, where he
thought he'd have a better chance running against Republicans. Many
white people moved to the suburbs for jobs, for newly paved roads and
subsidized mortgages, leaving Brooklyn behind. I understood what
happened there. But some explanations made less sense. Like one guy I
called, he did stay in Brooklyn. On the phone, he was telling me why he
believed it was important that I.S. 293 be integrated. But then he said
his own kids went to Brooklyn Friends, a Quaker private school. I said,
oh, they didn't go to I.S. 293.

\begin{itemize}
\tightlist
\item
  speaker\\
  No. As I said, I'm a Quaker, and ---
\end{itemize}

chana joffe-walt

But you were a Quaker when you wrote this letter, asking for an
integrated 293.

\begin{itemize}
\tightlist
\item
  speaker\\
  I believed it. I believed in it, but ---
\end{itemize}

chana joffe-walt

You weren't planning to send your kid there?

\begin{itemize}
\tightlist
\item
  speaker\\
  No, no, no.
\end{itemize}

chana joffe-walt

What to make of that? When you get what you say you want and then, given
the opportunity, don't take it. Maybe you never really wanted it in the
first place. Then I spoke to Elaine Hencke. Of all the people I spoke
with, everything about Elaine indicated someone who did believe in
integration, someone who would send her kids to 293. And yet, she
didn't. Elaine was a public school teacher. She taught in an integrated
elementary school, until she had her own kids. She was looking forward
to sending them to an integrated 293. When her daughter was old enough
for junior high school, Elaine visited the school. She was the only
letter writer I spoke with who actually went into the building. If this
was going to work with anyone, it was going to be Elaine.

\begin{itemize}
\tightlist
\item
  elaine hencke\\
  I didn't know quite what to make of it because the school had a nice
  plant. Physically, it was a nice school. But it just seemed chaotic
  and noisy, and kids were disruptive. And kids --- {[}LAUGHS{]} ---
  kids were doing the wrong things, you know? And kids do. I mean, it
  wasn't that they were nasty kids or doing --- it was not drugs. It was
  not drugs. It was just --- it just seemed too chaotic to me at the
  time.
\end{itemize}

chana joffe-walt

Elaine and I talked for a long time I pushed her --- not to make her
feel bad, but to get to what felt like a more real answer. At the time
that you are visiting, was it majority Black and Hispanic kids?

\begin{itemize}
\tightlist
\item
  elaine hencke\\
  Yes, I'm sure it was.
\end{itemize}

chana joffe-walt

And did that have anything to do with the way that you saw the classroom
as disruptive and chaotic?

\begin{itemize}
\item
  elaine hencke\\
  I would hope not.

  I'm not --- I'm not sure how well educated they were, or --- you know,
  I don't know. I don't know why I'm going into this.
\end{itemize}

chana joffe-walt

Well, did you have reason to think that they weren't well educated?

\begin{itemize}
\tightlist
\item
  elaine hencke\\
  Before 293? Well, their reading levels were way down. You know.
\end{itemize}

chana joffe-walt

I'm just --- when you say chaos and disruptive, I'm trusting that what
you saw was chaotic and disruptive. But I also know that those are words
white people use --- we use to express our racial fears, to express real
racial fears. Do you think that's what was happening with you?

\begin{itemize}
\tightlist
\item
  elaine hencke\\
  I don't think I would admit to that. I don't think that was true. But
  what I may have thought was that these kids are not expected to do so
  well in school, all the way from the beginning of school. And here
  they are, really unprepared in some way, for junior high school or ---
  I mean, the reading levels were low.
\end{itemize}

chana joffe-walt

Elaine told me when she wrote that letter to the Board of Education, she
pictured her children becoming friends with Black kids, learning
side-by-side, learning that all children are equal. That's what
motivated her to write that letter. She wanted the picture of
integration the Board of Ed was promoting --- the picture of harmonious
integration. But when she visited I.S. 293, that didn't seem possible.
The reading levels were low. The kids were not entering the school on
equal grounds. Her white children had received years of high quality
teaching at well-resourced schools. The kids coming from segregated
elementary schools had not had that experience.

\begin{itemize}
\tightlist
\item
  elaine hencke\\
  I mean, one of the problems is that many of the white kids had higher
  sort of academic skills, or skills. They could read better. I think
  --- I mean, if the white kids knew how to read in first grade and ---
  and I guess there were Black kids who also could. But it just seemed
  as if most of the black kids didn't really learn --- learn to read.
\end{itemize}

chana joffe-walt

But part of the --- part of the vocal complaints of black parents at
this period of time was that their kids were not learning how to read
because schools were segregated, and their kids were kept in schools
that were inferior. And that was part of the argument for integration.

\begin{itemize}
\tightlist
\item
  elaine hencke\\
  Yes, yes.
\end{itemize}

chana joffe-walt

That their kids were not going to get the resources, and quality
teaching, and good facilities unless they were in the same buildings
with kids like yours.

\begin{itemize}
\item
  elaine hencke\\
  Right.

  I don't know what to say to that. I just --- I guess I just began to
  feel that things were really difficult for these kids. Schools were
  not made for them. If the schools were made for them, with their
  background, what would they be like?

  I think there was --- and that's another whole thing. I don't know
  about it. I think there was sort of anger in the black community at
  the white community. A lot of the teachers were white. There were more
  white teachers, I suppose. People said that that was racism. And of
  course, it was racism. But maybe the kids were a little angry at the
  school. I wouldn't --- I couldn't fault them for that. But on the
  other hand, then they don't get as much from the school. I don't know.
  I thought the problems were kind of enormous. And I guess I just, at
  one point, I just decided that my kids should go --- went to Brooklyn
  Friends. And we could afford to pay for it. It wasn't easy, you know.
  It was --- {[}LAUGHS{]} but ---
\end{itemize}

chana joffe-walt

Did your feelings about integration change? Did you believe in it less?

\begin{itemize}
\item
  elaine hencke\\
  Maybe.

  I think I would have said no, theoretically. But maybe they did. I
  guess I saw it as a more difficult project then. I sort of did back
  off from it. I just ---
\end{itemize}

chana joffe-walt

Yeah. It felt when you guys wrote these letters like, this is ---
integration is this exciting ideal, and we can be part of it, and it's
going to be a meaningful project that's also going to be kind of easy.

\begin{itemize}
\tightlist
\item
  elaine hencke\\
  I certainly didn't think it would be so difficult. But I --- I was, I
  was innocent, you know? I don't know. I still believe in it. I do.
\end{itemize}

{[}music{]}

chana joffe-walt

I think what Elaine actually meant was not that she was innocent, but
that she was naive. She was naive about the reality of segregation, the
harm of it. And naive about what it would take to undo it. She did not
know. And I think she didn't want to know. When Elaine said the word
innocent, I felt a jolt of recognition. I felt like Elaine had walked me
right up to the truth about her, and about me.

When my own kids were old enough, I sent them to our zoned public
school. It was racially mixed and economically mixed. I was excited
about that. And it was nice walking to school with neighbors, people I
likely never would have gotten to know otherwise. My kid's first day of
school was another boy's first week in the country. He'd just moved from
China, and his mom asked the neighbor where the school was. When she
said goodbye that first morning, I think he thought I was a teacher, and
he crawled into my lap. We had no words in common, so I just held him
while he screamed and cried. By the holiday show three months later, I
watched that same boy belt out ``This Pretty Planet'' on a stage with
his classmates. He was the star. He nailed the hand motions. Every other
kid up on stage was just following his lead, just trying to keep up. It
was such a sweet picture, all of them up there --- Black kids, and
Mexican kids, and Colombian, and Asian and white kids. And all of us
adults supporting all of them. It's moving, to me, this picture of
integration. It is also, I'm realizing right now, writing these words
down, the very same picture the Board of Education put forth in 1954 ---
a multiracial choir singing together, building brotherhood. And it's
dangerous, I think, this picture of integration. It seems perfectly
designed to preserve my innocence, to make me comfortable, not to remedy
inequality, but a way to bypass it entirely. I can sit in that assembly
and feel good about the gauzy display of integration without ever being
asked to think about the fact that much of the time, white kids in the
school building are having a different educational experience than kids
of color. A large share of the white students at the school are
clustered in a gifted program. They have separate classrooms and
separate teachers. We all blithely call these white children gifted and
talented, G\&T, starting at four years old. White children are
performing better at the school than black children and Latino children.
White families are the loudest and most powerful voices in the building.
The advantages white kids had back in the 1950s, they're still in place.
When Elaine said she was innocent, I thought about the things we say,
nice, white parents, to each other about why we won't send our kids to
segregated schools --- because they're too strict, or too chaotic, or
too disruptive. Because the test scores are bad, because we want more
play. We want fewer worksheets. Because we don't want to ride a bus. We
don't want uniforms. We don't want tests. We want innocence. We need it,
to protect us from the reality that we are the ones creating the
segregation, and we're not sure we're ready to give it up.

{[}music{]}

Elaine was not for segregation. But in the end, she wasn't really for
integration, either. All of the choices she made, choices she had the
luxury of making, were meant to advantage her own kids. And I understand
that. That's what parents do.

\begin{itemize}
\tightlist
\item
  elaine\\
  I remember thinking very clearly that OK, I believe in this. But I
  don't sort of want to sacrifice my children to it. I have to look at
  what they will learn, and what they will do. And for people who sent
  their kids to 293, it seemed to work out well. So that made me think,
  well, maybe I made a mistake. Maybe they should have gone there. I
  know at one point it was very clear to me that I had beliefs that I
  thought were kind of contrary to my own children's best interests. And
  I decided that I wasn't going to use them to sort of extend my own
  beliefs. But then I regretted that, because that wasn't really true.
\end{itemize}

chana joffe-walt

You regretted what?

\begin{itemize}
\tightlist
\item
  elaine\\
  Well, I kind of wish I had sent them to 293 because Joan's kids had a
  good experience there.
\end{itemize}

chana joffe-walt

Elaine's friend Joan, another white mom who did send her kids to I.S.
293. Elaine still feels bad about her choice. But not everyone felt bad.

\begin{itemize}
\tightlist
\item
  carol netzer\\
  We were not pious, kind of, oh, the kids have to go to public school.
  Not at all. I went to public schools, and there's nothing to write
  about.
\end{itemize}

chana joffe-walt

Carol is the woman who wrote the letter about how she'd come to New York
City from the suburbs for integration. I had a hard time reconciling her
lack of piety with her letter, which I read back to her, about wanting
her kids to mix freely with children of other classes and races.
{[}READING{]} --- which we were not able to provide for them when we
lived in the Westchester suburb.

\begin{itemize}
\tightlist
\item
  carol netzer\\
  That was all true. Yeah, yeah.
\end{itemize}

chana joffe-walt

You remember feeling that way?

\begin{itemize}
\tightlist
\item
  carol netzer\\
  Well, I don't really remember feeling that way. And I think that we
  say a lot of things that are politically correct, without even
  realizing that we are not telling exactly how we feel. So I can't
  really guarantee that it was 100\% the way I felt. I don't really
  remember. Probably close to it, but I mean, I'm a liberal, you know?
\end{itemize}

chana joffe-walt

As a parent, did you --- do you remember feeling like, I hope my kid has
experiences outside of just people like them?

\begin{itemize}
\tightlist
\item
  carol netzer\\
  Not especially. I mean, we rushed right away to send them to private
  school, right? So what was most important to us was that they get the
  best education. But one of the things that changed it was St. Anne's
  School, a sort of progressive school with this man, headmaster, who
  was brilliant. Opened up St. Anne's. And if you keep working on this,
  you'll hear a lot about St. Anne's.
\end{itemize}

chana joffe-walt

I'm not going to tell you a lot about St. Anne's, except to say this ---
it's one of the most prominent private schools in Brooklyn. Upscale
neighborhood, prime real estate, lots of heavy-hitters send their kids
to St. Anne's. I had heard of it. What I didn't know is that St. Anne's
opened at the very same time that Black parents were waging their
strongest fight for integration in New York City, in 1965. Right when a
lot of the letter writers would have been looking for schools. And it
wasn't just St. Anne's. New progressive private schools were opening and
expanding all over the city. Brooklyn Friends School expanded into a new
building, and would double its enrollment. They were opening private
schools in the South, too. But down there, it was all very explicit.
They became known as quote, unquote, ``segregation academies,'' schools
for white people who were wholeheartedly committed to avoiding
integration. In the North, private schools opened as if they were
completely disconnected from everything else that was happening at that
very moment. St. Anne's marketed itself as a pioneer, a community of
like-minded, gifted kids, no grades. Lots of talk about progressive,
child-centered education, the whole child. At one point in my
conversation with Carol Netzer I was talking about how integration was
happening around his time. And she surprised me by saying, no, not at
that time.

\begin{itemize}
\tightlist
\item
  carol netzer\\
  I think the --- I think that you may be off on the timing for me,
  because it was too early. They didn't start really any kind of crusade
  about integrating until well after I had left the neighborhood.
\end{itemize}

chana joffe-walt

No, they were integrating the schools in the `60s, though.

\begin{itemize}
\tightlist
\item
  carol netzer\\
  Oh. It didn't make much of a splash. We weren't against it. There was
  --- it wasn't a big item.
\end{itemize}

chana joffe-walt

That's how easy it was to walk away from integration in New York City.
You could do it without even knowing you'd thrown a bomb over your
shoulder on the way out.

{[}music{]}

Here is what I think happened over those five years between the writing
of the letters in 1963 and not sending their kids to the school in 1968.
Those five years were a battle between the Board of Education's
definition of integration and the actual integration that black parents
wanted. For black parents, integration was about safe schools for their
children, with qualified teachers and functioning toilets, a full day of
school. For them, integration was a remedy for injustice. The Board of
Ed, though, took that definition and retooled it. Integration wasn't a
means to an end. It was about racial harmony and diversity. The Board
spun integration into a virtue that white parents could feel good about.
And their side triumphed. That's the definition of integration that
stuck, that's still with us today. It's the version of integration that
was being celebrated 50 years later, at the French Cultural Services
Building at the Gala for SAS.

In some of my calls with the white letter writers, a few people
mentioned that yes, they wanted integration. But also, they wanted the
school closer to them. They weren't comfortable sending their kids over
to the other side of the neighborhood. Which brings me to one final
letter from the other side of the neighborhood. One I haven't told you
about, from the I.S. 293 folder in the archives. It's one of the only
letters, as far as I can tell, that is not from a white parent. It's
from the Tenants Association for the Gowanus Houses, a housing project,
home to mostly Black and Puerto Rican families. They also wanted a
school closer to them. The letter from the Tenants Association is formal
and straightforward. It says, please build the school on the original
site you proposed, right next to the projects. That way, they explained,
our kids won't have to cross many streets. We'll get recreational
facilities, which we desperately need. And it'll be close to the people
who will actually use it. The letter says they represent over 1,000
families. The white families, they numbered a couple dozen. Still, in
the name of integration, the white letter writers got what they wanted
--- a new building close to where they lived, that they did not attend.
Note the Black and Puerto Rican families we're not asking to share a
school with white people. They were not seeking integration. That's not
what their letter was about. They were asking for a school, period. The
school they got was three blocks further than they wanted. And from the
moment it opened, I.S. 293 was de facto segregated --- an overwhelmingly
Black and Puerto Rican school. What were those years like, once the
white parents pushing their priorities went away? Once there were no
more efforts at feel-good integration, and the community was finally
left alone? Was that better? That's next time, on ``Nice White
Parents.''

``Nice White Parents'' is produced by Julie Snyder and me, with editing
on this episode from Sarah Koenig, Nancy Updike and Ira Glass. Neil
Drumming is our Managing Editor. Eve Ewing and Rachel Lissy are our
editorial consultants. Fact-checking and research by Ben Phelan, with
additional research from Lilly Sullivan. Archival research by Rebecca
Kent. Music supervision and mixing by Stowe Nelson. Our Director of
Operations is Seth Lind. Julie Whitaker is our Digital Manager. Finance
management by Cassie Howley and production management by Frances
Swanson. The original music for Nice White Parents is by The Bad Plus,
with additional music written and performed by Matt McGinley. A thank
you to all the people and organizations who helped provide archival
sound for this episode, including the Moorland-Spingarn Research Center,
Andy Lanset at WNYC, Ruta Abolins and the Walter J. Brown Media Archives
at the University of Georgia and David Ment, Dwight Johnson and all the
other people at the Board of Education archives. Special thanks to
Francine Almash, Jeanne Theoharis, Matt Delmont, Paula Marie Seniors,
Ashley Farmer, Sherrilyn Ifill, Monifa Edwards, Charles Isaacs, Noliwe
Rooks, Jerald Podair and Judith Kafka.

``Nice White Parents'' is produced by Serial Productions, a New York
Times Company.

\includegraphics{https://static01.nyt.com/images/2020/07/23/video/23NWP-IMAGE/23NWP-IMAGE-articleLarge.jpg?quality=75\&auto=webp\&disable=upscale}

``Nice White Parents'' was reported by Chana Joffe-Walt; produced by
Julie Snyder; edited by Sarah Koenig, Neil Drumming and Ira Glass;
editorial consulting by Eve L. Ewing and Rachel Lissy; and sound mix by
Stowe Nelson.

The original score for ``Nice White Parents'' was written and performed
by the jazz group The Bad Plus. The band consists of bassist Reid
Anderson, pianist Orrin Evans and drummer Dave King. Additional music
from Matt McGinley.

Special thanks to Sam Dolnick, Julie Whitaker, Seth Lind, Julia Simon
and Lauren Jackson.

Advertisement

\protect\hyperlink{after-bottom}{Continue reading the main story}

\hypertarget{site-index}{%
\subsection{Site Index}\label{site-index}}

\hypertarget{site-information-navigation}{%
\subsection{Site Information
Navigation}\label{site-information-navigation}}

\begin{itemize}
\tightlist
\item
  \href{https://help.nytimes.com/hc/en-us/articles/115014792127-Copyright-notice}{©~2020~The
  New York Times Company}
\end{itemize}

\begin{itemize}
\tightlist
\item
  \href{https://www.nytco.com/}{NYTCo}
\item
  \href{https://help.nytimes.com/hc/en-us/articles/115015385887-Contact-Us}{Contact
  Us}
\item
  \href{https://www.nytco.com/careers/}{Work with us}
\item
  \href{https://nytmediakit.com/}{Advertise}
\item
  \href{http://www.tbrandstudio.com/}{T Brand Studio}
\item
  \href{https://www.nytimes.com/privacy/cookie-policy\#how-do-i-manage-trackers}{Your
  Ad Choices}
\item
  \href{https://www.nytimes.com/privacy}{Privacy}
\item
  \href{https://help.nytimes.com/hc/en-us/articles/115014893428-Terms-of-service}{Terms
  of Service}
\item
  \href{https://help.nytimes.com/hc/en-us/articles/115014893968-Terms-of-sale}{Terms
  of Sale}
\item
  \href{https://spiderbites.nytimes.com}{Site Map}
\item
  \href{https://help.nytimes.com/hc/en-us}{Help}
\item
  \href{https://www.nytimes.com/subscription?campaignId=37WXW}{Subscriptions}
\end{itemize}
