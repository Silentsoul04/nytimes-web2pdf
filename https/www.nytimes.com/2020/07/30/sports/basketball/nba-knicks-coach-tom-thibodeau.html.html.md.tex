Sections

SEARCH

\protect\hyperlink{site-content}{Skip to
content}\protect\hyperlink{site-index}{Skip to site index}

\href{https://www.nytimes.com/section/sports/basketball}{Pro Basketball}

\href{https://myaccount.nytimes.com/auth/login?response_type=cookie\&client_id=vi}{}

\href{https://www.nytimes.com/section/todayspaper}{Today's Paper}

\href{/section/sports/basketball}{Pro Basketball}\textbar{}Thibodeau
Calls Knicks Job a `Dream Come True'

\url{https://nyti.ms/2Xb1Nqz}

\begin{itemize}
\item
\item
\item
\item
\item
\end{itemize}

Advertisement

\protect\hyperlink{after-top}{Continue reading the main story}

Supported by

\protect\hyperlink{after-sponsor}{Continue reading the main story}

\hypertarget{thibodeau-calls-knicks-job-a-dream-come-true}{%
\section{Thibodeau Calls Knicks Job a `Dream Come
True'}\label{thibodeau-calls-knicks-job-a-dream-come-true}}

Tom Thibodeau was announced as the latest coach of the Knicks on
Thursday. Eight different people have filled the role since 2011.

\includegraphics{https://static01.nyt.com/images/2020/07/30/sports/30nba-knicks-1/merlin_175113222_2bb4fcab-d54d-4491-b79e-714031be6065-articleLarge.jpg?quality=75\&auto=webp\&disable=upscale}

By \href{https://www.nytimes.com/by/sopan-deb}{Sopan Deb}

\begin{itemize}
\item
  July 30, 2020
\item
  \begin{itemize}
  \item
  \item
  \item
  \item
  \item
  \end{itemize}
\end{itemize}

There was an exhortation for patience and a newfound optimism for the
future. There were assurances about a disciplined style of play and the
willingness to build a winning culture. And there was a shout out to the
1990s Knicks.

It was not déjà vu. It was yet another introduction of a new Knicks
coach, this time
\href{https://www.nytimes.com/2020/07/25/sports/tom-thibodeau-nearing-agreement-to-become-knicks-coach.html}{Tom
Thibodeau}, a former Coach of the Year Award winner, whose hiring was
announced at a Zoom news conference on Thursday.

``This is a dream come true for me,'' Thibodeau, 62, said in prepared
remarks. ``This is my dream job. I've been there before. I have a great
understanding of New York. We have the best city in the world. We have
the best arena in the world. We have the best fans in the world.''

Thibodeau will become the team's
\href{https://www.basketball-reference.com/teams/NYK/}{eighth coach
since 2011}. He is the latest to try to revive the Knicks, who will miss
the playoffs for the seventh straight year, the longest streak for the
franchise since the 1960s.

The last time the Knicks were a perennial powerhouse, Thibodeau had an
inside view as an assistant coach from 1996 to 2003, including a finals
run, a point he made sure to emphasize to reporters. He thanked Jeff Van
Gundy, the former Knicks coach, for hiring him then and singled out
players like Patrick Ewing, Larry Johnson and Allan Houston, who are
considered near-deities among the Knicks fan base.

``We saw how hard they played. They gave us everything they had each and
every night,'' Thibodeau said.

He was joined in the news conference by Leon Rose, the
\href{https://www.nytimes.com/2020/02/06/sports/basketball/leon-rose-knicks-president.html}{recently
hired Knicks president}, and Scott Perry, the general manager, whom the
team
\href{https://www.nytimes.com/2020/04/29/sports/basketball/scott-perry-knicks.html}{extended
for at least another year}. Rose has had a two-decades long relationship
with Thibodeau, in part because Rose was a
\href{https://www.nytimes.com/2020/06/24/sports/basketball/knicks-world-wide-wes.html}{longtime
agent at Creative Artists Agency}, which represents Thibodeau.

``The relationship gives me a comfort level, knowing Tom for that
long,'' Rose said. He added, ``Tom was really the perfect candidate from
the standpoint of that he's going to demand accountability, he's going
to have development and he's going to create a winning culture.''

Thibodeau said he was drawn to the Knicks because of its roster ---
which he called ``young and talented'' --- and his connections to the
Knicks front office, as well as the team's abundance of draft picks and
cap space. He also said that, as far as on-court play, he would focus on
five pillars: defense, rebounding, low turnovers, having the ball touch
the paint and making the extra pass.

``You go step by step. I think you don't skip over anything,'' Thibodeau
said. ``The first thing is to lay the foundation, develop your plan and
then work the plan.''

While Thibodeau mentioned RJ Barrett and Mitchell Robinson as two
current Knicks players he was fond of, it's not clear, given the team's
draft picks and cap space, what the roster will look like next season,
which Rose acknowledged. Rose said that the Knicks ``had not set a
timeline'' for when the team would be a playoff contender again.

``We felt that Tom was that coach that can take us from development to
becoming a perennial winner,'' Rose said.

Thibodeau spent more than 20 years as an assistant coach in the N.B.A.,
developing a reputation for designing strong defenses, having an
obsessive work ethic and playing his stars for heavy minutes. He was the
architect of the defense for the 2008 Boston Celtics, who won a
championship.

His first head coaching job came in the 2010-11 season with the Chicago
Bulls, where he went to the Eastern Conference finals in his first year.
Thibodeau went 255-139
\href{https://www.basketball-reference.com/coaches/thiboto99c.html}{with
the Bulls} but was
\href{https://www.chicagotribune.com/sports/bulls/ct-tom-thibodeau-fired-20150528-story.html}{fired
in 2015} after clashes with the front office. He landed with a
rebuilding team --- the Minnesota Timberwolves --- in 2016 as coach and
president of basketball operations. Thibodeau was
\href{https://www.nytimes.com/2019/01/06/sports/tom-thibodeau-fired-timberwolves.html}{fired
abruptly} last season after only making the playoffs once and going
97-107.

He told reporters on Thursday that while he had caught a few Knicks
games this season, he also traveled quite a bit, including a trip to
unwind in Miami.

``I also went on vacation a couple of times,'' Thibodeau said. ``I know
people don't think I do that.''

Advertisement

\protect\hyperlink{after-bottom}{Continue reading the main story}

\hypertarget{site-index}{%
\subsection{Site Index}\label{site-index}}

\hypertarget{site-information-navigation}{%
\subsection{Site Information
Navigation}\label{site-information-navigation}}

\begin{itemize}
\tightlist
\item
  \href{https://help.nytimes.com/hc/en-us/articles/115014792127-Copyright-notice}{©~2020~The
  New York Times Company}
\end{itemize}

\begin{itemize}
\tightlist
\item
  \href{https://www.nytco.com/}{NYTCo}
\item
  \href{https://help.nytimes.com/hc/en-us/articles/115015385887-Contact-Us}{Contact
  Us}
\item
  \href{https://www.nytco.com/careers/}{Work with us}
\item
  \href{https://nytmediakit.com/}{Advertise}
\item
  \href{http://www.tbrandstudio.com/}{T Brand Studio}
\item
  \href{https://www.nytimes.com/privacy/cookie-policy\#how-do-i-manage-trackers}{Your
  Ad Choices}
\item
  \href{https://www.nytimes.com/privacy}{Privacy}
\item
  \href{https://help.nytimes.com/hc/en-us/articles/115014893428-Terms-of-service}{Terms
  of Service}
\item
  \href{https://help.nytimes.com/hc/en-us/articles/115014893968-Terms-of-sale}{Terms
  of Sale}
\item
  \href{https://spiderbites.nytimes.com}{Site Map}
\item
  \href{https://help.nytimes.com/hc/en-us}{Help}
\item
  \href{https://www.nytimes.com/subscription?campaignId=37WXW}{Subscriptions}
\end{itemize}
