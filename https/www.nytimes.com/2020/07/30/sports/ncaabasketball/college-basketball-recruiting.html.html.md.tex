Sections

SEARCH

\protect\hyperlink{site-content}{Skip to
content}\protect\hyperlink{site-index}{Skip to site index}

\href{https://www.nytimes.com/section/sports/ncaabasketball}{College
Basketball}

\href{https://myaccount.nytimes.com/auth/login?response_type=cookie\&client_id=vi}{}

\href{https://www.nytimes.com/section/todayspaper}{Today's Paper}

\href{/section/sports/ncaabasketball}{College Basketball}\textbar{}No
Summer Tournaments Means More Recruits Committing to Colleges

\url{https://nyti.ms/2XcmthL}

\begin{itemize}
\item
\item
\item
\item
\item
\end{itemize}

Advertisement

\protect\hyperlink{after-top}{Continue reading the main story}

Supported by

\protect\hyperlink{after-sponsor}{Continue reading the main story}

\hypertarget{no-summer-tournaments-means-more-recruits-committing-to-colleges}{%
\section{No Summer Tournaments Means More Recruits Committing to
Colleges}\label{no-summer-tournaments-means-more-recruits-committing-to-colleges}}

Villanova has already received verbal commitments from enough players to
build most of a starting lineup for future seasons.

\includegraphics{https://static01.nyt.com/images/2020/08/02/sports/28collegehoops-recruiting-1/merlin_169791603_d3412c58-3ce2-449c-8db8-1e1440bb4b48-articleLarge.jpg?quality=75\&auto=webp\&disable=upscale}

By Adam Zagoria

\begin{itemize}
\item
  July 30, 2020
\item
  \begin{itemize}
  \item
  \item
  \item
  \item
  \item
  \end{itemize}
\end{itemize}

In April, Jay Wright, the men's basketball coach at Villanova, worried
the coronavirus pandemic would hurt his ability to recruit players from
next year's high school class.

Because coaches could not watch players in person or have them visit
their campuses that month, Wright was not sure how he and his staff
would be able to properly evaluate the athletes.

``If it affects us and hurts us a little bit, so what?'' Wright said in
April. ``Suck it up. There's a lot more important things going on in our
world right now.''

Three months later, his fears seem almost quaint.

After two recent verbal commitments, Wright and his staff now have four
rising seniors committed and thus the No. 1 recruiting class for 2021,
according to the recruiting website 247Sports.com. They include players
who could start in the future: point guard Angelo Brizzi of Virginia,
shooting guard Jordan Longino of Pennsylvania, small forward Trey
Patterson of New Jersey and the big man Nnanna Njoku of Delaware. Under
Wright, Villanova has won two of the last four N.C.A.A. championships,
and these players are eager to be a part of the growing tradition.

``We don't have A.A.U. this year, and normally we would be traveling and
focusing on games,'' Patterson, out of Rutgers Preparatory School, said
Tuesday in a phone interview. ``But because of the pandemic, me and my
family, we've had more time to talk to schools and evaluate programs, so
I think that kind of expedited the process.''

In a normal summer, Patterson and many of the other top high school
prospects would be out playing this month at the Nike Peach Jam in North
Augusta, S.C., or at other events across the nation trying to impress
coaches and attract scholarship offers.

Now, with tournaments largely canceled because of the virus and the
N.C.A.A. imposing a dead period for recruiting through at least August,
Patterson verbally committed on June 18 and plans to sign his letter of
intent in the fall.

Several other schools are also benefiting. Baylor, Butler, Ohio State,
Southern California, Louisville, Michigan and Florida State each had
three players committed for 2021 as of Wednesday.

``What's really pushed up the whole recruiting process is the pandemic
because kids are uncertain as to whether they're ever going to be able
to make paid visits to these schools, so they've had these virtual
visits,'' Tom Konchalski, a longtime recruiting expert, said in a phone
interview. ``Obviously, that doesn't give them as much of a feel for the
team, the campus, the coach, the players and the whole culture of the
school as if they took a visit when school was actually in session. But
it's better than nothing.''

Jeff Ngandu, a Canadian big man originally from the Democratic Republic
of Congo, pledged to Seton Hall in May for the 2020-21 season without
ever having visited the New Jersey school. Saquan Singleton similarly
committed in April for 2020 to New Mexico out of Hutchinson (Kan.)
Community College. Both stayed in touch with their future coaching
staffs through videoconferencing and phone calls.

``Unfortunately, I couldn't get to see the visit but I just felt the
love and the connection,'' Singleton said.

Jordan Riley, a rising senior guard at Brentwood High School on Long
Island, committed on Friday to Georgetown after only a one-hour visit to
the university this month with his father, Monty. They were not able to
see any students or meet with the basketball team because the campus was
closed. They did not even let Georgetown's coach, Patrick Ewing, and his
staff know they were on campus.

Yet the visit --- along with Ewing's recruiting message on a daily basis
--- was enough to make the 6-foot-4, 185-pound guard pledge to the Hoyas
over Kansas, Florida State, Connecticut and St. John's. He had visited
St. John's and UConn, but was unable to visit the other campuses.

``I saw the campus, I saw what I liked and I'm just ready to go there,''
Riley said of Georgetown.

Monty Riley said he wanted to speed up the process because he was being
overwhelmed with calls that he normally would not have received in a
nonpandemic year.

``It was just too many phone calls,'' Monty Riley said. ``One day I got
30 phone calls. And I'm working. I made him shorten his list and made
him go on from there.''

Rising seniors and incoming college freshmen aren't the only ones
committing during the pandemic.

Emoni Bates, the No. 1 prospect in the class of 2022 out of Michigan,
verbally committed to Michigan State last month, more than two years
before he will play his first college basketball game.

\includegraphics{https://static01.nyt.com/images/2020/08/02/sports/28collegehoops-recruiting-2/28collegehoops-recruiting-2-articleLarge.jpg?quality=75\&auto=webp\&disable=upscale}

That could change if Bates reclassifies into the 2021 class. And if the
N.B.A.'s one-and-done rule is collectively bargained away by 2022,
Bates, who has been compared to a young Kevin Durant, could just enter
the N.B.A. draft in 2022 and skip college altogether. He could also
forgo college, and make money, by entering the G-League's pro pathway
program for elite prospects.

For now, though, his college decision has been made.

``The pandemic had no influence over the timing of our Michigan State
decision,'' Elgin Bates, Emoni's father, said in a text message. ``We
just chose to let it be known what we've been thinking since seventh
grade at this time. There was no point in waiting or wasting anyone's
time recruiting.''

Michigan State landed a second pledge in the junior class this week when
center Enoch Boakye verbally pledged to Coach Tom Izzo's team.

Some college coaches and others in the game believe there could be
another downside to these early commitments besides players changing
their minds before officially signing. They're bracing for more
transfers --- an interesting prospect considering there are already more
than 1,000 players in the N.C.A.A. transfer portal.

Patterson feels the early commitment will help Villanova and his future
teammates in the long run, giving them more time to get to know one
another and understand what they will expect from each other.

``It's good that we're almost done with our class now and we have the
opportunity to build relationships with each other over this next year
before we come in and build a relationship with the coaches even more as
well,'' he said. ``So everybody's pretty much on the same page.''

Advertisement

\protect\hyperlink{after-bottom}{Continue reading the main story}

\hypertarget{site-index}{%
\subsection{Site Index}\label{site-index}}

\hypertarget{site-information-navigation}{%
\subsection{Site Information
Navigation}\label{site-information-navigation}}

\begin{itemize}
\tightlist
\item
  \href{https://help.nytimes.com/hc/en-us/articles/115014792127-Copyright-notice}{©~2020~The
  New York Times Company}
\end{itemize}

\begin{itemize}
\tightlist
\item
  \href{https://www.nytco.com/}{NYTCo}
\item
  \href{https://help.nytimes.com/hc/en-us/articles/115015385887-Contact-Us}{Contact
  Us}
\item
  \href{https://www.nytco.com/careers/}{Work with us}
\item
  \href{https://nytmediakit.com/}{Advertise}
\item
  \href{http://www.tbrandstudio.com/}{T Brand Studio}
\item
  \href{https://www.nytimes.com/privacy/cookie-policy\#how-do-i-manage-trackers}{Your
  Ad Choices}
\item
  \href{https://www.nytimes.com/privacy}{Privacy}
\item
  \href{https://help.nytimes.com/hc/en-us/articles/115014893428-Terms-of-service}{Terms
  of Service}
\item
  \href{https://help.nytimes.com/hc/en-us/articles/115014893968-Terms-of-sale}{Terms
  of Sale}
\item
  \href{https://spiderbites.nytimes.com}{Site Map}
\item
  \href{https://help.nytimes.com/hc/en-us}{Help}
\item
  \href{https://www.nytimes.com/subscription?campaignId=37WXW}{Subscriptions}
\end{itemize}
