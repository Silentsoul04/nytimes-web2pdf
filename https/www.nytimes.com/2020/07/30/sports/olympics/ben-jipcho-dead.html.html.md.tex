Sections

SEARCH

\protect\hyperlink{site-content}{Skip to
content}\protect\hyperlink{site-index}{Skip to site index}

\href{https://www.nytimes.com/section/sports/olympics}{Olympics}

\href{https://myaccount.nytimes.com/auth/login?response_type=cookie\&client_id=vi}{}

\href{https://www.nytimes.com/section/todayspaper}{Today's Paper}

\href{/section/sports/olympics}{Olympics}\textbar{}Ben Jipcho, a Runner
Who Sacrificed Himself for a Teammate, Dies at 77

\url{https://nyti.ms/337sosh}

\begin{itemize}
\item
\item
\item
\item
\item
\end{itemize}

Advertisement

\protect\hyperlink{after-top}{Continue reading the main story}

Supported by

\protect\hyperlink{after-sponsor}{Continue reading the main story}

\hypertarget{ben-jipcho-a-runner-who-sacrificed-himself-for-a-teammate-dies-at-77}{%
\section{Ben Jipcho, a Runner Who Sacrificed Himself for a Teammate,
Dies at
77}\label{ben-jipcho-a-runner-who-sacrificed-himself-for-a-teammate-dies-at-77}}

At the 1968 Summer Olympics, he set a blistering early pace in the
1,500-meter race to help his teammate beat the American Jim Ryun. Jipcho
finished 10th.

\includegraphics{https://static01.nyt.com/images/2020/08/02/obituaries/02jipcho-obit2/30jipcho-sub2-articleLarge.jpg?quality=75\&auto=webp\&disable=upscale}

\href{https://www.nytimes.com/by/richard-sandomir}{\includegraphics{https://static01.nyt.com/images/2018/12/10/multimedia/author-richard-sandomir/author-richard-sandomir-thumbLarge.png}}

By \href{https://www.nytimes.com/by/richard-sandomir}{Richard Sandomir}

\begin{itemize}
\item
  Published July 30, 2020Updated July 31, 2020
\item
  \begin{itemize}
  \item
  \item
  \item
  \item
  \item
  \end{itemize}
\end{itemize}

Ben Jipcho, a Kenyan distance runner who set a torrid pace in the
1,500-meter race at the 1968 Summer Olympics in Mexico City to help his
countryman, Kip Keino, defeat Jim Ryun, the American favorite, died on
July 24 at a hospital in Eldoret, Kenya. He was 77.

His death was confirmed by Athletics Kenya, the country's governing body
of sport. His daughter Ruth Jipcho told the Kenyan news media that
t\href{https://www.pd.co.ke/sports/athletics/tributes-pour-in-for-olympian-jipcho-45426/}{he
cause was cancer}.

Taking advantage of the high altitude that was familiar to them from
home, Kenyan runners won eight medals in Mexico City. But Jipcho was not
among them. Instead, Kenyan officials used him in a plan to give Keino
an advantage after he had lost to Ryun in a 1,500-meter semifinal.

Ryun was formidable. He had not lost a mile or 1,500-meter race in three
years and held world records for the 880, the mile and the 1,500.

Kenyan coaches wanted Jipcho to run so fast from the start, as the
race's rabbit, that he would tire Ryun and catapult Keino to the lead.

``I was not really willing, but they convinced me --- they said I was
young and that Kip was getting old,'' Jipcho, 25, who was three years
younger than Keino, told The Associated Press in 1975. ``I had to
sacrifice to set the pace because we wanted the gold medal for Kenya.''

Jipcho did his job well. He ran the first 400 meters in a stunning 55.9
seconds, with Keino in third place and Ryun in the back of the field.
Jipcho kept up the pace through 800 meters, when Keino took the lead. He
won the gold medal with a time of 3 minutes 34.9 seconds, an Olympic
record. Ryun finished second, nearly three seconds behind. Jipcho faded
to 10th.

Several years later, Jipcho told Ryun that he regretted his role in the
race, that the thought the tactic had undermined the Olympic spirit and
Ryun's chances of winning.

\includegraphics{https://static01.nyt.com/images/2020/08/28/obituaries/28Jipcho1/merlin_174900969_6830a980-b3c9-478c-9654-2f67828eb9a5-articleLarge.jpg?quality=75\&auto=webp\&disable=upscale}

``He felt he owed me an apology for what happened in Mexico City,'' Ryun
told The A.P. in 1989. ``I said, `You didn't have to say that.' But he
felt he had to do it. I think if he had given it a second thought, he
might not have done it. But the pressures during international
competition are enormous.''

At the 1972 Summer Games in Munich, Jipcho began to shed his reputation
as a supporting player.
\href{https://web.archive.org/web/20200417174448/https://www.sports-reference.com/olympics/athletes/ji/ben-jipcho-1.html}{In
the 3,000-meter steeplechase,} he won the silver medal. Keino won the
gold.

Over the next few years, Jipcho would become one of the world's top
distance runners.

Benjamin Wabura Jipcho was born on March 1, 1943, in the Mount Elgon
district of western Kenya. He grew up in the village of Kapkateny, where
his father was a farmer. He did not start running until high school,
when he was encouraged by a British-born coach.

Jipcho --- who supported himself as a teacher in a prison while he was
running --- did not reach his athletic peak until after the 1972
Olympics. In early 1973,
\href{https://en.wikipedia.org/wiki/Athletics_at_the_1973_All-Africa_Games}{he
won gold medals at the All-Africa Games} in Nigeria in the 5,000 meters
and the 3,000-meter steeplechase.

And, in a two-week stretch that summer, he twice broke the world record
in the 3,000-meter steeplechase in Helsinki and ran the second-fastest
mile, 3:52.0, in Stockholm. Ryun held the record at 3:51.1.

``Jipcho has emerged this summer as his sport's most versatile and
impressive runner'' and the successor to Keino,
\href{https://vault.si.com/vault/1973/07/30/jipcho-is-hitting-his-stride}{Sports
Illustrated wrote in July 1973}.

After winning two gold medals (in the 5,000-meter and 3,000-meter
steeplechase races) and a bronze (in the 1,500 meters) at the
\href{https://en.wikipedia.org/wiki/Athletics_at_the_1974_British_Commonwealth_Games\#:~:text=The\%20QE\%20II\%20Park\%20was\%20purpose\%2Dbuilt\%20for\%20the\%201974\%20Games.\&text=At\%20the\%201974\%20British\%20Commonwealth\%20Games\%2C\%20the\%20athletics\%20events\%20were,25\%20January\%20and\%202\%20February.}{1974
British Commonwealth Games}in New Zealand, Jipcho joined the
International Track Association professional circuit. He became one of
its
stars\href{http://www.thesportsexaminer.com/lane-one-echoes-of-the-failed-intl-track-association-in-the-new-intl-swimming-league/}{until
it folded in 1976}.

At one meet in Los Angeles, he set an indoor professional record for the
two-mile; less than an hour later, he ran the third-fastest indoor mile
ever.

As a professional, Jipcho said, he was running not for records but for
money to help his family. After winning a mile race in a relatively slow
4:02.8 in El Paso in 1974,
\href{https://vault.si.com/vault/1974/05/06/the-pros-are-beginning-to-look-professional}{he
told Sports Illustrated:} ``The \$500 will buy some cows for my farm in
Kenya. A winning time is always a good time. If the I.T.A. people want a
sub-four-minute mile, all they have to do is come to me. With money.''

After his racing career ended, he worked as a teacher, a school
principal and a farmer and went into local politics.

In addition to his daughter Ruth, his survivors include his wife, Bilia;
two other daughters, Catherine and Jacky; five sons, Godfrey, Geoffrey,
Moses, Oliver and Anthony; seven grandchildren; and four
great-grandchildren.

Jipcho, whose athletic renown was tied partly to helping Keino beat Ryun
in 1968, died on the same day that Ryun
\href{https://kuathletics.com/jim-ryun-receives-presidential-medal-of-freedom/}{received
the Presidential Medal of Freedom}at the White House.

Advertisement

\protect\hyperlink{after-bottom}{Continue reading the main story}

\hypertarget{site-index}{%
\subsection{Site Index}\label{site-index}}

\hypertarget{site-information-navigation}{%
\subsection{Site Information
Navigation}\label{site-information-navigation}}

\begin{itemize}
\tightlist
\item
  \href{https://help.nytimes.com/hc/en-us/articles/115014792127-Copyright-notice}{©~2020~The
  New York Times Company}
\end{itemize}

\begin{itemize}
\tightlist
\item
  \href{https://www.nytco.com/}{NYTCo}
\item
  \href{https://help.nytimes.com/hc/en-us/articles/115015385887-Contact-Us}{Contact
  Us}
\item
  \href{https://www.nytco.com/careers/}{Work with us}
\item
  \href{https://nytmediakit.com/}{Advertise}
\item
  \href{http://www.tbrandstudio.com/}{T Brand Studio}
\item
  \href{https://www.nytimes.com/privacy/cookie-policy\#how-do-i-manage-trackers}{Your
  Ad Choices}
\item
  \href{https://www.nytimes.com/privacy}{Privacy}
\item
  \href{https://help.nytimes.com/hc/en-us/articles/115014893428-Terms-of-service}{Terms
  of Service}
\item
  \href{https://help.nytimes.com/hc/en-us/articles/115014893968-Terms-of-sale}{Terms
  of Sale}
\item
  \href{https://spiderbites.nytimes.com}{Site Map}
\item
  \href{https://help.nytimes.com/hc/en-us}{Help}
\item
  \href{https://www.nytimes.com/subscription?campaignId=37WXW}{Subscriptions}
\end{itemize}
