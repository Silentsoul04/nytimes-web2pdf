Sections

SEARCH

\protect\hyperlink{site-content}{Skip to
content}\protect\hyperlink{site-index}{Skip to site index}

\href{https://www.nytimes.com/section/sports/soccer}{Soccer}

\href{https://myaccount.nytimes.com/auth/login?response_type=cookie\&client_id=vi}{}

\href{https://www.nytimes.com/section/todayspaper}{Today's Paper}

\href{/section/sports/soccer}{Soccer}\textbar{}Saudi Arabia Withdraws
Bid to Buy Newcastle United

\url{https://nyti.ms/2P6IYjR}

\begin{itemize}
\item
\item
\item
\item
\item
\end{itemize}

Advertisement

\protect\hyperlink{after-top}{Continue reading the main story}

Supported by

\protect\hyperlink{after-sponsor}{Continue reading the main story}

\hypertarget{saudi-arabia-withdraws-bid-to-buy-newcastle-united}{%
\section{Saudi Arabia Withdraws Bid to Buy Newcastle
United}\label{saudi-arabia-withdraws-bid-to-buy-newcastle-united}}

The proposed takeover of a Premier League club had drawn criticism from
human rights groups and an important television partner.

\includegraphics{https://static01.nyt.com/images/2020/06/15/world/15soccer-newcastle-1/merlin_172582179_8f1911ea-3758-44b8-b326-34ab35ba63ce-articleLarge.jpg?quality=75\&auto=webp\&disable=upscale}

By \href{https://www.nytimes.com/by/tariq-panja}{Tariq Panja}

\begin{itemize}
\item
  July 30, 2020
\item
  \begin{itemize}
  \item
  \item
  \item
  \item
  \item
  \end{itemize}
\end{itemize}

Saudi Arabia's sovereign wealth fund on Thursday withdrew its bid to
become the latest foreign owner in England's Premier League, pulling out
of an agreement to buy Newcastle United after a tumultuous takeover
process and significant pressure on the league to block the sale.

Without criticizing the Premier League directly, the group led by Saudi
Arabia's Public Investment Fund attributed the collapse of the deal in
part to ``an unforeseen prolonged process.''

The Premier League, which had been vetting the proposed sale since
April, made no comment on the withdrawal.

The Saudi-led consortium, which included the British businesswoman
Amanda Staveley and a British-based property company in addition to the
kingdom's sovereign wealth fund, was set to pay about \$400 million for
the team and its stadium, which has been owned by the sportswear magnate
Mike Ashley since 2007.

While the Premier League's glamour and global reach have long made it a
magnet for the world's super rich --- its team owners currently include
American billionaires, a Russian oligarch, a Chinese holding company and
the brother of the ruler of the United Arab Emirates --- Saudi Arabia's
bid for a team led to a level of discord rarely seen.

\href{https://www.independent.co.uk/sport/football/premier-league/newcastle-takeover-human-rights-saudi-arabia-consortium-fa-disqualified-a9481361.html}{Human
rights groups} and even
t\href{https://www.theguardian.com/commentisfree/2020/may/18/saudi-regime-newcastle-united-jamal-khashoggi-mohammed-bin-salman}{he
widow of the murdered journalist Jamal Khashoggi} wrote to the Premier
League's chief executive, Richard Masters, to urge him to block the sale
because of the involvement of the Public Investment Fund, the Saudi
sovereign wealth fund led by Saudi Arabia's de facto ruler, Crown Prince
Mohammed bin Salman.

A more important challenge to the takeover, at least for top Premier
League officials, had come from beIN Media Group, the Qatar-owned
television network. The network, one of the Premier League's biggest
broadcaster partners, has for three years has accused Saudi Arabia of
being behind industrial-scale piracy of its programming.

Only weeks before it began considering the Saudi takeover bid, the
Premier League had
\href{https://www.nytimes.com/2020/04/30/sports/premier-league-saudi-arabia.html}{written
to the United States government} to urge it to keep the kingdom on a
watch list of countries that breach intellectual property rules.

Once the Saudi bid became public, senior beIN officials lobbied the
league and even the British government not to allow a Saudi state
vehicle to join the ranks of club owners, and the Premier League spent
months deliberating the so-called ``fit and proper test'' that is
applied to all new owners.

The Premier League was not known to have ever previously blocked a sale,
and with the Saudi group's withdrawal, it did not have to do so in this
case.

``Unfortunately, the prolonged process under the current circumstances
coupled with global uncertainty has rendered the potential investment no
longer commercially viable,'' the investment group said in a joint
statement. It said its agreement with Newcastle's owners to buy the team
had expired and appeared to blame uncertain economic conditions as the
reason to walk away. Ashley had collected more than \$25 million as a
nonrefundable deposit.

Premier League matches have a reach that surpasses any other similar
global sports competitions, with its teams counting millions of
passionate fans on continents thousands of miles away from the stadiums
where games take place. That reach has attracted perhaps the most
diverse ownership group in sports: Over the last two decades, British
businessmen who once dominated the league's ownership ranks have been
edged out by billionaires from the United States, Europe, Asia and
Africa --- a membership that currently boasts a Russian-Israeli oligarch
(Chelsea's Roman Abramovich), one of Africa's richest men
(\href{https://www.birminghammail.co.uk/sport/football/football-news/who-is-nassef-sawiris-villa-17082772}{Aston
Villa's Nassef Sawiris}) and the heirs to a Thai duty-free shopping
empire
(\href{https://www.leicestermercury.co.uk/sport/football/football-news/leicester-city-owner-aiyawatt-srivaddhanaprabha-3824465}{Leicester's
Srivaddhanaprabha family}).

The Saudi-led investors had proposed spending as much as \$320 million
over five years to turn Newcastle into a competitive force in the league
and to invest in infrastructure around its stadium.

The Saudi fund would not have been the league's first Gulf-state owner:
Manchester City, who won the league in two of the last three seasons, is
controlled by the ruling family of the United Arab Emirates.

While the league spent weeks in an uncomfortable spotlight created by
the Saudi bid, Newcastle fans had largely rejoiced at the prospect of
the unpopular Ashley's being replaced by deep-pocketed owners.

Since the first details of the proposed takeover emerged earlier this
year, many Newcastle promoted it on social media, with some even
changing their profile pictures to incorporate images of the Saudi flag
or Salman, the kingdom's crown prince.

Most seemed to hope that the Saudis' wealth would allow the team, whose
raucous home support endures despite a middling on-field record, to
compete for titles again. Newcastle narrowly missed winning the Premier
League title twice in the mid-1990s but has not won a major domestic
trophy since the 1955 F.A. Cup. The last of the club's four English
titles came in 1927.

Advertisement

\protect\hyperlink{after-bottom}{Continue reading the main story}

\hypertarget{site-index}{%
\subsection{Site Index}\label{site-index}}

\hypertarget{site-information-navigation}{%
\subsection{Site Information
Navigation}\label{site-information-navigation}}

\begin{itemize}
\tightlist
\item
  \href{https://help.nytimes.com/hc/en-us/articles/115014792127-Copyright-notice}{©~2020~The
  New York Times Company}
\end{itemize}

\begin{itemize}
\tightlist
\item
  \href{https://www.nytco.com/}{NYTCo}
\item
  \href{https://help.nytimes.com/hc/en-us/articles/115015385887-Contact-Us}{Contact
  Us}
\item
  \href{https://www.nytco.com/careers/}{Work with us}
\item
  \href{https://nytmediakit.com/}{Advertise}
\item
  \href{http://www.tbrandstudio.com/}{T Brand Studio}
\item
  \href{https://www.nytimes.com/privacy/cookie-policy\#how-do-i-manage-trackers}{Your
  Ad Choices}
\item
  \href{https://www.nytimes.com/privacy}{Privacy}
\item
  \href{https://help.nytimes.com/hc/en-us/articles/115014893428-Terms-of-service}{Terms
  of Service}
\item
  \href{https://help.nytimes.com/hc/en-us/articles/115014893968-Terms-of-sale}{Terms
  of Sale}
\item
  \href{https://spiderbites.nytimes.com}{Site Map}
\item
  \href{https://help.nytimes.com/hc/en-us}{Help}
\item
  \href{https://www.nytimes.com/subscription?campaignId=37WXW}{Subscriptions}
\end{itemize}
