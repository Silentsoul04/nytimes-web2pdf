Sections

SEARCH

\protect\hyperlink{site-content}{Skip to
content}\protect\hyperlink{site-index}{Skip to site index}

\href{https://www.nytimes.com/section/technology}{Technology}

\href{https://myaccount.nytimes.com/auth/login?response_type=cookie\&client_id=vi}{}

\href{https://www.nytimes.com/section/todayspaper}{Today's Paper}

\href{/section/technology}{Technology}\textbar{}The Economy Is in Record
Decline, but Not for the Tech Giants

\url{https://nyti.ms/30fLsCN}

\begin{itemize}
\item
\item
\item
\item
\item
\end{itemize}

\href{https://www.nytimes.com/news-event/coronavirus?action=click\&pgtype=Article\&state=default\&region=TOP_BANNER\&context=storylines_menu}{The
Coronavirus Outbreak}

\begin{itemize}
\tightlist
\item
  live\href{https://www.nytimes.com/2020/08/01/world/coronavirus-covid-19.html?action=click\&pgtype=Article\&state=default\&region=TOP_BANNER\&context=storylines_menu}{Latest
  Updates}
\item
  \href{https://www.nytimes.com/interactive/2020/us/coronavirus-us-cases.html?action=click\&pgtype=Article\&state=default\&region=TOP_BANNER\&context=storylines_menu}{Maps
  and Cases}
\item
  \href{https://www.nytimes.com/interactive/2020/science/coronavirus-vaccine-tracker.html?action=click\&pgtype=Article\&state=default\&region=TOP_BANNER\&context=storylines_menu}{Vaccine
  Tracker}
\item
  \href{https://www.nytimes.com/interactive/2020/07/29/us/schools-reopening-coronavirus.html?action=click\&pgtype=Article\&state=default\&region=TOP_BANNER\&context=storylines_menu}{What
  School May Look Like}
\item
  \href{https://www.nytimes.com/live/2020/07/31/business/stock-market-today-coronavirus?action=click\&pgtype=Article\&state=default\&region=TOP_BANNER\&context=storylines_menu}{Economy}
\end{itemize}

Advertisement

\protect\hyperlink{after-top}{Continue reading the main story}

Supported by

\protect\hyperlink{after-sponsor}{Continue reading the main story}

\hypertarget{the-economy-is-in-record-decline-but-not-for-the-tech-giants}{%
\section{The Economy Is in Record Decline, but Not for the Tech
Giants}\label{the-economy-is-in-record-decline-but-not-for-the-tech-giants}}

Even though the tech industry's four biggest companies were stung by a
slowdown in spending, they reported a combined \$28 billion in profits
on Thursday.

\includegraphics{https://static01.nyt.com/images/2020/07/30/technology/30tech-earnings/oakImage-1596148409520-articleLarge.jpg?quality=75\&auto=webp\&disable=upscale}

\href{https://www.nytimes.com/by/daisuke-wakabayashi}{\includegraphics{https://static01.nyt.com/images/2018/07/30/multimedia/author-daisuke-wakabayashi/author-daisuke-wakabayashi-thumbLarge.png}}\href{https://www.nytimes.com/by/karen-weise}{\includegraphics{https://static01.nyt.com/images/2019/04/11/multimedia/author-karen-weise/author-karen-weise-thumbLarge.png}}\href{https://www.nytimes.com/by/jack-nicas}{\includegraphics{https://static01.nyt.com/images/2018/11/26/multimedia/author-jack-nicas/author-jack-nicas-thumbLarge.png}}\href{https://www.nytimes.com/by/mike-isaac}{\includegraphics{https://static01.nyt.com/images/2018/02/16/multimedia/author-mike-isaac/author-mike-isaac-thumbLarge.jpg}}

By \href{https://www.nytimes.com/by/daisuke-wakabayashi}{Daisuke
Wakabayashi}, \href{https://www.nytimes.com/by/karen-weise}{Karen
Weise}, \href{https://www.nytimes.com/by/jack-nicas}{Jack Nicas} and
\href{https://www.nytimes.com/by/mike-isaac}{Mike Isaac}

\begin{itemize}
\item
  July 30, 2020
\item
  \begin{itemize}
  \item
  \item
  \item
  \item
  \item
  \end{itemize}
\end{itemize}

A day after lawmakers grilled the chief executives of the biggest tech
companies about their size and power, Amazon, Apple, Alphabet and
Facebook reported surprisingly healthy quarterly financial results,
defying one of the
\href{https://www.nytimes.com/live/2020/07/30/business/stock-market-today-coronavirus/the-us-economys-contraction-in-the-second-quarter-was-the-worst-on-record}{worst
economic downturns on record}.

Even though the companies felt some sting from the spending slowdown,
they demonstrated, as critics have argued, that they are operating on a
different playing field from the rest of the economy.

Amazon's sales were up 40 percent from a year ago and its profit
doubled. Facebook's profit jumped 98 percent. Even though the pandemic
\href{https://www.nytimes.com/2020/03/14/technology/apple-stores-coronavirus.html}{shuttered
many of its store}s, Apple increased sales of all its products in every
part of the world and posted \$11.25 billion in profit. Advertising
revenue dropped for Alphabet, the laggard of the bunch, but it still did
better than Wall Street had expected.

``The strong continue to get stronger,'' said Dan Ives, managing
director of equity research at Wedbush Securities. ``As many companies
are falling by the wayside, the tech stalwarts continue to gain muscle
and power in this environment.''

The tech companies' financial performance was a remarkable contrast to
the overall health of the U.S. economy. The Commerce Department said on
Thursday that the country's gross domestic product
\href{https://www.nytimes.com/2020/07/30/business/economy/q2-gdp-coronavirus-economy.html?action=click\&module=Top\%20Stories\&pgtype=Homepage}{fell
9.5 percent in the second quarter}of the year as consumers cut back
spending. It was the steepest drop on record.

Combined, the companies reported \$28.6 billion in quarterly net profit,
underscoring how regulatory scrutiny remains more background noise and a
distraction for them rather than an imminent threat to their businesses.

On Wednesday, a congressional antitrust panel questioned the companies'
leaders --- Jeff Bezos of Amazon, Tim Cook of Apple, Mark Zuckerberg of
Facebook and Sundar Pichai of Alphabet --- about their market power and
business practices.

It was part of a broader inquiry by regulators and lawmakers into the
dominance of the tech giants, with open investigations from the Justice
Department, the Federal Trade Commission and state attorneys general.

The spectacle of the chief executives of the four companies, worth
nearly \$5 trillion by market capitalization combined, appearing before
a House subcommittee was historic. But antitrust investigations often
take years, especially if regulators seek more drastic measures like
breaking up companies.

The pandemic has reinforced the advantages held by the big tech
companies. As consumers stay home, demand for Amazon's shopping site
surged, while companies are turning to its cloud computing products to
keep their services up and running. Apple said the shift to working and
learning from home had led more people to splurge on Apple's devices and
use its services.

\hypertarget{latest-updates-economy}{%
\section{\texorpdfstring{\href{https://www.nytimes.com/live/2020/07/31/business/stock-market-today-coronavirus?action=click\&pgtype=Article\&state=default\&region=MAIN_CONTENT_1\&context=storylines_live_updates}{Latest
Updates:
Economy}}{Latest Updates: Economy}}\label{latest-updates-economy}}

\href{https://www.nytimes.com/live/2020/07/31/business/stock-market-today-coronavirus?action=click\&pgtype=Article\&state=default\&region=MAIN_CONTENT_1\&context=storylines_live_updates\#kodaks-chief-executive-was-given-stock-options-then-the-share-price-spiked-1000-percent}{27h
ago}

\href{https://www.nytimes.com/live/2020/07/31/business/stock-market-today-coronavirus?action=click\&pgtype=Article\&state=default\&region=MAIN_CONTENT_1\&context=storylines_live_updates\#kodaks-chief-executive-was-given-stock-options-then-the-share-price-spiked-1000-percent}{Kodak's
chief executive was given stock options. Then the share price spiked
1,000 percent.}

\href{https://www.nytimes.com/live/2020/07/31/business/stock-market-today-coronavirus?action=click\&pgtype=Article\&state=default\&region=MAIN_CONTENT_1\&context=storylines_live_updates\#fitch-ratings-downgrades-its-outlook-on-us-debt}{30h
ago}

\href{https://www.nytimes.com/live/2020/07/31/business/stock-market-today-coronavirus?action=click\&pgtype=Article\&state=default\&region=MAIN_CONTENT_1\&context=storylines_live_updates\#fitch-ratings-downgrades-its-outlook-on-us-debt}{Fitch
Ratings downgrades its outlook on U.S. debt.}

\href{https://www.nytimes.com/live/2020/07/31/business/stock-market-today-coronavirus?action=click\&pgtype=Article\&state=default\&region=MAIN_CONTENT_1\&context=storylines_live_updates\#us-sanctions-more-chinese-officials-over-human-rights-violations-as-tensions-flare}{37h
ago}

\href{https://www.nytimes.com/live/2020/07/31/business/stock-market-today-coronavirus?action=click\&pgtype=Article\&state=default\&region=MAIN_CONTENT_1\&context=storylines_live_updates\#us-sanctions-more-chinese-officials-over-human-rights-violations-as-tensions-flare}{U.S.
sanctions more Chinese officials over human rights violations as
tensions flare}

\href{https://www.nytimes.com/live/2020/07/31/business/stock-market-today-coronavirus?action=click\&pgtype=Article\&state=default\&region=MAIN_CONTENT_1\&context=storylines_live_updates}{See
more updates}

More live coverage:
\href{https://www.nytimes.com/2020/08/01/world/coronavirus-covid-19.html?action=click\&pgtype=Article\&state=default\&region=MAIN_CONTENT_1\&context=storylines_live_updates}{Global}

``Our products and services are very relevant to our customers' lives,
and in some cases, even more during the pandemic than ever before,''
Luca Maestri, Apple's finance chief, said in an interview. He noted,
however, that Apple could have made several billion dollars more if not
for the pandemic.

Facebook and Google continue to be important to marketers and they are
weathering the downturn in advertising better than rivals. Facebook
shrugged off a spending slowdown, hailing record levels of engagement
with its products.

Alphabet said revenue from Google search ads fell 10 percent --- pushing
the company's overall revenue lower for the first time in the company's
history --- but that still was better than rivals. Last week, Microsoft
reported an 18 percent slide in search advertising revenue.

Since the beginning of March, the companies' stock prices have risen by
an average of 35 percent, compared with a 10 percent rise in the S.\&P.
500.

\hypertarget{amazon}{%
\subsection{Amazon}\label{amazon}}

Buoyed by a pandemic-induced surge in online shopping,
\href{https://www.nytimes.com/2020/05/22/technology/amazon-coronavirus-target-walmart.html}{Amazon
had \$88.9 billion in quarterly sales}, up 40 percent from a year
earlier. Profit doubled, to \$5.2 billion, even though the company
invested in expanding warehouses and other ways to increase capacity.

``Simply put, Covid-19, in our view, has injected Amazon with a growth
hormone,'' Tom Forte, an analyst at the investment bank D.A. Davidson \&
Company, wrote in a recent note to investors.

In April, Mr. Bezos
\href{https://www.nytimes.com/2020/04/30/technology/amazon-stock-earnings-report.html}{told}
investors to expect no operating profit, and maybe even a loss, as the
company planned to spend about \$4 billion on coronavirus-related
expenses like temporary pay increases, declines in warehouse efficiency
because of social distancing, and \$300 million for testing its work
force for the virus.

But even those costs did not compare to the immense surge in demand,
with online retail sales up 48 percent.

On a call with reporters, Amazon declined to say if it would give its
warehouse workers virus-related bonuses or raises in the current
quarter, but added that pandemic-related expenses would fall to \$2
billion in the quarter.

Sales at Amazon's lucrative cloud computing business, whose customers
include major corporations and small start-ups, grew 29 percent, to
\$10.8 billion, falling short of analyst expectations, though it was
more profitable than they had expected.

\hypertarget{facebook}{%
\subsection{Facebook}\label{facebook}}

Facebook's revenue for the second quarter rose 11 percent from a year
earlier to \$18.7 billion, while profits jumped 98 percent to \$5.2
billion. The results were well above analysts' estimates of \$17.3
billion in revenue with a profit of \$3.9 billion, according to data
provided by FactSet.

Despite increasing scrutiny from regulators, questions about
\href{https://slack-redir.net/link?url=https\%3A\%2F\%2Fwww.nytimes.com\%2F2018\%2F02\%2F17\%2Ftechnology\%2Findictment-russian-tech-facebook.html}{its
role in subverting elections} and how people use the platform to spread
misinformation, neither users nor advertisers have shown an inclination
to stop using Facebook.

More than three billion people now regularly come to Facebook or one of
its family of apps, as the services have overtaken much of the developed
world. And some 2.47 billion people use one or more of Facebook's apps
every day.

The company said its number of monthly active users rose 12 percent from
a year ago and added that it was seeing record levels of engagement and
usage this year because of shelter-in-place orders around the world.

In late June, a grass-roots campaign, Stop Hate for Profit, rallied many
of the top advertisers on Facebook to
\href{https://slack-redir.net/link?url=https\%3A\%2F\%2Fwww.nytimes.com\%2F2020\%2F06\%2F30\%2Ftechnology\%2Ffacebook-advertising-boycott.html}{reduce
their spending} because of issues with hate speech on the site.

Facebook cautioned investors on Thursday that fallout from the ad
boycott was noticeable in July and warned that greater economic turmoil
from the pandemic could eventually hurt Facebook's bottom line.

\hypertarget{apple}{%
\subsection{Apple}\label{apple}}

Despite the global
\href{https://slack-redir.net/link?url=https\%3A\%2F\%2Fwww.nytimes.com\%2Flive\%2F2020\%2F07\%2F30\%2Fbusiness\%2Fstock-market-today-coronavirus\%2Fthe-us-economys-contraction-in-the-second-quarter-was-the-worst-on-record}{economic
slowdown}, people kept buying Apple devices en masse and paid the tech
giant billions of dollars more for apps and services on those gadgets.

Apple said its sales rose 11 percent to \$59.7 billion and its profits
increased 12 percent to \$11.25 billion. Both figures handily beat
analysts' expectations, with Wall Street having forecast declines in
both areas.

Sales were particularly strong for iPads and Mac computers, as the
public was increasingly forced to work and socialize virtually. Revenue
also surged in its internet-services business, which include Apple's cut
of sales from the App Store, the subject of antitrust investigations in
\href{https://slack-redir.net/link?url=https\%3A\%2F\%2Fwww.nytimes.com\%2F2020\%2F07\%2F28\%2Ftechnology\%2Famazon-apple-facebook-google-antitrust-hearing.html}{the
United States} and
\href{https://slack-redir.net/link?url=https\%3A\%2F\%2Fwww.nytimes.com\%2F2020\%2F06\%2F16\%2Fbusiness\%2Fapple-app-store-european-union-antitrust.html}{Europe}.

Even the iPhone, which remains the company's biggest seller, had a
slight increase in sales for only the second time in the past seven
quarters.

Apple also announced a stock split on Thursday that would quadruple its
number of shares, allowing people to buy a share in the company for a
quarter of the current stock price, which closed at \$384.76 on
Thursday.

\hypertarget{alphabet}{%
\subsection{Alphabet}\label{alphabet}}

Google's parent company, Alphabet, reported its first-ever decline in
quarterly revenue, hurt by a slowdown in spending by advertisers. The
company posted revenue of \$38.3 billion and a profit of \$6.96 billion
--- significantly higher than what Wall Street analysts had predicted.

Ruth Porat, Alphabet's chief financial officer, said advertising revenue
``gradually improved'' as the quarter went on. The decline came largely
from lower sales of advertisements that run alongside Google's search
results, but the company's efforts to diversify its business paid off as
revenue from YouTube ads and its cloud computing business grew.

When asked in a call with financial analysts about the congressional
hearing, Mr. Pichai said the company would have to learn to live with
the investigations.

``The scrutiny is going to be here for a while and we're committed to
working through it,'' he said.

Advertisement

\protect\hyperlink{after-bottom}{Continue reading the main story}

\hypertarget{site-index}{%
\subsection{Site Index}\label{site-index}}

\hypertarget{site-information-navigation}{%
\subsection{Site Information
Navigation}\label{site-information-navigation}}

\begin{itemize}
\tightlist
\item
  \href{https://help.nytimes.com/hc/en-us/articles/115014792127-Copyright-notice}{©~2020~The
  New York Times Company}
\end{itemize}

\begin{itemize}
\tightlist
\item
  \href{https://www.nytco.com/}{NYTCo}
\item
  \href{https://help.nytimes.com/hc/en-us/articles/115015385887-Contact-Us}{Contact
  Us}
\item
  \href{https://www.nytco.com/careers/}{Work with us}
\item
  \href{https://nytmediakit.com/}{Advertise}
\item
  \href{http://www.tbrandstudio.com/}{T Brand Studio}
\item
  \href{https://www.nytimes.com/privacy/cookie-policy\#how-do-i-manage-trackers}{Your
  Ad Choices}
\item
  \href{https://www.nytimes.com/privacy}{Privacy}
\item
  \href{https://help.nytimes.com/hc/en-us/articles/115014893428-Terms-of-service}{Terms
  of Service}
\item
  \href{https://help.nytimes.com/hc/en-us/articles/115014893968-Terms-of-sale}{Terms
  of Sale}
\item
  \href{https://spiderbites.nytimes.com}{Site Map}
\item
  \href{https://help.nytimes.com/hc/en-us}{Help}
\item
  \href{https://www.nytimes.com/subscription?campaignId=37WXW}{Subscriptions}
\end{itemize}
