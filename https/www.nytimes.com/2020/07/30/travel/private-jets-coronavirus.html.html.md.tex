Sections

SEARCH

\protect\hyperlink{site-content}{Skip to
content}\protect\hyperlink{site-index}{Skip to site index}

\href{https://www.nytimes.com/section/travel}{Travel}

\href{https://myaccount.nytimes.com/auth/login?response_type=cookie\&client_id=vi}{}

\href{https://www.nytimes.com/section/todayspaper}{Today's Paper}

\href{/section/travel}{Travel}\textbar{}Afraid of Airlines? There's
Always the Private Jet

\url{https://nyti.ms/3jVASIL}

\begin{itemize}
\item
\item
\item
\item
\item
\end{itemize}

\href{https://www.nytimes.com/news-event/coronavirus?action=click\&pgtype=Article\&state=default\&region=TOP_BANNER\&context=storylines_menu}{The
Coronavirus Outbreak}

\begin{itemize}
\tightlist
\item
  live\href{https://www.nytimes.com/2020/08/04/world/coronavirus-cases.html?action=click\&pgtype=Article\&state=default\&region=TOP_BANNER\&context=storylines_menu}{Latest
  Updates}
\item
  \href{https://www.nytimes.com/interactive/2020/us/coronavirus-us-cases.html?action=click\&pgtype=Article\&state=default\&region=TOP_BANNER\&context=storylines_menu}{Maps
  and Cases}
\item
  \href{https://www.nytimes.com/interactive/2020/science/coronavirus-vaccine-tracker.html?action=click\&pgtype=Article\&state=default\&region=TOP_BANNER\&context=storylines_menu}{Vaccine
  Tracker}
\item
  \href{https://www.nytimes.com/2020/08/02/us/covid-college-reopening.html?action=click\&pgtype=Article\&state=default\&region=TOP_BANNER\&context=storylines_menu}{College
  Reopening}
\item
  \href{https://www.nytimes.com/live/2020/08/04/business/stock-market-today-coronavirus?action=click\&pgtype=Article\&state=default\&region=TOP_BANNER\&context=storylines_menu}{Economy}
\end{itemize}

Advertisement

\protect\hyperlink{after-top}{Continue reading the main story}

Supported by

\protect\hyperlink{after-sponsor}{Continue reading the main story}

\hypertarget{afraid-of-airlines-theres-always-the-private-jet}{%
\section{Afraid of Airlines? There's Always the Private
Jet}\label{afraid-of-airlines-theres-always-the-private-jet}}

Concerned about virus-related safety on commercial planes, many fliers
are turning to private jets for the first time. The catch, of course, is
the price.

\includegraphics{https://static01.nyt.com/images/2020/07/27/travel/29private-jets-virus/oakImage-1595868653679-articleLarge.jpg?quality=75\&auto=webp\&disable=upscale}

By \href{https://www.nytimes.com/by/sally-french}{Sally French}

\begin{itemize}
\item
  July 30, 2020
\item
  \begin{itemize}
  \item
  \item
  \item
  \item
  \item
  \end{itemize}
\end{itemize}

Lexi Shangraw, a San Francisco resident, flew to Phoenix in early March
for what was supposed to be a brief visit. But when lockdowns started,
she ended up staying longer than anticipated in hopes of waiting out
Covid-19.

Last month, she decided it was finally time to return home. Dubious
about the safety of big commercial airlines, she chose
\href{https://www.jsx.com/home/search}{JSX,} a hybrid private jet
service that departs from small, private terminals. In the world of
private jet travel, Ms. Shangraw got a good deal. Her one-way ticket on
a semiprivate jet to Oakland, Calif., cost \$159. That same day, a
flight to the Bay Area on American Airlines would have cost \$150, she
said.

Ms. Shangraw is among the growing number of Americans using private
jets, seeing them as a safer alternative to the often
\href{https://www.nytimes.com/2020/07/21/travel/crowded-flights-coronavirus.html}{cramped
commercial flights} filled with strangers during the pandemic. The day
after the Fourth of July, when commercial airline travel was down 74
percent year-over-year, private jet flights were up five percent,
according to
\href{https://privatejetcardcomparisons.com/2020/07/10/heres-where-the-private-jets-were-flying-for-the-july-4th-holiday/\#more-38454}{an
analysis of data from Argus, an aviation consulting firm}, by Doug
Gollan, who runs the website
\href{https://privatejetcardcomparisons.com/}{Private Jet Card
Comparisons.}

On JSX, passengers still fly with up to 29 strangers (though Ms.
Shangraw said there were fewer than 15 on her flight), but there's no
need to arrive two hours early (the company recommends 20 minutes),
because there are no security lines and no complex boarding procedures.
JSX flights tend to cost between \$300 and \$500 one way, per person,
but some shorter legs can cost less than \$100.

\hypertarget{the-price-of-exclusivity}{%
\subsection{The price of exclusivity}\label{the-price-of-exclusivity}}

Compared to most private jet services, JSX is downright affordable. Some
customers are opting for pricey, custom charter flights that can cost
anywhere from a few thousand dollars to more than \$10,000 per hour,
based on factors like aircraft type and in-flight service.

Even when paying top dollar, many travelers are seeing value in
springing for private flights amid a pandemic. That includes people like
Franklin Antoian, the founder of the personal training website iBodyFit,
who --- along with his wife and two kids --- took his first-ever private
jet ride last month from Palm Beach, Fla., to visit family in upstate
New York. It cost \$20,000, about six times more than Mr. Antoian's
usual first-class fares for his family of four. He justified the cost,
saying this may be his family's only flight this year.

A town car arrived at their home and shuttled them directly to the door
of a small airport with plush chairs and no blaring loudspeakers. The
plane left when the family was ready.

It's a far cry from
\href{https://www.nytimes.com/2020/07/10/world/canada/canada-airlines-coronavirus.html}{ever-changing
rules about middle seats},
\href{https://www.nytimes.com/2020/06/18/us/american-airlines-mask-brandon-straka.html}{passengers
refusing to wear masks} and
\href{https://www.nytimes.com/2020/07/08/travel/airplanes-social-distancing-coronavirus.html}{flight
attendants telling off passengers for sitting in an unoccupied exit row}
for more space. And while travelers on commercial airlines report
\href{https://www.nytimes.com/2020/06/04/travel/coronavirus-flying-face-masks.html}{confusion
over mask policies} not being enforced, flying private means everyone
has their face covered.

``I'd always wanted to fly on a private jet, and then I started thinking
about what travel might otherwise be like on a commercial airline,'' Mr.
Antoian said. ``I didn't want to wait in a crowded terminal. I didn't
want the chance that someone on the plane might complain about wearing a
mask and the plane might be delayed.''

``Flying private is much safer, and consistently so,'' said Sridhar
Tayur, founder of OrganJet, a company that provides private jet travel
for organ-transplant patients. ``Social distancing is easier. The pilots
wear masks. The passengers --- usually a small number --- know each
other.''

\hypertarget{latest-updates-global-coronavirus-outbreak}{%
\section{\texorpdfstring{\href{https://www.nytimes.com/2020/08/04/world/coronavirus-cases.html?action=click\&pgtype=Article\&state=default\&region=MAIN_CONTENT_1\&context=storylines_live_updates}{Latest
Updates: Global Coronavirus
Outbreak}}{Latest Updates: Global Coronavirus Outbreak}}\label{latest-updates-global-coronavirus-outbreak}}

Updated 2020-08-05T05:55:41.927Z

\begin{itemize}
\tightlist
\item
  \href{https://www.nytimes.com/2020/08/04/world/coronavirus-cases.html?action=click\&pgtype=Article\&state=default\&region=MAIN_CONTENT_1\&context=storylines_live_updates\#link-762df92}{As
  talks drag on, McConnell signals openness to jobless aid extension,
  and negotiators agree on a deadline.}
\item
  \href{https://www.nytimes.com/2020/08/04/world/coronavirus-cases.html?action=click\&pgtype=Article\&state=default\&region=MAIN_CONTENT_1\&context=storylines_live_updates\#link-1228a480}{Novavax
  sees encouraging results from two studies of its experimental
  vaccine.}
\item
  \href{https://www.nytimes.com/2020/08/04/world/coronavirus-cases.html?action=click\&pgtype=Article\&state=default\&region=MAIN_CONTENT_1\&context=storylines_live_updates\#link-794484ed}{Mississippians
  must now wear masks in public, governor says.}
\end{itemize}

\href{https://www.nytimes.com/2020/08/04/world/coronavirus-cases.html?action=click\&pgtype=Article\&state=default\&region=MAIN_CONTENT_1\&context=storylines_live_updates}{See
more updates}

More live coverage:
\href{https://www.nytimes.com/live/2020/08/04/business/stock-market-today-coronavirus?action=click\&pgtype=Article\&state=default\&region=MAIN_CONTENT_1\&context=storylines_live_updates}{Markets}

The major drawback for many travelers is, of course, the cost. A one-way
charter flight between New York and Miami with the private jet company
\href{https://www.silverair.com/}{Silver Air} costs between \$15,000 and
\$20,000 for the entire aircraft, depending on the jet (their planes
seat between four and 10). Bring nine friends, and that still amounts to
a few thousand dollars per person each way --- significantly more than
the cost of your average first-class ticket, and far more than the price
of a basic economy seat. Another company,
\href{https://gojetit.com/}{Jet It,} charges \$4,200 per hour (though
purchasing a membership reduces the per-hour rate to \$1,600), not
including airport fees. Their HondaJet Elite aircraft seats six.

To reduce the price of the \$8,000-to-\$10,000-per-hour flight, Jamie
Gibson, the founder of the website
\href{https://www.flightess.com/}{Flightess} and a high-end charter
flight attendant, says more groups of first-time fliers are chartering
planes with friends and family, and thus reducing the per-person cost.
Prepandemic, her regular clients were executives who tended to travel
alone. The cost is further reduced by the CARES Act tax break. Private
jet customers aren't required to pay the 7.5 percent Federal Excise Tax
between March 28 and Dec. 31, 2020, which is typically charged on all
private jet flights and hours. Additionally, companies don't have to pay
any fuel taxes during that period, which is one less cost they would
otherwise pass onto consumers.

\hypertarget{gaining-in-popularity}{%
\subsection{Gaining in popularity}\label{gaining-in-popularity}}

While
\href{https://www.nytimes.com/2020/05/10/business/airlines-coronavirus-bleak-future.html}{commercial
air travel} is getting pummeled, private jet travel has not been hit
nearly as hard, said Mr. Gollan.

In April, passenger count on commercial airlines fell 95 percent
year-over-year, while passenger count on private jet charters was down
67 percent, according to Mr. Gollan's analysis of Argus's data. By June,
private jet operators saw just a 22 percent decrease.

``With virtually no business travel, the rebound was fueled by existing
customers flying for personal reasons and newcomers to the market,'' Mr.
Gollan said. ``Private flying isn't fully back, but certainly the
industry is in much better shape than airlines. There is a strong flow
of new-to-private-aviation customers.''

XO, which offers both private charters and the ability to book
individual seats on private jets, saw a 19.8 percent decrease in hours
flown in the first half of 2020 versus the first half of 2019, according
to Argus data. But the company said monthly membership sales between
March and May 2020 among first-time private jet fliers averaged five
times higher than their monthly averages.

Two other companies have also seen increased interest.
\href{https://www.sentient.com/}{Sentient Jet} said more than 50 percent
of the 8,000 flight hours in June were sold to first-time customers, up
from about 25 to 30 percent in most months. And
\href{https://www.aircharterserviceusa.com/}{Air Charter Service} said
in a press release that in May and June, it saw a 75 percent increase in
year-over-year inquiries from potential customers.

The trend looks likely to continue as commercial air travel may only
become more painful. JetBlue is blocking middle seats through at least
Sept. 8 and Southwest Airlines is doing the same through at least Oct.
31 --- but it's unclear what happens after that. Luxuries like airport
lounges are closed with no indication when they'll reopen. And
passengers report
\href{https://www.nytimes.com/2020/06/18/travel/travel-refunds-airlines.html}{flights
being canceled at the last minute}.

\hypertarget{who-is-flying-private}{%
\subsection{Who is flying private}\label{who-is-flying-private}}

Ms. Gibson said in addition to families and friends on vacation, she's
recently flown students who needed to return from college or boarding
schools and older passengers who feel especially at risk flying
commercial airlines. And as airlines cut back on international flights
in response to
\href{https://www.nytimes.com/article/coronavirus-travel-restrictions.html}{countries
closing their borders} to some foreigners, including Americans, she's
also flying a number of repatriation trips.

\href{https://www.nytimes.com/news-event/coronavirus?action=click\&pgtype=Article\&state=default\&region=MAIN_CONTENT_3\&context=storylines_faq}{}

\hypertarget{the-coronavirus-outbreak-}{%
\subsubsection{The Coronavirus Outbreak
›}\label{the-coronavirus-outbreak-}}

\hypertarget{frequently-asked-questions}{%
\paragraph{Frequently Asked
Questions}\label{frequently-asked-questions}}

Updated August 4, 2020

\begin{itemize}
\item ~
  \hypertarget{i-have-antibodies-am-i-now-immune}{%
  \paragraph{I have antibodies. Am I now
  immune?}\label{i-have-antibodies-am-i-now-immune}}

  \begin{itemize}
  \tightlist
  \item
    As of right
    now,\href{https://www.nytimes.com/2020/07/22/health/covid-antibodies-herd-immunity.html?action=click\&pgtype=Article\&state=default\&region=MAIN_CONTENT_3\&context=storylines_faq}{that
    seems likely, for at least several months.} There have been
    frightening accounts of people suffering what seems to be a second
    bout of Covid-19. But experts say these patients may have a
    drawn-out course of infection, with the virus taking a slow toll
    weeks to months after initial exposure. People infected with the
    coronavirus typically
    \href{https://www.nature.com/articles/s41586-020-2456-9}{produce}
    immune molecules called antibodies, which are
    \href{https://www.nytimes.com/2020/05/07/health/coronavirus-antibody-prevalence.html?action=click\&pgtype=Article\&state=default\&region=MAIN_CONTENT_3\&context=storylines_faq}{protective
    proteins made in response to an
    infection}\href{https://www.nytimes.com/2020/05/07/health/coronavirus-antibody-prevalence.html?action=click\&pgtype=Article\&state=default\&region=MAIN_CONTENT_3\&context=storylines_faq}{.
    These antibodies may} last in the body
    \href{https://www.nature.com/articles/s41591-020-0965-6}{only two to
    three months}, which may seem worrisome, but that's perfectly normal
    after an acute infection subsides, said Dr. Michael Mina, an
    immunologist at Harvard University. It may be possible to get the
    coronavirus again, but it's highly unlikely that it would be
    possible in a short window of time from initial infection or make
    people sicker the second time.
  \end{itemize}
\item ~
  \hypertarget{im-a-small-business-owner-can-i-get-relief}{%
  \paragraph{I'm a small-business owner. Can I get
  relief?}\label{im-a-small-business-owner-can-i-get-relief}}

  \begin{itemize}
  \tightlist
  \item
    The
    \href{https://www.nytimes.com/article/small-business-loans-stimulus-grants-freelancers-coronavirus.html?action=click\&pgtype=Article\&state=default\&region=MAIN_CONTENT_3\&context=storylines_faq}{stimulus
    bills enacted in March} offer help for the millions of American
    small businesses. Those eligible for aid are businesses and
    nonprofit organizations with fewer than 500 workers, including sole
    proprietorships, independent contractors and freelancers. Some
    larger companies in some industries are also eligible. The help
    being offered, which is being managed by the Small Business
    Administration, includes the Paycheck Protection Program and the
    Economic Injury Disaster Loan program. But lots of folks have
    \href{https://www.nytimes.com/interactive/2020/05/07/business/small-business-loans-coronavirus.html?action=click\&pgtype=Article\&state=default\&region=MAIN_CONTENT_3\&context=storylines_faq}{not
    yet seen payouts.} Even those who have received help are confused:
    The rules are draconian, and some are stuck sitting on
    \href{https://www.nytimes.com/2020/05/02/business/economy/loans-coronavirus-small-business.html?action=click\&pgtype=Article\&state=default\&region=MAIN_CONTENT_3\&context=storylines_faq}{money
    they don't know how to use.} Many small-business owners are getting
    less than they expected or
    \href{https://www.nytimes.com/2020/06/10/business/Small-business-loans-ppp.html?action=click\&pgtype=Article\&state=default\&region=MAIN_CONTENT_3\&context=storylines_faq}{not
    hearing anything at all.}
  \end{itemize}
\item ~
  \hypertarget{what-are-my-rights-if-i-am-worried-about-going-back-to-work}{%
  \paragraph{What are my rights if I am worried about going back to
  work?}\label{what-are-my-rights-if-i-am-worried-about-going-back-to-work}}

  \begin{itemize}
  \tightlist
  \item
    Employers have to provide
    \href{https://www.osha.gov/SLTC/covid-19/standards.html}{a safe
    workplace} with policies that protect everyone equally.
    \href{https://www.nytimes.com/article/coronavirus-money-unemployment.html?action=click\&pgtype=Article\&state=default\&region=MAIN_CONTENT_3\&context=storylines_faq}{And
    if one of your co-workers tests positive for the coronavirus, the
    C.D.C.} has said that
    \href{https://www.cdc.gov/coronavirus/2019-ncov/community/guidance-business-response.html}{employers
    should tell their employees} -\/- without giving you the sick
    employee's name -\/- that they may have been exposed to the virus.
  \end{itemize}
\item ~
  \hypertarget{should-i-refinance-my-mortgage}{%
  \paragraph{Should I refinance my
  mortgage?}\label{should-i-refinance-my-mortgage}}

  \begin{itemize}
  \tightlist
  \item
    \href{https://www.nytimes.com/article/coronavirus-money-unemployment.html?action=click\&pgtype=Article\&state=default\&region=MAIN_CONTENT_3\&context=storylines_faq}{It
    could be a good idea,} because mortgage rates have
    \href{https://www.nytimes.com/2020/07/16/business/mortgage-rates-below-3-percent.html?action=click\&pgtype=Article\&state=default\&region=MAIN_CONTENT_3\&context=storylines_faq}{never
    been lower.} Refinancing requests have pushed mortgage applications
    to some of the highest levels since 2008, so be prepared to get in
    line. But defaults are also up, so if you're thinking about buying a
    home, be aware that some lenders have tightened their standards.
  \end{itemize}
\item ~
  \hypertarget{what-is-school-going-to-look-like-in-september}{%
  \paragraph{What is school going to look like in
  September?}\label{what-is-school-going-to-look-like-in-september}}

  \begin{itemize}
  \tightlist
  \item
    It is unlikely that many schools will return to a normal schedule
    this fall, requiring the grind of
    \href{https://www.nytimes.com/2020/06/05/us/coronavirus-education-lost-learning.html?action=click\&pgtype=Article\&state=default\&region=MAIN_CONTENT_3\&context=storylines_faq}{online
    learning},
    \href{https://www.nytimes.com/2020/05/29/us/coronavirus-child-care-centers.html?action=click\&pgtype=Article\&state=default\&region=MAIN_CONTENT_3\&context=storylines_faq}{makeshift
    child care} and
    \href{https://www.nytimes.com/2020/06/03/business/economy/coronavirus-working-women.html?action=click\&pgtype=Article\&state=default\&region=MAIN_CONTENT_3\&context=storylines_faq}{stunted
    workdays} to continue. California's two largest public school
    districts --- Los Angeles and San Diego --- said on July 13, that
    \href{https://www.nytimes.com/2020/07/13/us/lausd-san-diego-school-reopening.html?action=click\&pgtype=Article\&state=default\&region=MAIN_CONTENT_3\&context=storylines_faq}{instruction
    will be remote-only in the fall}, citing concerns that surging
    coronavirus infections in their areas pose too dire a risk for
    students and teachers. Together, the two districts enroll some
    825,000 students. They are the largest in the country so far to
    abandon plans for even a partial physical return to classrooms when
    they reopen in August. For other districts, the solution won't be an
    all-or-nothing approach.
    \href{https://bioethics.jhu.edu/research-and-outreach/projects/eschool-initiative/school-policy-tracker/}{Many
    systems}, including the nation's largest, New York City, are
    devising
    \href{https://www.nytimes.com/2020/06/26/us/coronavirus-schools-reopen-fall.html?action=click\&pgtype=Article\&state=default\&region=MAIN_CONTENT_3\&context=storylines_faq}{hybrid
    plans} that involve spending some days in classrooms and other days
    online. There's no national policy on this yet, so check with your
    municipal school system regularly to see what is happening in your
    community.
  \end{itemize}
\end{itemize}

Private jet travel allows citizens of other countries to find a way
home. For repatriation flights from the United States to a country where
travel is restricted to citizens only, the plane can land, but Ms.
Gibson and her crewmates can't set foot on foreign land. The passenger
departs, and the crew immediately leaves the country. It is not advised
to use a private jet to skirt entry restrictions --- just look at the
\href{https://www.nytimes.com/2020/07/07/world/europe/american-passport-privilege-coronavirus.html}{five
American travelers who chartered a private jet to Sardinia}, but were
turned away upon arrival.

Even dogs are flying on chartered planes. Elsa Chen, a Bernedoodle
puppy, was purchased by her owners through a website called
\href{https://www.puppyspot.com/}{PuppySpot}. They paid the company's
standard flat rate of \$799 to send dogs via air cargo. But when Elsa's
American Airlines flight from Chicago O'Hare to San Francisco was
canceled last month and could not be rebooked for several days,
PuppySpot rebooked Elsa on a private jet and had her arrive in San
Francisco nearly on schedule. As a result, PuppySpot is now flying all
of its dogs on private planes.

\hypertarget{mask-gloves-and-cleaning}{%
\subsection{Mask, gloves and cleaning}\label{mask-gloves-and-cleaning}}

These days, most passengers' biggest safety concerns center on Covid-19.
Ms. Gibson and her crewmates now wear a mask and gloves throughout the
flight, but she said some customers still opt out of most in-flight
service as a precaution. About 15 percent of her clients now prefer
plastic plates in lieu of fine porcelain china to minimize risk, and
about the same number ask to be mostly left alone in the cabin to
maintain distancing.

JSX said they've always wiped down high-touch areas like seats, armrests
and tray tables at the start and end of each day. Since the Covid-19
pandemic, they've ramped up cleaning to occur ``throughout the day''
with hospital-grade disinfectant.

With most of the modern stresses of commercial travel absent, Mr.
Antoian said his private jet experience harkened back to the 1950s era
of ``the Golden Age of plane travel,'' a time when flying felt
glamorous.

``You're not just offering coffee or tea,'' Ms. Gibson said. ``You're
offering a cappuccino or espresso. You're not just handing them a bag of
cookies or peanuts. I offer to bake them a souffle. Any custom food
requests, we can order or make.''

Mr. Antoian didn't ask for any custom orders (his kids ate sandwiches
from home) --- but it wasn't out of any coronavirus-related caution.

``I just didn't want to inconvenience anyone,'' he said. Even without a
souffle, Mr. Antoian said the experience was well worth it for (most of)
his family.

His 4-year-old son still prefers Delta.

``He knows how to work the Delta TV and how to navigate the Disney
movies,'' he said. ``He had to watch it on my wife's iPad. He was
disappointed.''

\begin{center}\rule{0.5\linewidth}{\linethickness}\end{center}

\emph{\textbf{Follow New York Times Travel}}
\emph{on}\href{https://www.instagram.com/nytimestravel/}{\emph{Instagram}}\emph{,}\href{https://twitter.com/nytimestravel}{\emph{Twitter}}
\emph{and}\href{https://www.facebook.com/nytimestravel/}{\emph{Facebook}}\emph{.
And}\href{https://www.nytimes.com/newsletters/traveldispatch}{\emph{sign
up for our weekly Travel Dispatch newsletter}} \emph{to receive expert
tips on traveling smarter and inspiration for your next vacation.}

Advertisement

\protect\hyperlink{after-bottom}{Continue reading the main story}

\hypertarget{site-index}{%
\subsection{Site Index}\label{site-index}}

\hypertarget{site-information-navigation}{%
\subsection{Site Information
Navigation}\label{site-information-navigation}}

\begin{itemize}
\tightlist
\item
  \href{https://help.nytimes.com/hc/en-us/articles/115014792127-Copyright-notice}{©~2020~The
  New York Times Company}
\end{itemize}

\begin{itemize}
\tightlist
\item
  \href{https://www.nytco.com/}{NYTCo}
\item
  \href{https://help.nytimes.com/hc/en-us/articles/115015385887-Contact-Us}{Contact
  Us}
\item
  \href{https://www.nytco.com/careers/}{Work with us}
\item
  \href{https://nytmediakit.com/}{Advertise}
\item
  \href{http://www.tbrandstudio.com/}{T Brand Studio}
\item
  \href{https://www.nytimes.com/privacy/cookie-policy\#how-do-i-manage-trackers}{Your
  Ad Choices}
\item
  \href{https://www.nytimes.com/privacy}{Privacy}
\item
  \href{https://help.nytimes.com/hc/en-us/articles/115014893428-Terms-of-service}{Terms
  of Service}
\item
  \href{https://help.nytimes.com/hc/en-us/articles/115014893968-Terms-of-sale}{Terms
  of Sale}
\item
  \href{https://spiderbites.nytimes.com}{Site Map}
\item
  \href{https://help.nytimes.com/hc/en-us}{Help}
\item
  \href{https://www.nytimes.com/subscription?campaignId=37WXW}{Subscriptions}
\end{itemize}
