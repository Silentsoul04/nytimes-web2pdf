Sections

SEARCH

\protect\hyperlink{site-content}{Skip to
content}\protect\hyperlink{site-index}{Skip to site index}

\href{https://myaccount.nytimes.com/auth/login?response_type=cookie\&client_id=vi}{}

\href{https://www.nytimes.com/section/todayspaper}{Today's Paper}

\href{/section/upshot}{The Upshot}\textbar{}Why Trump's Blunt Appeals to
Suburban Voters May Not Work

\url{https://nyti.ms/2Xbi26T}

\begin{itemize}
\item
\item
\item
\item
\item
\item
\end{itemize}

\begin{itemize}
\item
  \href{https://www.nytimes.com/2020/08/04/us/elections/primary-election-michigan-arizona-kansas.html?action=click\&pgtype=Article\&state=default\&region=TOP_BANNER\&context=storylines_menu}{Election
  Updates}
\item
  \href{https://www.nytimes.com/article/biden-vice-president-2020.html?action=click\&pgtype=Article\&state=default\&region=TOP_BANNER\&context=storylines_menu}{Biden's
  V.P. Search}
\item
  \href{https://www.nytimes.com/interactive/2020/07/24/us/politics/trump-biden-campaign-donors.html?action=click\&pgtype=Article\&state=default\&region=TOP_BANNER\&context=storylines_menu}{Map
  of Donations}
\item
  \href{https://www.nytimes.com/interactive/2020/us/elections/delegate-count-primary-results.html?action=click\&pgtype=Article\&state=default\&region=TOP_BANNER\&context=storylines_menu}{Delegate
  Count}
\item
  \href{https://www.nytimes.com/interactive/2019/us/politics/2020-presidential-candidates.html?action=click\&pgtype=Article\&state=default\&region=TOP_BANNER\&context=storylines_menu}{The
  Candidates}
\item
  \href{https://www.nytimes.com/newsletters/politics?action=click\&pgtype=Article\&state=default\&region=TOP_BANNER\&context=storylines_menu}{Politics
  Newsletter}
\end{itemize}

Advertisement

\protect\hyperlink{after-top}{Continue reading the main story}

Upshot

Supported by

\protect\hyperlink{after-sponsor}{Continue reading the main story}

\hypertarget{why-trumps-blunt-appeals-to-suburban-voters-may-not-work}{%
\section{Why Trump's Blunt Appeals to Suburban Voters May Not
Work}\label{why-trumps-blunt-appeals-to-suburban-voters-may-not-work}}

People living the ``Suburban Lifestyle Dream'' tend to support recent
protests and disapprove of the president's handling of race.

\href{https://www.nytimes.com/by/emily-badger}{\includegraphics{https://static01.nyt.com/images/2018/02/16/multimedia/author-emily-badger/author-emily-badger-thumbLarge-v2.png}}\href{https://www.nytimes.com/by/nate-cohn}{\includegraphics{https://static01.nyt.com/images/2018/06/13/multimedia/author-nate-cohn/author-nate-cohn-thumbLarge.jpg}}

By \href{https://www.nytimes.com/by/emily-badger}{Emily Badger} and
\href{https://www.nytimes.com/by/nate-cohn}{Nate Cohn}

\begin{itemize}
\item
  Published July 30, 2020Updated Aug. 1, 2020
\item
  \begin{itemize}
  \item
  \item
  \item
  \item
  \item
  \item
  \end{itemize}
\end{itemize}

\includegraphics{https://static01.nyt.com/images/2020/07/30/upshot/30up-unrest-suburbs/merlin_175027380_4d52b94c-d484-46af-b076-aefc27a1be02-articleLarge.jpg?quality=75\&auto=webp\&disable=upscale}

President Trump's latest campaign ads
\href{https://www.nytimes.com/2020/07/21/us/politics/trump-campaign-ads.html}{warn
of left-wing mobs destroying American cities}. His recent White House
comments have depicted
\href{https://www.whitehouse.gov/briefings-statements/remarks-president-trump-operation-legend-combatting-violent-crime-american-cities/}{a
rampage of violence} and a ``radical movement'' to dissolve the police.
His Twitter feed has sounded alarms over an Obama-era fair housing rule
he has framed as a threat to
``\href{https://twitter.com/realDonaldTrump/status/1286372175117791236}{The
Suburban Housewives of America}'' and the
``\href{https://twitter.com/realDonaldTrump/status/1288509568578777088?s=20}{Suburban
Lifestyle Dream}.''

It all amounts, with little subtlety, to a play on the perceived fears
of suburban voters. But there are several reasons to believe that a
strategy that worked for Richard Nixon on the heels of urban unrest in
1968 is less likely to be effective for Donald Trump in 2020.

For one, these are not the American suburbs of the 1960s (and they have
a lot fewer housewives). The scale of urban violence and the threats to
that suburban lifestyle are a faint echo of that time. And while polling
shows that suburban voters disapprove of the president's job in general,
they disapprove even more of his handling of the very issues he is
trying to elevate.

Over all, just 38 percent of voters in the suburbs approve of Mr.
Trump's job performance compared with 59 percent who disapprove,
according to a New York Times/Siena College poll in June. Suburban
voters disapproved of Mr. Trump's handling of recent protests and race
relations by an even wider margin, and 65 percent had a favorable view
of the Black Lives Matter movement.

The president's attention to suburban areas is understandable. Nearly
half of voters live in a suburb, defined here as the parts of
metropolitan areas that lie outside central cities, like Philadelphia or
Baltimore, and that
\href{https://www.census.gov/programs-surveys/geography/guidance/geo-areas/urban-rural/2010-urban-rural.html}{aren't
considered rural} by the Census Bureau. In the Times/Siena poll, Mr.
Trump trailed Joe Biden by 16 points, 51 percent to 35 percent, in
suburban areas, notably worse than his eight-point deficit in similar
areas against Hillary Clinton in 2016.

The president's disadvantage in the suburbs is underpinned by his
longstanding weakness among white voters with a four-year college
degree, who back Mr. Biden, 57-31, in the suburbs.

\hypertarget{latest-updates-2020-election}{%
\section{\texorpdfstring{\href{https://www.nytimes.com/2020/08/04/us/elections/primary-election-michigan-arizona-kansas.html?action=click\&pgtype=Article\&state=default\&region=MAIN_CONTENT_1\&context=storylines_live_updates}{Latest
Updates: 2020
Election}}{Latest Updates: 2020 Election}}\label{latest-updates-2020-election}}

Updated 2020-08-04T18:27:53.780Z

\begin{itemize}
\tightlist
\item
  \href{https://www.nytimes.com/2020/08/04/us/elections/primary-election-michigan-arizona-kansas.html?action=click\&pgtype=Article\&state=default\&region=MAIN_CONTENT_1\&context=storylines_live_updates\#link-3924dd44}{Two
  G.O.P. Senate primaries offer --- what else? --- a test of loyalty to
  Trump.}
\item
  \href{https://www.nytimes.com/2020/08/04/us/elections/primary-election-michigan-arizona-kansas.html?action=click\&pgtype=Article\&state=default\&region=MAIN_CONTENT_1\&context=storylines_live_updates\#link-32b39e33}{President
  Trump is suddenly a big supporter of mail-in voting --- in Florida.}
\item
  \href{https://www.nytimes.com/2020/08/04/us/elections/primary-election-michigan-arizona-kansas.html?action=click\&pgtype=Article\&state=default\&region=MAIN_CONTENT_1\&context=storylines_live_updates\#link-6d019753}{Election
  experts warn Congress about widespread disenfranchisement of voters of
  color in November.}
\end{itemize}

\href{https://www.nytimes.com/2020/08/04/us/elections/primary-election-michigan-arizona-kansas.html?action=click\&pgtype=Article\&state=default\&region=MAIN_CONTENT_1\&context=storylines_live_updates}{See
more updates}

Fifty years ago, white voters with a college degree were a relative
rarity, even in the suburbs. The white voters who fled cities often held
blue-collar industrial jobs, and many embraced Republican messages on
crime and race. To some extent they still do: White voters without a
degree in the suburbs back Mr. Trump in the Times/Siena survey, 52
percent to 35 percent, but they made up only 37 percent of registered
voters surveyed in the suburbs.

If white suburban voters in general are President Trump's audience for
his recent law-and-order messages, the scenes from Portland, Ore., have
only complicated his pitch. White suburban mothers looking on across the
country have seen not marauding criminals, but
\href{https://www.nytimes.com/2020/07/27/parenting/wall-of-moms-protests.html?action=click\&module=Top\%20Stories\&pgtype=Homepage}{women
who look a lot like them}.

``The images that are emerging as the most indelible in the public mind
are a line of mothers taking the tear gas,'' said Rick Perlstein, a
historian who has written extensively on the Nixon era. ``Or
\href{https://www.nytimes.com/2020/07/20/us/portland-protests-navy-christopher-david.html}{a
53-year-old Navy vet} asking people to honor their oath to the
Constitution of the United States.''

There have also been
\href{https://www.oregonlive.com/news/2020/07/dad-with-leaf-blower-arrested-tuesday-says-he-was-taken-to-ground-by-federal-officers-during-portland-protest.html}{the
dads with leaf blowers}. And the peaceful protesters who were violently
cleared from Lafayette Square outside the White House in June for a
presidential photo op may represent
\href{https://www.nytimes.com/2020/06/02/us/politics/trump-walk-lafayette-square.html}{one
of the enduring scenes of Mr. Trump's presidency}.

In this moment, President Trump also differs from Nixon in 1968 in a
crucial way. Nixon wasn't yet president; he wasn't in charge. It's much
harder to run against disorder when it happens on your watch, Mr.
Perlstein said. And fanning fears of crime and violence was less
effective for President Nixon later in his presidency for just that
reason.

In recent decades, cities have grown safer, and the suburbs have become
much more racially and economically diverse. They have been
\href{https://www.bloomberg.com/news/features/2020-06-19/protests-for-racial-justice-take-root-in-suburbia}{sites
of Black Lives Matter protests, too}. About one in 10 suburban voters in
the Times/Siena poll said they had participated in such a demonstration.
A clear majority of suburban voters also said they believed there were
broader patterns in America of excessive police violence toward
African-Americans and bias against them in the criminal justice system.

For white suburban voters who do still live in segregated communities,
the historian Matthew Lassiter said that threats today to suburban
exclusion are much weaker than they were when President Nixon was
elected. At the time, busing was still on the table. So was the
possibility that desegregation plans might send students
\href{https://www.oyez.org/cases/1973/73-434}{across city lines to
neighboring school districts}. Courts were still considering whether it
was constitutional \href{https://www.oyez.org/cases/1972/71-1332}{for
wealthy districts to spend far more on education} than poorer ones, or
for suburban municipalities
\href{https://supreme.justia.com/cases/federal/us/402/137/}{to keep out
low-income housing}.

``The threat of comprehensive restructuring of suburban privilege was
real in the late '60s and early '70s because it was coming from the
courts, and it was coming from civil rights litigants who had a federal
judiciary that was going to go all the way with them,'' said Mr.
Lassiter, a professor at the University of Michigan.

That was true until President Nixon put four justices on the Supreme
Court, who together killed many of those remedies to racial and economic
segregation. Today, it's simply less effective to warn that anyone is
coming to destroy the ``Suburban Lifestyle Dream'' of advantaged schools
and
\href{https://www.nytimes.com/interactive/2019/06/18/upshot/cities-across-america-question-single-family-zoning.html}{single-family
neighborhoods} because a previous generation of politicians and white
voters were so successful at protecting it.

President Trump's warnings today --- that Mr. Biden will defund the
police, and take federal control of local zoning laws --- carry
significantly less weight. Mr. Biden has said
\href{https://www.npr.org/sections/live-updates-protests-for-racial-justice/2020/06/08/872376757/biden-opposes-defunding-police-campaign-says}{he
doesn't support defunding the police}. And the Obama-era fair-housing
rule,
\href{https://www.nytimes.com/2020/07/23/us/politics/trump-housing-discrimination-suburbs.html}{which
the Trump administration announced it would end last week}, is both too
bureaucratic and incremental to be easily wielded as a boogeyman. Its
central goal was to prod local governments to consider segregation
patterns in their planning.

At times, President Trump himself has
\href{https://twitter.com/realDonaldTrump/status/1278136326647406593}{seemed
unsure how to describe} what's so scary about that,
\href{https://twitter.com/realDonaldTrump/status/1286372175117791236}{leaving
those arguments}
\href{https://www.nationalreview.com/corner/biden-and-dems-are-set-to-abolish-the-suburbs/}{to
op-eds by others}. If anything, his foray into the topic may teach some
moderate and liberal voters that their yard signs
\href{https://www.nytimes.com/2018/08/21/upshot/home-ownership-nimby-bipartisan.html}{opposing
affordable housing and denser zoning} put them in awkward alignment with
President Trump.

Historians who look back at the Nixon era add that this president is
unlikely to succeed with white suburban voters for one more reason: He's
not as subtle about it as President Nixon, or Vice President Spiro
Agnew, or Ronald Reagan after them.

``They understood something about race that Trump doesn't understand,''
Mr. Lassiter said. ``Voters don't want racial privilege challenged, but
they don't want to be explicitly reminded that racism is underneath
their position.''

Because that tension persists, the historian Lily Geismer is skeptical
that white suburban voters who support Black Lives Matter protests now
--- and may be Biden voters in the fall --- will also back affordable
housing that would diversify their neighborhoods or support city budgets
that would cut police funding. In the Times/Siena survey, 49 percent of
suburban voters said they strongly opposed reducing funding to police
departments.

Professor Geismer, who teaches at Claremont McKenna College, noted that
some of these same voters demonstrating for racial justice today are
also talking about pulling their children from public schools during the
coronavirus crisis and hiring private tutors.

``The idea that we support Black Lives Matter but we're trying to do
everything we can to protect our own children's educational well-being
--- that's the disconnect I see,'' she said.

Ultimately, these are two separate questions: how suburban voters will
respond to President Trump in the fall, and what they'll support after
the election, regardless of its outcome.

\hypertarget{our-2020-election-guide}{%
\section{Our 2020 Election Guide}\label{our-2020-election-guide}}

Updated Aug. 4, 2020

\begin{itemize}
\item
  \begin{center}\rule{0.5\linewidth}{\linethickness}\end{center}

  \hypertarget{the-latest}{%
  \subsection{The Latest}\label{the-latest}}

  \begin{itemize}
  \tightlist
  \item
    Five states are holding primary elections Tuesday, with voters in
    Arizona, Kansas, Michigan, Missouri and Washington State choosing
    nominees for Congress and local offices.
    \href{https://www.nytimes.com/2020/08/04/us/elections/primary-election-michigan-arizona-kansas.html?action=click\&pgtype=Article\&state=default\&region=BELOW_MAIN_CONTENT\&context=storylines_guide}{Follow
    live election updates here.}
  \end{itemize}
\item
  \begin{center}\rule{0.5\linewidth}{\linethickness}\end{center}

  \hypertarget{bidens-vp-search}{%
  \subsection{Biden's V.P. Search}\label{bidens-vp-search}}

  \begin{itemize}
  \tightlist
  \item
    \href{https://www.nytimes.com/article/biden-vice-president-2020.html?action=click\&pgtype=Article\&state=default\&region=BELOW_MAIN_CONTENT\&context=storylines_guide}{Here
    are 13 women} who have been under consideration to be Joe Biden's
    running mate, and why each might be chosen --- and might not be.
  \end{itemize}
\item
  \begin{center}\rule{0.5\linewidth}{\linethickness}\end{center}

  \hypertarget{keep-up-with-our-coverage}{%
  \subsection{Keep Up With Our
  Coverage}\label{keep-up-with-our-coverage}}

  \begin{itemize}
  \tightlist
  \item
    Get an
    \href{https://www.nytimes.com/newsletters/politics?action=click\&pgtype=Article\&state=default\&region=BELOW_MAIN_CONTENT\&context=storylines_guide}{email}
    recapping the day's news
  \end{itemize}

  \begin{itemize}
  \tightlist
  \item
    Download our mobile app on
    \href{https://apps.apple.com/us/app/nytimes/id284862083?ls=1\&mat_click_id=5c79ae7455014fd1bd66b5610c05b8f2-20191112-16948\&referrer=mat_click_id\%3D5c79ae7455014fd1bd66b5610c05b8f2-20191112-16948\%26link_click_id\%3D722930677036718082}{iOS}
    and
    \href{http://a.localytics.com/android?id=com.nytimes.android\&referrer=utm_source\%3Dother_nyt_mobile_web\%26utm_medium\%3DWeb\%2520page\%26utm_term\%3DGeneral\%2520Mobile\%2520Page\%26utm_campaign\%3DNYT\%2520Mobile\%2520General\%2520Page}{Android}
    and turn on Breaking News and Politics alerts
  \end{itemize}
\end{itemize}

Advertisement

\protect\hyperlink{after-bottom}{Continue reading the main story}

\hypertarget{site-index}{%
\subsection{Site Index}\label{site-index}}

\hypertarget{site-information-navigation}{%
\subsection{Site Information
Navigation}\label{site-information-navigation}}

\begin{itemize}
\tightlist
\item
  \href{https://help.nytimes.com/hc/en-us/articles/115014792127-Copyright-notice}{©~2020~The
  New York Times Company}
\end{itemize}

\begin{itemize}
\tightlist
\item
  \href{https://www.nytco.com/}{NYTCo}
\item
  \href{https://help.nytimes.com/hc/en-us/articles/115015385887-Contact-Us}{Contact
  Us}
\item
  \href{https://www.nytco.com/careers/}{Work with us}
\item
  \href{https://nytmediakit.com/}{Advertise}
\item
  \href{http://www.tbrandstudio.com/}{T Brand Studio}
\item
  \href{https://www.nytimes.com/privacy/cookie-policy\#how-do-i-manage-trackers}{Your
  Ad Choices}
\item
  \href{https://www.nytimes.com/privacy}{Privacy}
\item
  \href{https://help.nytimes.com/hc/en-us/articles/115014893428-Terms-of-service}{Terms
  of Service}
\item
  \href{https://help.nytimes.com/hc/en-us/articles/115014893968-Terms-of-sale}{Terms
  of Sale}
\item
  \href{https://spiderbites.nytimes.com}{Site Map}
\item
  \href{https://help.nytimes.com/hc/en-us}{Help}
\item
  \href{https://www.nytimes.com/subscription?campaignId=37WXW}{Subscriptions}
\end{itemize}
