Sections

SEARCH

\protect\hyperlink{site-content}{Skip to
content}\protect\hyperlink{site-index}{Skip to site index}

\href{https://www.nytimes.com/section/us}{U.S.}

\href{https://myaccount.nytimes.com/auth/login?response_type=cookie\&client_id=vi}{}

\href{https://www.nytimes.com/section/todayspaper}{Today's Paper}

\href{/section/us}{U.S.}\textbar{}Native Americans Feel Devastated by
the Virus Yet Overlooked in the Data

\begin{itemize}
\item
\item
\item
\item
\item
\item
\end{itemize}

\href{https://www.nytimes.com/news-event/coronavirus?action=click\&pgtype=Article\&state=default\&region=TOP_BANNER\&context=storylines_menu}{The
Coronavirus Outbreak}

\begin{itemize}
\tightlist
\item
  live\href{https://www.nytimes.com/2020/08/02/world/coronavirus-updates.html?action=click\&pgtype=Article\&state=default\&region=TOP_BANNER\&context=storylines_menu}{Latest
  Updates}
\item
  \href{https://www.nytimes.com/interactive/2020/us/coronavirus-us-cases.html?action=click\&pgtype=Article\&state=default\&region=TOP_BANNER\&context=storylines_menu}{Maps
  and Cases}
\item
  \href{https://www.nytimes.com/interactive/2020/science/coronavirus-vaccine-tracker.html?action=click\&pgtype=Article\&state=default\&region=TOP_BANNER\&context=storylines_menu}{Vaccine
  Tracker}
\item
  \href{https://www.nytimes.com/interactive/2020/07/29/us/schools-reopening-coronavirus.html?action=click\&pgtype=Article\&state=default\&region=TOP_BANNER\&context=storylines_menu}{What
  School May Look Like}
\item
  \href{https://www.nytimes.com/live/2020/07/31/business/stock-market-today-coronavirus?action=click\&pgtype=Article\&state=default\&region=TOP_BANNER\&context=storylines_menu}{Economy}
\end{itemize}

Advertisement

\protect\hyperlink{after-top}{Continue reading the main story}

Supported by

\protect\hyperlink{after-sponsor}{Continue reading the main story}

\hypertarget{native-americans-feel-devastated-by-the-virus-yet-overlooked-in-the-data}{%
\section{Native Americans Feel Devastated by the Virus Yet Overlooked in
the
Data}\label{native-americans-feel-devastated-by-the-virus-yet-overlooked-in-the-data}}

Statistical gaps can make it difficult to properly allocate public
resources to Native Americans. When that's the case, one leader said,
``tribal nations have an effective death sentence.''

\includegraphics{https://static01.nyt.com/images/2020/07/29/us/virus-nativeamericans04/merlin_174878256_eb1cb8a3-0c67-47fd-99e9-23c9820827e9-articleLarge.jpg?quality=75\&auto=webp\&disable=upscale}

By \href{https://www.nytimes.com/by/kate-conger}{Kate Conger},
\href{https://www.nytimes.com/by/robert-gebeloff}{Robert Gebeloff} and
\href{https://www.nytimes.com/by/richard-a-oppel-jr}{Richard A. Oppel
Jr.}

\begin{itemize}
\item
  Published July 30, 2020Updated July 31, 2020
\item
  \begin{itemize}
  \item
  \item
  \item
  \item
  \item
  \item
  \end{itemize}
\end{itemize}

HARRAH, Wash. --- As the coronavirus outbreak in Washington State's
Yakima County worsened last month, Tashina Nunez recognized more and
more of the patients who arrived in her hospital. They had coughs,
fevers and, in some severe cases, respiratory failure. And many of them
were her acquaintances and neighbors, members of the tribes that make up
the Yakama Nation.

Ms. Nunez, a nurse at a hospital in Yakima County and a Yakama Nation
descendant, noticed that Native Americans, who make up about 7 percent
of the county's population, seemed to account for many of the hospital's
virus patients. Because the hospital does not routinely record race and
ethnicity data, she said, it was hard for Ms. Nunez to know for certain.

``Not being counted is not new to us,'' she said. Without firm figures,
she and other health care providers for Native communities said they
struggled to know where or how to intervene to stop the spread. ``You
don't know how bad it is until it's too late,'' Ms. Nunez said.

By mid-July, more than 650 members of the Yakama Nation, in central
Washington State, had contracted the virus --- about 6 percent of the
total membership. Twenty-eight people have died, Delano Saluskin,
chairman of the Yakama Nation,
\href{https://www.facebook.com/YakamaNationInfo/posts/2901001803343892}{said
in a video update}.

``We all grieve those losses,'' he said. ``This has been devastating for
many families on the reservation and it means that, every week, a family
member is impacted.''

The situation among the Yakama Nation is not unique. Even with
significant gaps in the data that is available, there are strong
indications that Native Americans have been disproportionately affected
by the coronavirus.

The rate of known cases in the eight counties with the largest
populations of Native Americans is nearly double the national average, a
New York Times analysis has found. The analysis cannot determine which
individuals are testing positive for the virus, but these counties are
home to one in six U.S. residents who describe themselves in census
surveys as non-Hispanic and American Indian or Alaska Native.

\hypertarget{native-americans-at-risk}{%
\subsection{Native Americans at Risk}\label{native-americans-at-risk}}

Counties with large Native American populations with reported infection
rates above 1,500 cases per 100,000 residents.

Wash.

Yakima

S.D.

Buffalo

Thurston

Iowa

Neb.

Utah

San Juan

San Juan

Coconino

N.C.

Robeson

Apache

Ariz.

Okla.

McKinley

La Paz

Navajo

N.M.

Maricopa

Miss.

McCurtain

Pinal

Neshoba

Wash.

Yakima

S.D.

Buffalo

Lyman

Thurston

Neb.

Utah

D.C.

San Juan

San Juan

Coconino

N.C.

Ariz.

Robeson

McKinley

Apache

Okla.

La Paz

N.M.

Navajo

Miss.

Maricopa

Pinal

Neshoba

McCurtain

La.

Terrebonne

Yakima

Buffalo

Lyman

Thurston

Apache

San Juan

San Juan

Robeson

Coconino

McCurtain

McKinley

La Paz

Navajo

Neshoba

Maricopa

Pinal

Terrebonne

By Scott Reinhard \textbar{} Source: Times database of coronavirus cases
compiled from state and local health agencies as of July 24.

And there are many smaller counties with significant populations of
Native Americans that have elevated case rates, including Yakima County.
The Times identified at least 15 counties that have elevated case rates
and are home to sizable numbers of Native American residents. Those
counties ranged from large metropolitan areas in Arizona to rural
communities in Nebraska and Mississippi.

\hypertarget{latest-updates-global-coronavirus-outbreak}{%
\section{\texorpdfstring{\href{https://www.nytimes.com/2020/08/01/world/coronavirus-covid-19.html?action=click\&pgtype=Article\&state=default\&region=MAIN_CONTENT_1\&context=storylines_live_updates}{Latest
Updates: Global Coronavirus
Outbreak}}{Latest Updates: Global Coronavirus Outbreak}}\label{latest-updates-global-coronavirus-outbreak}}

Updated 2020-08-02T17:52:35.962Z

\begin{itemize}
\tightlist
\item
  \href{https://www.nytimes.com/2020/08/01/world/coronavirus-covid-19.html?action=click\&pgtype=Article\&state=default\&region=MAIN_CONTENT_1\&context=storylines_live_updates\#link-34047410}{The
  U.S. reels as July cases more than double the total of any other
  month.}
\item
  \href{https://www.nytimes.com/2020/08/01/world/coronavirus-covid-19.html?action=click\&pgtype=Article\&state=default\&region=MAIN_CONTENT_1\&context=storylines_live_updates\#link-780ec966}{Top
  U.S. officials work to break an impasse over the federal jobless
  benefit.}
\item
  \href{https://www.nytimes.com/2020/08/01/world/coronavirus-covid-19.html?action=click\&pgtype=Article\&state=default\&region=MAIN_CONTENT_1\&context=storylines_live_updates\#link-2bc8948}{Its
  outbreak untamed, Melbourne goes into even greater lockdown.}
\end{itemize}

\href{https://www.nytimes.com/2020/08/01/world/coronavirus-covid-19.html?action=click\&pgtype=Article\&state=default\&region=MAIN_CONTENT_1\&context=storylines_live_updates}{See
more updates}

More live coverage:
\href{https://www.nytimes.com/live/2020/07/31/business/stock-market-today-coronavirus?action=click\&pgtype=Article\&state=default\&region=MAIN_CONTENT_1\&context=storylines_live_updates}{Markets}

``I feel as though tribal nations have an effective death sentence when
the scale of this pandemic, if it continues to grow, exceeds the public
resources available,'' said Fawn Sharp, the president of the Quinault
Indian Nation and of the National Congress of American Indians.

\href{https://www.nytimes.com/2020/04/09/us/coronavirus-navajo-nation.html}{The
situation has been stark in the Navajo Nation}, where high infection
rates have created a crisis in the largest reservation in the United
States. But health officials say the same worrying trends are repeating
in Native communities across the country, and congressional leaders have
prompted the
\href{https://www.warren.senate.gov/newsroom/press-releases/us-commission-on-civil-rights-agrees-to-warren-haaland-request-to-update-broken-promises-report-and-examine-pandemic-impacts-on-indian-country}{U.S.
Commission on Civil Rights} to examine the health disparities compounded
by the pandemic.

\includegraphics{https://static01.nyt.com/images/2020/07/29/us/virus-nativeamericans02/virus-nativeamericans02-articleLarge-v2.jpg?quality=75\&auto=webp\&disable=upscale}

In New Mexico, Native American and Alaska Native people have accounted
for \href{https://cvprovider.nmhealth.org/public-dashboard.html}{nearly
40 percent of virus cases} even though they make up 9 percent of the
population.

Native Americans in the Phoenix area have been
\href{https://phdata.maricopa.gov/Dashboard/e10a16d8-921f-4aac-b921-26d95e638a45?e=false\&vo=viewonly}{infected
at four times the rate of their white neighbors}. The Fort McDowell
Yavapai Nation
\href{https://www.fmyn.org/tribal-member-letter-extending-shelter-in-place-order-and-tribal-govt-closure/}{extended
a shelter-in-place order} on July 18 because infections were continuing
to multiply. The Salt River Pima-Maricopa Indian Community
\href{https://oan.srpmic-nsn.gov/archives/2020/PDFs/OAN_JULY162020_FINAL.pdf}{also
reported mounting infections} this month.

Outbreaks
\href{https://www.wbtw.com/home/slowing-the-spread-lumbee-tribe-hosts-free-covid-19-testings/}{have
been reported} among the Lumbee Tribe in North Carolina, Choctaw
communities in
\href{https://oklahoman.com/article/5665246/mccurtain-countgy-sees-outbreaks-of-covid-19}{Oklahoma}
and
\href{https://www.clarionledger.com/story/news/2020/07/19/covid-toll-mississippi-band-choctaw-indians/5467655002/}{Mississippi},
and at two reservations in
\href{https://journalstar.com/news/state-and-regional/nebraska/covid-19-cases-increasing-on-omaha-winnebago-reservations/article_7da2cba1-4405-55aa-ab2b-69075aa42d40.html}{Thurston
County, Neb.}

\href{https://www.cdc.gov/coronavirus/2019-ncov/covid-data/covidview/index.html}{Hospitalization
rates published} by the Centers for Disease Control and Prevention also
suggest that Native Americans are overrepresented among those who become
seriously ill from the virus. The data about Covid-19 is collected from
a sample of counties and provides an incomplete picture, but the
conclusion is unsurprising to epidemiologists who study the health of
Native Americans.

``The disparities we see there with Covid are aligned with those that we
see for hospitalizations and deaths due to influenza and other
respiratory viruses,'' said Allison Barlow, director of the Center for
American Indian Health at Johns Hopkins University.

Native Americans --- particularly those living on reservations --- are
more prone to contract the virus because of crowded housing conditions
that make social distancing difficult, she said. And years of
underfunded health systems, food and water insecurity and other factors
contribute to underlying health conditions that can make the illness
more severe once contracted.

Yet understanding the extent of how Native American people have been
disproportionately affected by Covid-19 is extremely difficult.

Calculating how many people who identify as Native American have had the
virus and how many have died of it is nearly impossible because federal
data tracking individual coronavirus cases often omits information about
the race and ethnicity of people; such information is missing from about
half the cases reported to the C.D.C., which serves as a clearinghouse
for cases reported by state and local authorities.

Even when such information is collected, it is uncertain how accurate it
is. Miscounting can begin at testing sites and health clinics, public
health officials said, where health care workers sometimes do not record
a patient's race and ethnicity data, or simply guess without asking a
patient.

The Indian Health Service has identified at least 30,987 cases among
Native Americans and Alaska Natives, but tribal nations are not required
to share their data. Just under half of tribal health centers and 61
percent of urban health services serving Native Americans have provided
case information, an I.H.S. spokeswoman said.

After suing the C.D.C.,
\href{https://www.nytimes.com/interactive/2020/07/05/us/coronavirus-latinos-african-americans-cdc-data.html}{The
Times obtained a database with the characteristics of 1.5 million
individuals} who tested positive for the virus through the end of May.
The data showed that people who were Black or Latino were three times as
likely to become infected as people who were white.

The data provided only part of the picture, though, when it came to
Native Americans because of gaps in the data: It included geographic
information and racial classifications for just 974 of the 3,143
counties in the nation, and did not include some of the places where
Native American people make up large parts of the population. What
information there was did show a disparity: The infection rate for
Native Americans was 1.7 times the rate for white people over all, and
somewhat higher in younger age groups.

In the Yakama Nation, Haver Jim Ptxunu, a 42-year-old resident who works
for the tribal power company and helps run a nonprofit group called the
Peacekeeper Society, said he and his wife contracted the virus in June.

\href{https://www.nytimes.com/news-event/coronavirus?action=click\&pgtype=Article\&state=default\&region=MAIN_CONTENT_3\&context=storylines_faq}{}

\hypertarget{the-coronavirus-outbreak-}{%
\subsubsection{The Coronavirus Outbreak
›}\label{the-coronavirus-outbreak-}}

\hypertarget{frequently-asked-questions}{%
\paragraph{Frequently Asked
Questions}\label{frequently-asked-questions}}

Updated July 27, 2020

\begin{itemize}
\item ~
  \hypertarget{should-i-refinance-my-mortgage}{%
  \paragraph{Should I refinance my
  mortgage?}\label{should-i-refinance-my-mortgage}}

  \begin{itemize}
  \tightlist
  \item
    \href{https://www.nytimes.com/article/coronavirus-money-unemployment.html?action=click\&pgtype=Article\&state=default\&region=MAIN_CONTENT_3\&context=storylines_faq}{It
    could be a good idea,} because mortgage rates have
    \href{https://www.nytimes.com/2020/07/16/business/mortgage-rates-below-3-percent.html?action=click\&pgtype=Article\&state=default\&region=MAIN_CONTENT_3\&context=storylines_faq}{never
    been lower.} Refinancing requests have pushed mortgage applications
    to some of the highest levels since 2008, so be prepared to get in
    line. But defaults are also up, so if you're thinking about buying a
    home, be aware that some lenders have tightened their standards.
  \end{itemize}
\item ~
  \hypertarget{what-is-school-going-to-look-like-in-september}{%
  \paragraph{What is school going to look like in
  September?}\label{what-is-school-going-to-look-like-in-september}}

  \begin{itemize}
  \tightlist
  \item
    It is unlikely that many schools will return to a normal schedule
    this fall, requiring the grind of
    \href{https://www.nytimes.com/2020/06/05/us/coronavirus-education-lost-learning.html?action=click\&pgtype=Article\&state=default\&region=MAIN_CONTENT_3\&context=storylines_faq}{online
    learning},
    \href{https://www.nytimes.com/2020/05/29/us/coronavirus-child-care-centers.html?action=click\&pgtype=Article\&state=default\&region=MAIN_CONTENT_3\&context=storylines_faq}{makeshift
    child care} and
    \href{https://www.nytimes.com/2020/06/03/business/economy/coronavirus-working-women.html?action=click\&pgtype=Article\&state=default\&region=MAIN_CONTENT_3\&context=storylines_faq}{stunted
    workdays} to continue. California's two largest public school
    districts --- Los Angeles and San Diego --- said on July 13, that
    \href{https://www.nytimes.com/2020/07/13/us/lausd-san-diego-school-reopening.html?action=click\&pgtype=Article\&state=default\&region=MAIN_CONTENT_3\&context=storylines_faq}{instruction
    will be remote-only in the fall}, citing concerns that surging
    coronavirus infections in their areas pose too dire a risk for
    students and teachers. Together, the two districts enroll some
    825,000 students. They are the largest in the country so far to
    abandon plans for even a partial physical return to classrooms when
    they reopen in August. For other districts, the solution won't be an
    all-or-nothing approach.
    \href{https://bioethics.jhu.edu/research-and-outreach/projects/eschool-initiative/school-policy-tracker/}{Many
    systems}, including the nation's largest, New York City, are
    devising
    \href{https://www.nytimes.com/2020/06/26/us/coronavirus-schools-reopen-fall.html?action=click\&pgtype=Article\&state=default\&region=MAIN_CONTENT_3\&context=storylines_faq}{hybrid
    plans} that involve spending some days in classrooms and other days
    online. There's no national policy on this yet, so check with your
    municipal school system regularly to see what is happening in your
    community.
  \end{itemize}
\item ~
  \hypertarget{is-the-coronavirus-airborne}{%
  \paragraph{Is the coronavirus
  airborne?}\label{is-the-coronavirus-airborne}}

  \begin{itemize}
  \tightlist
  \item
    The coronavirus
    \href{https://www.nytimes.com/2020/07/04/health/239-experts-with-one-big-claim-the-coronavirus-is-airborne.html?action=click\&pgtype=Article\&state=default\&region=MAIN_CONTENT_3\&context=storylines_faq}{can
    stay aloft for hours in tiny droplets in stagnant air}, infecting
    people as they inhale, mounting scientific evidence suggests. This
    risk is highest in crowded indoor spaces with poor ventilation, and
    may help explain super-spreading events reported in meatpacking
    plants, churches and restaurants.
    \href{https://www.nytimes.com/2020/07/06/health/coronavirus-airborne-aerosols.html?action=click\&pgtype=Article\&state=default\&region=MAIN_CONTENT_3\&context=storylines_faq}{It's
    unclear how often the virus is spread} via these tiny droplets, or
    aerosols, compared with larger droplets that are expelled when a
    sick person coughs or sneezes, or transmitted through contact with
    contaminated surfaces, said Linsey Marr, an aerosol expert at
    Virginia Tech. Aerosols are released even when a person without
    symptoms exhales, talks or sings, according to Dr. Marr and more
    than 200 other experts, who
    \href{https://academic.oup.com/cid/article/doi/10.1093/cid/ciaa939/5867798}{have
    outlined the evidence in an open letter to the World Health
    Organization}.
  \end{itemize}
\item ~
  \hypertarget{what-are-the-symptoms-of-coronavirus}{%
  \paragraph{What are the symptoms of
  coronavirus?}\label{what-are-the-symptoms-of-coronavirus}}

  \begin{itemize}
  \tightlist
  \item
    Common symptoms
    \href{https://www.nytimes.com/article/symptoms-coronavirus.html?action=click\&pgtype=Article\&state=default\&region=MAIN_CONTENT_3\&context=storylines_faq}{include
    fever, a dry cough, fatigue and difficulty breathing or shortness of
    breath.} Some of these symptoms overlap with those of the flu,
    making detection difficult, but runny noses and stuffy sinuses are
    less common.
    \href{https://www.nytimes.com/2020/04/27/health/coronavirus-symptoms-cdc.html?action=click\&pgtype=Article\&state=default\&region=MAIN_CONTENT_3\&context=storylines_faq}{The
    C.D.C. has also} added chills, muscle pain, sore throat, headache
    and a new loss of the sense of taste or smell as symptoms to look
    out for. Most people fall ill five to seven days after exposure, but
    symptoms may appear in as few as two days or as many as 14 days.
  \end{itemize}
\item ~
  \hypertarget{does-asymptomatic-transmission-of-covid-19-happen}{%
  \paragraph{Does asymptomatic transmission of Covid-19
  happen?}\label{does-asymptomatic-transmission-of-covid-19-happen}}

  \begin{itemize}
  \tightlist
  \item
    So far, the evidence seems to show it does. A widely cited
    \href{https://www.nature.com/articles/s41591-020-0869-5}{paper}
    published in April suggests that people are most infectious about
    two days before the onset of coronavirus symptoms and estimated that
    44 percent of new infections were a result of transmission from
    people who were not yet showing symptoms. Recently, a top expert at
    the World Health Organization stated that transmission of the
    coronavirus by people who did not have symptoms was ``very rare,''
    \href{https://www.nytimes.com/2020/06/09/world/coronavirus-updates.html?action=click\&pgtype=Article\&state=default\&region=MAIN_CONTENT_3\&context=storylines_faq\#link-1f302e21}{but
    she later walked back that statement.}
  \end{itemize}
\end{itemize}

``It was physical torture,'' he said, adding that one of his most
debilitating symptoms was a constant eye irritation that he described
like ``a bad sunburn, but inside your eyes.'' Still, he felt fortunate
that he and his wife recovered after about three weeks, because he had
seen a few older couples on the reservation die.

The Peacekeeper Society operates a weekly food giveaway and delivers
food and cleaning supplies to households where people have fallen ill.
Mr. Jim said he suspected he caught the virus while out on such a
delivery.

Image

Hundreds of cars in Wapato waited for the potatoes, zucchini, chicken
and salmon being given to people affected by the
coronavirus.Credit...Mason Trinca for The New York Times

As soon as he recovered, Mr. Jim said, he returned to his work
distributing food. On a hot July afternoon, he helped distribute boxes
filled with potatoes, zucchini, cabbage and onions to a line of hundreds
of cars. Families could choose between chicken and salmon waiting in two
kiddie pools stocked with ice.

Adding to the toll of the virus among Native Americans has been swift
and grim economic fallout. ``People lost jobs really quick,'' he said.
``We went from serving a dozen people a week to hundreds.''

Tribal epidemiology centers have fought for months to obtain case
information from the C.D.C., and are only now receiving snippets of what
they requested, several of the dozen centers in the United States said.
Without an accurate portrait of the rates of illness within their
populations, tribal nations have struggled to receive federal funds
aimed at economic recovery and protective gear.

``I think this historic, deep neglect is just coming into sharper focus
because of Covid,'' said Liz Malerba, policy and legislative affairs
director for the United South and Eastern Tribes, a tribal epidemiology
center. ``It's always been there, but now you are seeing more clearly
what the depths are.''

A spokeswoman from the C.D.C. said the agency was working to fill gaps
in its data to better understand the impact of the virus.

``There is still more work to be done to ensure complete race and
ethnicity data in the case report forms,'' said the spokeswoman, Jasmine
Reed. Since April, the agency has increased its collection of race and
ethnicity data from patients tested for the coronavirus, she said.

Ms. Malerba said many tribes did not receive federal emergency funds
equal to their needs because the Treasury Department allocated the money
using census data that undercounted tribal memberships.

``If you eliminate us in the data, you have effectively eliminated us
for the allocation of resources,'' said Abigail Echo-Hawk, the director
of the Urban Indian Health Institute.

In California, tribal epidemiologists have tried to uncover cases
themselves. The California Department of Public Health publishes a daily
count of coronavirus cases, and California Tribal Epidemiology Center
pulls data from that tally in order to track the virus among the 87,000
Native people who access tribal health programs in the state.

``We can only see the number but we don't know more information about
them, where they reside, their specific symptoms,'' said Aurimar Ayala,
the center's epidemiology manager. ``It means we cannot further
investigate those cases.''

She added that the epidemiology center had created a workaround by
contacting local clinics and tracking down the cases, but said that it
was a cumbersome solution.

Although health officials are still struggling to fully understand the
impact of the coronavirus on Native American people, the severity of the
crisis in Yakama Nation is clear to residents, some said.

``It's devastating to our community,'' Ms. Nunez said. ``We have these
elders that have lived through residential schools and the outlawing of
their own religion --- they've been keeping this culture alive and now
Covid hits and it's taking them from us.''

Kate Conger reported from Harrah, and Robert Gebeloff and Richard A.
Oppel Jr. from New York. Sarah Cahalan contributed reporting from
Chicago.

Advertisement

\protect\hyperlink{after-bottom}{Continue reading the main story}

\hypertarget{site-index}{%
\subsection{Site Index}\label{site-index}}

\hypertarget{site-information-navigation}{%
\subsection{Site Information
Navigation}\label{site-information-navigation}}

\begin{itemize}
\tightlist
\item
  \href{https://help.nytimes.com/hc/en-us/articles/115014792127-Copyright-notice}{©~2020~The
  New York Times Company}
\end{itemize}

\begin{itemize}
\tightlist
\item
  \href{https://www.nytco.com/}{NYTCo}
\item
  \href{https://help.nytimes.com/hc/en-us/articles/115015385887-Contact-Us}{Contact
  Us}
\item
  \href{https://www.nytco.com/careers/}{Work with us}
\item
  \href{https://nytmediakit.com/}{Advertise}
\item
  \href{http://www.tbrandstudio.com/}{T Brand Studio}
\item
  \href{https://www.nytimes.com/privacy/cookie-policy\#how-do-i-manage-trackers}{Your
  Ad Choices}
\item
  \href{https://www.nytimes.com/privacy}{Privacy}
\item
  \href{https://help.nytimes.com/hc/en-us/articles/115014893428-Terms-of-service}{Terms
  of Service}
\item
  \href{https://help.nytimes.com/hc/en-us/articles/115014893968-Terms-of-sale}{Terms
  of Sale}
\item
  \href{https://spiderbites.nytimes.com}{Site Map}
\item
  \href{https://help.nytimes.com/hc/en-us}{Help}
\item
  \href{https://www.nytimes.com/subscription?campaignId=37WXW}{Subscriptions}
\end{itemize}
