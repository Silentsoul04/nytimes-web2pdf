Sections

SEARCH

\protect\hyperlink{site-content}{Skip to
content}\protect\hyperlink{site-index}{Skip to site index}

\href{https://www.nytimes.com/section/politics}{Politics}

\href{https://myaccount.nytimes.com/auth/login?response_type=cookie\&client_id=vi}{}

\href{https://www.nytimes.com/section/todayspaper}{Today's Paper}

\href{/section/politics}{Politics}\textbar{}Trump Floats an Election
Delay, and Republicans Shoot It Down

\url{https://nyti.ms/3gfV5ad}

\begin{itemize}
\item
\item
\item
\item
\item
\end{itemize}

\begin{itemize}
\item
  \href{https://www.nytimes.com/2020/08/03/us/elections/biden-vs-trump.html?action=click\&pgtype=Article\&state=default\&region=TOP_BANNER\&context=storylines_menu}{Election
  Updates}
\item
  \href{https://www.nytimes.com/article/biden-vice-president-2020.html?action=click\&pgtype=Article\&state=default\&region=TOP_BANNER\&context=storylines_menu}{Biden's
  V.P. Search}
\item
  \href{https://www.nytimes.com/interactive/2020/07/24/us/politics/trump-biden-campaign-donors.html?action=click\&pgtype=Article\&state=default\&region=TOP_BANNER\&context=storylines_menu}{Map
  of Donations}
\item
  \href{https://www.nytimes.com/interactive/2020/us/elections/delegate-count-primary-results.html?action=click\&pgtype=Article\&state=default\&region=TOP_BANNER\&context=storylines_menu}{Delegate
  Count}
\item
  \href{https://www.nytimes.com/interactive/2019/us/politics/2020-presidential-candidates.html?action=click\&pgtype=Article\&state=default\&region=TOP_BANNER\&context=storylines_menu}{The
  Candidates}
\item
  \href{https://www.nytimes.com/newsletters/politics?action=click\&pgtype=Article\&state=default\&region=TOP_BANNER\&context=storylines_menu}{Politics
  Newsletter}
\end{itemize}

Advertisement

\protect\hyperlink{after-top}{Continue reading the main story}

Supported by

\protect\hyperlink{after-sponsor}{Continue reading the main story}

\hypertarget{trump-floats-an-election-delay-and-republicans-shoot-it-down}{%
\section{Trump Floats an Election Delay, and Republicans Shoot It
Down}\label{trump-floats-an-election-delay-and-republicans-shoot-it-down}}

The president's suggestion that the Nov. 3 vote could be delayed ---
something he cannot do on his own --- drew unusually firm Republican
resistance and signaled worry about his re-election bid.

\includegraphics{https://static01.nyt.com/images/2020/07/30/us/politics/30trump-election1/merlin_175126236_6c5d937a-ab38-460a-87aa-892bfd358495-articleLarge.jpg?quality=75\&auto=webp\&disable=upscale}

\href{https://www.nytimes.com/by/maggie-haberman}{\includegraphics{https://static01.nyt.com/images/2018/07/12/multimedia/author-maggie-haberman/author-maggie-haberman-thumbLarge.png}}\href{https://www.nytimes.com/by/jonathan-martin}{\includegraphics{https://static01.nyt.com/images/2018/11/06/multimedia/author-jonathan-martin/author-jonathan-martin-thumbLarge.png}}\href{https://www.nytimes.com/by/reid-j-epstein}{\includegraphics{https://static01.nyt.com/images/2019/06/25/reader-center/author-reid-epstein/9e877853d8234217b58e5762253aa771-thumbLarge.png}}

By \href{https://www.nytimes.com/by/maggie-haberman}{Maggie Haberman},
\href{https://www.nytimes.com/by/jonathan-martin}{Jonathan Martin} and
\href{https://www.nytimes.com/by/reid-j-epstein}{Reid J. Epstein}

\begin{itemize}
\item
  July 30, 2020
\item
  \begin{itemize}
  \item
  \item
  \item
  \item
  \item
  \end{itemize}
\end{itemize}

Facing disastrous economic news and rising coronavirus deaths,
\href{https://www.nytimes.com/interactive/2020/us/elections/donald-trump.html}{President
Trump} on Thursday floated delaying the Nov. 3 election, a suggestion
that lacks legal authority and could undermine confidence in an election
that polls show him on course to lose.

Republican leaders in Congress, who often claim not to have seen Mr.
Trump's outlandish statements and tweets and who infrequently challenge
him in public, promptly and vocally condemned any notion that the
election would be moved.

It was a moment of striking political isolation for the president, as
Republicans felt no need to defend him, Democrats condemned him, and
three former presidents gathered in a rare moment together, paying
tribute at
\href{https://www.nytimes.com/2020/07/30/us/john-lewis-live-funeral.html}{the
funeral of Representative John Lewis} of Georgia.

Mr. Trump is facing about as dire a run-up to a presidential election as
any incumbent could imagine: the
\href{https://www.nytimes.com/live/2020/07/30/business/stock-market-today-coronavirus}{worst
quarter} in the economy on record, an unceasing health crisis, protests
nationwide and a country paralyzed by the lack of a financial recovery
plan with no solution in sight --- all compounded by his own inability
to curtail his behavior.

His remarks on Twitter about the election delay --- which he linked to
his baseless claims about the potential for mail-in voter fraud --- were
one of the few clear signs that the president now realizes how deep a
hole he has dug for himself in his re-election effort. Aides have
described him as pained by the widespread rejection he is seeing in
public opinion polls, even as he continues with self-sabotaging behavior
rather than taking steps that might help him, like getting involved in
negotiations for a deal on Capitol Hill to lift the economy.

``With Universal Mail-In Voting (not Absentee Voting, which is good),
2020 will be the most INACCURATE \& FRAUDULENT Election in history,''
Mr. Trump
\href{https://twitter.com/realDonaldTrump/status/1288818160389558273?s=20}{wrote}.
``It will be a great embarrassment to the USA. Delay the Election until
people can properly, securely and safely vote???''

Mr. Trump later pinned the tweet at the top of his Twitter feed,
ensuring people would continue to see it. Hours later, despite warnings
from his campaign officials that delays are likely in tabulating results
on Nov. 3, Mr. Trump said in a separate
\href{https://twitter.com/realDonaldTrump/status/1288933078287745024?s=20}{tweet},
``Must know Election results on the night of the Election, not days,
months or even years later!''

That second statement reflects a concern that Democrats have given voice
to --- that Mr. Trump will try to focus on the same-day voting tallies
to claim victory, even when the full results may be unknown for days.

At a late-afternoon briefing with reporters, Mr. Trump defended the
initial tweet, saying that he feared delays in counting votes. But he
declined to elaborate on whether he was seriously proposing moving the
election.

Mr. Trump posted the first tweet shortly after the Commerce Department
announced that the gross domestic product for the second quarter of the
year had fallen precipitously by 9.5 percent, reflecting the widespread
shutdown of businesses beginning in March to combat the spread of the
coronavirus.

\hypertarget{latest-updates-2020-election}{%
\section{\texorpdfstring{\href{https://www.nytimes.com/2020/08/03/us/elections/biden-vs-trump.html?action=click\&pgtype=Article\&state=default\&region=MAIN_CONTENT_1\&context=storylines_live_updates}{Latest
Updates: 2020
Election}}{Latest Updates: 2020 Election}}\label{latest-updates-2020-election}}

Updated 2020-08-03T13:14:26.522Z

\begin{itemize}
\tightlist
\item
  \href{https://www.nytimes.com/2020/08/03/us/elections/biden-vs-trump.html?action=click\&pgtype=Article\&state=default\&region=MAIN_CONTENT_1\&context=storylines_live_updates\#link-2a2c5488}{The
  vice-presidential watch begins in earnest this week.}
\item
  \href{https://www.nytimes.com/2020/08/03/us/elections/biden-vs-trump.html?action=click\&pgtype=Article\&state=default\&region=MAIN_CONTENT_1\&context=storylines_live_updates\#link-37287715}{Trump's
  campaign returns to the airwaves in North Carolina, Florida, Georgia
  and Arizona.}
\item
  \href{https://www.nytimes.com/2020/08/03/us/elections/biden-vs-trump.html?action=click\&pgtype=Article\&state=default\&region=MAIN_CONTENT_1\&context=storylines_live_updates\#link-33f706ee}{Biden
  steps up his push to make Ohio competitive in November, with help from
  Sherrod Brown.}
\end{itemize}

\href{https://www.nytimes.com/2020/08/03/us/elections/biden-vs-trump.html?action=click\&pgtype=Article\&state=default\&region=MAIN_CONTENT_1\&context=storylines_live_updates}{See
more updates}

Mr. Trump, who often tests the boundaries of his authority, has
increasingly used public comments to lay groundwork for arguing that the
election results are illegitimate if he loses. Though he does not have
the constitutional authority to unilaterally change the date of the
election, his tweet prompted a now-familiar round of assertions about
what his true intention was with his statement.

With Mr. Trump, that is frequently a guessing game. The president has
often posted remarks on Twitter that are aimed at sparking a reaction
from people. At other times, he posts in reaction to what he sees on
cable news shows. And sometimes he tries to change what those shows are
focusing on with his tweets, offering a diversion.

Whatever his motivation on Thursday, senior Republicans and an array of
senators wanted no part of it, diverging from their standard practice of
walking on eggshells after a Trump eruption.

``Never in the history of the federal elections have we not held an
election, and we should go forward,'' said Representative Kevin McCarthy
of California, the House minority leader and an enthusiastic supporter
of Mr. Trump's, adding that he understood ``the president's concern
about mail-in voting.''

\includegraphics{https://static01.nyt.com/images/2020/07/30/us/politics/30trump-election3/merlin_175125528_d9a5161a-4545-4d12-b133-08d3febe0a9d-articleLarge.jpg?quality=75\&auto=webp\&disable=upscale}

Senator Mitch McConnell, the majority leader, echoed Mr. McCarthy,
saying ``we'll find a way'' to hold the election on Nov. 3.

Senators Ted Cruz and Marco Rubio, rivals for the 2016 Republican
presidential nomination who have since become staunch Trump supporters,
both dismissed the idea that the date for the election could change.
Senator Lindsey Graham, Mr. Trump's foremost public defender in the
Senate, said there would be a secure vote in November. And officials in
key swing states showed little interest in engaging on the topic.

``We're going to have an election, it's going to be legitimate, it's
going to be credible, it's going to be the same as it's always been,''
Mr. Rubio told reporters at the Capitol in Washington.

Mr. Cruz agreed. ``I think election fraud is a serious problem,'' he
said. ``But, no, we should not delay the election.''

People close to Mr. Trump said that the president has at times discussed
with associates whether the election can be delayed, and has been told
definitively that only an amendment to the Constitution could change the
date. But his tweet was discomfiting to most of his aides, who tried to
clean up his statement later by contending that he had been referring to
the possibility that the outcome won't be known until weeks after the
election.

This is not the first time that Mr. Trump has raised the idea of
thwarting rules or laws that he finds objectionable, and he often fails
to follow through. He has repeatedly hurled threats, whether it is
defunding universities or blocking federal aid to states, the substance
of which he has no intent, or capacity, to fulfill.

The president, who did not serve in government before being elected to
the highest office in the country, has never fully absorbed what powers
he does and does not have, or how to wield his authority. What Mr. Trump
has always been mindful of, dating to his time as a real estate
developer, is the danger of being labeled a failure.

So in response to his weakened standing in the presidential race, Mr.
Trump has been reaching for arguments to explain his difficulties this
year, repeatedly noting how the virus undermined the booming economy for
which he claims credit.

In this vein, any uncertainty about the balloting offers him an opening
to raise questions about the legitimacy of his loss, regardless of
whether he challenges the results.

Trump-weary Republicans may make that a more difficult task, however.

Representative Liz Cheney, Republican of Wyoming, a sometime critic of
the president who is eyeing the top ranks of the House leadership, said:
``We are not moving the date of the election. The resistance to this
idea among Republicans is overwhelming.''

Scott Jennings, a Republican strategist and an adviser to Mr. McConnell,
called Mr. Trump's statement ``unfocused,'' and ``insecure,'' saying it
``separates him from his own party and most of mainstream political
thought at a time when he needs to be fully focused on coronavirus, the
economy, and defining Biden as out of the mainstream.''

``Republicans,'' Mr. Jennings added, ``have reacted correctly by
rejecting the notion of delay.''

To Mr. Jennings and other Republican strategists, Mr. Trump is playing
with fire by suggesting to his supporters that mail voting can't be
trusted, given that it may be the best option for some people in an era
in which almost every activity has been changed to combat the virus's
spread. Making Republican voters distrust mail voting could negatively
affect not just Mr. Trump, but a host of down-ballot candidates.

''The reality is,'' Mr. Jennings said, ``he needs every Republican vote
there is, and he needs them any way he can get them, no matter how they
are cast.''

The president has repeatedly railed against mail voting, creating
outlandish scenarios of ballot theft to undermine confidence in the
practice.

Image

Mr. Trump has frequently broken with presidential precedent in doubting
the legitimacy of elections.~Credit...Doug Mills/The New York Times

Even for Mr. Trump,
\href{https://twitter.com/realDonaldTrump/status/1288818160389558273?s=20}{suggesting
a delay in the election} was an extraordinary breach of presidential
decorum that will increase the chances that he and his core supporters
don't accept the legitimacy of the election should he lose to former
Vice President
\href{https://www.nytimes.com/interactive/2020/us/elections/joe-biden.html}{Joseph
R. Biden Jr.} Mr. Trump's comments about the election looked all the
more discordant coming just hours before the funeral for Mr. Lewis, a
Democrat who as a young man was beaten and jailed as he advocated voting
rights.

Without mentioning his successor by name, former President Barack Obama
used his eulogy of Mr. Lewis to rebuke Mr. Trump.

Speaking from the pulpit of Atlanta's Ebenezer Baptist Church, where the
Rev. Dr. Martin Luther King Jr. was reared and eventually preached, Mr.
Obama lashed Mr. Trump for ``even undermining the Postal Service in the
run-up to an election that's going to be dependent on mail-in ballots so
people don't get sick.''

For all the eye-rolling dismissals among Republicans, Mr. Trump's
remarks irritated and embarrassed his allies --- and represented the
latest illustration of how he is not only complicating his own campaign
but also compounding his party's challenge this fall.

Already burdened with an administration that only briefly attempted a
full-scale response to a public health crisis that has
\href{https://www.nytimes.com/interactive/2020/us/coronavirus-us-cases.html}{sickened
millions} of Americans and killed over 150,000 while ravaging the
economy, Republicans on the ballot are increasingly being undermined by
Mr. Trump's response to his misfortune.

Just this week, after he finally bowed to pressure to urge people to
take virus safety measures, the president lamented how unpopular he is
compared with his high-profile medical advisers.

And then he
\href{https://www.nytimes.com/2020/07/28/technology/virus-video-trump.html}{publicized
an online video} promoting an unproven virus treatment from a doctor who
has previously opined on alien DNA and the impact of having sex with
demons in one's dreams.

His growing desperation to close the gap with Mr. Biden has also caused
headaches for Republicans because he has increasingly employed
race-baiting language that few in the party care to defend.

``I am happy to inform all of the people living their Suburban Lifestyle
Dream that you will no longer be bothered or financially hurt by having
low income housing built in your neighborhood,''
\href{https://www.nytimes.com/2020/07/29/us/politics/trump-suburbs-housing-white-voters.html}{he
tweeted on Wednesday}.

Luke Broadwater, Emily Cochrane and Matt Stevens contributed reporting.

\hypertarget{our-2020-election-guide}{%
\section{Our 2020 Election Guide}\label{our-2020-election-guide}}

Updated July 31, 2020

\begin{itemize}
\item
  \begin{center}\rule{0.5\linewidth}{\linethickness}\end{center}

  \hypertarget{the-latest}{%
  \subsection{The Latest}\label{the-latest}}

  \begin{itemize}
  \tightlist
  \item
    The vice-presidential watch begins in earnest this week.
    \href{https://www.nytimes.com/2020/08/03/us/elections/biden-vs-trump.html?action=click\&pgtype=Article\&state=default\&region=BELOW_MAIN_CONTENT\&context=storylines_guide}{Follow
    the latest updates here.}
  \end{itemize}
\item
  \begin{center}\rule{0.5\linewidth}{\linethickness}\end{center}

  \hypertarget{bidens-vp-search}{%
  \subsection{Biden's V.P. Search}\label{bidens-vp-search}}

  \begin{itemize}
  \tightlist
  \item
    \href{https://www.nytimes.com/article/biden-vice-president-2020.html?action=click\&pgtype=Article\&state=default\&region=BELOW_MAIN_CONTENT\&context=storylines_guide}{Here
    are 13 women} who have been under consideration to be Joe Biden's
    running mate, and why each might be chosen --- and might not be.
  \end{itemize}
\item
  \begin{center}\rule{0.5\linewidth}{\linethickness}\end{center}

  \hypertarget{keep-up-with-our-coverage}{%
  \subsection{Keep Up With Our
  Coverage}\label{keep-up-with-our-coverage}}

  \begin{itemize}
  \tightlist
  \item
    Get an
    \href{https://www.nytimes.com/newsletters/politics?action=click\&pgtype=Article\&state=default\&region=BELOW_MAIN_CONTENT\&context=storylines_guide}{email}
    recapping the day's news
  \end{itemize}

  \begin{itemize}
  \tightlist
  \item
    Download our mobile app on
    \href{https://apps.apple.com/us/app/nytimes/id284862083?ls=1\&mat_click_id=5c79ae7455014fd1bd66b5610c05b8f2-20191112-16948\&referrer=mat_click_id\%3D5c79ae7455014fd1bd66b5610c05b8f2-20191112-16948\%26link_click_id\%3D722930677036718082}{iOS}
    and
    \href{http://a.localytics.com/android?id=com.nytimes.android\&referrer=utm_source\%3Dother_nyt_mobile_web\%26utm_medium\%3DWeb\%2520page\%26utm_term\%3DGeneral\%2520Mobile\%2520Page\%26utm_campaign\%3DNYT\%2520Mobile\%2520General\%2520Page}{Android}
    and turn on Breaking News and Politics alerts
  \end{itemize}
\end{itemize}

Advertisement

\protect\hyperlink{after-bottom}{Continue reading the main story}

\hypertarget{site-index}{%
\subsection{Site Index}\label{site-index}}

\hypertarget{site-information-navigation}{%
\subsection{Site Information
Navigation}\label{site-information-navigation}}

\begin{itemize}
\tightlist
\item
  \href{https://help.nytimes.com/hc/en-us/articles/115014792127-Copyright-notice}{©~2020~The
  New York Times Company}
\end{itemize}

\begin{itemize}
\tightlist
\item
  \href{https://www.nytco.com/}{NYTCo}
\item
  \href{https://help.nytimes.com/hc/en-us/articles/115015385887-Contact-Us}{Contact
  Us}
\item
  \href{https://www.nytco.com/careers/}{Work with us}
\item
  \href{https://nytmediakit.com/}{Advertise}
\item
  \href{http://www.tbrandstudio.com/}{T Brand Studio}
\item
  \href{https://www.nytimes.com/privacy/cookie-policy\#how-do-i-manage-trackers}{Your
  Ad Choices}
\item
  \href{https://www.nytimes.com/privacy}{Privacy}
\item
  \href{https://help.nytimes.com/hc/en-us/articles/115014893428-Terms-of-service}{Terms
  of Service}
\item
  \href{https://help.nytimes.com/hc/en-us/articles/115014893968-Terms-of-sale}{Terms
  of Sale}
\item
  \href{https://spiderbites.nytimes.com}{Site Map}
\item
  \href{https://help.nytimes.com/hc/en-us}{Help}
\item
  \href{https://www.nytimes.com/subscription?campaignId=37WXW}{Subscriptions}
\end{itemize}
