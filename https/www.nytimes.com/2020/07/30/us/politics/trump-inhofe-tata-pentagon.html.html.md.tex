Sections

SEARCH

\protect\hyperlink{site-content}{Skip to
content}\protect\hyperlink{site-index}{Skip to site index}

\href{https://www.nytimes.com/section/politics}{Politics}

\href{https://myaccount.nytimes.com/auth/login?response_type=cookie\&client_id=vi}{}

\href{https://www.nytimes.com/section/todayspaper}{Today's Paper}

\href{/section/politics}{Politics}\textbar{}Inflammatory Comments Delay
Confirmation of Retired General to Pentagon Post

\url{https://nyti.ms/2XdKk0C}

\begin{itemize}
\item
\item
\item
\item
\item
\end{itemize}

Advertisement

\protect\hyperlink{after-top}{Continue reading the main story}

Supported by

\protect\hyperlink{after-sponsor}{Continue reading the main story}

\hypertarget{inflammatory-comments-delay-confirmation-of-retired-general-to-pentagon-post}{%
\section{Inflammatory Comments Delay Confirmation of Retired General to
Pentagon
Post}\label{inflammatory-comments-delay-confirmation-of-retired-general-to-pentagon-post}}

Half an hour before a hearing for Anthony J. Tata was to begin, the
Oklahoma Republican who is the chairman of the Senate Armed Services
Committee delayed it.

\includegraphics{https://static01.nyt.com/images/2020/07/30/us/politics/30dc-tata/30dc-tata-articleLarge.jpg?quality=75\&auto=webp\&disable=upscale}

\href{https://www.nytimes.com/by/helene-cooper}{\includegraphics{https://static01.nyt.com/images/2018/08/24/multimedia/author-helene-cooper/author-helene-cooper-thumbLarge.png}}\href{https://www.nytimes.com/by/catie-edmondson}{\includegraphics{https://static01.nyt.com/images/2019/11/20/us/politics/catie-edmonson-twitter-chatblog/catie-edmonson-twitter-chatblog-thumbLarge.png}}\href{https://www.nytimes.com/by/maggie-haberman}{\includegraphics{https://static01.nyt.com/images/2018/07/12/multimedia/author-maggie-haberman/author-maggie-haberman-thumbLarge.png}}

By \href{https://www.nytimes.com/by/helene-cooper}{Helene Cooper},
\href{https://www.nytimes.com/by/catie-edmondson}{Catie Edmondson} and
\href{https://www.nytimes.com/by/maggie-haberman}{Maggie Haberman}

\begin{itemize}
\item
  July 30, 2020
\item
  \begin{itemize}
  \item
  \item
  \item
  \item
  \item
  \end{itemize}
\end{itemize}

WASHINGTON --- President Trump's nomination of a retired general with a
history of inflammatory comments to serve in the Pentagon's top policy
job was abruptly postponed on Thursday, as senators from both sides of
the aisle indicated an unwillingness to back Anthony J. Tata, a novelist
and Fox News commentator.

Half an hour before Mr. Tata's hearing was set to begin, Senator James
M. Inhofe, the Oklahoma Republican who serves as the chairman of the
Armed Services Committee, announced that he was delaying it.

``There are many Democrats and Republicans who didn't know enough about
Anthony Tata to consider him for a very significant position at this
time,'' Mr. Inhofe said in a statement. ``We didn't get the required
documentation in time; some documents, which we normally get before a
hearing, didn't arrive until yesterday.''

Mr. Inhofe said he talked to Mr. Trump on Wednesday night and told him
that ``we're simply out of time with the August recess coming, so it
wouldn't serve any useful purpose to have a hearing at this point, and
he agreed.''

The nomination of Mr. Tata, a retired Army brigadier general, was in
trouble, facing a wall of Democratic opposition and growing concerns
from vulnerable Republicans who are up for re-election in November.

At the same time, several senior retired military officers have dropped
their support for Mr. Tata. Gen. Joseph L. Votel, the former head of the
Central Command; Gen. Tony Thomas, the former head of the Special
Operations Command; and Lt. Gen. David A. Deptula, a former top Air
Force general, all asked in June that their names be removed from
\href{https://s.wsj.net/public/resources/documents/Tata-Letter_06-18-2020.pdf}{a
letter} sent by 36 current and former leaders to the Armed Services
Committee backing the nomination.

Mr. Tata's views, expressed in
\href{https://twitter.com/ajtata/status/1014278134185840640}{a series of
tweets}, strike a jarring note, particularly as the country is seized by
a growing movement for change. He called Islam ``the most oppressive
violent religion'' and referred to former President Barack Obama as a
``terrorist leader.'' He has since apologized for the remarks, which
were
\href{https://edition.cnn.com/2020/06/12/politics/pentagon-nominee-tata-trump-kfile/index.html}{first
reported by CNN}.

Mr. Tata was meant to succeed John Rood, who
\href{https://www.nytimes.com/2020/02/19/us/politics/john-rood-trump.html}{resigned
in February} at Mr. Trump's request. Mr. Rood had pushed back on efforts
to withhold military aid to Ukraine, a central issue in Mr. Trump's
impeachment hearings. But Mr. Tata's chances look bleak now,
congressional staff members said.

One Republican on the armed services panel, Senator Kevin Cramer of
North Dakota, had publicly indicated that he was prepared to block the
nomination. Mr. Cramer said he would oppose Mr. Tata's nomination unless
he reversed a policy that prevented adding the names of sailors who died
aboard a naval destroyer to the Vietnam Veterans Memorial, a
longstanding personal crusade of the senator's.

But other Republican lawmakers were privately unsettled by Mr. Tata's
inflammatory remarks, and taking a vote on the nomination would have put
four Republicans on the panel, who are facing difficult re-election
battles, in a particularly unsavory position: Senators Joni Ernst of
Iowa, Martha McSally of Arizona, Thom Tillis of North Carolina and David
Perdue of Georgia.

Democratic lawmakers on the panel were united in opposing Mr. Tata,
making the threat of Mr. Cramer's opposition potentially fatal to moving
the nomination out of committee.

``No one with a record of repeated, repugnant statements like yours
should be nominated to serve in a senior position of public trust at the
Pentagon,'' the Democratic lawmakers wrote in a letter to Mr. Tata.
``Your views are wholly incompatible with the U.S. military's values.''

Senator Jack Reed, the ranking Democrat on the Armed Services Committee,
said after Mr. Inhofe canceled the hearing that ``it's fair to say
members on both sides of the aisle have raised serious questions about
this nominee.''

``We had a closed-door session on Tuesday and today's public hearing has
now been canceled,'' Mr. Reed added. ``Chairman Inhofe did the right
thing here, and it's clear this nomination isn't going anywhere without
a full, fair, open hearing.''

It remained unclear whether Mr. Tata could eventually get a hearing, or
if his nomination was dead.

During his conversation with Mr. Inhofe on Wednesday night, Mr. Trump
could be heard indicating that he might give Mr. Tata a different
appointment.

The call was overheard because Mr. Inhofe put it on speakerphone to hear
better as he sat in the Trattoria Alberto restaurant in Washington.

The conversation, recorded by someone in the room, ranged from a
discussion about Mr. Tata to Mr. Trump's desire to preserve the name of
Robert E. Lee, the commander of the Confederate Army, on a military
base.

``We're going to keep the name of Robert E. Lee?'' Mr. Trump asked Mr.
Inhofe. The senator put the phone to his ear but put Mr. Trump on
speakerphone, and the president's voice was audible to people sitting at
other tables.

Mr. Inhofe replied: ``Just trust me. I'll make it happen.''

Mr. Trump went on. ``I had about 95,000 positive retweets on that.
That's a lot,'' he said, appearing to refer to a Twitter post last
Friday in which he said that Mr. Inhofe had assured him that he would
not change the names of ``military forts and bases'' and that the
senator ``is not a believer in `Cancel Culture.'''

Mr. Trump could be heard on the call criticizing cancel culture and told
Mr. Inhofe that people ``want to be able to go back to life.'' He then
appeared to dismiss the focus on the cultural shift taking place across
the country with an expletive.

An aide to Mr. Inhofe declined to comment on the conversation. Aides to
Mr. Trump did not immediately respond to requests for comment.

Advertisement

\protect\hyperlink{after-bottom}{Continue reading the main story}

\hypertarget{site-index}{%
\subsection{Site Index}\label{site-index}}

\hypertarget{site-information-navigation}{%
\subsection{Site Information
Navigation}\label{site-information-navigation}}

\begin{itemize}
\tightlist
\item
  \href{https://help.nytimes.com/hc/en-us/articles/115014792127-Copyright-notice}{©~2020~The
  New York Times Company}
\end{itemize}

\begin{itemize}
\tightlist
\item
  \href{https://www.nytco.com/}{NYTCo}
\item
  \href{https://help.nytimes.com/hc/en-us/articles/115015385887-Contact-Us}{Contact
  Us}
\item
  \href{https://www.nytco.com/careers/}{Work with us}
\item
  \href{https://nytmediakit.com/}{Advertise}
\item
  \href{http://www.tbrandstudio.com/}{T Brand Studio}
\item
  \href{https://www.nytimes.com/privacy/cookie-policy\#how-do-i-manage-trackers}{Your
  Ad Choices}
\item
  \href{https://www.nytimes.com/privacy}{Privacy}
\item
  \href{https://help.nytimes.com/hc/en-us/articles/115014893428-Terms-of-service}{Terms
  of Service}
\item
  \href{https://help.nytimes.com/hc/en-us/articles/115014893968-Terms-of-sale}{Terms
  of Sale}
\item
  \href{https://spiderbites.nytimes.com}{Site Map}
\item
  \href{https://help.nytimes.com/hc/en-us}{Help}
\item
  \href{https://www.nytimes.com/subscription?campaignId=37WXW}{Subscriptions}
\end{itemize}
