Sections

SEARCH

\protect\hyperlink{site-content}{Skip to
content}\protect\hyperlink{site-index}{Skip to site index}

\href{https://www.nytimes.com/section/world/asia}{Asia Pacific}

\href{https://myaccount.nytimes.com/auth/login?response_type=cookie\&client_id=vi}{}

\href{https://www.nytimes.com/section/todayspaper}{Today's Paper}

\href{/section/world/asia}{Asia Pacific}\textbar{}Lee Teng-hui, 97, Who
Led Taiwan's Turn to Democracy, Dies

\url{https://nyti.ms/2Xb5zQO}

\begin{itemize}
\item
\item
\item
\item
\item
\item
\end{itemize}

Advertisement

\protect\hyperlink{after-top}{Continue reading the main story}

Supported by

\protect\hyperlink{after-sponsor}{Continue reading the main story}

\hypertarget{lee-teng-hui-97-who-led-taiwans-turn-to-democracy-dies}{%
\section{Lee Teng-hui, 97, Who Led Taiwan's Turn to Democracy,
Dies}\label{lee-teng-hui-97-who-led-taiwans-turn-to-democracy-dies}}

Its first popularly elected president, he transformed a police state
into a vibrant country while angering Beijing by insisting that Taiwan
be treated as a sovereign state.

\includegraphics{https://static01.nyt.com/images/2020/07/31/world/00lee-obit-HFO1-print/merlin_136316991_2280ad0b-105d-4510-a006-742d47b7f665-articleLarge.jpg?quality=75\&auto=webp\&disable=upscale}

By Jonathan Kandell

\begin{itemize}
\item
  July 30, 2020
\item
  \begin{itemize}
  \item
  \item
  \item
  \item
  \item
  \item
  \end{itemize}
\end{itemize}

\href{https://cn.nytimes.com/asia-pacific/20200730/lee-teng-hui-dead/}{阅读简体中文版}\href{https://cn.nytimes.com/asia-pacific/20200730/lee-teng-hui-dead/zh-hant/}{閱讀繁體中文版}

Lee Teng-hui, who as president of Taiwan led its transformation from an
island in the grip of authoritarian rule to one of Asia's most vibrant
and prosperous democracies, died on Thursday in Taipei, the capital. He
was 97.

The office of Taiwan's president, Tsai Ing-wen, announced the death, at
Taipei Veterans Hospital. News reports said the cause was septic shock
and multiple organ failure.

Mr. Lee's insistence that Taiwan be treated as a sovereign state angered
the Chinese government in Beijing, which considered Taiwan part of its
territory and pushed for its unification with the mainland under
Communist rule. His stance posed a political quandary for the United
States as it sought to improve relations with Beijing while dissuading
it from taking military action to press its claims over the island.

As president from 1988 to 2000 --- the first to be elected by popular
vote in Taiwan --- Mr. Lee never backed down from disputes with the
mainland, and he continued to be a thorn in its side well into his later
years. In 2018 he called, unsuccessfully, for a referendum on declaring
the country's name to be Taiwan, not the Republic of China, as it is
formally known --- a move that would have paved the way for sovereignty.

``China's goal regarding Taiwan has never changed,'' he told
\href{https://www.nytimes.com/2018/05/25/world/asia/china-taiwan-identity-xi-jinping.html}{The
New York Times} in a rare interview at a time when the Chinese
government was trying to further isolate the island from the
international community. ``That goal is to swallow up Taiwan's
sovereignty, exterminate Taiwanese democracy and achieve ultimate
unification.''

President Tsai's office praised Mr. Lee's achievements, saying in a
statement, ``The president believes that former President Lee's
contribution to Taiwan's democratic journey is irreplaceable and his
death is a great loss to the country.''

Mr. Lee entered Taiwan's politics during the dictatorial Nationalist
Party regimes of Chiang Kai-shek and his son Chiang Ching-kuo, who
assumed power after his father's death in 1975. The Nationalists ruled
with brutality, which reached a peak in 1947 with what became known as
the February 28 incident, in which up to 28,000 Taiwanese were massacred
by Chiang Kai-shek's troops in response to street protests. The
Nationalists imposed martial law two years later, and it was not lifted
until 1987 by Chiang Ching-kuo.

Born in Taiwan, Mr. Lee joined the Nationalist Party, known as the
Kuomintang or KMT, in 1971 and became an agricultural minister. He was
later mayor of Taipei and governor of Taiwan Province before being
tapped as vice president in 1984.

When Chiang Ching-kuo died of a heart attack in 1988, Mr. Lee succeeded
him, becoming the first native Taiwanese president.

\includegraphics{https://static01.nyt.com/images/2020/07/28/world/00lee-obit-HFO-2/merlin_175055469_872bebcc-e57f-473c-9e8e-5ffb21ab8064-articleLarge.jpg?quality=75\&auto=webp\&disable=upscale}

Mr. Lee dismantled the dictatorship and worked to end the animosity
between those born on the mainland and the native Taiwanese. He pushed
the concept of ``New Taiwanese,'' a term suggesting that the islanders,
no matter their backgrounds, were forging a common identity based on a
democratic political system and growing prosperity.

He pursued a deliberately ambiguous policy with mainland China, shifting
between rigid hostility, tentative conciliation and defiant
independence. His attempts to demonstrate Taiwan's international
sovereignty sometimes provoked the mainland into saber-rattling military
exercises.

One such episode occurred after a trip by Mr. Lee to the United States
in 1995, ostensibly to visit Cornell University, his alma mater. China
accused the United States and Taiwan of colluding to raise the island's
diplomatic status. In a demonstration of Beijing's ire, Chinese military
forces fired test missiles into the Taiwan Strait, which separates the
island from the mainland. Washington countered by positioning warships
off the Taiwan coast. The affair strained relations between Washington
and Beijing for months.

Mr. Lee again infuriated Beijing in a German television interview in
1999 by suggesting that relations between Taiwan and China should be
conducted on a ``special state-to-state'' basis. That provoked tirades
in the official Chinese media. The People's Liberation Army Daily
denounced Mr. Lee as ``the No. 1 scum in the nation.'' The Xinhua News
Agency called him a ``deformed test-tube baby cultivated in the
political laboratory of hostile anti-China forces.''

Image

A picture released by China's state-run Xinhua news agency showed
People's Liberation Army military exercises in 1996 during Taiwan's
first open presidential election, which Mr. Lee won.~Credit...Xinhua,
via Reuters

But such attacks made Mr. Lee only more popular in Taiwan. A tall,
silver-haired, tough-minded campaigner with a dazzling smile, he used
his charisma to rally support. He spoke the slang of the ports and
factories, rode bullhorn trucks with local candidates and set off
firecrackers to please the deities of local temples.

``The people like Lee Teng-hui because he stands up for them in the face
of China's dictators,'' Chen Shui-bian, the mayor of Taipei at the time,
\href{https://www.nytimes.com/1996/03/22/world/tension-in-taiwan-the-politics-war-games-play-well-for-taiwan-s-leader.html}{said
in 1996},

Lee Teng-hui was born on Jan. 15, 1923, in Sanzhi, a village on the
outskirts of Taipei. His father was a police detective in the employ of
the Japanese authorities that ruled Taiwan as a colony from 1895 to
1945. Mr. Lee studied agronomy in Japan at the Kyoto Imperial University
and served as a second lieutenant in the Imperial Japanese Army during
World War II, though he never saw action.

He returned to Taiwan after the war and secretly joined the Communist
Party of China while completing his undergraduate work at the National
Taiwan University. ``I read everything I could get my hands on by Karl
Marx and Friedrich Engels,'' he wrote in his 1999 memoirs, ``The Road to
Democracy.''

He joined the protests in the February 28 incident in 1947, but he soon
renounced Marxism and joined the KMT. The party later destroyed his
Communist Party records when he became politically prominent.

Image

Mr. Lee inspecting troops in 1997. Once in office he ended decades of
state-of-emergency measures on Taiwan.Credit...Reuters

Mr. Lee married Tseng Wen-fui, the daughter of a prosperous landholding
family, in 1949, and both became devoted Presbyterians. They had two
daughters, Anna and Annie; their only son, Hsien-wen, died of cancer. He
is survived by his wife and daughters as well as a granddaughter and a
grandson.

Taiwan became a separate political entity in 1949 after the civil war in
China brought Mao's Communists to power, forcing Chiang's defeated
government to flee to the island, some 100 miles from the mainland.

For the next 30 years, Taiwan, with American support, maintained the
fiction that it was the seat of China's legitimate government in exile.
Washington finally recognized the Communist government in Beijing in
1979 and severed its formal diplomatic relations with Taiwan. But it
continued to guarantee Taiwan's security against a mainland invasion and
backed negotiations between both sides aimed at reunification.

Mr. Lee cultivated ties with the United States during two academic
stays, receiving a master's degree in agricultural economics from Iowa
State University in 1953 and a Ph.D. from Cornell in 1968. In between,
he taught in Taiwanese universities, gaining recognition as an
agricultural economics scholar and attracting the attention of Chiang
Ching-kuo, then a deputy prime minister under his father. On the younger
Chiang's recommendation, Mr. Lee was appointed minister without
portfolio. He distinguished himself by promoting programs that raised
health standards and farm incomes.

Image

Mr. Lee, center, and his wife, Tsang Wen-hui, foreground, on a goodwill
visit to the United States in 1983, when he was governor of Taiwan
Province. He was named vice president the next year.Credit...Associated
Press

With Chiang Ching-kuo installed as president, Mr. Lee was appointed
mayor of Taipei in 1978 and set about modernizing the capital's road and
sewer systems. As governor of Taiwan Province, from 1981 to 1984, he
pushed agrarian reforms that helped achieve a balanced growth between
urban and rural areas, still a hallmark of Taiwan.

Mr. Chiang selected Mr. Lee as his vice president in 1984. It was a
dramatic departure from the usual practice of appointing only former
mainland Chinese to top government posts. His selection was viewed as a
gesture toward the native Taiwanese, who had been politically powerless
despite accounting for 85 percent of the population.

When Mr. Lee became president in 1988 on Mr. Chiang's death, he moved to
break with the Chiang family's autocratic system, publicly deploring the
February 28 massacres. He ended decades of state-of-emergency measures,
allowed citizens to send mail to mainland relatives and visit them,
dropped bans on street demonstrations, eased press restrictions,
promoted a multiparty system and decreed open elections for the National
Assembly.

The KMT easily retained control of the legislature, but more than
three-fourths of the seats went to Taiwanese natives.

``What had been a tight police state under Chiang Kai-shek and his son
Chiang Ching-kuo is now the most democratic society in the
Chinese-speaking world,''
\href{https://www.nytimes.com/1992/02/22/opinion/don-t-stifle-democracy-in-taiwan.html}{The
Times declared} in a 1992 editorial.

Image

Mr. Lee campaigning in Taipei in 1996. After succeeding to the office as
vice president, he became Taiwan's first popularly elected
president.Credit...The Asahi Shimbun, via Getty Images

Mr. Lee was elected outright in 1996, in Taiwan's first open
presidential contest. Seeking to begin a dialogue with Beijing, he
supported a policy of ``one China, two equal governments.'' But he
insisted that Taiwan would rejoin the mainland only if China became a
democratic, capitalist society. In the meantime he again called for
``state to state'' relations between Taipei and Beijing, a policy that
the mainland rejected. Instead, Chinese officials tried to persuade
other countries to cut all ties with Taiwan, asserting that any
improvement in relations would come only after Mr. Lee had retired.

Mr. Lee was succeeded in 2000 by Chen Shui-bian, the Democratic
Progressive Party candidate whose election ended KMT rule. In his two
terms, Mr. Chen presided over a huge expansion of Taiwan's trade and
investment in China, a process that had already been underway during the
Lee presidency. But like his predecessor, Mr. Chen frustrated Beijing's
attempts to get Taipei to acknowledge the mainland's sovereignty and
embrace a timetable for unification.

Mr. Lee came out of retirement in 2018 to help create the Formosa
Alliance, a new party calling for the formal independence of Taiwan from
China. But the party did not go ahead with a promised referendum on
independence.

Late in life, Mr. Lee endured the ignominy of corruption charges. In
June 2011, he was indicted, along with a financier, Liu Tai-ying, on
charges
\href{https://www.nytimes.com/2011/07/01/world/asia/01taiwan.html}{of
embezzling} almost \$8 million in public funds during his presidency.
Mr. Lee was acquitted in 2013.

He took solace in proclaiming that he had helped his island of 23
million inhabitants serve as a beacon for the 1.4 billion people on the
mainland. Or, as he wrote in his memoirs, ``We have developed the
economy and have embraced democracy, becoming the model for a future
reunified China.''

Image

Mr. Lee in 2018. He came out of retirement that year to help create the
Formosa Alliance, a new party calling for the formal independence of
Taiwan from China. Credit...Lam Yik Fei for The New York Times

Austin Ramzy contributed reporting.

Advertisement

\protect\hyperlink{after-bottom}{Continue reading the main story}

\hypertarget{site-index}{%
\subsection{Site Index}\label{site-index}}

\hypertarget{site-information-navigation}{%
\subsection{Site Information
Navigation}\label{site-information-navigation}}

\begin{itemize}
\tightlist
\item
  \href{https://help.nytimes.com/hc/en-us/articles/115014792127-Copyright-notice}{©~2020~The
  New York Times Company}
\end{itemize}

\begin{itemize}
\tightlist
\item
  \href{https://www.nytco.com/}{NYTCo}
\item
  \href{https://help.nytimes.com/hc/en-us/articles/115015385887-Contact-Us}{Contact
  Us}
\item
  \href{https://www.nytco.com/careers/}{Work with us}
\item
  \href{https://nytmediakit.com/}{Advertise}
\item
  \href{http://www.tbrandstudio.com/}{T Brand Studio}
\item
  \href{https://www.nytimes.com/privacy/cookie-policy\#how-do-i-manage-trackers}{Your
  Ad Choices}
\item
  \href{https://www.nytimes.com/privacy}{Privacy}
\item
  \href{https://help.nytimes.com/hc/en-us/articles/115014893428-Terms-of-service}{Terms
  of Service}
\item
  \href{https://help.nytimes.com/hc/en-us/articles/115014893968-Terms-of-sale}{Terms
  of Sale}
\item
  \href{https://spiderbites.nytimes.com}{Site Map}
\item
  \href{https://help.nytimes.com/hc/en-us}{Help}
\item
  \href{https://www.nytimes.com/subscription?campaignId=37WXW}{Subscriptions}
\end{itemize}
