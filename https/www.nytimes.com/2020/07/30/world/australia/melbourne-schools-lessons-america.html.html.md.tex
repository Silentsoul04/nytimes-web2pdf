Sections

SEARCH

\protect\hyperlink{site-content}{Skip to
content}\protect\hyperlink{site-index}{Skip to site index}

\href{https://www.nytimes.com/section/world/australia}{Australia}

\href{https://myaccount.nytimes.com/auth/login?response_type=cookie\&client_id=vi}{}

\href{https://www.nytimes.com/section/todayspaper}{Today's Paper}

\href{/section/world/australia}{Australia}\textbar{}What Can Victorian
Schools Teach America About Reopening?

\url{https://nyti.ms/39LQ4Uu}

\begin{itemize}
\item
\item
\item
\item
\item
\end{itemize}

\href{https://www.nytimes.com/news-event/coronavirus?action=click\&pgtype=Article\&state=default\&region=TOP_BANNER\&context=storylines_menu}{The
Coronavirus Outbreak}

\begin{itemize}
\tightlist
\item
  live\href{https://www.nytimes.com/2020/08/04/world/coronavirus-cases.html?action=click\&pgtype=Article\&state=default\&region=TOP_BANNER\&context=storylines_menu}{Latest
  Updates}
\item
  \href{https://www.nytimes.com/interactive/2020/us/coronavirus-us-cases.html?action=click\&pgtype=Article\&state=default\&region=TOP_BANNER\&context=storylines_menu}{Maps
  and Cases}
\item
  \href{https://www.nytimes.com/interactive/2020/science/coronavirus-vaccine-tracker.html?action=click\&pgtype=Article\&state=default\&region=TOP_BANNER\&context=storylines_menu}{Vaccine
  Tracker}
\item
  \href{https://www.nytimes.com/2020/08/02/us/covid-college-reopening.html?action=click\&pgtype=Article\&state=default\&region=TOP_BANNER\&context=storylines_menu}{College
  Reopening}
\item
  \href{https://www.nytimes.com/live/2020/08/04/business/stock-market-today-coronavirus?action=click\&pgtype=Article\&state=default\&region=TOP_BANNER\&context=storylines_menu}{Economy}
\end{itemize}

Advertisement

\protect\hyperlink{after-top}{Continue reading the main story}

Supported by

\protect\hyperlink{after-sponsor}{Continue reading the main story}

letter 169

\hypertarget{what-can-victorian-schools-teach-america-about-reopening}{%
\section{What Can Victorian Schools Teach America About
Reopening?}\label{what-can-victorian-schools-teach-america-about-reopening}}

With around 100 schools closed because of students or teachers with
COVID-19, Australia has lessons for other countries aiming to reopen.

\includegraphics{https://static01.nyt.com/images/2020/07/31/world/31ausletter169-1/31ausletter169-1-articleLarge.jpg?quality=75\&auto=webp\&disable=upscale}

By Besha Rodell

\begin{itemize}
\item
  July 30, 2020
\item
  \begin{itemize}
  \item
  \item
  \item
  \item
  \item
  \end{itemize}
\end{itemize}

\href{https://www.nytimes.com/series/nyt-australia-newsletter?module=inline}{\emph{The
Australia Letter}} \emph{is a weekly newsletter from our Australia
bureau.}
\href{https://www.nytimes.com/newsletters/australia-letter?module=inline}{\emph{Sign
up}} \emph{to get it by email.}

\begin{center}\rule{0.5\linewidth}{\linethickness}\end{center}

Every night, we wait for the email. Sometimes it comes in the late
afternoon, but many nights it doesn't hit my inbox until 10 or 11 p.m.
Eventually, it arrives, written by a beleaguered school principal
letting us know that my son's high school is still closed.

My family is in the same position as thousands of others in Victoria,
where around 100 schools are dealing with similar situations.

After months of remote learning, year 11 and 12 students in
\href{https://www.nytimes.com/2020/08/04/world/australia/coronavirus-melbourne-lockdown.html}{Melbourne}
returned to the classroom on July 14. For my son, who is in 11th grade,
this in-person schooling lasted less than a week --- on July 20 we were
informed that a student at his school had tested positive for
coronavirus and all in-person learning would be suspended while the
school was cleaned and contact tracing conducted.

As of today, July 31, the school is clean but the contact tracing
continues. There has never been a timeline given to parents or students
about how long that tracing will take. We wait day-to-day for updates on
whether school will resume the next day. The principal waits on the
Department of Health to let him know when contact tracing is complete,
and the overburdened Department of Health does --- I assume --- its
best, probably with some waiting of its own for coronavirus test
results.

With debate in the United States raging about whether schools should
reopen after the annual summer break, there are some useful lessons in
the struggles of our schools in Victoria. An
\href{https://www.nytimes.com/2020/07/28/opinion/coronavirus-schools-reopening.html?searchResultPosition=2}{opinion
piece in the Times this week asked, ``what happens when there is a
Covid-19 case in a school?''} Well, here in Melbourne, many schools are
already answering that question.

\hypertarget{latest-updates-global-coronavirus-outbreak}{%
\section{\texorpdfstring{\href{https://www.nytimes.com/2020/08/04/world/coronavirus-cases.html?action=click\&pgtype=Article\&state=default\&region=MAIN_CONTENT_1\&context=storylines_live_updates}{Latest
Updates: Global Coronavirus
Outbreak}}{Latest Updates: Global Coronavirus Outbreak}}\label{latest-updates-global-coronavirus-outbreak}}

Updated 2020-08-04T20:57:54.346Z

\begin{itemize}
\tightlist
\item
  \href{https://www.nytimes.com/2020/08/04/world/coronavirus-cases.html?action=click\&pgtype=Article\&state=default\&region=MAIN_CONTENT_1\&context=storylines_live_updates\#link-1228a480}{Novavax
  sees encouraging results from two studies of its experimental
  vaccine.}
\item
  \href{https://www.nytimes.com/2020/08/04/world/coronavirus-cases.html?action=click\&pgtype=Article\&state=default\&region=MAIN_CONTENT_1\&context=storylines_live_updates\#link-4825b93}{Public
  and private schools in Maryland and elsewhere are divided over
  in-person instruction.}
\item
  \href{https://www.nytimes.com/2020/08/04/world/coronavirus-cases.html?action=click\&pgtype=Article\&state=default\&region=MAIN_CONTENT_1\&context=storylines_live_updates\#link-50f7386d}{The
  United Nations calls on policymakers to `plan thoroughly for school
  reopenings.'}
\end{itemize}

\href{https://www.nytimes.com/2020/08/04/world/coronavirus-cases.html?action=click\&pgtype=Article\&state=default\&region=MAIN_CONTENT_1\&context=storylines_live_updates}{See
more updates}

More live coverage:
\href{https://www.nytimes.com/live/2020/08/04/business/stock-market-today-coronavirus?action=click\&pgtype=Article\&state=default\&region=MAIN_CONTENT_1\&context=storylines_live_updates}{Markets}

I spoke to Times education reporter Eliza Shapiro today just as she was
finishing up a news brief about the plans drawn up by New York City's
school district --- the largest in the U.S. --- for reopening. It is one
of the only large districts in the country to be attempting in-person
learning any time soon, with most major districts opting for distance
learning for the foreseeable future.

Eliza's reporting, along with Dana Goldstein, has
\href{https://www.nytimes.com/2020/07/14/us/coronavirus-schools-fall.html}{shown
that most large school districts are in danger of major community
coronavirus spread if they reopen}, but New York is eager to move ahead
and the plans Eliza described to me are complex and ambitious, with
specific standards for when schools will close down and under what
conditions.

``It's really really complicated,'' she told me. ``We have so many
vulnerable kids, so many kids with disabilities, so many homeless kids,
so there's a lot of interest in getting as many kids back in the
classroom as possible. But once we actually open --- if we open --- real
life is going to collide with these plans and it's going to be really
difficult.''

What the Americans may not fully grasp is what we've already learned in
Victoria: Plans can disappear quickly when the unpredictability of the
virus comes into play. Each case or cluster becomes its own mystery,
demanding time and resources while raising anxiety to new levels.

To be clear: I place no blame on anyone for my son's school situation.
It's an overused word in these bizarre times, but the situation is
unprecedented and extremely complex. I applaud everyone involved for
trying to keep the community as safe as possible. But Victorian schools
are in a far better position than many American school systems by almost
every metric, and yet things here are messy and unpredictable and often
delayed for reasons that are unknown or not fully shared.

Just like our nightly ritual of learning on-the-fly what our situation
will be the following morning, the most disconcerting thing about this
virus is the extreme uncertainty and endurance that it demands. What
will tomorrow bring? Or the next day, month and year? I hope what we're
going through can, at the very least, help inform and prepare other
parents, students and school districts what their own future may hold.
And for now, it's mostly anticipation followed by disappointment.

What are your biggest concerns about schools reopening, in Australia or
elsewhere? Let us know at
\href{mailto:nytaustralia@nytimes.com}{nytaustralia@nytimes.com.}

Here are this week's stories:

\begin{center}\rule{0.5\linewidth}{\linethickness}\end{center}

\href{https://www.nytimes.com/news-event/coronavirus?action=click\&pgtype=Article\&state=default\&region=MAIN_CONTENT_3\&context=storylines_faq}{}

\hypertarget{the-coronavirus-outbreak-}{%
\subsubsection{The Coronavirus Outbreak
›}\label{the-coronavirus-outbreak-}}

\hypertarget{frequently-asked-questions}{%
\paragraph{Frequently Asked
Questions}\label{frequently-asked-questions}}

Updated August 4, 2020

\begin{itemize}
\item ~
  \hypertarget{i-have-antibodies-am-i-now-immune}{%
  \paragraph{I have antibodies. Am I now
  immune?}\label{i-have-antibodies-am-i-now-immune}}

  \begin{itemize}
  \tightlist
  \item
    As of right
    now,\href{https://www.nytimes.com/2020/07/22/health/covid-antibodies-herd-immunity.html?action=click\&pgtype=Article\&state=default\&region=MAIN_CONTENT_3\&context=storylines_faq}{that
    seems likely, for at least several months.} There have been
    frightening accounts of people suffering what seems to be a second
    bout of Covid-19. But experts say these patients may have a
    drawn-out course of infection, with the virus taking a slow toll
    weeks to months after initial exposure. People infected with the
    coronavirus typically
    \href{https://www.nature.com/articles/s41586-020-2456-9}{produce}
    immune molecules called antibodies, which are
    \href{https://www.nytimes.com/2020/05/07/health/coronavirus-antibody-prevalence.html?action=click\&pgtype=Article\&state=default\&region=MAIN_CONTENT_3\&context=storylines_faq}{protective
    proteins made in response to an
    infection}\href{https://www.nytimes.com/2020/05/07/health/coronavirus-antibody-prevalence.html?action=click\&pgtype=Article\&state=default\&region=MAIN_CONTENT_3\&context=storylines_faq}{.
    These antibodies may} last in the body
    \href{https://www.nature.com/articles/s41591-020-0965-6}{only two to
    three months}, which may seem worrisome, but that's perfectly normal
    after an acute infection subsides, said Dr. Michael Mina, an
    immunologist at Harvard University. It may be possible to get the
    coronavirus again, but it's highly unlikely that it would be
    possible in a short window of time from initial infection or make
    people sicker the second time.
  \end{itemize}
\item ~
  \hypertarget{im-a-small-business-owner-can-i-get-relief}{%
  \paragraph{I'm a small-business owner. Can I get
  relief?}\label{im-a-small-business-owner-can-i-get-relief}}

  \begin{itemize}
  \tightlist
  \item
    The
    \href{https://www.nytimes.com/article/small-business-loans-stimulus-grants-freelancers-coronavirus.html?action=click\&pgtype=Article\&state=default\&region=MAIN_CONTENT_3\&context=storylines_faq}{stimulus
    bills enacted in March} offer help for the millions of American
    small businesses. Those eligible for aid are businesses and
    nonprofit organizations with fewer than 500 workers, including sole
    proprietorships, independent contractors and freelancers. Some
    larger companies in some industries are also eligible. The help
    being offered, which is being managed by the Small Business
    Administration, includes the Paycheck Protection Program and the
    Economic Injury Disaster Loan program. But lots of folks have
    \href{https://www.nytimes.com/interactive/2020/05/07/business/small-business-loans-coronavirus.html?action=click\&pgtype=Article\&state=default\&region=MAIN_CONTENT_3\&context=storylines_faq}{not
    yet seen payouts.} Even those who have received help are confused:
    The rules are draconian, and some are stuck sitting on
    \href{https://www.nytimes.com/2020/05/02/business/economy/loans-coronavirus-small-business.html?action=click\&pgtype=Article\&state=default\&region=MAIN_CONTENT_3\&context=storylines_faq}{money
    they don't know how to use.} Many small-business owners are getting
    less than they expected or
    \href{https://www.nytimes.com/2020/06/10/business/Small-business-loans-ppp.html?action=click\&pgtype=Article\&state=default\&region=MAIN_CONTENT_3\&context=storylines_faq}{not
    hearing anything at all.}
  \end{itemize}
\item ~
  \hypertarget{what-are-my-rights-if-i-am-worried-about-going-back-to-work}{%
  \paragraph{What are my rights if I am worried about going back to
  work?}\label{what-are-my-rights-if-i-am-worried-about-going-back-to-work}}

  \begin{itemize}
  \tightlist
  \item
    Employers have to provide
    \href{https://www.osha.gov/SLTC/covid-19/standards.html}{a safe
    workplace} with policies that protect everyone equally.
    \href{https://www.nytimes.com/article/coronavirus-money-unemployment.html?action=click\&pgtype=Article\&state=default\&region=MAIN_CONTENT_3\&context=storylines_faq}{And
    if one of your co-workers tests positive for the coronavirus, the
    C.D.C.} has said that
    \href{https://www.cdc.gov/coronavirus/2019-ncov/community/guidance-business-response.html}{employers
    should tell their employees} -\/- without giving you the sick
    employee's name -\/- that they may have been exposed to the virus.
  \end{itemize}
\item ~
  \hypertarget{should-i-refinance-my-mortgage}{%
  \paragraph{Should I refinance my
  mortgage?}\label{should-i-refinance-my-mortgage}}

  \begin{itemize}
  \tightlist
  \item
    \href{https://www.nytimes.com/article/coronavirus-money-unemployment.html?action=click\&pgtype=Article\&state=default\&region=MAIN_CONTENT_3\&context=storylines_faq}{It
    could be a good idea,} because mortgage rates have
    \href{https://www.nytimes.com/2020/07/16/business/mortgage-rates-below-3-percent.html?action=click\&pgtype=Article\&state=default\&region=MAIN_CONTENT_3\&context=storylines_faq}{never
    been lower.} Refinancing requests have pushed mortgage applications
    to some of the highest levels since 2008, so be prepared to get in
    line. But defaults are also up, so if you're thinking about buying a
    home, be aware that some lenders have tightened their standards.
  \end{itemize}
\item ~
  \hypertarget{what-is-school-going-to-look-like-in-september}{%
  \paragraph{What is school going to look like in
  September?}\label{what-is-school-going-to-look-like-in-september}}

  \begin{itemize}
  \tightlist
  \item
    It is unlikely that many schools will return to a normal schedule
    this fall, requiring the grind of
    \href{https://www.nytimes.com/2020/06/05/us/coronavirus-education-lost-learning.html?action=click\&pgtype=Article\&state=default\&region=MAIN_CONTENT_3\&context=storylines_faq}{online
    learning},
    \href{https://www.nytimes.com/2020/05/29/us/coronavirus-child-care-centers.html?action=click\&pgtype=Article\&state=default\&region=MAIN_CONTENT_3\&context=storylines_faq}{makeshift
    child care} and
    \href{https://www.nytimes.com/2020/06/03/business/economy/coronavirus-working-women.html?action=click\&pgtype=Article\&state=default\&region=MAIN_CONTENT_3\&context=storylines_faq}{stunted
    workdays} to continue. California's two largest public school
    districts --- Los Angeles and San Diego --- said on July 13, that
    \href{https://www.nytimes.com/2020/07/13/us/lausd-san-diego-school-reopening.html?action=click\&pgtype=Article\&state=default\&region=MAIN_CONTENT_3\&context=storylines_faq}{instruction
    will be remote-only in the fall}, citing concerns that surging
    coronavirus infections in their areas pose too dire a risk for
    students and teachers. Together, the two districts enroll some
    825,000 students. They are the largest in the country so far to
    abandon plans for even a partial physical return to classrooms when
    they reopen in August. For other districts, the solution won't be an
    all-or-nothing approach.
    \href{https://bioethics.jhu.edu/research-and-outreach/projects/eschool-initiative/school-policy-tracker/}{Many
    systems}, including the nation's largest, New York City, are
    devising
    \href{https://www.nytimes.com/2020/06/26/us/coronavirus-schools-reopen-fall.html?action=click\&pgtype=Article\&state=default\&region=MAIN_CONTENT_3\&context=storylines_faq}{hybrid
    plans} that involve spending some days in classrooms and other days
    online. There's no national policy on this yet, so check with your
    municipal school system regularly to see what is happening in your
    community.
  \end{itemize}
\end{itemize}

\hypertarget{australia-and-new-zealand}{%
\subsection{\texorpdfstring{\href{https://www.nytimes.com/section/world/australia}{Australia
and New
Zealand}}{Australia and New Zealand}}\label{australia-and-new-zealand}}

\includegraphics{https://static01.nyt.com/images/2020/07/31/world/31ausletter169-2/31ausletter169-2-articleLarge.jpg?quality=75\&auto=webp\&disable=upscale}

\begin{itemize}
\tightlist
\item
  \textbf{\href{https://www.nytimes.com/2020/07/28/dining/melbourne-restaurants-coronavirus.html}{The
  Pandemic Could End the Age of Midpriced Dining}.} When Melbourne
  restaurants reopened after lockdown, owners got creative, and dinner
  got far more expensive.
\end{itemize}

\begin{itemize}
\tightlist
\item
  \textbf{\href{https://www.nytimes.com/2020/07/28/world/europe/british-australian-academic-jail-iran-qarchak.html}{British-Australian
  Academic Jailed in Iran Is Moved to Remote Prison}}. Kylie
  Moore-Gilbert, who has denied charges of espionage, is now in a
  facility where many have been infected with the coronavirus, rights
  activists say.
\end{itemize}

\begin{itemize}
\item
  \textbf{\href{https://www.nytimes.com/2020/07/28/world/australia/chinese-students-virtual-kidnapping.html}{Australia
  Says Chinese Students Are Targets in `Virtual Kidnapping' Scams}.}
  Recent cases reveal the evolution of a crime that often exploits worry
  over family members abroad with digital savvy and old-fashioned
  coercion.
\item
  \textbf{\href{https://www.nytimes.com/2020/07/24/world/australia/behrouz-boochani-asylum-new-zealand.html}{Refugee
  and Author Long Detained by Australia Gets Asylum in New Zealand}.}
  Behrouz Boochani, a Kurdish-Iranian exile, said the news showed the
  vast differences between the two neighboring countries on human
  rights.
\end{itemize}

\begin{center}\rule{0.5\linewidth}{\linethickness}\end{center}

\hypertarget{around-the-times}{%
\subsection{Around the Times}\label{around-the-times}}

Image

The president on Thursday tweeted that the 2020 election would be
``fraudulent'' if there is universal mail-in voting.Credit...Doug
Mills/The New York Times

\begin{itemize}
\item
  **\href{https://www.nytimes.com/2020/07/30/opinion/trump-delay-election-coronavirus.html}{Opinion:
  Trump Might Try to Postpone the Election. That's
  Unconstitutional.}**He should be removed unless he relents.
\item
  \textbf{\href{https://www.nytimes.com/2020/07/29/magazine/vesper-flights.html?action=click\&module=Editors\%20Picks\&pgtype=Homepage}{The
  Mysterious Life of Birds Who Never Come Down.}} **** Swifts spend all
  their time in the sky. What can their journeys tell us about the
  future?
\item
  \textbf{\href{https://www.nytimes.com/2020/07/30/health/diamond-princess-coronavirus-aerosol.html}{Aboard
  the Diamond Princess, a Case Study in Aerosol Transmission}}. A
  computer model of the cruise-ship outbreak found that the virus spread
  most readily in microscopic droplets light enough to linger in the
  air.
\item
  \textbf{\href{https://www.nytimes.com/2020/07/30/us/politics/herman-cain-gop-coronavirus.html}{Will
  Herman Cain's Death Change Republican Views on the Virus and Masks?}}
  **** His publicly dismissive attitude about the pandemic reflected the
  hands-off inconsistency of many party leaders.
\end{itemize}

\textbf{And Over to You \ldots{}}

\emph{Last week}
\href{https://www.nytimes.com/2020/07/24/world/australia/divisions-decency-and-the-plague.html}{\emph{we
wrote about pandemic reading}}\emph{, and asked what you were reading
right now. Here are some reader responses and suggestions:}

I am reading a novel that is not about pandemics, but, I think, nicely
captures the spirit of stay-at-home claustrophobia: ``A Gentleman in
Moscow,'' by Amor Towles.

\textbf{--- Kurt van der Walde}

An illuminating biography, ``Uncrowned Queen'' by Nicola Tallis has
really helped me in the ongoing situation of staying safe indoors. It's
about Tudor matriarch Margaret Beaufort, the mother of Henry V11, and
her extraordinary life.

\textbf{--- Peter James}

During Covid quarantine I've discovered Australian female authors and
been enjoying books that focus on station life in outback rural areas.
Authors like Fleur McDonald I found brilliant at developing complex
characters, relationships and behaviors particular to Australian rural
areas. I quite enjoyed a number of books by Karly Lane, also based in
rural Australia. I can recommend exploring books by Anne Rennie, Di
Morrissey and Kate Grenville; all accomplished writers of Australian
fiction.

\textbf{--- Wendy Williams}

\begin{center}\rule{0.5\linewidth}{\linethickness}\end{center}

Enjoying the Australia Letter?
\href{https://www.nytimes.com/newsletters/australia-letter?module=inline}{Sign
up here} or forward to a friend.

For more Australia coverage and discussion, start your day with your
local
\href{https://www.nytimes.com/interactive/2018/briefing/global-morning-briefing-newsletter-signup.html?utm_source=ausend}{Morning
Briefing} and join us in our
\href{https://www.facebook.com/groups/nytaustralia/}{Facebook group}.

Advertisement

\protect\hyperlink{after-bottom}{Continue reading the main story}

\hypertarget{site-index}{%
\subsection{Site Index}\label{site-index}}

\hypertarget{site-information-navigation}{%
\subsection{Site Information
Navigation}\label{site-information-navigation}}

\begin{itemize}
\tightlist
\item
  \href{https://help.nytimes.com/hc/en-us/articles/115014792127-Copyright-notice}{©~2020~The
  New York Times Company}
\end{itemize}

\begin{itemize}
\tightlist
\item
  \href{https://www.nytco.com/}{NYTCo}
\item
  \href{https://help.nytimes.com/hc/en-us/articles/115015385887-Contact-Us}{Contact
  Us}
\item
  \href{https://www.nytco.com/careers/}{Work with us}
\item
  \href{https://nytmediakit.com/}{Advertise}
\item
  \href{http://www.tbrandstudio.com/}{T Brand Studio}
\item
  \href{https://www.nytimes.com/privacy/cookie-policy\#how-do-i-manage-trackers}{Your
  Ad Choices}
\item
  \href{https://www.nytimes.com/privacy}{Privacy}
\item
  \href{https://help.nytimes.com/hc/en-us/articles/115014893428-Terms-of-service}{Terms
  of Service}
\item
  \href{https://help.nytimes.com/hc/en-us/articles/115014893968-Terms-of-sale}{Terms
  of Sale}
\item
  \href{https://spiderbites.nytimes.com}{Site Map}
\item
  \href{https://help.nytimes.com/hc/en-us}{Help}
\item
  \href{https://www.nytimes.com/subscription?campaignId=37WXW}{Subscriptions}
\end{itemize}
