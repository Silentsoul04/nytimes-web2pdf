Sections

SEARCH

\protect\hyperlink{site-content}{Skip to
content}\protect\hyperlink{site-index}{Skip to site index}

\href{https://www.nytimes.com/section/arts}{Arts}

\href{https://myaccount.nytimes.com/auth/login?response_type=cookie\&client_id=vi}{}

\href{https://www.nytimes.com/section/todayspaper}{Today's Paper}

\href{/section/arts}{Arts}\textbar{}The Most Soothing Man on TikTok

\url{https://nyti.ms/33chGk6}

\begin{itemize}
\item
\item
\item
\item
\item
\end{itemize}

Advertisement

\protect\hyperlink{after-top}{Continue reading the main story}

Supported by

\protect\hyperlink{after-sponsor}{Continue reading the main story}

Critic's Notebook

\hypertarget{the-most-soothing-man-on-tiktok}{%
\section{The Most Soothing Man on
TikTok}\label{the-most-soothing-man-on-tiktok}}

By slowing down and soaking it all in, Larry Scott makes the whole
TikTok experience \ldots{} nice.

\includegraphics{https://static01.nyt.com/images/2020/07/31/arts/31akumpo1/31akumpo1-articleLarge.jpg?quality=75\&auto=webp\&disable=upscale}

By \href{https://www.nytimes.com/by/jon-caramanica}{Jon Caramanica}

\begin{itemize}
\item
  July 31, 2020
\item
  \begin{itemize}
  \item
  \item
  \item
  \item
  \item
  \end{itemize}
\end{itemize}

Some of the most illuminating, purely pleasurable videos on TikTok this
month have been Larry Scott's awed observations of cooking, where the
teen from Florida looks on as a meal is lovingly prepared al fresco:
hand-rolled pasta dough, spices arranged by color, a knife assuredly
having its way with a pepper or onion. The recipe videos have quick
cuts, and with each new move, Scott's eyes widen. His brow furrows just
a bit while he tries to suss out what's being made. He eases into a
million-dollar smile when something catches his fancy. ``Oh,'' he says,
with a sparkle of realization. ``Nice.''

That's it. That's the thing.

TikTok is a decentralized medium, but Scott's gentle,
perspective-slowing reaction videos have a way of imposing just a touch
of reason to it, and untold joy. Using the duet --- the TikTok function
that allows a user to watch someone else's video and record a response
in real time --- as his métier, Scott is an equal opportunity reactor.
Dance videos, romantic montages, a call to arrest the police officers
who killed Breonna Taylor, weirdo nonsense quasi-art clips, an a
cappella group singing Alicia Keys, a rack of doughnuts getting
slathered in glaze: Scott has nice'd them all.

Under \href{https://www.tiktok.com/@larryakumpo}{the TikTok handle
@larryakumpo}, Scott posts several videos per week. They are maybe the
most calming thing on the internet and, on some days, maybe the only
calming thing on the internet. He radiates pure equanimity. No matter
how eye-popping the video is, he's never judgmental --- curious,
shocked, secondhand embarrassed, maybe a little worried, but he
basically never deviates from the sweetness of wonder.

And then there's the ``nice'' itself, which he rolls out with the
slithering embrace of a purr. It's not wry or ironically detached ---
it's the sort of utterance that slips out almost imperceptibly when
you're overcome by what you're seeing. Sometimes he adds an ``oh'' or a
``yeah'' --- it's like psychological A.S.M.R.

This earnest observational device is a pushback to TikTok's infinite
scroll. Scott is a watcher, trapped in the box just like the rest of us.
If we weren't already obsessed with our phones, the last few months of
isolation have made absorbing endless content the default national mode.
We are passive in our liminal misery --- waiting to be distracted,
entertained, vaccinated, liberated.

Unlike television, which requires a metacommentary that's pithy and
interruptive --- think ``Beavis and Butt-Head'' or ``Mystery Science
Theater 3000'' --- TikTok is already pithy and interruptive, which is
why the most effective sort of metacommentary slows down its rhythm,
encouraging reflection.

And Scott's clips are, without fail, beatifically tranquil. Sometimes
his hair is tied up, sometimes it falls in front of his face in a loose
tangle. Often he's reclined in bed or on a couch. His face fills up the
majority of the screen, so there's no ambiguity about how he's feeling.
When he lets out a ``whoa,'' his eyes get big, and he leans back, as if
a gust of wind has caught him off guard, nudging him gently. When his
face broadens into a smile, it has a way of almost obliterating the
video he's reacting to with its guilelessness. When he's frazzled, which
is very, very rarely, one single worry line creases his forehead.

\includegraphics{https://static01.nyt.com/images/2020/07/31/arts/31akumpo2/merlin_175164996_a14a3bf6-f832-4277-a6ca-4aab9a9173c1-articleLarge.jpg?quality=75\&auto=webp\&disable=upscale}

Even though the rhythm of his clips is familiar, Scott meets them with
full presence. In an interview with
\href{https://www.buzzfeednews.com/article/tanyachen/a-teen-has-become-famous-on-tiktok-simply-for-saying-whoa}{Buzzfeed}
last week, he said he doesn't pre-watch the videos he duets with, so as
to preserve the integrity of his reaction.

In an ecosystem as ruthless as TikTok, with creators jockeying for
likes, followers, clout and whatever monetary privileges follow those
things, Scott's videos are solely about encouragement, a dollop of pure
love. (The only time he's said ``not nice'' was to a freestyle by the
rapper Smokepurpp that went viral for its awkwardness.)

Scott started posting videos to the app last summer --- videos about
heartbreak, Frank Ocean, whether he looks like Bronny James. (He
doesn't.) His observational duets began in March, and the catchphrases
took hold in June, not long after he graduated from high school. Now
he's got 1.4 million followers, almost all of which he acquired this
month, as his wholesome nurturing has rapidly coursed through TikTok.

As happens often in the erratic and limitless world of social media,
Scott's ascent is accelerating rapidly. He's beginning to generate his
own meta-content --- other users riff on his ``nice,'' and in one post,
he talks about people alerting him to copycats who lack his ``natural
flow.''

Still, how much wonder can one young man express? Last week he appeared
in a video with the
\href{https://www.youtube.com/watch?v=YJdZQb8VOZk}{Pump Bros}, a Hans \&
Franz of social media who took Scott and a friend for a workout session.
After watching Scott work through some triceps kickbacks, one of them,
Will Savery, turns to the camera and declares, ``The `oh nice' guy is
getting swole.'' Elsewhere in the video, Savery runs through barbell
curls while Scott looks on and exclaims: ``Oh, nice. Yeah. Nice.''

There is, here, just the tiniest tiny bit of sourness, a light curdling.
The sentience that comes when a thing you've been doing unconsciously,
or at least without much scrutiny, suddenly becomes a catchphrase, a
meme, a thing. An albatross you'll carry for a week or a month or maybe
a lifetime.

In order for the ``nice'' to work, it has to be moving at a different
speed from everything else. It has to be the thing that reorients your
sense of time. In its steadfast but charming resistance, it's an
encouragement that maybe you, too, should slow down. Wonder is all
around you. Take it in. Nice.

Advertisement

\protect\hyperlink{after-bottom}{Continue reading the main story}

\hypertarget{site-index}{%
\subsection{Site Index}\label{site-index}}

\hypertarget{site-information-navigation}{%
\subsection{Site Information
Navigation}\label{site-information-navigation}}

\begin{itemize}
\tightlist
\item
  \href{https://help.nytimes.com/hc/en-us/articles/115014792127-Copyright-notice}{©~2020~The
  New York Times Company}
\end{itemize}

\begin{itemize}
\tightlist
\item
  \href{https://www.nytco.com/}{NYTCo}
\item
  \href{https://help.nytimes.com/hc/en-us/articles/115015385887-Contact-Us}{Contact
  Us}
\item
  \href{https://www.nytco.com/careers/}{Work with us}
\item
  \href{https://nytmediakit.com/}{Advertise}
\item
  \href{http://www.tbrandstudio.com/}{T Brand Studio}
\item
  \href{https://www.nytimes.com/privacy/cookie-policy\#how-do-i-manage-trackers}{Your
  Ad Choices}
\item
  \href{https://www.nytimes.com/privacy}{Privacy}
\item
  \href{https://help.nytimes.com/hc/en-us/articles/115014893428-Terms-of-service}{Terms
  of Service}
\item
  \href{https://help.nytimes.com/hc/en-us/articles/115014893968-Terms-of-sale}{Terms
  of Sale}
\item
  \href{https://spiderbites.nytimes.com}{Site Map}
\item
  \href{https://help.nytimes.com/hc/en-us}{Help}
\item
  \href{https://www.nytimes.com/subscription?campaignId=37WXW}{Subscriptions}
\end{itemize}
