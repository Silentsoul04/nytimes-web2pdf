Sections

SEARCH

\protect\hyperlink{site-content}{Skip to
content}\protect\hyperlink{site-index}{Skip to site index}

\href{https://www.nytimes.com/section/arts/music}{Music}

\href{https://myaccount.nytimes.com/auth/login?response_type=cookie\&client_id=vi}{}

\href{https://www.nytimes.com/section/todayspaper}{Today's Paper}

\href{/section/arts/music}{Music}\textbar{}Beyoncé's `Black Is King':
Let's Discuss

\url{https://nyti.ms/30gppf3}

\begin{itemize}
\item
\item
\item
\item
\item
\item
\end{itemize}

Advertisement

\protect\hyperlink{after-top}{Continue reading the main story}

Supported by

\protect\hyperlink{after-sponsor}{Continue reading the main story}

\hypertarget{beyoncuxe9s-black-is-king-lets-discuss}{%
\section{Beyoncé's `Black Is King': Let's
Discuss}\label{beyoncuxe9s-black-is-king-lets-discuss}}

Six critics on the visual album rooted in her ``Lion King''-inspired
record ``The Gift,'' a grand statement of African-diaspora pride and
creative power.

\includegraphics{https://static01.nyt.com/images/2020/08/03/arts/31beyonce3/merlin_175164408_4f5428a4-812f-4564-9e16-e2662dbd625c-articleLarge.jpg?quality=75\&auto=webp\&disable=upscale}

By \href{https://www.nytimes.com/by/jason-farago}{Jason Farago},
\href{https://www.nytimes.com/by/vanessa-friedman}{Vanessa Friedman},
\href{https://www.nytimes.com/by/gia-kourlas}{Gia Kourlas},
\href{https://www.nytimes.com/by/wesley-morris}{Wesley Morris},
\href{https://www.nytimes.com/by/jon-pareles}{Jon Pareles} and
\href{https://www.nytimes.com/by/salamishah-tillet}{Salamishah Tillet}

\begin{itemize}
\item
  July 31, 2020
\item
  \begin{itemize}
  \item
  \item
  \item
  \item
  \item
  \item
  \end{itemize}
\end{itemize}

When Beyoncé took a speaking role as Nala --- the eventual queen --- in
the 2019 remake of
\href{https://www.nytimes.com/2019/07/11/movies/the-lion-king-review.html}{``The
Lion King,''} she decided to delve beyond Disney's Hollywood version of
Africa. She added a new, gospel-charged song,
\href{https://www.nytimes.com/2019/07/12/arts/music/playlist-beyonce-billie-eilish-justin-bieber-ed-sheeran.html}{``Spirit,''}
to the film's soundtrack, and gathered an international coalition,
featuring up-and-coming African songwriters and producers, to join her
on a full-length album,
\href{https://www.nytimes.com/2019/07/24/arts/music/beyonce-the-lion-king-the-gift-review.html}{``The
Lion King: The Gift.''} Now she has turned songs from the album into a
film of her own, working with various directors as she did on her visual
albums
\href{https://www.nytimes.com/2013/12/14/arts/music/beyonces-new-album-is-steamy-and-sleek.html}{``Beyoncé''}
and
\href{https://www.nytimes.com/2016/04/25/arts/music/beyonce-lemonade.html}{``Lemonade.''}
Here, critics for The New York Times discuss the imagery and
implications of ``Black Is King.''

\hypertarget{wesley-morris-critic-at-large}{%
\subsection{Wesley Morris, critic at
large}\label{wesley-morris-critic-at-large}}

Let's take a moment, shall we, to appreciate that beauty will make you
tolerate anything, including waking up at the crack of dawn to behold
it. Very little compares to the rising sun. Often not much tops Beyoncé
and the extremes sometimes required to experience her (canceling an
evening, dropping everything, getting filthy at Coachella). ``Black Is
King'' is rather humane. You simply drag yourself from bed, head to
Disney+, and the beauty begins.

Well past the halfway point, Beyoncé is just facing Kelly Rowland,
serenading her, \emph{beaming} at her. The sincerity is so intense,
Rowland has to avert her eyes. She's girlishly overcome. The sunrise is
too much. Not for me. Most of this film ripples with that kind of love
--- of people, of bodies, of the elements, of self, of stuff. (Someone
involved really loved the death car from the great
``\href{https://sgtr.files.wordpress.com/2013/01/holy-motors2.png}{Holy
Motors}''; a tricked-out homage rides here, too.)

My usual qualm with the Beyoncé visual experience applies to this one:
The people who've edited it don't allow us to savor a single shot for
longer than a few seconds. It adheres to ancient music-video ideas of
chaos, incoherence and looks. Steadiness was part of the thrill of her
\href{https://www.nytimes.com/2018/04/15/arts/music/beyonce-coachella-review.html}{at
Coachella}. The stagecraft transfixed the cameras; the editing deferred
to motion. What if the songs here were wedded to full-blown set pieces,
in addition to kaleidoscopic exuberance? That, I suppose, would make the
project a musical. And that's not what this wants to be. But I'm greedy.
When I see a handful of dancers and Beyoncé awash in so much whiteness
that all the other color comes from skin and flowers, I just want five
minutes of that.

\includegraphics{https://static01.nyt.com/images/2020/07/31/arts/31beyonce5/merlin_175156821_2195e97c-2e77-4f45-8b1f-772502b5b4dc-articleLarge.jpg?quality=75\&auto=webp\&disable=upscale}

Tableaux do exist here, minced as they are. (That brown-on-white passage
is from ``Nile.'') The strongest come during ``My Power,'' and ``Mood 4
Eva.'' The latter finds itself on somebody's estate and features the
Knowles-Carters a-floss and a-flex. There's a real Baz Luhrmann zaniness
working here, from the synchronized,
\href{https://www.youtube.com/watch?v=gsp-LE_agns}{Esther Williams pool
party} (everybody side-dives in except our star) to the manic instant
grins that Beyoncé, the movie's wee boy-prince and her mother, Tina
Knowles-Lawson, flash. You could sense that those were good afternoons
for everybody. It hits the spot.

``Beyoncé'' and ``Lemonade'' were triple-impact shocks (new music, new
images, new ideas). ``Black Is King'' extends more than innovates. It's
playing. Beauty is a reason this film exists. The interstitial language
that Beyoncé recites hails, just as it did in ``Lemonade,'' in part,
from the earthen
\href{https://www.poetryfoundation.org/poets/warsan-shire}{poetry of
Warsan Shire}. ``We were beauty before they knew what beauty was'' and
``your skin is not only dark'' are two of the recital's most
exhilarating lines. They offer the beauty of correction. They approach
another of the film's strengths: rebuke --- of, in its title and closing
sequence, the gospel opportunism in Kanye West's film
\href{https://www.nytimes.com/2019/10/27/arts/music/kanye-west-jesus-is-king-review.html}{``Jesus
Is King.''}

And, perhaps, of ``The Lion King.'' What else is this but a restoration
of flesh and blood to cartoon landscapes? There are references to
\href{https://www.criterionchannel.com/daughters-of-the-dust}{Julie
Dash} and
\href{https://www.google.com/search?q=African+American+Flag+david+hammons\&client=safari\&rls=en\&sxsrf=ALeKk03IfUCPOgFtGYIYz_4RbNgFsPggpw:1596203835640\&source=lnms\&tbm=isch\&sa=X\&ved=2ahUKEwiX2PLU0vfqAhVsRN8KHVcvAb4Q_AUoAnoECBgQBA\&biw=1694\&bih=957\#imgrc=xTZg8oh2K3EHwM}{David
Hammons} and appearances by the musician
\href{https://www.youtube.com/watch?v=9RBsGd3eXBw}{Moonchild Sanelly},
the model \href{https://www.instagram.com/adutakech/?hl=en}{Adut Akech}
and the dancehall star
\href{https://en.wikipedia.org/wiki/Shatta_Wale}{Shatta Wale}: a
motherland connection. Many a notable Black American has managed
amazement in Africa: Malcolm X, James Brown and Muhammad Ali, Nina
Simone, her ashes. Beyoncé's trip feels like a search for confirmation:
a living myth roving terrain where myths were made.

\hypertarget{jon-pareles-chief-pop-critic}{%
\subsection{Jon Pareles, chief pop
critic}\label{jon-pareles-chief-pop-critic}}

``The Lion King: The Gift,'' Beyoncé's companion album to the ``Lion
King'' soundtrack, was a
\href{https://www.nytimes.com/2019/07/24/arts/music/beyonce-the-lion-king-the-gift-review.html}{grand
statement of African-diaspora unity}, pride and creative power. It
presented modern African voices and contemporary African sounds ---
among the most kinetic productions in pop --- not as exotic guests of
their American collaborators, but as equals reinforcing each other, an
international brotherhood and sisterhood.

``Black Is King,'' Beyoncé's visual album built on that album's songs,
goes even further. The deluxe version of ``The Lion King: The Gift''
only slightly extends the original album; its major addition is two
versions (one with marching band-style horns) of
\href{https://www.nytimes.com/2020/06/26/arts/music/playlist-beyonce-prince.html}{``Black
Parade,''} a song that addresses current Black Lives Matter protests and
much more. The deluxe version also, mercifully, eliminates the original
album's snippets of ``Lion King'' dialogue.

There's still some ``Lion King'' material in the ``Black Is King''
visual album to detail some of its messages, along with bits of lectures
that equate kingship with responsible manhood. Beyoncé also recites
Warsan Shire's poetry to insist on Africa's ancestral legacies and the
glories of Black beauty. Other transitions use African traditional music
from Smithsonian Folkways recordings, tacitly suggesting the continuity
of old and new. And now and then, there are glimpses within the music,
like a magnificent, purple-suited choir joining Beyoncé to sing
``Spirit'' a cappella.

Image

The fashion in ``Black Is King'' spans the famous and the little-known,
as well as the globe.Credit...Andrew White/Parkwood Entertainment via
Disney+, via Associated Press

Beyoncé is unquestionably the star of ``Black Is King.'' She's presented
as a panoply of archetypes --- mother, boss, clubgoer, biker, queen ---
with an apparently infinite wardrobe that draws on ancient African
iconography alongside extravagant haute couture. She places herself in
glorious open landscapes, a mansion, a gritty warehouse and a
leopard-patterned Rolls-Royce.

But she shares the screen with African and Black American faces:
dancers, tribal elders, city hustlers, judges in wigs and robes,
hoop-skirted debutantes and their beaus. And she willingly lets herself
be upstaged by African collaborators whose faces her American fans may
not yet have seen, like Busiswa from South Africa, Salatiel from
Cameroon and Yemi Alade, Tekno and Mr Eazi from Nigeria. It puts her
pan-African solidarity incontrovertibly onscreen.

\hypertarget{vanessa-friedman-fashion-director-and-chief-fashion-critic}{%
\subsection{Vanessa Friedman, fashion director and chief fashion
critic}\label{vanessa-friedman-fashion-director-and-chief-fashion-critic}}

To describe the amount of fashion on display in ``Black Is King'' as an
``extravaganza'' or a ``feast'' or any of the other words used generally
to convey exciting haute-runway content doesn't even begin to come close
to the reality of the production. ``Overwhelming'' might be more like
it. Beyoncé contains multitudes when it comes to artistic collaboration,
and when it comes to designers, too. They span the famous and the
little-known, as well as the globe.

An incomplete list of brands represented, for example, would include
Valentino couture (cheetah-print bodysuit); Erdem (rose-festooned giant
flounce tea dress); Burberry (cowhide cow print); Thierry Mugler
(rainbow printed jersey draped minidress); Molly Goddard (explosive
fuchsia tulle confection); and Marine Serre (moon-print bodysuit). Also
newish names such as the London-based
\href{https://www.instagram.com/p/CDTR8fsA1Y_/}{Michaela Stark} (denim
corset and puddling jeans), the Ivory Coast-based Loza Maléombho
(graphic print gold-buttoned jacket) and the Tel Aviv-based Alon Livné
(white crocheted gown). Also --- well. You get the idea.

There's not even one look per song; more like dozens. Especially when
you include the dancers and special guests like Naomi Campbell and Adut
Akech. I started taking notes and then gave up and just abandoned myself
to the visual excess.

It's dazzling, but also carefully calculated. Because what so much
muchness means is that no single designer ever reaches critical mass;
blink and you miss them as one more lavish creation strobes into the
next. All of them exist to serve the vision of one woman; to elevate the
imagery of Beyoncé, rather than their own.

As a result you are left with fleeting impressions rather than the
remembrance of any specific garment past: the tropes of majesty, Africa,
the natural world, the power shoulder, and the goddess, stretching from
the Nile to Versailles to Vegas.

They tap into our aesthetic memory archive via jewel tones, billowing
robes, drapes of diamanté and pearls. Via taffeta, silk and tulle;
fringe and cleavage and animal print. Via piles of accessories:
rhinestone sunglasses and gleaming, wearable circles of life.

Sorry, bangles and hoop earrings.

Image

Credit...Andrew White/Parkwood Entertainment and Disney+, via Associated
Press

It's a highly effective strategy in a world where artists tend to link
up with a single brand to define and redefine their public styles
(Ariana Grande and Versace;
\href{https://www.nytimes.com/2018/09/07/style/elton-john-farewell-tour-wardrobe-gucci.html}{Elton
John and Gucci)}, and one Beyoncé has been
\href{https://www.nytimes.com/2016/04/26/fashion/fashion-beyonce-lemonade.html}{honing
over the last decade}. She spreads her beneficence and beauty around,
which has the effect of both reinforcing her position as the ultimate
cultural tastemaker and rendering her subjects abjectly grateful for her
patronage.

It also serves to concentrate all the power in her own hands, making the
garments into tools to reinforce her message. Or part of it, anyway.

What the clothes in ``Black Is King'' do not do, though, unlike the rest
of the film, is reimagine or reclaim the narrative of fashion as written
by Black designers; many of the brands involved are run by white
creatives. Perhaps it's because the movie was made before George Floyd's
death transformed the summer, but in her
\href{https://www.instagram.com/p/CCAMxfrHjAL/}{Instagram statement} on
the work, Beyoncé has directly connected the film to the moment. Which
makes the fashion credits, fabulous as they are, seem like the rare
oversight on her part and that of her stylist and costume designer,
Zerina Akers.

Perhaps that's unfair; she does, after all, amalgamate them into a world
of her own making. But while Black may be king, this project and all its
trappings position its auteur, as the voice-over says in the film, as
the ``divine archetype.'' In that context, she raised the stakes
herself.

\hypertarget{salamishah-tillet-contributing-critic}{%
\subsection{Salamishah Tillet, contributing
critic}\label{salamishah-tillet-contributing-critic}}

A little over an hour into ``Black Is King,'' Beyoncé, with tears in her
eyes, places a baby boy, wrapped in a blanket, up a river inside a reed
basket. Unlike the mélange of sounds --- Afropop, dancehall, hip-hop,
and soul --- that I'd heard up to this point, the accompanying ballad,
``Otherside'' was such a sonic break from the high-tempo energy that I
paused the stream several times. I was moved by this scene of maternal
sacrifice, for even though I knew the plot of ``The Lion King,'' I found
myself hoping that this baby would survive the currents of the rushing
river.

This is because that baby was never just a baby, and this story was
never really simply the human version of Simba's journey into manhood,
much less kingship. On the surface, this river bed scene is an update of
that Old Testament story in which Jochebed, the mother of Moses, placed
him in the Nile River to protect him from being killed. But, the waters
here also invoke the Middle Passage, with each ripple break recalling
the fateful journey in which New World slavery, and America itself, was
born.

Moses has always loomed large among African-Americans seeking freedom.
It is why Harriet Tubman sang the spiritual ``Go Down, Moses'' as a code
to identify herself to those enslaved people who wanted to go with her
to the Promised Land. And while ``Black Is King'' shares those
19th-century aspirations of equality and Black dignity, it, in our age
of Black Lives Matter, knows it has to resort to mythmaking since racial
justice remains as firm as the shifting sands that backdrop so much of
this visual album.

Image

Moses has always loomed large among African-Americans seeking freedom.
Credit...Robin Harper/Parkwood Entertainment and Disney +, via
Associated Press

A few years before he sailed from Brooklyn for West Africa in 1923, the
young African-American writer Langston Hughes penned ``The Negro Speaks
of Rivers,'' an 11-line poem that traverses the Euphrates, the Nile and
the Mississippi River, and ends up in New Orleans. And Beyoncé would one
day feature that city in ``Lemonade,'' her film from 2016.

Much will be debated about whether ``Black Is King'' is an
African-American fantasy of Africa, or a homage to those contemporary
artists from Nigeria, Ghana, South Africa, Cameroon and Mali with whom
she collaborated, or whether the ``other side'' is the New World or a
prodigal return of the descendants of the enslaved to the Old World. I
saw her rivers, like Hughes's, as somewhere in between. Ancient. Dusky.
But also decidedly modern, and fuchsia, teal and gold. An in-between
space that is the hyphen, and the Diaspora, one that Black people have
had to continually create as resistance, and community. As Beyoncé says
in one scene, ``This is how we journey --- far --- and can still find
something like home.''

\hypertarget{jason-farago-art-critic}{%
\subsection{Jason Farago, art critic}\label{jason-farago-art-critic}}

It's been a long road for me and Beyoncé: We're now 20 years from the
day I leeched ``Bills, Bills, Bills'' from Napster. But this new film is
the kitschiest thing she's done in a while, and in ``Black Is King'' her
evident passion for African art keeps getting drowned in an ocean of
melodrama.

Ms. Knowles-Carter, and even more her husband, often showcase
contemporary art in their videos as markers of their cultural and
economic clout, and in the sequence devoted to ``Mood 4 Eva,'' a
Jay-and-Bey duet with samples from the great Malian diva Oumou Sangaré,
the walls of a hacienda are hung with a large portrait of Black models
by the American artist Derrick Adams, and another in the manner of the
British painter Lynette Yiadom-Boakye. I caught multiple direct
quotations of the French fashion photographer Jean-Paul Goude --- most
overtly his cover art for
\href{https://www.discogs.com/Grace-Jones-Island-Life/master/45107}{Grace
Jones's ``Island Life,''} remade by multiple dancers here in the film's
best sequence, for the gqom banger ``My Power.''

Other sequences seem to channel (to be generous) or crib (to be less so)
the work of contemporary African artists. The Ethiopian photographer
\href{https://www.theatlantic.com/magazine/archive/2019/06/aida-muluneh-the-world-is-9/588061/}{Aïda
Muluneh} is a clear influence on several tableaux of African models
posing in bright colors with painted faces. The film's recurrent
character of a topless, green-painted dancer seems to be borrowed from
the Nigerian artist Jelili Atiku, whose 2018 procession
``\href{http://m12.manifesta.org/festino-della-terra-alaraagbo-xiii-2018/}{Festival
of the Earth}'' brought performers slicked with green to the streets of
Sicily. The cinematography, throughout, is of a notably lower standard
than the careful lensing of her self-titled visual album and,
especially, ``Lemonade.'' The beachfront posing in ``Bigger,'' the
opening number, feels uncannily like a perfume ad.

Traditional African art, or imitations of it, gets screen time too.
Backup dancers in ``Find Your Way Back'' sport
\href{https://www.metmuseum.org/art/collection/search/315061}{kanaga
masks topped with crossbars}, worn by the Dogon people of Mali; ``Ja Ara
E'' features a spirit in a full-body raffia costume, familiar from Mende
masquerades. And there's a knowing flash of a catalog of Yoruba masks
and sculpture by Robert Farris Thompson, the influential historian of
West African art.

Late in ``Black Is King'' comes a maudlin apotheosis: The Simba
stand-in, sporting a leopard-print dinner jacket, arises to heaven
inside Johannesburg's apartheid-era Ponte Tower. It's a sequence
stripped of history, and confirms that we are nowhere near any
contemporary African city; we are in a cartoon fairyland, still rooted
in source material appropriate, per Disney, for children 6 years and
older. At least, then, there is Beyoncé's endless string of citations, a
rope ladder for those fans of hers ready to graduate into artistic
adulthood.

\hypertarget{gia-kourlas-dance-critic}{%
\subsection{Gia Kourlas, dance critic}\label{gia-kourlas-dance-critic}}

The choreographic feat of ``Black Is King'' isn't in its flashes of
dancing, exuberant as they are. Those fleeting infusions of footwork and
swirling arms leave behind rich afterimages, but what drives this lavish
visual spectacle is its rush of bodies and how the whole thing moves:
from swift changes of scenery, which are frequent yet never frenzied, to
boldly spare moments of stillness.

One seemingly quiet moment that made me gasp? An overhead shot during
``Brown Skin Girl,'' in which dancers playing debutantes etch a diagonal
line across the screen. The angle gives their voluminous ball gowns the
look of tutus and turns their white gloves into wings as they slowly
arch back. Opening their arms, they are transformed into beautiful Black
swans.

Later in the number, they return, reaching their gloved hands into the
center of a circle. ``Keep dancing/They can't control you,'' Beyoncé
sings. It's simply put, yet so empowering.

In this celebration of the Black body, there is music worthy of a
thousand dances (and, judging by the credits, 11 choreographers). In
``Already'' (performed by Beyoncé, Shatta Wale and Major Lazer), we see
the body on a pedestal, with sculptural moments that range from emphatic
to dreamy as women stand on wooden crates. Like Beyoncé, they wear
unitards that make it seem as if their bodies are covered in scales;
finding a hypnotic groove, they shift their weight from side to side
with elbows as bent as their knees.

Image

Beyoncé's dancing is luminous throughout ``Black Is
King.''Credit...Null/Parkwood Entertainment via Disney +, via Associated
Press

They also pause in arresting, stationary balancing poses, whether
kneeling or with a leg extended high to the side; when Beyoncé bends
backward, the others wrap around her body like a pile of tangled snakes.
In another scene, dancers from the DWP Academy in Ghana perform a
driving unison line dance with the intense, passionate
\href{https://www.instagram.com/p/CDTRa6wByKt/}{Dancegod Lloyd} front
and center. It points to the mix of African and American that Beyoncé
seems intent on getting right.

But she also looks at her own history. In the fantastic and fantastical
``Mood 4 Eva,'' she and Jay-Z stand before a painting, just like they
did in their \href{https://www.youtube.com/watch?v=kbMqWXnpXcA}{video
for ``Apes**t,''}set
\href{https://www.nytimes.com/2018/06/17/arts/design/louvre-jay-z-beyonce-video.html}{at
the Louvre}; here, instead of the Mona Lisa it's a rendering of Beyoncé
in Madonna and Child. Within the song's scene is another clever twist: a
Busby Berkeley-inspired synchronized swimming number led by Black
bodies. In that underwater dance, they slip sideways into the water like
jewels. Of course, Beyoncé rises from the center --- the most powerful
body of all.

Her dancing is luminous throughout ``Black Is King.'' I love the
contrast of how peaceful she remains as her hands perform a dazzling
dance with one wrist flitting over the other in ``Find Your Way Back''
and how, seconds later, her body follows, bowing and rippling to the
sweeping rhythm. In the majestic ``My Power,'' she pushes with force yet
not without freedom. She never holds back, but this time it's different:
It's as if she's trying to move beyond her body, and that brings a line
from Childish Gambino's bridge in ``Mood'' to life. She dances with
ancestors in her step.

Advertisement

\protect\hyperlink{after-bottom}{Continue reading the main story}

\hypertarget{site-index}{%
\subsection{Site Index}\label{site-index}}

\hypertarget{site-information-navigation}{%
\subsection{Site Information
Navigation}\label{site-information-navigation}}

\begin{itemize}
\tightlist
\item
  \href{https://help.nytimes.com/hc/en-us/articles/115014792127-Copyright-notice}{©~2020~The
  New York Times Company}
\end{itemize}

\begin{itemize}
\tightlist
\item
  \href{https://www.nytco.com/}{NYTCo}
\item
  \href{https://help.nytimes.com/hc/en-us/articles/115015385887-Contact-Us}{Contact
  Us}
\item
  \href{https://www.nytco.com/careers/}{Work with us}
\item
  \href{https://nytmediakit.com/}{Advertise}
\item
  \href{http://www.tbrandstudio.com/}{T Brand Studio}
\item
  \href{https://www.nytimes.com/privacy/cookie-policy\#how-do-i-manage-trackers}{Your
  Ad Choices}
\item
  \href{https://www.nytimes.com/privacy}{Privacy}
\item
  \href{https://help.nytimes.com/hc/en-us/articles/115014893428-Terms-of-service}{Terms
  of Service}
\item
  \href{https://help.nytimes.com/hc/en-us/articles/115014893968-Terms-of-sale}{Terms
  of Sale}
\item
  \href{https://spiderbites.nytimes.com}{Site Map}
\item
  \href{https://help.nytimes.com/hc/en-us}{Help}
\item
  \href{https://www.nytimes.com/subscription?campaignId=37WXW}{Subscriptions}
\end{itemize}
