\href{/section/books}{Books}\textbar{}Raven Leilani, a Flâneur Who Is
Going Places

\url{https://nyti.ms/3ffTMXt}

\begin{itemize}
\item
\item
\item
\item
\item
\end{itemize}

\href{https://www.nytimes.com/spotlight/at-home?action=click\&pgtype=Article\&state=default\&region=TOP_BANNER\&context=at_home_menu}{At
Home}

\begin{itemize}
\tightlist
\item
  \href{https://www.nytimes.com/2020/07/28/books/time-for-a-literary-road-trip.html?action=click\&pgtype=Article\&state=default\&region=TOP_BANNER\&context=at_home_menu}{Take:
  A Literary Road Trip}
\item
  \href{https://www.nytimes.com/2020/07/29/magazine/bored-with-your-home-cooking-some-smoky-eggplant-will-fix-that.html?action=click\&pgtype=Article\&state=default\&region=TOP_BANNER\&context=at_home_menu}{Cook:
  Smoky Eggplant}
\item
  \href{https://www.nytimes.com/2020/07/27/travel/moose-michigan-isle-royale.html?action=click\&pgtype=Article\&state=default\&region=TOP_BANNER\&context=at_home_menu}{Look
  Out: For Moose}
\item
  \href{https://www.nytimes.com/interactive/2020/at-home/even-more-reporters-editors-diaries-lists-recommendations.html?action=click\&pgtype=Article\&state=default\&region=TOP_BANNER\&context=at_home_menu}{Explore:
  Reporters' Obsessions}
\end{itemize}

\includegraphics{https://static01.nyt.com/images/2020/07/28/books/28Leilani2/28Leilani2-articleLarge-v4.jpg?quality=75\&auto=webp\&disable=upscale}

Sections

\protect\hyperlink{site-content}{Skip to
content}\protect\hyperlink{site-index}{Skip to site index}

\hypertarget{raven-leilani-a-fluxe2neur-who-is-going-places}{%
\section{Raven Leilani, a Flâneur Who Is Going
Places}\label{raven-leilani-a-fluxe2neur-who-is-going-places}}

The novelist's debut, ``Luster,'' is winning accolades for its
unfiltered depiction of sex, failure and a Black woman adrift in work
and life.

Raven Leilani in Brooklyn. For the protagonist of her book, ``Luster,''
she said, ``I didn't want her to be a pristine, neatly moral
character.''Credit...Miranda Barnes for The New York Times

Supported by

\protect\hyperlink{after-sponsor}{Continue reading the main story}

\href{https://www.nytimes.com/by/concepcion-de-leon}{\includegraphics{https://static01.nyt.com/images/2018/07/16/multimedia/author-concepcion-de-leon/author-concepcion-de-leon-thumbLarge.png}}

By \href{https://www.nytimes.com/by/concepcion-de-leon}{Concepción de
León}

\begin{itemize}
\item
  July 31, 2020
\item
  \begin{itemize}
  \item
  \item
  \item
  \item
  \item
  \end{itemize}
\end{itemize}

Raven Leilani has one of this summer's
\href{https://www.nytimes.com/2020/07/30/books/new-august-books.html}{most
anticipated fiction debuts}, but in some ways, she is already
anticipating the day the buzz dies down.

That is when she plans to take some time to grieve the loss of her
father, Warren, who died from Covid-19 in April. ``There's an aspect of
this moment --- because of the enormity of it, you see the number of
people who have died --- it feels abstract,'' she said in an interview.
``But it's not abstract at all. Every single number was a person, and
one of those was my dad.''

Because of their complicated relationship, her parents' separation when
Leilani was in college and the forced isolation of the coronavirus
pandemic, she has had to process the loss alone. That kind of solitude
is not what she is used to, having grown up first in the Bronx, then a
suburb of Albany, N.Y., in a family of West Indian artists who
encouraged her creativity.

For years, she juggled jobs and art, doing her writing at night or
during work shifts. ``The going was slow and the going was private,''
Leilani said. ``There was a frenzy to that grind.''

It's a frenzy she captures in her novel, ``Luster,'' out on Tuesday. It
follows Edie, a Black woman in her 20s scraping by on a publishing
salary while trying to self-actualize as an artist. When Edie meets
Eric, an older, married white man whose wife has agreed to an open
marriage, Edie becomes entangled with them and their daughter --- an
adopted Black 12-year-old named Akila --- in unexpected ways.

``I wanted to write a story about a Black woman who fails a lot and is
sort of grasping for human connection and making mistakes,'' Leilani,
now 29, said. ``I didn't want her to be a pristine, neatly moral
character.''

\includegraphics{https://static01.nyt.com/images/2020/08/18/books/18Leilani/18Leilani-articleLarge.jpg?quality=75\&auto=webp\&disable=upscale}

Farrar, Straus and Giroux, Leilani's publisher, has named ``Luster'' its
novel of August, part of \href{https://fsg2020.com/}{a campaign this
year} highlighting reading ``for solace, for protection, for
instruction, for survival, for music.'' ``She is exactly the kind of
writer that we've always published and that we've always been dedicated
to publishing --- someone who is an artist and a craftsman, but also
someone who is speaking to her moment and our cultural history,'' Jenna
Johnson, who acquired the book for the publisher, said.

Ahead of its publication, ``Luster'' has already been praised by other
writers, including Carmen Maria Machado, Brit Bennett and Angela
Flournoy. In an email, Zadie Smith, who taught Leilani in grad school,
called ``Luster'' a ``daring, perverse, wildly funny book about how we
use each other --- especially how the old use the young, socially,
economically and intimately.''

Machado, the author of
``\href{https://www.nytimes.com/2019/10/29/books/in-dream-house-memoir-carmen-maria-machado.html}{In
the Dream House}'' and
``\href{https://www.nytimes.com/2017/10/04/books/review-her-body-and-other-parties-carmen-maria-machado.html}{Her
Body and Other Parties},'' said ``Luster'' ``took me by the throat and
didn't really let me go,'' particularly when it came to the way Leilani
writes about sex. ``They were hot and real and also did all the things I
want sex scenes to do, which is feel realistic, sometimes be sexy,
sometimes be unpleasant or stressful, but allowing for both, allowing
for real bodies,'' Machado said.

For Leilani, those scenes were her way to capture ``a free Black girl''
and the ``perversity'' of sexual thoughts when allowed to roam free. She
also wanted to highlight a nonlinear artistic path, one that came in
contact with the real world. ``You talk to other writers and they're
sort of dogged by this specter of `I'm not making anything,''' she said,
``but for most of us, that's the reality of making art, is not making
it.''

That Edie is a painter is no coincidence. As a teenager, Leilani
expected that she would be a visual artist as well. She attended a high
school with a strong art program, where she said she and her classmates
engaged in serious critiques of their work. But when it came time to
apply for college, she realized that she wasn't quite good enough to
make a career out of painting.

``I still loved it a lot, and I think you can see that in a lot of my
writing, but with grappling with those artistic limits, I found that it
took the love out of it a little bit,'' she said. ``With writing, that's
not the case. Even when it's hard, I still love it.''

After graduating college in 2012, Leilani took the first job she could
find, as an imaging specialist at Ancestry.com. She went on to work at a
scientific journal, on a top-secret project for the Department of
Defense and as a Postmates delivery person. When she moved from
Washington to New York to pursue her M.F.A. at New York University in
2017, she joined Macmillan as a production associate.

``I'd write inside the HTML of the e-books, so it looked like I was
making corrections,'' she said, ``but I was writing `Luster.''' At other
jobs, she wrote on the backs of receipts or in email drafts. Leilani
started writing under her first and middle name --- her surname is
Baptiste ---~as a way to separate her literary work from her employment.

Those years are present in much of her work. The job at the Department
of Defense inspired the short story
``\href{https://cosmonautsavenue.com/raven-leilani-fiction/}{Hard
Water}.'' In 2016, she found herself having trouble breathing for half a
year and turned the experience into the story
``\href{https://cosmonautsavenue.com/raven-leilani-fiction/}{Breathing
Exercise},'' published in the Yale Review. And she was adamant that work
play a big role in the lives of the characters in ``Luster.''

``It was important to me,'' Leilani said, ``to have a book where
characters have work, where characters have something they do and care
about.''

Image

``Luster,'' Leilani said, ``was an experiment in speaking honestly and
in committing to a distinct point of view.''Credit...Miranda Barnes for
The New York Times

Her early writing years involved a lot of trial and error, including a
``sexy science fiction'' novel and another that drew on her love of
comic books and music. ``I felt preoccupied with the idea of an original
product. I wanted it to be weird and I wanted it to be strange and I
wanted it to feel new,'' she said. ``But when I was working on those
projects, they felt very opaque and without purpose.''

So when she got to grad school, she discarded them and thought, ``I can
do better. And not just do better, but write something that I really
mean.'' ``Luster,'' Leilani said, ``was an experiment in speaking
honestly and in committing to a distinct point of view.''

Her goal in developing the character of Edie was to melt away the
``studiedness'' that people --- especially Black people --- learn as a
survival mechanism in a world where they are constantly surveilled. ``I
wanted Edie to take up space,'' she said. ``I wanted her to always be
articulating to us, even though she's not articulating to the people in
her environment, what she wanted.''

In a
\href{https://www.vqronline.org/fiction-criticism/2020/06/sex-city}{review
of the book} in the Virginia Quarterly Review, the writer Kaitlyn
Greenidge described Edie as a Black flâneur, one who walks through the
city cataloging her surroundings, blending in as best she can~with the
crowd.

``She is playing with language in such an invigorating way,'' Greenidge
said in an interview. ``People say that about literary novels all the
time: `oh, the language, the language, the language.' But oftentimes
that ends up in inscrutable or not very exciting sentences. That is not
the case with Raven. Her use of language is truly surprising.''

Leilani, who often clears her head with long walks around New York City,
was struck by the flâneur comparison. She credits poetry as formative to
her writing. ``There's something beautiful about rhythm, about style,
about pattern,'' she said. ``I think because I started with a love of
poetry, the way I sort of transitioned into writing prose and
novel-length stuff and short fiction is that I still felt obsessed by
the part of writing that is about language.'' She often obsesses over
sentence-level changes and won't move on until she gets it just right.

Now she writes full-time, spending most of her days seated on her bed,
slowly making a dent in it while she writes until ``the sun is gone.''

``Because so much of my life has been work, has been a deferral of my
dream to make anything in terms of my art,'' Leilani said, ``it feels
incredible that my days right now can be about that. It feels magical.''

\emph{Follow New York Times Books on}
\href{https://www.facebook.com/nytbooks/}{\emph{Facebook}}\emph{,}
\href{https://twitter.com/nytimesbooks}{\emph{Twitter}} \emph{and}
\href{https://www.instagram.com/nytbooks/}{\emph{Instagram}}\emph{, sign
up for}
\href{https://www.nytimes.com/newsletters/books-review}{\emph{our
newsletter}} \emph{or}
\href{https://www.nytimes.com/interactive/2017/books/books-calendar.html}{\emph{our
literary calendar}}\emph{. And listen to us on the}
\href{https://www.nytimes.com/column/book-review-podcast}{\emph{Book
Review podcast}}\emph{.}

Advertisement

\protect\hyperlink{after-bottom}{Continue reading the main story}

\hypertarget{site-index}{%
\subsection{Site Index}\label{site-index}}

\hypertarget{site-information-navigation}{%
\subsection{Site Information
Navigation}\label{site-information-navigation}}

\begin{itemize}
\tightlist
\item
  \href{https://help.nytimes.com/hc/en-us/articles/115014792127-Copyright-notice}{©~2020~The
  New York Times Company}
\end{itemize}

\begin{itemize}
\tightlist
\item
  \href{https://www.nytco.com/}{NYTCo}
\item
  \href{https://help.nytimes.com/hc/en-us/articles/115015385887-Contact-Us}{Contact
  Us}
\item
  \href{https://www.nytco.com/careers/}{Work with us}
\item
  \href{https://nytmediakit.com/}{Advertise}
\item
  \href{http://www.tbrandstudio.com/}{T Brand Studio}
\item
  \href{https://www.nytimes.com/privacy/cookie-policy\#how-do-i-manage-trackers}{Your
  Ad Choices}
\item
  \href{https://www.nytimes.com/privacy}{Privacy}
\item
  \href{https://help.nytimes.com/hc/en-us/articles/115014893428-Terms-of-service}{Terms
  of Service}
\item
  \href{https://help.nytimes.com/hc/en-us/articles/115014893968-Terms-of-sale}{Terms
  of Sale}
\item
  \href{https://spiderbites.nytimes.com}{Site Map}
\item
  \href{https://help.nytimes.com/hc/en-us}{Help}
\item
  \href{https://www.nytimes.com/subscription?campaignId=37WXW}{Subscriptions}
\end{itemize}
