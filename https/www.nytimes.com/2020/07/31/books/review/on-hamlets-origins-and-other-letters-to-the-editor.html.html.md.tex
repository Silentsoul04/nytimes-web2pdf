Sections

SEARCH

\protect\hyperlink{site-content}{Skip to
content}\protect\hyperlink{site-index}{Skip to site index}

\href{https://www.nytimes.com/section/books/review}{Book Review}

\href{https://myaccount.nytimes.com/auth/login?response_type=cookie\&client_id=vi}{}

\href{https://www.nytimes.com/section/todayspaper}{Today's Paper}

\href{/section/books/review}{Book Review}\textbar{}On Hamlet's Origins
and Other Letters to the Editor

\url{https://nyti.ms/2DjXO3J}

\begin{itemize}
\item
\item
\item
\item
\item
\end{itemize}

Advertisement

\protect\hyperlink{after-top}{Continue reading the main story}

Supported by

\protect\hyperlink{after-sponsor}{Continue reading the main story}

\hypertarget{on-hamlets-origins-and-other-letters-to-the-editor}{%
\section{On Hamlet's Origins and Other Letters to the
Editor}\label{on-hamlets-origins-and-other-letters-to-the-editor}}

\includegraphics{https://static01.nyt.com/images/2020/07/19/books/review/19Brooks-COVER/19Brooks-COVER-articleLarge.jpg?quality=75\&auto=webp\&disable=upscale}

July 31, 2020

\begin{itemize}
\item
\item
\item
\item
\item
\end{itemize}

\hypertarget{not-to-be}{%
\subsection{Not to Be}\label{not-to-be}}

To the Editor:

Geraldine Brooks's July 19 review of Maggie O'Farrell's ``Hamnet'' asks
this question central to the book: Why did Shakespeare title his most
famous play for the son who had died several years earlier?

In fact, some scholars think that Shakespeare based the play on a
history of the Danes featuring the tale of Amleth (anagram of Hamlet), a
great ruler, by Saxo Grammaticus. Saxo's tale was translated into French
in 1514 by François de Belleforest, author of ``Histoires Tragiques.''
In English, his text was called ``The Historie of Hamblet.''

There is also speculation that Shakespeare's ``Hamlet'' was based on an
earlier play by the Elizabethan playwright Thomas Kyd that features a
character named Hamlet who seeks revenge.

Shakespeare's twins, Hamnet and Judith, born in 1585, were named after
his Stratford neighbors who, in turn, named one of their sons William.
Brooks also states as fact that ``Shakespeare was a grammar school
graduate.'' However, this is pure conjecture. No evidence has ever been
found that a William\\
Shakespeare of Stratford was ever enrolled in any school, the local
grammar school or

university.

We cannot assert with confidence then that Shakespeare's play ``Hamlet''
has really anything to do with the death of his son.

A. Birt\\
Uxbridge, Ontario

♦

To the Editor:

I'm certain that a number of your readers of Geraldine Brooks's
excellent review of ``Hamnet'' also recalled Anthony Burgess's novel
``Nothing Like the Sun,'' another fictionalized version of Shakespeare's
home life as well as his life in the London theater. While Shakespeare
has a Black mistress in London, his brother is busy on the home front.
Burgess's satire and brilliant imagination might be an excellent
complement to Maggie O'Farrell's creative work.

Marc Ratner\\
Ashland, Ore.

\hypertarget{between-the-lines}{%
\subsection{Between the Lines}\label{between-the-lines}}

To the Editor:

Every week I anticipate and thoroughly enjoy By the Book, your section's
questions and answers with well-known authors. In the July 19 issue,
Charlie Kaufman's responses to the stock questions left me
uncontrollably laughing. His comparing himself to a long-married couple
was brilliant, and I can't wait to read Andrei Zhdanov's tome on
Hollywood. Wow, would I love to be at his literary dinner party! Of
course, the bourbon I've been sipping might be coloring my reaction.

Rand Clark\\
Santa Barbara, Calif.

♦

To the Editor:

In his By the Book interview Charlie Kaufman praises Andrei Zhdanov. Was
this an attempt at being humorous?

Andrei Zhdanov was one of Stalin's most notorious henchmen. He was
responsible for hundreds, if not thousands, of deaths in Stalin's Great
Terror. He was responsible for purging Sergei Prokofiev, Aram
Khachaturian, Anna Akhmatova, Dmitri Shostakovich and many others after
the end of World War II. To praise Zhdanov for his insights into
Hollywood (or anything else for that matter) is absurd.

Paul Einstein\\
Teaneck, N.J.

Advertisement

\protect\hyperlink{after-bottom}{Continue reading the main story}

\hypertarget{site-index}{%
\subsection{Site Index}\label{site-index}}

\hypertarget{site-information-navigation}{%
\subsection{Site Information
Navigation}\label{site-information-navigation}}

\begin{itemize}
\tightlist
\item
  \href{https://help.nytimes.com/hc/en-us/articles/115014792127-Copyright-notice}{©~2020~The
  New York Times Company}
\end{itemize}

\begin{itemize}
\tightlist
\item
  \href{https://www.nytco.com/}{NYTCo}
\item
  \href{https://help.nytimes.com/hc/en-us/articles/115015385887-Contact-Us}{Contact
  Us}
\item
  \href{https://www.nytco.com/careers/}{Work with us}
\item
  \href{https://nytmediakit.com/}{Advertise}
\item
  \href{http://www.tbrandstudio.com/}{T Brand Studio}
\item
  \href{https://www.nytimes.com/privacy/cookie-policy\#how-do-i-manage-trackers}{Your
  Ad Choices}
\item
  \href{https://www.nytimes.com/privacy}{Privacy}
\item
  \href{https://help.nytimes.com/hc/en-us/articles/115014893428-Terms-of-service}{Terms
  of Service}
\item
  \href{https://help.nytimes.com/hc/en-us/articles/115014893968-Terms-of-sale}{Terms
  of Sale}
\item
  \href{https://spiderbites.nytimes.com}{Site Map}
\item
  \href{https://help.nytimes.com/hc/en-us}{Help}
\item
  \href{https://www.nytimes.com/subscription?campaignId=37WXW}{Subscriptions}
\end{itemize}
