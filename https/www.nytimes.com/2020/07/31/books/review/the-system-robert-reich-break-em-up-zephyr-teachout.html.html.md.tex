Sections

SEARCH

\protect\hyperlink{site-content}{Skip to
content}\protect\hyperlink{site-index}{Skip to site index}

\href{https://www.nytimes.com/section/books/review}{Book Review}

\href{https://myaccount.nytimes.com/auth/login?response_type=cookie\&client_id=vi}{}

\href{https://www.nytimes.com/section/todayspaper}{Today's Paper}

\href{/section/books/review}{Book Review}\textbar{}Why the Working Class
Votes Against Its Economic Interests

\url{https://nyti.ms/3jYHEhe}

\begin{itemize}
\item
\item
\item
\item
\item
\end{itemize}

Advertisement

\protect\hyperlink{after-top}{Continue reading the main story}

Supported by

\protect\hyperlink{after-sponsor}{Continue reading the main story}

nonfiction

\hypertarget{why-the-working-class-votes-against-its-economic-interests}{%
\section{Why the Working Class Votes Against Its Economic
Interests}\label{why-the-working-class-votes-against-its-economic-interests}}

\includegraphics{https://static01.nyt.com/images/2020/08/16/books/review/16Madrick-WEB/16Madrick-WEB-articleLarge.jpg?quality=75\&auto=webp\&disable=upscale}

By \href{https://www.nytimes.com/by/jeff-madrick}{Jeff Madrick}

\begin{itemize}
\item
  July 31, 2020
\item
  \begin{itemize}
  \item
  \item
  \item
  \item
  \item
  \end{itemize}
\end{itemize}

\textbf{THE SYSTEM}\\
\textbf{Who Rigged It, How We Fix It}\\
By Robert B. Reich\\
\textbf{BREAK 'EM UP}\\
\textbf{Recovering Our Freedom From Big Ag, Big Tech, and Big Money}\\
By Zephyr Teachout

One of the mysteries in politics for decades now has been why white
working-class Americans began to vote Republican in large numbers in the
1960s and 1970s. After all, it was Democrats who supported labor unions,
higher minimum wages, expanded unemployment insurance, Medicare and
generous Social Security, helping to lift workers into the middle class.

Of course, an alternative economic view, led by
\href{https://www.econlib.org/library/Enc/bios/Friedman.html}{economists
like Milton Friedman}, was that this turn toward the Republican Party
was rational and served workers' interests. He emphasized free markets,
entrepreneurialism and the maximization of profit. These, Friedman
argued, would raise wages for many and even most Americans.

But wages did not rise. And yet many in the working class kept voting
Republican, still seemingly angered by Lyndon Johnson's Great Society,
which was dedicated to helping the poor and assuring equal rights for
people of color. In the 1980s, under Ronald Reagan, income inequality
began to rise sharply; wages for typical Americans stagnated and poverty
and homelessness increased. Capital investment remained relatively weak
despite deep tax cuts (as it does today under Donald Trump). At the same
time, antitrust regulation was severely wounded, and giant corporations
began to monopolize industry after industry.

In 2004,
\href{https://www.nytimes.com/2004/06/13/books/heartland-security.html}{Thomas
Frank's book ``What's the Matter With Kansas?''} tried to explain why a
once Democratic state had turned resolutely Republican. His eloquent
review of the rhetoric of the age was instructive.

But the presidential election of 2016 sent the sharpest message yet.
Working-class voters in Michigan, Pennsylvania and Wisconsin opted for
Trump, and apparently against their economic interests. Trump had
succeeded in appealing to their anger and the Democrats were caught
flat-footed.

Two new books, ``The System,'' by the former labor secretary
\href{https://robertreich.org}{Robert B. Reich}, and ``Break 'Em Up,''
by the lawyer and activist
\href{https://www.chicagotribune.com/opinion/ct-nyt-zephyr-teachout-joe-biden-corrupt-op-ed-20200125-ny7pqixehrc7jbjdgts7pzny2u-story.html}{Zephyr
Teachout}, a onetime candidate for New York State attorney general, are
among the latest examples of an evolving set of explanations that try to
make sense of the 2016 results.

A powerful money-fueled oligarchy has emerged in America that is an
enemy of democracy, Reich writes. The self-interested power of the
nation's wealthy often goes unnoticed by voters, and is partly
misdirected by right-wing rhetoric about issues like immigration. But it
leads to lower wages, less product choice and abusive labor practices.
Trump has harnessed the frustration of the working class, Reich says,
but he was a ``smokescreen'' for the oligarchy. Reich has an almost
unmatched ability to make insightful observations about the nation's
inequities, and in ``The System,'' he observes that the question is no
longer Democrat versus Republican or left versus right, but ``democracy
versus oligarchy.''

To Teachout, what's behind our rigged system is the close cousin of
oligarchy: corporate monopoly. Teachout lists her culprits, among them
familiar names: Amazon, Google, Facebook, Monsanto, AT\&T, Verizon,
Walmart, Pfizer, Comcast, Apple and CVS. These companies ``represent a
new political phenomenon,'' she says, ``a 21st-century form of
centralized, authoritarian government.''

Two dramatic related facts underscore the claims of both Reich and
Teachout. The much discussed rise of wealth among the top 0.1 percent,
which now has 20 percent of the nation's wealth compared with only 10
percent 40 years ago, has been brought to light in recent years by the
innovative economists \href{https://www.nber.org/papers/w11955}{Thomas
Piketty and Emmanuel Saez}. The flip side is that wages for the large
majority of American workers have stagnated more or less over this same
period.

According to Reich, the ``anti-establishment fury'' that is the result
of such inequity supersedes racial prejudice as the cause of Trump's
success. In 2001, more than three out of four workers were satisfied
that they could get ahead by working hard. In 2014, only slightly more
than one out of two thought so. Voters wanted badly to blame it all on
the swamp Trump promised to clean up.

For Reich, the big oligarchical companies have the lobbying and
campaign-financing muscle to mold the rules in their own favor. They can
win enormous tax cuts, suppress financial and environmental regulations,
acquire new patents and subsidies, fight for free trade --- it is a long
list. For years, they successfully battled against higher minimum wages
and labor laws that restricted their union-busting efforts.

Teachout, a dogged scholar, lays out a comprehensive list of damage done
to American consumers by monopolized industries like Big Pharma, fossil
fuels, Silicon Valley, health insurance, banking and communications
giants from Verizon to Facebook and Google. She provides example after
example of how these companies limit consumer choice and suppress
regulation. Google and Facebook may make access to some news easier, but
they also undermine the profitability of the print news organizations,
putting many of them out of business. Big Pharma is protected from
competition by questionable patents and by ever lighter regulations. The
nation's private health care system, dominated by a relative handful of
insurance companies, keeps costs much higher in the United States than
in the rest of the rich world. For Teachout, the solution follows as
night follows day. Break up the big companies and reintroduce
competition. (Surprisingly, this is straightforward mainstream economic
theory.)

But both Reich and especially Teachout should temper their anticorporate
zeal, at least to a degree. Big companies have often done good while
also doing bad. In the 1800s, the A.\&P. grocery chain provided a wide
range of products, though it put countless mom and pop stores out of
business. Ford built a cheap functional car in the 1920s, and Apple an
affordable personal computer in recent years. Some balance is required.

Still, they are mostly right. Here is Teachout's general recommendation:
``Instead of protesting Pfizer on Tuesday for hiking drug prices,
Comcast on Wednesday for suppressing union voices and Amazon on Thursday
for getting billions in subsidies, we should unite behind a coherent
agenda, demanding that antitrust authorities break up Pfizer and
Comcast, Amazon and Facebook, Monsanto and Tyson.''

Both authors say that Ronald Reagan led the way to the swift undoing of
traditional antitrust regulation in the 1980s. But Reich is almost as
harsh on the Clinton and Obama administrations. Even when the Democrats
controlled both houses of Congress, he writes, they allowed antitrust
enforcement to ``ossify,'' let companies hammer away at trade unions and
went easy on Wall Street. They were also soft on the issue of campaign
contributions, failing to advocate for public financing of elections.

Why? Reich argues that the Democrats chose to turn their backs on the
working class and pursue suburban swing voters. He knows, he tells us.
He was there. And he reports that the Democrats ``drank from the same
campaign funding trough as the Republicans --- big corporations, Wall
Street and the very wealthy.''

Reich makes an example of
\href{https://institute.jpmorganchase.com/about/our-leadership/jamie-dimon}{Jamie
Dimon}, the chairman of JPMorgan Chase. For Reich, he is representative
of the C.E.O. class that talks about corporate social responsibility but
rarely practices it. A lifetime Democrat, Dimon was a major supporter of
the Trump tax cut and does not support an increase in the minimum wage.

Teachout by and large shares Reich's anger and may even exceed it. Yet
both find reasons for optimism in new laws and grass-roots movements.
America achieved marriage equality for gays and lesbians, elected a
Black man president and made the Affordable Care Act law. Reich insists
democracy will ultimately prevail over oligarchy. And Teachout sees
America embarking on a new antimonopoly moment.

These are valuable books, and the anger they will generate may prove
politically energizing. But Reich's claim that democracy will somehow
prevail underestimates the dangers we face. As for Teachout, more
competition may help alleviate some problems, but it is in fact an
idealized version of free market thinking.

Meanwhile, the current president is moving in exactly the opposite
direction. He is promising cuts in social policies that may well
increase middle-income and working-class frustration. He wants to
rewrite the official definition of poverty to claim that there are fewer
poor. He undermines the rule of law on a regular basis. The Supreme
Court has been stacked with extreme conservatives. Voter suppression is
common.

Is it any wonder that many fear democracy in America may not prevail?

Advertisement

\protect\hyperlink{after-bottom}{Continue reading the main story}

\hypertarget{site-index}{%
\subsection{Site Index}\label{site-index}}

\hypertarget{site-information-navigation}{%
\subsection{Site Information
Navigation}\label{site-information-navigation}}

\begin{itemize}
\tightlist
\item
  \href{https://help.nytimes.com/hc/en-us/articles/115014792127-Copyright-notice}{©~2020~The
  New York Times Company}
\end{itemize}

\begin{itemize}
\tightlist
\item
  \href{https://www.nytco.com/}{NYTCo}
\item
  \href{https://help.nytimes.com/hc/en-us/articles/115015385887-Contact-Us}{Contact
  Us}
\item
  \href{https://www.nytco.com/careers/}{Work with us}
\item
  \href{https://nytmediakit.com/}{Advertise}
\item
  \href{http://www.tbrandstudio.com/}{T Brand Studio}
\item
  \href{https://www.nytimes.com/privacy/cookie-policy\#how-do-i-manage-trackers}{Your
  Ad Choices}
\item
  \href{https://www.nytimes.com/privacy}{Privacy}
\item
  \href{https://help.nytimes.com/hc/en-us/articles/115014893428-Terms-of-service}{Terms
  of Service}
\item
  \href{https://help.nytimes.com/hc/en-us/articles/115014893968-Terms-of-sale}{Terms
  of Sale}
\item
  \href{https://spiderbites.nytimes.com}{Site Map}
\item
  \href{https://help.nytimes.com/hc/en-us}{Help}
\item
  \href{https://www.nytimes.com/subscription?campaignId=37WXW}{Subscriptions}
\end{itemize}
