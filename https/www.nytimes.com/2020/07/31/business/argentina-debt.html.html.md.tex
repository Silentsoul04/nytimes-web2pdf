Sections

SEARCH

\protect\hyperlink{site-content}{Skip to
content}\protect\hyperlink{site-index}{Skip to site index}

\href{https://www.nytimes.com/section/business}{Business}

\href{https://myaccount.nytimes.com/auth/login?response_type=cookie\&client_id=vi}{}

\href{https://www.nytimes.com/section/todayspaper}{Today's Paper}

\href{/section/business}{Business}\textbar{}In Argentina's Debt
Negotiations, a Kinder, Gentler Capitalism Faces a Test

\url{https://nyti.ms/3fgv2OH}

\begin{itemize}
\item
\item
\item
\item
\item
\item
\end{itemize}

\href{https://www.nytimes.com/news-event/coronavirus?action=click\&pgtype=Article\&state=default\&region=TOP_BANNER\&context=storylines_menu}{The
Coronavirus Outbreak}

\begin{itemize}
\tightlist
\item
  live\href{https://www.nytimes.com/2020/08/04/world/coronavirus-cases.html?action=click\&pgtype=Article\&state=default\&region=TOP_BANNER\&context=storylines_menu}{Latest
  Updates}
\item
  \href{https://www.nytimes.com/interactive/2020/us/coronavirus-us-cases.html?action=click\&pgtype=Article\&state=default\&region=TOP_BANNER\&context=storylines_menu}{Maps
  and Cases}
\item
  \href{https://www.nytimes.com/interactive/2020/science/coronavirus-vaccine-tracker.html?action=click\&pgtype=Article\&state=default\&region=TOP_BANNER\&context=storylines_menu}{Vaccine
  Tracker}
\item
  \href{https://www.nytimes.com/2020/08/02/us/covid-college-reopening.html?action=click\&pgtype=Article\&state=default\&region=TOP_BANNER\&context=storylines_menu}{College
  Reopening}
\item
  \href{https://www.nytimes.com/live/2020/08/04/business/stock-market-today-coronavirus?action=click\&pgtype=Article\&state=default\&region=TOP_BANNER\&context=storylines_menu}{Economy}
\end{itemize}

Advertisement

\protect\hyperlink{after-top}{Continue reading the main story}

Supported by

\protect\hyperlink{after-sponsor}{Continue reading the main story}

\hypertarget{in-argentinas-debt-negotiations-a-kinder-gentler-capitalism-faces-a-test}{%
\section{In Argentina's Debt Negotiations, a Kinder, Gentler Capitalism
Faces a
Test}\label{in-argentinas-debt-negotiations-a-kinder-gentler-capitalism-faces-a-test}}

BlackRock, the world's largest asset management company, is opposing a
debt settlement deal with Argentina as the country grapples with soaring
poverty and the pandemic.

\includegraphics{https://static01.nyt.com/images/2020/08/01/business/31JPargentinadebt2-print/31argentinadebt-1-articleLarge.jpg?quality=75\&auto=webp\&disable=upscale}

By \href{https://www.nytimes.com/by/peter-s-goodman}{Peter S. Goodman}
and Daniel Politi

\begin{itemize}
\item
  July 31, 2020
\item
  \begin{itemize}
  \item
  \item
  \item
  \item
  \item
  \item
  \end{itemize}
\end{itemize}

\href{https://www.nytimes.com/es/2020/07/31/espanol/negocios/argentina-deuda.html}{Leer
en español}

LONDON --- Laurence D. Fink presents himself as the vanguard of a
progressive form of capitalism in which profits are not everything: The
enlightened money is supposed to press for environmental and social
protection.

As the chief executive of BlackRock, the world's largest investment
management company, Mr. Fink oversees more than \$7 trillion. He has
steered some of that fortune to the crisis-wracked nation of Argentina,
purchasing government bonds.

But as Argentina --- in default since May --- seeks forgiveness on \$66
billion worth of bonds, Mr. Fink's oft-espoused faith in ``stakeholder
capitalism'' is colliding with traditional bottom line imperatives.
Though poverty is soaring in Argentina as the pandemic worsens a
punishing economic downturn, BlackRock is opposing a settlement proposed
by the government and rallying other creditors to reject it, while
holding out for a marginally improved deal.

Mr. Fink has inserted himself into the negotiations, speaking twice with
Argentina's economy minister, according to three people familiar with
the talks. The government and its creditors are only three pennies on
the dollar apart on their proposed terms.

``The BlackRock guys have gotten on the phone with a number of
significant creditors,'' said Hans Humes, president of Greylock Capital
Management, another creditor at the table. ``They convinced a lot of
people that if we all stepped up behind their deal, the Argentines would
take it. It's turned into a brutal standoff.''

BlackRock's stance has put it at odds with the International Monetary
Fund, which gave Argentina a
\href{https://www.nytimes.com/2018/06/07/business/argentina-imf-debt.html}{rescue
package} worth more than \$50 billion two years ago, and has supported
Argentina's proposal as an Aug. 4 deadline approaches.

\includegraphics{https://static01.nyt.com/images/2020/08/01/business/31argentinadebt3-print/merlin_146284194_cbc130c9-6ac7-407a-8d12-7206f9904c77-articleLarge.jpg?quality=75\&auto=webp\&disable=upscale}

The fund's managing director, Kristalina Georgieva, has
\href{https://www.imf.org/en/News/Articles/2020/02/04/pr2034-statement-by-imf-managing-director-kristalina-georgieva-on-argentina}{praised
Argentina's approach} and emphasized that bondholders must agree to
substantial debt forgiveness so Argentina can manage future payments.
Fund officials have assured the government that they will forge a new
bailout if Argentina cannot complete a deal.

The alternative would be an unruly default that would prevent Argentina
from tapping international markets, block its companies from gaining
access to capital and deepen the recession.

BlackRock's position has also put it crosswise with a group of prominent
economists, including a pair of Nobel laureates, Joseph Stiglitz and
Edmund Phelps. In May, they issued a
\href{https://www.reuters.com/article/us-argentina-bonds-economists/nobelist-stiglitz-economists-from-20-countries-back-argentina-in-debt-showdown-idUSKBN22I2V1?il=0}{public
letter} urging bondholders to come to terms with the government.

``Argentina has presented a responsible offer to creditors that reflects
the country's capacity to pay,'' declared the letter, which was signed
by 138 economists, among them Carmen Reinhart, now the chief economist
at the World Bank.

In a statement, BlackRock said it has been working diligently to achieve
a settlement, while recouping as much as possible for its clients.
Roughly two-thirds of the investments it manages comprise the retirement
savings of workers around the world.

``In this restructuring process, our fund managers are balancing a
fiduciary obligation to make decisions in the best interest of these
savers, while at the same time recognizing the difficult circumstances
facing the Argentine government, including the challenges posed by
Covid-19,'' the statement said.

Image

Argentinian officials said that paying more to creditors would amount to
transferring wealth from people who had almost nothing to international
investors.Credit...Juan Ignacio Roncoroni/EPA, via Shutterstock

The standoff in Argentina reflects the complexity of debt negotiations
in an era in which regular people are effectively at the table. In
decades past, bonds issued by developing countries were overwhelmingly
controlled by major banks. When governments could not pay, bank chiefs
hammered out a deal. Today, investors holding emerging market bonds run
the gamut from specialized funds with high tolerance for risk to
conservative pension funds.

\hypertarget{latest-updates-economy}{%
\section{\texorpdfstring{\href{https://www.nytimes.com/live/2020/08/04/business/stock-market-today-coronavirus?action=click\&pgtype=Article\&state=default\&region=MAIN_CONTENT_1\&context=storylines_live_updates}{Latest
Updates:
Economy}}{Latest Updates: Economy}}\label{latest-updates-economy}}

\href{https://www.nytimes.com/live/2020/08/04/business/stock-market-today-coronavirus?action=click\&pgtype=Article\&state=default\&region=MAIN_CONTENT_1\&context=storylines_live_updates\#fox-corporations-plunging-profit-is-cushioned-by-fox-news}{6h
ago}

\href{https://www.nytimes.com/live/2020/08/04/business/stock-market-today-coronavirus?action=click\&pgtype=Article\&state=default\&region=MAIN_CONTENT_1\&context=storylines_live_updates\#fox-corporations-plunging-profit-is-cushioned-by-fox-news}{Fox
Corporation's plunging profit is cushioned by Fox News.}

\href{https://www.nytimes.com/live/2020/08/04/business/stock-market-today-coronavirus?action=click\&pgtype=Article\&state=default\&region=MAIN_CONTENT_1\&context=storylines_live_updates\#trading-in-kodak-shares-comes-under-scrutiny}{6h
ago}

\href{https://www.nytimes.com/live/2020/08/04/business/stock-market-today-coronavirus?action=click\&pgtype=Article\&state=default\&region=MAIN_CONTENT_1\&context=storylines_live_updates\#trading-in-kodak-shares-comes-under-scrutiny}{Trading
in Kodak shares comes under scrutiny.}

\href{https://www.nytimes.com/live/2020/08/04/business/stock-market-today-coronavirus?action=click\&pgtype=Article\&state=default\&region=MAIN_CONTENT_1\&context=storylines_live_updates\#disney-lost-4-7-billion-last-quarter-but-its-newest-business-was-a-big-hit}{7h
ago}

\href{https://www.nytimes.com/live/2020/08/04/business/stock-market-today-coronavirus?action=click\&pgtype=Article\&state=default\&region=MAIN_CONTENT_1\&context=storylines_live_updates\#disney-lost-4-7-billion-last-quarter-but-its-newest-business-was-a-big-hit}{Disney
lost \$4.7 billion last quarter, but its newest business was a big hit.}

\href{https://www.nytimes.com/live/2020/08/04/business/stock-market-today-coronavirus?action=click\&pgtype=Article\&state=default\&region=MAIN_CONTENT_1\&context=storylines_live_updates}{See
more updates}

More live coverage:
\href{https://www.nytimes.com/2020/08/04/world/coronavirus-cases.html?action=click\&pgtype=Article\&state=default\&region=MAIN_CONTENT_1\&context=storylines_live_updates}{Global}

That Mr. Fink's company is playing a primary role in pressuring
Argentina contrasts with his campaign to make business a force for
social progress.

Two years ago, Mr. Fink --- who has been mentioned in
\href{https://www.cnbc.com/2020/04/06/biden-donors-float-elizabeth-warren-larry-fink-others-for-key-roles.html}{news
reports} as a potential Treasury secretary in a Biden administration ---
wrote
\href{http://www.corporance.es/wp-content/uploads/2018/01/Larry-Fink-letter-to-CEOs-2018-1.pdf}{an
open letter} to the chief executives of major corporations urging them
to focus on social, labor and environmental concerns.

``To prosper over time, every company must not only deliver financial
performance, but also show how it makes a positive contribution to
society,'' he wrote.

Last year, Mr. Fink signed
the\href{https://www.nytimes.com/2019/08/19/business/business-roundtable-ceos-corporations.html}{Statement
on the Purpose of a Corporation} crafted by the Business Roundtable, an
association of American chief executives. It pledged ``a fundamental
commitment to all of our stakeholders.''

In January, Mr. Fink wrote
another\href{https://www.blackrock.com/corporate/investor-relations/larry-fink-ceo-letter}{letter
to C.E.O.s} warning that companies that fail to address climate change
would be punished in the marketplace.

BlackRock has launched funds tailored to so-called impact investing,
with money directed at advancing social and environmental goals.

Image

The International Monetary Fund's managing director, Kristalina
Georgieva, with Argentina's economy minister, Martín Guzmán, in
February. She has supported Argentina's proposal to its
creditors.Credit...Remo Casilli/Reuters

Argentina is now consumed with stemming an alarming increase in poverty.
Once among the richest countries on earth, it has defaulted on its
government debt nine times.

Argentina's history has been dominated by populist governments that have
won political favor by dispensing subsidies and cash to the masses in
brazen disregard for budget arithmetic, yielding chronic inflation and
frequent crises.

The last government, headed by President Mauricio Macri, assumed power
in 2015 with a mandate to restore discipline toward regaining the
confidence of international markets, while also showing compassion to
the poor through social spending.

Among those impressed was Mr. Fink. Six months after Mr. Macri took
office, the BlackRock chief
\href{https://www.youtube.com/watch?v=TM_MC2Fj-JI}{said} his
administration ``has really shown what a government can do if it is
focusing on trying to change the future of its country.''

In the end, Mr. Macri acquired a reputation for
\href{https://www.nytimes.com/2019/05/10/business/argentina-economy-macri-populism.html}{muddling
through}, failing to produce growth while borrowing anew.

When a new president, Alberto Fernández, took office last year, many
assumed that populism was back. But Mr. Fernández quickly reassured the
I.M.F. and key creditors that he was a pragmatist intent on securing a
workable debt settlement.

The I.M.F. had long been accused of wielding a single blunt instrument
in the face of crisis --- austerity. Its rescue package in Argentina two
decades ago imposed crippling cuts to government programs, sowing
enduring bitterness. Ms. Georgieva, the fund's managing director, has
sharpened a focus on protecting countries from impossible debt burdens.

Image

A demonstration in Buenos Aires against the economic crisis. Argentina's
history has been dominated by populist governments that have dispensed
subsidies without regard for budget arithmetic.Credit...Juan Ignacio
Roncoroni/EPA, via Shutterstock

BlackRock is part of a consortium called the Ad Hoc Argentine Bondholder
Group, which controls about one-fourth of the bonds.

The Ad Hoc group has struck a unified front in rejecting the
government's latest offer, which would pay out 53 cents on the dollar
value of the bonds. Last week, it presented its own proposal seeking
improved terms --- more than 56 cents on the dollar.

In a letter sent Monday to Argentina's economy minister, Martín Guzmán,
the group said it had gained the support of a majority of all
bondholders, giving it the power to block the deal. Under the bond
covenants, an agreement to write down their value must win the support
of the holders of two-thirds of their value.

In a statement, the Ad Hoc group said it was operating in the interest
of the Argentine public by seeking a deal that would ``allow re-access
to capital markets and encourage further investment.''

But some creditors have publicly supported the government's proposal.

``Argentina has made a reasonable offer, which I believe the creditors
should accept, especially in light of the health and poverty situation
in the country,'' said Mohamed A. El-Erian, chief economic adviser at
Allianz SE, the parent company of Pacific Investment Management Company,
one of the world's largest bond managers. He has been advising a
creditor at the table, Gramercy Funds Management LLC, an emerging
markets specialist and serves as its chairman.

Gramercy has concluded that differences between the government's offer
and the Ad Hoc group's proposal are trivial compared with the risk of a
comprehensive default that would diminish the value of Argentine bonds,
subject creditors to years of potential litigation and intensify the
nation's crisis.

Image

After Alberto Fernández became Argentina's president last year, he
quickly reassured key creditors that he was a pragmatist intent on
securing a workable debt settlement.Credit...Esteban Collazo, via Agence
France-Presse --- Getty Images

Additional debt forgiveness also enhances the likelihood that Argentina
can manage its future payments, lifting the value of outstanding bonds,
and lowering borrowing costs for Argentine companies.

``For three points you're willing to lose 20 or 30,'' said Mr. Humes,
the Greylock president. ``It's just insanity. It's unfortunate when egos
and inexperience get in the way of a pragmatic solution.''

Some say the government overplayed its hand, antagonizing creditors with
an unreasonably low opening offer --- less than 40 cents on the dollar.

``Guzman started off with a very lowball offer,'' said Siobhan Morden, a
Latin America bond analyst at Amherst Pierpont Securities, an
independent broker. ``This has been an unnecessary distraction for
months that could have been avoided if the opening offer had been more
reasonable.''

Negotiations were conducted via Zoom, involving dozens of different
creditors. BlackRock's representatives clashed with Argentina's economy
minister, Mr. Guzmán, a 37-year-old economist who studied with Mr.
Stiglitz at Columbia University.

Image

Members of the Argentine Army served stew for residents of a Buenos
Aires shantytown. Soup kitchens are serving more people in the
pandemic.Credit...Juan Mabromata/Agence France-Presse --- Getty Images

In May, Mr. Fink called Mr. Guzmán to try to break the impasse,
suggesting that a deal could be had if the government lifted its offer
to the range of 50 to 55 cents on the dollar, the people familiar with
the talks said.

In private consultations with BlackRock, the government offered 50
cents. But BlackRock and its Ad Hoc group held out for more.

Mr. Fink complained that it was unfair that private creditors were
swallowing all the losses, arguing that the I.M.F. should forgive some
of its loans --- a non-starter.

In early July, Mr.
Guzmán\href{https://www.reuters.com/article/argentina-debt-proposal/argentina-unveils-sweetened-debt-offer-to-creditors-sets-aug-4-deadline-idUSE6N2BD07N}{sweetened
the terms}, offering 53 cents on the dollar. That won the support of
several creditors, including Gramercy and Greylock.

By then, the pandemic was deepening Argentina's recession just as the
government required extra funds for the public health emergency. But
BlackRock began a behind-the-scenes campaign to block the deal.

The government has insisted that its offer is final. With child poverty
exceeding 50 percent, officials say, paying more to creditors would
amount to transferring wealth from people who have almost nothing to
international investors.

On a recent morning, about 100 families showed up at a soup kitchen 25
miles west of Buenos Aires --- more than twice as many as in March.
Among them was Ángel Ariel Coronel, a plumber who lives nearby with his
wife and their 2-year-old son. A strict lockdown imposed by the
government has halted the construction projects where he has worked.

``My wife was a bit embarrassed about having to come here,'' said Mr.
Coronel as he waited for a portion of steaming lentils. ``But I don't
care. We need the help. I haven't worked a day since this whole thing
started.''

Image

Buenos Aires has been under a lockdown since March.Credit...Natacha
Pisarenko/Associated Press

Peter S. Goodman reported from London and Daniel Politi from Buenos
Aires.

Advertisement

\protect\hyperlink{after-bottom}{Continue reading the main story}

\hypertarget{site-index}{%
\subsection{Site Index}\label{site-index}}

\hypertarget{site-information-navigation}{%
\subsection{Site Information
Navigation}\label{site-information-navigation}}

\begin{itemize}
\tightlist
\item
  \href{https://help.nytimes.com/hc/en-us/articles/115014792127-Copyright-notice}{©~2020~The
  New York Times Company}
\end{itemize}

\begin{itemize}
\tightlist
\item
  \href{https://www.nytco.com/}{NYTCo}
\item
  \href{https://help.nytimes.com/hc/en-us/articles/115015385887-Contact-Us}{Contact
  Us}
\item
  \href{https://www.nytco.com/careers/}{Work with us}
\item
  \href{https://nytmediakit.com/}{Advertise}
\item
  \href{http://www.tbrandstudio.com/}{T Brand Studio}
\item
  \href{https://www.nytimes.com/privacy/cookie-policy\#how-do-i-manage-trackers}{Your
  Ad Choices}
\item
  \href{https://www.nytimes.com/privacy}{Privacy}
\item
  \href{https://help.nytimes.com/hc/en-us/articles/115014893428-Terms-of-service}{Terms
  of Service}
\item
  \href{https://help.nytimes.com/hc/en-us/articles/115014893968-Terms-of-sale}{Terms
  of Sale}
\item
  \href{https://spiderbites.nytimes.com}{Site Map}
\item
  \href{https://help.nytimes.com/hc/en-us}{Help}
\item
  \href{https://www.nytimes.com/subscription?campaignId=37WXW}{Subscriptions}
\end{itemize}
