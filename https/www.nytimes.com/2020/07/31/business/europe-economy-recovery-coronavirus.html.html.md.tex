Sections

SEARCH

\protect\hyperlink{site-content}{Skip to
content}\protect\hyperlink{site-index}{Skip to site index}

\href{https://www.nytimes.com/section/business}{Business}

\href{https://myaccount.nytimes.com/auth/login?response_type=cookie\&client_id=vi}{}

\href{https://www.nytimes.com/section/todayspaper}{Today's Paper}

\href{/section/business}{Business}\textbar{}Despite Historic Plunge,
Europe's Economy Flashes Signs of Recovery

\url{https://nyti.ms/3jYA9H0}

\begin{itemize}
\item
\item
\item
\item
\item
\end{itemize}

\href{https://www.nytimes.com/news-event/coronavirus?action=click\&pgtype=Article\&state=default\&region=TOP_BANNER\&context=storylines_menu}{The
Coronavirus Outbreak}

\begin{itemize}
\tightlist
\item
  live\href{https://www.nytimes.com/2020/08/01/world/coronavirus-covid-19.html?action=click\&pgtype=Article\&state=default\&region=TOP_BANNER\&context=storylines_menu}{Latest
  Updates}
\item
  \href{https://www.nytimes.com/interactive/2020/us/coronavirus-us-cases.html?action=click\&pgtype=Article\&state=default\&region=TOP_BANNER\&context=storylines_menu}{Maps
  and Cases}
\item
  \href{https://www.nytimes.com/interactive/2020/science/coronavirus-vaccine-tracker.html?action=click\&pgtype=Article\&state=default\&region=TOP_BANNER\&context=storylines_menu}{Vaccine
  Tracker}
\item
  \href{https://www.nytimes.com/interactive/2020/07/29/us/schools-reopening-coronavirus.html?action=click\&pgtype=Article\&state=default\&region=TOP_BANNER\&context=storylines_menu}{What
  School May Look Like}
\item
  \href{https://www.nytimes.com/live/2020/07/31/business/stock-market-today-coronavirus?action=click\&pgtype=Article\&state=default\&region=TOP_BANNER\&context=storylines_menu}{Economy}
\end{itemize}

Advertisement

\protect\hyperlink{after-top}{Continue reading the main story}

Supported by

\protect\hyperlink{after-sponsor}{Continue reading the main story}

\hypertarget{despite-historic-plunge-europes-economy-flashes-signs-of-recovery}{%
\section{Despite Historic Plunge, Europe's Economy Flashes Signs of
Recovery}\label{despite-historic-plunge-europes-economy-flashes-signs-of-recovery}}

European countries that have better contained the virus are poised for
speedier economic recovery than the United States.

\includegraphics{https://static01.nyt.com/images/2020/07/31/business/31EU-ECON-1/31EU-ECON-1-articleLarge.jpg?quality=75\&auto=webp\&disable=upscale}

\href{https://www.nytimes.com/by/peter-s-goodman}{\includegraphics{https://static01.nyt.com/images/2018/02/16/multimedia/author-peter-s-goodman/author-peter-s-goodman-thumbLarge-v2.png}}\href{https://www.nytimes.com/by/liz-alderman}{\includegraphics{https://static01.nyt.com/images/2019/08/08/business/author-liz-alderman-alt/author-liz-alderman-alt-thumbLarge.png}}\href{https://www.nytimes.com/by/jack-ewing}{\includegraphics{https://static01.nyt.com/images/2018/07/18/multimedia/author-jack-ewing/author-jack-ewing-thumbLarge.png}}

By \href{https://www.nytimes.com/by/peter-s-goodman}{Peter S. Goodman},
\href{https://www.nytimes.com/by/liz-alderman}{Liz Alderman} and
\href{https://www.nytimes.com/by/jack-ewing}{Jack Ewing}

\begin{itemize}
\item
  July 31, 2020
\item
  \begin{itemize}
  \item
  \item
  \item
  \item
  \item
  \end{itemize}
\end{itemize}

LONDON --- Before the pandemic, a traditional state of play prevailed in
the enormous economies on the opposite sides of the Atlantic. Europe ---
full of older people, and rife with bickering over policy --- appeared
stagnant. The United States, ruled by innovation and risk-taking, seemed
set to grow faster.

But that alignment has been reordered by contrasting approaches to a
terrifying global crisis. Europe has generally gotten a handle on the
spread of the coronavirus, enabling many
\href{https://www.nytimes.com/2020/07/14/business/europe-consumer-spending.html}{economies
to reopen} while
\href{https://www.nytimes.com/2020/07/03/business/economy/europe-us-jobless-coronavirus.html}{protecting
workers} whose livelihoods have been menaced. The United States has
become a symbol of fecklessness and discord in the face of a grave
emergency, yielding deepening worries about the fate of jobs and
sustenance.

On Friday, Europe released economic numbers that on their face were
terrible. The 19 nations that share the euro currency
\href{https://www.nytimes.com/live/2020/07/31/business/stock-market-today-coronavirus?action=click\&module=Top\%20Stories\&pgtype=Homepage}{contracted
by 12.1 percent} from April to June from the previous quarter --- the
sharpest decline since 1995, when the data was first collected. Spain
fell by a staggering 18.5 percent, and France, one of the eurozone's
largest economies, declined 13.8 percent. Italy shrunk by 12.4 percent.

Eurozone G.D.P.

+2\%

0

-2

-4

-6

-8

-10

--12.1\%

Percentage change from previous quarter

-12

2008

2010

2012

2014

2016

2018

2020

Eurozone G.D.P.

+2\%

0

-2

-4

-6

-8

-10

-12

--12.1\%

Percentage change from previous quarter

-14

2008

2010

2012

2014

2016

2018

2020

Note: Adjusted for inflation and seasonality.

Source: Eurostat

By The New York Times

Europe appeared even worse than the United States, which the day before
\href{https://www.nytimes.com/2020/07/30/business/economy/q2-gdp-coronavirus-economy.html?action=click\&module=Top\%20Stories\&pgtype=Homepage}{recorded
the single-worst} three-month stretch in its history, tumbling by 9.5
percent in the second quarter.

But beneath the headline figures, Europe flashed promising signs of
strength.

Germany saw a drop in the numbers of unemployed, surveys found evidence
of growing confidence amid an expansion in factory production, while the
euro continued to strengthen against the dollar as
\href{https://www.nytimes.com/2020/07/30/business/europes-markets-are-having-a-moment.html}{investment
flowed into European markets} --- signs of improving sentiment.

These contrasting fortunes underscored a central truth of a pandemic
that has killed more than 670,000 people worldwide: The most significant
cause of the economic pain is the virus itself. Governments that have
more adeptly controlled its spread have commanded greater confidence
from their citizens and investors, putting their economies in better
position to recuperate from the worst global downturn since the Great
Depression.

``There is no economic recovery without a controlled health situation,''
said Ángel Talavera, lead eurozone economist at Oxford Economics in
London. ``It's not a choice between the two.''

\includegraphics{https://static01.nyt.com/images/2020/07/31/business/31EU-ECON2/merlin_175146681_a203d848-e364-4e08-a9e3-d22cc3a647d6-articleLarge.jpg?quality=75\&auto=webp\&disable=upscale}

European confidence has been bolstered by a
\href{https://www.nytimes.com/2020/07/20/world/europe/eu-stimulus-coronavirus.html}{groundbreaking
agreement} struck in July within the European Union to sell 750 billion
euro (\$892 billion) worth of bonds that are backed collectively by its
members. Those funds will be deployed to the hardest hit countries like
Italy and Spain.

The deal transcended years of opposition from parsimonious northern
European countries like Germany and the Netherlands against issuing
common debt. They have balked at putting their taxpayers on the line to
bail out southern neighbors like Greece while indulging in crude
stereotypes of Mediterranean profligacy. The animosity perpetuated the
sense that Europe was a union in name only --- a critique that has been
muted.

\hypertarget{latest-updates-economy}{%
\section{\texorpdfstring{\href{https://www.nytimes.com/live/2020/07/31/business/stock-market-today-coronavirus?action=click\&pgtype=Article\&state=default\&region=MAIN_CONTENT_1\&context=storylines_live_updates}{Latest
Updates:
Economy}}{Latest Updates: Economy}}\label{latest-updates-economy}}

\href{https://www.nytimes.com/live/2020/07/31/business/stock-market-today-coronavirus?action=click\&pgtype=Article\&state=default\&region=MAIN_CONTENT_1\&context=storylines_live_updates\#kodaks-chief-executive-was-given-stock-options-then-the-share-price-spiked-1000-percent}{30h
ago}

\href{https://www.nytimes.com/live/2020/07/31/business/stock-market-today-coronavirus?action=click\&pgtype=Article\&state=default\&region=MAIN_CONTENT_1\&context=storylines_live_updates\#kodaks-chief-executive-was-given-stock-options-then-the-share-price-spiked-1000-percent}{Kodak's
chief executive was given stock options. Then the share price spiked
1,000 percent.}

\href{https://www.nytimes.com/live/2020/07/31/business/stock-market-today-coronavirus?action=click\&pgtype=Article\&state=default\&region=MAIN_CONTENT_1\&context=storylines_live_updates\#fitch-ratings-downgrades-its-outlook-on-us-debt}{33h
ago}

\href{https://www.nytimes.com/live/2020/07/31/business/stock-market-today-coronavirus?action=click\&pgtype=Article\&state=default\&region=MAIN_CONTENT_1\&context=storylines_live_updates\#fitch-ratings-downgrades-its-outlook-on-us-debt}{Fitch
Ratings downgrades its outlook on U.S. debt.}

\href{https://www.nytimes.com/live/2020/07/31/business/stock-market-today-coronavirus?action=click\&pgtype=Article\&state=default\&region=MAIN_CONTENT_1\&context=storylines_live_updates\#us-sanctions-more-chinese-officials-over-human-rights-violations-as-tensions-flare}{40h
ago}

\href{https://www.nytimes.com/live/2020/07/31/business/stock-market-today-coronavirus?action=click\&pgtype=Article\&state=default\&region=MAIN_CONTENT_1\&context=storylines_live_updates\#us-sanctions-more-chinese-officials-over-human-rights-violations-as-tensions-flare}{U.S.
sanctions more Chinese officials over human rights violations as
tensions flare}

\href{https://www.nytimes.com/live/2020/07/31/business/stock-market-today-coronavirus?action=click\&pgtype=Article\&state=default\&region=MAIN_CONTENT_1\&context=storylines_live_updates}{See
more updates}

More live coverage:
\href{https://www.nytimes.com/2020/08/01/world/coronavirus-covid-19.html?action=click\&pgtype=Article\&state=default\&region=MAIN_CONTENT_1\&context=storylines_live_updates}{Global}

\href{https://www.bruegel.org/publications/datasets/covid-national-dataset/}{The
United States has spent more than Europe} on programs to limit the
economic damage of the pandemic. But much of the spending has benefited
investors, spurring a
\href{https://www.nytimes.com/2020/06/08/business/recession-stock-market-coronavirus.html}{substantial
recovery in the stock market}.
\href{https://www.nytimes.com/2020/07/29/business/economy/unemployment-benefits-coronavirus.html}{Emergency
unemployment benefits} have proved crucial, enabling tens of millions of
jobless Americans to pay rent and buy groceries. But they were set to
expire on Friday and there were few signs that
\href{https://www.nytimes.com/2020/07/30/us/politics/senate-virus-aid.html}{Congress
would extend them}.

Europe's experience has underscored the virtues of its more generous
social welfare programs, including national health care systems.

Americans feel compelled to go to work, even at dangerous places like
\href{https://www.nytimes.com/2020/05/10/business/economy/coronavirus-tyson-plant-iowa.html}{meatpacking
plants}, and even when they are ill, because many lack paid sick leave.
Yet they also feel pressure to avoid shops, restaurants and other
crowded places of business because millions lack health insurance,
making hospitalization a financial catastrophe.

Image

People waiting for assistance filing unemployment insurance claims in
Tulsa, Okla.~Emergency unemployment benefits were set to expire on
Friday.Credit...Joseph Rushmore for The New York Times

``Europe has really benefited from having this system that is more
heavily dominated by welfare systems than the U.S.,'' said Kjersti
Haugland, chief economist at DNB Markets, an investment bank in Oslo.
``It keeps people less fearful.''

The more promising situation in Europe is neither certain nor
comprehensive.
\href{https://www.nytimes.com/2020/07/23/world/europe/spain-coronavirus-reopening.html}{Spain
remains a grave concern}, with the virus spreading, threatening lives
and livelihoods. Italy has emerged from the grim calculus of mass death
to the chronic condition of persistent economic troubles.
\href{https://www.nytimes.com/2020/07/30/world/europe/UK-deaths-coronavirus-europe.html}{Britain's
tragic mishandling} of the pandemic has
\href{https://www.reuters.com/article/us-health-coronavirus-poll/uk-leads-fall-in-global-trust-in-government-covid-responses-poll-idUSKBN23B0H4}{shaken
faith} in the government.

If short-term factors look more beneficial to European economies,
longer-term forces may favor the United States, with its younger
population and greater productivity.

A sense of European-American rivalry has been provoked by the bombast of
a nationalist American president, making the pandemic a morbid
opportunity to keep score.

``There is a certain amount of triumphalism,'' said Peter Dixon, a
global financial economist at Commerzbank in London. ``People are
saying, `Our economy has survived, we are doing OK.' There's a certain
amount of European \emph{schadenfreude}, if I can use that word, given
everything that Trump has said about the U.S.''

Image

People waited in line for social coupons for food in Barcelona in
March.~Spain's economy fell by a staggering 18.5 percent in the
April-to-June quarter.Credit...Samuel Aranda for The New York Times

But for now, Europe's moment of confidence is palpable, most prominently
in Germany, the continent's largest economy.

Though the
\href{https://www.nytimes.com/2020/07/30/business/the-german-economy-had-its-biggest-slump-in-50-years.html}{German
economy} shrank by 10.1 percent from March to June --- its worst drop in
at least half a century --- the number of officially jobless people fell
in July, in part because of government programs that have subsidized
furloughed workers.

\href{https://www.nytimes.com/2020/07/27/business/the-german-economy-shows-signs-of-rebounding.html}{Surveys}
show that German managers --- not a group inclined toward sunny optimism
--- have seen expectations for future sales return to nearly pre-virus
levels. That buoyancy translates directly into growth, emboldening
companies to rehire furloughed workers.

Ziehl-Abegg, a maker of ventilation systems for hospitals, factories and
large buildings, recently broke ground on a 16 million euro (\$19
million) expansion at a factory in southern Germany.

``If we wait to invest until the market recovers, that's too late,''
said Peter Fenkl, the company's chief executive. ``There are billions of
dollars in the market ready to be invested and just waiting for the
signal to kick off.''

The euro has gained more than 5 percent against the dollar so far this
year, according to FactSet.
\href{https://www.nytimes.com/2020/07/30/business/europes-markets-are-having-a-moment.html}{European
markets have been lifted} by international money flowing into so-called
exchange-traded funds that purchase European stocks. The Stoxx 600, an
index made up of companies in 17 European countries, appears set for a
second straight month of gains outpacing the S\&P 500.

Image

A shoe store in Berlin. Surveys have found evidence of growing
confidence amid an expansion in factory production.Credit...Lena Mucha
for The New York Times

The French oil giant Total saw demand for its products in Europe drop by
nearly one third in the second quarter of the year, but a powerful
recovery has been gaining momentum, said the company's chairman and
chief executive, Patrick Pouyanné.

``Since June, we have seen a rebound here in Europe,'' he said during a
call with analysts. ``Activity in our marketing networks is back to, I
would say, 90 percent of the pre-Covid levels.''

France, Europe's second largest economy, has been buttressed by
aggressive government spending. President Emmanuel Macron has mobilized
more than 400 billion euros (\$476 billion) in emergency aid and loan
guarantees since the start of the crisis, and is preparing an autumn
package worth another 100 billion euros.

Those funds paid businesses not to lay off workers, allowing more than
14 million employees to go on paid furlough, stay in their homes,
accumulate modest savings and continue spending. Delayed deadlines for
business taxes and loan payments spared companies from collapse.

In the second quarter, when France was still partially locked down, the
country's economy contracted by nearly 14 percent. Tourism, retail and
manufacturing, the main pillars of the economy, ground to a halt.

But **** services, industrial activity and consumer spending have all
shown signs of improvement. The Banque de France, which originally
expected the economy to shrink more than 10 percent this year, recently
forecast less damage.

Image

Parisians enjoying lunch and the outdoors in late July.~France's economy
has been buttressed by aggressive government
spending.Credit...Christophe Archambault/Agence France-Presse --- Getty
Images

In Spain, a sense of recovery remains distant. Its economy shrunk by
nearly 19 percent from April to June. The nation's unemployment rate
exceeds 15 percent, and could surge higher if a wage subsidy program for
furloughed workers is allowed to expire in September.

Spain officially ended its coronavirus state of emergency on June 21,
but has since suffered an increase in infections. The economic impacts
have been compounded by Britain's decision to force travelers returning
from Spain to quarantine for two weeks. Tourism accounts for 12 percent
of Spain's economy.

Italy is also highly exposed to tourism. Its industry is concentrated in
the north of the country, which saw the worst of coronavirus. The
central bank expects the Italian economy to contract by nearly 10
percent this year.

But exports surged more than one-third in May compared with the previous
month. That left them below pre-pandemic levels, yet on par with German
and American competitors, according to Confindustria, an Italian trade
association.

Image

A beach on Lampedusa, the largest of the Italian Pelagie Islands.~Italy
has faced persistent economic troubles. Credit...Alberto Pizzoli/Agence
France-Presse --- Getty Images

``We are starting to slowly recover after the most violent downfall in
the last 70 years,'' said Francesco Daveri, an economist at Bocconi
University in Milan.

Europe's fortunes appear on the mend because its people are more likely
to trust their governments.

Denmark acted early, imposing a strict lockdown while
paying\href{https://www.nytimes.com/2020/03/28/business/nordic-way-economic-rescue-virus.html}{wage
subsidies} that limited unemployment. Denmark suffered far fewer deaths
per capita than the United States and Britain.

With the virus largely controlled, Denmark lifted restrictions earlier,
while Danes heeded the call to resume commercial life. The Danish
economy is expected to contract by 5.25 percent this year, according to
the
\href{https://ec.europa.eu/economy_finance/forecasts/2020/summer/ecfin_forecast_summer_2020_dk_en.pdf}{European
Commission}, with a substantial improvement in the second half of the
year.

In the United States, people have wearied of bewildering and conflicting
advice from on high against a backdrop of
\href{https://www.nytimes.com/2020/07/29/us/coronavirus-deaths-150000.html}{more
than 150,000 deaths}.

President Trump first called the virus a
\href{https://www.nytimes.com/2020/02/28/us/politics/trump-accuses-media-democrats-coronavirus.html}{hoax},
then treated it as an emergency befitting
\href{https://www.nytimes.com/2020/03/22/us/politics/coronavirus-trump-wartime-president.html}{wartime
mobilization}, then
\href{https://www.nytimes.com/2020/04/16/us/politics/coronavirus-trump-guidelines.html}{urged
states to reopen} to spur the economy. He
\href{https://www.nytimes.com/2020/04/17/us/politics/trump-coronavirus-governors.html}{encouraged}protesters
who portrayed wearing masks as an affront to civil liberties.

Image

A shopping center in Copenhagen, Denmark.~The country imposed a strict
lockdown early in the pandemic while paying wage subsidies that limited
unemployment. Credit...Ritzau Scanpix/via Reuters

The result has been
\href{https://www.nytimes.com/2020/07/25/world/coronavirus-covid-19.html}{record
surges of new cases} along with a syndrome likely to persist --- an
aversion to being near other people. That spells leaner prospects for
retail, hotels, restaurants and other job-rich areas of the American
economy.

Liz Alderman reported from Paris. Emma Bubola contributed reporting from
Milan, Raphael Minder from Madrid and Stanley Reed and Eshe Nelson from
London.

Advertisement

\protect\hyperlink{after-bottom}{Continue reading the main story}

\hypertarget{site-index}{%
\subsection{Site Index}\label{site-index}}

\hypertarget{site-information-navigation}{%
\subsection{Site Information
Navigation}\label{site-information-navigation}}

\begin{itemize}
\tightlist
\item
  \href{https://help.nytimes.com/hc/en-us/articles/115014792127-Copyright-notice}{©~2020~The
  New York Times Company}
\end{itemize}

\begin{itemize}
\tightlist
\item
  \href{https://www.nytco.com/}{NYTCo}
\item
  \href{https://help.nytimes.com/hc/en-us/articles/115015385887-Contact-Us}{Contact
  Us}
\item
  \href{https://www.nytco.com/careers/}{Work with us}
\item
  \href{https://nytmediakit.com/}{Advertise}
\item
  \href{http://www.tbrandstudio.com/}{T Brand Studio}
\item
  \href{https://www.nytimes.com/privacy/cookie-policy\#how-do-i-manage-trackers}{Your
  Ad Choices}
\item
  \href{https://www.nytimes.com/privacy}{Privacy}
\item
  \href{https://help.nytimes.com/hc/en-us/articles/115014893428-Terms-of-service}{Terms
  of Service}
\item
  \href{https://help.nytimes.com/hc/en-us/articles/115014893968-Terms-of-sale}{Terms
  of Sale}
\item
  \href{https://spiderbites.nytimes.com}{Site Map}
\item
  \href{https://help.nytimes.com/hc/en-us}{Help}
\item
  \href{https://www.nytimes.com/subscription?campaignId=37WXW}{Subscriptions}
\end{itemize}
