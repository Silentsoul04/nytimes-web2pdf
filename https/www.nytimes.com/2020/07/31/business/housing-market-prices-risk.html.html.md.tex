Sections

SEARCH

\protect\hyperlink{site-content}{Skip to
content}\protect\hyperlink{site-index}{Skip to site index}

\href{https://www.nytimes.com/section/business}{Business}

\href{https://myaccount.nytimes.com/auth/login?response_type=cookie\&client_id=vi}{}

\href{https://www.nytimes.com/section/todayspaper}{Today's Paper}

\href{/section/business}{Business}\textbar{}How to Navigate the
Coronavirus Real Estate Market

\url{https://nyti.ms/33aRQwS}

\begin{itemize}
\item
\item
\item
\item
\item
\end{itemize}

\begin{itemize}
\item
  \href{https://www.nytimes.com/interactive/2020/08/04/us/elections/results-arizona-kansas-michigan-missouri-primaries.html?action=click\&pgtype=Article\&state=default\&region=TOP_BANNER\&context=storylines_menu}{Latest
  Results}
\item
  \href{https://www.nytimes.com/article/biden-vice-president-2020.html?action=click\&pgtype=Article\&state=default\&region=TOP_BANNER\&context=storylines_menu}{Biden's
  V.P. Search}
\item
  \href{https://www.nytimes.com/interactive/2020/07/24/us/politics/trump-biden-campaign-donors.html?action=click\&pgtype=Article\&state=default\&region=TOP_BANNER\&context=storylines_menu}{Map
  of Donations}
\item
  \href{https://www.nytimes.com/interactive/2020/us/elections/delegate-count-primary-results.html?action=click\&pgtype=Article\&state=default\&region=TOP_BANNER\&context=storylines_menu}{Delegate
  Count}
\item
  \href{https://www.nytimes.com/interactive/2019/us/politics/2020-presidential-candidates.html?action=click\&pgtype=Article\&state=default\&region=TOP_BANNER\&context=storylines_menu}{The
  Candidates}
\item
  \href{https://www.nytimes.com/newsletters/politics?action=click\&pgtype=Article\&state=default\&region=TOP_BANNER\&context=storylines_menu}{Politics
  Newsletter}
\end{itemize}

Advertisement

\protect\hyperlink{after-top}{Continue reading the main story}

Supported by

\protect\hyperlink{after-sponsor}{Continue reading the main story}

Economic View

\hypertarget{how-to-navigate-the-coronavirus-real-estate-market}{%
\section{How to Navigate the Coronavirus Real Estate
Market}\label{how-to-navigate-the-coronavirus-real-estate-market}}

Suburbs and fashionable exurbs are hot, but don't forget that home
prices have fallen before, a Nobel laureate warns.

\includegraphics{https://static01.nyt.com/images/2020/08/02/business/31View-illo-sub/31View-illo-sub-articleLarge.jpg?quality=75\&auto=webp\&disable=upscale}

By Robert J. Shiller

\begin{itemize}
\item
  July 31, 2020
\item
  \begin{itemize}
  \item
  \item
  \item
  \item
  \item
  \end{itemize}
\end{itemize}

There are signs that pockets of the U.S. housing market are
\href{https://www.nar.realtor/newsroom/existing-home-sales-climb-record-20-7-in-june}{heating
up},
\href{https://www.realtor.com/research/housing-market-rankings-in-suburban-communities-outpaced-urban-areas-in-may/}{particularly
in the suburbs} and fashionable exurbs, to which people have been
fleeing
\href{https://www.nytimes.com/2020/05/08/realestate/coronavirus-escape-city-to-suburbs.html}{to
escape the coronavirus}.

Some first-time buyers are feeling a sudden hurry to buy, fearing higher
prices if they wait. But they are also worried about the long-run
outlook for home prices.

For the United States, according to the
\href{https://www.spglobal.com/spdji/en/indices/indicators/sp-corelogic-case-shiller-us-national-home-price-nsa-index/\#data}{S\&P/CoreLogic/Case-Shiller
National Home Price Index}, adjusted for inflation, real home prices
rose 45 percent from February 2012 through May 2020, the latest data. (I
helped to create the index but have no financial interest in it.) Other
sources indicate that prices remain high. That is a remarkable record,
considering that the United States is grappling with the coronavirus
pandemic, a major recession and social upheaval. In that stretch, there
were no down years.

It would be easy to assume that the boom times for housing will go on
forever, but that would require ignoring the disaster that led to the
most recent great financial crisis, a little more than a decade ago.

Recall some recent history. Real home prices rose 75 percent from
February 1997 to a peak in December 2005, apparently unaffected by the
2001 U.S. recession and the steep stock market decline of 2000 to 2002.
Delusions of eternal price increases for houses --- thought to be much
more reliable than stocks --- sprouted in that era, not so long ago.

\hypertarget{latest-updates-2020-election}{%
\section{\texorpdfstring{\href{https://www.nytimes.com/2020/08/04/us/elections/primary-election-michigan-arizona-kansas.html?action=click\&pgtype=Article\&state=default\&region=MAIN_CONTENT_1\&context=storylines_live_updates}{Latest
Updates: 2020
Election}}{Latest Updates: 2020 Election}}\label{latest-updates-2020-election}}

Updated 2020-08-05T03:23:56.561Z

\begin{itemize}
\tightlist
\item
  \href{https://www.nytimes.com/2020/08/04/us/elections/primary-election-michigan-arizona-kansas.html?action=click\&pgtype=Article\&state=default\&region=MAIN_CONTENT_1\&context=storylines_live_updates\#link-3924dd44}{Two
  G.O.P. Senate primaries offer --- what else? --- a test of loyalty to
  Trump.}
\item
  \href{https://www.nytimes.com/2020/08/04/us/elections/primary-election-michigan-arizona-kansas.html?action=click\&pgtype=Article\&state=default\&region=MAIN_CONTENT_1\&context=storylines_live_updates\#link-62a8e06b}{The
  military-style uniforms of federal agents who responded to the unrest
  in Portland will be replaced.}
\item
  \href{https://www.nytimes.com/2020/08/04/us/elections/primary-election-michigan-arizona-kansas.html?action=click\&pgtype=Article\&state=default\&region=MAIN_CONTENT_1\&context=storylines_live_updates\#link-32b39e33}{President
  Trump is suddenly a big supporter of mail-in voting --- in Florida.}
\end{itemize}

\href{https://www.nytimes.com/2020/08/04/us/elections/primary-election-michigan-arizona-kansas.html?action=click\&pgtype=Article\&state=default\&region=MAIN_CONTENT_1\&context=storylines_live_updates}{See
more updates}

But housing prices crashed 36 percent from their 2005 high, to a low in
February 2012, and the impact of that decline spread into other
financial markets and crippled the global economy.

For a home buyer in 2005, who put up life savings for a 10 percent or 20
percent down payment, the price decline amounted to a devastating loss
by 2012. If they had been able to hold on until now, their real home
value would probably be mostly restored, but no one would want that
15-year experience again.

\begin{center}\rule{0.5\linewidth}{\linethickness}\end{center}

I'm not making a prediction. This is not 2005, and many things have
changed. The work-at-home trend today, aided by online communications,
is notably different. If one does not have to commute to work in a city,
there is so much land out there in America that many new suburban houses
can be built, supply can increase to meet the demand, and home prices in
the suburbs may never rise as much as they did in the previous boom.

But still, looking at the market cycle has to be instructive. In an
impressive new book,
``\href{https://www.hup.harvard.edu/catalog.php?isbn=9780674979659}{The
Great American Housing Bubble}\emph{,''}
\href{https://www.law.georgetown.edu/faculty/adam-j-levitin/}{Adam
Levitin of Georgetown University} and
\href{https://real-estate.wharton.upenn.edu/profile/wachter/}{Susan
Wachter of the Wharton School} summarized six possible causes of that
epic boom-and-bust cycle. Succinctly put, they are:

\begin{itemize}
\tightlist
\item
  Consumers' ``irrational exuberance,'' referring to an analysis that I
  made in the second edition of a book with that title in 2005.
\end{itemize}

\begin{itemize}
\tightlist
\item
  The ``fair lending and affordable housing policy'' starting with the
  1977 Community Reinvestment Act, which made it easier for poorer
  people to buy houses.
\end{itemize}

\begin{itemize}
\tightlist
\item
  Federal Reserve cuts in interest rates, which may have set off price
  speculation.
\end{itemize}

\begin{itemize}
\tightlist
\item
  A global savings glut --- excessive saving worldwide, given available
  investment opportunities, a theory proposed by Ben S. Bernanke, the
  former Fed chairman, in explanation of low interest rates in the early
  2000s.
\end{itemize}

\begin{itemize}
\tightlist
\item
  Excessive creation of securities that promoted subprime lending.
\end{itemize}

\begin{itemize}
\tightlist
\item
  A shift in mortgage lending to ``unregulated private-label
  securitization by private investment banks,'' a theory developed by
  the two authors.
\end{itemize}

All these factors, as well
as\href{https://www.bankrate.com/mortgages/30-year-mortgage-rates/?pointsChanged=false\&searchChanged=true\&mortgageType=Purchase\&zipCode=11968\&partnerId=br3\&ttcid\&userCreditScore=740\&userVeteranStatus=NoMilitaryService\&userHadPriorVaLoan=false\&userHasVaDisabilities=false\&userFirstTimeHomebuyer=false\&userQuickClosing=false\&userFha=false\&userLowUpfrontCosts=false\&userLowPayment=false\&purchasePrice=580000\&purchaseDownPayment=116000\&purchasePropertyType=SingleFamily\&purchasePropertyUse=PrimaryResidence\&purchaseLoanTerms=30yr\&purchasePoints=All\&refinancePropertyValue=580000\&refinanceLoanAmount=464000\&refinancePropertyType=SingleFamily\&refinancePropertyUse=PrimaryResidence\&refinanceCashOutAmount=0\&refinancePoints=All\&refinanceLoanTerms=30yr}{Federal
Reserve decisions affecting mortgage rates}, are part of the story of
the 1997 to 2012 boom and crash. So are the difficulties faced by the
Fed and other regulators, as described in a new and imposing 595-page
volume,
``\href{https://yalebooks.yale.edu/book/9780300244441/first-responders}{First
Responders},\emph{''} edited by Mr. Bernanke and two former U.S.
treasury secretaries, Timothy Geithner and Henry Paulson.

All of the theories point to a fragile boom-time mind-set that
underestimated home price risk, whether by home buyers, investors,
mortgage originators, securitizers, rating agencies or regulators.

So let us dig a little deeper. What caused all these errors back then?

Ultimately, it came down to unwarranted optimism and excitement about
home prices. There were, during the 1997-2005 boom, constellations of
narratives about housing that grew contagious over time, even
transcending national borders. Intense ``real estate voyeurism'' ---
envious online snooping of other peoples' home values --- became common.
The exuberant mind-set displaced thoughts of price declines.

Stories abounded of ``flippers,'' people who made fantastic profits
buying, fixing up, and selling homes within a matter of months. The
so-called experts in those days hardly ever mentioned that the high rate
of increase in home prices might one day be
\href{https://www.attomdata.com/news/market-trends/flipping/attom-data-solutions-q2-2019-u-s-home-flipping-report/}{reversed}.

In retrospect, it appears that there was a political component to the
housing craze. President George W. Bush said the United States was
becoming an ``ownership society'' in his successful 2004 re-election
\href{https://www.nytimes.com/2003/02/23/weekinreview/the-nation-focus-groups-to-bush-the-crowd-was-a-blur.html?searchResultPosition=1}{campaign}.
He promoted the idea of homeownership in a way that flattered the
apparent wisdom of people who bought houses.

Newspaper articles shortly after Mr. Bush won became much more
comfortable with the idea that something akin to an ``ownership
society'' was the country's future, part of a longer trend that defined
the ``American dream'' as owning a home. In that atmosphere, people
rarely even considered the possibility that home prices could ever fall.

Starting just before the 2005 peak, however, the news media started
\href{https://www.chicagotribune.com/news/ct-xpm-2004-09-19-0409190429-story.html}{discussing}
a new idea, the existence of a ``housing bubble'' for single-family
homes, whose prices had become obviously high. Before that, there just
wasn't much talk about the idea that a bubble could be forming in the
market for single-family homes. That sudden change is worth remembering.
It is a model for what might happen again one day.

That's not where the United States is now. The prevailing narratives are
different and the underlying economic situation is dominated by the
coronavirus pandemic. Furthermore, there tends to be a lot of momentum
in home prices. The boom that started in 2012 could conceivably go on
for years.

But there is a chance that inadequate public support for homeowners in
these difficult times will result in a rash of foreclosures, personal
bankruptcies and houses dumped on the market. From the intensity of
public reactions to current social and economic problems, it seems
possible that the United States is nearing a turning point in the
thinking of many people, and this could affect the housing market and
the economy. If President Trump and his ``Make America Great Again''
narrative continue to polarize the country as the election grows near,
confidence in the solidity of housing prices could drop sharply.

This poses a dilemma for prospective home buyers, who must make
life-changing decisions with imperfect knowledge of the future. History
suggests that it might be wise to avoid investing in too expensive of a
house or in taking on too much risk.

Yet the value of a good place to live for a family cannot be quantified.
If you can afford the cost, a house that you will live in for years to
come may be worthwhile, regardless of the short-term shifts in the
market.

Robert J. Shiller, Sterling Professor of Economics at Yale, shared the
Nobel Memorial Prize in Economic Science in
\href{https://www.nytimes.com/2013/10/15/business/3-american-professors-awarded-nobel-in-economic-sciences.html?_r=0}{2013}.

\begin{center}\rule{0.5\linewidth}{\linethickness}\end{center}

\hypertarget{our-2020-election-guide}{%
\section{Our 2020 Election Guide}\label{our-2020-election-guide}}

Updated Aug. 4, 2020

\begin{itemize}
\item
  \begin{center}\rule{0.5\linewidth}{\linethickness}\end{center}

  \hypertarget{the-latest}{%
  \subsection{The Latest}\label{the-latest}}

  \begin{itemize}
  \tightlist
  \item
    Kris Kobach, a polarizing figure in Kansas politics,
    \href{https://www.nytimes.com/2020/08/04/us/politics/kobach-tlaib.html?action=click\&pgtype=Article\&state=default\&region=BELOW_MAIN_CONTENT\&context=storylines_guide}{lost
    the Senate primary there}, relieving G.O.P. officials who feared he
    could jeopardize a safe seat.
  \end{itemize}
\item
  \begin{center}\rule{0.5\linewidth}{\linethickness}\end{center}

  \hypertarget{bidens-vp-search}{%
  \subsection{Biden's V.P. Search}\label{bidens-vp-search}}

  \begin{itemize}
  \tightlist
  \item
    \href{https://www.nytimes.com/article/biden-vice-president-2020.html?action=click\&pgtype=Article\&state=default\&region=BELOW_MAIN_CONTENT\&context=storylines_guide}{Here
    are 13 women} who have been under consideration to be Joe Biden's
    running mate, and why each might be chosen --- and might not be.
  \end{itemize}
\item
  \begin{center}\rule{0.5\linewidth}{\linethickness}\end{center}

  \hypertarget{keep-up-with-our-coverage}{%
  \subsection{Keep Up With Our
  Coverage}\label{keep-up-with-our-coverage}}

  \begin{itemize}
  \tightlist
  \item
    Get an
    \href{https://www.nytimes.com/newsletters/politics?action=click\&pgtype=Article\&state=default\&region=BELOW_MAIN_CONTENT\&context=storylines_guide}{email}
    recapping the day's news
  \end{itemize}

  \begin{itemize}
  \tightlist
  \item
    Download our mobile app on
    \href{https://apps.apple.com/us/app/nytimes/id284862083?ls=1\&mat_click_id=5c79ae7455014fd1bd66b5610c05b8f2-20191112-16948\&referrer=mat_click_id\%3D5c79ae7455014fd1bd66b5610c05b8f2-20191112-16948\%26link_click_id\%3D722930677036718082}{iOS}
    and
    \href{http://a.localytics.com/android?id=com.nytimes.android\&referrer=utm_source\%3Dother_nyt_mobile_web\%26utm_medium\%3DWeb\%2520page\%26utm_term\%3DGeneral\%2520Mobile\%2520Page\%26utm_campaign\%3DNYT\%2520Mobile\%2520General\%2520Page}{Android}
    and turn on Breaking News and Politics alerts
  \end{itemize}
\end{itemize}

Advertisement

\protect\hyperlink{after-bottom}{Continue reading the main story}

\hypertarget{site-index}{%
\subsection{Site Index}\label{site-index}}

\hypertarget{site-information-navigation}{%
\subsection{Site Information
Navigation}\label{site-information-navigation}}

\begin{itemize}
\tightlist
\item
  \href{https://help.nytimes.com/hc/en-us/articles/115014792127-Copyright-notice}{©~2020~The
  New York Times Company}
\end{itemize}

\begin{itemize}
\tightlist
\item
  \href{https://www.nytco.com/}{NYTCo}
\item
  \href{https://help.nytimes.com/hc/en-us/articles/115015385887-Contact-Us}{Contact
  Us}
\item
  \href{https://www.nytco.com/careers/}{Work with us}
\item
  \href{https://nytmediakit.com/}{Advertise}
\item
  \href{http://www.tbrandstudio.com/}{T Brand Studio}
\item
  \href{https://www.nytimes.com/privacy/cookie-policy\#how-do-i-manage-trackers}{Your
  Ad Choices}
\item
  \href{https://www.nytimes.com/privacy}{Privacy}
\item
  \href{https://help.nytimes.com/hc/en-us/articles/115014893428-Terms-of-service}{Terms
  of Service}
\item
  \href{https://help.nytimes.com/hc/en-us/articles/115014893968-Terms-of-sale}{Terms
  of Sale}
\item
  \href{https://spiderbites.nytimes.com}{Site Map}
\item
  \href{https://help.nytimes.com/hc/en-us}{Help}
\item
  \href{https://www.nytimes.com/subscription?campaignId=37WXW}{Subscriptions}
\end{itemize}
