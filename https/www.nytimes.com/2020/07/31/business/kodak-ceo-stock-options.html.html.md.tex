Sections

SEARCH

\protect\hyperlink{site-content}{Skip to
content}\protect\hyperlink{site-index}{Skip to site index}

\href{https://www.nytimes.com/section/business}{Business}

\href{https://myaccount.nytimes.com/auth/login?response_type=cookie\&client_id=vi}{}

\href{https://www.nytimes.com/section/todayspaper}{Today's Paper}

\href{/section/business}{Business}\textbar{}Kodak C.E.O. Got Stock
Options Day Before News of Loan Sent Stock Soaring

\url{https://nyti.ms/2Xgs8Dp}

\begin{itemize}
\item
\item
\item
\item
\item
\end{itemize}

Advertisement

\protect\hyperlink{after-top}{Continue reading the main story}

Supported by

\protect\hyperlink{after-sponsor}{Continue reading the main story}

\hypertarget{kodak-ceo-got-stock-options-day-before-news-of-loan-sent-stock-soaring}{%
\section{Kodak C.E.O. Got Stock Options Day Before News of Loan Sent
Stock
Soaring}\label{kodak-ceo-got-stock-options-day-before-news-of-loan-sent-stock-soaring}}

The stock options suddenly were worth about \$50 million --- the latest
instance of extraordinary good timing by corporate executives.

\includegraphics{https://static01.nyt.com/images/2020/08/01/business/31virus-kodak-print/merlin_77472113_4c707a84-2d62-47ed-bea4-1ad9ba7aae9f-articleLarge.jpg?quality=75\&auto=webp\&disable=upscale}

By \href{https://www.nytimes.com/by/jesse-drucker}{Jesse Drucker} and
\href{https://nytimes.com/by/ellen-gabler/}{Ellen Gabler}

\begin{itemize}
\item
  July 31, 2020
\item
  \begin{itemize}
  \item
  \item
  \item
  \item
  \item
  \end{itemize}
\end{itemize}

At the beginning of this week, the Eastman Kodak Company handed its
chief executive 1.75 million stock options.

It was the type of compensation decision that generally wouldn't attract
much notice, except for one thing: The day after the stock options were
granted, the White House announced that the company would receive
\href{https://www.nytimes.com/live/2020/07/28/business/stock-market-today-coronavirus\#the-united-states-will-lend-kodak-765-million-to-make-drug-components}{a
\$765 million federal loan} to produce ingredients to make
pharmaceuticals in the United States.

The news of the deal caused Kodak's shares to soar more than 1,000
percent. Within 48 hours of the options grants, their value had
ballooned, at least on paper, to about \$50 million.

The government loan is part of a broader federal effort to increase the
country's ability to respond to the coronavirus and future pandemics.

The options grant to Kodak's executive chairman and chief executive
officer, Jim Continenza, is the latest example of executives and board
members at companies receiving such federal support to benefit from
extraordinarily good timing. A number of those companies are involved in
the hunt for vaccines and treatments for Covid-19.

\href{https://www.nytimes.com/2020/07/25/business/coronavirus-vaccine-profits-vaxart.html}{Insiders
at Vaxart}, for example, received stock options shortly before the
California biotech company announced in June that its potential
coronavirus vaccine was being tested in a program organized by a federal
agency, causing its shares to instantly double.

A Kodak spokeswoman declined to comment on the timing of the
stock-options grants and emphasized that the value of the options could
change before Mr. Continenza uses them to buy Kodak shares.

Kodak, best known for its iconic camera and film business, has been
struggling for years to reinvent itself. The company emerged from
bankruptcy protection in 2013, and its shares in recent years have
mostly been trading at \$2 or \$3, giving it a market value of about
\$100 million.

Starting in May, Kodak began talks with the Trump administration about
manufacturing the ingredients for pharmaceuticals, Mr. Continenza said
in a television interview this week.

The deal was announced on Tuesday. President Trump said the federal loan
from the U.S. International Development Finance Corporation would help
reduce the United States' reliance on other countries, in particular
China and India, for the vast majority of ingredients used to make
generic drugs. Mr. Trump called the Kodak deal ``a breakthrough in
bringing pharmaceutical manufacturing back to the United States.''

Kodak said it was creating a new pharmaceuticals division and will
expand its facilities in Rochester, N.Y., and St. Paul, Minn. The
division will eventually have the capacity to produce as much as 25
percent of the active ingredients used in generic drugs in the United
States. Kodak has been in the chemicals business for more than a century
and ``has the facilities sitting there ready to go,'' Mr. Continenza
said in a TV interview this week.

It's unclear whether the ingredients that Kodak makes will have any role
in the fight against the coronavirus. Kodak will coordinate with the
federal government and other manufacturers to figure out which
ingredients to make, prioritizing those that are deemed critical to
Americans and national security.

The day before the loan was announced,
\href{https://www.nytimes.com/2020/07/30/business/whats-up-with-kodak.html}{trading
in Kodak shares surged}, and its stock jumped about 25 percent, closing
at \$2.62 a share. That activity raised suspicion about improper trading
ahead of the market-moving news, but
\href{https://www.wsj.com/articles/tweets-and-articles-sent-kodak-shares-surging-before-official-announcement-11596056729}{The
Wall Street Journal reported} that it was apparently the result of
reports by the media in Rochester, where Kodak is headquartered, about
the pending announcement.

Around the time that Kodak began talking with the federal government
this spring, Kodak insiders began receiving stock options. The pattern
was first reported
by\href{https://nongaap.substack.com/p/corp-governance-dark-arts-part-5}{Non-GAAP
Thoughts}, a digital newsletter.

On May 20, Kodak handed out 240,000 stock options to board members ---
an addition to its usual equity distribution in January.

The May stock options awarded to directors are now worth about \$4
million. Those options are eligible to be exercised gradually over the
course of this year.

Arielle Patrick, a spokeswoman for Kodak, declined to answer questions
about why the directors were granted stock options in May.

On the same day that Kodak was alerting the local media to its
about-to-be-announced deal with the Trump administration, the
compensation committee of the company's board voted to award Mr.
Continenza 1.75 million stock options that allow him to purchase shares
at prices ranging from \$3.03 to \$12.

By Wednesday morning, Kodak's shares had soared as high as \$60 each.
They have since retreated to about \$24, which means the stock options
give Mr. Continenza the right to buy shares at a deep discount.

Mr. Continenza can exercise some but not all of the options immediately.

Ms. Patrick said that the rapid increase in the values of Mr.
Continenza's new stock options ``are paper only. Mr. Continenza has not
received any proceeds nor does he have any intention of selling.''

She added that Kodak's board awarded the options to Mr. Continenza
because when the company last year issued a type of debt that converts
into equity, the value of the chief executive's stock and options were
diluted.

She said that Kodak received shareholder approval in May to issue
additional shares, and that the compensation committee approved the
options ``at the first meeting of this committee since the annual
stockholders meeting,'' which was on Monday, July 27.

She declined to comment on why Kodak did not wait until after the White
House announcement to grant the options.

The increase in Kodak's shares this week also transformed some stock
options that Mr. Continenza received when he became chief executive.
They had been effectively worthless because of Kodak's low stock price.
This week, their value grew to about \$59 million,
\href{https://www.nytimes.com/reuters/2020/07/29/business/29reuters-eastman-kodak-ceo.html}{Reuters
reported}.

Advertisement

\protect\hyperlink{after-bottom}{Continue reading the main story}

\hypertarget{site-index}{%
\subsection{Site Index}\label{site-index}}

\hypertarget{site-information-navigation}{%
\subsection{Site Information
Navigation}\label{site-information-navigation}}

\begin{itemize}
\tightlist
\item
  \href{https://help.nytimes.com/hc/en-us/articles/115014792127-Copyright-notice}{©~2020~The
  New York Times Company}
\end{itemize}

\begin{itemize}
\tightlist
\item
  \href{https://www.nytco.com/}{NYTCo}
\item
  \href{https://help.nytimes.com/hc/en-us/articles/115015385887-Contact-Us}{Contact
  Us}
\item
  \href{https://www.nytco.com/careers/}{Work with us}
\item
  \href{https://nytmediakit.com/}{Advertise}
\item
  \href{http://www.tbrandstudio.com/}{T Brand Studio}
\item
  \href{https://www.nytimes.com/privacy/cookie-policy\#how-do-i-manage-trackers}{Your
  Ad Choices}
\item
  \href{https://www.nytimes.com/privacy}{Privacy}
\item
  \href{https://help.nytimes.com/hc/en-us/articles/115014893428-Terms-of-service}{Terms
  of Service}
\item
  \href{https://help.nytimes.com/hc/en-us/articles/115014893968-Terms-of-sale}{Terms
  of Sale}
\item
  \href{https://spiderbites.nytimes.com}{Site Map}
\item
  \href{https://help.nytimes.com/hc/en-us}{Help}
\item
  \href{https://www.nytimes.com/subscription?campaignId=37WXW}{Subscriptions}
\end{itemize}
