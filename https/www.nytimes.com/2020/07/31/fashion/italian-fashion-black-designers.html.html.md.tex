Sections

SEARCH

\protect\hyperlink{site-content}{Skip to
content}\protect\hyperlink{site-index}{Skip to site index}

\href{https://www.nytimes.com/section/fashion}{Fashion}

\href{https://myaccount.nytimes.com/auth/login?response_type=cookie\&client_id=vi}{}

\href{https://www.nytimes.com/section/todayspaper}{Today's Paper}

\href{/section/fashion}{Fashion}\textbar{}Italian Fashion Hasn't Changed
Enough, Black Designers Say

\url{https://nyti.ms/312XgaE}

\begin{itemize}
\item
\item
\item
\item
\item
\end{itemize}

Advertisement

\protect\hyperlink{after-top}{Continue reading the main story}

Supported by

\protect\hyperlink{after-sponsor}{Continue reading the main story}

\hypertarget{italian-fashion-hasnt-changed-enough-black-designers-say}{%
\section{Italian Fashion Hasn't Changed Enough, Black Designers
Say}\label{italian-fashion-hasnt-changed-enough-black-designers-say}}

A new letter sent by independent designers to fashion executives demands
reform,~or at least a serious conversation about race.

\includegraphics{https://static01.nyt.com/images/2020/07/31/fashion/31stella-1/merlin_173300313_20e6e049-5489-41f9-be5a-9b7d5473ddd8-articleLarge.jpg?quality=75\&auto=webp\&disable=upscale}

By \href{https://www.nytimes.com/by/jessica-testa}{Jessica Testa}

\begin{itemize}
\item
  July 31, 2020
\item
  \begin{itemize}
  \item
  \item
  \item
  \item
  \item
  \end{itemize}
\end{itemize}

The designer Stella Jean wanted to make something clear about an
impassioned letter she sent to Italian fashion's governing body: ``It's
not a protest,'' she said. ``I'm not protesting. It's a proposal.''

Last week Ms. Jean sent the letter --- titled ``Do \#BLM in Italian
Fashion?'' --- to the president and 14 executive board members of the
Camera Nazionale della Moda Italiana, which organizes the Milan shows.
Those board members included the chief executives of Gucci, Dolce \&
Gabbana, Prada and more. The letter was co-signed by
\href{https://www.nytimes.com/2019/02/21/fashion/milan-edward-buchanan-race.html}{Edward
Buchanan}, an American designer based in Milan.

In the letter, shared with reporters on July 30, Ms. Jean and Mr.
Buchanan asked for ``a constructive, working dialogue'' about how to
best support the country's Black designers ahead of Milan Fashion Week
in September.

They acknowledged that in late 2019, the fashion council had published a
manifesto promising
``\href{https://www.cameramoda.it/en/associazione/news/1588/}{sweeping
reforms}'' in diversity and inclusion. They suggested that those reforms
had not yet come.

As evidence, Ms. Jean cited the recent
\href{https://www.nytimes.com/2020/07/13/fashion/milan-digital-fashion-week-video-stream.html}{Milan
Digital Fashion Week}, which had no Black designers on the main
streaming calendar, she said --- though a few were highlighted elsewhere
on the website, alongside other international or emerging brands.

``Let's change things,'' she and Mr. Buchanan wrote in the letter.
Instead of round tables on diversity, they proposed ``true work, true
collaboration.''

Carlo Capasa, the president of the Camera Nazionale della Moda Italiana,
said in an email Friday that the council's commitment to inclusion was
``real and measurable,'' pointing to its manifesto, its ``huge'' effort
to scout emerging brands and, in particular, its years of financial
assistance to Ms. Jean.

``I can say we supported her brand in an extraordinary way in comparison
to our standard support to emerging brands,'' he said.

The Italian fashion industry has been long criticized over incidences of
racism and cultural appropriation. In recent years, Dolce \& Gabbana
released ads that
\href{https://www.nytimes.com/2018/11/23/fashion/dolce-gabbana-china-disaster-backlash.html}{drove
Chinese customers} to burn their past purchases;
\href{https://www.nytimes.com/2020/02/04/style/Prada-racism-City-Commission-on-Human-Rights.html}{Prada}
made bag charms that evoked blackface;
\href{https://www.nytimes.com/2019/02/07/business/gucci-blackface-adidas-apologize.html}{Gucci}
made balaclava sweaters that were similarly suggestive.

These brands later shared
\href{https://www.nytimes.com/2020/06/06/style/fashion-racism-actions.html}{black
square statements} of support for Black Lives Matter, Ms. Jean pointed
out. Some have taken significant steps toward
\href{https://www.vogue.com/article/gucci-diversity-inclusion-renee-tirado}{changing}
their internal culture.

But the controversies in Italian fashion have kept coming: Marni
\href{https://www.instagram.com/p/CDPtl8EHYAr/}{apologized} Wednesday
for a ``Jungle Mood'' campaign that the fashion industry watchdog Diet
Prada \href{https://www.instagram.com/p/CDMR7AjnYUx/?hl=en}{called out}
for ``alluding to racist, colonial stereotypes.''

\includegraphics{https://static01.nyt.com/images/2020/07/31/fashion/31stella-2-sub/31stella-2-sub-articleLarge.jpg?quality=75\&auto=webp\&disable=upscale}

Ms. Jean, who is Haitian-Italian, had a suggestion for these brands in
her letter: ``For companies wishing to continue to draw free inspiration
from Black culture,'' an organization called Made in Africa will provide
a list of African artisans who can ``train and collaborate with Italian
companies,'' so that brands ``no longer create collections simply
inspired by Africa,'' but are ``consciously created with Africa.''

The letter also suggested creating a public database of Italian fashion
companies and their percentages of Black employees. (Mr. Capasa said the
council was conducting a survey to ``map and monitor'' brands' inclusion
and diversity efforts.) It emphasized the need to spotlight Black talent
and for young Black designers to have access to fashion schools in
Italy.

In responding to the letter, Mr. Capasa told Ms. Jean that he agreed it
was ``time to pass from words to action,'' according to a copy of the
response that he provided to The New York Times. He gave examples of
recent events and public discussions hosted by his organization
featuring Black voices from around the world.

He wrote that it was ``a pity'' Ms. Jean had never asked to be part of
the diversity and inclusion work group, and he maintained that she had
always been given support, attaching a list of discounts she had
received since 2013, totaling more than 175,000 euros.

``We've granted you various gratuities and preferential treatment in the
last few years,'' he said. ``I don't understand why you write as if none
of this ever happened.''

In an interview, Ms. Jean acknowledged the financial help but said that
Mr. Capasa had missed the point of her letter: It wasn't about her but
about all Black people in Italian fashion.

``We are still completely invisible to them,'' she said.

More than 100 companies make up the Camera Nazionale della Moda
Italiana, but Ms. Jean said her company is the only Black-owned brand.

This is not her first time speaking out about the marginalization of
people of color in Italy. During Milan Fashion Week in February, instead
of a runway show, she made a video featuring Italian women --- students,
lawyers, executives ---
\href{https://www.youtube.com/watch?v=ykhRrN0r5NI}{sharing racist
remarks} they had received. (It was more upbeat than it sounds.)

\href{https://www.nytimes.com/2013/09/24/fashion/Giorgio-Armani-Gives-Young-Designer-a-Hand.html?searchResultPosition=20}{Ms.
Jean} said she will not return to the official Milan Fashion Week
calendar until she is no longer the only Black designer on it. She has
grown tired of being an anomaly.

``I don't want to be the only one anymore,'' she said. ``But it's not
about boycotting. It's about asking for change.''

Advertisement

\protect\hyperlink{after-bottom}{Continue reading the main story}

\hypertarget{site-index}{%
\subsection{Site Index}\label{site-index}}

\hypertarget{site-information-navigation}{%
\subsection{Site Information
Navigation}\label{site-information-navigation}}

\begin{itemize}
\tightlist
\item
  \href{https://help.nytimes.com/hc/en-us/articles/115014792127-Copyright-notice}{©~2020~The
  New York Times Company}
\end{itemize}

\begin{itemize}
\tightlist
\item
  \href{https://www.nytco.com/}{NYTCo}
\item
  \href{https://help.nytimes.com/hc/en-us/articles/115015385887-Contact-Us}{Contact
  Us}
\item
  \href{https://www.nytco.com/careers/}{Work with us}
\item
  \href{https://nytmediakit.com/}{Advertise}
\item
  \href{http://www.tbrandstudio.com/}{T Brand Studio}
\item
  \href{https://www.nytimes.com/privacy/cookie-policy\#how-do-i-manage-trackers}{Your
  Ad Choices}
\item
  \href{https://www.nytimes.com/privacy}{Privacy}
\item
  \href{https://help.nytimes.com/hc/en-us/articles/115014893428-Terms-of-service}{Terms
  of Service}
\item
  \href{https://help.nytimes.com/hc/en-us/articles/115014893968-Terms-of-sale}{Terms
  of Sale}
\item
  \href{https://spiderbites.nytimes.com}{Site Map}
\item
  \href{https://help.nytimes.com/hc/en-us}{Help}
\item
  \href{https://www.nytimes.com/subscription?campaignId=37WXW}{Subscriptions}
\end{itemize}
