Sections

SEARCH

\protect\hyperlink{site-content}{Skip to
content}\protect\hyperlink{site-index}{Skip to site index}

\href{https://www.nytimes.com/section/health}{Health}

\href{https://myaccount.nytimes.com/auth/login?response_type=cookie\&client_id=vi}{}

\href{https://www.nytimes.com/section/todayspaper}{Today's Paper}

\href{/section/health}{Health}\textbar{}Covid-19 Tests Are in Short
Supply. Should You Still Get One?

\url{https://nyti.ms/39IVzTK}

\begin{itemize}
\item
\item
\item
\item
\item
\item
\end{itemize}

\href{https://www.nytimes.com/news-event/coronavirus?action=click\&pgtype=Article\&state=default\&region=TOP_BANNER\&context=storylines_menu}{The
Coronavirus Outbreak}

\begin{itemize}
\tightlist
\item
  live\href{https://www.nytimes.com/2020/08/01/world/coronavirus-covid-19.html?action=click\&pgtype=Article\&state=default\&region=TOP_BANNER\&context=storylines_menu}{Latest
  Updates}
\item
  \href{https://www.nytimes.com/interactive/2020/us/coronavirus-us-cases.html?action=click\&pgtype=Article\&state=default\&region=TOP_BANNER\&context=storylines_menu}{Maps
  and Cases}
\item
  \href{https://www.nytimes.com/interactive/2020/science/coronavirus-vaccine-tracker.html?action=click\&pgtype=Article\&state=default\&region=TOP_BANNER\&context=storylines_menu}{Vaccine
  Tracker}
\item
  \href{https://www.nytimes.com/interactive/2020/07/29/us/schools-reopening-coronavirus.html?action=click\&pgtype=Article\&state=default\&region=TOP_BANNER\&context=storylines_menu}{What
  School May Look Like}
\item
  \href{https://www.nytimes.com/live/2020/07/31/business/stock-market-today-coronavirus?action=click\&pgtype=Article\&state=default\&region=TOP_BANNER\&context=storylines_menu}{Economy}
\end{itemize}

Advertisement

\protect\hyperlink{after-top}{Continue reading the main story}

Supported by

\protect\hyperlink{after-sponsor}{Continue reading the main story}

\hypertarget{covid-19-tests-are-in-short-supply-should-you-still-get-one}{%
\section{Covid-19 Tests Are in Short Supply. Should You Still Get
One?}\label{covid-19-tests-are-in-short-supply-should-you-still-get-one}}

Public health officials once suggested only people with symptoms should
be tested. Now some say getting one is a civic duty.

\includegraphics{https://static01.nyt.com/images/2020/07/25/multimedia/00xp-virus-ethics-pix/00xp-virus-ethics-pix-articleLarge.jpg?quality=75\&auto=webp\&disable=upscale}

By \href{https://www.nytimes.com/by/maria-cramer}{Maria Cramer}

\begin{itemize}
\item
  July 31, 2020
\item
  \begin{itemize}
  \item
  \item
  \item
  \item
  \item
  \item
  \end{itemize}
\end{itemize}

First, health officials said
\href{https://www.washingtonpost.com/health/2020/03/21/coronavirus-testing-strategyshift/}{you
should not get tested} for the coronavirus unless you had symptoms. Then
the message shifted:
\href{https://www.nytimes.com/2020/07/01/health/coronavirus-pooled-testing.html}{Mass
testing was essential} to trace and contain the pandemic.

Now there is more confusion as
\href{https://www.nytimes.com/2020/07/23/health/coronavirus-testing-supply-shortage.html}{public
health officials warn} of shortages of testing supplies, soaring demand
and long wait times for results, which have
\href{https://www.nytimes.com/2020/07/31/health/covid-contact-tracing-tests.html}{thwarted
contact-tracing efforts in places with heavy outbreaks}. At the same
time, politicians in states where cases had fallen have urged people to
get tested, whether they have symptoms or not.

``Go get a
test,''\href{https://www.governor.ny.gov/news/video-photos-audio-rush-transcript-governor-cuomo-announces-phase-three-indoor-dining-postponed\#}{Gov.
Andrew M. Cuomo of New York said this month}. ``It doesn't cost you
anything. It doesn't hurt.''

And in New Jersey,
\href{https://nj.gov/governor/news/news/562020/approved/20200623b.shtml}{Gov.
Philip D. Murphy said during} a recent
briefing\href{https://nj.gov/governor/news/news/562020/approved/20200623b.shtml}{on
the coronavirus}: ``Getting tested gives you peace of mind that you are
not unknowingly carrying this virus and can spread it among your family
and friends. We just need you to go out and get tested.''

What is a conscientious person who already wears a mask and maintains
social distance to do?

\hypertarget{is-there-a-moral-obligation-to-get-a-test}{%
\subsection{Is there a moral obligation to get a
test?}\label{is-there-a-moral-obligation-to-get-a-test}}

Yes, said \href{https://law.wisc.edu/profiles/racharo}{R. Alta Charo}, a
professor of law and bioethics at the University of Wisconsin-Madison.

``One of the most important things to keep in mind when discussing
public health is the fact that this is fundamentally a community issue,
not merely an individual health concern,'' she said. ``We are all in
this together. What I do affects everyone around me, and what they do
affects me.''

If public health experts want people to be tested, they should comply,
especially if the goal is to gather critical information about how many
people are infected at a given point, Professor Charo said.

Epidemiologists can use the data to determine how fast the virus is
spreading and which measures are working, she said.

Taking a test, like wearing a mask, shows ``a desire to be a part of the
solution,'' said Dr.
\href{https://www.adelphi.edu/faculty/profiles/profile.php?PID=0328}{K.
C. Rondello}, an epidemiologist at Adelphi University in Garden City,
N.Y.

The virus has been difficult to control in large part because many
infected people without symptoms have unknowingly spread it, he said.

\hypertarget{latest-updates-global-coronavirus-outbreak}{%
\section{\texorpdfstring{\href{https://www.nytimes.com/2020/08/01/world/coronavirus-covid-19.html?action=click\&pgtype=Article\&state=default\&region=MAIN_CONTENT_1\&context=storylines_live_updates}{Latest
Updates: Global Coronavirus
Outbreak}}{Latest Updates: Global Coronavirus Outbreak}}\label{latest-updates-global-coronavirus-outbreak}}

Updated 2020-08-02T06:58:18.835Z

\begin{itemize}
\tightlist
\item
  \href{https://www.nytimes.com/2020/08/01/world/coronavirus-covid-19.html?action=click\&pgtype=Article\&state=default\&region=MAIN_CONTENT_1\&context=storylines_live_updates\#link-34047410}{The
  U.S. reels as July cases more than double the total of any other
  month.}
\item
  \href{https://www.nytimes.com/2020/08/01/world/coronavirus-covid-19.html?action=click\&pgtype=Article\&state=default\&region=MAIN_CONTENT_1\&context=storylines_live_updates\#link-780ec966}{Top
  U.S. officials work to break an impasse over the federal jobless
  benefit.}
\item
  \href{https://www.nytimes.com/2020/08/01/world/coronavirus-covid-19.html?action=click\&pgtype=Article\&state=default\&region=MAIN_CONTENT_1\&context=storylines_live_updates\#link-2bc8948}{Its
  outbreak untamed, Melbourne goes into even greater lockdown.}
\end{itemize}

\href{https://www.nytimes.com/2020/08/01/world/coronavirus-covid-19.html?action=click\&pgtype=Article\&state=default\&region=MAIN_CONTENT_1\&context=storylines_live_updates}{See
more updates}

More live coverage:
\href{https://www.nytimes.com/live/2020/07/31/business/stock-market-today-coronavirus?action=click\&pgtype=Article\&state=default\&region=MAIN_CONTENT_1\&context=storylines_live_updates}{Markets}

More testing will help identify these hidden cases, Dr. Rondello said.

But
\href{https://www.du.edu/ahss/philosophy/faculty_staff/upton_candace.html}{Candace
L. Upton}, a professor of philosophy at the University of Denver, said
people should not feel duty bound to get a test. It can even be argued
that it is morally wrong to go in for a test if you have no symptoms and
are not at a high risk, she said.

``Until there is no longer a shortage of test kits, it is morally
unjustified to test patients for Covid-19 solely for the purpose of
collecting data,'' Professor Upton said. ``Because of the deficit, labs
shouldn't be offering them to people who are just curious.''

The priority should remain testing only those with symptoms or
compromised immune systems, and essential workers and older people, she
said.

Professor Upton added that testing should be done selectively even in
locations where tests are readily available and where results can be
delivered quickly.

``The whole system is unfair,'' she said. ``And so to take advantage of
surpluses in certain places in the market is to add to the injustice to
people who didn't have availability in the first place.''

\hypertarget{i-dont-want-to-be-the-person-to-bring-covid-here}{%
\subsection{`I don't want to be the person to bring Covid
here.'}\label{i-dont-want-to-be-the-person-to-bring-covid-here}}

The national failure to coordinate testing efforts shouldn't cause
people with no symptoms to feel conflicted about being tested for the
coronavirus, said Dr.
\href{https://www.onemedical.com/providers/andrew-diamond/}{Andrew
Diamond}, chief medical officer at One Medical in San Francisco, a
membership-based primary care practice with offices around the country.

``If there is a way for you to get tested that does not clearly and
directly impair someone who is a priority, then you should get tested
for sure,'' he said.

Molly Wallace, 24, who grew up on Martha's Vineyard, was tested after
she moved back to the island from Boston in March.

She was furloughed from her job as a medical assistant and began
volunteering at a testing site, Test MV, at Martha's Vineyard Regional
High School, where she went to school.

\href{https://www.nytimes.com/news-event/coronavirus?action=click\&pgtype=Article\&state=default\&region=MAIN_CONTENT_3\&context=storylines_faq}{}

\hypertarget{the-coronavirus-outbreak-}{%
\subsubsection{The Coronavirus Outbreak
›}\label{the-coronavirus-outbreak-}}

\hypertarget{frequently-asked-questions}{%
\paragraph{Frequently Asked
Questions}\label{frequently-asked-questions}}

Updated July 27, 2020

\begin{itemize}
\item ~
  \hypertarget{should-i-refinance-my-mortgage}{%
  \paragraph{Should I refinance my
  mortgage?}\label{should-i-refinance-my-mortgage}}

  \begin{itemize}
  \tightlist
  \item
    \href{https://www.nytimes.com/article/coronavirus-money-unemployment.html?action=click\&pgtype=Article\&state=default\&region=MAIN_CONTENT_3\&context=storylines_faq}{It
    could be a good idea,} because mortgage rates have
    \href{https://www.nytimes.com/2020/07/16/business/mortgage-rates-below-3-percent.html?action=click\&pgtype=Article\&state=default\&region=MAIN_CONTENT_3\&context=storylines_faq}{never
    been lower.} Refinancing requests have pushed mortgage applications
    to some of the highest levels since 2008, so be prepared to get in
    line. But defaults are also up, so if you're thinking about buying a
    home, be aware that some lenders have tightened their standards.
  \end{itemize}
\item ~
  \hypertarget{what-is-school-going-to-look-like-in-september}{%
  \paragraph{What is school going to look like in
  September?}\label{what-is-school-going-to-look-like-in-september}}

  \begin{itemize}
  \tightlist
  \item
    It is unlikely that many schools will return to a normal schedule
    this fall, requiring the grind of
    \href{https://www.nytimes.com/2020/06/05/us/coronavirus-education-lost-learning.html?action=click\&pgtype=Article\&state=default\&region=MAIN_CONTENT_3\&context=storylines_faq}{online
    learning},
    \href{https://www.nytimes.com/2020/05/29/us/coronavirus-child-care-centers.html?action=click\&pgtype=Article\&state=default\&region=MAIN_CONTENT_3\&context=storylines_faq}{makeshift
    child care} and
    \href{https://www.nytimes.com/2020/06/03/business/economy/coronavirus-working-women.html?action=click\&pgtype=Article\&state=default\&region=MAIN_CONTENT_3\&context=storylines_faq}{stunted
    workdays} to continue. California's two largest public school
    districts --- Los Angeles and San Diego --- said on July 13, that
    \href{https://www.nytimes.com/2020/07/13/us/lausd-san-diego-school-reopening.html?action=click\&pgtype=Article\&state=default\&region=MAIN_CONTENT_3\&context=storylines_faq}{instruction
    will be remote-only in the fall}, citing concerns that surging
    coronavirus infections in their areas pose too dire a risk for
    students and teachers. Together, the two districts enroll some
    825,000 students. They are the largest in the country so far to
    abandon plans for even a partial physical return to classrooms when
    they reopen in August. For other districts, the solution won't be an
    all-or-nothing approach.
    \href{https://bioethics.jhu.edu/research-and-outreach/projects/eschool-initiative/school-policy-tracker/}{Many
    systems}, including the nation's largest, New York City, are
    devising
    \href{https://www.nytimes.com/2020/06/26/us/coronavirus-schools-reopen-fall.html?action=click\&pgtype=Article\&state=default\&region=MAIN_CONTENT_3\&context=storylines_faq}{hybrid
    plans} that involve spending some days in classrooms and other days
    online. There's no national policy on this yet, so check with your
    municipal school system regularly to see what is happening in your
    community.
  \end{itemize}
\item ~
  \hypertarget{is-the-coronavirus-airborne}{%
  \paragraph{Is the coronavirus
  airborne?}\label{is-the-coronavirus-airborne}}

  \begin{itemize}
  \tightlist
  \item
    The coronavirus
    \href{https://www.nytimes.com/2020/07/04/health/239-experts-with-one-big-claim-the-coronavirus-is-airborne.html?action=click\&pgtype=Article\&state=default\&region=MAIN_CONTENT_3\&context=storylines_faq}{can
    stay aloft for hours in tiny droplets in stagnant air}, infecting
    people as they inhale, mounting scientific evidence suggests. This
    risk is highest in crowded indoor spaces with poor ventilation, and
    may help explain super-spreading events reported in meatpacking
    plants, churches and restaurants.
    \href{https://www.nytimes.com/2020/07/06/health/coronavirus-airborne-aerosols.html?action=click\&pgtype=Article\&state=default\&region=MAIN_CONTENT_3\&context=storylines_faq}{It's
    unclear how often the virus is spread} via these tiny droplets, or
    aerosols, compared with larger droplets that are expelled when a
    sick person coughs or sneezes, or transmitted through contact with
    contaminated surfaces, said Linsey Marr, an aerosol expert at
    Virginia Tech. Aerosols are released even when a person without
    symptoms exhales, talks or sings, according to Dr. Marr and more
    than 200 other experts, who
    \href{https://academic.oup.com/cid/article/doi/10.1093/cid/ciaa939/5867798}{have
    outlined the evidence in an open letter to the World Health
    Organization}.
  \end{itemize}
\item ~
  \hypertarget{what-are-the-symptoms-of-coronavirus}{%
  \paragraph{What are the symptoms of
  coronavirus?}\label{what-are-the-symptoms-of-coronavirus}}

  \begin{itemize}
  \tightlist
  \item
    Common symptoms
    \href{https://www.nytimes.com/article/symptoms-coronavirus.html?action=click\&pgtype=Article\&state=default\&region=MAIN_CONTENT_3\&context=storylines_faq}{include
    fever, a dry cough, fatigue and difficulty breathing or shortness of
    breath.} Some of these symptoms overlap with those of the flu,
    making detection difficult, but runny noses and stuffy sinuses are
    less common.
    \href{https://www.nytimes.com/2020/04/27/health/coronavirus-symptoms-cdc.html?action=click\&pgtype=Article\&state=default\&region=MAIN_CONTENT_3\&context=storylines_faq}{The
    C.D.C. has also} added chills, muscle pain, sore throat, headache
    and a new loss of the sense of taste or smell as symptoms to look
    out for. Most people fall ill five to seven days after exposure, but
    symptoms may appear in as few as two days or as many as 14 days.
  \end{itemize}
\item ~
  \hypertarget{does-asymptomatic-transmission-of-covid-19-happen}{%
  \paragraph{Does asymptomatic transmission of Covid-19
  happen?}\label{does-asymptomatic-transmission-of-covid-19-happen}}

  \begin{itemize}
  \tightlist
  \item
    So far, the evidence seems to show it does. A widely cited
    \href{https://www.nature.com/articles/s41591-020-0869-5}{paper}
    published in April suggests that people are most infectious about
    two days before the onset of coronavirus symptoms and estimated that
    44 percent of new infections were a result of transmission from
    people who were not yet showing symptoms. Recently, a top expert at
    the World Health Organization stated that transmission of the
    coronavirus by people who did not have symptoms was ``very rare,''
    \href{https://www.nytimes.com/2020/06/09/world/coronavirus-updates.html?action=click\&pgtype=Article\&state=default\&region=MAIN_CONTENT_3\&context=storylines_faq\#link-1f302e21}{but
    she later walked back that statement.}
  \end{itemize}
\end{itemize}

Ms. Wallace said that she had never had coronavirus symptoms but that
she had still felt obligated to be tested. ``I don't want to be the
person to bring Covid here,'' she said.

All residents and visitors to the island are encouraged to be tested at
Test MV, where volunteers distribute free kits of self-administered
nasal swabs, said Ms. Wallace, who is now the site's outreach
coordinator.

During her interview last week, Ms. Wallace said that people typically
got their results within 72 hours, or more quickly if they test positive
for the virus. At the time, Martha's Vineyard stood in stark contrast to
states like New York and Arizona, where lines for tests have sometimes
\href{https://www.nytimes.com/2020/07/06/us/coronavirus-test-shortage.html}{stretched
around blocks} and
\href{https://www.azcentral.com/story/news/local/arizona-health/2020/07/06/slow-covid-19-test-results-have-cascade-effect-public-health/5386582002/}{the
turnaround time} for results has been days, if not weeks.

On Friday, she said the turnaround time for tests was now around seven
days, because of the shortage.

\hypertarget{dont-feel-guilty-if-you-dont-want-to-get-a-test}{%
\subsection{Don't feel guilty if you don't want to get a
test.}\label{dont-feel-guilty-if-you-dont-want-to-get-a-test}}

Wearing a mask should feel obligatory, Dr. Diamond said, but taking a
test should not.

If tests were widely available and turnaround times for results were
much faster, people would have a stronger sense of obligation to get
tested, he said.

``Under the current circumstance, I would say it's much more important
to continue to do what you're doing,'' Dr. Diamond said. That is, wear a
mask, keep six feet away from people and stay home as much as possible,
he said.

Dr. Diamond added, ``The behavior is really the thing that's going to
make the biggest difference.''

Remy Tumin contributed reporting.

Advertisement

\protect\hyperlink{after-bottom}{Continue reading the main story}

\hypertarget{site-index}{%
\subsection{Site Index}\label{site-index}}

\hypertarget{site-information-navigation}{%
\subsection{Site Information
Navigation}\label{site-information-navigation}}

\begin{itemize}
\tightlist
\item
  \href{https://help.nytimes.com/hc/en-us/articles/115014792127-Copyright-notice}{©~2020~The
  New York Times Company}
\end{itemize}

\begin{itemize}
\tightlist
\item
  \href{https://www.nytco.com/}{NYTCo}
\item
  \href{https://help.nytimes.com/hc/en-us/articles/115015385887-Contact-Us}{Contact
  Us}
\item
  \href{https://www.nytco.com/careers/}{Work with us}
\item
  \href{https://nytmediakit.com/}{Advertise}
\item
  \href{http://www.tbrandstudio.com/}{T Brand Studio}
\item
  \href{https://www.nytimes.com/privacy/cookie-policy\#how-do-i-manage-trackers}{Your
  Ad Choices}
\item
  \href{https://www.nytimes.com/privacy}{Privacy}
\item
  \href{https://help.nytimes.com/hc/en-us/articles/115014893428-Terms-of-service}{Terms
  of Service}
\item
  \href{https://help.nytimes.com/hc/en-us/articles/115014893968-Terms-of-sale}{Terms
  of Sale}
\item
  \href{https://spiderbites.nytimes.com}{Site Map}
\item
  \href{https://help.nytimes.com/hc/en-us}{Help}
\item
  \href{https://www.nytimes.com/subscription?campaignId=37WXW}{Subscriptions}
\end{itemize}
