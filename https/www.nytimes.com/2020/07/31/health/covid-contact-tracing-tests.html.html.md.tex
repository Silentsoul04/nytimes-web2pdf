Sections

SEARCH

\protect\hyperlink{site-content}{Skip to
content}\protect\hyperlink{site-index}{Skip to site index}

\href{https://www.nytimes.com/section/health}{Health}

\href{https://myaccount.nytimes.com/auth/login?response_type=cookie\&client_id=vi}{}

\href{https://www.nytimes.com/section/todayspaper}{Today's Paper}

\href{/section/health}{Health}\textbar{}Contact Tracing Is Failing in
Many States. Here's Why.

\url{https://nyti.ms/33cvxqQ}

\begin{itemize}
\item
\item
\item
\item
\item
\item
\end{itemize}

\href{https://www.nytimes.com/news-event/coronavirus?action=click\&pgtype=Article\&state=default\&region=TOP_BANNER\&context=storylines_menu}{The
Coronavirus Outbreak}

\begin{itemize}
\tightlist
\item
  live\href{https://www.nytimes.com/2020/08/01/world/coronavirus-covid-19.html?action=click\&pgtype=Article\&state=default\&region=TOP_BANNER\&context=storylines_menu}{Latest
  Updates}
\item
  \href{https://www.nytimes.com/interactive/2020/us/coronavirus-us-cases.html?action=click\&pgtype=Article\&state=default\&region=TOP_BANNER\&context=storylines_menu}{Maps
  and Cases}
\item
  \href{https://www.nytimes.com/interactive/2020/science/coronavirus-vaccine-tracker.html?action=click\&pgtype=Article\&state=default\&region=TOP_BANNER\&context=storylines_menu}{Vaccine
  Tracker}
\item
  \href{https://www.nytimes.com/interactive/2020/07/29/us/schools-reopening-coronavirus.html?action=click\&pgtype=Article\&state=default\&region=TOP_BANNER\&context=storylines_menu}{What
  School May Look Like}
\item
  \href{https://www.nytimes.com/live/2020/07/31/business/stock-market-today-coronavirus?action=click\&pgtype=Article\&state=default\&region=TOP_BANNER\&context=storylines_menu}{Economy}
\end{itemize}

Advertisement

\protect\hyperlink{after-top}{Continue reading the main story}

Supported by

\protect\hyperlink{after-sponsor}{Continue reading the main story}

\hypertarget{contact-tracing-is-failing-in-many-states-heres-why}{%
\section{Contact Tracing Is Failing in Many States. Here's
Why.}\label{contact-tracing-is-failing-in-many-states-heres-why}}

Inadequate testing and protracted delays in producing results have
crippled tracking and hampered efforts to contain major outbreaks.

\includegraphics{https://static01.nyt.com/images/2020/07/28/science/28VIRUS-TRACE3/merlin_173957862_3bc4680a-bae0-43db-bf6a-1888333bef2b-articleLarge.jpg?quality=75\&auto=webp\&disable=upscale}

By \href{https://www.nytimes.com/by/jennifer-steinhauer}{Jennifer
Steinhauer} and \href{https://www.nytimes.com/by/abby-goodnough}{Abby
Goodnough}

\begin{itemize}
\item
  July 31, 2020
\item
  \begin{itemize}
  \item
  \item
  \item
  \item
  \item
  \item
  \end{itemize}
\end{itemize}

In Arizona's most populated region, the coronavirus is so
\href{https://www.azfamily.com/news/continuing_coverage/coronavirus_coverage/contact-tracing-important-but-less-useful-with-spiking-cases-maricopa-county-says/article_57d55328-bb4b-11ea-8718-8b1cf4ab4137.html}{ubiquitous}
that contact tracers have been unable to reach a fraction of those
infected.

In Austin, Texas, the story is much the same. Just as it is in North
Carolina, where the state's health secretary recently told state
lawmakers that its tracking program was hiring outside workers to
\href{https://www.ncdhhs.gov/news/press-releases/ncdhhs-selects-first-vendors-expand-testing-and-contact-tracing-covid-19}{keep
up} with a steady rise in cases, as
\href{https://www.nashp.org/state-approaches-to-contact-tracing-covid-19/}{a
number of other states} have done.

Cities in Florida, another state where Covid-19 cases are surging, have
largely\href{https://www.nbcmiami.com/news/local/miami-beach-mayor-urges-desantis-to-address-failures-of-floridas-contact-tracing-program/2268324/}{given
up on tracking
cases}.\href{https://www.washingtonpost.com/national/coronavirus-ravaged-florida-as-ron-desantis-sidelined-scientists-and-followed-trump/2020/07/25/0b8008da-c648-11ea-b037-f9711f89ee46_story.htmlhttps://calmatters.org/health/coronavirus/2020/07/california-covid-contact-tracers-video-los-angeles/}{Things
are equally dismal in California.} And in
\href{https://www.nytimes.com/2020/07/29/nyregion/new-york-contact-tracing.html}{New
York City's tracing program}, workers complained of crippling
communication and training problems.

Contact tracing, a cornerstone of the public health arsenal to tamp down
the coronavirus across the world, has largely failed in the United
States; the virus's pervasiveness and major lags in testing have
rendered the system almost pointless. In some regions, large swaths of
the population have refused to participate or cannot even be located,
further hampering health care workers.

``We are not doing it to the level or extent that it should be done,''
said Steve Adler, the mayor of Austin, echoing the view of many state
and city leaders. ``There are three main reasons. One is the sheer
number of people, the second is the delay in getting test results back,
the third is the wide community spread of the disease.''

The
\href{https://www.cdc.gov/coronavirus/2019-ncov/php/contact-tracing/contact-tracing-plan/contact-tracing.html}{goal}of
contact tracing for Covid-19 is to reach people who have spent more than
15 minutes within six feet of an infected person and ask them to
quarantine at home voluntarily for two weeks even if they test negative,
monitoring themselves for symptoms during that time. But few places have
reported systemic success. And from the very beginning of the U.S.
epidemic, states and cities have struggled to detect the prevalence of
the virus because of spotty and sometimes rationed diagnostic testing
and long delays in getting results.

``I think it's easy to say contact tracing is broken,'' said Carolyn
Cannuscio, an expert on the method and an associate professor of family
medicine and community health at the University of Pennsylvania. ``It is
broken because so many parts of our prevention system are broken.''

Tracking those exposed is so far behind the virus raging in most places
that many public health officials believe the money and personnel
involved would be better spent on other resources, like increasing test
sites, helping schools prepare for reopening and educating the public
about mask wearing. Some public health experts now believe that, at the
very least, testing and contact tracing need to be scaled back in places
with major outbreaks. In some places, they say the effort may never
succeed.

``Contact tracing is the wrong tool for the wrong job at the wrong
time,'' said Dr. David Lakey, the former state health commissioner of
Texas who helped oversee the Ebola response in Dallas in 2014.

``Back when you had 10 cases here in Texas, it might have been useful,''
said Dr. Lakey, who is now the chief medical officer for the University
of Texas System. ``But if you don't have rapid testing, it is going to
be very difficult in a disease with 40 percent of people asymptomatic.
It is hard to see the benefit of it right now.''

\hypertarget{latest-updates-global-coronavirus-outbreak}{%
\section{\texorpdfstring{\href{https://www.nytimes.com/2020/08/01/world/coronavirus-covid-19.html?action=click\&pgtype=Article\&state=default\&region=MAIN_CONTENT_1\&context=storylines_live_updates}{Latest
Updates: Global Coronavirus
Outbreak}}{Latest Updates: Global Coronavirus Outbreak}}\label{latest-updates-global-coronavirus-outbreak}}

Updated 2020-08-02T06:58:18.835Z

\begin{itemize}
\tightlist
\item
  \href{https://www.nytimes.com/2020/08/01/world/coronavirus-covid-19.html?action=click\&pgtype=Article\&state=default\&region=MAIN_CONTENT_1\&context=storylines_live_updates\#link-34047410}{The
  U.S. reels as July cases more than double the total of any other
  month.}
\item
  \href{https://www.nytimes.com/2020/08/01/world/coronavirus-covid-19.html?action=click\&pgtype=Article\&state=default\&region=MAIN_CONTENT_1\&context=storylines_live_updates\#link-780ec966}{Top
  U.S. officials work to break an impasse over the federal jobless
  benefit.}
\item
  \href{https://www.nytimes.com/2020/08/01/world/coronavirus-covid-19.html?action=click\&pgtype=Article\&state=default\&region=MAIN_CONTENT_1\&context=storylines_live_updates\#link-2bc8948}{Its
  outbreak untamed, Melbourne goes into even greater lockdown.}
\end{itemize}

\href{https://www.nytimes.com/2020/08/01/world/coronavirus-covid-19.html?action=click\&pgtype=Article\&state=default\&region=MAIN_CONTENT_1\&context=storylines_live_updates}{See
more updates}

More live coverage:
\href{https://www.nytimes.com/live/2020/07/31/business/stock-market-today-coronavirus?action=click\&pgtype=Article\&state=default\&region=MAIN_CONTENT_1\&context=storylines_live_updates}{Markets}

Dr. Thomas R. Frieden, a former director of the C.D.C. who is a strong
advocate for robust contact tracing programs, largely agreed that it is
impossible to do meaningful or substantial contact tracing with huge
numbers of cases. He noted that when testing results lag as much as they
have, it becomes almost impossible to keep up with the high volume of
infected individuals and those who have been in contact with them.

``At some point when your cases are very high, you have to dial back
your testing and contact tracing,'' said Dr. Frieden, who now runs
Resolve to Save Lives, a nonprofit health advocacy initiative. ``We may
be in that situation in some parts of the country today.''

\includegraphics{https://static01.nyt.com/images/2020/07/28/science/28VIRUS-TRACE2/28VIRUS-TRACE2-articleLarge.jpg?quality=75\&auto=webp\&disable=upscale}

Others argue that contact tracing efforts around the country are still
nascent, and many workers fanning out in particular zones are still too
inexperienced to call it quits. These experts contend that tracking
remains an important mechanism that can help as flare-ups continue over
the next year and beyond.

\href{https://www.centerforhealthsecurity.org/our-people/C\%20Watson/}{Crystal
Watson}, a risk-assessment specialist at the Center for Health Security
at the Johns Hopkins Bloomberg School of Public Health, said she had
hoped more contact tracers would be trained and in place before states
started reopening. For now, she expects it to be feasible only in
Massachusetts, New York, North Dakota and the District of Columbia.
Massachusetts, where the nonprofit group Partners in Health leads the
efforts, has done particularly well.

Contact tracing has been used as a tool for hundreds of years to contain
diseases like
\href{https://www.who.int/tb/areas-of-work/laboratory/contact-investigation/en/}{tuberculosis},
yellow fever and Ebola. A rudimentary form was even used to track
\href{https://theconversation.com/contact-tracing-how-physicians-used-it-500-years-ago-to-control-the-bubonic-plague-139248}{the
route} of a syphilis outbreak in the 16th century. Countries like
\href{https://www.forbes.com/sites/alexandrasternlicht/2020/04/30/south-koreas-widespread-testing-and-contact-tracing-lead-to-first-day-with-no-new-cases/}{South
Korea},\href{https://www.theguardian.com/world/2020/jul/20/cheap-popular-and-it-works-irelands-contact-tracing-app-success?CMP=Share_iOSApp_Other}{Ireland}
and
\href{https://www.aei.org/technology-and-innovation/a-tale-of-two-contact-tracing-apps-lessons-from-australia-and-new-zealand/}{Australia}
used the method to successfully control the spread of the coronavirus,
too.

The C.D.C. has
\href{https://www.hhs.gov/about/news/2020/05/18/hhs-delivers-funding-to-expand-testing-capacity-for-states-territories-tribes.html}{sent
about \$11 billion} in relief funds to states and local jurisdictions
for expanding coronavirus testing and contact tracing. A survey of state
health departments
\href{https://www.npr.org/sections/health-shots/2020/06/18/879787448/as-states-reopen-do-they-have-the-workforce-they-need-to-stop-coronavirus-outbre}{by
National Public Radio} last month found they had roughly 37,000 contact
tracers in place, with an additional 31,000 in reserve for when they
would be needed. The work force --- a mix of government employees,
volunteers and contract workers hired by outside companies or nonprofit
organizations --- still falls short of the 100,000 people that the
C.D.C. has recommended.

The contact tracers, whose training varies considerably in length and
content depending on what state they are in, have struggled to keep up
with the rising number of cases.

``The challenge is that we are not dealing with ones and twos,'' said
Fran Phillips, a deputy Secretary for Public Health for Maryland, a
state that has largely kept the virus in check but still faces over 900
new cases daily. For every new case, there are several if not dozens of
people to contact, especially in large cities, which further strains the
system.

Contact tracing generally works best, public health experts say, when a
disease is easily detected from its onset. That is often impossible with
the coronavirus because a large percentage of those infected have no
symptoms.

``When you have a situation in which there are so many people who are
asymptomatic,'' said Dr. Anthony Fauci, the director of the National
Institute of Allergy and Infectious Diseases, at a recent Milken
Institute event, ``that makes that that much more difficult, which is
the reason you wanted to get it from the beginning and nip it in the
bud. Once you get what they call the logarithmic increase, then it
becomes very difficult to do contact tracing. It's not going well.''

Perhaps most harmful to the effort have been the persistent delays in
getting the results of diagnostic tests. Often by the time an individual
tests positive, it's too late for the health care workers tracking that
person to do anything.

``It's a race against time,'' Ms. Phillips said. ``And if we have lost
days and days of infectious period because we didn't get a lab result
back, that really diminishes our ability to do contact tracing.'' In
Maryland, like many states, some labs are taking as long as nine days to
turn around results. ``We are getting some assurances from national
manufacturers this lag is short term,'' she said. ``I am not
confident.''

In contrast, when sports teams and staff of the White House test people
constantly, with fast turnarounds, contact tracing is
i\href{https://www.nbcnews.com/politics/white-house/white-house-executive-office-cafeteria-closed-after-positive-coronavirus-test-n1234662?cid=sm_npd_nn_tw_ma}{nstant
and effective.}

\href{https://www.nytimes.com/news-event/coronavirus?action=click\&pgtype=Article\&state=default\&region=MAIN_CONTENT_3\&context=storylines_faq}{}

\hypertarget{the-coronavirus-outbreak-}{%
\subsubsection{The Coronavirus Outbreak
›}\label{the-coronavirus-outbreak-}}

\hypertarget{frequently-asked-questions}{%
\paragraph{Frequently Asked
Questions}\label{frequently-asked-questions}}

Updated July 27, 2020

\begin{itemize}
\item ~
  \hypertarget{should-i-refinance-my-mortgage}{%
  \paragraph{Should I refinance my
  mortgage?}\label{should-i-refinance-my-mortgage}}

  \begin{itemize}
  \tightlist
  \item
    \href{https://www.nytimes.com/article/coronavirus-money-unemployment.html?action=click\&pgtype=Article\&state=default\&region=MAIN_CONTENT_3\&context=storylines_faq}{It
    could be a good idea,} because mortgage rates have
    \href{https://www.nytimes.com/2020/07/16/business/mortgage-rates-below-3-percent.html?action=click\&pgtype=Article\&state=default\&region=MAIN_CONTENT_3\&context=storylines_faq}{never
    been lower.} Refinancing requests have pushed mortgage applications
    to some of the highest levels since 2008, so be prepared to get in
    line. But defaults are also up, so if you're thinking about buying a
    home, be aware that some lenders have tightened their standards.
  \end{itemize}
\item ~
  \hypertarget{what-is-school-going-to-look-like-in-september}{%
  \paragraph{What is school going to look like in
  September?}\label{what-is-school-going-to-look-like-in-september}}

  \begin{itemize}
  \tightlist
  \item
    It is unlikely that many schools will return to a normal schedule
    this fall, requiring the grind of
    \href{https://www.nytimes.com/2020/06/05/us/coronavirus-education-lost-learning.html?action=click\&pgtype=Article\&state=default\&region=MAIN_CONTENT_3\&context=storylines_faq}{online
    learning},
    \href{https://www.nytimes.com/2020/05/29/us/coronavirus-child-care-centers.html?action=click\&pgtype=Article\&state=default\&region=MAIN_CONTENT_3\&context=storylines_faq}{makeshift
    child care} and
    \href{https://www.nytimes.com/2020/06/03/business/economy/coronavirus-working-women.html?action=click\&pgtype=Article\&state=default\&region=MAIN_CONTENT_3\&context=storylines_faq}{stunted
    workdays} to continue. California's two largest public school
    districts --- Los Angeles and San Diego --- said on July 13, that
    \href{https://www.nytimes.com/2020/07/13/us/lausd-san-diego-school-reopening.html?action=click\&pgtype=Article\&state=default\&region=MAIN_CONTENT_3\&context=storylines_faq}{instruction
    will be remote-only in the fall}, citing concerns that surging
    coronavirus infections in their areas pose too dire a risk for
    students and teachers. Together, the two districts enroll some
    825,000 students. They are the largest in the country so far to
    abandon plans for even a partial physical return to classrooms when
    they reopen in August. For other districts, the solution won't be an
    all-or-nothing approach.
    \href{https://bioethics.jhu.edu/research-and-outreach/projects/eschool-initiative/school-policy-tracker/}{Many
    systems}, including the nation's largest, New York City, are
    devising
    \href{https://www.nytimes.com/2020/06/26/us/coronavirus-schools-reopen-fall.html?action=click\&pgtype=Article\&state=default\&region=MAIN_CONTENT_3\&context=storylines_faq}{hybrid
    plans} that involve spending some days in classrooms and other days
    online. There's no national policy on this yet, so check with your
    municipal school system regularly to see what is happening in your
    community.
  \end{itemize}
\item ~
  \hypertarget{is-the-coronavirus-airborne}{%
  \paragraph{Is the coronavirus
  airborne?}\label{is-the-coronavirus-airborne}}

  \begin{itemize}
  \tightlist
  \item
    The coronavirus
    \href{https://www.nytimes.com/2020/07/04/health/239-experts-with-one-big-claim-the-coronavirus-is-airborne.html?action=click\&pgtype=Article\&state=default\&region=MAIN_CONTENT_3\&context=storylines_faq}{can
    stay aloft for hours in tiny droplets in stagnant air}, infecting
    people as they inhale, mounting scientific evidence suggests. This
    risk is highest in crowded indoor spaces with poor ventilation, and
    may help explain super-spreading events reported in meatpacking
    plants, churches and restaurants.
    \href{https://www.nytimes.com/2020/07/06/health/coronavirus-airborne-aerosols.html?action=click\&pgtype=Article\&state=default\&region=MAIN_CONTENT_3\&context=storylines_faq}{It's
    unclear how often the virus is spread} via these tiny droplets, or
    aerosols, compared with larger droplets that are expelled when a
    sick person coughs or sneezes, or transmitted through contact with
    contaminated surfaces, said Linsey Marr, an aerosol expert at
    Virginia Tech. Aerosols are released even when a person without
    symptoms exhales, talks or sings, according to Dr. Marr and more
    than 200 other experts, who
    \href{https://academic.oup.com/cid/article/doi/10.1093/cid/ciaa939/5867798}{have
    outlined the evidence in an open letter to the World Health
    Organization}.
  \end{itemize}
\item ~
  \hypertarget{what-are-the-symptoms-of-coronavirus}{%
  \paragraph{What are the symptoms of
  coronavirus?}\label{what-are-the-symptoms-of-coronavirus}}

  \begin{itemize}
  \tightlist
  \item
    Common symptoms
    \href{https://www.nytimes.com/article/symptoms-coronavirus.html?action=click\&pgtype=Article\&state=default\&region=MAIN_CONTENT_3\&context=storylines_faq}{include
    fever, a dry cough, fatigue and difficulty breathing or shortness of
    breath.} Some of these symptoms overlap with those of the flu,
    making detection difficult, but runny noses and stuffy sinuses are
    less common.
    \href{https://www.nytimes.com/2020/04/27/health/coronavirus-symptoms-cdc.html?action=click\&pgtype=Article\&state=default\&region=MAIN_CONTENT_3\&context=storylines_faq}{The
    C.D.C. has also} added chills, muscle pain, sore throat, headache
    and a new loss of the sense of taste or smell as symptoms to look
    out for. Most people fall ill five to seven days after exposure, but
    symptoms may appear in as few as two days or as many as 14 days.
  \end{itemize}
\item ~
  \hypertarget{does-asymptomatic-transmission-of-covid-19-happen}{%
  \paragraph{Does asymptomatic transmission of Covid-19
  happen?}\label{does-asymptomatic-transmission-of-covid-19-happen}}

  \begin{itemize}
  \tightlist
  \item
    So far, the evidence seems to show it does. A widely cited
    \href{https://www.nature.com/articles/s41591-020-0869-5}{paper}
    published in April suggests that people are most infectious about
    two days before the onset of coronavirus symptoms and estimated that
    44 percent of new infections were a result of transmission from
    people who were not yet showing symptoms. Recently, a top expert at
    the World Health Organization stated that transmission of the
    coronavirus by people who did not have symptoms was ``very rare,''
    \href{https://www.nytimes.com/2020/06/09/world/coronavirus-updates.html?action=click\&pgtype=Article\&state=default\&region=MAIN_CONTENT_3\&context=storylines_faq\#link-1f302e21}{but
    she later walked back that statement.}
  \end{itemize}
\end{itemize}

Even as health care workers leap over these hurdles, they are also
finding that it can be difficult not just to reach people who were
potentially exposed to the virus but to get them to cooperate. Sometimes
there is no good phone number, and in the cellphone era, unrecognized
numbers are often ignored; 25 percent of those called in Maryland don't
pick up. Others, suspicious of contact tracers or fueled by
\href{https://www.npr.org/sections/health-shots/2020/07/14/890628203/conspiracy-theories-aside-heres-what-contact-tracers-really-do}{misinformation
about them}, decline to cooperate, a stark contrast with places like
Germany where compliance with contact tracers is viewed as a civic duty.

In Florida's Miami-Dade County, contact tracers employed by the state
have reached only 18 percent of those infected over the last two weeks,
according to Mayor Dan Gelber of Miami Beach; many of the others were
never even called. Mr. Gelber
\href{https://twitter.com/CBoomerVazquez/status/1287841499422629889?ref_src=twsrc\%5Etfw\%7Ctwcamp\%5Etweetembed\%7Ctwterm\%5E1287850787830468613\%7Ctwgr\%5E\&ref_url=https\%3A\%2F\%2Fwww.local10.com\%2Fnews\%2Flocal\%2F2020\%2F07\%2F27\%2Fcoronavirus-in-miami-dade-contact-tracing-failures-and-talk-of-how-to-spend-federal-money\%2F}{wrote
a letter} to Gov. Ron DeSantis on Monday decrying the state of the
program.

``You think it's a natural situation where people will say, `Oh of
course, I'll cooperate,''' Dr. Fauci said. ``But there's such pushback
on authority, on government, on all kinds of things like that. It makes
it very complicated.''

Image

A list of confirmed coronavirus cases at the Salt Lake County health
department in May.Credit...Rick Bowmer/Associated Press

In Seattle, tracers found 80 percent of the people they reached
\href{https://komonews.com/news/coronavirus/only-1-in-5-isolating-when-covid-symptoms-develop-king-county-says}{were
not in quarantine}, even if they had symptoms. And there is little
appetite in the United States for intrusive technology, such as
electronic bracelets or obligatory phone GPS signals, that has worked
well for contact tracing in parts of Asia. Although Americans are free
to cross state lines, no national tracing program exists.

``We need federal leadership for standards and privacy safeguards, and I
don't see that happening,'' said
\href{http://leighbureau.com/speakers/lborio}{Dr. Luciana Borio}, a
former director of medical and biodefense preparedness at the National
Security Council.

Many epidemiologists believe fixing the program in the United States to
combat and contain the coronavirus outbreaks is essential.

**``**We have to start by supporting people in getting tested, which
means making it easy enough for those exposed to someone or has symptoms
to just show up and not worry about a doctor's order,'' Ms. Cannuscio
said. ``People in the Covid era have a hard time telling you what day it
is.''

Dr. Joia Mukherjee, the chief medical officer at Partners in Health, the
group in charge of the Massachusetts effort, outlined the principles her
group insisted on: Tracers must come from the hardest-hit communities
and be able to speak Spanish, Haitian Creole or whatever language the
communities do.

Every tracer must be paid, not a volunteer. And Massachusetts had to put
in enough money to let the tracers ``support'' anyone expected to
self-quarantine.

``We ask: Do you need food? Infant formula? Diapers? Cab fare?
Unemployment insurance? And we help them get it,'' Dr. Mukherjee said.
``That way people feel it's care, not surveillance.''

Dr. Marcus Plescia, the chief medical officer at the Association of
State and Territorial Health Officials, said that despite the failures
so far, it was too soon to surrender. States need more time to build up
a tracing work force and the infrastructure to do it well, he said, and
Americans need to grow more comfortable with the concept, similar to
becoming accustomed to wearing masks.

Dr. William Foege, a former director of the C.D.C.,
\href{https://www.nytimes.com/2020/05/23/sunday-review/coronavirus-contact-tracing.html}{said
recently} that effective tracers should be ``psychiatrists, detectives
and problem solvers all at once,'' and that will also take time for many
who are new to the job.

But in the meantime, Dr. Plescia said, even finding a fraction of cases
through contact tracing will help slow the virus's spread.

``We don't have to strive for perfection on this,'' Dr. Plescia said.
``It's a heavy lift and it's going to take some time. We need to hang in
there and keep at it.''

Donald G. McNeil Jr. contributed reporting to this article.

Advertisement

\protect\hyperlink{after-bottom}{Continue reading the main story}

\hypertarget{site-index}{%
\subsection{Site Index}\label{site-index}}

\hypertarget{site-information-navigation}{%
\subsection{Site Information
Navigation}\label{site-information-navigation}}

\begin{itemize}
\tightlist
\item
  \href{https://help.nytimes.com/hc/en-us/articles/115014792127-Copyright-notice}{©~2020~The
  New York Times Company}
\end{itemize}

\begin{itemize}
\tightlist
\item
  \href{https://www.nytco.com/}{NYTCo}
\item
  \href{https://help.nytimes.com/hc/en-us/articles/115015385887-Contact-Us}{Contact
  Us}
\item
  \href{https://www.nytco.com/careers/}{Work with us}
\item
  \href{https://nytmediakit.com/}{Advertise}
\item
  \href{http://www.tbrandstudio.com/}{T Brand Studio}
\item
  \href{https://www.nytimes.com/privacy/cookie-policy\#how-do-i-manage-trackers}{Your
  Ad Choices}
\item
  \href{https://www.nytimes.com/privacy}{Privacy}
\item
  \href{https://help.nytimes.com/hc/en-us/articles/115014893428-Terms-of-service}{Terms
  of Service}
\item
  \href{https://help.nytimes.com/hc/en-us/articles/115014893968-Terms-of-sale}{Terms
  of Sale}
\item
  \href{https://spiderbites.nytimes.com}{Site Map}
\item
  \href{https://help.nytimes.com/hc/en-us}{Help}
\item
  \href{https://www.nytimes.com/subscription?campaignId=37WXW}{Subscriptions}
\end{itemize}
