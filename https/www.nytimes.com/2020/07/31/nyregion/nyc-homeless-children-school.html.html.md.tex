Sections

SEARCH

\protect\hyperlink{site-content}{Skip to
content}\protect\hyperlink{site-index}{Skip to site index}

\href{https://www.nytimes.com/section/nyregion}{New York}

\href{https://myaccount.nytimes.com/auth/login?response_type=cookie\&client_id=vi}{}

\href{https://www.nytimes.com/section/todayspaper}{Today's Paper}

\href{/section/nyregion}{New York}\textbar{}How 2 New York Schools
Became Models for Coping in a Pandemic

\url{https://nyti.ms/3f9XstO}

\begin{itemize}
\item
\item
\item
\item
\item
\item
\end{itemize}

\href{https://www.nytimes.com/news-event/coronavirus?action=click\&pgtype=Article\&state=default\&region=TOP_BANNER\&context=storylines_menu}{The
Coronavirus Outbreak}

\begin{itemize}
\tightlist
\item
  live\href{https://www.nytimes.com/2020/08/01/world/coronavirus-covid-19.html?action=click\&pgtype=Article\&state=default\&region=TOP_BANNER\&context=storylines_menu}{Latest
  Updates}
\item
  \href{https://www.nytimes.com/interactive/2020/us/coronavirus-us-cases.html?action=click\&pgtype=Article\&state=default\&region=TOP_BANNER\&context=storylines_menu}{Maps
  and Cases}
\item
  \href{https://www.nytimes.com/interactive/2020/science/coronavirus-vaccine-tracker.html?action=click\&pgtype=Article\&state=default\&region=TOP_BANNER\&context=storylines_menu}{Vaccine
  Tracker}
\item
  \href{https://www.nytimes.com/interactive/2020/07/29/us/schools-reopening-coronavirus.html?action=click\&pgtype=Article\&state=default\&region=TOP_BANNER\&context=storylines_menu}{What
  School May Look Like}
\item
  \href{https://www.nytimes.com/live/2020/07/31/business/stock-market-today-coronavirus?action=click\&pgtype=Article\&state=default\&region=TOP_BANNER\&context=storylines_menu}{Economy}
\end{itemize}

Advertisement

\protect\hyperlink{after-top}{Continue reading the main story}

Supported by

\protect\hyperlink{after-sponsor}{Continue reading the main story}

\hypertarget{how-2-new-york-schools-became-models-for-coping-in-a-pandemic}{%
\section{How 2 New York Schools Became Models for Coping in a
Pandemic}\label{how-2-new-york-schools-became-models-for-coping-in-a-pandemic}}

Vulnerable children need creative solutions, educators say: ``The entire
system has missed something if they don't rethink what the fall semester
looks like.''

\includegraphics{https://static01.nyt.com/images/2020/07/22/nyregion/00nyvirus-remotelearning1/00nyvirus-remotelearning1-articleLarge.jpg?quality=75\&auto=webp\&disable=upscale}

\href{https://www.nytimes.com/by/eliza-shapiro}{\includegraphics{https://static01.nyt.com/images/2018/12/28/multimedia/author-eliza-shapiro/author-eliza-shapiro-thumbLarge.png}}

By \href{https://www.nytimes.com/by/eliza-shapiro}{Eliza Shapiro}

\begin{itemize}
\item
  July 31, 2020
\item
  \begin{itemize}
  \item
  \item
  \item
  \item
  \item
  \item
  \end{itemize}
\end{itemize}

Around 3 a.m. one day this past spring, a math teacher at Mott Haven
Academy was stirred awake by the buzz of a text message. It was from a
student, who wrote that his mother had just died from the coronavirus.

The Bronx charter school, still in the process of switching to remote
learning, sprang into action. The child's teachers huddled to plan both
for his grieving process and how to soothe his classmates, and the
school helped the family cover funeral costs.

Mott Haven Academy was unusually well prepared to confront a death in
the community --- and to help its students weather the coronavirus ---
because it was built to handle crises. Mott Haven, which has an
elementary and middle school, reserves most of its roughly 460 seats for
students in the child welfare system and uses an array of social
services to offer them much more than academics.

Now, schools around the nation that serve vulnerable children may have
little choice but to adopt the unconventional measures that those like
Mott Haven and Broome Street Academy, a charter high school in Manhattan
that serves about 300 similarly at-risk students, have used since the
pandemic hit.

Doctors affiliated with the two New York schools mailed students
antidepressants or birth control pills, teachers added families who
could no longer pay bills to their personal cellphone plans, and
administrators gave cash grants to parents who had lost jobs.

All of this was in addition to
\href{https://www.nytimes.com/2020/03/29/nyregion/coronavirus-new-york-schools-remote-learning.html}{ensuring
schoolwork was adapted for remote learning}, underscoring how the
challenges facing schools in September include more than just preparing
English and math lessons, and instead encompass every facet of a
student's life.

Millions of students in large cities and suburban districts will not
physically return to classrooms this fall, and New York's 1.1 million
public school students will
\href{https://www.nytimes.com/2020/07/08/nyregion/nyc-schools-reopening-plan.html}{likely
attend school in person only a few days a week}.

Black, Hispanic and low-income children who have endured both
devastating health outcomes and serious economic devastation from the
virus need creative solutions when they cannot be in school, said Bill
Baccaglini, chief executive of the New York Foundling, the social
service agency that founded Mott Haven.

``I think the entire system has missed something if they don't rethink
what the fall semester looks like,'' he said.

Mott Haven's ties to the Foundling and Broome Street's to the Door,
based in SoHo, give both schools connections to large, nonprofit social
service agencies that focus on at-risk children.

\hypertarget{latest-updates-global-coronavirus-outbreak}{%
\section{\texorpdfstring{\href{https://www.nytimes.com/2020/08/01/world/coronavirus-covid-19.html?action=click\&pgtype=Article\&state=default\&region=MAIN_CONTENT_1\&context=storylines_live_updates}{Latest
Updates: Global Coronavirus
Outbreak}}{Latest Updates: Global Coronavirus Outbreak}}\label{latest-updates-global-coronavirus-outbreak}}

Updated 2020-08-02T06:58:18.835Z

\begin{itemize}
\tightlist
\item
  \href{https://www.nytimes.com/2020/08/01/world/coronavirus-covid-19.html?action=click\&pgtype=Article\&state=default\&region=MAIN_CONTENT_1\&context=storylines_live_updates\#link-34047410}{The
  U.S. reels as July cases more than double the total of any other
  month.}
\item
  \href{https://www.nytimes.com/2020/08/01/world/coronavirus-covid-19.html?action=click\&pgtype=Article\&state=default\&region=MAIN_CONTENT_1\&context=storylines_live_updates\#link-780ec966}{Top
  U.S. officials work to break an impasse over the federal jobless
  benefit.}
\item
  \href{https://www.nytimes.com/2020/08/01/world/coronavirus-covid-19.html?action=click\&pgtype=Article\&state=default\&region=MAIN_CONTENT_1\&context=storylines_live_updates\#link-2bc8948}{Its
  outbreak untamed, Melbourne goes into even greater lockdown.}
\end{itemize}

\href{https://www.nytimes.com/2020/08/01/world/coronavirus-covid-19.html?action=click\&pgtype=Article\&state=default\&region=MAIN_CONTENT_1\&context=storylines_live_updates}{See
more updates}

More live coverage:
\href{https://www.nytimes.com/live/2020/07/31/business/stock-market-today-coronavirus?action=click\&pgtype=Article\&state=default\&region=MAIN_CONTENT_1\&context=storylines_live_updates}{Markets}

The two schools offer a slew of services for tens of thousands of
children and families each year, including physical and mental health
care, legal assistance and help finding housing.

\href{https://www.nytimes.com/2020/07/14/us/coronavirus-schools-fall.html}{Most
public schools do not have those resources} or the ability to add new
services, especially with the significant budget cuts prompted by the
coronavirus crisis.

But schools that serve predominantly low-income children, long tasked
with providing nonacademic help, will have little choice but to redouble
that work whenever a profoundly traumatized class of students reports
back to school.

Melissa Silberman, the principal of Broome Street, started thinking
about how to protect her students and their families weeks before
schools were shuttered.

\includegraphics{https://static01.nyt.com/images/2020/07/22/nyregion/00nyvirus-remotelearning3/00nyvirus-remotelearning3-articleLarge.jpg?quality=75\&auto=webp\&disable=upscale}

``We have young people who have a high degree of trauma; even the very
act of shutting abruptly is traumatic for them since they see Broome
Street as a home,'' she said. Many of Broome Street's students came from
some of the worst-performing middle schools in the city.

In March, teachers at the school showed students how to log on to their
online learning portal, had them save their social workers' cellphone
numbers and created an email hotline for questions from families. Broome
Street distributed over 100 **** laptops for the many students who did
not have computers at home.

``Because we are a school built around a high degree of intervention, we
were able to pivot in a way that has just been shocking,'' Ms. Silberman
said.

Still, Broome Street's staff and students have continued to face
enormous challenges.

Shortly after protests against police brutality erupted in New York
City, a student reached out to the guidance staff, asking for an
emergency appointment. The teenager had recently lost a family member to
the coronavirus and had just left a protest where her brother was
pepper-sprayed by the police.

The school was able to schedule a remote therapy session quickly because
it has five social workers for its roughly 350 students.

In contrast, in 2018, the city's traditional public schools
\href{https://www.nydailynews.com/opinion/ny-oped-the-help-nyc-schoolkids-need-20200213-pii6zdvyebhhvoladadyacb2gi-story.html}{had
an average of one guidance counselor} for roughly every 380 students,
and one social worker for every 716 students,
\href{https://infohub.nyced.org/docs/default-source/default-document-library/guidance-counselor-report-and-summary-feb-2019.pdf}{according
to a city Department of Education report}.

About a third of the students at Broome Street receive weekly counseling
--- including a transgender teenager in a group home who has to use a
pay phone in a hallway for therapy sessions during the pandemic. Every
student at the school received a weekly wellness check-in via email in
the spring.

Aniyah Stoves, a recent graduate, relied on those meetings. Her stress
mounted after she lost a job she loved at the Guggenheim Museum when the
city shut down, and younger siblings running around the house made it
hard to focus on schoolwork. In the spring, she said, ``I got
overwhelmed and felt like I was bearing too much.''

Since 2014, Mayor Bill de Blasio has directed over 100 low-income
schools to add more social services. That started when the city's
economy was booming, but he cut about \$9 million for those so-called
community schools in the latest city budget. That move, activists say,
shows just how unprepared the city is to support its neediest children.

``You have this massive number of kids who will fall through the cracks
because the city didn't coordinate services for them,'' said Eric
Weingartner, the chief executive of the Door.

Mr. Weingartner said Broome Street was able to prevent calamity for its
students and families in the spring because it offered such a variety of
services.

``Whatever crisis kids are going through,'' he said, ``it's never just
one thing.''

Image

Aniyah Stoves said she came to rely on weekly check-ins with a school
therapist throughout New York's lockdown.Credit...Amr Alfiky/The New
York Times

That is why Jessica Nauiokas, Mott Haven's principal, spent the first
few weeks of the lockdown figuring out how to get cash and food into the
homes of the school's most vulnerable children, so that they could
eventually focus on their studies.

Jennifer Curet, a single mother of five who lives in the South Bronx,
said the school helped her manage some of the most difficult months of
her life.

\href{https://www.nytimes.com/news-event/coronavirus?action=click\&pgtype=Article\&state=default\&region=MAIN_CONTENT_3\&context=storylines_faq}{}

\hypertarget{the-coronavirus-outbreak-}{%
\subsubsection{The Coronavirus Outbreak
›}\label{the-coronavirus-outbreak-}}

\hypertarget{frequently-asked-questions}{%
\paragraph{Frequently Asked
Questions}\label{frequently-asked-questions}}

Updated July 27, 2020

\begin{itemize}
\item ~
  \hypertarget{should-i-refinance-my-mortgage}{%
  \paragraph{Should I refinance my
  mortgage?}\label{should-i-refinance-my-mortgage}}

  \begin{itemize}
  \tightlist
  \item
    \href{https://www.nytimes.com/article/coronavirus-money-unemployment.html?action=click\&pgtype=Article\&state=default\&region=MAIN_CONTENT_3\&context=storylines_faq}{It
    could be a good idea,} because mortgage rates have
    \href{https://www.nytimes.com/2020/07/16/business/mortgage-rates-below-3-percent.html?action=click\&pgtype=Article\&state=default\&region=MAIN_CONTENT_3\&context=storylines_faq}{never
    been lower.} Refinancing requests have pushed mortgage applications
    to some of the highest levels since 2008, so be prepared to get in
    line. But defaults are also up, so if you're thinking about buying a
    home, be aware that some lenders have tightened their standards.
  \end{itemize}
\item ~
  \hypertarget{what-is-school-going-to-look-like-in-september}{%
  \paragraph{What is school going to look like in
  September?}\label{what-is-school-going-to-look-like-in-september}}

  \begin{itemize}
  \tightlist
  \item
    It is unlikely that many schools will return to a normal schedule
    this fall, requiring the grind of
    \href{https://www.nytimes.com/2020/06/05/us/coronavirus-education-lost-learning.html?action=click\&pgtype=Article\&state=default\&region=MAIN_CONTENT_3\&context=storylines_faq}{online
    learning},
    \href{https://www.nytimes.com/2020/05/29/us/coronavirus-child-care-centers.html?action=click\&pgtype=Article\&state=default\&region=MAIN_CONTENT_3\&context=storylines_faq}{makeshift
    child care} and
    \href{https://www.nytimes.com/2020/06/03/business/economy/coronavirus-working-women.html?action=click\&pgtype=Article\&state=default\&region=MAIN_CONTENT_3\&context=storylines_faq}{stunted
    workdays} to continue. California's two largest public school
    districts --- Los Angeles and San Diego --- said on July 13, that
    \href{https://www.nytimes.com/2020/07/13/us/lausd-san-diego-school-reopening.html?action=click\&pgtype=Article\&state=default\&region=MAIN_CONTENT_3\&context=storylines_faq}{instruction
    will be remote-only in the fall}, citing concerns that surging
    coronavirus infections in their areas pose too dire a risk for
    students and teachers. Together, the two districts enroll some
    825,000 students. They are the largest in the country so far to
    abandon plans for even a partial physical return to classrooms when
    they reopen in August. For other districts, the solution won't be an
    all-or-nothing approach.
    \href{https://bioethics.jhu.edu/research-and-outreach/projects/eschool-initiative/school-policy-tracker/}{Many
    systems}, including the nation's largest, New York City, are
    devising
    \href{https://www.nytimes.com/2020/06/26/us/coronavirus-schools-reopen-fall.html?action=click\&pgtype=Article\&state=default\&region=MAIN_CONTENT_3\&context=storylines_faq}{hybrid
    plans} that involve spending some days in classrooms and other days
    online. There's no national policy on this yet, so check with your
    municipal school system regularly to see what is happening in your
    community.
  \end{itemize}
\item ~
  \hypertarget{is-the-coronavirus-airborne}{%
  \paragraph{Is the coronavirus
  airborne?}\label{is-the-coronavirus-airborne}}

  \begin{itemize}
  \tightlist
  \item
    The coronavirus
    \href{https://www.nytimes.com/2020/07/04/health/239-experts-with-one-big-claim-the-coronavirus-is-airborne.html?action=click\&pgtype=Article\&state=default\&region=MAIN_CONTENT_3\&context=storylines_faq}{can
    stay aloft for hours in tiny droplets in stagnant air}, infecting
    people as they inhale, mounting scientific evidence suggests. This
    risk is highest in crowded indoor spaces with poor ventilation, and
    may help explain super-spreading events reported in meatpacking
    plants, churches and restaurants.
    \href{https://www.nytimes.com/2020/07/06/health/coronavirus-airborne-aerosols.html?action=click\&pgtype=Article\&state=default\&region=MAIN_CONTENT_3\&context=storylines_faq}{It's
    unclear how often the virus is spread} via these tiny droplets, or
    aerosols, compared with larger droplets that are expelled when a
    sick person coughs or sneezes, or transmitted through contact with
    contaminated surfaces, said Linsey Marr, an aerosol expert at
    Virginia Tech. Aerosols are released even when a person without
    symptoms exhales, talks or sings, according to Dr. Marr and more
    than 200 other experts, who
    \href{https://academic.oup.com/cid/article/doi/10.1093/cid/ciaa939/5867798}{have
    outlined the evidence in an open letter to the World Health
    Organization}.
  \end{itemize}
\item ~
  \hypertarget{what-are-the-symptoms-of-coronavirus}{%
  \paragraph{What are the symptoms of
  coronavirus?}\label{what-are-the-symptoms-of-coronavirus}}

  \begin{itemize}
  \tightlist
  \item
    Common symptoms
    \href{https://www.nytimes.com/article/symptoms-coronavirus.html?action=click\&pgtype=Article\&state=default\&region=MAIN_CONTENT_3\&context=storylines_faq}{include
    fever, a dry cough, fatigue and difficulty breathing or shortness of
    breath.} Some of these symptoms overlap with those of the flu,
    making detection difficult, but runny noses and stuffy sinuses are
    less common.
    \href{https://www.nytimes.com/2020/04/27/health/coronavirus-symptoms-cdc.html?action=click\&pgtype=Article\&state=default\&region=MAIN_CONTENT_3\&context=storylines_faq}{The
    C.D.C. has also} added chills, muscle pain, sore throat, headache
    and a new loss of the sense of taste or smell as symptoms to look
    out for. Most people fall ill five to seven days after exposure, but
    symptoms may appear in as few as two days or as many as 14 days.
  \end{itemize}
\item ~
  \hypertarget{does-asymptomatic-transmission-of-covid-19-happen}{%
  \paragraph{Does asymptomatic transmission of Covid-19
  happen?}\label{does-asymptomatic-transmission-of-covid-19-happen}}

  \begin{itemize}
  \tightlist
  \item
    So far, the evidence seems to show it does. A widely cited
    \href{https://www.nature.com/articles/s41591-020-0869-5}{paper}
    published in April suggests that people are most infectious about
    two days before the onset of coronavirus symptoms and estimated that
    44 percent of new infections were a result of transmission from
    people who were not yet showing symptoms. Recently, a top expert at
    the World Health Organization stated that transmission of the
    coronavirus by people who did not have symptoms was ``very rare,''
    \href{https://www.nytimes.com/2020/06/09/world/coronavirus-updates.html?action=click\&pgtype=Article\&state=default\&region=MAIN_CONTENT_3\&context=storylines_faq\#link-1f302e21}{but
    she later walked back that statement.}
  \end{itemize}
\end{itemize}

``There were a lot of times I wanted to give up because it was too much
for me,'' Ms. Curet said. Had it not been for Mott Haven, ``I probably
wouldn't have home-schooled. I probably would have skipped it.''

A social worker helped make school schedules for her four children who
attend Mott Haven, which she then pasted on the walls of her apartment.
Ms. Curet said she received multiple Amazon packages a week filled with
crayons, construction paper and other supplies to keep her children
busy, and had frequent calls with the guidance counselor to talk about
her own anxiety and how to decompress.

``Every day was like Christmas,'' she said.

Ms. Nauiokas said dozens of single parents needed help with grocery
shopping, so school staff started making supermarket runs for those
families.

One family that was not eligible for governmental help because of its
immigration status had to close its small hair salon after the virus
hit. The school gave the family a cash grant for rent and sent gift
cards.

Ms. Nauiokas said the trauma and disruption of the pandemic had created
an opportunity to improve public schools drastically for students who
needed them to be their anchor. ``It would be a shame if we went back to
the same status-quo system that was underserving kids of color,'' she
said.

The Door, which houses Broome Street Academy and provides services for
its students and roughly 11,000 other children each year, also saw its
clients' lives begin to unravel, and rushed to adapt.

The organization took over part of a Midtown Manhattan hotel and
transformed it into an isolation unit for homeless children with the
coronavirus, complete with youth counselors and medical staff. The
Door's medical team has sent scores of uninsured teenagers medicine,
eyeglasses and H.I.V. tests. A team of lawyers who handle immigration
cases met with teenagers in parks so they could sign legal documents
while maintaining social distance.

But there was still so much adults could not do for the children they
serve.

Mr. Baccaglini of the Foundling said he sometimes struggled to sleep,
thinking about children in unstable homes who could no longer rely on
their school buildings as safe zones.

``Are we worried that abuse and neglect is higher? Without a doubt,'' he
said. ``It's the thing that keeps me up at night. I know what I know,
but what I don't know is what's going to hurt.''

Mr. Baccaglini said that the Foundling would pay for tutoring for Mott
Haven's eighth-grade graduates throughout their high school careers
elsewhere, and that it was mulling other ways to help.

Jardy Santana, a fourth-grade teacher at Mott Haven, said her
experiences teaching during the pandemic so far had left little doubt
that she would need to take a drastically different approach in
September.

Image

Jardy Santana, a teacher at Mott Haven Academy, said of her approach to
working with students this fall, ``Heart first, and we can work on the
mind later.''Credit...Amr Alfiky/The New York Times

Ms. Santana said she spent months trying to keep track of one student
who would for weeks cheerfully respond to calls, only to disappear for
days. And every morning in the spring, Ms. Santana retaught the day's
lesson by phone with one student who did not have internet access.

Ms. Santana said she had received countless text messages from families,
all with a similar plea: ``I need help'' --- with home-schooling, with
food and money, with anxiety and grief.

``Our students have faced losses of family members. I can't imagine just
jumping into the school year and saying, `Let's open up this book and
get started,''' Ms. Santana said. ``There has to be a place and a time
where we process this together, how it has impacted us, how we have
changed.''

Advertisement

\protect\hyperlink{after-bottom}{Continue reading the main story}

\hypertarget{site-index}{%
\subsection{Site Index}\label{site-index}}

\hypertarget{site-information-navigation}{%
\subsection{Site Information
Navigation}\label{site-information-navigation}}

\begin{itemize}
\tightlist
\item
  \href{https://help.nytimes.com/hc/en-us/articles/115014792127-Copyright-notice}{©~2020~The
  New York Times Company}
\end{itemize}

\begin{itemize}
\tightlist
\item
  \href{https://www.nytco.com/}{NYTCo}
\item
  \href{https://help.nytimes.com/hc/en-us/articles/115015385887-Contact-Us}{Contact
  Us}
\item
  \href{https://www.nytco.com/careers/}{Work with us}
\item
  \href{https://nytmediakit.com/}{Advertise}
\item
  \href{http://www.tbrandstudio.com/}{T Brand Studio}
\item
  \href{https://www.nytimes.com/privacy/cookie-policy\#how-do-i-manage-trackers}{Your
  Ad Choices}
\item
  \href{https://www.nytimes.com/privacy}{Privacy}
\item
  \href{https://help.nytimes.com/hc/en-us/articles/115014893428-Terms-of-service}{Terms
  of Service}
\item
  \href{https://help.nytimes.com/hc/en-us/articles/115014893968-Terms-of-sale}{Terms
  of Sale}
\item
  \href{https://spiderbites.nytimes.com}{Site Map}
\item
  \href{https://help.nytimes.com/hc/en-us}{Help}
\item
  \href{https://www.nytimes.com/subscription?campaignId=37WXW}{Subscriptions}
\end{itemize}
