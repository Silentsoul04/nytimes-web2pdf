Sections

SEARCH

\protect\hyperlink{site-content}{Skip to
content}\protect\hyperlink{site-index}{Skip to site index}

\href{https://myaccount.nytimes.com/auth/login?response_type=cookie\&client_id=vi}{}

\href{https://www.nytimes.com/section/todayspaper}{Today's Paper}

\href{/section/opinion}{Opinion}\textbar{}Scared That Covid-19 Immunity
Won't Last? Don't Be

\href{https://nyti.ms/2Pb6Dj1}{https://nyti.ms/2Pb6Dj1}

\begin{itemize}
\item
\item
\item
\item
\item
\end{itemize}

Advertisement

\protect\hyperlink{after-top}{Continue reading the main story}

\href{/section/opinion}{Opinion}

Supported by

\protect\hyperlink{after-sponsor}{Continue reading the main story}

\hypertarget{scared-that-covid-19-immunity-wont-last-dont-be}{%
\section{Scared That Covid-19 Immunity Won't Last? Don't
Be}\label{scared-that-covid-19-immunity-wont-last-dont-be}}

Dropping antibody counts aren't a sign that our immune system is failing
against the coronavirus, nor an omen that we can't develop a viable
vaccine.

By Akiko Iwasaki and Ruslan Medzhitov

Dr. Iwasaki and Dr. Medzhitov are professors of immunobiology at Yale.

\begin{itemize}
\item
  July 31, 2020
\item
  \begin{itemize}
  \item
  \item
  \item
  \item
  \item
  \end{itemize}
\end{itemize}

\includegraphics{https://static01.nyt.com/images/2020/07/30/opinion/30iwasaki/30iwasaki-articleLarge.jpg?quality=75\&auto=webp\&disable=upscale}

Within the last couple of months, several scientific studies have come
out --- some
\href{https://www.nature.com/articles/s41591-020-0965-6}{peer-reviewed},
\href{https://www.medrxiv.org/content/10.1101/2020.07.09.20148429v1.full.pdf}{others
not} --- indicating that the antibody response of people infected with
SARS-CoV-2 dropped significantly within two months. The news has
\href{https://www.sfchronicle.com/health/article/With-coronavirus-antibodies-fading-fast-focus-15414533.php}{sparked
fears} that the very immunity of patients with Covid-19 may be waning
fast --- dampening hopes for the development of an effective and durable
vaccine.

But these concerns are confused and mistaken.

Both our bodies' natural immunity and immunity acquired through
vaccination serve the same function, which is to inhibit a virus and
prevent it from causing a disease. But they don't always work quite the
same way.

And so a finding that naturally occurring antibodies in some Covid-19
patients are fading doesn't actually mean very much for the likely
efficacy of vaccines under development. Science, in this case, can be
more effective than nature.

The human immune system has evolved to serve two functions: expediency
and precision. Hence, we have
\href{https://www.nature.com/articles/ni.3123}{two types of immunity}:
innate immunity, which jumps into action within hours, sometimes just
minutes, of an infection; and adaptive immunity, which develops over
days and weeks.

Almost all the cells in the human body can detect a viral infection, and
when they do, they call on our white blood cells to deploy a defensive
response against the infectious agent.

When \href{https://www.ncbi.nlm.nih.gov/pmc/articles/PMC5480291/}{our
innate immune response} is successful at containing that pathogen, the
infection is resolved quickly and, generally, without many symptoms. In
the case of more sustained infections, though, it's our adaptive immune
system that kicks in to offer us protection.

The adaptive immune system consists of two types of white blood cells,
called T and B cells, that detect molecular details specific to the
virus and, based on that, mount a targeted response to it.

A virus causes disease by entering cells in the human body and hijacking
their genetic machinery so as to reproduce itself again and again: It
turns its hosts into viral factories.

T cells detect and kill those infected cells. B cells make antibodies, a
kind of protein that binds to the viral particles and blocks them from
entering our cells; this prevents the replication of the virus and stops
the infection in its tracks.

The body then stores the T and B cells that helped eliminate the
infection, in case it might need them in the future to fight off the
same virus again. These so-called memory cells are the main agents of
long-term immunity.

The antibodies produced in response to a common seasonal coronavirus
infection
\href{https://www.ncbi.nlm.nih.gov/pmc/articles/PMC2271881/pdf/epidinfect00023-0213.pdf}{last
for about a year}. But
\href{https://www.cdc.gov/Vaccines/pubs/surv-manual/chpt07-measles.pdf}{the
antibodies generated by a measles infection} last, and provide
protection, for a lifetime.

Yet it is also the case that with other viruses the amount of antibodies
in the blood peaks during an infection and drops after the infection has
cleared, often within a few months: This is the fact that has some
people worried about Covid-19, but it doesn't mean what it might seem.

That antibodies decrease once an infection recedes isn't a sign that
they are failing: It's a normal step in the usual course of an immune
response.

Nor does a waning antibody count mean waning immunity: The memory B
cells that first produced those antibodies are still around, and
standing ready to churn out new batches of antibodies on demand.

And that is why we should be hopeful about the prospects of a vaccine
for Covid-19.

A vaccine works by mimicking a natural infection, generating memory T
and B cells that can then provide long-lasting protection in the people
who are vaccinated. Yet the immunity created by vaccines differs from
the immunity created by a natural infection in several important ways.

Virtually all viruses that infect humans contain in their genomes
blueprints for producing proteins that help them evade detection by the
innate immune system. For example, SARS-CoV-2 appears to have
\href{https://www.biorxiv.org/content/10.1101/2020.05.11.088179v1}{a
gene dedicated to silencing the innate immune system}.

Among the viruses that have become endemic in humans, some have also
figured out ways to dodge the adaptive immune system:
\href{https://www.ncbi.nlm.nih.gov/pmc/articles/PMC2877745/}{H.I.V.-1}
mutates rapidly; \href{https://jvi.asm.org/content/76/18/9232}{herpes
viruses} deploy proteins that can trap and incapacitate antibodies.

Thankfully, SARS-CoV-2 does not seem to have evolved any such tricks yet
--- suggesting that we still have an opportunity to stem its spread and
the pandemic by pursuing a relatively straightforward vaccine approach.

Vaccines come in different flavors --- they can be based on killed or
live attenuated viral material, nucleic acids or recombinant proteins.
But all vaccines consist of two main components: an antigen and an
adjuvant.

The antigen is the part of the virus we want the adaptive immune
response to react to and target. The adjuvant is an agent that mimics
the infection and helps jump-start the immune response.

One beauty of vaccines --- and one of their great advantages over our
body's natural reaction to infections --- is that their antigens can be
designed to focus the immune response on a virus's Achilles heel
(whatever that may be).

Another advantage is that vaccines allow for different kinds and
different doses of adjuvants --- and so, for calibration and fine-tuning
that can help boost and lengthen immune responses.

The immune response generated against a virus during natural infection
is, to some degree, at the mercy of the virus itself. Not so with
vaccines.

Since many viruses evade the innate immune system, natural infections
sometimes do not result in robust or long-lasting immunity. The human
papillomavirus is one of them, which is why it can cause chronic
infections.
\href{https://www.sciencedirect.com/science/article/pii/S0090825817307746\#f0005}{The
papillomavirus vaccine} triggers a far better antibody response to its
viral antigen than does a natural HPV infection: It is
\href{https://www.cdc.gov/vaccines/vpd/hpv/hcp/vaccines.html}{almost 100
percent effective} in preventing HPV infection and disease.

Not only does vaccination protect against infection and disease; it also
blocks viral transmission --- and, if sufficiently widespread, can help
confer so-called
\href{https://academic.oup.com/cid/article/52/7/911/299077}{herd
immunity} to a population.

What proportion of individuals in a given population needs to be immune
to a new virus so that the whole group is, in effect, protected depends
on the virus's basic reproduction number --- broadly speaking: the
average number of people that a single infected person will, in turn,
infect.

For measles, which is highly contagious, more than 90 percent of a
population must be immunized in order for unvaccinated individuals to
also be protected. For Covid-19, the estimated figure --- which is
unsettled, understandably --- ranges between
\href{https://science.sciencemag.org/content/early/2020/06/22/science.abc6810}{43
percent} and
\href{https://science.sciencemag.org/content/369/6500/208?ijkey=805a30207015dd1c16dd7b169019e0de4f8b8fc4\&keytype2=tf_ipsecsha}{66
percent}.

Given the severe consequences of Covid-19 for many older patients, as
well as the disease's unpredictable course and consequences for the
young, the only safe way to achieve herd immunity is through
vaccination. That, combined with the fact that SARS-CoV-2 appears not to
have yet developed a mechanism to evade detection by our adaptive immune
system, is ample reason to double down on efforts to find a vaccine
fast.

So do not be alarmed by reports about Covid-19 patients' dropping
antibody counts; those are irrelevant to the prospects of finding a
viable vaccine.

Remember instead that
\href{https://www.nytimes.com/interactive/2020/science/coronavirus-vaccine-tracker.html}{more
than 165 vaccine candidates} already are in the pipeline, some showing
\href{https://www.nytimes.com/2020/07/20/world/covid-coronavirus-vaccine.html}{promising
early trial results}.

And start thinking about how best to ensure that when that vaccine
comes, it will be distributed efficiently and equitably.

Akiko Iwasaki is the Waldemar Von Zedtwitz Professor in the Department
of Immunobiology and a Professor in the Department of Molecular,
Cellular and Developmental Biology at Yale. Ruslan Medzhitov is a
Sterling Professor in the Department of Immunobiology at Yale School of
Medicine. Both are investigators at the Howard Hughes Medical Institute.

\emph{The Times is committed to publishing}
\href{https://www.nytimes.com/2019/01/31/opinion/letters/letters-to-editor-new-york-times-women.html}{\emph{a
diversity of letters}} \emph{to the editor. We'd like to hear what you
think about this or any of our articles. Here are some}
\href{https://help.nytimes.com/hc/en-us/articles/115014925288-How-to-submit-a-letter-to-the-editor}{\emph{tips}}\emph{.
And here's our email:}
\href{mailto:letters@nytimes.com}{\emph{letters@nytimes.com}}\emph{.}

\emph{Follow The New York Times Opinion section on}
\href{https://www.facebook.com/nytopinion}{\emph{Facebook}}\emph{,}
\href{http://twitter.com/NYTOpinion}{\emph{Twitter (@NYTopinion)}}
\emph{and}
\href{https://www.instagram.com/nytopinion/}{\emph{Instagram}}\emph{.}

Advertisement

\protect\hyperlink{after-bottom}{Continue reading the main story}

\hypertarget{site-index}{%
\subsection{Site Index}\label{site-index}}

\hypertarget{site-information-navigation}{%
\subsection{Site Information
Navigation}\label{site-information-navigation}}

\begin{itemize}
\tightlist
\item
  \href{https://help.nytimes.com/hc/en-us/articles/115014792127-Copyright-notice}{©~2020~The
  New York Times Company}
\end{itemize}

\begin{itemize}
\tightlist
\item
  \href{https://www.nytco.com/}{NYTCo}
\item
  \href{https://help.nytimes.com/hc/en-us/articles/115015385887-Contact-Us}{Contact
  Us}
\item
  \href{https://www.nytco.com/careers/}{Work with us}
\item
  \href{https://nytmediakit.com/}{Advertise}
\item
  \href{http://www.tbrandstudio.com/}{T Brand Studio}
\item
  \href{https://www.nytimes.com/privacy/cookie-policy\#how-do-i-manage-trackers}{Your
  Ad Choices}
\item
  \href{https://www.nytimes.com/privacy}{Privacy}
\item
  \href{https://help.nytimes.com/hc/en-us/articles/115014893428-Terms-of-service}{Terms
  of Service}
\item
  \href{https://help.nytimes.com/hc/en-us/articles/115014893968-Terms-of-sale}{Terms
  of Sale}
\item
  \href{https://spiderbites.nytimes.com}{Site Map}
\item
  \href{https://help.nytimes.com/hc/en-us}{Help}
\item
  \href{https://www.nytimes.com/subscription?campaignId=37WXW}{Subscriptions}
\end{itemize}
