Sections

SEARCH

\protect\hyperlink{site-content}{Skip to
content}\protect\hyperlink{site-index}{Skip to site index}

\href{https://myaccount.nytimes.com/auth/login?response_type=cookie\&client_id=vi}{}

\href{https://www.nytimes.com/section/todayspaper}{Today's Paper}

\href{/section/opinion}{Opinion}\textbar{}So Trump Wants to Postpone the
Election

\url{https://nyti.ms/30hiOkQ}

\begin{itemize}
\item
\item
\item
\item
\item
\end{itemize}

Advertisement

\protect\hyperlink{after-top}{Continue reading the main story}

\href{/section/opinion}{Opinion}

Supported by

\protect\hyperlink{after-sponsor}{Continue reading the main story}

letters

\hypertarget{so-trump-wants-to-postpone-the-election}{%
\section{So Trump Wants to Postpone the
Election}\label{so-trump-wants-to-postpone-the-election}}

One reader views the floating of the idea as ``a measure of how
desperate he is to stay in power.'' Also: Maskless shoppers; uncertain
in Australia.

July 31, 2020

\begin{itemize}
\item
\item
\item
\item
\item
\end{itemize}

\hypertarget{more-from-our-inbox}{%
\subsubsection{More from our inbox:}\label{more-from-our-inbox}}

\begin{itemize}
\tightlist
\item
  \protect\hyperlink{link-dcc4adc}{When Shoppers Refuse to Wear Masks}
\item
  \protect\hyperlink{link-597dc11b}{Australia Is Calling}
\end{itemize}

\includegraphics{https://static01.nyt.com/images/2020/07/30/us/politics/30TRUMP-ANALYSIS/30TRUMP-ANALYSIS-articleLarge-v2.jpg?quality=75\&auto=webp\&disable=upscale}

\textbf{To the Editor:}

Re
``\href{https://www.nytimes.com/2020/07/30/us/politics/trump-delay-election.html}{Crises
Abound, Yet Trump Chooses to Attack Election}'' (front page, July 31):

President Trump has finally played the delay-the-election card, and that
he has is a measure of how desperate he is to stay in power. He is, as
this and other articles point out, unmoored, isolated and sinking in the
polls. However, the fact that many Republicans did not join the
Democrats in condemning him shows how strong the political class feels
he still is.

Mr. Trump and his enablers are still powerful and capable of great
mischief between now and the election. We cannot let our guard down.
Isolated and sinking or not, Mr. Trump and his cronies remain a potent
menace to society and American ideals, and must be taken seriously right
up to the end.

Tim Shaw\\
Cambridge, Mass.

\textbf{To the Editor:}

President Trump should be called out for the absurd hypocrisy he is
showing by whining about the possibility of a rigged election. If he is
seriously worried about the election being rigged against him (as
opposed to being rigged for him), there is still time to fix the
situation.

He should allot more (not less) funding to the post office to handle the
absentee ballots. And he should give the states funds so they can open
up more and safer polling places, and so they can hire more monitors to
allow people to drop off their ballots in person, perhaps even a
drive-through.

It is obvious that he doesn't want to fix the situation, but would
instead rather prepare for some way to blame away his impending defeat.

Leslie Friedman\\
Mendham, N.J.

\hypertarget{when-shoppers-refuse-to-wear-masks}{%
\subsection{When Shoppers Refuse to Wear
Masks}\label{when-shoppers-refuse-to-wear-masks}}

Image

``I would be scared to confront people,'' said Christopher Vanderpool,
18, who works at a Walmart in Fayetteville, N.C.Credit...Jeremy M. Lange
for The New York Times

\textbf{To the Editor:}

Re
``\href{https://www.nytimes.com/2020/07/29/business/coronavirus-masks-stores-walmart.html?action=click\&module=Top\%20Stories\&pgtype=Homepage}{A
Wrinkle in Stores' Mask Policies: Enforcement}'' (news article,
nytimes.com, July 29):

Working as a grocery checker, I became a witness to how our society
sometimes seemed washed of decency, civility and emotional intelligence.
Even now with the pandemic, though, retailers coddle customers who curse
at, threaten, spit on or intimidate employees. Reinforcement --- not
lack of enforcement --- is the ``wrinkle.''

I'm not surprised that we're being idiots about wearing masks, because,
among other things, Americans are resistant to change. Any grocery store
that relocates a single item can tell you that.

Some of us will laugh at customer meltdowns, but if there's a joke in a
retail worker having to confront agitated and possibly armed people,
what do we think the punchline is going to be?

If only we had some kind of organized force in our communities that was
trained in verbal de-escalation techniques that could resolve
quality-of-life issues without stripping people of their dignity.

Dominic Andres Gonzalez\\
San Antonio

\textbf{To the Editor:}

Let an anti-masker imagine being a retail worker who is (rightly or
wrongly) terrified of this sometimes fatal disease, but who needs that
job to feed a family. And then, please just put a mask on inside a
store. Take your civil-disobedience protest, about being able to breathe
freely, out of buildings where employees are trapped.

Move it to the streets and parks, where at least the people you meet ---
like me, a teacher who's worked safely from home since March --- can
step out of your way.

I also wish the private sector had more and stronger unions, and that
retail workers had a hefty strike fund available. We the people love our
freedoms, but we also love our bread, butter and beer.

Eileen Gloster\\
North Adams, Mass.

\hypertarget{australia-is-calling}{%
\subsection{Australia Is Calling}\label{australia-is-calling}}

Image

The beach in Lennox Head.Credit...Paul Bamford for The New York Times

\textbf{To the Editor:}

Australians love Americans. We've supported you, and you've supported
us, over many decades. This is changing.

Why? Because you have a leader who has no idea what leadership is and
whose erratic, illogical, disrespectful and downright ugly behavior
leaves us with uncertainty about your country's reliability as an ally.

A pity. We look forward to renewing normality pronto.

David Brettell\\
Sydney, Australia

Advertisement

\protect\hyperlink{after-bottom}{Continue reading the main story}

\hypertarget{site-index}{%
\subsection{Site Index}\label{site-index}}

\hypertarget{site-information-navigation}{%
\subsection{Site Information
Navigation}\label{site-information-navigation}}

\begin{itemize}
\tightlist
\item
  \href{https://help.nytimes.com/hc/en-us/articles/115014792127-Copyright-notice}{©~2020~The
  New York Times Company}
\end{itemize}

\begin{itemize}
\tightlist
\item
  \href{https://www.nytco.com/}{NYTCo}
\item
  \href{https://help.nytimes.com/hc/en-us/articles/115015385887-Contact-Us}{Contact
  Us}
\item
  \href{https://www.nytco.com/careers/}{Work with us}
\item
  \href{https://nytmediakit.com/}{Advertise}
\item
  \href{http://www.tbrandstudio.com/}{T Brand Studio}
\item
  \href{https://www.nytimes.com/privacy/cookie-policy\#how-do-i-manage-trackers}{Your
  Ad Choices}
\item
  \href{https://www.nytimes.com/privacy}{Privacy}
\item
  \href{https://help.nytimes.com/hc/en-us/articles/115014893428-Terms-of-service}{Terms
  of Service}
\item
  \href{https://help.nytimes.com/hc/en-us/articles/115014893968-Terms-of-sale}{Terms
  of Sale}
\item
  \href{https://spiderbites.nytimes.com}{Site Map}
\item
  \href{https://help.nytimes.com/hc/en-us}{Help}
\item
  \href{https://www.nytimes.com/subscription?campaignId=37WXW}{Subscriptions}
\end{itemize}
