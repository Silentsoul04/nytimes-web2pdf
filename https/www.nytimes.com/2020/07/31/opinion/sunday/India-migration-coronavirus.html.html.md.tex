Sections

SEARCH

\protect\hyperlink{site-content}{Skip to
content}\protect\hyperlink{site-index}{Skip to site index}

\href{/section/opinion/sunday}{Sunday Review}\textbar{}A Friendship, a
Pandemic and a Death Beside the Highway

\href{https://nyti.ms/33cvATy}{https://nyti.ms/33cvATy}

\begin{itemize}
\item
\item
\item
\item
\item
\item
\end{itemize}

\includegraphics{https://static01.nyt.com/images/2020/08/02/opinion/31peer4/merlin_174902523_117ce866-d39c-4a06-95a7-ff67086fd8b1-articleLarge.jpg?quality=75\&auto=webp\&disable=upscale}

\href{/section/opinion}{Opinion}

\hypertarget{a-friendship-a-pandemic-and-a-death-beside-the-highway}{%
\section{A Friendship, a Pandemic and a Death Beside the
Highway}\label{a-friendship-a-pandemic-and-a-death-beside-the-highway}}

How a photograph of a young man cradling his dying friend sent me on a
journey across India.

Mohammad Saiyub at the grave of his friend Amrit Kumar, in the village
where they grew up.Credit...Vivek Singh for The New York Times

Supported by

\protect\hyperlink{after-sponsor}{Continue reading the main story}

\includegraphics{https://static01.nyt.com/images/2020/07/30/opinion/Basharat-Peer/Basharat-Peer-thumbLarge.png}

By Basharat Peer

Mr. Peer is a staff editor in Opinion.

\begin{itemize}
\item
  July 31, 2020
\item
  \begin{itemize}
  \item
  \item
  \item
  \item
  \item
  \item
  \end{itemize}
\end{itemize}

DEVARI, India --- Somebody took a photograph on the side of a highway in
India.

On a clearing of baked earth, a lithe, athletic man holds his friend in
his lap. A red bag and a half empty bottle of water are at his side. The
first man is leaning over his friend like a canopy, his face is anxious
and his eyes searching his friend's face for signs of life.

The friend is small and wiry, in a light green T-shirt and a faded pair
of jeans. He is sick, and seems barely conscious. His hair is soaked and
sticking to his scalp, a sparse stubble accentuates the deathlike pallor
of his face, his eyes are closed, and his darkened lips are half parted.
The lid of the water bottle is open. His friend's cupped hand is about
to pour some water on his feverish, dehydrated lips.

I saw this photo in May, as it was traveling across Indian social media.
News stories filled in some of the details: It was taken on May 15 on
the outskirts of Kolaras, a small town in the central Indian state of
Madhya Pradesh. The two young men were childhood friends: Mohammad
Saiyub, a 22-year-old Muslim, and Amrit Kumar, a 24-year-old Dalit, a
term for those once known as ``untouchables,'' people who have suffered
the
\href{https://www.nytimes.com/2020/07/14/opinion/caste-cisco-indian-americans-discrimination.html}{greatest
violence and discrimination} under the centuries-old Hindu caste system.

\includegraphics{https://static01.nyt.com/images/2020/08/02/opinion/31peer1a/merlin_175076706_da9a390c-0f6b-4028-90da-e631a4f4d0f1-articleLarge.jpg?quality=75\&auto=webp\&disable=upscale}

Over the next few weeks, I found myself returning to that moment
preserved and isolated by the photograph. I came across some details
about their lives in the Indian press: The two came from a small village
called Devari in the northern state of Uttar Pradesh. They had been
working in Surat, a city on the west coast, and were making their way
home, part of a mass migration that began when the Indian government
ordered a national lockdown to prevent the spread of the coronavirus.
Despite our image-saturated times, the photograph began assuming greater
meanings for me.

For the past six years, since Prime Minister Narendra Modi and his Hindu
nationalist Bharatiya Janata Party took power, it has seemed as if a
veil covering India's basest impulses has been removed. The ideas of
civility, grace and tolerance were replaced by triumphalist displays of
\href{https://www.orfonline.org/research/43665-digital-hatred-real-violence-majoritarian-radicalisation-and-social-media-in-india/}{prejudice,
sexism, hate speech and abuse} directed at women, minorities and
liberals. This culture of vilification dominates India's television
networks, social media and the immensely popular
\href{https://www.wired.com/story/indias-frightening-descent-social-media-terror/}{mobile
messaging service} WhatsApp. When you do come across acts of kindness
and compassion, they seem to be documented and calibrated to serve the
gods of exhibitionism and self-promotion.

The photograph of Amrit and Saiyub came like a gentle rain from heaven
on India's hate-filled public sphere. The gift of friendship and trust
it captured filled me with a certain sadness, as it felt so rare. I felt
compelled to find out more about their lives and journeys.

\begin{center}\rule{0.5\linewidth}{\linethickness}\end{center}

\textbf{On a June morning, I left New Delhi} for Devari. The highway was
unusually empty. I passed hulking gray towers --- tens of thousands of
\href{https://www.wsj.com/articles/indias-ghost-towns-saddle-middle-class-with-debtand-broken-dreams-11579189678}{unfinished
apartments}, monuments to the broken dreams of middle-class home buyers.

The landscape morphed into a monotonous expanse of paddies and drab
small towns off the new, impressive highway. I passed an exit sign for
Aligarh, a town where I had spent five years at an old public university
in the '90s. A voice on the radio promised a glorious future to
prospective students at a new private university. I knew those
operations; they took your money and years and left you unprepared for
the world.

To travel through a landscape that played a part in shaping you is to
also travel through the layers of memories, to revisit the concerns and
debates of an earlier life. I thought of my journeys as a reporter in
the 2000s on these roads --- the debates about India's economic growth,
the comparisons of its newfound wealth and inequality to the Roaring
Twenties in the United States, the debates about
\href{http://www.sacw.net/index.php?page=imprimir_articulo\&id_article=6295}{equal
distribution of opportunity},
\href{https://caravanmagazine.in/reportage/road-back-ayodhya}{equal
citizenship} and the campaigns against the violence of the caste system.

This time of hope and aspiration gave way to an aggressive
\href{https://www.theguardian.com/books/2014/may/16/what-next-india-pankaj-mishra}{Hindu
majoritarianism and strident nationalism} with the 2014 election of Mr.
Modi. Within a few years, even his electoral promises of
\href{https://www.nytimes.com/2019/02/01/opinion/india-unemployment-jobs-blackout.html}{economic
growth} proved to be a mirage.

Image

For the past six years, since Prime Minister Narendra Modi and his Hindu
nationalist Bharatiya Janata Party took power, it has seemed as if a
veil covering India's basest impulses has been removed. Credit...Saurabh
Das/Associated Press

As the highway crossed a massive bridge over the Sarayu River and past
the paddy-green fields and stacks of dried dung cakes, I could see the
outlines of the temple town of Ayodhya, where in 1992 a Hindu mob
destroyed a 16th-century mosque because they believed it had been built
on the exact birthplace of Rama, the Hindu deity.

Mr. Modi's party campaigned for building Rama's temple on the disputed
site for decades. In November, the Supreme Court of India cleared the
way for the temple to be built there, another step toward transforming
India into a majoritarian Hindu state. Next week, Mr. Modi will lay its
foundation stone.

Along with his devotion to the Hindu nationalist project, a consistent
feature of Mr. Modi's rule has been his
\href{https://www.nytimes.com/2020/05/27/opinion/india-modi-coronavirus.html}{penchant
for dramatic policy decisions} --- on everything from
\href{https://www.nytimes.com/2019/08/15/opinion/sunday/kashmir-siege-modi.html}{Kashmir}
to
\href{https://www.nytimes.com/2016/11/27/opinion/in-india-black-money-makes-for-bad-policy.html}{currency}
--- without serious consideration of their effects.

That trait was starkly illustrated by the
\href{https://www.nytimes.com/2020/03/25/opinion/india-coronavirus-lockdown.html}{imposition
of a lockdown on March 24}, which forced factories, offices and
educational institutions to close with only four hours' notice, at a
time when India had a mere 600 coronavirus cases compared to
\href{https://www.ndtv.com/india-news/coronavirus-over-50-000-cases-in-india-in-24-hours-for-the-first-time-15-83-lakh-total-cases-so-far-over-10-lakh-recoveries-2271144}{the
1.58 million} now.

The lockdown struck India's poor like a hammer. An overwhelming majority
of workers ---
\href{https://www.magzter.com/article/Business/Forbes-India/Work-In-Progress}{more
than 92 percent} --- lead precarious lives, getting paid after each
day's work, with no written contracts or job security, no paid leave or
health care benefits. Most had left their villages to work in faraway
cities. Living in Dickensian tenements, they would remit a
\href{https://www.livemint.com/news/india/why-india-s-migrants-deserve-a-better-deal-11589818749274.html}{significant
share of their earnings} to sustain their families back home.

Within weeks of the lockdown, multitudes who had been employed at
construction sites and brick kilns, in mines and factories, in hotels
and restaurants or as street vendors couldn't pay rent or buy enough to
eat.

The only place that would offer them shelter and share what it had was
the village, the home they had left. The Indian government, seeking to
contain the spread of the virus, tried to stop them from leaving the
cities,
\href{https://www.ndtv.com/india-news/coronavirus-trains-stopped-till-march-31-no-metros-interstate-buses-to-prevent-spread-of-coronavirus-2198784}{shutting
down} trains and buses.

The poor defied the government and hundreds of thousands walked or
caught rides to their villages: the first wave of coronavirus
``refugees'' in the world. Between April and June, the images of India's
poor workers returning to their villages
\href{https://www.theguardian.com/world/2020/mar/30/india-wracked-by-greatest-exodus-since-partition-due-to-coronavirus}{evoked
comparisons} to the great migration accompanying the partition of India
in 1947. It reminded me more of John Steinbeck's ``The Grapes of Wrath''
and the farmers of Oklahoma leaving the Dust Bowl to seek a future in
California, except the Indian workers were fleeing their Californias for
their impoverished villages.

Image

Migrant workers walking to their villages after the Indian government
imposed a hasty lockdown to slow the spread of the coronavirus outbreak
in the country.Credit...Danish Siddiqui/Reuters

Among the
\href{https://indianexpress.com/article/explained/coronavirus-how-many-migrant-workers-displaced-a-range-of-estimates-6447840/}{millions
of migrant workers} who made the desperate journey home were Amrit and
Saiyub. They were trying to reach Devari, about 920 miles away. It was
Mr. Modi's decision that brought them to that patch of baked earth by
the highway.

\begin{center}\rule{0.5\linewidth}{\linethickness}\end{center}

\textbf{About an hour from Ayodhya}, I got off the highway. I met Saiyub
in a bazaar a few miles from his village and he led the way on his
scooter. Devari is a smattering of mud and brick homes amid a few miles
of sugar cane and rice fields, children loitering about, cows and
buffaloes lazing under mahua trees. A visitor can fall for the romance
of pastoral community, but an Indian village is a hard place.

The immense expanses of land in rural India might suggest plenty, but
most land holdings in Indian villages are incredibly small. The yield of
wheat, rice and mustard does not fetch enough to sustain a family
through the year. Saiyub's family owns a third of an acre, which will be
divided among three brothers when his father dies. Amrit's family owns
even less: one-twelfth of an acre.

Saiyub and I sat on plastic chairs in the courtyard of his modest home.
Three goats reclined on a charpoy, a bed woven on a frame, nearby. He
had been in fifth grade when his father, a farmer, developed a severe
back problem and couldn't work. Two of his older brothers left for
Mumbai to find work. He helped with the chores at home, attended his
school indifferently and hung out with Amrit, who lived a few minutes
away. Interfaith friendships in India are not as uncommon as the regnant
political discourse might suggest.

Image

Mohammad Saiyub grieved alone in the quarantine ward for two days,
unable to see his deceased friend.Credit...Vivek Singh for The New York
Times

Amrit was the first to go. His father, Ram Charan, had struggled to make
enough from farming and working on construction sites to raise his five
children, and could no longer bear the hard labor. So Amrit dropped out
of high school and went to Surat to find work.

Surat is a mercantile city in the state of Gujarat, close to the Arabian
Sea, an ancient port that is now a major hub for India's textile
industry and the largest diamond polishing and processing center in the
world. The city of 4.5 million people employs
\href{https://www.financialexpress.com/india-news/lakhs-of-migrant-workers-in-surat-desperate-to-return-to-their-native-states/1943533/}{hundreds
of thousands} of migrant workers. Amrit found a job in a factory
manufacturing cloth and saris.

Every year, when the factory closed for the Diwali holidays, Amrit would
come back to visit. The friends would walk about the village, Saiyub
told me. He was working construction at the time, whenever there was an
opportunity. Amrit spoke about the factory, urging his friend to move to
the city. ``I will find you a job in Surat,'' Amrit promised.

Precise numbers are hard to arrive at, but scholars of urbanization and
migration estimate that India has
\href{https://indianexpress.com/article/explained/coronavirus-india-lockdown-migran-workers-mass-exodus-6348834/}{more
than a 100 million} migrant workers. The majority come from the
impoverished northern Indian states which, like the American Rust Belt,
have suffered decades of decline. They find work in the manufacturing
and services powerhouses in western India; the national capital region,
Delhi; and, increasingly, the fast-growing states in southern India.

``Way back from the 1960s Indian government policies encouraged industry
in the western and southern areas --- India's major capitalists came
from those regions and preferred investing there,'' said Rathin Roy, one
of India's leading economists. ``Most politicians in the north were
rural folk who saw the few pockets of industry as sites for
rent-seeking.''

\begin{center}\rule{0.5\linewidth}{\linethickness}\end{center}

\textbf{For Saiyub, there were few options} other than migrating. In the
winter of 2015, he left the village with Amrit. After a 36-hour train
journey, they arrived in Surat. They rented a room together for 2,000
rupees, or about \$27, a month near Amrit's factory. A few days later,
Saiyub got a job, with Amrit's help, at a factory that produced thread.

Saiyub started his work at 7 a.m., stopped for a lunch break and
continued till 7 p.m. ``We would go home for an hour, eat dinner and
return at 8 p.m.,'' he said. He worked a four more hours, till midnight,
returning to his room to sleep for six hours before setting out for the
factory again. I was struck by the 16 hour shifts, but he brushed that
off. ``We could stop for a bit. It is not that bad.''

On his arrival in Surat, Saiyub had some apprehensions about being
Muslim and working in Gujarat, Mr. Modi's home state and the strongest
bastion of Hindu nationalism. Throughout the five years he spent there,
he read the news of attacks on Muslims in India but avoided speaking
about politics in the factory. ``Nobody bothered me,'' he said. ``I did
my job. I got paid.''

On Sundays, Amrit and Saiyub washed their clothes, walked around the
city, and watched films and news on their phones. ``Amrit bought a
speaker and we lay on our beds and listened to music,'' said Saiyub.
They made about 15,000 rupees, or \$200, a month each and wired most of
it home to their parents. Amrit's family was able to upgrade from a
shack to a one-room brick house with a veranda and he was trying to save
enough for his sister's wedding in the fall.

\begin{center}\rule{0.5\linewidth}{\linethickness}\end{center}

\textbf{On March 25, the morning after Mr. Modi} announced the lockdown,
the factory owners told the workers the factories would close. They
wouldn't be paid while the factories remained shut. Saiyub's boss gave
everyone rice and lentils and about 1,500 rupees. Amrit's boss offered
his workers rice and lentils, but no cash.

Saiyub and Amrit resigned themselves to the situation and stayed in
their room most of the time, stepping out briefly to buy food. ``We
talked a lot and watched videos on our phones,'' he said. ``Amrit spoke
a lot about his sister's wedding.''

They watched the news of the explosion of the pandemic in India. The
dispatches were grim: Workers
\href{https://www.thehindu.com/news/national/other-states/migrant-workers-in-surat-take-to-the-streets-again/article31341846.ece}{protesting
about lack of food} and demanding to be allowed to return home; police
in Surat beating and arresting protesting workers; workers walking home
in desperation; bodies of people dying of the coronavirus being tossed
into hastily dug graves; cases rising steadily despite the lockdown
being extended; and even middle-class Indians, who live in spacious
homes and can bear the cost of treatment at private hospitals, being
\href{https://www.nytimes.com/2020/06/21/world/asia/coronavirus-india-hospitals-pregnant.html}{turned
away from hospitals} lacking beds and ventilators.

Image

After the Indian government imposed a strict and sudden lockdown,
jobless migrant laborers in the city of Surat in western India gathered
in April to demand transportation to their villages.Credit...Agence
France-Presse --- Getty Images

The Indian government spends just a little
\href{https://main.mohfw.gov.in/sites/default/files/NHA_Estimates_Report_2015-16_0.pdf}{over
1 percent} of its gross domestic product on health care, one of the
lowest rates in the world. Subsidized health care benefits are also tied
to a citizen's domicile --- that is, their village --- meaning many
migrant workers couldn't use them. Treatment costs because of an illness
push
\href{https://www.indiaspend.com/wp-content/uploads/2020/06/Draft_National_Hea_2263179a.pdf}{more
than 63 million Indians} into poverty every year.

``We had to get home,'' said Saiyub.

On May 1, after intense public criticism for ignoring the migrant worker
exodus, the Indian government started operations of
\href{https://timesofindia.indiatimes.com/videos/news/covid-19-lockdown-indian-railways-starts-first-special-train-for-migrant-workers-from-hyderabad-to-jharkhand/videoshow/75487379.cms}{the
state-owned railway network} to transport workers. Amrit and Saiyub
spoke to a travel agent to help them get two seats on the trains going
to Basti or Gorakhpur, the stations closest to their village. They paid
him. Two weeks passed but they could not get a spot. The travel agent
promised to call the moment he had their seats booked.

\begin{center}\rule{0.5\linewidth}{\linethickness}\end{center}

\textbf{Fifty-one days into the lockdown}, on May 14, the two friends
were restless, running out of savings and certain that they needed to
get home somehow. Amrit met some workers from their region in Uttar
Pradesh who had negotiated with a truck driver to drive them home. They
would have to each pay 4,000 Indian rupees, or \$53. They agreed.

The truck driver would wait for the workers at a secluded spot on NH-48
road, which they would follow north. The two friends packed a bag each,
locked their room and set out at 9 p.m. They walked 15 miles through the
humid night with about 60 other workers to the designated place on the
highway and waited. The truck arrived at 2 a.m.

The workers completely filled the bed of the truck, packed together like
sheep. Twelve men were still left, Amrit and Saiyub among them. They
were asked to climb into a balcony-like space above the driver's seat.
The journey began. ``We could feel the breeze and we were going home,''
Saiyub recalled. They caught snatches of sleep while sitting cramped
together and repeated their conversations about the pandemic, the loss
of work and the solace of home.

Image

Many migrant workers and their families attempted to return to their
villages by squeezing aboard trucks like this one in Ahmedbad,
India.Credit...Amit Dave/Reuters

The morning came. The truck groaned on through Madhya Pradesh, the huge
state in central India best known outside the country as home to the
forests and wildlife parks that inspired Rudyard Kipling's ``The Jungle
Book.'' Around noon they were passing by Kolaras, when Amrit turned to
Saiyub. ``I am feeling cold,'' he said. ``I have a fever.'' Saiyub
suggested they keep an eye on the road and stop the truck when they
spotted a pharmacy. The truck droned on. Amrit was shivering, his
temperature rising. They climbed down to the bed of the truck to shield
Amrit from the wind.

A little later, cramped in a corner among about 50 other workers, Amrit
started coughing and sweating. His fellow passengers were alarmed and
cries of protest rose: ``He is coughing. He has a fever. He has
corona.'' The voices turned angrier: ``We are running home to save
ourselves from corona.'' ``He will infect us all.'' ``We don't want to
die because of him.''

The driver stopped the truck. The passengers and the driver insisted
that Amrit get off. Saiyub asked the driver to stop at a hospital. The
driver and the workers were uncertain about the lockdown rules and
weren't ready to lose any time for Amrit. They refused and insisted
Amrit get off right there.

``Let him go. You should come home with us,'' the driver told Saiyub.

``I couldn't let Amrit be alone,'' he said. Saiyub picked up their bags
and helped Amrit off the truck.

\begin{center}\rule{0.5\linewidth}{\linethickness}\end{center}

\textbf{A blinding 109-degree afternoon sun baked the road}, the fields,
the trees in the distance. They sat in the clearing by the highway.
Scores of workers went past, following the highway toward their homes. A
politician arrived with a few cars and distributed food and water.
Saiyub rushed and collected a few bottles of water. Amrit babbled
incoherently; his temperature rose. ``I was holding him and he was
burning,'' Saiyub recalled. He poured water over Amrit's head but his
body wasn't cooling down.

Saiyub asked the politician to call an ambulance. As he waited, he
cradled Amrit in his lap, wiping his forehead with a wet handkerchief
and pouring handfuls of water on his lips. In that moment, somebody took
a photograph of the two friends.

An ambulance arrived and drove them to a small hospital in Kolaras. A
doctor found that Amrit had low blood sugar and a high temperature and
feared he had suffered a heat stroke. He tried
\href{https://indianexpress.com/article/india/lockdown-migrant-labourers-friends-deaths-coronavirus-lockdown-6414810/}{oral
rehydration therapy} to revive Amrit, whose consciousness was fading. A
few hours later, Amrit was transferred to a better-equipped hospital in
Shivpuri, a town about 15 miles away, where doctors diagnosed severe
dehydration and moved him into the intensive care unit.

He called Amrit's father. In the village, the news of his son's collapse
shook Ram Charan. He conferred with his family and set out for Basti,
the town where the government officials who administer the district were
based. The coronavirus lockdown in Uttar Pradesh forbade people from
traveling without official permission. Ram Charan requested from
officials a pass that would allow him to travel to the hospital in
Shivpuri to see his son. They turned him away.

Image

For Amrit's parents, the future is uncertain. ``He was all we had,''
said his father, Ram Charan.Credit...Vivek Singh for The New York Times

Saiyub stayed with Amrit in the I.C.U. The doctors tested the two
friends for coronavirus, sent their samples to a laboratory and put
Amrit on a ventilator. In the evening, they moved Saiyub to a quarantine
ward. ``I was not allowed to leave the quarantine ward and see Amrit
till our corona results would come,'' he said.

Sleep eluded Saiyub and nightmarish scenarios haunted him: He thought of
the reports of strangers burying the bodies of coronavirus victims,
tossing them into impromptu graves dug by backhoes. If Amrit died in the
hospital, how would he take his body home? How would he face Amrit's
parents, who had no financial support beyond their son's earnings?

``Around 3 in the morning, I felt terribly sad,'' Saiyub recalled. ``I
felt that Amrit, my friend, my brother, was not in this world anymore.''

In the morning, on May 16, a nurse came to the quarantine ward and
confirmed his fear. Amrit had died of severe dehydration. A doctor asked
Saiyub to inform Amrit's relatives of his death and have them collect
his body. ``His family can't come here,'' he replied. ``I will take him
home.''

The doctors moved Amrit's body to the hospital morgue, where it would
have to wait till the results of their coronavirus tests arrived. Saiyub
grieved alone in the quarantine ward for two days, unable to see his
deceased friend. He received several calls from officials who
administered Shivpuri, the district where the hospital was located.

The officials in Amrit and Saiyub's home district had made it clear to
the Shivpuri officials that they would not allow Amrit's body into
Devari if he tested positive for the coronavirus. They had urged them to
cremate him in Shivpuri itself.

For two days, Saiyub repeated a single prayer: ``Ya Allah! When the
results arrive let me and Amrit test negative for corona.''

\begin{center}\rule{0.5\linewidth}{\linethickness}\end{center}

\textbf{On the afternoon of May 18}, the reports came from a laboratory:
Both the friends had tested negative. In the evening, after a few hours
of paperwork, Saiyub was allowed to return home with Amrit's body. An
ambulance was ready. ``The freezer they had kept him had not been
working,'' Saiyub recalled. Amrit's body had turned black; his skin and
flesh were peeling off. ``He was already smelling.''

As Saiyub sat in the ambulance carrying him and the body to Devari, he
feared Amrit's parents wouldn't be able to bear the sight of their son's
corpse. ``I called his father. He agreed that I should take him straight
to the graveyard in the village.'' Most Hindus cremate their deceased
family members but some Dalits like Amrit's family
\href{https://www.reuters.com/article/us-india-landrights-caste-trfn/denied-in-life-indias-lower-caste-dalits-fight-for-land-in-death-idUSKBN20T0T1}{bury
their dead}.

The ambulance drove on. Saiyub ignored the numerous calls he was getting
from friends and family in the village and stayed in silence beside his
friend throughout the nightlong journey. About half a mile from Amrit's
home in Devari, the Dalit graveyard is a single acre of land lush with
wild grass and shaded by mahua trees. Amrit was buried there. The plain
brown mound of earth about six feet long and three feet wide has no
tombstone.

Saiyub walked home from the graveyard. A little later, his phone rang.
The travel agent from Surat was on the line. ``I have got tickets for
Amrit and you,'' he said. ``The train for your village leaves
tomorrow.''

\begin{center}\rule{0.5\linewidth}{\linethickness}\end{center}

\textbf{Five weeks had passed since they buried Amrit} when I met Saiyub
in the village. He was living with his parents, surviving off their
meager savings. There was no work in the village for him. He worried
more about the fate of Amrit's family: his parents, his four teenage
sisters, his 12-year-old brother.

The home Amrit had helped build with his remittances is a small
rectangle of brick walls: two rooms and a raised platform open to the
elements. A buffalo and a cow were tied to their pegs beside the house.
A few bales of cotton were stacked outside the bedroom; his mother and
sisters turn them into quilts for a vendor. Twigs of brushwood lie
around a mud oven used for cooking.

Image

Amrit Kumar's photograph from the Diwali holidays in 2016, hanging on a
wall in his parents' house.Credit...Vivek Singh for The New York Times

The sole adornment was a framed photograph of Amrit on a wall, a picture
taken during the festival of Diwali in the winter of 2016. He is posing
in a photo studio against the backdrop of a landscaped garden by water.
His eyes are bright, purposeful against his boyish face. His polka dot
shirt, his drainpipe denims, a smartphone daintily held in his right
hand are a statement of confidence and social mobility. His years of
toil in a faraway city had helped the poor young man earn a modicum of
freedom from the poverty, humiliation and violence that shadows every
Dalit body in the village.

Amrit's loss had left Ram Charan, his father, a shrunken shell of a man.
He spoke in monosyllables, struggling with his words. His eyes were
stony, coming alive with occasional flashes of anger and grief at the
hand fate and follies of powerful men he would never meet had dealt him.
His daughter's wedding was deferred. The villagers were talking about
pooling resources to help out.

Ram Charan gets between 30 to 40 days of work a year through a
\href{https://caravanmagazine.in/reportage/nregas-reality-check}{public
works program}. Since the pandemic began, he has found three days of
work overseeing laborers cleaning an irrigation canal in the village,
making 202 rupees, or about \$2.70, a day. The future seems uncertain
after Amrit's death. ``He was all we had. He kept our family going,''
Ram Charan said. ``He is not here anymore.''

A narrow muddy path led out of the village, to the town, to the highway,
to the cities. Saiyub and I walked together a while. The factory owner
in Surat had called the day before. Some of the workers were already
back. He wanted Saiyub to return.

``I have to go back. In a month, maybe two,'' he said. ``Not right now.
The heart is not ready yet.''

Image

A road leading out of Devari. The lack of work forces young men from
India's villages to seek jobs in cities.Credit...Vivek Singh for The New
York Times

Basharat Peer, a staff editor for Opinion, is the author of ``Curfewed
Night,'' a memoir of the conflict in Kashmir, and ``A Question of Order:
India, Turkey, and the Return of Strongmen.''

\emph{The Times is committed to publishing}
\href{https://www.nytimes.com/2019/01/31/opinion/letters/letters-to-editor-new-york-times-women.html}{\emph{a
diversity of letters}} \emph{to the editor. We'd like to hear what you
think about this or any of our articles. Here are some}
\href{https://help.nytimes.com/hc/en-us/articles/115014925288-How-to-submit-a-letter-to-the-editor}{\emph{tips}}\emph{.
And here's our email:}
\href{mailto:letters@nytimes.com}{\emph{letters@nytimes.com}}\emph{.}

\emph{Follow The New York Times Opinion section on}
\href{https://www.facebook.com/nytopinion}{\emph{Facebook}}\emph{,}
\href{http://twitter.com/NYTOpinion}{\emph{Twitter (@NYTopinion)}}
\emph{and}
\href{https://www.instagram.com/nytopinion/}{\emph{Instagram}}\emph{.}

Advertisement

\protect\hyperlink{after-bottom}{Continue reading the main story}

\hypertarget{site-index}{%
\subsection{Site Index}\label{site-index}}

\hypertarget{site-information-navigation}{%
\subsection{Site Information
Navigation}\label{site-information-navigation}}

\begin{itemize}
\tightlist
\item
  \href{https://help.nytimes.com/hc/en-us/articles/115014792127-Copyright-notice}{©~2020~The
  New York Times Company}
\end{itemize}

\begin{itemize}
\tightlist
\item
  \href{https://www.nytco.com/}{NYTCo}
\item
  \href{https://help.nytimes.com/hc/en-us/articles/115015385887-Contact-Us}{Contact
  Us}
\item
  \href{https://www.nytco.com/careers/}{Work with us}
\item
  \href{https://nytmediakit.com/}{Advertise}
\item
  \href{http://www.tbrandstudio.com/}{T Brand Studio}
\item
  \href{https://www.nytimes.com/privacy/cookie-policy\#how-do-i-manage-trackers}{Your
  Ad Choices}
\item
  \href{https://www.nytimes.com/privacy}{Privacy}
\item
  \href{https://help.nytimes.com/hc/en-us/articles/115014893428-Terms-of-service}{Terms
  of Service}
\item
  \href{https://help.nytimes.com/hc/en-us/articles/115014893968-Terms-of-sale}{Terms
  of Sale}
\item
  \href{https://spiderbites.nytimes.com}{Site Map}
\item
  \href{https://help.nytimes.com/hc/en-us}{Help}
\item
  \href{https://www.nytimes.com/subscription?campaignId=37WXW}{Subscriptions}
\end{itemize}
