Sections

SEARCH

\protect\hyperlink{site-content}{Skip to
content}\protect\hyperlink{site-index}{Skip to site index}

\href{https://www.nytimes.com/section/science}{Science}

\href{https://myaccount.nytimes.com/auth/login?response_type=cookie\&client_id=vi}{}

\href{https://www.nytimes.com/section/todayspaper}{Today's Paper}

\href{/section/science}{Science}\textbar{}The Romans Called it
`Alexandrian Glass.' Where Was It Really From?

\url{https://nyti.ms/3hPJy1J}

\begin{itemize}
\item
\item
\item
\item
\item
\item
\end{itemize}

Advertisement

\protect\hyperlink{after-top}{Continue reading the main story}

Supported by

\protect\hyperlink{after-sponsor}{Continue reading the main story}

Trilobites

\hypertarget{the-romans-called-it-alexandrian-glass-where-was-it-really-from}{%
\section{The Romans Called it `Alexandrian Glass.' Where Was It Really
From?}\label{the-romans-called-it-alexandrian-glass-where-was-it-really-from}}

Trace quantities of isotopes hint at the true origin of a kind of glass
that was highly prized in the Roman Empire.

\includegraphics{https://static01.nyt.com/images/2020/08/04/science/30TB-GLASS/30TB-GLASS-articleLarge.jpg?quality=75\&auto=webp\&disable=upscale}

By Katherine Kornei

\begin{itemize}
\item
  July 31, 2020
\item
  \begin{itemize}
  \item
  \item
  \item
  \item
  \item
  \item
  \end{itemize}
\end{itemize}

Glass was highly valued across the Roman Empire, particularly a
colorless, transparent version that resembled rock crystal. But the
source of this coveted material --- known as Alexandrian glass --- has
long remained a mystery. Now, by studying trace quantities of the
element hafnium within the glass, researchers have shown that this
prized commodity really did originate in ancient Egypt.

It was during the time of the Roman Empire that drinks and food were
served in glass vessels for the first time on a large scale, said
Patrick Degryse, an archaeometrist at KU Leuven in Belgium, who was not
involved in the new study. ``It was on every table,'' he said. Glass was
also used in windows and mosaics.

All that glass had to come from somewhere. Between the first and ninth
centuries A.D., Roman glassmakers in coastal regions of Egypt and the
Levant filled furnaces with sand. The enormous slabs of glass they
created tipped the scales at up to nearly 20 tons. That glass was then
broken up and distributed to glass workshops, where it was remelted and
shaped into final products.

But what many people really wanted was colorless glass, so glassmakers
experimented with adding different elements to their batches. Producers
in the Levant are known to have added manganese, which reacts with iron
impurities in sand. The manganese-treated glass still retained a bit of
color, however, said Gry Hoffmann Barfod, a geoscientist at Aarhus
University in Denmark who led the study, which was
\href{https://www.nature.com/articles/s41598-020-68089-w}{published this
month} in Scientific Reports. ``It wasn't perfect,'' she said.

Glassmakers also tried adding antimony, with much better results. ``That
made it completely crystal clear,'' Dr. Barfod said.

And expensive: A price list issued by the Roman emperor Diocletian in
the early fourth century A.D. refers to this colorless glass as
``Alexandrian'' and values it at
\href{https://www.cmog.org/article/glass-price-edict-diocletian}{nearly
double the price} of manganese-treated glass. But the provenance of
Alexandrian glass, despite its name, had never been conclusively pinned
to Egypt.

``We have the factories for the manganese-decolorized glass, but we
don't have them for the Alexandrian glass,'' Dr. Barfod said. ``It's
been a mystery that historians have dreamed of solving.''

Motivated by that enigma, Dr. Barfod and her colleagues analyzed 37
fragments of glass
\href{https://projects.au.dk/internationaljerashexcavation/}{excavated
in northern Jordan}. The sherds, each no longer than a finger, included
Alexandrian glass and manganese-treated glass from the first through the
fourth centuries A.D. The sample also included other specimens of glass
known to have been produced more recently in either Egypt or the Levant.

The researchers focused on hafnium, a trace element found in the mineral
zircon, a component of sand. They measured the concentration of hafnium
and the ratio of two hafnium isotopes in the sherds.

Glass forged in different geographic regions had different hafnium
signatures, Dr. Barfod and her collaborators showed. Egyptian glass
consistently contained more hafnium and had lower isotopic ratios than
glass produced in the Levant, the team found.

These differences make sense, Dr. Barfod and her colleagues propose,
because the zircon crystals within sand are inadvertently sorted by
nature.

After being expelled from the mouth of the Nile, sand sweeps east and
north up the coast of the Levant, propelled by water currents. The
zircon crystals within it are heavy, so they tend to settle out early in
the journey on Egyptian beaches. That explains why glass forged in
Egyptian furnaces tends to contain more hafnium than Levantine glass,
the researchers suggest.

When researchers analyzed the sherds of Alexandrian and
manganese-treated glass, they again found distinct differences in
hafnium. The manganese-treated glass had hafnium properties consistent
with being produced in the Levant, as expected. And Alexandrian glass,
the clearest of the clear when it came to transparent glass, chemically
resembled Egyptian glass.

It's rewarding to finally pin down the provenance of Alexandrian glass,
Dr. Barfod said, adding, ``This has been an open question for decades.''

But it's still a mystery why glasses from Egypt and the Levant exhibit
different ratios of hafnium isotopes. One possibility is that the
zircons containing certain isotopic ratios are bigger, denser, or
bulkier, which affects their movement, Dr. Barfod said. ``We don't
know.''

Analyzing the chemistry of Egyptian and Levantine beach sand would be a
logical way of confirming these findings, Dr. Barfod said. ``The next
step would obviously be to go out and get sand from both places.''

Advertisement

\protect\hyperlink{after-bottom}{Continue reading the main story}

\hypertarget{site-index}{%
\subsection{Site Index}\label{site-index}}

\hypertarget{site-information-navigation}{%
\subsection{Site Information
Navigation}\label{site-information-navigation}}

\begin{itemize}
\tightlist
\item
  \href{https://help.nytimes.com/hc/en-us/articles/115014792127-Copyright-notice}{©~2020~The
  New York Times Company}
\end{itemize}

\begin{itemize}
\tightlist
\item
  \href{https://www.nytco.com/}{NYTCo}
\item
  \href{https://help.nytimes.com/hc/en-us/articles/115015385887-Contact-Us}{Contact
  Us}
\item
  \href{https://www.nytco.com/careers/}{Work with us}
\item
  \href{https://nytmediakit.com/}{Advertise}
\item
  \href{http://www.tbrandstudio.com/}{T Brand Studio}
\item
  \href{https://www.nytimes.com/privacy/cookie-policy\#how-do-i-manage-trackers}{Your
  Ad Choices}
\item
  \href{https://www.nytimes.com/privacy}{Privacy}
\item
  \href{https://help.nytimes.com/hc/en-us/articles/115014893428-Terms-of-service}{Terms
  of Service}
\item
  \href{https://help.nytimes.com/hc/en-us/articles/115014893968-Terms-of-sale}{Terms
  of Sale}
\item
  \href{https://spiderbites.nytimes.com}{Site Map}
\item
  \href{https://help.nytimes.com/hc/en-us}{Help}
\item
  \href{https://www.nytimes.com/subscription?campaignId=37WXW}{Subscriptions}
\end{itemize}
