Sections

SEARCH

\protect\hyperlink{site-content}{Skip to
content}\protect\hyperlink{site-index}{Skip to site index}

\href{https://www.nytimes.com/section/sports/basketball}{Pro Basketball}

\href{https://myaccount.nytimes.com/auth/login?response_type=cookie\&client_id=vi}{}

\href{https://www.nytimes.com/section/todayspaper}{Today's Paper}

\href{/section/sports/basketball}{Pro Basketball}\textbar{}A Momentous
First Night Back for the N.B.A.

\url{https://nyti.ms/3jUOzYC}

\begin{itemize}
\item
\item
\item
\item
\item
\item
\end{itemize}

Advertisement

\protect\hyperlink{after-top}{Continue reading the main story}

Supported by

\protect\hyperlink{after-sponsor}{Continue reading the main story}

\hypertarget{a-momentous-first-night-back-for-the-nba}{%
\section{A Momentous First Night Back for the
N.B.A.}\label{a-momentous-first-night-back-for-the-nba}}

Close games, social justice protests and a principal role for Rudy
Gobert, again.

\includegraphics{https://static01.nyt.com/images/2020/07/31/sports/31nba-returnsWEB1/merlin_175137366_d5661bcf-71e1-4749-a5c8-1dfd3152b43a-articleLarge.jpg?quality=75\&auto=webp\&disable=upscale}

\href{https://www.nytimes.com/by/marc-stein}{\includegraphics{https://static01.nyt.com/images/2018/06/14/multimedia/author-marc-stein/author-marc-stein-thumbLarge.png}}

By \href{https://www.nytimes.com/by/marc-stein}{Marc Stein}

\begin{itemize}
\item
  July 31, 2020
\item
  \begin{itemize}
  \item
  \item
  \item
  \item
  \item
  \item
  \end{itemize}
\end{itemize}

LAKE BUENA VISTA, Fla. --- When his postgame Zoom interview was over,
before making a triumphant exit to the team bus, Rudy Gobert of the Utah
Jazz acknowledged that the historic play he was savoring did not go
exactly as planned.

``I wasn't supposed to get a post-up,'' Gobert said. ``I was supposed to
get a dunk.''

After using a Donovan Mitchell screen to shake free, finally corralling
a deflected pass and then spinning back toward the baseline, Gobert
dropped the ball in over his former teammate Derrick Favors inside the
first 20 seconds on Thursday night. Gobert's brief nod that followed
seemed to acknowledge the significance of the score.

What Gobert ultimately got was a layup that will be recorded as the
first N.B.A. basket in July that has ever counted. He scored the first
two points and the last two points in Utah's 106-104 victory over the
New Orleans Pelicans in the first game of the N.B.A. restart at Walt
Disney World --- 141 days after Gobert's positive coronavirus test on
March 11 led to the indefinite suspension of the season.

``Life works in a mysterious way,'' Gobert said.

That opening sequence and his clinching free throws, as a mere
62.1-percent foul shooter, helped make it a redemptive evening for
Gobert. His moment came swiftly after the pregame moment --- moving
social justice protest, in an arena without fans but teeming with unity
and purpose, that made this return engagement
\href{https://www.nytimes.com/2020/07/30/sports/basketball/clippers-lakers.html}{an
even bigger occasion for the N.B.A.}

\emph{\textbf{{[}Read:}}
\textbf{\href{https://www.nytimes.com/2020/07/30/sports/basketball/clippers-lakers.html}{\emph{How
the Jazz beat the Pelicans, and how the Lakers beat the
Clippers}}\emph{{]}}}

For more than four minutes before the Jazz and the Pelicans tipped off,
both teams' players, coaches and staff members, along with the referees,
congregated side by side, stretching from baseline to baseline. They
gathered near the BLACK LIVES MATTER lettering affixed to the floor near
the scorer's table, then knelt in unison during a playing of the
national anthem recorded by the musician Jon Batiste.

The Los Angeles Lakers and the Los Angeles Clippers, Staples Center
co-tenants and rivals, came together to do the same before their game,
during a recorded rendition of the anthem by the Compton Kidz Club from
the Los Angeles area. Later, after LeBron James had helped the Lakers
clinch a 103-101 victory with winning plays at both ends in the final
12.8 seconds, he told TNT in a postgame interview: ``I hope our fans are
proud of us.''

James wasn't talking about the basketball. Nor was he referring to the
league's comeback after a lengthy coronavirus-imposed absence, or the
hopeful start to the N.B.A.'s efforts to erect a so-called bubble on the
Disney campus (at a cost of at least \$180 million) with
made-for-television arena settings and daily coronavirus testing. Like
many players involved in Thursday's doubleheader, James was moved most
by the unity displayed in the anthem protests.

``I hope we made Kap proud,'' James said, referring to the former San
Francisco 49ers quarterback Colin Kaepernick, who began kneeling during
the anthem in the 2016 N.F.L. season to protest racial injustice. No
team has signed him since.

``I hope we continue making Kap proud every single day,'' James said.

Said the Pelicans' JJ Redick: ``The `stick to sports' crowd, `keep
politics out of sports,' all those things, they're meaningless now. You
can't. Politics and sports coexist now, and the league has recognized
that.''

N.B.A. Commissioner Adam Silver attended both of Thursday night's games,
wearing a blue hat and watching from behind plexiglass high above the
floor in both the HP Field House (Jazz-Pelicans) and The Arena
(Lakers-Clippers) because he has not yet been quarantined and thus
cannot be around any of the estimated 1,500 inhabitants of the league's
bubble. Silver, though, did issue a statement affirming that the league
will be not be enforcing its longstanding rule, dating to 1981, that
mandates all team personnel to stand for the national anthem in a
``dignified posture'' along a sideline or the foul line.

\hypertarget{the-games-resume}{%
\subsubsection{The Games Resume}\label{the-games-resume}}

\hypertarget{sports-and-the-virus}{%
\paragraph{Sports and the Virus}\label{sports-and-the-virus}}

Updated July 31, 2020

Here's what's happening as the world of sports slowly comes back to
life:

\begin{itemize}
\item
  \begin{itemize}
  \tightlist
  \item
    The
    \href{https://www.nytimes.com/2020/07/30/sports/basketball/clippers-lakers.html?action=click\&pgtype=Article\&state=default\&region=MAIN_CONTENT_2\&context=storylines_keepup}{N.B.A.
    returned}, and the Lakers held on to beat the Clippers in a
    thriller. Zion Williamson played in the first game of the night for
    the Pelicans.
  \item
    Players, coaches and analysts are watching this season's baseball
    games
    \href{https://www.nytimes.com/2020/07/31/sports/baseball/baseball-empty-stadiums-effects.html?action=click\&pgtype=Article\&state=default\&region=MAIN_CONTENT_2\&context=storylines_keepup}{to
    see what effect} the absence of fans has.
  \item
    With no summer tournaments to play in, top high school basketball
    stars are
    \href{https://www.nytimes.com/2020/07/30/sports/ncaabasketball/college-basketball-recruiting.html?action=click\&pgtype=Article\&state=default\&region=MAIN_CONTENT_2\&context=storylines_keepup}{committing
    to colleges earlier}. Villanova is one of the beneficiaries.
  \end{itemize}
\end{itemize}

``I respect our teams' unified act of peaceful protest for social
justice, and under these unique circumstances will not enforce our
longstanding rule requiring standing during the playing of our national
anthem,'' Silver said.

There was a lot for the commissioner to take in. The games were played
in two of the three venues at the ESPN Wide World of Sports Complex so
that Turner could broadcast them back to back without a delay in between
for cleaning.

In the first game, Utah overcame a 16-point deficit in front of the
``home'' team Pelicans' virtual fans . The players wore Black Lives
Matter T-shirts during pregame warm-ups and many had social justice
slogans on their backs of their uniforms in place of their names:
``Peace'' for New Orleans' prized rookie Zion Williamson; ``I am A Man''
for Utah's Mike Conley; ``Say Her Name'' for Utah's Donovan Mitchell.

Mitchell went even further in his protest against systemic racism,
entering the building clad in a bulletproof vest inscribed with the
names of numerous victims of police brutality.

\includegraphics{https://static01.nyt.com/images/2020/07/31/sports/31nba-returnsWEB2/merlin_175134909_22d715c2-44c5-410a-a844-f480630251db-articleLarge.jpg?quality=75\&auto=webp\&disable=upscale}

``The game was great --- we won by two --- but at the end of the day,
Breonna Taylor's killers are still free,'' Mitchell said. ``There are so
many different things that we could honestly talk about. I'm going to
continue to talk about Breonna Taylor because that's near and dear to
me.''

On March 13, Taylor was fatally shot when police officers burst into her
Louisville, Ky., apartment with a no-knock warrant they used as part of
a narcotics investigation. Mitchell played collegiately at Louisville.

Taylor's killing came two days after Gobert's positive coronavirus test
resulted in the N.B.A.'s shutdown. Gobert and Mitchell --- who also
tested positive for the coronavirus in March --- went weeks without
speaking. This was partly because of an infamous video clip of Gobert
touching a table full of reporters' recording devices before he knew he
had been infected, prompting many critics to assert that he was not
treating the virus seriously. It later emerged that tensions between the
two players had been bubbling for some time.

On this night, Mitchell scored eight consecutive Utah points in
crunchtime, then made the crucial drive and assist that set up Gobert's
game-winning free throws. Gobert finished with 14 points, 12 rebounds,
three blocked shots and the opportunity to reflect on the roller coaster
of the past four months when the N.B.A. was forced to go dormant.

``I'm just grateful to be back on the floor,'' Gobert said. ``Honestly,
a lot of things have been said, a lot of things happened, a lot of
things are happening in the world right now. To be able to do what we
love, to be able to do it at the highest level, in safe conditions, to
be able to have a positive impact on communities and inspire millions of
people and kids around the world --- it's really something that is
bigger than just the game.''

Advertisement

\protect\hyperlink{after-bottom}{Continue reading the main story}

\hypertarget{site-index}{%
\subsection{Site Index}\label{site-index}}

\hypertarget{site-information-navigation}{%
\subsection{Site Information
Navigation}\label{site-information-navigation}}

\begin{itemize}
\tightlist
\item
  \href{https://help.nytimes.com/hc/en-us/articles/115014792127-Copyright-notice}{©~2020~The
  New York Times Company}
\end{itemize}

\begin{itemize}
\tightlist
\item
  \href{https://www.nytco.com/}{NYTCo}
\item
  \href{https://help.nytimes.com/hc/en-us/articles/115015385887-Contact-Us}{Contact
  Us}
\item
  \href{https://www.nytco.com/careers/}{Work with us}
\item
  \href{https://nytmediakit.com/}{Advertise}
\item
  \href{http://www.tbrandstudio.com/}{T Brand Studio}
\item
  \href{https://www.nytimes.com/privacy/cookie-policy\#how-do-i-manage-trackers}{Your
  Ad Choices}
\item
  \href{https://www.nytimes.com/privacy}{Privacy}
\item
  \href{https://help.nytimes.com/hc/en-us/articles/115014893428-Terms-of-service}{Terms
  of Service}
\item
  \href{https://help.nytimes.com/hc/en-us/articles/115014893968-Terms-of-sale}{Terms
  of Sale}
\item
  \href{https://spiderbites.nytimes.com}{Site Map}
\item
  \href{https://help.nytimes.com/hc/en-us}{Help}
\item
  \href{https://www.nytimes.com/subscription?campaignId=37WXW}{Subscriptions}
\end{itemize}
