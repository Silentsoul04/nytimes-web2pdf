Sections

SEARCH

\protect\hyperlink{site-content}{Skip to
content}\protect\hyperlink{site-index}{Skip to site index}

\href{https://www.nytimes.com/section/style}{Style}

\href{https://myaccount.nytimes.com/auth/login?response_type=cookie\&client_id=vi}{}

\href{https://www.nytimes.com/section/todayspaper}{Today's Paper}

\href{/section/style}{Style}\textbar{}What Is Randonautica Really About?

\url{https://nyti.ms/3hUNxds}

\begin{itemize}
\item
\item
\item
\item
\item
\end{itemize}

\href{https://www.nytimes.com/spotlight/at-home?action=click\&pgtype=Article\&state=default\&region=TOP_BANNER\&context=at_home_menu}{At
Home}

\begin{itemize}
\tightlist
\item
  \href{https://www.nytimes.com/2020/08/03/well/family/the-benefits-of-talking-to-strangers.html?action=click\&pgtype=Article\&state=default\&region=TOP_BANNER\&context=at_home_menu}{Talk:
  To Strangers}
\item
  \href{https://www.nytimes.com/2020/08/01/at-home/coronavirus-make-pizza-on-a-grill.html?action=click\&pgtype=Article\&state=default\&region=TOP_BANNER\&context=at_home_menu}{Make:
  Grilled Pizza}
\item
  \href{https://www.nytimes.com/2020/07/31/arts/television/goldbergs-abc-stream.html?action=click\&pgtype=Article\&state=default\&region=TOP_BANNER\&context=at_home_menu}{Watch:
  'The Goldbergs'}
\item
  \href{https://www.nytimes.com/interactive/2020/at-home/even-more-reporters-editors-diaries-lists-recommendations.html?action=click\&pgtype=Article\&state=default\&region=TOP_BANNER\&context=at_home_menu}{Explore:
  Reporters' Google Docs}
\end{itemize}

Advertisement

\protect\hyperlink{after-top}{Continue reading the main story}

Supported by

\protect\hyperlink{after-sponsor}{Continue reading the main story}

\hypertarget{what-is-randonautica-really-about}{%
\section{What Is Randonautica Really
About?}\label{what-is-randonautica-really-about}}

An app that generates coordinates for adventurers claims to turn your
thoughts into reality. TikTok and YouTube creators want you to believe
it --- but you shouldn't.

\includegraphics{https://static01.nyt.com/images/2020/08/02/fashion/02RANDONAUTICA/02RANDONAUTICA-articleLarge.jpg?quality=75\&auto=webp\&disable=upscale}

By Lena Wilson

\begin{itemize}
\item
  July 31, 2020
\item
  \begin{itemize}
  \item
  \item
  \item
  \item
  \item
  \end{itemize}
\end{itemize}

The app led one person to a
\href{https://vm.tiktok.com/JNK12b4/}{friendly dog in the desert} and
another to a
\href{https://www.reddit.com/r/randonauts/comments/hjfwea/first_time_intention_was_pretty_flowers_took_me/}{field
of wildflowers}. One young woman, after making her college decision,
followed the app to a field where her school's initials had been
\href{https://vm.tiktok.com/JNKFyW3/}{mowed into the grass}.

And then there were the friends who followed the app to a
\href{https://vm.tiktok.com/JNKJHA5/}{suitcase full of human remains}.

That is the gamble one takes with
\href{https://www.randonautica.com/}{Randonautica}, which claims to
channel users' ``intentions'' to produce nearby coordinates for
exploration. Think: The
\href{https://www.nytimes.com/2010/09/26/books/review/Chabris-t.html?searchResultPosition=5}{law
of attraction} meets
\href{https://www.nytimes.com/2018/08/29/nyregion/new-york-city-geocachers-paradise.html?searchResultPosition=3}{geocaching}.

Randonautica makes a few asks of users --- ``What would you like to
get?'' ``Choose your entropy source'' --- before prompting them to
``focus on your intent'' while it fetches coordinates. This process
relies on location settings and a random number generator, which,
despite what the company says, cannot be directly affected by human
thoughts.

Many of the places users have been sent to since Randonautica became
available in February are unremarkable: parking lots, grasslands, many
bodies of water. However, interest has been driven by the spooky and
often synchronistic ``randonauting'' stories many have shared on social
media. While several of them appear to be fake, others have raised some
cause for concern.

The creators of Randonautica say the app has evolved beyond their
intentions. But what \emph{were} those intentions?

\hypertarget{a-brief-history-of-randonauting}{%
\subsection{A Brief History of
Randonauting}\label{a-brief-history-of-randonauting}}

Before Randonautica, there were the Randonauts: Strangers who swapped
stories about their bot-assisted adventures into the unknown. They
wanted to open their minds to the world around them and make meaning of
life's coincidences.

The bot's code came from a group of programmers called the Fatum Project
who were interested in, among other things, using the technology to
ensure the randomness of
\href{https://medium.com/@fatum_project/introducing-fatum-ab676e3c83d}{online
gambling} outcomes.

Joshua Lengfelder, 29, discovered the Fatum Project on the messenger app
Telegram in January 2019, in a fringe-science chat room. He absorbed the
project's theories about how random exploration could break people out
of their predetermined realities, and how people could influence random
outcomes with their minds.

Mr. Lengfelder, a former circus performer, thought the code and its
underlying ideas could be used to explore the relationship between
consciousness and technology. In February 2019, while caring for his
father, who had just suffered a stroke, he created a
\href{http://t.me/randonauts}{Telegram} bot that used the Fatum
Project's code to generate random coordinates. In March, he created a
\href{http://reddit.com/r/randonauts}{Randonauts subreddit}, which now
has 125,000 members. And in October, a developer named Simon Nishi
McCorkindale created a \href{https://bot.randonauts.com/}{web page} for
the bot.

That same month, Auburn Salcedo, the chief executive of
\href{https://www.presleymedia.com/}{Presley Media}, an agency that
creates brand integrations for TV, found the Randonauts on Reddit and
offered to help Mr. Lengfelder get the word out. On Jan. 24, Ms. Salcedo
and Mr. Lengfelder incorporated Randonauts, L.L.C., with her as C.O.O.
and him as C.E.O. (She remains the chief executive of Presley Media,
which handles P.R. for Randonautica.) They released a beta version of
the app on Feb. 22.

Since its release, Randonautica has been downloaded 10.8 million times
from the App Store and Google Play, according to the research firm
Sensor Tower. After a few months of rapid growth, much of it propelled
by TikTok, its downloads have started to taper off, according to data
from the analytics firm App Annie.

In an interview in July, Mr. Lengfelder described Randonautica as ``a
multimedia storytelling platform'' that encourages ``performance art.''
He said the overwhelming response has not surprised him.

``I kind of figured it was inevitable,'' he said. ``Because basically
what it is is like a machine that creates memes and legends, and it kind
of virally propagates on its own.''

On social media, the most popular randonauting videos feature eerie and
seemingly dangerous situations that are dramatized through editing. Some
creators have capitalized on the trend by posting exaggerated or false
accounts of their randonauting adventures. The 27-year-old YouTuber Josh
Yozura, for instance, claimed to have been led
\href{https://www.youtube.com/watch?v=Zdw_ykwq9Pg}{to a crime scene}.
(Mr. Yozura did not respond to multiple requests for comment.)

Ms. Salcedo denounced such videos in an interview with the YouTube
creator
\href{https://www.youtube.com/watch?v=z43KYpejDNQ}{Billschannel}. In a
phone interview this month, she spoke further about the proliferation of
fake videos. ``It's so hard to manage, because people are really taking
creative liberties after seeing how much traction the app is getting in
that fear factor,'' she said.

\hypertarget{so-how-does-it-work}{%
\subsection{So How Does It Work?}\label{so-how-does-it-work}}

On first use, Randonautica offers a brief intro and some tips (``Always
Randonaut with a charged phone,'' ``Never trespass'') before prompting
you to share your location.

Then it will ask you to choose which type of point you would like it to
generate (the differences between which only matter if you believe the
app can read your thoughts) before fetching coordinates from a random
number generator. The user can then open that location in Google Maps to
begin their journey.

Randonautica throws big words like ``quantum'' and ``entropy'' around a
lot. Its creators believe that quantum random numbers are more likely to
be influenced by human consciousness than non-quantum random numbers.
This hypothesis is part of a theory Mr. Lengfelder refers to as
``mind-machine interaction,'' or M.M.I.: It posits that when you focus
on your intent, you are influencing the numbers.

``Basically if you're looking for any kind of peer-reviewed, scientific
consensus, that does not exist yet in the literature,'' Mr. Lengfelder
said in a \href{https://vm.tiktok.com/JF7gotD/}{TikTok video} in June,
speaking about the theory. Instead, he pointed to the work of Dean
Radin, a prominent figure in the pseudoscientific field of
parapsychology, and the
\href{https://www.nytimes.com/2003/03/09/nyregion/mind-over-matter.html?searchResultPosition=1}{Princeton
Engineering Anomalies Research} (PEAR) program, which has cited Dr.
Radin's research, as evidence.

Randonautica
\href{https://old.reddit.com/r/randonauts/wiki/theory}{claims} that a
1998 PEAR experiment supported the idea that people can control random
number generation with their thoughts. That
\href{http://noosphere.princeton.edu/papers/pear/fieldreg2.pdf}{study}
was published in the Journal of Scientific Exploration, which includes
work about the paranormal, spirit possessions, poltergeists and
questions about Shakespeare's authorship. In the study, PEAR's
researchers wrote that the experiment was far from conclusive.

``It looks like they saw some kind of correlation, but they admit that
it was weak and it needed to have further research associated with it,''
said Casey Schwarz, an experimental physicist and assistant professor at
Ursinus College who reviewed Randonautica's claims for this article. She
said she did not know of any quantum system that could be influenced by
human thoughts.

Lisa Fazio, an assistant professor of psychology at Vanderbilt
University, said that the more synchronous experiences were likely
coincidences colored by confirmation bias, or the tendency to look for
information that affirms one's beliefs and tune out contradictory
evidence.

She pointed to a story shared on Reddit, in which an Australian poster
described being led to a map of the London underground. ``Things like
that happen all the time, it's just that you don't notice that map of
London if you didn't have the intention already to be thinking of
London,'' Dr. Fazio said. She also noted that coincidences are far more
common than people realize.

Mr. Lengfelder dismissed such criticisms, stating that the app was not
created to prove a hypothesis. ``I would say it's not some kind of
academic science work,'' he said. ``We're more like inventors than
academic scientists.''

An update coming in August will feature improved graphics and, Mr.
Lengfelder said, a custom random number generator that would have a
higher ``rate of entropy.'' ``So technically our M.M.I. effects should
be higher,'' he said. Of course, as noted above, M.M.I. is a theory that
is not supported by science.

Daniel J. Rogers, a physicist who has worked with quantum random number
generators, called Randonautica's M.M.I. theory ``completely absurd.''

``There is no quantum physics here,'' said Dr. Rogers, a founder of the
Global Disinformation Index. ``This is just people using big science
words to sound magical. There is no actual science here.''

\hypertarget{do-not-go-randonauting}{%
\subsection{`Do Not Go Randonauting'}\label{do-not-go-randonauting}}

Randonauting became popular partly because of reverse psychology; young
people approach it with a sense of foreboding. ``Do not go
randonauting'' has become a popular title for videos.

Several people who shared unsettling stories about the app say they have
since sworn it off. Adrian Chavez, 21, was led to an ominous beach near
his home in Orange County, Calif. A video of his journey, posted on
\href{https://vm.tiktok.com/JF1tNLR/}{TikTok} in early June, has been
viewed 4.5 million times.

``I deleted the app right after that and never used it again since,''
Mr. Chavez said in an interview in July.

The 18-year-old TikTok user who posted the viral video about finding a
suitcase of human remains on a Seattle beach, @UghHenry, wrote in the
comments of his video: ``The moment I got back home, I broke down. I
still can't sleep.''

In an interview with
\href{https://www.theatlantic.com/technology/archive/2020/07/randonautica-app-tiktok-body-reddit-quantum/614401/}{The
Atlantic}, Mr. Lengfelder was blasé about the story, which was covered
by news outlets including
\href{https://www.king5.com/article/news/local/remains-found-on-west-seattle-beach-identified/281-d3a505c4-1ea2-4e72-984a-f5bb4f8fa843}{KING
5 News} and The
\href{https://nypost.com/2020/07/08/reward-money-raised-after-seattle-tiktokers-find-bodies-in-suitcase/}{New
York Post}. ``It's not the best press, but I'm not really that upset
about it, because it's kind of cool,'' he said. ``I kind of wish it was
me who found it.''

Some adults have expressed concerns about the app's lack of safety
precautions for children. Though Randonautica's terms of use specify
that anyone who is a minor must obtain parental consent to use the app,
such consent is collected by email, making it easy for young users to
bypass.

Know and Tell, a child protection education program with the Granite
State Children's Alliance in New Hampshire, has posted on Instagram
telling parents to keep young people off the app, or at least supervise
their use.

``It was very apparent that these were young teenagers that were going
to undisclosed areas in the middle of the night,'' said Jana El-Sayed,
the outreach project manager for the Granite State Children's Alliance.
She described these circumstances as ``a perpetrator's dream.''

Concerns about human trafficking and personal data use are addressed in
Randonautica's
\href{https://www.randonautica.com/got-questions}{F.A.Q.}, which
specifies that all location data is anonymized and only made available
to developers, and that starting locations are never saved by the app.

Pokémon Go, which uses augmented reality to encourage local exploration,
has handled safety concerns by putting PokéStops and Gyms in notable,
public locations, and encouraging users to remain vigilant.

Randonautica's
\href{https://www.randonautica.com/be-a-responsible-randonaut}{safety
tips} are similar: Avoid dangerous areas, do not trespass, try to
explore during the day or with friends. Randonautica's
\href{https://www.randonautica.com/be-a-responsible-randonaut}{website}
repeatedly urges users to ``use common sense.'' The latest version of
the app will feature multiple screens and pop-ups reminding users to use
the app safely.

Randonautica's executives say they don't understand why people would use
the app to seek out risk or harm.

``You wouldn't go out on a walk and say, `Let me think about seeing
death,''' Ms. Salcedo said in an interview, referring to a
\href{https://www.tiktok.com/@mykenarae/video/6842171161585945862}{viral
TikTok video} in which an 18-year-old user claims she set her intention
as ``death'' and then happened upon a shooting victim.

``Yeah, `Let's see if I get stalked,''' Mr. Lengfelder added.

Ms. Salcedo said Randonautica's legal counsel reassured her and Mr.
Lengfelder that the app would not be liable for any user misconduct.

``Is Google Maps liable too, for giving them directions?'' Mr.
Lengfelder said. ``At a certain point, if somebody wants to really go
out of the way and harm themselves, they're going to do it. Whether it's
with Randonautica or not.''

\begin{center}\rule{0.5\linewidth}{\linethickness}\end{center}

Ben Decker contributed reporting.

Advertisement

\protect\hyperlink{after-bottom}{Continue reading the main story}

\hypertarget{site-index}{%
\subsection{Site Index}\label{site-index}}

\hypertarget{site-information-navigation}{%
\subsection{Site Information
Navigation}\label{site-information-navigation}}

\begin{itemize}
\tightlist
\item
  \href{https://help.nytimes.com/hc/en-us/articles/115014792127-Copyright-notice}{©~2020~The
  New York Times Company}
\end{itemize}

\begin{itemize}
\tightlist
\item
  \href{https://www.nytco.com/}{NYTCo}
\item
  \href{https://help.nytimes.com/hc/en-us/articles/115015385887-Contact-Us}{Contact
  Us}
\item
  \href{https://www.nytco.com/careers/}{Work with us}
\item
  \href{https://nytmediakit.com/}{Advertise}
\item
  \href{http://www.tbrandstudio.com/}{T Brand Studio}
\item
  \href{https://www.nytimes.com/privacy/cookie-policy\#how-do-i-manage-trackers}{Your
  Ad Choices}
\item
  \href{https://www.nytimes.com/privacy}{Privacy}
\item
  \href{https://help.nytimes.com/hc/en-us/articles/115014893428-Terms-of-service}{Terms
  of Service}
\item
  \href{https://help.nytimes.com/hc/en-us/articles/115014893968-Terms-of-sale}{Terms
  of Sale}
\item
  \href{https://spiderbites.nytimes.com}{Site Map}
\item
  \href{https://help.nytimes.com/hc/en-us}{Help}
\item
  \href{https://www.nytimes.com/subscription?campaignId=37WXW}{Subscriptions}
\end{itemize}
