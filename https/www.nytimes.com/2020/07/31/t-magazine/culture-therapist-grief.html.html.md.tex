How Should I Respond to Overzealous Gestures of Sympathy?

\url{https://nyti.ms/2BMLeK1}

\begin{itemize}
\item
\item
\item
\item
\item
\end{itemize}

\includegraphics{https://static01.nyt.com/images/2020/07/29/t-magazine/art/culture-therapist-slide-M29D/culture-therapist-slide-M29D-superJumbo.jpg}

Ivan Kramskoi's ``Inconsolable Grief'' (1884). HIP/Art Resource, NY

Sections

\protect\hyperlink{site-content}{Skip to
content}\protect\hyperlink{site-index}{Skip to site index}

\hypertarget{how-should-i-respond-to-overzealous-gestures-of-sympathy}{%
\section{How Should I Respond to Overzealous Gestures of
Sympathy?}\label{how-should-i-respond-to-overzealous-gestures-of-sympathy}}

``These people mire you in your sense of loss just as you are ready to
discover what remains to be found,'' writes one of our advice
columnists.

Ivan Kramskoi's ``Inconsolable Grief'' (1884). HIP/Art Resource,
NYCredit...

Supported by

\protect\hyperlink{after-sponsor}{Continue reading the main story}

By \href{https://www.nytimes.com/by/megan-o-grady}{Megan O'Grady}

\begin{itemize}
\item
  July 31, 2020
\item
  \begin{itemize}
  \item
  \item
  \item
  \item
  \item
  \end{itemize}
\end{itemize}

\emph{In T's advice column,}
\href{https://www.nytimes.com/column/culture-therapist?module=inline}{\emph{Culture
Therapist}}\emph{, either}
\href{https://www.nytimes.com/by/ligaya-mishan?module=inline}{\emph{Ligaya
Mishan}} \emph{or}
\href{https://www.nytimes.com/by/megan-o-grady?module=inline}{\emph{Megan
O'Grady}} \emph{solves your problems using art. Have a question? Need
some comfort? Email us at}
\href{mailto:advice@nytimes.com}{\emph{advice@nytimes.com}}\emph{.}

\emph{Q: My husband died recently. It was as good a death as could have
been expected: He had been ill for a year, so we had time to prepare and
to mourn together, he died at home under palliative care and in very
little pain (he said), and our family, close friends and I had time for
goodbyes. We had a good life together, and cleared up any prospective
messes in advance of his departure. We wrote his obituary together.}

\emph{My question is this: People who knew neither of us very well fall
upon me in public places with hugs and kisses and sad faces. I am not a
huggy sort of person, and I know they're trying to be sympathetic, but
the intrusions and the intimations that I must be very sad and lonely
drive me to distraction. I loved my husband, but we each grieve in our
own way, and mine is to get my life in order, circle around close
friends and move on to the next adventure, project or quiet time. How do
I best deal with these intrusions without hurting acquaintances'
feelings? --- Signed, Getting On With It}

A: All of us, but maybe especially women, have experienced what it's
like to feel like a screen for someone else's projections. The
over-the-top reactions of your acquaintances, perhaps well intentioned,
only serve to communicate to you that you have just suffered their worst
nightmare. I can understand your impatience with them as you open
yourself to the world again and consider your life, fundamentally
reordered. These people mire you in your sense of loss just as you are
ready to discover what remains to be found.

Such exchanges trouble you because they're about much more than
navigating social awkwardness. The fact is, when we suffer loss, we
often look to those around us to help affirm who we are, to give us a
sense of continuity and possibility. We need to have those conversations
in order to imagine a way forward, so that we don't feel like we're
stepping off a cliff, legs wheeling in thin air. How wonderful that you
were able to prepare for this, to the extent that one can, with your
husband. You were able to have a sense of an ending, something that
often, even under normal circumstances, just isn't possible. These days,
with families barred from hospitals and memorial services severely
limited, it's all the more rare and wonderful.

But, as you know, part of the complexity of grief is that it's not only
about missing the person who knew you in a way that no one else quite
does; it's also about grieving the loss of yourself in that shared
context. You aren't the same person you were during your marriage or
before you got married. The next few years will be filled with trial and
error, and having friends --- old and new --- who actively listen as you
test out new plans and ideas is essential. ``Starting a new chapter'' is
an appealing euphemism for a time like this because it implies a
continuity, a grander scheme, an awareness that life is what we live
along the way, and for that reason, it can never truly be ``lost.''

\includegraphics{https://static01.nyt.com/images/2020/07/29/t-magazine/art/culture-therapist-slide-4JHU/culture-therapist-slide-4JHU-articleLarge.jpg?quality=75\&auto=webp\&disable=upscale}

Once, in my 30s, faced with losses, I embarked on a new chapter in a
very mechanical way: I got on a plane to Berlin, a city well suited for
reinvention and its accompanying grief. I found there the kind of new
friends who would sit with me for dozens of hours over the course of a
year in cafes, drinking wine and helping me through the messy process of
eking out a new version of myself. I lived in a series of sublet
apartments, all of them with views of a cemetery. I dated an earnest
German who believed the internet would save democracy. How many wrong
turns, I often thought, grandiose with solipsism, had I made for it to
come to this.

In that time of odd suspension, I was spending a lot of time wandering
around museums, places that relieved me of my solipsism. At the
\href{https://www.nytimes.com/2018/04/24/arts/berlin-hamburger-banhof.html}{Hamburger
Bahnhof}, the popular former train station turned contemporary art
venue, the word ``Schmerz'' greeted me in enormous letters, the English
translation **** in smaller font below: ``Pain.'' I wandered numbly
through the meticulously curated exhibit, with its Anselm Kiefer
erosions in browns and grays, the contorted pale bodies of a Francis
Bacon, \emph{objets} like Nietzsche's death mask and a biting stick once
used in operations without anesthesia. ``Art is not just there to be
understood,'' Joseph Beuys once said; it is there to be experienced. I
didn't find the exhibit painful, though, just pedantic and depressing.

Image

The work of Hans Holbein the Younger in the old masters paintings
gallery at the Gemäldegalerie in Berlin, including, at center, ``The
Merchant Georg Gisze'' (circa 1532).Credit...© Staatliche Museen zu
Berlin/David von Becker

I found what I needed instead while visiting the
\href{https://www.smb.museum/museen-einrichtungen/gemaeldegalerie/home/}{Gemäldegalerie},
home to the (rather less fashionable) old masters: a Bruegel teeming
with life here, a weary-faced Mantegna Madonna there; Vermeer's women
aglow at their open windows, Holbein's officious burghers. The museum
was almost always empty, the silence inside had a velveteen density, and
there were little alcoves at the end of certain galleries where one
could sit and reflect. I thought I could really sense something of the
people in these portraits, who became familiar to me --- the tender
mothers and infants, the satin-sleeved merchants, the young women who
themselves seemed to be seeking a new view. The emotion of the subjects
was to me indistinguishable from that of their creators, each painting
representative of a moment in time and the effort of making it. I felt
myself poised there, on this tension between their time and mine,
absorbing the feeling of the artists, centuries after their deaths, the
canvases places to explore and amplify aspects of themselves that
perhaps went otherwise unexpressed. I returned many times, sometimes
several times a week, to be reminded that while they were now dead I was
still very much alive, that life was fragile and finite and not to be
squandered.

Image

Pieter Bruegel the Elder, ``The Dutch Proverbs'' (circa 1559).Credit...©
Staatliche Museen zu Berlin, Gemäldegalerie. Photo: Volker-H. Schneider

Image

Andrea Mantegna's ``Maria With the Sleeping Child'' (date
unknown).Credit...© Staatliche Museen zu Berlin, Gemäldegalerie. Photo:
Jörg P. Anders

Image

Jan Vermeer van Delft's ``Young Lady With Pearl Necklace'' (circa
1662).Credit...© Staatliche Museen zu Berlin, Gemäldegalerie. Photo:
Christoph Schmidt

The fact is, nothing really prepares us for loss; we outgrow our lives
faster than we can quite catch up. ``The art of losing isn't hard to
master,'' Elizabeth Bishop wrote, with pained irony, in her 1976
villanelle
``\href{https://www.poetryfoundation.org/poems/47536/one-art}{One
Art},'' in which she recounts a continuum of losses, from house keys to
those things immeasurable and unrecoverable. We will all, someday, lose
everything we love. But this impermanence is also what makes us
understand that the time we have is something to be cherished. No one
--- not new friends or old, and certainly not I --- can tell you
precisely what the next steps will be or just what experiences of art
and friendship might open a window in your mind. But I can tell you that
openness to life's possibilities is a stance, a perspective on the world
that we should all try to cultivate, wherever we are in it. All of us,
in ways small and vast, have been knocked off balance, and are
questioning the certainty of certain futures. We're all engaged in a
process of self-reckoning, reconsidering our lives and priorities.

Image

Image

While there are many beautiful memoirs capturing the textures of grief
--- Elizabeth Alexander's, Joan Didion's, Yiyun Li's, Sarah Manguso's
--- I often think there isn't enough written about what comes after: the
in-between times in life, before things are sorted, when we're still at
an impasse with ourselves. And here I'm thinking of Rachel Cohen's
``\href{https://www.nytimes.com/2020/07/21/books/review/austen-years-rachel-cohen.html}{Austen
Years}'' (2020), a memoir written in a period of dramatic change,
including the death of her father and birth of her children, time she
navigated with the companionship of an author seeking answers in her own
tumultuous era; and Helen Macdonald's
``\href{https://www.nytimes.com/2015/02/22/books/review/helen-macdonalds-h-is-for-hawk.html}{H
Is for Hawk}'' (2014), in which the author mourns her father while
training a goshawk, a hunting bird, a process that involves great
patience, a freezer-full of dead mice and a certain wildness of vision.
Both these books are forthright about the effort and risk, but also the
thrill and pleasure, of working out those next chapters. I'm not
suggesting you run out and acquire a flesh-eating pet (though that's one
way to keep those spaniel-eyed ``friends'' at arm's length), but that
something of your own searching thoughts may stand more fully
illuminated in the light of another's.

Another idea: to go a step further, via books that reexamine our common
present context in light of everything that has come before. Sarah
Broom's memoir,
``\href{https://www.nytimes.com/2019/08/05/books/review-yellow-house-sarah-broom.html}{The
Yellow House}'' (2019), does exactly that, connecting her family
history, and the story of her childhood home in New Orleans, to our
American story, with its entrenched racial inequality. Saidiya Hartman's
``\href{https://www.nytimes.com/2019/02/19/books/wayward-lives-beautiful-experiments-saidiya-hartman.html}{Wayward
Lives, Beautiful Experiments}'' (2019) also traces the echoes between
private losses and public life in a social history that explores the
lives of young Black American women in the early 20th century, unhailed
Modernists who were defining for themselves what life could be. These
were women who understood the imaginative work of self-authorship better
than any of us.

Image

Credit...Alessandra Montalto/The New York Times

Any real grieving process will always involve looking back to move
forward. I want you to honor this in-between time, but also to find good
company for it, literary or otherwise, of the kind that inspires you to
take up more space in the world, not less. Whether you are in lockdown
in your home or protesting in the streets, you can lay the groundwork
now as concretely as possible. You can make new friends by getting
involved in any number of initiatives, local or national. Your emotional
and energetic capital will be valuable in ways that you probably can't
yet know. Fear and freedom, like love and loss, will always be two sides
of the same coin, and there's no time to waste.

Advertisement

\protect\hyperlink{after-bottom}{Continue reading the main story}

\hypertarget{site-index}{%
\subsection{Site Index}\label{site-index}}

\hypertarget{site-information-navigation}{%
\subsection{Site Information
Navigation}\label{site-information-navigation}}

\begin{itemize}
\tightlist
\item
  \href{https://help.nytimes.com/hc/en-us/articles/115014792127-Copyright-notice}{©~2020~The
  New York Times Company}
\end{itemize}

\begin{itemize}
\tightlist
\item
  \href{https://www.nytco.com/}{NYTCo}
\item
  \href{https://help.nytimes.com/hc/en-us/articles/115015385887-Contact-Us}{Contact
  Us}
\item
  \href{https://www.nytco.com/careers/}{Work with us}
\item
  \href{https://nytmediakit.com/}{Advertise}
\item
  \href{http://www.tbrandstudio.com/}{T Brand Studio}
\item
  \href{https://www.nytimes.com/privacy/cookie-policy\#how-do-i-manage-trackers}{Your
  Ad Choices}
\item
  \href{https://www.nytimes.com/privacy}{Privacy}
\item
  \href{https://help.nytimes.com/hc/en-us/articles/115014893428-Terms-of-service}{Terms
  of Service}
\item
  \href{https://help.nytimes.com/hc/en-us/articles/115014893968-Terms-of-sale}{Terms
  of Sale}
\item
  \href{https://spiderbites.nytimes.com}{Site Map}
\item
  \href{https://help.nytimes.com/hc/en-us}{Help}
\item
  \href{https://www.nytimes.com/subscription?campaignId=37WXW}{Subscriptions}
\end{itemize}
