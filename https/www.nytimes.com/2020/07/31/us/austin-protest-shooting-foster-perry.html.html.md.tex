Sections

SEARCH

\protect\hyperlink{site-content}{Skip to
content}\protect\hyperlink{site-index}{Skip to site index}

\href{https://www.nytimes.com/section/us}{U.S.}

\href{https://myaccount.nytimes.com/auth/login?response_type=cookie\&client_id=vi}{}

\href{https://www.nytimes.com/section/todayspaper}{Today's Paper}

\href{/section/us}{U.S.}\textbar{}Motorist Who Shot a Protester in
Austin Claims Self-Defense

\url{https://nyti.ms/3fiHaP6}

\begin{itemize}
\item
\item
\item
\item
\item
\end{itemize}

\href{https://www.nytimes.com/news-event/george-floyd-protests-minneapolis-new-york-los-angeles?action=click\&pgtype=Article\&state=default\&region=TOP_BANNER\&context=storylines_menu}{Race
and America}

\begin{itemize}
\tightlist
\item
  \href{https://www.nytimes.com/2020/07/26/us/protests-portland-seattle-trump.html?action=click\&pgtype=Article\&state=default\&region=TOP_BANNER\&context=storylines_menu}{Protesters
  Return to Other Cities}
\item
  \href{https://www.nytimes.com/2020/07/24/us/portland-oregon-protests-white-race.html?action=click\&pgtype=Article\&state=default\&region=TOP_BANNER\&context=storylines_menu}{Portland
  at the Center}
\item
  \href{https://www.nytimes.com/2020/07/23/podcasts/the-daily/portland-protests.html?action=click\&pgtype=Article\&state=default\&region=TOP_BANNER\&context=storylines_menu}{Podcast:
  Showdown in Portland}
\item
  \href{https://www.nytimes.com/interactive/2020/07/16/us/black-lives-matter-protests-louisville-breonna-taylor.html?action=click\&pgtype=Article\&state=default\&region=TOP_BANNER\&context=storylines_menu}{45
  Days in Louisville}
\end{itemize}

Advertisement

\protect\hyperlink{after-top}{Continue reading the main story}

Supported by

\protect\hyperlink{after-sponsor}{Continue reading the main story}

\hypertarget{motorist-who-shot-a-protester-in-austin-claims-self-defense}{%
\section{Motorist Who Shot a Protester in Austin Claims
Self-Defense}\label{motorist-who-shot-a-protester-in-austin-claims-self-defense}}

A ride-share driver fatally shot an armed demonstrator, Garrett Foster,
over the weekend. The police are still trying to sort out what happened.

\includegraphics{https://static01.nyt.com/images/2020/07/31/us/31PROTEST-AUSTIN01/merlin_174988857_fa465cb9-d8d3-49eb-a98f-7788f2ef4cb5-articleLarge.jpg?quality=75\&auto=webp\&disable=upscale}

By David Montgomery and
\href{https://www.nytimes.com/by/manny-fernandez}{Manny Fernandez}

\begin{itemize}
\item
  July 31, 2020
\item
  \begin{itemize}
  \item
  \item
  \item
  \item
  \item
  \end{itemize}
\end{itemize}

AUSTIN, Texas --- The deadly confrontation between an armed motorist and
an armed protester during a street demonstration in downtown Austin over
the weekend began when the motorist made a turn toward a crowd of
marchers and came to a stop.

The protester was
\href{https://www.nytimes.com/2020/07/26/us/austin-shooting-texas-protests.html}{Garrett
Foster}, 28, a former aircraft mechanic for the U.S. Air Force who wore
a bandanna on his face and carried an AK-47-type rifle on a strap in
front of him. The driver who fired the fatal shots has now been
identified as Daniel S. Perry, 33, an active-duty sergeant with the U.S.
Army and a driver for the ride-hailing company Uber who had just dropped
off a customer nearby.

Days after the shooting that stunned Austin, details of the encounter
remain in dispute, with different points of view from the police,
demonstrators and Mr. Perry, who has not been charged with any crime.

Mr. Foster was at the demonstration that night with his fiancée, Whitney
Mitchell, a quadruple amputee who uses a wheelchair. Mr. Foster was
white, and Ms. Mitchell is Black. The two of them had frequently
attended protests against police brutality in Austin.

Demonstrators who witnessed the confrontation have said in interviews
that Mr. Perry was driving aggressively in the direction of the
protesters and that Mr. Foster approached the vehicle with his rifle
pointed downward. At that point, they said, Mr. Perry pulled out a
handgun and shot him.

But in a statement released late Thursday evening, Mr. Perry's lawyer
disputed that version of events.

Mr. Perry did not know that a Black Lives Matter demonstration was
taking place when he turned onto the street, said the lawyer, F. Clinton
Broden.

He said Mr. Foster approached the car and motioned with his rifle for
Mr. Perry to lower the window, and Mr. Perry complied because he
believed that Mr. Foster was associated with law enforcement. As Mr.
Perry realized that Mr. Foster was not a police officer, Mr. Foster
raised the rifle toward him, Mr. Broden said in the statement.

``It was only then that Sgt. Perry, who carried a handgun in his car for
his own protection while driving strangers in the ride-share program,
fired on the person to protect his own life,'' he said.

Mr. Broden said the police had interviewed witnesses who were marching
with Mr. Foster and who had confirmed that he had raised his rifle ``in
a direct threat to Sgt. Perry's life.'' Immediately after the shooting,
he said, a person in the crowd began firing on Mr. Perry's car, so he
``drove to safety and immediately called the police.''

Mr. Foster's family said they were certain that he had not threatened
the motorist.

``Everyone who was standing around said Garrett never raised his
weapon,'' his mother, Sheila Foster, said in an interview on Friday.
``That man took away one of the best people on this planet.''

A person who appeared to be Mr. Perry had posted in the past on Twitter
about using violence against protesters. The Twitter account has since
been deleted.

In June, President Trump posted a warning to protesters the day before
his rally in Tulsa, Okla., writing on Twitter that any ``protesters,
anarchists, agitators, looters or lowlifes who are going to Oklahoma
please understand, you will not be treated like you have been in New
York, Seattle, or Minneapolis. It will be a much different scene!''

The person who appeared to be Mr. Perry
\href{http://archive.vn/MddfF}{responded on Twitter}: ``Send them to
Texas we will show them why we say don't mess with Texas.''
\href{http://archive.vn/ylS2i}{In another tweet} in June, he wrote that
shooting someone in the ``center of mass,'' or chest area, was the best
way to take the person down.

Mr. Broden defended his client's tweets, saying they were being taken
out of context by protesters. ``I think they're being misused to serve
an agenda,'' Mr. Broden said, adding that Mr. Perry supports First
Amendment rights and has defended those rights as a member of the
military.

Mr. Foster's comments before the shooting are also being scrutinized.
Earlier that evening at the demonstration, Mr. Foster was interviewed by
an independent journalist
\href{https://twitter.com/stillgray/status/1287278680817823747}{on
Periscope} about why he brought his rifle, and he said that ``all the
people that hate us'' were too afraid to ``stop and actually do anything
about it.''

One police official who criticized that comment on social media has
since apologized. The official, Kenneth Casaday, the president of the
Austin police officers' union, wrote on Twitter that Mr. Foster ``was
looking for confrontation and he found it,'' but he later apologized in
\href{https://twitter.com/KennethCasaday/status/1288506200720650244}{another
tweet} ``for my offensive choice of words.''

Austin's police chief, Brian Manley, said investigators were told that
Mr. Foster was shot after he pointed his rifle at Mr. Perry. ``During
the initial investigation of this incident, it appears that Mr. Foster
may have pointed his weapon at the driver of this vehicle prior to being
shot,'' Chief Manley told reporters on Sunday.

Chief Manley said that a person in the crowd who had also opened fire
--- the gunfire that Mr. Perry had reported to the police --- had done
so after hearing the gunshots and seeing the car drive away.

Mr. Perry called 911 after leaving the scene and told dispatchers that
he had shot someone who had approached him and pointed a rifle at him.
He was instructed to pull over. Both he and the person in the crowd who
shot at the vehicle were interviewed by investigators and released. They
both had state-issued handgun licenses.

Mr. Broden said Mr. Perry had ``fully cooperated with the police
following the shooting and he continues to do so.'' Mr. Perry, who is
stationed at the Fort Hood Army base in Killeen, Texas, and had served
in Afghanistan, was driving for Uber as a way to make extra money, his
lawyer said.

``We simply ask that anybody who might want to criticize Sgt. Perry's
actions, picture themselves trapped in a car as a masked stranger raises
an assault rifle in their direction and reflect upon what they might
have done if faced with the split-second decision faced by Sgt. Perry
that evening,'' Mr. Broden said in the statement.

Mr. Foster's mother said her son enlisted in the Air Force in the weeks
after he graduated from high school in 2010 in the Dallas suburb of
Plano, where he grew up. Months later, he and Ms. Mitchell became
engaged, when they were both 19. In a matter of weeks, their lives
changed --- Ms. Mitchell collapsed at her grandmother's house. Her
organs began to fail, as an infection caused septic shock, a
life-threatening condition, Ms. Foster said. All four of her limbs were
amputated.

Mr. Foster spent two years in the Air Force but was discharged and began
taking care of Ms. Mitchell. The couple lived in a house in North Austin
that they renovated to accommodate her wheelchair and her health needs.

``From the moment he got out, two years after he went in, he never left
her side,'' Ms. Foster said. ``He brushed her teeth. He combed her hair.
He did her makeup.''

During a phone conversation she had with her son about a week before his
death, his mother expressed concern that the couple could be exposing
themselves to danger by participating in the demonstrations, which she
knew he had been attending with his rifle. She asked her son who would
take care of Ms. Mitchell if he were put in jail.

She said her son replied: ``Mom, it's not going to happen. I'm not
stupid. I know how to use a gun. I'm not going to point my gun at
anybody.''

David Montgomery reported from Austin and Manny Fernandez from Houston.
Bryan Pietsch contributed reporting from Andover, Minn.

Advertisement

\protect\hyperlink{after-bottom}{Continue reading the main story}

\hypertarget{site-index}{%
\subsection{Site Index}\label{site-index}}

\hypertarget{site-information-navigation}{%
\subsection{Site Information
Navigation}\label{site-information-navigation}}

\begin{itemize}
\tightlist
\item
  \href{https://help.nytimes.com/hc/en-us/articles/115014792127-Copyright-notice}{©~2020~The
  New York Times Company}
\end{itemize}

\begin{itemize}
\tightlist
\item
  \href{https://www.nytco.com/}{NYTCo}
\item
  \href{https://help.nytimes.com/hc/en-us/articles/115015385887-Contact-Us}{Contact
  Us}
\item
  \href{https://www.nytco.com/careers/}{Work with us}
\item
  \href{https://nytmediakit.com/}{Advertise}
\item
  \href{http://www.tbrandstudio.com/}{T Brand Studio}
\item
  \href{https://www.nytimes.com/privacy/cookie-policy\#how-do-i-manage-trackers}{Your
  Ad Choices}
\item
  \href{https://www.nytimes.com/privacy}{Privacy}
\item
  \href{https://help.nytimes.com/hc/en-us/articles/115014893428-Terms-of-service}{Terms
  of Service}
\item
  \href{https://help.nytimes.com/hc/en-us/articles/115014893968-Terms-of-sale}{Terms
  of Sale}
\item
  \href{https://spiderbites.nytimes.com}{Site Map}
\item
  \href{https://help.nytimes.com/hc/en-us}{Help}
\item
  \href{https://www.nytimes.com/subscription?campaignId=37WXW}{Subscriptions}
\end{itemize}
