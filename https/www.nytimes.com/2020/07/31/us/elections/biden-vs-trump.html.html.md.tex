Sections

SEARCH

\protect\hyperlink{site-content}{Skip to
content}\protect\hyperlink{site-index}{Skip to site index}

\href{https://www.nytimes.com/news-event/2020-election}{Elections}

\href{https://myaccount.nytimes.com/auth/login?response_type=cookie\&client_id=vi}{}

\href{https://www.nytimes.com/section/todayspaper}{Today's Paper}

\href{/news-event/2020-election}{Elections}\textbar{}Kamala Harris
Pushes Back on Criticism of `Ambition'

\url{https://nyti.ms/337Yt3c}

\begin{itemize}
\item
\item
\item
\item
\item
\item
\end{itemize}

\begin{itemize}
\item
  \href{https://www.nytimes.com/2020/07/31/us/elections/biden-vs-trump.html?action=click\&pgtype=Article\&state=default\&region=TOP_BANNER\&context=storylines_menu}{Election
  Updates}
\item
  \href{https://www.nytimes.com/article/biden-vice-president-2020.html?action=click\&pgtype=Article\&state=default\&region=TOP_BANNER\&context=storylines_menu}{Biden's
  V.P. Search}
\item
  \href{https://www.nytimes.com/interactive/2020/07/24/us/politics/trump-biden-campaign-donors.html?action=click\&pgtype=Article\&state=default\&region=TOP_BANNER\&context=storylines_menu}{Map
  of Donations}
\item
  \href{https://www.nytimes.com/interactive/2020/us/elections/delegate-count-primary-results.html?action=click\&pgtype=Article\&state=default\&region=TOP_BANNER\&context=storylines_menu}{Delegate
  Count}
\item
  \href{https://www.nytimes.com/interactive/2019/us/politics/2020-presidential-candidates.html?action=click\&pgtype=Article\&state=default\&region=TOP_BANNER\&context=storylines_menu}{The
  Candidates}
\item
  \href{https://www.nytimes.com/newsletters/politics?action=click\&pgtype=Article\&state=default\&region=TOP_BANNER\&context=storylines_menu}{Politics
  Newsletter}
\end{itemize}

Advertisement

\protect\hyperlink{after-top}{Continue reading the main story}

Supported by

\protect\hyperlink{after-sponsor}{Continue reading the main story}

\hypertarget{kamala-harris-pushes-back-on-criticism-of-ambition}{%
\section{Kamala Harris Pushes Back on Criticism of
`Ambition'}\label{kamala-harris-pushes-back-on-criticism-of-ambition}}

President Trump warned of an ``election disaster'' with mail-in voting,
but cited no evidence of fraud. Joe Biden has postponed his planned
announcement of a running mate.

July 31, 2020

\begin{itemize}
\item
\item
\item
\item
\item
\item
\end{itemize}

\emph{This briefing has ended. Follow our}
\href{https://www.nytimes.com/news-event/2020-election}{\emph{latest
coverage}} \emph{of the 2020 Election.}

\hypertarget{heres-what-you-need-to-know}{%
\subsubsection{Here's what you need to
know:}\label{heres-what-you-need-to-know}}

\begin{itemize}
\tightlist
\item
  \protect\hyperlink{link-29fdff45}{Kamala Harris, a top
  vice-presidential contender, confronts double standards.}
\item
  \protect\hyperlink{link-13ec3d9c}{Karen Bass and Susan Rice are rising
  on Biden's vice-presidential shortlist.}
\item
  \protect\hyperlink{link-49e9a016}{Trump says Russian bounties to kill
  U.S. troops `never took place.'}
\item
  \protect\hyperlink{link-48bfea37}{A Trump visit to the Tampa airport
  becomes a campaign rally with low turnout.}
\item
  \protect\hyperlink{link-2b118afb}{House committee ratchets up inquiry
  into whether Pompeo tried to help smear Biden and his son.}
\item
  \protect\hyperlink{link-47e04a0d}{Trump, citing no evidence of voter
  fraud, warns the election will be `rigged.'}
\item
  \protect\hyperlink{link-5c96ab71}{What Trump's `delay the election'
  tweet could set us up for come November.}
\end{itemize}

\includegraphics{https://static01.nyt.com/images/2020/07/31/us/politics/31elections-briefing-HARRIS/merlin_173849235_dee40a44-5e7a-4843-8cb8-4e23e27ebca2-articleLarge.jpg?quality=75\&auto=webp\&disable=upscale}

\hypertarget{kamala-harris-a-top-vice-presidential-contender-confronts-double-standards}{%
\subsection{Kamala Harris, a top vice-presidential contender, confronts
double
standards.}\label{kamala-harris-a-top-vice-presidential-contender-confronts-double-standards}}

With Joseph R. Biden Jr. considering her as his running mate, Senator
Kamala Harris of California on Friday pushed back against criticism that
she was too ambitious, a characterization often used as a double
standard against women and one that some allies of Mr. Biden have used
about her recently.

``There will be a resistance to your ambition,'' she said during Black
Girls Lead 2020, a virtual conference for young Black women. ``There
will be people who say to you, `You are out of your lane,' because they
are burdened by only having the capacity to see what has always been
instead of what can be. But don't you let that burden you.''

It was a broad statement of the sort that leaders often make in
motivational speeches to young people, and Ms. Harris never mentioned
the recent criticism specifically. But it carried an added subtext,
intentional or not, because of the language some Biden allies have used
to argue that Mr. Biden should not choose her.

Politico
\href{https://www.politico.com/news/2020/07/27/kamala-harris-biden-vp-381829}{reported
on Monday} that former Senator Christopher J. Dodd, a member of Mr.
Biden's search committee, had complained privately that Ms. Harris had
not apologized for
\href{https://www.nytimes.com/2019/06/27/us/politics/kamala-harris-busing-joe-biden.html}{attacking
Mr. Biden} in a Democratic primary debate last year. (``She had no
remorse,'' Mr. Dodd told a Biden supporter, according to Politico.) And
CNBC
\href{https://www.cnbc.com/2020/07/29/biden-allies-move-to-stop-kamala-harris-from-becoming-vice-president.html}{reported
on Wednesday} that unnamed allies of Mr. Biden considered Ms. Harris
``too ambitious'' because she ran for president herself and might want
to do so again.

This week, a Biden campaign official
\href{https://www.nytimes.com/2020/07/31/us/politics/joseph-biden-vice-president.html}{reached
out to The New York Times}, unprompted, to say that some of the former
vice president's own staff members did not support her as well.

``Too ambitious'' is a common criticism of women in politics,
\href{https://www.nytimes.com/2019/02/11/us/politics/sexism-double-standard-2020.html}{but
is rarely levied against men}. One study,
\href{https://gap.hks.harvard.edu/price-power-power-seeking-and-backlash-against-female-politicians}{released
by Harvard researchers in 2010}, found that voters expressed contempt
and anger toward women whom they perceived as ``power-seeking,'' but saw
power-seeking men as stronger and more competent.

In her appearance on Friday --- which
\href{https://www.cnn.com/2020/07/31/politics/kamala-harris-ambition-remarks/index.html}{was
first reported by CNN}, and audio of which was provided to The Times ---
Ms. Harris returned repeatedly to the theme of women, and especially
Black women, being told to ``wait their turn'' or not demand too much.
Her mother, she said, used to tell her, ``People will be fine when you
take what they give you, but oh, don't take more.''

``It has happened my entire career, my entire career, where people
literally look at you like, `How dare you literally walk in a room?' and
challenge your very existence, and challenge you when you exercise your
authority,'' she said. ``Where people would like to go to a place of
thinking that you are out of your lane, or that you're uppity, or you
need to go back to your place.''

``I want you to be ambitious,'' she told the students participating in
the event.

Ms. Harris is not the only vice-presidential contender who has spoken
out, whether directly or implicitly, about double standards in politics.

Representative Karen Bass of California, another top candidate in Mr.
Biden's search, expressed annoyance on Friday about the frequent
contrasts drawn between her and Ms. Harris.

``Why are you comparing me with her?'' she asked on ``The Breakfast
Club,'' a syndicated radio show. ``Why don't you compare Whitmer with
Warren?''

Gov. Gretchen Whitmer of Michigan and Senator Elizabeth Warren of
Massachusetts have both been vetted by Mr. Biden's team, and both are
white. Ms. Harris and Ms. Bass are both Black.

\hypertarget{karen-bass-and-susan-rice-are-rising-on-bidens-vice-presidential-shortlist}{%
\subsection{Karen Bass and Susan Rice are rising on Biden's
vice-presidential
shortlist.}\label{karen-bass-and-susan-rice-are-rising-on-bidens-vice-presidential-shortlist}}

Image

Representative Karen Bass has~risen recently to the top of Joseph R.
Biden Jr.'s shortlist for a running mate.Credit...Anna Moneymaker for
The New York Times

Mr. Biden, the presumptive Democratic nominee, plans to announce his
running mate the week before the Democratic convention --- which begins
on Aug. 17 --- later than his previously stated target of Aug. 1.

Senator Kamala Harris of California has long been seen as a
front-runner, but two lesser-known candidates --- Representative Karen
Bass of California and the former national security adviser Susan E.
Rice --- have also risen to the top of Mr. Biden's shortlist recently,
as the Times political reporters Jonathan Martin, Alexander Burns and
Katie Glueck
\href{https://www.nytimes.com/2020/07/31/us/politics/joseph-biden-vice-president.html}{wrote
on Friday}.

All three are Black, which has been an important consideration for Mr.
Biden and his selection team given the importance of Black voters, and
especially Black women, as a Democratic constituency.

Other factors in Mr. Biden's decision include whether he believes a
candidate will spend time in the White House pursuing their own
political goals, and how much of a target he thinks they will be for
President Trump.

\hypertarget{trump-says-russian-bounties-to-kill-us-troops-never-took-place}{%
\subsection{Trump says Russian bounties to kill U.S. troops `never took
place.'}\label{trump-says-russian-bounties-to-kill-us-troops-never-took-place}}

Image

President Trump at a coronavirus response and storm preparedness meeting
in Belleair, Fla.Credit...Al Drago for The New York Times

Mr. Trump on Friday dismissed intelligence that Russia had covertly
offered and paid bounties for the killing of American soldiers in
Afghanistan as ``another Russia hoax'' and suggested that he believes
that ``it never took place.''

\hypertarget{latest-updates-2020-election}{%
\section{\texorpdfstring{\href{https://www.nytimes.com/2020/07/31/us/elections/biden-vs-trump.html?action=click\&pgtype=Article\&state=default\&region=MAIN_CONTENT_1\&context=storylines_live_updates}{Latest
Updates: 2020
Election}}{Latest Updates: 2020 Election}}\label{latest-updates-2020-election}}

Updated 2020-08-01T01:26:45.732Z

\begin{itemize}
\tightlist
\item
  \href{https://www.nytimes.com/2020/07/31/us/elections/biden-vs-trump.html?action=click\&pgtype=Article\&state=default\&region=MAIN_CONTENT_1\&context=storylines_live_updates\#link-29fdff45}{Kamala
  Harris, a top vice-presidential contender, confronts double
  standards.}
\item
  \href{https://www.nytimes.com/2020/07/31/us/elections/biden-vs-trump.html?action=click\&pgtype=Article\&state=default\&region=MAIN_CONTENT_1\&context=storylines_live_updates\#link-13ec3d9c}{Karen
  Bass and Susan Rice are rising on Biden's vice-presidential
  shortlist.}
\item
  \href{https://www.nytimes.com/2020/07/31/us/elections/biden-vs-trump.html?action=click\&pgtype=Article\&state=default\&region=MAIN_CONTENT_1\&context=storylines_live_updates\#link-49e9a016}{Trump
  says Russian bounties to kill U.S. troops `never took place.'}
\end{itemize}

\href{https://www.nytimes.com/2020/07/31/us/elections/biden-vs-trump.html?action=click\&pgtype=Article\&state=default\&region=MAIN_CONTENT_1\&context=storylines_live_updates}{See
more updates}

Mr. Trump was asked about the matter during a round table on the
coronavirus and hurricane response in Tampa, Fla. A reporter asked the
president whether he regularly reads his daily intelligence briefing,
which notified him of the suspected bounties in February.

``I read it all the time. I see it all the time. It was never brought to
my attention,'' Mr. Trump replied, repeating his assertion that he was
not told about the intelligence. ``I think it's another Russia hoax.
They've been giving me the Russia hoax --- shifty Schiff, all these
characters --- from the day I got here,'' he said, using his nickname
for the top Democrat on the House Intelligence Committee, Representative
Adam Schiff of California.

Mr. Trump has not disputed that the intelligence appeared in his daily
briefing, which --- despite his claims on Friday --- he is said to read
infrequently. Officials say Mr. Trump absorbs most classified
information through verbal briefings, but the White House has said that
his C.I.A. briefer chose not to raise it with him.

Mr. Trump, always loath to criticize Russia and its president, Vladimir
V. Putin, then quickly raised an area of hoped-for cooperation with the
Kremlin. ``We're working with Russia right now on a nonproliferation
agreement --- nuclear nonproliferation, and if we get something like
that it'd be great,'' he said.

Mr. Trump repeated his
\href{https://www.nytimes.com/2020/07/29/us/politics/trump-putin-bounties.html}{past
assertions} that intelligence officials doubted whether Russia really
did pay bounties to Taliban-affiliated fighters. But the C.I.A. has
assessed with medium confidence that the payments were made, and the
U.S. government is still investigating the matter. National Security
Agency analysts had lower confidence in the assessment.

Mr. Trump approvingly cited unnamed ``people'' who he said believe that
the bounties were never paid. ``It was never brought to my attention and
--- because it didn't reach the level, there were a lot of people ---
including Democrats --- that said, it never took place,'' he said.

``Perhaps wasn't brought because they didn't consider it to be real,''
he later added. ``And if it is brought to my attention, I'll do
something about it.''

\hypertarget{a-trump-visit-to-the-tampa-airport-becomes-a-campaign-rally-with-low-turnout}{%
\subsection{A Trump visit to the Tampa airport becomes a campaign rally
with low
turnout.}\label{a-trump-visit-to-the-tampa-airport-becomes-a-campaign-rally-with-low-turnout}}

Image

Supporters of President Trump on the tarmac of Tampa International
Airport on Friday.Credit...Al Drago for The New York Times

Earlier Friday, Mr. Trump held a sparsely attended pseudo-rally on a
blazing airport tarmac in Florida, appearing with local sheriffs as he
denounced ``radical socialist Marxist anarchists'' and attacked Mr.
Biden's mental competence.

Mr. Trump's outdoor event at the Tampa airport, staged just yards away
from Air Force One, was billed as a ``Campaign Coalitions Event'' with
sheriffs, but had many of the trappings of the campaign rallies an
affection-starved Mr. Trump has been denied by the coronavirus. They
included a soundtrack drawn from those events --- including Lee
Greenwood's ``God Bless the U.S.A.,'' which blared as Mr. Trump walked
down the steps of Air Force One.

The event was attended by a few hundred supporters who braved
temperatures in the mid-90s. But crowd barriers set up for the event
bracketed off what turned out to be ample empty space, suggesting that
the Trump campaign had anticipated the possibility of many more
attendees.

Hailing the sheriffs who joined him as frontline heroes against the
specter of criminal mayhem, Mr. Trump vowed to ``drive out the fanatical
anti-police ideology of the radical socialist Marxist anarchists and
agitators.'' He also said that Mr. Biden ``has no idea what he's talking
about. He just reads whatever they put --- they put it there, reads it.
Sometimes he reads it and sometimes he doesn't. Pretty soon he won't be
able to do that, either.''

Mr. Trump also appeared to make light of an incident from last July,
when attendees at a rally in North Carolina
\href{https://www.nytimes.com/2019/07/17/us/politics/trump-send-her-back-ilhan-omar.html}{chanted
``send her back''} at the mention of a Somali-born congresswoman whom
the president has called anti-American.

After Mr. Trump mentioned Representative Ilhan Omar, Democrat of
Minnesota --- who, he said, ``doesn't love our country'' --- an audience
member called out something in response. That remark, inaudible to
reporters, was clearly audible to Mr. Trump.

``I won't say what you said,'' Mr. Trump replied. ``Last time that
happened, they said --- they made the exact same --- don't worry, they
made the exact same statement as you. And they criticized me for not
getting angry at the people in the audience. Remember?''

``So, we won't say it,'' he added. ``But they know what I mean.''

\hypertarget{house-committee-ratchets-up-inquiry-into-whether-pompeo-tried-to-help-smear-biden-and-his-son}{%
\subsection{House committee ratchets up inquiry into whether Pompeo
tried to help smear Biden and his
son.}\label{house-committee-ratchets-up-inquiry-into-whether-pompeo-tried-to-help-smear-biden-and-his-son}}

Image

The House Foreign Affairs Committee is investigating whether Secretary
of State Mike Pompeo helped smear President Trump's political rival, Mr.
Biden.Credit...Doug Mills/The New York Times

The Democratic-led House Foreign Affairs Committee has subpoenaed
Secretary of State Mike Pompeo, escalating its investigation into
whether Mr. Pompeo aided election-year attempts to smear Mr. Biden, the
president's political rival.

The subpoena, announced on Friday, demands that Mr. Pompeo turn over
records related to Mr. Biden and his son that the State Department has
already produced to Republican-led Senate panels.

Senate Republicans on the Judiciary Committee and the Homeland Security
and Governmental Affairs Committee in recent months
\href{https://www.nytimes.com/2020/05/20/us/politics/trump-biden-subpoena.html}{have
labored to resurrect unsubstantiated claims} that Mr. Biden's son helped
a Ukrainian energy firm curry favor with the Obama administration when
his father was vice president, and granted themselves subpoena power
aimed at uncovering potential wrongdoing.

Representative Eliot L. Engel, Democrat of New York and the chairman of
the panel, said on Friday that his committee's investigators had learned
that the State Department has produced 16,080 pages of material to the
Senate committees since February, but had declined to produce those same
documents to his panel.

``Secretary Pompeo has turned the State Department into an arm of the
Trump campaign and he's not even trying to disguise it,'' Mr. Engel
said. ``After trying to stonewall virtually every oversight effort by
the Foreign Affairs Committee in the last two years, Mr. Pompeo is more
than happy to help Senate Republicans advance their conspiracy theories
about the Bidens.''

Separately, members of Mr. Engel's committee said on Friday they were
``extremely alarmed'' by assertions that the American ambassador in
Brazil had signaled to Brazilian officials they could help get Mr. Trump
re-elected by changing their trade policies. Reports in the Brazilian
news media said the ambassador, Todd Chapman, made it clear that Mr.
Trump's electoral chances in Iowa, a potential swing state, could be
boosted if Brazil lifted its ethanol tariffs.

In a letter to the ambassador, Todd Chapman, Mr. Engel said that such
discussions could be a violation of the Hatch Act, a 1939 law that bars
federal officials from engaging in certain partisan activities.

\hypertarget{trump-citing-no-evidence-of-voter-fraud-warns-the-election-will-be-rigged}{%
\subsection{Trump, citing no evidence of voter fraud, warns the election
will be
`rigged.'}\label{trump-citing-no-evidence-of-voter-fraud-warns-the-election-will-be-rigged}}

\includegraphics{https://static01.nyt.com/images/2020/07/31/business/31elections-briefing-wh/merlin_175160550_5e5663db-98e4-4397-9668-8f28aaf0d61e-videoSixteenByNine3000.jpg}

A day after Mr. Trump raised the possibility of delaying the November
election, which he cannot do, he and his administration renewed their
attacks on voting by mail despite a lack of evidence that it leads to
significant voter fraud.

Mr. Trump, who polls show is facing the prospect of a decisive loss to
Mr. Biden, issued dire new warnings on Friday that the outcome of the
election might be delayed and unverifiable, using the coronavirus
pandemic to sow concerns about ballot integrity.

``This is going to be the greatest election disaster in history,'' Mr.
Trump told reporters at the White House after a meeting with national
police union leaders and before his trip to Florida.

``It'll be fixed,'' he added, although he did not repeat his call to
delay the Nov. 3 vote. ``It'll be rigged.''

Mr. Trump called absentee ballots ``great,'' saying, ``you have to go
through a process to get them.'' But he warned that the mail-in ballots
many states plan to distribute in large numbers are susceptible to
widespread fraud, even though few election experts agree.

Mr. Trump made several references to a June 23 primary election in New
York City whose
\href{https://www.nytimes.com/2020/07/17/nyregion/election-absentee-ballots-primary.html}{results
have been delayed} by problems counting a huge number of mail-in
ballots. But although that process has been painstakingly slow, and at
least one candidate has raised concerns about rejected ballots, there
has been no suggestion that voter fraud was involved.

``Look at New York,'' Mr. Trump said. ``They had a race, a small race by
comparison. By comparison, tiny. It's so messed up they have no idea.''

Kayleigh McEnany, the White House press secretary, echoed the
president's attacks during a media briefing on Friday and repeatedly
backed away from supporting any proposals to provide states funding for
elections to ensure the process could be made more secure.

``States run their elections, and it is up to states to make sure that
they have the capacity,'' Ms. McEnany said.

\href{https://www.nytimes.com/article/mail-in-voting-explained.html}{Voter
fraud is rare} in the United States, and mail-in voting is not a new or
untested idea. Five states --- Washington, Oregon, Colorado, Utah, and
Hawaii --- already conduct voting primarily by mail.

But Mr. Trump has repeatedly tried to sow doubt about the mail-in voting
process, despite having voted by mail himself before. His
administration's stance is at odds with an increasing number of
Americans who support the process, particularly as the coronavirus
pandemic
\href{https://www.nytimes.com/2020/07/31/us/politics/trump-mail-voting-fraud.html}{worsens}.

A growing number of White House officials have publicly spoken out
against mail-in voting in recent days, spreading falsehoods alongside
Mr. Trump. Stephen Miller, a senior adviser to Mr. Trump, falsely
claimed on Fox News on Friday morning that people who vote by mail do
not have their identities checked.

This is not true. There are several ways that states verify voter data,
including collecting signatures, birthdays, drivers license numbers and
partial Social Security numbers, which are then compared against
existing voter rolls.

Asked on Friday whether he wants the election delayed, as he suggested
on Thursday, Mr. Trump insisted that ``nobody wants that date more than
me.''

But he then seemed to call again for a new election date --- this time,
an earlier one.

``I wish we would move it up, OK? Move it up,'' he said.

\hypertarget{what-trumps-delay-the-election-tweet-could-set-us-up-for-come-november}{%
\subsection{What Trump's `delay the election' tweet could set us up for
come
November.}\label{what-trumps-delay-the-election-tweet-could-set-us-up-for-come-november}}

It is always risky to read too much into Mr. Trump's tweets and offhand
remarks to reporters. To what degree was he
\href{https://www.nytimes.com/2020/07/29/us/politics/trump-suburbs-housing-white-voters.html}{making
an explicitly race-baiting appeal to suburban homeowners} by promising
to block the construction of low-income housing in their backyards?
Would he actually try (the Constitution notwithstanding) to postpone the
election,
\href{https://www.nytimes.com/2020/07/30/us/politics/trump-delay-2020-election.html}{as
he suggested on Thursday}?

These could be the unpremeditated remarks of a public figure who knows
how to roil the water, and --- from his years playing the corners in the
famously raucous New York City media market --- how to change the
subject. Mr. Trump's tweet on elections came after the release of a
report that noted the economy was
\href{https://www.nytimes.com/2020/07/30/business/economy/q2-gdp-coronavirus-economy.html}{contracting
at a record rate}.

But whether by design or not, Mr. Trump's latest attack on voting, less
than 100 days before the election, sows distrust in one of the basic
pillars of the American system at a time when the country is culturally
and politically polarized, confronting regular demonstrations and
battered by an out-of-control pandemic.

These remarks set the groundwork for disputing the outcome of a close
election, should he lose to
\href{https://www.nytimes.com/interactive/2020/us/elections/joe-biden.html}{Joseph
R. Biden Jr.}, empowering his supporters, Republican politicians and
lawyers to reject the result if it is not to his liking. That could take
the form of recounts, court battles or protests.

The weeks after Election Day --- rather than being a time for transition
and healing if Mr. Biden wins, or preparations for a second term if Mr.
Trump wins --- could end up being a period of chaos that eclipses the
level of disruption Florida witnessed in the closing days of 2000 after
the disputed election between George W. Bush and Al Gore.

It seems noteworthy that when Mr. Trump questioned postponing the
election,
\href{https://www.nytimes.com/2020/07/30/us/politics/trump-delay-2020-election.html}{pushback
came from the Republicans who have been his most unquestioning
supporters}, among them, Senator Mitch McConnell of Kentucky, the
majority leader, and Senator Marco Rubio of Florida. The question now is
whether those men, and other congressional supporters of Mr. Trump's,
will be back at his side if the president comes to dispute the
legitimacy of the election.

\hypertarget{a-new-research-paper-predicts-a-wave-of-lost-votes-this-fall-further-complicating-the-election}{%
\subsection{A new research paper predicts a wave of `lost votes' this
fall, further complicating the
election.}\label{a-new-research-paper-predicts-a-wave-of-lost-votes-this-fall-further-complicating-the-election}}

Image

Election workers counting mail-in absentee ballots from New York's
primary.Credit...Victor J. Blue for The New York Times

A new paper by an M.I.T. elections expert predicts that the outcome of
this year's presidential election --- and the problem known as the
``lost vote,'' in which legitimate ballots go uncounted --- could fuel
postelection allegations of a rigged election. The
\href{https://papers.ssrn.com/sol3/papers.cfm?abstract_id=3660625}{paper,
by Charles Stewart III}, highlights how Mr. Trump's claims about
problems with mail balloting could drag on long after November if the
election is close.

A ``lost vote'' occurs when a voter does everything necessary to vote
but, thanks to administrative errors, the vote isn't counted in the
final tally, according to Mr. Stewart, a professor of political science.
In
\href{https://www.nytimes.com/2020/07/25/us/politics/georgia-election-voting-problems.html}{Georgia's
June primaries}, for example, ballot scanners did not count mail ballots
when voters used check marks instead of filling in the ovals.

In the paper, released this week, Mr. Stewart concludes that, for a
variety of reasons, while lost votes are rare, they occur more
frequently when mail-in ballots are used.

In the 2016 election, he writes, approximately 4 percent of the mail
ballots cast --- or 1.4 million votes --- went uncounted. With more
states now embracing mail ballots, including a number of states with
little experience with voting by mail, Mr. Stewart predicts a
``disproportionate growth'' in the number of lost votes in November.

``Despite the clear public health imperative that mail balloting be
increased in the 2020 primary season and general election, the expansion
of mail balloting comes with risks,'' Mr. Stewart wrote. ``To be sure,
these risks are small, and should not be sensationalized.''

In an interview Friday, Mr. Stewart said that some of his past writing
on the topic had been misinterpreted, including in recent legal
documents that used his research to sensationalize the risk of lost
votes. ``It's not a basket of votes being hidden or mailed in from
Siberia or name-your-fantasy story,'' he said.

Nevertheless, given the tenor of the 2020 election season so far,
uncounted mail votes will likely become fodder for legal disputes,
particularly if one or two swing states are too close to call, he
predicts.

``The great risk of an increase in mail ballots in 2020 may not befall
individual voters as much as it affects any postelection controversy
over whether the election was `rigged' or legitimate,'' the paper
concluded.

\hypertarget{the-trump-campaign-temporarily-suspends-tv-advertising-to-review-its-strategy}{%
\subsection{The Trump campaign temporarily suspends TV advertising to
review its
strategy.}\label{the-trump-campaign-temporarily-suspends-tv-advertising-to-review-its-strategy}}

The Trump campaign has completely gone off the air, temporarily
suspending all television advertising nationwide as the campaign
undertakes a ``review'' of its advertising strategy under the new
campaign manager, Bill Stepien.

``With the leadership change in the campaign, there's understandably a
review and fine-tuning of the campaign's strategy,'' a campaign official
said. ``We'll be back on the air shortly.''

The pause comes after several weeks of attacks against Mr. Trump's
opponent, Mr. Biden, on policing issues. The campaign spent more than
\$30 million since early July on television and digital ads that sought
to sow fear and division about the racial justice protests around the
country and falsely depict them as violent.

Mr. Trump's campaign has been prolific on the airwaves since last
September, when it began advertising during the impeachment process, and
it has continued at a significant pace. Since last January, the campaign
has spent \$202 million in television and digital advertising, according
to Advertising Analytics, an ad tracking firm.

Mr. Biden, by comparison, has spent about \$95 million over the same
period.

While it has temporarily paused its advertising, the Trump campaign
still has more than \$146 million in television and radio ads booked
through November, a number that far outpaces the Biden campaign. None of
those reservations have been altered or shifted yet as part of the
current review.

Mr. Trump
\href{https://twitter.com/realDonaldTrump/status/1289275311272521728}{wrote
on Twitter} on Friday that his campaign would roll out ``a new ad
campaign'' attacking Mr. Biden on Monday. ``He has been brought even
further LEFT than Crazy Bernie Sanders ever thought possible,'' the
president added.

Though the campaign is not on television at the moment, it is still
advertising on Facebook, with dozens of active ads. The campaign spent
nearly \$4 million on the platform over the past week.

The complete pause in advertising followed the campaign's recent
decision to
\href{https://www.nytimes.com/2020/07/29/us/politics/michigan-trump-biden-2020.html}{suspend
advertising in Michigan}, a battleground state that Mr. Trump won by
less than 11,000 votes in 2016.

\hypertarget{the-house-formally-reprimands-a-congressman-for-breaking-campaign-finance-laws}{%
\subsection{The House formally reprimands a congressman for breaking
campaign finance
laws.}\label{the-house-formally-reprimands-a-congressman-for-breaking-campaign-finance-laws}}

The House on Friday took the rare step of reprimanding one of its own
members for ethics violations, voting unanimously to fine Representative
David Schweikert, Republican of Arizona, \$50,000 for breaking campaign
finance laws.

The vote came after the House Ethics Committee
\href{https://www.nytimes.com/2020/07/30/us/david-schweikert-ethics-rules-violations.html}{found
that Mr. Schweikert violated 11 House ethics rules}. It marked the first
time since 2012 that the House has reprimanded one of its members for an
ethics violation.

``There is no joy in reprimanding one of our colleagues,'' said
Representative Dean Phillips, Democrat of Minnesota, who helped lead the
investigation into Mr. Schweikert's actions.

The committee said a two-year investigation had found
``\href{https://ethics.house.gov/press-releases/statement-chairman-and-ranking-member-committee-ethics-regarding-representative-13}{substantial
reason to believe}'' that Mr. Schweikert had violated House rules, the
Code of Ethics for Government Service, federal laws and other standards.

It cited Mr. Schweikert for campaign finance violations and errors in
reporting by his campaign committees; the misuse of his congressional
allowance; pressuring official staff members to perform campaign work;
and his ``lack of candor and due diligence'' during the investigation.

House investigators found that Mr. Schweikert's campaign had accepted
more than \$270,000 from his then-chief of staff, in violation of
campaign finance laws. The former chief of staff also testified that he
was pressured to raise money for the congressman's campaign,
\href{https://ethics.house.gov/sites/ethics.house.gov/files/documents/Committee\%20Report_19.pdf}{a
committee report} said.

Mr. Schweikert had agreed to a \$50,000 fine as part of an agreement to
end the investigation.

\hypertarget{wisconsins-new-state-supreme-court-justice-will-be-sworn-in-as-she-runs-a-100-mile-ultramarathon}{%
\subsection{Wisconsin's new State Supreme Court justice will be sworn in
as she runs a 100-mile
ultramarathon.}\label{wisconsins-new-state-supreme-court-justice-will-be-sworn-in-as-she-runs-a-100-mile-ultramarathon}}

Image

Jill Karofsky in November last year. After winning a spot on the
Wisconsin Supreme Court in April, Ms. Karofsky will be sworn in during a
100-mile ultramarathon on Saturday.Credit...John Hart/Wisconsin State
Journal, via Associated Press

Back in April,
\href{https://www.nytimes.com/2020/04/13/us/politics/wisconsin-primary-results.html}{Jill
Karofsky became just the second person in 50 years} to defeat a sitting
Wisconsin Supreme Court justice. A liberal judge on the state circuit
court, Ms. Karofsky is planning to be sworn in at mile 35 of a 100-mile
ultramarathon she is running in Wisconsin starting at 6 a.m. Saturday.

Ms. Karofsky, who has run several marathons, will have a say in cases
dealing with voting rights and Wisconsin's pandemic response, and she
spoke with our reporter Reid J. Epstein this week. The conversation was
edited and condensed.

\textbf{OK, why are you being sworn in during a 100-mile race?}

I was supposed to do a 100-mile race and I just thought, ``Why not get
sworn in in the middle of this 100-mile run and make it a little more of
a bigger deal than it might be otherwise and than other people have done
in the past?''

\textbf{What is special about the 35-mile mark to be sworn in there?}

Five miles past Belleville is an old bar called Dot's Tavern. And when
you're actually doing the run, the deal is you have to run into the
tavern, go down to the basement of the tavern, and get a coaster to
prove that you were there. So yeah. It's just for shtick. And that's at
35 miles.

\textbf{How long is it going to take you to run 100 miles?}

I'd be happy to finish around 30 hours, so I'd like to be, I'm hoping to
be done by noon on Sunday.

\textbf{So help me with the math, what sort of pace does that mean
you'll be running?}

Oh, it's about 13-minute miles. But you also have to factor in I'm going
to lose at least 30, 45 minutes getting sworn in.

\textbf{Are people running with you or is this all by yourself?}

I have a couple of friends who are going to meet me out there at
different sections of the course. One of my friends is going to run
through the night with me.

\textbf{How is keeping this sort of running regimen helpful to doing a
job like being a Supreme Court justice?}

No matter what your job is, when you sit down to do your job, to have
the clearest mind possible is how we all perform the best. And I think
that making important decisions on behalf of the state of Wisconsin, if
I can come at those decisions from a place where my mind is clear and
I'm not making decisions from a place of stress, then I can perform at
my best.

Reporting was contributed by Maggie Astor, Luke Broadwater, Alexander
Burns, Emily Cochrane, Nick Corasaniti, Michael Crowley, Catie
Edmondson, Reid J. Epstein, Katie Glueck, Mark Landler, Jonathan Martin,
Adam Nagourney, Jeremy W. Peters, Katie Rogers, Giovanni Russonello and
Stephanie Saul.

\hypertarget{our-2020-election-guide}{%
\section{Our 2020 Election Guide}\label{our-2020-election-guide}}

Updated July 31, 2020

\begin{itemize}
\item
  \begin{center}\rule{0.5\linewidth}{\linethickness}\end{center}

  \hypertarget{the-latest}{%
  \subsection{The Latest}\label{the-latest}}

  \begin{itemize}
  \tightlist
  \item
    President Trump's assault on the Postal Service is intersecting with
    his attacks on mail-in voting.
    \href{https://www.nytimes.com/2020/07/31/us/politics/trump-usps-mail-delays.html?action=click\&pgtype=Article\&state=default\&region=BELOW_MAIN_CONTENT\&context=storylines_guide}{Voting
    rights groups say it is a recipe for disaster.}
  \end{itemize}
\item
  \begin{center}\rule{0.5\linewidth}{\linethickness}\end{center}

  \hypertarget{bidens-vp-search}{%
  \subsection{Biden's V.P. Search}\label{bidens-vp-search}}

  \begin{itemize}
  \tightlist
  \item
    \href{https://www.nytimes.com/article/biden-vice-president-2020.html?action=click\&pgtype=Article\&state=default\&region=BELOW_MAIN_CONTENT\&context=storylines_guide}{Here
    are 13 women} who have been under consideration to be Joe Biden's
    running mate, and why each might be chosen --- and might not be.
  \end{itemize}
\item
  \begin{center}\rule{0.5\linewidth}{\linethickness}\end{center}

  \hypertarget{keep-up-with-our-coverage}{%
  \subsection{Keep Up With Our
  Coverage}\label{keep-up-with-our-coverage}}

  \begin{itemize}
  \tightlist
  \item
    Get an
    \href{https://www.nytimes.com/newsletters/politics?action=click\&pgtype=Article\&state=default\&region=BELOW_MAIN_CONTENT\&context=storylines_guide}{email}
    recapping the day's news
  \end{itemize}

  \begin{itemize}
  \tightlist
  \item
    Download our mobile app on
    \href{https://apps.apple.com/us/app/nytimes/id284862083?ls=1\&mat_click_id=5c79ae7455014fd1bd66b5610c05b8f2-20191112-16948\&referrer=mat_click_id\%3D5c79ae7455014fd1bd66b5610c05b8f2-20191112-16948\%26link_click_id\%3D722930677036718082}{iOS}
    and
    \href{http://a.localytics.com/android?id=com.nytimes.android\&referrer=utm_source\%3Dother_nyt_mobile_web\%26utm_medium\%3DWeb\%2520page\%26utm_term\%3DGeneral\%2520Mobile\%2520Page\%26utm_campaign\%3DNYT\%2520Mobile\%2520General\%2520Page}{Android}
    and turn on Breaking News and Politics alerts
  \end{itemize}
\end{itemize}

Advertisement

\protect\hyperlink{after-bottom}{Continue reading the main story}

\hypertarget{site-index}{%
\subsection{Site Index}\label{site-index}}

\hypertarget{site-information-navigation}{%
\subsection{Site Information
Navigation}\label{site-information-navigation}}

\begin{itemize}
\tightlist
\item
  \href{https://help.nytimes.com/hc/en-us/articles/115014792127-Copyright-notice}{©~2020~The
  New York Times Company}
\end{itemize}

\begin{itemize}
\tightlist
\item
  \href{https://www.nytco.com/}{NYTCo}
\item
  \href{https://help.nytimes.com/hc/en-us/articles/115015385887-Contact-Us}{Contact
  Us}
\item
  \href{https://www.nytco.com/careers/}{Work with us}
\item
  \href{https://nytmediakit.com/}{Advertise}
\item
  \href{http://www.tbrandstudio.com/}{T Brand Studio}
\item
  \href{https://www.nytimes.com/privacy/cookie-policy\#how-do-i-manage-trackers}{Your
  Ad Choices}
\item
  \href{https://www.nytimes.com/privacy}{Privacy}
\item
  \href{https://help.nytimes.com/hc/en-us/articles/115014893428-Terms-of-service}{Terms
  of Service}
\item
  \href{https://help.nytimes.com/hc/en-us/articles/115014893968-Terms-of-sale}{Terms
  of Sale}
\item
  \href{https://spiderbites.nytimes.com}{Site Map}
\item
  \href{https://help.nytimes.com/hc/en-us}{Help}
\item
  \href{https://www.nytimes.com/subscription?campaignId=37WXW}{Subscriptions}
\end{itemize}
