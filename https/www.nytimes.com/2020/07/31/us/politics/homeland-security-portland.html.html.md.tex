Sections

SEARCH

\protect\hyperlink{site-content}{Skip to
content}\protect\hyperlink{site-index}{Skip to site index}

\href{https://www.nytimes.com/section/politics}{Politics}

\href{https://myaccount.nytimes.com/auth/login?response_type=cookie\&client_id=vi}{}

\href{https://www.nytimes.com/section/todayspaper}{Today's Paper}

\href{/section/politics}{Politics}\textbar{}Homeland Security Shuts Down
`Intelligence' Reports on Journalists

\url{https://nyti.ms/3giO0Wf}

\begin{itemize}
\item
\item
\item
\item
\item
\end{itemize}

\href{https://www.nytimes.com/news-event/george-floyd-protests-minneapolis-new-york-los-angeles?action=click\&pgtype=Article\&state=default\&region=TOP_BANNER\&context=storylines_menu}{Race
and America}

\begin{itemize}
\tightlist
\item
  \href{https://www.nytimes.com/2020/07/26/us/protests-portland-seattle-trump.html?action=click\&pgtype=Article\&state=default\&region=TOP_BANNER\&context=storylines_menu}{Protesters
  Return to Other Cities}
\item
  \href{https://www.nytimes.com/2020/07/24/us/portland-oregon-protests-white-race.html?action=click\&pgtype=Article\&state=default\&region=TOP_BANNER\&context=storylines_menu}{Portland
  at the Center}
\item
  \href{https://www.nytimes.com/2020/07/23/podcasts/the-daily/portland-protests.html?action=click\&pgtype=Article\&state=default\&region=TOP_BANNER\&context=storylines_menu}{Podcast:
  Showdown in Portland}
\item
  \href{https://www.nytimes.com/interactive/2020/07/16/us/black-lives-matter-protests-louisville-breonna-taylor.html?action=click\&pgtype=Article\&state=default\&region=TOP_BANNER\&context=storylines_menu}{45
  Days in Louisville}
\end{itemize}

Advertisement

\protect\hyperlink{after-top}{Continue reading the main story}

Supported by

\protect\hyperlink{after-sponsor}{Continue reading the main story}

\hypertarget{homeland-security-shuts-down-intelligence-reports-on-journalists}{%
\section{Homeland Security Shuts Down `Intelligence' Reports on
Journalists}\label{homeland-security-shuts-down-intelligence-reports-on-journalists}}

The acting secretary of homeland security said that he would investigate
his department's dissemination of the tweets of journalists who
uncovered agency documents.

\includegraphics{https://static01.nyt.com/images/2020/07/31/us/politics/31dc-unrest-dhs/merlin_174842688_96e4e7ac-cc49-4672-ae63-4048294c934d-articleLarge.jpg?quality=75\&auto=webp\&disable=upscale}

\href{https://www.nytimes.com/by/zolan-kanno-youngs}{\includegraphics{https://static01.nyt.com/images/2019/12/13/reader-center/author-zolan-kanno-youngs/author-zolan-kanno-youngs-thumbLarge.png}}\href{https://www.nytimes.com/by/marc-tracy}{\includegraphics{https://static01.nyt.com/images/2018/02/20/multimedia/author-marc-tracy/author-marc-tracy-thumbLarge.jpg}}

By \href{https://www.nytimes.com/by/zolan-kanno-youngs}{Zolan
Kanno-Youngs} and \href{https://www.nytimes.com/by/marc-tracy}{Marc
Tracy}

\begin{itemize}
\item
  July 31, 2020
\item
  \begin{itemize}
  \item
  \item
  \item
  \item
  \item
  \end{itemize}
\end{itemize}

WASHINGTON --- The acting secretary of homeland security said on Friday
that he had shut down an intelligence examination of the work of
reporters covering the government's response to protests in Portland,
Ore., beginning an investigation into what he suggested was an
infringement on First Amendment rights.

The effort by the Department of Homeland Security's intelligence and
analysis directorate --- first revealed by
\href{https://www.washingtonpost.com/national-security/dhs-compiled-intelligence-reports-on-journalists-who-published-leaked-documents/2020/07/30/5be5ec9e-d25b-11ea-9038-af089b63ac21_story.html}{The
Washington Post} --- in part targeted
\href{https://www.nytimes.com/2020/07/28/us/federal-agents-portland-seattle-protests.html}{The
New York Times's release} of an intelligence analysis indicating that
even as federal agents in camouflage deployed to quell the protests in
Portland, the administration had little understanding of what it was
facing.

The acting secretary, Chad F. Wolf, ``is committed to ensuring that all
D.H.S. personnel uphold the principles of professionalism, impartiality
and respect for civil rights and civil liberties, particularly as it
relates to the exercise of First Amendment rights,'' said Alexei
Woltornist, the department's spokesman.

The intelligence office issued three ``open-source intelligence
reports'' in the past week that summarized the Twitter posts of a Times
reporter and the editor in chief for the blog Lawfare, noting that they
had published leaked unclassified documents.

Mr. Wolf ordered the intelligence arm to ``immediately discontinue
collecting information involving members of the press" once he found out
about the practice, Mr. Woltornist said.

One of the primary responsibilities of the Department of Homeland
Security is sharing information about national security threats to
state, local and federal law enforcement agencies. The gaps in
communication about such threats were among the motivating factors in
creating a central cabinet department to coordinate security efforts
after the Sept. 11, 2001, attacks. The department often distributes
reports to information-collecting ``fusion centers,'' which then
disseminate the intelligence to relevant agencies and police departments
across the United States.

But such efforts were intended to focus on those with connections to
terrorists or criminals, not journalists.

The directive comes as the agency and its leaders face backlash and
investigations for their actions in Portland, where tactical teams of
agents used tear gas and batons against protesters and forced
individuals into unmarked vehicles. Some in the large crowds have also
thrown rocks, bottles and commercial-grade fireworks at officers
guarding a federal courthouse in the city.

``This is highly disconcerting if true, which is why these things need
to be investigated,'' said John Cohen, who used to run the intelligence
office during the Obama administration. ``At the very least, they have a
perception problem because at no time should an intelligence community
organization be collecting and disseminating intelligence products on
U.S. journalists.''

``The politicization of intelligence or law enforcement activities is
highly problematic and there always has to be a separation between the
intelligence-gathering law enforcement activities from the political
agenda of the administration,'' added Mr. Cohen, who was also a senior
adviser for the Bush administration.

\includegraphics{https://static01.nyt.com/images/2020/07/31/us/politics/31dc-unrest-dhs2/merlin_175103346_337df78a-69c3-4411-88ac-262dd2553387-articleLarge.jpg?quality=75\&auto=webp\&disable=upscale}

It is not the first time the department has used resources to focus on
journalists. A Border Patrol agent, Jeffrey A. Rambo,
\href{https://www.nytimes.com/2018/07/12/business/jeffrey-rambo-james-wolfe-leaks.html}{was
the subject of a federal investigation} for obtaining the confidential
travel records of a Washington journalist and using them to press her
about her sources in 2017.

Emails obtained from Customs and Border Protection by the Reporters
Committee and the Committee to Protect Journalists showed that Mr. Rambo
was in touch with the F.B.I. around the time that he pressured the
journalist, who now works for The Times.

The reports issued by the department in recent days summarized tweets by
Benjamin Wittes, the editor in chief of Lawfare, who had shared multiple
internal documents, including one post on July 24 showing a memo warning
intelligence officers not to leak to the press.

The reports also included another tweet by Mr. Wittes showing an email
from Brian Murphy, the current acting under secretary for intelligence
and analysis, telling officers to begin referring to individuals
attacking federal facilities in Portland as ``VIOLENT ANTIFA
ANARCHISTS.''

Mr. Murphy said he and other leaders in the intelligence office made the
conclusion after reviewing so-called baseball cards of arrested
protesters to understand their motivations. It came just days after
officers with
\href{https://int.nyt.com/data/documenttools/portland-intelligence-assessment/65e0a41de45b7abc/full.pdf}{the
department issued an intelligence report} saying the agency had ``low
confidence'' that the attacks against the federal courthouse reflected a
broader threat.

``We lack insight into the motives for the most recent attacks,'' they
admitted in the briefing, which was first reported by The Times.

The Senate intelligence committee
\href{https://www.warner.senate.gov/public/_cache/files/e/a/eaa5a194-d357-475b-a880-f7aae8437437/204A4D77A82B5684AF04EACAB1D37719.2020-07-31-intelligence-to-ia-murphy-letter.pdf}{sent
a letter} on Friday to Mr. Murphy, pressing him for more information
about the matter.

Danielle Rhoades Ha, a spokeswoman for The Times, said that it was
critical that the department not use such tactics on journalists.

``The Department of Homeland Security has acknowledged that its
intelligence reporting system, designed to combat terrorism, has instead
been misused to target journalists who were reporting on the
controversial activities of federal law enforcement officers,'' she said
in a statement. ``It is imperative that D.H.S.'s investigation
determines how this happened and ensures it does not happen again.''

Mr. Wittes said there was nothing wrong with the agency sharing
information about his tweets, noting that it prepared daily news
clippings. ``But to frame it as intelligence work product is a really
odd thing,'' he said.

``If you're allowed to do this, what else are you allowed to do? If
you're allowed to keep tabs on and file reports about my social media,''
he added, ``are you allowed to gather all the other public record
material that exists about me and create a dossier?''

The American Civil Liberties Union denounced the agency's actions on
Friday, saying they were part of a larger pattern.

``Under Wolf's leadership, D.H.S. was
\href{https://www.aclu.org/blog/free-speech/freedom-press/government-detaining-and-interrogating-journalists-and-advocates-us}{caught
just last year} unconstitutionally targeting and building dossiers on
journalists reporting on conditions at the border,'' the group's senior
legislative counsel, Neema Signh Guliani, said in a statement. ``For
weeks, D.H.S. agents have been deliberately and brutally attacking
journalists covering the Portland protests. And documents show that
D.H.S. intelligence arm appears to be claiming authority it does not
have. This administration's assault on the First Amendment continues to
escalate.''

Gabe Rottman, a lawyer at the Reporters Committee for Freedom of the
Press, said that the department's focus on journalists had broader
implications.

``Federal law prohibits the creation of `dossiers' on journalists
precisely because doing so can morph into investigations of journalists
for news coverage that embarrasses the government, but that the public
has a right to know,'' he said in a statement.

Mr. Woltornist said Mr. Wolf found out about the practice from news
reports on Thursday night, prompting former officials to question the
stability of leadership in the agency. The Department of Homeland
Security has not had a Senate-confirmed secretary since President Trump
\href{https://www.nytimes.com/2019/04/07/us/politics/kirstjen-nielsen-dhs-resigns.html}{ousted
Kirstjen Nielsen in April 2019}.

``I'm concerned that what's happening is there's a lack of control and
when there's a lot of turnover, it takes a while to learn the DNA of the
department,'' said Michael Chertoff, a former secretary for homeland
security under President George W. Bush. ``It's very hard to get a hold
of the various components.''

Zolan Kanno-Youngs reported from Washington, and Marc Tracy from New
York.

Advertisement

\protect\hyperlink{after-bottom}{Continue reading the main story}

\hypertarget{site-index}{%
\subsection{Site Index}\label{site-index}}

\hypertarget{site-information-navigation}{%
\subsection{Site Information
Navigation}\label{site-information-navigation}}

\begin{itemize}
\tightlist
\item
  \href{https://help.nytimes.com/hc/en-us/articles/115014792127-Copyright-notice}{©~2020~The
  New York Times Company}
\end{itemize}

\begin{itemize}
\tightlist
\item
  \href{https://www.nytco.com/}{NYTCo}
\item
  \href{https://help.nytimes.com/hc/en-us/articles/115015385887-Contact-Us}{Contact
  Us}
\item
  \href{https://www.nytco.com/careers/}{Work with us}
\item
  \href{https://nytmediakit.com/}{Advertise}
\item
  \href{http://www.tbrandstudio.com/}{T Brand Studio}
\item
  \href{https://www.nytimes.com/privacy/cookie-policy\#how-do-i-manage-trackers}{Your
  Ad Choices}
\item
  \href{https://www.nytimes.com/privacy}{Privacy}
\item
  \href{https://help.nytimes.com/hc/en-us/articles/115014893428-Terms-of-service}{Terms
  of Service}
\item
  \href{https://help.nytimes.com/hc/en-us/articles/115014893968-Terms-of-sale}{Terms
  of Sale}
\item
  \href{https://spiderbites.nytimes.com}{Site Map}
\item
  \href{https://help.nytimes.com/hc/en-us}{Help}
\item
  \href{https://www.nytimes.com/subscription?campaignId=37WXW}{Subscriptions}
\end{itemize}
