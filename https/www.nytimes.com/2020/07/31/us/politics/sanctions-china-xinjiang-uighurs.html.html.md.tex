Sections

SEARCH

\protect\hyperlink{site-content}{Skip to
content}\protect\hyperlink{site-index}{Skip to site index}

\href{https://www.nytimes.com/section/politics}{Politics}

\href{https://myaccount.nytimes.com/auth/login?response_type=cookie\&client_id=vi}{}

\href{https://www.nytimes.com/section/todayspaper}{Today's Paper}

\href{/section/politics}{Politics}\textbar{}U.S. Adds Sanctions Over
Internment of Muslims in China

\url{https://nyti.ms/2PckmWL}

\begin{itemize}
\item
\item
\item
\item
\item
\end{itemize}

\href{https://www.nytimes.com/news-event/coronavirus?action=click\&pgtype=Article\&state=default\&region=TOP_BANNER\&context=storylines_menu}{The
Coronavirus Outbreak}

\begin{itemize}
\tightlist
\item
  live\href{https://www.nytimes.com/2020/08/04/world/coronavirus-covid-19.html?action=click\&pgtype=Article\&state=default\&region=TOP_BANNER\&context=storylines_menu}{Latest
  Updates}
\item
  \href{https://www.nytimes.com/interactive/2020/us/coronavirus-us-cases.html?action=click\&pgtype=Article\&state=default\&region=TOP_BANNER\&context=storylines_menu}{Maps
  and Cases}
\item
  \href{https://www.nytimes.com/interactive/2020/science/coronavirus-vaccine-tracker.html?action=click\&pgtype=Article\&state=default\&region=TOP_BANNER\&context=storylines_menu}{Vaccine
  Tracker}
\item
  \href{https://www.nytimes.com/2020/08/02/us/covid-college-reopening.html?action=click\&pgtype=Article\&state=default\&region=TOP_BANNER\&context=storylines_menu}{College
  Reopening}
\item
  \href{https://www.nytimes.com/live/2020/08/03/business/stock-market-today-coronavirus?action=click\&pgtype=Article\&state=default\&region=TOP_BANNER\&context=storylines_menu}{Economy}
\end{itemize}

Advertisement

\protect\hyperlink{after-top}{Continue reading the main story}

Supported by

\protect\hyperlink{after-sponsor}{Continue reading the main story}

\hypertarget{us-adds-sanctions-over-internment-of-muslims-in-china}{%
\section{U.S. Adds Sanctions Over Internment of Muslims in
China}\label{us-adds-sanctions-over-internment-of-muslims-in-china}}

The Treasury Department imposed sanctions on a powerful government
entity that runs companies and farms in the Xinjiang region, where
officials carry out the mass internment of Muslims.

\includegraphics{https://static01.nyt.com/images/2020/07/31/us/politics/31dc-china-sanctions/31dc-china-sanctions-articleLarge.jpg?quality=75\&auto=webp\&disable=upscale}

\href{https://www.nytimes.com/by/ana-swanson}{\includegraphics{https://static01.nyt.com/images/2018/12/10/multimedia/author-ana-swanson/author-ana-swanson-thumbLarge.png}}\href{https://www.nytimes.com/by/edward-wong}{\includegraphics{https://static01.nyt.com/images/2018/09/24/multimedia/author-edward-wong/author-edward-wong-thumbLarge-v5.png}}

By \href{https://www.nytimes.com/by/ana-swanson}{Ana Swanson} and
\href{https://www.nytimes.com/by/edward-wong}{Edward Wong}

\begin{itemize}
\item
  July 31, 2020
\item
  \begin{itemize}
  \item
  \item
  \item
  \item
  \item
  \end{itemize}
\end{itemize}

WASHINGTON --- The Trump administration announced sanctions Friday on a
powerful government entity and two senior officials who have helped
manage it, citing
\href{https://www.nytimes.com/2020/05/09/us/politics/china-uighurs-arrest.html}{systemic
human rights abuses} against
\href{https://www.nytimes.com/2018/10/18/world/asia/uighur-muslims-china-detainment.html}{predominantly
Muslim ethnic minorities} in the Xinjiang region in China's far
northwest.

The sanctions, imposed by the Treasury Department's Office of Foreign
Assets Control, name the
\href{https://www.nytimes.com/2009/08/07/world/asia/07xinjiang.html}{Xinjiang
Production and Construction Corps}, an economic and paramilitary
organization that plays
\href{https://www.andrewerickson.com/2019/11/the-xinjiang-production-construction-corps-key-policy-tool-from-mao-to-xi/}{a
central role} in the development of the Xinjiang region, and two
associated officials, Peng Jiarui and Sun Jinlong. The order is designed
to prevent them from accessing American property and the financial
system, as well as to ban any economic transactions between them and
American companies and citizens.

``The United States is committed to using the full breadth of its
financial powers to hold human rights abusers accountable in Xinjiang
and across the world,'' Steven T. Mnuchin, the Treasury Secretary, said
in a statement.

The sanctions most likely will have little or no practical impact on Mr.
Peng, the deputy party secretary and commander of the development group,
and Mr. Sun, one of its former political commissars. It was not
immediately clear what effect they would have on trade and international
commerce done by the group, which oversees some state-run companies that
export products such as tomato paste.

The group's direct exports to the United States were worth \$43 million
in 2018, the most recent year for which detailed official data were
available. Much more of its crops and other raw materials may end up
processed by Chinese companies for exports that are not counted under
the group's numbers.

\hypertarget{latest-updates-global-coronavirus-outbreak}{%
\section{\texorpdfstring{\href{https://www.nytimes.com/2020/08/04/world/coronavirus-covid-19.html?action=click\&pgtype=Article\&state=default\&region=MAIN_CONTENT_1\&context=storylines_live_updates}{Latest
Updates: Global Coronavirus
Outbreak}}{Latest Updates: Global Coronavirus Outbreak}}\label{latest-updates-global-coronavirus-outbreak}}

Updated 2020-08-04T10:03:05.885Z

\begin{itemize}
\tightlist
\item
  \href{https://www.nytimes.com/2020/08/04/world/coronavirus-covid-19.html?action=click\&pgtype=Article\&state=default\&region=MAIN_CONTENT_1\&context=storylines_live_updates\#link-6b644638}{`Long
  days, long nights': Washington prepares for a prolonged fight over
  virus relief.}
\item
  \href{https://www.nytimes.com/2020/08/04/world/coronavirus-covid-19.html?action=click\&pgtype=Article\&state=default\&region=MAIN_CONTENT_1\&context=storylines_live_updates\#link-7af9fca0}{Israel's
  rocky reopening of its schools may be a lesson for the U.S.}
\item
  \href{https://www.nytimes.com/2020/08/04/world/coronavirus-covid-19.html?action=click\&pgtype=Article\&state=default\&region=MAIN_CONTENT_1\&context=storylines_live_updates\#link-33bf9168}{Hurricane
  Isaias arrives in North Carolina as officials along the East Coast
  scramble.}
\end{itemize}

\href{https://www.nytimes.com/2020/08/04/world/coronavirus-covid-19.html?action=click\&pgtype=Article\&state=default\&region=MAIN_CONTENT_1\&context=storylines_live_updates}{See
more updates}

More live coverage:
\href{https://www.nytimes.com/live/2020/08/03/business/stock-market-today-coronavirus?action=click\&pgtype=Article\&state=default\&region=MAIN_CONTENT_1\&context=storylines_live_updates}{Markets}

In particular, the group is a big cotton grower. Last year, it produced
\href{http://www.xjbt.gov.cn/c/2020-04-26/7346731.shtml?ad_check=1}{two
million metric tons of cotton}, about a third of China's
\href{http://www.stats.gov.cn/tjsj/zxfb/201912/t20191217_1718007.html}{total
production}, according to official statistics. In 2018, its direct
imports from the United States were worth \$102 million.

Ties between the United States and China have been fraying as the Trump
administration takes an increasingly hard line on China's handling of
the initial
\href{https://www.nytimes.com/news-event/coronavirus}{coronavirus}
outbreak, its growing repression in Hong Kong, its maritime claims and
miliary expansionism in the South China Sea, its efforts to export 5G
next-generation telecommunications equipment and its systemic abuses of
largely Muslim ethnic minorities in Xinjiang.

The Chinese government has carried out a
\href{https://www.nytimes.com/interactive/2019/11/16/world/asia/china-xinjiang-documents.html}{campaign
of mass detentions} in Xinjiang, placing one million or more members of
the Uighur ethnic group and others into large internment camps that aim
to indoctrinate the detainees with propaganda about the Communist Party
and eradicate core parts of their Muslim and Uighur identities.

From 2018 onward, senior administration officials
\href{https://www.nytimes.com/2018/09/10/world/asia/us-china-sanctions-muslim-camps.html}{debated
whether and how to punish China} for the abuses.

China hawks in the administration blamed Mr. Trump and top economic
advisers, including Mr. Mnuchin, for holding back on sanctions in order
to avoid jeopardizing trade talks with China and to cozy up to Xi
Jinping, the Chinese leader. But now, as the pandemic roils the United
States and endangers the president's prospects of re-election, Mr. Trump
has begun to sour on maintaining cordial relations with China, and
\href{https://www.nytimes.com/2020/07/25/world/asia/us-china-trump-xi.html}{the
hawks have greater leeway} to pursue tougher actions on China and to try
to set the two nations
\href{https://www.nytimes.com/2020/07/29/podcasts/the-daily/china-trump-foreign-policy.html}{on
a long-term course for confrontation}.

Mr. Trump's campaign strategists have also urged him to attack China in
an attempt to turn the spotlight away from
\href{https://www.nytimes.com/2020/07/18/us/politics/trump-coronavirus-response-failure-leadership.html}{the
president's failures on the pandemic and the economy}.

``Today's designations are the latest U.S. government action in an
ongoing effort to deter human rights abuse in the Xinjiang region,''
Secretary of State Mike Pompeo, the most vocal of the China hawks, said
in a statement on Friday.

The
\href{https://www.nytimes.com/2009/08/07/world/asia/07xinjiang.html}{Xinjiang
Production and Construction Corps} was founded in 1954 as a group
entwined with the People's Liberation Army that would oversee the
deployment of large numbers of ethnic Han citizens, many of them
military veterans, to Xinjiang to build farms, factories and towns that
would allow China to consolidate control of the important border region
and the many ethnic minority groups there.

\href{https://www.nytimes.com/news-event/coronavirus?action=click\&pgtype=Article\&state=default\&region=MAIN_CONTENT_3\&context=storylines_faq}{}

\hypertarget{the-coronavirus-outbreak-}{%
\subsubsection{The Coronavirus Outbreak
›}\label{the-coronavirus-outbreak-}}

\hypertarget{frequently-asked-questions}{%
\paragraph{Frequently Asked
Questions}\label{frequently-asked-questions}}

Updated August 3, 2020

\begin{itemize}
\item ~
  \hypertarget{im-a-small-business-owner-can-i-get-relief}{%
  \paragraph{I'm a small-business owner. Can I get
  relief?}\label{im-a-small-business-owner-can-i-get-relief}}

  \begin{itemize}
  \tightlist
  \item
    The
    \href{https://www.nytimes.com/article/small-business-loans-stimulus-grants-freelancers-coronavirus.html?action=click\&pgtype=Article\&state=default\&region=MAIN_CONTENT_3\&context=storylines_faq}{stimulus
    bills enacted in March} offer help for the millions of American
    small businesses. Those eligible for aid are businesses and
    nonprofit organizations with fewer than 500 workers, including sole
    proprietorships, independent contractors and freelancers. Some
    larger companies in some industries are also eligible. The help
    being offered, which is being managed by the Small Business
    Administration, includes the Paycheck Protection Program and the
    Economic Injury Disaster Loan program. But lots of folks have
    \href{https://www.nytimes.com/interactive/2020/05/07/business/small-business-loans-coronavirus.html?action=click\&pgtype=Article\&state=default\&region=MAIN_CONTENT_3\&context=storylines_faq}{not
    yet seen payouts.} Even those who have received help are confused:
    The rules are draconian, and some are stuck sitting on
    \href{https://www.nytimes.com/2020/05/02/business/economy/loans-coronavirus-small-business.html?action=click\&pgtype=Article\&state=default\&region=MAIN_CONTENT_3\&context=storylines_faq}{money
    they don't know how to use.} Many small-business owners are getting
    less than they expected or
    \href{https://www.nytimes.com/2020/06/10/business/Small-business-loans-ppp.html?action=click\&pgtype=Article\&state=default\&region=MAIN_CONTENT_3\&context=storylines_faq}{not
    hearing anything at all.}
  \end{itemize}
\item ~
  \hypertarget{what-are-my-rights-if-i-am-worried-about-going-back-to-work}{%
  \paragraph{What are my rights if I am worried about going back to
  work?}\label{what-are-my-rights-if-i-am-worried-about-going-back-to-work}}

  \begin{itemize}
  \tightlist
  \item
    Employers have to provide
    \href{https://www.osha.gov/SLTC/covid-19/standards.html}{a safe
    workplace} with policies that protect everyone equally.
    \href{https://www.nytimes.com/article/coronavirus-money-unemployment.html?action=click\&pgtype=Article\&state=default\&region=MAIN_CONTENT_3\&context=storylines_faq}{And
    if one of your co-workers tests positive for the coronavirus, the
    C.D.C.} has said that
    \href{https://www.cdc.gov/coronavirus/2019-ncov/community/guidance-business-response.html}{employers
    should tell their employees} -\/- without giving you the sick
    employee's name -\/- that they may have been exposed to the virus.
  \end{itemize}
\item ~
  \hypertarget{should-i-refinance-my-mortgage}{%
  \paragraph{Should I refinance my
  mortgage?}\label{should-i-refinance-my-mortgage}}

  \begin{itemize}
  \tightlist
  \item
    \href{https://www.nytimes.com/article/coronavirus-money-unemployment.html?action=click\&pgtype=Article\&state=default\&region=MAIN_CONTENT_3\&context=storylines_faq}{It
    could be a good idea,} because mortgage rates have
    \href{https://www.nytimes.com/2020/07/16/business/mortgage-rates-below-3-percent.html?action=click\&pgtype=Article\&state=default\&region=MAIN_CONTENT_3\&context=storylines_faq}{never
    been lower.} Refinancing requests have pushed mortgage applications
    to some of the highest levels since 2008, so be prepared to get in
    line. But defaults are also up, so if you're thinking about buying a
    home, be aware that some lenders have tightened their standards.
  \end{itemize}
\item ~
  \hypertarget{what-is-school-going-to-look-like-in-september}{%
  \paragraph{What is school going to look like in
  September?}\label{what-is-school-going-to-look-like-in-september}}

  \begin{itemize}
  \tightlist
  \item
    It is unlikely that many schools will return to a normal schedule
    this fall, requiring the grind of
    \href{https://www.nytimes.com/2020/06/05/us/coronavirus-education-lost-learning.html?action=click\&pgtype=Article\&state=default\&region=MAIN_CONTENT_3\&context=storylines_faq}{online
    learning},
    \href{https://www.nytimes.com/2020/05/29/us/coronavirus-child-care-centers.html?action=click\&pgtype=Article\&state=default\&region=MAIN_CONTENT_3\&context=storylines_faq}{makeshift
    child care} and
    \href{https://www.nytimes.com/2020/06/03/business/economy/coronavirus-working-women.html?action=click\&pgtype=Article\&state=default\&region=MAIN_CONTENT_3\&context=storylines_faq}{stunted
    workdays} to continue. California's two largest public school
    districts --- Los Angeles and San Diego --- said on July 13, that
    \href{https://www.nytimes.com/2020/07/13/us/lausd-san-diego-school-reopening.html?action=click\&pgtype=Article\&state=default\&region=MAIN_CONTENT_3\&context=storylines_faq}{instruction
    will be remote-only in the fall}, citing concerns that surging
    coronavirus infections in their areas pose too dire a risk for
    students and teachers. Together, the two districts enroll some
    825,000 students. They are the largest in the country so far to
    abandon plans for even a partial physical return to classrooms when
    they reopen in August. For other districts, the solution won't be an
    all-or-nothing approach.
    \href{https://bioethics.jhu.edu/research-and-outreach/projects/eschool-initiative/school-policy-tracker/}{Many
    systems}, including the nation's largest, New York City, are
    devising
    \href{https://www.nytimes.com/2020/06/26/us/coronavirus-schools-reopen-fall.html?action=click\&pgtype=Article\&state=default\&region=MAIN_CONTENT_3\&context=storylines_faq}{hybrid
    plans} that involve spending some days in classrooms and other days
    online. There's no national policy on this yet, so check with your
    municipal school system regularly to see what is happening in your
    community.
  \end{itemize}
\item ~
  \hypertarget{is-the-coronavirus-airborne}{%
  \paragraph{Is the coronavirus
  airborne?}\label{is-the-coronavirus-airborne}}

  \begin{itemize}
  \tightlist
  \item
    The coronavirus
    \href{https://www.nytimes.com/2020/07/04/health/239-experts-with-one-big-claim-the-coronavirus-is-airborne.html?action=click\&pgtype=Article\&state=default\&region=MAIN_CONTENT_3\&context=storylines_faq}{can
    stay aloft for hours in tiny droplets in stagnant air}, infecting
    people as they inhale, mounting scientific evidence suggests. This
    risk is highest in crowded indoor spaces with poor ventilation, and
    may help explain super-spreading events reported in meatpacking
    plants, churches and restaurants.
    \href{https://www.nytimes.com/2020/07/06/health/coronavirus-airborne-aerosols.html?action=click\&pgtype=Article\&state=default\&region=MAIN_CONTENT_3\&context=storylines_faq}{It's
    unclear how often the virus is spread} via these tiny droplets, or
    aerosols, compared with larger droplets that are expelled when a
    sick person coughs or sneezes, or transmitted through contact with
    contaminated surfaces, said Linsey Marr, an aerosol expert at
    Virginia Tech. Aerosols are released even when a person without
    symptoms exhales, talks or sings, according to Dr. Marr and more
    than 200 other experts, who
    \href{https://academic.oup.com/cid/article/doi/10.1093/cid/ciaa939/5867798}{have
    outlined the evidence in an open letter to the World Health
    Organization}.
  \end{itemize}
\end{itemize}

As of 2009, the group, which reports directly to Beijing, had an annual
output of goods and services of \$7 billion, and the settlements and
entities overseen by the bingtuan, or soldiers corps, included five
cities, 180 farming communities and 1,000 companies. They also run their
own courts, universities and media organizations.

On July 9, the United States
\href{https://www.nytimes.com/2020/07/09/world/asia/trump-china-sanctions-uighurs.html}{imposed
sanctions} on four Chinese officials associated with Xinjiang policy,
including Chen Quanguo, the party chief of the region and a member of
the Chinese Communist Party's 25-member ruling Politburo. That move was
largely symbolic, but it sent a stronger message than an
\href{https://www.nytimes.com/2019/10/08/business/china-human-rights-technology-xinjiang.html}{October
2019 action} in which the administration placed 28 Chinese companies and
police departments deemed to be associated with Xinjiang abuses on a
blacklist that forbids American companies from selling technology and
other goods to them without a license. At that time, the State
Department also announced visa restrictions on some Chinese officials.

On July 20, the Trump administration
\href{https://www.nytimes.com/2020/07/20/business/economy/china-sanctions-uighurs-labor.html}{added
11 new Chinese entities}, including companies supplying major American
brands like Apple, Ralph Lauren and Tommy Hilfiger, to the list that
restricts them from purchasing American products, saying the firms were
complicit in human rights violations in Xinjiang. That brought to 48 the
total number of Chinese companies and security units on the U.S. entity
list for violations related to Xinjiang.

On July 1, the administration warned businesses with supply chains that
run through Xinjiang to consider the reputational, economic and legal
risks of doing so.

The Associated Press
\href{https://apnews.com/fff5fc7925f09916bf6b9d5f79bb4132}{reported} on
July 3 that agents of U.S. Customs and Border Protection in New York had
seized 13 tons of hair weaves and other beauty products suspected of
having been made by detainees in a Xinjiang internment camp. The
products were worth an estimated \$800,000. In May, the agency conducted
a seizure of similar products that were being imported by companies in
Georgia and Texas, to be sold to salons and individuals across the
United States.

Chris Buckley contributed reporting from Sydney, Australia.

Advertisement

\protect\hyperlink{after-bottom}{Continue reading the main story}

\hypertarget{site-index}{%
\subsection{Site Index}\label{site-index}}

\hypertarget{site-information-navigation}{%
\subsection{Site Information
Navigation}\label{site-information-navigation}}

\begin{itemize}
\tightlist
\item
  \href{https://help.nytimes.com/hc/en-us/articles/115014792127-Copyright-notice}{©~2020~The
  New York Times Company}
\end{itemize}

\begin{itemize}
\tightlist
\item
  \href{https://www.nytco.com/}{NYTCo}
\item
  \href{https://help.nytimes.com/hc/en-us/articles/115015385887-Contact-Us}{Contact
  Us}
\item
  \href{https://www.nytco.com/careers/}{Work with us}
\item
  \href{https://nytmediakit.com/}{Advertise}
\item
  \href{http://www.tbrandstudio.com/}{T Brand Studio}
\item
  \href{https://www.nytimes.com/privacy/cookie-policy\#how-do-i-manage-trackers}{Your
  Ad Choices}
\item
  \href{https://www.nytimes.com/privacy}{Privacy}
\item
  \href{https://help.nytimes.com/hc/en-us/articles/115014893428-Terms-of-service}{Terms
  of Service}
\item
  \href{https://help.nytimes.com/hc/en-us/articles/115014893968-Terms-of-sale}{Terms
  of Sale}
\item
  \href{https://spiderbites.nytimes.com}{Site Map}
\item
  \href{https://help.nytimes.com/hc/en-us}{Help}
\item
  \href{https://www.nytimes.com/subscription?campaignId=37WXW}{Subscriptions}
\end{itemize}
