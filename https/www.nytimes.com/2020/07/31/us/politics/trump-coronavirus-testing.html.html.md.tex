Sections

SEARCH

\protect\hyperlink{site-content}{Skip to
content}\protect\hyperlink{site-index}{Skip to site index}

\href{https://www.nytimes.com/section/politics}{Politics}

\href{https://myaccount.nytimes.com/auth/login?response_type=cookie\&client_id=vi}{}

\href{https://www.nytimes.com/section/todayspaper}{Today's Paper}

\href{/section/politics}{Politics}\textbar{}Trump's Coronavirus Testing
Chief Concedes a Lag in Test Results

\href{https://nyti.ms/2XhAWZL}{https://nyti.ms/2XhAWZL}

\begin{itemize}
\item
\item
\item
\item
\item
\end{itemize}

\href{https://www.nytimes.com/news-event/coronavirus?action=click\&pgtype=Article\&state=default\&region=TOP_BANNER\&context=storylines_menu}{The
Coronavirus Outbreak}

\begin{itemize}
\tightlist
\item
  live\href{https://www.nytimes.com/2020/08/08/world/coronavirus-updates.html?action=click\&pgtype=Article\&state=default\&region=TOP_BANNER\&context=storylines_menu}{Latest
  Updates}
\item
  \href{https://www.nytimes.com/interactive/2020/us/coronavirus-us-cases.html?action=click\&pgtype=Article\&state=default\&region=TOP_BANNER\&context=storylines_menu}{Maps
  and Cases}
\item
  \href{https://www.nytimes.com/interactive/2020/science/coronavirus-vaccine-tracker.html?action=click\&pgtype=Article\&state=default\&region=TOP_BANNER\&context=storylines_menu}{Vaccine
  Tracker}
\item
  \href{https://www.nytimes.com/interactive/2020/world/coronavirus-tips-advice.html?action=click\&pgtype=Article\&state=default\&region=TOP_BANNER\&context=storylines_menu}{F.A.Q.}
\item
  \href{https://www.nytimes.com/live/2020/08/07/business/stock-market-today-coronavirus?action=click\&pgtype=Article\&state=default\&region=TOP_BANNER\&context=storylines_menu}{Markets
  \& Economy}
\end{itemize}

Advertisement

\protect\hyperlink{after-top}{Continue reading the main story}

Supported by

\protect\hyperlink{after-sponsor}{Continue reading the main story}

\hypertarget{trumps-coronavirus-testing-chief-concedes-a-lag-in-test-results}{%
\section{Trump's Coronavirus Testing Chief Concedes a Lag in Test
Results}\label{trumps-coronavirus-testing-chief-concedes-a-lag-in-test-results}}

With the reopening plans of schools and businesses hinging on rapid test
results, the Trump administration's testing czar says a two- to
three-day turnaround ``is not possible.''

\includegraphics{https://static01.nyt.com/images/2020/07/31/us/politics/31dc-virus-hearing/merlin_175159080_b6d984cc-ecd9-412a-9c63-9fe3fb62b4cf-articleLarge.jpg?quality=75\&auto=webp\&disable=upscale}

By \href{https://www.nytimes.com/by/sheryl-gay-stolberg}{Sheryl Gay
Stolberg} and
\href{https://www.nytimes.com/by/katherine-j--wu}{Katherine J. Wu}

\begin{itemize}
\item
  July 31, 2020
\item
  \begin{itemize}
  \item
  \item
  \item
  \item
  \item
  \end{itemize}
\end{itemize}

WASHINGTON --- With schools, universities and businesses pinning their
hopes for reopening on rapid coronavirus testing, the Trump
administration's testing czar told Congress on Friday that getting test
results within two to three days ``is not a possible benchmark we can
achieve today.''

But even that sober assessment from Adm. Brett P. Giroir, the assistant
secretary for health, most likely did not fully reflect the mounting
frustration among patients and health professionals just as the school
year struggles to get started.

During a lengthy House hearing with top government health officials, Dr.
Giroir told lawmakers that the nation was averaging about 820,000 tests
daily, up from 550,000 earlier this month. But the raw numbers belie the
testing crunch that officials around the country are facing amid soaring
caseloads, particularly in the South and West.

``Turnaround times are definitely improving,'' Dr. Giroir insisted,
adding that it was ``very atypical'' to wait more than 12 days for
results.

But many researchers are still grappling with severe shortages of the
testing supplies needed to collect samples from patients and process
them in laboratories. That leaves state and local officials without
information they need to make critical health decisions, and it creates
lags in contact tracing --- a necessary tool for controlling the
pandemic's spread.

``We're doing so many tests, sometimes it takes seven to 10 days to get
the results back,'' Gov. Ron DeSantis of Florida noted on Friday evening
in a briefing there with President Trump.

Coronavirus testing is essential to opening the economy and getting
people back to work and school, but it is almost useless if long lag
times keep people unnecessarily quarantined for days or allow them to
spread the virus while they await their results.

Dr. Giroir insisted that over all, 59 percent of tests report results
within three days, and 76 percent within five.

``I'm sure there's an outlier at 12 to 16 days because that happens,''
he added, ``but that's very atypical.''

\hypertarget{latest-updates-the-coronavirus-outbreak}{%
\section{\texorpdfstring{\href{https://www.nytimes.com/2020/08/07/world/covid-19-news.html?action=click\&pgtype=Article\&state=default\&region=MAIN_CONTENT_1\&context=storylines_live_updates}{Latest
Updates: The Coronavirus
Outbreak}}{Latest Updates: The Coronavirus Outbreak}}\label{latest-updates-the-coronavirus-outbreak}}

Updated 2020-08-08T12:04:28.992Z

\begin{itemize}
\tightlist
\item
  \href{https://www.nytimes.com/2020/08/07/world/covid-19-news.html?action=click\&pgtype=Article\&state=default\&region=MAIN_CONTENT_1\&context=storylines_live_updates\#link-1f86d03a}{As
  the U.S. relief talks falter again, Trump says he is prepared to act
  on his own.}
\item
  \href{https://www.nytimes.com/2020/08/07/world/covid-19-news.html?action=click\&pgtype=Article\&state=default\&region=MAIN_CONTENT_1\&context=storylines_live_updates\#link-3f64a70a}{Cuomo
  says N.Y. schools can reopen in-person but leaves it up to districts
  to determine if, when and how.}
\item
  \href{https://www.nytimes.com/2020/08/07/world/covid-19-news.html?action=click\&pgtype=Article\&state=default\&region=MAIN_CONTENT_1\&context=storylines_live_updates\#link-14e70066}{Thousands
  of cases went unreported in California when a computer server failed.}
\end{itemize}

\href{https://www.nytimes.com/2020/08/07/world/covid-19-news.html?action=click\&pgtype=Article\&state=default\&region=MAIN_CONTENT_1\&context=storylines_live_updates}{See
more updates}

More live coverage:
\href{https://www.nytimes.com/live/2020/08/07/business/stock-market-today-coronavirus?action=click\&pgtype=Article\&state=default\&region=MAIN_CONTENT_1\&context=storylines_live_updates}{Markets}

Dr. Giroir's comments, during a hearing of the House Select Subcommittee
on the Coronavirus Crisis, were met with puzzlement by public health
experts, who say testing shortages persist. In some places, tests cannot
be processed at all because of a lack of reagents --- the chemicals
needed to detect whether the virus is present --- or lab capacity.

And anxious patients around the country paint a far bleaker picture.
Shawn Jain, who was tested along with several family members on July 7
in Nashville, waited 16 days for his results after they were processed
by Quest Diagnostics. Some of his family members still have not heard
back.

``I honestly thought they had lost the test,'' said Mr. Jain, who tested
negative for the virus. He added, ``It made me feel like, well if in the
future I do worry I have it, I can't even rely on something as basic as
testing.''

On Friday, the
\href{https://www.nih.gov/news-events/news-releases/nih-delivering-new-covid-19-testing-technologies-meet-us-demand}{National
Institutes of Health announced} awards totaling \$248.7 million for
seven companies to ramp up test production and deliver millions more
weekly tests as early as September. The N.I.H. director, Dr. Francis
Collins, described the announcement as the ``first of more awards to
come.'' Three of the tests are simple enough to deliver results in 30
minutes or less.

At the hearing, Dr. Anthony S. Fauci, the director of the National
Institute of Allergy and Infectious Diseases, said again that a safe and
effective coronavirus vaccine would most likely be ready by the end of
this year or early next, and cast doubt on efforts by Russia and China.

``I do hope that the Chinese and the Russians are actually testing the
vaccine before they're administering the vaccine to anyone,'' Dr. Fauci
said.

Until a vaccine is available, testing remains critical, but
\href{https://www.nytimes.com/2020/07/06/health/fast-coronavirus-tests.html}{new
diagnostic tools} will not come soon enough for the fall semester at
universities and colleges around the country. Many have decided to
transition to online classes in part because administrators cannot be
assured that enough testing will be available to keep students, faculty
and staff safe.

\includegraphics{https://static01.nyt.com/images/2020/07/31/us/politics/31dc-virus-hearing2/merlin_175126200_5134a356-bdd7-4f84-bc77-73b307c92c0d-articleLarge.jpg?quality=75\&auto=webp\&disable=upscale}

``Covid-19 testing capacity and delays in reporting results remain a
challenge,'' Sylvia M. Burwell, a health secretary in the Obama
administration who is now the president of American University,
\href{https://www.american.edu/president/announcements/july-30-2020.cfm}{wrote
this week}, in announcing that there would be ``no residential
experience'' for students this fall.

``The ability to test and support contact tracing is critical to
reducing community spread of Covid-19,'' she added, ``and the ambiguity
in this area presents a significant hurdle for all.''

Other colleges were in the same position.

``The increased spread of the virus nationwide, the impact that this
resurgence has had on the availability of testing supplies needed to
satisfy our testing protocols, and the strong national trend of rising
rates of infection in younger populations lead us to conclude that our
community is best served by maintaining social distancing in miles
rather than feet,'' Alison R. Byerly, the president of Lafayette College
in Pennsylvania, wrote last week.

\href{https://www.nytimes.com/news-event/coronavirus?action=click\&pgtype=Article\&state=default\&region=MAIN_CONTENT_3\&context=storylines_faq}{}

\hypertarget{the-coronavirus-outbreak-}{%
\subsubsection{The Coronavirus Outbreak
›}\label{the-coronavirus-outbreak-}}

\hypertarget{frequently-asked-questions}{%
\paragraph{Frequently Asked
Questions}\label{frequently-asked-questions}}

Updated August 6, 2020

\begin{itemize}
\item ~
  \hypertarget{why-are-bars-linked-to-outbreaks}{%
  \paragraph{Why are bars linked to
  outbreaks?}\label{why-are-bars-linked-to-outbreaks}}

  \begin{itemize}
  \tightlist
  \item
    Think about a bar. Alcohol is flowing. It can be loud, but it's
    definitely intimate, and you often need to lean in close to hear
    your friend. And strangers have way, way fewer reservations about
    coming up to people in a bar. That's sort of the point of a bar.
    Feeling good and close to strangers. It's no surprise, then, that
    \href{https://www.nytimes.com/2020/07/02/us/coronavirus-bars.html?action=click\&pgtype=Article\&state=default\&region=MAIN_CONTENT_3\&context=storylines_faq}{bars
    have been linked to outbreaks in several states.} Louisiana health
    officials have tied
    \href{https://www.nytimes.com/2020/06/22/us/new-coronavirus-phase.html?action=click\&pgtype=Article\&state=default\&region=MAIN_CONTENT_3\&context=storylines_faq}{at
    least 100 coronavirus cases} to bars in the Tigerland nightlife
    district in Baton Rouge. Minnesota has traced 328 recent cases to
    bars across the state.
    \href{https://www.boisestatepublicradio.org/post/bars-large-venues-close-ada-county-after-surge-coronavirus-prompts-rollback\#stream/0}{In
    Idaho}, health officials shut down bars in Ada County after
    reporting clusters of infections among young adults who had visited
    several bars in downtown Boise. Governors in
    \href{https://www.nytimes.com/2020/07/01/us/california-coronavirus-reopening.html?action=click\&pgtype=Article\&state=default\&region=MAIN_CONTENT_3\&context=storylines_faq}{California},
    \href{https://www.nytimes.com/2020/06/14/us/coronavirus-united-states.html?action=click\&pgtype=Article\&state=default\&region=MAIN_CONTENT_3\&context=storylines_faq}{Texas
    and Arizona}, where coronavirus cases are soaring, have ordered
    hundreds of newly reopened bars to shut down. Less than two weeks
    after Colorado's bars reopened at limited capacity, Gov. Jared Polis
    \href{https://www.denverpost.com/2020/06/30/colorado-bars-closed-coronavirus/}{ordered
    them to close}.
  \end{itemize}
\item ~
  \hypertarget{i-have-antibodies-am-i-now-immune}{%
  \paragraph{I have antibodies. Am I now
  immune?}\label{i-have-antibodies-am-i-now-immune}}

  \begin{itemize}
  \tightlist
  \item
    As of right now,
    \href{https://www.nytimes.com/2020/07/22/health/covid-antibodies-herd-immunity.html?action=click\&pgtype=Article\&state=default\&region=MAIN_CONTENT_3\&context=storylines_faq}{that
    seems likely, for at least several months.} There have been
    frightening accounts of people suffering what seems to be a second
    bout of Covid-19. But experts say these patients may have a
    drawn-out course of infection, with the virus taking a slow toll
    weeks to months after initial exposure. People infected with the
    coronavirus typically
    \href{https://www.nature.com/articles/s41586-020-2456-9}{produce}
    immune molecules called antibodies, which are
    \href{https://www.nytimes.com/2020/05/07/health/coronavirus-antibody-prevalence.html?action=click\&pgtype=Article\&state=default\&region=MAIN_CONTENT_3\&context=storylines_faq}{protective
    proteins made in response to an
    infection}\href{https://www.nytimes.com/2020/05/07/health/coronavirus-antibody-prevalence.html?action=click\&pgtype=Article\&state=default\&region=MAIN_CONTENT_3\&context=storylines_faq}{.
    These antibodies may} last in the body
    \href{https://www.nature.com/articles/s41591-020-0965-6}{only two to
    three months}, which may seem worrisome, but that's perfectly normal
    after an acute infection subsides, said Dr. Michael Mina, an
    immunologist at Harvard University. It may be possible to get the
    coronavirus again, but it's highly unlikely that it would be
    possible in a short window of time from initial infection or make
    people sicker the second time.
  \end{itemize}
\item ~
  \hypertarget{im-a-small-business-owner-can-i-get-relief}{%
  \paragraph{I'm a small-business owner. Can I get
  relief?}\label{im-a-small-business-owner-can-i-get-relief}}

  \begin{itemize}
  \tightlist
  \item
    The
    \href{https://www.nytimes.com/article/small-business-loans-stimulus-grants-freelancers-coronavirus.html?action=click\&pgtype=Article\&state=default\&region=MAIN_CONTENT_3\&context=storylines_faq}{stimulus
    bills enacted in March} offer help for the millions of American
    small businesses. Those eligible for aid are businesses and
    nonprofit organizations with fewer than 500 workers, including sole
    proprietorships, independent contractors and freelancers. Some
    larger companies in some industries are also eligible. The help
    being offered, which is being managed by the Small Business
    Administration, includes the Paycheck Protection Program and the
    Economic Injury Disaster Loan program. But lots of folks have
    \href{https://www.nytimes.com/interactive/2020/05/07/business/small-business-loans-coronavirus.html?action=click\&pgtype=Article\&state=default\&region=MAIN_CONTENT_3\&context=storylines_faq}{not
    yet seen payouts.} Even those who have received help are confused:
    The rules are draconian, and some are stuck sitting on
    \href{https://www.nytimes.com/2020/05/02/business/economy/loans-coronavirus-small-business.html?action=click\&pgtype=Article\&state=default\&region=MAIN_CONTENT_3\&context=storylines_faq}{money
    they don't know how to use.} Many small-business owners are getting
    less than they expected or
    \href{https://www.nytimes.com/2020/06/10/business/Small-business-loans-ppp.html?action=click\&pgtype=Article\&state=default\&region=MAIN_CONTENT_3\&context=storylines_faq}{not
    hearing anything at all.}
  \end{itemize}
\item ~
  \hypertarget{what-are-my-rights-if-i-am-worried-about-going-back-to-work}{%
  \paragraph{What are my rights if I am worried about going back to
  work?}\label{what-are-my-rights-if-i-am-worried-about-going-back-to-work}}

  \begin{itemize}
  \tightlist
  \item
    Employers have to provide
    \href{https://www.osha.gov/SLTC/covid-19/standards.html}{a safe
    workplace} with policies that protect everyone equally.
    \href{https://www.nytimes.com/article/coronavirus-money-unemployment.html?action=click\&pgtype=Article\&state=default\&region=MAIN_CONTENT_3\&context=storylines_faq}{And
    if one of your co-workers tests positive for the coronavirus, the
    C.D.C.} has said that
    \href{https://www.cdc.gov/coronavirus/2019-ncov/community/guidance-business-response.html}{employers
    should tell their employees} -\/- without giving you the sick
    employee's name -\/- that they may have been exposed to the virus.
  \end{itemize}
\item ~
  \hypertarget{what-is-school-going-to-look-like-in-september}{%
  \paragraph{What is school going to look like in
  September?}\label{what-is-school-going-to-look-like-in-september}}

  \begin{itemize}
  \tightlist
  \item
    It is unlikely that many schools will return to a normal schedule
    this fall, requiring the grind of
    \href{https://www.nytimes.com/2020/06/05/us/coronavirus-education-lost-learning.html?action=click\&pgtype=Article\&state=default\&region=MAIN_CONTENT_3\&context=storylines_faq}{online
    learning},
    \href{https://www.nytimes.com/2020/05/29/us/coronavirus-child-care-centers.html?action=click\&pgtype=Article\&state=default\&region=MAIN_CONTENT_3\&context=storylines_faq}{makeshift
    child care} and
    \href{https://www.nytimes.com/2020/06/03/business/economy/coronavirus-working-women.html?action=click\&pgtype=Article\&state=default\&region=MAIN_CONTENT_3\&context=storylines_faq}{stunted
    workdays} to continue. California's two largest public school
    districts --- Los Angeles and San Diego --- said on July 13, that
    \href{https://www.nytimes.com/2020/07/13/us/lausd-san-diego-school-reopening.html?action=click\&pgtype=Article\&state=default\&region=MAIN_CONTENT_3\&context=storylines_faq}{instruction
    will be remote-only in the fall}, citing concerns that surging
    coronavirus infections in their areas pose too dire a risk for
    students and teachers. Together, the two districts enroll some
    825,000 students. They are the largest in the country so far to
    abandon plans for even a partial physical return to classrooms when
    they reopen in August. For other districts, the solution won't be an
    all-or-nothing approach.
    \href{https://bioethics.jhu.edu/research-and-outreach/projects/eschool-initiative/school-policy-tracker/}{Many
    systems}, including the nation's largest, New York City, are
    devising
    \href{https://www.nytimes.com/2020/06/26/us/coronavirus-schools-reopen-fall.html?action=click\&pgtype=Article\&state=default\&region=MAIN_CONTENT_3\&context=storylines_faq}{hybrid
    plans} that involve spending some days in classrooms and other days
    online. There's no national policy on this yet, so check with your
    municipal school system regularly to see what is happening in your
    community.
  \end{itemize}
\end{itemize}

Democrats seized on reports of testing delays to demand a national
testing strategy.

``We once again call upon the president to get serious about this ---
no, testing is not overrated,'' Speaker Nancy Pelosi told reporters as
the hearing was underway.

Testing delays hurt efforts to contain the spread of the virus.
Diagnostic tests reflect only a person's health status on the day a
sample is collected. While those who visit testing sites are typically
told to quarantine at home while they await their results, that advice
becomes harder to take the longer people are forced to wait ---
especially for those who work essential jobs that cannot be done
remotely.

``It is an issue if you can't get it within a 24-to-48-hour period,''
Dr. Fauci said.

And diagnostic tests alone are not enough. Experts say that in order to
stop the pandemic, the country will have to expand testing in the
broader community --- not just to identify sick people, but to assess
the prevalence of disease in the general population and to catch
asymptomatic people who might be unknowingly carrying the virus.

About half of all coronavirus tests, Dr. Giroir said, are conducted in
so-called point-of-care settings --- like doctor's offices or urgent
care clinics, without the need to route samples through laboratories ---
or hospitals. Point-of-care tests, intended to be fast and simple enough
to obviate the need for specialized equipment or highly trained
personnel, can yield results in 15 minutes, he said, while hospital
tests take around a day.

The remainder of coronavirus tests are conducted by large-scale
commercial laboratory companies, like LabCorp and Quest Diagnostics,
which are strained near their limits.

Image

From left, Robert Redfield, Dr. Anthony Fauci, and Mr. Giroir,
testifying during a House Select Subcommittee hearing on Friday. ``It is
an issue if you cant get it within a 24 to 48 hour period,'' Dr. Fauci
said.Credit...Pool photo by Erin Scott

This month, Daniel Larremore, a mathematician and infectious disease
modeler at the University of Colorado, Boulder, collected data about
testing delays via an
\href{https://larremorelab.github.io/covid19testgroup}{informal survey
on Twitter}, showing that residents in multiple states were experiencing
prolonged test result turnaround times.

What is more, turnaround times do not reflect ``how long it takes for
people to get a test in the first place, and how convenient it is to get
the testing done,'' Dr. Larremore said. He also pointed out that many
places were still prioritizing patients with symptoms for testing.

But mounting evidence suggests that about 40 percent of coronavirus
infections could present without symptoms entirely. Tests are also still
failing to reach many of those who need it the most, including
communities marginalized by race and ethnicity, who have been
\href{https://www.nytimes.com/interactive/2020/07/05/us/coronavirus-latinos-african-americans-cdc-data.html}{disproportionately
affected by the coronavirus}.

Michael T. Osterholm, the director of the Center for Infectious Disease
Research and Policy at the University of Minnesota, said the
administration needed a ``national dashboard for testing'' where data
was collected and made publicly available.

``We need to know how many people are tested, by which test, how long
does it take to get a result back and where there is testing capacity
available, but they can't be done because there is an absence of reagent
or other critical components,'' he said.

While the country's capacity for testing has certainly increased, Dr.
Larremore said, ``That doesn't mean that we are where we need to be ---
just that we're continuing to accelerate.''

Sheryl Gay Stolberg reported from Washington, and Katherine J. Wu from
Boston.

Advertisement

\protect\hyperlink{after-bottom}{Continue reading the main story}

\hypertarget{site-index}{%
\subsection{Site Index}\label{site-index}}

\hypertarget{site-information-navigation}{%
\subsection{Site Information
Navigation}\label{site-information-navigation}}

\begin{itemize}
\tightlist
\item
  \href{https://help.nytimes.com/hc/en-us/articles/115014792127-Copyright-notice}{©~2020~The
  New York Times Company}
\end{itemize}

\begin{itemize}
\tightlist
\item
  \href{https://www.nytco.com/}{NYTCo}
\item
  \href{https://help.nytimes.com/hc/en-us/articles/115015385887-Contact-Us}{Contact
  Us}
\item
  \href{https://www.nytco.com/careers/}{Work with us}
\item
  \href{https://nytmediakit.com/}{Advertise}
\item
  \href{http://www.tbrandstudio.com/}{T Brand Studio}
\item
  \href{https://www.nytimes.com/privacy/cookie-policy\#how-do-i-manage-trackers}{Your
  Ad Choices}
\item
  \href{https://www.nytimes.com/privacy}{Privacy}
\item
  \href{https://help.nytimes.com/hc/en-us/articles/115014893428-Terms-of-service}{Terms
  of Service}
\item
  \href{https://help.nytimes.com/hc/en-us/articles/115014893968-Terms-of-sale}{Terms
  of Sale}
\item
  \href{https://spiderbites.nytimes.com}{Site Map}
\item
  \href{https://help.nytimes.com/hc/en-us}{Help}
\item
  \href{https://www.nytimes.com/subscription?campaignId=37WXW}{Subscriptions}
\end{itemize}
