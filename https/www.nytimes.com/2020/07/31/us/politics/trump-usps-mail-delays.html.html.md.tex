Sections

SEARCH

\protect\hyperlink{site-content}{Skip to
content}\protect\hyperlink{site-index}{Skip to site index}

\href{https://www.nytimes.com/section/politics}{Politics}

\href{https://myaccount.nytimes.com/auth/login?response_type=cookie\&client_id=vi}{}

\href{https://www.nytimes.com/section/todayspaper}{Today's Paper}

\href{/section/politics}{Politics}\textbar{}Mail Delays Fuel Concern
Trump Is Undercutting Postal System Ahead of Voting

\href{https://nyti.ms/318HPh8}{https://nyti.ms/318HPh8}

\begin{itemize}
\item
\item
\item
\item
\item
\item
\end{itemize}

\begin{itemize}
\item
  \href{https://www.nytimes.com/2020/08/07/us/elections/biden-vs-trump.html?action=click\&pgtype=Article\&state=default\&region=TOP_BANNER\&context=storylines_menu}{Election
  Updates}
\item
  \href{https://www.nytimes.com/interactive/2020/08/06/us/elections/results-tennessee-primary-elections.html?action=click\&pgtype=Article\&state=default\&region=TOP_BANNER\&context=storylines_menu}{Tennessee
  Results}
\item
  \href{https://www.nytimes.com/article/biden-vice-president-2020.html?action=click\&pgtype=Article\&state=default\&region=TOP_BANNER\&context=storylines_menu}{Biden's
  V.P. Search}
\item
  \href{https://www.nytimes.com/interactive/2019/us/politics/2020-presidential-candidates.html?action=click\&pgtype=Article\&state=default\&region=TOP_BANNER\&context=storylines_menu}{The
  Candidates}
\item
  \href{https://www.nytimes.com/newsletters/politics?action=click\&pgtype=Article\&state=default\&region=TOP_BANNER\&context=storylines_menu}{Politics
  Newsletter}
\end{itemize}

Advertisement

\protect\hyperlink{after-top}{Continue reading the main story}

Supported by

\protect\hyperlink{after-sponsor}{Continue reading the main story}

\hypertarget{mail-delays-fuel-concern-trump-is-undercutting-postal-system-ahead-of-voting}{%
\section{Mail Delays Fuel Concern Trump Is Undercutting Postal System
Ahead of
Voting}\label{mail-delays-fuel-concern-trump-is-undercutting-postal-system-ahead-of-voting}}

The president's long campaign against the Postal Service is intersecting
with his assault on mail-in voting amid concerns that he has politicized
oversight of the agency.

\includegraphics{https://static01.nyt.com/images/2020/07/31/multimedia/31dc-postal/merlin_172107963_a2c16fe7-bd41-4b56-8445-524d4cce31e7-articleLarge.jpg?quality=75\&auto=webp\&disable=upscale}

By \href{https://www.nytimes.com/by/michael-d-shear}{Michael D. Shear},
\href{https://www.nytimes.com/by/hailey-fuchs}{Hailey Fuchs} and
\href{https://www.nytimes.com/by/kenneth-p-vogel}{Kenneth P. Vogel}

\begin{itemize}
\item
  July 31, 2020
\item
  \begin{itemize}
  \item
  \item
  \item
  \item
  \item
  \item
  \end{itemize}
\end{itemize}

WASHINGTON --- Welcome to the next election battleground: the post
office.

President Trump's yearslong assault on the Postal Service and his
increasingly dire warnings about the dangers of
\href{https://www.nytimes.com/2020/08/03/us/politics/trump-mail-in-voting.html}{voting
by mail} are colliding as the presidential campaign enters its final
months. The result has been to generate new concerns about how he could
influence an election conducted during a pandemic in which
greater-than-ever numbers of voters will submit their ballots by mail.

In tweet after all-caps tweet, Mr. Trump has warned that allowing people
to vote by mail will result in a
``\href{https://twitter.com/realDonaldTrump/status/1285540318503407622?s=20}{CORRUPT
ELECTION}'' that will
``\href{https://twitter.com/realDonaldTrump/status/1266172570983940101?s=20}{LEAD
TO THE END OF OUR GREAT REPUBLICAN PARTY}'' and become the
``\href{https://twitter.com/realDonaldTrump/status/1275024974579982336?s=20}{SCANDAL
OF OUR TIMES}.'' He has predicted that children will steal ballots out
of mailboxes. On Thursday, he
\href{https://twitter.com/realDonaldTrump/status/1288818160389558273}{dangled
the idea of delaying the election} instead.

Members of Congress and state officials in
\href{https://www.nytimes.com/2020/07/30/us/politics/trump-delay-2020-election.html}{both
parties rejected the president's suggestion} and his claim that mail-in
ballots would result in widespread fraud. But they are warning that a
huge wave of ballots could overwhelm mail carriers unless the Postal
Service, in financial difficulty for years, receives emergency funding
that Republicans are blocking during negotiations over another pandemic
relief bill.

At the same time, the mail system is being undercut in ways set in
motion by Mr. Trump. Fueled by animus for Jeff Bezos, the founder of
Amazon, and surrounded by advisers who have long called for privatizing
the post office, Mr. Trump and his appointees have begun taking
cost-cutting steps that appear to have led to slower and less reliable
delivery.

In recent weeks, at the direction of a Trump campaign megadonor who was
\href{https://www.nytimes.com/2020/05/07/us/politics/postmaster-general-louis-dejoy.html}{recently
named the postmaster general}, the service has stopped paying mail
carriers and clerks the overtime necessary to ensure that deliveries can
be completed each day. That and other changes have led to reports of
letters and packages being delayed by as many as several days.

Voting rights groups say it is a recipe for disaster.

``We have an underfunded state and local election system and a
deliberate slowdown in the Postal Service,'' said Wendy Fields, the
executive director of the Democracy Initiative, a coalition of voting
and civil rights groups. She said the president was ``deliberately
orchestrating suppression and using the post office as a tool to do
it.''

Kim Wyman, the Republican secretary of state in Washington, one of five
states where mail-in balloting is universal, said Wednesday on NPR's
``1A'' program that ``election officials are very concerned, if the post
office is reducing service, that we will be able to get ballots to
people in time.''

During
\href{https://www.nytimes.com/2020/07/30/us/obama-eulogy-john-lewis-full-transcript.html}{his
eulogy on Thursday for Representative John Lewis}, former President
Barack Obama lamented what he said was a continuing effort to attack
voting rights ``with surgical precision, even undermining the Postal
Service in the run-up to an election that is going to be dependent on
mailed-in ballots so people don't get sick.''

Louis DeJoy, the postmaster general, defended the changes, saying in a
statement that the ban on overtime was intended to ``improve operational
efficiency'' and to ``ensure that we meet our service standards.''

\includegraphics{https://static01.nyt.com/images/2020/07/31/multimedia/31dc-postal2/31dc-postal2-articleLarge.jpg?quality=75\&auto=webp\&disable=upscale}

Mr. DeJoy declined to be interviewed. David Partenheimer, a spokesman
for the Postal Service, said that the nation's post offices had ``ample
capacity to adjust our nationwide processing and delivery network to
meet projected election and political mail volume, including any
additional volume that may result as a response to the Covid-19
pandemic.''

A plunge in the amount of mail because of a recession --- which the
\href{https://www.nytimes.com/2020/06/08/business/economy/us-economy-recession-2020.html}{United
States entered into in February} --- has cost the Postal Service
billions of dollars in revenue, with some analysts predicting that the
agency will run out of money by spring. Democrats have proposed an
infusion of \$25 billion. On Friday, Speaker Nancy Pelosi accused
Republicans, who are opposed to the funding, of wanting to ``diminish
the capacity of the Postal System to work in a timely fashion.''

Arthur B. Sackler, who runs the Coalition for a 21st Century Postal
Service, a group representing the biggest bulk mailers, said the changes
were concerning even though his organization did not take a position on
voting by mail.

``Like any other mail, this could complicate what is already going to be
a complicated process,'' Mr. Sackler said. ``A huge number of
jurisdictions are totally inexperienced in vote by mail. They have never
had the avalanche of interest that they have this year.''

Many states have already loosened restrictions on who can vote by mail:
In Kentucky, mail-in ballots
\href{https://www.whas11.com/article/news/local/kentucky-election-absentee-vote-turnout/417-23f2bb1e-ea9a-4c7e-8202-c33f54063ab6}{accounted
for 85 percent} of the vote in June's primary. In Vermont, requests for
mail-in ballots are
\href{https://www.sevendaysvt.com/OffMessage/archives/2020/07/03/absentee-ballot-requests-surge-in-vermont}{up
1,000 percent} over 2018.

Michigan voters had requested nearly
\href{https://www.michigan.gov/sos/0,4670,7-127--534590--,00.html}{1.8
million mail-in ballots} by the end of July, compared with about 500,000
by the similar time four years ago, after the secretary of state mailed
absentee ballot applications to all
\href{https://mvic.sos.state.mi.us/}{7.7 million registered voters}.

In the suburban Virginia district of Representative Gerald E. Connolly,
a Democrat who leads the House subcommittee that oversees the Postal
Service, 1,300 people voted by mail in a 2019 primary --- last month,
more than 34,000 did.

``We are worried about new management at the Postal Service that is
carrying out Trump's avowed opposition to voting by mail,'' Mr. Connolly
said. ``I don't think that's speculation. I think we are witnessing that
in front of our own eyes.''

Erratic service could delay the delivery of blank ballots to people who
request them. And in 34 states, completed ballots that are not received
by Election Day --- this year it is Nov. 3 --- are invalidated, raising
the prospect that some voters could be disenfranchised if the mail
system buckles.

In other states, ballots can be tallied as long as they are postmarked
by Election Day, but voting rights groups say ballots are often
erroneously delivered without a postmark, which prevents them from being
counted.

The ability of the Postal Service ``to timely deliver and return
absentee ballots and their work to postmark those ballots will literally
determine whether or not voters are disenfranchised during the
pandemic,'' said Kristen Clarke, the president of the National Lawyers'
Committee for Civil Rights Under Law.

In New York, where officials urged people not to cast ballots in person
during June's primary, counting of mail-in ballots is still underway
weeks later, leaving some crucial races undecided. In some cases,
ballots received without postmarks are being discarded.

Making the problem worse, New York law requires that election officials
wait to begin counting mail-in ballots until the polls close on Election
Day. Other states
\href{https://www.ncsl.org/research/elections-and-campaigns/vopp-table-16-when-absentee-mail-ballot-processing-and-counting-can-begin.aspx}{allow
counting to begin earlier}, though most insist that no results be
revealed until after voting ends. In Arizona, officials can begin
tallying votes 14 days early. In Florida, officials can begin verifying
signatures on ballots 22 days before the election.

Mr. Trump and his allies have seized upon the New York debacle as
evidence that he is right to oppose mail-in ballots. Kayleigh McEnany,
the White House press secretary, called it an ``absolute catastrophe,''
and the president
\href{https://twitter.com/realDonaldTrump/status/1287554056727040008?s=20}{referred
to New York in a tweet} that said, ``Rigged Election, and EVERYONE knows
it!''

But Mr. Trump --- who himself has repeatedly voted by mail in recent
elections --- has set in motion changes at the Postal Service that could
make the problem worse.

Image

New York City Board of Elections staff members counting absentee ballots
this month from June's primary. Some crucial races are still undecided.
Credit...Victor J. Blue for The New York Times

A series of Postal Service documents titled ``PMGs expectations,'' a
reference to the postmaster general, describe how Mr. Trump's new
leadership team is trying to cut costs.

``Overtime will be eliminated,'' says the document, which was
\href{https://www.washingtonpost.com/business/2020/07/14/postal-service-trump-dejoy-delay-mail/}{first
reported by The Washington Post}. ``Again, we are paying too much
overtime, and it is not cost effective and will soon be taken off the
table. More to come on this.''

The document continues: ``The U.S.P.S. will no longer use excessive cost
to get the basic job done. If the plants run late, they will keep the
mail for the next day.''

Another document, dated July 10, says, ``One aspect of these changes
that may be difficult for employees is that --- temporarily --- we may
see mail left behind or on the workroom floor or docks.''

With the agency under financial pressure, some offices have also begun
to cut back on hours. The result, according to postal workers, members
of Congress and major post office customers, is a noticeable slowdown in
delivery.

``The policies that the new postmaster general is putting into place ---
they couldn't lead to anything but degradation of service,'' said Mark
Dimondstein, the president of the
\href{https://www.apwu.org/mark-dimondstein}{American Postal Workers
Union}. ``Anything that slows down the mail could have a negative impact
on everything we do, including vote by mail.''

The Postal Service, which runs more than 31,000 post offices in the
United States, has struggled financially for years, in part because of
its legal obligation to deliver mail everywhere, even remote locations
that would be unprofitable for a private company.

\href{https://home.treasury.gov/system/files/136/USPS_A_Sustainable_Path_Forward_report_12-04-2018.pdf}{A
2018 report by the Treasury Department} recommended an overhaul of the
Postal Service, which the report said accumulated losses of \$69 billion
from 2007 to 2018.

But the administration's critics say the changes being put in place by
Mr. DeJoy are part of a political agenda to move toward privatization of
the Postal Service.

In mid-July, Representative Carolyn B. Maloney, Democrat of New York and
the chairwoman of the House Oversight and Reform Committee, and Mr.
Connolly
\href{https://oversight.house.gov/sites/democrats.oversight.house.gov/files/2020-07-20.CBM\%20GEC\%20to\%20DeJoy\%20-PMG\%20re\%20Postal\%20Service\%20Changes.pdf}{wrote
a letter to Mr. DeJoy} raising questions about the ban on overtime and
the other changes.

``While these changes in a normal year would be drastic,'' the lawmakers
wrote, ``in a presidential election year when many states are relying
heavily on absentee mail-in ballots, increases in mail delivery timing
would impair the ability of ballots to be received and counted in a
timely manner --- an unacceptable outcome for a free and fair
election.''

Image

Mail-in ballots at the Bucks County Board of Elections office before the
primary in May in Doylestown, Pa.~Mr. Trump has warned that allowing
people to vote by mail will result in a ``CORRUPT
ELECTION.''Credit...Matt Slocum/Associated Press

Mr. Trump has been assailing the Postal Service since early in his
presidency,
\href{https://twitter.com/realDonaldTrump/status/946728546633953285?s=20}{tweeting
in 2017 that the agency was becoming ``dumber and poorer''} because it
charged big companies too little for delivering their packages.

The president has repeatedly blamed Mr. Bezos, who is also the owner of
The Washington Post, for the financial plight of the Postal Service,
insisting that the post office charges Amazon too little, an assertion
that many experts have rejected as false.

In the past three years, the president has replaced all six members of
the Postal Service Board of Governors.

In May, the board, which includes two Democrats, selected Mr. DeJoy, a
longtime Republican fund-raiser who has contributed more than \$1.5
million to Mr. Trump's 2016 and 2020 campaigns, to be postmaster
general. According to financial disclosures, Mr. DeJoy and his wife,
Aldona Wos, who has been nominated to be the ambassador to Canada, have
\$115,002 to \$300,000 invested in the Postal Service's major
competitor, UPS.

Two board members have since departed. David C. Williams, the vice
chairman, left in April over concerns that the Postal Service was
becoming increasingly politicized by the Trump administration, according
to two people familiar with his thinking. Ronald Stroman, who oversaw
mail-in voting and relations with election officials, resigned in May.

One of the remaining members, Robert M. Duncan, is a former Republican
National Committee chairman who has been a campaign donor to Mr. Trump.

In accusing the administration of politicizing the Postal Service, the
president's critics point to a recent decision to send a mailer
detailing guidelines to protect against the
\href{https://www.nytimes.com/interactive/2020/us/coronavirus-us-cases.htmlhttps://www.nytimes.com/interactive/2020/us/coronavirus-us-cases.html}{coronavirus}.
The mailer, which featured Mr. Trump's name in a campaignlike style, was
sent in March to 130 million **** American households at a
\href{https://www.usatoday.com/story/news/politics/2020/05/28/coronavirus-post-card-trump-cost-post-office-28-million/5274034002/}{reported
cost of \$28 million}.

According to Postal Service emails obtained by The New York Times under
the Freedom of Information Act, Mr. Trump was personally involved.

``I know that POTUS personally approved this postcard and is aware of
the USPS effort in service to the nation --- pushing information out to
every household, urban and rural,'' John M. Barger, a governor of the
postal system, wrote
\href{https://www.documentcloud.org/documents/7010858-USPS-Email-Trump-Personally-Approved-28M.html}{in
an email} to the postmaster general at the time.

In
\href{https://www.documentcloud.org/documents/7010863-USPS-Emails-Show-Trump-Coronavirus-Postcard-Was.html}{another
email}, **** Dr. Deborah L. Birx, the White House coronavirus response
coordinator, told a member of the board that Dr. Stephen C. Redd, a
deputy director at the Centers for Disease Control and Prevention,
``will make this happen.'' The mailer received a go-ahead from the White
House before it was sent out, the emails show.

S. David Fineman, who served on the board under Presidents Bill Clinton
and George W. Bush, said that during his time, the board rarely if ever
had contact with the White House.

``I've never seen anything quite like this,'' he said. ``No one would
have thought that we would have sought the input of the
administration.''

\hypertarget{our-2020-election-guide}{%
\section{Our 2020 Election Guide}\label{our-2020-election-guide}}

Updated Aug. 7, 2020

\begin{itemize}
\item
  \begin{center}\rule{0.5\linewidth}{\linethickness}\end{center}

  \hypertarget{the-latest}{%
  \subsection{The Latest}\label{the-latest}}

  \begin{itemize}
  \tightlist
  \item
    \href{https://www.nytimes.com/2020/08/07/us/politics/russia-china-trump-biden-election-interference.html?action=click\&pgtype=Article\&state=default\&region=BELOW_MAIN_CONTENT\&context=storylines_guide}{Russia
    is using a range of techniques to denigrate Joe Biden}, American
    intelligence officials said, declaring that Moscow continues to try
    to interfere in the 2020 campaign to help President Trump.
  \end{itemize}
\item
  \begin{center}\rule{0.5\linewidth}{\linethickness}\end{center}

  \hypertarget{bidens-vp-search}{%
  \subsection{Biden's V.P. Search}\label{bidens-vp-search}}

  \begin{itemize}
  \tightlist
  \item
    \href{https://www.nytimes.com/article/biden-vice-president-2020.html?action=click\&pgtype=Article\&state=default\&region=BELOW_MAIN_CONTENT\&context=storylines_guide}{Here
    are 13 women} who have been under consideration to be Joe Biden's
    running mate, and why each might be chosen --- and might not be.
  \end{itemize}
\item
  \begin{center}\rule{0.5\linewidth}{\linethickness}\end{center}

  \hypertarget{keep-up-with-our-coverage}{%
  \subsection{Keep Up With Our
  Coverage}\label{keep-up-with-our-coverage}}

  \begin{itemize}
  \tightlist
  \item
    Get an
    \href{https://www.nytimes.com/newsletters/politics?action=click\&pgtype=Article\&state=default\&region=BELOW_MAIN_CONTENT\&context=storylines_guide}{email}
    recapping the day's news
  \end{itemize}

  \begin{itemize}
  \tightlist
  \item
    Download our mobile app on
    \href{https://apps.apple.com/us/app/nytimes/id284862083?ls=1\&mat_click_id=5c79ae7455014fd1bd66b5610c05b8f2-20191112-16948\&referrer=mat_click_id\%3D5c79ae7455014fd1bd66b5610c05b8f2-20191112-16948\%26link_click_id\%3D722930677036718082}{iOS}
    and
    \href{http://a.localytics.com/android?id=com.nytimes.android\&referrer=utm_source\%3Dother_nyt_mobile_web\%26utm_medium\%3DWeb\%2520page\%26utm_term\%3DGeneral\%2520Mobile\%2520Page\%26utm_campaign\%3DNYT\%2520Mobile\%2520General\%2520Page}{Android}
    and turn on Breaking News and Politics alerts
  \end{itemize}
\end{itemize}

Advertisement

\protect\hyperlink{after-bottom}{Continue reading the main story}

\hypertarget{site-index}{%
\subsection{Site Index}\label{site-index}}

\hypertarget{site-information-navigation}{%
\subsection{Site Information
Navigation}\label{site-information-navigation}}

\begin{itemize}
\tightlist
\item
  \href{https://help.nytimes.com/hc/en-us/articles/115014792127-Copyright-notice}{©~2020~The
  New York Times Company}
\end{itemize}

\begin{itemize}
\tightlist
\item
  \href{https://www.nytco.com/}{NYTCo}
\item
  \href{https://help.nytimes.com/hc/en-us/articles/115015385887-Contact-Us}{Contact
  Us}
\item
  \href{https://www.nytco.com/careers/}{Work with us}
\item
  \href{https://nytmediakit.com/}{Advertise}
\item
  \href{http://www.tbrandstudio.com/}{T Brand Studio}
\item
  \href{https://www.nytimes.com/privacy/cookie-policy\#how-do-i-manage-trackers}{Your
  Ad Choices}
\item
  \href{https://www.nytimes.com/privacy}{Privacy}
\item
  \href{https://help.nytimes.com/hc/en-us/articles/115014893428-Terms-of-service}{Terms
  of Service}
\item
  \href{https://help.nytimes.com/hc/en-us/articles/115014893968-Terms-of-sale}{Terms
  of Sale}
\item
  \href{https://spiderbites.nytimes.com}{Site Map}
\item
  \href{https://help.nytimes.com/hc/en-us}{Help}
\item
  \href{https://www.nytimes.com/subscription?campaignId=37WXW}{Subscriptions}
\end{itemize}
