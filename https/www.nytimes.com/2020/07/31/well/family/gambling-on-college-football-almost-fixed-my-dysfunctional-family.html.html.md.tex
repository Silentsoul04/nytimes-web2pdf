Sections

SEARCH

\protect\hyperlink{site-content}{Skip to
content}\protect\hyperlink{site-index}{Skip to site index}

\href{https://www.nytimes.com/section/well/family}{Family}

\href{https://myaccount.nytimes.com/auth/login?response_type=cookie\&client_id=vi}{}

\href{https://www.nytimes.com/section/todayspaper}{Today's Paper}

\href{/section/well/family}{Family}\textbar{}Gambling on College
Football Almost Fixed My Dysfunctional Family

\url{https://nyti.ms/3gfQiW5}

\begin{itemize}
\item
\item
\item
\item
\item
\item
\end{itemize}

\href{https://www.nytimes.com/news-event/coronavirus?action=click\&pgtype=Article\&state=default\&region=TOP_BANNER\&context=storylines_menu}{The
Coronavirus Outbreak}

\begin{itemize}
\tightlist
\item
  live\href{https://www.nytimes.com/2020/08/01/world/coronavirus-covid-19.html?action=click\&pgtype=Article\&state=default\&region=TOP_BANNER\&context=storylines_menu}{Latest
  Updates}
\item
  \href{https://www.nytimes.com/interactive/2020/us/coronavirus-us-cases.html?action=click\&pgtype=Article\&state=default\&region=TOP_BANNER\&context=storylines_menu}{Maps
  and Cases}
\item
  \href{https://www.nytimes.com/interactive/2020/science/coronavirus-vaccine-tracker.html?action=click\&pgtype=Article\&state=default\&region=TOP_BANNER\&context=storylines_menu}{Vaccine
  Tracker}
\item
  \href{https://www.nytimes.com/interactive/2020/07/29/us/schools-reopening-coronavirus.html?action=click\&pgtype=Article\&state=default\&region=TOP_BANNER\&context=storylines_menu}{What
  School May Look Like}
\item
  \href{https://www.nytimes.com/live/2020/07/31/business/stock-market-today-coronavirus?action=click\&pgtype=Article\&state=default\&region=TOP_BANNER\&context=storylines_menu}{Economy}
\end{itemize}

Advertisement

\protect\hyperlink{after-top}{Continue reading the main story}

Supported by

\protect\hyperlink{after-sponsor}{Continue reading the main story}

Ties

\hypertarget{gambling-on-college-football-almost-fixed-my-dysfunctional-family}{%
\section{Gambling on College Football Almost Fixed My Dysfunctional
Family}\label{gambling-on-college-football-almost-fixed-my-dysfunctional-family}}

If Covid takes the football season away from us, we're in danger of
losing the bond we've taken so many years to build.

\includegraphics{https://static01.nyt.com/images/2020/07/31/science/31TIES-FOOTBALL-illo/31TIES-FOOTBALL-illo-articleLarge.jpg?quality=75\&auto=webp\&disable=upscale}

By Mike Evans

\begin{itemize}
\item
  July 31, 2020
\item
  \begin{itemize}
  \item
  \item
  \item
  \item
  \item
  \item
  \end{itemize}
\end{itemize}

My first mistake was feeling sorry for him.

The first season my brother and I bet on college football against each
other, I beat him so badly I often bragged I could have lost every
single game we gambled on for the rest of the decade and still finished
in the money.

Each week, we would agree to disagree on five games across the N.C.A.A.
schedule. Each win was worth a dollar. Whoever won the most games of the
five we selected cashed an additional five bucks. Best out of five,
winner takes all for a maximum potential profit of \$10 for the weekend.

He couldn't have owed me more than \$100 --- we weren't kids anymore,
making outrageous wagers on games of blackjack at the kitchen table
neither of us could have paid off in three lifetimes --- but I still
didn't have the heart to make him pay up.

The next year, after torching him a second season in a row, I gave him a
book as a joke --- ``Handicapping College Football for Beginners,''
which he told me he relegated to the washroom magazine basket.

I didn't realize it then, but he was setting me up.

Later he admitted to reading it every chance he got. Studying. Formulas,
strategies, all of it. By season three, he cleaned my clock. Our father
soon inserted himself into the competition, which, over the past almost
20 years, came to represent our relationship: We went from being a
dysfunctional trio of man-children who didn't have the language to
express our feelings to discovering that our mutual love of competition
and one-upmanship gave us the language we needed to reconnect.

And then came the coronavirus.

As of June, in response to concerns over the coronavirus, the N.C.A.A.
Division I Football Oversight Committee
\href{https://www.espn.com/college-football/story/_/id/29325130/ncaa-division-council-approves-six-week-practice-plan-football}{announced}
their approval of a plan that would allow teams to transition from
voluntary workouts to mandatory meetings and preseason camps --- just
like any other year. But by the end of
July,\href{https://www.sportingnews.com/us/ncaa-football/news/covid-19-college-football-conferences-canceled-season-2020/19mroowdygu9z1tv54mul2vxgw}{five
Division I conferences had canceled their seasons} outright. Others, in
a last-ditch effort to play \emph{something} in 2020, are leaning toward
playing
``\href{https://www.cbssports.com/college-football/news/sec-acc-big-12-considering-plus-one-scheduling-model-with-shortened-2020-season-becoming-an-inevitability/}{conference
only'' or ``plus one'' schedules} to minimize travel and mitigate risk.
The closer we got to August, the more it seemed that Dr. Anthony Fauci,
who has been clear in his position from the outset, may have been right
after all:
``\href{https://www.cnn.com/2020/06/18/us/football-happen-fauci-spt-trnd/index.html}{Football
may not happen} this year.''

My little brother and I remain hopeful that won't be the case. Five
years apart, we were never especially close. Growing up, I'd put him
through the wringer.

\hypertarget{latest-updates-global-coronavirus-outbreak}{%
\section{\texorpdfstring{\href{https://www.nytimes.com/2020/08/01/world/coronavirus-covid-19.html?action=click\&pgtype=Article\&state=default\&region=MAIN_CONTENT_1\&context=storylines_live_updates}{Latest
Updates: Global Coronavirus
Outbreak}}{Latest Updates: Global Coronavirus Outbreak}}\label{latest-updates-global-coronavirus-outbreak}}

Updated 2020-08-02T06:58:18.835Z

\begin{itemize}
\tightlist
\item
  \href{https://www.nytimes.com/2020/08/01/world/coronavirus-covid-19.html?action=click\&pgtype=Article\&state=default\&region=MAIN_CONTENT_1\&context=storylines_live_updates\#link-34047410}{The
  U.S. reels as July cases more than double the total of any other
  month.}
\item
  \href{https://www.nytimes.com/2020/08/01/world/coronavirus-covid-19.html?action=click\&pgtype=Article\&state=default\&region=MAIN_CONTENT_1\&context=storylines_live_updates\#link-780ec966}{Top
  U.S. officials work to break an impasse over the federal jobless
  benefit.}
\item
  \href{https://www.nytimes.com/2020/08/01/world/coronavirus-covid-19.html?action=click\&pgtype=Article\&state=default\&region=MAIN_CONTENT_1\&context=storylines_live_updates\#link-2bc8948}{Its
  outbreak untamed, Melbourne goes into even greater lockdown.}
\end{itemize}

\href{https://www.nytimes.com/2020/08/01/world/coronavirus-covid-19.html?action=click\&pgtype=Article\&state=default\&region=MAIN_CONTENT_1\&context=storylines_live_updates}{See
more updates}

More live coverage:
\href{https://www.nytimes.com/live/2020/07/31/business/stock-market-today-coronavirus?action=click\&pgtype=Article\&state=default\&region=MAIN_CONTENT_1\&context=storylines_live_updates}{Markets}

When I was 8, and he was 3, I nearly took his eye out with a dead tree
branch. He still has a scar above his brow. In high school, my friends
and I would wrestle him to the ground, strip him down to his Fruit of
the Looms, force him onto the front lawn, and make him run around the
block in his skivvies before we let him back in the house. He still
delights in telling that story to showcase what kind of brother I was,
but there are plenty of other examples. I've made Baby Bro steal beer
from a convenience store ice cooler, thrown him in the trunk of a
friend's car and done doughnuts in a snowy church parking lot, and run
him over with a golf cart.

As adults, even when we both became dads, we weren't doing much better,
and I felt guilty. College football seemed like a good way to connect.
But I had no idea what I was in for. It was payback time, and every win
he tallied was sweet revenge.

``Hey. Who's winning this week?'' he would call any Saturday he was
ahead, pretending not to know.

``Really,'' I'd say. ``You know good and well who's winning.''

As much as I hated losing, I did my best to be happy for him.

The kid was due.

When he won in Season Four, evening the series at 2-2, I wasn't bothered
(much), and I wasn't all that surprised either. After all, we'd both
been raised in the same ultracompetitive, winner-takes-all environment.

Our dad never let us win at anything when we were kids. Not golf, not Go
Fish. I tell myself now, he only wanted his boys to succeed --- his
desire to win was that great --- but to say that my dad was an
\emph{enthusiastic} spectator was putting it mildly.

Looking back, I imagine in my dad's mind he was only teaching us to be
tough, to never quit or back down --- it was the 1970s and '80s when a
spanking was considered a valuable life lesson. So, it made sense after
watching our competition from the sidelines for a couple of years the
old man wanted in.

``You donkeys worried I'll beat you too badly?'' my dad goaded my
brother one summer afternoon as he casually flipped through the pages of
his Street \& Smith's ``College Football Annual.''

I knew this was going to be a problem.

The man loved sports almost as much as he loved being right, which was
\emph{a lot}. Not only did we have to mastermind a way to manage a
three-person, round robin format, but also keep our heads as my father
continued what he'd done our entire childhood: reveling in every moment
he won.

After every victory he took great pains to remind us, it would be a
\emph{long} time before we beat him at anything.

We were all supposed to be grown-ups, but most of the time we acted like
6-year-olds upset over a game of Chutes and Ladders that didn't go our
way.

\href{https://www.nytimes.com/news-event/coronavirus?action=click\&pgtype=Article\&state=default\&region=MAIN_CONTENT_3\&context=storylines_faq}{}

\hypertarget{the-coronavirus-outbreak-}{%
\subsubsection{The Coronavirus Outbreak
›}\label{the-coronavirus-outbreak-}}

\hypertarget{frequently-asked-questions}{%
\paragraph{Frequently Asked
Questions}\label{frequently-asked-questions}}

Updated July 27, 2020

\begin{itemize}
\item ~
  \hypertarget{should-i-refinance-my-mortgage}{%
  \paragraph{Should I refinance my
  mortgage?}\label{should-i-refinance-my-mortgage}}

  \begin{itemize}
  \tightlist
  \item
    \href{https://www.nytimes.com/article/coronavirus-money-unemployment.html?action=click\&pgtype=Article\&state=default\&region=MAIN_CONTENT_3\&context=storylines_faq}{It
    could be a good idea,} because mortgage rates have
    \href{https://www.nytimes.com/2020/07/16/business/mortgage-rates-below-3-percent.html?action=click\&pgtype=Article\&state=default\&region=MAIN_CONTENT_3\&context=storylines_faq}{never
    been lower.} Refinancing requests have pushed mortgage applications
    to some of the highest levels since 2008, so be prepared to get in
    line. But defaults are also up, so if you're thinking about buying a
    home, be aware that some lenders have tightened their standards.
  \end{itemize}
\item ~
  \hypertarget{what-is-school-going-to-look-like-in-september}{%
  \paragraph{What is school going to look like in
  September?}\label{what-is-school-going-to-look-like-in-september}}

  \begin{itemize}
  \tightlist
  \item
    It is unlikely that many schools will return to a normal schedule
    this fall, requiring the grind of
    \href{https://www.nytimes.com/2020/06/05/us/coronavirus-education-lost-learning.html?action=click\&pgtype=Article\&state=default\&region=MAIN_CONTENT_3\&context=storylines_faq}{online
    learning},
    \href{https://www.nytimes.com/2020/05/29/us/coronavirus-child-care-centers.html?action=click\&pgtype=Article\&state=default\&region=MAIN_CONTENT_3\&context=storylines_faq}{makeshift
    child care} and
    \href{https://www.nytimes.com/2020/06/03/business/economy/coronavirus-working-women.html?action=click\&pgtype=Article\&state=default\&region=MAIN_CONTENT_3\&context=storylines_faq}{stunted
    workdays} to continue. California's two largest public school
    districts --- Los Angeles and San Diego --- said on July 13, that
    \href{https://www.nytimes.com/2020/07/13/us/lausd-san-diego-school-reopening.html?action=click\&pgtype=Article\&state=default\&region=MAIN_CONTENT_3\&context=storylines_faq}{instruction
    will be remote-only in the fall}, citing concerns that surging
    coronavirus infections in their areas pose too dire a risk for
    students and teachers. Together, the two districts enroll some
    825,000 students. They are the largest in the country so far to
    abandon plans for even a partial physical return to classrooms when
    they reopen in August. For other districts, the solution won't be an
    all-or-nothing approach.
    \href{https://bioethics.jhu.edu/research-and-outreach/projects/eschool-initiative/school-policy-tracker/}{Many
    systems}, including the nation's largest, New York City, are
    devising
    \href{https://www.nytimes.com/2020/06/26/us/coronavirus-schools-reopen-fall.html?action=click\&pgtype=Article\&state=default\&region=MAIN_CONTENT_3\&context=storylines_faq}{hybrid
    plans} that involve spending some days in classrooms and other days
    online. There's no national policy on this yet, so check with your
    municipal school system regularly to see what is happening in your
    community.
  \end{itemize}
\item ~
  \hypertarget{is-the-coronavirus-airborne}{%
  \paragraph{Is the coronavirus
  airborne?}\label{is-the-coronavirus-airborne}}

  \begin{itemize}
  \tightlist
  \item
    The coronavirus
    \href{https://www.nytimes.com/2020/07/04/health/239-experts-with-one-big-claim-the-coronavirus-is-airborne.html?action=click\&pgtype=Article\&state=default\&region=MAIN_CONTENT_3\&context=storylines_faq}{can
    stay aloft for hours in tiny droplets in stagnant air}, infecting
    people as they inhale, mounting scientific evidence suggests. This
    risk is highest in crowded indoor spaces with poor ventilation, and
    may help explain super-spreading events reported in meatpacking
    plants, churches and restaurants.
    \href{https://www.nytimes.com/2020/07/06/health/coronavirus-airborne-aerosols.html?action=click\&pgtype=Article\&state=default\&region=MAIN_CONTENT_3\&context=storylines_faq}{It's
    unclear how often the virus is spread} via these tiny droplets, or
    aerosols, compared with larger droplets that are expelled when a
    sick person coughs or sneezes, or transmitted through contact with
    contaminated surfaces, said Linsey Marr, an aerosol expert at
    Virginia Tech. Aerosols are released even when a person without
    symptoms exhales, talks or sings, according to Dr. Marr and more
    than 200 other experts, who
    \href{https://academic.oup.com/cid/article/doi/10.1093/cid/ciaa939/5867798}{have
    outlined the evidence in an open letter to the World Health
    Organization}.
  \end{itemize}
\item ~
  \hypertarget{what-are-the-symptoms-of-coronavirus}{%
  \paragraph{What are the symptoms of
  coronavirus?}\label{what-are-the-symptoms-of-coronavirus}}

  \begin{itemize}
  \tightlist
  \item
    Common symptoms
    \href{https://www.nytimes.com/article/symptoms-coronavirus.html?action=click\&pgtype=Article\&state=default\&region=MAIN_CONTENT_3\&context=storylines_faq}{include
    fever, a dry cough, fatigue and difficulty breathing or shortness of
    breath.} Some of these symptoms overlap with those of the flu,
    making detection difficult, but runny noses and stuffy sinuses are
    less common.
    \href{https://www.nytimes.com/2020/04/27/health/coronavirus-symptoms-cdc.html?action=click\&pgtype=Article\&state=default\&region=MAIN_CONTENT_3\&context=storylines_faq}{The
    C.D.C. has also} added chills, muscle pain, sore throat, headache
    and a new loss of the sense of taste or smell as symptoms to look
    out for. Most people fall ill five to seven days after exposure, but
    symptoms may appear in as few as two days or as many as 14 days.
  \end{itemize}
\item ~
  \hypertarget{does-asymptomatic-transmission-of-covid-19-happen}{%
  \paragraph{Does asymptomatic transmission of Covid-19
  happen?}\label{does-asymptomatic-transmission-of-covid-19-happen}}

  \begin{itemize}
  \tightlist
  \item
    So far, the evidence seems to show it does. A widely cited
    \href{https://www.nature.com/articles/s41591-020-0869-5}{paper}
    published in April suggests that people are most infectious about
    two days before the onset of coronavirus symptoms and estimated that
    44 percent of new infections were a result of transmission from
    people who were not yet showing symptoms. Recently, a top expert at
    the World Health Organization stated that transmission of the
    coronavirus by people who did not have symptoms was ``very rare,''
    \href{https://www.nytimes.com/2020/06/09/world/coronavirus-updates.html?action=click\&pgtype=Article\&state=default\&region=MAIN_CONTENT_3\&context=storylines_faq\#link-1f302e21}{but
    she later walked back that statement.}
  \end{itemize}
\end{itemize}

We showed we cared by needling each other unmercifully anytime one of us
wound up on the wrong end of the point spread.

Like the year my dad gave my brother and me second and third place
medals to make sure we didn't forget who had come out on top that
season.

Or when visiting my parents once, my father introduced me to friends of
his and my mother's as ``the one who finished in last place'' the year
before.

I still don't know half of what I should about my brother, or agree with
all the things he believes in. But I'm learning. That ratio skews much
higher when it comes to my dad. I've realized my brother, dad and I
aren't all that different. We all want to be heard, each of us wants to
be seen, and above all, each of us wants to win. After almost 20 years
of this, our bonds are stronger than ever.

As disappointing as the prospect may be, whether college football
happens this year or not, at least now I have a reason to call.

The bonds we've worked so hard to build --- even if they've come from
trash talking each other over our latest win-loss records --- are in
danger of being lost. If Covid takes that away from us, we'll just have
to find something else to fight, I mean, connect over.

\emph{Mike Evans is a writer and television producer living in Los
Angeles. He is currently at work on a memoir.}

\begin{center}\rule{0.5\linewidth}{\linethickness}\end{center}

Advertisement

\protect\hyperlink{after-bottom}{Continue reading the main story}

\hypertarget{site-index}{%
\subsection{Site Index}\label{site-index}}

\hypertarget{site-information-navigation}{%
\subsection{Site Information
Navigation}\label{site-information-navigation}}

\begin{itemize}
\tightlist
\item
  \href{https://help.nytimes.com/hc/en-us/articles/115014792127-Copyright-notice}{©~2020~The
  New York Times Company}
\end{itemize}

\begin{itemize}
\tightlist
\item
  \href{https://www.nytco.com/}{NYTCo}
\item
  \href{https://help.nytimes.com/hc/en-us/articles/115015385887-Contact-Us}{Contact
  Us}
\item
  \href{https://www.nytco.com/careers/}{Work with us}
\item
  \href{https://nytmediakit.com/}{Advertise}
\item
  \href{http://www.tbrandstudio.com/}{T Brand Studio}
\item
  \href{https://www.nytimes.com/privacy/cookie-policy\#how-do-i-manage-trackers}{Your
  Ad Choices}
\item
  \href{https://www.nytimes.com/privacy}{Privacy}
\item
  \href{https://help.nytimes.com/hc/en-us/articles/115014893428-Terms-of-service}{Terms
  of Service}
\item
  \href{https://help.nytimes.com/hc/en-us/articles/115014893968-Terms-of-sale}{Terms
  of Sale}
\item
  \href{https://spiderbites.nytimes.com}{Site Map}
\item
  \href{https://help.nytimes.com/hc/en-us}{Help}
\item
  \href{https://www.nytimes.com/subscription?campaignId=37WXW}{Subscriptions}
\end{itemize}
