Sections

SEARCH

\protect\hyperlink{site-content}{Skip to
content}\protect\hyperlink{site-index}{Skip to site index}

\href{https://www.nytimes.com/section/world/africa}{Africa}

\href{https://myaccount.nytimes.com/auth/login?response_type=cookie\&client_id=vi}{}

\href{https://www.nytimes.com/section/todayspaper}{Today's Paper}

\href{/section/world/africa}{Africa}\textbar{}Zimbabwe Locks Down
Capital, Thwarting Planned Protests

\url{https://nyti.ms/315W0DF}

\begin{itemize}
\item
\item
\item
\item
\item
\item
\end{itemize}

\href{https://www.nytimes.com/news-event/coronavirus?action=click\&pgtype=Article\&state=default\&region=TOP_BANNER\&context=storylines_menu}{The
Coronavirus Outbreak}

\begin{itemize}
\tightlist
\item
  live\href{https://www.nytimes.com/2020/08/01/world/coronavirus-covid-19.html?action=click\&pgtype=Article\&state=default\&region=TOP_BANNER\&context=storylines_menu}{Latest
  Updates}
\item
  \href{https://www.nytimes.com/interactive/2020/us/coronavirus-us-cases.html?action=click\&pgtype=Article\&state=default\&region=TOP_BANNER\&context=storylines_menu}{Maps
  and Cases}
\item
  \href{https://www.nytimes.com/interactive/2020/science/coronavirus-vaccine-tracker.html?action=click\&pgtype=Article\&state=default\&region=TOP_BANNER\&context=storylines_menu}{Vaccine
  Tracker}
\item
  \href{https://www.nytimes.com/interactive/2020/07/29/us/schools-reopening-coronavirus.html?action=click\&pgtype=Article\&state=default\&region=TOP_BANNER\&context=storylines_menu}{What
  School May Look Like}
\item
  \href{https://www.nytimes.com/live/2020/07/31/business/stock-market-today-coronavirus?action=click\&pgtype=Article\&state=default\&region=TOP_BANNER\&context=storylines_menu}{Economy}
\end{itemize}

Advertisement

\protect\hyperlink{after-top}{Continue reading the main story}

Supported by

\protect\hyperlink{after-sponsor}{Continue reading the main story}

\hypertarget{zimbabwe-locks-down-capital-thwarting-planned-protests}{%
\section{Zimbabwe Locks Down Capital, Thwarting Planned
Protests}\label{zimbabwe-locks-down-capital-thwarting-planned-protests}}

Three years after the fall of the strongman Robert Mugabe, the country
is in free fall and his successor is clamping down by arresting
opposition activists --- including an author just nominated for the
Booker Prize.

\includegraphics{https://static01.nyt.com/images/2020/07/31/world/31zimbabwe-unrest001sub/31zimbabwe-unrest001sub-articleLarge.jpg?quality=75\&auto=webp\&disable=upscale}

By Jeffrey Moyo and
\href{https://www.nytimes.com/by/patrick-kingsley}{Patrick Kingsley}

\begin{itemize}
\item
  July 31, 2020
\item
  \begin{itemize}
  \item
  \item
  \item
  \item
  \item
  \item
  \end{itemize}
\end{itemize}

HARARE, Zimbabwe --- When Robert Mugabe was ousted as president of
Zimbabwe in
\href{https://www.nytimes.com/2017/11/21/world/africa/zimbabwe-mugabe-mnangagwa.html}{a
coup in 2017}, many in the country hoped for an end to the repression
and mismanagement that had characterized his 37 years in office.

But when protesters tried to gather on Friday in the capital, Harare,
the security services reacted in a manner reminiscent of the Mugabe era.
They shut down most of the city, arrested several government critics and
forced more than a dozen others into hiding --- highlighting how the
country has, in the eyes of the opposition, slipped from bad to worse
under Mr. Mugabe's successor,
\href{https://www.nytimes.com/2019/08/10/world/africa/zimbabwe-president-emmerson-mnangagwa-mugabe.html}{President
Emmerson Mnangagwa}.

Mr. Mnangagwa took power on a promise of renewal, but his critics
associate him with the same excesses that defined Mr. Mugabe's legacy:
\href{https://www.nytimes.com/2019/01/23/world/africa/zimbabwe-protests-emmerson-mnangagwa.html}{authoritarian
rule}, financial missteps, rampant graft,
\href{https://www.nytimes.com/2019/07/31/world/africa/zimbabwe-water-crisis.html}{plummeting
living standards} and a teetering economy.

``In reality, there is nothing new,'' said Obey Sithole, a leading
opposition campaigner who went into hiding days before the planned
protests. ``Instead, we have seen the perfection of the art of
repression.''

And the government's halting response to the coronavirus pandemic, which
has exposed the awful state of the country's health care system and led
to further allegations of corruption, has fueled widespread anger.

\includegraphics{https://static01.nyt.com/images/2020/07/31/world/31Zimbabwe-unrest02/merlin_158613132_646dc255-9048-4610-936b-98957e464acd-articleLarge.jpg?quality=75\&auto=webp\&disable=upscale}

\href{https://www.nytimes.com/2019/08/10/world/africa/zimbabwe-president-emmerson-mnangagwa-mugabe.html?action=click\&module=RelatedLinks\&pgtype=Article}{In
an interview with The Times last year}, Mr. Mnangagwa, 77, described
himself as a leader with a ``new dispensation.'' But in some respects,
Mr. Mnangagwa --- a veteran of the guerrilla war that ended
white-minority rule, and one of Mr. Mugabe's most trusted sidekicks ---
has proved to be an even harsher president than Mr. Mugabe.

The number of opposition activists charged with a form of treason during
Mr. Mnangagwa's three years in office is already higher than during Mr.
Mugabe's entire tenure, according to research by a coalition of 22
Zimbabwean rights groups.

Opposition activists hoped to hold mass rallies on Friday, partly in
response to a new wave of arrests and abductions that began in May, when
three female opposition activists, including a lawmaker, were abducted,
beaten and sexually assaulted by people they say were plainclothes
government agents. (The government denied involvement and, after being
treated in hospital, the women were charged with false accusations.)

But Mr. Mnangagwa's government refused to allow even this largely
symbolic expression of dissent.

To deter protesters in the prelude to the demonstrations, the police
seized the leader of an opposition group who had been helping to
organize the rally and a prominent investigative journalist who had
helped reveal a possible corruption scheme involving the country's
health minister, Obadiah Moyo. More than a dozen opposition politicians,
activists and union leaders, including Mr. Sithole, then went into
hiding after the police named them on a wanted list.

\hypertarget{latest-updates-global-coronavirus-outbreak}{%
\section{\texorpdfstring{\href{https://www.nytimes.com/2020/08/01/world/coronavirus-covid-19.html?action=click\&pgtype=Article\&state=default\&region=MAIN_CONTENT_1\&context=storylines_live_updates}{Latest
Updates: Global Coronavirus
Outbreak}}{Latest Updates: Global Coronavirus Outbreak}}\label{latest-updates-global-coronavirus-outbreak}}

Updated 2020-08-01T18:23:51.652Z

\begin{itemize}
\tightlist
\item
  \href{https://www.nytimes.com/2020/08/01/world/coronavirus-covid-19.html?action=click\&pgtype=Article\&state=default\&region=MAIN_CONTENT_1\&context=storylines_live_updates\#link-3ac56579}{Top
  officials work to break impasse over jobless benefit.}
\item
  \href{https://www.nytimes.com/2020/08/01/world/coronavirus-covid-19.html?action=click\&pgtype=Article\&state=default\&region=MAIN_CONTENT_1\&context=storylines_live_updates\#link-8796723}{The
  virus picks up dangerous speed in the Midwest, and in areas that had
  seen success.}
\item
  \href{https://www.nytimes.com/2020/08/01/world/coronavirus-covid-19.html?action=click\&pgtype=Article\&state=default\&region=MAIN_CONTENT_1\&context=storylines_live_updates\#link-25930521}{Thousands
  in Berlin protest Germany's coronavirus measures.}
\end{itemize}

\href{https://www.nytimes.com/2020/08/01/world/coronavirus-covid-19.html?action=click\&pgtype=Article\&state=default\&region=MAIN_CONTENT_1\&context=storylines_live_updates}{See
more updates}

More live coverage:
\href{https://www.nytimes.com/live/2020/07/31/business/stock-market-today-coronavirus?action=click\&pgtype=Article\&state=default\&region=MAIN_CONTENT_1\&context=storylines_live_updates}{Markets}

And on Friday, the police deployed personnel across the city, shutting
down most major transit routes and deterring most would-be protesters
from gathering. Several of those who did try to assemble said that they
had been detained, including the author Tsitsi Dangarembga, three days
after her novel ``This Mournable Body'' was
\href{https://thebookerprizes.com/books/mournable-body-by-tsitsi-dangarembga}{longlisted
for the Booker Prize}, a prestigious British literary award.

``During President Mugabe's era, there were serious, gross human rights
violations,'' said Robson Chere, the head of a major teachers' union,
and one of those now in hiding. ``But the current so-called new
dispensation has gone several gears up.''

The situation has led to tensions within Mr. Mnangagwa's political
party. A party official was fired this week after being accused of
helping to promote the protests, and the government itself has been
forced to deny that the military that brought Mr. Mnangagwa to power was
now seeking to oust him.

Those hoping to demonstrate on Friday had an extra grievance --- Mr.
Mnangagwa's handling of the coronavirus crisis, which critics see as
reflective of his government's wider faults.

Already close to collapse before the pandemic, hospitals lack enough
drugs, ventilators, personal protection equipment and staff, because
many doctors and nurses have moved abroad in search of better pay or
gone on strike to protest their low wages.

Police officers have used coronavirus restrictions as a pretext to
arrest the government's political opponents, according to Zimbabwe
Lawyers for Human Rights, a watchdog group based in Harare.

Mr. Moyo, the health minister, was fired in July after buying
coronavirus supplies at inflated prices through a multimillion-dollar
contract with an obscure foreign firm that was signed without the
approval of the relevant state authorities, according to court
documents.

Image

Nurses protested in Harare this month to demand a salary increase.
Hospitals in Zimbabwe lack enough drugs, ventilators, personal
protection equipment and staff.Credit...Aaron Ufumeli/EPA, via
Shutterstock

Mr. Moyo, who has yet to be replaced, was arrested and is on trial.

``The system has collapsed under our president,'' said Dr. Peter
Magombeyi, a former head of the Zimbabwe Hospital Doctors Association, a
labor union. ``We are noticing that there are no doctors, there are no
nurses, there are no drugs, not enough personal protective equipment, no
hospital C.E.O.s, no health care minister.''

The leadership of Mr. Mnangagwa's party had urged supporters to ``take
on'' the protesters, and attempted to portray the possibility of
demonstrations as a Western plot.

``None of it is coming from Zimbabweans,'' said Tafadzwa Mugwadi, the
party's director of information.

Another party official, Patrick Chinamasa, said that the American
ambassador to Zimbabwe, Brian A. Nichols, might be expelled from the
country, after the U.S. Embassy
\href{https://twitter.com/usembassyharare/status/1285856222579699713?s=20}{issued
a series of tweets} criticizing some of the arrests this month.

Mr. Chinamasa said at a news briefing on Monday that if Mr. Nichols were
to continue ``engaging in acts of mobilizing and funding disturbances,
coordinating violence and training fighters,'' then ``our leadership
will not hesitate to give him marching orders.''

Image

Shops were closed and street almost empty in Harare on Thursday ahead of
anti-government protests.Credit...Aaron Ufumeli/EPA, via Shutterstock

But in reality, there is plenty of homegrown anger --- not only about
rights violations, but also about the dire state of the health system
and deteriorating living conditions. Corruption and mismanagement have
led to the collapse of the economy and vast underinvestment in
infrastructure.

\href{https://www.nytimes.com/news-event/coronavirus?action=click\&pgtype=Article\&state=default\&region=MAIN_CONTENT_3\&context=storylines_faq}{}

\hypertarget{the-coronavirus-outbreak-}{%
\subsubsection{The Coronavirus Outbreak
›}\label{the-coronavirus-outbreak-}}

\hypertarget{frequently-asked-questions}{%
\paragraph{Frequently Asked
Questions}\label{frequently-asked-questions}}

Updated July 27, 2020

\begin{itemize}
\item ~
  \hypertarget{should-i-refinance-my-mortgage}{%
  \paragraph{Should I refinance my
  mortgage?}\label{should-i-refinance-my-mortgage}}

  \begin{itemize}
  \tightlist
  \item
    \href{https://www.nytimes.com/article/coronavirus-money-unemployment.html?action=click\&pgtype=Article\&state=default\&region=MAIN_CONTENT_3\&context=storylines_faq}{It
    could be a good idea,} because mortgage rates have
    \href{https://www.nytimes.com/2020/07/16/business/mortgage-rates-below-3-percent.html?action=click\&pgtype=Article\&state=default\&region=MAIN_CONTENT_3\&context=storylines_faq}{never
    been lower.} Refinancing requests have pushed mortgage applications
    to some of the highest levels since 2008, so be prepared to get in
    line. But defaults are also up, so if you're thinking about buying a
    home, be aware that some lenders have tightened their standards.
  \end{itemize}
\item ~
  \hypertarget{what-is-school-going-to-look-like-in-september}{%
  \paragraph{What is school going to look like in
  September?}\label{what-is-school-going-to-look-like-in-september}}

  \begin{itemize}
  \tightlist
  \item
    It is unlikely that many schools will return to a normal schedule
    this fall, requiring the grind of
    \href{https://www.nytimes.com/2020/06/05/us/coronavirus-education-lost-learning.html?action=click\&pgtype=Article\&state=default\&region=MAIN_CONTENT_3\&context=storylines_faq}{online
    learning},
    \href{https://www.nytimes.com/2020/05/29/us/coronavirus-child-care-centers.html?action=click\&pgtype=Article\&state=default\&region=MAIN_CONTENT_3\&context=storylines_faq}{makeshift
    child care} and
    \href{https://www.nytimes.com/2020/06/03/business/economy/coronavirus-working-women.html?action=click\&pgtype=Article\&state=default\&region=MAIN_CONTENT_3\&context=storylines_faq}{stunted
    workdays} to continue. California's two largest public school
    districts --- Los Angeles and San Diego --- said on July 13, that
    \href{https://www.nytimes.com/2020/07/13/us/lausd-san-diego-school-reopening.html?action=click\&pgtype=Article\&state=default\&region=MAIN_CONTENT_3\&context=storylines_faq}{instruction
    will be remote-only in the fall}, citing concerns that surging
    coronavirus infections in their areas pose too dire a risk for
    students and teachers. Together, the two districts enroll some
    825,000 students. They are the largest in the country so far to
    abandon plans for even a partial physical return to classrooms when
    they reopen in August. For other districts, the solution won't be an
    all-or-nothing approach.
    \href{https://bioethics.jhu.edu/research-and-outreach/projects/eschool-initiative/school-policy-tracker/}{Many
    systems}, including the nation's largest, New York City, are
    devising
    \href{https://www.nytimes.com/2020/06/26/us/coronavirus-schools-reopen-fall.html?action=click\&pgtype=Article\&state=default\&region=MAIN_CONTENT_3\&context=storylines_faq}{hybrid
    plans} that involve spending some days in classrooms and other days
    online. There's no national policy on this yet, so check with your
    municipal school system regularly to see what is happening in your
    community.
  \end{itemize}
\item ~
  \hypertarget{is-the-coronavirus-airborne}{%
  \paragraph{Is the coronavirus
  airborne?}\label{is-the-coronavirus-airborne}}

  \begin{itemize}
  \tightlist
  \item
    The coronavirus
    \href{https://www.nytimes.com/2020/07/04/health/239-experts-with-one-big-claim-the-coronavirus-is-airborne.html?action=click\&pgtype=Article\&state=default\&region=MAIN_CONTENT_3\&context=storylines_faq}{can
    stay aloft for hours in tiny droplets in stagnant air}, infecting
    people as they inhale, mounting scientific evidence suggests. This
    risk is highest in crowded indoor spaces with poor ventilation, and
    may help explain super-spreading events reported in meatpacking
    plants, churches and restaurants.
    \href{https://www.nytimes.com/2020/07/06/health/coronavirus-airborne-aerosols.html?action=click\&pgtype=Article\&state=default\&region=MAIN_CONTENT_3\&context=storylines_faq}{It's
    unclear how often the virus is spread} via these tiny droplets, or
    aerosols, compared with larger droplets that are expelled when a
    sick person coughs or sneezes, or transmitted through contact with
    contaminated surfaces, said Linsey Marr, an aerosol expert at
    Virginia Tech. Aerosols are released even when a person without
    symptoms exhales, talks or sings, according to Dr. Marr and more
    than 200 other experts, who
    \href{https://academic.oup.com/cid/article/doi/10.1093/cid/ciaa939/5867798}{have
    outlined the evidence in an open letter to the World Health
    Organization}.
  \end{itemize}
\item ~
  \hypertarget{what-are-the-symptoms-of-coronavirus}{%
  \paragraph{What are the symptoms of
  coronavirus?}\label{what-are-the-symptoms-of-coronavirus}}

  \begin{itemize}
  \tightlist
  \item
    Common symptoms
    \href{https://www.nytimes.com/article/symptoms-coronavirus.html?action=click\&pgtype=Article\&state=default\&region=MAIN_CONTENT_3\&context=storylines_faq}{include
    fever, a dry cough, fatigue and difficulty breathing or shortness of
    breath.} Some of these symptoms overlap with those of the flu,
    making detection difficult, but runny noses and stuffy sinuses are
    less common.
    \href{https://www.nytimes.com/2020/04/27/health/coronavirus-symptoms-cdc.html?action=click\&pgtype=Article\&state=default\&region=MAIN_CONTENT_3\&context=storylines_faq}{The
    C.D.C. has also} added chills, muscle pain, sore throat, headache
    and a new loss of the sense of taste or smell as symptoms to look
    out for. Most people fall ill five to seven days after exposure, but
    symptoms may appear in as few as two days or as many as 14 days.
  \end{itemize}
\item ~
  \hypertarget{does-asymptomatic-transmission-of-covid-19-happen}{%
  \paragraph{Does asymptomatic transmission of Covid-19
  happen?}\label{does-asymptomatic-transmission-of-covid-19-happen}}

  \begin{itemize}
  \tightlist
  \item
    So far, the evidence seems to show it does. A widely cited
    \href{https://www.nature.com/articles/s41591-020-0869-5}{paper}
    published in April suggests that people are most infectious about
    two days before the onset of coronavirus symptoms and estimated that
    44 percent of new infections were a result of transmission from
    people who were not yet showing symptoms. Recently, a top expert at
    the World Health Organization stated that transmission of the
    coronavirus by people who did not have symptoms was ``very rare,''
    \href{https://www.nytimes.com/2020/06/09/world/coronavirus-updates.html?action=click\&pgtype=Article\&state=default\&region=MAIN_CONTENT_3\&context=storylines_faq\#link-1f302e21}{but
    she later walked back that statement.}
  \end{itemize}
\end{itemize}

The situation is compounded by international sanctions on Zimbabwean
individuals and institutions, which are potential obstacles to loans
from the World Bank and the International Monetary Fund.

Dr. Magombeyi, the former doctors' union head, provoked widespread
horror this week when he shared an image on social media of what he said
were the corpses of seven stillborn infants at a single hospital in
Harare. Citing doctors at the hospital, Dr. Magombeyi said that the
seven had all died on the same night, after a staffing shortage caused
delays to their mothers' prenatal care.

Stagnant wages and rampant inflation have made basic medicines
unaffordable to most patients.

In June, the annual inflation rate was more than 700 percent, devaluing
salaries and making common household goods beyond the reach of many
citizens. Since the end of Mr. Mugabe's rule, which was itself marked by
profound economic upheaval, the cost of a loaf of bread has risen
roughly 70-fold, turning it from a staple into a luxury.

Image

Harare in March. Rampant inflation has devalued salaries and put common
household goods beyond the reach of many citizens.Credit...Jekesai
Njikizana/Agence France-Presse --- Getty Images

In recent months, the country's electricity crisis has ebbed: Most
households no longer face daily power outages, partly because the
coronavirus restrictions have caused a drop in demand.

But Zimbabweans still face daily water shortages. Parts of Harare
receive running water only once or twice a week, forcing many to line up
for hours at wells, springs and streams.

Against this backdrop, Mr. Mnangagwa caused further anger this week by
announcing a plan to raise \$3.5 billion through government bond sales
to compensate white farmers who were violently expelled from their land
under Mr. Mugabe. But there has been no similarly ambitious plan to
finance improvements to the health system or water infrastructure.

``We have suffered enough,'' said Denis Chengeto, a 55-year-old
unemployed teacher, speaking ahead of the protests on Thursday. ``We
have a government that doesn't care.''

But after the government locked down the city on Friday, Mr. Chengeto
said he was now too frightened to protest.

``Nobody may hear my voice today,'' he said on Friday. ``I know soldiers
won't hesitate to shoot at anyone if we go on the streets.''

Jeffrey Moyo reported from Harare; and Patrick Kingsley from Berlin.

Advertisement

\protect\hyperlink{after-bottom}{Continue reading the main story}

\hypertarget{site-index}{%
\subsection{Site Index}\label{site-index}}

\hypertarget{site-information-navigation}{%
\subsection{Site Information
Navigation}\label{site-information-navigation}}

\begin{itemize}
\tightlist
\item
  \href{https://help.nytimes.com/hc/en-us/articles/115014792127-Copyright-notice}{©~2020~The
  New York Times Company}
\end{itemize}

\begin{itemize}
\tightlist
\item
  \href{https://www.nytco.com/}{NYTCo}
\item
  \href{https://help.nytimes.com/hc/en-us/articles/115015385887-Contact-Us}{Contact
  Us}
\item
  \href{https://www.nytco.com/careers/}{Work with us}
\item
  \href{https://nytmediakit.com/}{Advertise}
\item
  \href{http://www.tbrandstudio.com/}{T Brand Studio}
\item
  \href{https://www.nytimes.com/privacy/cookie-policy\#how-do-i-manage-trackers}{Your
  Ad Choices}
\item
  \href{https://www.nytimes.com/privacy}{Privacy}
\item
  \href{https://help.nytimes.com/hc/en-us/articles/115014893428-Terms-of-service}{Terms
  of Service}
\item
  \href{https://help.nytimes.com/hc/en-us/articles/115014893968-Terms-of-sale}{Terms
  of Sale}
\item
  \href{https://spiderbites.nytimes.com}{Site Map}
\item
  \href{https://help.nytimes.com/hc/en-us}{Help}
\item
  \href{https://www.nytimes.com/subscription?campaignId=37WXW}{Subscriptions}
\end{itemize}
