Sections

SEARCH

\protect\hyperlink{site-content}{Skip to
content}\protect\hyperlink{site-index}{Skip to site index}

\href{https://www.nytimes.com/section/world/asia}{Asia Pacific}

\href{https://myaccount.nytimes.com/auth/login?response_type=cookie\&client_id=vi}{}

\href{https://www.nytimes.com/section/todayspaper}{Today's Paper}

\href{/section/world/asia}{Asia Pacific}\textbar{}Hong Kong Delays
Election, Citing Coronavirus. The Opposition Isn't Buying It.

\url{https://nyti.ms/2XdiHVr}

\begin{itemize}
\item
\item
\item
\item
\item
\item
\end{itemize}

\href{https://www.nytimes.com/news-event/coronavirus?action=click\&pgtype=Article\&state=default\&region=TOP_BANNER\&context=storylines_menu}{The
Coronavirus Outbreak}

\begin{itemize}
\tightlist
\item
  live\href{https://www.nytimes.com/2020/08/01/world/coronavirus-covid-19.html?action=click\&pgtype=Article\&state=default\&region=TOP_BANNER\&context=storylines_menu}{Latest
  Updates}
\item
  \href{https://www.nytimes.com/interactive/2020/us/coronavirus-us-cases.html?action=click\&pgtype=Article\&state=default\&region=TOP_BANNER\&context=storylines_menu}{Maps
  and Cases}
\item
  \href{https://www.nytimes.com/interactive/2020/science/coronavirus-vaccine-tracker.html?action=click\&pgtype=Article\&state=default\&region=TOP_BANNER\&context=storylines_menu}{Vaccine
  Tracker}
\item
  \href{https://www.nytimes.com/interactive/2020/07/29/us/schools-reopening-coronavirus.html?action=click\&pgtype=Article\&state=default\&region=TOP_BANNER\&context=storylines_menu}{What
  School May Look Like}
\item
  \href{https://www.nytimes.com/live/2020/07/31/business/stock-market-today-coronavirus?action=click\&pgtype=Article\&state=default\&region=TOP_BANNER\&context=storylines_menu}{Economy}
\end{itemize}

Advertisement

\protect\hyperlink{after-top}{Continue reading the main story}

Supported by

\protect\hyperlink{after-sponsor}{Continue reading the main story}

\hypertarget{hong-kong-delays-election-citing-coronavirus-the-opposition-isnt-buying-it}{%
\section{Hong Kong Delays Election, Citing Coronavirus. The Opposition
Isn't Buying
It.}\label{hong-kong-delays-election-citing-coronavirus-the-opposition-isnt-buying-it}}

Pro-democracy politicians, who had hoped to ride widespread discontent
to big gains in the fall, saw the yearlong delay as an attempt to thwart
their momentum.

\includegraphics{https://static01.nyt.com/images/2020/07/30/world/00hongkong-election/merlin_175101768_b26b54b4-9bc7-438c-9d80-e2644122f7c3-articleLarge.jpg?quality=75\&auto=webp\&disable=upscale}

\href{https://www.nytimes.com/by/austin-ramzy}{\includegraphics{https://static01.nyt.com/images/2018/10/15/multimedia/author-austin-ramzy/author-austin-ramzy-thumbLarge.png}}

By \href{https://www.nytimes.com/by/austin-ramzy}{Austin Ramzy}

\begin{itemize}
\item
  July 31, 2020
\item
  \begin{itemize}
  \item
  \item
  \item
  \item
  \item
  \item
  \end{itemize}
\end{itemize}

HONG KONG --- The Hong Kong government said on Friday that it would
postpone the city's September legislative election by one year because
of the coronavirus pandemic, a decision seen by the pro-democracy
opposition as a brazen attempt to thwart its electoral momentum and
avoid the defeat of pro-Beijing candidates.

``It is a really tough decision to delay, but we want to ensure
fairness, public safety and public health,'' said Carrie Lam, Hong
Kong's chief executive.

She cited the risk of infections, with as many as three million or more
people expected to vote on the same day; the inability of candidates to
hold campaign events due to social distancing rules; and the
difficulties faced by eligible voters who are overseas or in mainland
China and cannot return to cast ballots because of travel restrictions.

The delay was a blow to opposition politicians, who had hoped to ride to
victory in the fall on a wave of deep-seated dissatisfaction with the
government and concerns about a sweeping new national security law
imposed by Beijing on Hong Kong. And it was the latest in a quick series
of aggressive moves by the pro-Beijing establishment that had the effect
of sidelining the pro-democracy movement.

On Thursday,
\href{https://www.nytimes.com/2020/07/29/world/asia/hong-kong-arrests-security-law.html}{12
pro-democracy candidates said they had been barred from running},
including four sitting lawmakers and several prominent activists like
Joshua Wong. Mr. Wong said he was barred in part because of his
criticism of the new security law.

``Clearly it is the largest election fraud in \#HK's history,'' Mr. Wong
\href{https://twitter.com/joshuawongcf/status/1289147160009887745}{wrote
on Twitter} after Mrs. Lam announced the postponement.

\includegraphics{https://static01.nyt.com/images/2020/07/31/world/31hongkong-election-wong/merlin_175140705_2e968632-1c50-43bc-af28-3d8e2048403e-articleLarge.jpg?quality=75\&auto=webp\&disable=upscale}

Even before Friday, the city's pro-democracy opposition had accused the
government of using social-distancing rules to clamp down on the protest
movement that began more than a year ago.

Earlier this week, amid reports that the vote might be delayed, Eddie
Chu, a pro-democracy legislator running for re-election, said that
China's ruling Communist Party was ordering ``a strategic retreat.''
They ``want to avoid a potential devastating defeat'' in the election,
\href{https://twitter.com/ChuHoiDick/status/1287939279302193157}{he
wrote} on Twitter.

\hypertarget{latest-updates-global-coronavirus-outbreak}{%
\section{\texorpdfstring{\href{https://www.nytimes.com/2020/08/01/world/coronavirus-covid-19.html?action=click\&pgtype=Article\&state=default\&region=MAIN_CONTENT_1\&context=storylines_live_updates}{Latest
Updates: Global Coronavirus
Outbreak}}{Latest Updates: Global Coronavirus Outbreak}}\label{latest-updates-global-coronavirus-outbreak}}

Updated 2020-08-01T18:23:51.652Z

\begin{itemize}
\tightlist
\item
  \href{https://www.nytimes.com/2020/08/01/world/coronavirus-covid-19.html?action=click\&pgtype=Article\&state=default\&region=MAIN_CONTENT_1\&context=storylines_live_updates\#link-3ac56579}{Top
  officials work to break impasse over jobless benefit.}
\item
  \href{https://www.nytimes.com/2020/08/01/world/coronavirus-covid-19.html?action=click\&pgtype=Article\&state=default\&region=MAIN_CONTENT_1\&context=storylines_live_updates\#link-8796723}{The
  virus picks up dangerous speed in the Midwest, and in areas that had
  seen success.}
\item
  \href{https://www.nytimes.com/2020/08/01/world/coronavirus-covid-19.html?action=click\&pgtype=Article\&state=default\&region=MAIN_CONTENT_1\&context=storylines_live_updates\#link-25930521}{Thousands
  in Berlin protest Germany's coronavirus measures.}
\end{itemize}

\href{https://www.nytimes.com/2020/08/01/world/coronavirus-covid-19.html?action=click\&pgtype=Article\&state=default\&region=MAIN_CONTENT_1\&context=storylines_live_updates}{See
more updates}

More live coverage:
\href{https://www.nytimes.com/live/2020/07/31/business/stock-market-today-coronavirus?action=click\&pgtype=Article\&state=default\&region=MAIN_CONTENT_1\&context=storylines_live_updates}{Markets}

The explanation that Hong Kong must delay the vote because of the
pandemic is likely to fall flat among the wider public, said Ma Ngok, an
associate professor of political science at the Chinese University of
Hong Kong.

``I think it will be seen as a kind of manipulation, that the government
is afraid of losing the majority and that is why they postponed the
election,'' he said.

Mrs. Lam denied that the decision had been influenced by political
concerns. ``It is purely on the basis of protecting the health and
safety of the Hong Kong people and to ensure that the elections are held
in a fair and open manner,'' she said.

While Hong Kong has been a world leader in controlling the coronavirus,
in recent days it has seen its worst surge of infections yet, with more
than 100 new cases reported daily for more than a week. The government
has unfurled several new lockdown and social-distancing
\href{https://www.nytimes.com/2020/07/31/world/asia/hongkong-dining-lunch-coronavirus.html}{measures}.

``We face a dire situation in our fight against the virus,'' Mrs. Lam
said.

Under Hong Kong law, an election can be delayed for up to 14 days if
there is a ``danger to public health or safety.'' But Mrs. Lam postponed
the election until Sept. 5, 2021, under emergency powers that allow the
chief executive to make any regulations considered to be ``desirable in
the public interest.''

Those powers, which date to the British colonial era, were invoked last
year when the government
\href{https://www.nytimes.com/2019/10/04/world/asia/hong-kong-emergency-powers.html}{banned
the wearing of masks in an effort to stem protests}.

China's central government said it supported Mrs. Lam's decision to
delay the election,
\href{https://xhpfmapi.zhongguowangshi.com/vh512/share/9292212}{the
state-run Xinhua News Agency reported}.

Mrs. Lam acknowledged that the move created a ``rather thorny issue''
under the Basic Law, Hong Kong's constitution, which limits the terms of
Legislative Council members to four years --- meaning that the current
lawmakers' terms will soon expire.

That matter will be referred to the standing committee of the National
People's Congress in Beijing, which has the power to interpret the Basic
Law, for a decision on how to deal with the gap, Xinhua reported.

Image

Opposition candidates took control of 17 out of 18 district councils,
which had normally been controlled by pro-Beijing parties, in elections
in November.Credit...Lam Yik Fei for The New York Times

The postponement will likely be met with criticism from the United
States and other countries that have expressed sharp disapproval of
China's tightening grip on Hong Kong.
\href{https://www.nytimes.com/2020/05/29/us/politics/trump-hong-kong-china-WHO.html}{This
month, President Trump said} that because of the national security law,
the United States would begin to curb its special treatment of Hong Kong
and deal with it more in line with the rest of China.

The elections ``must proceed on time,''
\href{https://www.state.gov/secretary-michael-r-pompeo-with-simon-conway-of-the-joe-pags-show/}{Secretary
of State Mike Pompeo said} on Thursday in a U.S. radio interview. ``They
must be held. The people of Hong Kong deserve to have their voice
represented by the elected officials that they choose in those
elections.''

``If they destroy that, if they take that down, it will be another
marker that will simply prove that the Chinese Communist Party has now
made Hong Kong just another Communist-run city,'' he added.

Wang Wenbin, a spokesman for China's Ministry of Foreign Affairs, said
on Friday that the Hong Kong election was ``a local election in China
and is purely China's internal affair.''

The national security law targets activity that it describes as
secession, subversion, terrorism and collusion with foreign powers. It
has stirred concerns in Hong Kong because it allows mainland security
services to operate openly in the city and makes some speech, such as
advocating Hong Kong's independence, illegal.

\href{https://www.nytimes.com/news-event/coronavirus?action=click\&pgtype=Article\&state=default\&region=MAIN_CONTENT_3\&context=storylines_faq}{}

\hypertarget{the-coronavirus-outbreak-}{%
\subsubsection{The Coronavirus Outbreak
›}\label{the-coronavirus-outbreak-}}

\hypertarget{frequently-asked-questions}{%
\paragraph{Frequently Asked
Questions}\label{frequently-asked-questions}}

Updated July 27, 2020

\begin{itemize}
\item ~
  \hypertarget{should-i-refinance-my-mortgage}{%
  \paragraph{Should I refinance my
  mortgage?}\label{should-i-refinance-my-mortgage}}

  \begin{itemize}
  \tightlist
  \item
    \href{https://www.nytimes.com/article/coronavirus-money-unemployment.html?action=click\&pgtype=Article\&state=default\&region=MAIN_CONTENT_3\&context=storylines_faq}{It
    could be a good idea,} because mortgage rates have
    \href{https://www.nytimes.com/2020/07/16/business/mortgage-rates-below-3-percent.html?action=click\&pgtype=Article\&state=default\&region=MAIN_CONTENT_3\&context=storylines_faq}{never
    been lower.} Refinancing requests have pushed mortgage applications
    to some of the highest levels since 2008, so be prepared to get in
    line. But defaults are also up, so if you're thinking about buying a
    home, be aware that some lenders have tightened their standards.
  \end{itemize}
\item ~
  \hypertarget{what-is-school-going-to-look-like-in-september}{%
  \paragraph{What is school going to look like in
  September?}\label{what-is-school-going-to-look-like-in-september}}

  \begin{itemize}
  \tightlist
  \item
    It is unlikely that many schools will return to a normal schedule
    this fall, requiring the grind of
    \href{https://www.nytimes.com/2020/06/05/us/coronavirus-education-lost-learning.html?action=click\&pgtype=Article\&state=default\&region=MAIN_CONTENT_3\&context=storylines_faq}{online
    learning},
    \href{https://www.nytimes.com/2020/05/29/us/coronavirus-child-care-centers.html?action=click\&pgtype=Article\&state=default\&region=MAIN_CONTENT_3\&context=storylines_faq}{makeshift
    child care} and
    \href{https://www.nytimes.com/2020/06/03/business/economy/coronavirus-working-women.html?action=click\&pgtype=Article\&state=default\&region=MAIN_CONTENT_3\&context=storylines_faq}{stunted
    workdays} to continue. California's two largest public school
    districts --- Los Angeles and San Diego --- said on July 13, that
    \href{https://www.nytimes.com/2020/07/13/us/lausd-san-diego-school-reopening.html?action=click\&pgtype=Article\&state=default\&region=MAIN_CONTENT_3\&context=storylines_faq}{instruction
    will be remote-only in the fall}, citing concerns that surging
    coronavirus infections in their areas pose too dire a risk for
    students and teachers. Together, the two districts enroll some
    825,000 students. They are the largest in the country so far to
    abandon plans for even a partial physical return to classrooms when
    they reopen in August. For other districts, the solution won't be an
    all-or-nothing approach.
    \href{https://bioethics.jhu.edu/research-and-outreach/projects/eschool-initiative/school-policy-tracker/}{Many
    systems}, including the nation's largest, New York City, are
    devising
    \href{https://www.nytimes.com/2020/06/26/us/coronavirus-schools-reopen-fall.html?action=click\&pgtype=Article\&state=default\&region=MAIN_CONTENT_3\&context=storylines_faq}{hybrid
    plans} that involve spending some days in classrooms and other days
    online. There's no national policy on this yet, so check with your
    municipal school system regularly to see what is happening in your
    community.
  \end{itemize}
\item ~
  \hypertarget{is-the-coronavirus-airborne}{%
  \paragraph{Is the coronavirus
  airborne?}\label{is-the-coronavirus-airborne}}

  \begin{itemize}
  \tightlist
  \item
    The coronavirus
    \href{https://www.nytimes.com/2020/07/04/health/239-experts-with-one-big-claim-the-coronavirus-is-airborne.html?action=click\&pgtype=Article\&state=default\&region=MAIN_CONTENT_3\&context=storylines_faq}{can
    stay aloft for hours in tiny droplets in stagnant air}, infecting
    people as they inhale, mounting scientific evidence suggests. This
    risk is highest in crowded indoor spaces with poor ventilation, and
    may help explain super-spreading events reported in meatpacking
    plants, churches and restaurants.
    \href{https://www.nytimes.com/2020/07/06/health/coronavirus-airborne-aerosols.html?action=click\&pgtype=Article\&state=default\&region=MAIN_CONTENT_3\&context=storylines_faq}{It's
    unclear how often the virus is spread} via these tiny droplets, or
    aerosols, compared with larger droplets that are expelled when a
    sick person coughs or sneezes, or transmitted through contact with
    contaminated surfaces, said Linsey Marr, an aerosol expert at
    Virginia Tech. Aerosols are released even when a person without
    symptoms exhales, talks or sings, according to Dr. Marr and more
    than 200 other experts, who
    \href{https://academic.oup.com/cid/article/doi/10.1093/cid/ciaa939/5867798}{have
    outlined the evidence in an open letter to the World Health
    Organization}.
  \end{itemize}
\item ~
  \hypertarget{what-are-the-symptoms-of-coronavirus}{%
  \paragraph{What are the symptoms of
  coronavirus?}\label{what-are-the-symptoms-of-coronavirus}}

  \begin{itemize}
  \tightlist
  \item
    Common symptoms
    \href{https://www.nytimes.com/article/symptoms-coronavirus.html?action=click\&pgtype=Article\&state=default\&region=MAIN_CONTENT_3\&context=storylines_faq}{include
    fever, a dry cough, fatigue and difficulty breathing or shortness of
    breath.} Some of these symptoms overlap with those of the flu,
    making detection difficult, but runny noses and stuffy sinuses are
    less common.
    \href{https://www.nytimes.com/2020/04/27/health/coronavirus-symptoms-cdc.html?action=click\&pgtype=Article\&state=default\&region=MAIN_CONTENT_3\&context=storylines_faq}{The
    C.D.C. has also} added chills, muscle pain, sore throat, headache
    and a new loss of the sense of taste or smell as symptoms to look
    out for. Most people fall ill five to seven days after exposure, but
    symptoms may appear in as few as two days or as many as 14 days.
  \end{itemize}
\item ~
  \hypertarget{does-asymptomatic-transmission-of-covid-19-happen}{%
  \paragraph{Does asymptomatic transmission of Covid-19
  happen?}\label{does-asymptomatic-transmission-of-covid-19-happen}}

  \begin{itemize}
  \tightlist
  \item
    So far, the evidence seems to show it does. A widely cited
    \href{https://www.nature.com/articles/s41591-020-0869-5}{paper}
    published in April suggests that people are most infectious about
    two days before the onset of coronavirus symptoms and estimated that
    44 percent of new infections were a result of transmission from
    people who were not yet showing symptoms. Recently, a top expert at
    the World Health Organization stated that transmission of the
    coronavirus by people who did not have symptoms was ``very rare,''
    \href{https://www.nytimes.com/2020/06/09/world/coronavirus-updates.html?action=click\&pgtype=Article\&state=default\&region=MAIN_CONTENT_3\&context=storylines_faq\#link-1f302e21}{but
    she later walked back that statement.}
  \end{itemize}
\end{itemize}

On Wednesday, in a sign that officials would strictly enforce the
law,\href{https://www.nytimes.com/2020/07/29/world/asia/hong-kong-arrests-security-law.html}{the
police arrested four activists, ages 16 to 21,}who were accused of
supporting separatism in social media posts.

And the next day, in barring the 12 opposition candidates, the Hong Kong
government said that the grounds for disqualifying them included
advocating for Hong Kong's independence or self-determination,
soliciting intervention from foreign governments, expressing an
objection in principle to the national security law Beijing imposed last
month, or vowing to indiscriminately vote against government proposals.

Opposition candidates say the moves suggested that pro-Beijing officials
were concerned about a resounding defeat in the September election. Even
establishment candidates have been quietly discussing the potential for
a pan-democratic wave.

Elections for neighborhood-level offices,
\href{https://www.nytimes.com/2019/11/24/world/asia/hong-kong-election-results.html}{held
last November,} were seen as a warning: The opposition took control of
17 out of 18 district councils, which had normally been controlled by
pro-Beijing parties.

This year, the opposition set its sights on a bigger target: to take at
least half the 70 seats in the Legislative Council, the top lawmaking
body in the territory.

Image

Banners on a barge in Victoria Harbor in Hong Kong welcoming the
national security law imposed on July 1.~Credit...Lam Yik Fei for The
New York Times

While the protests have abated in recent weeks under the authorities'
crackdown, discontent with the government has remained strong since
Beijing imposed the security law on Hong Kong, a semiautonomous city
that maintains its own local government and legal system.

Two weeks ago,
\href{https://www.nytimes.com/2020/07/13/world/asia/hong-kong-elections-security.html}{more
than 600,000 people participated} in the opposition camp's primary
election, despite warnings from local officials that it might be
illegal. Voters generally preferred candidates closely associated with
the past year's protests.

In barring the candidates for the September elections, election
officials questioned whether candidates who had previously lobbied
foreign governments would continue to do so, which could potentially
violate the new security law's prohibitions on foreign influence.
Another question asked was whether candidates, if elected, would veto
the government's budget. Under Hong Kong's system, if the legislature
blocks the budget twice in a row, the chief executive is forced to step
down.

Kwok Ka-ki, a legislator who was one of the 12 candidates disqualified
Thursday, replied that such a question was political in nature, and that
he was unsure why an election official had any business asking it.
``After all, this is why there are elections in the first place,'' he
wrote.

Just half the seats in the legislature represent geographic districts in
Hong Kong, another barrier for the pro-democracy camp. The other half
are functional constituencies largely set aside for candidates from
various commercial sectors, which tend to vote for establishment
candidates.

The opposition has pointed to other places that have held successful
elections during the pandemic, including
\href{https://www.nytimes.com/2020/04/10/world/asia/coronavirus-south-korea-election.html}{South
Korea} and
\href{https://www.nytimes.com/2020/07/10/world/asia/singapore-election-results.html}{Singapore}.

``I don't think many people in Hong Kong will be convinced,'' Mr. Ma
said, referring to the official justification for delaying the election.
``They are allowed to go to work, take the subway, take the bus, stand
in long queues and then not allowed to vote? It won't be very
convincing.''

Elaine Yu and Tiffany May contributed reporting from Hong Kong. Keith
Bradsher contributed reporting, and Claire Fu contributed research, from
Beijing.

Advertisement

\protect\hyperlink{after-bottom}{Continue reading the main story}

\hypertarget{site-index}{%
\subsection{Site Index}\label{site-index}}

\hypertarget{site-information-navigation}{%
\subsection{Site Information
Navigation}\label{site-information-navigation}}

\begin{itemize}
\tightlist
\item
  \href{https://help.nytimes.com/hc/en-us/articles/115014792127-Copyright-notice}{©~2020~The
  New York Times Company}
\end{itemize}

\begin{itemize}
\tightlist
\item
  \href{https://www.nytco.com/}{NYTCo}
\item
  \href{https://help.nytimes.com/hc/en-us/articles/115015385887-Contact-Us}{Contact
  Us}
\item
  \href{https://www.nytco.com/careers/}{Work with us}
\item
  \href{https://nytmediakit.com/}{Advertise}
\item
  \href{http://www.tbrandstudio.com/}{T Brand Studio}
\item
  \href{https://www.nytimes.com/privacy/cookie-policy\#how-do-i-manage-trackers}{Your
  Ad Choices}
\item
  \href{https://www.nytimes.com/privacy}{Privacy}
\item
  \href{https://help.nytimes.com/hc/en-us/articles/115014893428-Terms-of-service}{Terms
  of Service}
\item
  \href{https://help.nytimes.com/hc/en-us/articles/115014893968-Terms-of-sale}{Terms
  of Sale}
\item
  \href{https://spiderbites.nytimes.com}{Site Map}
\item
  \href{https://help.nytimes.com/hc/en-us}{Help}
\item
  \href{https://www.nytimes.com/subscription?campaignId=37WXW}{Subscriptions}
\end{itemize}
