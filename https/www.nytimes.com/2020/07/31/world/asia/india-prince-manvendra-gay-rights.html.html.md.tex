Sections

SEARCH

\protect\hyperlink{site-content}{Skip to
content}\protect\hyperlink{site-index}{Skip to site index}

\href{https://www.nytimes.com/section/world/asia}{Asia Pacific}

\href{https://myaccount.nytimes.com/auth/login?response_type=cookie\&client_id=vi}{}

\href{https://www.nytimes.com/section/todayspaper}{Today's Paper}

\href{/section/world/asia}{Asia Pacific}\textbar{}In India, a Gay
Prince's Coming Out Earns Accolades, and Enemies

\url{https://nyti.ms/3jX3EZN}

\begin{itemize}
\item
\item
\item
\item
\item
\item
\end{itemize}

Advertisement

\protect\hyperlink{after-top}{Continue reading the main story}

Supported by

\protect\hyperlink{after-sponsor}{Continue reading the main story}

The Saturday Profile

\hypertarget{in-india-a-gay-princes-coming-out-earns-accolades-and-enemies}{%
\section{In India, a Gay Prince's Coming Out Earns Accolades, and
Enemies}\label{in-india-a-gay-princes-coming-out-earns-accolades-and-enemies}}

Prince Manvendra's journey from an excruciatingly lonely child to a
global L.G.B.T.Q. advocate included death threats and disinheritance.

\includegraphics{https://static01.nyt.com/images/2020/07/31/world/31profile-prince-manvendra-1/merlin_175102827_1051cde2-1285-4e95-b75f-23450f96203e-articleLarge.jpg?quality=75\&auto=webp\&disable=upscale}

\href{https://www.nytimes.com/by/shalini-venugopal-bhagat}{\includegraphics{https://static01.nyt.com/images/2019/11/26/reader-center/author-shalini-venugopal-bhagat/author-shalini-venugopal-bhagat-thumbLarge.png}}

By \href{https://www.nytimes.com/by/shalini-venugopal-bhagat}{Shalini
Venugopal Bhagat}

\begin{itemize}
\item
  Published July 31, 2020Updated Aug. 1, 2020
\item
  \begin{itemize}
  \item
  \item
  \item
  \item
  \item
  \item
  \end{itemize}
\end{itemize}

NEW DELHI --- Born into a royal family that once ruled the kingdom of
Rajpipla in India, he was raised in the family's palaces and mansions
and was being groomed to take over a dynasty that goes back 600 years.

But then he gave an interview that prompted his mother to disown him and
set off protests in his hometown, where he was burned in effigy.

Since coming out as gay in that 2006 interview, Prince Manvendra Singh
Gohil has faced a torrent of bullying and threats, and was disinherited
by his family for a period.

But he has also earned global accolades for his L.G.B.T.Q. advocacy,
becoming one of the few gay-rights activists in the world with such
royal ties.

As part of his efforts, Prince Manvendra, 55, has
\href{https://www.youtube.com/watch?v=i6a39D0PtPM}{appeared on ``The
Oprah Winfrey Show'' three times}, swapped life stories with Kris Jenner
on ``Keeping Up With the Kardashians'' and is working to establish a
shelter for L.G.B.T.Q. people on his property in the Indian state of
Gujarat. He is also working with several aid agencies to prevent the
spread of H.I.V. among gay men.

Prince Manvendra and his husband, deAndre Richardson, have spent the
last few months in lockdown getting the shelter ready. They envision a
safe space where those who have been disowned by their families can get
back on their feet and learn job skills.

``I know how important it is to have a safe space after coming out,''
the prince said.

Although India abolished the princely order in 1971, the honorary titles
are still commonly used for royal descendants, and traditional
responsibilities are still carried out.

\includegraphics{https://static01.nyt.com/images/2020/07/24/world/24profile-prince-manvendra1/merlin_174892248_a8b77fbc-1825-4403-a66f-c8ce51d5b470-articleLarge.jpg?quality=75\&auto=webp\&disable=upscale}

When the prince shared that he was gay in that front-page newspaper
interview 14 years ago, it created a storm of mostly negative publicity.
It was shocking for a member of an Indian royal family, especially one
from the rigidly conservative Rajput warrior clan that once ruled over
large parts of northern and central India, to come out so publicly.
Being gay was a criminal offense in India under the
\href{https://www.nytimes.com/2018/06/02/world/asia/gay-in-india-where-progress-has-come-only-with-risk.html?searchResultPosition=5}{archaic
British law in effect at the time}. The law was
\href{https://www.nytimes.com/2018/09/06/world/asia/india-gay-sex-377.html}{struck
down} in 2018.

The fallout from his announcement was brutal, beginning with protests in
his hometown, Rajpipla, where he was burned in effigy. His mother took
out a newspaper advertisement to announce she was disowning him.

The government offered him security after he received several death
threats, but he turned down the offer and refused to back down. ``I
decided that I would continue fighting because I have truth on my
side,'' he said.

Image

Prince Manvendra playing the harmonium for his husband, deAndre
Richardson.Credit...Atul Loke for The New York Times

Prince Manvendra was born in 1965 to Raghubir Singh Gohil, the current
honorary maharajah of Rajpipla, and Rukmani Devi Gohil, the daughter of
the former maharajah of Jaisalmer.

By that time, the era of fabulously rich Indian maharajahs had already
waned. His great-grandfather's ostentatious display of wealth, with
\href{https://economictimes.indiatimes.com/erstwhile-royals-rally-to-bring-back-vintage-rolls-royce-sold-overseas/articleshow/20608569.cms?from=mdr}{stables
of racehorses and garages filled with Rolls-Royces}(nearly a dozen), was
no longer welcome in a newly independent India where socialism,
austerity and self-sufficiency were the new mantras.

Although Prince Manvendra's family no longer ruled a kingdom, the old
ways still largely prevailed. He spent most of his childhood in his
family's seven-bedroom mansion in Mumbai, staffed by servants who had
worked for the family for generations. He barely saw his parents and was
raised primarily by the same nanny who had raised his mother.

``Until I was 9 or 10, I thought my nanny was my mother,'' he said. ``I
didn't realize that the glamorous woman who appeared once in a while was
actually my mother.''

The lack of parental love still wounds him. ``Why do parents give birth
to children if they don't want to take care of them?'' he said.

His childhood was excruciatingly lonely. His only friends were the birds
and other animals he rescued as a young child. ``I grew up with
literally no friends, because I knew I couldn't invite anyone home,'' he
said, because he was allowed to socialize only with children from a
similar background.

Image

The prince at 2. He said his only friends as a young child were the
birds and other animals he rescued.

He earned a college degree in commerce and accounting and went on to
complete law school, although he has never practiced law.

In 1991, he married Chandrika Kumari, a princess from the royal family
of Jhabua, a match entered into voluntarily, he emphasized.

``I was attracted to men but I thought it was just a passing phase,'' he
said. ``I had never been allowed to spend time alone with a girl, and
sex before marriage was out of the question.''

Being gay was not a possibility that ever crossed his mind, he said,
because he knew nothing about it.

``Once we got married, it became clear to me that I wasn't interested in
women sexually,'' he said. ``We were very good friends, we got along
very well, but there was no sexual attraction.''

The couple called it quits 15 months later, a split that caused an
uproar in royal circles. After the divorce, he said, he was wracked with
guilt and confused about his sexuality. He moved back to Mumbai, a
26-year-old divorced virgin, and started exploring his sexuality for the
first time.

``I started reading books and magazines. I saw an article about Ashok
Row Kavi and his gay magazine Bombay Dost. I decided to get in touch
with him and ask him if I could possibly be gay,'' he recalled.

Mr. Kavi is a father of India's gay-rights movement. In 1977, he came
out publicly and went on to found Bombay Dost, India's first gay
magazine, in 1990. He founded the Humsafar Trust, the first group to
provide health services and advocacy for gay men, in 1994.

Mr. Kavi introduced Prince Manvendra to other people in the community
and trained him as a counselor. He remembers the young prince as a
painfully shy introvert, who was slowly starting to become comfortable
with his identity. He said the prince quietly funded the first telephone
help line for gay people in India.

In 2000, with Mr. Kavi's encouragement, the prince started the
\href{https://www.lakshyatrust.com/}{Lakshya Trust} in Gujarat to help
the gay community there.

Image

The young prince with his parents, Raghubir Singh Gohil and Rukmani Devi
Gohil, and his sister Minaxi in Rajpipla, India, in 1976.

The work was fulfilling, but as a closeted gay man, the prince said, it
became increasingly difficult to do the advocacy work needed for
Lakshya. And there was growing pressure to remarry.

After he suffered a nervous breakdown in 2002, his psychiatrist
convinced him the first step in his recovery was to come out to his
parents.

It was the beginning of a long and bitter ordeal. ``My parents were in
an absolute state of denial,'' Prince Manvendra said. ``They declared
that science must have a cure for my condition, a surgery perhaps or
shock therapy to cure my `disease.'''

But every doctor his parents consulted told them the same thing ---
homosexuality was not a disease or a mental disorder. His parents
finally gave up on medical science and decided to try religion instead.
For three years, they took him to dozens of religious leaders around the
country.

``Ashok told me to cooperate with them completely,'' the prince said.
``To let them be satisfied that they'd tried their best.''

There were financial consequences to his coming out. He says that he was
removed from several family businesses and that his mother threatened to
persuade the government to cancel funding for the Lakshya Trust.

``I finally reached a point in my life where I couldn't take it
anymore,'' he said. ``I decided to tell the whole world.''

Image

The couple at home.Credit...Atul Loke for The New York Times

Over the past 14 years, the once-shy royal has grown accustomed to the
spotlight and become a vocal activist for the gay rights movement. Apart
from his work with the Lakshya Trust, he is a founding member of the
Asia Pacific Coalition on Male Sexual Health and is an ambassador
consultant of the AIDS Healthcare Foundation.

``He was living a very troubled life, under a lot of pressure,'' said
Chirantana Bhatt, a close friend. ``But now it's a life of pride, in the
true sense.''

He has also found love. In 2013, he married Mr. Richardson, an American
he met online in 2009, in the United States. The couple live on an
estate in Gujarat given to the prince by his father. His modest brick
house there is a far cry from the opulent palace of his ancestors, but
he says he could not be happier.

His father, the maharajah, acknowledged in an interview that it was
difficult for the family to come to terms with his son's sexuality and
the constant media attention on the family.

``But it's his decision,'' the maharajah said.

His relationship with his mother remains frosty, but other members of
the family have been supportive, he says. His grandmother, on her
deathbed, expressed her happiness that he had found a partner to share
his life with.

Prince Manvendra is cautiously optimistic about the future. He is not
sure if he will become the next honorary maharajah of Rajpipla. ``I have
left it to my family members,'' he said. ``I would prefer to keep
working for my cause because the role of maharajah comes with a lot of
responsibilities and duties that would divert me from my activism.''

Image

Prince Manvendra in front of the remains of his royal palace on the
banks of Narmada in Gujarat.Credit...Atul Loke for The New York Times

Advertisement

\protect\hyperlink{after-bottom}{Continue reading the main story}

\hypertarget{site-index}{%
\subsection{Site Index}\label{site-index}}

\hypertarget{site-information-navigation}{%
\subsection{Site Information
Navigation}\label{site-information-navigation}}

\begin{itemize}
\tightlist
\item
  \href{https://help.nytimes.com/hc/en-us/articles/115014792127-Copyright-notice}{©~2020~The
  New York Times Company}
\end{itemize}

\begin{itemize}
\tightlist
\item
  \href{https://www.nytco.com/}{NYTCo}
\item
  \href{https://help.nytimes.com/hc/en-us/articles/115015385887-Contact-Us}{Contact
  Us}
\item
  \href{https://www.nytco.com/careers/}{Work with us}
\item
  \href{https://nytmediakit.com/}{Advertise}
\item
  \href{http://www.tbrandstudio.com/}{T Brand Studio}
\item
  \href{https://www.nytimes.com/privacy/cookie-policy\#how-do-i-manage-trackers}{Your
  Ad Choices}
\item
  \href{https://www.nytimes.com/privacy}{Privacy}
\item
  \href{https://help.nytimes.com/hc/en-us/articles/115014893428-Terms-of-service}{Terms
  of Service}
\item
  \href{https://help.nytimes.com/hc/en-us/articles/115014893968-Terms-of-sale}{Terms
  of Sale}
\item
  \href{https://spiderbites.nytimes.com}{Site Map}
\item
  \href{https://help.nytimes.com/hc/en-us}{Help}
\item
  \href{https://www.nytimes.com/subscription?campaignId=37WXW}{Subscriptions}
\end{itemize}
