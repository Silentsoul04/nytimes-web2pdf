Sections

SEARCH

\protect\hyperlink{site-content}{Skip to
content}\protect\hyperlink{site-index}{Skip to site index}

\href{https://www.nytimes.com/section/world/asia}{Asia Pacific}

\href{https://myaccount.nytimes.com/auth/login?response_type=cookie\&client_id=vi}{}

\href{https://www.nytimes.com/section/todayspaper}{Today's Paper}

\href{/section/world/asia}{Asia Pacific}\textbar{}In Indonesia, False
Virus Cures Pushed by Those Who Should Know Better

\url{https://nyti.ms/30dNOlt}

\begin{itemize}
\item
\item
\item
\item
\item
\item
\end{itemize}

\href{https://www.nytimes.com/news-event/coronavirus?action=click\&pgtype=Article\&state=default\&region=TOP_BANNER\&context=storylines_menu}{The
Coronavirus Outbreak}

\begin{itemize}
\tightlist
\item
  live\href{https://www.nytimes.com/2020/08/01/world/coronavirus-covid-19.html?action=click\&pgtype=Article\&state=default\&region=TOP_BANNER\&context=storylines_menu}{Latest
  Updates}
\item
  \href{https://www.nytimes.com/interactive/2020/us/coronavirus-us-cases.html?action=click\&pgtype=Article\&state=default\&region=TOP_BANNER\&context=storylines_menu}{Maps
  and Cases}
\item
  \href{https://www.nytimes.com/interactive/2020/science/coronavirus-vaccine-tracker.html?action=click\&pgtype=Article\&state=default\&region=TOP_BANNER\&context=storylines_menu}{Vaccine
  Tracker}
\item
  \href{https://www.nytimes.com/interactive/2020/07/29/us/schools-reopening-coronavirus.html?action=click\&pgtype=Article\&state=default\&region=TOP_BANNER\&context=storylines_menu}{What
  School May Look Like}
\item
  \href{https://www.nytimes.com/live/2020/07/31/business/stock-market-today-coronavirus?action=click\&pgtype=Article\&state=default\&region=TOP_BANNER\&context=storylines_menu}{Economy}
\end{itemize}

Advertisement

\protect\hyperlink{after-top}{Continue reading the main story}

Supported by

\protect\hyperlink{after-sponsor}{Continue reading the main story}

\hypertarget{in-indonesia-false-virus-cures-pushed-by-those-who-should-know-better}{%
\section{In Indonesia, False Virus Cures Pushed by Those Who Should Know
Better}\label{in-indonesia-false-virus-cures-pushed-by-those-who-should-know-better}}

In the absence of a unified message from the national government, local
officials and opportunists have filled the gap with misinformation and
quack remedies.

\includegraphics{https://static01.nyt.com/images/2020/07/31/world/31virus-indonesia-1/merlin_175139715_c58ed554-ab30-4f39-a5cb-94e4e0ab8d8f-articleLarge.jpg?quality=75\&auto=webp\&disable=upscale}

\href{https://www.nytimes.com/by/richard-c-paddock}{\includegraphics{https://static01.nyt.com/images/2018/10/15/multimedia/author-richard-c-paddock/author-richard-c-paddock-thumbLarge.png}}

By \href{https://www.nytimes.com/by/richard-c-paddock}{Richard C.
Paddock}

\begin{itemize}
\item
  July 31, 2020
\item
  \begin{itemize}
  \item
  \item
  \item
  \item
  \item
  \item
  \end{itemize}
\end{itemize}

First, Indonesia's agriculture minister promoted wearing a necklace
containing a eucalyptus potion to cure the coronavirus. Not to be
outdone, the governor of Bali, a popular resort island, pushed his own
remedy: inhaling the steam from boiled arak, a traditional alcohol made
from coconuts.

So-called influencers and self-styled experts have also pushed their own
quack cures and misinformation on Indonesian social media, including a
widely spread rumor that popular infrared thermometer guns cause brain
damage.

As Indonesia steadily loses ground to the pandemic, the government has
had difficulty delivering a consistent, science-based message about the
coronavirus and the disease it causes, Covid-19.

As of Friday, Indonesia had reported more than 108,000 cases and more
than 5,130 deaths, surpassing China in both categories.

\includegraphics{https://static01.nyt.com/images/2020/07/31/world/31virus-indonesia-2/merlin_174790395_67c93f83-8eda-41eb-8893-a37c0d5ad8ea-articleLarge.jpg?quality=75\&auto=webp\&disable=upscale}

Yet even in hard-hit provinces, as many as 70 percent of people go
without masks and ignore social distancing requirements, according to
the government, often crowding into shops and markets and hanging out at
busy cafes and restaurants.

Indonesia is not the only country battling misinformation or whose
leaders have promoted quack remedies. The World Health Organization has
called the ubiquity of hazardous false information an
\href{https://www.nytimes.com/2020/02/06/health/coronavirus-misinformation-social-media.html}{``infodemic.''}

In Kenya, the
\href{https://www.nytimes.com/2020/04/17/world/coronavirus-news-updates.html\#link-5699a22e}{governor
of Nairobi} has pushed cognac as a miracle cure. President Trump has
continued to promote hydroxychloroquine, a drug used for treating
malaria, as a coronavirus remedy despite medical evidence to the
contrary. He has even suggested that an ``injection inside'' the human
body with a disinfectant like bleach could help combat the virus.

But Indonesia is unique because of its large population, expansive
geography across thousands of islands and mix of cultural identities. It
would be difficult enough for the government to implement a clear and
unified plan for combating the virus, but matters have been made worse
by the promotion of muddled and often dangerous information.

The country's president, Joko Widodo, had initially downplayed the
pandemic and has delivered mixed messages. He admitted in March that he
had
\href{https://www.nytimes.com/2020/03/17/world/asia/coronavirus-southeast-asia.html}{misled
the public} about the virus to prevent a panic. After that, he was slow
to close businesses and schools and to limit travel, but was quick to
lift restrictions even as cases continued to rise.

\hypertarget{latest-updates-global-coronavirus-outbreak}{%
\section{\texorpdfstring{\href{https://www.nytimes.com/2020/08/01/world/coronavirus-covid-19.html?action=click\&pgtype=Article\&state=default\&region=MAIN_CONTENT_1\&context=storylines_live_updates}{Latest
Updates: Global Coronavirus
Outbreak}}{Latest Updates: Global Coronavirus Outbreak}}\label{latest-updates-global-coronavirus-outbreak}}

Updated 2020-08-02T07:14:05.841Z

\begin{itemize}
\tightlist
\item
  \href{https://www.nytimes.com/2020/08/01/world/coronavirus-covid-19.html?action=click\&pgtype=Article\&state=default\&region=MAIN_CONTENT_1\&context=storylines_live_updates\#link-34047410}{The
  U.S. reels as July cases more than double the total of any other
  month.}
\item
  \href{https://www.nytimes.com/2020/08/01/world/coronavirus-covid-19.html?action=click\&pgtype=Article\&state=default\&region=MAIN_CONTENT_1\&context=storylines_live_updates\#link-780ec966}{Top
  U.S. officials work to break an impasse over the federal jobless
  benefit.}
\item
  \href{https://www.nytimes.com/2020/08/01/world/coronavirus-covid-19.html?action=click\&pgtype=Article\&state=default\&region=MAIN_CONTENT_1\&context=storylines_live_updates\#link-2bc8948}{Its
  outbreak untamed, Melbourne goes into even greater lockdown.}
\end{itemize}

\href{https://www.nytimes.com/2020/08/01/world/coronavirus-covid-19.html?action=click\&pgtype=Article\&state=default\&region=MAIN_CONTENT_1\&context=storylines_live_updates}{See
more updates}

More live coverage:
\href{https://www.nytimes.com/live/2020/07/31/business/stock-market-today-coronavirus?action=click\&pgtype=Article\&state=default\&region=MAIN_CONTENT_1\&context=storylines_live_updates}{Markets}

In May, he said Indonesia should learn to live with the virus. A month
later, however, he threatened to fire cabinet ministers for not doing
more to bring the pandemic under control.

This month, he called for a national campaign to promote better
discipline in social distancing, mask wearing and hand washing.

Image

Police officers making hand sanitizer from palm alcohol, known locally
as arak, in Bali, Indonesia. One official has promoted inhaling the
steam from boiled arak as a cure for the virus.~Credit...~Bali
Police,via~Agence France-Presse --- Getty Images

In the absence of a unified message from the national government, local
officials and opportunists have filled the gap.

One official who has promoted a questionable remedy is the agriculture
minister, Syahrul Yasin Limpo. He told reporters this month that a
ministry laboratory had developed a potion made from eucalyptus that
when worn on a necklace could kill 80 percent of virus particles in half
an hour.

``From 700 species of eucalyptus, our lab test results showed that one
kind could kill the corona,'' he said. ``We are certain. We will produce
it next month.''

His claim was quickly contradicted by health experts, including the head
of the laboratory that developed the aromatic potion, who said it was
not effective against the coronavirus. But that didn't stop others from
promoting it.

A popular singer, Iis Dahlia, met with Mr. Joko as he sought to recruit
celebrities to help in his health campaign. Soon after, she informed her
12 million Instagram followers that she was proud to be wearing the
amulet.

Image

The country's president, Joko Widodo, has downplayed the pandemic and
delivered mixed messages.Credit...Pool photo by Sigid Kurniawan/EPA, via
Shutterstock

``This eucalyptus necklace,'' she said, ``makes me feel safe and
protected from the virus.''

In Bali, the governor, I Wayan Koster, has promoted a local treatment:
inhaling the steam of boiled arak, a traditional alcoholic beverage. As
if to stay on trend, he too recommends adding a dash of eucalyptus oil.

The governor, who has a Ph.D. in education and described himself as a
former ``researcher,'' told a news conference last week that nearly 80
percent of those who inhaled the concoction tested negative sooner than
would have been expected.

The treatment has not been subjected to scientific testing, but he said
he hoped that Bali could patent and produce it.

The government's top coronavirus spokesman, Wiku Adisasmito, urged the
public to follow health guidelines and not to rely on superstition and
half-baked treatments, even when they emanate from public officials and
celebrities.

``At times of emergency, we all need honest, scientifically based, real
facts to bring us hope, calm and clarity,'' said Mr. Adisasmito, a
University of Indonesia health policy professor.

Jusuf Kalla, a former vice president who now heads the Indonesian Red
Cross, said the country got off to a slow start in fighting the pandemic
in part because the health minister, Terawan Agus Putranto, minimized
its severity.

\href{https://www.nytimes.com/news-event/coronavirus?action=click\&pgtype=Article\&state=default\&region=MAIN_CONTENT_3\&context=storylines_faq}{}

\hypertarget{the-coronavirus-outbreak-}{%
\subsubsection{The Coronavirus Outbreak
›}\label{the-coronavirus-outbreak-}}

\hypertarget{frequently-asked-questions}{%
\paragraph{Frequently Asked
Questions}\label{frequently-asked-questions}}

Updated July 27, 2020

\begin{itemize}
\item ~
  \hypertarget{should-i-refinance-my-mortgage}{%
  \paragraph{Should I refinance my
  mortgage?}\label{should-i-refinance-my-mortgage}}

  \begin{itemize}
  \tightlist
  \item
    \href{https://www.nytimes.com/article/coronavirus-money-unemployment.html?action=click\&pgtype=Article\&state=default\&region=MAIN_CONTENT_3\&context=storylines_faq}{It
    could be a good idea,} because mortgage rates have
    \href{https://www.nytimes.com/2020/07/16/business/mortgage-rates-below-3-percent.html?action=click\&pgtype=Article\&state=default\&region=MAIN_CONTENT_3\&context=storylines_faq}{never
    been lower.} Refinancing requests have pushed mortgage applications
    to some of the highest levels since 2008, so be prepared to get in
    line. But defaults are also up, so if you're thinking about buying a
    home, be aware that some lenders have tightened their standards.
  \end{itemize}
\item ~
  \hypertarget{what-is-school-going-to-look-like-in-september}{%
  \paragraph{What is school going to look like in
  September?}\label{what-is-school-going-to-look-like-in-september}}

  \begin{itemize}
  \tightlist
  \item
    It is unlikely that many schools will return to a normal schedule
    this fall, requiring the grind of
    \href{https://www.nytimes.com/2020/06/05/us/coronavirus-education-lost-learning.html?action=click\&pgtype=Article\&state=default\&region=MAIN_CONTENT_3\&context=storylines_faq}{online
    learning},
    \href{https://www.nytimes.com/2020/05/29/us/coronavirus-child-care-centers.html?action=click\&pgtype=Article\&state=default\&region=MAIN_CONTENT_3\&context=storylines_faq}{makeshift
    child care} and
    \href{https://www.nytimes.com/2020/06/03/business/economy/coronavirus-working-women.html?action=click\&pgtype=Article\&state=default\&region=MAIN_CONTENT_3\&context=storylines_faq}{stunted
    workdays} to continue. California's two largest public school
    districts --- Los Angeles and San Diego --- said on July 13, that
    \href{https://www.nytimes.com/2020/07/13/us/lausd-san-diego-school-reopening.html?action=click\&pgtype=Article\&state=default\&region=MAIN_CONTENT_3\&context=storylines_faq}{instruction
    will be remote-only in the fall}, citing concerns that surging
    coronavirus infections in their areas pose too dire a risk for
    students and teachers. Together, the two districts enroll some
    825,000 students. They are the largest in the country so far to
    abandon plans for even a partial physical return to classrooms when
    they reopen in August. For other districts, the solution won't be an
    all-or-nothing approach.
    \href{https://bioethics.jhu.edu/research-and-outreach/projects/eschool-initiative/school-policy-tracker/}{Many
    systems}, including the nation's largest, New York City, are
    devising
    \href{https://www.nytimes.com/2020/06/26/us/coronavirus-schools-reopen-fall.html?action=click\&pgtype=Article\&state=default\&region=MAIN_CONTENT_3\&context=storylines_faq}{hybrid
    plans} that involve spending some days in classrooms and other days
    online. There's no national policy on this yet, so check with your
    municipal school system regularly to see what is happening in your
    community.
  \end{itemize}
\item ~
  \hypertarget{is-the-coronavirus-airborne}{%
  \paragraph{Is the coronavirus
  airborne?}\label{is-the-coronavirus-airborne}}

  \begin{itemize}
  \tightlist
  \item
    The coronavirus
    \href{https://www.nytimes.com/2020/07/04/health/239-experts-with-one-big-claim-the-coronavirus-is-airborne.html?action=click\&pgtype=Article\&state=default\&region=MAIN_CONTENT_3\&context=storylines_faq}{can
    stay aloft for hours in tiny droplets in stagnant air}, infecting
    people as they inhale, mounting scientific evidence suggests. This
    risk is highest in crowded indoor spaces with poor ventilation, and
    may help explain super-spreading events reported in meatpacking
    plants, churches and restaurants.
    \href{https://www.nytimes.com/2020/07/06/health/coronavirus-airborne-aerosols.html?action=click\&pgtype=Article\&state=default\&region=MAIN_CONTENT_3\&context=storylines_faq}{It's
    unclear how often the virus is spread} via these tiny droplets, or
    aerosols, compared with larger droplets that are expelled when a
    sick person coughs or sneezes, or transmitted through contact with
    contaminated surfaces, said Linsey Marr, an aerosol expert at
    Virginia Tech. Aerosols are released even when a person without
    symptoms exhales, talks or sings, according to Dr. Marr and more
    than 200 other experts, who
    \href{https://academic.oup.com/cid/article/doi/10.1093/cid/ciaa939/5867798}{have
    outlined the evidence in an open letter to the World Health
    Organization}.
  \end{itemize}
\item ~
  \hypertarget{what-are-the-symptoms-of-coronavirus}{%
  \paragraph{What are the symptoms of
  coronavirus?}\label{what-are-the-symptoms-of-coronavirus}}

  \begin{itemize}
  \tightlist
  \item
    Common symptoms
    \href{https://www.nytimes.com/article/symptoms-coronavirus.html?action=click\&pgtype=Article\&state=default\&region=MAIN_CONTENT_3\&context=storylines_faq}{include
    fever, a dry cough, fatigue and difficulty breathing or shortness of
    breath.} Some of these symptoms overlap with those of the flu,
    making detection difficult, but runny noses and stuffy sinuses are
    less common.
    \href{https://www.nytimes.com/2020/04/27/health/coronavirus-symptoms-cdc.html?action=click\&pgtype=Article\&state=default\&region=MAIN_CONTENT_3\&context=storylines_faq}{The
    C.D.C. has also} added chills, muscle pain, sore throat, headache
    and a new loss of the sense of taste or smell as symptoms to look
    out for. Most people fall ill five to seven days after exposure, but
    symptoms may appear in as few as two days or as many as 14 days.
  \end{itemize}
\item ~
  \hypertarget{does-asymptomatic-transmission-of-covid-19-happen}{%
  \paragraph{Does asymptomatic transmission of Covid-19
  happen?}\label{does-asymptomatic-transmission-of-covid-19-happen}}

  \begin{itemize}
  \tightlist
  \item
    So far, the evidence seems to show it does. A widely cited
    \href{https://www.nature.com/articles/s41591-020-0869-5}{paper}
    published in April suggests that people are most infectious about
    two days before the onset of coronavirus symptoms and estimated that
    44 percent of new infections were a result of transmission from
    people who were not yet showing symptoms. Recently, a top expert at
    the World Health Organization stated that transmission of the
    coronavirus by people who did not have symptoms was ``very rare,''
    \href{https://www.nytimes.com/2020/06/09/world/coronavirus-updates.html?action=click\&pgtype=Article\&state=default\&region=MAIN_CONTENT_3\&context=storylines_faq\#link-1f302e21}{but
    she later walked back that statement.}
  \end{itemize}
\end{itemize}

``Until March, Minister Terawan was like Trump, saying, `Oh, this is
only a simple flu,''' Mr. Kalla said. ``But now, Minister Terawan is
very realistic. Ministers and governors are trying to come up with
solutions in an uncertain situation. It is trial and error.''

Indonesia is the world's largest Muslim-majority country, and some
citizens and officials have leaned on their faith to promote cures and
guide their understanding of the disease.

On Lombok Island, a top official suggested that niqabs, loose Islamic
veils worn by women, were as effective in preventing the spread of the
virus as snug-fitting medical face masks.

Image

An official suggested that niqabs, loose Islamic veils worn by Muslim
women, were as effective in preventing the spread of the virus as
snug-fitting medical face masks.Credit...Moh El Sasaky/Agence
France-Presse --- Getty Images

``The advantage of the niqab is more ease in breathing,'' Suhaili Fadhil
Thohir, the regent of Central Lombok, explained in an interview.

Nevertheless, the Covid task force for the province, West Nusa Tenggara,
continues to call for face masks, said Artanto, a police spokesman and
task force member.

``The regent still wears a mask, not a niqab,'' said Mr. Artanto, who
like many Indonesians uses one name. ``We keep educating people to wear
a mask.''

For many Muslims, the Covid-19 burial protocol of wrapping the body
tightly in plastic and burying it in a designated cemetery has been
difficult to accept. By tradition, Muslim family members wash the body
of the deceased and wrap it in cloth for burial.

Image

The authorities say there have been many cases around the country of
families rejecting doctors' warnings and taking Covid-positive bodies
home for burial.Credit...Juni Kriswanto/Agence France-Presse --- Getty
Images

The authorities say there have been many cases around the country of
families rejecting doctors' warnings and taking Covid-positive bodies
home for burial.

In Mataram, Lombok's main city, relatives of a woman who died in a
motorcycle accident this month refused to believe doctors who said she
had tested positive.

About 100 men stormed into Mataram's government hospital to claim the
body. Officers tried to explain the importance of the burial protocols.
But they were badly outnumbered, and the men took the body, put it in a
taxi and drove away.

``It happens all over Indonesia,'' Mr. Artanto said. ``Their
understanding as people who live in the village is different from those
of us who live in the city.''

Mr. Adisasmito said that Islamic burial traditions were deeply
ingrained, and that it was hard for people to accept that they must
change. He likened it to Americans who refuse to wear a mask because it
obstructs their ``pre-pandemic liberty, habits and way of life.''

``We live in a diverse globe,'' he said, ``and different communities
have distinctive values that they hold on to.''

Muktita Suhartono and Dera Menra Sijabat contributed reporting.

Advertisement

\protect\hyperlink{after-bottom}{Continue reading the main story}

\hypertarget{site-index}{%
\subsection{Site Index}\label{site-index}}

\hypertarget{site-information-navigation}{%
\subsection{Site Information
Navigation}\label{site-information-navigation}}

\begin{itemize}
\tightlist
\item
  \href{https://help.nytimes.com/hc/en-us/articles/115014792127-Copyright-notice}{©~2020~The
  New York Times Company}
\end{itemize}

\begin{itemize}
\tightlist
\item
  \href{https://www.nytco.com/}{NYTCo}
\item
  \href{https://help.nytimes.com/hc/en-us/articles/115015385887-Contact-Us}{Contact
  Us}
\item
  \href{https://www.nytco.com/careers/}{Work with us}
\item
  \href{https://nytmediakit.com/}{Advertise}
\item
  \href{http://www.tbrandstudio.com/}{T Brand Studio}
\item
  \href{https://www.nytimes.com/privacy/cookie-policy\#how-do-i-manage-trackers}{Your
  Ad Choices}
\item
  \href{https://www.nytimes.com/privacy}{Privacy}
\item
  \href{https://help.nytimes.com/hc/en-us/articles/115014893428-Terms-of-service}{Terms
  of Service}
\item
  \href{https://help.nytimes.com/hc/en-us/articles/115014893968-Terms-of-sale}{Terms
  of Sale}
\item
  \href{https://spiderbites.nytimes.com}{Site Map}
\item
  \href{https://help.nytimes.com/hc/en-us}{Help}
\item
  \href{https://www.nytimes.com/subscription?campaignId=37WXW}{Subscriptions}
\end{itemize}
