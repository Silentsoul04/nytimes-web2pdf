Sections

SEARCH

\protect\hyperlink{site-content}{Skip to
content}\protect\hyperlink{site-index}{Skip to site index}

\href{https://www.nytimes.com/section/world/canada}{Canada}

\href{https://myaccount.nytimes.com/auth/login?response_type=cookie\&client_id=vi}{}

\href{https://www.nytimes.com/section/todayspaper}{Today's Paper}

\href{/section/world/canada}{Canada}\textbar{}Canada's Key Role in
Creating a Once Awaited Vaccine

\url{https://nyti.ms/3fciVlS}

\begin{itemize}
\item
\item
\item
\item
\item
\end{itemize}

Advertisement

\protect\hyperlink{after-top}{Continue reading the main story}

Supported by

\protect\hyperlink{after-sponsor}{Continue reading the main story}

CANADA LeTTER

\hypertarget{canadas-key-role-in-creating-a-once-awaited-vaccine}{%
\section{Canada's Key Role in Creating a Once Awaited
Vaccine}\label{canadas-key-role-in-creating-a-once-awaited-vaccine}}

An American researcher created the polio vaccine, but a Toronto lab and
a pioneering female scientist made its mass production possible.

\href{https://www.nytimes.com/by/ian-austen}{\includegraphics{https://static01.nyt.com/images/2019/07/18/reader-center/author-ian-austen/author-ian-austen-thumbLarge.png}}

By \href{https://www.nytimes.com/by/ian-austen}{Ian Austen}

\begin{itemize}
\item
  July 31, 2020
\item
  \begin{itemize}
  \item
  \item
  \item
  \item
  \item
  \end{itemize}
\end{itemize}

Canadians don't have to go back to 1918 and the start of the Spanish flu
pandemic to find an analogy to today. For decades, waves of polio
outbreaks gripped the country with fear, death and uncertainty, as
recently as the 1950s.

\includegraphics{https://static01.nyt.com/images/2020/07/31/world/31canadaletter/31canadaletter-articleLarge.jpg?quality=75\&auto=webp\&disable=upscale}

At times, the outbreaks caused Canada to limit travel from the United
States. Special hospitals were set up in some provinces to help children
paralyzed by polio when physiotherapy was established. The iron lung
began appearing in hospitals to assist patients' breathing. School
openings were delayed in many communities in a bid to reduce polio's
spread.

Ultimately, about 50,000 Canadian children were infected with polio
during four major epidemics, and 4,000 of them died.

During the 1940s and '50s, an era before publicly funded health care,
the federal government and many provinces began pouring money and
resources into efforts to eliminate polio through a vaccine.

\href{https://www.sciencehistory.org/historical-profile/jonas-salk-and-albert-bruce-sabin}{Dr.
Jonas Salk, an American, became a global celebrity} for developing that
vaccine. But much less well known is the critical role the
\href{https://connaught.research.utoronto.ca/history/}{Connaught
Laboratories} in Toronto and Dr. Leone N. Farrell, one of its
researchers, played in making testing and then mass production of that
vaccine a reality.

``We're kind of at a similar stage with the Covid vaccines as the one
where she came involved with polio,'' Christopher Rutty, a medical
historian who is an adjunct professor at the University of Toronto's
Dalla Lana School of Public Health, told me this week. ``A vaccine may
work at a small scale, but upscaling is a major, different challenge.''

Dr. Salk's breakthrough was taking the live polio vaccine and killing it
with formaldehyde. When injected into people, the killed virus still
produced an antibody reaction that provided immunity.

The polio virus was grown in cultures of monkey kidney cells. But Dr.
Salk was initially only able to create a few grams of virus at a time in
test tubes. At the time, the substance of choice for growing viruses or
bacteria, Dr. Rutty said, was meat, a method that could lead to allergic
reactions in patients who received the vaccines.

Connaught, however, had come up with a synthetic, liquid growth mixture,
known as Medium 199, for cancer cell research that produced more virus,
more quickly and without contamination. It was provided to Dr. Salk for
his polio efforts.

Image

Dr. Farrell was a trailblazing Canadian research chemist. During much of
her career, she was one of a few women in the field.Credit...Sanofi
Pasteur Canada Archives

It was Dr. Farrell, one of a very small number of women then working as
research chemists in Canada, who figured out how to safely produce vast
quantities of virus in Medium 199. Adapting earlier work, she developed
what came to be known as the Toronto Method. Racks of specially designed
machines gently rocked bottles of Method 199 and the virus.

Her next task was to get enough machines built and to hire enough
qualified staff to make not only enough virus for the tests in the
United States, Canada and Finland, but also to create enough vaccine to
inoculate all of Canada's children. In a bid to accelerate vaccination,
the Canadian government gambled and placed an order with Connaught
before knowing if the Salk vaccine would prove safe and effective in
tests.

It did, with the result made public on April 12, 1955, the day before
Dr. Farrell's birthday. ``I could not help feeling that I had received a
pretty fine present,'' she said in a speech that fall.

Variations of the Toronto Method were used until the 1970s to make polio
vaccines, Dr. Rutty told me. Apparently, at Dr. Farrell's request,
Connaught decided not to patent the process.

Dr. Rutty, who is the expert when it comes to Canada's role in polio
research and who serves as the historian for Connaught's successor
company, Sanofi Pasteur Canada, said that frustratingly little is known
about Dr. Farrell's personal life. She never married, as was the case
with many other women in Canadian medical research, nor had children.

In 1941, when Dr. Farrell was inquiring about a post in naval
intelligence, she seemed to try to head off any potential sexism
byportraying herself as someone who could become one of the guys. ``My
intention has always been `to be a lady chemist --- and not look like
it,''' she wrote in the letter.

She added: ``So conscious am I of my environment and keenly aware of
people in all their phases as persons that I have been charged with
being a chameleon.''

Before retiring in 1969, Dr. Farrell took on several other major
projects including one that greatly increased penicillin production.

But she received relatively little public recognition in her lifetime
and was buried in an unmarked grave after her death in 1986.

In 2009, Dr. Farrell's name and a tribute to her work were added to a
family tombstone.

Dr. Rutty said that he hopes to do more research about her life.

``Farrell is a unique person,'' he said. ``Without her, there really
wouldn't have been a vaccine, at least not then.''

\begin{center}\rule{0.5\linewidth}{\linethickness}\end{center}

\hypertarget{trans-canada}{%
\subsection{Trans Canada}\label{trans-canada}}

Image

Prime Minister Justin Trudeau arriving at his office before testifying
on Thursday.Credit...Dave Chan/Agence France-Presse --- Getty Images

\begin{itemize}
\item
  There was no corruption, just a government working to save lives
  during a pandemic. That's
  \href{https://www.nytimes.com/2020/07/30/world/canada/justin-trudeau-we-charity.html}{the
  message Prime Minister Justin Trudeau delivered} on Thursday to
  skeptical Parliamentarians, and to Canadians, about the government's
  decision to award a hefty no-bid contract to a charity with ties with
  his family, Catherine Porter reports.
\item
  It's known as sled head. Matthew Futterman has investigated how skull
  rattling rides and high speed crashes may be
  \href{https://www.nytimes.com/2020/07/26/sports/olympics/olympics-bobsled-suicide-brain-injuries.html}{linked
  to a shocking wave of suicides} and suicide attempts by sliding sport
  athletes in Canada and in the United States.
\item
  My counterparts at the Climate Fwd: newsletter have
  \href{https://www.nytimes.com/2020/07/29/climate/skating-hockey-climate-change.html}{a
  grim forecast for outdoor hockey} in Canada: ``Hockey could become a
  sport for the privileged few.''
\item
  Tobias Carroll, a book reviewer for The Times, found that
  \href{https://www.nytimes.com/2020/07/28/books/review/empire-of-wild-cherie-dimaline.html}{Cherie
  Dimaline's new novel, ``Empire of Wild,''} ``turns an old story into
  something newly haunting and resonant.'' In it, the Vancouver-based
  writer, who is a member of Ontario's Georgian Bay Métis Community,
  tells the story of a Métis woman who is grappling with the loss of her
  husband. She doesn't know whether he's dead or has simply left town
  after a heated argument between them.
\item
  Eddie Shack,who was a
  \href{https://www.nytimes.com/2020/07/27/sports/hockey/eddie-shack-feisty-wing-for-powerful-maple-leafs-dies-at-83.html}{fan
  favorite at Maple Leaf Gardens} during the Leafs' now-distant glory
  days and a leading villain in Quebec, has died at the age of 83.
\item
  The International Real Estate column took a
  \href{https://www.nytimes.com/2020/07/29/realestate/house-hunting-in-nova-scotia-a-sprawling-seaside-villa-for-2-million.html}{tour
  of a villa on Nova Scotia's Mahone Bay.}
\end{itemize}

\begin{center}\rule{0.5\linewidth}{\linethickness}\end{center}

\emph{A native of Windsor, Ontario, Ian Austen was educated in Toronto,
lives in Ottawa and has reported about Canada for The New York Times for
the past 16 years. Follow him on Twitter at @ianrausten.}

\begin{center}\rule{0.5\linewidth}{\linethickness}\end{center}

\hypertarget{how-are-we-doing}{%
\subsubsection{\texorpdfstring{\textbf{How are we
doing?}}{How are we doing?}}\label{how-are-we-doing}}

We're eager to have your thoughts about this newsletter and events in
Canada in general. Please send them to
\href{mailto:nytcanada@nytimes.com?\%20subject=Canada\%20Letter\%20Newsletter\%20Feedback}{nytcanada@nytimes.com}.

\hypertarget{like-this-email}{%
\subsubsection{\texorpdfstring{\textbf{Like this
email?}}{Like this email?}}\label{like-this-email}}

Forward it to your friends, and let them know they can sign up
\href{https://www.nytimes.com/newsletters/canada-letter?smid=nytemail\&smvar=canadaletter\&te=1\&nl=canada-today\&emc=edit_cnda_20190622}{here}.

Advertisement

\protect\hyperlink{after-bottom}{Continue reading the main story}

\hypertarget{site-index}{%
\subsection{Site Index}\label{site-index}}

\hypertarget{site-information-navigation}{%
\subsection{Site Information
Navigation}\label{site-information-navigation}}

\begin{itemize}
\tightlist
\item
  \href{https://help.nytimes.com/hc/en-us/articles/115014792127-Copyright-notice}{©~2020~The
  New York Times Company}
\end{itemize}

\begin{itemize}
\tightlist
\item
  \href{https://www.nytco.com/}{NYTCo}
\item
  \href{https://help.nytimes.com/hc/en-us/articles/115015385887-Contact-Us}{Contact
  Us}
\item
  \href{https://www.nytco.com/careers/}{Work with us}
\item
  \href{https://nytmediakit.com/}{Advertise}
\item
  \href{http://www.tbrandstudio.com/}{T Brand Studio}
\item
  \href{https://www.nytimes.com/privacy/cookie-policy\#how-do-i-manage-trackers}{Your
  Ad Choices}
\item
  \href{https://www.nytimes.com/privacy}{Privacy}
\item
  \href{https://help.nytimes.com/hc/en-us/articles/115014893428-Terms-of-service}{Terms
  of Service}
\item
  \href{https://help.nytimes.com/hc/en-us/articles/115014893968-Terms-of-sale}{Terms
  of Sale}
\item
  \href{https://spiderbites.nytimes.com}{Site Map}
\item
  \href{https://help.nytimes.com/hc/en-us}{Help}
\item
  \href{https://www.nytimes.com/subscription?campaignId=37WXW}{Subscriptions}
\end{itemize}
