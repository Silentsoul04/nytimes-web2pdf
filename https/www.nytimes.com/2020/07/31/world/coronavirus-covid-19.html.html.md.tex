Sections

SEARCH

\protect\hyperlink{site-content}{Skip to
content}\protect\hyperlink{site-index}{Skip to site index}

\href{https://www.nytimes.com/section/world}{World}

\href{https://myaccount.nytimes.com/auth/login?response_type=cookie\&client_id=vi}{}

\href{https://www.nytimes.com/section/todayspaper}{Today's Paper}

\href{/section/world}{World}\textbar{}A \$600-a-Week Lifeline for
Unemployed Americans Expires After an Impasse in Washington

\url{https://nyti.ms/2CW4m96}

\begin{itemize}
\item
\item
\item
\item
\item
\item
\end{itemize}

\href{https://www.nytimes.com/news-event/coronavirus?action=click\&pgtype=Article\&state=default\&region=TOP_BANNER\&context=storylines_menu}{The
Coronavirus Outbreak}

\begin{itemize}
\tightlist
\item
  live\href{https://www.nytimes.com/2020/08/01/world/coronavirus-covid-19.html?action=click\&pgtype=Article\&state=default\&region=TOP_BANNER\&context=storylines_menu}{Latest
  Updates}
\item
  \href{https://www.nytimes.com/interactive/2020/us/coronavirus-us-cases.html?action=click\&pgtype=Article\&state=default\&region=TOP_BANNER\&context=storylines_menu}{Maps
  and Cases}
\item
  \href{https://www.nytimes.com/interactive/2020/science/coronavirus-vaccine-tracker.html?action=click\&pgtype=Article\&state=default\&region=TOP_BANNER\&context=storylines_menu}{Vaccine
  Tracker}
\item
  \href{https://www.nytimes.com/interactive/2020/07/29/us/schools-reopening-coronavirus.html?action=click\&pgtype=Article\&state=default\&region=TOP_BANNER\&context=storylines_menu}{What
  School May Look Like}
\item
  \href{https://www.nytimes.com/live/2020/07/31/business/stock-market-today-coronavirus?action=click\&pgtype=Article\&state=default\&region=TOP_BANNER\&context=storylines_menu}{Economy}
\end{itemize}

Advertisement

\protect\hyperlink{after-top}{Continue reading the main story}

Supported by

\protect\hyperlink{after-sponsor}{Continue reading the main story}

\hypertarget{a-600-a-week-lifeline-for-unemployed-americans-expires-after-an-impasse-in-washington}{%
\section{A \$600-a-Week Lifeline for Unemployed Americans Expires After
an Impasse in
Washington}\label{a-600-a-week-lifeline-for-unemployed-americans-expires-after-an-impasse-in-washington}}

California became the first state to reach 500,000 total coronavirus
cases. Once the site of a major outbreak, Italy now offers lessons for
keeping the virus in check.

\begin{itemize}
\item
  Published July 31, 2020Updated Aug. 1, 2020, 3:09 a.m. ET
\item
  \begin{itemize}
  \item
  \item
  \item
  \item
  \item
  \item
  \end{itemize}
\end{itemize}

\hypertarget{heres-what-you-need-to-know}{%
\subsubsection{Here's what you need to
know:}\label{heres-what-you-need-to-know}}

\begin{itemize}
\tightlist
\item
  \protect\hyperlink{link-7c4d159d}{Tens of millions of jobless
  Americans are losing a benefit that helped keep them afloat.}
\item
  \protect\hyperlink{link-4e17d805}{California's summer outbreak makes
  it the first state with half a million cases.}
\item
  \protect\hyperlink{link-65fa7f74}{Giroir, Trump's testing czar, said
  most virus test results were coming back quickly. Public health
  experts disagree.}
\item
  \protect\hyperlink{link-2b88e858}{South Korea arrests the leader of a
  church where the virus spread rapidly.}
\item
  \protect\hyperlink{link-3bb771a7}{Florida, already reeling from the
  virus, faces a new threat from Hurricane Isaias.}
\item
  \protect\hyperlink{link-747b61fb}{Contact tracing, a process critical
  for managing the virus, falters from testing shortages and backlogs.}
\item
  \protect\hyperlink{link-19b57b6f}{A large outbreak at a Georgia summer
  camp adds to the evidence that children are susceptible to the virus.}
\end{itemize}

\includegraphics{https://static01.nyt.com/images/2020/07/31/us/31virus-briefing-relief/merlin_175181316_602a1744-c139-44e8-9e19-20e8bd9d39cb-articleLarge.jpg?quality=75\&auto=webp\&disable=upscale}

\hypertarget{tens-of-millions-of-jobless-americans-are-losing-a-benefit-that-helped-keep-them-afloat}{%
\subsection{Tens of millions of jobless Americans are losing a benefit
that helped keep them
afloat.}\label{tens-of-millions-of-jobless-americans-are-losing-a-benefit-that-helped-keep-them-afloat}}

A \$600 weekly jobless benefit from the federal government that became a
lifeline for tens of millions of unemployed Americans, while also
helping prop up the coronavirus-ravaged economy, expired at midnight as
officials in Washington failed to agree on a new relief bill.

The loss of the aid will leave millions struggling to make ends meet at
a precarious moment when nearly 11 percent of Americans have said that
they live in households where there is not enough to eat, according to a
\href{https://www.census.gov/programs-surveys/household-pulse-survey/data.html?utm_campaign=20200727mspuls1ccdtanl\&utm_medium=email\&utm_source=govdelivery}{recent
Census Bureau survey}, and more than a quarter have missed a rent or
mortgage payment.

And it comes as unemployment remains at record levels. More than 1.4
million Americans filed new for state unemployment benefits last week,
the \href{https://oui.doleta.gov/press/2020/073020.pdf}{Labor Department
said Thursday}. It was the
\href{https://www.nytimes.com/2020/07/30/business/economy/q2-gdp-coronavirus-economy.html}{19th
straight week that the tally exceeded one million,}an unheard-of figure
before the pandemic. Some 30 million people are receiving unemployment
benefits.

The benefit's expiration will force Louise Francis, who worked as a
banquet cook at the Sheraton Hotel in New Orleans for nearly two decades
before being furloughed last
spring,\href{https://www.nytimes.com/2020/07/30/business/economy/q2-gdp-coronavirus-economy.html}{to
get by on just state unemployment benefits, which for her come to \$247
a week.}

``With the \$600, you could see your way a little bit,'' said Ms.
Francis, 59. ``You could feel a little more comfortable. You could pay
three or four bills and not feel so far behind.''

The aid lapsed as Republicans and Democrats in Washington
\href{https://www.nytimes.com/2020/07/28/us/politics/coronavirus-relief-bills-house-senate.html}{remained
far apart on what the next round of virus relief should look like}.

Democrats wanted to extend the \$600 weekly payments through the end of
the year, as part of an expansive \$3 trillion aid package that would
also prop up state and local governments that are weighing layoffs and
service cuts to offset dwindling tax revenues. Republicans, worried that
the \$600 benefit left some people with more money than they earned when
they were working, sought to scale it back to \$200 per week as part of
a \$1 trillion proposal.

White House officials and Democrats blamed each other on Friday for the
benefit's expiration.

At a White House news conference, Mark Meadows, President Trump's chief
of staff, accused Democrats of playing ``politics as usual.'' At the
Capitol, Nancy Pelosi, the House Speaker, declared that administration
officials ``do not understand the gravity of the situation.''

Both said they planned to continue discussions, possibly into the
weekend, to find a compromise. But the talks will come too late to help
laid-off workers set to lose their aid.

As the deadline neared, Republicans proposed continuing the \$600
benefit for one week while talks continue. Democrats rejected the
short-term extension.

``When you have a six-day, one-week extension on a provision, it is
usually --- has always been --- to accommodate a legislative topic if
you're on the verge of having an agreement,'' Ms. Pelosi said. ``Why
don't we just get the job done? Why don't we just get the job done?''

\hypertarget{californias-summer-outbreak-makes-it-the-first-state-with-half-a-million-cases}{%
\subsection{California's summer outbreak makes it the first state with
half a million
cases.}\label{californias-summer-outbreak-makes-it-the-first-state-with-half-a-million-cases}}

Image

A line outside a coronavirus testing site in Los Angeles this
month.Credit...Philip Cheung for The New York Times

California passed a grim milestone on Friday, becoming the first state
to report more than 500,000 cases of the coronavirus, according to a
\href{https://www.nytimes.com/interactive/2020/us/coronavirus-us-cases.html\#states}{New
York Times database}.

In per capita terms, both the infections and deaths in California ---
the country's most populous state, with 40 million residents --- remain
lower than in many other states, including Florida, where the
concentration of cases is the worst in the nation. Three more states
have reported more than 400,000 cases --- Texas, Florida and New York
--- and no other had more than 200,000 as of Friday.

And though California has the third-highest number of
coronavirus-related deaths, with slightly over 9,000, its total is
significantly lower than that of New York, which has over 32,000. New
Jersey has the country's second-highest total, with more than 15,000. On
Friday, California officials reported 213 new deaths for its single-day
record, surpassing the previous high, 192, recorded on Wednesday.

California locked down its residents relatively early, on March 19,
buying time for hospitals and public health workers to prepare for an
expected onslaught. The state's weekly average number of infections in
late April was less than 20 percent of what it is today.

But while the restrictions led to
\href{https://www.nytimes.com/2020/04/14/us/california-coronavirus-shutdown.html}{early
success} in the state, which has the world's fifth-largest economy, they
eventually wore on residents reeling from spikes in unemployment.
Resistance mounted to the restrictions.

After a phased reopening that began in May, which some health officials
warned was premature, the number of infections
\href{https://www.nytimes.com/2020/06/29/us/california-coronavirus-reopening.html}{began
to soar}. Gov. Gavin Newsom has since
\href{https://www.nytimes.com/interactive/2020/07/17/upshot/coronavirus-face-mask-map.html}{made
face masks mandatory}, closed the state's bars and banned indoor dining,
\href{https://www.nytimes.com/2020/07/14/us/california-counties-reopening.html}{rolled
back reopening plans} for most Californians and begun withholding
federal relief funds from cities that refuse to enforce public health
orders.

Municipalities have stepped up enforcement as well. Los Angeles County
this week
\href{https://www.latimes.com/california/story/2020-07-29/county-shuts-three-businesses-for-failing-to-report-coronavirus-outbreaks}{shut
down} three food distribution facilities for failing to report
outbreaks, and Palm Springs ordered a
\href{https://www.palmspringsca.gov/home/showdocument?id=75670}{midnight
curfew}.

Nonetheless, the state reported a record 197 new coronavirus deaths on
Wednesday. The average weekly fatalities have doubled since the
beginning of July. **** The virus also officially spread to the last of
the state's 58 counties, with two cases reported in remote Modoc County,
which is at the Nevada and Oregon borders.

``It's here,'' the county's director of health services
\href{http://modochealthservices.org/corona-virus}{said in a news
release}, ``and we could see the number of cases increase in the next
few weeks.''

\hypertarget{giroir-trumps-testing-czar-said-most-virus-test-results-were-coming-back-quickly-public-health-experts-disagree}{%
\subsection{Giroir, Trump's testing czar, said most virus test results
were coming back quickly. Public health experts
disagree.}\label{giroir-trumps-testing-czar-said-most-virus-test-results-were-coming-back-quickly-public-health-experts-disagree}}

\includegraphics{https://static01.nyt.com/images/2020/07/31/business/31virus-briefing-erin/31virus-briefing-erin-videoSixteenByNine3000-v2.jpg}

As schools, universities and businesses struggle to reopen without the
coronavirus testing they need to curb outbreaks, the Trump
administration's testing czar testified to Congress on Friday that it
was currently impossible to get all tests back within three days.

The testing czar, Adm. Brett P. Giroir, told lawmakers that getting all
coronavirus tests back between 48 and 72 hours, which many health
officials have said is critical, ``is not a possible benchmark we can
achieve today, given the demand and the supply.''

Admiral Giroir said that it would be ``absolutely'' achievable in the
future, and that half of all test results were being processed within 24
hours. While not all tests can be turned around within three days, he
said, the average wait time for the rest was around that time or less
--- an assessment that is sharply at odds with what patients and health
professionals around the country say they are experiencing.

He told lawmakers that the nation was now averaging about 820,000 tests
each day, and that roughly half were ``done in either point-of-care
technologies with results in 15 minutes or less or at local hospitals
for which the turnaround time is generally within 24 hours.''

And he said that three-quarters of tests from commercial labs were
coming back within five days.

The remainder, he said, are processed by commercial labs like Quest
Diagnostics and LabCorp. Three-quarters of those tests were coming back
within five days, he said.

Admiral Giroir spoke alongside Dr. Anthony S. Fauci, the nation's top
infectious disease expert, and Dr. Robert R. Redfield, the director of
the Centers for Disease Control and Prevention, during a hearing of the
House Select Subcommittee on the Coronavirus Crisis, a special panel
created by Speaker Nancy Pelosi to oversee the Trump administration's
coronavirus response.

His comments on testing turnaround times were met with puzzlement by
public health experts, who say that even if the figures are accurate,
they do not reflect the reality on the ground. Reporting test results
and wait times in aggregate, these experts say, does not indicate things
are getting better. Testing shortages persist. And in some places, tests
cannot be processed at all because of a lack of reagents --- the
chemicals needed to detect whether the virus is present --- or lab
capacity.

``Across the board, the supply chain is still fragile and fragmented,''
said Amanda Harrington, director of the Clinical Microbiology Laboratory
at Loyola University Medical Center in Maywood, Ill. ``We have assays we
don't know if we can run tomorrow.

Dr. Michael T. Osterholm, director of the Center for Infectious Disease
Research and Policy at the University of Minnesota, said the
administration needed a ``national dashboard for testing'' where data is
collected and made publicly available.

Later Friday in an evening briefing in Florida with President Trump,
Gov. Ron DeSantis of Florida noted: ``We're doing so many tests,
sometimes it takes seven to ten days to get the results back, ''He said
that the state was trying to speed tests for symptomatic people, and
that new point-of-care tests from the federal government should help the
state get faster results.

In Alabama, the average wait time for coronavirus test results is
currently seven days --- significantly longer than the two or three-day
turnaround window advised by public health officials for making
quarantine and care decisions.

In \href{https://www.alabamapublichealth.gov/news/2020/07/31e.html}{a
statement} released by the state's department of public health on
Friday, officials asked health care providers to limit testing to ``the
elderly, those in congregate living settings, health care personnel,
those with symptoms consistent with COVID-19 and those with underlying
medical conditions that place them most at risk.''

Democrats on the House panel wasted little time in pointing out that the
caseload is much lower in Europe and Asia than in the United States. Dr.
Fauci said countries in those parts of the world were more aggressive
about shutting down as the pandemic raged.

``When they shut down, they shut down to the tune of about 95 percent,
getting their baseline down to tens or hundreds of cases a day,'' Dr.
Fauci said. By contrast, he said, only about 50 percent of the United
States shut down, and the baseline of daily cases was much higher --- as
many as 20,000 new cases a day --- even at its lowest. More recently,
the United States has recorded as many as 70,000 new cases a day.

Dr. Fauci also cast doubt on a study promoted by Mr. Trump and other
conservatives. Conducted by Henry Ford Hospital in Detroit, it showed an
apparent benefit for hydroxychloroquine, the anti-malaria drug that
President Trump has touted as a Covid-19 treatment. ``That study is a
flawed study,'' Mr. Fauci said.
(\href{https://www.nytimes.com/interactive/2020/science/coronavirus-drugs-treatments.html}{Read
more about the most-talked-about treatments for the coronavirus.})

\href{https://www.nytimes.com/interactive/2020/us/coronavirus-us-cases.html}{Tracking
the Coronavirus~›}

\href{https://www.nytimes.com/interactive/2020/us/coronavirus-us-cases.html}{}

\hypertarget{where-cases-are-rising-fastest}{%
\subsubsection{\texorpdfstring{Where cases are \textbf{rising}
fastest}{Where cases are rising fastest}}\label{where-cases-are-rising-fastest}}

\href{https://www.nytimes.com/interactive/2020/us/hawaii-coronavirus-cases.html}{}

Hawaii

\href{https://www.nytimes.com/interactive/2020/us/alaska-coronavirus-cases.html}{}

Alaska

\href{https://www.nytimes.com/interactive/2020/us/new-jersey-coronavirus-cases.html}{}

N.J.

\href{https://www.nytimes.com/interactive/2020/us/missouri-coronavirus-cases.html}{}

Mo.

\href{https://www.nytimes.com/interactive/2020/us/rhode-island-coronavirus-cases.html}{}

R.I.

\href{https://www.nytimes.com/interactive/2020/us/massachusetts-coronavirus-cases.html}{}

Mass.

\href{https://www.nytimes.com/interactive/2020/us/mississippi-coronavirus-cases.html}{}

Miss.

\href{https://www.nytimes.com/interactive/2020/us/maryland-coronavirus-cases.html}{}

Md.

\href{https://www.nytimes.com/interactive/2020/us/oklahoma-coronavirus-cases.html}{}

Okla.

\href{https://www.nytimes.com/interactive/2020/us/south-dakota-coronavirus-cases.html}{}

S.D.

\href{https://www.nytimes.com/interactive/2020/us/kentucky-coronavirus-cases.html}{}

Ky.

\href{https://www.nytimes.com/interactive/2020/us/nebraska-coronavirus-cases.html}{}

Neb.

\href{https://www.nytimes.com/interactive/2020/us/coronavirus-us-cases.html}{}

\hypertarget{us-hot-spots-}{%
\subsubsection{U.S. hot spots~›}\label{us-hot-spots-}}

\includegraphics{https://static01.nyt.com/newsgraphics/2020/03/16/coronavirus-maps/ede29db66044639aecd12edb0a3b7696a30be7e6/images/orphan_usa-threeByTwoSmallAt2X.png}

\href{https://www.nytimes.com/interactive/2020/world/coronavirus-maps.html}{}

\hypertarget{worldwide-}{%
\subsubsection{Worldwide~›}\label{worldwide-}}

\includegraphics{https://static01.nyt.com/newsgraphics/2020/03/16/coronavirus-maps/ede29db66044639aecd12edb0a3b7696a30be7e6/images/orphan_world-threeByTwoSmallAt2X.png}

\hypertarget{south-korea-arrests-the-leader-of-a-church-where-the-virus-spread-rapidly}{%
\subsection{South Korea arrests the leader of a church where the virus
spread
rapidly.}\label{south-korea-arrests-the-leader-of-a-church-where-the-virus-spread-rapidly}}

Image

Lee Man-hee, founder of the Shincheonji Church of Jesus, during a news
conference in March.~Credit...Yonhap/Reuters

The leader of a secretive religious sect in South Korea was arrested
Saturday on charges of embezzling church money and conspiring to impede
the government's efforts to fight the coronavirus.

\href{https://www.nytimes.com/2020/03/02/world/asia/coronavirus-south-korea-shincheonji.html?searchResultPosition=1}{Lee
Man-hee,} the founder of the Shincheonji Church of Jesus, was taken to
jail in Suwon, south of Seoul, early Saturday after a judge issued a
warrant for prosecutors to arrest him.

\hypertarget{latest-updates-global-coronavirus-outbreak}{%
\section{\texorpdfstring{\href{https://www.nytimes.com/2020/08/01/world/coronavirus-covid-19.html?action=click\&pgtype=Article\&state=default\&region=MAIN_CONTENT_1\&context=storylines_live_updates}{Latest
Updates: Global Coronavirus
Outbreak}}{Latest Updates: Global Coronavirus Outbreak}}\label{latest-updates-global-coronavirus-outbreak}}

Updated 2020-08-01T18:23:51.652Z

\begin{itemize}
\tightlist
\item
  \href{https://www.nytimes.com/2020/08/01/world/coronavirus-covid-19.html?action=click\&pgtype=Article\&state=default\&region=MAIN_CONTENT_1\&context=storylines_live_updates\#link-3ac56579}{Top
  officials work to break impasse over jobless benefit.}
\item
  \href{https://www.nytimes.com/2020/08/01/world/coronavirus-covid-19.html?action=click\&pgtype=Article\&state=default\&region=MAIN_CONTENT_1\&context=storylines_live_updates\#link-8796723}{The
  virus picks up dangerous speed in the Midwest, and in areas that had
  seen success.}
\item
  \href{https://www.nytimes.com/2020/08/01/world/coronavirus-covid-19.html?action=click\&pgtype=Article\&state=default\&region=MAIN_CONTENT_1\&context=storylines_live_updates\#link-25930521}{Thousands
  in Berlin protest Germany's coronavirus measures.}
\end{itemize}

\href{https://www.nytimes.com/2020/08/01/world/coronavirus-covid-19.html?action=click\&pgtype=Article\&state=default\&region=MAIN_CONTENT_1\&context=storylines_live_updates}{See
more updates}

More live coverage:
\href{https://www.nytimes.com/live/2020/07/31/business/stock-market-today-coronavirus?action=click\&pgtype=Article\&state=default\&region=MAIN_CONTENT_1\&context=storylines_live_updates}{Markets}

The rapid spread of the virus this winter among the church's worshipers
in Daegu, a city in the southeast, briefly made South Korea home to the
world's largest coronavirus outbreak outside China. As of Friday, more
than 36 percent of the country's 14,300 coronavirus patients were
members of Shincheonji or their contacts, according to government data.

Prosecutors say Mr. Lee and other church officials obstructed the
government's efforts to fight the epidemic by not fully disclosing the
number of worshipers and their gathering places. Seven church officials
were indicted last month on the same charge.

Mr. Lee, 88, has also been accused of embezzling 5.6 billion won, or
\$4.7 million, from church funds to build a luxurious ``peace palace''
north of Seoul. The Shincheonji church has broadly denied all the
charges against him, and he could face years in prison if convicted.

Intense criticism from the South Korean public
\href{https://www.nytimes.com/2020/03/02/world/asia/coronavirus-south-korea-shincheonji.html}{forced
Mr. Lee to apologize} in March.

In a statement on Saturday, the church said that Mr. Lee never intended
to hamper the government's efforts to control the epidemic, and that he
had only expressed concern over what he felt were excessive demands for
personal data on church worshipers.

``He has emphasized the importance of disease control and urged the
church members to cooperate with the authorities,'' the church said.
``We will do our best to let the truth be known through trial.''

But parents who accused the church of luring and brainwashing their
children with its unorthodox teachings welcomed his arrest on Saturday,
calling Mr. Lee a ``religious con artist.''

\hypertarget{florida-already-reeling-from-the-virus-faces-a-new-threat-from-hurricane-isaias}{%
\subsection{Florida, already reeling from the virus, faces a new threat
from Hurricane
Isaias.}\label{florida-already-reeling-from-the-virus-faces-a-new-threat-from-hurricane-isaias}}

Image

In preparation on Friday for the storm, people filled sand bags for
distribution to the residents of Palmetto Bay, near
Miami.~Credit...Chandan Khanna/Agence France-Presse --- Getty Images

Florida's Atlantic coast braced for the arrival of Hurricane Isaias this
weekend after the storm raked the Bahamas, parts of Puerto Rico and the
Dominican Republic on Friday.

Preparations for the storm were complicated by the state's battle with
the coronavirus, which could make evacuating homes and entering
community shelters especially risky. Friday was the third consecutive
day that Florida set its record for the most deaths reported in a single
day, according to a New York Times database.

Gov. Ron DeSantis said at a news conference on Friday that the division
of emergency management had been working at
\href{https://www.floridadisaster.org/sert/eoc-activation-levels/}{its
most active level} since March, ``allowing them to actively plan for
hurricane season even while responding to the Covid-19 pandemic.''

Early on in the pandemic, the governor said, the division created a
reserve of protective equipment for hurricane season, including 20
million masks, 22 million gloves and 1.6 million face shields.

\hypertarget{contact-tracing-a-process-critical-for-managing-the-virus-falters-from-testing-shortages-and-backlogs}{%
\subsection{Contact tracing, a process critical for managing the virus,
falters from testing shortages and
backlogs.}\label{contact-tracing-a-process-critical-for-managing-the-virus-falters-from-testing-shortages-and-backlogs}}

Image

El, who worked as a contact tracer in New York, said, ``I have never had
a more dysfunctional workplace.''Credit...Hiroko Masuike/The New York
Times

Considered a cornerstone of the public health arsenal to suppress the
virus,
\href{https://www.nytimes.com/2020/07/31/health/covid-contact-tracing-tests.html}{contact
tracing has largely failed in the United States}, as the virus's
pervasiveness and major lags in testing have rendered the system almost
pointless.

The
\href{https://www.cdc.gov/coronavirus/2019-ncov/php/contact-tracing/contact-tracing-plan/contact-tracing.html}{goal}of
contact tracing is to reach people who have spent more than 15 minutes
within six feet of an infected person and ask them to voluntarily
quarantine at home for two weeks, even
\href{https://www.cdc.gov/coronavirus/2019-ncov/symptoms-testing/testing.html}{if
they test negative}, monitoring themselves for symptoms during that
time. On Friday, Dr. Fauci said that if someone gets tested, ``they
should assume that it might be positive and should essentially isolate
themselves before they go back and get the result of the test.''

In some of the hardest-hit regions, contact-tracing efforts seem futile,
as many people have refused to participate or cannot even be located,
further hampering health care workers.

In Arizona's most populated region, for example, the virus is so
\href{https://www.azfamily.com/news/continuing_coverage/coronavirus_coverage/contact-tracing-important-but-less-useful-with-spiking-cases-maricopa-county-says/article_57d55328-bb4b-11ea-8718-8b1cf4ab4137.html}{ubiquitous}
that contact tracers have been unable to reach a fraction of those
infected. In Austin, Texas, the story is much the same. Cities in
Florida, which has been seeing an average of more than 10,000 new cases
a day in the past week, have
largely\href{https://www.nbcmiami.com/news/local/miami-beach-mayor-urges-desantis-to-address-failures-of-floridas-contact-tracing-program/2268324/}{given
up on contact tracing}. Things are equally dismal in California. And in
\href{https://www.nytimes.com/2020/07/29/nyregion/new-york-contact-tracing.html}{New
York City's tracing program}, workers have complained of crippling
communication and training problems.

From the very beginning, states and cities have struggled to detect the
prevalence of the virus because of spotty and sometimes rationed
diagnostic testing and long delays in getting results. For the tests
currently available and in high demand, there is not a consensus on
\href{https://www.nytimes.com/2020/07/31/health/coronavirus-test-ethics.html}{who
should get them.} Some experts say everyone should get tested, even
those without symptoms. Others say the tests should be reserved for the
people who have symptoms or are more vulnerable to infection.

There is broad consensus, however, that more tests are needed.

On Friday, the
\href{https://www.nih.gov/news-events/news-releases/nih-delivering-new-covid-19-testing-technologies-meet-us-demand}{National
Institutes of Health announced} that seven companies have received
\$248.7 million to ramp up test production and deliver millions more
weekly tests as early as September.

The tests, which include three simple ``point of care'' tests
that\href{https://www.nytimes.com/2020/07/06/health/fast-coronavirus-tests.html}{don't
need to be shipped to laboratories}, were selected as promising
candidates to address the serious shortages that have plagued testing
efforts since March.

\hypertarget{a-large-outbreak-at-a-georgia-summer-camp-adds-to-the-evidence-that-children-are-susceptible-to-the-virus}{%
\subsection{A large outbreak at a Georgia summer camp adds to the
evidence that children are susceptible to the
virus.}\label{a-large-outbreak-at-a-georgia-summer-camp-adds-to-the-evidence-that-children-are-susceptible-to-the-virus}}

As schools and universities plan for the new academic year, and
administrators grapple with complex questions about how to keep young
people safe, a
new\href{https://www.cdc.gov/mmwr/volumes/69/wr/mm6931e1.htm?s_cid=mm6931e1_w}{report
about a coronavirus outbreak at a sleepaway camp in Georgia} provides
fresh reasons for concern.

The camp implemented several precautionary measures against the virus,
but stopped short of requiring campers to wear masks. The virus blazed
through the community of about 600 campers and counselors, the Centers
for Disease Control and Prevention reported on Friday.

The study is notable because few outbreaks in schools or child care
settings have been described to date, said Caitlin Rivers, an
epidemiologist at the Johns Hopkins Bloomberg School of Public Health.

``The study affirms that group settings can lead to large outbreaks,
even when they are primarily attended by children,'' she said.

``The fact that so many children at this camp were infected after just a
few days together underscores the importance of mitigation measures in
schools that do reopen for in person learning,'' Dr. Rivers added.

While the role children play in the spread of the virus has been
questioned, the authors of the report said the research adds to evidence
that children of all ages are not only susceptible to infection, but may
play an important role in transmission.

Of the 344 campers and staff for whom test results were available, 260
tested positive, meaning at least 43 percent were infected, though the
figure may well be higher, the C.D.C. said.

Of children ages 6 to 10, over half were infected; 44 percent of those
ages 11 to 17 were infected, as were one-third of those ages 18 to 21.
Only seven staffers were older than 22, and two of them tested positive.

Those who had been at the camp longest had the highest rate of
infection; overall, more than half of the staff, who had arrived before
the campers, were infected.

\hypertarget{fitch-ratings-downgrades-its-outlook-on-us-debt-as-the-deficit-soars}{%
\subsection{Fitch Ratings downgrades its outlook on U.S. debt as the
deficit
soars.}\label{fitch-ratings-downgrades-its-outlook-on-us-debt-as-the-deficit-soars}}

The credit rating firm Fitch left the United States' AAA rating
untouched, but downgraded its outlook on what's effectively the national
credit score, suggesting the country's status as one of the world's most
trustworthy borrowers could be put at risk by the enormous deficits the
federal government is running to combat fallout from the pandemic.

``The outlook has been revised to negative to reflect the ongoing
deterioration in the U.S. public finances and the absence of a credible
fiscal consolidation plan,''
\href{https://www.fitchratings.com/research/sovereigns/fitch-revises-united-states-outlook-to-negative-affirms-at-aaa-31-07-2020}{Fitch
analysts wrote} on Friday in a report announcing the decision.

Cratering tax revenues and surging expenditures have driven record
levels of red ink for the federal government in recent months. The
\href{https://www.nytimes.com/live/2020/07/13/business/stock-market-today-coronavirus\#the-us-budget-deficit-hits-another-monthly-record}{United
States budget deficit hit a record} \$864 billion in June as the
government continued pumping money into the economy to support workers
and businesses slammed by the pandemic. Some analysts expect monthly
deficits to soon top \$1 trillion.

Ballooning deficits have led to an explosion of new borrowing. Fitch
noted that the Treasury Department borrowed just under \$3 trillion
dollars from the end of February to the end of June.

Much of the supply of new government bonds was,
\href{https://www.nytimes.com/2020/04/15/business/coronavirus-stimulus-money.html}{essentially,
purchased by the Federal Reserve}, which has bought \$2.6 trillion in
financial assets since the middle of March, Fitch noted.

The presence of the Federal Reserve, which can essentially create
whatever money it wants and use it to buy assets, such as U.S.
government debt, has depressed yields on government bonds even as its
debts and deficits rise sharply.

On Friday, the yield on the 10-year note fell to 0.53 percent, one of
\href{https://www.marketwatch.com/story/10-year-treasury-yield-plunged-to-its-lowest-in-234-years-says-deutsche-bank-11596214464\#:~:text=The\%2010\%2Dyear\%20Treasury\%20note,scurrying\%20into\%20safe\%20haven\%20assets.}{the
lowest levels in recorded history}, suggesting there is virtually no
concern among investors about the country's ability to service its
growing debts.

Global roundup

\hypertarget{once-an-out-of-control-center-italy-now-offers-lessons-for-keeping-the-virus-in-check}{%
\subsection{Once an out-of-control center, Italy now offers lessons for
keeping the virus in
check.}\label{once-an-out-of-control-center-italy-now-offers-lessons-for-keeping-the-virus-in-check}}

Image

Italy has consolidated, or at least maintained, the rewards of a tough
nationwide lockdown through a mix of vigilance and painfully gained
medical expertise.Credit...Gianni Cipriano for The New York Times

When the virus erupted in the West, Italy was
\href{https://www.nytimes.com/interactive/2020/03/27/world/europe/coronavirus-italy-bergamo.html}{the
nightmarish epicenter}, a place to avoid at all costs and a shorthand in
the United States and much of Europe for uncontrolled contagion.

Fast forward a few months, and the United States has had tens of
thousands more deaths than any other country in the world. European
states that once looked smugly at Italy are facing new flare-ups.

And Italy? Its hospitals are basically empty of Covid-19 patients. Daily
deaths attributed to the virus in Lombardy, the region that bore the
brunt of the pandemic, hover around zero. The number of new daily cases
has plummeted to ``one of the lowest in Europe and the world,'' said
Giovanni Rezza, director of the infective illness department at the
National Institute of Health.

How Italy has gone from being a global pariah to a model --- however
imperfect --- of viral containment holds fresh lessons for the rest of
the world, including the United States.

Italy has consolidated, or at least maintained, the rewards of a tough
nationwide lockdown through a mix of vigilance and painfully gained
medical expertise.

\begin{itemize}
\item
  Its government has been guided by scientific and technical committees.
\item
  The country set aside economic pressures and only began easing its
  exceptionally tight lockdown based on case counts.
\item
  Italy continues to limit travel from elsewhere.
\item
  Local doctors, hospitals and health officials collect more than 20
  indicators on the virus daily and send them to regional authorities,
  who then forward them to the National Institute of Health.
\end{itemize}

The result is a weekly X-ray of the country's health upon which policy
decisions are based.

Here are other developments from around the globe:

\begin{itemize}
\item
  Across \textbf{Europe}, the economy tumbled into its worst recession
  on record in the second quarter. From April to June, gross domestic
  product fell by 11.9 percent from the first quarter in the European
  Union, and by 12.1 percent in the core group of countries that use the
  euro currency. On an annualized basis, European Union economies shrank
  by 14.4 percent, and eurozone economies by 15 percent, the sharpest
  contractions since statistics started being kept in 1995.
\item
  As of Saturday morning, \textbf{Mexico's confirmed death toll} of
  46,688 was
  \href{https://www.nytimes.com/interactive/2020/world/americas/mexico-coronavirus-cases.html}{the
  world's third highest} behind the United States and Brazil.
  \href{https://www.nytimes.com/interactive/2020/world/europe/united-kingdom-coronavirus-cases.html}{Britain
  ranked fourth}, with 569 fewer deaths. The number of new reported
  infections in Mexico has been climbing since May and topped 8,000 on
  Friday, bringing the country's caseload to nearly 425,000.
\item
  Britain has barred millions of people in northern \textbf{England}
  from meeting other members of other households at their homes, paused
  reopenings set for Aug. 1 and moved to make face masks mandatory in
  more places, after a day on which it reported 38
  \href{https://www.nytimes.com/interactive/2020/world/europe/united-kingdom-coronavirus-cases.html}{new
  coronavirus deaths} and nearly 900 known new infections, its highest
  case numbers in a month.
\end{itemize}

\includegraphics{https://static01.nyt.com/images/2020/07/31/business/31virus-video-boris/31virus-video-boris-videoSixteenByNine3000.jpg}

\begin{itemize}
\item
  \href{https://www.santepubliquefrance.fr/maladies-et-traumatismes/maladies-et-infections-respiratoires/infection-a-coronavirus/documents/bulletin-national/covid-19-point-epidemiologique-du-30-juillet-2020}{French
  health authorities} on Friday reported a 54 increase in new cases over
  the past week and a rise in hospitalizations. Community transmission
  is accelerating most among young adults aged 20 to 30, according to
  Santé Publique \textbf{France}. The authority called for increased
  vigilance in preventive measures and cited a decline in social
  distancing and avoidance of hand-shaking and hugs, while reporting an
  increase in public mask-wearing.
\item
  A stark lack of testing in many \textbf{African} countries has kept
  officials from being able to track the pandemic, prompting fears that
  a recent surge in cases across the continent may be just the ``tip of
  the iceberg,'' according to the International Rescue Committee. The
  organization said many African nations needed international support to
  increase their testing capacity or the continent could face ``an
  undetected and uncontrolled spread.''
\item
  \textbf{Vietnam's}
  \href{https://www.nytimes.com/2020/07/29/world/asia/coronavirus-vietnam.html}{new
  outbreak of the coronavirus} claimed a third victim on Saturday, a
  68-year-old patient in the central city of Danang who was suffering
  from late-stage leukemia. The Communist country has been one of the
  most successful in battling the coronavirus and went more than three
  months without a case of local transmission. But in late July, a more
  severe strain of the virus appeared in Danang and quickly spread to
  other parts of the country. Vietnam reported its first death from the
  coronavirus on Friday. It now has 558 cases, although many are
  returnees from other countries who tested positive in quarantine.
\item
  The \textbf{Hong Kong} government said on Friday that it would
  \href{https://www.nytimes.com/2020/07/31/world/asia/hong-kong-election-delayed.html}{postpone
  the city's September legislative election}by a year because of the
  coronavirus pandemic, a decision seen by the pro-democracy opposition
  as a brazen attempt to thwart its electoral momentum and avoid the
  defeat of pro-Beijing candidates.

  ``It is a really tough decision to delay but we want to ensure
  fairness, public safety and public health,'' said Carrie Lam, Hong
  Kong's chief executive.
\item
  On Saturday, \textbf{Japan} announced 1,579 new cases, breaking a
  record set the day before. The country now has more than 1,000 deaths
  related to the coronavirus, reporting 1,011 on Saturday.
\end{itemize}

\hypertarget{health-care-workers-who-described-their-ppe-as-inadequate-had-a-higher-infection-rate-a-study-says}{%
\subsection{Health care workers who described their P.P.E. as inadequate
had a higher infection rate, a study
says.}\label{health-care-workers-who-described-their-ppe-as-inadequate-had-a-higher-infection-rate-a-study-says}}

Image

Lab jackets and personal protective equipment hanging from a fence at a
coronavirus testing site in Milwaukee this month.Credit...Joshua Lott
for The New York Times

As the virus surged this spring, health care workers in the United
States and the United Kingdom scrambled to make do with scarce personal
protective equipment. The consequences to their own health were stark.

According to
a\href{https://www.thelancet.com/journals/lanpub/article/PIIS2468-2667(20)30164-X/fulltext}{new
study,} these workers were 3.4 times more likely to report a positive
coronavirus test than the general population. Workers who described
their equipment --- including masks, gloves and gowns --- as
insufficient were 1.3 times more likely to report positive tests than
their colleagues who deemed their equipment appropriate.

Using self-reported data collected
through\href{https://covid.joinzoe.com/us}{a Covid-symptom monitoring
app}, researchers at Harvard Medical School and Massachusetts General
Hospital surveyed 99,795 front-line health care workers and two million
other people from March 24 through April 23. Health care workers who
were Black, Asian or other races were 1.81 times more likely to report a
positive test result than non-Hispanic white health care workers, the
study found.

``Minority front-line health care workers tend to be in higher-risk
settings and have less access to protective equipment,'' said
\href{https://cgvh.harvard.edu/people/erica-warner}{Dr. Erica T.
Warner,} a co-author of the study and an assistant professor of medicine
at Harvard Medical School. ``This is a microcosm of the larger disparity
we see in health care in general.''

Even though protective gear is now more readily available, shortages are
\href{https://www.nytimes.com/2020/07/08/health/coronavirus-masks-ppe-doc.html}{still
common}.

U.S. ROUNDUP

\hypertarget{nyc-sets-a-positivity-rate-threshold-for-reopening-schools-and-a-strategy-if-someone-tests-positive-at-school}{%
\subsection{N.Y.C. sets a positivity rate threshold for reopening
schools and a strategy if someone tests positive at
school.}\label{nyc-sets-a-positivity-rate-threshold-for-reopening-schools-and-a-strategy-if-someone-tests-positive-at-school}}

Image

P.S. 241 in the Crown Heights neighborhood of Brooklyn closed in April
because of the pandemic.Credit...Kirsten Luce for The New York Times

New York City public schools, the nation's largest school system, will
be able to reopen its school buildings in September only if the city
maintains a test positivity rate below 3 percent, Mayor Bill de Blasio
announced on Friday. That conservative threshold is even lower than the
5 percent test positivity rate which has been set by Gov. Andrew M.
Cuomo as
\href{https://www.nytimes.com/2020/07/14/us/coronavirus-schools-fall.html}{a
cut-off for school reopening and recommended by public health experts}.

The average positivity rate for New York City has generally remained
lower even than the new city threshold, according to city and state
figures. But even a modest uptick in cases over the next few weeks could
nudge that rate higher, which raises fresh questions about
\href{https://www.nytimes.com/2020/07/08/nyregion/nyc-schools-reopening-plan.html}{whether
city schools will open part-time on Sept. 10 as planned}. On Friday, the
school system
\href{https://infohub.nyced.org/docs/default-source/default-document-library/nyc-doe---state-doh-reopening-plan-7-31.pdf}{submitted
its reopening plan to the state}.

New York is one of the only large districts in the country that is
planning to reopen its buildings at all: Children will report to school
one to three days a week to allow for social distancing. All staff
members will be asked to take tests before the start of school, with
expedited results. Education officials in the city laid out a plan on
Thursday for what would happen in the seemingly inevitable event that
cases are confirmed in a classroom.

The protocol means it is likely that at many of the city's 1,800
schools, individual classrooms or even entire buildings will be closed
at points during the school year.

Officials said confirmed infections among students, teachers and staff
members would be treated the same. One or two cases in a single
classroom would require those classes to close for 14 days; all students
and staff members in that classroom would be ordered to self-quarantine,
and students would learn remotely. The rest of the school would continue
to operate.

But if two or more people in different classrooms in the same school
tested positive, the entire building would close while city disease
detectives were brought in to investigate the cases, which could take
several days. Depending on the results of the investigation, the
building could reopen, but the classrooms with positive cases would
remain closed for 14 days.

Elsewhere in the U.S.:

\begin{itemize}
\item
  \textbf{Cases in New Jersey}, which just a week ago had plunged to
  their lowest levels since the pandemic began,
  \href{https://www.nytimes.com/2020/07/30/nyregion/coronavirus-cases-nj.html}{are
  rising again}, fueled in part by outbreaks among young adults along
  the Jersey Shore. As of Thursday, the state had recorded an average of
  434 cases per day over the last week, an increase of 35 percent from
  the average two weeks earlier, according to a Times database. On
  Friday, there were 699 new cases, the governor said.
\item
  Airbnb will start cracking down on house parties in \textbf{New
  Jersey} after state health officials warned that the parties were
  leading to Covid-19 clusters. The vacation rental company said it
  would remove 35 listings from the site, according to
  \href{https://apnews.com/c64053bb7f7b60001a314526da06732e}{The
  Associated Press}. It took police nearly five hours to break up a
  gathering of more than 700 people at an Airbnb rental property in
  Jackson, N.J., last weekend.
\item
  The U.S. Food and Drug Administration authorized the first two tests
  capable of estimating the quantity of coronavirus antibodies in a
  patient's blood. Both tests were developed by Siemens, according to
  \href{https://www.fda.gov/news-events/press-announcements/coronavirus-covid-19-update-fda-authorizes-first-tests-estimate-patients-antibodies-past-sars-cov-2}{a
  statement} released by the F.D.A. on Friday. The agency cautioned
  against interpreting the results from the tests, or any serology test,
  as a sign of immunity to the virus.
\item
  The French drug maker Sanofi said on Friday that it had secured an
  agreement of up to \$2.1 billion to supply the \textbf{U.S. federal
  governmen}t with 100 million doses of its experimental coronavirus
  vaccine, the largest such deal announced to date. The arrangement with
  Sanofi and its partner, the British pharmaceutical company
  GlaxoSmithKline, brings the Trump administration's investment in
  coronavirus vaccine projects
  \href{https://medicalcountermeasures.gov/app/barda/coronavirus/COVID19.aspx?filter=vaccine}{to
  more than \$8 billion}. This effort, known as Operation Warp Speed, is
  placing bets on multiple vaccines and is paying companies to
  manufacture millions of doses before clinical trials have been
  completed.
\item
  \textbf{The Trump administration} wasted around \$500 million by
  overpaying for ventilators through negotiations that were ``inept,'' a
  panel of the House Oversight and Reform Committee said in a report
  released Friday. It faulted Peter Navarro, Mr. Trump's top trade
  adviser, and Jared Kushner, his son-in-law and senior adviser, for
  negotiating a deal in which the panel said they paid almost five times
  the price per device than under a previous contract with the same
  vendor.
\item
  \textbf{Florida} broke a record --- the most deaths the state reported
  in a single day --- for the fourth day in a row: On Friday, the state
  announced 257 additional fatalities. \textbf{Mississippi, Idaho and
  California} all reported their highest numbers of deaths in a single
  day from the virus. \textbf{North Dakota} and \textbf{Tennessee}
  reported new single-day case records.
\item
  Even with significant gaps in the available data, there are strong
  indications that \textbf{Native American people}
  \href{https://www.nytimes.com/2020/07/30/us/native-americans-coronavirus-data.html}{have
  been disproportionately affected by the virus}. The rate of known
  cases in the eight counties with the largest populations of Native
  Americans is nearly double the national average, a Times analysis has
  found.
\item
  \textbf{Greenwich, Conn.}, one of the wealthiest suburbs in the
  country, is experiencing what health officials have called a ``mini
  surge'' of infections, an outbreak that has cascaded through the
  community and underscored how social gatherings among young people are
  posing fresh challenges to containing the virus. More than 20 people
  between the ages 16 and 21 have tested positive for the virus, with
  more cases expected as testing continues, according to Greenwich
  health officials.
\item
  \textbf{\href{https://www.nytimes.com/2020/07/30/us/politics/juvenile-detainees-coronavirus.html}{Black
  youth}}\href{https://www.nytimes.com/2020/07/30/us/politics/juvenile-detainees-coronavirus.html}{detained
  in juvenile justice facilities} have been released at a far slower
  rate than their white peers in response to the coronavirus, according
  to a new report that also found that the gap in release rates between
  the two groups had nearly doubled over the course of the pandemic. The
  \href{https://www.aecf.org/blog/youth-detention-admissions-remain-low-but-releases-stall-despite-covid-19/}{report},
  released this month by the Annie E. Casey Foundation, illustrates one
  more disparity the coronavirus has exacerbated for Black children, who
  are disproportionately funneled into the juvenile justice system.
\end{itemize}

\hypertarget{with-jobless-aid-set-to-lapse-at-midnight-white-house-and-democrats-trade-blame}{%
\subsection{With jobless aid set to lapse at midnight, White House and
Democrats trade
blame.}\label{with-jobless-aid-set-to-lapse-at-midnight-white-house-and-democrats-trade-blame}}

Image

The talks will come too late to help laid-off workers set to lose their
aid. Many state unemployment systems have already stopped sending
payments.Credit...Stefani Reynolds for The New York Times

White House officials and Democrats blamed each other on Friday for the
looming expiration by day's end of a \$600 weekly jobless aid payment
that has become a critical lifeline for tens of millions of Americans,
as they remained at an impasse on passing another round of federal
pandemic relief.

At a news conference at the White House, Mark Meadows, the chief of
staff, accused Democrats of playing ``politics as usual'' on Capitol
Hill. At the other end of Pennsylvania at the Capitol, Speaker Nancy
Pelosi of California declared that administration officials ``do not
understand the gravity of the situation.''

Both said they planned to continue discussions on Friday, and
potentially into the weekend to find a compromise. But the talks will
come too late to help laid-off workers set to lose their aid. Many state
unemployment systems have already stopped sending payments.

Economist have said the faltering economy is likely to face further
devastation without the security of additional payments.

Grasping for more time to reach an agreement, Republicans on Thursday at
first proposed extending the benefit at a much lower rate through the
end of the year, and then proposed continuing the \$600-per-week benefit
for one week. But Democrats, who want to extend the \$600 weekly
payments through the end of the year, rejected those alternatives. Ms.
Pelosi said on Friday that they would do nothing to address the
magnitude of the problem or bridge the deep divides separating her
party's \$3 trillion aid proposal with at \$1 trillion plan endorsed by
Republicans.

``When you have a six-day, one-week extension on a provision, it is
usually --- has always been --- to accommodate a legislative topic if
you're on the verge of having an agreement,'' Ms. Pelosi said. ``Why
don't we just get the job done? Why don't we just get the job done?''

Representative Steny H. Hoyer of Maryland, the majority leader, said on
Friday that lawmakers who were to begin their annual August recess would
be on call to return to Washington for potential votes on the recovery
package, should lawmakers and the White House reached an agreement.

\hypertarget{baseball-grapples-with-more-outbreaks-as-it-adjusts-to-cardboard-fans-and-piped-in-crowd-noise}{%
\subsection{Baseball grapples with more outbreaks as it adjusts to
cardboard fans and piped-in crowd
noise.}\label{baseball-grapples-with-more-outbreaks-as-it-adjusts-to-cardboard-fans-and-piped-in-crowd-noise}}

Image

The Cardinals played in Minnesota on Tuesday and Wednesday and were to
play in Milwaukee on Friday night.Credit...Jordan Johnson/USA Today
Sports, via Reuters

Major League Baseball's
\href{https://www.nytimes.com/2020/07/31/sports/baseball/cardinals-twins-coronavirus-mlb.html}{outbreak
spread into another clubhouse} on Friday when the St. Louis Cardinals
postponed their game in Milwaukee after two players tested positive.

The Cardinals-Brewers game is the third postponement on baseball's
Friday night schedule, following earlier ones involving the Miami
Marlins, who were to play the Washington Nationals, and the Philadelphia
Phillies, who were to host Toronto. The Marlins have had 18 players
(including another on Friday) and two staff members test positive this
week; those cases have already
\href{https://www.nytimes.com/2020/07/28/sports/baseball/marlins-outbreak-mlb-coronavirus.html}{upended
baseball's revised schedule}.

One of the biggest adjustments for major leaguers during this 60-game
season will be
\href{https://www.nytimes.com/2020/07/31/sports/baseball/baseball-empty-stadiums-effects.html}{playing
in empty, cavernous stadiums, at least for the time being}. While
baseball has attempted to fill the void with cardboard fans, artificial
noise and \href{https://www.youtube.com/watch?v=q_FQcKH4xL4}{even
virtual ``crowds'' on broadcasts}, there is no denying that games are
being held in an atmosphere that is far from normal.

``I think it's going to affect things in weird ways that we can't even
fully anticipate right now,'' Russell Carleton, a psychologist and
analyst who has consulted with the Cleveland Indians and the Mets, said
of 2020's empty stadiums. ``And it's going to vary from guy to guy.''

Other developments related to sports and culture:

\begin{itemize}
\item
  The
  \textbf{\href{https://www.nytimes.com/2020/07/31/arts/music/salzburg-festival-coronavirus-cosi.html}{Salzburg
  Festival}}\href{https://www.nytimes.com/2020/07/31/arts/music/salzburg-festival-coronavirus-cosi.html}{is
  still going on}but its centennial season is abbreviated and has come
  with an elaborate protection plan.
\item
  The \textbf{Barrington Stage Company} in Massachusetts was planning to
  become the first theater in the U.S. to stage an indoor show featuring
  an Actors' Equity performer since the outbreak closed theaters. The
  company removed many seats in its theater, reconfigured its
  air-conditioning system, and redesigned bathrooms. But the state of
  Massachusetts decided not to permit indoor theater, so the show,
  ``Harry Clarke,''
  \href{https://www.nytimes.com/2020/07/30/theater/the-first-equity-authorized-indoor-theater-is-moving-outdoors.html}{is
  moving outdoors}.
\item
  \textbf{Bryan Cranston},
  \href{https://www.nytimes.com/2020/07/31/arts/television/bryan-cranston-coronavirus-plasma.html}{the
  star of ``Breaking Bad,'' has posted a video of himself donating
  plasma} following his recovery from Covid-19. The actor called plasma
  ``liquid gold'' and said it could be rich in antibodies and could
  benefit others in their recovery.
\end{itemize}

Image

Members of the Civic Party barred from Hong Kong's Legislative Council
elections speaking at a news conference on Thursday.Credit...Lam Yik Fei
for The New York Times

The Hong Kong government said on Friday that it would
\href{https://www.nytimes.com/2020/07/31/world/asia/hong-kong-election-delayed.html}{postpone
the city's September legislative election}by a year because of the
coronavirus pandemic, a decision seen by the pro-democracy opposition as
a brazen attempt to thwart its electoral momentum and avoid the defeat
of pro-Beijing candidates.

``It is a really tough decision to delay but we want to ensure fairness,
public safety and public health,'' said Carrie Lam, Hong Kong's chief
executive.

The delay was a blow to opposition politicians, who had expected to ride
to victory in the fall on a wave of deep-seated dissatisfaction with the
government and concerns about a sweeping
\href{https://www.nytimes.com/2020/07/05/world/asia/hong-kong-security-law.html}{new
national security law imposed on the city by Beijing}. And it was the
latest in a quick series of aggressive moves by the pro-Beijing
establishment to sideline the pro-democracy movement.

On Thursday,
\href{https://www.nytimes.com/2020/07/29/world/asia/hong-kong-arrests-security-law.html}{12
pro-democracy candidates said they had been barred from running},
including four sitting lawmakers and several prominent activists like
Joshua Wong. Mr. Wong said he was barred in part because of his
criticism of the new security law. He called the disqualifications ``the
most scandalous election fraud ever in Hong Kong history.''

\href{https://www.nytimes.com/news-event/coronavirus?action=click\&pgtype=Article\&state=default\&region=MAIN_CONTENT_3\&context=storylines_faq}{}

\hypertarget{the-coronavirus-outbreak-}{%
\subsubsection{The Coronavirus Outbreak
›}\label{the-coronavirus-outbreak-}}

\hypertarget{frequently-asked-questions}{%
\paragraph{Frequently Asked
Questions}\label{frequently-asked-questions}}

Updated July 27, 2020

\begin{itemize}
\item ~
  \hypertarget{should-i-refinance-my-mortgage}{%
  \paragraph{Should I refinance my
  mortgage?}\label{should-i-refinance-my-mortgage}}

  \begin{itemize}
  \tightlist
  \item
    \href{https://www.nytimes.com/article/coronavirus-money-unemployment.html?action=click\&pgtype=Article\&state=default\&region=MAIN_CONTENT_3\&context=storylines_faq}{It
    could be a good idea,} because mortgage rates have
    \href{https://www.nytimes.com/2020/07/16/business/mortgage-rates-below-3-percent.html?action=click\&pgtype=Article\&state=default\&region=MAIN_CONTENT_3\&context=storylines_faq}{never
    been lower.} Refinancing requests have pushed mortgage applications
    to some of the highest levels since 2008, so be prepared to get in
    line. But defaults are also up, so if you're thinking about buying a
    home, be aware that some lenders have tightened their standards.
  \end{itemize}
\item ~
  \hypertarget{what-is-school-going-to-look-like-in-september}{%
  \paragraph{What is school going to look like in
  September?}\label{what-is-school-going-to-look-like-in-september}}

  \begin{itemize}
  \tightlist
  \item
    It is unlikely that many schools will return to a normal schedule
    this fall, requiring the grind of
    \href{https://www.nytimes.com/2020/06/05/us/coronavirus-education-lost-learning.html?action=click\&pgtype=Article\&state=default\&region=MAIN_CONTENT_3\&context=storylines_faq}{online
    learning},
    \href{https://www.nytimes.com/2020/05/29/us/coronavirus-child-care-centers.html?action=click\&pgtype=Article\&state=default\&region=MAIN_CONTENT_3\&context=storylines_faq}{makeshift
    child care} and
    \href{https://www.nytimes.com/2020/06/03/business/economy/coronavirus-working-women.html?action=click\&pgtype=Article\&state=default\&region=MAIN_CONTENT_3\&context=storylines_faq}{stunted
    workdays} to continue. California's two largest public school
    districts --- Los Angeles and San Diego --- said on July 13, that
    \href{https://www.nytimes.com/2020/07/13/us/lausd-san-diego-school-reopening.html?action=click\&pgtype=Article\&state=default\&region=MAIN_CONTENT_3\&context=storylines_faq}{instruction
    will be remote-only in the fall}, citing concerns that surging
    coronavirus infections in their areas pose too dire a risk for
    students and teachers. Together, the two districts enroll some
    825,000 students. They are the largest in the country so far to
    abandon plans for even a partial physical return to classrooms when
    they reopen in August. For other districts, the solution won't be an
    all-or-nothing approach.
    \href{https://bioethics.jhu.edu/research-and-outreach/projects/eschool-initiative/school-policy-tracker/}{Many
    systems}, including the nation's largest, New York City, are
    devising
    \href{https://www.nytimes.com/2020/06/26/us/coronavirus-schools-reopen-fall.html?action=click\&pgtype=Article\&state=default\&region=MAIN_CONTENT_3\&context=storylines_faq}{hybrid
    plans} that involve spending some days in classrooms and other days
    online. There's no national policy on this yet, so check with your
    municipal school system regularly to see what is happening in your
    community.
  \end{itemize}
\item ~
  \hypertarget{is-the-coronavirus-airborne}{%
  \paragraph{Is the coronavirus
  airborne?}\label{is-the-coronavirus-airborne}}

  \begin{itemize}
  \tightlist
  \item
    The coronavirus
    \href{https://www.nytimes.com/2020/07/04/health/239-experts-with-one-big-claim-the-coronavirus-is-airborne.html?action=click\&pgtype=Article\&state=default\&region=MAIN_CONTENT_3\&context=storylines_faq}{can
    stay aloft for hours in tiny droplets in stagnant air}, infecting
    people as they inhale, mounting scientific evidence suggests. This
    risk is highest in crowded indoor spaces with poor ventilation, and
    may help explain super-spreading events reported in meatpacking
    plants, churches and restaurants.
    \href{https://www.nytimes.com/2020/07/06/health/coronavirus-airborne-aerosols.html?action=click\&pgtype=Article\&state=default\&region=MAIN_CONTENT_3\&context=storylines_faq}{It's
    unclear how often the virus is spread} via these tiny droplets, or
    aerosols, compared with larger droplets that are expelled when a
    sick person coughs or sneezes, or transmitted through contact with
    contaminated surfaces, said Linsey Marr, an aerosol expert at
    Virginia Tech. Aerosols are released even when a person without
    symptoms exhales, talks or sings, according to Dr. Marr and more
    than 200 other experts, who
    \href{https://academic.oup.com/cid/article/doi/10.1093/cid/ciaa939/5867798}{have
    outlined the evidence in an open letter to the World Health
    Organization}.
  \end{itemize}
\item ~
  \hypertarget{what-are-the-symptoms-of-coronavirus}{%
  \paragraph{What are the symptoms of
  coronavirus?}\label{what-are-the-symptoms-of-coronavirus}}

  \begin{itemize}
  \tightlist
  \item
    Common symptoms
    \href{https://www.nytimes.com/article/symptoms-coronavirus.html?action=click\&pgtype=Article\&state=default\&region=MAIN_CONTENT_3\&context=storylines_faq}{include
    fever, a dry cough, fatigue and difficulty breathing or shortness of
    breath.} Some of these symptoms overlap with those of the flu,
    making detection difficult, but runny noses and stuffy sinuses are
    less common.
    \href{https://www.nytimes.com/2020/04/27/health/coronavirus-symptoms-cdc.html?action=click\&pgtype=Article\&state=default\&region=MAIN_CONTENT_3\&context=storylines_faq}{The
    C.D.C. has also} added chills, muscle pain, sore throat, headache
    and a new loss of the sense of taste or smell as symptoms to look
    out for. Most people fall ill five to seven days after exposure, but
    symptoms may appear in as few as two days or as many as 14 days.
  \end{itemize}
\item ~
  \hypertarget{does-asymptomatic-transmission-of-covid-19-happen}{%
  \paragraph{Does asymptomatic transmission of Covid-19
  happen?}\label{does-asymptomatic-transmission-of-covid-19-happen}}

  \begin{itemize}
  \tightlist
  \item
    So far, the evidence seems to show it does. A widely cited
    \href{https://www.nature.com/articles/s41591-020-0869-5}{paper}
    published in April suggests that people are most infectious about
    two days before the onset of coronavirus symptoms and estimated that
    44 percent of new infections were a result of transmission from
    people who were not yet showing symptoms. Recently, a top expert at
    the World Health Organization stated that transmission of the
    coronavirus by people who did not have symptoms was ``very rare,''
    \href{https://www.nytimes.com/2020/06/09/world/coronavirus-updates.html?action=click\&pgtype=Article\&state=default\&region=MAIN_CONTENT_3\&context=storylines_faq\#link-1f302e21}{but
    she later walked back that statement.}
  \end{itemize}
\end{itemize}

Even as Hong Kong cast the decision as one made for public health
reasons, to curb the spread of the virus, the pro-democracy opposition
has accused the government of using social-distancing rules to clamp
down on the protest movement that began more than a year ago.

\hypertarget{sanofi-and-glaxosmithkline-strike-the-biggest-vaccine-deal-yet-with-the-us-government}{%
\subsection{Sanofi and GlaxoSmithKline strike the biggest vaccine deal
yet with the U.S.
government.}\label{sanofi-and-glaxosmithkline-strike-the-biggest-vaccine-deal-yet-with-the-us-government}}

Image

A production facility of the French pharmaceutical company Sanofi in
Val-de-Reuil, France.Credit...Joel Saget/Agence France-Presse --- Getty
Images

The French drug maker Sanofi said on Friday that it had secured an
agreement of up to \$2.1 billion to supply the U.S. federal government
with 100 million doses of its experimental coronavirus vaccine, the
largest such deal announced to date.

The arrangement with Sanofi and its partner, the British pharmaceutical
company GlaxoSmithKline, brings the Trump administration's investment in
coronavirus vaccine projects
\href{https://medicalcountermeasures.gov/app/barda/coronavirus/COVID19.aspx?filter=vaccine}{to
more than \$8 billion}. This effort, known as Operation Warp Speed, is
placing bets on multiple vaccines and is paying companies to manufacture
millions of doses before clinical trials have been completed.

``The global need for a vaccine to help prevent Covid-19 is massive, and
no single vaccine or company will be able to meet the global demand
alone,'' Thomas Triomphe, executive vice president and global head of
Sanofi Pasteur, the company's vaccine division, said in a statement.

Also on Friday, the European Union said that it was working on a deal
with Sanofi to buy up to 300 million doses of potential vaccines to
distribute to citizens in its 27 member countries. The announcements
came two days after Sanofi said it had
\href{https://www.sanofi.com/en/media-room/press-releases/2020/2020-07-29-07-00-00}{a
deal} with the British government to supply up to 60 million doses of
the vaccine. Financial details about those deals were not disclosed.

Under the U.S. deal, the companies will receive federal funding to pay
for clinical trials as well as for manufacturing the vaccine. The
company expects to begin clinical trials to test for safety in
September, followed by late-stage efficacy trials before the end of this
year. Sanofi said it could apply for regulatory approval in the first
half of next year.

If the vaccine is successful, it would be made available to Americans at
no cost, other than what providers charge to administer it, the U.S.
federal government said in a statement.

The head of Operation Warp Speed, Moncef Slaoui, is a former GSK
executive who
\href{https://www.nytimes.com/2020/05/20/health/coronavirus-vaccine-czar.html}{as
of May held just under \$10 million} in GSK stock. Dr. Slaoui said in an
interview in May that he was determined to avoid any conflicts of
interest, but that his GSK stock represented his retirement from 29
years at the company, and that he had told federal officials he would
not take the job if he had to sell it.

A handful of other vaccine candidates are already in late-stage clinical
trials and some, such as AstraZeneca and Moderna, have said a vaccine
could be ready before the end of this year.

\hypertarget{europes-economic-contraction-is-its-worst-on-record}{%
\subsection{Europe's economic contraction is its worst on
record.}\label{europes-economic-contraction-is-its-worst-on-record}}

Image

The European economy tumbled into its worst recession on record in the
second quarter, as quarantines across the continent brought business,
trade and consumer spending to a grinding halt.

From April to June, gross domestic product fell by 11.9 percent from the
first quarter in the 27 member states of the European Union, and by 12.1
percent in the core group of countries that use the euro currency,
according to figures released on Friday by Eurostat, the E.U. statistics
agency.

On an annualized basis, European Union economies shrank by 14.4 percent,
and eurozone economies by 15 percent, the sharpest contractions since
statistics started being kept in 1995.

Over the same period, the United States economy shrank by 9.5 percent on
the previous quarter and by 32.9 percent on an annual basis,
\href{https://www.nytimes.com/2020/07/30/business/economy/q2-gdp-coronavirus-economy.html}{according
to figures published on Thursday}.

But in Europe, there were signs that the worst may have passed, and that
a tentative recovery has been gaining some traction as governments
unleash enormous stimulus spending. Lengthy lockdowns, while painful for
business and industry, have helped curb a widespread resurgence of the
pandemic in most countries, easing reopening.

The figures were especially grim for nations on Europe's southern rim,
which were among the worst affected by the virus and which faced longer
quarantine periods than northern European countries.

In Spain, which has had one of Europe's highest death tolls, the economy
shrank by a staggering 22.1 percent from a year ago and by 18.5 percent
from the first quarter. France, the eurozone's second-largest economy,
shrank by 19 percent from a year ago and by 13.8 percent from the first
quarter; and Italy, the third-largest economy in the zone, contracted by
17.3 percent from a year ago and by 12.4 percent from the first quarter.
France is officially in recession, with three straight quarters of
contraction.

On Thursday, the authorities reported that the German economy, Europe's
largest, shrank by 11.7 percent from the same period last year and by
10.1 percent from the previous quarter.

European Union leaders last week agreed to
\href{https://www.nytimes.com/2020/07/20/world/europe/eu-stimulus-coronavirus.html}{a
landmark stimulus of 750 billion euros}, or about \$884 billion, to
rescue their economies and to anchor a mild turnaround that had started
to take hold after lockdowns began to be lifted.

But risks abound as surges in new cases are reported, increasing the
possibility of more quarantines.

``The hard part of this recovery is set to start about now,'' Bert
Colijn, senior economist for the eurozone at ING Bank, said in a note to
clients.

\hypertarget{the-house-oversight-committee-charged-that-the-trump-administration-wasted-500-million-by-overpaying-for-ventilators}{%
\subsection{The House Oversight Committee charged that the Trump
administration wasted \$500 million by overpaying for
ventilators.}\label{the-house-oversight-committee-charged-that-the-trump-administration-wasted-500-million-by-overpaying-for-ventilators}}

The Trump administration wasted around \$500 million by overpaying for
ventilators through negotiations that were ``inept,'' a panel of the
House Oversight and Reform Committee said in a report released Friday.

``The American people got ripped off, and Donald Trump and his team got
taken to the cleaners,'' said Rep. Raja Krishnamoorthi, Democrat of
Illinois and the chairman of the panel's economic and consumer policy
subcommittee. ``The Trump administration's mishandling of ventilator
procurement for the nation's stockpile cost the American people dearly
during the worst public health crisis of our generation.''

The report faulted Peter Navarro, Mr. Trump's top trade adviser, and
Jared Kushner, his son-in-law and senior adviser, for negotiating a deal
to acquire ventilators quickly in which the panel said they paid almost
five times the price per device than under a previous contract with the
same vendor.

``The Trump negotiators
\href{https://oversight.house.gov/sites/democrats.oversight.house.gov/files/1122-1125_Redacted.pdf}{appeared}
gullible and conceded to Philips on all significant matters, including
price,'' the report said, referring to Philips North America
Corporation, which had a federal contract to supply ventilators to the
national stockpile. ``The
\href{https://oversight.house.gov/sites/democrats.oversight.house.gov/files/298-299_Redacted.pdf}{documents}
show that the administration accepted Philips' first offer without even
trying to
\href{https://oversight.house.gov/sites/democrats.oversight.house.gov/files/1251-1280_Redacted.pdf}{negotiate}
a lower price.''

The committee launched an investigation in April to determine why the
country was without much-needed ventilators during the initial months of
the coronavirus pandemic.

In January, Philips approached the Trump administration about speeding
up the long-delayed delivery of ventilators it had agreed to supply, but
the administration failed to respond to for six weeks, the panel found.
When it did, the report said, rather than insist on the delivery of the
devices by the deadlines in its original contract, officials led by Mr.
Navarro and Mr. Kushner negotiated a new deal at an inflated rate.

\hypertarget{nyc-sets-a-positivity-rate-threshold-for-reopening-schools-and-a-strategy-if-someone-tests-positive-at-school-1}{%
\subsection{N.Y.C. sets a positivity rate threshold for reopening
schools and a strategy if someone tests positive at
school.}\label{nyc-sets-a-positivity-rate-threshold-for-reopening-schools-and-a-strategy-if-someone-tests-positive-at-school-1}}

Image

P.S. 241 in the Crown Heights neighborhood of Brooklyn closed in April
because of the pandemic.Credit...Kirsten Luce for The New York Times

New York City public schools, the nation's largest school system, will
be able to reopen its school buildings in September only if the city
maintains a test positivity rate below 3 percent, Mayor Bill de Blasio
announced on Friday. That conservative threshold is even lower than the
5 percent test positivity rate which has been set by Gov. Andrew M.
Cuomo as
\href{https://www.nytimes.com/2020/07/14/us/coronavirus-schools-fall.html}{a
cut-off for school reopening and recommended by public health experts}.

The average positivity for New York City has generally remained lower
even than the new city threshold, according to city and state figures.
But even a modest uptick in cases over the next few weeks could nudge
that rate closer to the new threshold, which raises fresh questions
about
\href{https://www.nytimes.com/2020/07/08/nyregion/nyc-schools-reopening-plan.html}{whether
city schools will open part-time on Sept. 10 as planned in a hybrid
model}.

``I want to set that very very tough standard,'' the mayor said, adding,
``this is a way of proving that we will do things the right way.''

New York is one of the only large districts in the country that is
currently planning to reopen its buildings at all: Children will report
to school one to three days a week to allow for social distancing. All
staff members will be asked to take tests before the start of school,
with expedited results. Education officials in the city laid out a plan
on Thursday for what would happen in the seemingly inevitable event that
cases are confirmed in a classroom.

The protocol means it is likely that at many of the city's 1,800
schools, individual classrooms or even entire buildings will be closed
at points during the school year.

Officials said confirmed infections among students, teachers and staff
members would be treated the same. One or two cases in a single
classroom would require those classes to close for 14 days; all students
and staff members in that classroom would be ordered to self-quarantine,
and students would learn remotely. The rest of the school would continue
to operate.

But if two or more people in different classrooms in the same school
tested positive, the entire building would close while disease
detectives from the city's Department of Health were brought in to
investigate the cases, which could take several days. Depending on the
results of the investigation, the building could reopen, but the
classrooms with positive cases would remain closed for 14 days.

If disease detectives were not able to find a link between two or more
confirmed cases in a building, including exposure to the virus outside
of school, the entire building would remain shuttered for two weeks.

Mr. de Blasio's
administration\href{https://www.nytimes.com/2020/07/06/nyregion/nyc-school-reopening-plan.html}{faced
enormous criticism for waiting until mid-March to close schools}, after
the virus had already begun to spread significantly throughout the city,
which soon became a global center of the crisis. Throughout March, when
a student or staff member tested positive, the school would
automatically close for 24 hours for cleaning, a protocol that many
parents and teachers said was too lax.

Other states, including California, have announced less stringent
policies for how to manage positive cases in schools. But most
California school districts will begin the academic year exclusively
online because of the high numbers of cases in their communities.

\subsection{}

\hypertarget{here-are-5-key-developments-you-may-have-missed}{%
\subsection{Here are 5 key developments you may have
missed.}\label{here-are-5-key-developments-you-may-have-missed}}

\begin{itemize}
\item
  The pandemic's toll on businesses in the United States became
  emphatically clearer as the government detailed the most devastating
  three-month \textbf{economic collapse} on record, which wiped away
  nearly five years of growth.
  \href{https://www.nytimes.com/2020/07/30/business/economy/q2-gdp-coronavirus-economy.html}{Read
  more on the economic crisis}.
\item
  Mr. Trump, whose unsteady handling of the virus has left him trailing
  in the polls, floated the idea of changing the date of the
  \textbf{2020 election} --- a suggestion he has no authority to enact,
  and which instantly drew rare rebukes from top Republicans.
  \href{https://www.nytimes.com/2020/07/30/us/politics/trump-delay-2020-election.html?action=click\&module=Top\%20Stories\&pgtype=Homepage}{Read
  more on Mr. Trump's words and the reaction to them}.
\item
  U.S. lawmakers failed to extend \textbf{jobless benefits} that are
  expiring today. On Thursday, the Senate dissolved into partisan
  bickering over a sweeping economic stabilization package, clashing
  over dueling proposals. Tens of millions of Americans have depended on
  the \$600-a-week unemployment aid for months.
  \href{https://www.nytimes.com/2020/07/30/us/politics/senate-virus-aid.html}{Read
  more about the impasse}.
\item
  \textbf{Herman Cain}, who ran for the 2012 Republican presidential
  nomination and was a recent contender for a top Federal Reserve job,
  died after being hospitalized with the coronavirus.
  \href{https://www.nytimes.com/2020/07/30/us/politics/herman-cain-dead.html}{Read
  Mr. Cain's obituary}.
\item
  Cases in \textbf{New Jersey} are rising again: On Friday, there were
  699 new additional cases, according to the state, sending the week's
  average number of daily cases to 550, well above the rate from a month
  ago.

  ``The numbers are setting off alarms,'' Gov. Philip D. Murphy said at
  a briefing on Friday. ``We are standing in a very dangerous place.''

  The increase has worried elected leaders and public health officials
  who say that young people who are enjoying summer parties are not
  taking enough precautions.
  \href{https://www.nytimes.com/2020/07/30/nyregion/coronavirus-cases-nj.html}{Read
  more about the uptick in cases in New Jersey}.
\item
  \textbf{Florida} broke a record --- the most deaths reported in a
  single day --- for the fourth day in a row: On Friday the state
  announced 257 additional fatalities.
\end{itemize}

The French drug maker Sanofi said on Friday that it had
\href{https://www.nytimes.com/2020/07/31/health/covid-19-vaccine-sanofi-gsk.html}{secured
an agreement of up to \$2.1 billion} to supply the United States
government with 100 million doses of its experimental coronavirus
vaccine, the largest such deal announced to date.

The arrangement brings the Trump administration's investment in
coronavirus vaccine projects
\href{https://medicalcountermeasures.gov/app/barda/coronavirus/COVID19.aspx?filter=vaccine}{to
more than \$8 billion}. This sprawling, multiagency effort, known as
Operation Warp Speed, is placing bets on multiple vaccines and is paying
companies to manufacture millions of doses before clinical trials have
been completed.

``The global need for a vaccine to help prevent Covid-19 is massive, and
no single vaccine or company will be able to meet the global demand
alone,'' Thomas Triomphe, executive vice president and global head of
Sanofi Pasteur, the company's vaccine division, said in a statement.

Under the deal announced, Sanofi and its partner, the British
pharmaceutical company GlaxoSmithKline, will receive federal funding to
pay for clinical trials as well as for manufacturing the vaccine. Sanofi
said the deal also includes an option for the company to supply an
additional 500 million doses. The company expects to begin clinical
trials to test for safety in September, followed by late-stage efficacy
trials before the end of this year. Sanofi said it could apply for
regulatory approval in the first half of next year.

If the vaccine is successful, it would be made available to Americans at
no cost, other than what providers charge to administer it, the federal
government said in a statement.

The head of Operation Warp Speed, Moncef Slaoui, is a former GSK
executive who
\href{https://www.nytimes.com/2020/05/20/health/coronavirus-vaccine-czar.html}{as
of May held just under \$10 million} in GSK stock. Dr. Slaoui's
financial ties to some of the companies that are pursuing coronavirus
vaccines
\href{https://www.nytimes.com/2020/07/15/us/politics/vaccine-Slaoui-coronavirus-trump.html}{have
raised questions} about conflicts of interest.

Sanofi and GSK did not say how much of the federal money would go to
each company --- only that Sanofi would receive the most. GSK did not
comment on whether Dr. Slaoui had recused himself from negotiations over
the deal. A senior administration official said all agreements were
negotiated by federal ``acquisition professionals'' and that Dr. Slaoui
did not play a role in the negotiations.

Global roundup

\hypertarget{britain-halts-reopenings-and-brings-in-new-restrictions-for-millions-after-signs-of-increasing-spread}{%
\subsection{Britain halts reopenings and brings in new restrictions for
millions after signs of increasing
spread.}\label{britain-halts-reopenings-and-brings-in-new-restrictions-for-millions-after-signs-of-increasing-spread}}

Image

In Manchester, England, on Friday. Different households in the city
aren't allowed to meet at home or in a private garden, with potential
fines of up to 100 pounds, about \$130.Credit...Phil Noble/Reuters

Britain has barred millions of people in northern England from meeting
other households at their homes, paused reopenings set for Aug. 1 and
moved to make face masks mandatory in more places, after a day on which
it reported 38
\href{https://www.nytimes.com/interactive/2020/world/europe/united-kingdom-coronavirus-cases.html}{new
coronavirus deaths} and nearly 900 known new infections, its highest
case numbers in a month.

At a news conference on Friday, Prime Minister Boris Johnson said he had
promised to put on the brakes at any sign of an increase in cases, and
added: ``Our assessment is that we should now squeeze that brake
pedal.''

Britain --- which has suffered
\href{https://www.nytimes.com/2020/07/30/world/europe/UK-deaths-coronavirus-europe.html}{Europe's
worst coronavirus outbreak}, with nearly 56,000 confirmed deaths --- has
been gradually easing restrictions, with
\href{https://www.nytimes.com/2020/06/23/world/europe/uk-coronavirus-reopening.html}{pubs,
restaurants, museums and hair salons} allowed to reopen early this
month.

On Saturday, the government had planned to allow reopening of
higher-risk settings in England including casinos, bowling alleys and
skating rinks, and to permit small wedding receptions and some indoor
performances. All that will now be pushed back until at least August 15.

``We simply cannot take the risk,'' Mr. Johnson said. ``We will of
course study the data carefully and pay attention to open up as soon as
we can.''

Measures to encourage more people to return to their places of work
would go ahead, he said. Mr. Johnson also added that mask wearing,
already mandatory in shops and supermarkets in England, would be
extended to include more indoor settings where social distancing was not
an option.

The moves came suddenly, with the restrictions in northern England
implemented at midnight, less than three hours after
\href{https://twitter.com/MattHancock/status/1288931464168591371}{the
authorities' initial announcement} Thursday night, and with
\href{https://www.gov.uk/guidance/north-west-of-england-local-restrictions-what-you-can-and-cannot-do}{official
guidance on what the rules cover} not published until the following
morning.

Those restrictions affect Manchester and its surrounding towns and
suburbs, plus areas in East Lancashire and West Yorkshire.The
announcement came just before Eid al-Adha, and several of the affected
areas have large Muslim communities. Places of worship will remain open
with social distancing measures but the authorities recommended praying
outdoors.

Here are other developments from around the globe:

\begin{itemize}
\tightlist
\item
  \textbf{Vietnam}, which has been
  \href{https://www.nytimes.com/2020/07/29/world/asia/coronavirus-vietnam.html}{fighting
  a fresh virus outbreak} after more than three months without reporting
  a locally transmitted case, has announced its first death from the
  coronavirus. The victim was a 70-year-old resident of the city of Hoi
  An who had been living with kidney disease for more than a decade. The
  man was admitted to a hospital on July 9 with chest tightness and
  fatigue, and tested positive for the virus on Sunday. He died Friday
  morning.
\end{itemize}

\begin{itemize}
\tightlist
\item
  On Friday, \textbf{Japan} announced 1,305 new cases, breaking a record
  set the day before. As cases spike in Tokyo, Gov. Yuriko Koike has
  requested that karaoke venues and bars and restaurants serving alcohol
  close by 10 p.m. from Aug. 3 through the end of the month. Businesses
  that cooperate will be offered 200,000 yen, or about \$1,900.
\end{itemize}

\hypertarget{many-african-countries-are-testing-too-little-to-track-the-virus-an-aid-group-warns}{%
\subsection{Many African countries are testing too little to track the
virus, an aid group
warns.}\label{many-african-countries-are-testing-too-little-to-track-the-virus-an-aid-group-warns}}

Image

A testing site in Bujumbura,~Burundi, this month. The country has one of
the lowest testing rates in Africa at 563 screenings per million
people.Credit...Berthier Mugiraneza/Associated Press

A stark lack of testing in many African countries has kept officials
from being able to track the pandemic, prompting fears that a recent
surge in cases across the continent may be just the ``tip of the
iceberg,'' according to the International Rescue Committee.

Each country in Africa where the committee works has conducted fewer
than 8,000 tests per million people, the group said. By contrast,
Britain has conducted 205,782 tests per million, the United Arab
Emirates 472,590 per million, and Singapore 199,904 per million, the
committee said.

The committee cited Tanzania (63 tests per million), Niger (373 tests
per million), Chad (383 tests per million), Democratic Republic of Congo
(467 tests per million) and Burundi (563 tests per million) as having
the lowest testing rates among the African countries where it works.

The committee, a global humanitarian aid organization, said that testing
in many African countries was falling far short of the rate of at least
one test per 1,000 people per week recommended by the World Health
Organization.

The organization said many African nations needed international support
to increase their testing capacity or the continent could face ``an
undetected and uncontrolled spread --- and a response fighting with a
hand tied behind its back.''

``The testing shortfalls make it nearly impossible to understand the
extent of the pandemic --- let alone put measures in place to stop it,''
Stacey Mearns, a senior technical adviser on emergency health at the
committee, said in a statement.

\hypertarget{native-americans-appear-to-be-hit-particularly-hard-by-the-virus-though-the-data-is-incomplete}{%
\subsection{Native Americans appear to be hit particularly hard by the
virus, though the data is
incomplete.}\label{native-americans-appear-to-be-hit-particularly-hard-by-the-virus-though-the-data-is-incomplete}}

Image

A food giveaway by the Peacekeeper Society,~a tribal nonprofit that
works around the Yakama Nation reservation in Washington
State.Credit...Mason Trinca for The New York Times

Even with significant gaps in the available data, there are strong
indications that Native American people
\href{https://www.nytimes.com/2020/07/30/us/native-americans-coronavirus-data.html}{have
been disproportionately affected by the virus}.

The rate of known cases in the eight counties with the largest
populations of Native Americans is nearly double the national average, a
New York Times analysis has found. The analysis cannot determine which
individuals are testing positive for the virus, but these counties are
home to one in six U.S. residents who describe themselves in census
surveys as non-Hispanic and American Indian or Alaska Native.

And there are many other smaller counties with significant populations
of Native Americans that have elevated case rates, including Yakima
County, Wash. The Times identified at least 15 counties that have
elevated case rates and are home to sizable numbers of Native American
residents, ranging from large metropolitan areas in Arizona to rural
communities in Nebraska and Mississippi.

``I feel as though tribal nations have an effective death sentence when
the scale of this pandemic, if it continues to grow, exceeds the public
resources available,'' said Fawn Sharp, the president of the Quinault
Indian Nation and of the National Congress of American Indians.

The trends are troubling enough that congressional leaders have asked
the
\href{https://www.warren.senate.gov/newsroom/press-releases/us-commission-on-civil-rights-agrees-to-warren-haaland-request-to-update-broken-promises-report-and-examine-pandemic-impacts-on-indian-country}{U.S.
Commission on Civil Rights} to examine them.

In New Mexico, Native American and Alaska Native people have accounted
for \href{https://cvprovider.nmhealth.org/public-dashboard.html}{nearly
40 percent of virus cases,} though they make up 9 percent of the
population.

\href{https://www.cdc.gov/coronavirus/2019-ncov/covid-data/covidview/index.html}{Hospitalization
rates published} by the Centers for Disease Control and Prevention also
suggest that Native American people are overrepresented among those who
become seriously ill from the virus. Federal data tracking individual
coronavirus cases often omits race and ethnicity information.

Native Americans --- particularly those living on reservations --- are
more prone to contract the virus because of crowded housing conditions
that make social distancing difficult, said Allison Barlow, director of
the Center for American Indian Health at Johns Hopkins University. And
years of underfunded health systems, food and water insecurity and other
factors contribute to underlying health conditions that can make the
illness more severe once contracted.

Reporting was contributed by Liz Alderman, Ian Austen, Luke Broadwater,
Julia Calderone, Emily Cochrane, Kate Conger, Michael Cooper, Michael
Crowley, Johnny Diaz, Robert Gebeloff, Erica L. Green, Jan Hoffman,
Rebecca Halleck, Jan Hoffman, Shawn Hubler, Mike Ives, Michael Levenson,
Giulia McDonnell Nieto del Rio, Eshe Nelson, Richard A. Oppel Jr.,
Richard C. Paddock, Elian Peltier, Matt Phillips, Austin Ramzy, Motoko
Rich, Amanda Rosa, Eliza Shapiro, Megan Specia, Sheryl Gay Stolberg,
Eileen Sullivan, Katie Thomas, Tracey Tulley, Hisako Ueno, Neil Vigdor,
Katherine J. Wu ****** and Mihir Zaveri.

Advertisement

\protect\hyperlink{after-bottom}{Continue reading the main story}

\hypertarget{site-index}{%
\subsection{Site Index}\label{site-index}}

\hypertarget{site-information-navigation}{%
\subsection{Site Information
Navigation}\label{site-information-navigation}}

\begin{itemize}
\tightlist
\item
  \href{https://help.nytimes.com/hc/en-us/articles/115014792127-Copyright-notice}{©~2020~The
  New York Times Company}
\end{itemize}

\begin{itemize}
\tightlist
\item
  \href{https://www.nytco.com/}{NYTCo}
\item
  \href{https://help.nytimes.com/hc/en-us/articles/115015385887-Contact-Us}{Contact
  Us}
\item
  \href{https://www.nytco.com/careers/}{Work with us}
\item
  \href{https://nytmediakit.com/}{Advertise}
\item
  \href{http://www.tbrandstudio.com/}{T Brand Studio}
\item
  \href{https://www.nytimes.com/privacy/cookie-policy\#how-do-i-manage-trackers}{Your
  Ad Choices}
\item
  \href{https://www.nytimes.com/privacy}{Privacy}
\item
  \href{https://help.nytimes.com/hc/en-us/articles/115014893428-Terms-of-service}{Terms
  of Service}
\item
  \href{https://help.nytimes.com/hc/en-us/articles/115014893968-Terms-of-sale}{Terms
  of Sale}
\item
  \href{https://spiderbites.nytimes.com}{Site Map}
\item
  \href{https://help.nytimes.com/hc/en-us}{Help}
\item
  \href{https://www.nytimes.com/subscription?campaignId=37WXW}{Subscriptions}
\end{itemize}
