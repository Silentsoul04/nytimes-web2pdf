Sections

SEARCH

\protect\hyperlink{site-content}{Skip to
content}\protect\hyperlink{site-index}{Skip to site index}

\href{https://www.nytimes.com/section/your-money}{Your Money}

\href{https://myaccount.nytimes.com/auth/login?response_type=cookie\&client_id=vi}{}

\href{https://www.nytimes.com/section/todayspaper}{Today's Paper}

\href{/section/your-money}{Your Money}\textbar{}Can't Afford a Birkin
Bag or a Racehorse? You Can Invest in One

\url{https://nyti.ms/2PpbPA9}

\begin{itemize}
\item
\item
\item
\item
\item
\end{itemize}

\href{https://www.nytimes.com/news-event/coronavirus?action=click\&pgtype=Article\&state=default\&region=TOP_BANNER\&context=storylines_menu}{The
Coronavirus Outbreak}

\begin{itemize}
\tightlist
\item
  live\href{https://www.nytimes.com/2020/08/04/world/coronavirus-cases.html?action=click\&pgtype=Article\&state=default\&region=TOP_BANNER\&context=storylines_menu}{Latest
  Updates}
\item
  \href{https://www.nytimes.com/interactive/2020/us/coronavirus-us-cases.html?action=click\&pgtype=Article\&state=default\&region=TOP_BANNER\&context=storylines_menu}{Maps
  and Cases}
\item
  \href{https://www.nytimes.com/interactive/2020/science/coronavirus-vaccine-tracker.html?action=click\&pgtype=Article\&state=default\&region=TOP_BANNER\&context=storylines_menu}{Vaccine
  Tracker}
\item
  \href{https://www.nytimes.com/2020/08/02/us/covid-college-reopening.html?action=click\&pgtype=Article\&state=default\&region=TOP_BANNER\&context=storylines_menu}{College
  Reopening}
\item
  \href{https://www.nytimes.com/live/2020/08/04/business/stock-market-today-coronavirus?action=click\&pgtype=Article\&state=default\&region=TOP_BANNER\&context=storylines_menu}{Economy}
\end{itemize}

Advertisement

\protect\hyperlink{after-top}{Continue reading the main story}

Supported by

\protect\hyperlink{after-sponsor}{Continue reading the main story}

Wealth Matters

\hypertarget{cant-afford-a-birkin-bag-or-a-racehorse-you-can-invest-in-one}{%
\section{Can't Afford a Birkin Bag or a Racehorse? You Can Invest in
One}\label{cant-afford-a-birkin-bag-or-a-racehorse-you-can-invest-in-one}}

Interest in fractional investments has grown as the pandemic has forced
more people to spend time at home, but advisers say the strategy has
risks.

\includegraphics{https://static01.nyt.com/images/2020/08/01/business/01Wealth-01/31Wealth-01-articleLarge.jpg?quality=75\&auto=webp\&disable=upscale}

By \href{https://www.nytimes.com/by/paul-sullivan}{Paul Sullivan}

\begin{itemize}
\item
  July 31, 2020
\item
  \begin{itemize}
  \item
  \item
  \item
  \item
  \item
  \end{itemize}
\end{itemize}

Antonella Carbonaro, a consultant to financial technology companies,
saved up to buy her Birkin bag, a luxury tote made by Hermès that sells
new for tens of thousands of dollars. Since getting her bag in 2018, Ms.
Carbonaro has stored it in her closet, bringing it out only on special
occasions.

But when she heard that there was a marketplace to buy shares in other
Birkins, including more exotic versions that can fetch six figures, she
was in. It is not a lark. Ms. Carbonaro, 30, sees her shares in an
exclusive bag as an alternative investment, no different than stakes in
private equity funds that invest in a basket of companies.

``This is a visual way to participate in different asset classes that
aren't as accessible,'' Ms. Carbonaro said. ``Investing in shares of
Birkin bags, even though I have one, is getting more exposure.''

She bought 10 shares in a Bleu Lézard Birkin bag that was valued at
\$61,500 in an offering last year. Earlier this year, she bought 25
shares in a gray Himalaya Birkin. It was valued at \$140,000 in an
offering in May.

Unlike owning a fractional share of a condominium, she will never be
able to use her investment. Shares are traded until the owner of the
marketplace sells the asset.

Ms. Carbonaro's first Birkin investment is trading up 6 percent from the
purchase price on Rally Rd., a platform that deals in fractional
investments in collectible items. The other one is still in the lockup
period and its shares cannot be traded yet.

The market for investing in fractions of items otherwise seen as
collectibles --- and largely reserved for the wealthiest people --- has
seen an uptick in interest during
\href{https://www.nytimes.com/news-event/coronavirus}{the pandemic} as
people spend more time at home.

\includegraphics{https://static01.nyt.com/images/2020/07/31/business/31Wealth-birkin-himalaya/31Wealth-birkin-himalaya-articleLarge.jpg?quality=75\&auto=webp\&disable=upscale}

Rally Rd. began by selling shares in exotic cars several years ago but
has expanded to art, books, wine and whiskey, memorabilia and Birkin
bags.

``In the beginning, it was like equity markets: just safe, blue-chip
investments,'' said Rob Petrozzo, a founder and the chief product
officer at Rally Rd. ``Over the past few months, we've seen with people
being inside, they've gotten access to more information and they have
been exploring the app more fully.''

\hypertarget{latest-updates-global-coronavirus-outbreak}{%
\section{\texorpdfstring{\href{https://www.nytimes.com/2020/08/04/world/coronavirus-cases.html?action=click\&pgtype=Article\&state=default\&region=MAIN_CONTENT_1\&context=storylines_live_updates}{Latest
Updates: Global Coronavirus
Outbreak}}{Latest Updates: Global Coronavirus Outbreak}}\label{latest-updates-global-coronavirus-outbreak}}

Updated 2020-08-04T20:57:54.346Z

\begin{itemize}
\tightlist
\item
  \href{https://www.nytimes.com/2020/08/04/world/coronavirus-cases.html?action=click\&pgtype=Article\&state=default\&region=MAIN_CONTENT_1\&context=storylines_live_updates\#link-1228a480}{Novavax
  sees encouraging results from two studies of its experimental
  vaccine.}
\item
  \href{https://www.nytimes.com/2020/08/04/world/coronavirus-cases.html?action=click\&pgtype=Article\&state=default\&region=MAIN_CONTENT_1\&context=storylines_live_updates\#link-4825b93}{Public
  and private schools in Maryland and elsewhere are divided over
  in-person instruction.}
\item
  \href{https://www.nytimes.com/2020/08/04/world/coronavirus-cases.html?action=click\&pgtype=Article\&state=default\&region=MAIN_CONTENT_1\&context=storylines_live_updates\#link-50f7386d}{The
  United Nations calls on policymakers to `plan thoroughly for school
  reopenings.'}
\end{itemize}

\href{https://www.nytimes.com/2020/08/04/world/coronavirus-cases.html?action=click\&pgtype=Article\&state=default\&region=MAIN_CONTENT_1\&context=storylines_live_updates}{See
more updates}

More live coverage:
\href{https://www.nytimes.com/live/2020/08/04/business/stock-market-today-coronavirus?action=click\&pgtype=Article\&state=default\&region=MAIN_CONTENT_1\&context=storylines_live_updates}{Markets}

He said existing investors on the platform had doubled the number of
items they owned shares in. Initial offerings have sold out five times
faster than before the pandemic, as new investors on the platform began
buying up shares more quickly.

To accommodate growing interest, MyRacehorse, which sells shares in
racehorses that are far smaller stakes than those sold by traditional
racing syndicates, has partnered with a top stud farm, Spendthrift, to
extend the length of the investments. Before, its model had been to sell
the horse when it was done racing. Now, investors can participate in the
breeding fees, which can be many times any racetrack winnings.

The fractional movement is not limited to luxury items. Fidelity, the
mutual fund giant, offers ``stocks by the slice'' where you can buy a
portion of a share starting at \$1. And many private equity funds, which
have high minimum investments and long lockup-periods, have created
mutual fund versions of their funds.

Eugene Olmstead, a retired internet technology executive, said he had 1
percent to 1.5 percent in 11 horses, all bought through his
self-directed individual retirement account.

``You're not going to get a worthwhile return on your investment unless
you have a certain percentage,'' said Mr. Olmstead, 58. ``I've done my
research, and I'm investing in ones that I think in the long run will
give me a decent return.''

Of the 11 horses he has bought shares in, only two are old enough to
race. He said both had average winnings of \$12,000 a race. He has
received some dividends from those races, but said the money was not
substantial yet.

``It's money I don't need right now,'' he said. ``It gives me a chance
to wait for those returns.''

Another owner of fractional shares in horses, David Falo, 58, compared
buying stakes in young horses to investing in companies on private
platforms before their initial public offering. ``The horse may not do
well, or the horse could get injured,'' he said, ``but it gives you a
little thrill along the way.''

Image

From left, Max Niederste-Ostholt, Rob Petrozzo and Chris Bruno of Rally
Rd. Mr. Petrozzo said that existing investors on the platform had
doubled the number of items they owned shares in.Credit...Jeenah Moon
for The New York Times

There are many caveats. Trading through Rally Rd. and MyRacehorse are
done through apps, which makes buying and selling easier and creates a
community. But the apps turn investing into games, as has happened with
\href{https://www.nytimes.com/2020/07/08/technology/robinhood-risky-trading.html}{the
stock-trading app Robinhood}. That can distort the financial
consequences of ill-considered investments.

Compounding the risk, an asset typically bought for personal enjoyment
or bragging rights cannot be analyzed the same way that a private equity
investment would be.

``There could be return potential, but who knows?'' said Jack Ablin,
chief investment officer of Cresset Capital. ``There's no liquidity and
no control. When do you get your money back? You don't know. The other
is the carrying costs could be high.''

In the case of the shares in the racehorses, expenses like training and
boarding are shared just as profits are. ``You own full equity in the
horse,'' said Michael Behrens, founder of MyRacehorse.

\href{https://www.nytimes.com/news-event/coronavirus?action=click\&pgtype=Article\&state=default\&region=MAIN_CONTENT_3\&context=storylines_faq}{}

\hypertarget{the-coronavirus-outbreak-}{%
\subsubsection{The Coronavirus Outbreak
›}\label{the-coronavirus-outbreak-}}

\hypertarget{frequently-asked-questions}{%
\paragraph{Frequently Asked
Questions}\label{frequently-asked-questions}}

Updated August 4, 2020

\begin{itemize}
\item ~
  \hypertarget{i-have-antibodies-am-i-now-immune}{%
  \paragraph{I have antibodies. Am I now
  immune?}\label{i-have-antibodies-am-i-now-immune}}

  \begin{itemize}
  \tightlist
  \item
    As of right
    now,\href{https://www.nytimes.com/2020/07/22/health/covid-antibodies-herd-immunity.html?action=click\&pgtype=Article\&state=default\&region=MAIN_CONTENT_3\&context=storylines_faq}{that
    seems likely, for at least several months.} There have been
    frightening accounts of people suffering what seems to be a second
    bout of Covid-19. But experts say these patients may have a
    drawn-out course of infection, with the virus taking a slow toll
    weeks to months after initial exposure. People infected with the
    coronavirus typically
    \href{https://www.nature.com/articles/s41586-020-2456-9}{produce}
    immune molecules called antibodies, which are
    \href{https://www.nytimes.com/2020/05/07/health/coronavirus-antibody-prevalence.html?action=click\&pgtype=Article\&state=default\&region=MAIN_CONTENT_3\&context=storylines_faq}{protective
    proteins made in response to an
    infection}\href{https://www.nytimes.com/2020/05/07/health/coronavirus-antibody-prevalence.html?action=click\&pgtype=Article\&state=default\&region=MAIN_CONTENT_3\&context=storylines_faq}{.
    These antibodies may} last in the body
    \href{https://www.nature.com/articles/s41591-020-0965-6}{only two to
    three months}, which may seem worrisome, but that's perfectly normal
    after an acute infection subsides, said Dr. Michael Mina, an
    immunologist at Harvard University. It may be possible to get the
    coronavirus again, but it's highly unlikely that it would be
    possible in a short window of time from initial infection or make
    people sicker the second time.
  \end{itemize}
\item ~
  \hypertarget{im-a-small-business-owner-can-i-get-relief}{%
  \paragraph{I'm a small-business owner. Can I get
  relief?}\label{im-a-small-business-owner-can-i-get-relief}}

  \begin{itemize}
  \tightlist
  \item
    The
    \href{https://www.nytimes.com/article/small-business-loans-stimulus-grants-freelancers-coronavirus.html?action=click\&pgtype=Article\&state=default\&region=MAIN_CONTENT_3\&context=storylines_faq}{stimulus
    bills enacted in March} offer help for the millions of American
    small businesses. Those eligible for aid are businesses and
    nonprofit organizations with fewer than 500 workers, including sole
    proprietorships, independent contractors and freelancers. Some
    larger companies in some industries are also eligible. The help
    being offered, which is being managed by the Small Business
    Administration, includes the Paycheck Protection Program and the
    Economic Injury Disaster Loan program. But lots of folks have
    \href{https://www.nytimes.com/interactive/2020/05/07/business/small-business-loans-coronavirus.html?action=click\&pgtype=Article\&state=default\&region=MAIN_CONTENT_3\&context=storylines_faq}{not
    yet seen payouts.} Even those who have received help are confused:
    The rules are draconian, and some are stuck sitting on
    \href{https://www.nytimes.com/2020/05/02/business/economy/loans-coronavirus-small-business.html?action=click\&pgtype=Article\&state=default\&region=MAIN_CONTENT_3\&context=storylines_faq}{money
    they don't know how to use.} Many small-business owners are getting
    less than they expected or
    \href{https://www.nytimes.com/2020/06/10/business/Small-business-loans-ppp.html?action=click\&pgtype=Article\&state=default\&region=MAIN_CONTENT_3\&context=storylines_faq}{not
    hearing anything at all.}
  \end{itemize}
\item ~
  \hypertarget{what-are-my-rights-if-i-am-worried-about-going-back-to-work}{%
  \paragraph{What are my rights if I am worried about going back to
  work?}\label{what-are-my-rights-if-i-am-worried-about-going-back-to-work}}

  \begin{itemize}
  \tightlist
  \item
    Employers have to provide
    \href{https://www.osha.gov/SLTC/covid-19/standards.html}{a safe
    workplace} with policies that protect everyone equally.
    \href{https://www.nytimes.com/article/coronavirus-money-unemployment.html?action=click\&pgtype=Article\&state=default\&region=MAIN_CONTENT_3\&context=storylines_faq}{And
    if one of your co-workers tests positive for the coronavirus, the
    C.D.C.} has said that
    \href{https://www.cdc.gov/coronavirus/2019-ncov/community/guidance-business-response.html}{employers
    should tell their employees} -\/- without giving you the sick
    employee's name -\/- that they may have been exposed to the virus.
  \end{itemize}
\item ~
  \hypertarget{should-i-refinance-my-mortgage}{%
  \paragraph{Should I refinance my
  mortgage?}\label{should-i-refinance-my-mortgage}}

  \begin{itemize}
  \tightlist
  \item
    \href{https://www.nytimes.com/article/coronavirus-money-unemployment.html?action=click\&pgtype=Article\&state=default\&region=MAIN_CONTENT_3\&context=storylines_faq}{It
    could be a good idea,} because mortgage rates have
    \href{https://www.nytimes.com/2020/07/16/business/mortgage-rates-below-3-percent.html?action=click\&pgtype=Article\&state=default\&region=MAIN_CONTENT_3\&context=storylines_faq}{never
    been lower.} Refinancing requests have pushed mortgage applications
    to some of the highest levels since 2008, so be prepared to get in
    line. But defaults are also up, so if you're thinking about buying a
    home, be aware that some lenders have tightened their standards.
  \end{itemize}
\item ~
  \hypertarget{what-is-school-going-to-look-like-in-september}{%
  \paragraph{What is school going to look like in
  September?}\label{what-is-school-going-to-look-like-in-september}}

  \begin{itemize}
  \tightlist
  \item
    It is unlikely that many schools will return to a normal schedule
    this fall, requiring the grind of
    \href{https://www.nytimes.com/2020/06/05/us/coronavirus-education-lost-learning.html?action=click\&pgtype=Article\&state=default\&region=MAIN_CONTENT_3\&context=storylines_faq}{online
    learning},
    \href{https://www.nytimes.com/2020/05/29/us/coronavirus-child-care-centers.html?action=click\&pgtype=Article\&state=default\&region=MAIN_CONTENT_3\&context=storylines_faq}{makeshift
    child care} and
    \href{https://www.nytimes.com/2020/06/03/business/economy/coronavirus-working-women.html?action=click\&pgtype=Article\&state=default\&region=MAIN_CONTENT_3\&context=storylines_faq}{stunted
    workdays} to continue. California's two largest public school
    districts --- Los Angeles and San Diego --- said on July 13, that
    \href{https://www.nytimes.com/2020/07/13/us/lausd-san-diego-school-reopening.html?action=click\&pgtype=Article\&state=default\&region=MAIN_CONTENT_3\&context=storylines_faq}{instruction
    will be remote-only in the fall}, citing concerns that surging
    coronavirus infections in their areas pose too dire a risk for
    students and teachers. Together, the two districts enroll some
    825,000 students. They are the largest in the country so far to
    abandon plans for even a partial physical return to classrooms when
    they reopen in August. For other districts, the solution won't be an
    all-or-nothing approach.
    \href{https://bioethics.jhu.edu/research-and-outreach/projects/eschool-initiative/school-policy-tracker/}{Many
    systems}, including the nation's largest, New York City, are
    devising
    \href{https://www.nytimes.com/2020/06/26/us/coronavirus-schools-reopen-fall.html?action=click\&pgtype=Article\&state=default\&region=MAIN_CONTENT_3\&context=storylines_faq}{hybrid
    plans} that involve spending some days in classrooms and other days
    online. There's no national policy on this yet, so check with your
    municipal school system regularly to see what is happening in your
    community.
  \end{itemize}
\end{itemize}

Another issue is that buying these assets in slices can mean a person is
paying more than she or he might if the person could buy the whole
asset, and that could dampen returns or make it hard to resell the
asset.

``You're buying an overvalued slice of the whole,'' said David Abate,
senior wealth adviser with Strategic Wealth Partners. ``If you decide
you want to get out of this investment, you'd better understand how the
secondary market works.''

The fees are disclosed but baked in. With MyRacehorse, 15 percent of the
offering of a horse goes to the company upfront. But each horse is part
of an entity that has been registered with the Securities and Exchange
Commission.

``This is high risk; I'd never tell people otherwise,'' Mr. Behrens
said. ``We're not trying to build a platform that says this is going to
be a really good asset class. Many horses have been bought for \$1
million and never made it to the racetrack.''

As with other alternative investments, buyers are restricted from the
selling of these fractions until after the lockup period ends. But when
the asset itself --- the bag or the horse --- is sold is determined by
the platform, not the individual investors.

Image

Other luxury items, like Rolex watches, are available for fractional
shares from sites like Rally Rd.Credit...Jeenah Moon for The New York
Times

Jimmy Lee, chief executive of the Wealth Consulting Group, a wealth
adviser, questions the notion of buying a passion asset with an eye
toward profit. ``When it comes to art, you only see the ones that go up
in value,'' he said. ``If someone buys a piece of art for \$1 million
and it doesn't go up in value, it's not going to be sold.''

There are other drawbacks. These marketplaces do offer the possibility
of a return on the investment, but they deprive people of the joy of
owning a painting or a fast car: having it in your possession. (Although
with MyRacehorse, investors can at least go to the track and see their
horses.)

``You lose the intimacy of what it's meant to be,'' Mr. Ablin said.
``It's normally an asset you can touch, enjoy, ride in, ride on or
drink.''

But many investors in shares seem unbothered by this. Ms. Carbonaro said
not being able to touch or hold the bags she had invested in was not an
issue for her. ``If I had a Michael Jordan rookie card, I don't think
I'd want to touch it,'' she said.

John Cochran, who works in sales in Baltimore, has invested in shares of
76 different collectibles including a shirt Mr. Jordan wore in a
basketball game, a Muhammad Ali fight contract, a portrait of Abraham
Lincoln and a 2006 Ferrari f430 manual.

He said he was happy receiving a photo and some information on the
object and was unfazed that he could not hold or touch it. ``I like the
idea that, just like my stocks, it's all in an electronic portfolio,''
he said. ``I don't have to have the resources to store these things.''

Advertisement

\protect\hyperlink{after-bottom}{Continue reading the main story}

\hypertarget{site-index}{%
\subsection{Site Index}\label{site-index}}

\hypertarget{site-information-navigation}{%
\subsection{Site Information
Navigation}\label{site-information-navigation}}

\begin{itemize}
\tightlist
\item
  \href{https://help.nytimes.com/hc/en-us/articles/115014792127-Copyright-notice}{©~2020~The
  New York Times Company}
\end{itemize}

\begin{itemize}
\tightlist
\item
  \href{https://www.nytco.com/}{NYTCo}
\item
  \href{https://help.nytimes.com/hc/en-us/articles/115015385887-Contact-Us}{Contact
  Us}
\item
  \href{https://www.nytco.com/careers/}{Work with us}
\item
  \href{https://nytmediakit.com/}{Advertise}
\item
  \href{http://www.tbrandstudio.com/}{T Brand Studio}
\item
  \href{https://www.nytimes.com/privacy/cookie-policy\#how-do-i-manage-trackers}{Your
  Ad Choices}
\item
  \href{https://www.nytimes.com/privacy}{Privacy}
\item
  \href{https://help.nytimes.com/hc/en-us/articles/115014893428-Terms-of-service}{Terms
  of Service}
\item
  \href{https://help.nytimes.com/hc/en-us/articles/115014893968-Terms-of-sale}{Terms
  of Sale}
\item
  \href{https://spiderbites.nytimes.com}{Site Map}
\item
  \href{https://help.nytimes.com/hc/en-us}{Help}
\item
  \href{https://www.nytimes.com/subscription?campaignId=37WXW}{Subscriptions}
\end{itemize}
