Sections

SEARCH

\protect\hyperlink{site-content}{Skip to
content}\protect\hyperlink{site-index}{Skip to site index}

\href{https://www.nytimes.com/spotlight/at-home}{At Home}

\href{https://myaccount.nytimes.com/auth/login?response_type=cookie\&client_id=vi}{}

\href{https://www.nytimes.com/section/todayspaper}{Today's Paper}

\href{/spotlight/at-home}{At Home}\textbar{}Make Pizza \ldots{} On Your
Grill

\url{https://nyti.ms/3gklcge}

\begin{itemize}
\item
\item
\item
\item
\item
\end{itemize}

\href{https://www.nytimes.com/spotlight/at-home?action=click\&pgtype=Article\&state=default\&region=TOP_BANNER\&context=at_home_menu}{At
Home}

\begin{itemize}
\tightlist
\item
  \href{https://www.nytimes.com/2020/07/28/books/time-for-a-literary-road-trip.html?action=click\&pgtype=Article\&state=default\&region=TOP_BANNER\&context=at_home_menu}{Take:
  A Literary Road Trip}
\item
  \href{https://www.nytimes.com/2020/07/29/magazine/bored-with-your-home-cooking-some-smoky-eggplant-will-fix-that.html?action=click\&pgtype=Article\&state=default\&region=TOP_BANNER\&context=at_home_menu}{Cook:
  Smoky Eggplant}
\item
  \href{https://www.nytimes.com/2020/07/27/travel/moose-michigan-isle-royale.html?action=click\&pgtype=Article\&state=default\&region=TOP_BANNER\&context=at_home_menu}{Look
  Out: For Moose}
\item
  \href{https://www.nytimes.com/interactive/2020/at-home/even-more-reporters-editors-diaries-lists-recommendations.html?action=click\&pgtype=Article\&state=default\&region=TOP_BANNER\&context=at_home_menu}{Explore:
  Reporters' Obsessions}
\end{itemize}

Advertisement

\protect\hyperlink{after-top}{Continue reading the main story}

Supported by

\protect\hyperlink{after-sponsor}{Continue reading the main story}

\hypertarget{make-pizza--on-your-grill}{%
\section{Make Pizza \ldots{} On Your
Grill}\label{make-pizza--on-your-grill}}

Bear with us. We know this sounds bonkers, but it really does work.

\includegraphics{https://static01.nyt.com/images/2020/08/02/multimedia/02ah-grilledpizza1/02ah-grilledpizza1-articleLarge.jpg?quality=75\&auto=webp\&disable=upscale}

By \href{https://www.nytimes.com/by/amelia-nierenberg}{Amelia
Nierenberg}

\begin{itemize}
\item
  Aug. 1, 2020
\item
  \begin{itemize}
  \item
  \item
  \item
  \item
  \item
  \end{itemize}
\end{itemize}

Homemade pizza is great. A 500-degree oven in August is not.

But if you have a grill, you can turn it into a makeshift pizza oven.
You won't end up with the most traditional pie --- it'll be more like a
flatbread --- but that's OK.

\href{https://www.nytimes.com/2013/07/01/health/forgot-tofu-hot-dogs-pizza-on-the-grill.html}{Grilled
pizza}, done right, will be ready in minutes. It's delicious, and your
friends will be impressed. It's also a great way to use up all the
\href{https://slate.com/business/2020/04/yeast-shortage-supermarkets-coronavirus.html}{yeast}
and flour you panic-bought in March.

\hypertarget{make-and-shape-the-dough}{%
\subsection{Make and shape the dough.}\label{make-and-shape-the-dough}}

Pizza dough is pretty straightforward: flour, water, salt, yeast. Some
recipes call for a little sugar and olive oil. If you're looking for
inspiration, look to the Cooking section of The New York Times, which
includes
\href{https://cooking.nytimes.com/recipes/1014838-pizza-on-the-grill}{a
recipe} specifically for grilled pizza.

You'll want a thin dough, but one that isn't floppy: You don't want it
to droop between the grill grates. So use your hands instead of a
rolling pin to stretch it out. Stop when it's about as thick as the
length of your fingernail.

Don't try to make a separate crust. It probably won't cook through, and
isn't necessary to the recipe.

\hypertarget{get-that-grill-hot}{%
\subsection{Get that grill hot.}\label{get-that-grill-hot}}

Pizza is a high-heat food. If you have a thermometer, aim for around 500
degrees. If not, just hold your palm about six inches away from the
heat. When it's uncomfortable after a few seconds, you're ready to go.

The heat distribution in a pizza oven is sort of like surround sound.
``It's above it, below it, swirling around it,'' explained Anthony
Falco, 41, \href{https://www.piz.za.com/}{an international pizza
consultant}.

But on a grill, the heat comes from the bottom. You can, of course,
mimic a pizza oven by closing the lid. Or you can embrace the
flatbread-y quality and keep the lid up.

\includegraphics{https://static01.nyt.com/images/2020/08/02/multimedia/02ah-grilledpizza2/02grilledpizza2-articleLarge.jpg?quality=75\&auto=webp\&disable=upscale}

\hypertarget{a-pizza-stone-works-or-you-can-go-straight-on-the-grate}{%
\subsection{A pizza stone works, or you can go straight on the
grate.}\label{a-pizza-stone-works-or-you-can-go-straight-on-the-grate}}

If you have a pizza stone --- or even
\href{https://cooking.nytimes.com/recipes/1020233-sheet-pan-pizza-with-asparagus-and-arugula}{a
metal sheet-pan} --- you can just put it on top of the grill. It'll give
you a more consistent crust and might help with heat distribution.

Slapping the dough down straight on the grate, though, might be more
fun.

``When something has those char marks, you can taste the barbecue
aspect,'' said Audrey Kelly, 34, a pizzaiola and the owner of
\href{https://www.thepizzagarage.com/}{Audrey Jane's Pizza Garage} in
Boulder, Colo.

Either way, put the dough down fast to keep it from sticking. A dusting
of flour should do the trick. A little olive oil will, too, but go easy.
You don't want your pizza to burst into flames.

\hypertarget{keep-an-eye-on-the-dough}{%
\subsection{Keep an eye on the dough.}\label{keep-an-eye-on-the-dough}}

If you're cooking it on a stove or a sheet-pan, you can probably follow
the recipe with minimal changes.

But if you're a straight-to-the-grate maverick, think of the preparation
in two stages: a dough stage and a topping stage, with a flip in
between. Watch the surface of the dough. Once it has some big air
bubbles and the char marks are starting to brown, flip the dough so it
cooks evenly on both sides. This when the toppings are added.

If you want a benchmark, Ms. Kelly suggested shooting for an eight
minute cook time: four to five for the dough, flip, then three to four
for the toppings.

\hypertarget{consider-your-toppings}{%
\subsection{Consider your toppings.}\label{consider-your-toppings}}

Sauce, regrettably, is probably an iffy bet. Too little will leave the
pizza under-flavored. But too much might make it soggy. If you are going
to sauce, a small amount should suffice.

If you're adding meat, cook it beforehand, as well as some vegetables.
Summer markets are flourishing right now --- take advantage.

``The key is you're getting the best produce, and you're treating it
very simply,'' Mr. Falco said, 41.

If you're looking for a guiding cheese principle, try to think of what
your other toppings don't yet fulfill. If you're salt-less, try some
shaved pecorino. If you need something creamy-ish, dollop some ricotta.
If you're looking for texture, think mozzarella.

\hypertarget{portion-your-pie-size}{%
\subsection{Portion your pie size.}\label{portion-your-pie-size}}

A big pizza is going to be harder to flip and harder to evenly heat.

So make a few dinner-plate-sized ones instead. This is a great thing to
do with kids --- set up a topping station and let them make their own
pizza.

And remember, it's summer during a pandemic. If it works out, it'll be
fun to share. If not, the worst thing that happens is that you just call
for pizza delivery instead.

Advertisement

\protect\hyperlink{after-bottom}{Continue reading the main story}

\hypertarget{site-index}{%
\subsection{Site Index}\label{site-index}}

\hypertarget{site-information-navigation}{%
\subsection{Site Information
Navigation}\label{site-information-navigation}}

\begin{itemize}
\tightlist
\item
  \href{https://help.nytimes.com/hc/en-us/articles/115014792127-Copyright-notice}{©~2020~The
  New York Times Company}
\end{itemize}

\begin{itemize}
\tightlist
\item
  \href{https://www.nytco.com/}{NYTCo}
\item
  \href{https://help.nytimes.com/hc/en-us/articles/115015385887-Contact-Us}{Contact
  Us}
\item
  \href{https://www.nytco.com/careers/}{Work with us}
\item
  \href{https://nytmediakit.com/}{Advertise}
\item
  \href{http://www.tbrandstudio.com/}{T Brand Studio}
\item
  \href{https://www.nytimes.com/privacy/cookie-policy\#how-do-i-manage-trackers}{Your
  Ad Choices}
\item
  \href{https://www.nytimes.com/privacy}{Privacy}
\item
  \href{https://help.nytimes.com/hc/en-us/articles/115014893428-Terms-of-service}{Terms
  of Service}
\item
  \href{https://help.nytimes.com/hc/en-us/articles/115014893968-Terms-of-sale}{Terms
  of Sale}
\item
  \href{https://spiderbites.nytimes.com}{Site Map}
\item
  \href{https://help.nytimes.com/hc/en-us}{Help}
\item
  \href{https://www.nytimes.com/subscription?campaignId=37WXW}{Subscriptions}
\end{itemize}
