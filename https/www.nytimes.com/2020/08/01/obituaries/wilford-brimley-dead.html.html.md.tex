Sections

SEARCH

\protect\hyperlink{site-content}{Skip to
content}\protect\hyperlink{site-index}{Skip to site index}

\href{https://www.nytimes.com/section/obituaries}{Obituaries}

\href{https://myaccount.nytimes.com/auth/login?response_type=cookie\&client_id=vi}{}

\href{https://www.nytimes.com/section/todayspaper}{Today's Paper}

\href{/section/obituaries}{Obituaries}\textbar{}Wilford Brimley,
`Cocoon' Star and Quaker Oats Pitchman, Is Dead at 85

\url{https://nyti.ms/3jWEk6d}

\begin{itemize}
\item
\item
\item
\item
\item
\item
\end{itemize}

Advertisement

\protect\hyperlink{after-top}{Continue reading the main story}

Supported by

\protect\hyperlink{after-sponsor}{Continue reading the main story}

\hypertarget{wilford-brimley-cocoon-star-and-quaker-oats-pitchman-is-dead-at-85}{%
\section{Wilford Brimley, `Cocoon' Star and Quaker Oats Pitchman, Is
Dead at
85}\label{wilford-brimley-cocoon-star-and-quaker-oats-pitchman-is-dead-at-85}}

Recognizable by his walrus mustache, the actor specialized in playing
cantankerous characters in ``Absence of Malice,'' ``The Natural'' and
other films.

\includegraphics{https://static01.nyt.com/images/2020/08/03/world/00brimley/00brimley-articleLarge-v2.jpg?quality=75\&auto=webp\&disable=upscale}

By \href{https://www.nytimes.com/by/william-grimes}{William Grimes}

\begin{itemize}
\item
  Aug. 1, 2020
\item
  \begin{itemize}
  \item
  \item
  \item
  \item
  \item
  \item
  \end{itemize}
\end{itemize}

Wilford Brimley, a portly actor with a walrus mustache who found his
niche playing cantankerous coots in ``Absence of Malice,'' ``The
Natural,'' ``Cocoon'' and other films, died on Saturday in a hospital in
St. George, Utah. He was 85.

He had been sick for two months with a kidney ailment, said his agent,
Lynda Bensky.

Mr. Brimley had played the Walton Mountain resident Horace Brimley in a
recurring role on the television series ``The Waltons'' when Michael
Douglas, the producer of ``The China Syndrome,'' gave him his
breakthrough role: Ted Spindler, an assistant engineer at a nuclear
plant.

In the film's climactic scene, in which he is being interviewed by a
crusading television reporter played by Jane Fonda, Mr. Brimley
delivered an impassioned defense of his boss (Jack Lemmon), who had
precipitated a crisis to draw public attention to defects at the plant.

In an article for The New York Times singling out Mr. Brimley as a
talent to watch, Janet Maslin called him ``the mustachioed man who very
nearly steals the ending of `China Syndrome' from Jane Fonda.''

Mr. Brimley followed up with a small but memorable performance as a
pugnacious district attorney in
\href{https://www.youtube.com/watch?v=btqBJJF2yvE}{``Absence of
Malice''} and with supporting roles in ``The Natural,'' as the put-upon
manager of a losing baseball team, and ``The Firm,'' in which he played
the sinister head of security at an unsavory law firm.

In Ron Howard's 1985 fantasy film ``Cocoon,'' Mr. Brimley delivered one
of his most engaging performances, as a Florida retiree who, with Don
Ameche and Hume Cronyn, regains his youth after swimming in a magic
pool.

``Wilford's a testy guy, not an easy guy to work with all the time, but
he has great instincts,'' Mr. Howard told The Times in 1985. ``Many of
his scenes were totally improvised.''

In the 1980s and 1990s Mr. Brimley was a television fixture as a
spokesman for \href{https://www.youtube.com/watch?v=GOLXnkbfEuo}{Quaker
Oats}, gruffly telling viewers to eat the cereal because ``it's the
right thing to do,'' and
\href{https://www.youtube.com/watch?v=1K93EPoO7is}{Liberty Medical}, a
company selling diabetes-testing supplies. Mr. Brimley learned that he
had the disease in the late 1970s.

When interviewed, Mr. Brimley played down his talent; he described
himself as ``just a guy, just a feller'' to The
\href{http://www.powelltribune.com/news/item/12188-\%E2\%80\%98just-a-feller\%E2\%80\%99-actor-wilford-brimley-reflects-on-long-career-stars-he\%E2\%80\%99s-known-and-the-music-he-loves-to-sing}{Powell
Tribune} of Wyoming in 2014. ``I can't talk about acting,'' he said. ``I
don't know anything about it. I was just lucky enough to get hired.''

Anthony Wilford Brimley was born on Sept. 27, 1934, in Salt Lake City.
His father, a real estate broker, sold the family farm in 1939 and moved
his family to Santa Monica, Calif.

Tony, as he was known, dropped out of school at 14 and worked as a
cowboy in Idaho, Nevada and Arizona before enlisting in the Marine
Corps, which sent him to the Aleutian Islands. After leaving the
service, he worked as a ranch hand, wrangler and blacksmith. Briefly, he
was a bodyguard for Howard Hughes.

He began shoeing horses for television and film westerns, and gradually
took nonspeaking roles on horseback. He appeared as a stuntman in
``Bandolero!,'' in an uncredited role in ``True Grit'' and as a
blacksmith in the television series ``Kung Fu.''

\includegraphics{https://static01.nyt.com/images/2020/08/03/obituaries/01brimley-cocoon/01brimley-cocoon-articleLarge.jpg?quality=75\&auto=webp\&disable=upscale}

After ``The China Syndrome,'' he worked steadily. He played Harry, the
former manager of the country singer played by Robert Duvall, in
\href{https://www.youtube.com/watch?v=d40BjDgpdwU}{``Tender Mercies,''}
and the eccentric tycoon Bradley Tozer in the Tom Selleck adventure film
``High Road to China,'' before returning to the role of Ben Luckett in
``Cocoon: The Return.''

From 1986 to 1988 he had a starring role as Gus Witherspoon, the
opinionated but lovable grandfather in the NBC series ``Our House,'' yet
again confounding the usual Hollywood aging process by portraying, in
his early 50s, a character who was 65.

``I'm never the leading man,'' he told The Dallas Morning News in 1993.
``I never get the girl. And I never get to take my shirt off. I started
by playing fathers to guys who were 25 years older than I was.''

In part because of his television commercials, Mr. Brimley made the
transition from actor to comic source material. John Goodman did a
parody of his diabetes commercial on
\href{https://video.yahoo.com/liberty-medical-000000755.html}{``Saturday
Night Live,''} and in 1997 he appeared in a cameo role on ``Seinfeld''
as the short-tempered postmaster general, Henry Atkins.

He had a pleasant singing voice and recorded several albums of jazz
standards, including ``This Time the Dream's on Me'' and ``Wilford
Brimley With the Jeff Hamilton Trio.'' He could more than hold his own
as a \href{https://www.youtube.com/watch?v=WAft2naOgGc}{guitarist} too.

Mr. Brimley's first wife, the former Lynne Bagley, died in 2000. He is
survived by his wife, Beverly, and three sons from his first marriage,
James, John and William. Another son, Lawrence, died in infancy.
Complete information on other survivors was not immediately available.

As Mr. Howard noted, Mr. Brimley came by his
\href{https://www.youtube.com/watch?v=DLqX7Vi9yT8\&list=PLy1Yuw2wBXtCMd3A4mbt2x4Z8ty3S_YFF}{cussedness}naturally.
In ``Miracles and Mercies,'' a documentary about the making of ``Tender
Mercies,'' Mr. Duvall recalled a set-to between Mr. Brimley and the
director Bruce Beresford, who had made a suggestion about how Mr.
Brimley might play the role of Harry.

``Now look, let me tell you something, I'm Harry,'' he recalled Mr.
Brimley telling Mr. Beresford. ``Harry's not over there, Harry's not
over here. Until you fire me or get another actor, I'm Harry, and
whatever I do is fine `cause I'm Harry.''

Aimee Ortiz contributed reporting.

Advertisement

\protect\hyperlink{after-bottom}{Continue reading the main story}

\hypertarget{site-index}{%
\subsection{Site Index}\label{site-index}}

\hypertarget{site-information-navigation}{%
\subsection{Site Information
Navigation}\label{site-information-navigation}}

\begin{itemize}
\tightlist
\item
  \href{https://help.nytimes.com/hc/en-us/articles/115014792127-Copyright-notice}{©~2020~The
  New York Times Company}
\end{itemize}

\begin{itemize}
\tightlist
\item
  \href{https://www.nytco.com/}{NYTCo}
\item
  \href{https://help.nytimes.com/hc/en-us/articles/115015385887-Contact-Us}{Contact
  Us}
\item
  \href{https://www.nytco.com/careers/}{Work with us}
\item
  \href{https://nytmediakit.com/}{Advertise}
\item
  \href{http://www.tbrandstudio.com/}{T Brand Studio}
\item
  \href{https://www.nytimes.com/privacy/cookie-policy\#how-do-i-manage-trackers}{Your
  Ad Choices}
\item
  \href{https://www.nytimes.com/privacy}{Privacy}
\item
  \href{https://help.nytimes.com/hc/en-us/articles/115014893428-Terms-of-service}{Terms
  of Service}
\item
  \href{https://help.nytimes.com/hc/en-us/articles/115014893968-Terms-of-sale}{Terms
  of Sale}
\item
  \href{https://spiderbites.nytimes.com}{Site Map}
\item
  \href{https://help.nytimes.com/hc/en-us}{Help}
\item
  \href{https://www.nytimes.com/subscription?campaignId=37WXW}{Subscriptions}
\end{itemize}
