Sections

SEARCH

\protect\hyperlink{site-content}{Skip to
content}\protect\hyperlink{site-index}{Skip to site index}

\href{https://www.nytimes.com/section/opinion/sunday}{Sunday Review}

\href{https://myaccount.nytimes.com/auth/login?response_type=cookie\&client_id=vi}{}

\href{https://www.nytimes.com/section/todayspaper}{Today's Paper}

\href{/section/opinion/sunday}{Sunday Review}\textbar{}Voting by Mail Is
Crucial for Democracy

\href{https://nyti.ms/3gkyJUR}{https://nyti.ms/3gkyJUR}

\begin{itemize}
\item
\item
\item
\item
\item
\item
\end{itemize}

Advertisement

\protect\hyperlink{after-top}{Continue reading the main story}

\href{/section/opinion}{Opinion}

Supported by

\protect\hyperlink{after-sponsor}{Continue reading the main story}

\hypertarget{voting-by-mail-is-crucial-for-democracy}{%
\section{Voting by Mail Is Crucial for
Democracy}\label{voting-by-mail-is-crucial-for-democracy}}

Especially amid the pandemic, it's the surest path to a more inclusive,
more accurate and more secure election.

By
\href{https://www.nytimes.com/interactive/opinion/editorialboard.html}{The
Editorial Board}

The editorial board is a group of opinion journalists whose views are
informed by expertise, research, debate and certain longstanding ****
\href{https://www.nytimes.com/interactive/2018/opinion/editorialboard.html}{values}.
It is separate from the newsroom.

\begin{itemize}
\item
  Aug. 1, 2020
\item
  \begin{itemize}
  \item
  \item
  \item
  \item
  \item
  \item
  \end{itemize}
\end{itemize}

\includegraphics{https://static01.nyt.com/images/2020/08/02/opinion/01Voting/01Voting-articleLarge.jpg?quality=75\&auto=webp\&disable=upscale}

For a man who
\href{https://www.nytimes.com/2020/08/03/us/politics/trump-mail-in-voting.html}{votes
by mail} himself, Donald Trump is strangely obsessed with the idea that
it is the most dangerous method of casting a ballot.

The president was at it again this week. ``Rigged Election,'' he
\href{https://twitter.com/realDonaldTrump/status/1288602262567153664}{tweeted}
of New York's well-publicized struggles with counting mail-in votes.
``Same thing would happen, but on massive scale, with USA.''

Voting by mail is a
\href{https://twitter.com/realDonaldTrump/status/1288809157722877952}{``catastrophic
disaster,''} he later said, ``an easy way for foreign countries to enter
the race.'' Any election conducted by mail would be
\href{https://twitter.com/realDonaldTrump/status/1288818160389558273}{``INACCURATE
AND FRAUDULENT.''}

Finally, the
\href{https://twitter.com/realDonaldTrump/status/1288818160389558273}{hammer}:
``Delay the Election until people can properly, securely and safely
vote???''

In a word, Mr. President: No.

The election will not be delayed --- because the president can't legally
delay it. Its date is set by federal law, as is the date on which the
presidential electors must cast their ballots. Then there's the backstop
of Inauguration Day,
\href{https://constitutioncenter.org/interactive-constitution/amendment/amendment-xx}{set
by the Constitution} as Jan. 20.

Mr. Trump says things like this often enough that it can be easy to
brush him off. He even claimed that the 2016 election, \emph{which he
won}, was rigged. But the president's words, however misleading, carry
weight. So it is necessary to say it again: Especially in the midst of a
raging pandemic, voting by mail is the surest path to a more inclusive,
more accurate and more secure election.

The good news is that the primary season gave states a chance to run
their elections with far more mail-in ballots than usual, and in many
places the system worked well. But there were multiple high-profile
examples of mail voting gone wrong. In Wisconsin,
\href{https://www.nytimes.com/2020/04/09/us/politics/wisconsin-election-absentee-coronavirus.html}{thousands}
of absentee ballots were requested and never received. In New Jersey,
\href{https://www.njspotlight.com/2020/06/one-in-10-ballots-rejected-in-last-months-vote-by-mail-elections/}{10
percent of mail ballots} were thrown out for arriving too late or for
being otherwise deficient. In Pennsylvania,
\href{https://www.inquirer.com/politics/election/pa-mail-ballot-deadlines-disenfranchisement-20200730.html}{tens
of thousands of absentee votes} were either not cast or not counted,
especially among voters who requested their absentee ballots closer to
the election.

Mr. Trump and his allies have exploited these bungles to the hilt,
claiming that they reveal how dangerous it is to vote by mail. Ignore
them. Voting by mail --- or absentee voting, which Mr. Trump pretends is
something different even though it isn't --- has risks like any other
method, but overall it is safe and accurate. So safe and accurate, in
fact, that in
\href{https://www.ncsl.org/research/elections-and-campaigns/all-mail-elections.aspx}{five
states} most or all voters use it, and in three other states more than
half do. In those states, elections go off without a hitch.

That's why as soon as the pandemic hit, it was clear that expanding
access to mail voting across the country would be essential for the
November election to succeed. Voting experts pleaded with Congress to
supply the necessary funds to help states with less experience in
processing absentee ballots.

More than four months later, only a fraction of that money has been
handed out. As Congress
\href{https://www.nytimes.com/2020/07/30/opinion/mitch-mcconnell-coronavirus-economy.html}{battles
over the latest} stimulus bill, it's not clear if any more is on the
way. This is a dereliction of Congress's duty to ensure the functioning
of American democracy.

The American people need to be able to vote in the November election,
and they need to be able to trust the outcome of that vote. What can be
done over the next three months to make the process as accessible,
accurate and secure as possible? Here are three relatively
straightforward tasks.

First, aggressively counter misinformation about mail voting, which
continues to be spread not just by President Trump, but also by top
members of his administration.

On Tuesday, Attorney General William Barr testified in Congress that he
believed mail voting on a large scale presented a
\href{https://www.washingtonpost.com/video/politics/barr-states-that-mail-in-voting-could-lead-to-a-high-risk-of-fraud/2020/07/28/2db47f91-2c5f-41e3-904e-79072b68547d_video.html}{``high
risk''} for massive voter fraud.

As Mr. Barr well knows, voter fraud is rare and is
\href{https://docs.wixstatic.com/ugd/ef45f5_81a3affd554e4b5b9b5852f8fb3c10fd.pdf}{virtually
nonexistent} in the states where most or all voters cast their ballots
by mail.

The problem is that many people in the Republican Party are convinced of
its own unpopularity: Some openly
\href{https://www.washingtonpost.com/politics/2020/03/30/trump-voting-republicans/}{admit
their belief} that when more people vote, Republicans are more likely to
lose. In March, Mr. Trump complained about a proposal by House Democrats
to expand access to the ballot. ``They had levels of voting, that if you
ever agreed to it, you'd never have a Republican elected in this country
again,'' he said.

It's true that mail voting
\href{https://www.vox.com/policy-and-politics/2018/5/23/17383400/vote-by-mail-home-california-alaska-nebraska}{increases
turnout}, particularly among groups that tend not to vote, like young
people. Colorado saw a turnout increase of 9 percent when it switched to
all-mail voting, and the increase was
\href{https://www.nytimes.com/2020/05/04/opinion/coronavirus-vote-by-mail.html}{nearly
double} that among young voters.

Republicans may be thinking about numbers like these when they rail
against mail voting. But the turnout increases from mail voting
\href{https://www.nytimes.com/2020/04/10/us/politics/vote-by-mail.html}{don't
appear} to change the results.

The real reason to make mail voting widely accessible isn't to help one
party or another --- it's to help the American people participate in
their own democracy as fully as possible. That's why voters of both
parties
\href{https://news.gallup.com/poll/310586/americans-favor-voting-mail-option-november.aspx}{like
it so much}, which may be the best evidence of all that it has no
built-in partisan bias.

Second, public officials must educate voters.

In 2016, nearly
\href{https://www.eac.gov/documents/2017/10/17/eavs-deep-dive-early-absentee-and-mail-voting-data-statutory-overview}{one
in four voters} cast their ballots by mail. Still, voting by mail
remains a novelty for most Americans, who are used to walking into their
polling place on Election Day, registering their vote and handing their
ballot to another human being --- or at least feeding it into a scanner.
It's understandable that people would be wary of or confused by a new
method.

That's why public-education efforts will be critical over the next few
months. State and local officials need to explain, in clear and simple
terms, when and how to request an absentee ballot and how to fill one
out, sign it and send it back. This will make the process more secure
and also reduce the number of ballots rejected because they weren't
properly filled out or signed. When ballots are rejected, states must
give voters a fair opportunity to fix any errors.

Of course, all the education in the world won't help if ballots are
rejected or uncounted through no fault of the voter --- say, because
mail backups delay their arrival. By
\href{https://papers.ssrn.com/sol3/papers.cfm?abstract_id=3660625}{one
estimate}, as many as 4 percent of all mail ballots went uncounted in
2016. At a minimum, states that don't already accept ballots that arrive
after Election Day must update their election laws and rules to do so.
Whether they allow for a week or 10 days, the window needs to be long
enough to account for delays in mail handling and postmark mix-ups that
led to the dumping of so many absentee ballots in New York's primary.
(The postmaster general, a Trump donor named Louis DeJoy, is
\href{https://www.npr.org/2020/07/29/894799516/pending-postal-service-changes-could-delay-mail-and-deliveries-advocates-warn}{making
matters worse} by slashing overtime and slowing the delivery of regular
mail. Perhaps not coincidentally, Mr. Trump
\href{https://twitter.com/realDonaldTrump/status/1288933078287745024?s=20}{has
started insisting} that a winner be called on election night itself, and
not a moment later.)

The crush on the Postal Service will be real, and it can be eased by
providing more places for voters to drop off their ballots in person ---
like dedicated drop boxes, which are popular in Colorado.

The pressure on election workers to process all those extra mail ballots
can be alleviated by hiring more of them, paying them a decent wage and,
critically, reminding voters not to get antsy when final results aren't
immediately clear. Counting absentee ballots can take time. That's not
fraud.

Third, officials need to ensure that in-person voting is safe and
available for those who either do not receive a mail ballot or are not
comfortable voting that way.

This is made more difficult because many polling places are being shut
down for public-health reasons, and thousands of poll workers --- many
of whom are older and at increased risk for severe illness --- are
declining to volunteer. That's all the more reason to hire more (and
younger) poll workers, and to provide large spaces where voters can stay
socially distanced.

Nathaniel Persily and Charles Stewart, two voting experts,
\href{https://www.theatlantic.com/ideas/archive/2020/06/looming-threat-voting-person/613552/?utm_source=newsletter\&utm_medium=email\&utm_campaign=atlantic-daily-newsletter\&utm_content=20200701\&silverid-ref=NjEyOTYyMjM5Njg3S0}{suggest}
making Election Day a school holiday and turning big-box retailers into
polling places. That advice has been taken up by N.B.A. teams in three
battleground states,
\href{https://www.npr.org/2020/07/02/886566523/need-a-polling-place-with-social-distancing-3-nba-teams-offer-venues}{who
have offered} their arenas as polling sites. Other venues ought to
follow suit.

Early voting and same-day voter registration --- reforms that have
proved to increase turnout --- are all the more important this year, as
millions of Americans either have moved or have temporarily relocated as
a result of the pandemic, and may not have their registration in order.

In the end, most of these fixes come down to money: to educate voters,
to print more mail ballots and envelopes, to hire more poll workers and
election workers, to provide masks and other protective gear, to rent
large spaces as one-time precincts.

Voting experts have said for months that it's in the range of \$4
billion --- a lot of money, to be sure, but a rounding error in the
context of the trillions already allocated in the stimulus bills passed
by Congress.

Despite all the obstacles in this unprecedented moment, Americans will
vote this year, possibly in record numbers. It's not a matter of whether
tens of millions of them will do so by mail, but whether they will have
their voices heard, and whether we can all be patient enough to get
through what may well be the most extraordinary election in our
lifetime.

\emph{The Times is committed to publishing}
\href{https://www.nytimes.com/2019/01/31/opinion/letters/letters-to-editor-new-york-times-women.html}{\emph{a
diversity of letters}} \emph{to the editor. We'd like to hear what you
think about this or any of our articles. Here are some}
\href{https://help.nytimes.com/hc/en-us/articles/115014925288-How-to-submit-a-letter-to-the-editor}{\emph{tips}}\emph{.
And here's our email:}
\href{mailto:letters@nytimes.com}{\emph{letters@nytimes.com}}\emph{.}

\emph{Follow The New York Times Opinion section on}
\href{https://www.facebook.com/nytopinion}{\emph{Facebook}}\emph{,}
\href{http://twitter.com/NYTOpinion}{\emph{Twitter (@NYTopinion)}}
\emph{and}
\href{https://www.instagram.com/nytopinion/}{\emph{Instagram}}\emph{.}

Advertisement

\protect\hyperlink{after-bottom}{Continue reading the main story}

\hypertarget{site-index}{%
\subsection{Site Index}\label{site-index}}

\hypertarget{site-information-navigation}{%
\subsection{Site Information
Navigation}\label{site-information-navigation}}

\begin{itemize}
\tightlist
\item
  \href{https://help.nytimes.com/hc/en-us/articles/115014792127-Copyright-notice}{©~2020~The
  New York Times Company}
\end{itemize}

\begin{itemize}
\tightlist
\item
  \href{https://www.nytco.com/}{NYTCo}
\item
  \href{https://help.nytimes.com/hc/en-us/articles/115015385887-Contact-Us}{Contact
  Us}
\item
  \href{https://www.nytco.com/careers/}{Work with us}
\item
  \href{https://nytmediakit.com/}{Advertise}
\item
  \href{http://www.tbrandstudio.com/}{T Brand Studio}
\item
  \href{https://www.nytimes.com/privacy/cookie-policy\#how-do-i-manage-trackers}{Your
  Ad Choices}
\item
  \href{https://www.nytimes.com/privacy}{Privacy}
\item
  \href{https://help.nytimes.com/hc/en-us/articles/115014893428-Terms-of-service}{Terms
  of Service}
\item
  \href{https://help.nytimes.com/hc/en-us/articles/115014893968-Terms-of-sale}{Terms
  of Sale}
\item
  \href{https://spiderbites.nytimes.com}{Site Map}
\item
  \href{https://help.nytimes.com/hc/en-us}{Help}
\item
  \href{https://www.nytimes.com/subscription?campaignId=37WXW}{Subscriptions}
\end{itemize}
