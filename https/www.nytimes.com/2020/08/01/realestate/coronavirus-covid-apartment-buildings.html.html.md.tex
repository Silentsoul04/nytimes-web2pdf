Sections

SEARCH

\protect\hyperlink{site-content}{Skip to
content}\protect\hyperlink{site-index}{Skip to site index}

\href{https://www.nytimes.com/section/realestate}{Real Estate}

\href{https://myaccount.nytimes.com/auth/login?response_type=cookie\&client_id=vi}{}

\href{https://www.nytimes.com/section/todayspaper}{Today's Paper}

\href{/section/realestate}{Real Estate}\textbar{}How Do I Get My
Landlord to Follow Covid-19 Rules?

\url{https://nyti.ms/3hQRNdM}

\begin{itemize}
\item
\item
\item
\item
\item
\item
\end{itemize}

\href{https://www.nytimes.com/news-event/coronavirus?action=click\&pgtype=Article\&state=default\&region=TOP_BANNER\&context=storylines_menu}{The
Coronavirus Outbreak}

\begin{itemize}
\tightlist
\item
  live\href{https://www.nytimes.com/2020/08/03/world/coronavirus-covid-19.html?action=click\&pgtype=Article\&state=default\&region=TOP_BANNER\&context=storylines_menu}{Latest
  Updates}
\item
  \href{https://www.nytimes.com/interactive/2020/us/coronavirus-us-cases.html?action=click\&pgtype=Article\&state=default\&region=TOP_BANNER\&context=storylines_menu}{Maps
  and Cases}
\item
  \href{https://www.nytimes.com/interactive/2020/science/coronavirus-vaccine-tracker.html?action=click\&pgtype=Article\&state=default\&region=TOP_BANNER\&context=storylines_menu}{Vaccine
  Tracker}
\item
  \href{https://www.nytimes.com/2020/08/02/us/covid-college-reopening.html?action=click\&pgtype=Article\&state=default\&region=TOP_BANNER\&context=storylines_menu}{College
  Reopening}
\item
  \href{https://www.nytimes.com/live/2020/08/03/business/stock-market-today-coronavirus?action=click\&pgtype=Article\&state=default\&region=TOP_BANNER\&context=storylines_menu}{Economy}
\end{itemize}

Advertisement

\protect\hyperlink{after-top}{Continue reading the main story}

Supported by

\protect\hyperlink{after-sponsor}{Continue reading the main story}

Ask Real Estate

\hypertarget{how-do-i-get-my-landlord-to-follow-covid-19-rules}{%
\section{How Do I Get My Landlord to Follow Covid-19
Rules?}\label{how-do-i-get-my-landlord-to-follow-covid-19-rules}}

New York issued guidelines for how residential buildings should safely
operate --- but there no consequences for failing to comply.

\includegraphics{https://static01.nyt.com/images/2020/08/02/realestate/01Ask/01Ask-articleLarge.jpg?quality=75\&auto=webp\&disable=upscale}

\href{https://www.nytimes.com/by/ronda-kaysen}{\includegraphics{https://static01.nyt.com/images/2018/07/16/multimedia/author-ronda-kaysen/author-ronda-kaysen-thumbLarge-v2.png}}

By \href{https://www.nytimes.com/by/ronda-kaysen}{Ronda Kaysen}

\begin{itemize}
\item
  Aug. 1, 2020
\item
  \begin{itemize}
  \item
  \item
  \item
  \item
  \item
  \item
  \end{itemize}
\end{itemize}

\textbf{Q: I live in a Manhattan rental building and staff aren't
wearing masks anymore, much less screening visitors. The building has
been lax the whole pandemic, but it's gotten particularly bad in the
last two weeks. I can't force our neighbors to wear masks, but I am
often on the elevator with three or four people, without enough room to
safely socially distance. In response to an email, management said, ``We
can't enforce anything,'' and ignored my second email. I am feeling very
unsafe. What can I do?}

\textbf{A:}
\href{https://www.governor.ny.gov/sites/governor.ny.gov/files/atoms/files/realestate-masterguidance.pdf}{The
state's reopening plan} includes detailed guidelines for how residential
buildings should safely operate during the pandemic: Building staff
should wear face coverings when they can't safely socially distance;
post signs and remind residents of the rules; sanitize surfaces; and
screen visitors.

But the guidelines do not dictate consequences for failing to comply. So
if your landlord refuses to compel building staff to follow the rules,
your recourse is limited. Consider masks: ``There are no penalties or
fines for not wearing one,'' said Jennifer Rozen, a lawyer who
represents tenants. ``Enforcement is at the discretion of local
jurisdictions.''

The city Health Department will not send out an inspector if you report
the kinds of conditions you described, according to Michael Lanza, a
Health Department spokesman.

\begin{center}\rule{0.5\linewidth}{\linethickness}\end{center}

\begin{center}\rule{0.5\linewidth}{\linethickness}\end{center}

Avery Cohen, deputy press secretary for Mayor Bill de Blasio, said the
office is ``focused on education and outreach, and have seen
overwhelming compliance from buildings and management companies alike.''
Calls to 311 about such issues would be directed to the sheriff's
office.

You could try a different approach. Your landlord is supposed to provide
you with safe and habitable living conditions. By failing to take the
minimum steps to keep the common areas safe, your landlord is putting
you and your family at risk. You could withhold your rent, arguing that
your landlord is violating your
\href{https://nycourts.gov/courts/nyc/housing/pdfs/warrantyofhabitability.pdf}{warranty
of habitability}, a state statute, by failing to make sure tenants and
staff follow the guidelines. When your landlord sues you in housing
court for nonpayment of rent, you bring your claim to the judge as a
countersuit.

Before you take such a step, consult a lawyer who could first write the
landlord a strongly worded letter demanding the building provide you
with safe living conditions. This might be enough to get management to
listen.

Standing up to your landlord has risks, particularly if you are a
market-rate tenant. The landlord could refuse to offer you a new lease.
However, you are protected by the
\href{https://www1.nyc.gov/site/fairhousing/rights-responsibilities/retaliation.page}{retaliatory
eviction law}, so you could contest such a nonrenewal. But it may not
come to this.

``It's a renter's market with tenants fleeing the city, and this
complaint is reasonable to make,'' Ms. Rozen said. So you may decide
you're in a position to take the risk of confronting your landlord to
protect yourself.

For weekly email updates on residential real estate news,
\href{http://www.nytimes.com/newsletters/realestate/}{sign up here}.
Follow us on Twitter:
\href{https://twitter.com/nytrealestate}{@nytrealestate}.

Advertisement

\protect\hyperlink{after-bottom}{Continue reading the main story}

\hypertarget{site-index}{%
\subsection{Site Index}\label{site-index}}

\hypertarget{site-information-navigation}{%
\subsection{Site Information
Navigation}\label{site-information-navigation}}

\begin{itemize}
\tightlist
\item
  \href{https://help.nytimes.com/hc/en-us/articles/115014792127-Copyright-notice}{©~2020~The
  New York Times Company}
\end{itemize}

\begin{itemize}
\tightlist
\item
  \href{https://www.nytco.com/}{NYTCo}
\item
  \href{https://help.nytimes.com/hc/en-us/articles/115015385887-Contact-Us}{Contact
  Us}
\item
  \href{https://www.nytco.com/careers/}{Work with us}
\item
  \href{https://nytmediakit.com/}{Advertise}
\item
  \href{http://www.tbrandstudio.com/}{T Brand Studio}
\item
  \href{https://www.nytimes.com/privacy/cookie-policy\#how-do-i-manage-trackers}{Your
  Ad Choices}
\item
  \href{https://www.nytimes.com/privacy}{Privacy}
\item
  \href{https://help.nytimes.com/hc/en-us/articles/115014893428-Terms-of-service}{Terms
  of Service}
\item
  \href{https://help.nytimes.com/hc/en-us/articles/115014893968-Terms-of-sale}{Terms
  of Sale}
\item
  \href{https://spiderbites.nytimes.com}{Site Map}
\item
  \href{https://help.nytimes.com/hc/en-us}{Help}
\item
  \href{https://www.nytimes.com/subscription?campaignId=37WXW}{Subscriptions}
\end{itemize}
