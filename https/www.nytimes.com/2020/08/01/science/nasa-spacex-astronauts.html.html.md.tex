Sections

SEARCH

\protect\hyperlink{site-content}{Skip to
content}\protect\hyperlink{site-index}{Skip to site index}

\href{https://www.nytimes.com/section/science}{Science}

\href{https://myaccount.nytimes.com/auth/login?response_type=cookie\&client_id=vi}{}

\href{https://www.nytimes.com/section/todayspaper}{Today's Paper}

\href{/section/science}{Science}\textbar{}SpaceX Crew Dragon Departs,
Carrying NASA Astronauts Toward Home

\url{https://nyti.ms/2D7F5Zm}

\begin{itemize}
\item
\item
\item
\item
\item
\end{itemize}

\href{https://www.nytimes.com/2020/08/02/science/spacex-astronauts-splashdown.html?action=click\&pgtype=Article\&state=default\&region=TOP_BANNER\&context=storylines_menu}{SpaceX's
Astronaut Trip}

\begin{itemize}
\tightlist
\item
  \href{https://www.nytimes.com/2020/08/02/science/spacex-astronauts-splashdown.html?action=click\&pgtype=Article\&state=default\&region=TOP_BANNER\&context=storylines_menu}{`Thanks
  for Flying SpaceX'}
\item
  \href{https://www.nytimes.com/2020/05/26/science/spacex-launch-nasa.html?action=click\&pgtype=Article\&state=default\&region=TOP_BANNER\&context=storylines_menu}{Why
  NASA Picked SpaceX}
\item
  \href{https://www.nytimes.com/interactive/2020/05/26/science/spacex-nasa.html?action=click\&pgtype=Article\&state=default\&region=TOP_BANNER\&context=storylines_menu}{Inside
  the Capsule}
\item
  \href{https://www.nytimes.com/2020/05/27/science/bob-behnken-doug-hurley.html?action=click\&pgtype=Article\&state=default\&region=TOP_BANNER\&context=storylines_menu}{Meet
  the Astronauts}
\end{itemize}

Advertisement

\protect\hyperlink{after-top}{Continue reading the main story}

Supported by

\protect\hyperlink{after-sponsor}{Continue reading the main story}

\hypertarget{spacex-crew-dragon-departs-carrying-nasa-astronauts-toward-home}{%
\section{SpaceX Crew Dragon Departs, Carrying NASA Astronauts Toward
Home}\label{spacex-crew-dragon-departs-carrying-nasa-astronauts-toward-home}}

Bob Behnken and Doug Hurley are getting ready to splash down after two
months in orbit.

\includegraphics{https://static01.nyt.com/images/2020/08/01/science/01sci-astronauts-capsule/merlin_175216434_9d678aba-9c92-426a-bf3f-45b5f10187b9-articleLarge.jpg?quality=75\&auto=webp\&disable=upscale}

\href{https://www.nytimes.com/by/kenneth-chang}{\includegraphics{https://static01.nyt.com/images/2018/02/16/multimedia/author-kenneth-chang/author-kenneth-chang-thumbLarge.jpg}}

By \href{https://www.nytimes.com/by/kenneth-chang}{Kenneth Chang}

\begin{itemize}
\item
  Aug. 1, 2020
\item
  \begin{itemize}
  \item
  \item
  \item
  \item
  \item
  \end{itemize}
\end{itemize}

Two astronauts who took the first commercial trip to orbit have left the
International Space Station. They are scheduled to return home on
Sunday.

The astronauts, Robert L. Behnken and Douglas G. Hurley, traveled to the
space station in May aboard a Crew Dragon capsule built and run by
SpaceX, the private rocket company started by Elon Musk.

The Crew Dragon undocked from the space station at 7:35 p.m. Eastern
time on Saturday, with brief thruster firings pushing the spacecraft
back.

As the capsule backed away from the station, Mr. Hurley thanked the
current crew of the space station and the teams on the ground that
helped manage their mission.

``We look forward to splashdown tomorrow,'' he said.

If the weather conditions remain favorable, it will splash down in the
Gulf of Mexico off Pensacola, Fla., at 2:48 p.m. on Sunday, NASA
announced.

A safe return would open up more trips to and from orbit for future
astronaut crews, and possibly space tourists, aboard the spacecraft.

\href{https://www.nytimes.com/2020/08/01/us/hurricane-isaias-track.html}{Isaias
is forecast to sweep up along the Atlantic coast of Florida} over the
weekend. NASA and SpaceX have seven splashdown sites in the Gulf of
Mexico and the Atlantic, but the track of the storm ruled out the three
in the Atlantic.

``We have confidence that the teams on the ground are, of course,
watching that much more closely than we are,'' Mr. Behnken said during a
news conference on Friday, ``and we won't leave the space station
without some good landing opportunities in front of us, good splashdown
weather in front of us.''

\hypertarget{how-can-i-watch-the-return-of-the-astronauts}{%
\subsection{How can I watch the return of the
astronauts?}\label{how-can-i-watch-the-return-of-the-astronauts}}

\href{https://www.nasa.gov/multimedia/nasatv/\#public}{NASA
Television}'s coverage of the undocking will continue through
splashdown. You can watch it in the video player below.

\hypertarget{what-will-happen-after-they-leave-the-station}{%
\subsection{What will happen after they leave the
station?}\label{what-will-happen-after-they-leave-the-station}}

The capsule is now performing a series of burns to move away from the
station and then line up with the splashdown site.

For much of the trip, Mr. Behnken and Mr. Hurley will be sleeping. Their
\href{https://twitter.com/NASA/status/1289245570565992449}{schedule sets
aside}a full night of rest.

Any return journey that exceeds six hours has to be long enough for the
crew to get some sleep between undocking and splashdown, Daniel Huot, a
NASA spokesman, said in an email.

Otherwise, because of the extended process that leads up to undocking,
the crew would end up working more than 20 hours straight, ``which is
not safe for dynamic operations like water splashdown and recovery,''
Mr. Huot said.

Just before a final burn that will drop the Crew Dragon out of orbit on
Sunday afternoon, it will jettison the bottom part of the spacecraft,
known as the trunk, which will then burn up in the atmosphere.

At re-entry, the Crew Dragon will be traveling at about 17,500 miles per
hour. Two small parachutes will deploy at an altitude of 18,000 feet
when the spacecraft has already been slowed by Earth's atmosphere to
about 350 miles per hour. The four main parachutes deploy at an altitude
of about 6,000 feet.

Once the capsule splashes in the water, it is expected to take 45 to 60
minutes to pluck them out.

\hypertarget{why-does-isaias-affect-the-departure}{%
\subsection{Why does Isaias affect the
departure?}\label{why-does-isaias-affect-the-departure}}

The storm complicated where splashdown could take place. At the
splashdown site, winds must be less than 10 miles per hour for the
capsule to land safely. There are additional constraints on waves, rain
and lightning. In addition, helicopters that take part in the recovery
of the capsule must be able to fly and land safely.

The first landing opportunity will aim for only the primary site,
Pensacola. If weather there is inconsistent with the rules, the capsule
and the astronauts will remain in orbit for another day or two, and
managers will consider the backup site, which is Panama City.

\hypertarget{returning-to-earth}{%
\subsection{Returning to Earth}\label{returning-to-earth}}

The SpaceX Crew Dragon is scheduled to splash down near Florida on
Sunday, though
\href{https://www.nytimes.com/interactive/2020/07/31/us/hurricane-isaias-tracker-map.html}{Hurricane
Isaias} could change those plans.

Category

2

1

Tropical storm

Forecasted path of Hurricane Isaias

Atlantic Ocean

Mon. 2 a.m.

Seven possible

splashdown sites

(approximate)

Gulf of Mexico

Sun. 2 a.m.

Sat. 2 a.m.

Last updated Sat. 9 a.m.

Category

2

1

Tropical storm

Forecasted path of Hurricane Isaias

Mon. 2 a.m.

Seven possible

splashdown sites

(approximate)

Gulf of Mexico

Sun. 2 a.m.

Sat. 2 a.m.

Last updated Sat. 9 a.m.

Category

2

1

Tropical storm

Forecasted path of Hurricane Isaias

Mon.

2 a.m.

Seven possible

splashdown sites

(approximate)

Gulf of Mexico

Sun.

2 a.m.

Last updated Sat. 9 a.m.

By The New York Times \textbar{} Sources: NASA, National Hurricane
Center, Mapbox, OpenStreetMap

\hypertarget{is-it-safer-to-land-on-water-or-on-land}{%
\subsection{Is it safer to land on water or on
land?}\label{is-it-safer-to-land-on-water-or-on-land}}

Spacecraft can safely return to Earth in either environment.

During the 1960s and 1970s, NASA's Mercury, Gemini and Apollo capsules
all splashed down in the ocean while Soviet capsules all ended their
trips on land. Russia's current Soyuz capsules continue to make ground
landings, as do China's astronaut-carrying Shenzhou capsules.

When Boeing's Starliner capsule begins carrying crews to the space
station, it will return on land, in New Mexico. SpaceX had originally
planned for the Crew Dragon to do ground landings, but decided that
water landings, employed for the earlier version of Dragon for taking
cargo, simplified the development of the capsule.

\hypertarget{why-is-the-return-trip-an-important-part-of-the-crew-dragons-first-flight}{%
\subsection{Why is the return trip an important part of the Crew
Dragon's first
flight?}\label{why-is-the-return-trip-an-important-part-of-the-crew-dragons-first-flight}}

After launch, re-entry through Earth's atmosphere is the second most
dangerous phase of spaceflight. Friction of air rushing past will heat
the bottom of the capsule to about 3,500 degrees Fahrenheit. A test
flight of the Crew Dragon last year successfully splashed down, so
engineers know the system works.

A successful conclusion to the trip opens the door to more people flying
to space. Some companies have already announced plans to use Crew
Dragons to lift wealthy tourists to orbit.

In the past, NASA astronauts launched on spacecraft like the Saturn 5
moon rocket and the space shuttles that NASA itself operated. After the
retirement of the space shuttles in 2011, NASA had to rely on Russia,
buying seats on the Soyuz capsules for trips to and from orbit.

Under the Obama administration, NASA hired two companies, SpaceX and
Boeing, to build spacecraft to take astronauts to the space station.
NASA financed much of the work to develop the spacecraft but will now
buy rides at fixed prices. For SpaceX, the trip by Mr. Behnken and Mr.
Hurley --- the first launch of astronauts from American soil since the
last space shuttle flight --- was the last major demonstration needed
before NASA officially certifies that the Crew Dragon is ready to begin
regular flights.

\hypertarget{who-are-the-astronauts}{%
\subsection{Who are the astronauts?}\label{who-are-the-astronauts}}

The astronauts are
\href{https://www.nytimes.com/2020/05/27/science/bob-behnken-doug-hurley.html}{Robert
L. Behnken and Douglas G. Hurley}, who have been friends and colleagues
since both were selected by NASA to be astronauts in 2000.

Both men have backgrounds as military test pilots and each has flown
twice before on space shuttle missions, although this is the first time
they have worked together on a mission. Mr. Hurley flew on the space
shuttle's final mission in 2011.

In 2015, they were among the astronauts chosen to work with Boeing and
SpaceX on the commercial space vehicles that the companies were
developing. In 2018, they were assigned to the first SpaceX flight.

\includegraphics{https://static01.nyt.com/images/2020/08/01/science/01sci-astronauts02/merlin_175087110_502c4955-3a19-4af0-ae99-6828fd764386-articleLarge.jpg?quality=75\&auto=webp\&disable=upscale}

\hypertarget{what-have-the-astronauts-been-doing-aboard-the-space-station}{%
\subsection{What have the astronauts been doing aboard the space
station?}\label{what-have-the-astronauts-been-doing-aboard-the-space-station}}

Originally, the mission was to last only up to two weeks, but Mr.
Behnken and Mr. Hurley ended up with a longer and busier stay at the
space station. Because of repeated delays by SpaceX and Boeing, NASA
ended up short-handed, with only one astronaut, Christopher J. Cassidy,
aboard the space station when the Crew Dragon and its two passengers
docked.

They stayed two months, helping Mr. Cassidy with space station chores.
Mr. Behnken and Mr. Cassidy performed four spacewalks to complete the
installation of new batteries on the space station. Mr. Hurley helped by
operating the station's robotic arm.

The men have also been contributing to science experiments in low earth
orbit. They assisted in
\href{https://www.nasa.gov/mission_pages/station/research/behnken-hurley-science-scrapbook}{a
study of water droplet formation} in the low gravity environment of the
space station using a shower head, and another that used fruit punch and
foam to look at \href{https://www.youtube.com/watch?v=2Dzx6b6vSK4}{how
to manage fluids in space}. They also
\href{https://www.nasa.gov/mission_pages/station/research/behnken-hurley-science-scrapbook}{helped
install new equipment inside the station} that will be used in future
scientific research.

Mr. Cassidy will remain aboard the station with two Russian astronauts,
Anatoly Ivanishin and Ivan Vagner. All three are to stay
\href{https://www.nasa.gov/sites/default/files/atoms/files/exp-63-summary.pdf}{on
board through October} when another crew of one American and two Russian
astronauts
\href{https://www.nasa.gov/press-release/nasa-astronaut-kate-rubins-crewmates-to-discuss-upcoming-spaceflight}{will
replace them}.

\hypertarget{when-are-the-next-crew-dragon-flights-and-who-will-they-carry}{%
\subsection{When are the next Crew Dragon flights, and who will they
carry?}\label{when-are-the-next-crew-dragon-flights-and-who-will-they-carry}}

The first operational flight of the Crew Dragon will launch no earlier
than late September. It will take three NASA astronauts --- Michael S.
Hopkins, Victor J. Glover and Shannon Walker --- and one Japanese
astronaut, Soichi Noguchi, to the space station.

The second operational flight, tentatively scheduled for February 2021,
will carry two NASA astronauts, Robert S. Kimbrough and K. Megan
McArthur; Akihiko Hoshide of Japan; and Thomas Pesquet of the European
Space Agency.

Ms. McArthur is married to Mr. Behnken.

Image

Credit...John Raoux/Associated Press

Advertisement

\protect\hyperlink{after-bottom}{Continue reading the main story}

\hypertarget{site-index}{%
\subsection{Site Index}\label{site-index}}

\hypertarget{site-information-navigation}{%
\subsection{Site Information
Navigation}\label{site-information-navigation}}

\begin{itemize}
\tightlist
\item
  \href{https://help.nytimes.com/hc/en-us/articles/115014792127-Copyright-notice}{©~2020~The
  New York Times Company}
\end{itemize}

\begin{itemize}
\tightlist
\item
  \href{https://www.nytco.com/}{NYTCo}
\item
  \href{https://help.nytimes.com/hc/en-us/articles/115015385887-Contact-Us}{Contact
  Us}
\item
  \href{https://www.nytco.com/careers/}{Work with us}
\item
  \href{https://nytmediakit.com/}{Advertise}
\item
  \href{http://www.tbrandstudio.com/}{T Brand Studio}
\item
  \href{https://www.nytimes.com/privacy/cookie-policy\#how-do-i-manage-trackers}{Your
  Ad Choices}
\item
  \href{https://www.nytimes.com/privacy}{Privacy}
\item
  \href{https://help.nytimes.com/hc/en-us/articles/115014893428-Terms-of-service}{Terms
  of Service}
\item
  \href{https://help.nytimes.com/hc/en-us/articles/115014893968-Terms-of-sale}{Terms
  of Sale}
\item
  \href{https://spiderbites.nytimes.com}{Site Map}
\item
  \href{https://help.nytimes.com/hc/en-us}{Help}
\item
  \href{https://www.nytimes.com/subscription?campaignId=37WXW}{Subscriptions}
\end{itemize}
