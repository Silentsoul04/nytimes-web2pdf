Sections

SEARCH

\protect\hyperlink{site-content}{Skip to
content}\protect\hyperlink{site-index}{Skip to site index}

\href{https://www.nytimes.com/section/science}{Science}

\href{https://myaccount.nytimes.com/auth/login?response_type=cookie\&client_id=vi}{}

\href{https://www.nytimes.com/section/todayspaper}{Today's Paper}

\href{/section/science}{Science}\textbar{}How Woody Vines Do the Twist

\url{https://nyti.ms/3fmIWii}

\begin{itemize}
\item
\item
\item
\item
\item
\end{itemize}

Advertisement

\protect\hyperlink{after-top}{Continue reading the main story}

Supported by

\protect\hyperlink{after-sponsor}{Continue reading the main story}

Trilobites

\hypertarget{how-woody-vines-do-the-twist}{%
\section{How Woody Vines Do the
Twist}\label{how-woody-vines-do-the-twist}}

Slowly, scientists are learning how lianas quickly climb.

\includegraphics{https://static01.nyt.com/images/2020/08/04/science/00SCI-VINES1/merlin_173207808_dc81372e-088e-4bad-b732-3d78f5d2c922-articleLarge.jpg?quality=75\&auto=webp\&disable=upscale}

By \href{https://www.nytimes.com/by/devi-lockwood}{Devi Lockwood}

\begin{itemize}
\item
  Aug. 1, 2020
\item
  \begin{itemize}
  \item
  \item
  \item
  \item
  \item
  \end{itemize}
\end{itemize}

Wood is typically thought of as stiff and rigid, but some wood, in the
race upward to access the best sunlight, twists.
\href{http://www.lianaecologyproject.com/}{Lianas}, or woody vines, are
concentrated in tropical forests; they possess a narrow stem that lets
them climb to the top of the canopy, more than 100 feet above the
ground, as quickly as possible by twisting their way around tree trunks.
Basking in the sun at the top, these vines flower, fruit and lay out new
leaves as they photosynthesize.

But the number of lianas is increasing in tropical forests
\href{https://www.nytimes.com/2011/05/24/science/24vine.html}{relative
to trees}, and their overabundance can hamper a forest's ability to
\href{https://www.nytimes.com/2015/10/13/science/study-quantifies-liana-vines-threat-to-forests.html}{store
carbon}, so botanists are eager to learn as much about these plants as
they can.

``We understand a lot about their ecology, but we don't understand how
these diverse and strange wood forms evolved,'' said Joyce Chery, a
botanist at Cornell, and the lead author of a study
\href{https://www.cell.com/current-biology/fulltext/S0960-9822(19)31442-3?_returnURL=https\%3A\%2F\%2Flinkinghub.elsevier.com\%2Fretrieve\%2Fpii\%2FS0960982219314423\%3Fshowall\%3Dtrue}{published}
earlier this year in the journal Current Biology.

In early 2017, as a graduate student, Dr. Chery visited the Smithsonian
Tropical Research Institute in Panama, where she collected cross-section
samples of various species of Paullinia, a lineage of liana. Those
samples are now part of the herbaria at the University of California,
Berkeley, and the University of Panama.

Dr. Chery extracted DNA from the leaves and analyzed the molecular
sequence of each sample, and of similar samples stored at herbaria at
the University of Panama, Universidad Nacional Autónoma de México and
the Smithsonian Institution. She also studied the configuration of cells
in 148 samples of cross-sections of the stems.

\begin{center}\rule{0.5\linewidth}{\linethickness}\end{center}

\includegraphics{https://static01.nyt.com/images/2020/08/04/science/00SCI-VINES2/00SCI-VINES2-articleLarge.jpg?quality=75\&auto=webp\&disable=upscale}

From this analysis, Dr. Chery and the co-authors on the recent paper
identified five patterns of stem growth, ranging from circular to lobed,
to star-shaped cross-sections.

The driving force behind each of these patterns is a bundle of cells
behind the bark called the vascular cambium. To survive, a woody vine
must be both strong and flexible --- variant shapes allow woody vines to
make the twists and turns they need to be successful in the tropics.
Their sugar- and water-conducting cells are positioned in irregular
ways, far different than they would be in run-of-the-mill trees or
shrubs.

``Whereas trees all tend to be the same shape, lianas are all over the
place,'' said
\href{https://www.marquette.edu/biology/directory/schnitzer.php}{Stefan
Schnitzer}, a botanist at Marquette University who was not involved in
the study.

These strange stem variations give the vines an advantage. ``Being
asymmetrical helps you to anchor in the trees you're growing on,'' said
\href{http://www.ib.unam.mx/directorio/234}{Marcelo Rodrigo Pace}, a
botanist at Universidad Nacional Autónoma de México and a co-author of
the study. ``These lianas also have tendrils that let them grab pieces
of stems and leaves and start growing.''

This adaptation is ``purely mechanical, architectural,'' he said. ``It's
better than being slippery and cylindrical.''

The study considered two scales of time: an individual plant's life, and
a longer, evolutionary breadth. Dr. Chery and her colleagues found that
in a single plant's early development, when the liana is leafy, green
and small, woody vines already have an unusual tissue formation. The
stem is star-shaped rather than circular; the vascular bundles are
scattered in the lobes of the star-shaped body and absent in the arcs.
At later stages, this lobed structure can lead to more unusual growth
patterns.

Over evolutionary time, vines of different groups developed various
mechanisms to contort their stems. The paper's authors found that the
five different atypical forms found in mature liana stems trace their
evolutionary history back to a common disturbance to the young plant's
development: the lobed stem.

``This is exciting because it's one step away from saying that this
leads in perfectly to understanding how lianas do what they do,'' Dr.
Schnitzer said. While lianas share most characteristics with trees, like
producing wood and thriving in similar environmental conditions, the two
plant types invest differently in certain parts of their composition.
Lianas have more cells related to being flexible, whereas trees
prioritize being stiff and tough. Both have cells responsible for
stiffness and flexibility in differing ratios.

``They have the same ingredients, but the proportion of those
ingredients is distributed differently,'' Dr. Chery said.

\textbf{\emph{{[}}\href{http://on.fb.me/1paTQ1h}{\emph{Like the Science
Times page on Facebook.}}} ****** \emph{\textbar{} Sign up for the}
\textbf{\href{http://nyti.ms/1MbHaRU}{\emph{Science Times
newsletter.}}\emph{{]}}}

Advertisement

\protect\hyperlink{after-bottom}{Continue reading the main story}

\hypertarget{site-index}{%
\subsection{Site Index}\label{site-index}}

\hypertarget{site-information-navigation}{%
\subsection{Site Information
Navigation}\label{site-information-navigation}}

\begin{itemize}
\tightlist
\item
  \href{https://help.nytimes.com/hc/en-us/articles/115014792127-Copyright-notice}{©~2020~The
  New York Times Company}
\end{itemize}

\begin{itemize}
\tightlist
\item
  \href{https://www.nytco.com/}{NYTCo}
\item
  \href{https://help.nytimes.com/hc/en-us/articles/115015385887-Contact-Us}{Contact
  Us}
\item
  \href{https://www.nytco.com/careers/}{Work with us}
\item
  \href{https://nytmediakit.com/}{Advertise}
\item
  \href{http://www.tbrandstudio.com/}{T Brand Studio}
\item
  \href{https://www.nytimes.com/privacy/cookie-policy\#how-do-i-manage-trackers}{Your
  Ad Choices}
\item
  \href{https://www.nytimes.com/privacy}{Privacy}
\item
  \href{https://help.nytimes.com/hc/en-us/articles/115014893428-Terms-of-service}{Terms
  of Service}
\item
  \href{https://help.nytimes.com/hc/en-us/articles/115014893968-Terms-of-sale}{Terms
  of Sale}
\item
  \href{https://spiderbites.nytimes.com}{Site Map}
\item
  \href{https://help.nytimes.com/hc/en-us}{Help}
\item
  \href{https://www.nytimes.com/subscription?campaignId=37WXW}{Subscriptions}
\end{itemize}
