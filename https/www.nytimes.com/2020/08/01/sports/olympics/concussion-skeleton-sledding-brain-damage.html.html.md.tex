Sections

SEARCH

\protect\hyperlink{site-content}{Skip to
content}\protect\hyperlink{site-index}{Skip to site index}

\href{/section/sports/olympics}{Olympics}\textbar{}This Sledding Team
Trained Hard for Gold in 2010. Some Members Regret It.

\url{https://nyti.ms/2Peb1Oj}

\begin{itemize}
\item
\item
\item
\item
\item
\end{itemize}

\includegraphics{https://static01.nyt.com/images/2020/07/29/sports/29SKELETON-top-1/merlin_174528426_06653e10-f239-4dc0-ad27-840a38122d93-articleLarge.jpg?quality=75\&auto=webp\&disable=upscale}

\hypertarget{this-sledding-team-trained-hard-for-gold-in-2010-some-members-regret-it}{%
\section{This Sledding Team Trained Hard for Gold in 2010. Some Members
Regret
It.}\label{this-sledding-team-trained-hard-for-gold-in-2010-some-members-regret-it}}

In skeleton, the headfirst Olympic sledding sport, the opportunity for
unlimited training on the track can be a huge advantage. But Canadian
Olympians who had such access believe it was bad for their brains.

Credit...Doug Pensinger/Getty Images

Supported by

\protect\hyperlink{after-sponsor}{Continue reading the main story}

\href{https://www.nytimes.com/by/matthew-futterman}{\includegraphics{https://static01.nyt.com/images/2020/02/24/reader-center/author-matthew-futterman/author-matthew-futterman-thumbLarge.png}}

By \href{https://www.nytimes.com/by/matthew-futterman}{Matthew
Futterman}

\begin{itemize}
\item
  Aug. 1, 2020
\item
  \begin{itemize}
  \item
  \item
  \item
  \item
  \item
  \end{itemize}
\end{itemize}

The chance to host the
\href{https://www.nytimes.com/2010/02/10/sports/olympics/10podium.html}{2010
Winter Games} was supposed to be a godsend for Canadian athletes who
compete in skeleton, the headfirst sled run down a twisting track.

While most competitors get access to the track for just a handful of
days leading up to the Olympics, the host country gets to practice far
more, because its athletes are logistically closer and the sport's rules
allow it. The home team can memorize every detail of every turn on run
after treacherous run.

Mellisa Hollingsworth, who was favored to win a medal that year in
skeleton, said she and her teammates took as many as 11 runs a day down
the track, the fastest in the world, at Whistler, British Columbia,
about 75 miles north of Vancouver. When a training session ended, they
were so worn out they struggled to put sentences together. Noise was
intolerable. Their brains felt scrambled.

And that's how Hollingsworth, now 39, and her teammates became case
studies in a process that is beginning to realign how
\href{https://www.ncbi.nlm.nih.gov/pmc/articles/PMC6153360/}{neuroscientists
and a handful of coaches and athletes understand} the connection between
brain injury and sliding sports.

\includegraphics{https://static01.nyt.com/images/2020/08/03/sports/29SKELETON3-print/merlin_174568515_ff3a1aa7-4b71-49ae-b12a-5f11795575ee-articleLarge.jpg?quality=75\&auto=webp\&disable=upscale}

``We overdid it,'' said
\href{https://olympic.ca/2013/08/07/how-the-cfl-is-redefining-bobsleigh-in-canada/}{Nathan
Cicoria, a high performance coach for Canada's bobsled and skeleton team
from 2006-14}. ``I just wish we knew then what we know now. You learn
things.''

During the last decade, football and other contact sports have received
most of the attention and research interest for traumatic brain injuries
in sports.

By comparison, sliding sports, niche activities that require athletes to
careen down twisting tracks of ice on sleds at 80 miles per hour, have
been largely ignored. And yet, for years, elite competitors have talked
about the mental fog, headaches, inability to eat or speak effectively,
and sensitivity to light and sound that a day of training, or, for some,
even a single routine run can produce.

They called it
\href{https://www.nytimes.com/2020/07/26/sports/olympics/olympics-bobsled-suicide-brain-injuries.html}{``sled
head.''} It was just something they had to accept, like cold weather, or
sore muscles.

Now, in retirement, many of these athletes continue to struggle with
many of those same symptoms, as well as forgetfulness, depression and
mental illness.

Former top competitors like Hollingsworth, who finished fifth in
skeleton at the Vancouver Games, Pascal Richard, also of Canada, and
\href{https://www.nytimes.com/2020/06/06/sports/olympics/coronavirus-olympic-training-tokyo.html}{Katie
Uhlaender, a four-time Olympian from the United States} who wants to
make one last Olympic team, wonder whether those symptoms are connected
to their dramatic crashes and the brain-rattling runs.

They have watched teammates descend into
\href{https://www.teamusa.org/News/2020/May/09/Pavle-Jovanovic-2006-Olympic-Bobsledder-Dies-By-Suicide}{depression
and die by suicide}. Since 2013, three former elite North American
bobsledders have taken their lives. Another attempted it, and two others
died of overdoses, a remarkable number given that just a few hundred
athletes participate seriously in sliding sports at any level at once.

Image

Katie Uhlaender wonders about the effects of brain-rattling runs on the
track.Credit...Doug Mills/The New York Times

``It's almost like the boxers all over again,'' said Peter McCarthy, a
neurophysiologist at the University of South Wales who has studied the
dynamics of skeleton by attaching motion sensors to the athletes. ``What
you are doing is taking someone's head and giving it a really good shake
around, but in this case it lasts for a minute.''

McCarthy has been working closely with
\href{https://www.ibsf.org/en/news/9-skeleton/20677-head-coach-talent-mark-wood-leaves-british-bobsleigh-skeleton-association}{Mark
Wood of Britain}, who has coached multiple medalists in skeleton and is
now on a crusade to make people understand that allowing an athlete to
train or compete with ``sled head'' is akin to subjecting someone with
concussion-like symptoms to 500 more slaps to the head.

People within the sport keep telling him he is going to ruin it.

``I say, `No I'm not. I'm going to make it safer,''' said Wood, who has
coached for Canada, Britain and China. ``The more data we get, the
better information we can give.''

For many athletes though, the data is arriving too late.

In 1998, Pascal Richard was heading into the sixth of 19 curves, about
one-third of the way down the track in La Plagne, France, the same one
used for the 1992 Albertville Games. The gravitational acceleration
forces spiked and slammed his face into the ice. The impact knocked him
out. He remained unconscious the whole way down as he crossed the
finish. Richard returned to training the next day.

Image

``I have lost part of my life,'' said Pascal Richard, left, a former
Olympic athlete from Canada, whose cognitive and psychological problems
forced him to retire from the Royal Canadian Mounted
Police.Credit...Amber Bracken for The New York Times

Neck pain and problems with concentration lasted through the following
summer, and the chronic fogginess increased as Richard pushed to make
the 2002 Olympics in Salt Lake City, where he finished 15th. He retired
after those Games, returning to his full-time job as a member of the
Royal Canadian Mounted Police.

Richard soon started falling in and out of depression. He lost his
temper easily. A single hit in a beer-league hockey game would put him
out for the season. Work became too challenging, as he struggled to
remember details of investigations and Canada's penal code, forcing him
to retire.

``My wife would tell you I'm not the person I used to be,'' said
Richard, who lives outside Calgary, Alberta. ``I could have a great
friend who called me on the phone and it could take me awhile to figure
out who it is. I have lost part of my life.''

He is 48, has young children and would like to find something else to
do. He said he doesn't have the energy.

No one can say for sure whether skeleton is solely responsible for
Richard's downfall or anyone else's, or how many runs it took Richard to
get where he is today. He played other contact sports growing up. He
suffers from post-traumatic stress disorder from coming upon so many
grisly death scenes during his career with the Mounties, especially one
in which he could not rescue a man stuck in the driver's seat of a van
that was on fire.

Image

Richard retired after the 2002 Winter Games but still struggles with
cognitive problems.Credit...Amber Bracken for The New York Times

All of that could contribute to brain injury and depression.

But Tyson Plesuk has seen enough skeleton to be convinced that too many
runs can pose serious danger to the brain.

Plesuk, a top sports physiotherapist in Canada, grew up playing hockey.
He suffered three diagnosed concussions, and probably many that went
undiagnosed. In 2010 when he became a physiotherapist with Canada's
skeleton team, he knew little about the sport.

As Plesuk began spending time with Hollingsworth and other team members,
he noticed how much they needed to sleep when they were not training,
how sometimes they could not eat or talk to each other during their
lunch breaks. ``It's not normal behavior, but we needed someone from the
outside for us to understand that,'' Hollingsworth said.

At the beginning of the season, the athletes had taken a test to get a
baseline for their cognitive functions. If they crashed and suffered a
head injury they would have to take the test again, and they could not
train or compete until their performance had returned to the baseline,
even if scans of their brains looked clear.

Plesuk detected a problem though: The athletes could pass the test even
when they had other symptoms of a concussion. Fearing they might miss a
chance to train or lose coveted spots on the team, they wouldn't dare
mention feeling weak to their coaches.

Image

Duff Gibson won the gold medal in skeleton in 2006. As a coach he pushed
for limits on how many runs athletes can take each day.Credit...Domenico
Stinellis/Associated Press

As Plesuk and Duff Gibson, the team's head coach and the 2006 Olympic
skeleton champion, got to know the athletes better during the 2010-11
season, they noticed that many who struggled the most with the
concussion symptoms had participated in the high-volume training leading
up to the Vancouver Games.

Gibson can still remember when it was a point of pride for an athlete to
finish a skeleton run with a bloody nose from banging their face on the
ice. ``The further back you go in history, the more cave man it
becomes,'' he said.

Now he understood how all the training likely left his athletes more
vulnerable to repeated brain injuries and its symptoms, as run after run
over tracks that look smooth but are really covered with bumps and
divots can cause micro-tears in brain tissue, even if there is never a
crash.

``The big thing is the repetitive shaking,'' Plesuk said during a recent
interview.

Image

Uhlaender, competing in Sochi in 2014, is trying to make her fifth
Olympic team in 2022.Credit...Doug Mills/The New York Times

Gibson and Plesuk decided to limit runs to three per day for every
athlete who competed for Canada. If an athlete didn't seem ``right,''
they pulled her from competition, no matter the circumstances.

``If you see stars, that is not normal, and if you have a headache after
a run, that is not a normal condition,'' Gibson said.

Heading into the 2014 Sochi Olympics, Hollingsworth got pulled from a
race, which resulted in a lower starting slot and may have contributed
to her 11th-place finish.

Hollingsworth knew Gibson and Plesuk had made the right call. Leading up
to those Games in Russia, she struggled to wake from naps after a hard
morning of training. Hours passed before she could walk 10 normal steps.
One afternoon she came down with vertigo while visiting a sporting goods
store and ended up curled in a ball on the floor.

She retired after Sochi. She can struggle to remember details of even
recent experiences. She recalls little of what happened during the few
years leading up to Vancouver; even races she won, moments that should
stand out, are a blur, or have disappeared altogether. She has no
recollection of her first skeleton run when she was a teenager.

She can't be in loud or busy places. After a concert, she can't sleep
for a night or two. A small restaurant with a lot of chatter can make
her ears ring.

She will not recruit athletes to compete in the sport that was once her
life.

Last year, WinSport, Canada's winter sports organization, began
dismantling the Calgary bobsled and skeleton track where Hollingsworth
started. After 30 years, it was deemed at the end of its life cycle. As
the track came down, Hollingsworth said she felt nostalgia, but also
something else --- comfort that no one would get hurt there anymore.

Image

Hollingsworth keeps her sled on display at her home. She does not
recruit young athletes to her sport.Credit...Amber Bracken for The New
York Times

Advertisement

\protect\hyperlink{after-bottom}{Continue reading the main story}

\hypertarget{site-index}{%
\subsection{Site Index}\label{site-index}}

\hypertarget{site-information-navigation}{%
\subsection{Site Information
Navigation}\label{site-information-navigation}}

\begin{itemize}
\tightlist
\item
  \href{https://help.nytimes.com/hc/en-us/articles/115014792127-Copyright-notice}{©~2020~The
  New York Times Company}
\end{itemize}

\begin{itemize}
\tightlist
\item
  \href{https://www.nytco.com/}{NYTCo}
\item
  \href{https://help.nytimes.com/hc/en-us/articles/115015385887-Contact-Us}{Contact
  Us}
\item
  \href{https://www.nytco.com/careers/}{Work with us}
\item
  \href{https://nytmediakit.com/}{Advertise}
\item
  \href{http://www.tbrandstudio.com/}{T Brand Studio}
\item
  \href{https://www.nytimes.com/privacy/cookie-policy\#how-do-i-manage-trackers}{Your
  Ad Choices}
\item
  \href{https://www.nytimes.com/privacy}{Privacy}
\item
  \href{https://help.nytimes.com/hc/en-us/articles/115014893428-Terms-of-service}{Terms
  of Service}
\item
  \href{https://help.nytimes.com/hc/en-us/articles/115014893968-Terms-of-sale}{Terms
  of Sale}
\item
  \href{https://spiderbites.nytimes.com}{Site Map}
\item
  \href{https://help.nytimes.com/hc/en-us}{Help}
\item
  \href{https://www.nytimes.com/subscription?campaignId=37WXW}{Subscriptions}
\end{itemize}
