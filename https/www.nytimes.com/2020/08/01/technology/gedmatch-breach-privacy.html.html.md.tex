Sections

SEARCH

\protect\hyperlink{site-content}{Skip to
content}\protect\hyperlink{site-index}{Skip to site index}

\href{https://www.nytimes.com/section/technology}{Technology}

\href{https://myaccount.nytimes.com/auth/login?response_type=cookie\&client_id=vi}{}

\href{https://www.nytimes.com/section/todayspaper}{Today's Paper}

\href{/section/technology}{Technology}\textbar{}Why a Data Breach at a
Genealogy Site Has Privacy Experts Worried

\url{https://nyti.ms/2BMBLCj}

\begin{itemize}
\item
\item
\item
\item
\item
\end{itemize}

Advertisement

\protect\hyperlink{after-top}{Continue reading the main story}

Supported by

\protect\hyperlink{after-sponsor}{Continue reading the main story}

\hypertarget{why-a-data-breach-at-a-genealogy-site-has-privacy-experts-worried}{%
\section{Why a Data Breach at a Genealogy Site Has Privacy Experts
Worried}\label{why-a-data-breach-at-a-genealogy-site-has-privacy-experts-worried}}

Nearly two-thirds of GEDmatch's users opt out of helping law
enforcement. For a brief window this month, that didn't matter.

\includegraphics{https://static01.nyt.com/images/2020/07/30/arts/00xp-GEDmatch-pix-sub/00xp-GEDmatch-pix-sub-articleLarge.jpg?quality=75\&auto=webp\&disable=upscale}

By \href{https://www.nytimes.com/by/heather-murphy}{Heather Murphy}

\begin{itemize}
\item
  Aug. 1, 2020
\item
  \begin{itemize}
  \item
  \item
  \item
  \item
  \item
  \end{itemize}
\end{itemize}

The peculiar matches began early on a Sunday morning. Across the world,
genealogists found that they had numerous new relatives on GEDmatch, a
website known for its role in helping crack the
\href{https://www.nytimes.com/2020/06/29/us/golden-state-killer-joseph-deangelo.html}{Golden
State Killer case}.

New relatives are typically cause for celebration among genealogists.
But upon close inspection, experienced users noticed that some of the
new relatives seemed to be the DNA equivalent of a Twitter bot or a
Match.com scammer; the DNA did things that actual people's DNA should
not be able to do.

Others seemed to be suspected murderers and rapists, uploaded by
genealogists working with law enforcement. Users knew that the police
sometimes used the site to try to identify DNA found at crime scenes.
But users found the new profiles strange because they also knew that
profiles made for law enforcement purposes were supposed to be hidden to
prevent tipping off or upsetting a suspect's relatives amid an
investigation. What really drew attention, however, was the fact that
all one million or so users who had opted not to help law enforcement
had been forced to opt in.

\begin{quote}
Gedmatch back up and all kits are still currently switched to police
accessible \url{https://t.co/nh91rxpIBI}
\href{https://t.co/rN9wHdqSM9}{pic.twitter.com/rN9wHdqSM9}

--- Graham Coop (@Graham\_Coop)
\href{https://twitter.com/Graham_Coop/status/1284882121014702080?ref_src=twsrc\%5Etfw}{July
19, 2020}
\end{quote}

GEDmatch, a longstanding family history site containing around 1.4
million people's genetic information, had experienced a data breach. The
peculiar matches were not new uploads but rather the result of two
back-to-back hacks, which overrode existing user settings, according to
Brett Williams, the chief executive of Verogen, a forensic company that
has owned GEDmatch
\href{https://slate.com/technology/2019/12/gedmatch-verogen-genetic-genealogy-law-enforcement.html}{since
December}.

Though the growth of genealogy sites has
\href{https://www.cnbc.com/2020/01/23/23andme-lays-off-100-people-ceo-anne-wojcicki-explains-why.html}{slowed
slightly} in recent years, their
\href{https://www.nytimes.com/2019/04/25/us/golden-state-killer-dna.html}{use
by the police} has increased. After the authorities in California used
GEDmatch in 2018 to identify
\href{https://www.nytimes.com/2018/04/26/us/joseph-james-deangelo.html}{a
suspect} in the decades-long Golden State Killer case, police
departments across the country began to dig through their cold case
files in the hopes that this new technique could solve old crimes.

And GEDmatch was often their preferred site. Unlike the genealogy
services Ancestry and 23andMe, which are marketed to people who are new
to using DNA to learn about themselves, GEDmatch caters to more advanced
researchers. The site appeals to the police because it allows DNA that
has been processed elsewhere to be uploaded. Verogen has a long history
of working with law enforcement, and the acquisition of GEDmatch further
solidified this collaboration.

Scientists and genealogists say the GEDmatch breach ---~which exposed
more than a million additional profiles to law enforcement officials ---
offers an important window into what can go wrong when those responsible
for storing genetic information fail to take necessary precautions.

In an interview, Mr. Williams said that the first breach occurred early
on July 19. After shutting down the site, his team ``covered up the
vulnerability,'' he said, and brought it back online, but only briefly.
``On Monday we took the site down again because it was clear the hackers
were trying again,'' he said.

This time the site remained down for nearly a week. ``We're taking an
abundance of caution because we don't want to end up in the same
situation again,'' Mr. Williams said.

Mr. Williams said he had hired an outside security team and contacted
the F.B.I. to see if the agency would investigate. The F.B.I. did not
respond to a request for comment.

All was far from resolved when the site's settings were restored, said
Debbie Kennett, a genealogist in England,
who\href{https://cruwys.blogspot.com/2020/07/major-privacy-breach-at-gedmatch.html}{wrote
about} the breach on her blog. We're stuck with our DNA for life, she
said. ``Once it's out there it's not like an email address you can
change,'' she said in an interview. Because of its interconnected
nature, she added, when any one person's genetic information is exposed,
the exposed DNA can potentially affect their family members too.

In a paper published
\href{https://www.ucdavis.edu/news/hobbyist-dna-services-may-be-open-genetic-hacking/}{last
year}, Michael Edge, a professor of biological sciences at the
University of Southern California, and fellow researchers warned several
genealogy websites that they were vulnerable to data breaches.

``Of course, hacks happen to lots of companies, even entities that take
security very seriously,'' he said. ``At the same time, GEDmatch's, and
eventually Verogen's, response to our paper didn't inspire much
confidence that they were taking it seriously.'' Other genealogy
websites, he added, seemed more open to the researchers' recommendations
for improving security.

For many, the presence of fake users in GEDmatch was as alarming as the
breach itself. Genealogists know that they cannot trust names or emails.
They also know that a user can easily upload someone else's genetic
profile. But the breach exposed that behind the scenes, hidden by
privacy settings, were all kinds of profiles of people who were not even
real.

The giveaway that the matches were not actual relatives was that their
DNA was too good to be true, said Leah Larkin, a biologist who runs
\href{https://thednageek.com/about/}{DNA Geek,} a genealogical research
company. People who managed profiles for many clients and relatives
repeatedly found that these fake users somehow were displayed as close
relatives across the unrelated profiles. Their visible ancestry
information reinforced the matches were impossible and suggested the
fake profiles had been designed to trick the site's search algorithm for
some reason.

In Dr. Edge's paper, he warned that it was possible to create fake
profiles to identify people with genetic variants associated with
Alzheimer's and other diseases.

``If something is just a geeky genealogist messing around, there is no
concern,'' Dr. Larkin said. But it becomes a problem, she said, if users
are trying to find people who all share a particular genetic mutation or
trait, as Dr. Edge cautioned. Such information could be abused by
insurance companies, pharmaceutical companies or others, she said.

The breach also reinforced something that genealogists have been saying
for years: Mixing genealogy and law enforcement is messy, even when you
try to draw clear lines. Until two years ago, the primary DNA databases
that law enforcement used for investigations were maintained by the
F.B.I. and the police. That changed with the Golden State Killer case in
2018.

As police departments rushed to reinvestigate cold cases, GEDmatch,
which at the time was
\href{https://www.nytimes.com/2018/10/15/science/gedmatch-genealogy-cold-cases.html}{run
by two family history hobbyists as a sort of passion project}, tried to
serve two audiences: genealogists who simply wanted to trace their
family tree and law enforcement officials who wanted to know if a murder
or a rapist was hiding in one of its branches. Amid a backlash, GEDmatch
\href{https://www.legalgenealogist.com/2019/05/19/gedmatch-reverses-course/}{changed
its policy} in May 2019 so that only users who explicitly opted to help
law enforcement would show up in police searches. Still, there
\href{https://www.nytimes.com/2019/10/05/us/genetic-genealogy-guidelines-privacy.html}{is
little regulation} around how the authorities can use GEDmatch and other
genealogy databases, so it's largely up to the companies and their users
to police themselves.

And as the breach demonstrated, users' wishes could be quickly
overridden.

For some users, the reason for keeping their profiles private is
philosophical. Even if helping law enforcement could mean helping catch
a killer, they do not want their genetic information used to incriminate
their relatives. Others, like Carolynn ni Lochlainn, a genealogist from
Huntington, N.Y., keep their profiles private because they worry the
data will be improperly used to arrest innocent people.

``I work with a lot of Black clients and cousins, and I was most angered
by the inexcusable risk at which they were placed,'' Ms. ni Lochlainn,
said.

Colleen Fitzpatrick, the founder of Identifinders International, which
applies forensic genealogy techniques toward identifying unclaimed
remains and suspects in crimes, oversees a team that relies heavily on
GEDmatch.

Her team was affected differently than the genealogists' clients. They
had uploaded DNA from crime scenes and unidentified babies who had been
abandoned by their mothers. Because they'd checked the law enforcement
box, these profiles were not supposed to show up in their relative's
searches. For a brief window in time, ``the whole database, they could
see us,'' she said.

She said it was unlikely that anyone working with law enforcement had
exploited the breach to obtain a match against a relative's will, given
the short amount of time involved. ``It wasn't this magnificent reveal
that we're going to cash in on,'' she said.

Nonetheless, the breach undeniably undermined trust for all, she said.
``I think Verogen needs to up its game,'' she said.

Advertisement

\protect\hyperlink{after-bottom}{Continue reading the main story}

\hypertarget{site-index}{%
\subsection{Site Index}\label{site-index}}

\hypertarget{site-information-navigation}{%
\subsection{Site Information
Navigation}\label{site-information-navigation}}

\begin{itemize}
\tightlist
\item
  \href{https://help.nytimes.com/hc/en-us/articles/115014792127-Copyright-notice}{©~2020~The
  New York Times Company}
\end{itemize}

\begin{itemize}
\tightlist
\item
  \href{https://www.nytco.com/}{NYTCo}
\item
  \href{https://help.nytimes.com/hc/en-us/articles/115015385887-Contact-Us}{Contact
  Us}
\item
  \href{https://www.nytco.com/careers/}{Work with us}
\item
  \href{https://nytmediakit.com/}{Advertise}
\item
  \href{http://www.tbrandstudio.com/}{T Brand Studio}
\item
  \href{https://www.nytimes.com/privacy/cookie-policy\#how-do-i-manage-trackers}{Your
  Ad Choices}
\item
  \href{https://www.nytimes.com/privacy}{Privacy}
\item
  \href{https://help.nytimes.com/hc/en-us/articles/115014893428-Terms-of-service}{Terms
  of Service}
\item
  \href{https://help.nytimes.com/hc/en-us/articles/115014893968-Terms-of-sale}{Terms
  of Sale}
\item
  \href{https://spiderbites.nytimes.com}{Site Map}
\item
  \href{https://help.nytimes.com/hc/en-us}{Help}
\item
  \href{https://www.nytimes.com/subscription?campaignId=37WXW}{Subscriptions}
\end{itemize}
