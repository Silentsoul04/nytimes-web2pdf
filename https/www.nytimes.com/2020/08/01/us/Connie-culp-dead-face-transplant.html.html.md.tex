Sections

SEARCH

\protect\hyperlink{site-content}{Skip to
content}\protect\hyperlink{site-index}{Skip to site index}

\href{https://www.nytimes.com/section/us}{U.S.}

\href{https://myaccount.nytimes.com/auth/login?response_type=cookie\&client_id=vi}{}

\href{https://www.nytimes.com/section/todayspaper}{Today's Paper}

\href{/section/us}{U.S.}\textbar{}Connie Culp, First Face Transplant
Recipient in U.S., Dies at 57

\url{https://nyti.ms/3jVN6kK}

\begin{itemize}
\item
\item
\item
\item
\item
\end{itemize}

Advertisement

\protect\hyperlink{after-top}{Continue reading the main story}

Supported by

\protect\hyperlink{after-sponsor}{Continue reading the main story}

\hypertarget{connie-culp-first-face-transplant-recipient-in-us-dies-at-57}{%
\section{Connie Culp, First Face Transplant Recipient in U.S., Dies at
57}\label{connie-culp-first-face-transplant-recipient-in-us-dies-at-57}}

Her near-total face transplant in 2008 was the most complex at the time.
She was the fourth patient in the world to undergo such a procedure.

\includegraphics{https://static01.nyt.com/images/2020/08/01/multimedia/01xp-connieculp-pix1/merlin_175200339_febffdf5-1c8c-47cb-96b2-27c133295c8d-articleLarge.jpg?quality=75\&auto=webp\&disable=upscale}

By \href{https://www.nytimes.com/by/bryan-pietsch}{Bryan Pietsch}

\begin{itemize}
\item
  Aug. 1, 2020
\item
  \begin{itemize}
  \item
  \item
  \item
  \item
  \item
  \end{itemize}
\end{itemize}

Connie Culp, the first patient in the United States to receive a face
transplant, died on Wednesday at the Cleveland Clinic, which had
performed her procedure in 2008. She was 57.

She died of complications from an infection that was unrelated to her
transplant, a spokeswoman for the hospital said. The clinic confirmed
Ms. Culp's death
\href{https://twitter.com/CleClinicNews/status/1289217253158731777?s=20}{on
Twitter}.

Ms. Culp was the longest-living face transplant patient in the world,
the spokeswoman said.

``She was a great pioneer and her decision to undergo a
sometimes-daunting procedure is an enduring gift for all of humanity,''
said Dr. Frank Papay, chairman of the Cleveland Clinic's Dermatology and
Plastic Surgery Institute.

Dr. Papay was part of the surgical team that performed Ms. Culp's
\href{https://www.nytimes.com/2008/12/18/health/s18face.html}{23-hour
operation} in 2008, which replaced her damaged face with that of a
recently deceased woman.

It was the most extensive and complicated face transplant
\href{https://www.nytimes.com/2008/12/17/health/17face.html}{at the
time}. Three face transplants had been done before Ms. Culp's: two in
France and one in China.

A Cleveland Clinic ethics committee had approved such a procedure only
in 2004, the first such committee to do so. Dr. Eric Kodish, the then
chairman of the clinic's bioethics department,
\href{https://www.nytimes.com/2008/12/18/health/s18face.html}{told The
New York Times} after the transplant that Ms. Culp had undergone
psychological testing before the surgery.

She was asked whether it was she or a family member who wanted the
transplant, and about how she felt about living with the face from a
dead person, Dr. Kodish said.

Ms. Culp was shot by her husband, Thomas Culp, in 2004, damaging most of
her face and leaving her unable to breathe or eat on her own. Her
husband, with whom she had a common-law marriage, according to
\href{https://www.cleveland.com/healthfit/2010/11/woman_who_underwent_first_near.html}{The
Plain Dealer}, was sentenced to seven years in prison for aggravated
attempted murder and released in 2011.

After shooting his wife, Mr. Culp turned the shotgun on himself but only
lost a few teeth and some of the vision in his left eye. He still looked
the same, Ms. Culp told The Plain Dealer.

She said she had forgiven her husband. ``I still love my husband,'' she
told
``\href{https://abcnews.go.com/Health/MindMoodNews/story?id=7535591}{Good
Morning America}'' in 2009. ``I forgave him the day he did it. I have
to.''

Around 40 such surgeries have been done worldwide since Ms. Culp's, said
Dr. Thomas Romo III, director of facial plastic and reconstructive
surgery at Lenox Hill Hospital and Manhattan Eye, Ear and Throat
Hospital.

\href{https://www.nytimes.com/2008/12/18/health/s18face.html}{Her
procedure was immediately successful} and Ms. Culp's body did not reject
the transplant, though she needed to take anti-rejection drugs for the
rest of her life because her body could have rejected the transplant at
any time. The drugs suppress a patient's immune system to prevent it
from rejecting the transplant but also make the person more susceptible
to infections.

Most face transplant procedures are related to gunshot wounds or
accidents involving animals, Dr. Romo said.

Charla Nash, who received a full face transplant in 2011 after being
mauled by her friend's pet chimpanzee in Stamford, Conn.,
\href{https://www.nytimes.com/2016/05/06/nyregion/chimpanzee-attack-victim-who-got-face-transplant-is-hospitalized.html}{was
hospitalized} in 2016 after she participated in a trial to determine if
transplant patients could be weaned off the drugs.

Face transplants are more than just cosmetic improvements for patients,
Dr. Romo said. After successful transplants, most patients are able to
speak, eat and otherwise live a more normal life.

If not for the surgery, Ms. Culp would not have been able to smile or
talk, Dr. Romo said, adding that face transplants can have positive
psychological effects for patients.

Ms. Culp is ``a milestone in medical history, and will be forever,'' he
said.

She was chosen for the then-experimental surgery because of her optimism
and willingness to follow medical orders, according to a
\href{https://www.cleveland.com/healthfit/2010/11/woman_who_underwent_first_near.html}{2010
profile} in The Plain Dealer.

Ms. Culp and her husband previously ran a drywall, painting and
wallpapering business before they bought a restaurant and bar in 2004,
where she often worked from the early morning until late at night, The
Plain Dealer reported.

She was born on March 26, 1963. Details about survivors were not
immediately available.

At a news conference
\href{https://www.nytimes.com/2009/05/06/science/06face.html}{unveiling
her new face} in 2009, Ms. Culp asked others to be kind to people with
facial disfigurements.

``Don't judge people who don't look the same as you do,'' she said.
``Because you never know. One day it might be all taken away.''

Advertisement

\protect\hyperlink{after-bottom}{Continue reading the main story}

\hypertarget{site-index}{%
\subsection{Site Index}\label{site-index}}

\hypertarget{site-information-navigation}{%
\subsection{Site Information
Navigation}\label{site-information-navigation}}

\begin{itemize}
\tightlist
\item
  \href{https://help.nytimes.com/hc/en-us/articles/115014792127-Copyright-notice}{©~2020~The
  New York Times Company}
\end{itemize}

\begin{itemize}
\tightlist
\item
  \href{https://www.nytco.com/}{NYTCo}
\item
  \href{https://help.nytimes.com/hc/en-us/articles/115015385887-Contact-Us}{Contact
  Us}
\item
  \href{https://www.nytco.com/careers/}{Work with us}
\item
  \href{https://nytmediakit.com/}{Advertise}
\item
  \href{http://www.tbrandstudio.com/}{T Brand Studio}
\item
  \href{https://www.nytimes.com/privacy/cookie-policy\#how-do-i-manage-trackers}{Your
  Ad Choices}
\item
  \href{https://www.nytimes.com/privacy}{Privacy}
\item
  \href{https://help.nytimes.com/hc/en-us/articles/115014893428-Terms-of-service}{Terms
  of Service}
\item
  \href{https://help.nytimes.com/hc/en-us/articles/115014893968-Terms-of-sale}{Terms
  of Sale}
\item
  \href{https://spiderbites.nytimes.com}{Site Map}
\item
  \href{https://help.nytimes.com/hc/en-us}{Help}
\item
  \href{https://www.nytimes.com/subscription?campaignId=37WXW}{Subscriptions}
\end{itemize}
