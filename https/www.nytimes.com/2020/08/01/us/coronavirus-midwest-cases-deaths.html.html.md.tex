Sections

SEARCH

\protect\hyperlink{site-content}{Skip to
content}\protect\hyperlink{site-index}{Skip to site index}

\href{https://www.nytimes.com/section/us}{U.S.}

\href{https://myaccount.nytimes.com/auth/login?response_type=cookie\&client_id=vi}{}

\href{https://www.nytimes.com/section/todayspaper}{Today's Paper}

\href{/section/us}{U.S.}\textbar{}After Plummeting, the Virus Soars Back
in the Midwest

\url{https://nyti.ms/319iyDx}

\begin{itemize}
\item
\item
\item
\item
\item
\item
\end{itemize}

\href{https://www.nytimes.com/news-event/coronavirus?action=click\&pgtype=Article\&state=default\&region=TOP_BANNER\&context=storylines_menu}{The
Coronavirus Outbreak}

\begin{itemize}
\tightlist
\item
  live\href{https://www.nytimes.com/2020/08/01/world/coronavirus-covid-19.html?action=click\&pgtype=Article\&state=default\&region=TOP_BANNER\&context=storylines_menu}{Latest
  Updates}
\item
  \href{https://www.nytimes.com/interactive/2020/us/coronavirus-us-cases.html?action=click\&pgtype=Article\&state=default\&region=TOP_BANNER\&context=storylines_menu}{Maps
  and Cases}
\item
  \href{https://www.nytimes.com/interactive/2020/science/coronavirus-vaccine-tracker.html?action=click\&pgtype=Article\&state=default\&region=TOP_BANNER\&context=storylines_menu}{Vaccine
  Tracker}
\item
  \href{https://www.nytimes.com/interactive/2020/07/29/us/schools-reopening-coronavirus.html?action=click\&pgtype=Article\&state=default\&region=TOP_BANNER\&context=storylines_menu}{What
  School May Look Like}
\item
  \href{https://www.nytimes.com/live/2020/07/31/business/stock-market-today-coronavirus?action=click\&pgtype=Article\&state=default\&region=TOP_BANNER\&context=storylines_menu}{Economy}
\end{itemize}

Advertisement

\protect\hyperlink{after-top}{Continue reading the main story}

Supported by

\protect\hyperlink{after-sponsor}{Continue reading the main story}

\hypertarget{after-plummeting-the-virus-soars-back-in-the-midwest}{%
\section{After Plummeting, the Virus Soars Back in the
Midwest}\label{after-plummeting-the-virus-soars-back-in-the-midwest}}

States like Missouri, Illinois and Wisconsin are riding a frustrating
seesaw during the pandemic, with new coronavirus cases rising again
after apparent progress.

\includegraphics{https://static01.nyt.com/images/2020/08/02/us/SUB02virus-stateofthevirus/01virus-stateofthevirus01-articleLarge.jpg?quality=75\&auto=webp\&disable=upscale}

By \href{https://www.nytimes.com/by/julie-bosman}{Julie Bosman},
\href{https://www.nytimes.com/by/manny-fernandez}{Manny Fernandez} and
\href{https://www.nytimes.com/by/thomas-fuller}{Thomas Fuller}

\begin{itemize}
\item
  Aug. 1, 2020
\item
  \begin{itemize}
  \item
  \item
  \item
  \item
  \item
  \item
  \end{itemize}
\end{itemize}

CHICAGO --- First, the Pacific Northwest and the Northeast were hit
hardest as the coronavirus tore through the nation. Then it surged
across the South. Now the virus is again picking up dangerous speed in
much of the Midwest --- and in states from Mississippi to Florida to
California that thought they had already seen the worst of it.

As the United States rides what amounts to a second wave of cases, with
daily new infections leveling off at an alarming higher mark, there is a
deepening national sense that the progress made in fighting the pandemic
is coming undone and no patch of America is safe.

In Missouri, Wisconsin and Illinois, distressed government officials are
retightening restrictions on residents and businesses, and sounding
warnings about a surge in coronavirus-related hospitalizations.

In the South and the West, several states are reporting their highest
levels of new coronavirus cases, with outbreaks overwhelming urban and
rural areas alike.

Across the country, communities including Snohomish County, Wash.,
Jackson, Miss., and Baton Rouge, La., have seen coronavirus numbers fall
and then shoot back up --- not unlike the two ends of a seesaw.

In Illinois, Gov. J.B. Pritzker sounded an unusually somber note this
past week as he delivered a warning that reverberated across the state:
Even though Illinoisans had battled an early flood of coronavirus
infections and then managed to reduce the virus's spread, their
successes were fleeting. As of Thursday, the state was averaging more
than 1,400 cases a day, up from about 800 at the start of July.

``We're at a danger point,'' Mr. Pritzker said in Peoria County, where
the total number of cases has doubled in the last month.

Gone is any sense that the country may soon gain control of the
pandemic. Instead, the seven-day average for new infections hovered
around 65,000 for two weeks. Progress in some states has been mostly
offset by growing outbreaks in parts of the South and the Midwest.

\includegraphics{https://static01.nyt.com/images/2020/08/01/us/01virus-stateofthevirus02/01virus-stateofthevirus02-articleLarge.jpg?quality=75\&auto=webp\&disable=upscale}

``There's a sort of collective tiredness and frustration, and of course
I feel it, too --- we all feel it,'' said County Judge Lina Hidalgo, the
top elected official in Harris County, which includes Houston. ``So it's
difficult to know that there's no real end in sight.''

On Friday, Dr. Anthony S. Fauci, the nation's top infectious-disease
expert, told Congress he was cautiously optimistic that a safe and
effective coronavirus vaccine would be available by the end of the year
or early 2021, though the federal government's ability to speedily
immunize most Americans was unclear.

\hypertarget{latest-updates-global-coronavirus-outbreak}{%
\section{\texorpdfstring{\href{https://www.nytimes.com/2020/08/01/world/coronavirus-covid-19.html?action=click\&pgtype=Article\&state=default\&region=MAIN_CONTENT_1\&context=storylines_live_updates}{Latest
Updates: Global Coronavirus
Outbreak}}{Latest Updates: Global Coronavirus Outbreak}}\label{latest-updates-global-coronavirus-outbreak}}

Updated 2020-08-02T07:42:09.613Z

\begin{itemize}
\tightlist
\item
  \href{https://www.nytimes.com/2020/08/01/world/coronavirus-covid-19.html?action=click\&pgtype=Article\&state=default\&region=MAIN_CONTENT_1\&context=storylines_live_updates\#link-34047410}{The
  U.S. reels as July cases more than double the total of any other
  month.}
\item
  \href{https://www.nytimes.com/2020/08/01/world/coronavirus-covid-19.html?action=click\&pgtype=Article\&state=default\&region=MAIN_CONTENT_1\&context=storylines_live_updates\#link-780ec966}{Top
  U.S. officials work to break an impasse over the federal jobless
  benefit.}
\item
  \href{https://www.nytimes.com/2020/08/01/world/coronavirus-covid-19.html?action=click\&pgtype=Article\&state=default\&region=MAIN_CONTENT_1\&context=storylines_live_updates\#link-2bc8948}{Its
  outbreak untamed, Melbourne goes into even greater lockdown.}
\end{itemize}

\href{https://www.nytimes.com/2020/08/01/world/coronavirus-covid-19.html?action=click\&pgtype=Article\&state=default\&region=MAIN_CONTENT_1\&context=storylines_live_updates}{See
more updates}

More live coverage:
\href{https://www.nytimes.com/live/2020/07/31/business/stock-market-today-coronavirus?action=click\&pgtype=Article\&state=default\&region=MAIN_CONTENT_1\&context=storylines_live_updates}{Markets}

Even finding out who has the virus is a challenge, as testing programs
have frustrated many Americans with lengthy delays in providing results.

The picture is similarly depressing overseas, where even governments
that would seem well suited to combating the virus are seeing
resurgences.

New daily infections in Japan, a country with a long tradition of
wearing face masks, rose more than 50 percent in July. Australia, which
can cut itself off from the rest of the world more easily than most, is
battling a wave of infections in and around Melbourne. Hong Kong, Israel
and Spain are also fighting second waves.

None of those places has an infection rate as high as the United States,
which has the most cases and deaths in the world.

In American communities that saw improvement in June, such as Milwaukee
County in Wisconsin, there was a widespread feeling of relief, said Dr.
Ben Weston, the director of medical services for the Milwaukee County
Office of Emergency Management.

But then mask-wearing and social distancing began to relax.

``There was a sense of complacency, like, `We're finally beyond this,
it's finally getting better,''' he said. ``We were seeing our numbers go
down, but the reason is because of physical distancing. It's because
people were being so careful. There was no reason to think that cases
weren't going to rise.''

Image

Customers leaving a store in Tulsa, Okla., on Thursday.Credit...Chris
Creese for The New York Times

On Thursday, Gov. Tony Evers, a Democrat, made another attempt to get a
handle on the outbreaks in his state, issuing an order that every
Wisconsinite wear a mask indoors in public beginning Saturday.

Many states have traced new outbreaks to the loosening of the
economically costly restrictions aimed at stopping the spread of the
virus.

In California, which has had more than 500,000 coronavirus cases, more
than any other state, the reopening has proved disastrous. When the
pandemic was ravaging the Northeast in March and April, California kept
its daily case count around 2,000, and the state was praised for its
early and aggressive actions to combat the virus.

The state is now averaging more than four times as many cases --- 8,500
a day. Los Angeles County and other Southern California counties account
for the majority of the state's infections, but the virus is now
everywhere.

That notion was reinforced on Tuesday when health officials in one of
the most remote parts of the state, Modoc County, which had been the
last of California's 58 counties without a known case, announced that
the virus had arrived.

A waitress at the Brass Rail, a Basque restaurant and bar, tested
positive, raising concerns about the virus's spread in a tight-knit
county with a population of 8,800 and where cows outnumber people five
to one. (A billboard there warning residents of the coronavirus tells
people to stand one cow's length apart.)

The waitress and her husband recently returned from a trip to the
Central Valley, according to the co-owner of the Brass Rail, Jodie
Larranaga, who said she assumed that the waitress was infected during
her journey.

That the virus is now present in the evergreen forests of the
northeastern corner of the state is testament to its inexorable spread,
say the county's residents. Alturas, the only incorporated city in Modoc
County, is so isolated that its high school football team must drive as
long as five hours to reach its opponents.

``We all felt very safe for a while,'' said Juan Ledezma, the owner of a
thrift shop on Main Street in Alturas. ``Right now, it's a little bit
scary.''

\href{https://www.nytimes.com/news-event/coronavirus?action=click\&pgtype=Article\&state=default\&region=MAIN_CONTENT_3\&context=storylines_faq}{}

\hypertarget{the-coronavirus-outbreak-}{%
\subsubsection{The Coronavirus Outbreak
›}\label{the-coronavirus-outbreak-}}

\hypertarget{frequently-asked-questions}{%
\paragraph{Frequently Asked
Questions}\label{frequently-asked-questions}}

Updated July 27, 2020

\begin{itemize}
\item ~
  \hypertarget{should-i-refinance-my-mortgage}{%
  \paragraph{Should I refinance my
  mortgage?}\label{should-i-refinance-my-mortgage}}

  \begin{itemize}
  \tightlist
  \item
    \href{https://www.nytimes.com/article/coronavirus-money-unemployment.html?action=click\&pgtype=Article\&state=default\&region=MAIN_CONTENT_3\&context=storylines_faq}{It
    could be a good idea,} because mortgage rates have
    \href{https://www.nytimes.com/2020/07/16/business/mortgage-rates-below-3-percent.html?action=click\&pgtype=Article\&state=default\&region=MAIN_CONTENT_3\&context=storylines_faq}{never
    been lower.} Refinancing requests have pushed mortgage applications
    to some of the highest levels since 2008, so be prepared to get in
    line. But defaults are also up, so if you're thinking about buying a
    home, be aware that some lenders have tightened their standards.
  \end{itemize}
\item ~
  \hypertarget{what-is-school-going-to-look-like-in-september}{%
  \paragraph{What is school going to look like in
  September?}\label{what-is-school-going-to-look-like-in-september}}

  \begin{itemize}
  \tightlist
  \item
    It is unlikely that many schools will return to a normal schedule
    this fall, requiring the grind of
    \href{https://www.nytimes.com/2020/06/05/us/coronavirus-education-lost-learning.html?action=click\&pgtype=Article\&state=default\&region=MAIN_CONTENT_3\&context=storylines_faq}{online
    learning},
    \href{https://www.nytimes.com/2020/05/29/us/coronavirus-child-care-centers.html?action=click\&pgtype=Article\&state=default\&region=MAIN_CONTENT_3\&context=storylines_faq}{makeshift
    child care} and
    \href{https://www.nytimes.com/2020/06/03/business/economy/coronavirus-working-women.html?action=click\&pgtype=Article\&state=default\&region=MAIN_CONTENT_3\&context=storylines_faq}{stunted
    workdays} to continue. California's two largest public school
    districts --- Los Angeles and San Diego --- said on July 13, that
    \href{https://www.nytimes.com/2020/07/13/us/lausd-san-diego-school-reopening.html?action=click\&pgtype=Article\&state=default\&region=MAIN_CONTENT_3\&context=storylines_faq}{instruction
    will be remote-only in the fall}, citing concerns that surging
    coronavirus infections in their areas pose too dire a risk for
    students and teachers. Together, the two districts enroll some
    825,000 students. They are the largest in the country so far to
    abandon plans for even a partial physical return to classrooms when
    they reopen in August. For other districts, the solution won't be an
    all-or-nothing approach.
    \href{https://bioethics.jhu.edu/research-and-outreach/projects/eschool-initiative/school-policy-tracker/}{Many
    systems}, including the nation's largest, New York City, are
    devising
    \href{https://www.nytimes.com/2020/06/26/us/coronavirus-schools-reopen-fall.html?action=click\&pgtype=Article\&state=default\&region=MAIN_CONTENT_3\&context=storylines_faq}{hybrid
    plans} that involve spending some days in classrooms and other days
    online. There's no national policy on this yet, so check with your
    municipal school system regularly to see what is happening in your
    community.
  \end{itemize}
\item ~
  \hypertarget{is-the-coronavirus-airborne}{%
  \paragraph{Is the coronavirus
  airborne?}\label{is-the-coronavirus-airborne}}

  \begin{itemize}
  \tightlist
  \item
    The coronavirus
    \href{https://www.nytimes.com/2020/07/04/health/239-experts-with-one-big-claim-the-coronavirus-is-airborne.html?action=click\&pgtype=Article\&state=default\&region=MAIN_CONTENT_3\&context=storylines_faq}{can
    stay aloft for hours in tiny droplets in stagnant air}, infecting
    people as they inhale, mounting scientific evidence suggests. This
    risk is highest in crowded indoor spaces with poor ventilation, and
    may help explain super-spreading events reported in meatpacking
    plants, churches and restaurants.
    \href{https://www.nytimes.com/2020/07/06/health/coronavirus-airborne-aerosols.html?action=click\&pgtype=Article\&state=default\&region=MAIN_CONTENT_3\&context=storylines_faq}{It's
    unclear how often the virus is spread} via these tiny droplets, or
    aerosols, compared with larger droplets that are expelled when a
    sick person coughs or sneezes, or transmitted through contact with
    contaminated surfaces, said Linsey Marr, an aerosol expert at
    Virginia Tech. Aerosols are released even when a person without
    symptoms exhales, talks or sings, according to Dr. Marr and more
    than 200 other experts, who
    \href{https://academic.oup.com/cid/article/doi/10.1093/cid/ciaa939/5867798}{have
    outlined the evidence in an open letter to the World Health
    Organization}.
  \end{itemize}
\item ~
  \hypertarget{what-are-the-symptoms-of-coronavirus}{%
  \paragraph{What are the symptoms of
  coronavirus?}\label{what-are-the-symptoms-of-coronavirus}}

  \begin{itemize}
  \tightlist
  \item
    Common symptoms
    \href{https://www.nytimes.com/article/symptoms-coronavirus.html?action=click\&pgtype=Article\&state=default\&region=MAIN_CONTENT_3\&context=storylines_faq}{include
    fever, a dry cough, fatigue and difficulty breathing or shortness of
    breath.} Some of these symptoms overlap with those of the flu,
    making detection difficult, but runny noses and stuffy sinuses are
    less common.
    \href{https://www.nytimes.com/2020/04/27/health/coronavirus-symptoms-cdc.html?action=click\&pgtype=Article\&state=default\&region=MAIN_CONTENT_3\&context=storylines_faq}{The
    C.D.C. has also} added chills, muscle pain, sore throat, headache
    and a new loss of the sense of taste or smell as symptoms to look
    out for. Most people fall ill five to seven days after exposure, but
    symptoms may appear in as few as two days or as many as 14 days.
  \end{itemize}
\item ~
  \hypertarget{does-asymptomatic-transmission-of-covid-19-happen}{%
  \paragraph{Does asymptomatic transmission of Covid-19
  happen?}\label{does-asymptomatic-transmission-of-covid-19-happen}}

  \begin{itemize}
  \tightlist
  \item
    So far, the evidence seems to show it does. A widely cited
    \href{https://www.nature.com/articles/s41591-020-0869-5}{paper}
    published in April suggests that people are most infectious about
    two days before the onset of coronavirus symptoms and estimated that
    44 percent of new infections were a result of transmission from
    people who were not yet showing symptoms. Recently, a top expert at
    the World Health Organization stated that transmission of the
    coronavirus by people who did not have symptoms was ``very rare,''
    \href{https://www.nytimes.com/2020/06/09/world/coronavirus-updates.html?action=click\&pgtype=Article\&state=default\&region=MAIN_CONTENT_3\&context=storylines_faq\#link-1f302e21}{but
    she later walked back that statement.}
  \end{itemize}
\end{itemize}

Businesses across the country have abandoned their own plans to return
to normal in light of the virus's resurgence.

The company that operates a popular water taxi on the Chicago River,
ferrying commuters to work each day, had hoped to reopen by Labor Day.
This week, officials postponed those plans until March.

The historic Berghoff restaurant in Chicago's Loop reopened at the end
of June after months of closure, a sign that the coronavirus curve had
flattened and the city's downtown was ready to start humming again.

This week, as coronavirus infections surged in Illinois, the restaurant
abruptly shut its doors for the second time.

``It broke my heart,'' said Pete Berghoff, whose family has owned the
restaurant since 1898. ``We reopened, and after about three weeks my
enthusiasm was beaten out of me.''

From state to state and region to region, the picture of coronavirus
spread is shifting daily as some communities see gradual improvement and
others suddenly struggle.

Image

Renee Leonard, Delisa Craig and Miriam Girata help one another put on
personal protective equipment at~ a testing site in Orlando, Fla., on
Tuesday.Credit...Eve Edelheit for The New York Times

A few places, including Arizona, South Carolina and Texas, have started
to see new case reports drop after huge surges. California, Florida and
Louisiana continue to report some of their highest daily totals of the
pandemic.

The Rio Grande Valley in Texas is suffering through perhaps the worst
current outbreak in the country, with hundreds of new cases and dozens
of deaths a day. In more than half of states, outbreaks continue to
grow.

In Missouri and Oklahoma, cases have grown to alarming levels, with both
states now averaging more than 1,000 each day. And in Maryland, daily
case numbers are ticking upward again after periods of sustained
progress.

The Northeast, once the virus's biggest hot spot, has improved
considerably since its peak in April, when the region suffered more than
any other region of the country. Yet cases are now increasing slightly
in New Jersey, Rhode Island and Massachusetts, as residents move around
more freely and gather more frequently in groups.

Across the country, deaths from the coronavirus continue to rise. The
country was averaging about 500 per day at the start of July. Over the
last week, it has averaged more than 1,000 daily, with many of those
concentrated in Sun Belt states. On Wednesday, California, Florida and
Texas reported a combined 724 deaths, about half the national total.

Houston, the fourth-largest city in the country, has been adjusting to a
new normal where the only thing certain is that nothing is certain.
After cases and hospitalizations seemed to level off and even decrease
in recent days, Harris County on Friday broke a single-day record with
2,100 new cases.

``I think to a certain extent, we saw a spike because people were
fatigued over it,'' said Alan Rosen, who leads the Harris County
Precinct One constable's office. ``They were fatigued over hearing about
it every day. They were fatigued about being cooped up in their house
and being away from people.''

People there have been coping with the lulls and peaks of a physical,
emotional, fiscal and logistical crisis from an invisible foe nearly
three years after surviving Hurricane Harvey, one of the worst disasters
in American history.

*``*It is a roller coaster,'' said Mr. Rosen, who recovered after
getting infected with the virus in May. ``It's not like a hurricane
that's coming through and we know what to do. We know we got to clean up
and rebuild and everybody is accustomed to the time frame. But with
this, there are just so many unknowns.''

Julie Bosman reported from Chicago, Manny Fernandez from Houston and
Thomas Fuller from Alturas, Calif. Mitch Smith contributed reporting
from Chicago.

Advertisement

\protect\hyperlink{after-bottom}{Continue reading the main story}

\hypertarget{site-index}{%
\subsection{Site Index}\label{site-index}}

\hypertarget{site-information-navigation}{%
\subsection{Site Information
Navigation}\label{site-information-navigation}}

\begin{itemize}
\tightlist
\item
  \href{https://help.nytimes.com/hc/en-us/articles/115014792127-Copyright-notice}{©~2020~The
  New York Times Company}
\end{itemize}

\begin{itemize}
\tightlist
\item
  \href{https://www.nytco.com/}{NYTCo}
\item
  \href{https://help.nytimes.com/hc/en-us/articles/115015385887-Contact-Us}{Contact
  Us}
\item
  \href{https://www.nytco.com/careers/}{Work with us}
\item
  \href{https://nytmediakit.com/}{Advertise}
\item
  \href{http://www.tbrandstudio.com/}{T Brand Studio}
\item
  \href{https://www.nytimes.com/privacy/cookie-policy\#how-do-i-manage-trackers}{Your
  Ad Choices}
\item
  \href{https://www.nytimes.com/privacy}{Privacy}
\item
  \href{https://help.nytimes.com/hc/en-us/articles/115014893428-Terms-of-service}{Terms
  of Service}
\item
  \href{https://help.nytimes.com/hc/en-us/articles/115014893968-Terms-of-sale}{Terms
  of Sale}
\item
  \href{https://spiderbites.nytimes.com}{Site Map}
\item
  \href{https://help.nytimes.com/hc/en-us}{Help}
\item
  \href{https://www.nytimes.com/subscription?campaignId=37WXW}{Subscriptions}
\end{itemize}
