Sections

SEARCH

\protect\hyperlink{site-content}{Skip to
content}\protect\hyperlink{site-index}{Skip to site index}

\href{https://www.nytimes.com/section/us}{U.S.}

\href{https://myaccount.nytimes.com/auth/login?response_type=cookie\&client_id=vi}{}

\href{https://www.nytimes.com/section/todayspaper}{Today's Paper}

\href{/section/us}{U.S.}\textbar{}Celebrating Eid al-Adha Amid a
Pandemic

\url{https://nyti.ms/3i21FBH}

\begin{itemize}
\item
\item
\item
\item
\item
\end{itemize}

\href{https://www.nytimes.com/news-event/coronavirus?action=click\&pgtype=Article\&state=default\&region=TOP_BANNER\&context=storylines_menu}{The
Coronavirus Outbreak}

\begin{itemize}
\tightlist
\item
  live\href{https://www.nytimes.com/2020/08/01/world/coronavirus-covid-19.html?action=click\&pgtype=Article\&state=default\&region=TOP_BANNER\&context=storylines_menu}{Latest
  Updates}
\item
  \href{https://www.nytimes.com/interactive/2020/us/coronavirus-us-cases.html?action=click\&pgtype=Article\&state=default\&region=TOP_BANNER\&context=storylines_menu}{Maps
  and Cases}
\item
  \href{https://www.nytimes.com/interactive/2020/science/coronavirus-vaccine-tracker.html?action=click\&pgtype=Article\&state=default\&region=TOP_BANNER\&context=storylines_menu}{Vaccine
  Tracker}
\item
  \href{https://www.nytimes.com/interactive/2020/07/29/us/schools-reopening-coronavirus.html?action=click\&pgtype=Article\&state=default\&region=TOP_BANNER\&context=storylines_menu}{What
  School May Look Like}
\item
  \href{https://www.nytimes.com/live/2020/07/31/business/stock-market-today-coronavirus?action=click\&pgtype=Article\&state=default\&region=TOP_BANNER\&context=storylines_menu}{Economy}
\end{itemize}

Advertisement

\protect\hyperlink{after-top}{Continue reading the main story}

Supported by

\protect\hyperlink{after-sponsor}{Continue reading the main story}

Race/Related

\hypertarget{celebrating-eid-al-adha-amid-a-pandemic}{%
\section{Celebrating Eid al-Adha Amid a
Pandemic}\label{celebrating-eid-al-adha-amid-a-pandemic}}

With socially distanced gatherings now the norm, we talked to a few
people about how they planned to celebrate one of Islam's holiest days.

\includegraphics{https://static01.nyt.com/images/2020/08/01/us/01racerelated-eid/01racerelated-eid-articleLarge.jpg?quality=75\&auto=webp\&disable=upscale}

\href{https://www.nytimes.com/by/fahima-haque}{\includegraphics{https://static01.nyt.com/images/2020/07/03/reader-center/author-fahima-haque/author-fahima-haque-thumbLarge.png}}

By \href{https://www.nytimes.com/by/fahima-haque}{Fahima Haque}

\begin{itemize}
\item
  Aug. 1, 2020
\item
  \begin{itemize}
  \item
  \item
  \item
  \item
  \item
  \end{itemize}
\end{itemize}

\emph{{[}Race/Related is available as a newsletter.}
\href{https://www.nytimes.com/newsletters/race-related}{\emph{Sign up
here to get it delivered to your inbox}}\emph{.{]}}

\hypertarget{celebrating-eid-al-adha}{%
\subsection{Celebrating Eid al-Adha}\label{celebrating-eid-al-adha}}

Most countries observed Eid al-Adha on Friday, and so a belated Eid
Mubarak to all who celebrated one of the holiest days in Islam. It is
meant to remind Muslims of their faithfulness to Allah and each other,
through Zakat, one of the five pillars of the religion that focuses on
charity.

The holiday commemorates the story of the Prophet Ibrahim's devotion to
Allah and his willingness to sacrifice his son Ismail. Allah ultimately
spared Ismail, and instead sacrificed a ram. You might know the story as
Abraham and Isaac, per the Judeo-Christian tradition. This is why Eid
al-Adha is known as the festival of sacrifice and why families slaughter
an animal --- often a goat, sheep or a cow --- to give to a family in
need.

Eid al-Adha also comes right after Hajj, a sacred pilgrimage to Mecca
that is a mandate for Muslims who are able-bodied and can afford the
five-day trip. Usually 2.5 million Muslims make the journey. But this
year, because of the global pandemic, Saudi Arabia
\href{https://www.nytimes.com/2020/07/30/world/middleeast/pilgrims-hajj-mecca-coronavirus-pandemic.html}{said
it would allow just 1,000} people, and all from within the kingdom.

\hypertarget{latest-updates-global-coronavirus-outbreak}{%
\section{\texorpdfstring{\href{https://www.nytimes.com/2020/08/01/world/coronavirus-covid-19.html?action=click\&pgtype=Article\&state=default\&region=MAIN_CONTENT_1\&context=storylines_live_updates}{Latest
Updates: Global Coronavirus
Outbreak}}{Latest Updates: Global Coronavirus Outbreak}}\label{latest-updates-global-coronavirus-outbreak}}

Updated 2020-08-02T07:42:09.613Z

\begin{itemize}
\tightlist
\item
  \href{https://www.nytimes.com/2020/08/01/world/coronavirus-covid-19.html?action=click\&pgtype=Article\&state=default\&region=MAIN_CONTENT_1\&context=storylines_live_updates\#link-34047410}{The
  U.S. reels as July cases more than double the total of any other
  month.}
\item
  \href{https://www.nytimes.com/2020/08/01/world/coronavirus-covid-19.html?action=click\&pgtype=Article\&state=default\&region=MAIN_CONTENT_1\&context=storylines_live_updates\#link-780ec966}{Top
  U.S. officials work to break an impasse over the federal jobless
  benefit.}
\item
  \href{https://www.nytimes.com/2020/08/01/world/coronavirus-covid-19.html?action=click\&pgtype=Article\&state=default\&region=MAIN_CONTENT_1\&context=storylines_live_updates\#link-2bc8948}{Its
  outbreak untamed, Melbourne goes into even greater lockdown.}
\end{itemize}

\href{https://www.nytimes.com/2020/08/01/world/coronavirus-covid-19.html?action=click\&pgtype=Article\&state=default\&region=MAIN_CONTENT_1\&context=storylines_live_updates}{See
more updates}

More live coverage:
\href{https://www.nytimes.com/live/2020/07/31/business/stock-market-today-coronavirus?action=click\&pgtype=Article\&state=default\&region=MAIN_CONTENT_1\&context=storylines_live_updates}{Markets}

I'm celebrating --- socially distanced --- with my parents and one of my
sisters, who lives in the New York City borough of Queens, not far from
our parents and from where I grew up. My dad went to a socially
distanced prayer service in the morning and my mom and I prayed at home.
My mom usually makes a feast --- pulao, biryani, kebabs and much more
--- but made much less food this year.

Our pared down celebration got me thinking: How else are American
Muslims observing Eid al-Adha this year? I talked to a few people across
the country about how they planned to celebrate. Here's what they say
had to say, edited lightly for length and clarity:

\textbf{Ahmed Ali Akbar}, a journalist and host of the podcast ``See
Something, Say Something,'' has been in quarantine in Michigan with his
wife and his father since March.

\begin{quote}
We're going to pray in our house (the local mosque is open but we are
choosing not to go), my wife, my dad and I on Zoom. We'll probably take
a lot of pictures. Dressing up and looking nice is definitely a huge
part of Eid; it's a renewal kind of thing.

We're going to go to the drive-through --- our mosque is doing barbecue
--- and pick up some food. We're going to do a socially distanced picnic
and do a socially distanced photo shoot. The other thing I'm going to
cook is achar gosht (pickled meat stew) because during Eid al-Adha meat
is a very central part, in a way.
\end{quote}

\includegraphics{https://static01.nyt.com/images/2020/08/01/us/01racerelated-03/01racerelated-03-articleLarge.jpg?quality=75\&auto=webp\&disable=upscale}

\begin{quote}
Ever since my mother passed, Eid has changed its meaning. Our mother was
responsible for a lot of the excitement and cooking. So now that's
fallen on me, actually. I called up my dad and I think we decided on
achar gosht and I have some mango ice cream that I've been making out of
these mangoes that we imported from Pakistan. It will be a restrained
menu. I think when you compare to both when my mother was alive and when
there was no quarantine, the expectations have simplified.

This Eid, I'm asking, can I take the spirit of generosity here and try
to use whatever I have for good? I'm trying to figure what local
organizations and people I can support.
\end{quote}

\textbf{Kima Jones}, the founder of a book publicity agency committed to
literature by Black writers and writers of color, lives in Los Angeles
and will be celebrating with her two brothers who have been in
quarantine with her.

\begin{quote}
The Eids are two of my favorite holidays. My father was Muslim, and
growing up, my mother was Southern Baptist; she's since converted. It's
really just all about the food for me. We lived in New York and my
father would drive to New Jersey and pick up Halal sausage, bean pies,
in bulk, because there were eight of us children. My father, my
brothers, my older male cousins, they always slaughtered lamb, sheep,
and once or twice, cows.

My father owned a Halal farm during his lifetime. I grew up with him
going out and sacrificing and cleaning the designated animal. We paid
Zakat the way that we needed to, but really it was just three or four
days of extremely good eating. I won't be sacrificing an animal this
year because of Covid-19. Whenever I can't get meat, either I can't do
it myself or if a family member can't, I try to order from Honest Chops,
a Halal meat market in Manhattan. You can actually buy an animal and
donate it to a family and they will do the ritual for you and get the
meat cleaned, packaged and shipped out.

This year in the pandemic, I'm going to do our Eid prayers here at the
house. We're going to cook five or six courses, which I know sounds like
a lot, but I come from a big family and so I'm used to very big
portions. We're going to have lamb, red snapper, something with shrimp,
a vegetable, grill some corn, make a fruit salad.
\end{quote}

Image

Like many Muslims across America, Ms. Jones planned to celebrate Eid
al-Adha at home. She prepared several dishes for her and her two
brothers.Credit...Philip Cheung for The New York Times

\begin{quote}
There are two major ways that I try to look at time, and I measure it
for my birthday to my birthday, like it's my own personal calendar year,
but I also measure progress, Eid to Eid, Ramadan to Ramadan. In addition
to having material resolutions, to-do lists or goals, I also have my
spiritual resolutions and I want to make sure that I'm checking in with
myself each Ramadan, whether that is to learn a new Surah, whether that
is to finally memorize the 99 names of Allah, whatever the thing is.

Eid al-Adha also specifically makes me think, what is my divine
assignment? What have I been asked to do? Am I doing it? Am I doing it
in a way that's a reflection of what's the best for me, what's best for
the people that I serve? It really makes me sit with myself, course
correct and be self-aware. The story of Ibrahim is forcing us to check
in with ourselves, and the quarantine is forcing us to check in with
ourselves, our friends, our family more often.
\end{quote}

\textbf{Shahana Hanif}, is running for a seat on the New York City
Council to represent District 39 in Brooklyn. She lives with her parents
in the Kensington neighborhood of Brooklyn.

\begin{quote}
Kensington is quite festive because it's one of the largest
Bangladeshi-Muslim enclaves in our city. The circumstances of
celebration during this moment are hard because of not being able to be
as mobile as I'd want to be. Having been born and raised in the
diaspora, we've built traditions that are rooted in going away or
traveling about and taking on the outdoors. But I don't think that
component will be gone, like one thing that we do always is go to our
local hookah spot and that's still on the agenda. They have outdoor
hookah, and so we're continuing that.

Eid is very low key in my household. For my family, it's making sure
that family back home (in Bangladesh) have what they need to celebrate
and making sure that the financial contributions are met in both of my
parents' hometowns.
\end{quote}

\hypertarget{invite-your-friends}{%
\subsubsection{\texorpdfstring{\textbf{Invite your
friends.}}{Invite your friends.}}\label{invite-your-friends}}

Invite someone to subscribe to the
\href{https://www.nytimes.com/newsletters/race-related?te=1\&nl=race-related\&emc=edit_rr_20190628}{Race/Related}
newsletter. Or email your thoughts and suggestions to
\href{mailto:racerelated@nytimes.com}{\nolinkurl{racerelated@nytimes.com}}.

\hypertarget{want-more-racerelated}{%
\subsubsection{\texorpdfstring{\textbf{Want more
Race/Related?}}{Want more Race/Related?}}\label{want-more-racerelated}}

\href{http://instagram.com/racerelatednyt}{Follow us on Instagram},
where we continue the conversation about race through visuals.

Advertisement

\protect\hyperlink{after-bottom}{Continue reading the main story}

\hypertarget{site-index}{%
\subsection{Site Index}\label{site-index}}

\hypertarget{site-information-navigation}{%
\subsection{Site Information
Navigation}\label{site-information-navigation}}

\begin{itemize}
\tightlist
\item
  \href{https://help.nytimes.com/hc/en-us/articles/115014792127-Copyright-notice}{©~2020~The
  New York Times Company}
\end{itemize}

\begin{itemize}
\tightlist
\item
  \href{https://www.nytco.com/}{NYTCo}
\item
  \href{https://help.nytimes.com/hc/en-us/articles/115015385887-Contact-Us}{Contact
  Us}
\item
  \href{https://www.nytco.com/careers/}{Work with us}
\item
  \href{https://nytmediakit.com/}{Advertise}
\item
  \href{http://www.tbrandstudio.com/}{T Brand Studio}
\item
  \href{https://www.nytimes.com/privacy/cookie-policy\#how-do-i-manage-trackers}{Your
  Ad Choices}
\item
  \href{https://www.nytimes.com/privacy}{Privacy}
\item
  \href{https://help.nytimes.com/hc/en-us/articles/115014893428-Terms-of-service}{Terms
  of Service}
\item
  \href{https://help.nytimes.com/hc/en-us/articles/115014893968-Terms-of-sale}{Terms
  of Sale}
\item
  \href{https://spiderbites.nytimes.com}{Site Map}
\item
  \href{https://help.nytimes.com/hc/en-us}{Help}
\item
  \href{https://www.nytimes.com/subscription?campaignId=37WXW}{Subscriptions}
\end{itemize}
