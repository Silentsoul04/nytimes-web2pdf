Sections

SEARCH

\protect\hyperlink{site-content}{Skip to
content}\protect\hyperlink{site-index}{Skip to site index}

\href{https://www.nytimes.com/section/us}{U.S.}

\href{https://myaccount.nytimes.com/auth/login?response_type=cookie\&client_id=vi}{}

\href{https://www.nytimes.com/section/todayspaper}{Today's Paper}

\href{/section/us}{U.S.}\textbar{}Isaias Live Updates: Storm Weakens as
It Churns Toward Florida After Hitting Bahamas

\begin{itemize}
\item
\item
\item
\item
\item
\item
\end{itemize}

\href{https://www.nytimes.com/news-event/coronavirus?action=click\&pgtype=Article\&state=default\&region=TOP_BANNER\&context=storylines_menu}{The
Coronavirus Outbreak}

\begin{itemize}
\tightlist
\item
  live\href{https://www.nytimes.com/2020/08/01/world/coronavirus-covid-19.html?action=click\&pgtype=Article\&state=default\&region=TOP_BANNER\&context=storylines_menu}{Latest
  Updates}
\item
  \href{https://www.nytimes.com/interactive/2020/us/coronavirus-us-cases.html?action=click\&pgtype=Article\&state=default\&region=TOP_BANNER\&context=storylines_menu}{Maps
  and Cases}
\item
  \href{https://www.nytimes.com/interactive/2020/science/coronavirus-vaccine-tracker.html?action=click\&pgtype=Article\&state=default\&region=TOP_BANNER\&context=storylines_menu}{Vaccine
  Tracker}
\item
  \href{https://www.nytimes.com/interactive/2020/07/29/us/schools-reopening-coronavirus.html?action=click\&pgtype=Article\&state=default\&region=TOP_BANNER\&context=storylines_menu}{What
  School May Look Like}
\item
  \href{https://www.nytimes.com/live/2020/07/31/business/stock-market-today-coronavirus?action=click\&pgtype=Article\&state=default\&region=TOP_BANNER\&context=storylines_menu}{Economy}
\end{itemize}

Advertisement

\protect\hyperlink{after-top}{Continue reading the main story}

Supported by

\protect\hyperlink{after-sponsor}{Continue reading the main story}

LIVE UPDATES

Updated~

Aug. 2, 2020, 2:14 a.m. ET

Aug. 2, 2020, 2:14 a.m. ET

\hypertarget{isaias-live-updates-storm-weakens-as-it-churns-toward-florida-after-hitting-bahamas}{%
\section{Isaias Live Updates: Storm Weakens as It Churns Toward Florida
After Hitting
Bahamas}\label{isaias-live-updates-storm-weakens-as-it-churns-toward-florida-after-hitting-bahamas}}

The tropical storm is expected to be upgraded to a hurricane again
overnight and may hit Florida's coast, further testing a state that has
been battered by the coronavirus pandemic.

Right Now

Isaias was downgraded to a tropical storm but is expected to grow
stronger overnight. It is about 70 miles southeast of Fort Lauderdale,
moving slowly toward Florida with winds swirling at about 70 miles per
hour.

\hypertarget{heres-what-you-need-to-know}{%
\subsubsection{Here's what you need to
know:}\label{heres-what-you-need-to-know}}

\begin{itemize}
\tightlist
\item
  \protect\hyperlink{link-1ab916c}{Isaias was downgraded to a tropical
  storm as it continues to move toward Florida's coastline.}
\item
  \protect\hyperlink{link-45d987d2}{Shelters began opening in Palm Beach
  County.}
\item
  \protect\hyperlink{link-1257f12e}{Nursing homes, already tested by the
  pandemic, may be vulnerable.}
\item
  \protect\hyperlink{link-65b0758c}{Florida is closing state-run
  coronavirus testing sites in the storm's path.}
\item
  \protect\hyperlink{link-41916cd2}{Forecasters predicted an active
  hurricane season, and it seems they were right.}
\item
  \protect\hyperlink{link-5019ef0e}{Astronauts avoided delay return from
  the International Space Station and are headed home.}
\end{itemize}

\includegraphics{https://static01.nyt.com/images/2020/08/01/business/01floriday-vid/01floriday-vid-videoSixteenByNine3000.jpg}

\subsection{}

Isaias was downgraded to a tropical storm as it continues to move toward
Florida's coastline.

After
\href{https://www.nytimes.com/2020/07/31/us/hurricane-isaias.html}{battering
the Bahamas and raking parts of Puerto Rico and the Dominican Republic}
with hurricane strength wind and rain, Isaias was downgraded Saturday
evening to a tropical storm.

Still, it continued its slow churn toward Florida's coastline, and state
officials said the storm would likely regain its strength as the evening
progressed.

``Don't be fooled by the downgrade,'' warned Gov. Ron DeSantis at a news
conference.

Floridians spent Saturday preparing for wind gusts up to 80 miles per
hour and dangerous coastal surf. And forecasters said they expected
Isaias to again become a Category 1 hurricane before scraping Florida's
coast.

\href{https://www.nytimes.com/interactive/2020/07/31/us/hurricane-isaias-tracker-map.html}{}

\includegraphics{https://static01.nyt.com/images/2020/07/31/us/hurricane-isaias-tracker-map-promo-1596209917104/hurricane-isaias-tracker-map-promo-1596209917104-articleLarge-v4.jpg}

\hypertarget{hurricane-isaias-tracking-map}{%
\subsection{Hurricane Isaias Tracking
Map}\label{hurricane-isaias-tracking-map}}

Follow the storm's path as it approaches the Florida coast.

Isaias's projected path shifted slightly eastward, forecasters said, and
was expected to possibly make landfall near Palm Beach, Jacksonville and
other coastal areas in the storm's possible path.

The Federal Emergency Management Agency offered federal disaster
assistance to the state on Saturday, a move approved by President Trump,
the agency announced in a statement.

Miami is no longer in the
\href{https://www.nytimes.com/2018/09/11/climate/hurricane-evacuation-path-forecasts.html}{``cone''
that signals the storm's possible paths}, but the National Weather
Service
\href{https://forecast.weather.gov/showsigwx.php?warnzone=FLZ073\&warncounty=FLC086\&firewxzone=FLZ073\&local_place1=3\%20Miles\%20W\%20Palm\%20Springs\%20North\%20FL\&product1=Hurricane+Local+Statement\&lat=25.9366\&lon=-80.3793\#.XyWaz_hKjOQ}{warned}
that the region could still see floods from heavy rain and damage from
strong winds.

Up the coast, officials in Georgia, South Carolina and North Carolina
were closely monitoring the storm, which is expected to move north and
could scrape the coasts of any of those states.

Like Florida, those three states have seen a dramatic rise in new
reported cases of the coronavirus since mid-June, and more recently,
health officials have warned that their health-care systems could be
strained beyond capacity with the flood of new patients. Emergency
management officials have been drawing up new plans to deal with people
fleeing amid the virus, including placing people in hotel rooms instead
of congregate shelters like basketball gyms.

Even so, Keith Acree, a North Carolina emergency management spokesman,
said the state was urging coastal residents to make plans to stay with
family or friends further inland. ``A shelter this year is not really
where you want to be this year during a pandemic,'' he said.

As of 2 a.m. Sunday, the tropical storm was about 70 miles southeast of
Fort Lauderdale and 90 miles southeast of West Palm Beach, moving toward
the coast at about 8 miles per hour. Its winds were swirling at about 70
m.p.h., just under the 74 m.p.h. threshold that would make it a Category
1 hurricane.

In Florida, a hurricane warning remains in effect from Boca Raton to the
northern edge of Volusia County, which includes Daytona Beach. A
tropical storm warning extends from the northern border of Volusia
County to the coast just southeast of Jacksonville. And a tropical storm
watch reaches about 40 miles north of Charleston, S.C.

Weather forecasters said dangerous storm surges up to four feet high in
some parts of Florida were possible.

Flooding caused by Isaias led to at least one death in Puerto Rico,
where a woman drowned in the municipality of Rincón, in the northwest of
the island, the Puerto Rico Department of Public Safety announced
Saturday in a statement. The woman had gone missing on Thursday, the
authorities said.

The storm is expected to weaken and be off the coast of Georgia on
Monday morning, and off the coast of South Carolina by Monday evening.

\hypertarget{-1}{%
\subsection{}\label{-1}}

Shelters began opening in Palm Beach County.

\includegraphics{https://static01.nyt.com/images/2020/08/02/us/02isaias-briefing-pets/merlin_175204521_aa8980a4-eb2d-4784-9feb-121268eb6362-articleLarge.jpg?quality=75\&auto=webp\&disable=upscale}

Palm Beach County opened four general population centers at one middle
school and three high schools Saturday morning as Isaias continued to
stretch up the Atlantic Coast. The shelters are available only to
residents who live in mobile homes or ``sub-standard'' housing, the
county
\href{https://discover.pbcgov.org/Lists/Newsroom/NewsDispForm.aspx?ID=3014}{said}.

As of Saturday afternoon, there were 150 people at the shelters,
officials said.

Individuals older than 2 will be required to wear face coverings, and
temperature screenings will be conducted for all residents who want to
enter the shelters. The county noted that social distancing protocols
would be in effect, and families staying at the shelters would be kept
further apart from each other. The county also said it would open one
pet friendly shelter for animals.

Still, county officials on Saturday urged residents to stay home and
avoid congregating in settings like shelters, if possible. For those
living in less stable housing, such as mobile homes, officials
recommended sheltering with a family member or friend who resides in a
sturdy home, or relocating to a hotel.

Bill Johnson, the emergency management director for Palm Beach County,
said on Saturday that storm conditions would begin impacting the county
in the evening. The county has issued a voluntary evacuation order for
residents who live in evacuation Zone A, which includes mobile and
manufactured homes.

``There is Covid in every aspect of your hurricane preparedness needs,''
Mr. Johnson said at a news conference on Friday. ``Shelters should be
considered your last resort.''

\hypertarget{-2}{%
\subsection{}\label{-2}}

Nursing homes, already tested by the pandemic, may be vulnerable.

Three years after
\href{https://www.nytimes.com/2019/08/24/us/4-charged-holywood-hills-deaths-hurricane-irma-florida.html}{a
dozen nursing home residents died during Hurricane Irma}, Florida's
former ombudsman warned that Isaias will test nursing homes already
battling the coronavirus pandemic.

Lawmakers passed regulations after air-conditioners failed at one home
in 2017, leading to heat-related deaths. They mandated that nursing
homes install backup generators in case of severe weather.

But this May, the state issued 95 variances --- passes that allow
facilities to operate despite noncompliance --- to nursing homes that
had not met the emergency requirements,
\href{https://www.miamiherald.com/article242595251.html}{according to
The Miami Herald}.

The former ombudsman, Brian Lee, who now runs Families for Better Care,
an advocacy group for nursing home and elder-care residents, said he
doubted state officials who said the nursing homes were prepared.

``I can't imagine that these facilities are prepped and ready to handle
a pandemic and a hurricane simultaneously,'' Mr. Lee said. ``They are
going to be over their heads and under water. It is a total recipe for
disaster.''

Kristen Knapp, a spokeswoman for the Florida Health Care Association,
which lobbies for nursing homes, said that not every nursing home that
applied for a variance was without a generator. Some facilities applied
for other reasons, including that they had not been able to perform
inspections because of the coronavirus, she said.

According to
\href{http://apps.ahca.myflorida.com/dm_web/(S(1v0gkxzpdxlo2x3vtgyf1qxk))/doc_results_fo.aspx}{records
from Florida's Agency for Health Care Administration}, some nursing
homes that were approved for variances were without generators as
recently as March, and were allowed to operate without generators until
June 1.

An agency spokesman said in an email that all nursing homes and assisted
living facilities have generators on site.

Some facilities plan to shelter in place rather than evacuate residents.
Mr. Lee said he was concerned that social distancing would be impossible
if dozens of residents were gathered in a common space.

``You get 120 residents and their caregivers in a large room in the
middle of a pandemic --- social distancing is out the window,'' Mr. Lee
said. ``This pandemic is really a threat to the residents in these
facilities, not just from a health care perspective, but for natural
disasters as well.''

\hypertarget{-3}{%
\subsection{}\label{-3}}

Florida is closing state-run coronavirus testing sites in the storm's
path.

Image

A coronavirus testing center at Hard Rock Stadium in Miami was closed
because of the threat of Hurricane Isaias.Credit...Daniel A.
Varela/Miami Herald, via Associated Press

Gov. Ron DeSantis said on Friday that state-run coronavirus testing
sites, which are mostly housed under tents at outdoor venues, will be
closed if they are within Isaias's anticipated path.

Many testing sites would be unsafe for lab personnel during the storm's
wind and rain, Mr. DeSantis said
\href{https://www.youtube.com/watch?v=FsKIF5je_bo}{during a news
conference} on Friday. Labs run by private companies, hospitals and
local county health departments
\href{https://www.floridadisaster.org/news-media/news/20200729-all-state-supported-testing-sites-temporarily-close-for-potential-tropical-cyclone-nine/}{will
not be affected} by the state's closure.

The governor, a Republican, had planned to close all of the state's
testing sites from Friday to Wednesday. But the Division of Emergency
Management eventually said it would keep testing sites open in counties
that should be unaffected.

In Miami-Dade County, the center of Florida's coronavirus outbreak,
Mayor Carlos Gimenez ordered the county-run sites to close from Friday
until at least Monday.

The county has recorded more than 20,000 cases in the past seven days.

``We have thousands of tests that will not be conducted until we get
these test sites up and running again,'' Mr. Gimenez said during a news
conference on Friday. ``We have to put safety first.''

On Thursday, Florida recorded 253 deaths, the state's most deaths in a
single day. While the number of daily new cases has declined in the
second half of July,
\href{https://www.nytimes.com/interactive/2020/us/florida-coronavirus-cases.html}{the
number of daily deaths} has trended upward.

\hypertarget{-4}{%
\subsection{}\label{-4}}

Forecasters predicted an active hurricane season, and it seems they were
right.

Because of warm ocean temperatures and other conditions this year,
\href{https://www.nytimes.com/2020/05/21/climate/hurricane-season-2020-noaa.html}{weather
experts said in May} that there would probably be more than the average
of 12 named storms.

The season, which runs from June 1 to November 30, is only one-third
over, and Isaias is already its ninth named storm, which requires
maximum sustained winds above 38 miles per hour.

June and July are usually quiet, which means the 2020 season could
approach
\href{https://slack-redir.net/link?url=https\%3A\%2F\%2Fwww.nhc.noaa.gov\%2Fdata\%2Ftcr\%2Findex.php\%3Fseason\%3D2005\%26basin\%3Datl}{the
record of 27 named storms set in 2005} --- the only time the National
Hurricane Center had to use Greek letters for some names.

Two factors combine to make the August to October period more active.
Ocean warmth, which provides the energy that fuels tropical storms and
hurricanes, reaches its peak in late summer. And differential winds that
can weaken storms by disrupting their rotating, or cyclonic, structure
are at their quietest.

Of the nine storms so far, seven were tropical storms, with wind speeds
below 74 miles an hour. The first hurricane, Hanna,
\href{https://www.nytimes.com/2020/07/26/us/virus-texas-storm.html}{which
struck South Texas last week}, was a Category 1 storm, with winds below
96 m.p.h. and so far Isaias's strength is about the same.

So while the season is busy, it remains to be seen whether another of
the predictions by the experts at the National Oceanic and Atmospheric
Administration are borne out. They forecast that three to six storms
this season would be major hurricanes, with sustained winds of 111
m.p.h. or higher.

\hypertarget{-5}{%
\subsection{}\label{-5}}

Astronauts avoided delay return from the International Space Station and
are headed home.

\hypertarget{returning-to-earth}{%
\subsection{Returning to Earth}\label{returning-to-earth}}

The SpaceX Crew Dragon is scheduled to splash down near Florida on
Sunday, though
\href{https://www.nytimes.com/interactive/2020/07/31/us/hurricane-isaias-tracker-map.html}{Hurricane
Isaias} could change those plans.

Category

2

1

Tropical storm

Forecasted path of Hurricane Isaias

Atlantic Ocean

Mon. 2 a.m.

Seven possible

splashdown sites

(approximate)

Gulf of Mexico

Sun. 2 a.m.

Sat. 2 a.m.

Last updated Sat. 9 a.m.

Category

2

1

Tropical storm

Forecasted path of Hurricane Isaias

Mon. 2 a.m.

Seven possible

splashdown sites

(approximate)

Gulf of Mexico

Sun. 2 a.m.

Sat. 2 a.m.

Last updated Sat. 9 a.m.

Category

2

1

Tropical storm

Forecasted path of Hurricane Isaias

Mon.

2 a.m.

Seven possible

splashdown sites

(approximate)

Gulf of Mexico

Sun.

2 a.m.

Last updated Sat. 9 a.m.

By The New York Times \textbar{} Sources: NASA, National Hurricane
Center, Mapbox, OpenStreetMap

Though Isaias threatened to delay the return trip of two Florida-bound
\href{https://www.nytimes.com/2020/08/01/science/nasa-spacex-astronauts.html}{astronauts
who have been aboard the International Space Station} since June, they
were able to undock and begin their long journey home.

The astronauts, Robert L. Behnken and Douglas G. Hurley, blasted off to
the space station in May in the Crew Dragon capsule built and operated
by SpaceX, the rocket company started by Elon Musk. They are scheduled
to splash down on Sunday afternoon.

NASA and SpaceX have seven potential sites in the Atlantic Ocean and
Gulf of Mexico where the capsule and its passengers can splash down. But
the track of Isaias ruled out the three in the Atlantic.

At the splashdown site, winds must be less than 10 miles per hour, and
there are additional constraints on waves and rain. In addition,
helicopters that take part in the recovery of the capsule must be able
to fly and land safely.

The spacecraft undocked from the space station at about 7:35 p.m. on
Saturday and, if the weather conditions remain favorable, it is
scheduled to splash down in the Gulf of Mexico off Pensacola, Fla., at
2:41 p.m. on Sunday, NASA announced.

Reporting was contributed by Nicholas Bogel-Burroughs, Kenneth Chang,
Richard Fausset, Henry Fountain, Patricia Mazzei, Giulia McDonnell Nieto
del Rio,Frances Robles and Will Wright.

Advertisement

\protect\hyperlink{after-bottom}{Continue reading the main story}

\hypertarget{site-index}{%
\subsection{Site Index}\label{site-index}}

\hypertarget{site-information-navigation}{%
\subsection{Site Information
Navigation}\label{site-information-navigation}}

\begin{itemize}
\tightlist
\item
  \href{https://help.nytimes.com/hc/en-us/articles/115014792127-Copyright-notice}{©~2020~The
  New York Times Company}
\end{itemize}

\begin{itemize}
\tightlist
\item
  \href{https://www.nytco.com/}{NYTCo}
\item
  \href{https://help.nytimes.com/hc/en-us/articles/115015385887-Contact-Us}{Contact
  Us}
\item
  \href{https://www.nytco.com/careers/}{Work with us}
\item
  \href{https://nytmediakit.com/}{Advertise}
\item
  \href{http://www.tbrandstudio.com/}{T Brand Studio}
\item
  \href{https://www.nytimes.com/privacy/cookie-policy\#how-do-i-manage-trackers}{Your
  Ad Choices}
\item
  \href{https://www.nytimes.com/privacy}{Privacy}
\item
  \href{https://help.nytimes.com/hc/en-us/articles/115014893428-Terms-of-service}{Terms
  of Service}
\item
  \href{https://help.nytimes.com/hc/en-us/articles/115014893968-Terms-of-sale}{Terms
  of Sale}
\item
  \href{https://spiderbites.nytimes.com}{Site Map}
\item
  \href{https://help.nytimes.com/hc/en-us}{Help}
\item
  \href{https://www.nytimes.com/subscription?campaignId=37WXW}{Subscriptions}
\end{itemize}
