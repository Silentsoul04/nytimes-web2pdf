Sections

SEARCH

\protect\hyperlink{site-content}{Skip to
content}\protect\hyperlink{site-index}{Skip to site index}

\href{https://www.nytimes.com/section/politics}{Politics}

\href{https://myaccount.nytimes.com/auth/login?response_type=cookie\&client_id=vi}{}

\href{https://www.nytimes.com/section/todayspaper}{Today's Paper}

\href{/section/politics}{Politics}\textbar{}Tammy Duckworth Is Nothing
and Everything Like Joe Biden

\url{https://nyti.ms/30hGfKy}

\begin{itemize}
\item
\item
\item
\item
\item
\end{itemize}

\begin{itemize}
\item
  \href{https://www.nytimes.com/2020/07/31/us/elections/biden-vs-trump.html?action=click\&pgtype=Article\&state=default\&region=TOP_BANNER\&context=storylines_menu}{Election
  Updates}
\item
  \href{https://www.nytimes.com/article/biden-vice-president-2020.html?action=click\&pgtype=Article\&state=default\&region=TOP_BANNER\&context=storylines_menu}{Biden's
  V.P. Search}
\item
  \href{https://www.nytimes.com/interactive/2020/07/24/us/politics/trump-biden-campaign-donors.html?action=click\&pgtype=Article\&state=default\&region=TOP_BANNER\&context=storylines_menu}{Map
  of Donations}
\item
  \href{https://www.nytimes.com/interactive/2020/us/elections/delegate-count-primary-results.html?action=click\&pgtype=Article\&state=default\&region=TOP_BANNER\&context=storylines_menu}{Delegate
  Count}
\item
  \href{https://www.nytimes.com/interactive/2019/us/politics/2020-presidential-candidates.html?action=click\&pgtype=Article\&state=default\&region=TOP_BANNER\&context=storylines_menu}{The
  Candidates}
\item
  \href{https://www.nytimes.com/newsletters/politics?action=click\&pgtype=Article\&state=default\&region=TOP_BANNER\&context=storylines_menu}{Politics
  Newsletter}
\end{itemize}

Advertisement

\protect\hyperlink{after-top}{Continue reading the main story}

Supported by

\protect\hyperlink{after-sponsor}{Continue reading the main story}

\hypertarget{tammy-duckworth-is-nothing-and-everything-like-joe-biden}{%
\section{Tammy Duckworth Is Nothing and Everything Like Joe
Biden}\label{tammy-duckworth-is-nothing-and-everything-like-joe-biden}}

Despite their disparate backgrounds, the Illinois Democrat has carved
out a public life most evocative of the man she could join on the
presidential ticket.

\includegraphics{https://static01.nyt.com/images/2020/08/02/us/politics/02duckworth-A1/00duckworth1-articleLarge.jpg?quality=75\&auto=webp\&disable=upscale}

\href{https://www.nytimes.com/by/matt-flegenheimer}{\includegraphics{https://static01.nyt.com/images/2018/10/02/multimedia/author-matt-flegenheimer/author-matt-flegenheimer-thumbLarge.png}}

By \href{https://www.nytimes.com/by/matt-flegenheimer}{Matt
Flegenheimer}

\begin{itemize}
\item
  Aug. 1, 2020, 5:00 a.m. ET
\item
  \begin{itemize}
  \item
  \item
  \item
  \item
  \item
  \end{itemize}
\end{itemize}

Senator Tammy Duckworth, like the man she might serve as vice president,
prizes loyalty in her ranks and occasional mischief in her workplace.

So when a top communications aide prepared to defect last year to the
presidential campaign of Pete Buttigieg, Ms. Duckworth recognized an
opportunity. She recorded a faux media interview trashing Mr. Buttigieg
for hiring her staff away, recruiting an intern to pose as a journalist
on the tape. The file was sent to the departing aide, Sean Savett, who
called the Buttigieg team in a panic.

Soon, Mr. Savett was summoned to the Illinois senator's office, where
she fumed theatrically, stalling as other staff members filed in quietly
for the reveal: It was all a ruse. Ms. Duckworth handed him a parting
gift --- a Smirnoff Ice, the centerpiece of a viral drinking game known
as ``icing'' --- and gave a final senatorial directive:

``Get down on one knee and chug.''

A year later, Ms. Duckworth is the one thinking about a new job and
submitting to the attendant rituals.
\href{https://www.nytimes.com/interactive/2020/us/elections/joe-biden.html}{Joseph
R. Biden Jr.}, the presumptive Democratic nominee, is vetting her to be
his running mate, and many of his allies see the freshman senator as a
model contrast to
\href{https://www.nytimes.com/interactive/2020/us/elections/donald-trump.html}{President
Trump}: a death-cheating, double-amputee Iraq war veteran whose life
story --- whose very appearance, whooshing by wheelchair through the
Capitol --- defines the decency and service that the president's
opponents have found lacking in this White House.

There are more accomplished legislators than Ms. Duckworth under
consideration. There are more prolific policy thinkers and more electric
campaigners.

But in bearing and biography, Ms. Duckworth, 52, is almost certainly the
Biden-est choice --- the would-be lieutenant who has, despite their
disparate backgrounds, carved out a public life most evocative of his
own. Though both are known as reliable Democrats whose more moderate
instincts can sometimes disappoint progressives, they are also the kinds
of politicians whose politics can feel beside the point to many voters.

Like Mr. Biden, who entered the national consciousness as a 30-year-old
senator-elect left to mourn his wife and daughter, Ms. Duckworth has
forged a political identity around trauma and personal resilience, her
status as a wounded warrior shadowing every inch of her professional arc
since her Black Hawk helicopter was shot down outside Baghdad in 2004.

In an interview, Ms. Duckworth suggested the two share a perspective
that can flow only from confronting unfathomable pain, from sitting with
loss and slogging through Plan B anyway.

``Why did some troops come home from a trauma and survive and thrive?
And why do some come home and kill themselves?'' Ms. Duckworth asked,
without answering. ``You could almost say that I'm a success story of
someone who survived a trauma. But it wasn't easy. And I think that's
what Vice President Biden and I have in common. We've been able to face
the demons. We've been able to face the fear, the doubts and all of
that, and we're still here. But we both know that it's not easy.''

\includegraphics{https://static01.nyt.com/images/2020/08/02/us/politics/00duckworth8/merlin_63188633_07860d57-5a7e-4023-b531-f8b43e1aa9df-articleLarge.jpg?quality=75\&auto=webp\&disable=upscale}

Less weighty parallels, in style and political substance, likewise imply
an intuitive partnership.

Like Mr. Biden --- whose
\href{https://www.nytimes.com/2019/10/30/us/politics/joe-biden-debate-gaffes.html}{decades
of verbal blunders} have not kept him from six Senate terms, the vice
presidency and the Democratic presidential nomination --- Ms. Duckworth
can at times sound less than smooth at a microphone but has rarely paid
much of a penalty for it. Past rivals say this owes, in part, to the
campaign perils of insulting someone so visibly marked as a survivor of
war. Most recently, after Ms. Duckworth suggested clumsily that removing
monuments of George Washington merited discussion,
\href{https://www.nytimes.com/2020/07/08/us/politics/tucker-carlson-tammy-duckworth.html}{attacks
on her patriotism} from conservatives like Tucker Carlson seemed to only
boost her reputation among Democrats.

And ideologically, Ms. Duckworth would appear closely attuned to Mr.
Biden. She has spent much of her career positioned to the right of
liberal Democrats, retaining some centrist muscle memory from her
unsuccessful first congressional race in 2006 --- when she pledged
fiscal conservatism and punishments for ``illegal immigrants'' --- and
occasionally leading Republicans to wonder if they are looking at a
kindred soul.

``I had a chance to develop a friendship with Tammy about 15 years ago
while we were both out at Walter Reed,'' Bob Dole, the former Republican
senator and presidential nominee, said in an emailed statement,
recalling
\href{https://www.nytimes.com/2005/04/10/politics/dole-discloses-emergency-that-nearly-took-his-life.html}{his
time} as a patient at the veterans hospital during Ms. Duckworth's stay
there. ``In hindsight, I wish I had brought up politics. She could have
run as a Republican.''

Yet Ms. Duckworth's is a worldview that has long defied easy labeling.
She is at once the product of a globe-trotting conservative military
family sustained by food stamps in her youth and a soldier who gave her
limbs to a war whose wisdom she came to question. She is a woman well
acquainted with male-dominated worlds --- fellow pilots called her
``Mommy Platoon Leader'' long before
\href{https://www.nytimes.com/2018/04/09/us/politics/tammy-duckworth-birth.html}{she
became the first sitting senator to give birth}, at age 50 --- and a
canny politician whose connections helped guide her to the upper reaches
of her party.

Those close to Ms. Duckworth still describe her present career as
something of a consolation prize. Plan A was flying helicopters, and she
did not surrender the vision easily.

Recovering in 2005, Ms. Duckworth vowed that ``some guy who got lucky
one day in Baghdad'' would not dictate her future.

Nine years later, concluding her first congressional term, she
reconsidered.

``I mean, it did,'' she
\href{https://www.dailyherald.com/article/20141111/news/141119763}{conceded}
to a reporter. ``I'm in politics.''

Image

Ms. Duckworth went through flight school and entered the Illinois
National Guard in 1996. She deployed for Iraq in 2004.Credit...via
Senator Tammy Duckworth

\hypertarget{plan-a-flying-helicopters}{%
\subsection{Plan A: Flying
Helicopters}\label{plan-a-flying-helicopters}}

The campus misogynist was enjoying his soapbox. Ms. Duckworth wanted to
keep it that way.

It was the early 1990s at Northern Illinois University, where Ms.
Duckworth was pursuing a Ph.D. in political science, and a traveling
evangelist had been lamenting the evils of skirt-wearing women in a
public square.

\hypertarget{latest-updates-2020-election}{%
\section{\texorpdfstring{\href{https://www.nytimes.com/2020/07/31/us/elections/biden-vs-trump.html?action=click\&pgtype=Article\&state=default\&region=MAIN_CONTENT_1\&context=storylines_live_updates}{Latest
Updates: 2020
Election}}{Latest Updates: 2020 Election}}\label{latest-updates-2020-election}}

Updated 2020-08-01T01:26:45.732Z

\begin{itemize}
\tightlist
\item
  \href{https://www.nytimes.com/2020/07/31/us/elections/biden-vs-trump.html?action=click\&pgtype=Article\&state=default\&region=MAIN_CONTENT_1\&context=storylines_live_updates\#link-29fdff45}{Kamala
  Harris, a top vice-presidential contender, confronts double
  standards.}
\item
  \href{https://www.nytimes.com/2020/07/31/us/elections/biden-vs-trump.html?action=click\&pgtype=Article\&state=default\&region=MAIN_CONTENT_1\&context=storylines_live_updates\#link-13ec3d9c}{Karen
  Bass and Susan Rice are rising on Biden's vice-presidential
  shortlist.}
\item
  \href{https://www.nytimes.com/2020/07/31/us/elections/biden-vs-trump.html?action=click\&pgtype=Article\&state=default\&region=MAIN_CONTENT_1\&context=storylines_live_updates\#link-49e9a016}{Trump
  says Russian bounties to kill U.S. troops `never took place.'}
\end{itemize}

\href{https://www.nytimes.com/2020/07/31/us/elections/biden-vs-trump.html?action=click\&pgtype=Article\&state=default\&region=MAIN_CONTENT_1\&context=storylines_live_updates}{See
more updates}

``I came in and said, `I wish somebody would shut that guy up,'''
recalled Patricia Henry, one of Ms. Duckworth's professors. ``She said:
`No, no, no. You can't do that.'''

Friends say such earnest alarm over would-be speech infringement
reflects Ms. Duckworth's itinerant youth across Southeast Asia, which
often exposed her to repressive governments and introduced her to the
tenets of American democracy through the rose-colored lens of a child
expat.

Born in Bangkok to a white American veteran father and a Thai mother of
Chinese descent, Ms. Duckworth did not learn English until she was 8.
(Some Democrats suspect that the president and his allies would make an
issue of her birthplace if Mr. Biden chooses her, recalling Mr. Trump
\href{https://www.politico.com/story/2016/04/donald-trump-ted-cruz-canada-222347}{questioning}
the presidential eligibility of Senator Ted Cruz, another American
citizen born outside the country, when the two competed for the
Republican nomination in 2016.)

Some of Ms. Duckworth's earliest memories involve the Khmer Rouge
seizing control of Cambodia, where her father was working for the United
Nations. She remembers watching bombs go off in Phnom Penh from their
rooftop. Her upbringing, she said, gave her ``an idealized version of
America.''

More than that, these seminomadic years seemed to enforce a certain
comfort level with short-notice upheaval.

``There's a built-in flexibility with children who've grown up as
expats,'' said Alison Parsons, a close friend who attended school with
Ms. Duckworth in Jakarta and Bangkok. ``You have to be able to reinvent
yourself. I'm not talking about flip-flopping, but you have to be able
to make friends, to make connections on a dime.''

Facing financial distress, Ms. Duckworth's father moved the family to
Hawaii in her teens, finding space in a down-market hotel and leaning on
public assistance.

Imagining a life in the foreign service, she graduated from the
University of Hawaii before moving to the mainland for an international
affairs program at George Washington University. She held up Madeleine
Albright as a role model.

Image

Ms. Duckworth with her father, Franklin, in the 1990s.Credit...via
Senator Tammy Duckworth

But while in school, Ms. Duckworth joined the Army Reserve Officers'
Training Corps, partly because she noticed that many of her friends had
military backgrounds.

She found herself taken with the ostensible meritocracy, she said, that
allowed a ``little Asian girl'' to rise so long as she could shoot
straight, even as one fellow cadet, Bryan Bowlsbey, tested her nerves.

``He made a comment that I thought was derogatory about the role of
women in the Army,'' she told C-SPAN years later. ``But he came over and
apologized very nicely and then helped me clean my M16.''

They have been married since 1993. Mr. Bowlsbey now works as an
information technology consultant.

Though Ms. Duckworth moved to Illinois to pursue a doctorate, she went
through flight school and entered the Illinois National Guard in 1996.

Before her deployment eight years later, Ms. Duckworth had been working
at Rotary International, helping to manage offices in its Asia-Pacific
region. When the Guard sought out commissioned officers for a mission to
Iraq, she volunteered, arriving in March 2004. (Ms. Duckworth
\href{https://www.wbez.org/stories/ep-106-sen-tammy-duckworth/aa3e6284-2b53-44e5-9312-83e82aef723a}{has
said} she always believed the Bush administration ``started this war for
themselves,'' but as a soldier, ``you keep your personal opinions to
yourself.'')

Ms. Duckworth spent much of her time there inside an operations center,
coordinating missions. She flew herself about twice a week.

Her last waking day in Iraq, Nov. 12, 2004, began unremarkably. Ms.
Duckworth's crew was conducting ``taxi service,'' in her telling:
shuttling people and supplies, with a stop at a base in Baghdad to
acquire Christmas ornaments.

She had been at the controls all day. A colleague, Dan Milberg,
playfully called her a ``stick pig,'' requesting to take the lead on a
final flight. She obliged.

They were about 10 minutes from their destination when an explosion
scorched through the right side of the cockpit, where Ms. Duckworth sat:

A rocket-propelled grenade. A fireball blast at her lower body.

She does not remember feeling pain immediately. She does remember the
black smoke --- and an aircraft suddenly impervious to her prompts. By
this point, Ms. Duckworth learned later, she had no feet.

Mr. Milberg was able to land on a plot of open woods. Ms. Duckworth, on
the cusp of losing consciousness, has retained a snapshot from the haze
of her rescue: a cluster of tall grass poking through the base of the
Black Hawk. She wondered how it had gotten there.

Image

Ms. Duckworth at a town hall event during her first campaign for
Congress in Illinois in 2006.~Credit...Sally Ryan for The New York Times

\hypertarget{plan-b-politics}{%
\subsection{Plan B: Politics}\label{plan-b-politics}}

Ms. Duckworth awoke over a week later at Walter Reed. Her legs were
gone.

The next days passed in a whir of continuous trauma: surgeries,
hallucinations from morphine, flashes of guilt that she had somehow
crashed herself.

Ms. Duckworth's mother and her husband took turns counting to 60 at her
side, guiding her from one minute to the next. And soon, there was
another patient on the hospital grounds: Her father, who had suffered a
heart attack in Hawaii shortly before his daughter's injuries, had
another after traveling to see her. He died a few weeks after Christmas.

Around the same time, a new mentor figure entered Ms. Duckworth's life.
Senator Dick Durbin, Democrat of Illinois, had been looking for local
veterans to invite to President George W. Bush's State of the Union
address. Ms. Duckworth attended with an IV drip running beneath her
clothes.

The senator asked her to stay in touch. ``I gave her my personal
cellphone number,'' he remembered, ``which she greatly abused by calling
me --- I say that in jest, of course --- by calling me incessantly to do
constituent work for all of her fellow vets at Walter Reed.''

The rehab process was painful and often slow-going. Her left leg was
amputated below the knee. Her right was an inches-long stump that Ms.
Duckworth had asked doctors to leave, despite the complications of
fitting a prosthetic to it, because she believed it would help her fly
again.

Image

Ms. Duckworth during a physical therapy session at Walter Reed in
January 2005.Credit...Michael Chow/The Arizona Republic, via Associated
Press

It was not until later that year, she said, that a call from Mr. Durbin
made her consider an alternate path. There was a congressional seat
coming open in the Chicago suburbs with the retirement of a long-tenured
Republican, Henry Hyde.

``I said, `Tammy would you ever consider running?''' Mr. Durbin
recalled. ``She didn't say no.''

By the summer, with a full return to combat looking remote, Ms.
Duckworth had been casting about for her next ``mission,'' she said. A
campaign seemed as good an option as any.

The transition was not frictionless. Like many first-time candidates,
Ms. Duckworth could be tempted to act as her own campaign manager,
former advisers say, seeking to impose military efficiency on overlong
phone calls. Unlike many first-time candidates, she was still learning
to walk in her new legs.

One focus group of Democratic primary voters bristled when Ms. Duckworth
wore a skirt, saying that the prominence of her prosthetics felt like
the calculating work of operatives.

``There was a big negative reaction,'' said John Kupper, an adviser to
the campaign. ``They thought they were being manipulated.'' (Ms.
Duckworth has said she prefers skirts because they make bathroom visits
less logistically complicated.)

Her military background was more of an asset in the general election for
a right-leaning district. She remarked to voters that she had been shot
down ``18 months after the mission was accomplished,'' nodding at the
Bush administration's
\href{https://www.nytimes.com/2003/10/29/world/bush-steps-away-from-victory-banner.html}{infamous
premature victory lap}.

She patiently identified herself in calls to would-be donors, who often
interrupted her health care pitch with questions about her life.

``Yes,'' she would tell them, ``I'm the one who was injured.''

Ms. Duckworth would ultimately lose, narrowly, to Peter Roskam, a local
Republican legislator. But the contest drew national attention and
enshrined Ms. Duckworth as a potential star in the party.

Image

Ms. Duckworth lost her first congressional race for a seat in the
Chicago suburbs.Credit...Peter Wynn Thompson for The New York Times

Rod Blagojevich, the not-yet-jailed governor of Illinois, appointed her
to lead the state's veterans department. Her name was floated as a
possible Senate replacement as Barack Obama chased the presidency.

And at the 2008 Democratic convention in Denver, Ms. Duckworth was
invited to speak in prime time on the night Mr. Biden accepted the
vice-presidential nomination. She joined the Biden family backstage
beforehand, convening ``soldier to soldier'' with Beau Biden, she
recalled, just shy of his own deployment.

``It was a family moment,'' she said, ``and they allowed me to join.''

The speech seemed to erase any doubt that Ms. Duckworth was a politician
now --- or, at least, that she would be again before long. After joining
the Obama administration in 2009 as an assistant secretary for veterans
affairs, she took notice as a favorable district redrawing supplied a
cleaner shot at a House seat.

When Ms. Duckworth decided to run again, in 2012, she was the one
picking up the phone.

``There are some candidates you have to recruit,'' said Steve Israel,
then the chairman of the Democratic Congressional Campaign Committee.
``She called me.''

Image

Senator Dick Durbin of Illinois became a mentor for Ms. Duckworth after
meeting her at Walter Reed.~Credit...Scott Olson/Getty Images

\hypertarget{the-plan-from-here-on-out}{%
\subsection{The plan from here on out}\label{the-plan-from-here-on-out}}

Ms. Duckworth's years in Congress since then --- four in the House,
nearly four in the Senate --- have done little to eclipse the central
facts of her biography.

Perhaps this was inevitable. Major policy feats can be elusive in the
minority party. Voters who know much about Ms. Duckworth nationally seem
likelier to recall her path to Washington than her work while there.
Since defeating Mark Kirk, the incumbent Republican senator, in 2016,
she has probably received the most attention for another personal turn:
\href{https://www.nytimes.com/2018/04/19/us/politics/baby-duckworth-senate-floor.html}{bringing
her newborn to a Senate vote}, a first for the chamber.

Colleagues praise Ms. Duckworth as a forceful advocate for veterans and
people with disabilities but sometimes struggle to name her signature
legislative triumphs.

She is not considered a foremost national voice in some policy areas of
particular significance in this moment, like policing and the economy
--- a potential weakness in her case to be vice president.

Ms. Duckworth has generally opposed the legislative priorities and
high-profile nominations of this White House, with a handful of
exceptions, including a vote supporting Wilbur Ross for commerce
secretary, which a majority of Democrats opposed, and another for John
Kelly as homeland security secretary.

Mr. Trump has signed into law legislation that Ms. Duckworth pushed
involving veteran entrepreneurship and expanded access to lactation
rooms in airports. Her office is quick to cite
\href{https://news.vanderbilt.edu/2019/02/28/grassley-klobuchar-most-effective-senators-of-115th-congress-according-to-study/}{an
analysis} last year identifying her as the most effective freshman
Democratic senator.

Some peers say she has been especially valuable during private sessions
on foreign policy. Senator Richard Blumenthal, Democrat of Connecticut
and a fellow member of the Armed Services Committee, recalled Ms.
Duckworth's lacerating questions recently at a classified briefing about
intelligence community assessments of
\href{https://www.nytimes.com/2020/06/26/us/politics/russia-afghanistan-bounties.html}{apparent
Russian bounties on American troops}.

``She was pummeling them,'' Mr. Blumenthal said.

Among staff, Ms. Duckworth can be more puckish, known to celebrate
``Talk Like a Pirate Day'' and razz communications aides by suggesting
that she has just uttered something damaging to congressional reporters:
``Don't really know what I said,'' she has bluffed upon returning to the
office. ``You might want to track them down.''

Image

Ms. Duckworth arrived at the Capitol with her daughter Maile, who was 10
days old, for a vote in 2018.Credit...Erin Schaff for The New York Times

It is true, though, that Ms. Duckworth can seem less practiced than some
other senators when speaking to the press, mixing self-deprecation with
political self-assessments that might dishearten the left.

In the interview, Ms. Duckworth by turns explained why the vetting
process had been uncomplicated (``I was a soldier for 23 years, and I
don't have a lot of money''), said she remained a fiscal conservative
(with an aside about wasteful defense contracts) and appeared to
acknowledge that her coordinates on the ideological spectrum were
difficult to track.

``People talk to me, and they're like, `So are you lefty, or are you
ultra-conservative and a hawk?''' she said. ``I'm like, `I'm just about
the strength of America.'''

Ms. Duckworth is not the sort of senator who had been discussed as an
instant presidential hopeful, like Kamala Harris, another freshman. Many
Democrats believe that vice-presidential contenders with more experience
in a national race, like Ms. Harris or Senator Elizabeth Warren, would
be wiser picks.

Yet in recent weeks, Ms. Duckworth said, she has been compelled to
consider a life one septuagenarian's heartbeat away from the presidency
--- and whether she might be ready for the highest promotion, if
required.

She defaulted to military imagery (``every soldier is taught to be able
to pick up the rifle of a fallen comrade in front of them'') and ticked
through her credentials, sounding for the first time like a job
applicant: Senate, House, V.A., Ph.D., speaker of ``a bunch of
languages.''

And then Ms. Duckworth cut herself off, abandoning the hypothetical with
a promise: ``I'm going to do everything I can to keep Joe Biden as
healthy as he can possibly be.''

She let a long laugh fly, imagining her place in the command.

``I'll be the one like, `Here, here, take your vitamins,''' she said.
```Let's go work out together.'''

\hypertarget{our-2020-election-guide}{%
\section{Our 2020 Election Guide}\label{our-2020-election-guide}}

Updated July 31, 2020

\begin{itemize}
\item
  \begin{center}\rule{0.5\linewidth}{\linethickness}\end{center}

  \hypertarget{the-latest}{%
  \subsection{The Latest}\label{the-latest}}

  \begin{itemize}
  \tightlist
  \item
    President Trump's assault on the Postal Service is intersecting with
    his attacks on mail-in voting.
    \href{https://www.nytimes.com/2020/07/31/us/politics/trump-usps-mail-delays.html?action=click\&pgtype=Article\&state=default\&region=BELOW_MAIN_CONTENT\&context=storylines_guide}{Voting
    rights groups say it is a recipe for disaster.}
  \end{itemize}
\item
  \begin{center}\rule{0.5\linewidth}{\linethickness}\end{center}

  \hypertarget{bidens-vp-search}{%
  \subsection{Biden's V.P. Search}\label{bidens-vp-search}}

  \begin{itemize}
  \tightlist
  \item
    \href{https://www.nytimes.com/article/biden-vice-president-2020.html?action=click\&pgtype=Article\&state=default\&region=BELOW_MAIN_CONTENT\&context=storylines_guide}{Here
    are 13 women} who have been under consideration to be Joe Biden's
    running mate, and why each might be chosen --- and might not be.
  \end{itemize}
\item
  \begin{center}\rule{0.5\linewidth}{\linethickness}\end{center}

  \hypertarget{keep-up-with-our-coverage}{%
  \subsection{Keep Up With Our
  Coverage}\label{keep-up-with-our-coverage}}

  \begin{itemize}
  \tightlist
  \item
    Get an
    \href{https://www.nytimes.com/newsletters/politics?action=click\&pgtype=Article\&state=default\&region=BELOW_MAIN_CONTENT\&context=storylines_guide}{email}
    recapping the day's news
  \end{itemize}

  \begin{itemize}
  \tightlist
  \item
    Download our mobile app on
    \href{https://apps.apple.com/us/app/nytimes/id284862083?ls=1\&mat_click_id=5c79ae7455014fd1bd66b5610c05b8f2-20191112-16948\&referrer=mat_click_id\%3D5c79ae7455014fd1bd66b5610c05b8f2-20191112-16948\%26link_click_id\%3D722930677036718082}{iOS}
    and
    \href{http://a.localytics.com/android?id=com.nytimes.android\&referrer=utm_source\%3Dother_nyt_mobile_web\%26utm_medium\%3DWeb\%2520page\%26utm_term\%3DGeneral\%2520Mobile\%2520Page\%26utm_campaign\%3DNYT\%2520Mobile\%2520General\%2520Page}{Android}
    and turn on Breaking News and Politics alerts
  \end{itemize}
\end{itemize}

Advertisement

\protect\hyperlink{after-bottom}{Continue reading the main story}

\hypertarget{site-index}{%
\subsection{Site Index}\label{site-index}}

\hypertarget{site-information-navigation}{%
\subsection{Site Information
Navigation}\label{site-information-navigation}}

\begin{itemize}
\tightlist
\item
  \href{https://help.nytimes.com/hc/en-us/articles/115014792127-Copyright-notice}{©~2020~The
  New York Times Company}
\end{itemize}

\begin{itemize}
\tightlist
\item
  \href{https://www.nytco.com/}{NYTCo}
\item
  \href{https://help.nytimes.com/hc/en-us/articles/115015385887-Contact-Us}{Contact
  Us}
\item
  \href{https://www.nytco.com/careers/}{Work with us}
\item
  \href{https://nytmediakit.com/}{Advertise}
\item
  \href{http://www.tbrandstudio.com/}{T Brand Studio}
\item
  \href{https://www.nytimes.com/privacy/cookie-policy\#how-do-i-manage-trackers}{Your
  Ad Choices}
\item
  \href{https://www.nytimes.com/privacy}{Privacy}
\item
  \href{https://help.nytimes.com/hc/en-us/articles/115014893428-Terms-of-service}{Terms
  of Service}
\item
  \href{https://help.nytimes.com/hc/en-us/articles/115014893968-Terms-of-sale}{Terms
  of Sale}
\item
  \href{https://spiderbites.nytimes.com}{Site Map}
\item
  \href{https://help.nytimes.com/hc/en-us}{Help}
\item
  \href{https://www.nytimes.com/subscription?campaignId=37WXW}{Subscriptions}
\end{itemize}
