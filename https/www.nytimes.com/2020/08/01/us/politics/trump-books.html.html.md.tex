Sections

SEARCH

\protect\hyperlink{site-content}{Skip to
content}\protect\hyperlink{site-index}{Skip to site index}

\href{https://www.nytimes.com/section/politics}{Politics}

\href{https://myaccount.nytimes.com/auth/login?response_type=cookie\&client_id=vi}{}

\href{https://www.nytimes.com/section/todayspaper}{Today's Paper}

\href{/section/politics}{Politics}\textbar{}In Trumpworld, the Grown-Ups
in the Room All Left, and Got Book Deals

\url{https://nyti.ms/3jZ0Uel}

\begin{itemize}
\item
\item
\item
\item
\item
\end{itemize}

Advertisement

\protect\hyperlink{after-top}{Continue reading the main story}

Supported by

\protect\hyperlink{after-sponsor}{Continue reading the main story}

News Analysis

\hypertarget{in-trumpworld-the-grown-ups-in-the-room-all-left-and-got-book-deals}{%
\section{In Trumpworld, the Grown-Ups in the Room All Left, and Got Book
Deals}\label{in-trumpworld-the-grown-ups-in-the-room-all-left-and-got-book-deals}}

A large club of Trump administration evictees have turned their
bracingly bad experiences into a new genre: political revenge
literature.

\includegraphics{https://static01.nyt.com/images/2020/08/01/us/politics/01trump-books-final/01trump-books-final-articleLarge.jpg?quality=75\&auto=webp\&disable=upscale}

\href{https://www.nytimes.com/by/sarah-lyall}{\includegraphics{https://static01.nyt.com/images/2018/02/20/multimedia/author-sarah-lyall/author-sarah-lyall-thumbLarge.jpg}}

By \href{https://www.nytimes.com/by/sarah-lyall}{Sarah Lyall}

\begin{itemize}
\item
  Aug. 1, 2020, 11:56 a.m. ET
\item
  \begin{itemize}
  \item
  \item
  \item
  \item
  \item
  \end{itemize}
\end{itemize}

It was the summer of 2016, and the Republican Party was about to
nominate Donald J. Trump for president. Until then, many party members
had aggressively opposed his candidacy. ``I think he's crazy,'' Senator
Lindsey Graham, Republican of South Carolina, said earlier that year.
``I think he's unfit for office.''

But faced with the inevitable reality of Mr. Trump, the party was forced
to perform what the British tabloids call a
``\href{https://www.forbes.com/sites/andrewbusby/2019/03/29/the-reverse-ferret-and-the-department-store-how-today-became-debs-day/\#6eebddc017de}{reverse-ferret}''
--- a messaging U-turn in which you abruptly take the opposite position
of the one you espoused a moment earlier.

Contrary to what they said before, the Republicans announced, Mr. Trump
was totally suited for the presidency. He would rise to the occasion.
Being president would render him, tautologically, presidential. In any
case, at least he would be surrounded by adults who would steer him in
the right direction.

``It began to dawn on me,'' Anthony Scaramucci, who went on to (briefly)
work in the Trump White House, wrote of hearing about the
then-candidate's tax proposals. ``Donald J. Trump wasn't the extreme,
unhinged, unserious candidate that I thought he was.''

Mr. Scaramucci spent just 11 days as the White House communications
director in 2017 before being
\href{https://www.nytimes.com/2017/07/31/us/politics/trump-white-house-obamacare-health.html}{unceremoniously
removed}, a victim of his own operatic ineptitude as well as the
dysfunction of the White House. He now regrets the error, as he sees it,
of ever having admired Mr. Trump. ``The guy stinks,'' he
\href{https://www.theguardian.com/tv-and-radio/2020/jul/17/the-guy-stinks-and-hes-a-racist-anthony-scaramucci-on-donald-trump}{said
recently}.

As it happens, Mr. Scaramucci wrote a book about his brief, unhappy
White House experience, joining a large club of Trump administration
evictees who have turned their bracingly bad experiences into a new
genre of political revenge literature. These include
\href{https://www.nytimes.com/2019/10/12/us/politics/james-comey-trump.html}{James
Comey}, former F.B.I. director; Omarosa Manigault Newman, former
assistant to the president; Andrew McCabe, former deputy F.B.I.
director; John Bolton, former national security adviser; Cliff Sims,
former White House communications aide; and Anonymous, current senior
figure, at least by his or her own account, in the Trump administration.

(There's also Sean Spicer, former press secretary, who wrote a mostly
complimentary book about his fleeting White House tenure; and Mary
Trump, not an ex-staffer but the president's niece, whose scathing
portrait of Trump family pathology came out in July and sold 1.35
million copies across all formats in its first week. That book is
currently No. 1 on hardback nonfiction best-seller lists in the United
States, Britain, Canada and Ireland, and No. 2 in Australia.)

\includegraphics{https://static01.nyt.com/images/2020/08/02/us/politics/trump-books-covers/oakImage-1595963353180-articleLarge.jpg?quality=75\&auto=webp\&disable=upscale}

Taken en masse, the books paint a damning portrait of the 45th president
of the United States. But the sheer volume of unflattering material they
contain can have the paradoxical danger of blunting their collective
impact. After the 10th time you read about Mr. Trump's short attention
span, your own attention is in danger of wandering.

``There is only so much the public can absorb,'' Anonymous writes in ``A
Warning.''

There are even more memoirs scheduled for the fall: one by Michael
Cohen, the president's disgraced ex-personal lawyer, which federal
officials tried to block but then said could proceed, and another by
H.R. McMaster, who was Mr. Trump's second national security adviser and
is no fan of the president.

But at this point, nearly four years in, is there anything left to say
about Mr. Trump that might surprise us? Or, as Mr. McCabe writes in
``The Threat'': ``What more could a person do to erode the credibility
of the presidency?''

Reading all these books, one after the other, is like swimming for days
in a greasy, brackish canal whose bottom is teeming with shards of
broken-down old industrial equipment. The experience is not pleasant,
you might hurt yourself, and it leaves you covered in grime. The picture
they paint of their protagonist --- Mr. Trump --- is so outrageous that
if they were fiction they would be dismissed as too broad, too much of a
caricature.

As different as the authors are, the books share a number of common
observations about the president. And so, with the Republican Party set
to renominate him this month, here is a reminder of what sort of leader
Mr. Trump has turned out to be, according to his growing band of
disgruntled former employees.

\hypertarget{trump-vs-his-employees}{%
\subsection{Trump vs. his employees}\label{trump-vs-his-employees}}

Mr. Trump is universally presented in the memoirs as a flamboyantly mean
and intemperately indiscreet boss, wrong-footing and humiliating cabinet
members and aides with constant criticism, sometimes to their faces,
sometimes behind their backs.

The president dismisses Jim Mattis, his first secretary of defense as
``a liberal Democrat,'' yells at him in meetings and notes that ``I
never really liked him.'' (``I felt sorry for Mattis, not to mention the
country as a whole,'' Mr. Bolton writes.)

He derides Kirstjen Nielsen, his second secretary of homeland security
as ineffectual and ``not mentally able'' to handle her job and then, in
a fit of pique, futilely attempts to reassign her responsibilities first
to Jared Kushner, his son-in-law, and then to Mr. Bolton.

He muses aloud on multiple occasions about dumping Vice President Mike
Pence from the ticket in 2020 and replacing him with Nikki Haley, the UN
ambassador. ``Did we make a mistake with Gina?'' he asks, referring to
his decision to make Gina Haspel director of the C.I.A.

``Rex was terrible,'' he says about Rex Tillerson, his original
secretary of state. ``What good is he?'' he asks rhetorically about
Steven Mnuchin, his treasury secretary. ``I thought we had the right guy
at Treasury. But now I don't know.''

He yells at his trade adviser, Peter Navarro, when Mr. Navarro attempts
to show him a complicated chart outlining a policy point. (``I have no
idea what I'm even looking at,'' the president snaps.) He tells Mr.
Kushner in meetings: ``Jared, you don't know what you're talking
about.'' He mocks his original chief economic adviser, Gary Cohn, as a
``globalist,'' elongating the ``O'' in a sneering tone, as if the word
were akin to ``Antifa member.''

Just as the president uses derisive nicknames for his political enemies,
so he does for his own subordinates. He mocks Jeff Sessions his first
attorney general, as ``Benjamin Button.'' He calls Betsy DeVos, the
education secretary, ``Ditzy DeVos.'' ``This place is really taking a
toll on Kellyanne,'' he says of Kellyanne Conway, counselor to the
president, implying that she looks tired and worn out.

\hypertarget{the-president-and-the-truth}{%
\subsection{The president and the
truth}\label{the-president-and-the-truth}}

The Trump administration surged into life with a whomping great Trumpian
untruth: that Mr. Trump's inauguration crowd was the largest in history.
Even Mr. Spicer did not believe it, though he had to pretend otherwise.

``It was hard to keep a straight face as Sean proceeded to lie to the
American people,'' Ms. Manigault Newman writes.

All the memoirists present Mr. Trump as supremely untrustworthy. He is
``a deliberate liar, someone who will say whatever he pleased to get
whatever he wishes,'' Mr. McCabe writes. ``People who've known him for
years accept it as common knowledge,'' Anonymous writes.

Sometimes Mr. Trump asserts one thing and then, a few minutes later,
just the opposite.

On other occasions, he conjures pieces of misinformation designed to
bolster his thesis, as when he insisted that ``three to five million
people'' voted illegally in the 2016 election. He has a habit of
plucking figures from thin air --- first \$20 billion, for instance,
then \$38 billion, to drive home his point about trade deficits in a
meeting with President Moon Jae-in of South Korea --- regardless of the
numbers' relationship to fact.

The memoirists have different ways of dealing with all this presidential
slipperiness. Mr. Comey and Mr. McCabe start keeping detailed logs of
their encounters with the president, the way you would if you had an
unstable spouse and wanted to catalog his erratic behavior for use in
future divorce proceedings.

Too bad, is the apparent view of Reince Priebus, the original chief of
staff.

``The directive came down from Reince,'' Ms. Manigault Newman writes,
``that our default position was to back up whatever the president said
or tweeted, regardless of its accuracy.''

\hypertarget{how-to-describe-the-experience}{%
\subsection{How to describe the
experience}\label{how-to-describe-the-experience}}

Striving for new ways to characterize the head-spinning unreality of the
Trump White House, the authors of the memoirs turn to a variety of vivid
figures of speech.

\textbf{Mr. Spicer:} ``I sometimes felt like a scuba diver, abandoned in
the middle of the ocean, treading water.''

\textbf{Mr. Comey:} ``The demand was like Sammy the Bull's Cosa Nostra
induction ceremony --- with Trump, in the role of the family boss,
asking me if I have what it takes to be a `made man.'''

\textbf{Ms. Manigault Newman:} ``The selection process for his cabinet
was like an episode of `The Bachelor.'''

\textbf{Mr. Bolton:} ``It was like making and executing policy inside a
pinball machine.''

\textbf{Anonymous:} Working for Mr. Trump was like ``showing up at the
nursing home at daybreak to find your elderly uncle running pantsless
across the courtyard and cursing loudly about the cafeteria food.''

\hypertarget{trump-as-instigator}{%
\subsection{Trump as instigator}\label{trump-as-instigator}}

To read these books is to read of a chaotic, paranoiac workplace, where
the boss delights in fomenting discord and instability among the
employees.

He encourages them to keep tabs on one another. ``Give me their names,''
he tells Mr. Sims, wielding a Sharpie and a White House note card,
vowing to rid the White House of nonloyalists.

He praises their rivals. ``Keith Kellogg knows all about NATO,'' the
president says airily to Mr. Bolton, speaking with ominous intent of Mr.
Pence's national security adviser. ``He never offers his opinions unless
I ask.''

(This causes some merriment between Mr. Bolton and Mike Pompeo, the
current secretary of state, who plays dual roles in Mr. Bolton's drama:
as a rival he suspects of conspiring and leaking against him, and as his
partner in anti-Trump incredulity and black-humored job insecurity.

``As Pompeo and I reflected later, this statement told us exactly who my
likely replacement would be if I resigned soon,'' Mr. Bolton writes. ``I
said, `Of course, if you resign, maybe Keith would be Secretary of
State.''' To which Mr. Pompeo responds: ``Or, if we both resign, Keith
could become Henry Kissinger and have both jobs.''')

\hypertarget{the-presidents-verbal-style}{%
\subsection{The president's verbal
style}\label{the-presidents-verbal-style}}

Mr. Trump likes to talk, the memoirists agree, and he does not like to
listen.

He meanders from topic to topic, loops back around, adds new topics,
repeats himself, boasts, mixes facts with fake facts, throws in his
latest obsession, continuing on and on according to some labyrinthine
stream-of-consciousness impulse in which whatever is on his mind is
worthy of public utterance. He does this in rallies and at campaign
events; he also does it in briefings, in one-on-one conversations and at
policy meetings.

``I don't use the word `conversation' because the term doesn't apply
when one person speaks nearly the entire time,'' Mr. Comey writes of the
experience.

\hypertarget{the-presidential-attention-span}{%
\subsection{The presidential attention
span}\label{the-presidential-attention-span}}

It is true that Mr. Trump successfully repeated the words
``\href{https://www.nytimes.com/2020/07/23/us/politics/person-woman-man-camera-tv-trump.html}{person,
woman, man, camera, TV}'' on television in an effort to demonstrate the
superiority of his mental acuity, but it is also true, the books argue,
that he rarely reads, gets bored easily, is irritable and distracted,
has trouble remembering complicated things, has no intellectual
curiosity and is ignorant not just about his job but about things
generally considered common knowledge.

With his short attention span, he is averse to learning anything at
briefings if he finds the information difficult to follow, boring, or in
contravention of what he already thinks. Staff members are told to to
stick to a single point and repeat it often, and to boil complex
proposals down to a single page --- or, better, a single paragraph. They
are told not to present Mr. Trump with too-long briefing papers, lest he
shout at them, or with too many slides, lest his eyes glaze over.

``Any time somebody new came in to brief him, he'd get angry and say,
``Who's that guy? What's he want?'' Ms. Manigault Newman writes.

\hypertarget{the-presidential-schedule}{%
\subsection{The presidential schedule}\label{the-presidential-schedule}}

The president keeps unconventional office hours, is often late to
meetings and events and watches a lot of TV.

``At 9:35 I called Trump, who was as usual still in the residence,'' Mr.
Bolton writes.

``He often doesn't start the day in the Oval Office until 10 or 11
a.m.,'' Anonymous writes. He is ``channel-surfing his way through the
presidency.''

``His official schedule was more of a loose outline than a strict
regimen,'' Mr. Sims writes.

\hypertarget{the-presidential-ego}{%
\subsection{The presidential ego}\label{the-presidential-ego}}

In ``Too Much and Never Enough,'' Mary Trump describes her uncle as ``a
savant of self-promotion'' with a ``delusional belief in his own
brilliance and superiority'' stemming from a bottomless insecurity that
needs to be assuaged with a constant stream of ego-boosting compliments.

That is why the president often asserts that he is the best at
everything.

``It was the most presidential act in decades,'' he says, after he
directs the Pentagon to bomb Iran and then calls it off at the last
minute. (Mr. Bolton has a different take: ``In my government experience,
this was the most irrational thing I ever witnessed any president do.''
)

``They say I might be the world's greatest brander,'' he says to Mr.
Sims, before unveiling his marketing idea for his tax-cut plan: calling
the legislation the ``Cutting Cutting Cutting Bill'' (it ended up being
called something else).

Several memoirists describe how Mr. Trump, to soothe a wounded psyche
bruised by his failure to win the popular vote in 2016, continually
invited visitors to admire posters illustrating how he had won the
election anyway.

``Trump kept big charts in his private dining room, in his den, in his
study, that showed the electoral map color coded in red and blue,'' Ms.
Manigault Newman writes. ``When anyone walked in, he'd point to the
chart and talk about the election results.''

Anonymous was familiar with the maps, as well. ``Trump carried around
maps outlining his electoral victory, which he would pull out at odd
times,'' he writes. ``He would beckon guests, as well as aides, advisers
and incoming cabinet officers, to gaze at the sea of red on the map.''

\hypertarget{does-mr-trump-use-a-tanning-bed}{%
\subsection{Does Mr. Trump use a tanning
bed?}\label{does-mr-trump-use-a-tanning-bed}}

``His face appeared slightly orange, with bright white half-moons under
his eyes where I assumed he placed small tanning goggles,'' Mr. Comey
writes.

Ms. Manigault Newman mentions the tanning-adjacent chatter around the
\href{https://www.washingtonpost.com/news/post-politics/wp/2017/05/05/white-house-fires-its-chief-usher-the-first-woman-in-that-job/}{abrupt
firing} of Angella Reid, chief usher of the White House, several months
into Mr. Trump's administration.

``Allegedly, Trump didn't approve of her handling of his tanning bed,''
she says. ``I'd heard he was unhappy with her efforts to procure the
bed, to bring it into the East Wing securely, to find a discreet place
for it, and to set it up properly.''

\hypertarget{aides-on-the-presidents-conduct}{%
\subsection{Aides on the president's
conduct}\label{aides-on-the-presidents-conduct}}

The memoirs paint a picture of the West Wing as a place of baroque
workplace dysfunction, where workers gather to trade ``Guess what he did
now'' stories about their boss and to save him (and themselves, and the
country) from his worst impulses.

And so, in ``A Warning,'' Anonymous writes that cabinet-level
administration officials contemplated ``a midnight self-massacre,''
which would entail ``resigning en masse to call attention to Trump's
misconduct and erratic leadership.''

Many staffers are perpetually on the brink of quitting, keeping
resignation letters on hand should the time come. And if his colleagues
hate working for the Trump administration, John Kelly, the president's
second chief of staff, apparently hates it the most.

``This is the worst'' (insert expletive here) ``job I've ever had,'' Mr.
Kelly tells Mr. Sims.

``You can't imagine how desperate I am to get out of here,'' he tells
Mr. Bolton. ``This is a very bad place to work.''

\hypertarget{things-john-bolton-claims-people-said-to-him}{%
\subsection{Things John Bolton claims people said to
him}\label{things-john-bolton-claims-people-said-to-him}}

A striking aspect of ``The Room Where it Happened'' is how frequently
cabinet-level officials confide incredulously in Mr. Bolton about the
president's irrationality and narcissism, as if they and the former
national security adviser formed a gang of rebellious high-school
students, quietly plotting resistance against the incompetent autocrats
running the school.

``As McGahn often whispered to me,'' Mr. Bolton writes, speaking of
Donald McGahn, who served for a while as White House counsel, ``This is
not the Bush Administration.''

``Has there ever been a presidency like this?'' Mr. Kelly asks Mr.
Bolton, mentioning that the president has just said, apropos of nothing,
that it would be ``cool'' to invade Venezuela. (``I assured him there
had not,'' Mr. Bolton responds.)

``This is getting pretty silly,'' Mr. Mattis says to Mr. Bolton as the
men listen to Mr. Trump rail at Jens Stoltenberg, the secretary general
of NATO, about how America's allies mock it behind its back because it
pays too much in annual dues.

As for Mr. Pompeo, Mr. Bolton describes how the secretary of state
passed him a snarky anti-Trump note in the middle of the president's
summit with Kim Jong-un ****** of North Korea in 2018. And he describes
how, after listening to the president yell at Ms. Nielsen about border
security in a particularly fruitless meeting, Mr. Pompeo whispers to Mr.
Bolton: ``Why are we still here?''

\hypertarget{leaving-trumps-orbit}{%
\subsection{Leaving Trump's orbit}\label{leaving-trumps-orbit}}

``What Donald can do in order to offset the powerlessness and rage he
feels is punish the rest of us,'' Mary Trump writes.

This is clear by the way he behaves when he has fired someone or they
have quit, frequent occurrences in an administration with such a high
turnover. After he fires Mr. Comey, the FBI director, while he is in
California, for instance, Mr. Trump is incensed to learn that Mr. Comey
has returned to Washington on the same government plane that he traveled
out on.

``That's not right! I didn't approve of that!'' he rants to Mr. McCabe.
Then he decrees that Mr. Comey should never be allowed to enter the
F.B.I. headquarters again, not even to clean out his desk. ``I'm banning
him from the building,'' the president says.

After Mr. Mattis resigns as defense secretary, Anonymous writes, the
wounded president throws ``a temper tantrum,'' insists that Mr. Mattis
leave the job immediately, before his successor has been named, and then
falsely claims that in fact he fired Mr. Mattis, rather than the other
way around.

\hypertarget{how-the-trump-administration-said-youre-fired}{%
\subsection{How the Trump administration said `you're
fired'}\label{how-the-trump-administration-said-youre-fired}}

\textbf{Mr. Comey:} Saw the news reported on TV in the back of the
auditorium while he was in the middle of making a speech in California.

\textbf{Mr. McCabe:} Saw the news on TV, followed by a presidential
tweet: ``Andrew McCabe FIRED, a great day for the hard working men and
women of the FBI.''

\textbf{Ms. Manigault Newman:} Called into the Situation Room before the
2017 White House Christmas Party, was informed by Mr. Kelly that ``there
are significant integrity violations related to you,'' and was not
allowed to leave until the stress of the encounter triggered an asthma
attack and she went home.

\textbf{Mr. Sims:} Was fired by Mr. Kelly, who said: ``In the past 40
years, I don't think I've ever had a subordinate whose reputation is
worse than yours.''

\textbf{Mr. Priebus:} Idling in the presidential motorcade after a trip
to New York on the day after he had submitted his resignation, learned
that his removal was effective immediately when he read on Twitter that
Mr. Kelly was replacing him. The motorcade went on to the White House;
his car peeled away and drove off into oblivion.

\hypertarget{trump-on-the-authors}{%
\subsection{Trump on the authors}\label{trump-on-the-authors}}

\textbf{Mr. Comey:}
\href{https://www.businessinsider.com/trump-tweets-james-comey-book-higher-loyalty-slime-ball-2018-4}{``A
weak and untruthful slime ball''}

\textbf{Mr. McCabe:}
\href{https://twitter.com/realdonaldtrump/status/1150011125347627009?lang=en}{``A
major sleazebag.''}

\textbf{Ms. Manigault Newman:}
\href{https://www.cnbc.com/2018/08/13/trump-launches-twitter-attack-on-omarosa-she-was-vicious-but-not-smart.html}{``Vicious,
but not smart.''}

\textbf{Mary Trump:}
\href{https://twitter.com/realdonaldtrump/status/1284255473424891905}{``A
seldom seen niece who knows little about me, says untruthful things
about my wonderful parents (who couldn't stand her!) and me, and
violated her NDA. \ldots{} She's a mess!''}

\textbf{Mr. Sims:}
\href{https://twitter.com/realdonaldtrump/status/1090244651578204160?lang=en}{``A
low level staffer that I hardly knew. \ldots{} He is a mess!''}

\textbf{Mr. Bolton:} A
\href{https://twitter.com/realdonaldtrump/status/1273603410340843520}{``sick
puppy.''}

Advertisement

\protect\hyperlink{after-bottom}{Continue reading the main story}

\hypertarget{site-index}{%
\subsection{Site Index}\label{site-index}}

\hypertarget{site-information-navigation}{%
\subsection{Site Information
Navigation}\label{site-information-navigation}}

\begin{itemize}
\tightlist
\item
  \href{https://help.nytimes.com/hc/en-us/articles/115014792127-Copyright-notice}{©~2020~The
  New York Times Company}
\end{itemize}

\begin{itemize}
\tightlist
\item
  \href{https://www.nytco.com/}{NYTCo}
\item
  \href{https://help.nytimes.com/hc/en-us/articles/115015385887-Contact-Us}{Contact
  Us}
\item
  \href{https://www.nytco.com/careers/}{Work with us}
\item
  \href{https://nytmediakit.com/}{Advertise}
\item
  \href{http://www.tbrandstudio.com/}{T Brand Studio}
\item
  \href{https://www.nytimes.com/privacy/cookie-policy\#how-do-i-manage-trackers}{Your
  Ad Choices}
\item
  \href{https://www.nytimes.com/privacy}{Privacy}
\item
  \href{https://help.nytimes.com/hc/en-us/articles/115014893428-Terms-of-service}{Terms
  of Service}
\item
  \href{https://help.nytimes.com/hc/en-us/articles/115014893968-Terms-of-sale}{Terms
  of Sale}
\item
  \href{https://spiderbites.nytimes.com}{Site Map}
\item
  \href{https://help.nytimes.com/hc/en-us}{Help}
\item
  \href{https://www.nytimes.com/subscription?campaignId=37WXW}{Subscriptions}
\end{itemize}
