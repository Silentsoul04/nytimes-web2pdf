Sections

SEARCH

\protect\hyperlink{site-content}{Skip to
content}\protect\hyperlink{site-index}{Skip to site index}

\href{https://www.nytimes.com/section/politics}{Politics}

\href{https://myaccount.nytimes.com/auth/login?response_type=cookie\&client_id=vi}{}

\href{https://www.nytimes.com/section/todayspaper}{Today's Paper}

\href{/section/politics}{Politics}\textbar{}Alienated by Trump, Suburban
Voters Sour on G.O.P. in Battle for the House

\url{https://nyti.ms/2EFW2e2}

\begin{itemize}
\item
\item
\item
\item
\item
\end{itemize}

Advertisement

\protect\hyperlink{after-top}{Continue reading the main story}

Supported by

\protect\hyperlink{after-sponsor}{Continue reading the main story}

\hypertarget{alienated-by-trump-suburban-voters-sour-on-gop-in-battle-for-the-house}{%
\section{Alienated by Trump, Suburban Voters Sour on G.O.P. in Battle
for the
House}\label{alienated-by-trump-suburban-voters-sour-on-gop-in-battle-for-the-house}}

House Republicans are on the defensive in suburban strongholds as voters
reject President Trump's handling of the coronavirus.

\includegraphics{https://static01.nyt.com/images/2020/07/30/us/politics/01dc-repubs1/merlin_174667644_fff3f748-bcbc-42a8-a554-bbb8b804abef-articleLarge.jpg?quality=75\&auto=webp\&disable=upscale}

\href{https://www.nytimes.com/by/emily-cochrane}{\includegraphics{https://static01.nyt.com/images/2018/11/28/multimedia/author-emily-cochrane/author-emily-cochrane-thumbLarge-v3.png}}\href{https://www.nytimes.com/by/catie-edmondson}{\includegraphics{https://static01.nyt.com/images/2019/11/20/us/politics/catie-edmonson-twitter-chatblog/catie-edmonson-twitter-chatblog-thumbLarge.png}}

By \href{https://www.nytimes.com/by/emily-cochrane}{Emily Cochrane} and
\href{https://www.nytimes.com/by/catie-edmondson}{Catie Edmondson}

\begin{itemize}
\item
  Aug. 1, 2020
\item
  \begin{itemize}
  \item
  \item
  \item
  \item
  \item
  \end{itemize}
\end{itemize}

BALLWIN, Mo. --- For Heather Vaughn, a substitute teacher and graduate
student, the decision last month to place the black sign with colorful
lettering in her front yard --- the one that said, ``Black Lives
Matter'' and ``Science is Real'' --- felt like an act of courage.

In previous years, such a placard might have drawn unwanted attention in
her suburban, tree-lined neighborhood, where expansive homes with
manicured gardens had been decked out with blue ribbons and signs of
support for the police. But now it is one of three on her block that
reflect support for nationwide protests against police brutality and a
growing sense of unease with President Trump's handling of the
coronavirus.

A self-described independent, Ms. Vaughn, 41, had supported
Representative Ann Wagner, her Republican congresswoman, in past years,
but more recently soured on her. This year, given her frustration and
anger with Mr. Trump, Ms. Vaughn is confident she will not vote for Ms.
Wagner and is wrestling with whether she in good conscience can vote
again for any of the local Republicans down the ballot whom she would
normally back.

``That is an issue that we've had my entire life and we still haven't
solved,'' she said of the systemic racism that drove recent protests
around the country, much as it did
\href{https://www.nytimes.com/interactive/2014/08/13/us/ferguson-missouri-town-under-siege-after-police-shooting.html}{in
2014 in nearby Ferguson, Mo}. ``It's just going to get swept under the
rug again unless we do something significant at the polls in November.''

Suburban districts like these have long been critical bases of
Republican support, packed with affluent white voters who reliably chose
Republicans to represent them in Congress. Democrats
\href{https://www.nytimes.com/2018/11/06/us/politics/midterm-elections-results.html}{seized
control of the House in 2018} by making inroads in communities like
these, and Republicans have tied their hopes of reclaiming power to
preserving their remaining footholds there. But as Mr. Trump continues
to stumble in his response to the pandemic and seeks to
\href{https://www.nytimes.com/2020/07/29/us/politics/trump-suburbs-housing-white-voters.html}{stir
up racist fears} with pledges to preserve the ``Suburban Lifestyle
Dream,'' such districts are
\href{https://www.nytimes.com/2020/07/10/us/politics/trump-white-voters-in-suburbs.html}{slipping
further from the party's grasp}, and threatening to drag down
congressional Republicans in November's elections.

Interviews with more than two dozen party officials, strategists and
voters in areas like these help explain what recent polls have found:
that Mr. Trump's strategy is alienating independent and even some
conservative voters --- particularly women and better-educated Americans
--- who are turned off by his partisan appeals and disappointed in his
leadership. From the suburbs of St. Louis to Omaha to Houston, they
expressed deep concern about Mr. Trump's approach to twin national
crises, lamenting his confident declarations that the coronavirus was
under control and his move to stoke racial divides after nationwide
protests over police brutality against Black Americans.

One result is that House Republicans, who began the election cycle
hoping to win an uphill battle to recapture their majority --- or at the
very least, claw back some of the competitive districts they lost to
Democrats in 2018 --- are instead scrambling to shore up seats that once
would have required little effort to hold. Analysts at the nonpartisan
Cook Political Report
\href{https://cookpolitical.com/analysis/national/national-politics/new-july-2020-electoral-college-ratings}{recently
forecast} that November could bring ``a Democratic tsunami,'' and
\href{https://twitter.com/Redistrict/status/1284116710866538496?s=20}{placed
once safe Republican incumbents} on an ``anti-Trump wave watch list.''

``We feel that we're not only going to hold the House, we are going to
grow the majority that we have,'' Representative Cheri Bustos of
Illinois, the chairwoman of House Democrats' campaign arm, said in an
interview. ``With each passing month, that number of seats that we think
that we can gain continues to grow.

Michael McAdams, a spokesman for House Republicans' campaign arm,
contended in a statement that incumbent Republicans in those districts
would be able to rise above the national trends and noted, ``Voters
don't cast their ballots in July.''

Republicans ``can and will win no matter the environment, and this cycle
will be no different,'' Mr. McAdams said, ``especially with President
Trump turning out Republican voters who stayed home in 2018 and
Democrats embracing an extreme agenda.''

\includegraphics{https://static01.nyt.com/images/2020/07/30/us/politics/00dc-repubs2/merlin_173290452_acca4a88-7f6b-4d77-961e-37a506044109-articleLarge.jpg?quality=75\&auto=webp\&disable=upscale}

Still, some voters who in the past supported Mr. Trump and his
Republican allies said they had seen subtle shifts in recent months
among their co-workers and friends. People who had once shied away from
any political commentary were now openly criticizing the failures in the
pandemic response, they said, or displaying ``Black Lives Matter'' signs
and posters outside their homes.

Ms. Wagner, who last year began the House Suburban Caucus in part to be
``a voice for the growing suburban areas in our country,'' is among
those facing a more difficult than expected re-election race.

In the suburbs of Douglas County in Nebraska, Derek Oden, 23, the
executive director of the local Republican Party, said he was working
feverishly to expand his party's outreach, acknowledging that the
national rhetoric fueled in part by Mr. Trump's inflammatory language
``definitely convolutes things.'' Representative Don Bacon, Republican
of Nebraska, has recently begun to distance himself from the president,
openly breaking with him by
\href{https://www.nytimes.com/2020/07/20/us/politics/congress-trump-confederate-base-names.html}{leading
the charge} to remove the names of Confederate figures from military
bases, a move that Mr. Trump has condemned.

``I think they're leaning away from what used to be Republican standards
--- instead of leading the culture, they're letting the culture lead
them,'' Nora Haury, 87, said of Republicans in an interview outside her
home in Omaha. ``I feel a bit discouraged,'' she added, though she said
her concerns about how much the Democrats were influenced by their
party's left flank would keep her voting red come November.

Mr. Bacon will again face Kara Eastman, a progressive activist and
nonprofit organizer, after defeating her by two points in 2018. Armed
with fliers and an arsenal of pork-related puns, the congressman spent
one recent afternoon knocking on doors in blistering heat, trying to
persuade moderate and independent voters that he deserved their votes.

Cheerfully reminding those who answered the door that their votes could
make a difference, he made little unsolicited mention of the president,
responding to entreaties to make the pandemic go away with reassurances
about the promising, yet early, success of
\href{https://www.nytimes.com/2020/07/14/health/cornavirus-vaccine-moderna.html}{a
vaccine trial} and pointing to the \$2.2 trillion stimulus law that
Congress
\href{https://www.nytimes.com/2020/03/27/us/politics/coronavirus-house-voting.html}{approved
in March}.

Image

Mr. Bacon knocking on doors in blistering heat, hoping to persuade
moderate and independent voters that he deserved their
vote.Credit...Terry Ratzlaff for The New York Times

``I can just control my message and control my work ethic,'' Mr. Bacon
said, adding that he believed Ms. Eastman's support for ``Medicare for
all'' and other progressive proposals would repel independent voters.
``Trump will be a factor in this discussion, and I don't know where it
will be in four months, so I can't worry about that.''

In Texas, where Democrats are targeting five seats that once were
Republican strongholds explicitly gerrymandered to capture large
sections of the suburbs, some steadfast conservative voters are now
preparing to cast their first votes for Democratic congressional
candidates, infuriated by the administration's handling of the pandemic.

Cass Mattison and his wife, Samantha Mattison, who live in Sugar Land,
just southwest of Houston, say they usually vote Republican, but they
both plan to vote for Sri Kulkarni, a Democratic former Foreign Service
officer running to replace Representative Pete Olson, a Republican who
is retiring. They cite their party's ``very poor handling'' of the
pandemic ``from top to bottom.''

Ms. Mattison, who runs an in-home day care, said that she was
particularly infuriated by how long Mr. Trump had waited
\href{https://www.nytimes.com/interactive/2020/us/texas-coronavirus-cases.html}{to
take the virus seriously}, and upset that he
\href{https://www.nytimes.com/2020/07/01/us/coronavirus-masks.html}{refused
for so long to wear a mask}.

``The lack of accountability kills me,'' Ms. Mattison said.

Image

A coronavirus testing site in Houston, where cases of known infections
surged this summer.Credit...Callaghan O'Hare for The New York Times

As she and her husband watched hospitalizations skyrocket in Houston,
they turned their attention to the election, and began to research the
two Republican candidates in their district vying to succeed Mr. Olson,
only to be disappointed.

``Houston was just out of control, and not one of those candidates
talked about what we're going to do about Covid,'' Mr. Mattison, an
engineer and Army veteran, said in a phone interview.

Farha Ahmed, a lawyer in Sugar Land, said she has consistently voted
Republican for the past 30 years and previously served as general
counsel for her county's local Republican Party. She plans to support
Mr. Kulkarni in November.

``I don't see a lot of leadership" from Republicans, she said in an
interview. ``The megaphone is really with the president and that is what
has translated to all the Texas Republican leaders. It makes it very
difficult for them to carry out what they need to do for health and
safety reasons.''

In Houston's northern suburbs, Representative Michael McCaul, the top
Republican on the Foreign Affairs Committee who won re-election in 2018
by five points, is facing a rematch from Mike Siegel, a progressive
civil rights lawyer. Republican strategists say that Mr. McCaul's
campaign this cycle is far stronger, but privately acknowledge Mr.
McCaul could fall if an exceptionally strong Democratic wave sweeps
across the country.

They are worried about voters like Wade Miller, 51, in Cypress. Mr.
Miller, in an interview, described himself as a longtime Republican, but
said he was reluctant to support Republicans in the coming election,
citing their response to the pandemic. He and his wife had stopped
watching national television news because listening to the president's
talk ``made us angry for a little bit there,'' he said.

``I have always been a mostly straight-ticket voter --- I don't think I
will be this coming election,'' Mr. Miller said. ``We're talking about
human lives here, and if people aren't willing to do what it takes to
save lives, what else aren't they willing to do? I will definitely be
changing my vote come November.''

Advertisement

\protect\hyperlink{after-bottom}{Continue reading the main story}

\hypertarget{site-index}{%
\subsection{Site Index}\label{site-index}}

\hypertarget{site-information-navigation}{%
\subsection{Site Information
Navigation}\label{site-information-navigation}}

\begin{itemize}
\tightlist
\item
  \href{https://help.nytimes.com/hc/en-us/articles/115014792127-Copyright-notice}{©~2020~The
  New York Times Company}
\end{itemize}

\begin{itemize}
\tightlist
\item
  \href{https://www.nytco.com/}{NYTCo}
\item
  \href{https://help.nytimes.com/hc/en-us/articles/115015385887-Contact-Us}{Contact
  Us}
\item
  \href{https://www.nytco.com/careers/}{Work with us}
\item
  \href{https://nytmediakit.com/}{Advertise}
\item
  \href{http://www.tbrandstudio.com/}{T Brand Studio}
\item
  \href{https://www.nytimes.com/privacy/cookie-policy\#how-do-i-manage-trackers}{Your
  Ad Choices}
\item
  \href{https://www.nytimes.com/privacy}{Privacy}
\item
  \href{https://help.nytimes.com/hc/en-us/articles/115014893428-Terms-of-service}{Terms
  of Service}
\item
  \href{https://help.nytimes.com/hc/en-us/articles/115014893968-Terms-of-sale}{Terms
  of Sale}
\item
  \href{https://spiderbites.nytimes.com}{Site Map}
\item
  \href{https://help.nytimes.com/hc/en-us}{Help}
\item
  \href{https://www.nytimes.com/subscription?campaignId=37WXW}{Subscriptions}
\end{itemize}
