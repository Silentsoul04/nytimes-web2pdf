Sections

SEARCH

\protect\hyperlink{site-content}{Skip to
content}\protect\hyperlink{site-index}{Skip to site index}

\href{https://www.nytimes.com/section/world/americas}{Americas}

\href{https://myaccount.nytimes.com/auth/login?response_type=cookie\&client_id=vi}{}

\href{https://www.nytimes.com/section/todayspaper}{Today's Paper}

\href{/section/world/americas}{Americas}\textbar{}Under Pressure,
Brazil's Bolsonaro Forced to Fight Deforestation

\url{https://nyti.ms/3k5rRxc}

\begin{itemize}
\item
\item
\item
\item
\item
\end{itemize}

Advertisement

\protect\hyperlink{after-top}{Continue reading the main story}

Supported by

\protect\hyperlink{after-sponsor}{Continue reading the main story}

\hypertarget{under-pressure-brazils-bolsonaro-forced-to-fight-deforestation}{%
\section{Under Pressure, Brazil's Bolsonaro Forced to Fight
Deforestation}\label{under-pressure-brazils-bolsonaro-forced-to-fight-deforestation}}

After fending off international criticism on rainforest destruction,
President Jair Bolsonaro caved to pressure and took steps to curb
deforestation and forest fires.

\includegraphics{https://static01.nyt.com/images/2020/08/02/world/02brazil-amazon/merlin_160086585_797d8284-14b9-4098-83bc-3d9e148ee8a1-articleLarge.jpg?quality=75\&auto=webp\&disable=upscale}

By \href{https://www.nytimes.com/by/ernesto-londono}{Ernesto Londoño}
and Letícia Casado

\begin{itemize}
\item
  Aug. 1, 2020
\item
  \begin{itemize}
  \item
  \item
  \item
  \item
  \item
  \end{itemize}
\end{itemize}

RIO DE JANEIRO --- A year ago, as fires engulfed the Amazon, President
Jair Bolsonaro of Brazil reacted to criticism from abroad with
indignation:
\href{https://www.nytimes.com/2019/07/28/world/americas/brazil-deforestation-amazon-bolsonaro.html}{``The
Amazon is ours,''} he said, arguing that the fate of the rainforest was
for his country to decide.

Much has changed in a year.

Under pressure from European governments, foreign investors and
Brazilian companies concerned about the country's reputation, Mr.
Bolsonaro has banned forest fires for the four months of the dry season
and set up a military operation against deforestation.

The new stance represents a notable turnaround by a government that has
drawn widespread global condemnation over its environmental policies.

Environmentalists, experts and foreign officials who have pressed Brazil
on conservation matters are skeptical of the government's commitment,
afraid these actions amount to little more than damage control at a time
when the economy is in deep trouble.

Mr. Bolsonaro and many of his political allies have long favored opening
the Amazon to miners, farmers and loggers, and his government has openly
worked to undermine the land rights of Indigenous communities.
Deforestation has spiked under his tenure.

But as the political and business costs of policies that prioritize
exploration over conservation escalate, some activists see an
opportunity to slow, or even reverse, that trend by promoting private
sector support for greener policies.

``Brazil is becoming an environmental pariah on the global stage,
destroying a positive reputation that took decades to build,'' said
Sueley Araújo, a veteran environmental policy expert who was dismissed
as the head of the country's main environmental protection agency soon
after Mr. Bolsonaro took office.

\includegraphics{https://static01.nyt.com/images/2020/07/30/world/00brazil-amazon2/merlin_164845956_f6770388-33c5-46b9-84e1-4473efa1428d-articleLarge.jpg?quality=75\&auto=webp\&disable=upscale}

Brazil's worsening reputation on the environment has also put in
jeopardy two important foreign policy goals: the implementation of a
\href{https://www.dw.com/en/austria-deals-first-blow-to-eu-mercosur-trade-pact/a-50489747}{trade
deal with the European Union} and its ambition to join the Organization
for Economic Cooperation and Development, a 37-country group. Both deals
require Brazil to meet baseline standards on labor and environmental
policies.

A striking sign of the potential economic damage to Brazil's interests
came in late June, when more than two dozen financial institutions that
collectively control some \$3.7 billion in assets warned the Brazilian
government in a letter that investors were steering away from countries
that are accelerating the degradation of ecosystems.

``We recognize the crucial role that tropical forests play in tackling
climate change, protecting biodiversity and ensuring ecosystem
services,'' the investors wrote.

This week, Nordea Asset Management, a major European investment firm,
announced it has dropped from its funds the Brazilian meat giant JBS SA
over the company's role in deforestation and other concerns,
\href{https://www.wsj.com/articles/nordea-asset-management-drops-jbs-over-deforestation-corruption-worker-health-11595963107}{according
to The Wall Street Journal.} JBS, one of the leading meat suppliers in
the world, has
\href{https://www.nytimes.com/2019/10/10/world/americas/amazon-fires-brazil-cattle.html}{come
under criticism} for failing to keep meat from cattle grazed in
protected lands out of its supply chains.

Norway's Minister of Climate and Environment, Sveinung Rotevatn, said
Brazil has managed in the past to rein in deforestation by protecting
Indigenous communities, shielding natural forests and aggressively
enforcing the law.

``Brazil was a world leader in dramatically reducing deforestation, and
showed the world that they could significantly increase agricultural
production at the same time,'' he said in an email. ``They can do so
again.''

The message has clearly registered within Brazil. The country's three
largest banks announced this week a joint effort to press for and fund
sustainable development projects in the Amazon.

And a group of former Brazilian finance ministers and central bank
presidents argued in a joint statement earlier this month that the best
way to jump-start the economy is by investing in greener technologies,
ending fuel subsidies and drastically reducing the deforestation rate.

Image

Burned forest near a cattle ranch in Mato Grosso.Credit...Victor
Moriyama for The New York Times

But the clearest sign of the shifting politics on the issue lies in the
fate of Ricardo Salles, Mr. Bolsonaro's environment minister, who is
fighting for his political survival amid criticism of Brazil's growing
deforestation.

Mr. Salles, the face of the Bolsonaro administration's efforts to weaken
environmental protections, was expelled from his party in May over his
leadership of the ministry. He is also facing a legal complaint from
federal prosecutors who are seeking his removal, arguing that Mr.
Salles' actions in office amounted to a dereliction of duty.

Brazilian leaders have often bristled at foreign-led campaigns to save
the rainforest, regarding such efforts as an underhanded way to hinder
the economic potential of the vast nation, which is a leading exporter
of food and other commodities.

Last July, Mr. Bolsonaro told a round table of international journalists
that the rate of deforestation in the Amazon should concern Brazil
alone.
\href{https://www.theguardian.com/world/2019/jul/19/jair-bolsonaro-brazil-amazon-rainforest-deforestation}{``The
Amazon is ours,''} he snapped.

The next month, in early August, Mr. Bolsonaro fired the head of the
government agency that tracks deforestation trends, and Mr. Salles
raised doubts about his own government's data, which showed a clear rise
in destruction of the forest.

Image

President Jair Bolsonaro of Brazil, left, and the country's environment
minister, Ricardo Salles. Ms. Salles is fighting efforts to bar him from
public office.Credit...Jose Cruz/Agencia Brasil, via Reuters

Later that month, world leaders, celebrities and people on social media
reacted with horror as photos and videos of an unusually intense fire
season in the Amazon went viral. Such fires are intentionally set in
July and August to clear land for cattle grazing and to plant crops, but
several last year, which was unusually dry, raged out of control.

Mr. Bolsonaro sparred with President Emmanuel Macron of France after the
European leader drew attention to the fires by asserting that ``our
house is burning. Literally.''

Since then, experts say, deforestation has continued to rise as the
government has hobbled its environmental protection agencies, allowing
illegal miners and loggers to go deeper into the Amazon with broad
impunity.

During the first six months of this year, loggers razed approximately
1,184 square miles of the Amazon, according to Brazil's National
Institute for Space Research. That area --- slightly smaller than the
state of Rhode Island --- is 25 percent larger than the forest cover
lost during the same time period in 2019.

Environmental experts say the military operation to curb deforestation,
which includes more than 3,600 troops and law enforcement agents, will
at best make a dent in deforestation and fire trends this year. To
fundamentally reverse them, they say the government would need to make
sweeping changes to bolster the staffing level, tools and political
backing of the environmental protection agencies.

Image

An illegal logging operation in Pará state last year.Credit...Victor
Moriyama for The New York Times

The association of government environmental protection agents and
federal prosecutors say Mr. Salles is largely responsible for the rise
in deforestation during the Bolsonaro administration.

On his watch, they asserted in separate statements issued recently,
career specialists have lost tools and autonomy. Career law enforcement
agents at the main environmental agencies were demoted or dismissed
earlier this year after operations against land invaders that drew a
political backlash.

Criticism of Mr. Salles reached a boiling point in May following the
release of a video recording of a cabinet meeting during which he said
the coronavirus pandemic had created an opportune distraction to make
headway on environmental deregulation without drawing much scrutiny from
the press.

In a
\href{http://www.mpf.mp.br/df/sala-de-imprensa/docs/aia-salles-1}{126-page
complaint} filed in early July, federal prosecutors accused Mr. Salles
of spending money inefficiently, retaliating against effective
enforcement agents and issuing the fewest fines for environmental crimes
in 20 years, even as invasion of protected lands surged.

``The destruction of the system of Brazil's environmental protection
system was the result of the acts, omissions and statements by the
accused,'' federal prosecutors
\href{http://www.mpf.mp.br/df/sala-de-imprensa/noticias-df/mpf-pede-afastamento-de-ricardo-salles-do-ministerio-do-meio-ambiente-por-improbidade-administrativa}{wrote
in their complaint}, which seeks to prevent Mr. Salles from occupying
public office.

Mr. Salles, who did not respond to a request for an interview, called
the allegations baseless and accused prosecutors of meddling in policies
of the executive branch.

Mr. Bolsonaro's office referred a request for comment to the office of
the vice president, Hamilton Mourão, which also did not respond.

Mr. Mourão, a former Army general and the head of the government's
recent military deployment to the Amazon, has billed the effort a sign
of the administration's commitment to reduce deforestation and other
environmental crimes.

``Rest assured that enforcement is continuing,'' Mr. Mourão
\href{https://www.gov.br/pt-br/noticias/meio-ambiente-e-clima/2020/07/em-programa-de-radio-mourao-fala-sobre-fiscalizacao-na-amazonia}{said
earlier this month} in remarks to a public radio station, ``and that it
is having good results.''

Image

Timber on a ferry crossing the Pacajá River, a popular transportation
route for legal and illegal logging, in Pará last year.Credit...Victor
Moriyama for The New York Times

Advertisement

\protect\hyperlink{after-bottom}{Continue reading the main story}

\hypertarget{site-index}{%
\subsection{Site Index}\label{site-index}}

\hypertarget{site-information-navigation}{%
\subsection{Site Information
Navigation}\label{site-information-navigation}}

\begin{itemize}
\tightlist
\item
  \href{https://help.nytimes.com/hc/en-us/articles/115014792127-Copyright-notice}{©~2020~The
  New York Times Company}
\end{itemize}

\begin{itemize}
\tightlist
\item
  \href{https://www.nytco.com/}{NYTCo}
\item
  \href{https://help.nytimes.com/hc/en-us/articles/115015385887-Contact-Us}{Contact
  Us}
\item
  \href{https://www.nytco.com/careers/}{Work with us}
\item
  \href{https://nytmediakit.com/}{Advertise}
\item
  \href{http://www.tbrandstudio.com/}{T Brand Studio}
\item
  \href{https://www.nytimes.com/privacy/cookie-policy\#how-do-i-manage-trackers}{Your
  Ad Choices}
\item
  \href{https://www.nytimes.com/privacy}{Privacy}
\item
  \href{https://help.nytimes.com/hc/en-us/articles/115014893428-Terms-of-service}{Terms
  of Service}
\item
  \href{https://help.nytimes.com/hc/en-us/articles/115014893968-Terms-of-sale}{Terms
  of Sale}
\item
  \href{https://spiderbites.nytimes.com}{Site Map}
\item
  \href{https://help.nytimes.com/hc/en-us}{Help}
\item
  \href{https://www.nytimes.com/subscription?campaignId=37WXW}{Subscriptions}
\end{itemize}
