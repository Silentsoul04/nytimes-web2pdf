Sections

SEARCH

\protect\hyperlink{site-content}{Skip to
content}\protect\hyperlink{site-index}{Skip to site index}

\href{https://www.nytimes.com/section/arts}{Arts}

\href{https://myaccount.nytimes.com/auth/login?response_type=cookie\&client_id=vi}{}

\href{https://www.nytimes.com/section/todayspaper}{Today's Paper}

\href{/section/arts}{Arts}\textbar{}The Calm Voice Asking Thorny
Questions in `Death, Sex \& Money'

\url{https://nyti.ms/31don2O}

\begin{itemize}
\item
\item
\item
\item
\item
\end{itemize}

\href{https://www.nytimes.com/spotlight/at-home?action=click\&pgtype=Article\&state=default\&region=TOP_BANNER\&context=at_home_menu}{At
Home}

\begin{itemize}
\tightlist
\item
  \href{https://www.nytimes.com/2020/08/03/well/family/the-benefits-of-talking-to-strangers.html?action=click\&pgtype=Article\&state=default\&region=TOP_BANNER\&context=at_home_menu}{Talk:
  To Strangers}
\item
  \href{https://www.nytimes.com/2020/08/01/at-home/coronavirus-make-pizza-on-a-grill.html?action=click\&pgtype=Article\&state=default\&region=TOP_BANNER\&context=at_home_menu}{Make:
  Grilled Pizza}
\item
  \href{https://www.nytimes.com/2020/07/31/arts/television/goldbergs-abc-stream.html?action=click\&pgtype=Article\&state=default\&region=TOP_BANNER\&context=at_home_menu}{Watch:
  'The Goldbergs'}
\item
  \href{https://www.nytimes.com/interactive/2020/at-home/even-more-reporters-editors-diaries-lists-recommendations.html?action=click\&pgtype=Article\&state=default\&region=TOP_BANNER\&context=at_home_menu}{Explore:
  Reporters' Google Docs}
\end{itemize}

Advertisement

\protect\hyperlink{after-top}{Continue reading the main story}

Supported by

\protect\hyperlink{after-sponsor}{Continue reading the main story}

\hypertarget{the-calm-voice-asking-thorny-questions-in-death-sex--money}{%
\section{The Calm Voice Asking Thorny Questions in `Death, Sex \&
Money'}\label{the-calm-voice-asking-thorny-questions-in-death-sex--money}}

In her podcast, Anna Sale inspires her guests to share their darkest
thoughts and deepest secrets. The intimate conversations feel more
urgent than ever.

\includegraphics{https://static01.nyt.com/images/2020/08/03/arts/00anna-sale1/merlin_174742218_9e2cdbb1-6ad7-4ce9-a241-9e02024a2048-articleLarge.jpg?quality=75\&auto=webp\&disable=upscale}

\href{https://www.nytimes.com/by/reggie-ugwu}{\includegraphics{https://static01.nyt.com/images/2018/06/13/multimedia/author-reggie-ugwu/author-reggie-ugwu-thumbLarge.jpg}}

By \href{https://www.nytimes.com/by/reggie-ugwu}{Reggie Ugwu}

\begin{itemize}
\item
  Published Aug. 2, 2020Updated Aug. 3, 2020
\item
  \begin{itemize}
  \item
  \item
  \item
  \item
  \item
  \end{itemize}
\end{itemize}

If you want to know what it feels like to be listened to, if, in our
moment of detachment and division, you've forgotten the basic pleasure
of revealing something delicate about yourself to another person, and of
having that person respond by taking a sincere and sustained interest,
allow Anna Sale to remind you.

I experienced it earlier this summer, when I made Sale pretend that I
was a guest on her acclaimed interview podcast,
``\href{https://www.wnycstudios.org/podcasts/deathsexmoney}{Death, Sex
\& Money}.'' With little more knowledge of her subject than could be
gleaned from an email signature and a few minutes of small talk, she
felt her way toward a line of questioning that left a lump in my throat
and a storm of memories flashing before my eyes.

\emph{What was the career arc that led you to The New York Times at this
moment? When did you feel like ``I'm uncertain if I can get paid writing
about the things that I love and think are important?'' Have there been
moments when it didn't feel like that was going to be possible? How did
you figure that out? Were there people in your life who were there to
support you?}

Listeners to Sale's show are familiar with questions like these,
questions that lock on to moments of unease, irresolution or tenderness
that we don't always put into words. Since she created ``Death, Sex \&
Money'' for WNYC in 2014, Sale has asked them weekly of both famous
people (Bill Withers, Jane Fonda) and nonfamous people, many of whom
send in letters and voice memos inspired by the show's tagline: ``The
things we think about a lot and need to talk about more.''

Guests have included a copywriter who
\href{https://www.wnycstudios.org/podcasts/deathsexmoney/episodes/seeking-arrangement-sugar-daddy-death-sex-money}{paid
her bills by working as a ``sugar baby,''} a woman who'd
\href{https://www.wnycstudios.org/podcasts/deathsexmoney/episodes/stillbirth-death-sex-money}{recently
given birth to a stillborn child} and a Black man in Chicago who was
\href{https://www.wnycstudios.org/podcasts/deathsexmoney/episodes/darrell-cannon-2020-death-sex-money}{tortured
by the police}, to name just a few.

In the era of Covid-19 and mandated social isolation, the show's
intimate conversations feel more urgent than ever. Several recent
episodes --- including a series of interviews with essential workers and
``\href{https://www.wnycstudios.org/podcasts/deathsexmoney/episodes/skin-hunger-love-radio-part-1}{Skin
Hunger,}'' a two-part collaboration with the podcast
``\href{https://loveandradio.org/}{Love + Radio}'' about the longing for
physical touch --- have confronted our pandemic reality explicitly.

But the show is perhaps most valuable as a long-running investigation
into interpersonal estrangement of all kinds. If no human experience
should be regarded as alien, to paraphrase the Roman playwright Terence,
then ``Death, Sex \& Money'' offers a fuller view of what being human
can mean.

Sale, 39, has straight, shoulder-length brown hair and the
enthusiastically analytical manner of a therapist at happy hour. In
March, she left her home in Berkeley, Calif., to shelter with her
husband, two young daughters and in-laws at her in-laws' house in Cody,
Wyo. During our video call, she sat on the floor of a closet that has
been serving as a temporary ``Death, Sex \& Money'' studio.

Sale grew up in Charleston, W.Va., the fourth of five daughters, with a
father who was an orthopedic surgeon and a mother who was a physical
therapist. Both of her parents were regular listeners of NPR, and Sale,
an observer born into a family of talkers, loved to listen to Terry
Gross while riding in the back seat. She moved away for college in 1999
--- she studied history at Stanford and worked at the Martin Luther King
Jr. Papers Project there --- but returned home after graduation without
a clear vision for her future.

``I had all of this energy and didn't know where to direct it,'' she
said.

\includegraphics{https://static01.nyt.com/images/2020/07/31/arts/00anna-sale2/merlin_174742251_972f9a5c-1227-4f9a-8ad1-3b394f914291-articleLarge.jpg?quality=75\&auto=webp\&disable=upscale}

One day, her aunt told her to close her eyes and imagine someone who
made her feel jealous. She pictured Gross. Soon after, she got her first
break in journalism, as a local politics reporter for West Virginia
Public Radio. She spent three years there, plus one in Connecticut,
before moving to New York, where she cold-called her way into a job at
WNYC.

From 2010 to 2013, Sale reported on politics for the WNYC show ``The
Takeaway.'' During the 2012 presidential election, she led a series of
candid, in-depth conversations with voters in swing states. She had
hoped they might provide a counterbalance to the red meat of political
rallies and professional pundits, but the stories struggled to penetrate
the din of the horse race.

While covering Anthony Weiner's second sexting scandal and ill-fated
mayoral bid in New York the following year, pangs of doubt about the
direction of her life returned. But not long after, she learned of an
internal WNYC contest soliciting ideas for its nascent podcast
operation. Sale, who was 33 and divorced at the time, realized that she
had one --- a show where people would be given room to talk about hard
things that had shaped their lives. One day, while walking the dog, she
heard herself say the words ``death, sex and money.''

The secret ingredient of the show is Sale's empathic persona. Nick van
der Kolk, the host and director of ``Love + Radio'' and co-producer of
``Skin Hunger,'' first noticed it in an early episode about
\href{https://www.wnycstudios.org/podcasts/deathsexmoney/episodes/sex-worker-next-door}{a
massage therapist} who also did sex work.

``Usually, when you hear a story like that, it becomes either a tragic
thing or the flip-side, which is like militantly sex-positive,'' he
said. ``But their discussion was incredibly nuanced. The woman was
completely honest about not liking the job, but also about how she
didn't feel like it was this horrendous thing that was destroying her
life.''

Often, as in an
\href{https://www.wnycstudios.org/podcasts/deathsexmoney/episodes/porn-death-sex-money}{episode
about pornography} featuring a man using the pseudonym Daniel, who
reported intrusive, upsetting thoughts during sex, Sale's forthright
questioning --- in a finely tuned, feather-soft voice --- elicits
equally forthright answers.

\begin{quote}
\textbf{SALE} Is it possible for you to have sex with your girlfriend
that doesn't feel hard?

\textbf{DANIEL} Sometimes, yeah. Is there ever a time when we have sex
that I don't have to talk to my brain? Where I don't have to use the
conscious part to talk to the unconscious part? No. But it doesn't mean
it's not good.

\textbf{SALE} So what's a sentence that you have to tell yourself?

\textbf{DANIEL} I'll be like, ``That's not real, that doesn't mean
anything, that's not what you really want, think about what you really
want.''
\end{quote}

``She's a master of the craft,'' said Stella Bugbee, editor in chief of
The Cut and a longtime fan of the show. ``You can hear the generosity in
her voice, and it's very genuine. But she doesn't beat around the bush
and she doesn't back away from pain.''

Sale, who said her experience covering politicians taught her to embrace
tough questions, doesn't work from a script during interviews. ``I'm
listening and editing at the same time that I'm interviewing,'' she
said. ``If someone is opening up to me about something, I keep chasing
the thread until I can picture it and it feels real to me. \emph{Where
were you? Who was there? What was that like?}''

Over the show's six years, listeners have come to trust it as a vessel
for their most vulnerable selves. That has placed a particular burden on
Sale and her team of producers.

When I asked Sale if she ever felt that the emotional toll was too much
to bear, she brought up
\href{https://www.wnycstudios.org/podcasts/deathsexmoney/episodes/stillbirth-death-sex-money}{the
episode} about the woman whose child had been stillborn. ``It was the
kind of loss that our society is so paralyzed about and unable to figure
out how to acknowledge,'' Sale said.

After conducting the interview, in which Sale, who had recently given
birth to her second daughter, asked the woman about deciding to hold the
child and what she planned to do with her milk, she took the rest of the
day off, called a close friend and went home to her family. Once the
episode had aired, she began to hear from listeners.

``There was a woman who donated 50 trees to be planted in the child's
name, a man in our building who said he'd never thought about this
subject before, and a woman who said that it had happened to her 25
years ago and it's still the most painful thing she's ever gone
through,'' she said. ``I was moved that we had been a place where people
could encounter that kind of experience and think about how it exists in
the world that they live in. It made me proud that we hadn't looked
away.''

Advertisement

\protect\hyperlink{after-bottom}{Continue reading the main story}

\hypertarget{site-index}{%
\subsection{Site Index}\label{site-index}}

\hypertarget{site-information-navigation}{%
\subsection{Site Information
Navigation}\label{site-information-navigation}}

\begin{itemize}
\tightlist
\item
  \href{https://help.nytimes.com/hc/en-us/articles/115014792127-Copyright-notice}{©~2020~The
  New York Times Company}
\end{itemize}

\begin{itemize}
\tightlist
\item
  \href{https://www.nytco.com/}{NYTCo}
\item
  \href{https://help.nytimes.com/hc/en-us/articles/115015385887-Contact-Us}{Contact
  Us}
\item
  \href{https://www.nytco.com/careers/}{Work with us}
\item
  \href{https://nytmediakit.com/}{Advertise}
\item
  \href{http://www.tbrandstudio.com/}{T Brand Studio}
\item
  \href{https://www.nytimes.com/privacy/cookie-policy\#how-do-i-manage-trackers}{Your
  Ad Choices}
\item
  \href{https://www.nytimes.com/privacy}{Privacy}
\item
  \href{https://help.nytimes.com/hc/en-us/articles/115014893428-Terms-of-service}{Terms
  of Service}
\item
  \href{https://help.nytimes.com/hc/en-us/articles/115014893968-Terms-of-sale}{Terms
  of Sale}
\item
  \href{https://spiderbites.nytimes.com}{Site Map}
\item
  \href{https://help.nytimes.com/hc/en-us}{Help}
\item
  \href{https://www.nytimes.com/subscription?campaignId=37WXW}{Subscriptions}
\end{itemize}
