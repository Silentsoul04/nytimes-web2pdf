Sections

SEARCH

\protect\hyperlink{site-content}{Skip to
content}\protect\hyperlink{site-index}{Skip to site index}

\href{https://www.nytimes.com/section/arts/music}{Music}

\href{https://myaccount.nytimes.com/auth/login?response_type=cookie\&client_id=vi}{}

\href{https://www.nytimes.com/section/todayspaper}{Today's Paper}

\href{/section/arts/music}{Music}\textbar{}Leon Fleisher, 92, Dies;
Spellbinding Pianist Using One Hand or Two

\url{https://nyti.ms/2BQplcD}

\begin{itemize}
\item
\item
\item
\item
\item
\end{itemize}

Advertisement

\protect\hyperlink{after-top}{Continue reading the main story}

Supported by

\protect\hyperlink{after-sponsor}{Continue reading the main story}

\hypertarget{leon-fleisher-92-dies-spellbinding-pianist-using-one-hand-or-two}{%
\section{Leon Fleisher, 92, Dies; Spellbinding Pianist Using One Hand or
Two}\label{leon-fleisher-92-dies-spellbinding-pianist-using-one-hand-or-two}}

Unable to use his right hand, he performed pieces written for left hand
only, conducted and taught. Years later, he made a triumphant two-handed
comeback.

\includegraphics{https://static01.nyt.com/images/2020/08/04/obituaries/00fleisher1/merlin_18504701_f19acbe9-202c-4815-bd7c-0e86390beb86-articleLarge.jpg?quality=75\&auto=webp\&disable=upscale}

By \href{https://www.nytimes.com/by/allan-kozinn}{Allan Kozinn}

\begin{itemize}
\item
  Aug. 2, 2020
\item
  \begin{itemize}
  \item
  \item
  \item
  \item
  \item
  \end{itemize}
\end{itemize}

Leon Fleisher, a leading American pianist in the 1950s and early '60s
who was forced by an injury to his right hand to channel his career into
conducting, teaching and mastering the left-hand repertoire, died on
Sunday in Baltimore. He was 92.

His death, in a hospice, was confirmed by his son Julian, who said that
Mr. Fleisher had been teaching and conducting master classes online as
recently as last week.

Mr. Fleisher came to believe that his career-altering malady, focal
dystonia, was caused by overpracticing --- ``seven or eight hours a day
of pumping ivory,'' as he told The New York Times in 1996 --- and for 30
years he tried virtually any cure that looked promising: shots of
lidocaine, rehabilitation therapy, psychotherapy, shock treatments,
Rolfing, EST. At times, he said, he was so despondent that he considered
suicide.

But he realized that the musicality and incisiveness that had been so
widely admired in his early years could be mined in other ways. Joining
the faculty of the Peabody Conservatory, in Baltimore, in 1959, he
devoted himself more fully to teaching, both there and at the Tanglewood
Music Center, where he was artistic director from 1986 to 1997.

He made his way through the estimable left-hand catalog of works
composed by Ravel, Prokofiev and many others for the pianist Paul
Wittgenstein (the brother of the philosopher Ludwig Wittgenstein), who
had lost his right arm during World War I, and commissioned new
left-hand works from American composers. He helped start the Theater
Chamber Players in Washington. And he began conducting.

\includegraphics{https://static01.nyt.com/images/2020/08/04/obituaries/Fleisher-2/merlin_9068922_c7b55747-d4eb-465b-bf57-f1a76b484386-articleLarge.jpg?quality=75\&auto=webp\&disable=upscale}

Eventually, a combination of Rolfing --- a deep massage technique ---
and Botox injections provided sufficient relief that he was able to
resume his career as a two-handed pianist in 1995. He continued to play
recitals and concertos and to make recordings until last year.

Mr. Fleisher pointed out after his comeback that he was not fully cured
and never would be. But he acknowledged late in life that the
incapacitation of his right hand in 1964 had given him a far more varied
musical life than he might have had if he had been able to pursue a
conventional career as a virtuoso pianist.

That realization is implicit in the title of his autobiography, ``My
Nine Lives: A Memoir of Many Careers in Music'' (2010), which he wrote
with the music critic Anne Midgette.

Early in his career as a pianist Mr. Fleisher produced a warm, sharply
etched and thoughtfully contoured sound that was ideally suited to
19th-century Viennese classics --- Beethoven, Brahms and Schubert, most
notably --- but that also yielded illuminating readings of Rachmaninoff,
Debussy and Liszt and of contemporary American composers like Roger
Sessions and Aaron Copland.

His recordings of the Brahms and Beethoven piano concertos with
\href{https://www.nytimes.com/1970/07/31/archives/george-szell-conductor-is-dead-george-szell-of-cleveland-orchestra.html\#:~:text=George\%20Szell\%2C\%20conductor\%20of\%20the,He\%20was\%2073\%20years\%20old.}{George
Szell} and the Cleveland Orchestra, made from 1958 to 1963, are
considered among the most vivid and moving accounts of those works.

In the 1990s, he recorded spellbinding performances of the peaks of the
left-hand repertoire, including concertos by Ravel, Prokofiev and
Britten, chamber music by Korngold and Schmidt, and solo works by
Saint-Saëns, Godowsky and Bach (Brahms's left-hand arrangement of the
Chaconne from the Partita No. 2 for solo violin).

Even after he returned to recording two-hand works, on the albums ``Two
Hands'' (2004) and ``The Journey'' (2006), he continued to revisit the
left-hand works that had kept him going for three decades.

His album ``All the Things You Are'' (2014) included not only left-hand
arrangements of Gershwin's ``The Man I Love'' and the Jerome Kern song
that gave the collection its title, but also pieces composed for Mr.
Fleisher by George Perle and Leon Kirchner, and a spacious
reconsideration of the Bach-Brahms Chaconne.

\hypertarget{at-4-playing-by-ear}{%
\subsection{At 4, Playing by Ear}\label{at-4-playing-by-ear}}

Leon Fleisher was born in San Francisco on July 23, 1928, to Isidore and
Bertha Fleisher. His parents, Jewish immigrants --- his father was from
Odessa, then in Russia, now in Ukraine; his mother was from Poland ---
each managed one of the family's two hat shops.

Leon was drawn to the piano from an early age. Though he showed little
interest when an older brother, Raymond, was given piano lessons, Leon
would go to the piano when Raymond went out to play after his lessons
and repeat, by ear, everything he had heard. He was 4 years old.

His mother soon decided that Leon, rather than Raymond, should study
piano. She made her intentions for her younger son clear: He would be
either the first Jewish president of the United States or a concert
pianist.

So devoted was his mother to Leon's musical training that after two
weeks of kindergarten, during which he objected strenuously to nap time,
she withdrew him from public school and hired tutors so that he could
devote his time to practicing the piano. She also found ways of bringing
him to the attention of two San Francisco conductors, Pierre Monteux and
Alfred Hertz, who in turn persuaded the pianist
\href{https://www.nytimes.com/1951/08/16/archives/artur-schnabel-69-famed-pianist-dies-best-known-for-interpretation.html}{Artur
Schnabel} to take Leon on as a student in 1938, when he was 9, despite
Schnabel's policy of not teaching children.

By then Leon had already played a few concerts, but Schnabel's single
condition for teaching him was that there be no more concerts. Schnabel
relaxed the rule in 1944 and allowed his teenage pupil to play the
Brahms Piano Concerto No. 1 in D minor with Monteux and the San
Francisco Symphony and then with the New York Philharmonic at Carnegie
Hall, also with Monteux conducting.

Noel Strauss, reviewing the Carnegie Hall performance for The Times,
wrote that Mr. Fleisher, making his New York debut, had ``established
himself as one of the most remarkably gifted of the younger generation
of American keyboard artists.''

In 1945, at \href{https://www.ravinia.org/Page/History}{the Ravinia
summer festival}in Illinois, Mr. Fleisher played the Brahms again --- it
became one of his signature pieces --- as well as the Liszt Concerto No.
2 in A, with Leonard Bernstein conducting the Chicago Symphony
Orchestra. The next summer at Ravinia, he performed four concertos under
the direction of
\href{https://www.nytimes.com/1978/05/17/archives/william-steinberg-orchestral-conductor-dies-at-78-former-music.html}{William
Steinberg} and Szell, who soon engaged Mr. Fleisher to perform with the
Cleveland Orchestra, which Szell took over later that year.

By 1949, however, though he had played with many of the major American
orchestras and had given recitals across the country, engagements began
to dry up for Mr. Fleisher. The next year he moved to Paris and remained
in Europe until 1958, relocating first to the Netherlands and then to
Italy.

As an expatriate, Mr. Fleisher became the first American to win the gold
medal at the Queen Elisabeth Competition in Brussels, in 1952. The
victory led to a long list of engagements in Europe and revived interest
in him among American orchestras, managers and concert promoters.

When Szell and the Cleveland Orchestra were signed to a new recording
contract with the Epic label in 1954, he invited Mr. Fleisher to be his
main soloist for recordings of the great piano concertos.

\hypertarget{always-more-to-attain}{%
\subsection{`Always More to Attain'}\label{always-more-to-attain}}

It was shortly after his return to the United States, in the late 1950s,
that Mr. Fleisher accepted an offer to teach at the Peabody
Conservatory, though he continued to pursue a heavy performing and
recording schedule.

``I was driven, if anything, even harder by all of my successes,'' he
wrote in his memoir. ``There was always more to attain, and more to
achieve, and more musical depths to plumb, and lurking behind it all,
the terrifying risk of failure.''

Image

Mr. Fleisher in concert at Carnegie Hall in 2003.Credit...Chris Lee for
The New York Times

Failure was not far away. During the winter of 1963, he noticed what he
described as laziness in his right index finger, as well as ``a creeping
numbness'' in his right hand. By the summer, the fourth and fifth
fingers of his right hand had begun to curl toward his palm.

The timing was disastrous. He had planned to celebrate the 20th
anniversary of his New York debut with a busy season that included 20
performances in New York alone and a spring 1964 tour of the Soviet
Union, in which he was to be the soloist in Mozart's Concerto No. 25 in
C (K. 503) with Szell and the Cleveland Orchestra.

Shortly before the tour, Mr. Fleisher performed the Mozart in Cleveland.
Szell noted the strain Mr. Fleisher was under and told him that he did
not feel he could undertake the tour. The pianist
\href{https://www.nytimes.com/2005/03/30/arts/music/grant-johannesen-unorthodox-pianist-is-dead-at-83.html\#:~:text=Grant\%20Johannesen\%2C\%20a\%20pianist\%20best,David\%20Johannesen\%2C\%20announced\%20the\%20death.}{Grant
Johannesen} traveled with the orchestra instead.

``The initial problem was a very stupid kind of overwork,'' Mr. Fleisher
said in 1996, cautioning young pianists against following his path. ``I
see kids still falling into this, and there are many reasons for it. The
perfection that they're bombarded with from recordings. The kind of
sound a {[}Vladimir{]} Horowitz produced, which is wonderful, but people
don't realize that he had his technician work very hard on the piano, so
the piano itself helped. So when kids go to an acoustically dead hall,
and get a dead piano, and try to make these Horowitz kinds of sounds,
they end up brutalizing themselves.''

Mr. Fleisher resisted taking up the left-hand repertoire, partly because
he felt that to do so would be an admission that he would never regain
the use of his right hand. But after two years without playing concerts,
he reconsidered, agreeing to play both Ravel's Concerto for the Left
Hand and Benjamin Britten's left-hand work ``Diversions'' with Seiji
Ozawa and the Toronto Symphony in 1967.

The next year, with the pianist and composer
\href{https://www.rogershapirofund.org/founders/dina-koston/}{Dina
Koston}, he started the Theater Chamber Players, a flexible chamber
group meant to present both contemporary music and classics.

The ensemble --- initially based at the Washington Theater Club, later
at the Smithsonian National Museum of Natural History and ultimately at
the Kennedy Center in Washington --- provided an opportunity for Mr.
Fleisher both to play and to conduct. And an invitation to be music
director of the Annapolis Symphony Orchestra in Maryland, a
semiprofessional community group, gave him a chance to work on the
symphonic repertoire.

Soon he was guest-conducting around the country --- his debut at the
head of a professional orchestra took place at Lincoln Center's Mostly
Mozart Festival in 1970 --- and in 1973 he became associate conductor of
the Baltimore Symphony Orchestra.

Mr. Fleisher held that post for only five years, but he maintained a
close relationship with the orchestra thereafter. When the ensemble was
preparing to inaugurate the new Joseph Meyerhoff Symphony Hall in 1982,
its music director, Sergiu Comissiona, invited him to be the
opening-night soloist.

\hypertarget{a-two-hand-return}{%
\subsection{A Two-Hand Return}\label{a-two-hand-return}}

Having recently had an operation to relieve carpal tunnel syndrome, Mr.
Fleisher began to regain the use of his right hand, if only partly and
inconsistently. But he felt he could make the jump back to two-handed
playing, using the televised opening of Meyerhoff Hall as the occasion
for his comeback.

In a bold moment, he told the orchestra that he would play Beethoven's
Fourth Piano Concerto. But as the occasion drew near, he decided to play
Franck's Symphonic Variations instead, a shorter and less pianistically
exposed work.

Image

Mr. Fleisher performing in 2001 at the New York String Orchestera's
annual~ Christmas Eve concert at Carnegie Hall.Credit...Chris Lee/The
New York Times

Most listeners thought the performance went well. But Mr. Fleisher was
not satisfied. In his view, the amount of effort he had expended working
to control his right hand precluded the kind of interpretive depth he
had hoped for, and he dropped plans for a broader return to two-handed
playing.

Shortly after the Baltimore performance, Mr. Fleisher married Katherine
Jacobson, a pianist who had been a student of his at Peabody. His two
previous marriages --- to Dorothy Druzinsky and Rikki Rosenthal ---
ended in divorce.

Ms. Jacobson survives him, as do his children from his first marriage,
Deborah, Richard and Leah Fleisher; his children from his second
marriage, Julian and Paula Fleisher; and two grandchildren.

In 1991, Mr. Fleisher found a doctor who was experimenting with Botox
injections for injuries like his. At first he found that the injections
loosened up his still-cramped fourth and fifth fingers, to the point
where he could play. But the injections wore off, and he was still
looking for a permanent cure.

Having tried Rolfing in the 1970s, he decided to try again in 1994. This
time he found that a regimen of Rolfing and Botox injections was enough
to keep him in playing trim.

As an experiment, he played Mozart's Piano Concerto No. 12 (K. 414) with
the Theater Chamber Players in April 1995, and with the Cleveland
Orchestra and at Tanglewood shortly thereafter.

``Nothing felt sweeter than the feeling of those notes falling into
place,'' he wrote in his memoir, ``the right hand singing, the left hand
balancing it on the lower part of the keyboard, and the piece growing
into something whole and complete, a dream become reality.''

Mr. Fleisher cautiously reclaimed the repertoire he had been unable to
play for more than 30 years, building his recital programs with both
two-hand and left-hand works and playing programs of piano four-hand
works with his wife.

Image

Mr. Fleisher and his wife, the pianist Katherine Jacobson Fleisher,
arriving at the Kennedy Center Honors gala in Washington in 2007. He was
among that year's recipients.~Credit...Nicholas Kamm/Agence
France-Presse --- Getty Images

Mr. Fleisher was made a commander of the Order of Arts and Letters by
the French government in 2006 and received a Kennedy Center Honor the
next year. A film about his struggle with focal dystonia, ``Two Hands,''
directed by Nathaniel Kahn, was nominated for an Academy Award for best
short documentary in 2006.

Toward the end of his life, Mr. Fleisher spoke about the level of
despair he had felt when he was unable to use his right hand. But having
regained that ability he was philosophical about the challenges life
presents.

``There are forces out there,'' he told The International Herald Tribune
in 2007, ``and if you keep yourself open to them, if you go along with
them, there are wondrous surprises.''

Jack Kadden contributed reporting.

Advertisement

\protect\hyperlink{after-bottom}{Continue reading the main story}

\hypertarget{site-index}{%
\subsection{Site Index}\label{site-index}}

\hypertarget{site-information-navigation}{%
\subsection{Site Information
Navigation}\label{site-information-navigation}}

\begin{itemize}
\tightlist
\item
  \href{https://help.nytimes.com/hc/en-us/articles/115014792127-Copyright-notice}{©~2020~The
  New York Times Company}
\end{itemize}

\begin{itemize}
\tightlist
\item
  \href{https://www.nytco.com/}{NYTCo}
\item
  \href{https://help.nytimes.com/hc/en-us/articles/115015385887-Contact-Us}{Contact
  Us}
\item
  \href{https://www.nytco.com/careers/}{Work with us}
\item
  \href{https://nytmediakit.com/}{Advertise}
\item
  \href{http://www.tbrandstudio.com/}{T Brand Studio}
\item
  \href{https://www.nytimes.com/privacy/cookie-policy\#how-do-i-manage-trackers}{Your
  Ad Choices}
\item
  \href{https://www.nytimes.com/privacy}{Privacy}
\item
  \href{https://help.nytimes.com/hc/en-us/articles/115014893428-Terms-of-service}{Terms
  of Service}
\item
  \href{https://help.nytimes.com/hc/en-us/articles/115014893968-Terms-of-sale}{Terms
  of Sale}
\item
  \href{https://spiderbites.nytimes.com}{Site Map}
\item
  \href{https://help.nytimes.com/hc/en-us}{Help}
\item
  \href{https://www.nytimes.com/subscription?campaignId=37WXW}{Subscriptions}
\end{itemize}
