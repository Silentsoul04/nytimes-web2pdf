Sections

SEARCH

\protect\hyperlink{site-content}{Skip to
content}\protect\hyperlink{site-index}{Skip to site index}

\href{https://www.nytimes.com/section/arts/television}{Television}

\href{https://myaccount.nytimes.com/auth/login?response_type=cookie\&client_id=vi}{}

\href{https://www.nytimes.com/section/todayspaper}{Today's Paper}

\href{/section/arts/television}{Television}\textbar{}`Immigration
Nation' Review: The Banality of Deportation

\url{https://nyti.ms/2DeUuqR}

\begin{itemize}
\item
\item
\item
\item
\item
\end{itemize}

Advertisement

\protect\hyperlink{after-top}{Continue reading the main story}

Supported by

\protect\hyperlink{after-sponsor}{Continue reading the main story}

\hypertarget{immigration-nation-review-the-banality-of-deportation}{%
\section{`Immigration Nation' Review: The Banality of
Deportation}\label{immigration-nation-review-the-banality-of-deportation}}

This Netflix documentary looks at the bureaucracy of immigration
enforcement --- an intriguing investigation that probably won't change
any minds.

\includegraphics{https://static01.nyt.com/images/2020/08/02/arts/02immigration-review/02immigration-review-articleLarge.jpg?quality=75\&auto=webp\&disable=upscale}

\href{https://www.nytimes.com/by/mike-hale}{\includegraphics{https://static01.nyt.com/images/2018/02/16/multimedia/author-mike-hale/author-mike-hale-thumbLarge.jpg}}

By \href{https://www.nytimes.com/by/mike-hale}{Mike Hale}

\begin{itemize}
\item
  Aug. 2, 2020
\item
  \begin{itemize}
  \item
  \item
  \item
  \item
  \item
  \end{itemize}
\end{itemize}

If you watch only one documentary about immigration, then by all means
make it ``Immigration Nation,'' a six-hour Netflix series that mixes
reporting with an impressive amount of vivid ride-along observation.

Parts of it may start to drag or feel padded, but its
see-the-whole-elephant approach to one of America's most divisive issues
has inherent value. It will almost certainly leave you better informed
than you were before, even if its net effect may be to further entrench
people on whichever side of the debate they already occupy.

Immigration to the United States is a story spread across thousands of
miles, a variety of faceless government agencies and a tapestry of
determined, often desperate petitioners, and ``Immigration Nation''
tries to cover as many of its facets as it can cram in. This includes
the widely known ones,
\href{https://www.nytimes.com/2019/03/09/us/migrant-family-separations-border.html}{like
child separation at the border}, as well as less familiar angles, such
as the exploitation of migrants who take on the work of natural-disaster
recovery and federal attempts to co-opt local law enforcement into
immigration agencies.

Much of the time, especially after its more fluid and immersive initial
episodes, the series takes a standard television current-affairs
approach, and as you watch its segments you may recall sharper or more
evocative reports on the same stories by shows like ``Frontline,''
``Vice'' and ``Last Week Tonight With John Oliver.''

But the makers of ``Immigration Nation,'' Christina Clusiau and Shaul
Schwarz, benefited from time --- they filmed for nearly three years ---
and a startling degree of access, particularly to agents of Immigration
and Customs Enforcement as they rounded up immigrants, processed them
for (mostly) deportation and spoke to the camera about how it made them
feel. And in the series' first two hours, the results of that embedding,
with ICE operations in New York, Charlotte, N.C., and El Paso can be
startling and engrossing.

Part of that effect comes from seeing agents push the boundaries of
legality --- most strikingly, how they routinely enter apartments when
``invited'' by cowed, uncomprehending immigrants, in a way that's
surprisingly similar to what you'd see in a TV cop drama. (Maybe that's
where they learned it.) Once inside the home of the target, probably an
immigrant accused of a crime, they frequently find ``collaterals,''
additional people who can be rounded up simply because they're
undocumented.

Material like that, and worse --- like an agent picking an apartment
building's lock --- gained ``Immigration Nation'' some prerelease
publicity, particularly when The New York Times reported that
\href{https://www.nytimes.com/2020/07/23/us/trump-immigration-nation-netflix.html}{ICE
had pressured the filmmakers} to delay the release and remove footage.

But the real impact of the show's early episodes isn't the outrage you
may feel over the thuggish tactics. It's the wearying, demoralizing
depiction of a self-perpetuating bureaucracy, one that churns through
the lives of people it takes little notice of --- as if your trip to the
D.M.V. meant not just standing in an endless line, but then being
shackled and put on a plane to Central America.

The scenes inside field offices and detention centers, as agents bluffly
banter with the people whose lives they're destroying and then joke with
one another about funny accents and kung pao chicken, might have been
written by Kafka, except his dialogue would have been better. The
series' hallmark is not an image but a sound bite --- the agents'
endless variations on ``I may not like it, but it's the job.'' The
human-rights lawyer Becca Heller sums it up nicely: ``When you add up
all the people just doing their job, it becomes this crazy, terrorizing
system.''

``Immigration Nation'' provides abundant evidence for things that some
might call fake news, like the determination of ICE, under the Trump
administration, to remove immigrants from the United States in bulk
regardless of whether they pose any danger. As one of the disarmingly
honest agents says, ``They want to get rid of everybody, I guess.''

That will be the takeaway for those who want to make political points
from the series, from either direction. And in the later episodes there
are wrenching individual stories, like that of a Guatemalan grandmother
seeking asylum and sitting for more than a year in a Texas detention
center, though these segments tend to indulge in superfluous scenes of
inspiration and tearful condolence.

But what sticks with you from ``Immigration Nation'' is its up-close
depiction of the banality of deportation --- of the huge disconnect
between the everyday people of ICE and the Border Patrol and the
everyday people they detain, arrest and ``process.'' (In El Paso, a
morning meeting at a detention center ends with the chant, ``1, 2, 3,
processing!'')

Agent after agent expresses an ambivalence about the job that's given
its most extreme expression by an Arizona ICE investigator who says, ``I
put my personal feelings aside, which, yeah, maybe that's what every
Nazi said, right?'' But he immediately adds, ``I actually believe in the
cause of trying to enforce some sort of sovereignty over our borders,
and no one's figured out a better way to do it yet.''

It's a nice summation of the schism, within the country at large, that
will keep us talking past one another despite the filmmakers' best
efforts.

Advertisement

\protect\hyperlink{after-bottom}{Continue reading the main story}

\hypertarget{site-index}{%
\subsection{Site Index}\label{site-index}}

\hypertarget{site-information-navigation}{%
\subsection{Site Information
Navigation}\label{site-information-navigation}}

\begin{itemize}
\tightlist
\item
  \href{https://help.nytimes.com/hc/en-us/articles/115014792127-Copyright-notice}{©~2020~The
  New York Times Company}
\end{itemize}

\begin{itemize}
\tightlist
\item
  \href{https://www.nytco.com/}{NYTCo}
\item
  \href{https://help.nytimes.com/hc/en-us/articles/115015385887-Contact-Us}{Contact
  Us}
\item
  \href{https://www.nytco.com/careers/}{Work with us}
\item
  \href{https://nytmediakit.com/}{Advertise}
\item
  \href{http://www.tbrandstudio.com/}{T Brand Studio}
\item
  \href{https://www.nytimes.com/privacy/cookie-policy\#how-do-i-manage-trackers}{Your
  Ad Choices}
\item
  \href{https://www.nytimes.com/privacy}{Privacy}
\item
  \href{https://help.nytimes.com/hc/en-us/articles/115014893428-Terms-of-service}{Terms
  of Service}
\item
  \href{https://help.nytimes.com/hc/en-us/articles/115014893968-Terms-of-sale}{Terms
  of Sale}
\item
  \href{https://spiderbites.nytimes.com}{Site Map}
\item
  \href{https://help.nytimes.com/hc/en-us}{Help}
\item
  \href{https://www.nytimes.com/subscription?campaignId=37WXW}{Subscriptions}
\end{itemize}
