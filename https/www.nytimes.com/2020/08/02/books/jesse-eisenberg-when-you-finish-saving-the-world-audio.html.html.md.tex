Sections

SEARCH

\protect\hyperlink{site-content}{Skip to
content}\protect\hyperlink{site-index}{Skip to site index}

\href{https://www.nytimes.com/section/books}{Books}

\href{https://myaccount.nytimes.com/auth/login?response_type=cookie\&client_id=vi}{}

\href{https://www.nytimes.com/section/todayspaper}{Today's Paper}

\href{/section/books}{Books}\textbar{}It's a Book. It's a Podcast. It's
a Three-Act Play, in Your Ears.

\url{https://nyti.ms/2DmisAh}

\begin{itemize}
\item
\item
\item
\item
\item
\end{itemize}

\href{https://www.nytimes.com/spotlight/at-home?action=click\&pgtype=Article\&state=default\&region=TOP_BANNER\&context=at_home_menu}{At
Home}

\begin{itemize}
\tightlist
\item
  \href{https://www.nytimes.com/2020/07/28/books/time-for-a-literary-road-trip.html?action=click\&pgtype=Article\&state=default\&region=TOP_BANNER\&context=at_home_menu}{Take:
  A Literary Road Trip}
\item
  \href{https://www.nytimes.com/2020/07/29/magazine/bored-with-your-home-cooking-some-smoky-eggplant-will-fix-that.html?action=click\&pgtype=Article\&state=default\&region=TOP_BANNER\&context=at_home_menu}{Cook:
  Smoky Eggplant}
\item
  \href{https://www.nytimes.com/2020/07/27/travel/moose-michigan-isle-royale.html?action=click\&pgtype=Article\&state=default\&region=TOP_BANNER\&context=at_home_menu}{Look
  Out: For Moose}
\item
  \href{https://www.nytimes.com/interactive/2020/at-home/even-more-reporters-editors-diaries-lists-recommendations.html?action=click\&pgtype=Article\&state=default\&region=TOP_BANNER\&context=at_home_menu}{Explore:
  Reporters' Obsessions}
\end{itemize}

Advertisement

\protect\hyperlink{after-top}{Continue reading the main story}

Supported by

\protect\hyperlink{after-sponsor}{Continue reading the main story}

\hypertarget{its-a-book-its-a-podcast-its-a-three-act-play-in-your-ears}{%
\section{It's a Book. It's a Podcast. It's a Three-Act Play, in Your
Ears.}\label{its-a-book-its-a-podcast-its-a-three-act-play-in-your-ears}}

Jesse Eisenberg's audio drama, ``When You Finish Saving the World,'' is
coming to Audible ahead of a film adaptation with Julianne Moore.

\includegraphics{https://static01.nyt.com/images/2020/08/04/books/04Eisenberg1/merlin_174776955_5e1cd9f2-e7e2-4c7a-b8bc-e3a2a9787aa9-articleLarge.jpg?quality=75\&auto=webp\&disable=upscale}

By Elisabeth Egan

\begin{itemize}
\item
  Aug. 2, 2020
\item
  \begin{itemize}
  \item
  \item
  \item
  \item
  \item
  \end{itemize}
\end{itemize}

If you're missing the chilly joy of ducking into a movie theater on a
sweltering day, welcome to the club.

Ditto for attending concerts, plays, sporting events and awkward variety
shows on the last day of summer camp. Our usual forms of entertainment
are scarce right now, but here's a fresh alternative: Jesse Eisenberg's
\href{https://www.nytimes.com/2018/12/06/books/curtis-sittenfeld-audible-original.html}{Audible
Original}, ``When You Finish Saving the World.''

The idea for the five-hour, 17-minute audio drama, available on Tuesday,
grew out of a conversation between Eisenberg --- the star of movies such
as ``The Social Network'' as well as an
\href{https://www.nytimes.com/2015/09/06/books/review/jesse-eisenberg-by-the-book.html}{author}
and
\href{https://www.nytimes.com/2015/06/03/theater/review-the-spoils-stars-jesse-eisenberg-as-narcissist.html}{playwright}
--- and a friend who confessed that he had no emotional connection to
his newborn daughter.

``He was mortified and felt terribly guilty. I thought this was an
interesting dynamic to explore,'' Eisenberg said in a phone interview.
``Then I met these great producers who told me about a new format which
is fiction created exclusively for audio. The internal struggle of a
character who is emotionally a bit stifled seemed perfect for that
medium.''

``When You Finish Saving the World'' tells the story of the Katz family
over 30 years. First, we hear from Nathan (voiced by Eisenberg), a young
father struggling to connect with his newborn son; then Ziggy
(\href{https://www.nytimes.com/2017/09/28/style/stranger-things-finn-wolfhard-rough-trade-vinyl.html}{Finn
Wolfhard}), that same baby, now grown into a 15-year-old blundering
through adolescent angst in 2032, which makes the present look downright
blissful; and, finally, Rachel
(\href{https://www.nytimes.com/2014/10/19/movies/kaitlyn-dever-talks-about-men-women-children.html}{Kaitlyn
Dever}), a wide-eyed, well-intentioned student trying to get her
bearings at Indiana University in 2002. Her path is about to make a
zigzag that will lead her to become Nathan's wife and Ziggy's mother.

Each character takes shape through his or her own series of audio files.
Nathan's are intended for a couples' therapist and Ziggy's for a
futuristic bot therapist he has been ``sentenced'' to see. Rachel's
cassette tapes are intended for her high school boyfriend, who is
awaiting deployment to Afghanistan.

These dispatches are whispered and wept from a variety of locations,
including a guest room, a bathroom and a Subway sandwich shop. They give
you the forbidden thrill of reading someone else's mail, with the added
bonus of being able to hear the sender's voice. The experience is
reminiscent of watching a play --- the intimacy and urgency of
``\href{https://www.nytimes.com/2016/12/04/theater/dear-evan-hansen-review.html}{Dear
Evan Hansen}'' come to mind --- to the extent that brief pauses between
sections are as jarring as the house lights coming up in a hushed
theater.

\includegraphics{https://static01.nyt.com/images/2020/08/04/books/04Eisenberg2/merlin_174776958_516fd49d-2e1e-4c40-942c-8a43a94cbbe7-articleLarge.jpg?quality=75\&auto=webp\&disable=upscale}

Rachel Ghiazza, the head of U.S. content at Audible, said Eisenberg's
approach is ``genre-bending'' and ``pushes the boundaries of what audio
storytelling can do.'' This podcast-weary walker would have to agree.

One might wonder about the logistics of producing anything during a
pandemic, let alone a three-part drama with music and sound effects such
as a baby crying, a party raging, a button clicking on an old-fashioned
tape recorder. (Keen-eared Subway enthusiasts may question the crinkly
noise of a sandwich being unwrapped --- it sounds like it may be the
wrong paper.)

Here's how it all came together. Eisenberg spent several months writing
the script, even meeting with veterans to find the right military base
for Rachel's story. ``When a friend who was stationed in
\href{https://www.nytimes.com/2005/07/31/world/uzbeks-order-us-from-base-in-refugee-rift.html}{Karshi-Khanabad
in Uzbekistan} told me about his experiences, I knew I had the right
location,'' Eisenberg said. ``All of the military stories are based on
friends' experiences, and I lined them up on the same timeline as U.S.
politics in 2002, so Rachel would have to struggle to navigate two
opposing worlds: a boyfriend stationed overseas and an antiwar, liberal
college campus.''

In the early weeks of 2020, Eisenberg recorded his part in a Newark
studio, then traveled to Vancouver's Gastown neighborhood to record with
Wolfhard in a studio owned by the singer-songwriter
\href{https://www.nytimes.com/1994/03/08/arts/review-pop-bryan-adams-more-mr-nice-guy.html}{Bryan
Adams}. Wolfhard's section happens to include singing and a slew of
made-up slang delivered with a fluency only an actual teenager could
muster.

Wolfhard, 17 and best known as Mike Wheeler on the Netflix series
``Stranger Things,'' said: ``It was therapeutic. I got to be kind of a
brat for a change. Hopefully I'm not as much of one in real life.''

Of his invention of the hipster lexicon of the future, Eisenberg said,
``Ziggy is one of these pretentious kids who adopts something before
general society agrees that it's palatable. My only inconvenience was,
anytime I came up with a new word, I would immediately search in
\href{https://www.nytimes.com/2009/07/05/magazine/05FOB-medium-t.html}{Urban
Dictionary} and discover that it meant something that was horribly
sexually perverse.''

Wolfhard put Eisenberg in touch with Dever (a Golden Globe nominee for
``\href{https://www.nytimes.com/2019/09/13/arts/television/review-netflix-unbelievable.html}{Unbelievable}'')
and, in March, she and Eisenberg met up at a Los Angeles coffee shop,
Joan's on Third, to discuss the project. ``Jesse was the last person I
shared a cookie with in the real world,'' Dever said.

``It was classic chocolate chip,'' Eisenberg recalled. ``Had I known the
world was about to change, I would have gotten something else.''

With California on lockdown, logistics presented a challenge. ``Where
the suspense came in was figuring out how to record Kaitlyn's part of
the story,'' Ghiazza said. The team at Audible put together a kit
containing ``everything she needed to turn her home into a professional
recording studio'' --- including a microphone, audio interface, monitor,
Bluetooth mouse, pop filter, microphone stand, headphones and cabling.

Dever said she could hear Eisenberg in her headphones, but otherwise she
was on her own in a bedroom closet. ``There was something about being in
the comfort of my own home that made everything more relaxed and casual.
It took the pressure off,'' she said.

``When You Finish Saving the World'' is also being made into a movie,
with some adjustments --- like taking place in the present day. Other
details will remain the same:
\href{https://www.nytimes.com/2019/03/06/movies/julianne-moore-gloria-bell.html}{Julianne
Moore}, who plays Ziggy's mother, runs a shelter for victims of domestic
violence in the film, just as Rachel does in the audio version.

``She's a mother who is a hero to thousands of people but feels less
comfortable as a mother to one,'' Eisenberg said.

As for what listeners take away from ``When You Finish Saving the
World,'' Eisenberg hopes it is empathy. ``In stories that take place
from multiple characters' perspectives, where you see the same world
through different eyes, I think there's a macro message that the world
is full of complicated people, not heroes and villains,'' he said.
``Everybody's trying their best. If you try to understand their
intentions, you might understand their behavior better.''

\emph{\textbf{Correction: Aug. 2, 2020}}\\
\emph{An earlier version of this article misstated where Jesse Eisenberg
recorded his part of ``When You Finish Saving the World.'' It was
Newark, not Jersey City.}

\emph{Follow New York Times Books on}
\href{https://www.facebook.com/nytbooks/}{\emph{Facebook}}\emph{,}
\href{https://twitter.com/nytimesbooks}{\emph{Twitter}} \emph{and}
\href{https://www.instagram.com/nytbooks/}{\emph{Instagram}}\emph{, sign
up for}
\href{https://www.nytimes.com/newsletters/books-review}{\emph{our
newsletter}} \emph{or}
\href{https://www.nytimes.com/interactive/2017/books/books-calendar.html}{\emph{our
literary calendar}}\emph{. And listen to us on the}
\href{https://www.nytimes.com/column/book-review-podcast}{\emph{Book
Review podcast}}\emph{.}

Advertisement

\protect\hyperlink{after-bottom}{Continue reading the main story}

\hypertarget{site-index}{%
\subsection{Site Index}\label{site-index}}

\hypertarget{site-information-navigation}{%
\subsection{Site Information
Navigation}\label{site-information-navigation}}

\begin{itemize}
\tightlist
\item
  \href{https://help.nytimes.com/hc/en-us/articles/115014792127-Copyright-notice}{©~2020~The
  New York Times Company}
\end{itemize}

\begin{itemize}
\tightlist
\item
  \href{https://www.nytco.com/}{NYTCo}
\item
  \href{https://help.nytimes.com/hc/en-us/articles/115015385887-Contact-Us}{Contact
  Us}
\item
  \href{https://www.nytco.com/careers/}{Work with us}
\item
  \href{https://nytmediakit.com/}{Advertise}
\item
  \href{http://www.tbrandstudio.com/}{T Brand Studio}
\item
  \href{https://www.nytimes.com/privacy/cookie-policy\#how-do-i-manage-trackers}{Your
  Ad Choices}
\item
  \href{https://www.nytimes.com/privacy}{Privacy}
\item
  \href{https://help.nytimes.com/hc/en-us/articles/115014893428-Terms-of-service}{Terms
  of Service}
\item
  \href{https://help.nytimes.com/hc/en-us/articles/115014893968-Terms-of-sale}{Terms
  of Sale}
\item
  \href{https://spiderbites.nytimes.com}{Site Map}
\item
  \href{https://help.nytimes.com/hc/en-us}{Help}
\item
  \href{https://www.nytimes.com/subscription?campaignId=37WXW}{Subscriptions}
\end{itemize}
