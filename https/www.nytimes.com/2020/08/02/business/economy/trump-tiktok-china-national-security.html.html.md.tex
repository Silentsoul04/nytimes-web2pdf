Sections

SEARCH

\protect\hyperlink{site-content}{Skip to
content}\protect\hyperlink{site-index}{Skip to site index}

\href{https://www.nytimes.com/section/business/economy}{Economy}

\href{https://myaccount.nytimes.com/auth/login?response_type=cookie\&client_id=vi}{}

\href{https://www.nytimes.com/section/todayspaper}{Today's Paper}

\href{/section/business/economy}{Economy}\textbar{}Microsoft Says It'll
Continue Pursuit of TikTok

\url{https://nyti.ms/2DxNvJl}

\begin{itemize}
\item
\item
\item
\item
\item
\end{itemize}

Advertisement

\protect\hyperlink{after-top}{Continue reading the main story}

Supported by

\protect\hyperlink{after-sponsor}{Continue reading the main story}

\hypertarget{microsoft-says-itll-continue-pursuit-of-tiktok}{%
\section{Microsoft Says It'll Continue Pursuit of
TikTok}\label{microsoft-says-itll-continue-pursuit-of-tiktok}}

The announcement came after the company consulted with President Trump,
who has threatened to ban the app and expressed national security
concerns about it in recent weeks.

\includegraphics{https://static01.nyt.com/images/2020/08/02/us/politics/02dc-tiktok1/02dc-tiktok1-articleLarge.jpg?quality=75\&auto=webp\&disable=upscale}

\href{https://www.nytimes.com/by/mike-isaac}{\includegraphics{https://static01.nyt.com/images/2018/02/16/multimedia/author-mike-isaac/author-mike-isaac-thumbLarge.jpg}}\href{https://www.nytimes.com/by/ana-swanson}{\includegraphics{https://static01.nyt.com/images/2018/12/10/multimedia/author-ana-swanson/author-ana-swanson-thumbLarge.png}}\href{https://www.nytimes.com/by/maggie-haberman}{\includegraphics{https://static01.nyt.com/images/2018/07/12/multimedia/author-maggie-haberman/author-maggie-haberman-thumbLarge.png}}

By \href{https://www.nytimes.com/by/mike-isaac}{Mike Isaac},
\href{https://www.nytimes.com/by/ana-swanson}{Ana Swanson} and
\href{https://www.nytimes.com/by/maggie-haberman}{Maggie Haberman}

\begin{itemize}
\item
  Aug. 2, 2020
\item
  \begin{itemize}
  \item
  \item
  \item
  \item
  \item
  \end{itemize}
\end{itemize}

Microsoft said on Sunday that it would continue to pursue the
\href{https://www.nytimes.com/2020/08/03/technology/tiktok-microsoft-tweens.html}{purchase
of TikTok} in the United States after consulting with President Trump,
clearing the way for a potential blockbuster deal between the software
giant and the viral social media phenomenon.

\href{https://blogs.microsoft.com/blog/2020/08/02/microsoft-to-continue-discussions-on-potential-tiktok-purchase-in-the-united-states/}{The
announcement} came as Mr. Trump has expressed repeated concerns about
\href{https://www.nytimes.com/2020/08/03/technology/trump-tiktok-microsoft.html}{TikTok}
and national security in recent weeks because of the app's Chinese
origins and backing; on Friday,
\href{https://www.nytimes.com/2020/08/02/style/tiktok-ban-threat-trump.html}{Mr.
Trump threatened to ban the app} entirely within the United States,
saying any decision could come as soon as Saturday.

Those plans appeared to change after several of Mr. Trump's allies and
Satya Nadella, the chief executive of
\href{https://www.nytimes.com/2020/08/03/us/navy-seal-museum-kaepernick.html}{Microsoft},
spoke over the weekend with the president.

``Microsoft fully appreciates the importance of addressing the
president's concerns,'' the company
\href{https://blogs.microsoft.com/blog/2020/08/02/microsoft-to-continue-discussions-on-potential-tiktok-purchase-in-the-united-states/}{said
in a statement}. ``It is committed to acquiring TikTok subject to a
complete security review and providing proper economic benefits to the
United States, including the United States Treasury.''

Microsoft said it would pursue the deal over the coming weeks, and
expected to complete the discussions no later than Sept. 15. Such a deal
would involve purchasing the TikTok service in the United States,
Canada, Australia and New Zealand;
\href{https://www.nytimes.com/2020/08/03/technology/tiktok-trump-sale-microsoft.html}{ByteDance},
the parent company of TikTok, would continue to own the social media
app's operations in Beijing and other markets.

Microsoft may also bring on a series of outside investors, which would
hold minority stakes in any deal. In recent weeks, investors from
Sequoia Capital, SoftBank and General Atlantic have all held talks with
TikTok to discuss participating in an acquisition of the company,
according to two people familiar with the discussions.

Such a deal would be a boon for the Redmond, Wash.-based Microsoft,
which has pursued corporate and enterprise computing lines of business
under the leadership of Mr. Nadella, who took over as chief executive in
2014. Though it has dabbled in consumer acquisitions --- Microsoft
purchased Minecraft in 2014 and bought LinkedIn in 2017 --- the purchase
of TikTok would be largely new ground for Mr. Nadella. More than 800
million people regularly use the app to watch viral videos, with some
100 million of those users in the United States.

Acquiring TikTok would also pit Microsoft directly against social media
titans like Twitter, Pinterest, Reddit and the mighty Facebook, the
latter used by more than three billion people regularly. All of the
companies compete for user attention and billions in digital advertising
dollars. Administration officials emphasized on Sunday that as is
frequently the case with Mr. Trump, no decision is final until paperwork
was signed.

The forced sale is the latest in a series of punitive actions the Trump
administration has taken against China, which the president blames for
allowing the coronavirus pandemic to spread and damage the American
economy, diminishing his re-election chances. As the election nears, Mr.
Trump has increasingly
\href{https://www.nytimes.com/2020/07/25/world/asia/us-china-trump-xi.html}{challenged
China over security, technology and commercial relations} in an attempt
to persuade voters that he will be tougher in taking on Beijing than
former Vice President Joseph R. Biden Jr.

But a ban on TikTok, which could target its presence in the Apple and
Google app stores, would come with other difficulties, including irking
millions of young Americans who share viral videos and dance clips on
the service. It also most likely would prompt legal challenges, anger
prominent Republican lawmakers and dismay the business community.

Those tensions spilled into the open over the weekend as Washington
awaited a decision from Mr. Trump.

Surrounded by few White House aides on Friday night as he returned to
Washington aboard Air Force One, Mr. Trump caught several advisers by
surprise when he told reporters he planned to ``terminate'' the ability
of TikTok to operate in the United States using emergency economic
powers or an executive order.

Several advisers were furious, and suggested that Peter Navarro, the top
trade adviser who often has Mr. Trump's ear, and other people in the
president's inner circle had helped short-circuit the president's
approval of a possible sale to Microsoft, according to White House
officials and others close to the president. Those who opposed the deal
had focused on the idea of punishing China, not what could happen to a
popular platform. Mr. Navarro did not immediately respond to a request
for comment.

As the president played golf at his club in Virginia on Saturday, his
advisers discussed how to persuade him to sign off on the Microsoft deal
--- and to convey the political repercussions of simply turning off a
service for tens of millions of people in the United States, according
to a person familiar with what took place.

Several people, including Treasury Secretary Steven Mnuchin, reached out
to Senator Lindsey Graham, Republican of South Carolina and an informal
adviser to Mr. Trump, to ask him to intervene. The Treasury Department
declined to comment on Sunday.

After speaking with Microsoft officials a few times, Mr. Graham
eventually tweeted about the deal, saying that Mr. Trump was ``right to
want to make sure that the Chinese Communist Party doesn't own TikTok
and most importantly --- all of your private data.''

He added: ``What's the right answer? Have an American company like
Microsoft take over TikTok. Win-win. Keeps competition alive and data
out of the hands of the Chinese Communist Party.''

The tweet caught Mr. Trump's eye, prompting a call between the two in
which Mr. Graham told the president that he agreed that the platform was
a national security risk, but he stressed the political risks of banning
the app.

Some of Mr. Trump's closest political advisers, including Mr. Mnuchin
and Larry Kudlow, the chief of the National Economic Council, had also
been urging the president to allow a sale of TikTok.

In a bid to sway the president, several business leaders and prominent
Republican lawmakers, including Senators John Cornyn of Texas and Marco
Rubio of Florida weighed in on Sunday.

``I was among the first to warn of danger posed by TikTok last year,''
Mr. Rubio wrote on Twitter. ``As I have shared with POTUS \&
\href{https://twitter.com/WhiteHouse}{@WhiteHouse} if the company \&
data can be purchased \& secured by a trusted U.S. company that would be
a positive \& acceptable outcome.''

Myron Brilliant, the executive vice president of the U.S. Chamber of
Commerce, tweeted that a sale would ``be a good solution that helps to
address some security concerns, strengthens the US
\href{https://twitter.com/hashtag/digitaleconomy?src=hashtag_click}{\#digitaleconomy},
and preserves an app enjoyed by millions of Americans.''

On Sunday, Mr. Trump spoke with Microsoft officials after becoming
convinced that he was heading off a political issue and solving a
security risk, according to a person familiar with what took place.

If a deal ultimately materializes, any agreement would be contingent
upon strict data security measures. Microsoft said that it would ensure
all private data of American users would be transferred to and held on
servers within the United States. ``To the extent that any such data is
currently stored or backed-up outside the United States, Microsoft would
ensure that this data is deleted from servers outside the country after
it is transferred,'' the company said.

Such a deal would also be contingent upon ``providing proper economic
benefits to the United States,'' according to Microsoft's statement; one
of those benefits could include creating a number of U.S. jobs as a
result of the deal, according to a person familiar with the matter.

But the move is unlikely to please all of Mr. Trump's advisers. Some of
Mr. Trump's most hawkish advisers, including Mr. Navarro, have objected
to TikTok's sale, seeing the moment as an opportunity to push through
more expansive measures that could curtail the influence of Chinese apps
more broadly.

Speaking
\href{https://www.foxnews.com/media/peter-navarro-on-tiktok-china-use-these-social-media-apps-to-track-you-and-surveil-you-and-monitor-your-movements}{on
Fox News on Saturday night}, Mr. Navarro, a noted China critic,
criticized the attempted acquisition, saying that Microsoft was ``the
software that the People's Liberation Army and Chinese government run
on'' and that Microsoft helped China build its Great Firewall.

``What this president and the White House is going to be doing is look
at any kind of software that sends the information for Americans back to
servers in China,'' Mr. Navarro said. ``They're going to come under
scrutiny.''

Michael Crowley, David E. Sanger and Alan Rappeport contributed
reporting.

Advertisement

\protect\hyperlink{after-bottom}{Continue reading the main story}

\hypertarget{site-index}{%
\subsection{Site Index}\label{site-index}}

\hypertarget{site-information-navigation}{%
\subsection{Site Information
Navigation}\label{site-information-navigation}}

\begin{itemize}
\tightlist
\item
  \href{https://help.nytimes.com/hc/en-us/articles/115014792127-Copyright-notice}{©~2020~The
  New York Times Company}
\end{itemize}

\begin{itemize}
\tightlist
\item
  \href{https://www.nytco.com/}{NYTCo}
\item
  \href{https://help.nytimes.com/hc/en-us/articles/115015385887-Contact-Us}{Contact
  Us}
\item
  \href{https://www.nytco.com/careers/}{Work with us}
\item
  \href{https://nytmediakit.com/}{Advertise}
\item
  \href{http://www.tbrandstudio.com/}{T Brand Studio}
\item
  \href{https://www.nytimes.com/privacy/cookie-policy\#how-do-i-manage-trackers}{Your
  Ad Choices}
\item
  \href{https://www.nytimes.com/privacy}{Privacy}
\item
  \href{https://help.nytimes.com/hc/en-us/articles/115014893428-Terms-of-service}{Terms
  of Service}
\item
  \href{https://help.nytimes.com/hc/en-us/articles/115014893968-Terms-of-sale}{Terms
  of Sale}
\item
  \href{https://spiderbites.nytimes.com}{Site Map}
\item
  \href{https://help.nytimes.com/hc/en-us}{Help}
\item
  \href{https://www.nytimes.com/subscription?campaignId=37WXW}{Subscriptions}
\end{itemize}
