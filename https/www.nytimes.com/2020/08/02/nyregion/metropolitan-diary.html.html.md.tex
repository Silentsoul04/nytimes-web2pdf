Sections

SEARCH

\protect\hyperlink{site-content}{Skip to
content}\protect\hyperlink{site-index}{Skip to site index}

\href{https://www.nytimes.com/section/nyregion}{New York}

\href{https://myaccount.nytimes.com/auth/login?response_type=cookie\&client_id=vi}{}

\href{https://www.nytimes.com/section/todayspaper}{Today's Paper}

\href{/section/nyregion}{New York}\textbar{}`As I Started to Walk Away,
the Second Man Reached Out His Hand'

\url{https://nyti.ms/33fDUBW}

\begin{itemize}
\item
\item
\item
\item
\item
\item
\end{itemize}

Advertisement

\protect\hyperlink{after-top}{Continue reading the main story}

Supported by

\protect\hyperlink{after-sponsor}{Continue reading the main story}

METROPOLITAN DIARY

\hypertarget{as-i-started-to-walk-away-the-second-man-reached-out-his-hand}{%
\section{`As I Started to Walk Away, the Second Man Reached Out His
Hand'}\label{as-i-started-to-walk-away-the-second-man-reached-out-his-hand}}

Giving directions, a noisy Chinatown dining room and more reader tales
of New York City in this week's Metropolitan Diary.

\begin{itemize}
\item
  Aug. 2, 2020
\item
  \begin{itemize}
  \item
  \item
  \item
  \item
  \item
  \item
  \end{itemize}
\end{itemize}

Image

\hypertarget{giving-directions}{%
\subsection{Giving Directions}\label{giving-directions}}

Dear Diary:

As I left the No. 6 train station at Bleecker Street, I noticed two
young men on the corner. One was holding a map, and they both had
puzzled looks on their faces as they scanned the nearby street signs.

I asked whether they needed help.

The one with the map said he knew where they were but couldn't find the
spot on the map.

I pointed out Houston Street a block away, and then showed them where it
was on the map. They thanked me.

As I started to walk away, the second man reached out his hand --- not
to shake mine, but to give me a \$1 bill.

It was the only time I'd been offered a tip for giving directions.

\emph{--- John F. Backe}

\begin{center}\rule{0.5\linewidth}{\linethickness}\end{center}

Image

\hypertarget{in-chinatown}{%
\subsection{In Chinatown}\label{in-chinatown}}

Dear Diary:

It was the early '70s, and we were at our favorite Chinatown restaurant.
Our party of five included a former New York City junior high school
teacher, Cathleen McDonnell Pietronuto.

As usual, the restaurant was crowded, and the noise level was getting
out of control.

At some point, Cathy stood up.

``There is entirely too much noise in this room,'' she announced in her
classroom-tested voice.

Total silence.

Then a tentative voice piped up from across the dining room.

``Miss McDonnell?''

\emph{--- Joseph Demas}

\begin{center}\rule{0.5\linewidth}{\linethickness}\end{center}

Image

\hypertarget{a-tourist}{%
\subsection{A Tourist}\label{a-tourist}}

Dear Diary:

I was a tourist in town, walking up the east side of Central Park, when
I saw a woman sitting on a stone bench inside the park.

It was midday, but she was dressed as if for a party. Her face was in
her hands, and she was sobbing as if she had just lost everything in the
world.

I make my living with words and pictures, and this one was perfect: the
woman's bright red dress, her perfectly coifed blonde hair, the gray
stone and green leaves, the jarring contrast of beauty and grief.

The woman's head was down, and my camera was ready. It would take only a
second, and she would never know.

I couldn't do it, not even to snap a photo for my eyes only. In this
very public place, it was her private moment, and it could not belong to
me. I longed to console her, but even that feel like a trespass.

I kept on walking, and silently wished her well.

\emph{--- Jil McIntosh}

\begin{center}\rule{0.5\linewidth}{\linethickness}\end{center}

Image

\hypertarget{sideways}{%
\subsection{Sideways}\label{sideways}}

Dear Diary:

Moving to the West Village from the suburbs was a dream that finally
became reality when my children went to college in the early '90s.

As I moved into the tiny apartment, I realized that my beloved antique
iron headboard would only fit in the bedroom if I was willing to live
with the door opening just enough for me to pass through sideways. I
decided I was.

I lived that way for a few years before finally bringing the headboard
to the curb and replacing it with a much smaller one that allowed me to
open the bedroom door all the way.

About six months ago, I heard some banging and scraping outside my
apartment. I opened the door and saw my next-door neighbor dragging my
old headboard out of his apartment.

Apologizing for the noise, he said that he had found it at the curb
years ago and loved it, but that he grown tired of walking into his
bedroom sideways so he was returning it to the street.

\emph{--- Ellen Myers}

\begin{center}\rule{0.5\linewidth}{\linethickness}\end{center}

Image

\hypertarget{checking-in}{%
\subsection{Checking In}\label{checking-in}}

Dear Diary:

My wife and I came to New York in November 2002 for my second New York
City Marathon. We splurged and booked a room at a boutique hotel near
the New York Public Library, where runners board early morning buses
that take them to where the race starts on Staten Island.

We registered at the desk with an assistant manager, who struck us as
the type of well-mannered, middle-age gentleman one might encounter at a
traditional European hotel.

I made conversation by mentioning the huge number of international
runners I had seen. He volunteered that he was from what had been known
as Czechoslovakia. We fell into an easy, extended chat about distance
running.

Eventually, he insisted on personally showing us to the room we had
booked on a lower floor. He seemed intent on continuing our
conversation.

As he pulled our luggage trolley onto the elevator, a twinkle came to
his eyes.

``I bet you don't know the name of the greatest Czech distance runner of
all time,'' he said.

Somehow, my usually unreliable memory jumped to life.

``Um, yeah, Emil Zatopek,'' I stammered.

His face lit up, and he beamed with pride.

After a moment's reflection, he spoke again.

``The room you reserved just isn't right for you,'' he said. ``Allow me
to upgrade you to a larger suite on an upper floor.''

\emph{--- Geoffrey Vincent}

\emph{Read}
\href{http://www.nytimes.com/column/metropolitan-diary}{\emph{all recent
entries}} \emph{and our}
\href{http://www.nytimes.com/2015/10/21/nyregion/how-to-submit-to-metropolitan-diary.html}{\emph{submissions
guidelines}}\emph{. Reach us via email}
\href{mailto:diary@nytimes.com}{\emph{diary@nytimes.com}} \emph{or
follow} \href{https://twitter.com/\#\%21/nytmetro}{\emph{@NYTMetro}}
\emph{on Twitter.}

Illustrations by Agnes Lee

Advertisement

\protect\hyperlink{after-bottom}{Continue reading the main story}

\hypertarget{site-index}{%
\subsection{Site Index}\label{site-index}}

\hypertarget{site-information-navigation}{%
\subsection{Site Information
Navigation}\label{site-information-navigation}}

\begin{itemize}
\tightlist
\item
  \href{https://help.nytimes.com/hc/en-us/articles/115014792127-Copyright-notice}{©~2020~The
  New York Times Company}
\end{itemize}

\begin{itemize}
\tightlist
\item
  \href{https://www.nytco.com/}{NYTCo}
\item
  \href{https://help.nytimes.com/hc/en-us/articles/115015385887-Contact-Us}{Contact
  Us}
\item
  \href{https://www.nytco.com/careers/}{Work with us}
\item
  \href{https://nytmediakit.com/}{Advertise}
\item
  \href{http://www.tbrandstudio.com/}{T Brand Studio}
\item
  \href{https://www.nytimes.com/privacy/cookie-policy\#how-do-i-manage-trackers}{Your
  Ad Choices}
\item
  \href{https://www.nytimes.com/privacy}{Privacy}
\item
  \href{https://help.nytimes.com/hc/en-us/articles/115014893428-Terms-of-service}{Terms
  of Service}
\item
  \href{https://help.nytimes.com/hc/en-us/articles/115014893968-Terms-of-sale}{Terms
  of Sale}
\item
  \href{https://spiderbites.nytimes.com}{Site Map}
\item
  \href{https://help.nytimes.com/hc/en-us}{Help}
\item
  \href{https://www.nytimes.com/subscription?campaignId=37WXW}{Subscriptions}
\end{itemize}
