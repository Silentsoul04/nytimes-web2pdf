Sections

SEARCH

\protect\hyperlink{site-content}{Skip to
content}\protect\hyperlink{site-index}{Skip to site index}

\href{https://www.nytimes.com/section/nyregion}{New York}

\href{https://myaccount.nytimes.com/auth/login?response_type=cookie\&client_id=vi}{}

\href{https://www.nytimes.com/section/todayspaper}{Today's Paper}

\href{/section/nyregion}{New York}\textbar{}Is the Subway Risky? It May
Be Safer Than You Think

\url{https://nyti.ms/3jWXmtb}

\begin{itemize}
\item
\item
\item
\item
\item
\item
\end{itemize}

\href{https://www.nytimes.com/news-event/coronavirus?action=click\&pgtype=Article\&state=default\&region=TOP_BANNER\&context=storylines_menu}{The
Coronavirus Outbreak}

\begin{itemize}
\tightlist
\item
  live\href{https://www.nytimes.com/2020/08/03/world/coronavirus-covid-19.html?action=click\&pgtype=Article\&state=default\&region=TOP_BANNER\&context=storylines_menu}{Latest
  Updates}
\item
  \href{https://www.nytimes.com/interactive/2020/us/coronavirus-us-cases.html?action=click\&pgtype=Article\&state=default\&region=TOP_BANNER\&context=storylines_menu}{Maps
  and Cases}
\item
  \href{https://www.nytimes.com/interactive/2020/science/coronavirus-vaccine-tracker.html?action=click\&pgtype=Article\&state=default\&region=TOP_BANNER\&context=storylines_menu}{Vaccine
  Tracker}
\item
  \href{https://www.nytimes.com/2020/08/02/us/covid-college-reopening.html?action=click\&pgtype=Article\&state=default\&region=TOP_BANNER\&context=storylines_menu}{College
  Reopening}
\item
  \href{https://www.nytimes.com/live/2020/08/03/business/stock-market-today-coronavirus?action=click\&pgtype=Article\&state=default\&region=TOP_BANNER\&context=storylines_menu}{Economy}
\end{itemize}

Advertisement

\protect\hyperlink{after-top}{Continue reading the main story}

Supported by

\protect\hyperlink{after-sponsor}{Continue reading the main story}

\hypertarget{is-the-subway-risky-it-may-be-safer-than-you-think}{%
\section{Is the Subway Risky? It May Be Safer Than You
Think}\label{is-the-subway-risky-it-may-be-safer-than-you-think}}

New studies in Europe and Asia suggest that riding public transportation
is not a major source of transmission for the coronavirus.

\includegraphics{https://static01.nyt.com/images/2020/08/02/nyregion/02nyvirus-subway/merlin_173243304_7c0db049-d11d-4e86-89b8-1fce3576fe5e-articleLarge.jpg?quality=75\&auto=webp\&disable=upscale}

\href{https://www.nytimes.com/by/christina-goldbaum}{\includegraphics{https://static01.nyt.com/images/2019/11/22/reader-center/author-christina-goldbaum/author-christina-goldbaum-thumbLarge.png}}

By \href{https://www.nytimes.com/by/christina-goldbaum}{Christina
Goldbaum}

\begin{itemize}
\item
  Aug. 2, 2020
\item
  \begin{itemize}
  \item
  \item
  \item
  \item
  \item
  \item
  \end{itemize}
\end{itemize}

Five months after the coronavirus outbreak engulfed New York City,
riders are still staying away from public transportation in enormous
numbers, often because they are concerned that sharing enclosed places
with strangers is simply too dangerous.

But the picture emerging in major cities across the world suggests that
public transportation may not be as risky as nervous New Yorkers
believe.

In countries where the pandemic has ebbed, ridership has rebounded in
far greater numbers than in New York City --- yet there have been no
notable superspreader events linked to mass transit, according to a
survey of transportation agencies conducted by The New York Times.

Those findings could be evidence that subways, commuter railways and
buses may not be a significant source of transmission, as long as riders
wear masks and train cars or buses never become as intensely crowded as
they did in pre-pandemic rush hours.

If the risks of mass transit can be addressed, that could have sweeping
implications for many large American cities, particularly New York,
where one of the biggest challenges in a recovery will be coaxing riders
back onto subways, buses and suburban trains --- a vast system that is
the backbone of the region's economy.

When the city shut down in March, over 90 percent of the subway's 5.5
million weekday riders abandoned the system. Even now, as the city has
largely contained the virus and reopened some businesses, ridership is
still just 20 percent of pre-pandemic levels, adding to the financial
strain of New York's transit agency, which relies on fare revenue for 40
percent of its operating budget.

``What we are seeing in other cities makes me optimistic,'' said Toph
Allen, an epidemiologist who co-wrote a report on coronavirus
transmission and public transportation with the Tri-State Transportation
Campaign, a transit advocacy group. ``If you know that you have a
transit system that is functioning in an area where there are no major
outbreaks, you know transit can be safe.''

In Paris, public health authorities conducting contact tracing found
that none of the 386 infection clusters identified between early May and
mid-July were linked to the city's public transportation.

A study of coronavirus clusters in April and May in Austria did not tie
any to public transit. And in Tokyo, where public health authorities
have aggressively traced virus clusters, none have been linked to the
city's famously crowded rail lines.

But public health experts warn that the evidence so far should be
considered with caution. Ridership in other major cities is still well
below pre-pandemic levels, tracing clusters directly to public transit
is difficult, the quality of ventilation systems used to filter air
varies, and the level of threat depends to a high degree on how well a
city has reduced its overall infection rate.

\hypertarget{latest-updates-global-coronavirus-outbreak}{%
\section{\texorpdfstring{\href{https://www.nytimes.com/2020/08/03/world/coronavirus-covid-19.html?action=click\&pgtype=Article\&state=default\&region=MAIN_CONTENT_1\&context=storylines_live_updates}{Latest
Updates: Global Coronavirus
Outbreak}}{Latest Updates: Global Coronavirus Outbreak}}\label{latest-updates-global-coronavirus-outbreak}}

Updated 2020-08-04T05:55:16.339Z

\begin{itemize}
\tightlist
\item
  \href{https://www.nytimes.com/2020/08/03/world/coronavirus-covid-19.html?action=click\&pgtype=Article\&state=default\&region=MAIN_CONTENT_1\&context=storylines_live_updates\#link-4547638f}{Fauci
  defends Birx after she is criticized by Trump.}
\item
  \href{https://www.nytimes.com/2020/08/03/world/coronavirus-covid-19.html?action=click\&pgtype=Article\&state=default\&region=MAIN_CONTENT_1\&context=storylines_live_updates\#link-15e7f995}{Trump
  derides Democrats as lawmakers and administration officials try to
  break stimulus impasse.}
\item
  \href{https://www.nytimes.com/2020/08/03/world/coronavirus-covid-19.html?action=click\&pgtype=Article\&state=default\&region=MAIN_CONTENT_1\&context=storylines_live_updates\#link-e5a2cda}{The
  deadline for 2020 census counting has been moved up by a month.}
\end{itemize}

\href{https://www.nytimes.com/2020/08/03/world/coronavirus-covid-19.html?action=click\&pgtype=Article\&state=default\&region=MAIN_CONTENT_1\&context=storylines_live_updates}{See
more updates}

More live coverage:
\href{https://www.nytimes.com/live/2020/08/03/business/stock-market-today-coronavirus?action=click\&pgtype=Article\&state=default\&region=MAIN_CONTENT_1\&context=storylines_live_updates}{Markets}

``There are so many other factors that go into levels of risk and how
you assess risk,'' said Dr. Michael Reid, an assistant professor at the
University of California, San Francisco School of Medicine and a
contact-tracing expert. ``They are not equal comparisons.''

In fact, state and city officials have been unable to determine whether
mass transit in New York contributed to the surge in March and April
that devastated the city, killing more than 20,000 people.

The outbreak has exacted an especially devastating toll on transit
workers. To date, over 4,000 have tested positive and 131 workers have
died from the virus --- nearly 90 percent of whom worked for the
division that runs the city's subways and buses.

For much of that time, riders were not required to wear masks, and the
infection rate in the city was much higher than it is today, likely
making public transportation a riskier venue. (One study at M.I.T.
purported to show that the subway was a superspreader early in the
pandemic, but its methodology was widely disputed.)

Still, some public health experts believe the experiences of other
cities offer a blueprint for how to minimize the potential for
transmission on public transit systems.

Among the range of urban activities, the experts say, riding the subway
is probably riskier than walking outdoors but safer than indoor dining.

The low infection rates on some public transportation systems can be
attributed, in part, to measures transit agencies have adopted,
including mandating face masks; disinfecting trains and buses; and
ramping up service and asking businesses to stagger work hours to reduce
rush-hour crowding.

New York officials are trying to balance two goals: drawing as many
riders back as possible while also avoiding sardine-can crowding at rush
hour. They have appealed to business leaders to have employees start at
different hours, though the pressure on the system has eased notably
since the shift toward working from home is expected to last for months,
if not longer.

\includegraphics{https://static01.nyt.com/images/2020/07/28/nyregion/00nyvirus-mta2/merlin_172484904_f08ae0f4-4145-4878-8899-53d16669a0f3-articleLarge.jpg?quality=75\&auto=webp\&disable=upscale}

``Each of these things layers one on top of the other to make things
safer,'' said Dr. Don Milton, an environmental health researcher and
aerosol transmission expert at the University of Maryland.

The nature of how people use public transit also may help explain why
potential exposure levels might not be as high as some riders believe.

People tend to stay on trains or buses for relatively short amounts of
time, compared with a day's work in an office or an outing to a bar to
see friends. Riders tend not to talk on the train, reducing the amount
of aerosols they release. In many cities, lockdown orders and new
work-from-home norms have minimized crowds on trains, making it easier
to keep some social distance.

Riders seem to be wearing masks and adhering to new guidelines,
officials said.

``We were pleasantly surprised that Berliners accepted it so quickly,''
said Jannes Schwentu, a spokesman for the Berliner Verkehrsbetriebe,
which operates Berlin's subway and buses, referring to mask compliance.

Image

Public health experts say that mandatory mask rules have helped to keep
the virus from spreading on public transit. Here, commuters ride a train
in Hong Kong.Credit...Anthony Wallace/Agence France-Presse --- Getty
Images

In New York, transit officials say that a recent observational study of
over 220,000 riders found that over 90 percent were wearing masks. The
transit agency has handed out free masks to passengers.

Though some veteran riders might be surprised, the subway system also
benefits from a robust ventilation system that is effective at removing
viral particles from the air.

In New York's subway trains, transit officials say, the filtered air
that circulates through a car is replaced with fresh air at least 18
times an hour. That is a much higher than the recommended air-exchange
rates in restaurants, where recycled air is replaced eight to 12 times
per hour, or in offices, where it is replaced six to eight times an
hour.

This sharply reduces the chances of a superspreader event on trains, as
long as they do not become overly crowded, said Linsey Marr, an expert
on the airborne transmission of viruses at Virginia Tech.

But once too many people pack a train, the ability to provide proper
ventilation to prevent the spread of viral aerosols diminishes
significantly. When riders are standing shoulder to shoulder, any viral
particles a sick passenger exhales could be readily inhaled by another
passenger --- which is possible even if both are wearing masks.

\href{https://www.nytimes.com/news-event/coronavirus?action=click\&pgtype=Article\&state=default\&region=MAIN_CONTENT_3\&context=storylines_faq}{}

\hypertarget{the-coronavirus-outbreak-}{%
\subsubsection{The Coronavirus Outbreak
›}\label{the-coronavirus-outbreak-}}

\hypertarget{frequently-asked-questions}{%
\paragraph{Frequently Asked
Questions}\label{frequently-asked-questions}}

Updated August 3, 2020

\begin{itemize}
\item ~
  \hypertarget{im-a-small-business-owner-can-i-get-relief}{%
  \paragraph{I'm a small-business owner. Can I get
  relief?}\label{im-a-small-business-owner-can-i-get-relief}}

  \begin{itemize}
  \tightlist
  \item
    The
    \href{https://www.nytimes.com/article/small-business-loans-stimulus-grants-freelancers-coronavirus.html?action=click\&pgtype=Article\&state=default\&region=MAIN_CONTENT_3\&context=storylines_faq}{stimulus
    bills enacted in March} offer help for the millions of American
    small businesses. Those eligible for aid are businesses and
    nonprofit organizations with fewer than 500 workers, including sole
    proprietorships, independent contractors and freelancers. Some
    larger companies in some industries are also eligible. The help
    being offered, which is being managed by the Small Business
    Administration, includes the Paycheck Protection Program and the
    Economic Injury Disaster Loan program. But lots of folks have
    \href{https://www.nytimes.com/interactive/2020/05/07/business/small-business-loans-coronavirus.html?action=click\&pgtype=Article\&state=default\&region=MAIN_CONTENT_3\&context=storylines_faq}{not
    yet seen payouts.} Even those who have received help are confused:
    The rules are draconian, and some are stuck sitting on
    \href{https://www.nytimes.com/2020/05/02/business/economy/loans-coronavirus-small-business.html?action=click\&pgtype=Article\&state=default\&region=MAIN_CONTENT_3\&context=storylines_faq}{money
    they don't know how to use.} Many small-business owners are getting
    less than they expected or
    \href{https://www.nytimes.com/2020/06/10/business/Small-business-loans-ppp.html?action=click\&pgtype=Article\&state=default\&region=MAIN_CONTENT_3\&context=storylines_faq}{not
    hearing anything at all.}
  \end{itemize}
\item ~
  \hypertarget{what-are-my-rights-if-i-am-worried-about-going-back-to-work}{%
  \paragraph{What are my rights if I am worried about going back to
  work?}\label{what-are-my-rights-if-i-am-worried-about-going-back-to-work}}

  \begin{itemize}
  \tightlist
  \item
    Employers have to provide
    \href{https://www.osha.gov/SLTC/covid-19/standards.html}{a safe
    workplace} with policies that protect everyone equally.
    \href{https://www.nytimes.com/article/coronavirus-money-unemployment.html?action=click\&pgtype=Article\&state=default\&region=MAIN_CONTENT_3\&context=storylines_faq}{And
    if one of your co-workers tests positive for the coronavirus, the
    C.D.C.} has said that
    \href{https://www.cdc.gov/coronavirus/2019-ncov/community/guidance-business-response.html}{employers
    should tell their employees} -\/- without giving you the sick
    employee's name -\/- that they may have been exposed to the virus.
  \end{itemize}
\item ~
  \hypertarget{should-i-refinance-my-mortgage}{%
  \paragraph{Should I refinance my
  mortgage?}\label{should-i-refinance-my-mortgage}}

  \begin{itemize}
  \tightlist
  \item
    \href{https://www.nytimes.com/article/coronavirus-money-unemployment.html?action=click\&pgtype=Article\&state=default\&region=MAIN_CONTENT_3\&context=storylines_faq}{It
    could be a good idea,} because mortgage rates have
    \href{https://www.nytimes.com/2020/07/16/business/mortgage-rates-below-3-percent.html?action=click\&pgtype=Article\&state=default\&region=MAIN_CONTENT_3\&context=storylines_faq}{never
    been lower.} Refinancing requests have pushed mortgage applications
    to some of the highest levels since 2008, so be prepared to get in
    line. But defaults are also up, so if you're thinking about buying a
    home, be aware that some lenders have tightened their standards.
  \end{itemize}
\item ~
  \hypertarget{what-is-school-going-to-look-like-in-september}{%
  \paragraph{What is school going to look like in
  September?}\label{what-is-school-going-to-look-like-in-september}}

  \begin{itemize}
  \tightlist
  \item
    It is unlikely that many schools will return to a normal schedule
    this fall, requiring the grind of
    \href{https://www.nytimes.com/2020/06/05/us/coronavirus-education-lost-learning.html?action=click\&pgtype=Article\&state=default\&region=MAIN_CONTENT_3\&context=storylines_faq}{online
    learning},
    \href{https://www.nytimes.com/2020/05/29/us/coronavirus-child-care-centers.html?action=click\&pgtype=Article\&state=default\&region=MAIN_CONTENT_3\&context=storylines_faq}{makeshift
    child care} and
    \href{https://www.nytimes.com/2020/06/03/business/economy/coronavirus-working-women.html?action=click\&pgtype=Article\&state=default\&region=MAIN_CONTENT_3\&context=storylines_faq}{stunted
    workdays} to continue. California's two largest public school
    districts --- Los Angeles and San Diego --- said on July 13, that
    \href{https://www.nytimes.com/2020/07/13/us/lausd-san-diego-school-reopening.html?action=click\&pgtype=Article\&state=default\&region=MAIN_CONTENT_3\&context=storylines_faq}{instruction
    will be remote-only in the fall}, citing concerns that surging
    coronavirus infections in their areas pose too dire a risk for
    students and teachers. Together, the two districts enroll some
    825,000 students. They are the largest in the country so far to
    abandon plans for even a partial physical return to classrooms when
    they reopen in August. For other districts, the solution won't be an
    all-or-nothing approach.
    \href{https://bioethics.jhu.edu/research-and-outreach/projects/eschool-initiative/school-policy-tracker/}{Many
    systems}, including the nation's largest, New York City, are
    devising
    \href{https://www.nytimes.com/2020/06/26/us/coronavirus-schools-reopen-fall.html?action=click\&pgtype=Article\&state=default\&region=MAIN_CONTENT_3\&context=storylines_faq}{hybrid
    plans} that involve spending some days in classrooms and other days
    online. There's no national policy on this yet, so check with your
    municipal school system regularly to see what is happening in your
    community.
  \end{itemize}
\item ~
  \hypertarget{is-the-coronavirus-airborne}{%
  \paragraph{Is the coronavirus
  airborne?}\label{is-the-coronavirus-airborne}}

  \begin{itemize}
  \tightlist
  \item
    The coronavirus
    \href{https://www.nytimes.com/2020/07/04/health/239-experts-with-one-big-claim-the-coronavirus-is-airborne.html?action=click\&pgtype=Article\&state=default\&region=MAIN_CONTENT_3\&context=storylines_faq}{can
    stay aloft for hours in tiny droplets in stagnant air}, infecting
    people as they inhale, mounting scientific evidence suggests. This
    risk is highest in crowded indoor spaces with poor ventilation, and
    may help explain super-spreading events reported in meatpacking
    plants, churches and restaurants.
    \href{https://www.nytimes.com/2020/07/06/health/coronavirus-airborne-aerosols.html?action=click\&pgtype=Article\&state=default\&region=MAIN_CONTENT_3\&context=storylines_faq}{It's
    unclear how often the virus is spread} via these tiny droplets, or
    aerosols, compared with larger droplets that are expelled when a
    sick person coughs or sneezes, or transmitted through contact with
    contaminated surfaces, said Linsey Marr, an aerosol expert at
    Virginia Tech. Aerosols are released even when a person without
    symptoms exhales, talks or sings, according to Dr. Marr and more
    than 200 other experts, who
    \href{https://academic.oup.com/cid/article/doi/10.1093/cid/ciaa939/5867798}{have
    outlined the evidence in an open letter to the World Health
    Organization}.
  \end{itemize}
\end{itemize}

Hong Kong is one city where public transit ridership is still lower than
before the pandemic, and it has not ``seen a big outbreak associated
with public transit,'' said David Hui, the director of the Stanley Ho
Center for Emerging Infectious Diseases at the Chinese University of
Hong Kong.

But, he added: ``If not for the work-from-home measure, both buses and
the subway would be full of people. In that case, I believe there could
have been a serious outbreak.''

In some places, ridership has rebounded more so than in New York, but
none have had to grapple with overflowing public transit and how that
could test their ability to keep the virus at bay.

In Beijing, subway ridership has risen to 59 percent of pre-pandemic
levels; in Tokyo, Metro ridership has increased to 63 percent; in
Berlin, ridership on buses and subways is between 60 to 70 percent of
normal rates; and in Paris, ridership on the Metro has returned to 45
percent of usual levels.

Image

A bus in Berlin, where transportation officials said riders had
generally complied with rules to wear masks.~Credit...Fabrizio
Bensch/Reuters

``I am more vigilant in the Metro and careful not touching the bar or
sitting on seats,'' said Alain Raphael, 28, an engineer in a tech
company in Paris. ``I am less confident in bars, cafes and restaurants
than riding the Metro.''

So far in New York State, where contact-tracing efforts are not as
robust as in European and Asian countries, public health officials have
not linked any new clusters to public transportation, according to state
and city officials.

Contact-tracing experts warn that tracking an infection cluster to
public transportation is particularly challenging because the chances of
infected people remembering the precise train cars they were riding is
unlikely and reaching those who were in that same car is nearly
impossible.

``Transit is much more anonymous and relatively fleeting,'' said Crystal
Watson, a senior scholar at the Johns Hopkins Center for Health
Security.

In the months since the height of the outbreak in New York, the
Metropolitan Transportation Authority, which runs the city's subway and
buses, has invested hundreds of millions of dollars on the daily
disinfection of train cars, distributed over a million masks to riders,
and started public service campaigns encouraging riders to maintain
social distance.

These efforts are as much about swaying public perceptions and regaining
the confidence of commuters as they are about safeguarding public
health, officials said.

``There is both a substantive public health goal and there's a messaging
and assuring goal as well,'' Patrick J. Foye, the M.T.A.'s chairman,
said.

Dr. Joan Stroud, 61, a family medicine doctor at N.Y.U. Langone Brooklyn
Heights Medical, started driving from her home in Bedford-Stuyvesant,
Brooklyn, to work in early May rather than taking the subway.

``New York City trains were already filthy,'' she said. ``I wasn't going
to get on one every day during a major wave of infection.''

But a month ago, she got back on the subway, which provides a faster
commute, and has been impressed with the system.

``The trains are as clean as I've ever seen them,'' she said.

Reporting was contributed by Théophile Larcher from Paris, Bella Huang
from Hong Kong, Melissa Eddy from Berlin and Makiko Inoue from Tokyo.

Advertisement

\protect\hyperlink{after-bottom}{Continue reading the main story}

\hypertarget{site-index}{%
\subsection{Site Index}\label{site-index}}

\hypertarget{site-information-navigation}{%
\subsection{Site Information
Navigation}\label{site-information-navigation}}

\begin{itemize}
\tightlist
\item
  \href{https://help.nytimes.com/hc/en-us/articles/115014792127-Copyright-notice}{©~2020~The
  New York Times Company}
\end{itemize}

\begin{itemize}
\tightlist
\item
  \href{https://www.nytco.com/}{NYTCo}
\item
  \href{https://help.nytimes.com/hc/en-us/articles/115015385887-Contact-Us}{Contact
  Us}
\item
  \href{https://www.nytco.com/careers/}{Work with us}
\item
  \href{https://nytmediakit.com/}{Advertise}
\item
  \href{http://www.tbrandstudio.com/}{T Brand Studio}
\item
  \href{https://www.nytimes.com/privacy/cookie-policy\#how-do-i-manage-trackers}{Your
  Ad Choices}
\item
  \href{https://www.nytimes.com/privacy}{Privacy}
\item
  \href{https://help.nytimes.com/hc/en-us/articles/115014893428-Terms-of-service}{Terms
  of Service}
\item
  \href{https://help.nytimes.com/hc/en-us/articles/115014893968-Terms-of-sale}{Terms
  of Sale}
\item
  \href{https://spiderbites.nytimes.com}{Site Map}
\item
  \href{https://help.nytimes.com/hc/en-us}{Help}
\item
  \href{https://www.nytimes.com/subscription?campaignId=37WXW}{Subscriptions}
\end{itemize}
