\href{/section/opinion}{Opinion}\textbar{}Baseball Is Playing for Its
Life, and Ours

\url{https://nyti.ms/3hXzRye}

\begin{itemize}
\item
\item
\item
\item
\item
\end{itemize}

\href{https://www.nytimes.com/spotlight/at-home?action=click\&pgtype=Article\&state=default\&region=TOP_BANNER\&context=at_home_menu}{At
Home}

\begin{itemize}
\tightlist
\item
  \href{https://www.nytimes.com/2020/07/28/books/time-for-a-literary-road-trip.html?action=click\&pgtype=Article\&state=default\&region=TOP_BANNER\&context=at_home_menu}{Take:
  A Literary Road Trip}
\item
  \href{https://www.nytimes.com/2020/07/29/magazine/bored-with-your-home-cooking-some-smoky-eggplant-will-fix-that.html?action=click\&pgtype=Article\&state=default\&region=TOP_BANNER\&context=at_home_menu}{Cook:
  Smoky Eggplant}
\item
  \href{https://www.nytimes.com/2020/07/27/travel/moose-michigan-isle-royale.html?action=click\&pgtype=Article\&state=default\&region=TOP_BANNER\&context=at_home_menu}{Look
  Out: For Moose}
\item
  \href{https://www.nytimes.com/interactive/2020/at-home/even-more-reporters-editors-diaries-lists-recommendations.html?action=click\&pgtype=Article\&state=default\&region=TOP_BANNER\&context=at_home_menu}{Explore:
  Reporters' Obsessions}
\end{itemize}

\includegraphics{https://static01.nyt.com/images/2020/08/02/opinion/02glanvilleWeb/merlin_174959136_6782ff80-a02a-4f91-a997-a37792efcefc-articleLarge.jpg?quality=75\&auto=webp\&disable=upscale}

Sections

\protect\hyperlink{site-content}{Skip to
content}\protect\hyperlink{site-index}{Skip to site index}

\href{/section/opinion}{Opinion}

\hypertarget{baseball-is-playing-for-its-life-and-ours}{%
\section{Baseball Is Playing for Its Life, and
Ours}\label{baseball-is-playing-for-its-life-and-ours}}

An outbreak of infections soon after reopening has struck a blow at more
than just the Marlins.

The Miami Marlins shortstop Miguel Rojas leaping over Bryce Harper of
the Philadelphia Phillies during a game on July 25.Credit...Chris
Szagola/Associated Press

Supported by

\protect\hyperlink{after-sponsor}{Continue reading the main story}

By \href{https://www.nytimes.com/by/doug-glanville}{Doug Glanville}

Mr. Glanville is a former Major League Baseball player and a sports
commentator.

\begin{itemize}
\item
  Aug. 2, 2020
\item
  \begin{itemize}
  \item
  \item
  \item
  \item
  \item
  \end{itemize}
\end{itemize}

In 1996, my rookie year in the major leagues with the Chicago Cubs, our
manager Jim Riggleman often reminded us that we should play for the name
on the front of our uniforms, not the name on the back. It was a noble
idea --- team before self --- reinforced by the power big league
athletes can feel when they realize they are representing entire cities,
states, even countries, and taking a place in a long history of the
sport they are so passionate about.

Last week, we learned once again that Covid-19 does not care about such
loyalties. Only a few days after the Major League Baseball season
opened, the Miami Marlins were dealing with an outbreak, with the number
of players and staff infected with the coronavirus jumping from four to
17 in a matter of days. Quick action was taken: Games were postponed or
canceled and the league announced it would require each team to have a
``compliance officer'' to enforce health safety rules.

Then Marlins infections hit 20; more games were postponed and canceled.
ESPN reported that the league's commissioner, Rob Manfred, told the
players' union executive director Tony Clark that if safety rules were
not more stringently followed, shutting down the season could become
possible. Then the St. Louis Cardinals reported positive tests for one
player and three staff members. On the record, Mr. Manfred told ESPN's
Karl Ravech that ``there is no reason to quit now''and that the
situation was ``manageable.''

Baseball teams were in the midst of spring training on March 11 when the
N.B.A. halted its season because of the pandemic (the basketball league
resumed games without fans in attendance on July 30), unleashing a
ripple effect in the sports world and beyond. Baseball shut down soon
after and quickly went to work trying to find the right time and way to
get back on the field safely.

The usual rancorous labor tensions made for a protracted timeline as
they fought to agree on the terms of re-engagement, but eventually, it
was ``play ball.'' Unlike the N.B.A. and some other leagues, baseball
did not create a ``bubble'' for its players, given the real estate
required and the size of these traveling teams. They were allowed to
return home after games, undoubtedly increasing their chances of
exposure and the risk for others beyond the clubhouse. Players and staff
have taken precautions to protect their families and some players have
even opted out of the season entirely. Still, day by day, the league is
scrambling to contain the opponent that had put them out of business for
so long, and is threatening to do so again.

Most Americans, even those who are not fans, are watching professional
sports closely, not only to see our favorite team compete for a
championship, but to see these institutions, whose power grew out of our
collective imagination, fighting to win a real-time battle over this
threat to our current existence. Even when humility tells us that
triumph is simply prolonging our ability to safely play another day.

As usual, baseball is never just about baseball. It is called our
national pastime for a reason. The virus has dealt a serious blow not
just to the league's operation but, in some sense, to the nation itself:
Our confidence has been shaken, our helplessness reinforced, our anxiety
and caution ramped up yet again. Baseball was entering the war against
the pandemic, and the world was positioned to benefit from the
information that would be gathered. The league, armed to the teeth with
power and privilege, access to testing, cash flow, precision data
collection, and high-powered, lower-risk athletes playing outdoors, was
supposed to prevail.

Baseball's success, then, will be our success; its failure, our failure.
We want to know we can win this fight, without being curled up in a ball
while waiting for a vaccine, even though we quietly understand that many
variables that give these sports advantages in this fight are not fully
available to the vast majority of people. Still, we hope that baseball's
eventual victory will wash over us as one.

I remember my time on the Phillies in wake of the Sept. 11 attacks. As
both a player and a player representative, I wondered how we would
justify coming back to play at all. In the grand scheme of things, we
were only playing a game. We were nonessential on paper. But when we did
return, we found that some of what we recaptured \emph{was} essential
--- to the uplift of our spirit and to the restoration of the society we
created with our inspiration and our passion for fair competition and
gamesmanship. It had become larger than the scoreboard.

We know this game is still in progress. As the players and the league
grapple with the outbreaks, and perhaps more infections to come, they
have been forced to reconsider all of their protocols while still trying
to keep the game recognizable and fair. As in any game, we do not know
if they will succeed, but we can take some comfort from the fact that
baseball is willing to find out for us.

We hope to come out of this season with a new world champion, the
fulfillment of a baseball season completed. We hope that we will one day
retire this virus, with an asterisk, to our scariest of histories, and
that baseball and other sports will help get us there by aggressively
gathering information about the risks we are all facing. In the end,
this will be prove to be more valuable than anything normalcy can
provide.

In the game we are playing now, there are no names on the front or the
back of our jerseys. We are playing to survive.

Doug Glanville
(\href{https://twitter.com/dougglanville}{@dougglanville}), a former
Major League Baseball player, an ESPN baseball analyst and the author of
``The Game From Where I Stand.''

\emph{The Times is committed to publishing}
\href{https://www.nytimes.com/2019/01/31/opinion/letters/letters-to-editor-new-york-times-women.html}{\emph{a
diversity of letters}} \emph{to the editor. We'd like to hear what you
think about this or any of our articles. Here are some}
\href{https://help.nytimes.com/hc/en-us/articles/115014925288-How-to-submit-a-letter-to-the-editor}{\emph{tips}}\emph{.
And here's our email:}
\href{mailto:letters@nytimes.com}{\emph{letters@nytimes.com}}\emph{.}

\emph{Follow The New York Times Opinion section on}
\href{https://www.facebook.com/nytopinion}{\emph{Facebook}}\emph{,}
\href{http://twitter.com/NYTOpinion}{\emph{Twitter (@NYTopinion)}}
\emph{and}
\href{https://www.instagram.com/nytopinion/}{\emph{Instagram}}\emph{.}

Advertisement

\protect\hyperlink{after-bottom}{Continue reading the main story}

\hypertarget{site-index}{%
\subsection{Site Index}\label{site-index}}

\hypertarget{site-information-navigation}{%
\subsection{Site Information
Navigation}\label{site-information-navigation}}

\begin{itemize}
\tightlist
\item
  \href{https://help.nytimes.com/hc/en-us/articles/115014792127-Copyright-notice}{©~2020~The
  New York Times Company}
\end{itemize}

\begin{itemize}
\tightlist
\item
  \href{https://www.nytco.com/}{NYTCo}
\item
  \href{https://help.nytimes.com/hc/en-us/articles/115015385887-Contact-Us}{Contact
  Us}
\item
  \href{https://www.nytco.com/careers/}{Work with us}
\item
  \href{https://nytmediakit.com/}{Advertise}
\item
  \href{http://www.tbrandstudio.com/}{T Brand Studio}
\item
  \href{https://www.nytimes.com/privacy/cookie-policy\#how-do-i-manage-trackers}{Your
  Ad Choices}
\item
  \href{https://www.nytimes.com/privacy}{Privacy}
\item
  \href{https://help.nytimes.com/hc/en-us/articles/115014893428-Terms-of-service}{Terms
  of Service}
\item
  \href{https://help.nytimes.com/hc/en-us/articles/115014893968-Terms-of-sale}{Terms
  of Sale}
\item
  \href{https://spiderbites.nytimes.com}{Site Map}
\item
  \href{https://help.nytimes.com/hc/en-us}{Help}
\item
  \href{https://www.nytimes.com/subscription?campaignId=37WXW}{Subscriptions}
\end{itemize}
