Sections

SEARCH

\protect\hyperlink{site-content}{Skip to
content}\protect\hyperlink{site-index}{Skip to site index}

\href{https://myaccount.nytimes.com/auth/login?response_type=cookie\&client_id=vi}{}

\href{https://www.nytimes.com/section/todayspaper}{Today's Paper}

\href{/section/opinion}{Opinion}\textbar{}Why Did Hong Kong Delay Its
Election --- by a Year?

\url{https://nyti.ms/2DvMthi}

\begin{itemize}
\item
\item
\item
\item
\item
\end{itemize}

Advertisement

\protect\hyperlink{after-top}{Continue reading the main story}

\href{/section/opinion}{Opinion}

Supported by

\protect\hyperlink{after-sponsor}{Continue reading the main story}

\hypertarget{why-did-hong-kong-delay-its-election--by-a-year}{%
\section{Why Did Hong Kong Delay Its Election --- by a
Year?}\label{why-did-hong-kong-delay-its-election--by-a-year}}

The government blames the pandemic. More likely, it was afraid to lose.

By Fernando Cheung

Mr. Cheung is a member of the Hong Kong Legislative Council.

\begin{itemize}
\item
  Aug. 2, 2020
\item
  \begin{itemize}
  \item
  \item
  \item
  \item
  \item
  \end{itemize}
\end{itemize}

\includegraphics{https://static01.nyt.com/images/2020/08/04/opinion/04cheung/merlin_175103946_901ecb4f-473c-4826-8ac7-a4771835ac6b-articleLarge.jpg?quality=75\&auto=webp\&disable=upscale}

HONG KONG --- And now, it's election fraud.

The Chinese Communist Party's onslaught against the rights and the
freedoms of the Hong Kong people continues. On June 30, it imposed on
the city a new
\href{https://www.gld.gov.hk/egazette/pdf/20202448e/egn2020244872.pdf}{national
security law}. Within hours the police arrested people simply for
possessing banners that said
``\href{https://twitter.com/hkpoliceforce/status/1278201222457987073}{Hong
Kong Independence}.''

On Thursday, the Hong Kong authorities
\href{https://www.nytimes.com/2020/07/29/world/asia/hong-kong-arrests-security-law.html}{disqualified
12 candidates from the pro-democracy camp}, including four sitting
legislators, from running in the election for the Legislative Council,
known as LegCo, scheduled for early September: They questioned the
candidates' sincerity in pledging allegiance to the government.

An official government statement listed
``\href{https://www.info.gov.hk/gia/general/202007/30/P2020073000481.htm}{expressing
an objection in principle}'' to the new national security law as one of
the grounds for disqualification --- adding, ``There is no question of
any political censorship, restriction of the freedom of speech or
deprivation of the right to stand for elections as alleged by some
members of the community.''

And then on Friday, the Hong Kong authorities announced that the LegCo
election
\href{https://www.nytimes.com/2020/07/31/world/asia/hong-kong-election-delayed.html}{would
be delayed by a year}.

They cited the coronavirus pandemic as an excuse, but the truth is that
they are afraid their camp would lose the race.

Why else postpone the election so early before voting day? Or for so
long? Over the past few months, elections were successfully held in
\href{https://www.nytimes.com/2020/04/15/world/asia/south-korea-election.html}{South
Korea},
\href{https://www.nytimes.com/2020/07/05/world/asia/tokyo-governor-election.html}{Tokyo}
and
\href{https://www.nytimes.com/2020/07/10/world/asia/singapore-election-results.html}{Singapore}
in the middle of coronavirus outbreaks.

The pro-government camp was humiliated in district council elections in
November, when the opposition won
\href{https://www.nytimes.com/2019/11/24/world/asia/hong-kong-election-results.html}{86
percent of contested seats.} In mid-July more than 600,000 people turned
up
for\href{https://hongkongfp.com/2020/07/14/serious-provocation-beijing-blasts-hong-kong-democrat-primaries-after-initial-results-reveal/}{informal
primaries for the opposition camp} --- which Beijing's representatives
in the city later called ``a serious provocation to the current election
system.'' Not wanting to lose another election, the pro-government
forces have, in effect, canceled it.

And now this postponed election creates a dangerous legislative void ---
as well as a gutting dilemma for some of us who are pro-democracy
members of LegCo.

Hong Kong's election cycle is fixed: Under
\href{https://www.basiclaw.gov.hk/en/basiclawtext/images/basiclaw_full_text_en.pdf}{the
Basic Law}, the city's mini-constitution, legislative elections are to
be held every four years in September. The embattled chief executive,
Carrie Lam, has conceded that this delay does not conform to the law ---
and so she
\href{https://www.info.gov.hk/gia/general/202007/31/P2020073101081.htm?fontSize=1}{is
deferring to the Chinese government in Beijing} to decide how a
provisional Legislature here should operate.

It is not known whether that body will function as LegCo usually does,
or if it will meet only in emergency situations. Even if LegCo's current
session is simply extended, its composition is now unclear: Will the
sitting legislators who were disqualified from contesting the next
election be allowed to continue to serve?

I am one of the 24 members of LegCo from the pro-democracy camp, out of
a total of 70 legislators.

Throughout the years, our camp --- which comprises different parties
with different views, though all committed to democratic rights and
freedoms --- has received a majority of the popular vote for the seats
decided by direct suffrage. But the Legislature's design, which reserves
\href{https://www.reo.gov.hk/en/voter/FC.htm}{35 seats for special
interest groups} --- many by now dominated or co-opted by pro-Beijing
parties --- has ensured that nonetheless we are a minority.

During LegCo's current term, the government had already disqualified
\href{https://www.nytimes.com/2016/11/08/world/asia/china-hong-kong-sixtus-leung-yau-wai-ching.html}{a
total} of
\href{https://www.nytimes.com/2017/07/14/world/asia/hong-kong-court-pro-democracy-lawmakers.html}{six
elected pro-democracy legislators}, essentially arguing that their
allegiance to the idea that Hong Kong is an integral part of China was
in doubt.

If four more pro-democracy members of LegCo are ousted from the
provisional Legislature, we will be reduced to less than one-third of
the seats --- the threshold for vetoing major bills, such as changes in
the election system or decisions to impeach legislators.

So what should pro-democracy legislators do?

Do we boycott the interim Legislature in protest or in anticipation that
if we participate, we will be run roughshod over --- and adding our
unwilling imprimatur to laws we oppose?

Or do we participate in a sham and do our best to stand our ground,
knowing that if we don't, grievous laws will be passed for sure?

In the lead-up to Hong Kong's handover from Britain to China in 1997, a
provisional Legislature was established to transition away from
colonial-era institutions. Many saw it as
\href{https://www.hrw.org/legacy/press/pro-legi.htm}{undemocratic} ---
its
\href{https://www.scmp.com/news/hong-kong/education-community/article/2012978/explained-how-hong-kongs-legislative-council-has}{members
were selected by a Beijing-appointed committee} --- and the
pro-democracy camp at the time
\href{http://www.ipsnews.net/1996/12/hong-kong-two-legislatures-one-in-hong-kong-one-in-shenzhen/}{refused
to take par}t.

Within a year and a half, that provisional LegCo had passed laws that
\href{https://www.justicecentre.org.hk/framework/uploads/2018/11/HKUPR-Coalition-Fact-Sheet-Freedom-of-Assembly-Rights-and-Public-Order-Ordinance.pdf}{restricted
freedom of assembly} and
\href{https://hongkongfp.com/2018/07/19/explainer-hong-kong-seeking-ban-pro-independence-party-using-existing-national-security-laws/}{freedom
of association}, and it had
\href{https://www.legco.gov.hk/yr98-99/chinese/hc/papers/hc2711-m1-ec.pdf}{repealed
a law granting collective bargaining powers to trade unions}. In 1997,
it also passed
\href{https://www.legco.gov.hk/general/chinese/procedur/companion/chapter_3/mcp-part1-ch3-n11-ce.pdf}{the
Legislative Council Ordinance}, which helped create the unfair
structural design of LegCo today.

What more evils will this next provisional Legislature do to Hong Kong?
How can the Chinese Communist Party be prevented from passing laws to
manipulate future elections here --- perhaps even allowing people on the
mainland to vote in them?

Beijing's overarching intentions with Hong Kong are clear, and it's also
clear by now that the Hong Kong government is doing nothing but
Beijing's bidding.

Last week, the police arrested four students, ages 16 to 21,
\href{https://www.scmp.com/news/hong-kong/law-and-crime/article/3095240/least-three-core-members-hong-kong-pro-independence}{from
a disbanded pro-independence group}; under the new national security
law, they could face a life sentence. On Friday, the Hong Kong
authorities
\href{https://www.nytimes.com/reuters/2020/07/31/world/asia/31reuters-hongkong-security-exiles.html?searchResultPosition=9}{issued
arrest warrants for six activists abroad}, including one American
citizen.

Academics affiliated with the democracy movement are
\href{https://www.nytimes.com/2020/07/28/world/asia/benny-tai-hong-kong-university.html?campaign_id=7\&emc=edit_MBAE_p_20200728\&instance_id=20720\&nl=morning-briefing\&regi_id=65413713\&section=whatElse\&segment_id=34578\&te=1\&user_id=bd32fbf008e5183a7928ed61}{being
sacked} by
\href{https://www.scmp.com/news/hong-kong/politics/article/3046632/occupy-ringleader-shiu-ka-chun-accuses-hong-kong-university}{their
universities}.
\href{https://www.scmp.com/news/hong-kong/law-and-crime/article/3092957/hong-kong-media-tycoon-jimmy-lai-and-12-others-face}{Independent
media} outlets are hounded. A popular satirical TV show produced by Hong
Kong's public broadcaster
\href{https://hongkongfp.com/2020/05/19/hong-kong-public-broadcaster-axes-satirical-show-hours-after-govt-demands-apology-for-insulting-police/}{was
terminated for mocking the police}.

Officials are talking about
\href{https://www.scmp.com/news/hong-kong/education/article/3095434/hong-kong-national-security-law-schools-get-new-teaching}{revising
the management of schools and the curriculum} to promote patriotism and
a sense of national identity.

Every way I turn, I see red lines being drawn. Anyone who dares to step
over one will be heavily punished.

The Chinese Communist Party is well aware of
\href{https://www.state.gov/on-the-postponement-of-hong-kongs-legislative-council-elections/}{the
international outcry} over what it is doing to Hong Kong. No matter; it
presses on. Is that about saving face? Is China truly insecure about its
national security? Does it want to change the world order? I can't tell.

Whatever the motives, Hong Kong has become a battlefield for a contest
between much larger forces, and the immediate casualties are the rule of
law here and the rightful freedoms of the city's people.

But all this only gives Hong Kongers more reason, and more conviction,
to fight on, and defeat vested interests by defending our values.

Fernando Cheung is a member of the Hong Kong Legislative Council from
the Labour Party.

\emph{The Times is committed to publishing}
\href{https://www.nytimes.com/2019/01/31/opinion/letters/letters-to-editor-new-york-times-women.html}{\emph{a
diversity of letters}} \emph{to the editor. We'd like to hear what you
think about this or any of our articles. Here are some}
\href{https://help.nytimes.com/hc/en-us/articles/115014925288-How-to-submit-a-letter-to-the-editor}{\emph{tips}}\emph{.
And here's our email:}
\href{mailto:letters@nytimes.com}{\emph{letters@nytimes.com}}\emph{.}

\emph{Follow The New York Times Opinion section on}
\href{https://www.facebook.com/nytopinion}{\emph{Facebook}}\emph{,}
\href{http://twitter.com/NYTOpinion}{\emph{Twitter (@NYTopinion)}}
\emph{and}
\href{https://www.instagram.com/nytopinion/}{\emph{Instagram}}\emph{.}

Advertisement

\protect\hyperlink{after-bottom}{Continue reading the main story}

\hypertarget{site-index}{%
\subsection{Site Index}\label{site-index}}

\hypertarget{site-information-navigation}{%
\subsection{Site Information
Navigation}\label{site-information-navigation}}

\begin{itemize}
\tightlist
\item
  \href{https://help.nytimes.com/hc/en-us/articles/115014792127-Copyright-notice}{©~2020~The
  New York Times Company}
\end{itemize}

\begin{itemize}
\tightlist
\item
  \href{https://www.nytco.com/}{NYTCo}
\item
  \href{https://help.nytimes.com/hc/en-us/articles/115015385887-Contact-Us}{Contact
  Us}
\item
  \href{https://www.nytco.com/careers/}{Work with us}
\item
  \href{https://nytmediakit.com/}{Advertise}
\item
  \href{http://www.tbrandstudio.com/}{T Brand Studio}
\item
  \href{https://www.nytimes.com/privacy/cookie-policy\#how-do-i-manage-trackers}{Your
  Ad Choices}
\item
  \href{https://www.nytimes.com/privacy}{Privacy}
\item
  \href{https://help.nytimes.com/hc/en-us/articles/115014893428-Terms-of-service}{Terms
  of Service}
\item
  \href{https://help.nytimes.com/hc/en-us/articles/115014893968-Terms-of-sale}{Terms
  of Sale}
\item
  \href{https://spiderbites.nytimes.com}{Site Map}
\item
  \href{https://help.nytimes.com/hc/en-us}{Help}
\item
  \href{https://www.nytimes.com/subscription?campaignId=37WXW}{Subscriptions}
\end{itemize}
