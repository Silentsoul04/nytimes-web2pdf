Sections

SEARCH

\protect\hyperlink{site-content}{Skip to
content}\protect\hyperlink{site-index}{Skip to site index}

\href{https://www.nytimes.com/section/sports/ncaafootball}{College
Football}

\href{https://myaccount.nytimes.com/auth/login?response_type=cookie\&client_id=vi}{}

\href{https://www.nytimes.com/section/todayspaper}{Today's Paper}

\href{/section/sports/ncaafootball}{College Football}\textbar{}A Group
of Pac-12 Football Players Threaten to Opt Out of the Season

\url{https://nyti.ms/33l2y4g}

\begin{itemize}
\item
\item
\item
\item
\item
\end{itemize}

Advertisement

\protect\hyperlink{after-top}{Continue reading the main story}

Supported by

\protect\hyperlink{after-sponsor}{Continue reading the main story}

\hypertarget{a-group-of-pac-12-football-players-threaten-to-opt-out-of-the-season}{%
\section{A Group of Pac-12 Football Players Threaten to Opt Out of the
Season}\label{a-group-of-pac-12-football-players-threaten-to-opt-out-of-the-season}}

The athletes from 10 schools said they were dissatisfied with how their
universities were handling the coronavirus, an approach they say
prioritizes money over safety.

\includegraphics{https://static01.nyt.com/images/2020/08/02/sports/02collegefootball-web-2/merlin_162985113_bb1f697e-55d7-4324-8afb-7361ce90afcb-articleLarge.jpg?quality=75\&auto=webp\&disable=upscale}

\href{https://www.nytimes.com/by/billy-witz}{\includegraphics{https://static01.nyt.com/images/2018/02/16/multimedia/author-billy-witz/author-billy-witz-thumbLarge.jpg}}

By \href{https://www.nytimes.com/by/billy-witz}{Billy Witz}

\begin{itemize}
\item
  Aug. 2, 2020
\item
  \begin{itemize}
  \item
  \item
  \item
  \item
  \item
  \end{itemize}
\end{itemize}

Thirteen Pac-12 Conference football players threatened Sunday to opt out
of the coming season, saying they would not play until systemic
inequities that have been highlighted by
\href{https://www.nytimes.com/2020/07/16/sports/ncaafootball/ncaa-guidelines-fall-sports.html}{college
athletics' response} to
\href{https://www.nytimes.com/news-event/coronavirus}{the coronavirus
pandemic} were addressed.

The players, who are from 10 schools and include All-American and honor
roll candidates, said that playing a contact sport like football during
the outbreak would be reckless because of what they described as
inadequate transparency about the health risks, a lack of uniform safety
measures and an absence of ample enforcement.

Those shortcomings, they added, are emblematic of a system in which
players have little standing to address social, economic or racial
inequalities --- and, they said, far more of the millions of dollars
they help generate should go toward addressing them.

``The people who are deciding whether we are going to play football are
going to prioritize money over health and safety 10 times out of 10,''
Jaydon Grant, a senior defensive back at Oregon State who graduated with
a degree in digital communication arts said in an interview.

The announcement comes as
\href{https://www.nytimes.com/2020/07/10/sports/ncaafootball/coronavirus-college-football-season-canceled.html}{the
college football season is increasingly in doubt} as the coronavirus
bounces around the country --- including
\href{https://www.nytimes.com/2020/08/01/sports/baseball/coronavirus-cardinals.html}{infiltrating
Major League Baseball}--- no more under control than it was
\href{https://www.nytimes.com/2020/03/12/sports/ncaabasketball/ncaa-basketball-tournament-coronavirus.html}{in
March, when college sports} and
\href{https://www.nytimes.com/2020/03/12/sports/coronavirus-sports-canceled.html}{professional
leagues in the United States} began shutting down.

This has led many universities to
\href{https://www.nytimes.com/2020/07/08/upshot/virus-colleges-harvard-reopening.html}{keep
students off campus} and some conferences, like the Ivy League, to
postpone fall sports until at least January. But the schools at the
lucrative top of the football food chain, which heavily leans on
television revenue, are forging ahead. Four major conferences --- the
Southeastern, Big Ten, Pac-12 and Atlantic Coast --- have pared their
schedules mainly to conference games.

Still, there is pushback gathering over whether universities should be
conscripting unpaid college athletes to keep hundreds of millions of
dollars flowing into athletic departments' coffers by largely assuming
whatever risks come with Covid-19, the disease caused by the
coronavirus. Particularly when there are no N.C.A.A.-wide standards on
the frequency of testing or other protocols, which some schools could
resist because they would be costly.
(\href{https://www.nytimes.com/2020/07/16/sports/ncaafootball/ncaa-guidelines-fall-sports.html}{The
N.C.A.A. has made recommendations} but decisions have been left up to
the universities themselves.)

The N.C.A.A. Board of Governors, which largely comprises university
presidents, will consider fall sports when it meets Tuesday.

While some athletes have expressed trepidation about playing football
during the pandemic --- including SEC players during a recent call with
league officials,
\href{https://www.washingtonpost.com/sports/2020/08/01/sec-football-players-safety-meeting/}{according
to The Washington Post} --- and a handful have opted out, the Pac-12
players represent the first collective effort to question why players
are assuming so much risk.

The Pac-12 players, who include Oregon safety Jevon Holland, considered
a possible first-round N.F.L. draft pick, and Washington linebacker Joe
Tryon, a preseason All-American, are taking advantage of the
conference's recent announcement that it will allow all students to
retain their athletic scholarships if they opt out. The players said the
conditions for their return not only included increased health and
safety protections, but measures that would redistribute some of the
millions of dollars that college football generates.

\includegraphics{https://static01.nyt.com/images/2020/08/02/sports/02collegefootball-web-3/merlin_162380622_73c86245-5a40-4532-9c06-1a3b00de6781-articleLarge.jpg?quality=75\&auto=webp\&disable=upscale}

The players asked that Commissioner Larry Scott, who is paid \$5.3
million per year, and other coaches and administrators drastically
reduce their pay and end lavish facility spending. They also demanded
increased medical insurance coverage, six-year scholarships, the freedom
to hire marketing agents, and that 50 percent of each sport's conference
revenue be distributed evenly among athletes in their sport, akin to how
professional sports leagues share revenue with players.

Scott declined an interview request. A conference spokesman referred to
a statement that said the group had not contacted the Pac-12 or its
schools.

\hypertarget{the-games-resume}{%
\subsubsection{The Games Resume}\label{the-games-resume}}

\hypertarget{sports-and-the-virus}{%
\paragraph{Sports and the Virus}\label{sports-and-the-virus}}

Updated Aug. 4, 2020

Here's what's happening as the world of sports slowly comes back to
life:

\begin{itemize}
\item
  \begin{itemize}
  \tightlist
  \item
    As the virus spreads through baseball,
    \href{https://www.nytimes.com/2020/08/03/sports/baseball/mlb-coronavirus-outbreak.html?action=click\&pgtype=Article\&state=default\&region=MAIN_CONTENT_2\&context=storylines_keepup}{so
    does frustration}. Series have been postponed, teams have been
    quarantined and road trips have been rerouted in a season that has
    been defined above all by its precariousness.
  \item
    On all but the two biggest courts, automated line calls
    \href{https://www.nytimes.com/2020/08/03/sports/tennis/us-open-hawkeye-line-judges.html?action=click\&pgtype=Article\&state=default\&region=MAIN_CONTENT_2\&context=storylines_keepup}{will
    replace human judges} at the U.S. Open to reduce the number of
    people on site during the pandemic.
  \item
    Mets star Yoenis Cespedes is healthy, but
    \href{https://www.nytimes.com/2020/08/02/sports/baseball/Yoenis-cespedes-opt-out-rule.html?action=click\&pgtype=Article\&state=default\&region=MAIN_CONTENT_2\&context=storylines_keepup}{has
    decided to opt out} of the 2020 baseball season for Covid-related
    reasons.
  \end{itemize}
\end{itemize}

At least one head coach was not happy with the players' stance.
Washington State Coach Nick Rolovich told players who had health
concerns he was fine if they opted out, but he did not want them around
the team if they expressed support for \#WeStandUnited, according to
John Woods Jr., the father of the sophomore receiver Kassidy Woods.

In an interview Sunday night, John Woods Jr. said Rolovich told his son,
who opted out for health reasons, to clean out his locker on Monday
after he also said he supported the \#WeStandUnited players. The lone
Washington State player to sign the statement, Dallas Hobbs, a junior
defensive lineman, was told the same, the receiver's father said.

Washington State did not immediately respond to a request to comment.

``These are discussions and topics that are talked about in locker rooms
around the country weekly,'' said Valentino Daltoso, a senior three-year
starter on the offensive line at California, where he recently graduated
in legal studies. ``This isn't some new idea out of left field.''

Daltoso, one of three Cal players among the 13, said the idea took a
foothold about a month ago during a Zoom call his teammates had in the
wake of protests over the police killing of George Floyd in Minneapolis.
As the discussions developed, they reached out to players around the
Pac-12 and to others, like Ramogi Huma, the director of the National
College Players Association, which advocates for players' rights.

The players say there are hundreds of others in the Pac-12 who share
their concerns, and indeed dozens, including
\href{https://twitter.com/peneisewell58/status/1289974930155569153}{Penei
Sewell}, an Oregon offensive tackle who is considered a likely top draft
pick next year, retweeted a Twitter post on Sunday with the hashtag
\#WeAreUnited.

Daltoso expects there are also hundreds of players in other conferences
who feel similarly, noting the questions the SEC players raised in their
conference call with Commissioner Greg Sankey and the conference's
medical advisers.

When MoMo Sanogo, a linebacker at Mississippi, wondered why colleges
were bringing students back to campus, according to The Post, an
official replied, ``It's one of those things where if students don't
come back to campus, then the chances of having a football season are
almost zero.'' Another player wondered about the long-term effects of
contracting the virus.

``Those guys in the SEC are not alone in how they feel,'' Daltoso said.
``Good for them for advocating for themselves. Our power as players
comes from being knowledgeable of each other's struggles.''

Image

Oregon's Jevon Holland, a potential first-round pick in the N.F.L.
draft, was among those who threatened to opt out.Credit...Abbie
Parr/Getty Images

Grant, the Oregon State player and a son of former N.B.A. player Brian
Grant, said that his school has taken extensive measures to keep players
safe during workouts, but he doesn't see a way to play games, where
social distancing would be impossible. A restricted environment at a
limited number of sites, an approach that
\href{https://www.nytimes.com/2020/07/30/sports/basketball/sports-bubble-nba-mlb.html}{seems
to have worked in the nascent returns of pro basketball and soccer},
would be impractical for teams of 120 players --- and antithetical to
college sports.

And the sidelining of Boston Red Sox pitcher Eduardo Rodriguez with
heart inflammation related to the coronavirus has also been chilling.

``Do you want to wait until something happens to us or do you want
ensure that there's a system in place that will help keep us safe?''
Grant said. ``The system has failed to provide our insurance. That's why
we're united.''

Advertisement

\protect\hyperlink{after-bottom}{Continue reading the main story}

\hypertarget{site-index}{%
\subsection{Site Index}\label{site-index}}

\hypertarget{site-information-navigation}{%
\subsection{Site Information
Navigation}\label{site-information-navigation}}

\begin{itemize}
\tightlist
\item
  \href{https://help.nytimes.com/hc/en-us/articles/115014792127-Copyright-notice}{©~2020~The
  New York Times Company}
\end{itemize}

\begin{itemize}
\tightlist
\item
  \href{https://www.nytco.com/}{NYTCo}
\item
  \href{https://help.nytimes.com/hc/en-us/articles/115015385887-Contact-Us}{Contact
  Us}
\item
  \href{https://www.nytco.com/careers/}{Work with us}
\item
  \href{https://nytmediakit.com/}{Advertise}
\item
  \href{http://www.tbrandstudio.com/}{T Brand Studio}
\item
  \href{https://www.nytimes.com/privacy/cookie-policy\#how-do-i-manage-trackers}{Your
  Ad Choices}
\item
  \href{https://www.nytimes.com/privacy}{Privacy}
\item
  \href{https://help.nytimes.com/hc/en-us/articles/115014893428-Terms-of-service}{Terms
  of Service}
\item
  \href{https://help.nytimes.com/hc/en-us/articles/115014893968-Terms-of-sale}{Terms
  of Sale}
\item
  \href{https://spiderbites.nytimes.com}{Site Map}
\item
  \href{https://help.nytimes.com/hc/en-us}{Help}
\item
  \href{https://www.nytimes.com/subscription?campaignId=37WXW}{Subscriptions}
\end{itemize}
