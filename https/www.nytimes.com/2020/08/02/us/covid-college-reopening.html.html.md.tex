Sections

SEARCH

\protect\hyperlink{site-content}{Skip to
content}\protect\hyperlink{site-index}{Skip to site index}

\href{https://www.nytimes.com/section/us}{U.S.}

\href{https://myaccount.nytimes.com/auth/login?response_type=cookie\&client_id=vi}{}

\href{https://www.nytimes.com/section/todayspaper}{Today's Paper}

\href{/section/us}{U.S.}\textbar{}Covid Tests and Quarantines: Colleges
Brace for an Uncertain Fall

\url{https://nyti.ms/33kjgjW}

\begin{itemize}
\item
\item
\item
\item
\item
\end{itemize}

\href{https://www.nytimes.com/news-event/coronavirus?action=click\&pgtype=Article\&state=default\&region=TOP_BANNER\&context=storylines_menu}{The
Coronavirus Outbreak}

\begin{itemize}
\tightlist
\item
  live\href{https://www.nytimes.com/2020/08/02/world/coronavirus-updates.html?action=click\&pgtype=Article\&state=default\&region=TOP_BANNER\&context=storylines_menu}{Latest
  Updates}
\item
  \href{https://www.nytimes.com/interactive/2020/us/coronavirus-us-cases.html?action=click\&pgtype=Article\&state=default\&region=TOP_BANNER\&context=storylines_menu}{Maps
  and Cases}
\item
  \href{https://www.nytimes.com/interactive/2020/science/coronavirus-vaccine-tracker.html?action=click\&pgtype=Article\&state=default\&region=TOP_BANNER\&context=storylines_menu}{Vaccine
  Tracker}
\item
  \href{https://www.nytimes.com/interactive/2020/07/29/us/schools-reopening-coronavirus.html?action=click\&pgtype=Article\&state=default\&region=TOP_BANNER\&context=storylines_menu}{What
  School May Look Like}
\item
  \href{https://www.nytimes.com/live/2020/07/31/business/stock-market-today-coronavirus?action=click\&pgtype=Article\&state=default\&region=TOP_BANNER\&context=storylines_menu}{Economy}
\end{itemize}

Advertisement

\protect\hyperlink{after-top}{Continue reading the main story}

Supported by

\protect\hyperlink{after-sponsor}{Continue reading the main story}

\hypertarget{covid-tests-and-quarantines-colleges-brace-for-an-uncertain-fall}{%
\section{Covid Tests and Quarantines: Colleges Brace for an Uncertain
Fall}\label{covid-tests-and-quarantines-colleges-brace-for-an-uncertain-fall}}

Colleges are racing to reconfigure dorms, expand testing programs and
establish detailed social distancing rules. And then, what to do about
sex?

\includegraphics{https://static01.nyt.com/images/2020/07/31/us/00virus-collegeprep01/merlin_175178988_969779f2-30fe-46fa-bc7f-fa9c7a176a5c-articleLarge.jpg?quality=75\&auto=webp\&disable=upscale}

\href{https://www.nytimes.com/by/anemona-hartocollis}{\includegraphics{https://static01.nyt.com/images/2018/06/13/multimedia/author-anemona-hartocollis/author-anemona-hartocollis-thumbLarge-v3.jpg}}\href{https://www.nytimes.com/by/shawn-hubler}{\includegraphics{https://static01.nyt.com/images/2020/06/05/reader-center/author-shawn-hubler/author-shawn-hubler-thumbLarge.png}}

By \href{https://www.nytimes.com/by/anemona-hartocollis}{Anemona
Hartocollis} and \href{https://www.nytimes.com/by/shawn-hubler}{Shawn
Hubler}

\begin{itemize}
\item
  Aug. 2, 2020
\item
  \begin{itemize}
  \item
  \item
  \item
  \item
  \item
  \end{itemize}
\end{itemize}

This month, many colleges around the country plan to welcome back
thousands of students into something they hope will resemble normal
campus life. But they face challenges unlike any other American
institution --- containing the coronavirus among a young, impulsive
population that not only studies together, but lives together, parties
together, and, if decades of history are any guide, sleeps together.

It will be a hugely complex and costly endeavor requiring far more than
just the reconfiguring of dorm rooms and cafeterias and the construction
of annexes and tent classrooms to increase social distancing. It also
crucially involves the creation of testing programs capable of serving
communities the size of small cities and the enforcement of codes of
conduct among students not eager to be policed.

Who will be tested for the coronavirus and how quickly can they get
results? Will mask wearing be mandated? And what will happen to
tailgating, keg parties and sneaking into your partner's dorm room?
Colleges are mapping strategies as varied as the contrasting Covid
regulations enacted by the states, reflecting the culture and leadership
of their schools.

Syracuse is vowing to play the strict parent, requiring students to sign
codes of conduct with penalties for violating Covid-19 rules more severe
than the punishment for smoking marijuana. But the University of
Kentucky is presenting a more lenient front, adopting existing honor
codes that urge students to ``promote personal responsibility and peer
accountability.''

And the University of Texas-Austin has prohibited students from holding
parties on or off campus, banned overnight guests in dorm rooms and
warned students that they can be disciplined for ``purposefully invading
the personal space of others,'' at least without a face mask on.

All of these efforts are coming at great cost, potentially adding more
than \$70 billion to the budgets of the nation's 5,000 colleges. Yet
college administrators say giving their constituents --- students and
their families --- at least a taste of college life is worth it, if done
in the safest possible way. Whether those constituents agree is an open
question, and complaints about tuition have led a growing number of
schools to offer rebates.

It is still possible that the frantic planning will come to naught.
Almost daily, universities that had released detailed plans for
in-person classes this semester have reversed themselves and said they
will go almost entirely online. On Friday, the University of
Pennsylvania became the latest, announcing that almost all undergraduate
classes would be taught online and that undergraduates returning to
Philadelphia, regardless of whether they were living on or off campus,
would have to take a minimum of two Covid tests to participate in any
Penn activities this fall.

``We have learned how to close safely,'' Hiram Chodosh, president of
Claremont McKenna College, a liberal arts school in Claremont, Calif.,
said. ``But the big question now is, can we open safely?''

Testing capacity, a problem in communities throughout the country,
varies widely among schools and could play a major role in whether they
can remain open during an outbreak.

Big schools, from Syracuse University to the University of California,
San Diego, that have connections to labs, health programs or medical
schools say they are capable of processing large numbers of Covid tests
in 24 to 48 hours.

\hypertarget{latest-updates-global-coronavirus-outbreak}{%
\section{\texorpdfstring{\href{https://www.nytimes.com/2020/08/01/world/coronavirus-covid-19.html?action=click\&pgtype=Article\&state=default\&region=MAIN_CONTENT_1\&context=storylines_live_updates}{Latest
Updates: Global Coronavirus
Outbreak}}{Latest Updates: Global Coronavirus Outbreak}}\label{latest-updates-global-coronavirus-outbreak}}

Updated 2020-08-02T17:52:35.962Z

\begin{itemize}
\tightlist
\item
  \href{https://www.nytimes.com/2020/08/01/world/coronavirus-covid-19.html?action=click\&pgtype=Article\&state=default\&region=MAIN_CONTENT_1\&context=storylines_live_updates\#link-34047410}{The
  U.S. reels as July cases more than double the total of any other
  month.}
\item
  \href{https://www.nytimes.com/2020/08/01/world/coronavirus-covid-19.html?action=click\&pgtype=Article\&state=default\&region=MAIN_CONTENT_1\&context=storylines_live_updates\#link-780ec966}{Top
  U.S. officials work to break an impasse over the federal jobless
  benefit.}
\item
  \href{https://www.nytimes.com/2020/08/01/world/coronavirus-covid-19.html?action=click\&pgtype=Article\&state=default\&region=MAIN_CONTENT_1\&context=storylines_live_updates\#link-2bc8948}{Its
  outbreak untamed, Melbourne goes into even greater lockdown.}
\end{itemize}

\href{https://www.nytimes.com/2020/08/01/world/coronavirus-covid-19.html?action=click\&pgtype=Article\&state=default\&region=MAIN_CONTENT_1\&context=storylines_live_updates}{See
more updates}

More live coverage:
\href{https://www.nytimes.com/live/2020/07/31/business/stock-market-today-coronavirus?action=click\&pgtype=Article\&state=default\&region=MAIN_CONTENT_1\&context=storylines_live_updates}{Markets}

In a typical big-school plan, the University of California, Berkeley,
will test all residential **** students within 24 hours of their
arrival, for free, using either a standard nasal swab
or\href{https://news.berkeley.edu/2020/06/30/uc-berkeley-launches-trial-of-saliva-test-for-covid-19/}{a
saliva test} being developed by an internationally renowned genomics
research lab on campus. Students will subsequently be sequestered for 7
to 10 days, leaving their single dorm rooms only to go (masked) to the
bathroom or to pick up a meal from a central location in the building or
outside, then retested. If they test positive, they'll be isolated in a
special dorm. (Some schools hope to create supportive communities, along
the lines of an old-fashioned TB sanitarium, for students who test
positive.) After that, everyone living on campus will be tested
regularly, twice a month, if the spit test proves to be accurate enough.

But little Cornell College in Iowa, with only 1,000 students, is not
doing universal testing on arrival, believing that it would give a false
sense of security because of the incubation period. The school will be
doing randomized rapid testing of 3 percent of its asymptomatic students
per week through its health center, which will take just a few minutes
to get results. It will reserve the more sophisticated testing, with the
help of the county health department, for students who show symptoms.
Other small schools in similar situations are finding themselves at the
mercy of private labs that can take days to deliver results, making
results almost meaningless.

But even some big schools are worried about testing backlogs. ``If we
have to wait days for a result,'' said Michael Haynie, Syracuse's vice
chancellor of Strategic Initiatives and Innovation, ``the quarantine
requirements will overwhelm us before we even get started.''

Alison Byerly, president of Lafayette College, in Easton, Pa., cited
worries about testing supplies as a reason to shift all classes online,
and to ask most students to study from home.

Cost is an issue. Delaware State University, an historically black
college, is among several that have enlisted the nonprofit Testing for
America and the Thurgood Marshall College Fund, among others, to help
finance its testing program.

So is personal freedom. Despite Florida's high infection rate, the
University of Florida has declined to force students to be tested,
worrying some local officials and residents in Gainesville who fear that
students could cause an outbreak in the city. Although Florida has among
\href{https://www.nytimes.com/interactive/2020/us/coronavirus-us-cases.html}{the
highest per capita} rates of infection in the country, the university is
mandating testing only for athletes, those who report Covid-19 symptoms
and a few other exceptions. ``The Gator Nation will not be deterred,''
says the school's
\href{https://coronavirus.ufl.edu/media/coronavirusufledu/Reopening-Plan.pdf}{reopening
plan}.

\includegraphics{https://static01.nyt.com/images/2020/07/31/us/00virus-collegeprep02/merlin_175179009_bb934f23-92a8-4858-9d0b-63018046d166-articleLarge.jpg?quality=75\&auto=webp\&disable=upscale}

``We're a public institution, so constitutional considerations come into
play in terms of what we require --- and how we will be able to enforce
that requirement,'' said Ken Garcia, a campus spokesman, in an email.
And testing backlogs are a major issue, university officials said in a
\href{https://mediasite.video.ufl.edu/Mediasite/Play/175687a86d7f49069f03f9e60e3ed70b1d}{university
webcast}.

Equally daunting is the task of regulating the behavior of an age group
known for its risk-taking behavior.

Many schools have adopted social compacts and behavior codes. Masks are
a key part of almost every code, to be worn except in situations like
brushing teeth, walking alone outside, or being alone in a dorm room.

Most ban partying or socializing outside ``social pods'' --- the small
groups of students that some colleges are assigning students to, usually
based on their dorms. Penalties for code violations range from being
kicked out of class and counseled, to eviction from campus housing and
expulsion.

The word ``sex'' is not mentioned in the typical behavior code. Some
colleges may try to prohibit overnight visits in dorms, and many are
stressing the obvious risks intimate contact poses of spreading the
virus. But most administrators seem to believe that a rule banning sex
is unrealistic, and are quietly hoping that students will use common
sense and refrain from, say, having it with people outside their pod.

\href{https://www.nytimes.com/news-event/coronavirus?action=click\&pgtype=Article\&state=default\&region=MAIN_CONTENT_3\&context=storylines_faq}{}

\hypertarget{the-coronavirus-outbreak-}{%
\subsubsection{The Coronavirus Outbreak
›}\label{the-coronavirus-outbreak-}}

\hypertarget{frequently-asked-questions}{%
\paragraph{Frequently Asked
Questions}\label{frequently-asked-questions}}

Updated July 27, 2020

\begin{itemize}
\item ~
  \hypertarget{should-i-refinance-my-mortgage}{%
  \paragraph{Should I refinance my
  mortgage?}\label{should-i-refinance-my-mortgage}}

  \begin{itemize}
  \tightlist
  \item
    \href{https://www.nytimes.com/article/coronavirus-money-unemployment.html?action=click\&pgtype=Article\&state=default\&region=MAIN_CONTENT_3\&context=storylines_faq}{It
    could be a good idea,} because mortgage rates have
    \href{https://www.nytimes.com/2020/07/16/business/mortgage-rates-below-3-percent.html?action=click\&pgtype=Article\&state=default\&region=MAIN_CONTENT_3\&context=storylines_faq}{never
    been lower.} Refinancing requests have pushed mortgage applications
    to some of the highest levels since 2008, so be prepared to get in
    line. But defaults are also up, so if you're thinking about buying a
    home, be aware that some lenders have tightened their standards.
  \end{itemize}
\item ~
  \hypertarget{what-is-school-going-to-look-like-in-september}{%
  \paragraph{What is school going to look like in
  September?}\label{what-is-school-going-to-look-like-in-september}}

  \begin{itemize}
  \tightlist
  \item
    It is unlikely that many schools will return to a normal schedule
    this fall, requiring the grind of
    \href{https://www.nytimes.com/2020/06/05/us/coronavirus-education-lost-learning.html?action=click\&pgtype=Article\&state=default\&region=MAIN_CONTENT_3\&context=storylines_faq}{online
    learning},
    \href{https://www.nytimes.com/2020/05/29/us/coronavirus-child-care-centers.html?action=click\&pgtype=Article\&state=default\&region=MAIN_CONTENT_3\&context=storylines_faq}{makeshift
    child care} and
    \href{https://www.nytimes.com/2020/06/03/business/economy/coronavirus-working-women.html?action=click\&pgtype=Article\&state=default\&region=MAIN_CONTENT_3\&context=storylines_faq}{stunted
    workdays} to continue. California's two largest public school
    districts --- Los Angeles and San Diego --- said on July 13, that
    \href{https://www.nytimes.com/2020/07/13/us/lausd-san-diego-school-reopening.html?action=click\&pgtype=Article\&state=default\&region=MAIN_CONTENT_3\&context=storylines_faq}{instruction
    will be remote-only in the fall}, citing concerns that surging
    coronavirus infections in their areas pose too dire a risk for
    students and teachers. Together, the two districts enroll some
    825,000 students. They are the largest in the country so far to
    abandon plans for even a partial physical return to classrooms when
    they reopen in August. For other districts, the solution won't be an
    all-or-nothing approach.
    \href{https://bioethics.jhu.edu/research-and-outreach/projects/eschool-initiative/school-policy-tracker/}{Many
    systems}, including the nation's largest, New York City, are
    devising
    \href{https://www.nytimes.com/2020/06/26/us/coronavirus-schools-reopen-fall.html?action=click\&pgtype=Article\&state=default\&region=MAIN_CONTENT_3\&context=storylines_faq}{hybrid
    plans} that involve spending some days in classrooms and other days
    online. There's no national policy on this yet, so check with your
    municipal school system regularly to see what is happening in your
    community.
  \end{itemize}
\item ~
  \hypertarget{is-the-coronavirus-airborne}{%
  \paragraph{Is the coronavirus
  airborne?}\label{is-the-coronavirus-airborne}}

  \begin{itemize}
  \tightlist
  \item
    The coronavirus
    \href{https://www.nytimes.com/2020/07/04/health/239-experts-with-one-big-claim-the-coronavirus-is-airborne.html?action=click\&pgtype=Article\&state=default\&region=MAIN_CONTENT_3\&context=storylines_faq}{can
    stay aloft for hours in tiny droplets in stagnant air}, infecting
    people as they inhale, mounting scientific evidence suggests. This
    risk is highest in crowded indoor spaces with poor ventilation, and
    may help explain super-spreading events reported in meatpacking
    plants, churches and restaurants.
    \href{https://www.nytimes.com/2020/07/06/health/coronavirus-airborne-aerosols.html?action=click\&pgtype=Article\&state=default\&region=MAIN_CONTENT_3\&context=storylines_faq}{It's
    unclear how often the virus is spread} via these tiny droplets, or
    aerosols, compared with larger droplets that are expelled when a
    sick person coughs or sneezes, or transmitted through contact with
    contaminated surfaces, said Linsey Marr, an aerosol expert at
    Virginia Tech. Aerosols are released even when a person without
    symptoms exhales, talks or sings, according to Dr. Marr and more
    than 200 other experts, who
    \href{https://academic.oup.com/cid/article/doi/10.1093/cid/ciaa939/5867798}{have
    outlined the evidence in an open letter to the World Health
    Organization}.
  \end{itemize}
\item ~
  \hypertarget{what-are-the-symptoms-of-coronavirus}{%
  \paragraph{What are the symptoms of
  coronavirus?}\label{what-are-the-symptoms-of-coronavirus}}

  \begin{itemize}
  \tightlist
  \item
    Common symptoms
    \href{https://www.nytimes.com/article/symptoms-coronavirus.html?action=click\&pgtype=Article\&state=default\&region=MAIN_CONTENT_3\&context=storylines_faq}{include
    fever, a dry cough, fatigue and difficulty breathing or shortness of
    breath.} Some of these symptoms overlap with those of the flu,
    making detection difficult, but runny noses and stuffy sinuses are
    less common.
    \href{https://www.nytimes.com/2020/04/27/health/coronavirus-symptoms-cdc.html?action=click\&pgtype=Article\&state=default\&region=MAIN_CONTENT_3\&context=storylines_faq}{The
    C.D.C. has also} added chills, muscle pain, sore throat, headache
    and a new loss of the sense of taste or smell as symptoms to look
    out for. Most people fall ill five to seven days after exposure, but
    symptoms may appear in as few as two days or as many as 14 days.
  \end{itemize}
\item ~
  \hypertarget{does-asymptomatic-transmission-of-covid-19-happen}{%
  \paragraph{Does asymptomatic transmission of Covid-19
  happen?}\label{does-asymptomatic-transmission-of-covid-19-happen}}

  \begin{itemize}
  \tightlist
  \item
    So far, the evidence seems to show it does. A widely cited
    \href{https://www.nature.com/articles/s41591-020-0869-5}{paper}
    published in April suggests that people are most infectious about
    two days before the onset of coronavirus symptoms and estimated that
    44 percent of new infections were a result of transmission from
    people who were not yet showing symptoms. Recently, a top expert at
    the World Health Organization stated that transmission of the
    coronavirus by people who did not have symptoms was ``very rare,''
    \href{https://www.nytimes.com/2020/06/09/world/coronavirus-updates.html?action=click\&pgtype=Article\&state=default\&region=MAIN_CONTENT_3\&context=storylines_faq\#link-1f302e21}{but
    she later walked back that statement.}
  \end{itemize}
\end{itemize}

``I think at some point, if you treat young people like adults, they are
going to act like adults,'' Gordon Gee, the president of West Virginia
University, said. ``In the end, we're not going to patrol every aspect
of their lives.''

Or, as one official at another college, put it: ``Could there be love in
the pod? I guess so.''

The behavior codes generally apply both on and off campus, though they
are clearly harder to enforce off-campus, and some students say that
they immediately began looking for off-campus housing when they realized
where rules would be strictly supervised.

The rules of local governments also apply.

``In Berkeley, indoor gatherings which would constitute a party or are
outside of your social pod are forbidden,'' Dan Mogulof, a spokesman for
the university, said. ``So we are and are going to remain consistent
with what the city's rules are, and we have to run everything through
them.''

But students say social pods, especially when assigned by
administrators, could quickly fracture if one or two students have a
falling out.

West Virginia University has persuaded the governor to shut down bars
serving students at its Morgantown campus after a Covid outbreak in the
area, Mr. Gee said. It has been in effect for about two weeks, and he
would like to renew it.

``Bars get people together in small places, and they cause these kids to
really really, really get too damn close to each other,'' Mr. Gee said.

Image

Maria Gray, who is from Portland, Ore., had gotten an apartment in
Portland, Maine, to wait out the coronavirus for the summer and return
to classes at Bates College in the fall. Now, she will be driving back
to the west coast in two weeks to live and attend classes online at
Bates in September.Credit...Sarah Rice for The New York Times

Travel restrictions are also common. In an email to students at the
University of Pennsylvania's Wharton School two weeks ago, before the
school went mainly online, Maryellen Reilly, deputy vice dean, said that
students would be expected to limit all unnecessary travel.

``Does this mean that if your spouse or partner lives in D.C. or N.Y.
that you can't go visit for the weekend?,'' her email said.
``Unfortunately, yes. The risk of bringing germs back and forth is too
great --- this also means we ask that you don't have visitors who could
be traveling with the virus.''

Already some students are pushing back against codes of conduct and
choosing either to skip the semester or live off-campus, where they can
control their own environment.

Maria Gray, a junior at Bates College in Maine, was horrified when she
paged through enrollment documents and found that she was being asked to
sign a legal document with her digital PIN. ``I acknowledge and agree
that by committing to attend Bates College as an on-campus residential
student, I am voluntarily assuming any and all risks,'' the statement
said, ending with a warning that the outcome of getting sick with
Covid-19 could be ``disability, or even death.''

That document was scary enough. But then on Friday, the school sent her
an email saying that students could have to evacuate campus within 24-48
hours if there were an outbreak, and to bring only what they could
easily pack. That made a closing seem inevitable.

``I have faith in people to be responsible and understand the stakes,''
said Ms. Gray, who now plans to study online at her home in Portland,
Ore. ``But also, this shouldn't be a life or death thing. The stakes
just got really high really fast.''

Advertisement

\protect\hyperlink{after-bottom}{Continue reading the main story}

\hypertarget{site-index}{%
\subsection{Site Index}\label{site-index}}

\hypertarget{site-information-navigation}{%
\subsection{Site Information
Navigation}\label{site-information-navigation}}

\begin{itemize}
\tightlist
\item
  \href{https://help.nytimes.com/hc/en-us/articles/115014792127-Copyright-notice}{©~2020~The
  New York Times Company}
\end{itemize}

\begin{itemize}
\tightlist
\item
  \href{https://www.nytco.com/}{NYTCo}
\item
  \href{https://help.nytimes.com/hc/en-us/articles/115015385887-Contact-Us}{Contact
  Us}
\item
  \href{https://www.nytco.com/careers/}{Work with us}
\item
  \href{https://nytmediakit.com/}{Advertise}
\item
  \href{http://www.tbrandstudio.com/}{T Brand Studio}
\item
  \href{https://www.nytimes.com/privacy/cookie-policy\#how-do-i-manage-trackers}{Your
  Ad Choices}
\item
  \href{https://www.nytimes.com/privacy}{Privacy}
\item
  \href{https://help.nytimes.com/hc/en-us/articles/115014893428-Terms-of-service}{Terms
  of Service}
\item
  \href{https://help.nytimes.com/hc/en-us/articles/115014893968-Terms-of-sale}{Terms
  of Sale}
\item
  \href{https://spiderbites.nytimes.com}{Site Map}
\item
  \href{https://help.nytimes.com/hc/en-us}{Help}
\item
  \href{https://www.nytimes.com/subscription?campaignId=37WXW}{Subscriptions}
\end{itemize}
