Sections

SEARCH

\protect\hyperlink{site-content}{Skip to
content}\protect\hyperlink{site-index}{Skip to site index}

\href{https://www.nytimes.com/section/us}{U.S.}

\href{https://myaccount.nytimes.com/auth/login?response_type=cookie\&client_id=vi}{}

\href{https://www.nytimes.com/section/todayspaper}{Today's Paper}

\href{/section/us}{U.S.}\textbar{}Florida's Summer of Dread

\url{https://nyti.ms/30iWIOI}

\begin{itemize}
\item
\item
\item
\item
\item
\end{itemize}

\href{https://www.nytimes.com/news-event/coronavirus?action=click\&pgtype=Article\&state=default\&region=TOP_BANNER\&context=storylines_menu}{The
Coronavirus Outbreak}

\begin{itemize}
\tightlist
\item
  live\href{https://www.nytimes.com/2020/08/01/world/coronavirus-covid-19.html?action=click\&pgtype=Article\&state=default\&region=TOP_BANNER\&context=storylines_menu}{Latest
  Updates}
\item
  \href{https://www.nytimes.com/interactive/2020/us/coronavirus-us-cases.html?action=click\&pgtype=Article\&state=default\&region=TOP_BANNER\&context=storylines_menu}{Maps
  and Cases}
\item
  \href{https://www.nytimes.com/interactive/2020/science/coronavirus-vaccine-tracker.html?action=click\&pgtype=Article\&state=default\&region=TOP_BANNER\&context=storylines_menu}{Vaccine
  Tracker}
\item
  \href{https://www.nytimes.com/interactive/2020/07/29/us/schools-reopening-coronavirus.html?action=click\&pgtype=Article\&state=default\&region=TOP_BANNER\&context=storylines_menu}{What
  School May Look Like}
\item
  \href{https://www.nytimes.com/live/2020/07/31/business/stock-market-today-coronavirus?action=click\&pgtype=Article\&state=default\&region=TOP_BANNER\&context=storylines_menu}{Economy}
\end{itemize}

Advertisement

\protect\hyperlink{after-top}{Continue reading the main story}

Supported by

\protect\hyperlink{after-sponsor}{Continue reading the main story}

MIAMI JOURNAL

\hypertarget{floridas-summer-of-dread}{%
\section{Florida's Summer of Dread}\label{floridas-summer-of-dread}}

The coronavirus had entrenched itself in communities from Pensacola to
Key West, killing more than 7,000 Floridians. Then came Tropical Storm
Isaias.

\includegraphics{https://static01.nyt.com/images/2020/08/02/us/politics/02virus-floridastorm-1/merlin_175209375_a3ff0f8e-a302-461a-a795-c76e4cf3f1ab-articleLarge.jpg?quality=75\&auto=webp\&disable=upscale}

\href{https://www.nytimes.com/by/patricia-mazzei}{\includegraphics{https://static01.nyt.com/images/2018/11/28/multimedia/author-patricia-mazzei/author-patricia-mazzei-thumbLarge.png}}

By \href{https://www.nytimes.com/by/patricia-mazzei}{Patricia Mazzei}

\begin{itemize}
\item
  Aug. 2, 2020, 5:00 a.m. ET
\item
  \begin{itemize}
  \item
  \item
  \item
  \item
  \item
  \end{itemize}
\end{itemize}

MIAMI --- The crowded grocery stores, empty shelves and barren streets
of South Florida in the dawning days of the coronavirus pandemic felt
unsettlingly familiar: They resembled the rush of preparations and then
the tense silence that precede a hurricane.

Maybe the tough residents of a state used to dealing with unpredictable
forces of nature would have an edge in handling the deadly coronavirus.
In theory, the people of Florida know a thing or two about how to follow
orders during an emergency and stay at home.

Oh, were we naïve.

The virus has entrenched itself in communities from Pensacola to
\href{https://www.nytimes.com/2020/07/31/us/coronavirus-masks-enforcement-key-west.html?referringSource=articleShare}{Key
West}, killing more than 7,000 Floridians. For four consecutive days
last week, the daily number of fatalities broke state records. Florida's
257 deaths on Friday accounted for nearly one-fifth of all of the deaths
attributed to Covid-19 that day in the United States.

With the scourge of virus death came Tropical Storm Isaias to stalk the
Atlantic Coast. The calendar had barely turned to August --- too early,
in a normal year, to worry much about storms. But this annus horribilis
would not have it any other way.

A public health crisis. An economic calamity, with more than a million
\href{https://www.nytimes.com/2020/04/23/us/florida-coronavirus-unemployment.html}{Floridians
out of work} and an unemployment payment system that was one of the
slowest in the country. And now an early debut of hurricane season to
remind the state that the inevitable convergence of the pandemic and the
weather is likely to play out again, and perhaps much more seriously
than this relatively mild storm, before this nightmare season ends.

``It's just kind of been the way 2020's gone so far,'' said Howard
Tipton, the administrator for St. Lucie County, on Florida's Treasure
Coast. ``But we roll with it, right? We don't get to determine the cards
that we're dealt.''

Tropical Storm Isaias threatens the entire East Coast all the way up to
Maine, but it is the South that has seen a recent dramatic increase in
new coronavirus cases. Health officials in Georgia, South Carolina and
North Carolina have warned that hospitals could be strained beyond
capacity with the flood of new patients.

Meantime, emergency management officials
\href{https://www.nytimes.com/2020/05/24/us/hurricane-pandemic-coronavirus-florida.html}{have
drawn up special plans} to deal with people fleeing or displaced by
storms. To avoid virus exposure in shelters, the first choice is for
coastal residents in homes vulnerable to flooding to stay with relatives
or friends farther inland, being careful to wear masks and remain
socially distant.

``Because of Covid, we feel that you are safer at home,'' said Bill
Johnson, the emergency management director for Palm Beach County.
``Shelters should be considered your last resort.''

Summer in Florida, with its routine thunderstorms, sweaty nights and
unforgiving mosquitoes, is not for the faint of heart. (At least 11
suspected cases of coronavirus in the Florida Keys last month turned out
to be mosquito-borne dengue fever.) Sometimes it feels as though the
season's only rewards are royal poinciana blooms, ripened mangoes and
fewer tourists.

\hypertarget{latest-updates-global-coronavirus-outbreak}{%
\section{\texorpdfstring{\href{https://www.nytimes.com/2020/08/01/world/coronavirus-covid-19.html?action=click\&pgtype=Article\&state=default\&region=MAIN_CONTENT_1\&context=storylines_live_updates}{Latest
Updates: Global Coronavirus
Outbreak}}{Latest Updates: Global Coronavirus Outbreak}}\label{latest-updates-global-coronavirus-outbreak}}

Updated 2020-08-02T07:42:09.613Z

\begin{itemize}
\tightlist
\item
  \href{https://www.nytimes.com/2020/08/01/world/coronavirus-covid-19.html?action=click\&pgtype=Article\&state=default\&region=MAIN_CONTENT_1\&context=storylines_live_updates\#link-34047410}{The
  U.S. reels as July cases more than double the total of any other
  month.}
\item
  \href{https://www.nytimes.com/2020/08/01/world/coronavirus-covid-19.html?action=click\&pgtype=Article\&state=default\&region=MAIN_CONTENT_1\&context=storylines_live_updates\#link-780ec966}{Top
  U.S. officials work to break an impasse over the federal jobless
  benefit.}
\item
  \href{https://www.nytimes.com/2020/08/01/world/coronavirus-covid-19.html?action=click\&pgtype=Article\&state=default\&region=MAIN_CONTENT_1\&context=storylines_live_updates\#link-2bc8948}{Its
  outbreak untamed, Melbourne goes into even greater lockdown.}
\end{itemize}

\href{https://www.nytimes.com/2020/08/01/world/coronavirus-covid-19.html?action=click\&pgtype=Article\&state=default\&region=MAIN_CONTENT_1\&context=storylines_live_updates}{See
more updates}

More live coverage:
\href{https://www.nytimes.com/live/2020/07/31/business/stock-market-today-coronavirus?action=click\&pgtype=Article\&state=default\&region=MAIN_CONTENT_1\&context=storylines_live_updates}{Markets}

This summer has been made harder by the virus, which brought a sense of
despair and helplessness that seemed especially acute in the days
leading up to Tropical Storm Isaias. The storm goes away. The virus has
not.

``It's really stretching our limits,'' said Kevin Cho, 31, a Florida
National Guard captain and a nurse practitioner who treats Covid-19
patients in the intensive care units of several Miami public hospitals.
Among them have been a doctor, who died, and a fellow nurse, who lived.

Many poor people contracting the disease ``are losing their jobs, and
now they're faced with a hurricane,'' he added. ``How could they prepare
for a hurricane when they have been exhausted of every resource they
have? This hurricane is only going to make things worse.''

In Miami-Dade County, where the coronavirus has hit worse than anywhere
else in Florida, the emergency operations center has been outfitted with
plexiglass desk dividers and fans equipped with ultraviolet lights to
try to kill the virus. Many employees who would normally be in the
building worked from home, at least as long as their internet did not go
out.

``It's not as good as being here,'' said Frank K. Rollason, the county's
emergency management director. ``But right now, it's better than being
here.''

Some South Floridians hurried to supermarkets, gas stations and hardware
stores to stock up on canned food, water bottles and plywood. But
others, unfazed by the relatively weak and disorganized storm, did not
bother. My building in a Miami suburb, which was not in the storm's
direct path, did not even bring in the patio furniture, and my potted
plants remained on the balcony. One neighbor on my street put up window
shutters.

``We usually would be assuming, `This is terrible,' I think, except
we're already so busy assuming that Covid is terrible that we don't have
any room,'' said the humor writer Dave Barry, a fellow veteran of the
Miami press corps whom I have known since we both worked at The Miami
Herald. ``We go through this every year, where we always overreact to
it, and maybe this time we underreact to it. Or maybe this is just 2020
lulling us into: `OK, you guys think you had a hurricane. Now you can
relax!' Then the big hurricane comes.''

Local officials worried that the usual spike in alcohol sales before the
storm would entice people to invite friends and relatives over.

Verdenia C. Baker, the Palm Beach County administrator, warned: ``I know
we've been cooped up. Now we have a storm. And some of us normally would
have hurricane parties. This is not the time.''

Florida's relentless coronavirus surge has been driven by a
\href{https://www.nytimes.com/2020/06/26/us/coronavirus-florida-texas-bars-closing.html}{rapid
economic reopening} that exposed people to infection in bars and
\href{https://www.nytimes.com/2020/07/06/us/coronavirus-florida-miami.html}{house
parties}. Contact tracers in Miami-Dade County have found that about 30
percent of people who tested positive for the virus were exposed by
someone else in their household, the biggest source of infection after
``don't know.''

The huge growth in case numbers, which is finally starting to dip, came
even though South Florida had locked down earlier and longer than the
rest of the state. Gov. Ron DeSantis, a Republican, has not issued a
statewide mask order, but Miami-Dade County imposed its first facial
covering requirement back in April.

Dr. Mary Jo Trepka, chairwoman of the epidemiology department at Florida
International University, attributed the contagion in part to Miami's
larger-than-average household sizes and higher poverty rates, as well as
to uneven mask use. The prevalence of the virus is declining so slowly
that it might take until December to get down to a 5 percent positivity
rate, she estimated.

\href{https://www.nytimes.com/news-event/coronavirus?action=click\&pgtype=Article\&state=default\&region=MAIN_CONTENT_3\&context=storylines_faq}{}

\hypertarget{the-coronavirus-outbreak-}{%
\subsubsection{The Coronavirus Outbreak
›}\label{the-coronavirus-outbreak-}}

\hypertarget{frequently-asked-questions}{%
\paragraph{Frequently Asked
Questions}\label{frequently-asked-questions}}

Updated July 27, 2020

\begin{itemize}
\item ~
  \hypertarget{should-i-refinance-my-mortgage}{%
  \paragraph{Should I refinance my
  mortgage?}\label{should-i-refinance-my-mortgage}}

  \begin{itemize}
  \tightlist
  \item
    \href{https://www.nytimes.com/article/coronavirus-money-unemployment.html?action=click\&pgtype=Article\&state=default\&region=MAIN_CONTENT_3\&context=storylines_faq}{It
    could be a good idea,} because mortgage rates have
    \href{https://www.nytimes.com/2020/07/16/business/mortgage-rates-below-3-percent.html?action=click\&pgtype=Article\&state=default\&region=MAIN_CONTENT_3\&context=storylines_faq}{never
    been lower.} Refinancing requests have pushed mortgage applications
    to some of the highest levels since 2008, so be prepared to get in
    line. But defaults are also up, so if you're thinking about buying a
    home, be aware that some lenders have tightened their standards.
  \end{itemize}
\item ~
  \hypertarget{what-is-school-going-to-look-like-in-september}{%
  \paragraph{What is school going to look like in
  September?}\label{what-is-school-going-to-look-like-in-september}}

  \begin{itemize}
  \tightlist
  \item
    It is unlikely that many schools will return to a normal schedule
    this fall, requiring the grind of
    \href{https://www.nytimes.com/2020/06/05/us/coronavirus-education-lost-learning.html?action=click\&pgtype=Article\&state=default\&region=MAIN_CONTENT_3\&context=storylines_faq}{online
    learning},
    \href{https://www.nytimes.com/2020/05/29/us/coronavirus-child-care-centers.html?action=click\&pgtype=Article\&state=default\&region=MAIN_CONTENT_3\&context=storylines_faq}{makeshift
    child care} and
    \href{https://www.nytimes.com/2020/06/03/business/economy/coronavirus-working-women.html?action=click\&pgtype=Article\&state=default\&region=MAIN_CONTENT_3\&context=storylines_faq}{stunted
    workdays} to continue. California's two largest public school
    districts --- Los Angeles and San Diego --- said on July 13, that
    \href{https://www.nytimes.com/2020/07/13/us/lausd-san-diego-school-reopening.html?action=click\&pgtype=Article\&state=default\&region=MAIN_CONTENT_3\&context=storylines_faq}{instruction
    will be remote-only in the fall}, citing concerns that surging
    coronavirus infections in their areas pose too dire a risk for
    students and teachers. Together, the two districts enroll some
    825,000 students. They are the largest in the country so far to
    abandon plans for even a partial physical return to classrooms when
    they reopen in August. For other districts, the solution won't be an
    all-or-nothing approach.
    \href{https://bioethics.jhu.edu/research-and-outreach/projects/eschool-initiative/school-policy-tracker/}{Many
    systems}, including the nation's largest, New York City, are
    devising
    \href{https://www.nytimes.com/2020/06/26/us/coronavirus-schools-reopen-fall.html?action=click\&pgtype=Article\&state=default\&region=MAIN_CONTENT_3\&context=storylines_faq}{hybrid
    plans} that involve spending some days in classrooms and other days
    online. There's no national policy on this yet, so check with your
    municipal school system regularly to see what is happening in your
    community.
  \end{itemize}
\item ~
  \hypertarget{is-the-coronavirus-airborne}{%
  \paragraph{Is the coronavirus
  airborne?}\label{is-the-coronavirus-airborne}}

  \begin{itemize}
  \tightlist
  \item
    The coronavirus
    \href{https://www.nytimes.com/2020/07/04/health/239-experts-with-one-big-claim-the-coronavirus-is-airborne.html?action=click\&pgtype=Article\&state=default\&region=MAIN_CONTENT_3\&context=storylines_faq}{can
    stay aloft for hours in tiny droplets in stagnant air}, infecting
    people as they inhale, mounting scientific evidence suggests. This
    risk is highest in crowded indoor spaces with poor ventilation, and
    may help explain super-spreading events reported in meatpacking
    plants, churches and restaurants.
    \href{https://www.nytimes.com/2020/07/06/health/coronavirus-airborne-aerosols.html?action=click\&pgtype=Article\&state=default\&region=MAIN_CONTENT_3\&context=storylines_faq}{It's
    unclear how often the virus is spread} via these tiny droplets, or
    aerosols, compared with larger droplets that are expelled when a
    sick person coughs or sneezes, or transmitted through contact with
    contaminated surfaces, said Linsey Marr, an aerosol expert at
    Virginia Tech. Aerosols are released even when a person without
    symptoms exhales, talks or sings, according to Dr. Marr and more
    than 200 other experts, who
    \href{https://academic.oup.com/cid/article/doi/10.1093/cid/ciaa939/5867798}{have
    outlined the evidence in an open letter to the World Health
    Organization}.
  \end{itemize}
\item ~
  \hypertarget{what-are-the-symptoms-of-coronavirus}{%
  \paragraph{What are the symptoms of
  coronavirus?}\label{what-are-the-symptoms-of-coronavirus}}

  \begin{itemize}
  \tightlist
  \item
    Common symptoms
    \href{https://www.nytimes.com/article/symptoms-coronavirus.html?action=click\&pgtype=Article\&state=default\&region=MAIN_CONTENT_3\&context=storylines_faq}{include
    fever, a dry cough, fatigue and difficulty breathing or shortness of
    breath.} Some of these symptoms overlap with those of the flu,
    making detection difficult, but runny noses and stuffy sinuses are
    less common.
    \href{https://www.nytimes.com/2020/04/27/health/coronavirus-symptoms-cdc.html?action=click\&pgtype=Article\&state=default\&region=MAIN_CONTENT_3\&context=storylines_faq}{The
    C.D.C. has also} added chills, muscle pain, sore throat, headache
    and a new loss of the sense of taste or smell as symptoms to look
    out for. Most people fall ill five to seven days after exposure, but
    symptoms may appear in as few as two days or as many as 14 days.
  \end{itemize}
\item ~
  \hypertarget{does-asymptomatic-transmission-of-covid-19-happen}{%
  \paragraph{Does asymptomatic transmission of Covid-19
  happen?}\label{does-asymptomatic-transmission-of-covid-19-happen}}

  \begin{itemize}
  \tightlist
  \item
    So far, the evidence seems to show it does. A widely cited
    \href{https://www.nature.com/articles/s41591-020-0869-5}{paper}
    published in April suggests that people are most infectious about
    two days before the onset of coronavirus symptoms and estimated that
    44 percent of new infections were a result of transmission from
    people who were not yet showing symptoms. Recently, a top expert at
    the World Health Organization stated that transmission of the
    coronavirus by people who did not have symptoms was ``very rare,''
    \href{https://www.nytimes.com/2020/06/09/world/coronavirus-updates.html?action=click\&pgtype=Article\&state=default\&region=MAIN_CONTENT_3\&context=storylines_faq\#link-1f302e21}{but
    she later walked back that statement.}
  \end{itemize}
\end{itemize}

``It's really important that we don't open the tap in any way,'' she
said of the possibility of further reopening. ``I hope we're not going
to be having exposures related to that --- or to any of the future
hurricanes we might face over the next couple of months.''

Florida's shockingly high coronavirus case numbers came after it
initially appeared that the state had weathered the first two months of
the outbreak with success. Instead, after most counties returned to
business and holidays prompted people to hold gatherings, the infections
got out of control.

Gus Perez, 32, whom I met at a party last year, thinks he contracted the
virus three weeks ago, over a weekend on which he hung out with a few
friends and went to an outdoor event late one night at a brewery.

He wore a mask and was careful --- his friend who had leukemia and his
friend's mother had both succumbed earlier to the virus.

``I thought I was very on top of it, and it still got me,'' he said.

The hospitals have not collapsed, but only because they have added
scores of beds, straining doctors and nurses.

The Rev. Maria Anderson, 64, an interfaith Miami hospital chaplain, has
been tending to exhausted medical workers treating Covid-19 patients and
to family members allowed to visit their loved ones shortly before or
after they die.

``I've actually lost track of time,'' she said. ``We're in a timeline
limbo. The end doesn't seem to be in sight, and we have no hope that it
will end.''

Ms. Anderson said that coming home to watch news coverage of political
fighting over masks and the virus has been frustrating, underscoring the
distance between the elected officials making decisions and the
professionals toiling in hospitals every day.

``It's such a huge moral disconnect that state and federal leaders have,
and that's what makes me angry,'' she said. ``I look up and say, `Sorry,
God' --- but the anger is there.''

And now storm season may imperil the tenuous new normal that businesses
have tried to forge as they confront the virus.

When Hurricane Irma lashed Florida in 2017, Mike Beltran kept Ariete,
his restaurant in Miami's Coconut Grove neighborhood, open until the
last minute, cooking rice and black beans for customers. He then worked
out of a borrowed food truck while the electricity was out.

This year, Mr. Beltran, 34, was so consumed by staying afloat amid the
virus, which has forced him to close one of his three restaurants and
lay off some of his staff, that he did not know about Tropical Storm
Isaias until late on Thursday.

``It's like, `Oh, something \emph{else},''' he said. ``I'm just waiting
for the year to be over.''

Patricia Mazzei is the Miami bureau chief for The New York Times. She
has lived in Miami for the past 17 years; before joining the Times, was
The Miami Herald's political writer.

Advertisement

\protect\hyperlink{after-bottom}{Continue reading the main story}

\hypertarget{site-index}{%
\subsection{Site Index}\label{site-index}}

\hypertarget{site-information-navigation}{%
\subsection{Site Information
Navigation}\label{site-information-navigation}}

\begin{itemize}
\tightlist
\item
  \href{https://help.nytimes.com/hc/en-us/articles/115014792127-Copyright-notice}{©~2020~The
  New York Times Company}
\end{itemize}

\begin{itemize}
\tightlist
\item
  \href{https://www.nytco.com/}{NYTCo}
\item
  \href{https://help.nytimes.com/hc/en-us/articles/115015385887-Contact-Us}{Contact
  Us}
\item
  \href{https://www.nytco.com/careers/}{Work with us}
\item
  \href{https://nytmediakit.com/}{Advertise}
\item
  \href{http://www.tbrandstudio.com/}{T Brand Studio}
\item
  \href{https://www.nytimes.com/privacy/cookie-policy\#how-do-i-manage-trackers}{Your
  Ad Choices}
\item
  \href{https://www.nytimes.com/privacy}{Privacy}
\item
  \href{https://help.nytimes.com/hc/en-us/articles/115014893428-Terms-of-service}{Terms
  of Service}
\item
  \href{https://help.nytimes.com/hc/en-us/articles/115014893968-Terms-of-sale}{Terms
  of Sale}
\item
  \href{https://spiderbites.nytimes.com}{Site Map}
\item
  \href{https://help.nytimes.com/hc/en-us}{Help}
\item
  \href{https://www.nytimes.com/subscription?campaignId=37WXW}{Subscriptions}
\end{itemize}
