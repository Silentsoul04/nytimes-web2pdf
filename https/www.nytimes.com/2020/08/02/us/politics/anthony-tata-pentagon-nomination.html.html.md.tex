Sections

SEARCH

\protect\hyperlink{site-content}{Skip to
content}\protect\hyperlink{site-index}{Skip to site index}

\href{https://www.nytimes.com/section/politics}{Politics}

\href{https://myaccount.nytimes.com/auth/login?response_type=cookie\&client_id=vi}{}

\href{https://www.nytimes.com/section/todayspaper}{Today's Paper}

\href{/section/politics}{Politics}\textbar{}Trump Pick for Pentagon Post
Sidesteps Senate Ire for Different Defense Job

\url{https://nyti.ms/3k87izV}

\begin{itemize}
\item
\item
\item
\item
\item
\end{itemize}

Advertisement

\protect\hyperlink{after-top}{Continue reading the main story}

Supported by

\protect\hyperlink{after-sponsor}{Continue reading the main story}

\hypertarget{trump-pick-for-pentagon-post-sidesteps-senate-ire-for-different-defense-job}{%
\section{Trump Pick for Pentagon Post Sidesteps Senate Ire for Different
Defense
Job}\label{trump-pick-for-pentagon-post-sidesteps-senate-ire-for-different-defense-job}}

A retired Army general whose inflammatory comments appeared to doom his
nomination was tapped for a job that does not require Senate approval.

\includegraphics{https://static01.nyt.com/images/2020/08/02/us/politics/02DC-TATA1/02DC-TATA1-articleLarge.jpg?quality=75\&auto=webp\&disable=upscale}

\href{https://www.nytimes.com/by/eric-schmitt}{\includegraphics{https://static01.nyt.com/images/2018/06/12/multimedia/author-eric-schmitt/author-eric-schmitt-thumbLarge-v2.png}}

By \href{https://www.nytimes.com/by/eric-schmitt}{Eric Schmitt}

\begin{itemize}
\item
  Aug. 2, 2020
\item
  \begin{itemize}
  \item
  \item
  \item
  \item
  \item
  \end{itemize}
\end{itemize}

WASHINGTON --- President Trump's choice to fill the Pentagon's top
policy job withdrew his name from consideration on Sunday after senators
from both parties
\href{https://www.nytimes.com/2020/07/30/us/politics/trump-inhofe-tata-pentagon.html}{voiced
opposition to the official's nomination}, largely because of his history
of inflammatory comments.

But in an end run around the skeptical senators, the Trump
administration appointed the official, Anthony J. Tata, a retired Army
one-star general turned Fox News commentator, to a temporary senior
position in the same Defense Department office that does not require
Senate approval.

A Pentagon representative said that Mr. Tata had formally withdrawn his
nomination to be the under secretary of defense for policy and instead
``has been designated as the official performing the duties of the
deputy under secretary of defense for policy.'' He will report to the
acting under secretary, James H. Anderson, who is filling the job for
which Mr. Tata was nominated.

Senior congressional Democrats expressed outrage at what they said was a
subterfuge that amounted to the White House and the Pentagon thumbing
their noses at Congress.

``This method of appointment is an insult to our troops, professionals
at the Pentagon, the Senate and the American people,'' Senator Jack Reed
of Rhode Island, the senior Democrat on the Armed Services Committee,
said in a statement Sunday night. ``Clearly, President Trump wants
people who will swear allegiance to him over the Constitution. This is a
flagrant end run around the confirmation process.''

Mr. Tata's nomination appeared to be on life support on Thursday when
minutes before his hearing was set to begin, Senator James M. Inhofe, an
Oklahoma Republican who leads the Armed Services Committee, announced
that he was delaying it.

``There are many Democrats and Republicans who didn't know enough about
Anthony Tata to consider him for a very significant position at this
time,'' Mr. Inhofe said in a statement that papered over the fierce
opposition from all the committee's Democrats and at least one of the
panel's Republicans.

Mr. Inhofe said he had talked to Mr. Trump on Wednesday night and told
him that ``we're simply out of time with the August recess coming, so it
wouldn't serve any useful purpose to have a hearing at this point, and
he agreed.''

Mr. Tata's views, expressed in
\href{https://twitter.com/ajtata/status/1014278134185840640}{a series of
tweets}, drew angry denunciations from both Democrats and Republicans,
particularly as the country is seized by a growing movement for change.
He called Islam ``the most oppressive violent religion'' and referred to
former President Barack Obama as a ``terrorist leader.'' Mr. Tata has
since apologized for the remarks, which were
\href{https://edition.cnn.com/2020/06/12/politics/pentagon-nominee-tata-trump-kfile/index.html}{first
reported by CNN}.

At least three senior retired officers dropped their support for Mr.
Tata after his tweets were made public.

Gen. Joseph L. Votel, the former head of the Central Command; Gen. Tony
Thomas, the former head of the Special Operations Command; and Lt. Gen.
David A. Deptula, a former top Air Force general, asked that their names
be removed from
\href{https://s.wsj.net/public/resources/documents/Tata-Letter_06-18-2020.pdf}{a
letter} sent by 36 current and former leaders to the Senate Armed
Services Committee in support of Mr. Tata.

Despite Mr. Tata's comments and the senators' opposition, Mr. Trump
threw him a lifeline last week. During his conversation with Mr. Inhofe
on Wednesday, Mr. Trump could be heard indicating that he might give Mr.
Tata a different appointment.

The call was overheard because Mr. Inhofe put it on speakerphone as he
sat in the restaurant in Washington.

Mr. Tata was meant to succeed John Rood,
\href{https://www.nytimes.com/2020/02/19/us/politics/john-rood-trump.html}{who
resigned} as the under secretary for policy in February at Mr. Trump's
request. Mr. Rood had pushed back on efforts to withhold military aid to
Ukraine, a central issue in Mr. Trump's impeachment hearings.

Advertisement

\protect\hyperlink{after-bottom}{Continue reading the main story}

\hypertarget{site-index}{%
\subsection{Site Index}\label{site-index}}

\hypertarget{site-information-navigation}{%
\subsection{Site Information
Navigation}\label{site-information-navigation}}

\begin{itemize}
\tightlist
\item
  \href{https://help.nytimes.com/hc/en-us/articles/115014792127-Copyright-notice}{©~2020~The
  New York Times Company}
\end{itemize}

\begin{itemize}
\tightlist
\item
  \href{https://www.nytco.com/}{NYTCo}
\item
  \href{https://help.nytimes.com/hc/en-us/articles/115015385887-Contact-Us}{Contact
  Us}
\item
  \href{https://www.nytco.com/careers/}{Work with us}
\item
  \href{https://nytmediakit.com/}{Advertise}
\item
  \href{http://www.tbrandstudio.com/}{T Brand Studio}
\item
  \href{https://www.nytimes.com/privacy/cookie-policy\#how-do-i-manage-trackers}{Your
  Ad Choices}
\item
  \href{https://www.nytimes.com/privacy}{Privacy}
\item
  \href{https://help.nytimes.com/hc/en-us/articles/115014893428-Terms-of-service}{Terms
  of Service}
\item
  \href{https://help.nytimes.com/hc/en-us/articles/115014893968-Terms-of-sale}{Terms
  of Sale}
\item
  \href{https://spiderbites.nytimes.com}{Site Map}
\item
  \href{https://help.nytimes.com/hc/en-us}{Help}
\item
  \href{https://www.nytimes.com/subscription?campaignId=37WXW}{Subscriptions}
\end{itemize}
