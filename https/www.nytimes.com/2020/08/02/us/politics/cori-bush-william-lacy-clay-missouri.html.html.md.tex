Sections

SEARCH

\protect\hyperlink{site-content}{Skip to
content}\protect\hyperlink{site-index}{Skip to site index}

\href{https://www.nytimes.com/section/politics}{Politics}

\href{https://myaccount.nytimes.com/auth/login?response_type=cookie\&client_id=vi}{}

\href{https://www.nytimes.com/section/todayspaper}{Today's Paper}

\href{/section/politics}{Politics}\textbar{}In St. Louis, Testing
Liberal Might Against a Democratic Fixture

\url{https://nyti.ms/3hZnT7d}

\begin{itemize}
\item
\item
\item
\item
\item
\end{itemize}

\href{https://www.nytimes.com/news-event/george-floyd-protests-minneapolis-new-york-los-angeles?action=click\&pgtype=Article\&state=default\&region=TOP_BANNER\&context=storylines_menu}{Race
and America}

\begin{itemize}
\tightlist
\item
  \href{https://www.nytimes.com/2020/07/26/us/protests-portland-seattle-trump.html?action=click\&pgtype=Article\&state=default\&region=TOP_BANNER\&context=storylines_menu}{Protesters
  Return to Other Cities}
\item
  \href{https://www.nytimes.com/2020/07/24/us/portland-oregon-protests-white-race.html?action=click\&pgtype=Article\&state=default\&region=TOP_BANNER\&context=storylines_menu}{Portland
  at the Center}
\item
  \href{https://www.nytimes.com/2020/07/23/podcasts/the-daily/portland-protests.html?action=click\&pgtype=Article\&state=default\&region=TOP_BANNER\&context=storylines_menu}{Podcast:
  Showdown in Portland}
\item
  \href{https://www.nytimes.com/interactive/2020/07/16/us/black-lives-matter-protests-louisville-breonna-taylor.html?action=click\&pgtype=Article\&state=default\&region=TOP_BANNER\&context=storylines_menu}{45
  Days in Louisville}
\end{itemize}

Advertisement

\protect\hyperlink{after-top}{Continue reading the main story}

Supported by

\protect\hyperlink{after-sponsor}{Continue reading the main story}

\hypertarget{in-st-louis-testing-liberal-might-against-a-democratic-fixture}{%
\section{In St. Louis, Testing Liberal Might Against a Democratic
Fixture}\label{in-st-louis-testing-liberal-might-against-a-democratic-fixture}}

Cori Bush, an activist backed by the progressive group Justice
Democrats, is trying to unseat 10-term Representative William Lacy Clay
in a bid to turn protest-movement fervor into hard political power.

\includegraphics{https://static01.nyt.com/images/2020/08/03/us/politics/03dc-missouri1/merlin_175215426_0cf385fa-4f65-4647-a37c-380d53779281-articleLarge.jpg?quality=75\&auto=webp\&disable=upscale}

\href{https://www.nytimes.com/by/nicholas-fandos}{\includegraphics{https://static01.nyt.com/images/2018/11/06/multimedia/author-nicholas-fandos/author-nicholas-fandos-thumbLarge-v2.png}}

By \href{https://www.nytimes.com/by/nicholas-fandos}{Nicholas Fandos}

\begin{itemize}
\item
  Aug. 2, 2020
\item
  \begin{itemize}
  \item
  \item
  \item
  \item
  \item
  \end{itemize}
\end{itemize}

FERGUSON, Mo. --- As an activist who jumped into the political arena
after the police shooting of Michael Brown here six years ago, Cori Bush
is accustomed to hard fights. She has been maced, shot at with rubber
bullets and cloaked in tear gas at so many protests against police
brutality that they have blurred together.

So when she heard that Representative William Lacy Clay, the 10-term
Democrat she is challenging in Missouri's Democratic primary on Tuesday,
had called her ``a prop'' for the Justice Democrats, a national
progressive group that exists to knock off titans of the party
establishment such as himself, Ms. Bush did not miss a beat.

``I had no title, no name, came out of the Ferguson uprising and people
know who I am across the world,'' Ms. Bush said on Saturday, responding
to comments Mr. Clay made about her in an interview with The New York
Times. ``Not because I took money from some group --- none of that. It
is because I stayed true to a message of change for real people.''

Of Mr. Clay, she added, ``He doesn't understand that, because he doesn't
understand fighting for people.''

All over the country this summer, progressive candidates like Ms. Bush,
44, are doing battle with veteran incumbents over the identity of the
Democratic Party. In New York City,
\href{https://www.nytimes.com/2020/07/17/nyregion/jamaal-bowman-eliot-engel.html}{Jamaal
Bowman defeated Representative Eliot L. Engel,} a 16-term incumbent and
powerful committee chairman. In western Massachusetts, Alex Morse, the
mayor of Holyoke, is trying to unseat another long-serving chairman,
Representative Richard E. Neal.

Emboldened by a lethal pandemic that has shone a spotlight on systemic
racial and economic inequality, and the swell of public support for the
Black Lives Matter movement, they are seeking to sustain the momentum
gathered in 2018 by insurgents, like Representative Alexandria
Ocasio-Cortez of New York, who felled establishment figures. In St.
Louis, Ms. Bush's candidacy is a test of whether the protest movement
can translate into hard electoral power on the federal level.

For Democratic leaders watching warily from Washington, Mr. Clay's fate
will also indicate whether the rise in progressive energy that has cost
powerful white incumbents in places like the Bronx, Queens and Boston
their seats can also dislodge a Black representative deep in the
heartland of the country.

\includegraphics{https://static01.nyt.com/images/2020/08/03/us/politics/03dc-missouri2/merlin_175215447_1051afbe-05e7-4432-b675-76ac02f3af0d-articleLarge.jpg?quality=75\&auto=webp\&disable=upscale}

In few places have the intraparty battle lines glowed as brightly as
greater St. Louis, a once-mighty industrial city plagued by economic
malaise, a legacy of racial segregation and now spiking coronavirus
cases, where the death of Mr. Brown in 2014 helped give rise to the
movement on the left demanding a replacement for an incremental approach
to governing.

The contest has grown exceedingly bitter, and Mr. Clay, 64, has come to
view it not only as a fight for his own survival, but a chance to snuff
out an upstart movement he sees as dangerously divisive. In an interview
last week, the congressman suggested that the effort to unseat him by
Ms. Bush, who is also Black, rests on a racist premise.

``The easy, racist way to lay it out is, `Look at Clay --- what has he
done for his district?''' he said, adding, ``I fight for that district
every single day.''

Mr. Clay accused the groups like Justice Democrats and Brand New
Congress that have helped groom progressive primary challengers of
targeting members of the Congressional Black Caucus specifically because
``they think we are easy targets.''

``She's a prop,'' Mr. Clay said of Ms. Bush. ``They use her to raise
money to support their infrastructure.''

Mr. Clay has a powerful infrastructure of his own.

A Clay has represented part of St. Louis in Congress since 1969. William
Lacy Clay Sr. was an icon of the civil rights movement in the city and a
founding member of the Congressional Black Caucus. When he retired
nearly two decades ago, his son, William Lacy Clay Jr., inherited the
seat and the loyalty of Black St. Louisans who have sent him back to
Washington every two years since.

By some estimates, a majority of voters in the city have never voted for
a congressman by any other name. Because Democrats so dominate this
district, the real contest is fought each term in the Democratic
primary, not the general election.

Mr. Clay is not bashful about his seniority in the Black Caucus and
among the intensely hierarchical House Democratic Caucus, arguing that
his easy access to the levers of power helps his district. He has the
backing of Speaker Nancy Pelosi and many of the party's establishment
pillars, like the Planned Parenthood Action Fund.

``There is no substitute in life for substance,'' he said. ``Substance
is so relevant to people. That's why there's been a Clay there for the
last 52 years.''

After
\href{https://www.nytimes.com/elections/results/missouri-house-district-1-primary-election}{falling
about 20 points short against Mr. Clay in 2018}, Ms. Bush has come back
with a better-funded and more aggressive campaign. A documentary that
chronicled her 2018 campaign, as well as those of Ms. Ocasio-Cortez and
other progressives,
``\href{https://www.nytimes.com/2019/04/30/movies/knock-down-the-house-review.html}{Knock
Down the House},'' helped burnish her profile. Perhaps more important,
her aides argue that in the current moment of national upheaval, more
voters are beginning to understand the need for policy prescriptions she
has long championed, like Medicare for all, a \$15-an-hour minimum wage,
a universal basic income and the wholesale dismantling of police
departments.

At campaign events, Ms. Bush speaks vividly about her own battle with
the coronavirus this spring --- how her fingertips turned blue as she
was deprived of oxygen, and her fear of the medical bills that would
follow her two hospital stays --- to bolster her arguments.

Image

After falling to Mr. Clay in 2018, Ms. Bush has come back with a
better-funded and more aggressive campaign.Credit...MichaelB Thomas for
The New York Times

But her case against Mr. Clay centers on the protests that have rippled
through St. Louis after the deaths of George Floyd and Breonna Taylor at
the hands of the police in Minneapolis and Louisville. Ms. Bush's brand
of politics is built on being on the streets with everyday people. By
her account, Mr. Clay has simply not showed up and proposed only
half-measures to fix things.

``When we were getting our butts kicked and I was maced in the face in
Florissant a few weeks ago and people were getting beat on by police
officers --- no, no,'' she said. ``Did he show up the next day to say,
`You can't do this in my district, you can't treat people this way?' No,
no.''

That message has won Ms. Bush the avid backing of a loosely affiliated
coalition of activists, young people of color and white progressives
enraged by the events of the last several months.

Jasnaam Singh, 23, who was among almost a dozen volunteers who showed up
at a school parking lot in Ferguson to canvass for Ms. Bush on an
unseasonably cool Saturday morning, said he first encountered her
through a network of supporters for Senator Bernie Sanders of Vermont
and then noticed her showing up on the streets again and again this
summer, after the death of Mr. Floyd.

``Right then and there, I knew that she was a voice that the movement
desperately needed to be heard in D.C.,'' he said.

For those who have tended the protest movement since Ferguson ---
watching as Black activists and reformers have slowly gained footholds
in City Hall here, the St. Louis County prosecutor's office and in
Jefferson City, the state capital --- a victory by Ms. Bush would mark a
milestone of another magnitude.

``She would fit right in as somebody who is pushing for the systematic
change that we need, and not the small tedious change that we see,''
said Rasheen Aldridge, an activist who won a seat in the Missouri
Legislature last year.

The challenge for Ms. Bush has been persuading more moderate voters ---
Black and otherwise --- to take a chance on a relative political novice
who is unapologetically pushing for far-left policies like defunding the
police.

Mr. Clay may not be wildly popular in the district, political analysts
said, but he is not unpopular either and is considered a safe choice
among the older Black voters who make up the voting base. And unlike
Representative Joseph Crowley, who lost to Ms. Ocasio-Cortez in 2018, or
Mr. Engel, Mr. Clay is a Black man in a plurality Black district.

Image

Mr. Clay has the backing of Speaker Nancy Pelosi and many of his party's
establishment pillars.Credit...Andrew Harnik/Associated Press

Antonio French, a former alderman and mayoral candidate from St. Louis's
North Side, said he had noticed a ``disconnect'' between the progressive
politics of many white voters and young activists of color rallying
behind Ms. Bush and those of Black voters in his neighborhood.

``Defund and get rid of the police is not a message I hear from average
voters in my ward or districts like mine,'' Mr. French said. ``It's
quite the opposite. If you go to a Black neighborhood ward meeting,
primarily you are hearing people complain about the lack of police in
that neighborhood.''

Mr. Clay points out that after Ferguson, he encouraged the Justice
Department to investigate the city's Police Department and force
changes. He wrote key provisions of the sweeping police overhaul bill
House Democrats passed this summer and heads an influential subcommittee
with jurisdiction over eviction law.

And even as he lacerates Ms. Bush, Mr. Clay has made direct overtures to
her allies. He signed onto the Green New Deal, the liberal climate and
economic agenda that has been among their most prominent demands. It
appears to have worked. Ms. Ocasio-Cortez,
\href{https://www.nytimes.com/2018/08/05/us/politics/st-louis-cori-bush-ocasio-cortez.html}{who
campaigned for Ms. Bush in 2018}, has sat on the sidelines this time
around.

Ms. Bush rejected Mr. Clay's efforts to portray her campaign as racially
divisive and accused him of using dirty tactics with ``racist
undertones'' when he sent a mailer including an image of Ms. Bush
altered to make her skin appear darker.

Justice Democrats backed only two challengers to Black incumbents this
cycle. Both were Black themselves.

``No one is targeting C.B.C. members,'' Ms. Bush said. ``What they are
targeting is people who are not doing the work of the communities ---
and communities are suffering.''

Advertisement

\protect\hyperlink{after-bottom}{Continue reading the main story}

\hypertarget{site-index}{%
\subsection{Site Index}\label{site-index}}

\hypertarget{site-information-navigation}{%
\subsection{Site Information
Navigation}\label{site-information-navigation}}

\begin{itemize}
\tightlist
\item
  \href{https://help.nytimes.com/hc/en-us/articles/115014792127-Copyright-notice}{©~2020~The
  New York Times Company}
\end{itemize}

\begin{itemize}
\tightlist
\item
  \href{https://www.nytco.com/}{NYTCo}
\item
  \href{https://help.nytimes.com/hc/en-us/articles/115015385887-Contact-Us}{Contact
  Us}
\item
  \href{https://www.nytco.com/careers/}{Work with us}
\item
  \href{https://nytmediakit.com/}{Advertise}
\item
  \href{http://www.tbrandstudio.com/}{T Brand Studio}
\item
  \href{https://www.nytimes.com/privacy/cookie-policy\#how-do-i-manage-trackers}{Your
  Ad Choices}
\item
  \href{https://www.nytimes.com/privacy}{Privacy}
\item
  \href{https://help.nytimes.com/hc/en-us/articles/115014893428-Terms-of-service}{Terms
  of Service}
\item
  \href{https://help.nytimes.com/hc/en-us/articles/115014893968-Terms-of-sale}{Terms
  of Sale}
\item
  \href{https://spiderbites.nytimes.com}{Site Map}
\item
  \href{https://help.nytimes.com/hc/en-us}{Help}
\item
  \href{https://www.nytimes.com/subscription?campaignId=37WXW}{Subscriptions}
\end{itemize}
