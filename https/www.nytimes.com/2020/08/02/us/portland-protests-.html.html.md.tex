Sections

SEARCH

\protect\hyperlink{site-content}{Skip to
content}\protect\hyperlink{site-index}{Skip to site index}

\href{https://www.nytimes.com/section/us}{U.S.}

\href{https://myaccount.nytimes.com/auth/login?response_type=cookie\&client_id=vi}{}

\href{https://www.nytimes.com/section/todayspaper}{Today's Paper}

\href{/section/us}{U.S.}\textbar{}As Federal Agents Retreat in Portland,
Protesters Return to Original Foe: Local Police

\url{https://nyti.ms/39TxUzZ}

\begin{itemize}
\item
\item
\item
\item
\item
\end{itemize}

\href{https://www.nytimes.com/news-event/george-floyd-protests-minneapolis-new-york-los-angeles?action=click\&pgtype=Article\&state=default\&region=TOP_BANNER\&context=storylines_menu}{Race
and America}

\begin{itemize}
\tightlist
\item
  \href{https://www.nytimes.com/2020/07/26/us/protests-portland-seattle-trump.html?action=click\&pgtype=Article\&state=default\&region=TOP_BANNER\&context=storylines_menu}{Protesters
  Return to Other Cities}
\item
  \href{https://www.nytimes.com/2020/07/24/us/portland-oregon-protests-white-race.html?action=click\&pgtype=Article\&state=default\&region=TOP_BANNER\&context=storylines_menu}{Portland
  at the Center}
\item
  \href{https://www.nytimes.com/2020/07/23/podcasts/the-daily/portland-protests.html?action=click\&pgtype=Article\&state=default\&region=TOP_BANNER\&context=storylines_menu}{Podcast:
  Showdown in Portland}
\item
  \href{https://www.nytimes.com/interactive/2020/07/16/us/black-lives-matter-protests-louisville-breonna-taylor.html?action=click\&pgtype=Article\&state=default\&region=TOP_BANNER\&context=storylines_menu}{45
  Days in Louisville}
\end{itemize}

Advertisement

\protect\hyperlink{after-top}{Continue reading the main story}

Supported by

\protect\hyperlink{after-sponsor}{Continue reading the main story}

\hypertarget{as-federal-agents-retreat-in-portland-protesters-return-to-original-foe-local-police}{%
\section{As Federal Agents Retreat in Portland, Protesters Return to
Original Foe: Local
Police}\label{as-federal-agents-retreat-in-portland-protesters-return-to-original-foe-local-police}}

While protests around the federal courthouse have remained calm for
three consecutive nights, Portland police officers chased demonstrators
through the streets near a local precinct.

\includegraphics{https://static01.nyt.com/images/2020/08/02/us/00Portland-protests-3/00Portland-protests-3-articleLarge.jpg?quality=75\&auto=webp\&disable=upscale}

\href{https://www.nytimes.com/by/mike-baker}{\includegraphics{https://static01.nyt.com/images/2020/05/19/reader-center/author-mike-baker/author-mike-baker-thumbLarge.png}}

By \href{https://www.nytimes.com/by/mike-baker}{Mike Baker}

\begin{itemize}
\item
  Aug. 2, 2020
\item
  \begin{itemize}
  \item
  \item
  \item
  \item
  \item
  \end{itemize}
\end{itemize}

PORTLAND, Ore. --- Late on Saturday night, with protests in Portland
continuing into their third month, one crowd of demonstrators gathered
yet again in front of the city's fortified federal courthouse while
another group traveled miles east to a precinct used by local law
enforcement.

At the federal courthouse, the crowd saw a third consecutive night of
calm since
\href{https://www.nytimes.com/2020/07/29/us/protests-portland-federal-withdrawal.html}{the
start of a plan to withdraw federal agents} who had brought a
militarized crackdown to the city. But at the police precinct, officers
pointed bright lights into the crowd, warned protesters to disperse,
then chased them through the streets, knocking people to the ground,
using pepper spray and making arrests.

While the arrival of federal agents wearing camouflage last month
outraged protesters and local government leaders alike, their presence
also masked the more personal grievances that protesters have long had
with their local police force.

Gia Naranjo-Rivera, who had been protesting for weeks, said that while
she was appalled by the arrival of federal agents, she believed local
police officers brought their own form of hyper-militarization and
repressive tactics. She was arrested on Thursday by local police after
breaking caution tape the police had put up to close two locked parks
next to the federal courthouse.

Ms. Naranjo-Rivera said the protests needed to continue.

``If this movement doesn't succeed right now, we are just kicking the
can down the road to the next civil rights uprising,'' Ms.
Naranjo-Rivera said.

The city's protests in June were largely about local policing, with
crowds denouncing a criminal justice system that disproportionately
harms Black people and a Portland Police Bureau that has embraced
aggressive tactics to contain unruly crowds. The police have said the
crowd had flung objects at officers, including bottles and fireworks.

Mayor Ted Wheeler, who serves as police commissioner and is largely
reviled among protesters, said last week that he believed police had at
times made mistakes in the past, including using tear gas
indiscriminately, and he hoped the departure of federal officers brought
a chance to bring renewed peace.

``My hope is we will all do an outstanding job of de-escalating
tensions,'' Mr. Wheeler said.

At the precinct on Saturday night, the crowd stood on the street and
chanted. Officers set up bright lights to shine into the crowd, angering
some. When it appeared one of the officers had brought out what looked
like a camera to film the crowd, some protesters pointed lasers at the
device. Police said someone threw a glass jar or bottle at officers.

The protest crowds have remained much larger than they had been in the
days before federal agents had arrived.

While protest crowds numbered in the thousands in early June, those
figures waned over the month. But with the federal courthouse among the
targets of some of the remaining demonstrators, and President Trump
issuing an executive order to protect statues and government property
around the country, federal agents deployed at the beginning of July.

\includegraphics{https://static01.nyt.com/images/2020/08/02/us/00Portland-protests-2/merlin_175225215_b62eb4be-8665-4644-ac82-cb803c3d4294-articleLarge.jpg?quality=75\&auto=webp\&disable=upscale}

Their presence and tactics, including firing crowd-control munitions and
swinging batons, infuriated the city, drawing thousands out to the
streets once again to stand against what many saw as a troubling federal
incursion into a city that didn't want them. At that point, the
epicenter of the protests shifted from a county justice center to the
federal courthouse across the street.

Clashes at the courthouse, with nightly volleys of tear gas, continued
to draw more people out to stand against the federal presence, including
lines of mothers linking arms and a group of military veterans.

Last week, in an agreement between Gov. Kate Brown of Oregon and leaders
from the Department of Homeland Security, the state found a pathway to
withdraw federal agents, with Ms. Brown vowing to put state troopers in
place to provide security around the courthouse.

Since that plan went into effect on Thursday, there has been a minimal
law enforcement presence on the streets. Protesters have continued to
show up outside the fenced courthouse, chanting and giving speeches
around a bonfire in the middle of the street. While some have
occasionally thrown bottles over the fence toward the empty courthouse
entrance or burned flags, others in the crowd have confronted them to
keep things peaceful and focused on the Black Lives Matter cause that
drew millions to the streets after the death of George Floyd in
Minneapolis.

Demetria Hester, who led a group of women in chants in front of the
federal courthouse over the weekend, said she was going to continue
calling out people who lit fires, threw objects or burned flags.

``How is that helping?'' Ms. Hester said. ``The protest right now is
about Black Lives Matter. Burning a flag is not about Black Lives
Matter.''

Federal agents haven't fully retreated. Federal leaders, including
President Trump, have said the agents won't be gone until local
officials contain the unrest.

On Saturday night, as protesters downtown marched peacefully through the
streets, they noticed through the windows of a different federal
building that Homeland Security agents were standing inside watching
them. Some in the crowd stopped to flash lights through the window. One
agent appeared to respond by raising a middle finger.

Then the crowd continued on.

Advertisement

\protect\hyperlink{after-bottom}{Continue reading the main story}

\hypertarget{site-index}{%
\subsection{Site Index}\label{site-index}}

\hypertarget{site-information-navigation}{%
\subsection{Site Information
Navigation}\label{site-information-navigation}}

\begin{itemize}
\tightlist
\item
  \href{https://help.nytimes.com/hc/en-us/articles/115014792127-Copyright-notice}{©~2020~The
  New York Times Company}
\end{itemize}

\begin{itemize}
\tightlist
\item
  \href{https://www.nytco.com/}{NYTCo}
\item
  \href{https://help.nytimes.com/hc/en-us/articles/115015385887-Contact-Us}{Contact
  Us}
\item
  \href{https://www.nytco.com/careers/}{Work with us}
\item
  \href{https://nytmediakit.com/}{Advertise}
\item
  \href{http://www.tbrandstudio.com/}{T Brand Studio}
\item
  \href{https://www.nytimes.com/privacy/cookie-policy\#how-do-i-manage-trackers}{Your
  Ad Choices}
\item
  \href{https://www.nytimes.com/privacy}{Privacy}
\item
  \href{https://help.nytimes.com/hc/en-us/articles/115014893428-Terms-of-service}{Terms
  of Service}
\item
  \href{https://help.nytimes.com/hc/en-us/articles/115014893968-Terms-of-sale}{Terms
  of Sale}
\item
  \href{https://spiderbites.nytimes.com}{Site Map}
\item
  \href{https://help.nytimes.com/hc/en-us}{Help}
\item
  \href{https://www.nytimes.com/subscription?campaignId=37WXW}{Subscriptions}
\end{itemize}
