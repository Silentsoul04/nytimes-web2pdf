Sections

SEARCH

\protect\hyperlink{site-content}{Skip to
content}\protect\hyperlink{site-index}{Skip to site index}

\href{https://www.nytimes.com/section/us}{U.S.}

\href{https://myaccount.nytimes.com/auth/login?response_type=cookie\&client_id=vi}{}

\href{https://www.nytimes.com/section/todayspaper}{Today's Paper}

\href{/section/us}{U.S.}\textbar{}Tropical Storm Isaias, Grazing
Florida, Takes Aim at Carolinas

\url{https://nyti.ms/33wyGlB}

\begin{itemize}
\item
\item
\item
\item
\item
\end{itemize}

Advertisement

\protect\hyperlink{after-top}{Continue reading the main story}

Supported by

\protect\hyperlink{after-sponsor}{Continue reading the main story}

\hypertarget{tropical-storm-isaias-grazing-florida-takes-aim-at-carolinas}{%
\section{Tropical Storm Isaias, Grazing Florida, Takes Aim at
Carolinas}\label{tropical-storm-isaias-grazing-florida-takes-aim-at-carolinas}}

The Florida coast was spared severe damage on Sunday, but much of the
Eastern Seaboard is threatened with flooding rains.

\includegraphics{https://static01.nyt.com/images/2020/08/02/us/02storm-1/merlin_175215639_9253d98f-4c76-4648-9f8c-8436a3490656-articleLarge.jpg?quality=75\&auto=webp\&disable=upscale}

By \href{https://www.nytimes.com/by/rick-rojas}{Rick Rojas} and Rebecca
Halleck

\begin{itemize}
\item
  Aug. 2, 2020
\item
  \begin{itemize}
  \item
  \item
  \item
  \item
  \item
  \end{itemize}
\end{itemize}

ATLANTA
---~\href{https://www.nytimes.com/2020/08/03/us/isaias-storm-updates.html}{Tropical
Storm Isaias} buffeted Florida's eastern edge on Sunday with more heavy
rainfall and powerful winds as it skirted the Atlantic Coast, leaving
many people bracing for the threat of flash floods, storm surges and
even tornadoes as the storm made its way north.

The storm failed to deliver the punch in Florida that state officials
had feared. But that has not been enough to allay the concerns of
officials and residents in its path.

``It's a wait-and-see game,'' said Jay Slevin, the manager of a pizzeria
about a mile from the shore in Myrtle Beach, S.C., where the center of
the storm appeared to be heading.

Isaias, the ninth named storm in what has become a busy hurricane
season, has come at a time when many people in the Southeast are already
beleaguered by the coronavirus outbreak. Officials in the region are
juggling the response to a storm with a pandemic, and business owners
are wary of being dealt yet another crippling blow.

Isaias, which is written as Isaías in Spanish and pronounced
ees-ah-EE-ahs, clobbered the Bahamas with hurricane conditions over the
weekend after hitting parts of Puerto Rico and the Dominican Republic.

It was downgraded to a tropical storm on Saturday evening when its
sustained winds slipped below 74 m.p.h.; they were about 65 m.p.h. most
of Sunday but picked up again to 70 m.p.h. late in the afternoon.
Forecasters said some minor fluctuations in the strength of the storm
were possible over the next few days, and they posted hurricane watches
for areas in its immediate path and tropical storm watches all the way
to Rhode Island.

The storm, which has largely run parallel to the Florida coast since
leaving the Bahamas, is expected to give the Georgia coast only a
glancing blow but to strike the Carolinas more directly.

After pummeling the Bahamas for the better part of the weekend, the
storm blew away on Sunday morning, leaving parts of low-lying Grand
Bahama soaked with more than a foot of rain and other islands in the
archipelago with minor flooding, downed trees and power outages.

No storm-related deaths were reported in the country, which remains
haunted by the devastation caused last year by Hurricane Dorian. That
storm killed at least 74 people. Many storm victims are still living in
tents and damaged homes.

The coronavirus pandemic has made rebuilding more difficult and weakened
the country's tourism-dependent economy, leaving the Bahamas
particularly vulnerable this hurricane season.

\includegraphics{https://static01.nyt.com/images/2020/08/02/us/02storm-3/merlin_175235652_c156b00d-b2dc-4f9f-bd92-13e93c28f6df-articleLarge.jpg?quality=75\&auto=webp\&disable=upscale}

Tropical Storm Isaias's path now includes vacation destinations in the
Carolinas that are usually popular with tourists at this time of year
---~but the coronavirus outbreak has left them struggling.

The Two Meeting Street Inn, a waterfront bed-and-breakfast in
Charleston, S.C., closed in March because of the pandemic. It was
planning to reopen on Aug. 15, but that might now be delayed until
September. ``It's been devastating for us,'' said Julie Spell Roberts,
whose family has owned and operated the inn since 1946.

In preparation for the storm, her family has removed furniture from the
porch and cleared the property of anything that might break a window or
damage the inn --- measures that Ms. Spell Roberts called ``Stage 1.''

``It can change in a minute,'' she said of the weather. ``What we have
learned over time is that you're foolish if you don't think that Mother
Nature is a formidable foe, because she is.''

Myrtle Beach is preparing for a ``lower to moderate threat,'' said Steve
Pfaff, a National Weather Service meteorologist. Sustained winds are
expected to be around 50 to 60 miles an hour, with gusts of up to 70.

Those wind speeds can knock down trees, cause minor structural damage
and litter roads with debris. Rainfall will range from four to six
inches in most areas, with a few areas getting up to eight inches, which
could lead to flash flooding. Myrtle Beach will probably see the brunt
of the storm on Monday night, when the rain will increase and the risk
of flash floods will be greatest. There could also be a storm surge of
two to four feet, and a possibility of tornadoes.

Gov. Henry McMaster of South Carolina said on Friday that he did not
expect a need for an emergency declaration or mandatory evacuations.

``It looks like it will not be necessary --- we certainly hope not,''
Mr. McMaster, a Republican, said at a briefing. ``We're hoping this
storm will not hit us hard, if at all.''

But Gov. Roy Cooper of North Carolina, where the storm is expected to
arrive by Tuesday, has declared a state of emergency, with forecasters
predicting up to six inches of rain in some areas.
\href{https://alerts.weather.gov/cap/wwacapget.php?x=NC125F5DA2F1C4.HurricaneLocalStatement.125F5DA42AA8NC.ILMHLSILM.3d2f501b896f05f00166a758bf00288a}{Flash
flood warnings were in effect} in some areas on Sunday, and some islands
in the vulnerable Outer Banks were under mandatory evacuation orders.

``The storm continued its march toward North Carolina overnight,'' Mr.
Cooper, a Democrat, said in a briefing on Sunday. ``We're asking North
Carolinians in the storm's path to make sure they are prepared.''

The residents of inland counties saw
\href{https://www.nytimes.com/2018/09/18/us/north-carolina-hurricanes-storms-history.html}{flooding
inundate streets in 2016 during Hurricane Matthew and in 2018 during
Hurricane Florence}. The past few years of inclement weather have been
especially brutal for Carolinians.

Image

Utility crews worked on power lines that were damaged by the storm in
Palm Beach, Fla., on Sunday.Credit...Saul Martinez for The New York
Times

In Palm Beach County, Fla., the authorities said they had braced for
hurricane-force winds, but the storm caused only limited damage,
including some power outages and fallen trees.

``I think we can all agree, we've all been very fortunate, very lucky in
this county,'' Dave Kerner, the Palm Beach County mayor, said in a news
conference on Sunday.

Some Florida officials said that Isaias had served as somewhat of a
drill. Bill Johnson, the emergency management director for Palm Beach
County, described the combination of the coronavirus pandemic and a
tropical storm as ``something we have never done or been faced with
before.''

``We are blessed that Hurricane Isaias spared us of significant
damage,'' he said. ``I am pleased that this was more of an exercise than
a real event.''

Rick Rojas reported from Atlanta, and Rebecca Halleck from New York.
Melina Delkic contributed reporting from Cleveland; Christina Morales
from Hialeah. Fla.; Rachel Knowles Scott from Nassau, the Bahamas; and
Lucy Tompkins from Bozeman, Mont.

Advertisement

\protect\hyperlink{after-bottom}{Continue reading the main story}

\hypertarget{site-index}{%
\subsection{Site Index}\label{site-index}}

\hypertarget{site-information-navigation}{%
\subsection{Site Information
Navigation}\label{site-information-navigation}}

\begin{itemize}
\tightlist
\item
  \href{https://help.nytimes.com/hc/en-us/articles/115014792127-Copyright-notice}{©~2020~The
  New York Times Company}
\end{itemize}

\begin{itemize}
\tightlist
\item
  \href{https://www.nytco.com/}{NYTCo}
\item
  \href{https://help.nytimes.com/hc/en-us/articles/115015385887-Contact-Us}{Contact
  Us}
\item
  \href{https://www.nytco.com/careers/}{Work with us}
\item
  \href{https://nytmediakit.com/}{Advertise}
\item
  \href{http://www.tbrandstudio.com/}{T Brand Studio}
\item
  \href{https://www.nytimes.com/privacy/cookie-policy\#how-do-i-manage-trackers}{Your
  Ad Choices}
\item
  \href{https://www.nytimes.com/privacy}{Privacy}
\item
  \href{https://help.nytimes.com/hc/en-us/articles/115014893428-Terms-of-service}{Terms
  of Service}
\item
  \href{https://help.nytimes.com/hc/en-us/articles/115014893968-Terms-of-sale}{Terms
  of Sale}
\item
  \href{https://spiderbites.nytimes.com}{Site Map}
\item
  \href{https://help.nytimes.com/hc/en-us}{Help}
\item
  \href{https://www.nytimes.com/subscription?campaignId=37WXW}{Subscriptions}
\end{itemize}
