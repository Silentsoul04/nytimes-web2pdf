Sections

SEARCH

\protect\hyperlink{site-content}{Skip to
content}\protect\hyperlink{site-index}{Skip to site index}

\href{https://www.nytimes.com/section/world/asia}{Asia Pacific}

\href{https://myaccount.nytimes.com/auth/login?response_type=cookie\&client_id=vi}{}

\href{https://www.nytimes.com/section/todayspaper}{Today's Paper}

\href{/section/world/asia}{Asia Pacific}\textbar{}Militants Attack
Afghan Prison as Brief Cease-Fire Expires

\url{https://nyti.ms/3k3NH3Q}

\begin{itemize}
\item
\item
\item
\item
\item
\end{itemize}

Advertisement

\protect\hyperlink{after-top}{Continue reading the main story}

Supported by

\protect\hyperlink{after-sponsor}{Continue reading the main story}

\hypertarget{militants-attack-afghan-prison-as-brief-cease-fire-expires}{%
\section{Militants Attack Afghan Prison as Brief Cease-Fire
Expires}\label{militants-attack-afghan-prison-as-brief-cease-fire-expires}}

The Islamic State reportedly took responsibility for an assault at a
time when releasing insurgents from prisons has become a major issue in
the Afghan peace process.

\includegraphics{https://static01.nyt.com/images/2020/08/02/world/02afghan-prison-sub/02afghan-prison-sub-articleLarge-v2.jpg?quality=75\&auto=webp\&disable=upscale}

By Zabihullah Ghazi and
\href{https://www.nytimes.com/by/mujib-mashal}{Mujib Mashal}

\begin{itemize}
\item
  Published Aug. 2, 2020Updated Aug. 3, 2020, 1:23 a.m. ET
\item
  \begin{itemize}
  \item
  \item
  \item
  \item
  \item
  \end{itemize}
\end{itemize}

JALALABAD, Afghanistan --- Militants attacked a major prison in eastern
Afghanistan on Sunday, detonating a car bomb and waging a gun battle
against guards for hours, as dozens of inmates managed to escape, Afghan
officials said.

The attack, which continued into late Monday morning in the city of
Jalalabad, came as the issue of releasing insurgent fighters from prison
has moved to the forefront of efforts to strike a peace deal and end
Afghanistan's long war.

Disagreement over
\href{https://www.nytimes.com/2020/07/28/world/asia/afghanistan-cease-fire-taliban.html}{the
last batch of a prisoner release} has delayed the next steps of an
agreement reached in February between
\href{https://www.nytimes.com/2020/02/29/world/asia/us-taliban-deal.html}{the
United States and the Taliban}, and the start of direct talks between
the Taliban and the Afghan government.

But other militant groups that are also fighting the government were not
party to that agreement, and one of those, the Islamic State, claimed
responsibility for the attack on Sunday through its Amaq news agency.

Zabihullah Mujahid, a spokesman for the Taliban, said the group was not
behind the assault on the prison, which holds about 1,500 inmates.

Whoever the attackers were, they could put a strain on a fragile peace
process that has
\href{https://www.nytimes.com/2019/09/08/world/asia/afghanistan-trump-camp-david-taliban.html}{broken
down}
\href{https://www.nytimes.com/2020/04/07/world/asia/afghan-prisoner-talks-collapse.html}{several
times}, often with renewed violence.

The attack came during the final hours of a three-day cease-fire between
the Taliban and the Afghan government for the Muslim festival of Eid
al-Adha. Afghan officials said that violence during the cease-fire had
dropped significantly, with fewer than a dozen incidents reported over
the first two days.

At least 13 people were killed and 42 others wounded in the prison
attack so far, according to Attaullah Khogyani, a spokesman for the
government of Nangarhar Province.

Insurgents have repeatedly tried to stage prison breaks to free their
compatriots during the war,
\href{https://www.nytimes.com/2011/04/26/world/asia/26afghanistan.html}{sometimes
successfully}.

Nangarhar has been a stronghold of the Islamic State in Afghanistan.
Intense operations by Afghan forces, often backed by American air power,
significantly shrank the group's presence. Afghan officials said
Saturday that they had killed a senior leader of the group in the
province.

On Sunday, militants exploded a car bomb at the prison entrance in
Jalalabad before engaging in a gunfight with guards. One official said
militants were holed up in towers around the prison, while fighting was
reported inside.

The United States-Taliban deal called for the Afghan government to free
5,000 Taliban prisoners in exchange for 1,000 Taliban-held members of
the Afghan security forces.

The swap was supposed to take place early this year over the course of
10 days, after which the Taliban and the government were expected to sit
for direct negotiations.

The Afghan government at first resisted the prisoner release, and then
gave in to a phased release under much pressure from the Trump
administration. More recently, President Ashraf Ghani said he would not
release the last 400 of the 5,000 people on a list provided by the
Taliban, as they are accused of serious crimes.

While the Taliban have completed the release of the 1,000 prisoners they
had committed to, Mr. Ghani has offered a compromise: He is releasing
500 other Taliban members instead of the 400 on the list presented by
the insurgents, and he is calling a council of elders from across
Afghanistan to consult on whether to free the 400 accused of grave
crimes as well. The grand consultation, called a Loya Jirga, is expected
to happen this month.

It was not clear whether Mr. Ghani's compromise was acceptable to the
Taliban to open the way for direct negotiations, expected around Aug.
10.

Zabihullah Ghazi reported from Jalalabad, and Mujib Mashal from Kabul,
Afghanistan.

Advertisement

\protect\hyperlink{after-bottom}{Continue reading the main story}

\hypertarget{site-index}{%
\subsection{Site Index}\label{site-index}}

\hypertarget{site-information-navigation}{%
\subsection{Site Information
Navigation}\label{site-information-navigation}}

\begin{itemize}
\tightlist
\item
  \href{https://help.nytimes.com/hc/en-us/articles/115014792127-Copyright-notice}{©~2020~The
  New York Times Company}
\end{itemize}

\begin{itemize}
\tightlist
\item
  \href{https://www.nytco.com/}{NYTCo}
\item
  \href{https://help.nytimes.com/hc/en-us/articles/115015385887-Contact-Us}{Contact
  Us}
\item
  \href{https://www.nytco.com/careers/}{Work with us}
\item
  \href{https://nytmediakit.com/}{Advertise}
\item
  \href{http://www.tbrandstudio.com/}{T Brand Studio}
\item
  \href{https://www.nytimes.com/privacy/cookie-policy\#how-do-i-manage-trackers}{Your
  Ad Choices}
\item
  \href{https://www.nytimes.com/privacy}{Privacy}
\item
  \href{https://help.nytimes.com/hc/en-us/articles/115014893428-Terms-of-service}{Terms
  of Service}
\item
  \href{https://help.nytimes.com/hc/en-us/articles/115014893968-Terms-of-sale}{Terms
  of Sale}
\item
  \href{https://spiderbites.nytimes.com}{Site Map}
\item
  \href{https://help.nytimes.com/hc/en-us}{Help}
\item
  \href{https://www.nytimes.com/subscription?campaignId=37WXW}{Subscriptions}
\end{itemize}
