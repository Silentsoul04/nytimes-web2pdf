Sections

SEARCH

\protect\hyperlink{site-content}{Skip to
content}\protect\hyperlink{site-index}{Skip to site index}

\href{https://www.nytimes.com/section/world}{World}

\href{https://myaccount.nytimes.com/auth/login?response_type=cookie\&client_id=vi}{}

\href{https://www.nytimes.com/section/todayspaper}{Today's Paper}

\href{/section/world}{World}\textbar{}Birx Warns That U.S. Epidemic Is
in a `New Phase'

\url{https://nyti.ms/3i13OO4}

\begin{itemize}
\item
\item
\item
\item
\item
\end{itemize}

\href{https://www.nytimes.com/news-event/coronavirus?action=click\&pgtype=Article\&state=default\&region=TOP_BANNER\&context=storylines_menu}{The
Coronavirus Outbreak}

\begin{itemize}
\tightlist
\item
  live\href{https://www.nytimes.com/2020/08/03/world/coronavirus-covid-19.html?action=click\&pgtype=Article\&state=default\&region=TOP_BANNER\&context=storylines_menu}{Latest
  Updates}
\item
  \href{https://www.nytimes.com/interactive/2020/us/coronavirus-us-cases.html?action=click\&pgtype=Article\&state=default\&region=TOP_BANNER\&context=storylines_menu}{Maps
  and Cases}
\item
  \href{https://www.nytimes.com/interactive/2020/science/coronavirus-vaccine-tracker.html?action=click\&pgtype=Article\&state=default\&region=TOP_BANNER\&context=storylines_menu}{Vaccine
  Tracker}
\item
  \href{https://www.nytimes.com/2020/08/02/us/covid-college-reopening.html?action=click\&pgtype=Article\&state=default\&region=TOP_BANNER\&context=storylines_menu}{College
  Reopening}
\item
  \href{https://www.nytimes.com/live/2020/08/03/business/stock-market-today-coronavirus?action=click\&pgtype=Article\&state=default\&region=TOP_BANNER\&context=storylines_menu}{Economy}
\end{itemize}

Advertisement

\protect\hyperlink{after-top}{Continue reading the main story}

Supported by

\protect\hyperlink{after-sponsor}{Continue reading the main story}

\hypertarget{birx-warns-that-us-epidemic-is-in-a-new-phase}{%
\section{Birx Warns That U.S. Epidemic Is in a `New
Phase'}\label{birx-warns-that-us-epidemic-is-in-a-new-phase}}

Colleges are preparing to welcome students back to a complicated campus
setting. India's home minister tests positive.

\begin{itemize}
\item
  Aug. 2, 2020
\item
  \begin{itemize}
  \item
  \item
  \item
  \item
  \item
  \end{itemize}
\end{itemize}

This briefing has ended. Read live coronavirus updates
\href{https://www.nytimes.com/2020/08/03/world/coronavirus-covid-19.html}{here}.

\hypertarget{heres-what-you-need-to-know}{%
\subsubsection{Here's what you need to
know:}\label{heres-what-you-need-to-know}}

\begin{itemize}
\tightlist
\item
  \protect\hyperlink{link-4c637647}{Birx urges Americans in hot spots to
  consider wearing a mask at home if they live with someone who is
  especially vulnerable.}
\item
  \protect\hyperlink{link-18bff79b}{Scientists are worried about
  political influence over the U.S. coronavirus vaccine project.}
\item
  \protect\hyperlink{link-79292e3}{Will sex in a social pod be OK?
  Colleges prepare to welcome students back to a new reality.}
\item
  \protect\hyperlink{link-41ce99eb}{Manila returns to lockdown after
  opening up leads to a resurgence.}
\item
  \protect\hyperlink{link-74f9a2b2}{Its outbreak untamed, Melbourne,
  Australia, escalates a lockdown.}
\item
  \protect\hyperlink{link-1a62dbb2}{Lawmakers and White House officials
  remain at an impasse on a new relief package.}
\item
  \protect\hyperlink{link-479653b6}{Hurricane season in a pandemic:
  Florida is grazed by a tropical storm as virus cases surge.}
\end{itemize}

\includegraphics{https://static01.nyt.com/images/2020/08/02/business/02virus-briefing-birz/merlin_173957070_557fcd4f-d678-4732-b872-17a4142a718a-articleLarge.jpg?quality=75\&auto=webp\&disable=upscale}

\hypertarget{birx-urges-americans-in-hot-spots-to-consider-wearing-a-mask-at-home-if-they-live-with-someone-who-is-especially-vulnerable}{%
\subsection{Birx urges Americans in hot spots to consider wearing a mask
at home if they live with someone who is especially
vulnerable.}\label{birx-urges-americans-in-hot-spots-to-consider-wearing-a-mask-at-home-if-they-live-with-someone-who-is-especially-vulnerable}}

Dr. Deborah L. Birx, the Trump administration's coronavirus coordinator,
said on the CNN program ``State of the Union'' on Sunday that the
country is in a ``new phase'' of the coronavirus pandemic, and that it
is much more extensive than the spring outbreaks in major cities like
New York and Seattle.

She recommended that people living in communities where cases are
surging consider wearing a mask at home if they live with someone who is
especially vulnerable because of age or underlying medical conditions.

``What we are seeing today is different from March and April. It is
extraordinarily widespread,'' Dr. Birx said, adding that rural areas
have not been spared. ``So everybody who lives in a rural area, you are
not immune.''

She emphasized the significance of asymptomatic transmission. ``If you
have an outbreak in your rural area or in your city, you need to really
consider wearing a mask at home, assuming that you're positive if you
have individuals in your household with co-morbidities,'' she said.

Both she and Adm. Brett Giroir, an assistant secretary at the Department
of Health and Human Services, emphasized the importance of mask wearing,
hand washing and avoiding crowds. On the NBC program ``Meet the Press,''
Admiral Giroir said some of the efforts seemed to be helping in recent
weeks to reduce the number of cases in Arizona and some other states
that have been hard hit this summer.

He repeatedly returned to mask wearing as perhaps the most effective
preventive measure in communities experiencing an outbreak. ``Wearing a
mask is incredibly important, but we have to have like 85 or 90 percent
of individuals wearing a mask and avoiding crowds,'' he said. Those
percentages, he said, give ``you the same outcome as a complete
shutdown.''

Asked if he was recommending a national mask mandate, Admiral Giroir
said, ``The public health message is we've got to have mask wearing.''
He added: ``If we don't do that, and if we don't limit the indoor
crowded spaces, the virus will continue to run.''

Dr. Ashish Jha, the director of the Harvard Global Health Institute,
said on CNN that in many areas where cases are surging, the availability
of tests was badly lagging. ``In 18, 20 states, the number of tests
being done is actually falling and falling because our testing system is
under such strain that we just can't even deliver the test today that we
were doing two weeks ago. That's very concerning because when cases are
rising, and your number of tests are falling, that's a recipe for
disaster,'' he said.

Admiral Giroir defended the nation's testing program, noting it has been
increased exponentially in recent months. He said that both testing and
\href{https://www.nytimes.com/2020/07/31/health/covid-contact-tracing-tests.html}{contact
tracing} were crucial responses, but not particularly helpful in large,
communitywide outbreaks.

He was also asked about the president's endorsement
\href{https://www.nytimes.com/2020/07/28/technology/virus-video-trump.html}{of
the anti-malarial drug hydroxychloroquine} as a treatment for the virus.
He discounted its utility, citing results from several recent clinical
trials that showed no benefit.

``At this point in time we don't recommend that as a treatment,'' he
said. ``There's no evidence to show that it is.''

\hypertarget{scientists-are-worried-about-political-influence-over-the-us-coronavirus-vaccine-project}{%
\subsection{Scientists are worried about political influence over the
U.S. coronavirus vaccine
project.}\label{scientists-are-worried-about-political-influence-over-the-us-coronavirus-vaccine-project}}

Image

President Trump has been relentlessly promoting the administration's
vaccine efforts, including during an appearance at a biotechnology
laboratory in North Carolina last week.Credit...Anna Moneymaker for The
New York Times

In April, with hospitals overwhelmed and much of the United States in
lockdown, the Department of Health and Human Services produced a
presentation for the White House arguing that rapid development of a
\href{https://www.nytimes.com/interactive/2020/science/coronavirus-vaccine-tracker.html}{coronavirus
vaccine} was the best hope to control the pandemic.

``DEADLINE: Enable broad access to the public by October 2020,'' the
first slide read, with the date in bold.

Given that it typically takes years to develop a vaccine, the timetable
for the initiative, called
\href{https://www.nytimes.com/2020/04/29/us/politics/trump-coronavirus-vaccine-operation-warp-speed.html}{Operation
Warp Speed}, was incredibly ambitious. With tens of thousands dying and
tens of millions out of work, the crisis demanded an all-out
public-private response, with the government supplying billions of
dollars to pharmaceutical and biotechnology companies, providing
logistical support and cutting through red tape.

It escaped no one that the proposed deadline also intersected nicely
with President Trump's need to curb the virus before the election in
November.

The ensuing race for a vaccine --- in the middle of a campaign in which
the president's handling of the pandemic is the key issue after he has
spent his time in office
\href{https://www.nytimes.com/2020/04/28/climate/trump-coronavirus-climate-science.html}{undermining
science} and
\href{https://www.nytimes.com/2020/07/09/climate/trump-hurricane-dorian-noaa.html}{the
expertise of the federal bureaucracy} --- is now testing the system set
up to ensure safe and effective drugs to a degree never before seen.

Under
\href{https://www.nytimes.com/2020/08/02/us/politics/coronavirus-vaccine.html}{constant
pressure from a White House anxious for good news} and a public
desperate for a silver bullet to end the crisis, the government's
researchers are fearful of political intervention in the coming months
and are struggling to ensure that the government maintains the right
balance between speed and rigorous regulation, according to interviews
with administration officials, federal scientists and outside experts.

\hypertarget{will-sex-in-a-social-pod-be-ok-colleges-prepare-to-welcome-students-back-to-a-new-reality}{%
\subsection{Will sex in a social pod be OK? Colleges prepare to welcome
students back to a new
reality.}\label{will-sex-in-a-social-pod-be-ok-colleges-prepare-to-welcome-students-back-to-a-new-reality}}

Image

Students moving into North Carolina State University in Raleigh, N.C.,
on Friday.Credit...Gerry Broome/Associated Press

As many U.S. colleges plan to welcome back students this month, they
face challenges unlike any other industry --- containing the coronavirus
among a young, carefree population that not only studies together, but
also lives together, parties together and, if decades of history are any
guide, sleeps together.

It will be a complex endeavor requiring far more than just the
reconfiguring of dorm rooms and cafeterias. It also involves coronavirus
testing programs capable of serving communities the size of small cities
and the enforcement of codes of conduct among students not eager to be
policed.

\hypertarget{latest-updates-global-coronavirus-outbreak}{%
\section{\texorpdfstring{\href{https://www.nytimes.com/2020/08/03/world/coronavirus-covid-19.html?action=click\&pgtype=Article\&state=default\&region=MAIN_CONTENT_1\&context=storylines_live_updates}{Latest
Updates: Global Coronavirus
Outbreak}}{Latest Updates: Global Coronavirus Outbreak}}\label{latest-updates-global-coronavirus-outbreak}}

Updated 2020-08-04T07:33:06.428Z

\begin{itemize}
\tightlist
\item
  \href{https://www.nytimes.com/2020/08/03/world/coronavirus-covid-19.html?action=click\&pgtype=Article\&state=default\&region=MAIN_CONTENT_1\&context=storylines_live_updates\#link-4547638f}{Fauci
  defends Birx after she is criticized by Trump.}
\item
  \href{https://www.nytimes.com/2020/08/03/world/coronavirus-covid-19.html?action=click\&pgtype=Article\&state=default\&region=MAIN_CONTENT_1\&context=storylines_live_updates\#link-15e7f995}{Trump
  derides Democrats as lawmakers and administration officials try to
  break stimulus impasse.}
\item
  \href{https://www.nytimes.com/2020/08/03/world/coronavirus-covid-19.html?action=click\&pgtype=Article\&state=default\&region=MAIN_CONTENT_1\&context=storylines_live_updates\#link-e5a2cda}{The
  deadline for 2020 census counting has been moved up by a month.}
\end{itemize}

\href{https://www.nytimes.com/2020/08/03/world/coronavirus-covid-19.html?action=click\&pgtype=Article\&state=default\&region=MAIN_CONTENT_1\&context=storylines_live_updates}{See
more updates}

More live coverage:
\href{https://www.nytimes.com/live/2020/08/03/business/stock-market-today-coronavirus?action=click\&pgtype=Article\&state=default\&region=MAIN_CONTENT_1\&context=storylines_live_updates}{Markets}

Colleges are mapping strategies as varied as the contrasting Covid
regulations enacted by the states, and the efforts could add more than
\$70 billion to the budgets of the nation's 5,000 colleges.

Yet administrators say giving students at least a taste of college life
is worth it, if done in a safe way. Whether those constituents agree is
an open question, and complaints about tuition have led a growing number
of schools to offer rebates.

In one of the more elaborate plans, the University of California,
Berkeley, will test all residential **** students within 24 hours of
their arrival. After that, everyone living on campus will be tested
twice a month if its test proves accurate enough.

But Cornell College in Iowa, with only 1,000 students, is counting on
its humble health center and county health department to do its testing.
Small schools in similar situations are finding themselves at the mercy
of private labs that can take two weeks to deliver results, making
results almost meaningless.

It is still possible that the frantic planning will come to naught.

\hypertarget{manila-returns-to-lockdown-after-opening-up-leads-to-a-resurgence}{%
\subsection{Manila returns to lockdown after opening up leads to a
resurgence.}\label{manila-returns-to-lockdown-after-opening-up-leads-to-a-resurgence}}

Image

A coronavirus testing facility in the Philippines last
week.Credit...Mark R Cristino/EPA, via Shutterstock

President Rodrigo Duterte of the Philippines on Sunday ordered Manila
and nearby suburban areas to re-enter lockdown for two weeks as the
health department reported 5,032 new cases.

Group gatherings were prohibited, and the population was advised to stay
home. Public transportation was halted, domestic flights and
inter-island ferries remained suspended, and the government encouraged
biking. Schools will remain shut.

Infections spiked after the government eased lockdown rules and
gradually opened up in an effort to jump start the economy. But instead
of managing the numbers, it has resulted in grim results, with hospitals
overwhelmed and doctors warning they were reaching a breaking point. In
an appeal to the government on Saturday, the Philippine College of
Physicians, the country's main organization of doctors, warned that the
health system ``has been overwhelmed.''

This came shortly after Manila's city government ordered the temporary
closure of its two hospitals, citing the growing number of health care
staff members who have been infected. It said that the city's health
care workers are burned out ``with the seemingly endless number of
patients trooping to our hospitals for emergency care and admission.''

Total cases in the country now stand at 103,185, with 2,059 deaths.

Mr. Duterte told officials to ``strictly enforce the quarantine,
especially the lockdown.''

``I have heard the call of different groups from the medical community
for a two-week enhanced community quarantine in mega Manila,'' he said.
``I fully understand why your health workers would like to ask for such
a timeout period. They have been in the front lines for months and are
exhausted.''

global roundup

\hypertarget{its-outbreak-untamed-melbourne-australia-escalates-a-lockdown}{%
\subsection{Its outbreak untamed, Melbourne, Australia, escalates a
lockdown.}\label{its-outbreak-untamed-melbourne-australia-escalates-a-lockdown}}

\includegraphics{https://static01.nyt.com/images/2020/08/02/world/02virus-briefing-melbourne/merlin_175141296_dccccb19-1179-4660-a45f-369462798c5f-videoSixteenByNine3000.jpg}

Officials in Melbourne, Australia's second-largest city, announced
stricter measures on Sunday in an effort to stem a coronavirus outbreak
that is raging despite a lockdown that began four weeks ago.

For six weeks starting on Sunday, residents of metropolitan Melbourne
will be under curfew from 8 p.m. to 5 a.m. except for purposes of work
or giving and receiving care.

As under the current lockdown, permitted reasons for leaving the house
include: shopping for essential goods and services; medical care and
caregiving; and necessary exercise, work and study. Food shopping is
limited to one person per household per day, and outdoor exercise is
limited to one hour per person per day, both within about three miles of
home. Public gatherings are limited to two people, including household
members.

In explaining the new measures, Premier Daniel Andrews said the high
rate of community transmission, including 671 new cases reported in the
state of Victoria on Sunday, suggested that the virus was more
widespread than had been known.

``You've got to err on the side of caution and go further and go
harder,'' he said.

Victoria has had 11,557 confirmed cases, almost all of them in
metropolitan Melbourne, and 123 deaths.

Here is what else is happening around the world:

\begin{itemize}
\item
  \textbf{Kosovo}'s prime minister Avdullah Hoti said on Sunday he has
  contracted Covid-19 and will self-isolate at home for two weeks,
  \href{https://www.reuters.com/article/us-health-coronavirus-kosovo-primeminist/kosovo-prime-minister-says-he-has-covid-19-idUSKBN24Y0ON}{Reuters
  reports}. ``I have no signs, except a very easy cough,''
  \href{https://www.facebook.com/avdullah.hoti/posts/3508635622531100}{he
  wrote on Facebook}. He said he will be ``in isolation'' for two weeks
  and ``fulfill my obligations from home.''
\item
  Many \textbf{Ethiopians} who found work in other parts of Africa or in
  the Persian Gulf before the pandemic are
  \href{https://www.nytimes.com/2020/08/01/world/africa/ethiopian-migrant-workers-coronavirus.html?action=click\&module=RelatedLinks\&pgtype=Article}{heading
  home unemployed and possibly infected} with the virus, representing a
  major strain on Ethiopia's fragile health system. More than 30,000
  laborers have re-entered Ethiopia since mid-March. Of those, at least
  927 had the virus, according to the government, though that figure has
  not been updated in over a month and is almost certainly an
  undercount.
\item
  India's biggest film star, \textbf{Amitabh Bachchan}, was discharged
  from the hospital on Sunday after recovering from Covid-19, and the
  country's powerful home minister, Amit Shah, announced that he tested
  positive. Mr. Bachchan, 77, was hospitalized for three weeks. His son,
  Abhishek Bachchan, also a movie star, remains in the hospital. Mr.
  Shah announced on Twitter on Sunday he tested positive. The
  announcement came one day after his government allowed the reopening
  of hotels and weekly markets in New Delhi, one of the worst-hit
  regions in the country.
\end{itemize}

\hypertarget{lawmakers-and-white-house-officials-remain-at-an-impasse-on-a-new-relief-package}{%
\subsection{Lawmakers and White House officials remain at an impasse on
a new relief
package.}\label{lawmakers-and-white-house-officials-remain-at-an-impasse-on-a-new-relief-package}}

Image

Members of the Army oversaw a drive-through coronavirus testing site in
Opelousas, La., on Thursday.Credit...William Widmer for The New York
Times

With coronavirus cases soaring across the United States, the debate in
Washington over a
\href{https://www.nytimes.com/2020/07/28/us/politics/coronavirus-relief-bills-house-senate.html}{new
relief package} to help people and businesses weather the crisis is set
to take center stage in the coming week, and negotiators were meeting
over the weekend in hopes of making progress on a deal.

``The president's determined to spend what we need to spend,'' said
Steven Mnuchin, the Treasury secretary, speaking on the ABC program
``This Week.'' ``We're acting very quickly now.''

Unemployment benefits lapsed this week for tens of millions of people,
but officials have struggled to agree over new aid. Mr. Mnuchin's
remarks came after he and Mark Meadows, the White House chief of staff,
met with top congressional Democrats in a rare Saturday meeting on
Capitol Hill.

Speaker Nancy Pelosi, who hosted the meeting with Senator Chuck Schumer
of New York, said that staff members would meet on Sunday and that the
main negotiators would convene again on Monday. They called the
discussion on Saturday productive but said that the sides remained far
apart on several matters.

``We must defeat this virus, and that's one of the points that we still
have not come to any agreement on,'' Ms. Pelosi said, speaking on
\href{https://abcnews.go.com/ThisWeek/video/speaker-house-nancy-pelosi-72130729}{``This
Week.''} (Mr. Mnuchin, appearing afterward, refuted the suggestion that
the administration is not invested in defeating the virus.)

At issue is
\href{https://www.nytimes.com/2020/07/28/us/politics/coronavirus-relief-bills-house-senate.html}{the
gap between the latest relief packages} put forward by Democrats and
Republicans.

A \$1 trillion proposal issued by Senate Republicans and administration
officials last week includes cutting by two-thirds the \$600-per-week
unemployment payments that workers had received since April and
providing tax cuts and liability protections for businesses.

A \$3 trillion relief package approved by House Democrats in May
includes an extension of the jobless aid, nearly \$200 billion for
rental assistance and mortgage relief, \$3.6 billion to bolster election
security and additional aid for food assistance.

U.S. Roundup

\hypertarget{hurricane-season-in-a-pandemic-florida-is-grazed-by-a-tropical-storm-as-virus-cases-surge}{%
\subsection{Hurricane season in a pandemic: Florida is grazed by a
tropical storm as virus cases
surge.}\label{hurricane-season-in-a-pandemic-florida-is-grazed-by-a-tropical-storm-as-virus-cases-surge}}

Image

Tropical Storm Isaias approaching Palm Beach, Fla., on
Sunday.Credit...Saul Martinez for The New York Times

Virus-battered Florida is confronting a new challenge: Tropical Storm
Isaias, which is whipping the coast with high winds and creating the
risk of flash flooding as it makes its way up the East Coast.

At 2 p.m. Eastern time, the center of the storm was about 30 miles
offshore, east of Port St. Lucie, Fla., and was moving north-northwest
at about eight miles an hour, according to the
\href{https://www.nhc.noaa.gov/text/refresh/MIATCPAT4+shtml/020856.shtml?}{National
Hurricane Center}.

Isaias --- which is written Isaías in Spanish and pronounced
ees-ah-EE-ahs --- had clobbered the Bahamas with hurricane conditions on
Saturday after hitting parts of Puerto Rico and the Dominican Republic.
As it advances northward, the center of the storm is skirting close to
the coast of Florida without making landfall so far, but its track is
likely to bring it ashore in the Carolinas early in the week.

Complicating the emergency response to the storm, reported coronavirus
cases continue to rise sharply in Florida, Georgia and the Carolinas,
and health officials have warned that their health care systems could be
strained beyond capacity. To avoid virus exposure in shelters, the first
choice is for coastal residents in homes vulnerable to flooding to stay
with relatives or friends farther inland, being careful to wear masks
and remain socially distant.

``Because of Covid, we feel that you are safer at home,'' said Bill
Johnson, the emergency management director for Palm Beach County.
``Shelters should be considered your last resort.''

Here is what else is happening around the country:

\begin{itemize}
\item
  Lord \& Taylor, the floundering department store company, on Sunday
  became the latest retailer to
  \href{https://www.nytimes.com/2020/08/02/business/Lord-and-Taylor-Bankruptcy.html}{file
  for bankruptcy protection} as the coronavirus outbreak accelerates the
  demise of teetering chains. Lord \& Taylor and parent company Le Tote
  said in a filing on Sunday that they operated 38 locations, which had
  been temporarily closed since March 2020, and that they had about 651
  employees.
\item
  The two owners of the Liberty Belle, a party boat,
  \href{https://www.nytimes.com/2020/08/02/nyregion/liberty-belle-illegal-party.html}{were
  arrested on Saturday night} after it was used that day to host an
  event with more than 170 guests, violating state and local
  social-distancing rules, according to the New York Sheriff's office.
  State officials in recent weeks have increasingly cracked down on bars
  and other businesses that violate social-distancing and other safety
  measures. Dozens of businesses have had their liquor licenses
  suspended.
\item
  Five months after the coronavirus engulfed New York City, subway
  ridership is 20 percent of pre-pandemic levels, even as the city has
  largely contained the virus and reopened some businesses. But public
  transportation may not be as risky as New Yorkers believed. There has
  been no notable superspreader event linked to mass transit, according
  to a survey of transportation agencies
  \href{https://www.nytimes.com/2020/08/02/nyregion/nyc-subway-coronavirus-safety.html}{conducted
  by The New York Times}.
\end{itemize}

\hypertarget{a-fed-president-endorses-strict-lockdowns-to-avoid-many-more-job-losses-and-many-more-bankruptcies}{%
\subsection{A Fed president endorses strict lockdowns to avoid ``many
more job losses and many more
bankruptcies.''}\label{a-fed-president-endorses-strict-lockdowns-to-avoid-many-more-job-losses-and-many-more-bankruptcies}}

Image

Pedestrians in Las Vegas, where restrictions have been
eased.Credit...Bridget Bennett for The New York Times

A top economic official and the governor of Arkansas used appearances on
the Sunday talk shows to discuss the financial toll of the virus as it
rages through much of the country.

Neel Kashkari, the president of the Federal Reserve Bank of Minneapolis,
argued that it would be better for the economy if the United States
instituted strict lockdown policies for a month to six weeks to stop the
spread of the virus.

If the country cannot control the spread, ``we're going to have
flare-ups, lockdowns and a very halting recovery with many more job
losses and many more bankruptcies,'' Mr. Kashkari said on the CBS
program ``Face the Nation'' on Sunday.

``If we were to lock down hard for a month or six weeks, we could get
the case count down, so that our testing and our contact tracing was
actually enough to control it,'' he said. ``If we don't do that, and we
have this raging virus spreading throughout the country with flare-ups
and local lockdowns for the next year or two, which is entirely
possible, we're going to see many, many more business bankruptcies.''

He also said that given the low cost of issuing debt, the government has
room to spend to support the American economy.

``Congress should use this opportunity to support the American people,
and the American economy,'' he said. ``If we get the economy growing, we
will be able to pay off the debt.''

His argument for a longer shutdown stands in contrast to others' views.
On the CNN program ``State of the Union,'' Gov. Asa Hutchinson of
Arkansas defended his decision not to impose a statewide stay-at-home
order earlier this year. Mr. Hutchinson emphasized the economic
ramifications of extended shutdowns.

\href{https://www.nytimes.com/news-event/coronavirus?action=click\&pgtype=Article\&state=default\&region=MAIN_CONTENT_3\&context=storylines_faq}{}

\hypertarget{the-coronavirus-outbreak-}{%
\subsubsection{The Coronavirus Outbreak
›}\label{the-coronavirus-outbreak-}}

\hypertarget{frequently-asked-questions}{%
\paragraph{Frequently Asked
Questions}\label{frequently-asked-questions}}

Updated August 3, 2020

\begin{itemize}
\item ~
  \hypertarget{im-a-small-business-owner-can-i-get-relief}{%
  \paragraph{I'm a small-business owner. Can I get
  relief?}\label{im-a-small-business-owner-can-i-get-relief}}

  \begin{itemize}
  \tightlist
  \item
    The
    \href{https://www.nytimes.com/article/small-business-loans-stimulus-grants-freelancers-coronavirus.html?action=click\&pgtype=Article\&state=default\&region=MAIN_CONTENT_3\&context=storylines_faq}{stimulus
    bills enacted in March} offer help for the millions of American
    small businesses. Those eligible for aid are businesses and
    nonprofit organizations with fewer than 500 workers, including sole
    proprietorships, independent contractors and freelancers. Some
    larger companies in some industries are also eligible. The help
    being offered, which is being managed by the Small Business
    Administration, includes the Paycheck Protection Program and the
    Economic Injury Disaster Loan program. But lots of folks have
    \href{https://www.nytimes.com/interactive/2020/05/07/business/small-business-loans-coronavirus.html?action=click\&pgtype=Article\&state=default\&region=MAIN_CONTENT_3\&context=storylines_faq}{not
    yet seen payouts.} Even those who have received help are confused:
    The rules are draconian, and some are stuck sitting on
    \href{https://www.nytimes.com/2020/05/02/business/economy/loans-coronavirus-small-business.html?action=click\&pgtype=Article\&state=default\&region=MAIN_CONTENT_3\&context=storylines_faq}{money
    they don't know how to use.} Many small-business owners are getting
    less than they expected or
    \href{https://www.nytimes.com/2020/06/10/business/Small-business-loans-ppp.html?action=click\&pgtype=Article\&state=default\&region=MAIN_CONTENT_3\&context=storylines_faq}{not
    hearing anything at all.}
  \end{itemize}
\item ~
  \hypertarget{what-are-my-rights-if-i-am-worried-about-going-back-to-work}{%
  \paragraph{What are my rights if I am worried about going back to
  work?}\label{what-are-my-rights-if-i-am-worried-about-going-back-to-work}}

  \begin{itemize}
  \tightlist
  \item
    Employers have to provide
    \href{https://www.osha.gov/SLTC/covid-19/standards.html}{a safe
    workplace} with policies that protect everyone equally.
    \href{https://www.nytimes.com/article/coronavirus-money-unemployment.html?action=click\&pgtype=Article\&state=default\&region=MAIN_CONTENT_3\&context=storylines_faq}{And
    if one of your co-workers tests positive for the coronavirus, the
    C.D.C.} has said that
    \href{https://www.cdc.gov/coronavirus/2019-ncov/community/guidance-business-response.html}{employers
    should tell their employees} -\/- without giving you the sick
    employee's name -\/- that they may have been exposed to the virus.
  \end{itemize}
\item ~
  \hypertarget{should-i-refinance-my-mortgage}{%
  \paragraph{Should I refinance my
  mortgage?}\label{should-i-refinance-my-mortgage}}

  \begin{itemize}
  \tightlist
  \item
    \href{https://www.nytimes.com/article/coronavirus-money-unemployment.html?action=click\&pgtype=Article\&state=default\&region=MAIN_CONTENT_3\&context=storylines_faq}{It
    could be a good idea,} because mortgage rates have
    \href{https://www.nytimes.com/2020/07/16/business/mortgage-rates-below-3-percent.html?action=click\&pgtype=Article\&state=default\&region=MAIN_CONTENT_3\&context=storylines_faq}{never
    been lower.} Refinancing requests have pushed mortgage applications
    to some of the highest levels since 2008, so be prepared to get in
    line. But defaults are also up, so if you're thinking about buying a
    home, be aware that some lenders have tightened their standards.
  \end{itemize}
\item ~
  \hypertarget{what-is-school-going-to-look-like-in-september}{%
  \paragraph{What is school going to look like in
  September?}\label{what-is-school-going-to-look-like-in-september}}

  \begin{itemize}
  \tightlist
  \item
    It is unlikely that many schools will return to a normal schedule
    this fall, requiring the grind of
    \href{https://www.nytimes.com/2020/06/05/us/coronavirus-education-lost-learning.html?action=click\&pgtype=Article\&state=default\&region=MAIN_CONTENT_3\&context=storylines_faq}{online
    learning},
    \href{https://www.nytimes.com/2020/05/29/us/coronavirus-child-care-centers.html?action=click\&pgtype=Article\&state=default\&region=MAIN_CONTENT_3\&context=storylines_faq}{makeshift
    child care} and
    \href{https://www.nytimes.com/2020/06/03/business/economy/coronavirus-working-women.html?action=click\&pgtype=Article\&state=default\&region=MAIN_CONTENT_3\&context=storylines_faq}{stunted
    workdays} to continue. California's two largest public school
    districts --- Los Angeles and San Diego --- said on July 13, that
    \href{https://www.nytimes.com/2020/07/13/us/lausd-san-diego-school-reopening.html?action=click\&pgtype=Article\&state=default\&region=MAIN_CONTENT_3\&context=storylines_faq}{instruction
    will be remote-only in the fall}, citing concerns that surging
    coronavirus infections in their areas pose too dire a risk for
    students and teachers. Together, the two districts enroll some
    825,000 students. They are the largest in the country so far to
    abandon plans for even a partial physical return to classrooms when
    they reopen in August. For other districts, the solution won't be an
    all-or-nothing approach.
    \href{https://bioethics.jhu.edu/research-and-outreach/projects/eschool-initiative/school-policy-tracker/}{Many
    systems}, including the nation's largest, New York City, are
    devising
    \href{https://www.nytimes.com/2020/06/26/us/coronavirus-schools-reopen-fall.html?action=click\&pgtype=Article\&state=default\&region=MAIN_CONTENT_3\&context=storylines_faq}{hybrid
    plans} that involve spending some days in classrooms and other days
    online. There's no national policy on this yet, so check with your
    municipal school system regularly to see what is happening in your
    community.
  \end{itemize}
\item ~
  \hypertarget{is-the-coronavirus-airborne}{%
  \paragraph{Is the coronavirus
  airborne?}\label{is-the-coronavirus-airborne}}

  \begin{itemize}
  \tightlist
  \item
    The coronavirus
    \href{https://www.nytimes.com/2020/07/04/health/239-experts-with-one-big-claim-the-coronavirus-is-airborne.html?action=click\&pgtype=Article\&state=default\&region=MAIN_CONTENT_3\&context=storylines_faq}{can
    stay aloft for hours in tiny droplets in stagnant air}, infecting
    people as they inhale, mounting scientific evidence suggests. This
    risk is highest in crowded indoor spaces with poor ventilation, and
    may help explain super-spreading events reported in meatpacking
    plants, churches and restaurants.
    \href{https://www.nytimes.com/2020/07/06/health/coronavirus-airborne-aerosols.html?action=click\&pgtype=Article\&state=default\&region=MAIN_CONTENT_3\&context=storylines_faq}{It's
    unclear how often the virus is spread} via these tiny droplets, or
    aerosols, compared with larger droplets that are expelled when a
    sick person coughs or sneezes, or transmitted through contact with
    contaminated surfaces, said Linsey Marr, an aerosol expert at
    Virginia Tech. Aerosols are released even when a person without
    symptoms exhales, talks or sings, according to Dr. Marr and more
    than 200 other experts, who
    \href{https://academic.oup.com/cid/article/doi/10.1093/cid/ciaa939/5867798}{have
    outlined the evidence in an open letter to the World Health
    Organization}.
  \end{itemize}
\end{itemize}

``We've got to take on two emergencies here,'' he said. ``One is our
virus, the other is the economy.''

\hypertarget{high-fiving-and-spitting-major-league-baseball-has-an-outbreak-the-commissioner-wants-players-to-behave}{%
\subsection{High-fiving and spitting: Major League Baseball has an
outbreak. The commissioner wants players to
behave.}\label{high-fiving-and-spitting-major-league-baseball-has-an-outbreak-the-commissioner-wants-players-to-behave}}

Image

Mets players Pete Alonso and Michael Conforto high fived after their
victory over the Boston Red Sox last week.Credit...Adam Glanzman/Getty
Images

Amid a slow but steady stream of new coronavirus cases, the Major League
Baseball season becoming more precarious seemingly by the day.

Then on Saturday the league's commissioner issued a rallying cry. ``We
are playing,'' Rob Manfred told ESPN. ``The players need to be better,
but I am not a quitter in general and there is no reason to quit now. We
have had to be fluid, but it is manageable.''

Players on many teams have been spotted high-fiving or spitting or
getting too close too often in the dugout --- all in violation of the
manual.

And already, 20 cases among the Miami Marlins and six among the St.
Louis Cardinals less than two weeks into the season have wreaked havoc
on the schedules of eight teams. It also raised questions about M.L.B.'s
protocols.

In saying the games would go on, Manfred thrust the onus on the players.

Kathleen Bachynski, an assistant professor of public health at
Muhlenberg College,
\href{https://twitter.com/bachyns/status/1289665507117772800}{took
issue} with Manfred's comments, writing on Twitter that the virus
thrives ``when people insist on sticking with a poor plan to the bitter
end.''

In a phone interview, Dr. Bachynski said ``the responsibility has to be
on the league to provide safe conditions to play in.''

Four players announced they would not play this season since the
Marlin's outbreak; a dozen players opted out of the season before
opening day.

On Sunday, Mets outfielder
\href{https://www.nytimes.com/2020/08/02/sports/baseball/Yoenis-cespedes-opt-out-rule.html}{Yoenis
Cespedes opted out} of the rest of the 2020 Major League Baseball season
for ``Covid-related'' reasons, Brodie Van Wagenen, the team's general
manager, announced Sunday afternoon. The announcement came after
Cespedes had failed to show up on Sunday for the Mets' game in Atlanta
against the Braves.

\hypertarget{russia-has-set-a-mass-vaccination-for-october-after-a-shortened-trial}{%
\subsection{Russia has set a mass vaccination for October after a
shortened
trial.}\label{russia-has-set-a-mass-vaccination-for-october-after-a-shortened-trial}}

Image

Russia is one of a number of countries rushing to develop and administer
a vaccine, and it is determined to get there first.Credit...Sechenov
Medical University Press Office, via Getty Images

\href{https://www.nytimes.com/2020/08/02/world/europe/russia-trials-vaccine-October.html}{Russia
plans to launch a nationwide vaccination campaign} in October with a
coronavirus vaccine that has yet to complete clinical trials, raising
international concern about the methods the country is using to compete
in the global race to inoculate the public.

The minister of health, Mikhail Murashko, said Saturday that the plan
was to begin by vaccinating teachers and health care workers. He also
\href{https://ria.ru/20200801/1575248763.html}{told the RIA} state news
agency that amid accelerated testing, the laboratory that developed the
vaccine was already seeking regulatory approval for it.

Russia is one of a number of countries rushing to develop and administer
a vaccine, and it is determined to get there first.

Not only would a vaccine help alleviate a worldwide health crisis that
has killed more than 680,000 people and badly wounded the global
economy, it would also become a symbol of national pride and a valuable
propaganda tool for the country that produces it. It could be a
lucrative commodity, as well.

``I do hope that the Chinese and the Russians are actually testing the
vaccine before they are administering the vaccine to anyone,'' Dr.
Anthony Fauci, director of the National Institute of Allergy and
Infectious Diseases in the United States, warned a congressional hearing
on Friday.

A Russian regulatory agency is expected to approve that vaccine for the
October campaign by mid-August, far earlier than timelines suggested by
Western regulators, who have often said a vaccine would become available
no sooner than the end of the year.

But with limited transparency in the Russian program, separating the
science from the politics and propaganda could prove impossible. Critics
have already drawn attention to Russia's tradition of cutting corners in
research on other pharmaceutical products and accusations of
intellectual property theft.

\hypertarget{slickers-face-masks-what-to-do-if-youre-caught-in-the-rain-during-the-pandemic}{%
\subsection{Slickers? Face masks? What to do if you're caught in the
rain during the
pandemic.}\label{slickers-face-masks-what-to-do-if-youre-caught-in-the-rain-during-the-pandemic}}

Image

People wearing masks during a storm in Nagaland, India, in
July.Credit...Yirmiyan Arthur/Associated Press

As the coronavirus has resurged in many parts of the country in recent
weeks, experts and politicians alike have implored people to protect
themselves and others by wearing a face mask in public.

Does that apply when you have to be out in the gusting wind and driving
rain of a tropical storm like Isaias? Our health columnist Tara
Parker-Pope says, probably not: Face masks
\href{https://www.nursingtimes.net/clinical-archive/infection-control/the-effectiveness-of-surgical-face-masks-what-the-literature-shows-30-09-2003/}{aren't
as effective} when they are wet.

For one thing, it's much harder to breathe through a wet mask than a dry
one, Ms. Parker-Pope notes. And on top of that, a moist or wet mask
doesn't filter as well as a dry mask. The Centers for Disease Control
and Prevention, which recommends mask-wearing in general, says they
\href{https://www.cdc.gov/coronavirus/2019-ncov/prevent-getting-sick/cloth-face-cover-guidance.html}{should
not be worn when doing things that may get the mask wet.}

It doesn't take a tropical storm to drench a mask, of course. They can
become soaked with condensation from your breath or sweat from your
face, and some people think of wetting them deliberately to cool off in
hot weather. But the harm done is the same, wherever the moisture comes
from.

A paper surgical mask that gets soaked should probably be discarded, Ms.
Parker-Pope advises, but a cloth mask can be washed, dried and reused.

If rain is coming down in buckets, social distancing is not likely to be
a problem, and any viral particles exhaled by an infected person
probably would be quickly diluted by gusting wind and rain. So there is
little need to wear a mask out in a rainstorm, Ms. Parker-Pope notes:
``In fact, you should take it off and keep it dry, so if you need to
duck into a store to wait out the storm, you have a dry mask to wear
indoors.''

\hypertarget{scientists-study-whether-people-with-the-virus-can-infect-bats-and-other-wildlife}{%
\subsection{Scientists study whether people with the virus can infect
bats and other
wildlife.}\label{scientists-study-whether-people-with-the-virus-can-infect-bats-and-other-wildlife}}

Image

A recent paper tracing the genetic lineage of the virus found evidence
that it probably evolved in bats into its current form.Credit...Kim Raff
for The New York Times

Could humans pass the coronavirus to wildlife, specifically North
American bats?

It may seem like a minor worry --- far down the list from concerns like
getting sick, losing a loved one or staying employed. But as the
pandemic has made clear, the more careful people are about viruses
passing among species, the better.

The scientific consensus is that the coronavirus originated in bats in
China or neighboring countries. A recent paper tracing the genetic
lineage of the virus
\href{https://www.nature.com/articles/s41564-020-0771-4}{found evidence
that it probably evolved in bats into its current form}. The researchers
also concluded that either this coronavirus or others that could make
the jump to humans may be present in bat populations.

So why worry about infecting more bats with the current virus?

The U.S. government considers it a legitimate concern both for bat
populations, which have been devastated by a fungal disease called
white-nose syndrome, and for humans, given potential problems down the
road. If the virus can pass easily between species, it could potentially
spill back over to humans.

Another concern is how readily the coronavirus may spread from bats to
other kinds of wildlife or domestic animals, including pets. Much
attention has been paid to the small number of pets that have been
infected, but public health authorities like the Centers for Disease
Control and Prevention have said that, although information is limited,
\href{https://www.cdc.gov/coronavirus/2019-ncov/daily-life-coping/pets.html}{the
risk of pets spreading the virus to people} is low.

They do recommend that any person who has Covid-19 take the same
precautions with their pets that they would with human family members.

\hypertarget{viewers-flock-to-mexicos-flagging-telenovelas-seeking-the-familiar-in-a-time-of-distress}{%
\subsection{Viewers flock to Mexico's flagging telenovelas, seeking the
familiar in a time of
distress.}\label{viewers-flock-to-mexicos-flagging-telenovelas-seeking-the-familiar-in-a-time-of-distress}}

Image

Mexico's Televisa network has continued to film telenovelas in the
pandemic.Credit...Meghan Dhaliwal for The New York Times

Mexico's love affair with melodrama appeared to be over. Now, thanks
partly to the pandemic, the telenovela is roaring back.

Confined to their homes, millions of Mexicans have devoted their
evenings to the traditional melodramas and other kitschy classics,
finding in the familiar faces and happy endings a balm for anxieties
raised by a health crisis that has left at least 43,000 dead and
millions unemployed in the country.

The resurgence has been a boon to Televisa, a onetime media monopoly
that had taken a beating from streaming services. During the second
quarter, 6.6 million people watched Televisa's flagship channel during
prime time each evening, when telenovelas and other melodramas air.
Viewership was around five million in that period last year, according
to the network.

Miguel Ángel Herros, the executive producer of the melodrama ``La Rosa
de Guadalupe,'' has been filming for shorter periods, in locations that
leave ample space for his crew. Actors have their temperatures taken
when they arrive on set, and rehearse with masks and face shields.

It is unclear whether the success will last through a pandemic that has
forced physical displays of affection out of telenovelas.

``There are no kisses, no hugs, no caresses, no scenes in bed,'' Mr.
Herros said.

\hypertarget{is-it-feasible-to-travel-this-year}{%
\subsection{Is it feasible to travel this
year?}\label{is-it-feasible-to-travel-this-year}}

Travel looks very different in 2020.
\href{https://www.nytimes.com/interactive/2020/07/31/travel/coronavirus-travel-risk.html}{Here
are some questions} to help you decide whether you would feel
comfortable taking a trip during the pandemic.

Reporting was contributed by Kevin Armstrong, Peter Baker, Benedict
Carey, Emily Cochrane, Melina Delkic, Tess Felder, Christina Goldbaum,
James Gorman, Jason Gutierrez, Andrew Higgins, Jennifer Jett, Annie
Karni, Natalie Kitroeff, Sharon LaFraniere, **** Patrick J. Lyons, Simon
Marks, Patricia Mazzei, Tara Parker-Pope, Kate Phillips, Jeanna Smialek,
Katie Thomas, Noah Weiland and Sameer Yasir.

Advertisement

\protect\hyperlink{after-bottom}{Continue reading the main story}

\hypertarget{site-index}{%
\subsection{Site Index}\label{site-index}}

\hypertarget{site-information-navigation}{%
\subsection{Site Information
Navigation}\label{site-information-navigation}}

\begin{itemize}
\tightlist
\item
  \href{https://help.nytimes.com/hc/en-us/articles/115014792127-Copyright-notice}{©~2020~The
  New York Times Company}
\end{itemize}

\begin{itemize}
\tightlist
\item
  \href{https://www.nytco.com/}{NYTCo}
\item
  \href{https://help.nytimes.com/hc/en-us/articles/115015385887-Contact-Us}{Contact
  Us}
\item
  \href{https://www.nytco.com/careers/}{Work with us}
\item
  \href{https://nytmediakit.com/}{Advertise}
\item
  \href{http://www.tbrandstudio.com/}{T Brand Studio}
\item
  \href{https://www.nytimes.com/privacy/cookie-policy\#how-do-i-manage-trackers}{Your
  Ad Choices}
\item
  \href{https://www.nytimes.com/privacy}{Privacy}
\item
  \href{https://help.nytimes.com/hc/en-us/articles/115014893428-Terms-of-service}{Terms
  of Service}
\item
  \href{https://help.nytimes.com/hc/en-us/articles/115014893968-Terms-of-sale}{Terms
  of Sale}
\item
  \href{https://spiderbites.nytimes.com}{Site Map}
\item
  \href{https://help.nytimes.com/hc/en-us}{Help}
\item
  \href{https://www.nytimes.com/subscription?campaignId=37WXW}{Subscriptions}
\end{itemize}
