Sections

SEARCH

\protect\hyperlink{site-content}{Skip to
content}\protect\hyperlink{site-index}{Skip to site index}

\href{https://www.nytimes.com/section/arts/dance}{Dance}

\href{https://myaccount.nytimes.com/auth/login?response_type=cookie\&client_id=vi}{}

\href{https://www.nytimes.com/section/todayspaper}{Today's Paper}

\href{/section/arts/dance}{Dance}\textbar{}What Is It Like to Watch Live
Dance Again? Amazing

\url{https://nyti.ms/3i7BGJ7}

\begin{itemize}
\item
\item
\item
\item
\item
\item
\end{itemize}

Advertisement

\protect\hyperlink{after-top}{Continue reading the main story}

Supported by

\protect\hyperlink{after-sponsor}{Continue reading the main story}

Critic's Notebook

\hypertarget{what-is-it-like-to-watch-live-dance-again-amazing}{%
\section{What Is It Like to Watch Live Dance Again?
Amazing}\label{what-is-it-like-to-watch-live-dance-again-amazing}}

Kaatsbaan hosts a welcome summer festival in the Hudson Valley, where
nature and dancers join forces to put on a show.

\includegraphics{https://static01.nyt.com/images/2020/08/04/arts/03kaatsbaan-review-1/merlin_175232118_afdb4fdf-430c-4771-b3a8-d339648f4590-articleLarge.jpg?quality=75\&auto=webp\&disable=upscale}

By \href{https://www.nytimes.com/by/gia-kourlas}{Gia Kourlas}

\begin{itemize}
\item
  Aug. 3, 2020
\item
  \begin{itemize}
  \item
  \item
  \item
  \item
  \item
  \item
  \end{itemize}
\end{itemize}

TIVOLI, N.Y. --- It didn't bode well that the first live dance I was
going to see since mid-March was one I had seen many times before.
``Sunshine,'' a Larry Keigwin war horse set to the Bill Withers's
classic ``Ain't No Sunshine,'' can give a dancer the opportunity to
really \emph{feel} the music in all the worst ways. It's treacly stuff.

So I'm happy to say that as soon as Melvin Lawovi began to move, my
chest tightened; I even sensed --- the horror --- some tears. Lately,
for self-preservation, I've been talking myself into believing that I
can live without watching dance in person, and while that is true, I
clearly miss it. A lot. ``Sunshine,'' which opened the outdoor
\href{https://www.nytimes.com/2020/07/29/arts/dance/kaatsbaan-dance-festival-stella-abrera.html}{Kaatsbaan
Summer Festival} under beautiful blue skies on Saturday, worked out just
fine.

That was also to the credit of Mr. Lawovi, a recent addition to American
Ballet Theater, who never delivered a treacly moment as he traversed the
stage with the lightest touch. Instead of dwelling on the lyrics or
giving into angst, he danced with an unparalleled polish, as if his body
were clearing the air.

\includegraphics{https://static01.nyt.com/images/2020/08/04/arts/03kaatsbaan-review-2/03kaatsbaan-review-2-articleLarge.jpg?quality=75\&auto=webp\&disable=upscale}

But repertory alone doesn't seem the be all end all of this summer
festival, the first of its kind in Kaatsbaan's 30 years as a cultural
park. From the performances to Brandon Stirling Baker's light-and-sound
installation in a rustic barn to the peace of being surrounded by so
much open space and air, the festival is not only about live dance. It's
a package. The best choreographic moments came in the dancers' simple
yet courtly walks across the grass to the stage.

Kaatsbaan's artistic director, Stella Abrera, and its executive
director, Sonja Kostich, aren't messing around when it comes to safety,
and that was comforting, too, at this socially distanced performance.
The experience included frantically filling out the health check survey
in the car while thinking hard about the questions: \emph{Was} that a
touch of a sore throat this morning?

I loved the elegant firmness of the handwritten signs telling us to wear
masks; the raised stage that seemed like it was dropped from the sky
onto a field; and the optional post-performance walk, on the grounds of
what was originally a farm, with live music (instead of a meandering or
self-congratulatory post-performance talk).

Image

Leonardo Sandoval, with Gregory Richardson on bass, in ``Laying the
Ground.''Credit...Sara Krulwich/The New York Times

In honor of the Black Lives Matter movement, the festival has been
curated by three respected Black dance artists:
\href{https://www.abt.org/people/calvin-royal/?type=performer\&gclid=Cj0KCQjwyJn5BRDrARIsADZ9ykGukwgjIfy5GS0uVFO0vpgqra8aXLyIqvL8ApaOarDXLchnTxIYBTMaAqtmEALw_wcB}{Calvin
Royal III, a soloist at Ballet Theater}, who programmed the first
weekend; Alicia Graf Mack, who leads the dance division of the Juilliard
School; and Lloyd Knight, of the Martha Graham Dance Company.

For the first program, the selections were brief and unassuming --- less
about innovative choreography than watching bodies in motion. (Programs
are short: 20 to 30 minutes, and feature solos and duets.) At the start
of the tap solo ``Laying the Ground,'' Leonardo Sandoval, accompanied by
the bassist Greg Richardson, used his body as an instrument, contrasting
soft taps of his feet with gentle slaps on his thighs and chest as he
made his way to a wooden platform on the stage. His footwork was hushed
--- an articulate, musical whisper --- as he somehow managed to convey
the idea that he was gliding just above his feet.

``The Dividing Line,'' a premiere by Mr. Royal set to Gershwin, wavered
between sensations --- abutting stillness and alienation were glimpses
of hope. Mr. Royal, cutting a figure both introspective and heroic,
stood with his back to us at first, his arms bound from behind until he
suddenly released them and skittered across the stage with quick
backward steps.

Image

Mr. Royal, who curated the first weekend's programs, in his dance ``The
Dividing Line.''~Credit...Sara Krulwich/The New York Times

That feeling of push and pull continued throughout this dance, in which
Mr. Royal, in stops and starts, unleashed his body in space. While
personal and poetic, its power might have been amplified had certain
gestures --- the reach of an arm, the somber bow of the head --- been
toned down, more accidental.

The 30-minute program ended with another solo by Mr. Royal, ``The Dream
Continues,'' set to excerpts from the Rev. Dr. Martin Luther King Jr.'s
``I Have a Dream'' speech, and danced by the elegant Courtney Lavine,
Mr. Royal's Ballet Theater colleague. During the transition between the
solos, Mr. Royal's voice was heard reciting
\href{https://www.poetryfoundation.org/poems/46548/harlem}{Langston
Hughes's poem ``Harlem''} as he slowly moved across the stage: ``What
happens to a dream deferred? Does it dry up like a raisin in the sun?''

Image

Ms. Lavine in Mr. Royal's ``The Dream Continues,'' which was danced to
the words of the Rev. Dr. Martin Luther King's ``I Have a Dream''
speech.Credit...Sara Krulwich/The New York Times

He hopped off, in essence handing the space to Ms. Lavine --- and to
King's words, which beamed into the air with extra clarity. Ms. Lavine,
in a purple leotard and skirt that danced around her knees, used her
luxurious arms and expansiveness to spin seemingly out of control then
stop on a dime as she held still, giving beauty and breadth to King's
message. Her capacity to move clearly isn't suffering under quarantine,
but at the same time, it wasn't about the steps: She was a spirit,
clearly dancing for something bigger than herself.

\textbf{Kaatsbaan Summer Festival}

Weekends through Sept. 27 at Kaatsbaan Cultural Park, Tivoli, N.Y.;
\href{https://kaatsbaan.org/spring-summer-events}{kaatsbaan.org}.

Advertisement

\protect\hyperlink{after-bottom}{Continue reading the main story}

\hypertarget{site-index}{%
\subsection{Site Index}\label{site-index}}

\hypertarget{site-information-navigation}{%
\subsection{Site Information
Navigation}\label{site-information-navigation}}

\begin{itemize}
\tightlist
\item
  \href{https://help.nytimes.com/hc/en-us/articles/115014792127-Copyright-notice}{©~2020~The
  New York Times Company}
\end{itemize}

\begin{itemize}
\tightlist
\item
  \href{https://www.nytco.com/}{NYTCo}
\item
  \href{https://help.nytimes.com/hc/en-us/articles/115015385887-Contact-Us}{Contact
  Us}
\item
  \href{https://www.nytco.com/careers/}{Work with us}
\item
  \href{https://nytmediakit.com/}{Advertise}
\item
  \href{http://www.tbrandstudio.com/}{T Brand Studio}
\item
  \href{https://www.nytimes.com/privacy/cookie-policy\#how-do-i-manage-trackers}{Your
  Ad Choices}
\item
  \href{https://www.nytimes.com/privacy}{Privacy}
\item
  \href{https://help.nytimes.com/hc/en-us/articles/115014893428-Terms-of-service}{Terms
  of Service}
\item
  \href{https://help.nytimes.com/hc/en-us/articles/115014893968-Terms-of-sale}{Terms
  of Sale}
\item
  \href{https://spiderbites.nytimes.com}{Site Map}
\item
  \href{https://help.nytimes.com/hc/en-us}{Help}
\item
  \href{https://www.nytimes.com/subscription?campaignId=37WXW}{Subscriptions}
\end{itemize}
