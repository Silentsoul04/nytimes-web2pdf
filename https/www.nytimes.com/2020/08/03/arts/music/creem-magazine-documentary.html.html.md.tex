Sections

SEARCH

\protect\hyperlink{site-content}{Skip to
content}\protect\hyperlink{site-index}{Skip to site index}

\href{https://www.nytimes.com/section/arts/music}{Music}

\href{https://myaccount.nytimes.com/auth/login?response_type=cookie\&client_id=vi}{}

\href{https://www.nytimes.com/section/todayspaper}{Today's Paper}

\href{/section/arts/music}{Music}\textbar{}The Wild Story of Creem, Once
`America's Only Rock 'n' Roll Magazine'

\url{https://nyti.ms/30oUSMi}

\begin{itemize}
\item
\item
\item
\item
\item
\item
\end{itemize}

Advertisement

\protect\hyperlink{after-top}{Continue reading the main story}

Supported by

\protect\hyperlink{after-sponsor}{Continue reading the main story}

The Great Read

\hypertarget{the-wild-story-of-creem-once-americas-only-rock-n-roll-magazine}{%
\section{The Wild Story of Creem, Once `America's Only Rock 'n' Roll
Magazine'}\label{the-wild-story-of-creem-once-americas-only-rock-n-roll-magazine}}

A new documentary traces the rise and fall of the irreverent,
boundary-smashing music publication where Lester Bangs did some of his
most famous work.

\includegraphics{https://static01.nyt.com/images/2020/08/04/arts/03creem1/merlin_175133154_635eccac-ba2f-45e8-abca-05e02cd73da5-articleLarge.jpg?quality=75\&auto=webp\&disable=upscale}

By Mike Rubin

\begin{itemize}
\item
  Aug. 3, 2020
\item
  \begin{itemize}
  \item
  \item
  \item
  \item
  \item
  \item
  \end{itemize}
\end{itemize}

On Jaan Uhelszki's first day at Creem magazine in October 1970, she met
a fellow new hire: Lester Bangs, a freelance writer freshly arrived from
California to fill the post of record reviews editor. His plaid
three-piece suit made him look like an awkward substitute teacher, she
thought, and certainly out of place among the hippies and would-be
revolutionaries using the publication's decrepit Detroit office as a
crash pad.

Uhelszki, still a teenager, was majoring in journalism at nearby Wayne
State University, and had been sent to the fledgling rock magazine by
editors at the student newspaper. ``They said with a sneer, `We can't
publish you, you don't have any clips, but Creem will publish anybody,
why don't you go walk down the street,''' Uhelszki said in a phone
interview. ``So my first clips were Creem. I started at the top.''

She'd arrived at the headquarters of ``America's Only Rock 'n' Roll
Magazine,'' as Creem's front covers would soon proclaim. What began as
an underground newspaper soon evolved under Bangs, the editor Dave Marsh
and the publisher Barry Kramer into a boisterous, irreverent,
boundary-smashing monthly that was equal parts profound and profane.
During his half-decade at Creem, Bangs would publish many of the
pharmaceutically fueled exegeses that made him ``America's greatest rock
critic'' --- including his epic three-part interview with his
hero/nemesis Lou Reed. By 1976, it had a circulation of over 210,000,
second only to Rolling Stone.

\includegraphics{https://static01.nyt.com/images/2020/08/04/arts/03creem2/03creem2-articleLarge.jpg?quality=75\&auto=webp\&disable=upscale}

The magazine's roller-coaster arc and its lasting impact on the culture
is the subject of a spirited new documentary directed by Scott Crawford,
\href{https://www.creemmag.com/pages/documentary}{``Creem: America's
Only Rock 'n' Roll Magazine,''} which Uhelszki co-wrote and helped
produce. The film opens Friday for virtual cinema and limited theatrical
release, and comes to VOD on Aug. 28.

As a teenager, Crawford bought old issues of Creem from used bookstores
near his home outside Washington, D.C. His first film was ``Salad
Days,'' a 2014 documentary about the city's hardcore punk scene.

``I was aware of the personalities involved,'' he said of the Creem
crew. ``I'd heard stories over the years of their fights, literal
fistfights, so I knew that this would make for a hell of a film because
in addition to how much they contributed to music journalism, a lot of
the writers were just as interesting as the artists that they covered.''

The documentary traces how Creem's high-intensity environment mirrored
that of the late 1960s Detroit rock scene, which was centered around the
heavy guitar assault of bands like the MC5, the Stooges and Alice
Cooper. Barry Kramer, a working-class Jewish kid with a chip on his
shoulder and a volatile temper, was a key local figure: He owned the
record store-cum-head shops Mixed Media and Full Circle.

``I liked Barry a great deal, and in fact I wanted him to manage the
MC5,'' the band's guitarist Wayne Kramer, who is not a relation, said in
a phone interview. (He also handled original music for the film.) ``He
had a vision and saw ways that this emerging counterculture could be
monetized.''

Image

Jaan Uhelszki with Paul Stanley from Kiss while working on her hands-on
story.Credit...Barry Levine

The original idea for Creem came from a clerk at Mixed Media, Tony Reay,
who persuaded Barry Kramer to put \$1,200 into the venture, which began
in March 1969. When the cartoonist Robert Crumb wandered into Mixed
Media in need of cash, Reay offered him \$50 to draw the cover of issue
No. 2. Crumb's illustration included an anthropomorphized bottle of
cream exclaiming ``Boy Howdy!,'' which became the magazine's mascot and
catchphrase.

Reay soon departed over creative differences, and the magazine briefly
took on a more political flavor, thanks to Marsh, a 19-year-old Wayne
State student. The arrival of Bangs in 1970 was explosive.

``They both had different ideas of what Creem should be,'' Uhelszki
said. ``Lester just saw us as bozo provocateurs, and David wanted it to
be a more political magazine and saw us as foot soldiers of the
counterculture.''

In 1971, a robbery at the Cass Corridor offices spurred Barry Kramer to
move the magazine to a 120-acre farm in the rural suburb Walled Lake.
The staff lived there communally for two years: sharing three rooms and
one bathroom, working and socializing around the clock amid a menagerie
of dogs, cats and horses. In the film, Uhelszki reveals that a trip to
the bathroom in the middle of the night meant possibly encountering
Kramer and getting a lecture about copy while half-awake, and that Marsh
once deposited wayward excrement from Bangs's dog onto Bangs's
typewriter.

``We had rolled out into the driveway,'' Marsh recalls of the ensuing
fistfight, ``and I got my head smacked into an open car door. That's OK,
he wasn't trying to hurt me, he was just trying to win.''

In 1973, the commune experiment ended and Creem relocated into a proper
office in Birmingham, one of Detroit's toniest suburbs. Still, the
city's scrappy, underdog spirit remained a crucial element of the
magazine's aesthetic. ``I don't think it could have existed anywhere
else,'' Alice Cooper said in a phone interview. ``In New York it would
have been more sophisticated; in L.A. it would have been a lot slicker.
Detroit was the perfect place for it, because it was somewhere between a
teen magazine and Mad magazine and a hard rock magazine.''

Image

Bangs was known for tangling with musicians and brawling with his
colleagues.Credit...Charles Auringer

Rolling Stone felt comparatively stuffy, preoccupied with movies and
politics and reluctant to cover loud and snotty subcultural movements
like punk and metal, whereas Creem's pages first coined those genre's
names: ``punk rock'' by Marsh, about ? and the Mysterians, and ``heavy
metal'' by Mike Saunders, about Sir Lord Baltimore, both in the May 1971
issue.

The reader mail page provided a ribald frisson between the writers and
their audience. The most infamous exchange came in 1977, after the
writer Rick Johnson opened his review of the second Runaways album,
``Queens of Noise,'' by declaring ``These bitches suck. That's all there
is to it.'' An infuriated Joan Jett visited the Creem office to confront
him; when told Johnson wasn't there, she settled the score in the
letters column.

Musicians were not only the subject of the publication, they were often
its authors; Patti Smith and Lenny Kaye became contributors. And Creem
writers sometimes scaled the fourth wall themselves. The J. Geils Band
singer Peter Wolf invited Bangs to ``play'' his typewriter onstage;
Uhelszki was gussied up by Kiss in full ``Hotter Than Hell'' makeup and
played (unplugged) guitar onstage for her August 1975 story
\href{https://iaspm-us.net/rocks-backpages-rewind-jaan-uhelszki-i-dreamed-i-was-onstage-with-kiss-in-my-maidenform-bra-creem-august-1975/}{``I
Dreamed I Was Onstage With Kiss in My Maidenform Bra.''}

Subversive humor was the Creem lingua franca. Snarky photo captions and
regular features like the Creem Dreems (tongue-in-cheek pinups of
artists like Debbie Harry and Bebe Buell) were clearly intended for ---
and driven by --- adolescent hormones, but the magazine provided
opportunities for women writers like Roberta Cruger, Cynthia Dagnal,
Lisa Robinson and Penny Valentine at a time when the music industry was
intensely misogynist. ``We had so many women who were empowered and were
editors at the time,'' Susan Whitall says in the film. ``When I came in,
Jaan mentored me, and then I mentored other women.''

Image

An assortment of covers from Creem's history.Credit...Clockwise from top
left: Gary Ciccarelli, Andy Kent,~Gary Cooley, Ric Siegel

Still, seen through today's eyes, some of the old Creem content can seem
puerile, even offensive. The casual sexism and homophobia is sadly
typical of its time, and racial sensitivity was nonexistent. Yet its
anarchic attitude and early embrace of new wave and punk inspired future
musicians like Sonic Youth's Thurston Moore, Pearl Jam's Jeff Ament and
Metallica's Kirk Hammett, who all appear in the film. In one scene,
R.E.M.'s Michael Stipe recalls the first time he ever saw a copy of
Creem, during detention in high school, and being mesmerized by a photo
of Patti Smith.

``From that moment forward my entire life shifted and changed
dramatically,'' Stipe says. ``I was like, what world is this? Most
people want to fit in somewhere. Because of my otherness, because of my
queerness, I was trying to find that gang. I wasn't going to find it in
my high school. I found it in Creem magazine.''

Uhelszki said making the documentary revealed that musicians devoted to
the magazine were empowered by what they read. ``The people who made the
magazine, we thought we were equals to the bands in the early years,''
she said. ``Rock stars were just like us but they had better clothes
than we did.''

As the 1970s expired, all the hard partying took a toll. Bangs, who had
departed Detroit and Creem in 1976 for New York, died of an accidental
painkiller overdose in 1982. After a long spiral of drinking and
drugging, Barry Kramer overdosed on nitrous oxide in 1981. He left the
magazine to his 4-year-old son, JJ, who was listed on the masthead as
the chairman of the board.

Image

Barry Kramer and his infant son, JJ, who went on to take over
Creem.Credit...Connie Kramer

With the magazine heavily in debt, Barry Kramer's former wife, Connie,
sold Creem to an investor in 1986 who moved it to Los Angeles. The
company changed ownership several times before the magazine finally
ceased publication in 1989. Considerable litigation lingered into the
2000s, until the Creem brand was acquired by JJ Kramer's company in
2017.

For JJ Kramer, also one of the movie's producers, the documentary
project was more than a film; it was an opportunity to discover the
father he never knew. ``Before we did the movie, I think I can remember
actually hearing his voice maybe once or twice,'' he said in a phone
interview. ``That's why the film was such an incredible experience for
me, just getting to know who he was as a person --- the good, the bad,
the ugly and the crazy --- which in turn taught me a lot about myself as
well.''

An intellectual-property lawyer, JJ Kramer spent 20 years gathering the
rights to the old material, and is eager to make the magazine's archive
available for a new generation. To coincide with the film's release, a
limited-edition best-of-Creem issue will be available on newsstands on
Nov. 1, and additional print editions are being considered, as well as a
TV show.

``I view the documentary as very much the beginning, not the end,'' he
said. ``We're all looking for something to capture our attention and our
passion, so to me that feels like a really strong signal that the world
might need Creem more than ever.''

Advertisement

\protect\hyperlink{after-bottom}{Continue reading the main story}

\hypertarget{site-index}{%
\subsection{Site Index}\label{site-index}}

\hypertarget{site-information-navigation}{%
\subsection{Site Information
Navigation}\label{site-information-navigation}}

\begin{itemize}
\tightlist
\item
  \href{https://help.nytimes.com/hc/en-us/articles/115014792127-Copyright-notice}{©~2020~The
  New York Times Company}
\end{itemize}

\begin{itemize}
\tightlist
\item
  \href{https://www.nytco.com/}{NYTCo}
\item
  \href{https://help.nytimes.com/hc/en-us/articles/115015385887-Contact-Us}{Contact
  Us}
\item
  \href{https://www.nytco.com/careers/}{Work with us}
\item
  \href{https://nytmediakit.com/}{Advertise}
\item
  \href{http://www.tbrandstudio.com/}{T Brand Studio}
\item
  \href{https://www.nytimes.com/privacy/cookie-policy\#how-do-i-manage-trackers}{Your
  Ad Choices}
\item
  \href{https://www.nytimes.com/privacy}{Privacy}
\item
  \href{https://help.nytimes.com/hc/en-us/articles/115014893428-Terms-of-service}{Terms
  of Service}
\item
  \href{https://help.nytimes.com/hc/en-us/articles/115014893968-Terms-of-sale}{Terms
  of Sale}
\item
  \href{https://spiderbites.nytimes.com}{Site Map}
\item
  \href{https://help.nytimes.com/hc/en-us}{Help}
\item
  \href{https://www.nytimes.com/subscription?campaignId=37WXW}{Subscriptions}
\end{itemize}
