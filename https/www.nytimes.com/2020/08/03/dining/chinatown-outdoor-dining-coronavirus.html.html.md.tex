Sections

SEARCH

\protect\hyperlink{site-content}{Skip to
content}\protect\hyperlink{site-index}{Skip to site index}

\href{https://www.nytimes.com/section/food}{Food}

\href{https://myaccount.nytimes.com/auth/login?response_type=cookie\&client_id=vi}{}

\href{https://www.nytimes.com/section/todayspaper}{Today's Paper}

\href{/section/food}{Food}\textbar{}Chinatown Is Coming Back, One Noodle
at a Time

\url{https://nyti.ms/3i11kz5}

\begin{itemize}
\item
\item
\item
\item
\item
\item
\end{itemize}

\href{https://www.nytimes.com/spotlight/at-home?action=click\&pgtype=Article\&state=default\&region=TOP_BANNER\&context=at_home_menu}{At
Home}

\begin{itemize}
\tightlist
\item
  \href{https://www.nytimes.com/2020/08/03/well/family/the-benefits-of-talking-to-strangers.html?action=click\&pgtype=Article\&state=default\&region=TOP_BANNER\&context=at_home_menu}{Talk:
  To Strangers}
\item
  \href{https://www.nytimes.com/2020/08/01/at-home/coronavirus-make-pizza-on-a-grill.html?action=click\&pgtype=Article\&state=default\&region=TOP_BANNER\&context=at_home_menu}{Make:
  Grilled Pizza}
\item
  \href{https://www.nytimes.com/2020/07/31/arts/television/goldbergs-abc-stream.html?action=click\&pgtype=Article\&state=default\&region=TOP_BANNER\&context=at_home_menu}{Watch:
  'The Goldbergs'}
\item
  \href{https://www.nytimes.com/interactive/2020/at-home/even-more-reporters-editors-diaries-lists-recommendations.html?action=click\&pgtype=Article\&state=default\&region=TOP_BANNER\&context=at_home_menu}{Explore:
  Reporters' Google Docs}
\end{itemize}

Advertisement

\protect\hyperlink{after-top}{Continue reading the main story}

Supported by

\protect\hyperlink{after-sponsor}{Continue reading the main story}

Critic's Notebook

\hypertarget{chinatown-is-coming-back-one-noodle-at-a-time}{%
\section{Chinatown Is Coming Back, One Noodle at a
Time}\label{chinatown-is-coming-back-one-noodle-at-a-time}}

Restaurants in the Manhattan neighborhood suffered early in the
pandemic. Some are just now experimenting with outdoor service.

\includegraphics{https://static01.nyt.com/images/2020/08/05/dining/03Chinatown1/merlin_175184061_ed1541c0-f5dc-4ab2-b5a5-e03d4b499648-articleLarge.jpg?quality=75\&auto=webp\&disable=upscale}

By \href{https://www.nytimes.com/by/pete-wells}{Pete Wells}

\begin{itemize}
\item
  Aug. 3, 2020
\item
  \begin{itemize}
  \item
  \item
  \item
  \item
  \item
  \item
  \end{itemize}
\end{itemize}

\href{https://cn.nytimes.com/style/20200804/chinatown-outdoor-dining-coronavirus/}{阅读简体中文版}\href{https://cn.nytimes.com/style/20200804/chinatown-outdoor-dining-coronavirus/zh-hant/}{閱讀繁體中文版}

With an amber chip of roast duck skin on one street and a tangle of lo
mein on another, Chinatown took a few steps out of its long hibernation
last week.

Before New York City had its
\href{https://www.nytimes.com/2020/03/01/nyregion/new-york-coronvirus-confirmed.html}{first
confirmed case} of Covid-19 on March 1, the disease had already hurt the
neighborhood indirectly, as baseless fears of Chinese-owned businesses
\href{https://www.nytimes.com/2020/01/29/nyregion/coronavirus-nyc.html}{kept
some visitors away.}

Like other Chinese neighborhoods in the city, Manhattan's Chinatown has
been
\href{https://www.nytimes.com/2020/06/30/nyregion/chinatown-coronavirus-nyc.html?searchResultPosition=1}{slow
to recover}. When the city began allowing restaurants to place tables on
sidewalks and in parking lanes, and patio umbrellas and potted palms
began to sprout north of Canal Street and west of the Tombs, the streets
of Chinatown remained largely, eerily, deserted.

Then on Wednesday, on the south end of Mott Street, seating for 120
people opened at a string of communal
\href{https://www.rockwellgroup.com/projects/dineout-nyc\#:~:text=Our\%20solution\%2C\%20DineOut\%20NYC\%2C\%20designed,to\%20feel\%20safe\%20and\%20comfortable.}{dining
platforms designed to sit on the pavement by the Rockwell Group}
architecture firm. The installation, done in bright primary colors and
then painted by local artists and students from the nearby
\href{https://www.transfigurationschoolnyc.org/}{Transfiguration
School}, is essentially a socially distanced food court.

Anyone who finds an empty table can sit down with food bought at one of
the dozen or so businesses that face the new tables. These include two
restaurants, \href{http://www.hop-kee-nyc.com/}{Hop Kee} and
\href{https://www.nytimes.com/2010/07/07/dining/reviews/07dinbriefs.html}{Wo
Hop}, where generations of New Yorkers and tourists got their first
taste of Chinatown.

\includegraphics{https://static01.nyt.com/images/2020/08/05/dining/03Chinatown6/03Chinatown6-articleLarge-v2.jpg?quality=75\&auto=webp\&disable=upscale}

The project, a nonprofit effort by Rockwell Group and other donors, had
an immediate ripple effect. After seeing the construction earlier last
week, the owner of \href{https://www.pekingduckhousenyc.com/}{Peking
Duck House} began to build a dining platform on Mott Street for his
restaurant, which is just north of the Rockwell Group cluster. Some
businesses on the block had already built dining areas in the street;
once the red, yellow and blue pavilions arrived, they no longer needed
their ad hoc street furniture, and sold it at a discount to other local
restaurants, including
\href{http://www.goldenunicornrestaurant.com/}{Golden Unicorn} and
\href{https://www.nytimes.com/2018/02/13/dining/hwa-yuan-szechuan-review-chinese-food.html}{Hwa
Yuan Szechuan}.

On Tuesday night, Hwa Yuan served its first meals since March that were
not packed in takeout containers. True, customers sat at three tables
next to the curb on East Broadway rather than in one of the newly
decorated dining rooms on the restaurant's three floors. (Indoor dining
is still prohibited in the city.)

But Hwa Yuan had set the tables as grandly as the circumstances would
permit, nesting delicate white porcelain appetizer plates into porcelain
chargers, and supplying each place setting with two pairs of lacquered
chopsticks propped against a ceramic chopstick rest.

The city had shaken off most of the day's heat by the time I arrived for
dinner. Still, the first thing I ordered was a bowl of chilled sesame
noodles.

These are, of course, the noodles that Hwa Yuan began serving when it
was founded by the Sichuan-born chef
\href{https://dinersjournal.blogs.nytimes.com/2010/09/30/shorty-tang-sesame-noodle-king-isremembered/}{Shorty
Tang} in 1967. They are the noodles that restaurants across the city
have tried to copy, often without the run of dark vinegar or orange
splash of chile oil that made Mr. Tang's recipe so thrilling. One day
\href{https://cooking.nytimes.com/recipes/9558-takeout-style-sesame-noodles}{these
noodles} will be commemorated by a bronze plaque on the building. For
now, the best way to honor them is to eat them.

Image

Hwa Yuan never closed during the pandemic, but only last week began
serving cold sesame noodles and other dishes outdoors.Credit...Jeenah
Moon for The New York Times

For the sake of comparison, you might also get the shredded bean-curd
salad, for which cold, pressed tofu is sliced into square strands, like
\href{https://www.nytimes.com/2004/05/26/dining/a-guitar-that-makes-beautiful-pasta.html}{spaghetti
alla chitarra}. Outfitted with Chinese celery and dressed with sesame
oil, it is at least as refreshing on a hot day as the sesame noodles.

Some people always look as if they should be in a suit, even at the
beach. Hwa Yuan is like that. In its attempt to transfer its elegant
style to the new asphalt-and-concrete ambience, it may have pioneered
the concept of curbside Peking duck carving. A waiter set up a small
table on the sidewalk and, holding a cleaver in one hand and a roast
duck in the other, proceeded to whittle off tiles of skin and flesh
under the LED beam of the streetlights.

Chien Lieh Tang, Shorty Tang's son and the current chef, said in a phone
interview that he had resisted outdoor dining. Instead, he had lobbied
the city to allow indoor service.

Image

Chien Lieh Tang, Hwa Yuan's owner, reluctantly brought his restaurant's
formal service style to the pavement.Credit...Jeenah Moon for The New
York Times

``A Chinese restaurant is different because we have a lot of plate
service,'' he said. ``You have appetizer, soup, rice, a lot of things, a
lot of sauce, so it's not easy for Chinese restaurants to put it
outside.''

There are other challenges in Chinatown. The tourists are gone, and
while jury duty has resumed at some downtown courts, many office
buildings are empty. A handful of sidewalk tables would make little
difference to banquet and dim sum restaurants with space for hundreds of
people at a time, including Jing Fong and 88 Palace, which have not
embraced outdoor dining.

Mr. Tang held out for a month before giving in. ``We don't have a
choice,'' he said. ``We have to survive.''

Dinner on Mott Street the next night was a less formal affair. A waiter
handed me a paper menu pulled off a ring hanging by the front door. Then
he gestured toward an empty patio table and chairs --- donated by the
companies that made them --- inside a three-sided plywood pen painted
daisy yellow and separated by plexiglass partitions from the two nearest
tables.

The steamed dumplings, scallops in black-bean sauce and pork chow fun
showed up in plastic takeout containers. The lids were snapped on tight.
Knives, forks and napkins were sealed in plastic. Disposable chopsticks
were handed out if you asked.

It wasn't until after I'd sat down that I realized my menu wasn't from
\href{https://www.wohop17.com/}{Wo Hop}, where I'd meant to eat, but
\href{https://www.wohop15.com/menu}{Wo Hop Next Door}, the younger
spinoff run by another branch of the family. Walking down the red-tiled
passageway to Wo Hop has always felt like entering a noir film. Part of
it is the period décor and the rest is the period cuisine --- midcentury
chow mein. There's nothing cinematic about Wo Hop Next Door.

But I've never been one of those people who think the cooking at the two
Wo Hops is different enough to be worth arguing about. And the cooking
registered differently outside, under the forest of fire escapes and
bilingual signage. It is less a step back in time and more a stop along
a guided tour of local history.

At your outdoor table, you could start with lobster enrobed in a glossy,
harmless sauce at Wo Hop, founded in 1938; move on to crab fried rice
and an entire steamed fish under scallion threads from
\href{https://www.nytimes.com/2000/08/09/dining/restaurants-diners-and-dinner-exchanging-glances.html?searchResultPosition=1}{Ping's
Seafood}, open since 1998; then take on a bowl of the trusty soup
dumplings from \href{https://shanghai21togo.com/\#}{Shanghai 21}, also
known as 21 Shanghai House, which has been there less than a decade.

Image

Pinklady Cheese Tart, a bakery, opened on Mott Street during the
pandemic.Credit...Jeenah Moon for The New York Times

You could bring your tour up to this year with dessert from
\href{https://pinkladycheesetart.business.site/}{Pinklady Cheese Tart}.
Jean Lim, a young entrepreneur who moved to New York from Malaysia,
signed the lease on her narrow bakery in February.

She moved ahead with her plan, even when there was almost nobody on Mott
Street to buy her crisp, twice-baked pastry shells filled with whipped,
sweetened fresh cheese. She makes some plain and flavors others with,
for instance, matcha or chocolate. A single palm-size tart is \$2.75,
and will disappear in a minute or less.

Chinatown's geologic layering of successive generations of immigrants,
cuisines and sensibilities is one source of its enduring allure to
hungry, curious visitors. Tradition anchors the neighborhood, but also
makes it vulnerable. Although restaurants there have long offered
takeout, online ordering is unheard of at many of them, or was until the
pandemic forced some owners to adapt.

Narrow sidewalks and streets that seem to have more than their share of
restricted parking spaces made it hard for some Chinatown restaurants to
take advantage of the new outdoor dining rules. And the frugality that
helps businesses there come through lean times also made them reluctant
to spend money on outdoor seating that may end up in the trash in a few
months, according to Wellington Chen, the executive director of the
\href{http://supportchinatownbid.org/}{Chinatown Business Improvement
District}.

Image

Wun Yin Wu, left, helped paint a new dining pavilion for Peking Duck
House, his Mott Street restaurant.Credit...Jeenah Moon for The New York
Times

His group has lent restaurants its canopies, outdoor lights, tables,
chairs and umbrellas, along with sandbags to weigh them down. It trucked
in 6,400 pounds of rich soil to fill the planter boxes that restaurants
use as a buffer between traffic lanes and in-street seating, and moved
the boxes from restaurants that no longer need them to ones that did,
like Golden Unicorn. (That restaurant and Peking Duck House did not
return calls about their reopenings, but Mr. Chen believes that both
will begin serving outdoors within the next week.)

Image

For its first communal-seating structures, Rockwell Group looked for
areas where additional resources could help restaurants take advantage
of the new outdoor dining rules. The next installation will be in
Jackson Heights, Queens.Credit...Jeenah Moon for The New York Times

The new seating on Mott Street is Rockwell Group's sixth outdoor-dining
installation, and the first one shared by several restaurants.

Another communal dining cluster is scheduled for later this month. It
will radiate from the intersection of 37th Road and 74th Street in
Jackson Heights, in the part of Queens that had the highest
concentration of illness from Covid-19. Like the businesses of
Chinatown, the Nepalese, Bangladeshi and Indian restaurants on those
blocks have been slow to exploit the new outdoor dining rules.

Working with the city Department of Transportation, the firm looked in
the five boroughs to identify ``locations where the operators weren't
capable of rallying resources to help themselves,'' in the words of
David Rockwell, the firm's founder.

``It's been so terrifying to look at the empty city and see it just as
hardware,'' Mr. Rockwell said. ``In theater, when there's not a
performance, the art form doesn't exist. In some ways, cities are like
that. Walking around the city you see these big gaping wounds. And you
see these pockets where people have started to dine out.''

\emph{Follow} \href{https://twitter.com/nytfood}{\emph{NYT Food on
Twitter}} \emph{and}
\href{https://www.instagram.com/nytcooking/}{\emph{NYT Cooking on
Instagram}}\emph{,}
\href{https://www.facebook.com/nytcooking/}{\emph{Facebook}}\emph{,}
\href{https://www.youtube.com/nytcooking}{\emph{YouTube}} \emph{and}
\href{https://www.pinterest.com/nytcooking/}{\emph{Pinterest}}\emph{.}
\href{https://www.nytimes.com/newsletters/cooking}{\emph{Get regular
updates from NYT Cooking, with recipe suggestions, cooking tips and
shopping advice}}\emph{.}

Advertisement

\protect\hyperlink{after-bottom}{Continue reading the main story}

\hypertarget{site-index}{%
\subsection{Site Index}\label{site-index}}

\hypertarget{site-information-navigation}{%
\subsection{Site Information
Navigation}\label{site-information-navigation}}

\begin{itemize}
\tightlist
\item
  \href{https://help.nytimes.com/hc/en-us/articles/115014792127-Copyright-notice}{©~2020~The
  New York Times Company}
\end{itemize}

\begin{itemize}
\tightlist
\item
  \href{https://www.nytco.com/}{NYTCo}
\item
  \href{https://help.nytimes.com/hc/en-us/articles/115015385887-Contact-Us}{Contact
  Us}
\item
  \href{https://www.nytco.com/careers/}{Work with us}
\item
  \href{https://nytmediakit.com/}{Advertise}
\item
  \href{http://www.tbrandstudio.com/}{T Brand Studio}
\item
  \href{https://www.nytimes.com/privacy/cookie-policy\#how-do-i-manage-trackers}{Your
  Ad Choices}
\item
  \href{https://www.nytimes.com/privacy}{Privacy}
\item
  \href{https://help.nytimes.com/hc/en-us/articles/115014893428-Terms-of-service}{Terms
  of Service}
\item
  \href{https://help.nytimes.com/hc/en-us/articles/115014893968-Terms-of-sale}{Terms
  of Sale}
\item
  \href{https://spiderbites.nytimes.com}{Site Map}
\item
  \href{https://help.nytimes.com/hc/en-us}{Help}
\item
  \href{https://www.nytimes.com/subscription?campaignId=37WXW}{Subscriptions}
\end{itemize}
