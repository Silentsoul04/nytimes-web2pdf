Sections

SEARCH

\protect\hyperlink{site-content}{Skip to
content}\protect\hyperlink{site-index}{Skip to site index}

\href{https://www.nytimes.com/section/nyregion}{New York}

\href{https://myaccount.nytimes.com/auth/login?response_type=cookie\&client_id=vi}{}

\href{https://www.nytimes.com/section/todayspaper}{Today's Paper}

\href{/section/nyregion}{New York}\textbar{}Judge Whose Son Was Killed
by Misogynistic Lawyer Speaks Out

\url{https://nyti.ms/3gp1WOp}

\begin{itemize}
\item
\item
\item
\item
\item
\item
\end{itemize}

Advertisement

\protect\hyperlink{after-top}{Continue reading the main story}

Supported by

\protect\hyperlink{after-sponsor}{Continue reading the main story}

\hypertarget{judge-whose-son-was-killed-by-misogynistic-lawyer-speaks-out}{%
\section{Judge Whose Son Was Killed by Misogynistic Lawyer Speaks
Out}\label{judge-whose-son-was-killed-by-misogynistic-lawyer-speaks-out}}

``Two weeks ago, my life as I knew it changed in an instant, and my
family will never be the same,'' Judge Esther Salas said in a video
statement.

\includegraphics{https://static01.nyt.com/images/2020/08/03/nyregion/03salas-1/merlin_175108872_f48b9331-2145-4010-8684-75c41c4a349b-videoSixteenByNineJumbo1600.jpg}

By \href{https://www.nytimes.com/by/tracey-tully}{Tracey Tully}

\begin{itemize}
\item
  Aug. 3, 2020
\item
  \begin{itemize}
  \item
  \item
  \item
  \item
  \item
  \item
  \end{itemize}
\end{itemize}

The federal judge whose son was killed by a misogynistic lawyer spoke
out Monday for the first time about the shooting, describing the horror
that unfolded as her only child ran to answer the door and a ``madman''
opened fire.

The judge, Esther Salas, also issued a call for increased privacy
protections for federal judges, saying the death of her 20-year-old son,
Daniel, should not be in vain. Her husband, Mark Anderl, who was
\href{https://www.nytimes.com/2020/07/20/nyregion/esther-salas.html}{shot
three times}, remains hospitalized.

``Two weeks ago, my life as I knew it changed in an instant, and my
family will never be the same,'' Judge Salas said in her video
statement. ``A madman, who I believe was targeting me because of my
position as a federal judge, came to my house.''

She described a weekend celebration at their New Jersey home for
Daniel's 20th birthday that included several of his friends from
Catholic University of America, who had stayed overnight.

``The weekend was a glorious one,'' Judge Salas added, choking back
tears. ``It was filled with love, laughter and smiles.''

She and her son were in the basement talking when the doorbell rang.

``Daniel looked at me and said, `Who is that?'''

``And before I could say a word, he sprinted upstairs. Within seconds, I
heard the sound of bullets and someone screaming, `No!'''

Daniel's final act, she said, was to protect his father from the man she
described as a monster.

``He took the shooter's first bullet directly to the chest,'' she said.
``The monster then turned his attention to my husband and began to shoot
at my husband, one shot after another.''

Judge Salas said the man, believed to have been
\href{https://www.nytimes.com/2020/07/25/nyregion/roy-den-hollander-esther-salas-list.html}{Roy
Den Hollander}, who later killed himself, was carrying a FedEx package
--- an apparent ruse to coax the family to open the door.

Until that moment on July 19, it had been an otherwise routine Sunday:
Judge Salas and her husband went to church, and Daniel, who was about to
start his junior year in college, caught up on some sleep after his
friends left for the weekend.

Image

Roy Den HollanderCredit...via Agence France-Presse --- Getty Images

She said
\href{https://www.nytimes.com/2020/07/26/nyregion/roy-den-hollander-judge.html}{Mr.
Hollander had compiled a dossier} on her and her family, including their
address in North Brunswick, N.J., and the church they attended.

Days before, Mr. Den Hollander, 72, had traveled by train to San
Bernardino County, Calif., where he shot and killed a rival men's rights
lawyer,
\href{https://www.nytimes.com/2020/07/22/nyregion/roy-den-hollander-esther-salas.html}{Marc
E. Angelucci}, at his home, the authorities said.

Hours after the shooting in New Jersey, the police found Mr. Den
Hollander's body off a road in upstate New York with a single gunshot to
the head.

Mr. Den Hollander was a self-described ``anti-feminist'' with a record
of virulently misogynistic and hateful writing. He represented the most
extreme element of the men's rights movement whose online discussions in
recent years have become increasingly menacing toward women.

He was apparently angry at Judge Salas for not moving quickly enough on
a lawsuit he had brought challenging the constitutionality of the
male-only draft.

Judge Salas said she understood that judges' decisions would be
scrutinized.

``We know that our job requires us to make tough calls, and sometimes
those calls can leave people angry and upset,'' she said. ``That comes
with the territory and we accept that.

``But what we cannot accept is when we are forced to live in fear for
our lives because personal information, like our home addresses, can be
easily obtained by anyone seeking to do us or our families harm.''

She called for a national conversation on ways to safeguard the privacy
of federal judges.

Judge Salas said it was a ``complicated issue,'' but urged those in
power to ``do something to help my brothers and sisters on the bench.''

She specifically cited companies that sell personal details, which she
said ``can be leveraged for nefarious purposes.''

Senator Robert Menendez of New Jersey, who recommended Judge Salas for
the federal bench, said he was drafting legislation with Senator Cory
Booker to keep personal information about federal judges outside of the
public domain.

``No parent should have to go through the devastating tragedy that she
has,'' Mr. Menendez said of Judge Salas, who was appointed in 2011 after
being nominated for the lifetime position by President Barack Obama.

``If a federal judge has to worry that his or her decisions at the end
of the day could cause a loss of the life of a loved one, then I'm not
sure how that full independence --- even when one works hard to maintain
it --- can ever be achieved,'' Mr. Menendez said Monday during a
\href{https://www.facebook.com/senatormenendez/videos/721816791697781}{news
conference} on an unrelated topic. ``No federal judge should have to
worry about the writs that they issue, the decisions that they make.''

After investigators found Mr. Den Hollander's body, they discovered a
list in his rental car that named more than a dozen possible targets,
including Judge Salas and three other female judges. The list also
included the names of two oncologists; Mr. Den Hollander had told a
former rugby teammate that he was dying from a rare form of cancer.

His beliefs teetered between the views of self-proclaimed anti-feminists
and
\href{https://www.nytimes.com/2018/07/13/style/mens-rights-movement.html}{men's
rights activists}. The final version of his autobiography, a 1,698-page
manifesto, ended with a vow to fight ``feminazis'' until his last
breath.

Mr. Den Hollander's connection to Judge Salas and Mr. Angelucci involved
similar cases.

In 2015, Mr. Den Hollander brought a legal challenge to the male-only
military draft that was assigned to Judge Salas in Newark federal court.

Mr. Angelucci had filed a similar lawsuit in a different jurisdiction. A
federal court in Houston
\href{https://www.nytimes.com/2019/02/24/us/military-draft-men-unconstitutional.html}{ruled
in Mr. Angelucci's favor} in February 2019, angering Mr. Den Hollander,
who complained in his online writings that Judge Salas was moving too
slowly.

Mr. Den Hollander also had a photo of New York State's chief judge,
Janet M. DiFiore, in his car, according to her spokesman, Lucian
Chalfen.

A former federal judge in Manhattan,
\href{https://www.nytimes.com/2016/03/24/nyregion/shira-scheindlin-judge-behind-stop-and-frisk-ruling-will-step-down.html?searchResultPosition=2}{Shira
A. Scheindlin}, agreed that access to judges' home addresses and phone
numbers too often puts them and their families at risk. She said she
supported Judge Salas's call for greater safeguards at a time when
private details are often only a few computer clicks away.

``It only takes one crazy person,'' said Ms. Scheindlin, who resigned
from the federal bench in 2016.

In 2005, a Chicago
\href{https://www.nytimes.com/2005/03/11/us/electrician-says-in-suicide-note-that-he-killed-judges-family.html}{federal
judge's husband and mother} were killed at home by a delusional man
whose lawsuit had been dismissed by the judge, Joan Humphrey Lefkow. In
1988, a New York federal judge, Richard J. Daronco, was
\href{https://www.nytimes.com/1988/05/22/nyregion/federal-judge-slain-by-a-gunman-in-westchester.html}{shot
and killed} as he worked in the garden of his home in Pelham, N.Y. Both
of the killers later took their own lives.

Ms. Scheindlin recalled getting threatening telephone calls after she
issued a controversial ruling in 2013 rejecting New York City's former
\href{https://www.nytimes.com/2013/08/13/nyregion/stop-and-frisk-practice-violated-rights-judge-rules.html}{stop-and-frisk
policy}, which she had concluded was ``indirect racial profiling.''

``This is getting to be a very serious problem,'' Ms. Scheindlin said,
``and a very scary thing for judges.''

Advertisement

\protect\hyperlink{after-bottom}{Continue reading the main story}

\hypertarget{site-index}{%
\subsection{Site Index}\label{site-index}}

\hypertarget{site-information-navigation}{%
\subsection{Site Information
Navigation}\label{site-information-navigation}}

\begin{itemize}
\tightlist
\item
  \href{https://help.nytimes.com/hc/en-us/articles/115014792127-Copyright-notice}{©~2020~The
  New York Times Company}
\end{itemize}

\begin{itemize}
\tightlist
\item
  \href{https://www.nytco.com/}{NYTCo}
\item
  \href{https://help.nytimes.com/hc/en-us/articles/115015385887-Contact-Us}{Contact
  Us}
\item
  \href{https://www.nytco.com/careers/}{Work with us}
\item
  \href{https://nytmediakit.com/}{Advertise}
\item
  \href{http://www.tbrandstudio.com/}{T Brand Studio}
\item
  \href{https://www.nytimes.com/privacy/cookie-policy\#how-do-i-manage-trackers}{Your
  Ad Choices}
\item
  \href{https://www.nytimes.com/privacy}{Privacy}
\item
  \href{https://help.nytimes.com/hc/en-us/articles/115014893428-Terms-of-service}{Terms
  of Service}
\item
  \href{https://help.nytimes.com/hc/en-us/articles/115014893968-Terms-of-sale}{Terms
  of Sale}
\item
  \href{https://spiderbites.nytimes.com}{Site Map}
\item
  \href{https://help.nytimes.com/hc/en-us}{Help}
\item
  \href{https://www.nytimes.com/subscription?campaignId=37WXW}{Subscriptions}
\end{itemize}
