Sections

SEARCH

\protect\hyperlink{site-content}{Skip to
content}\protect\hyperlink{site-index}{Skip to site index}

\href{https://www.nytimes.com/section/nyregion}{New York}

\href{https://myaccount.nytimes.com/auth/login?response_type=cookie\&client_id=vi}{}

\href{https://www.nytimes.com/section/todayspaper}{Today's Paper}

\href{/section/nyregion}{New York}\textbar{}Disputed Ballots Must Be
Counted in N.Y. Congressional Race, U.S. Judge Rules

\url{https://nyti.ms/3hZwdnA}

\begin{itemize}
\item
\item
\item
\item
\item
\item
\end{itemize}

Advertisement

\protect\hyperlink{after-top}{Continue reading the main story}

Supported by

\protect\hyperlink{after-sponsor}{Continue reading the main story}

\hypertarget{disputed-ballots-must-be-counted-in-ny-congressional-race-us-judge-rules}{%
\section{Disputed Ballots Must Be Counted in N.Y. Congressional Race,
U.S. Judge
Rules}\label{disputed-ballots-must-be-counted-in-ny-congressional-race-us-judge-rules}}

Delays continue in a race that remains undecided six weeks after a
Democratic primary in which Representative Carolyn B. Maloney faced off
against a challenger.

\includegraphics{https://static01.nyt.com/images/2020/08/03/nyregion/03NYVIRUS-MALONEY1/merlin_175075239_62139f30-e9b4-491f-a064-2eb09e167f11-articleLarge.jpg?quality=75\&auto=webp\&disable=upscale}

\href{https://www.nytimes.com/by/jesse-mckinley}{\includegraphics{https://static01.nyt.com/images/2018/02/20/multimedia/author-jesse-mckinley/author-jesse-mckinley-thumbLarge.jpg}}

By \href{https://www.nytimes.com/by/jesse-mckinley}{Jesse McKinley}

\begin{itemize}
\item
  Aug. 3, 2020
\item
  \begin{itemize}
  \item
  \item
  \item
  \item
  \item
  \item
  \end{itemize}
\end{itemize}

A federal judge in Manhattan ruled late Monday that at least 1,000
disputed ballots in a closely watched Democratic congressional primary
should be counted, upending a race that remains undecided six weeks
after the election and that has drawn the attention of President Trump
and
\href{https://www.nytimes.com/2020/08/03/nyregion/nyc-mail-ballots-voting.html}{embarrassed
the New York City Board of Elections}.

The ruling, by Judge Analisa Torres of Federal District Court in
Manhattan, may not affect the outcome in the June 23 primary.
Representative Carolyn B. Maloney is leading her Democratic challenger,
Suraj Patel, by some 3,700 votes, and the judge's decision was narrowly
drawn to force the counting of only a portion of the 12,500 disputed
absentee ballots.

Still, the judge's decision is the latest twist in a race that has been
used by the president to cast doubts on the efficacy of vote-by-mail
systems nationwide, even as he trails in polls in his bid for
re-election. The coronavirus outbreak has prompted states across the
nation to consider expanding mail-in voting for the general election in
November, as public health officials worry that convening at polling
locations may spread the disease.

\includegraphics{https://static01.nyt.com/images/2020/08/03/nyregion/03NYVIRUS-MALONEY2/merlin_173559594_e41ae70e-5b4f-4e11-85cb-612556f38755-articleLarge.jpg?quality=75\&auto=webp\&disable=upscale}

On Monday, Mr. Trump said that Ms. Maloney's race was ``a mess'' and ``a
total disaster,'' and suggested that it should be ``rerun.''

``They're six weeks into it now,'' Mr. Trump said. ``They have no clue
what's going on.''

Under Judge Torres's decision, ballots received the day after Election
Day --- June 24 --- will be counted ``without regard to whether such
ballots are postmarked by June 23.'' Mr. Patel estimates that this
amounts to about 1,200 ballots, which would not be enough for him to
overtake Ms. Maloney. Ballots that were received by June 25 --- two days
after the election deadline --- would also be valid, so long as they did
not have a postmark later than June 23.

In sworn testimony last week, postal officials conceded that their
system of identifying and postmarking ballots --- a critical element in
determining whether ballots were sent by the Election Day deadline ---
was not foolproof, and that some ballots had not been postmarked.

Late Monday, Mr. Patel lauded the judge's decision, casting it as a
warning about possible complications in the general election. ``This is
no longer a Democratic or a Republican fight, this is not an
establishment versus progressive fight,'' he said. ``This is now a fight
for the voting rights of millions in a pandemic.''

Ms. Maloney said in a statement released on Tuesday morning that she
welcomed the judge's decision, and called on Mr. Patel to concede.

``It is regrettable that my former opponent has become President Trump's
mouthpiece in disparaging mail voting by making unsupported claims of
many thousands of ballots being invalidated,'' she said. ``The true
facts show a smaller number that had no effect on the results.''

Mr. Trump has cited the primary contest in New York City as evidence for
his unfounded claims that mail-in voting is susceptible to fraud. There
is no evidence that the primary results were tainted by criminal
malfeasance, according to a wide array of election officials and
representatives of campaigns.

While the outcomes have been unknown for weeks in Ms. Maloney's race ---
in the 12th Congressional District, which includes parts of Manhattan,
Queens and Brooklyn --- and another race in a nearby district in the
Bronx, the cause of the delays is clear.

In late April, Gov. Andrew M. Cuomo, a third-term Democrat,
\href{https://www.governor.ny.gov/news/amid-ongoing-covid-19-pandemic-governor-cuomo-issues-executive-order-make-sure-every-new-yorker}{ordered
a wide expansion} of access to absentee voting as the state reeled from
the pandemic, which has killed more than 30,000 people in New York. But
local boards of election were unprepared for the crush of ballots they
received: Over 400,000 were cast in New York City alone, more than were
submitted across the entire state in 2018.

Working with a model geared toward machine counts of in-person ballots,
the city's Board of Elections was overwhelmed by the number of mailed
ballots and constricted by state law that sets precise guidelines for
counting such ballots, which typically make up just a small fraction of
the total number of votes.

While Judge Torres ruled on ballots that were disputed over a postmark,
thousands of others have already been disqualified for minor errors such
as missing signatures on envelopes or envelopes sealed with tape rather
than saliva.

In her decision, Judge Torres also opened the door for possible
consideration of ballots received as late as June 30 if they could
decide the race.

In the June 23 primary, the outcomes of some races were determined or
codified by absentee votes, including several victories for
\href{https://www.nytimes.com/2020/07/24/nyregion/progressive-primaries-ny-legislature.html?searchResultPosition=1}{the
progressive wing of the Democratic Party} in New York, a deep blue
state.

But the slow pace of counting in New York City has drawn the scorn of
the president, as well as concern from Mr. Cuomo, who has clashed with
Mr. Trump on multiple occasions and predicted his defeat in November.

On Saturday, Mr. Cuomo said that he had offered to assist local election
boards --- such as New York City's --- with national guard personnel in
the June primary, but had not sent any such personnel for help.

Advertisement

\protect\hyperlink{after-bottom}{Continue reading the main story}

\hypertarget{site-index}{%
\subsection{Site Index}\label{site-index}}

\hypertarget{site-information-navigation}{%
\subsection{Site Information
Navigation}\label{site-information-navigation}}

\begin{itemize}
\tightlist
\item
  \href{https://help.nytimes.com/hc/en-us/articles/115014792127-Copyright-notice}{©~2020~The
  New York Times Company}
\end{itemize}

\begin{itemize}
\tightlist
\item
  \href{https://www.nytco.com/}{NYTCo}
\item
  \href{https://help.nytimes.com/hc/en-us/articles/115015385887-Contact-Us}{Contact
  Us}
\item
  \href{https://www.nytco.com/careers/}{Work with us}
\item
  \href{https://nytmediakit.com/}{Advertise}
\item
  \href{http://www.tbrandstudio.com/}{T Brand Studio}
\item
  \href{https://www.nytimes.com/privacy/cookie-policy\#how-do-i-manage-trackers}{Your
  Ad Choices}
\item
  \href{https://www.nytimes.com/privacy}{Privacy}
\item
  \href{https://help.nytimes.com/hc/en-us/articles/115014893428-Terms-of-service}{Terms
  of Service}
\item
  \href{https://help.nytimes.com/hc/en-us/articles/115014893968-Terms-of-sale}{Terms
  of Sale}
\item
  \href{https://spiderbites.nytimes.com}{Site Map}
\item
  \href{https://help.nytimes.com/hc/en-us}{Help}
\item
  \href{https://www.nytimes.com/subscription?campaignId=37WXW}{Subscriptions}
\end{itemize}
