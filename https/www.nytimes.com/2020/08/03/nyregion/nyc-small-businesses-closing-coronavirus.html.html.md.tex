Sections

SEARCH

\protect\hyperlink{site-content}{Skip to
content}\protect\hyperlink{site-index}{Skip to site index}

\href{https://www.nytimes.com/section/nyregion}{New York}

\href{https://myaccount.nytimes.com/auth/login?response_type=cookie\&client_id=vi}{}

\href{https://www.nytimes.com/section/todayspaper}{Today's Paper}

\href{/section/nyregion}{New York}\textbar{}One-Third of New York's
Small Businesses May Be Gone Forever

\url{https://nyti.ms/2D4c54Z}

\begin{itemize}
\item
\item
\item
\item
\item
\item
\end{itemize}

Advertisement

\protect\hyperlink{after-top}{Continue reading the main story}

Supported by

\protect\hyperlink{after-sponsor}{Continue reading the main story}

\hypertarget{one-third-of-new-yorks-small-businesses-may-be-gone-forever}{%
\section{One-Third of New York's Small Businesses May Be Gone
Forever}\label{one-third-of-new-yorks-small-businesses-may-be-gone-forever}}

Small-business owners said they have exhausted federal and local
assistance and see no end in sight after months of sharp revenue drops.
Now, many are closing their shops and restaurants for good.

\includegraphics{https://static01.nyt.com/images/2020/07/27/nyregion/00nyvirus-smallbusiness/00nyvirus-smallbusiness-articleLarge-v3.jpg?quality=75\&auto=webp\&disable=upscale}

\href{https://www.nytimes.com/by/matthew-haag}{\includegraphics{https://static01.nyt.com/images/2018/06/14/multimedia/author-matthew-haag/author-matthew-haag-thumbLarge.jpg}}

By \href{https://www.nytimes.com/by/matthew-haag}{Matthew Haag}

\begin{itemize}
\item
  Published Aug. 3, 2020Updated Aug. 4, 2020, 12:05 a.m. ET
\item
  \begin{itemize}
  \item
  \item
  \item
  \item
  \item
  \item
  \end{itemize}
\end{itemize}

In early March, Glady's, a Caribbean restaurant in Brooklyn, was
bringing in about \$35,000 a week in revenue. The Bank Street Bookstore,
a 50-year-old children's shop in Manhattan, was preparing for busy
spring and summer shopping seasons. And Busy Bodies, a play space for
children in Brooklyn, had just wrapped up months of packed classes with
long waiting lists.

Five months later, those once prosperous businesses have evaporated.
Glady's and Busy Bodies are closed for good and Bank Street, one of the
city's last children's bookstores, will shut down permanently in August.

The three are victims of the economic destruction that threatens to
derail New York City's recovery from the financial collapse triggered by
the coronavirus pandemic.

An expanding universe of distinctive small businesses --- from coffee
shops to dry cleaners to hardware stores --- that give New York's
neighborhoods their unique personalities and are key to the city's
economy are starting to topple.

More than 2,800 businesses in New York City have permanently closed
since March 1, according to data from Yelp, the business listing and
review site, a higher number than in any other large American city.

About half the closings have been in Manhattan, where office buildings
have been hollowed out, its wealthier residents have left for second
homes and tourists have stayed away.

When the pandemic eventually subsides, roughly one-third of the city's
240,000 small businesses may never reopen,
\href{https://pfnyc.org/wp-content/uploads/2020/07/actionandcollaboration.pdf}{according
to a report} by the Partnership for New York City, an influential
business group. So far, those businesses have shed 520,000 jobs.

While New York is home to more Fortune 500 headquarters than any city in
the country, small businesses are the city's backbone. They represent
roughly 98 percent of the employers in the city and provide jobs to more
than 3 million people, which is about half of its work force, according
to the city.

When New York's economic lockdown started in March the hope was that the
closing of businesses would be temporary and many could weather the
financial blow.

But the devastation to small businesses has become both widespread and
permanent as the economy reopens at a slow pace. Emergency federal aid
has failed to provide enough of a cushion, people remain leery of
resuming normal lives and the threat of a second wave of the virus
looms.

The first to fall were businesses, especially retail shops, that
depended on New York City's massive flow of commuters. And months into
the crisis, established businesses that once seemed invincible,
including some that had ambitious expansion plans, are cratering under a
sustained collapse in consumer spending.

One business that will not reopen is Bank Street Bookstore, a nonprofit
on the Upper West Side run by the Bank Street College of Education. More
than 90 percent of its revenue was in-store sales, mostly to
neighborhood parents, the college's students and elementary
schoolteachers.

``We had to keep reinventing the business every week to two weeks, based
on new guidelines,'' Caitlyn Morrissey, the store's manager, said about
the past months. ``Our cornerstone was in-person sales, not web sales.''

\includegraphics{https://static01.nyt.com/images/2020/07/27/nyregion/00nyvirus-smallbusiness-02/00nyvirus-smallbusiness-02-articleLarge.jpg?quality=75\&auto=webp\&disable=upscale}

Unlike larger firms, small businesses --- bookstores, bodegas, bars,
dental practices, gyms and day care centers --- typically do not have
the financial resources to overcome a few rough days or weeks, let alone
months.

There is no clearinghouse for reliable data on the number of small
businesses that have closed in New York or nationwide. The actual number
of permanent closings in New York is probably higher than Yelp's tally
since it largely focuses on consumer-facing businesses. A small business
is broadly defined by economists as those with under 500 employees.

From March 1 to the end of April, during the height of the pandemic in
New York City, businesses in the city that use the payment company
Square saw their revenues drop by half, according to an analysis the
company provided to The New York Times. The most significant revenue
declines were in the Bronx and Manhattan, the company said.

As part of a \$2.2 trillion emergency aid package adopted in March, the
federal government set aside about \$500 billion in small-business loans
to keep workers employed and companies afloat. But business owners said
they have spent all or most of their loans, paying salaries and bills,
including rent.

More help for small businesses is part of negotiations as the Trump
administration and Republicans and Democrats in Congress try to iron out
another rescue package.

While the worst of the pandemic in the United States struck New York
City first, small businesses across the country have been clobbered.

Between early March and early May,
\href{https://www.nytimes.com/2020/07/13/business/small-businesses-coronavirus.html}{roughly
110,000 small businesses nationwide} shut down, according to
\href{https://www.nber.org/papers/w26989.pdf}{researchers at Harvard}.

In New York, the restaurant and hospitality industry has been one of the
hardest hit. More than 80 percent of the city's restaurants and bars did
not pay full rent in June, according to the NYC Hospitality Alliance.

Among those restaurants was Glady's in Brooklyn. Its revenue plummeted
by two-thirds since March, to about \$12,000 per week in June. The
majority of its sales were from tropical rum drinks served through a
side window of the restaurant.

The owner, William Garfield, said he decided to close in June before
officials started allowing outdoor dining after his landlord said he had
to start paying the full monthly rent, \$8,000, starting in July. Mr.
Garfield said the healthy revenue from drink sales was still not enough
to make ends meet.

``We were thriving,'' said Mr. Garfield, 32, said about Glady's business
before March. ``I would disagree with the sentiment that if someone had
a thriving business they should be able to survive this.''

Mr. Garfield has another restaurant, Mo's Original, and a bar next door,
both of which he plans to keep open. His staff among his businesses has
shrunk from 56 to seven.

He has spent almost all of his small-business stimulus loan, known as
Payroll Protection Assistance, about \$72,000. His insurance company
denied his business interruption claim, citing New York State's order
that restaurants were ``essential businesses'' and could stay open.

``It's the most frustrating situation because it's not about passion
anymore or the work you put in or the hours you put in,'' he said.
``It's all about the mitigating circumstances that are out of your
control.''

In recent weeks, ``For Lease'' signs have started to appear on
storefronts on streets throughout New York, evidence that businesses
that tried to ride out the initial months or abruptly shift to new
online business models could no longer survive.

Business owners said they are at a tipping point. They have exhausted
their federal, state and local aid. And while some landlords have
offered breaks on rent, some business owners say others have been less
flexible.

Owners say they also have to cope with constant uncertainty --- not just
the threat of a resurgence of the virus but also having to navigate
shifting reopening plans.

Restaurants in New York City were expecting to restart indoor dining in
July. Owners bought food and supplies for what they thought would be
larger crowds. But days before the restrictions were to be lifted,
officials halted the plans, citing rising cases in other states that had
allowed indoor dining.

Nearly a third of the 2,800 businesses in New York City that have
permanently closed were restaurants, according to Yelp.

The remaining businesses represent a broad swath of the city's economy,
including small law firms, beauty stores, spas and cleaning companies.

``As a small-business owner, I'm surprised that more businesses have not
closed yet,'' said Andrea Dillon, the owner of Busy Bodies, a day care
she opened on Fulton Street in Brooklyn in 2016.

Ms. Dillon said she noticed the ripple effect of the pandemic in late
February, a few weeks before the city shut down. Parents and caregivers
were canceling upcoming birthday parties and classes.

By early March, she realized that her entire business model --- in which
up to 70 children and adults cram into a play space with toys and live
music --- could not coexist with the coronavirus.

She asked her landlord for a break on her \$6,000 a month rent, but he
refused. Ms. Dillon said she decided in early April to close down.

``The face of New York City storefronts, they will not be forever
changed,'' she said. ``But they will be changed for the foreseeable
future.''

While her management company did not offer a break on rent, another
landlord, Brian Steinwurtzel, said he was doing just that for some of
his roughly 2,000 tenants in New York City, many of them small
businesses. Mr. Steinwurtzel, the co-chief executive at GFP Real Estate,
said he helped them apply for federal assistance and lowered their rents
while business is down.

``It doesn't make any sense to kick them out or fight with them as long
as we are all working together,'' Mr. Steinwurtzel said. ``We believe we
are all in it together, and we all have to help each other out.''

The most vulnerable small businesses in New York City might be those
operated by minority or female owners.
\href{https://www.nytimes.com/2020/05/18/business/minority-businesses-coronavirus-loans.html}{Recent
studies} have shown that these
\href{https://www.nytimes.com/2020/04/10/business/minority-business-coronavirus-loans.html}{were
largely shut out} of federal aid. There are about 10,500 business that
New York City has certified as minority- or female-owned.

A
\href{https://comptroller.nyc.gov/newsroom/comptroller-stringer-analysis-85-percent-of-m-wbes-report-they-will-be-out-of-business-in-six-months-due-to-economic-distress-of-covid-19-pandemic/}{survey
of such businesses} released by the New York City Comptroller's Office
found that 30 percent of them believed they were likely to fold within
the next 30 days.

Among those businesses is ThroughMyKitchen, a catering and snack company
owned by Evelyn Echevarria. Before March, she derived most of her income
from selling goods at street fairs and catering. Her last event was in
March, catering a 120-person wedding in South Carolina.

She is surviving on unemployment benefits, but the largest portion of
that, the federal stimulus of \$600 per week, expired at the end of
July. She also received \$2,000 in assistance from the city.

``It's been very, very hard,'' Ms. Echevarria, 58, said. ``The small
businesses won't be able to survive this. This, to me and many others,
is devastating. It's devastating.''

Sheelagh McNeill contributed research.

Advertisement

\protect\hyperlink{after-bottom}{Continue reading the main story}

\hypertarget{site-index}{%
\subsection{Site Index}\label{site-index}}

\hypertarget{site-information-navigation}{%
\subsection{Site Information
Navigation}\label{site-information-navigation}}

\begin{itemize}
\tightlist
\item
  \href{https://help.nytimes.com/hc/en-us/articles/115014792127-Copyright-notice}{©~2020~The
  New York Times Company}
\end{itemize}

\begin{itemize}
\tightlist
\item
  \href{https://www.nytco.com/}{NYTCo}
\item
  \href{https://help.nytimes.com/hc/en-us/articles/115015385887-Contact-Us}{Contact
  Us}
\item
  \href{https://www.nytco.com/careers/}{Work with us}
\item
  \href{https://nytmediakit.com/}{Advertise}
\item
  \href{http://www.tbrandstudio.com/}{T Brand Studio}
\item
  \href{https://www.nytimes.com/privacy/cookie-policy\#how-do-i-manage-trackers}{Your
  Ad Choices}
\item
  \href{https://www.nytimes.com/privacy}{Privacy}
\item
  \href{https://help.nytimes.com/hc/en-us/articles/115014893428-Terms-of-service}{Terms
  of Service}
\item
  \href{https://help.nytimes.com/hc/en-us/articles/115014893968-Terms-of-sale}{Terms
  of Sale}
\item
  \href{https://spiderbites.nytimes.com}{Site Map}
\item
  \href{https://help.nytimes.com/hc/en-us}{Help}
\item
  \href{https://www.nytimes.com/subscription?campaignId=37WXW}{Subscriptions}
\end{itemize}
