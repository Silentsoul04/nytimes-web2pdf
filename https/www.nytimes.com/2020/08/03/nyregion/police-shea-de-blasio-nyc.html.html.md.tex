Sections

SEARCH

\protect\hyperlink{site-content}{Skip to
content}\protect\hyperlink{site-index}{Skip to site index}

\href{https://www.nytimes.com/section/nyregion}{New York}

\href{https://myaccount.nytimes.com/auth/login?response_type=cookie\&client_id=vi}{}

\href{https://www.nytimes.com/section/todayspaper}{Today's Paper}

\href{/section/nyregion}{New York}\textbar{}These Remarks Might Get a
Police Chief Fired. Not in New York.

\url{https://nyti.ms/3i6iUBB}

\begin{itemize}
\item
\item
\item
\item
\item
\item
\end{itemize}

\href{https://www.nytimes.com/news-event/george-floyd-protests-minneapolis-new-york-los-angeles?action=click\&pgtype=Article\&state=default\&region=TOP_BANNER\&context=storylines_menu}{Race
and America}

\begin{itemize}
\tightlist
\item
  \href{https://www.nytimes.com/2020/07/26/us/protests-portland-seattle-trump.html?action=click\&pgtype=Article\&state=default\&region=TOP_BANNER\&context=storylines_menu}{Protesters
  Return to Other Cities}
\item
  \href{https://www.nytimes.com/2020/07/24/us/portland-oregon-protests-white-race.html?action=click\&pgtype=Article\&state=default\&region=TOP_BANNER\&context=storylines_menu}{Portland
  at the Center}
\item
  \href{https://www.nytimes.com/2020/07/23/podcasts/the-daily/portland-protests.html?action=click\&pgtype=Article\&state=default\&region=TOP_BANNER\&context=storylines_menu}{Podcast:
  Showdown in Portland}
\item
  \href{https://www.nytimes.com/interactive/2020/07/16/us/black-lives-matter-protests-louisville-breonna-taylor.html?action=click\&pgtype=Article\&state=default\&region=TOP_BANNER\&context=storylines_menu}{45
  Days in Louisville}
\end{itemize}

Advertisement

\protect\hyperlink{after-top}{Continue reading the main story}

Supported by

\protect\hyperlink{after-sponsor}{Continue reading the main story}

\hypertarget{these-remarks-might-get-a-police-chief-fired-not-in-new-york}{%
\section{These Remarks Might Get a Police Chief Fired. Not in New
York.}\label{these-remarks-might-get-a-police-chief-fired-not-in-new-york}}

The police commissioner's pointed criticism --- and the fact that he
still has his job --- speaks to the mayor's fraught relationship with
the Police Department.

\includegraphics{https://static01.nyt.com/images/2020/07/30/nyregion/00shea-bdb1/00shea-bdb1-articleLarge-v2.jpg?quality=75\&auto=webp\&disable=upscale}

\href{https://www.nytimes.com/by/emma-g-fitzsimmons}{\includegraphics{https://static01.nyt.com/images/2018/07/18/multimedia/author-emma-g-fitzsimmons/author-emma-g-fitzsimmons-thumbLarge.png}}

By \href{https://www.nytimes.com/by/emma-g-fitzsimmons}{Emma G.
Fitzsimmons}

\begin{itemize}
\item
  Aug. 3, 2020
\item
  \begin{itemize}
  \item
  \item
  \item
  \item
  \item
  \item
  \end{itemize}
\end{itemize}

The criticism of Mayor Bill de Blasio's law enforcement policies was
stinging.

A law banning the use of chokeholds and similar types of restraints by
police officers was ``insane.'' Agreeing to cut the Police Department
budget was a ``bow to mob rule.'' Those who failed to ``stand up for
what's right'' were ``cowards.''

But the outspoken critic was not a rival of the mayor's or one of the
candidates vying to succeed him. It was Dermot F. Shea, Mr. de Blasio's
own police commissioner, a trusted ally who went rogue in media
interviews and in a private address to police brass.

The commissioner's comments --- and the fact that he still has his job
--- speak to the fraught relationship that Mr. de Blasio has maintained
with the Police Department throughout his tenure.

Mr. de Blasio has made racial justice and an overhaul of police
practices central to his political brand, from his initial mayoral
campaign in 2013 to his brief candidacy for president last year.

But as mayor, Mr. de Blasio has often shown surprising deference to his
police commissioners --- three Irish-American veterans of the department
--- adopting a hands-off approach that affords the commissioners an
unusual amount of leeway.

The mayor's approach has frustrated advocacy groups that favor broad
changes to policing in New York and that contend he is not doing enough
to hold the police accountable, especially after a wave of Black Lives
Matter protests. Some are calling for Commissioner Shea to resign or be
removed, but the mayor has dismissed those suggestions amid a recent
\href{https://www.nytimes.com/2020/07/16/nyregion/nyc-shootings-nypd.html}{spike
in violence in the city}.

A video that surfaced last week that showed officers pulling a protester
into an unmarked van --- evoking the practices of aggressive federal
agents in Portland, Ore. --- intensified the backlash against the
police.

Image

After a video depicted New York police officers arresting a protester
and throwing her in an unmarked van on Tuesday, Mayor Bill de Blasio
said it was the ``wrong time and the wrong place'' to make such an
arrest.Credit...@Naddleez on Twitter via Reuters

The Police Department
\href{https://twitter.com/NYPDnews/status/1288270680614739968}{said in a
statement that the protester} had been taken into custody by officers
from the warrant squad in connection with ``damaging police cameras
during five separate criminal incidents in and around City Hall Park.''

Mr. de Blasio said that it was ``the wrong time and the wrong place'' to
make that arrest, and that any scenes similar to Portland were
``troubling.'' The mayor said he would talk to Commissioner Shea about
``a better way to get that done,'' though he said destroying police
property was not acceptable.

In cities like Atlanta and Louisville, Ky., police chiefs have
\href{https://www.nytimes.com/2020/06/15/us/police-chiefs-fatal-shooting-atlanta.html}{lost
their jobs} after episodes of police violence in the wake of protests
over the death of George Floyd in Minneapolis. But Commissioner Shea's
job security seems far more assured.

``I've been very clear about my faith in Commissioner Shea,'' Mr. de
Blasio said at a recent news conference. ``I have known him over these
whole seven years of the administration, and I've seen what he can do.''

On Monday, the mayor again praised Commissioner Shea after a violent
weekend in the city, saying: ``There's no doubt in my mind he will
succeed'' in bringing crime down.

All of Mr. de Blasio's police chiefs have been acolytes of his first
commissioner, William J. Bratton, who became a policing celebrity in the
1990s for his ``broken windows'' approach to fighting crime. The second
commissioner, James P. O'Neill, was a protégé of Mr. Bratton's, and
offered continuity, as has Commissioner Shea, known for overseeing the
data-driven Compstat program.

Mr. de Blasio kept going back to the Bratton orbit because the results
were good, said Chuck Wexler, executive director of the Police Executive
Research Forum, a law enforcement policy nonprofit.

``Crime has continued to go down, and people were generally satisfied,''
Mr. Wexler said. ``If you're de Blasio, you're like, `Why would I make a
dramatic change?'''

Still, the mayor's opponents say it took far too long to
\href{https://www.nytimes.com/2019/08/19/nyregion/daniel-pantaleo-fired.html}{fire
Daniel Pantaleo}, the officer whose chokehold led to
\href{https://www.nytimes.com/2015/06/14/nyregion/eric-garner-police-chokehold-staten-island.html}{Eric
Garner's death} in 2014, and argue that the mayor was too slow to fix a
process that allowed officers' disciplinary records to remain secret
under a state measure known as 50-a.

\includegraphics{https://static01.nyt.com/images/2020/07/30/nyregion/00shea-bdb4/merlin_158085954_ce8b9923-1d77-4b2d-b4fa-f1f8b4aa4043-articleLarge.jpg?quality=75\&auto=webp\&disable=upscale}

Policing has been a
\href{https://www.nytimes.com/2015/01/12/nyregion/in-police-rift-mayor-de-blasios-missteps-included-thinking-it-would-pass.html}{persistently
thorny issue} for Mr. de Blasio, emerging early in his tenure. His
election in 2013 was fueled in part by his opposition to the
stop-and-frisk policies under Mayor Michael R. Bloomberg, along with a
television ad starring Mr. de Blasio's son, Dante, who is Black and
pledged that his father would end the discriminatory policing practice.

Late in Mr. de Blasio's first year as mayor, a Staten Island grand jury
refused to bring charges against Mr. Pantaleo. The mayor chose not to
criticize the decision, disappointing many of his supporters. But in his
response, he also angered the police rank and file when he disclosed
that he had urged his son
\href{http://www.nytimes.com/2014/12/04/nyregion/de-blasio-reacts-as-mayor-and-a-father-to-chokehold-case-decision.html?_r=0}{to
take special precautions} when dealing with police officers.

The police unions accused Mr. de Blasio of creating an anti-police
environment, which they said contributed to the
\href{https://www.nytimes.com/2014/12/21/nyregion/two-police-officers-shot-in-their-patrol-car-in-brooklyn.html}{fatal
shootings} of two police officers in December 2014. Officers
\href{https://www.nytimes.com/2015/01/05/nyregion/police-officers-gather-for-the-funeral-of-wenjian-liu-killed-in-an-ambush.html}{turned
their backs} on the mayor at the men's funerals.

Since then, the mayor has been careful in trying not to alienate the
police, and that could be one reason he gives so much independence to
his police commissioners.

Black leaders have repeatedly called on Mr. de Blasio to select a
commissioner who is a person of color. Yet last year, he
\href{https://www.nytimes.com/2019/11/05/nyregion/nypd-police-commissioner-de-blasio.html}{passed
over Benjamin Tucker}, who was the second-highest-ranking police leader
and is Black, in favor of Commissioner Shea.

Mr. de Blasio appears to genuinely like Commissioner Shea, who joined
the department in 1991, saying he chose him because he was a ``proven
change agent.''

Image

Mr. de Blasio, left, with Commissioner Shea and Benjamin Tucker, right,
the first deputy commissioner. Black leaders have pushed for a person of
color, like Mr. Tucker, as commissioner.Credit...Dave Sanders for The
New York Times

The mayor and the police commissioner may not seem like natural allies.
Mr. de Blasio is a Democrat in the mold of progressives like Senator
Bernie Sanders of Vermont; Commissioner Shea is a registered Republican
who has declined to say if he voted for President Trump in 2016.

Devora Kaye, a Police Department spokeswoman, said Commissioner Shea's
relationship with the mayor ``continues to be strong and productive and,
integrally, very open and honest.''

That frankness was recently on display after Mr. de Blasio signed the
chokehold bill, which also banned actions by police officers that
compress a person's diaphragm.

The next day, July 16, Commissioner Shea railed against city leaders as
``cowards'' at a CompStat meeting. He cited political pressure to push
some people out of jail and keep others out, in an apparent reference to
recent changes in bail laws and the court system that he has blamed for
the spike in crime.

When
\href{https://www.nydailynews.com/new-york/nyc-crime/ny-nypd-commissioner-dermot-shea-calls-city-leaders-cowards-20200718-jqh3qagax5cjda73izzuq3t4xu-story.html}{The
New York Daily News} published a video of his comments, Mr. de Blasio
told reporters that Commissioner Shea's ``language wasn't
constructive.''

``I've understood it was important for him to express some of those
concerns,'' the mayor said, ``but now it's time to move forward.''

At least three City Council members have called on Commissioner Shea to
resign, as has Maya D. Wiley, a former lawyer for Mr. de Blasio who is
\href{https://www.nytimes.com/2020/07/28/nyregion/maya-wiley-mayor-nyc.html}{considering
a run for mayor}.

``I've never seen a commissioner, or any head of any agency, be so
outwardly insubordinate and disdainful of the rule of law, the City
Council and the democratically elected mayor who is their boss,'' said
Councilman Rory Lancman, a Queens Democrat.

Ms. Kaye, the police spokeswoman, said that Commissioner Shea had not
been talking about Mr. de Blasio when he referred to cowards at the
CompStat meeting. She declined to say whom he was admonishing; the
commissioner was not made available for an interview.

It is unusual for a police commissioner to speak as bluntly as
Commissioner Shea did at the CompStat meeting, said Kenneth Sherrill, a
professor emeritus of political science at Hunter College.

``A police commissioner speaking among friends might say what Shea said,
but to say it in a large forum indicates a level of exasperation that
shows either a loss of self-control or a lack of political judgment,''
Professor Sherrill said.

Commissioner Shea has been under immense pressure during the pandemic,
and
\href{https://www1.nyc.gov/site/nypd/about/memorials/covid-19-memorial.page}{more
than 40 members of the Police Department have died} of Covid-19, said
Mr. Wexler of the Police Executive Research Forum, who has spoken to
Commissioner Shea several times in recent months. The
\href{https://www.nytimes.com/2020/06/15/nyregion/nypd-plainclothes-cops.html}{decision
to disband the department's anti-crime units} shows that Commissioner
Shea understands the need to restore public trust, he said.

``Police chiefs face a tough balancing act: recognizing the need for
reform --- and he gets it --- and you can't lose the cops,'' Mr. Wexler
said.

Jumaane Williams, the city's public advocate, pointed out that the mayor
was much harder on his health commissioner, Dr. Oxiris Barbot, who had
to apologize for
\href{https://www.nytimes.com/2020/05/19/nyregion/barbot-police-health-coronavirus.html}{comments
she made about not giving masks to the police} during a shortage in
March.

``What's most stark is that Dr. Barbot was forced to apologize,'' said
Mr. Williams, who added that he was not yet calling for Commissioner
Shea's resignation. ``But there seems to be no accountability for the
leadership of the N.Y.P.D.''

Advertisement

\protect\hyperlink{after-bottom}{Continue reading the main story}

\hypertarget{site-index}{%
\subsection{Site Index}\label{site-index}}

\hypertarget{site-information-navigation}{%
\subsection{Site Information
Navigation}\label{site-information-navigation}}

\begin{itemize}
\tightlist
\item
  \href{https://help.nytimes.com/hc/en-us/articles/115014792127-Copyright-notice}{©~2020~The
  New York Times Company}
\end{itemize}

\begin{itemize}
\tightlist
\item
  \href{https://www.nytco.com/}{NYTCo}
\item
  \href{https://help.nytimes.com/hc/en-us/articles/115015385887-Contact-Us}{Contact
  Us}
\item
  \href{https://www.nytco.com/careers/}{Work with us}
\item
  \href{https://nytmediakit.com/}{Advertise}
\item
  \href{http://www.tbrandstudio.com/}{T Brand Studio}
\item
  \href{https://www.nytimes.com/privacy/cookie-policy\#how-do-i-manage-trackers}{Your
  Ad Choices}
\item
  \href{https://www.nytimes.com/privacy}{Privacy}
\item
  \href{https://help.nytimes.com/hc/en-us/articles/115014893428-Terms-of-service}{Terms
  of Service}
\item
  \href{https://help.nytimes.com/hc/en-us/articles/115014893968-Terms-of-sale}{Terms
  of Sale}
\item
  \href{https://spiderbites.nytimes.com}{Site Map}
\item
  \href{https://help.nytimes.com/hc/en-us}{Help}
\item
  \href{https://www.nytimes.com/subscription?campaignId=37WXW}{Subscriptions}
\end{itemize}
