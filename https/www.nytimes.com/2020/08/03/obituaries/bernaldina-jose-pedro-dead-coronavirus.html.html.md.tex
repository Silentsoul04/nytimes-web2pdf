Sections

SEARCH

\protect\hyperlink{site-content}{Skip to
content}\protect\hyperlink{site-index}{Skip to site index}

\href{https://www.nytimes.com/section/obituaries}{Obituaries}

\href{https://myaccount.nytimes.com/auth/login?response_type=cookie\&client_id=vi}{}

\href{https://www.nytimes.com/section/todayspaper}{Today's Paper}

\href{/section/obituaries}{Obituaries}\textbar{}Bernaldina José Pedro,
Repository of Indigenous Culture, Dies at 75

\url{https://nyti.ms/3i4iAne}

\begin{itemize}
\item
\item
\item
\item
\item
\end{itemize}

\href{https://www.nytimes.com/news-event/coronavirus?action=click\&pgtype=Article\&state=default\&region=TOP_BANNER\&context=storylines_menu}{The
Coronavirus Outbreak}

\begin{itemize}
\tightlist
\item
  live\href{https://www.nytimes.com/2020/08/04/world/coronavirus-cases.html?action=click\&pgtype=Article\&state=default\&region=TOP_BANNER\&context=storylines_menu}{Latest
  Updates}
\item
  \href{https://www.nytimes.com/interactive/2020/us/coronavirus-us-cases.html?action=click\&pgtype=Article\&state=default\&region=TOP_BANNER\&context=storylines_menu}{Maps
  and Cases}
\item
  \href{https://www.nytimes.com/interactive/2020/science/coronavirus-vaccine-tracker.html?action=click\&pgtype=Article\&state=default\&region=TOP_BANNER\&context=storylines_menu}{Vaccine
  Tracker}
\item
  \href{https://www.nytimes.com/2020/08/02/us/covid-college-reopening.html?action=click\&pgtype=Article\&state=default\&region=TOP_BANNER\&context=storylines_menu}{College
  Reopening}
\item
  \href{https://www.nytimes.com/live/2020/08/04/business/stock-market-today-coronavirus?action=click\&pgtype=Article\&state=default\&region=TOP_BANNER\&context=storylines_menu}{Economy}
\end{itemize}

Advertisement

\protect\hyperlink{after-top}{Continue reading the main story}

Supported by

\protect\hyperlink{after-sponsor}{Continue reading the main story}

Those We've Lost

\hypertarget{bernaldina-josuxe9-pedro-repository-of-indigenous-culture-dies-at-75}{%
\section{Bernaldina José Pedro, Repository of Indigenous Culture, Dies
at
75}\label{bernaldina-josuxe9-pedro-repository-of-indigenous-culture-dies-at-75}}

Ms. Pedro was a leader of Brazil's Macuxi people and carried a message
of alarm for their fate to Pope Francis. She died of the novel
coronavirus.

\includegraphics{https://static01.nyt.com/images/2020/08/05/obituaries/30Pedro/merlin_175120329_f9fbbf3d-bbbd-43ac-bb1f-0ee5a4d7ee77-articleLarge.jpg?quality=75\&auto=webp\&disable=upscale}

By Michael Astor

\begin{itemize}
\item
  Aug. 3, 2020
\item
  \begin{itemize}
  \item
  \item
  \item
  \item
  \item
  \end{itemize}
\end{itemize}

\emph{This obituary is part of a series about people who have died in
the coronavirus pandemic. Read about others}
\href{https://www.nytimes.com/interactive/2020/obituaries/people-died-coronavirus-obituaries.html}{\emph{here}}\emph{.}

Bernaldina José Pedro possessed a wealth of knowledge about the songs,
dances, crafts, medicines and prayers of the Macuxi Indigenous people,
who live in northernmost Brazil. And she was a respected voice in the
successful struggle to establish a 4 million-acre Indigenous territory
on the border with Guyana.

But what was especially satisfying to her was a 2018 trip to Rome to
meet Pope Francis, because to her it showed the world that a woman could
be an Indigenous leader.

``It made a great impression on her,'' said her son, Jaider Esbell. ``It
was the first time she left Brazil, and she was proud to be performing
what would usually be a man's role, usually something a chief would
do.''

Ms. Pedro died on June 24 at a hospital in Boa Vista, the capital of
Roraima state in Brazil. She was 75. Mr. Esbell said the cause was
Covid-19.

Better known as Vovó Bernaldina (Portuguese for Grandma Bernaldina), Ms.
Pedro was Roman Catholic and a big fan of the Argentine-born pope, the
first from Latin America, even without renouncing her traditional
beliefs --- a common practice in the region. During her short meeting
with Francis, in a general audience in St. Peter's Square, she gave him
a letter warning that the Brazilian government might try to reclaim the
Indigenous reservation on which her people lived. She asked for his
help.

The
\href{https://www.ecoamazonia.org.br/2018/11/indigenas-macuxi-recebidos-papa-francisco/}{meeting}was
captured in the 2020 documentary ``Amazonian Cosmos,'' by Daniel
Schweizer, which recounted her son's efforts to raise awareness about
his people.

Ms. Pedro was born Koko Meriná Eremunkon on March 25, 1945, to Samuel
José de Souza and Marina José in the Indigenous village of Pedra Preta,
in Guyana, where they hunted, farmed and fished for a living. In her 20s
she married Marcelo Pedro and moved to his village, Mataruca, just
across the border in Brazil.

Her husband's family were tribal leaders, and she rose in influence
through marriage, but she also commanded respect for the wisdom,
charisma and guidance she offered while the Macuxi fought off court
challenges, squatters and often violent attacks until their reservation,
the Raposa Serra do Sol, was officially established in 2009.

Ms. Pedro was the author of ``Chants and Enchantment --- Meriná
Eremunkon,'' a 2019 book written with Devair Fiorotti.

In addition to Mr. Esbell, she is survived by six other children,
Marcilio, Benjamin, Aguinês, Jorge, Eldina and Charles; and 15
grandchildren. Her sons Jaime and Horacio died earlier.

When Covid-19 first appeared in her village, Ms. Pedro was the one whom
people called on to fight the new disease. She would perform a
shamanistic ritual that involved filling the hut of an ailing person
with smoke, chanting and dance. That was how she may have become
infected, her son said.

``She died doing what she liked to do,'' Mr. Esbell added. He said his
mother had imparted much of her cultural knowledge to her daughter
Eldina, who will now assume Ms. Pedro's role as a keeper of Macuxi
traditions.

\href{https://www.nytimes.com/interactive/2020/obituaries/people-died-coronavirus-obituaries.html?action=click\&pgtype=Article\&state=default\&region=BELOW_MAIN_CONTENT\&context=covid_obits_promo}{}

\hypertarget{those-weve-lost}{%
\section{Those We've Lost}\label{those-weve-lost}}

The coronavirus pandemic has taken an incalculable death toll. This
series is designed to put names and faces to the numbers.

Read more

\includegraphics{https://static01.nyt.com/images/2020/08/05/obituaries/03Woods/03Woods-square640.jpg}

\hypertarget{helen-jones-woods}{%
\section{Helen Jones Woods}\label{helen-jones-woods}}

d. Sarasota, Fla.

Musician in all-female, multi-racial jazz band

\includegraphics{https://static01.nyt.com/images/2020/08/05/obituaries/30Pedro/30Pedro-square640.jpg}

\hypertarget{bernaldina-josuxe9-pedro}{%
\section{Bernaldina José Pedro}\label{bernaldina-josuxe9-pedro}}

d. Boa Vista, Brazil

Leader among the Indigenous Macuxi

\includegraphics{https://static01.nyt.com/images/2020/08/05/obituaries/31Swing/merlin_175167783_8913bc90-0d64-43f3-a655-1bb1bf1601c9-square640.jpg}

\hypertarget{john-eric-swing}{%
\section{John Eric Swing}\label{john-eric-swing}}

d. Fountain Valley, Calif.

Champion of Filipino-Americans

\includegraphics{https://static01.nyt.com/images/2020/07/27/obituaries/27Victor/merlin_175001436_38b11f8e-227a-4e2c-9821-7618af9b2524-square640.jpg}

\hypertarget{victor-victor}{%
\section{Victor Victor}\label{victor-victor}}

d. Santo Domingo, Dominican Republic

Beloved musician of the Dominican Republic

\includegraphics{https://static01.nyt.com/images/2020/08/05/obituaries/31Negron/merlin_175160169_516322ae-fd23-4969-b6b2-193ced371105-square640.jpg}

\hypertarget{dr-eddie-negruxf3n}{%
\section{Dr. Eddie Negrón}\label{dr-eddie-negruxf3n}}

d. Fort Walton Beach, Fla.

Internist on Florida's Emerald Coast

\includegraphics{https://static01.nyt.com/images/2020/07/30/obituaries/30Dobson/merlin_175115928_f6b9271c-8f05-4fe1-a38a-5ca4a58f8935-square640.jpg}

\hypertarget{dobby-dobson}{%
\section{Dobby Dobson}\label{dobby-dobson}}

d. Coral Springs, Fla.

Jamaican singer and songwriter

Advertisement

\protect\hyperlink{after-bottom}{Continue reading the main story}

\hypertarget{site-index}{%
\subsection{Site Index}\label{site-index}}

\hypertarget{site-information-navigation}{%
\subsection{Site Information
Navigation}\label{site-information-navigation}}

\begin{itemize}
\tightlist
\item
  \href{https://help.nytimes.com/hc/en-us/articles/115014792127-Copyright-notice}{©~2020~The
  New York Times Company}
\end{itemize}

\begin{itemize}
\tightlist
\item
  \href{https://www.nytco.com/}{NYTCo}
\item
  \href{https://help.nytimes.com/hc/en-us/articles/115015385887-Contact-Us}{Contact
  Us}
\item
  \href{https://www.nytco.com/careers/}{Work with us}
\item
  \href{https://nytmediakit.com/}{Advertise}
\item
  \href{http://www.tbrandstudio.com/}{T Brand Studio}
\item
  \href{https://www.nytimes.com/privacy/cookie-policy\#how-do-i-manage-trackers}{Your
  Ad Choices}
\item
  \href{https://www.nytimes.com/privacy}{Privacy}
\item
  \href{https://help.nytimes.com/hc/en-us/articles/115014893428-Terms-of-service}{Terms
  of Service}
\item
  \href{https://help.nytimes.com/hc/en-us/articles/115014893968-Terms-of-sale}{Terms
  of Sale}
\item
  \href{https://spiderbites.nytimes.com}{Site Map}
\item
  \href{https://help.nytimes.com/hc/en-us}{Help}
\item
  \href{https://www.nytimes.com/subscription?campaignId=37WXW}{Subscriptions}
\end{itemize}
