Sections

SEARCH

\protect\hyperlink{site-content}{Skip to
content}\protect\hyperlink{site-index}{Skip to site index}

\href{https://www.nytimes.com/section/world/europe}{Europe}

\href{https://myaccount.nytimes.com/auth/login?response_type=cookie\&client_id=vi}{}

\href{https://www.nytimes.com/section/todayspaper}{Today's Paper}

\href{/section/world/europe}{Europe}\textbar{}John Hume, Nobel Laureate
for Work in Northern Ireland, Dies at 83

\url{https://nyti.ms/2D4gipl}

\begin{itemize}
\item
\item
\item
\item
\item
\item
\end{itemize}

Advertisement

\protect\hyperlink{after-top}{Continue reading the main story}

Supported by

\protect\hyperlink{after-sponsor}{Continue reading the main story}

\hypertarget{john-hume-nobel-laureate-for-work-in-northern-ireland-dies-at-83}{%
\section{John Hume, Nobel Laureate for Work in Northern Ireland, Dies at
83}\label{john-hume-nobel-laureate-for-work-in-northern-ireland-dies-at-83}}

The politician's campaign for peace was seen as a driving force behind
an end to 25 years of sectarian conflict in the territory.

\includegraphics{https://static01.nyt.com/images/2020/08/04/world/03hume-obit3/03hume-obit3-articleLarge-v2.jpg?quality=75\&auto=webp\&disable=upscale}

By \href{https://www.nytimes.com/by/alan-cowell}{Alan Cowell}

\begin{itemize}
\item
  Aug. 3, 2020
\item
  \begin{itemize}
  \item
  \item
  \item
  \item
  \item
  \item
  \end{itemize}
\end{itemize}

John Hume, a moderate Roman Catholic politician who was
\href{https://www.nytimes.com/1998/10/17/world/2-ulster-peacemakers-win-the-nobel-prize.html}{awarded
the Nobel Peace Prize} for his dogged and ultimately successful campaign
to end decades of bloodshed in his native Northern Ireland, died on
Monday in the northern city of Derry. He was 83.

His death, at a nursing home, was announced by his family in a
statement, which did not give the cause, though his wife, Pat Hume, had
earlier acknowledged that he was struggling with dementia.

``It seems particularly apt for these strange and fearful days to
remember the phrase that gave hope to John and so many of us through
dark times: We shall overcome,'' his family said.

Mr. Hume, a former French teacher who was known for a sharp wit but
rarely for rhetorical flourishes, rose from hardscrabble beginnings to
become the longtime leader of the Social Democratic and Labour Party and
a towering figure in the grinding and oft-thwarted drive to end
\href{https://www.nytimes.com/2018/10/04/world/europe/northern-ireland-troubles.html}{25
years of ``The Troubles,''} as Northern Ireland's strife was known.

In his campaign for peace, inspired by the example of the Rev. Dr.
Martin Luther King Jr., he employed a winning combination of public
exhortation against the violence of the Irish Republican Army and secret
diplomacy with its political leadership, sitting down for talks in his
modest rowhouse over coffee. Deftly and persistently he enlisted the
White House to help him reach his goal.

His efforts were recognized when he shared the Nobel with the Protestant
leader David Trimble in 1998, the year of the
\href{https://www.nytimes.com/1998/04/11/world/irish-accord-overview-irish-talks-produce-accord-stop-decades-bloodshed-with.html}{Good
Friday peace agreement}, which crowned his commitment to ending the
unrest that had claimed more than 3,000 lives.

A television poll in the Irish Republic in 2010 proclaimed Mr. Hume
``Ireland's Greatest,'' ahead of prominent contenders like the rock star
Bono. In 2012, Pope Benedict XVI awarded him a papal knighthood.

Paradoxically, in bringing more radical Roman Catholic figures to the
negotiating table --- notably Gerry Adams, the head of the I.R.A.'s
political wing --- Mr. Hume undermined his own party's appeal to voters.
Battling poor health, he resigned in 2001 as leader of the Social
Democratic and Labour Party, which he had led since 1979, without
enjoying the high office that might normally reward an architect of
historic change.

\includegraphics{https://static01.nyt.com/images/2020/08/04/world/03hume-obit1/merlin_175256247_75e064a6-2386-464d-85fe-24f127ae3ee1-articleLarge.jpg?quality=75\&auto=webp\&disable=upscale}

In 2004, he said he would no longer seek election to the European and
British Parliaments, which he joined in 1979 and 1983, respectively. In
late 2015, his wife, who was also his political manager, told the BBC
that he was experiencing ``severe difficulties'' with dementia.

Throughout a career in Northern Ireland politics, in which
finger-pointing and recrimination amplified a drumbeat of bombings and
killings, Mr. Hume stood as a voice of reason, counseling against the
cycles of bloodshed between the Protestant majority and the Roman
Catholic minority.

``An eye for an eye leaves everyone blind,'' he said, attributing the
comment to Dr. King.

He argued instead for dialogue and reconciliation to still the furious
conflict that pitted the I.R.A. against Protestant paramilitary groups
and thousands of British Army soldiers. ``We have to start spilling our
sweat, not our blood,'' he declared.

In the parlance of Northern Ireland, Mr. Hume was a ``nationalist''
whose dream of a reunited Ireland had no place for the violence embraced
by ``republicans'' like the I.R.A., with its armed fighters and networks
of financiers, bomb-makers and sympathizers in the region and in the
United States. Rather, he foresaw a time when Northern Ireland's divide
would give way to peace and economic self-interest.

Mr. Hume was so concerned about multimillion-dollar funding for the
I.R.A. by Irish Americans that he traveled frequently to Washington to
convince American leaders, from President Jimmy Carter onward, that a
majority of Northern Irish people rejected the I.R.A.'s violent methods.
It was a message that culminated in a more active role in Northern
Ireland adopted by President Bill Clinton.

In one of three of visits to the Clinton White House by Mr. Hume, Mr.
Clinton lauded him as ``Ireland's most tireless champion for civil
rights and its most eloquent spokesman for peace.'' Back home, Mr. Hume
had a parallel reputation as a man who did not suffer fools gladly.

``Question: What is the difference between John Hume and God?'' one joke
asked. ``Answer: God doesn't think he is John Hume.''

Image

President Bill Clinton with Mr. Hume and Mr. Hume's wife, Pat, on the
Peace Bridge in Derry, Northern Ireland, in 2014.Credit...Peter
Morrison/Associated Press

Mr. Hume's most dramatic initiative played out in the late 1980s and
mid-'90s, when he held secret peace talks with Mr. Adams at a rowhouse
in Derry, which those seeking to retain close ties to Britain refer to
as Londonderry.

The house itself was attacked several times over the years by
firebombers --- some Protestants, others Catholic supporters of the
I.R.A. --- a token of the hazards and threats from both sides that
persisted during the quest for peace. Mr. Hume was hospitalized several
times in the mid-1990s for what he called ``a case of nerves.''

He said the talks, over cups of coffee and glasses of Ballygowan mineral
water, had begun in the early 1990s, a resumption of discussions dating
to 1988.

For many Britons and Northern Irish Protestants, Mr. Adams was a pariah
at the time, with a reputed history as an I.R.A. commander, a role he
has denied. As president of Sinn Fein --- the political wing of the
outlawed I.R.A., which the British authorities and many others viewed as
a terrorist organization --- Mr. Adams was depicted by his critics as no
more than a front for the ``hard men'' of violence. And in talking to
him, Mr. Hume risked the accusation that he was treating with
terrorists.

``One was a man of peace and the other a man of war,'' the correspondent
John Darnton wrote
in\href{http://www.nytimes.com/1994/09/05/world/turning-point-ira-cease-fire-special-report-2-irish-foes-journey-deeds-words.html?pagewanted=1}{The
New York Times} in 1994.

Mr. Hume's essential achievement was to convince Mr. Adams that if the
I.R.A. renounced violence, Sinn Fein could join peace talks from which
it had long been excluded, gaining a yearned-for political legitimacy.
The effort was part of a complicated international process. The British
government had itself been conducting unpublicized back-channel contacts
with Sinn Fein.

``Central to the discussions from my point of view was violence,'' Mr.
Hume said. ``I kept asking for the reason for it. I had said publicly
that the I.R.A. had been discussed as criminals and gangsters. I said I
wish they were. If they were, we could have gotten rid of them in a
fortnight. The problem was they believed in what they were saying.''

He added: ``The whole objective was to bring about a total cessation of
violence. We eventually agreed on that. Then the question was, How to
get there?''

The contacts led to a ``complete cessation of military operations''
announced by the I.R.A. in 1994 --- a critical steppingstone on the way
to the 1998 peace accord, though an
\href{https://www.nytimes.com/1996/02/10/world/bomb-wounds-100-in-london-as-ira-truce-is-said-to-end.html}{I.R.A.
bombing campaign in London in 1996} would shatter the cease-fire before
it was restored.

Finally, in September 1997, Sinn Fein, representing the I.R.A., and the
leaders of Protestant parties sat at the same negotiating table for the
first time since 1922, when Ireland was partitioned into an independent
Irish Republic in the south and the British-run province in the north.
Mr. Hume dismissed widespread suggestions that the I.R.A. had bombed its
way to the peace table. Without the violence, Mr. Hume argued, Sinn Fein
would have been admitted to the talks years earlier.

But the relationship came with a heavy political cost.

In the early 1990s, Mr. Hume's Social Democratic and Labour Party had
controlled about two-thirds of the Catholic vote, Sinn Fein one third.
By mid-1997 Sinn Fein's share had risen to about 40 percent. The trend
continued. In the 2011 elections to the Northern Ireland Assembly, Sinn
Fein won twice as many seats as the S.D.L.P. By helping to give Sinn
Fein a place at the peace table, Mr. Hume had hurt his own party, and
many of its members resented him for it.

Image

Mr. Hume in Dublin in 1997 with Gerry Adams, left, the Sinn Fein leader,
and Bertie Ahern, center, Ireland's prime minister at the
time.Credit...Pat Maxwell/Associated Press

John Hume was born in Derry on Jan. 18, 1937, the eldest of seven
children of Sam Hume, a shipyard riveter who lived for many years on
state welfare, and Annie Doherty Hume.

In a memoir, ``John Hume --- Personal Views: Politics, Peace and
Reconciliation in Ireland,'' he recalled his father taking him to a
Republican meeting in the late 1940s.

``They were all waving flags and stirring up emotion for the united
Ireland and an end to partition,'' he wrote. ``When my father saw that I
was affected, he put his hand gently on my shoulder and said, `Son,
don't get involved in that stuff,' and I said, `Why not, Da?' He
answered simply, `Because you can't eat the flag.' That was my first
lesson in politics, and it has stayed with me to this day.''

He won a scholarship to St. Columb's College, a grammar school in Derry
for the small elite of middle-class Catholic professionals, and studied
for the priesthood before switching to a degree course in French and
history. He taught French in his 20s and became a leader in both the
civil rights movement and the fledgling credit union movement.

In 1960, after three years of courtship, he married Pat Hone, a fellow
teacher. At one point, alongside their teaching, the couple ran a modest
smoked-salmon business.

He is survived by his wife; their five children, Terese, Áine, Aidan,
John and Mo; as well as siblings and grandchildren, the family statement
said.

As a rising politician, Mr. Hume was instrumental in preparing the
\href{https://www.nytimes.com/1985/11/24/weekinreview/anglo-irish-agreement-pits-both-ends-against-the-middle.html}{Anglo-Irish
agreement of 1985}. The pact gave the Irish Republic, for the first
time, a consultative role in the affairs of the North, but it also
guaranteed that no change in the territory's political status could be
made without the consent of its Protestant majority. He remained close
to leading political figures in the United States and was an energetic
salesman for the territory, helping to persuade companies to move there.

When Jean Kennedy Smith, the older sister of Senator Edward M. Kennedy,
was appointed ambassador to the Irish Republic in 1993, Mr. Hume became
one of her constant advisers. She responded by helping to persuade
President Clinton to end American sanctions against Sinn Fein and to
support the inclusion of Mr. Adams and Sinn Fein at the peace talks.
\href{https://www.nytimes.com/2020/06/18/us/politics/jean-kennedy-smith-dead.html}{(Ms.
Smith died in June at 92.)}

A committed European, Mr. Hume believed that just as Western European
borders were weakened to encourage trade, so could the border between
Northern Ireland and the Irish Republic be gradually eliminated as their
economies became interdependent.

``I am a teacher,'' he said. ``You keep saying the same things over and
over. Then you know you're getting through when someone in a pub gives
you back your own words.''

James F. Clarity, who died in 2007, contributed reporting.

Advertisement

\protect\hyperlink{after-bottom}{Continue reading the main story}

\hypertarget{site-index}{%
\subsection{Site Index}\label{site-index}}

\hypertarget{site-information-navigation}{%
\subsection{Site Information
Navigation}\label{site-information-navigation}}

\begin{itemize}
\tightlist
\item
  \href{https://help.nytimes.com/hc/en-us/articles/115014792127-Copyright-notice}{©~2020~The
  New York Times Company}
\end{itemize}

\begin{itemize}
\tightlist
\item
  \href{https://www.nytco.com/}{NYTCo}
\item
  \href{https://help.nytimes.com/hc/en-us/articles/115015385887-Contact-Us}{Contact
  Us}
\item
  \href{https://www.nytco.com/careers/}{Work with us}
\item
  \href{https://nytmediakit.com/}{Advertise}
\item
  \href{http://www.tbrandstudio.com/}{T Brand Studio}
\item
  \href{https://www.nytimes.com/privacy/cookie-policy\#how-do-i-manage-trackers}{Your
  Ad Choices}
\item
  \href{https://www.nytimes.com/privacy}{Privacy}
\item
  \href{https://help.nytimes.com/hc/en-us/articles/115014893428-Terms-of-service}{Terms
  of Service}
\item
  \href{https://help.nytimes.com/hc/en-us/articles/115014893968-Terms-of-sale}{Terms
  of Sale}
\item
  \href{https://spiderbites.nytimes.com}{Site Map}
\item
  \href{https://help.nytimes.com/hc/en-us}{Help}
\item
  \href{https://www.nytimes.com/subscription?campaignId=37WXW}{Subscriptions}
\end{itemize}
