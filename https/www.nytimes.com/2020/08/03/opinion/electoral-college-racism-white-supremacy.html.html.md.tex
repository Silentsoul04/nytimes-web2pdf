\href{/section/opinion}{Opinion}\textbar{}How Has the Electoral College
Survived for This Long?

\url{https://nyti.ms/33fVXYK}

\begin{itemize}
\item
\item
\item
\item
\item
\end{itemize}

\includegraphics{https://static01.nyt.com/images/2020/08/03/opinion/03keyssarWeb/03keyssarWeb-articleLarge.jpg?quality=75\&auto=webp\&disable=upscale}

Sections

\protect\hyperlink{site-content}{Skip to
content}\protect\hyperlink{site-index}{Skip to site index}

\href{/section/opinion}{Opinion}

\hypertarget{how-has-the-electoral-college-survived-for-this-long}{%
\section{How Has the Electoral College Survived for This
Long?}\label{how-has-the-electoral-college-survived-for-this-long}}

Resistance to eliminating it has long been connected to the idea of
white supremacy.

Voters casting ballots in Mississippi in 1946.Credit...Bettmann
Archive/Getty Images

Supported by

\protect\hyperlink{after-sponsor}{Continue reading the main story}

By Alexander Keyssar

Mr. Keyssar is a professor of history and social policy at Harvard and
the author of
``\href{https://www.hup.harvard.edu/catalog.php?isbn=9780674660151\#:~:text=After\%20tracing\%20the\%20Electoral\%20College's,showing\%20why\%20each\%20has\%20failed.}{Why
Do We Still Have the Electoral College}?''

\begin{itemize}
\item
  Aug. 3, 2020
\item
  \begin{itemize}
  \item
  \item
  \item
  \item
  \item
  \end{itemize}
\end{itemize}

\href{https://www.nytimes.com/es/2020/08/03/espanol/opinion/colegio-electoral-estados-unidos.html}{Leer
en español}

As our revived national conversation on race has made clear, the
legacies of slavery and white supremacy run wide and deep in American
society and political life. One such legacy --- which is particularly
noteworthy in a presidential election season --- has been the survival
and preservation of the Electoral College, an institution that has been
under fire for more than 200 years. Our complicated method of electing
presidents has been the target of recurrent reform attempts since the
early 19th century, and the politics of race and region have figured
prominently in their defeat.

It is, of course, no secret that slavery played a role in the original
design of our presidential election system --- although
\href{https://www.nytimes.com/2019/04/04/opinion/the-electoral-college-slavery-myth.html?action=click\&module=RelatedLinks\&pgtype=Article}{historians}
\href{https://www.nytimes.com/2019/04/06/opinion/electoral-college-slavery.html}{disagree}
about the centrality of that role. The notorious formula that gave
states representation in Congress for three-fifths of their slaves was
carried over into the allocation of electoral votes; the number of
electoral votes granted to each state was (and remains) equivalent to
that state's representation in both branches of Congress. This
constitutional design gave white Southerners disproportionate influence
in the choice of presidents, an edge that could and did affect the
outcome of elections.

Not surprisingly, the slave states strenuously opposed any changes to
the system that would diminish their advantage. In 1816, when a
resolution calling for a national popular vote was introduced in
Congress for the first time, it was derailed by the protestations of
Southern senators. The slaveholding states ``would lose the privilege
the Constitution now allows them, of votes upon three-fifths of their
population other than freemen,'' objected William Wyatt Bibb of Georgia
on the floor of the Senate. ``It would be deeply injurious to them.''

What is far less known, or recognized, is that long after the abolition
of slavery, Southern political leaders continued to resist any attempts
to replace the Electoral College with a national popular vote. (They
sometimes supported other reforms, like the proportional division of
each state's electoral votes, but those are different strands of a
multifaceted tale.) The reasoning behind this opposition was
straightforward, if disturbing. After Reconstruction, the white
``Redeemer'' governments that came to power in Southern states became
the political beneficiaries of what amounted to a ``five-fifths''
clause: African-Americans counted fully toward representation (and thus
electoral votes), but they were again disenfranchised --- despite the
formal protections outlined in the 15th Amendment, ratified in 1870,
which stated that the right to vote could not be denied ``on account of
race, color, or previous condition of servitude.'' White Southerners
consequently derived an even greater benefit from the Electoral College
than they had before the Civil War.

A national popular vote would have eliminated that benefit. As the
region's political leaders recognized, passage of a constitutional
amendment instituting a national popular vote would have spawned strong
legal and political pressures to enfranchise African-Americans. Even if
those pressures could be resisted, an Alabama campaign pamphlet noted in
1914, ``with the Negro half of our people not voting, our voice in the
national elections, which is now based upon total population, would then
be based solely on our voting population and, therefore reduced by
half.'' The political consequences of a national popular vote could
simply not be countenanced.

By the 1940s, many Southerners also came to believe that their
disproportionate weight in presidential elections, thanks to the
Electoral College, was a critical bulwark against mounting Northern
pressures to enlarge the civil and political rights of
African-Americans. In 1947 Charles Collins's ``Whither Solid South?,''
an influential states' rights and segregationist treatise, implored
Southerners to repel ``any attempt to do away with the College because
it alone can enable the Southern States to preserve their rights within
the Union.'' The book, which became must reading among the Dixiecrats
who bolted from the Democratic Party in 1948, was highly praised and
freely distributed by (among others) the Mississippi segregationist
James Eastland, who served in the Senate from 1943 until 1978.

Driven by such convictions, the white supremacist regimes of the South
stood as a roadblock in the path of a national popular vote from the
latter decades of the 19th century into the 1960s, when the Voting
Rights Act and other measures compelled the region to enfranchise
African-Americans. There was, of course, resistance to the idea of a
national vote elsewhere in the country, but it was the South's
well-known adamance --- and the fact that Southern states alone could
come close to blocking a constitutional amendment in Congress --- that
kept the idea on the outskirts of public debate for decades. Numerous
political leaders who personally favored a national popular vote, like
Senator Henry Cabot Lodge, Jr. of Massachusetts, a Republican, in the
1940s, concluded that such a reform had no realistic chance of success,
and they shifted their advocacy to less sweeping measures.

The politics of race and region also figured prominently in the stinging
defeat of a national popular vote amendment in the Senate in 1970 ---
the closest that the United States has come to transforming its
presidential election system since 1821. Popular and elite support for
the idea had mushroomed in the 1960s, leading in 1969 to the House of
Representatives voting overwhelmingly in favor of a constitutional
amendment that would have abolished the Electoral College. The proposal
then got bogged down in the Senate during a year when regional tensions
were high: two Southern nominees to the Supreme Court were rejected by
the Senate, and the Voting Rights Act was renewed over the vocal
opposition of Southern senators. Meanwhile, the national popular vote
amendment was stalled in the Judiciary Committee, which was chaired by
none other than Senator Eastland.

When the amendment resolution finally came to the floor of the Senate in
September 1970, thanks to the prodigious efforts of an Indiana senator,
Birch Bayh, it was greeted by a filibuster led by segregationists Sam
Ervin and Strom Thurmond (with an assist from the Nebraska Republican
Roman Hruska). Although things were changing in the South, its political
leaders remained steeped in the values and perspectives that had
informed their hostility to the civil rights movement and the Voting
Rights Act. ``The Electoral College,'' wrote Senator James Allen of
Alabama in 1969, ``is one of the South's few remaining political
safeguards. Let's keep it.''

The filibuster succeeded, dooming the proposal: attempts to invoke
cloture --- to end the debate and vote on the amendment itself --- fell
a few votes short of the two-thirds majority then needed to break a
filibuster. The regional lineups in the crucial cloture votes (there
were two) were starkly visible. More than 75 percent of Southern
senators voted against cloture; a similar proportion of senators from
outside the South voted favorably.

Southern political leaders, shaped by segregation and white supremacist
beliefs, thus kept the idea of a national popular vote off the table for
many decades and played a crucial role in blocking its passage through
Congress at a historical juncture when change actually seemed possible.
To be sure, electoral reform is almost always a complex, difficult
process, with diverse actors competing to defend their ideas and
interests. But had the politics of race been less salient, both in the
19th century and the 20th, the Electoral College would most likely have
been relegated long ago to the status of a historical curiosity. We
might want to keep that sobering fact in mind as we look ahead to an
election whose outcome is in question only because of the peculiar
manner in which we choose our presidents.

Alexander Keyssar
(\href{https://twitter.com/alexkeyssar?lang=en}{@AlexKeyssar}), a
professor of history and social policy at Harvard, is the author of
``\href{https://www.hup.harvard.edu/catalog.php?isbn=9780674660151\#:~:text=After\%20tracing\%20the\%20Electoral\%20College's,showing\%20why\%20each\%20has\%20failed.}{Why
Do We Still Have the Electoral College}?'' and
``\href{https://www.basicbooks.com/titles/alexander-keyssar/the-right-to-vote/9780465005024/}{The
Right to Vote: The Contested History of Democracy in the United
States}.''

\emph{The Times is committed to publishing}
\href{https://www.nytimes.com/2019/01/31/opinion/letters/letters-to-editor-new-york-times-women.html}{\emph{a
diversity of letters}} \emph{to the editor. We'd like to hear what you
think about this or any of our articles. Here are some}
\href{https://help.nytimes.com/hc/en-us/articles/115014925288-How-to-submit-a-letter-to-the-editor}{\emph{tips}}\emph{.
And here's our email:}
\href{mailto:letters@nytimes.com}{\emph{letters@nytimes.com}}\emph{.}

\emph{Follow The New York Times Opinion section on}
\href{https://www.facebook.com/nytopinion}{\emph{Facebook}}\emph{,}
\href{http://twitter.com/NYTOpinion}{\emph{Twitter (@NYTopinion)}}
\emph{and}
\href{https://www.instagram.com/nytopinion/}{\emph{Instagram}}\emph{.}

Advertisement

\protect\hyperlink{after-bottom}{Continue reading the main story}

\hypertarget{site-index}{%
\subsection{Site Index}\label{site-index}}

\hypertarget{site-information-navigation}{%
\subsection{Site Information
Navigation}\label{site-information-navigation}}

\begin{itemize}
\tightlist
\item
  \href{https://help.nytimes.com/hc/en-us/articles/115014792127-Copyright-notice}{©~2020~The
  New York Times Company}
\end{itemize}

\begin{itemize}
\tightlist
\item
  \href{https://www.nytco.com/}{NYTCo}
\item
  \href{https://help.nytimes.com/hc/en-us/articles/115015385887-Contact-Us}{Contact
  Us}
\item
  \href{https://www.nytco.com/careers/}{Work with us}
\item
  \href{https://nytmediakit.com/}{Advertise}
\item
  \href{http://www.tbrandstudio.com/}{T Brand Studio}
\item
  \href{https://www.nytimes.com/privacy/cookie-policy\#how-do-i-manage-trackers}{Your
  Ad Choices}
\item
  \href{https://www.nytimes.com/privacy}{Privacy}
\item
  \href{https://help.nytimes.com/hc/en-us/articles/115014893428-Terms-of-service}{Terms
  of Service}
\item
  \href{https://help.nytimes.com/hc/en-us/articles/115014893968-Terms-of-sale}{Terms
  of Sale}
\item
  \href{https://spiderbites.nytimes.com}{Site Map}
\item
  \href{https://help.nytimes.com/hc/en-us}{Help}
\item
  \href{https://www.nytimes.com/subscription?campaignId=37WXW}{Subscriptions}
\end{itemize}
