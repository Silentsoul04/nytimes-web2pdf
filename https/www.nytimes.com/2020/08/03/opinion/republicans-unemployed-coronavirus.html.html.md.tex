Sections

SEARCH

\protect\hyperlink{site-content}{Skip to
content}\protect\hyperlink{site-index}{Skip to site index}

\href{https://myaccount.nytimes.com/auth/login?response_type=cookie\&client_id=vi}{}

\href{https://www.nytimes.com/section/todayspaper}{Today's Paper}

\href{/section/opinion}{Opinion}\textbar{}The Unemployed Stare Into the
Abyss. Republicans Look Away.

\href{https://nyti.ms/30nmzFe}{https://nyti.ms/30nmzFe}

\begin{itemize}
\item
\item
\item
\item
\item
\item
\end{itemize}

Advertisement

\protect\hyperlink{after-top}{Continue reading the main story}

\href{/section/opinion}{Opinion}

Supported by

\protect\hyperlink{after-sponsor}{Continue reading the main story}

\hypertarget{the-unemployed-stare-into-the-abyss-republicans-look-away}{%
\section{The Unemployed Stare Into the Abyss. Republicans Look
Away.}\label{the-unemployed-stare-into-the-abyss-republicans-look-away}}

The cruelty and ignorance of Trump and his allies are creating another
gratuitous disaster.

\href{https://www.nytimes.com/by/paul-krugman}{\includegraphics{https://static01.nyt.com/images/2018/04/02/opinion/paul-krugman/paul-krugman-thumbLarge.png}}

By \href{https://www.nytimes.com/by/paul-krugman}{Paul Krugman}

Opinion Columnist

\begin{itemize}
\item
  Aug. 3, 2020
\item
  \begin{itemize}
  \item
  \item
  \item
  \item
  \item
  \item
  \end{itemize}
\end{itemize}

\includegraphics{https://static01.nyt.com/images/2020/08/03/opinion/03krugmanWe/03krugmanWe-articleLarge-v2.jpg?quality=75\&auto=webp\&disable=upscale}

In case you haven't noticed, the coronavirus is still very much with us.
Around a thousand Americans are dying from Covid-19 each day, 10 times
the rate in the
\href{https://ourworldindata.org/coronavirus-data-explorer?zoomToSelection=true\&deathsMetric=true\&interval=smoothed\&smoothing=7\&country=USA~EuropeanUnion\&pickerMetric=location\&pickerSort=asc}{European
Union}. Thanks to our failure to control the pandemic, we're still
suffering from Great Depression levels of
\href{https://www.nytimes.com/2020/08/06/business/economy/unemployment-claims.html}{unemployment};
a brief recovery driven by premature attempts to resume business as
usual appears to have
\href{https://www.calculatedriskblog.com/2020/08/forecasts-for-july-employment-report.html}{petered
out} as states pause or reverse their opening.

Yet enhanced unemployment benefits, a crucial lifeline for tens of
millions of Americans, have expired. And negotiations over how --- or
even whether --- to restore aid appear to be
\href{https://www.nytimes.com/2020/08/02/us/politics/coronavirus-jobless-aid.html?action=click\&module=Top\%20Stories\&pgtype=Homepage}{stalled}.

You sometimes see headlines describing this crisis as a result of
``congressional dysfunction.'' Such headlines reveal a severe case of
bothsidesism --- the almost pathological aversion of some in the media
to placing blame where it belongs.

For House Democrats passed a bill specifically designed to deal with
this mess \emph{two and a half months ago}. The Trump administration and
Senate Republicans had plenty of time to propose an alternative.
Instead, they didn't even focus on the issue until days before the
benefits ended. And even now they're refusing to offer anything that
might significantly alleviate workers' plight.

This is an astonishing failure of governance, right up there with the
mishandling of the pandemic itself. But what explains it?

Well, I'm of two minds. Was it ignorant malevolence, or malevolent
ignorance?

Let's talk first about the ignorance.

The Covid recession that began in February may have been the simplest,
most comprehensible business downturn in history. Much of the U.S.
economy was put on hold to contain a pandemic. Job losses were
concentrated in services that were either inessential or could be
postponed, and were highly likely to spread the coronavirus:
restaurants, air travel, dentists' visits.

The main goal of economic policy was to make this temporary lockdown
tolerable, sustaining the incomes of those unable to work.

Republicans, however, have shown no sign of understanding any of this.
The policy proposals being floated by White House aides and advisers are
almost surreal in their disconnect from reality. Cutting
\href{https://www.wsj.com/articles/how-trump-can-deliver-tax-relief-without-congress-11596396830}{payroll
taxes} on workers who can't work? Letting businesspeople deduct the full
cost of
\href{https://www.msnbc.com/rachel-maddow-show/gop-includes-three-martini-lunch-deduction-aid-package-n1235110}{three-martini
lunches} they can't eat?

They don't even seem to understand the mechanics of how unemployment
checks are paid out. They proposed
\href{https://www.cbsnews.com/news/unemployment-600-benefits-extension-senate-democrats-reject-white-house-short-term-extension/}{continuing
benefits} for a brief period while negotiations continue --- but this
literally can't be done, because the state offices that disburse
unemployment aid couldn't handle the necessary reprogramming.

Above all, Republicans seem obsessed with the idea that unemployment
benefits are making workers lazy and unwilling to accept jobs.

This would be a bizarre claim even if unemployment benefits really were
reducing the incentive to seek work. After all, there are more than
\href{https://www.dol.gov/ui/data.pdf}{30 million workers} receiving
benefits, but only \href{https://fred.stlouisfed.org/series/JTSJOL}{five
million} job openings. No matter how harshly you treat the unemployed,
they can't take jobs that don't exist.

It's almost a secondary concern to note that there's almost no evidence
that unemployment benefits are, in fact, discouraging workers from
taking jobs.
\href{https://twitter.com/ernietedeschi/status/1289919537538674689}{Multiple
studies} find no significant incentive effect.

And unemployment benefits didn't prevent the U.S. from adding seven
million jobs, most of them for low-wage workers --- that is, precisely
the workers often receiving more in unemployment than from their normal
jobs --- during the abortive spring recovery.

By the way, a
\href{https://fivethirtyeight.com/features/economists-think-congress-should-keep-paying-unemployed-workers-600-a-week-or-even-more/}{great
majority} of economists believe that unemployment benefits have helped
sustain the economy as a whole, by supporting consumer spending.

So the attack on unemployment aid is rooted in deep ignorance. But
there's also a strong element of malice.

Republicans have a long history of suggesting that the jobless are moral
failures --- that they'd rather sit home
\href{https://www.epi.org/blog/ugly-views-about-the-unemployed-by-congressional-republicans/}{watching
TV} than work. And the Trump years have been marked by a relentless
assault on programs that help the less fortunate, from Obamacare to food
stamps.

One indicator of G.O.P. disingenuousness is the sudden re-emergence of
``deficit hawks'' claiming that helping the unemployed will add too much
to the national debt. I use the scare quotes because as far as I can
tell not one of the politicians claiming that we can't afford to help
the unemployed raised any objections to Donald Trump's \$2 trillion tax
cut for corporations and the wealthy.

Nor was disdain for the unlucky the only reason the G.O.P. didn't want
to help Americans in need. The recent
\href{https://www.vanityfair.com/news/2020/07/how-jared-kushners-secret-testing-plan-went-poof-into-thin-air}{Vanity
Fair} report about why we don't have a national testing strategy fits
with a lot of evidence that Republicans spent months believing that
Covid-19 was a blue-state problem, not relevant to people they cared
about. By the time they realized that the pandemic was exploding in the
\href{https://covidtracking.com/data/charts/regional-cases}{Sun Belt},
it was too late to avoid disaster.

At this point, then, it's hard to see how we avoid another gratuitous
catastrophe. The fecklessness of the Trump administration and its allies
means that millions of Americans will soon be in dire financial straits.

\emph{The Times is committed to publishing}
\href{https://www.nytimes.com/2019/01/31/opinion/letters/letters-to-editor-new-york-times-women.html}{\emph{a
diversity of letters}} \emph{to the editor. We'd like to hear what you
think about this or any of our articles. Here are some}
\href{https://help.nytimes.com/hc/en-us/articles/115014925288-How-to-submit-a-letter-to-the-editor}{\emph{tips}}\emph{.
And here's our email:}
\href{mailto:letters@nytimes.com}{\emph{letters@nytimes.com}}\emph{.}

\emph{Follow The New York Times Opinion section on}
\href{https://www.facebook.com/nytopinion}{\emph{Facebook}}\emph{,}
\href{http://twitter.com/NYTOpinion}{\emph{Twitter (@NYTopinion)}}
\emph{and}
\href{https://www.instagram.com/nytopinion/}{\emph{Instagram}}\emph{.}

Advertisement

\protect\hyperlink{after-bottom}{Continue reading the main story}

\hypertarget{site-index}{%
\subsection{Site Index}\label{site-index}}

\hypertarget{site-information-navigation}{%
\subsection{Site Information
Navigation}\label{site-information-navigation}}

\begin{itemize}
\tightlist
\item
  \href{https://help.nytimes.com/hc/en-us/articles/115014792127-Copyright-notice}{©~2020~The
  New York Times Company}
\end{itemize}

\begin{itemize}
\tightlist
\item
  \href{https://www.nytco.com/}{NYTCo}
\item
  \href{https://help.nytimes.com/hc/en-us/articles/115015385887-Contact-Us}{Contact
  Us}
\item
  \href{https://www.nytco.com/careers/}{Work with us}
\item
  \href{https://nytmediakit.com/}{Advertise}
\item
  \href{http://www.tbrandstudio.com/}{T Brand Studio}
\item
  \href{https://www.nytimes.com/privacy/cookie-policy\#how-do-i-manage-trackers}{Your
  Ad Choices}
\item
  \href{https://www.nytimes.com/privacy}{Privacy}
\item
  \href{https://help.nytimes.com/hc/en-us/articles/115014893428-Terms-of-service}{Terms
  of Service}
\item
  \href{https://help.nytimes.com/hc/en-us/articles/115014893968-Terms-of-sale}{Terms
  of Sale}
\item
  \href{https://spiderbites.nytimes.com}{Site Map}
\item
  \href{https://help.nytimes.com/hc/en-us}{Help}
\item
  \href{https://www.nytimes.com/subscription?campaignId=37WXW}{Subscriptions}
\end{itemize}
