Sections

SEARCH

\protect\hyperlink{site-content}{Skip to
content}\protect\hyperlink{site-index}{Skip to site index}

\href{https://www.nytimes.com/section/sports/baseball}{Baseball}

\href{https://myaccount.nytimes.com/auth/login?response_type=cookie\&client_id=vi}{}

\href{https://www.nytimes.com/section/todayspaper}{Today's Paper}

\href{/section/sports/baseball}{Baseball}\textbar{}Aaron Judge Is Nearly
Unstoppable. He Thinks He Can Be Better.

\url{https://nyti.ms/3fqxGBx}

\begin{itemize}
\item
\item
\item
\item
\item
\end{itemize}

Advertisement

\protect\hyperlink{after-top}{Continue reading the main story}

Supported by

\protect\hyperlink{after-sponsor}{Continue reading the main story}

\hypertarget{aaron-judge-is-nearly-unstoppable-he-thinks-he-can-be-better}{%
\section{Aaron Judge Is Nearly Unstoppable. He Thinks He Can Be
Better.}\label{aaron-judge-is-nearly-unstoppable-he-thinks-he-can-be-better}}

Judge hit home runs in five consecutive games entering Monday night's
action, and he did it without one of his favorite tools: the video
replay room.

\includegraphics{https://static01.nyt.com/images/2020/08/03/sports/03yankees-1/merlin_175181169_127d39c2-665c-4eed-bb0c-1efbaaa25cf4-articleLarge.jpg?quality=75\&auto=webp\&disable=upscale}

\href{https://www.nytimes.com/by/james-wagner}{\includegraphics{https://static01.nyt.com/images/2018/06/13/multimedia/author-james-wagner/author-james-wagner-thumbLarge.jpg}}

By \href{https://www.nytimes.com/by/james-wagner}{James Wagner}

\begin{itemize}
\item
  Aug. 3, 2020
\item
  \begin{itemize}
  \item
  \item
  \item
  \item
  \item
  \end{itemize}
\end{itemize}

Six home runs over five games is more than enough proof that a batter is
locked in at the plate. But the Yankees star outfielder Aaron Judge,
perhaps guided by humility or a constant search for perfection, insisted
it was not.

After blasting two more home runs on Sunday, including a two-run shot in
a 9-7 win over the Boston Red Sox, Judge said he was still searching for
that final bit of comfort while hitting. Baseball, after all, is a sport
of daily failure.

``Locked in for me is if I'm going 5 for 5 every night,'' he said. ``I
still got out a couple times and chased a couple pitches. So there's
some times where I'm not really locked in.''

Those around him, though, felt otherwise.

``There's a reason he almost won the M.V.P., and he's in a groove right
now,'' said Yankees first baseman Luke Voit, referring to the award
Judge narrowly missed out on in 2017. ``I'm excited to see what he can
do in 60 games this year with how he's raking right now. He's a guy you
don't want to take your eyes off when he's hitting.''

Had the Major League Baseball season started as planned on March 26,
Judge would not have been on the field, still
\href{https://www.nytimes.com/2020/03/06/sports/baseball/aaron-judge-yankees.html}{recuperating
from an injury}. He fractured his rib during a play in the outfield in
September, which led to a partially collapsed lung. He fought through
some lingering discomfort in his side and shoulder the rest of the
season, the playoffs and throughout the off-season.

The fracture, however, was not identified until spring training, after
nearly a dozen tests. So while the start of the M.L.B. season was
delayed for four months because of the coronavirus pandemic, Judge used
the extra time to recover. When he was cleared by doctors just before
summer workouts began on July 4, it ``lit another fuse'' for Judge,
Manager Aaron Boone said.

Injuries have kept Judge off the field for parts of the previous two
seasons. He missed 45 games in 2018 after a pitch fractured his wrist
and was out for
\href{https://www.nytimes.com/2019/06/21/sports/aaron-judge-yankees.html}{54
games last year} with an oblique strain. He still hit 27 home runs with
an on-base-plus-slugging percentage over .900 in each of those seasons.

``He's really on a mission right now,'' Boone said. ``When he got that
clean bill of health right before summer camp started and started
ramping up, there's just been an intensity level and an energy level to
the work. He's just a great player that you can tell is feeling really
good.''

Entering Monday's 6-3 win over the Philadelphia Phillies, who had not
played since July 26 because of
\href{https://www.nytimes.com/2020/07/29/sports/baseball/yankees-schedule.html}{the
ripple effects of the Miami Marlins' coronavirus outbreak}, no one in
baseball had more home runs (six) or runs batted in (14) or was hitting
the ball harder (an
\href{https://baseballsavant.mlb.com/leaderboard/statcast?type=batter\&year=2020\&position=\&team=\&min=q\&sort=6\&sortDir=asc}{average
exit velocity} of 98.3 miles per hour) than Judge. He was on a pace to
smash 45 home runs this season --- which would amount to (a very
unrealistic) 122 over a normal 162-game season.

Judge accomplished all of this without one of his favorite hitting
tools: in-game video. Because of
\href{https://www.nytimes.com/2020/06/24/sports/baseball/mlb-coronavirus-rules.html}{M.L.B.'s
health and safety protocols} for this season, the replay review room at
each stadium is closed to players and coaches at all times to ensure
social distancing and to keep them isolated from other personnel. (After
the Houston Astros' cheating scandal, M.L.B. and the players' union
\href{https://www.nytimes.com/2020/02/24/sports/baseball/astros-cheating-scandal.html}{have
worked on new rules} governing these rooms.)

During games in past years, Judge said that he, like many other players,
would run to that room after an at-bat to check his swing or the pitch
he swung at or --- in his words --- slam his fist down and get mad at
himself. But now after he makes an out, Judge turns to his companions in
the dugout for their feedback. He said it might even be more helpful
than the video that has become so prevalent in modern baseball.

\includegraphics{https://static01.nyt.com/images/2020/08/03/sports/03yankees-2/merlin_175268655_aac6aa9e-008c-40f2-9979-55456a798ecd-articleLarge.jpg?quality=75\&auto=webp\&disable=upscale}

``This is kind of taking us back to the travel ball days,'' Judge said.

Case in point: Judge, 28, said he was riding home with his fellow
Yankees slugger Giancarlo Stanton, who is also off to a resounding start
after an
\href{https://www.nytimes.com/2019/09/18/sports/baseball/giancarlo-stanton-yankees.html}{injury-marred
2019 season}, after a recent game and mentioned how he was not hitting
some breaking balls properly. Stanton offered a small tip --- keep your
head down a click longer --- which Judge said proved fairly useful.

``We don't have the video like we usually do,'' Judge said. ``But now
it's just us, using your teammates' eyes and your own eyes, and just
talking some baseball.''

(Players and coaches can still watch video on M.L.B.-supplied tablets,
but not real-time footage from a live game since the content is loaded
only before or after games.)

Judge has a few factors working in his favor, too: After starting the
season against the defending champion Washington Nationals, the Yankees
have faced two opponents (the Baltimore Orioles and the Red Sox) with
poor pitching, and he is sandwiched in the lineup between other talented
hitters --- the 2019 All-Star infielders D.J. LeMahieu and Gleyber
Torres, and Stanton.

After possessing M.L.B.'s highest-scoring offense last season, the 8-1
Yankees are off to a similar start this year. Judge said he was simply
trying to do his part. He more than has: Five of his six home runs have
given the Yankees the lead. Entering Monday night's game, he had homered
in five straight games --- the first Yankee to do so since Alex
Rodriguez in 2007.

Judge did not extend that streak on Monday against the Phillies, but he
went 2 for 4 --- raising his season average to .314 --- in support of
ace Gerrit Cole, who allowed one run over six innings in his home debut
as a Yankee. Gio Urshela had a pivotal three-run homer in the sixth
inning.

``For me, right now, it's about not missing my pitch,'' Judge said after
Sunday's game. ``Pitchers are making really good pitches and hitting
their corners. But when there's times they leave one over the plate,
I've got to do some damage on it. Fortunately enough, I've been able to
do that.''

Advertisement

\protect\hyperlink{after-bottom}{Continue reading the main story}

\hypertarget{site-index}{%
\subsection{Site Index}\label{site-index}}

\hypertarget{site-information-navigation}{%
\subsection{Site Information
Navigation}\label{site-information-navigation}}

\begin{itemize}
\tightlist
\item
  \href{https://help.nytimes.com/hc/en-us/articles/115014792127-Copyright-notice}{©~2020~The
  New York Times Company}
\end{itemize}

\begin{itemize}
\tightlist
\item
  \href{https://www.nytco.com/}{NYTCo}
\item
  \href{https://help.nytimes.com/hc/en-us/articles/115015385887-Contact-Us}{Contact
  Us}
\item
  \href{https://www.nytco.com/careers/}{Work with us}
\item
  \href{https://nytmediakit.com/}{Advertise}
\item
  \href{http://www.tbrandstudio.com/}{T Brand Studio}
\item
  \href{https://www.nytimes.com/privacy/cookie-policy\#how-do-i-manage-trackers}{Your
  Ad Choices}
\item
  \href{https://www.nytimes.com/privacy}{Privacy}
\item
  \href{https://help.nytimes.com/hc/en-us/articles/115014893428-Terms-of-service}{Terms
  of Service}
\item
  \href{https://help.nytimes.com/hc/en-us/articles/115014893968-Terms-of-sale}{Terms
  of Sale}
\item
  \href{https://spiderbites.nytimes.com}{Site Map}
\item
  \href{https://help.nytimes.com/hc/en-us}{Help}
\item
  \href{https://www.nytimes.com/subscription?campaignId=37WXW}{Subscriptions}
\end{itemize}
