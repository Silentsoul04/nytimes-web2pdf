Sections

SEARCH

\protect\hyperlink{site-content}{Skip to
content}\protect\hyperlink{site-index}{Skip to site index}

\href{https://www.nytimes.com/section/sports/tennis}{Tennis}

\href{https://myaccount.nytimes.com/auth/login?response_type=cookie\&client_id=vi}{}

\href{https://www.nytimes.com/section/todayspaper}{Today's Paper}

\href{/section/sports/tennis}{Tennis}\textbar{}Automated Line Calls Will
Replace Human Judges at U.S. Open

\url{https://nyti.ms/33jnfNJ}

\begin{itemize}
\item
\item
\item
\item
\item
\end{itemize}

Advertisement

\protect\hyperlink{after-top}{Continue reading the main story}

Supported by

\protect\hyperlink{after-sponsor}{Continue reading the main story}

On Tennis

\hypertarget{automated-line-calls-will-replace-human-judges-at-us-open}{%
\section{Automated Line Calls Will Replace Human Judges at U.S.
Open}\label{automated-line-calls-will-replace-human-judges-at-us-open}}

On all but the two biggest show courts, line calls will be made by
Hawk-Eye Live to reduce the number of people on site during the
pandemic.

\includegraphics{https://static01.nyt.com/images/2020/08/03/sports/03tennis-hawkeye/merlin_126623084_4be80037-6147-4b2e-9e5b-e9a730a701cf-articleLarge.jpg?quality=75\&auto=webp\&disable=upscale}

\href{https://www.nytimes.com/by/christopher-clarey}{\includegraphics{https://static01.nyt.com/images/2018/09/10/multimedia/author-christopher-clarey/author-christopher-clarey-thumbLarge.png}}

By \href{https://www.nytimes.com/by/christopher-clarey}{Christopher
Clarey}

\begin{itemize}
\item
  Aug. 3, 2020
\item
  \begin{itemize}
  \item
  \item
  \item
  \item
  \item
  \end{itemize}
\end{itemize}

Three weeks of World TeamTennis at the Greenbrier resort in West
Virginia had come down to a single point on Sunday.

In the last match of the final, the New York Empire and the Chicago
Smash had a simultaneous championship point at 6-6 in the decisive
women's doubles tiebreaker.

Sloane Stephens of the Smash hit her first serve in play. Coco
Vandeweghe of the Empire took a bold forehand cut and her return flew
well out of Stephens's reach, landing deep near the baseline.

There was no call by a line judge, because there were no line judges on
the court.

Instead, the critical call was made electronically, and though Stephens
and the Smash asked to see a replay of the virtual ball mark, it only
confirmed the judgment of the machine.

The replay showed
\href{https://twitter.com/WorldTeamTennis/status/1290004012100714497}{Vandeweghe's
shot had landed on the back half of the baseline.} The Empire had a
21-20 victory and the celebration --- no model of social distancing with
group hugs galore --- could begin in earnest.

\href{https://www.nytimes.com/2020/07/11/sports/tennis/Bryan-brothers.html}{World
TeamTennis} was using
\href{https://www.nytimes.com/2018/03/01/sports/tennis/hawkeye-live-line-calling.html?searchResultPosition=1}{Hawk-Eye
Live}, an automated system that not only eliminates line judges but also
eliminates the now-familiar challenge setup in which players can ask for
human calls to be reviewed by an electronic system.

With Hawk-Eye Live, the electronic system makes all the calls, even if
there are some familiar touches like the recorded voices that shout
``out,'' ``fault'' or ``foot fault.''

When a line call is particularly close, the system automatically uses a
recorded voice that projects more urgency. As in GPS systems, different
voices (and languages) can be used and during World TeamTennis, both
male voices and female voices were used during matches.

``For us, actually having a human voice still call `out' rather than
using a beep or some other sound was an important part of making sure
the feel of the sport didn't change,'' said James Japhet, the managing
director of Hawk-Eye North America.

But there is no doubt that Hawk-Eye Live represents major change and
later this month it is set to make its Grand Slam tournament debut. The
United States Tennis Association plans to deploy it on all but its two
biggest show courts at
\href{https://www.nytimes.com/2020/06/15/sports/tennis/us-open.html}{the
United States Open}, scheduled
\href{https://www.nytimes.com/2020/06/16/sports/us-open-tennis-cuomo.html}{for
Aug. 31-Sept. 13}. The U.S. Open was the first Grand Slam event to use
electronic line calling for its challenge system in 2006. In 2018, it
became the first Grand Slam event to make that available on all its
courts. Now comes the next phase as Hawk-Eye goes from serving as
quality control and a broadcast tool to being the first and final word.

The system also will be used at the Western \& Southern Open, the
combined WTA and ATP event transplanted from the Cincinnati suburbs that
is scheduled for the week before the U.S. Open at the U.S.T.A. Billie
Jean King National Tennis Center in Queens.

``I'm happy to see the U.S. Open using Hawk-Eye Live,'' said Carlos
Silva, chief executive of World TeamTennis. ``Is the system perfect?
Probably not. Is it close to perfect? Yes. Is it more perfect than
humans? 100 percent yes.''

It also has the potential to put quite a few human line judges out of
work, which is partly why the sport as a whole has been slow to adopt
Hawk-Eye Live. There is also concern that it could make it more
difficult to develop quality chair umpires because line judging is the
typical pathway to the chair.

``I imagine I'm off a few Christmas card lists,'' Japhet said. ``We're
not in the business of trying to remove people from the sport. It just
happens to have been a byproduct of this particular advancement of the
technology. So I think there certainly have been questions asked on our
side and the sport's side as to whether this is the right thing to do.''

\includegraphics{https://static01.nyt.com/images/2020/08/03/sports/03tennis-hawkeye02/merlin_38446243_01b0f324-d6fd-4593-8a99-d1962b82264a-articleLarge.jpg?quality=75\&auto=webp\&disable=upscale}

World TeamTennis chose to forgo line judges and use Hawk-Eye Live for
the last three years. The men's tour has done the same since 2017 at its
Next Gen ATP Finals, an experimental event for the best players under
the age of 22. But what is driving the U.S. Open's decision above all is
the coronavirus pandemic and the need to maintain safety and social
distancing.

``Every functional area of the tournament has been asked to limit the
number of people who physically need to be on-site,'' said Stacey
Allaster, the U.S. Open tournament director.

That includes officials, and by using Hawk-Eye Live on 15 of the 17
match courts, the U.S. Open can drastically reduce the number of line
judges on site: from approximately 350 to well under 100. Only Arthur
Ashe Stadium and Louis Armstrong Stadium will still feature full,
officiating crews of nine line judges who work rotating one-hour shifts.
The other courts will have only a chair umpire, who will call the score
after Hawk-Eye Live makes the call and who will focus more on monitoring
player behavior and the pace of play. The umpires will not be allowed to
overrule the machines on line calls, only taking over if the system
breaks down during a point and fails to make a call. If the audio system
were to fail, a light attached to the umpire's chair would still
indicate when Hawk-Eye has determined a shot is out.

The system is not entirely glitch-free. During this World TeamTennis
season, Jessica Pegula of the Orlando Storm and Bernarda Pera of the
Washington Kastles were playing a tiebreaker in a women's singles match.
With Pera leading 2-1, she hit a ball that was not called out but that
Pegula and her teammates were convinced had landed wide.

They asked to see a replay, and it suspiciously said the ball had landed
well within the court.

``We were like, this obviously isn't right,'' Pegula said. ``Hawk-Eye
clearly messed up. If you saw the ball land, that's not where the mark
was at all. We switched sides and were arguing with them and the umpire
got a call from whoever works the Hawk-Eye and said, `Actually you are
correct, Hawk-Eye was wrong. The ball was out.'''

She continued: ``If we wouldn't have fought about it, it probably
wouldn't have happened because the umpire just goes with what Hawk-Eye
says. So there have been some discrepancies here.''

Japhet said Hawk-Eye officials monitoring the system also have access to
a broadcast feed as an additional tool for such rare occasions. But he
said the automated system had been tested and shown to be accurate
within two millimeters.

Donald Young, a veteran American who first played in World TeamTennis in
2016, remains a convert.

``Obviously with the Covid situation, it's particularly useful, but
apart from that, it's just great,'' he said. ``The ball is coming fast,
so you can see it sometimes faster with Hawk-Eye than with a lot of
eyes. Sometimes it can be a little off. A couple calls have been inside
the box, and the guys had to correct it, but it's definitely gotten a
lot better over the years for sure. I think it's more accurate now than
ever.''

The ATP Tour, which until now had only authorized the use of Hawk-Eye
Live at the Next-Gen Finals, has temporarily approved the system's use
at all ATP events because of the pandemic. The women's tour has for now
only approved its use at the Western \& Southern Open, which will be the
first WTA event to use the system.

Japhet said he expects a significant increase in Hawk-Eye Live use over
the next two years in part because of the pandemic and the system's
precision but also because of economics. Though operating the system is
expensive with its 18 cameras, six of them used by a review official to
monitor foot faults, it is also costly to house, feed, transport and pay
daily wages to hundreds of line judges.

``I think the numbers do stack up for tournaments,'' Japhet said. ``They
have a net savings in using it.''

Technology is ever more pervasive in professional sports. But Pegula, a
26-year-old American, hopes line judges do not go the way of net-cord
judges, who were gradually replaced by sensors mounted on the net in the
1990s.

``It's a fun part of our sport, and obviously adding the challenges in
to kind of question them makes it exciting and more entertaining for
fans,'' she said. ``I don't know if I would want to eliminate linesmen
forever. It's part of tennis, part of its culture. It's more interactive
that way.''

But Silva believes more technology and less human error are inevitable.

``I think that ship sailed a long time ago in the world we live in,'' he
said. ``We're all living on iPods and iPhones and asking Google to be
our memories. I think it's long overdue to have the lines get called
automatically, and I think there are a bunch of new technologies around,
not just the cameras and sensors doing it now. You might even see active
paint and things like that for the lines, which might make it even more
accurate than what we have now.''

Advertisement

\protect\hyperlink{after-bottom}{Continue reading the main story}

\hypertarget{site-index}{%
\subsection{Site Index}\label{site-index}}

\hypertarget{site-information-navigation}{%
\subsection{Site Information
Navigation}\label{site-information-navigation}}

\begin{itemize}
\tightlist
\item
  \href{https://help.nytimes.com/hc/en-us/articles/115014792127-Copyright-notice}{©~2020~The
  New York Times Company}
\end{itemize}

\begin{itemize}
\tightlist
\item
  \href{https://www.nytco.com/}{NYTCo}
\item
  \href{https://help.nytimes.com/hc/en-us/articles/115015385887-Contact-Us}{Contact
  Us}
\item
  \href{https://www.nytco.com/careers/}{Work with us}
\item
  \href{https://nytmediakit.com/}{Advertise}
\item
  \href{http://www.tbrandstudio.com/}{T Brand Studio}
\item
  \href{https://www.nytimes.com/privacy/cookie-policy\#how-do-i-manage-trackers}{Your
  Ad Choices}
\item
  \href{https://www.nytimes.com/privacy}{Privacy}
\item
  \href{https://help.nytimes.com/hc/en-us/articles/115014893428-Terms-of-service}{Terms
  of Service}
\item
  \href{https://help.nytimes.com/hc/en-us/articles/115014893968-Terms-of-sale}{Terms
  of Sale}
\item
  \href{https://spiderbites.nytimes.com}{Site Map}
\item
  \href{https://help.nytimes.com/hc/en-us}{Help}
\item
  \href{https://www.nytimes.com/subscription?campaignId=37WXW}{Subscriptions}
\end{itemize}
