Sections

SEARCH

\protect\hyperlink{site-content}{Skip to
content}\protect\hyperlink{site-index}{Skip to site index}

\href{https://www.nytimes.com/section/technology}{Technology}

\href{https://myaccount.nytimes.com/auth/login?response_type=cookie\&client_id=vi}{}

\href{https://www.nytimes.com/section/todayspaper}{Today's Paper}

\href{/section/technology}{Technology}\textbar{}The Strange Saga of
TikTok

\href{https://nyti.ms/2XEBGZp}{https://nyti.ms/2XEBGZp}

\begin{itemize}
\item
\item
\item
\item
\item
\end{itemize}

Advertisement

\protect\hyperlink{after-top}{Continue reading the main story}

Supported by

\protect\hyperlink{after-sponsor}{Continue reading the main story}

on tech

\hypertarget{the-strange-saga-of-tiktok}{%
\section{The Strange Saga of TikTok}\label{the-strange-saga-of-tiktok}}

The chief executive of a big tech company negotiated directly with the
U.S. president over an app. Say what?

\includegraphics{https://static01.nyt.com/images/2020/08/03/business/03ontech-videostill/03ontech-videostill-threeByTwoMediumAt2X-v2.png}

\href{https://www.nytimes.com/by/shira-ovide}{\includegraphics{https://static01.nyt.com/images/2020/03/18/reader-center/author-shira-ovide/author-shira-ovide-thumbLarge-v2.png}}

By \href{https://www.nytimes.com/by/shira-ovide}{Shira Ovide}

\begin{itemize}
\item
  Aug. 3, 2020
\item
  \begin{itemize}
  \item
  \item
  \item
  \item
  \item
  \end{itemize}
\end{itemize}

\emph{This article is part of the On Tech newsletter. You can}
\href{https://www.nytimes.com/newsletters/signup/OT}{\emph{sign up
here}} \emph{to receive it weekdays.}

What's happening with TikTok is one of the strangest things I've seen.

Let me catch you up. One of the world's hottest apps has become a
\href{https://www.nytimes.com/2020/07/27/technology/tiktok-data-privacy.html}{political
hot potato} because it's owned by the Chinese internet giant ByteDance
at a time when relations between the United States and China are
\href{https://www.nytimes.com/2020/07/22/world/asia/us-china-cold-war.html}{at
a low point}.

U.S. government officials say they are worried that the Chinese
government might force TikTok to hand over information it collects about
Americans and use the app to spread a Chinese-friendly view of the
world.

TikTok has
\href{https://www.nytimes.com/2020/08/03/technology/tiktok-trump-sale-microsoft.html}{tried
and failed to calm those fears}.

That brings us to this weekend's strange scene: The chief executive of
Microsoft negotiated directly with the president of the United States
\href{https://www.nytimes.com/2020/08/01/technology/tiktok-trump-microsoft-bytedance-china-ban.html}{over
the purchase of an app} from China.

Here are some thoughts about what's going on:

\textbf{U.S. officials are getting the outcome they want. Probably.} My
New York Times colleagues wrote that there are people inside the White
House who
\href{https://www.nytimes.com/2020/08/02/business/economy/trump-tiktok-china-national-security.html}{want
TikTok banned} rather than sold to an American company. President Trump
said as much himself on Friday. It seemed that a phone call between him
and Satya Nadella, Microsoft's chief executive, put a TikTok takeover by
Microsoft back on the front burner.

Microsoft
\href{https://blogs.microsoft.com/blog/2020/08/02/microsoft-to-continue-discussions-on-potential-tiktok-purchase-in-the-united-states/}{said}
on Sunday that it's negotiating to buy the TikTok app, but only in the
United States, Canada, Australia and New Zealand.

While some Trump officials might have preferred to have the app banned
outright, having it owned by a U.S. company would still be a win.

\textbf{Huh? Microsoft?} Microsoft is both an odd and perfect fit. A new
owner of TikTok needs to be palatable to the Trump administration, and
Microsoft has mostly stayed in the government's good graces. A buyer
also needs to be rich enough to purchase TikTok but not so powerful ---
ahem, Facebook or Google --- that buying the app might tip it into
monopoly territory.

And Microsoft has the technical expertise to untangle TikTok from
ByteDance and make sure that information about Americans stays in U.S.
computer systems.

\textbf{This is not a done deal.} Microsoft has about six weeks to
haggle over price with ByteDance, figure out how to safeguard the
information of TikTok users, and keep U.S. government officials on
board.

Some people who make a living on TikTok are
\href{https://www.nytimes.com/2020/08/02/style/tiktok-ban-threat-trump.html}{freaking
out} about the app's fate, my colleague Taylor Lorenz reported. There is
a cloud of uncertainty about all of this.

The Times's Karen Weise
\href{https://www.nytimes.com/2020/08/03/technology/tiktok-microsoft-tweens.html}{wrote}
that Microsoft has a recent record of buying businesses and not
interfering in them too much --- maybe a sliver of good news for TikTok
fans worried about the app becoming boring.

\textbf{Is every Chinese technology a no-go zone?} I wonder what will
happen to other Chinese technology companies in the United States.
Secretary of State Mike Pompeo hinted in a
\href{https://www.state.gov/secretary-michael-r-pompeo-with-maria-bartiromo-of-fox-news-sunday-morning-futures-2/}{weekend
interview} that the administration was looking at other Chinese software
companies that he said fed data to China's government.

He didn't mention other kinds of technology from Chinese companies that
already operate in the United States --- Lenovo, for example, is one of
the country's biggest sellers of laptops, and it owns the mobile phone
maker Motorola. U.S. officials have
\href{https://www.nytimes.com/2020/07/23/us/politics/dji-drones-security-vulnerability.html}{expressed
concerns previousl}y about DJI, which makes the popular Mavic drones.

If tensions between the United States and China continue to escalate,
all technology companies based in China --- maybe Chinese companies in
any industry, really --- may find it difficult to operate in the United
States.

\emph{If you don't already get this newsletter in your inbox,}
\href{https://www.nytimes.com/newsletters/signup/OT}{\emph{please sign
up here}}\emph{.}

\begin{center}\rule{0.5\linewidth}{\linethickness}\end{center}

\hypertarget{tip-of-the-week}{%
\subsubsection{Tip of the Week}\label{tip-of-the-week}}

\hypertarget{how-to-request-a-record-of-your-data-from-the-big-four}{%
\subsection{How to request a record of your data from the Big
Four}\label{how-to-request-a-record-of-your-data-from-the-big-four}}

\href{https://www.nytimes.com/by/brian-x-chen}{\emph{Brian X.
Chen}}\emph{, our personal tech columnist, tells us how you can obtain a
record of your data from the tech giants.}

Last week's
\href{https://www.nytimes.com/2020/07/29/technology/big-tech-hearing-apple-amazon-facebook-google.html}{antitrust
hearing} made it clear that Google, Apple, Facebook and Amazon touch
every aspect of our digital lives, including our messaging apps, virtual
assistants and hardware. That may make you curious what data each of
those companies has collected about you.

Here's how you can find out. These instructions are to be followed from
a laptop or desktop computer, not a mobile app:

\textbf{Facebook:}

\begin{itemize}
\tightlist
\item
  On Facebook.com, click the arrow pointing downward in the top-right
  corner.
\end{itemize}

\begin{itemize}
\tightlist
\item
  Click Settings \& Privacy \textgreater{} Settings. In the left column,
  click Your Facebook Information.
\end{itemize}

\begin{itemize}
\tightlist
\item
  Here, follow the steps to request a copy of your Facebook data.
\end{itemize}

\textbf{Google:}

\begin{itemize}
\tightlist
\item
  Visit Google Takeout at takeout.google.com
\end{itemize}

\begin{itemize}
\tightlist
\item
  Here, select the categories for the data you would like to download.
\end{itemize}

\textbf{Apple:}

\begin{itemize}
\tightlist
\item
  Visit privacy.apple.com and log in with your Apple ID credentials.
\end{itemize}

\begin{itemize}
\tightlist
\item
  Click Request a Copy of Your Data, to access the data portal.
\end{itemize}

\textbf{Amazon:}

\begin{itemize}
\tightlist
\item
  Visit the
  \href{https://www.amazon.com/gp/privacycentral/dsar/preview.html}{Request
  My Data portal}. You'll need to log into your Amazon account.
\end{itemize}

\begin{itemize}
\tightlist
\item
  From the drop-down menu, click Request All Your Data, and submit the
  request.
\end{itemize}

In 2018, I downloaded copies of my information from each of the Big
Four, and I was most disturbed by the
\href{https://www.nytimes.com/2018/04/11/technology/personaltech/i-downloaded-the-information-that-facebook-has-on-me-yikes.html}{incredible
amount of data} that Facebook was hoarding about me, including
information on my friends and exes. The Facebook data also revealed that
hundreds of advertisers, many that I had never heard of, had my contact
information.

You can guess what I did next: I deleted my Facebook account. I haven't
regretted it.

\begin{center}\rule{0.5\linewidth}{\linethickness}\end{center}

\hypertarget{before-we-go-}{%
\subsection{Before we go \ldots{}}\label{before-we-go-}}

\begin{itemize}
\item
  \textbf{The path of Twitter's alleged teenage hacker:} My New York
  Times colleagues
  \href{https://www.nytimes.com/2020/08/02/technology/florida-teenager-twitter-hack.html}{traced}
  the life of Graham Ivan Clark, the 17-year-old charged with
  orchestrating a hack of Twitter last month that resulted in the
  takeover of accounts of some of the world's most famous people. My
  colleagues write that Clark loved the Minecraft video game as a kid,
  becoming known as a scammer who cheated people out of their money.
\item
  \textbf{Living online without getting swept into perpetual
  surveillance:} One lesson from my colleague Kashmir Hill is that a
  lifetime of our online photos are fodder for searchable databases that
  can be used to identify us by our faces. Now Kash writes about a team
  of computer engineers who say they
  \href{https://www.nytimes.com/2020/08/03/technology/fawkes-tool-protects-photos-from-facial-recognition.html}{found
  a way to disguise digital photos} enough to confuse facial recognition
  systems. One problem: It might not work.
\item
  \textbf{Nostalgia for those ridiculous cellphone ringtones:} People of
  a certain age might remember how important it was to find just the
  right tune to signal that your boyfriend was calling your flip phone.
  The tech publication OneZero
  \href{https://onezero.medium.com/how-the-custom-ringtone-industry-paved-the-way-for-the-app-store-and-then-vanished-11f0d2a1e53b}{explains}
  how the 2000s ringtone industry made today's smartphone app stores and
  music streaming services possible.
\end{itemize}

\hypertarget{hugs-to-this}{%
\subsubsection{Hugs to this}\label{hugs-to-this}}

Awwww. A
\href{https://www.tiktok.com/@wags_and_whiskers/video/6848325208785341701}{tiny
kitten} enjoying the heck out of a snack (on TikTok).

\begin{center}\rule{0.5\linewidth}{\linethickness}\end{center}

\emph{We want to hear from you. Tell us what you think of this
newsletter and what else you'd like us to explore. You can reach us at}
\href{mailto:ontech@nytimes.com?subject=On\%20Tech\%20Feedback}{\emph{ontech@nytimes.com.}}
**

\emph{If you don't already get this newsletter in your inbox,}
\href{https://www.nytimes.com/newsletters/signup/OT}{\emph{please sign
up here}}\emph{.}

Advertisement

\protect\hyperlink{after-bottom}{Continue reading the main story}

\hypertarget{site-index}{%
\subsection{Site Index}\label{site-index}}

\hypertarget{site-information-navigation}{%
\subsection{Site Information
Navigation}\label{site-information-navigation}}

\begin{itemize}
\tightlist
\item
  \href{https://help.nytimes.com/hc/en-us/articles/115014792127-Copyright-notice}{©~2020~The
  New York Times Company}
\end{itemize}

\begin{itemize}
\tightlist
\item
  \href{https://www.nytco.com/}{NYTCo}
\item
  \href{https://help.nytimes.com/hc/en-us/articles/115015385887-Contact-Us}{Contact
  Us}
\item
  \href{https://www.nytco.com/careers/}{Work with us}
\item
  \href{https://nytmediakit.com/}{Advertise}
\item
  \href{http://www.tbrandstudio.com/}{T Brand Studio}
\item
  \href{https://www.nytimes.com/privacy/cookie-policy\#how-do-i-manage-trackers}{Your
  Ad Choices}
\item
  \href{https://www.nytimes.com/privacy}{Privacy}
\item
  \href{https://help.nytimes.com/hc/en-us/articles/115014893428-Terms-of-service}{Terms
  of Service}
\item
  \href{https://help.nytimes.com/hc/en-us/articles/115014893968-Terms-of-sale}{Terms
  of Sale}
\item
  \href{https://spiderbites.nytimes.com}{Site Map}
\item
  \href{https://help.nytimes.com/hc/en-us}{Help}
\item
  \href{https://www.nytimes.com/subscription?campaignId=37WXW}{Subscriptions}
\end{itemize}
