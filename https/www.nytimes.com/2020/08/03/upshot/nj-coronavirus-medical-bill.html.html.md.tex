Sections

SEARCH

\protect\hyperlink{site-content}{Skip to
content}\protect\hyperlink{site-index}{Skip to site index}

\href{https://myaccount.nytimes.com/auth/login?response_type=cookie\&client_id=vi}{}

\href{https://www.nytimes.com/section/todayspaper}{Today's Paper}

\href{/section/upshot}{The Upshot}\textbar{}A Hospital Forgot to Bill
Her Coronavirus Test. It Cost Her \$1,980.

\url{https://nyti.ms/3i7ADZL}

\begin{itemize}
\item
\item
\item
\item
\item
\item
\end{itemize}

\href{https://www.nytimes.com/news-event/coronavirus?action=click\&pgtype=Article\&state=default\&region=TOP_BANNER\&context=storylines_menu}{The
Coronavirus Outbreak}

\begin{itemize}
\tightlist
\item
  live\href{https://www.nytimes.com/2020/08/04/world/coronavirus-cases.html?action=click\&pgtype=Article\&state=default\&region=TOP_BANNER\&context=storylines_menu}{Latest
  Updates}
\item
  \href{https://www.nytimes.com/interactive/2020/us/coronavirus-us-cases.html?action=click\&pgtype=Article\&state=default\&region=TOP_BANNER\&context=storylines_menu}{Maps
  and Cases}
\item
  \href{https://www.nytimes.com/interactive/2020/science/coronavirus-vaccine-tracker.html?action=click\&pgtype=Article\&state=default\&region=TOP_BANNER\&context=storylines_menu}{Vaccine
  Tracker}
\item
  \href{https://www.nytimes.com/2020/08/02/us/covid-college-reopening.html?action=click\&pgtype=Article\&state=default\&region=TOP_BANNER\&context=storylines_menu}{College
  Reopening}
\item
  \href{https://www.nytimes.com/live/2020/08/04/business/stock-market-today-coronavirus?action=click\&pgtype=Article\&state=default\&region=TOP_BANNER\&context=storylines_menu}{Economy}
\end{itemize}

Advertisement

\protect\hyperlink{after-top}{Continue reading the main story}

Upshot

Supported by

\protect\hyperlink{after-sponsor}{Continue reading the main story}

\hypertarget{a-hospital-forgot-to-bill-her-coronavirus-test-it-cost-her-1980}{%
\section{A Hospital Forgot to Bill Her Coronavirus Test. It Cost Her
\$1,980.}\label{a-hospital-forgot-to-bill-her-coronavirus-test-it-cost-her-1980}}

Send us your medical bills. We'll use them to investigate hospital and
doctor billing practices.

\includegraphics{https://static01.nyt.com/images/2020/08/03/business/03up-virus-bills/03up-virus-bills-articleLarge-v2.jpg?quality=75\&auto=webp\&disable=upscale}

By \href{https://www.nytimes.com/by/sarah-kliff}{Sarah Kliff}

\begin{itemize}
\item
  Aug. 3, 2020
\item
  \begin{itemize}
  \item
  \item
  \item
  \item
  \item
  \item
  \end{itemize}
\end{itemize}

\emph{The New York Times is investigating the costs associated with
testing and treatment for the coronavirus and how the pandemic is
changing health care in America. You can read more about the project
and}
\href{https://www.nytimes.com/2020/08/03/reader-center/coronavirus-medical-bills.html}{\emph{submit
your medical bills here}}\emph{.}

When Debbie Krebs got the bill for a March emergency room visit, she
immediately noticed something was missing: her coronavirus test.

Ms. Krebs, a lawyer who focuses on insurance issues, had gone to the
Valley Hospital in Ridgewood, N.J., with lung pain and a cough. A doctor
ran tests and scans to rule out other diseases before swabbing her nose.
A week later, the medical laboratory called, telling her it was
negative.

Ms. Krebs had a clear memory of the experience, particularly the doctor
saying the coronavirus test would make her feel as if she had to sneeze.
She wondered whether the doctor could have lied about performing the
test, or if her swab could have gone missing. (But if so, why had the
laboratory called her with results?)

The absence of the coronavirus test made a big price difference.
Insurers, Ms. Krebs had heard, were not charging patients for visits
meant to diagnose coronavirus. Without the test, Ms. Krebs didn't
qualify for that protection and owed \$1,980. She called the hospital to
explain the situation but immediately ran into roadblocks.

``When I called the hospital, they said, `You did not get a coronavirus
test,''' she said. ``I told them I absolutely did.''

Across the country, Americans like Ms. Krebs are receiving surprise
bills for care connected with coronavirus. Tests can cost between
\href{https://www.nytimes.com/2020/06/29/upshot/coronavirus-tests-unpredictable-prices.html}{\$199
and \$6,408} at the same location. A coming wave of treatment bills
could be hundreds of multiples higher, especially for those who receive
intensive care or have
\href{https://www.nytimes.com/2020/07/01/health/coronavirus-recovery-survivors.html}{symptoms
that linger for months}. Services that patients expect to be covered
\href{https://khn.org/news/bill-of-the-month-covid19-tests-are-free-except-when-theyre-not/}{often
aren't}.

This patchwork of medical billing is one reason we're starting something
new today: soliciting your medical bills. We're asking you to
\href{https://www.nytimes.com/2020/08/03/reader-center/coronavirus-medical-bills.html}{send
us copies} of your bills for coronavirus testing and treatment, so we
can understand what costs look like across the country. We want to know
how patients are managing their medical bills in the midst of a
pandemic. This is part of our larger effort to understand how the
pandemic is reshaping American health care.

American medical billing is unlike that of any other developed country.
The government does not regulate health care prices, but instead lets
insurers and hospitals negotiate fees. Those deliberations happen in
secret, and patients often do not learn the cost of their care until a
bill shows up in the mail.

Sometimes, insurers give reporters a peek at their data. That's how I
learned that a laboratory in Texas had charged
\href{https://www.nytimes.com/2020/06/16/upshot/coronavirus-test-cost-varies-widely.html}{\$2,315}
for individual coronavirus tests. But more often, they keep that
information confidential, which is why we need readers' bills and
explanation-of-benefit documents for any care related to coronavirus.

\hypertarget{latest-updates-global-coronavirus-outbreak}{%
\section{\texorpdfstring{\href{https://www.nytimes.com/2020/08/04/world/coronavirus-cases.html?action=click\&pgtype=Article\&state=default\&region=MAIN_CONTENT_1\&context=storylines_live_updates}{Latest
Updates: Global Coronavirus
Outbreak}}{Latest Updates: Global Coronavirus Outbreak}}\label{latest-updates-global-coronavirus-outbreak}}

Updated 2020-08-04T19:28:21.450Z

\begin{itemize}
\tightlist
\item
  \href{https://www.nytimes.com/2020/08/04/world/coronavirus-cases.html?action=click\&pgtype=Article\&state=default\&region=MAIN_CONTENT_1\&context=storylines_live_updates\#link-4825b93}{Public
  and private schools in Maryland and elsewhere are divided over
  in-person instruction.}
\item
  \href{https://www.nytimes.com/2020/08/04/world/coronavirus-cases.html?action=click\&pgtype=Article\&state=default\&region=MAIN_CONTENT_1\&context=storylines_live_updates\#link-4d1eafa8}{N.Y.C.'s
  health commissioner resigns after clashing with the mayor over the
  virus.}
\item
  \href{https://www.nytimes.com/2020/08/04/world/coronavirus-cases.html?action=click\&pgtype=Article\&state=default\&region=MAIN_CONTENT_1\&context=storylines_live_updates\#link-6b644638}{`Long
  days, long nights': Washington prepares for a prolonged fight over
  virus relief.}
\end{itemize}

\href{https://www.nytimes.com/2020/08/04/world/coronavirus-cases.html?action=click\&pgtype=Article\&state=default\&region=MAIN_CONTENT_1\&context=storylines_live_updates}{See
more updates}

More live coverage:
\href{https://www.nytimes.com/live/2020/08/04/business/stock-market-today-coronavirus?action=click\&pgtype=Article\&state=default\&region=MAIN_CONTENT_1\&context=storylines_live_updates}{Markets}

Readers' bills have already shown that surprise medical bills for
coronavirus have been in the United States nearly as long as the disease
itself.

In late February, an American man and his 3-year-old daughter were hit
with medical bills totaling thousands of dollars for care received
during a government-mandated quarantine. This was only weeks after
Washington State announced the country's first known case.

``I assumed it was all being paid for,'' Frank Wucinski, the patient,
said at the time. ``We didn't have a choice. When the bills showed up,
it was just a pit in my stomach, like, `How do I pay for this?'''

The federal government has since resolved to give Americans special
protections against outlandish medical bills. Congress enacted new rules
to make the tests a rare oasis within the American health care system
--- the price had to be public; and co-payments, deductibles or other
charges weren't allowed.

Or at least, Congress tried to. The experiences of patients who had or
suspected they might have Covid-19 show how hard it is to write
different billing rules for a tiny sliver of the country's \$3 trillion
in health spending. Numerous doctor's offices and hospitals do not post
the cash prices for their coronavirus tests, despite the federal
requirement to do so. Some patients have encountered unwarranted
co-payments as doctors and hospitals have stuck to their regular billing
habits. Others have failed to qualify for the protections because they
\href{https://khn.org/news/bill-of-the-month-covid19-tests-are-free-except-when-theyre-not/}{did
not receive a coronavirus test} as part of their care ---~or, in the
case of Ms. Krebs, had it left off the bill.

Aside from mandating that Covid-19 tests cost the patient nothing, there
are no new rules to protect insured Americans from coronavirus treatment
bills. Health policy experts worry that even those with good insurance
could end up facing high costs. One outcome they envision: A patient
goes to an in-network hospital for coronavirus treatment, but that
hospital is overwhelmed and has no beds left. The patient is transferred
to an out-of-network hospital, and gets significant bills as a result.

``Our system is so complicated,'' said Karen Pollitz, a senior fellow at
the Kaiser Family Foundation. ``If things aren't exactly right or
weren't coded correctly, you get thrown into the blizzard.''

The protections that do exist are based on the receipt of something that
can be in short supply: a coronavirus test. If doctors can't obtain a
test and turn to other diagnostic methods --- testing for other
diseases, for example --- the patient will have to cover the visit's
cost.

The Trump administration has also set aside an undisclosed sum to pay
for uninsured Americans' testing and treatment, a program that has
become increasingly important as millions
\href{https://www.nytimes.com/2020/07/13/us/politics/coronavirus-health-insurance-trump.html}{have
lost coverage} in the economic downturn. So far, that fund has paid out
\href{https://data.cdc.gov/Administrative/Claims-Reimbursement-to-Health-Care-Providers-and-/rksx-33p3}{\$348
million} to providers, but it is unknown how much money remains or what
happens when it runs out.

Billing challenges have persisted, despite these new rules and programs.
Many stem from the decision by legislators to condition aid on receipt
of a test.

Dr. Kao-Ping Chua, a pediatrician in Michigan,
\href{https://www.healthaffairs.org/do/10.1377/hblog20200413.783118/full/}{started
running into problems} in March when he had patients with
coronavirus-like symptoms seeking tests. His health system, like many
others, required patients to undergo testing for other conditions before
coronavirus.

``I had to tell my patients that, if the test I run first comes back
positive and says you have the common cold, you'll have to pay for it,''
he said. ``But if you test negative, that allows you to get the Covid
test, and that waives your cost sharing.''

\includegraphics{https://static01.nyt.com/images/2020/08/03/business/03up-virus-bills2/merlin_175087872_74a9e78d-f95d-4282-869e-bf4e66a083e4-articleLarge.jpg?quality=75\&auto=webp\&disable=upscale}

Luciano Aita, 35, sought treatment in early July at St. Mary's Medical
Center in San Francisco after his ``chest started closing up'' and he
felt as if he couldn't breathe.

\href{https://www.nytimes.com/news-event/coronavirus?action=click\&pgtype=Article\&state=default\&region=MAIN_CONTENT_3\&context=storylines_faq}{}

\hypertarget{the-coronavirus-outbreak-}{%
\subsubsection{The Coronavirus Outbreak
›}\label{the-coronavirus-outbreak-}}

\hypertarget{frequently-asked-questions}{%
\paragraph{Frequently Asked
Questions}\label{frequently-asked-questions}}

Updated August 4, 2020

\begin{itemize}
\item ~
  \hypertarget{i-have-antibodies-am-i-now-immune}{%
  \paragraph{I have antibodies. Am I now
  immune?}\label{i-have-antibodies-am-i-now-immune}}

  \begin{itemize}
  \tightlist
  \item
    As of right
    now,\href{https://www.nytimes.com/2020/07/22/health/covid-antibodies-herd-immunity.html?action=click\&pgtype=Article\&state=default\&region=MAIN_CONTENT_3\&context=storylines_faq}{that
    seems likely, for at least several months.} There have been
    frightening accounts of people suffering what seems to be a second
    bout of Covid-19. But experts say these patients may have a
    drawn-out course of infection, with the virus taking a slow toll
    weeks to months after initial exposure. People infected with the
    coronavirus typically
    \href{https://www.nature.com/articles/s41586-020-2456-9}{produce}
    immune molecules called antibodies, which are
    \href{https://www.nytimes.com/2020/05/07/health/coronavirus-antibody-prevalence.html?action=click\&pgtype=Article\&state=default\&region=MAIN_CONTENT_3\&context=storylines_faq}{protective
    proteins made in response to an
    infection}\href{https://www.nytimes.com/2020/05/07/health/coronavirus-antibody-prevalence.html?action=click\&pgtype=Article\&state=default\&region=MAIN_CONTENT_3\&context=storylines_faq}{.
    These antibodies may} last in the body
    \href{https://www.nature.com/articles/s41591-020-0965-6}{only two to
    three months}, which may seem worrisome, but that's perfectly normal
    after an acute infection subsides, said Dr. Michael Mina, an
    immunologist at Harvard University. It may be possible to get the
    coronavirus again, but it's highly unlikely that it would be
    possible in a short window of time from initial infection or make
    people sicker the second time.
  \end{itemize}
\item ~
  \hypertarget{im-a-small-business-owner-can-i-get-relief}{%
  \paragraph{I'm a small-business owner. Can I get
  relief?}\label{im-a-small-business-owner-can-i-get-relief}}

  \begin{itemize}
  \tightlist
  \item
    The
    \href{https://www.nytimes.com/article/small-business-loans-stimulus-grants-freelancers-coronavirus.html?action=click\&pgtype=Article\&state=default\&region=MAIN_CONTENT_3\&context=storylines_faq}{stimulus
    bills enacted in March} offer help for the millions of American
    small businesses. Those eligible for aid are businesses and
    nonprofit organizations with fewer than 500 workers, including sole
    proprietorships, independent contractors and freelancers. Some
    larger companies in some industries are also eligible. The help
    being offered, which is being managed by the Small Business
    Administration, includes the Paycheck Protection Program and the
    Economic Injury Disaster Loan program. But lots of folks have
    \href{https://www.nytimes.com/interactive/2020/05/07/business/small-business-loans-coronavirus.html?action=click\&pgtype=Article\&state=default\&region=MAIN_CONTENT_3\&context=storylines_faq}{not
    yet seen payouts.} Even those who have received help are confused:
    The rules are draconian, and some are stuck sitting on
    \href{https://www.nytimes.com/2020/05/02/business/economy/loans-coronavirus-small-business.html?action=click\&pgtype=Article\&state=default\&region=MAIN_CONTENT_3\&context=storylines_faq}{money
    they don't know how to use.} Many small-business owners are getting
    less than they expected or
    \href{https://www.nytimes.com/2020/06/10/business/Small-business-loans-ppp.html?action=click\&pgtype=Article\&state=default\&region=MAIN_CONTENT_3\&context=storylines_faq}{not
    hearing anything at all.}
  \end{itemize}
\item ~
  \hypertarget{what-are-my-rights-if-i-am-worried-about-going-back-to-work}{%
  \paragraph{What are my rights if I am worried about going back to
  work?}\label{what-are-my-rights-if-i-am-worried-about-going-back-to-work}}

  \begin{itemize}
  \tightlist
  \item
    Employers have to provide
    \href{https://www.osha.gov/SLTC/covid-19/standards.html}{a safe
    workplace} with policies that protect everyone equally.
    \href{https://www.nytimes.com/article/coronavirus-money-unemployment.html?action=click\&pgtype=Article\&state=default\&region=MAIN_CONTENT_3\&context=storylines_faq}{And
    if one of your co-workers tests positive for the coronavirus, the
    C.D.C.} has said that
    \href{https://www.cdc.gov/coronavirus/2019-ncov/community/guidance-business-response.html}{employers
    should tell their employees} -\/- without giving you the sick
    employee's name -\/- that they may have been exposed to the virus.
  \end{itemize}
\item ~
  \hypertarget{should-i-refinance-my-mortgage}{%
  \paragraph{Should I refinance my
  mortgage?}\label{should-i-refinance-my-mortgage}}

  \begin{itemize}
  \tightlist
  \item
    \href{https://www.nytimes.com/article/coronavirus-money-unemployment.html?action=click\&pgtype=Article\&state=default\&region=MAIN_CONTENT_3\&context=storylines_faq}{It
    could be a good idea,} because mortgage rates have
    \href{https://www.nytimes.com/2020/07/16/business/mortgage-rates-below-3-percent.html?action=click\&pgtype=Article\&state=default\&region=MAIN_CONTENT_3\&context=storylines_faq}{never
    been lower.} Refinancing requests have pushed mortgage applications
    to some of the highest levels since 2008, so be prepared to get in
    line. But defaults are also up, so if you're thinking about buying a
    home, be aware that some lenders have tightened their standards.
  \end{itemize}
\item ~
  \hypertarget{what-is-school-going-to-look-like-in-september}{%
  \paragraph{What is school going to look like in
  September?}\label{what-is-school-going-to-look-like-in-september}}

  \begin{itemize}
  \tightlist
  \item
    It is unlikely that many schools will return to a normal schedule
    this fall, requiring the grind of
    \href{https://www.nytimes.com/2020/06/05/us/coronavirus-education-lost-learning.html?action=click\&pgtype=Article\&state=default\&region=MAIN_CONTENT_3\&context=storylines_faq}{online
    learning},
    \href{https://www.nytimes.com/2020/05/29/us/coronavirus-child-care-centers.html?action=click\&pgtype=Article\&state=default\&region=MAIN_CONTENT_3\&context=storylines_faq}{makeshift
    child care} and
    \href{https://www.nytimes.com/2020/06/03/business/economy/coronavirus-working-women.html?action=click\&pgtype=Article\&state=default\&region=MAIN_CONTENT_3\&context=storylines_faq}{stunted
    workdays} to continue. California's two largest public school
    districts --- Los Angeles and San Diego --- said on July 13, that
    \href{https://www.nytimes.com/2020/07/13/us/lausd-san-diego-school-reopening.html?action=click\&pgtype=Article\&state=default\&region=MAIN_CONTENT_3\&context=storylines_faq}{instruction
    will be remote-only in the fall}, citing concerns that surging
    coronavirus infections in their areas pose too dire a risk for
    students and teachers. Together, the two districts enroll some
    825,000 students. They are the largest in the country so far to
    abandon plans for even a partial physical return to classrooms when
    they reopen in August. For other districts, the solution won't be an
    all-or-nothing approach.
    \href{https://bioethics.jhu.edu/research-and-outreach/projects/eschool-initiative/school-policy-tracker/}{Many
    systems}, including the nation's largest, New York City, are
    devising
    \href{https://www.nytimes.com/2020/06/26/us/coronavirus-schools-reopen-fall.html?action=click\&pgtype=Article\&state=default\&region=MAIN_CONTENT_3\&context=storylines_faq}{hybrid
    plans} that involve spending some days in classrooms and other days
    online. There's no national policy on this yet, so check with your
    municipal school system regularly to see what is happening in your
    community.
  \end{itemize}
\end{itemize}

``I was super scared and worried about Covid, since I never had
experienced anything like that before,'' he said. A doctor checked his
blood pressure, listened to his lungs and took his temperature --- but
did not administer a coronavirus test. He recalls being told that the
emergency room was giving the test only to ``critically ill'' patients,
and he did not qualify.

Mr. Aita, who lost his job at a record store at the start of the
pandemic and is uninsured, received a document at the end of his visit
estimating he would owe \$1,157. If the hospital had tested him for
coronavirus, the federal fund could have covered the visit entirely.

Last week, he received a medical bill for the visit that was only \$350.
He initially thought this was good news --- that the hospital had
dropped his charge. But when he looked into the issue, he learned this
was an additional charge from the doctor who saw him.

``I understood that if it was related to Covid, it would be taken care
of,'' Mr. Aita said. ``It's a pandemic, I'm unemployed, and now I'm
dealing with the stress of this situation.''

A spokesman for Dignity Health, which owns St. Mary's Medical Center,
said the hospital uses C.D.C. protocols to decide who is tested, but he
declined to comment on Mr. Aita's case.

``We have suspended billing patients for their portion of their bill for
the testing and treatment of Covid-19 while we work with insurers and
the government to exhaust financial assistance options for patients,''
Chad Burns, the spokesman, said. Mr. Aita, however, does not appear to
qualify for those protections because he did not receive a test.

As coronavirus spreads and hospitalizations mount, so will the ranks of
those managing unexpected bills.

The Kaiser Family Foundation estimates that a
\href{https://www.kff.org/coronavirus-covid-19/issue-brief/potential-costs-of-coronavirus-treatment-for-people-with-employer-coverage/}{fifth}
of all coronavirus hospitalizations could result in a surprise medical
bill from an out-of-network doctor who became involved in the patient's
care. The nonpartisan foundation also
\href{https://www.kff.org/coronavirus-covid-19/issue-brief/five-things-to-know-about-the-cost-of-covid-19-testing-and-treatment/}{projects}
that, on average, an American with employer-sponsored coverage would
face \$1,300 in costs for a coronavirus hospitalization.

Congressional staffers working on the issue say they've come across
cases in which health providers are not following the new rules on
coronavirus billing. The providers are charging patients for services
when they shouldn't, or not posting their cash prices for testing online
as they are legally required to.

``Billing offices may just be doing what they're used to --- looking at
your card, seeing that it says \$30 co-pay and collecting it,'' Ms.
Pollitz said. ``The person at the front desk may not know you got a
test. The protections aren't airtight.''

Congress is currently split over how far to go in protecting coronavirus
patients from surprise medical bills. House Democrats have supported
mandating that insurers cover all costs related to treatment as part of
the HEROES Act, a larger stimulus package.

Senate Republicans introduced their stimulus proposal, the HEALS Act,
last week. It does not include a similar mandate.

In the case of Ms. Krebs, she shared her bill with me after reading
another article I wrote about coronavirus test billing. Together, we
tracked down a record for her coronavirus test to prove that it did
indeed occur.

Two days after I inquired about the case, the Valley Hospital
resubmitted her bill with the coronavirus test included. Her insurer,
Aetna, reprocessed the bill and confirmed that she would no longer be
charged.

``We were trying to come up with extraordinary processes quickly to
react to the many changes placed on all of us, including payer
requirements of coverage,'' Josette Portalatin, an assistant vice
president at the hospital, wrote in an email to Ms. Krebs. ``We
apologize that your lab Covid test was not on your original claim, but
happy to report we tracked down the issue.''

Advertisement

\protect\hyperlink{after-bottom}{Continue reading the main story}

\hypertarget{site-index}{%
\subsection{Site Index}\label{site-index}}

\hypertarget{site-information-navigation}{%
\subsection{Site Information
Navigation}\label{site-information-navigation}}

\begin{itemize}
\tightlist
\item
  \href{https://help.nytimes.com/hc/en-us/articles/115014792127-Copyright-notice}{©~2020~The
  New York Times Company}
\end{itemize}

\begin{itemize}
\tightlist
\item
  \href{https://www.nytco.com/}{NYTCo}
\item
  \href{https://help.nytimes.com/hc/en-us/articles/115015385887-Contact-Us}{Contact
  Us}
\item
  \href{https://www.nytco.com/careers/}{Work with us}
\item
  \href{https://nytmediakit.com/}{Advertise}
\item
  \href{http://www.tbrandstudio.com/}{T Brand Studio}
\item
  \href{https://www.nytimes.com/privacy/cookie-policy\#how-do-i-manage-trackers}{Your
  Ad Choices}
\item
  \href{https://www.nytimes.com/privacy}{Privacy}
\item
  \href{https://help.nytimes.com/hc/en-us/articles/115014893428-Terms-of-service}{Terms
  of Service}
\item
  \href{https://help.nytimes.com/hc/en-us/articles/115014893968-Terms-of-sale}{Terms
  of Sale}
\item
  \href{https://spiderbites.nytimes.com}{Site Map}
\item
  \href{https://help.nytimes.com/hc/en-us}{Help}
\item
  \href{https://www.nytimes.com/subscription?campaignId=37WXW}{Subscriptions}
\end{itemize}
