Sections

SEARCH

\protect\hyperlink{site-content}{Skip to
content}\protect\hyperlink{site-index}{Skip to site index}

\href{https://www.nytimes.com/section/us}{U.S.}

\href{https://myaccount.nytimes.com/auth/login?response_type=cookie\&client_id=vi}{}

\href{https://www.nytimes.com/section/todayspaper}{Today's Paper}

\href{/section/us}{U.S.}\textbar{}A Historic Supreme Court Ruling Upends
Courts in Oklahoma

\url{https://nyti.ms/33jJbbJ}

\begin{itemize}
\item
\item
\item
\item
\item
\end{itemize}

Advertisement

\protect\hyperlink{after-top}{Continue reading the main story}

Supported by

\protect\hyperlink{after-sponsor}{Continue reading the main story}

\hypertarget{a-historic-supreme-court-ruling-upends-courts-in-oklahoma}{%
\section{A Historic Supreme Court Ruling Upends Courts in
Oklahoma}\label{a-historic-supreme-court-ruling-upends-courts-in-oklahoma}}

Local prosecutors are referring criminal cases to the federal and tribal
courts, which are now flooded with new cases.

\includegraphics{https://static01.nyt.com/images/2020/07/28/us/00tribes-justice01/merlin_174879450_f5946007-96ca-4a36-bfa4-5af71c874eeb-articleLarge.jpg?quality=75\&auto=webp\&disable=upscale}

By \href{https://www.nytimes.com/by/jack-healy}{Jack Healy}

\begin{itemize}
\item
  Published Aug. 3, 2020Updated Aug. 4, 2020, 12:16 a.m. ET
\item
  \begin{itemize}
  \item
  \item
  \item
  \item
  \item
  \end{itemize}
\end{itemize}

TULSA, Okla. --- Kelsey Lipp was sitting in jail, charged with robbery
and murder, when her lawyer walked into court with three pieces of paper
and a new plan to get her case thrown out.

The documentation he had looked sparse: A letter identifying Ms. Lipp as
a citizen of the Cherokee Nation and grainy photocopies of her tribal
identification card. But under a landmark Supreme Court decision last
month declaring that a huge patch of Oklahoma sits on a Native American
reservation, those papers now meant that the state could not prosecute
Ms. Lipp or thousands of other tribal citizens like her.

``It's a no-brainer,'' her Tulsa County public defender, Jack Gordon,
said.

The Supreme Court ruling
\href{https://www.nytimes.com/2020/07/09/us/supreme-court-oklahoma-mcgirt-creek-nation.html}{recognizing
the lands of the Muscogee (Creek) Nation} was hailed as a historic win
for tribes and their long struggle for sovereignty. On the ground, it
has upended Oklahoma's justice system, forcing lawyers and the police to
rewrite the rules of who they can and cannot prosecute inside the newly
recognized borders of a reservation that stretches across 11 counties
and includes Tulsa, the state's second-largest city.

Prosecutors are giving police officers laminated index cards that spell
out how to proceed depending on whether suspects and victims are
``Indian'' or ``non-Indian.''

``It's unprecedented,'' said R. Trent Shores, the United States attorney
for the Northern District of Oklahoma in Tulsa.

Elected district attorneys handle most criminal cases in America, but
they generally have little to no authority over tribal citizens for
crimes committed on reservations. So now, from downtown Tulsa through
rolling farms and dozens of small towns in eastern Oklahoma, local
prosecutors are handing off hundreds of criminal cases involving tribal
victims and defendants.

``My voice mail got filled up in two hours,'' said Stephen Lee, a
criminal defense lawyer in Tulsa. ``People with loved ones who are
locked up, people with pending cases.''

Local prosecutors are referring dozens of murders, robberies and sexual
assaults to federal prosecutors, who have responsibility for major
crimes on tribal lands. Lesser cases are being handed over to tribal
courts, which can only hand down smaller fines and sentences of a year
or less in most cases.

The flood of new cases is threatening to overwhelm the smaller rosters
of judges, attorneys and victims' advocates in federal and tribal
courts. There are just two judges on the Muscogee Nation's court, and
tribal officials say they will need more money and staff to handle
hundreds of additional cases.

The fatal shooting that led to Ms. Lipp's arrest began when a
25-year-old man was lured to her apartment in July 2018 on the promise
he would get a kiss in exchange for \$100, investigators say. The
victim, Dustin Barham, was robbed and shot, bleeding to death,
prosecutors say. Ms. Lipp, her cousin and cousin's boyfriend have been
charged in his killing.

Mr. Gordon, Ms. Lipp's lawyer, said Ms. Lipp denied any role in the
murder, and hoped that moving the case from state court to federal court
could lead to a plea deal or re-examination of what he called a flawed
case against Ms. Lipp. ``We're better off over there,'' he said.

Mr. Barham's mother, Andra, said she had already waited two years for
justice for her dead son, whom she called a ``good-hearted person,'' and
worried that refiling the criminal case in federal court would add years
of additional delays.

``We're looking at starting over,'' she said. ``It's frustrating.''

The Muscogee Nation established its court system in 1867, and tribal
prosecutors and judges say their courtrooms are the best forums for
Indigenous people to get justice and a fair hearing. ``We understand
these people are going back into our community,'' said Gregory Bigler,
one of the Muscogee district judges.

But they are now confronting a thicket of complications: How will the
tribal court in the small town of Okmulgee, home of the Muscogee (Creek)
Nation's headquarters, handle cases when people are arrested an hour
away in Tulsa for shoplifting or low-level drug possession? Does it make
sense to spend money jailing them or transporting them to hearings?

``We're going to have to grow exponentially,'' said Shannon Prescott,
the other Muscogee district judge.

One recent morning, the tribal court was shuffling through the day's
criminal charges and pleas through a video hearing when a bald man in an
orange jumpsuit shuffled in front of the camera. He had been arrested in
Tulsa on a charge of threatening violence but was brought to the
Okmulgee County Jail and handed over to tribal court when the police
realized he had an Osage ancestry.

``That would have been a Tulsa case,'' Mark Thetford, a Muscogee
prosecutor, said. ``It's kind of crazy right now.''

In Tulsa, federal prosecutors have vowed ``seamless jurisdiction'' and
said tribes and law enforcement agencies have a long history of
cooperation. Nevertheless, the federal government is scrambling to find
more lawyers and staff members to handle the surge. The U.S. attorney's
office in Tulsa files about 250 felony cases annually, compared with the
6,000 felonies that churn through Tulsa's county courts each year.

``It's a lot more than we normally do,'' Mr. Shores, the United States
attorney, said. ``There's only so much we're able to take.''

Native Americans convicted by state courts have begun filing appeals
arguing the state did not have the power to try them. Four Cherokee
citizens have filed a class-action lawsuit demanding that Oklahoma
return millions of dollars in court fees and fines that Indigenous
defendants have been ordered to pay over the years.

Some criminal cases have been upended when the victim, not the
defendant, turns out to be a tribal member.

Dustin Dennis, who prosecutors said was not a tribal member, was charged
with second-degree murder in July after his young son and daughter,
Teagan, 4, and Ryan, 3, were found dead in his sweltering pickup. The
children climbed into the car and were apparently overcome by the heat
while Mr. Dennis slept, prosecutors said.

Tulsa County prosecutors had to drop the case when it turned out the
children were Cherokee on their mother's side. Mr. Dennis was charged
federally with child neglect, but the Tulsa district attorney, Steve
Kunzweiler, said it had been devastating to tell the children's mother
he was dropping the case.

``She thinks she's on her path to justice, and I'm telling her I have to
dismiss this charge,'' Mr. Kunzweiler said. ``I'm just worried about all
these victims out there who've believed they're getting justice only to
have justice interrupted.''

Mr. Shores, the U.S. attorney in Tulsa, said his office had reached out
to the children's mother to assure her they were continuing the case. In
a brief interview, the mother, Cheyenne Trent, said that ``I just want
justice for my two babies, that's it.''

Beyond crime scenes and courtrooms, the ripples are radiating to other
reservations across Oklahoma.

\includegraphics{https://static01.nyt.com/images/2020/07/28/us/00tribes-justice02/merlin_175007745_8198c0f1-2185-48b5-9174-138de6362531-articleLarge.jpg?quality=75\&auto=webp\&disable=upscale}

The Supreme Court's decision dealt with the boundaries of the Muscogee
(Creek) Nation, but nearly half of Oklahoma rests on land of five tribes
whose members were forced west along the Trail of Tears in the 1800s ---
an expanse with nearly 2 million residents.

Legal experts say that eastern Oklahoma's other tribes --- the Choctaw,
Chickasaw, Seminole and Cherokee nations --- now have strong arguments
that their lands should also be legally recognized as reservations.

The question now, Indigenous leaders and activists said, is whether they
will be able to hold on to their recent gains or see them undone.

To address the ``unpredictability'' created by the Supreme Court
decision, Gov. Kevin Stitt, a Republican, created a 10-member commission
to study the fallout and make recommendations to the state. But tribal
leaders say they were excluded from the panel, which is led by a former
oil executive and made up of Republican politicians and business
leaders.

Indigenous activists say they are worried that industry leaders, to
protect their interests against any new regulations, will push through
legislation that could dilute tribal powers or even basically dissolve
their reservations.

The question of whether to work with Congress on a new law addressing
tribal sovereignty has divided Oklahoma's tribes. The Muscogee and
Seminole nations are
\href{https://tulsaworld.com/news/local/crime-and-courts/creek-seminole-tribes-deny-pact-with-state-on-jurisdictional-issues/article_d5e11094-2cd8-53a6-9686-13768e05ccce.html}{opposed}.
But Principal Chief Chuck Hoskin Jr. of the Cherokee Nation said the
tribes could either work with Congress or become the victims of yet
another law stripping them of land and power.

``I know my history, and I know when we've made advances, Congress can
push back,'' Mr. Hoskin said. ``They possess the power to do injury to
us. I don't have the luxury of closing my eyes and covering my ears and
hoping for the best.''

Alison Arkeketa is among those hoping their loved ones can get a fresh
chance at justice from a different court. Her fiancé is facing up to 10
years in prison for illegally possessing a gun as a felon convicted of
assault, but Ms. Arkeketa said he needed substance-abuse counseling and
not another decade in prison --- ``to be treated like a human.''

That decision will now likely lie with a federal or tribal judge. A
lawyer for her fiancé recently filed a motion arguing for a dismissal
because his Creek citizenship put him out of reach of the local county
court.

Advertisement

\protect\hyperlink{after-bottom}{Continue reading the main story}

\hypertarget{site-index}{%
\subsection{Site Index}\label{site-index}}

\hypertarget{site-information-navigation}{%
\subsection{Site Information
Navigation}\label{site-information-navigation}}

\begin{itemize}
\tightlist
\item
  \href{https://help.nytimes.com/hc/en-us/articles/115014792127-Copyright-notice}{©~2020~The
  New York Times Company}
\end{itemize}

\begin{itemize}
\tightlist
\item
  \href{https://www.nytco.com/}{NYTCo}
\item
  \href{https://help.nytimes.com/hc/en-us/articles/115015385887-Contact-Us}{Contact
  Us}
\item
  \href{https://www.nytco.com/careers/}{Work with us}
\item
  \href{https://nytmediakit.com/}{Advertise}
\item
  \href{http://www.tbrandstudio.com/}{T Brand Studio}
\item
  \href{https://www.nytimes.com/privacy/cookie-policy\#how-do-i-manage-trackers}{Your
  Ad Choices}
\item
  \href{https://www.nytimes.com/privacy}{Privacy}
\item
  \href{https://help.nytimes.com/hc/en-us/articles/115014893428-Terms-of-service}{Terms
  of Service}
\item
  \href{https://help.nytimes.com/hc/en-us/articles/115014893968-Terms-of-sale}{Terms
  of Sale}
\item
  \href{https://spiderbites.nytimes.com}{Site Map}
\item
  \href{https://help.nytimes.com/hc/en-us}{Help}
\item
  \href{https://www.nytimes.com/subscription?campaignId=37WXW}{Subscriptions}
\end{itemize}
