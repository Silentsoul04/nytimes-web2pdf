Sections

SEARCH

\protect\hyperlink{site-content}{Skip to
content}\protect\hyperlink{site-index}{Skip to site index}

\href{https://www.nytimes.com/section/politics}{Politics}

\href{https://myaccount.nytimes.com/auth/login?response_type=cookie\&client_id=vi}{}

\href{https://www.nytimes.com/section/todayspaper}{Today's Paper}

\href{/section/politics}{Politics}\textbar{}What Happened to the Young
Voters Focused on Guns?

\url{https://nyti.ms/3fmK9WV}

\begin{itemize}
\item
\item
\item
\item
\item
\end{itemize}

\href{https://www.nytimes.com/news-event/george-floyd-protests-minneapolis-new-york-los-angeles?action=click\&pgtype=Article\&state=default\&region=TOP_BANNER\&context=storylines_menu}{Race
and America}

\begin{itemize}
\tightlist
\item
  \href{https://www.nytimes.com/2020/07/26/us/protests-portland-seattle-trump.html?action=click\&pgtype=Article\&state=default\&region=TOP_BANNER\&context=storylines_menu}{Protesters
  Return to Other Cities}
\item
  \href{https://www.nytimes.com/2020/07/24/us/portland-oregon-protests-white-race.html?action=click\&pgtype=Article\&state=default\&region=TOP_BANNER\&context=storylines_menu}{Portland
  at the Center}
\item
  \href{https://www.nytimes.com/2020/07/23/podcasts/the-daily/portland-protests.html?action=click\&pgtype=Article\&state=default\&region=TOP_BANNER\&context=storylines_menu}{Podcast:
  Showdown in Portland}
\item
  \href{https://www.nytimes.com/interactive/2020/07/16/us/black-lives-matter-protests-louisville-breonna-taylor.html?action=click\&pgtype=Article\&state=default\&region=TOP_BANNER\&context=storylines_menu}{45
  Days in Louisville}
\end{itemize}

Advertisement

\protect\hyperlink{after-top}{Continue reading the main story}

Supported by

\protect\hyperlink{after-sponsor}{Continue reading the main story}

\hypertarget{what-happened-to-the-young-voters-focused-on-guns}{%
\section{What Happened to the Young Voters Focused on
Guns?}\label{what-happened-to-the-young-voters-focused-on-guns}}

It's not their top issue anymore, but many see it as a part of other,
larger concerns.

\includegraphics{https://static01.nyt.com/images/2020/08/03/us/politics/03guns-politics1/merlin_167475906_aee99c28-c574-4b77-abf0-4ba1c3d9f256-articleLarge.jpg?quality=75\&auto=webp\&disable=upscale}

\href{https://www.nytimes.com/by/giovanni-russonello}{\includegraphics{https://static01.nyt.com/images/2019/04/03/multimedia/author-giovanni-russonello/author-giovanni-russonello-thumbLarge.png}}

By \href{https://www.nytimes.com/by/giovanni-russonello}{Giovanni
Russonello}

\begin{itemize}
\item
  Published Aug. 3, 2020Updated Aug. 4, 2020, 12:20 a.m. ET
\item
  \begin{itemize}
  \item
  \item
  \item
  \item
  \item
  \end{itemize}
\end{itemize}

After the
\href{https://www.nytimes.com/2019/02/13/us/parkland-anniversary-marjory-stoneman-douglas.html}{mass
shooting in 2018 at Marjory Stoneman Douglas High School} left 17 people
dead in Parkland, Fla., the massacre's young survivors converted their
outrage into political organizing. Alongside other student leaders
across the country, they brought
\href{https://www.nytimes.com/2018/03/24/us/march-for-our-lives.html}{hundreds
of thousands of people to Washington for the March for Our Lives},
pressed the case for tougher gun laws in the Florida legislature and at
the U.S. Senate, registered 50,000 new voters nationally, and helped
drive a
\href{https://www.nytimes.com/2018/11/07/us/elections-gun-control-florida.html}{surge
in turnout} by young people in that year's midterm elections.

But Congress passed no gun legislation, even as measures like universal
background checks consistently win the support of about nine in 10
Americans, according to
\href{http://maristpoll.marist.edu/npr-pbs-newshour-marist-poll-results-and-analysis-4/\#sthash.d89KlpFU.5oyynxDB.dpbs}{surveys}.

Another stark reminder of the country's vulnerability to gun violence
arrived
\href{https://www.nytimes.com/2019/08/03/us/el-paso-shooting.html}{one
year ago}, when a gunman in El Paso, Texas --- a state with some of the
country's most permissive gun laws --- opened fire at a Walmart
Supercenter,
\href{https://www.nytimes.com/2019/08/03/us/el-paso-shooting.html}{killing
23 people} in a rampage driven by anti-Hispanic hatred.

But now, with the country swept up in protests over racial justice
driven largely by young people, the youthful voices that propelled a
movement just two years ago find themselves less squarely focused on
issues around gun violence. Polls show that racial justice, the
coronavirus pandemic and the related economic downturn far outpace guns
as top issues of concern for young people. When asked about gun control
measures, it is in fact the oldest Americans who now most often express
support, according to some
\href{https://poll.qu.edu/national/release-detail?ReleaseID=3639}{polls}.

The activists who organized after the Parkland shooting say they have
built up their organizing capacity since then, and they remain committed
to making at least as significant a difference in 2020 as they did in
2018. But this year, they say, a big part of that will mean building
solidarity with organizers confronting racial injustice.

``For us, we recognize how gun violence is such an intersectional
issue,'' said Kelly Choi, 20, a member of the executive board at
\href{https://marchforourlives.com/}{March for Our Lives}, the national
nonprofit that grew out of the Parkland students' organizing. ``Gun
violence is the symptom of other things, like poverty, racism, housing
insecurity, domestic violence.''

\href{https://studentsdemandaction.org/}{Students Demand Action}, a
grass-roots network affiliated with Michael Bloomberg's nonprofit
Everytown for Gun Safety, sprouted more than 400 chapters of its own
after the Parkland shooting. Its student members are at the core of
Everytown's voter-registration and other campaign operations, part of
\href{https://www.nytimes.com/2020/07/23/us/politics/bloomberg-guns.html}{a
planned \$60 million investment} by the nonprofit in federal and state
races this fall.

In their work, too, Students Demand Action organizers are emphasizing
collaboration. ``When I first got involved, it was in the wake of a
school shooting, and there was this thought that gun violence is school
shootings and mass shootings,'' said Alanna Miller, 19, who founded a
Students Demand Action chapter at her Texas high school after the
Parkland attack, and then another chapter at Duke University --- where
she is now a rising sophomore --- in the wake of El Paso. ``Yes, that's
true, but it's also so much more than that. It's domestic-partner
violence, it's inner-city violence.''

``I don't know if I would still be in this movement, organizing, if I
hadn't expanded my worldview in thinking about this issue and how
insidious and how pervasive it is,'' she added.

March for Our Lives has established partnerships with older
organizations like the longstanding Brady Campaign as well as Black
community-based groups like the
\href{https://www.cjactionfund.org/}{Community Justice Action Fund}. In
April, March for Our Lives joined a number of other youth-led groups,
including the Sunrise Movement and Justice Democrats, in writing a
\href{https://marchforourlives.com/earn-our-vote/}{letter to Joseph R.
Biden Jr.'s campaign} demanding action on a range of progressive
policies.

On Monday evening in El Paso, the local March for Our Lives chapter
plans to convene a vigil at the site of last year's massacre, with
participation from other local groups. The next day, young organizers
with the nearby Houston chapter will gather alongside representatives
from racial-justice organizations and other activist groups to promote a
City Council bill
\href{https://www.khou.com/article/news/local/police-reform-proposal-involves-cutting-199-unfilled-hpd-positions/285-9a6b0853-7ce2-4638-aa35-65e0e12be3d8}{proposed
by the council member Letitia Plummer} that would reallocate funding
away from the Houston Police Department and toward social programs.

A year after the Parkland shooting, with Democrats enjoying a newly
strengthened majority in the House, young adults remained more likely
than older Americans to say that gun control legislation should be an
immediate priority for Congress, with half of respondents under 30
saying so, according to an
\href{http://maristpoll.marist.edu/wp-content/uploads/2019/02/NPR_PBS-NewsHour_Marist-Poll_USA-NOS-and-Tables_1902121446.pdf\#page=3}{NPR/PBS
NewsHour/Marist College poll}.

In that survey, conducted in February 2019, two-thirds of those under 30
said that they prioritized strengthening gun laws over preserving the
rights of firearm owners --- a heavier gun-control tilt than in any
other age group. Young adults were also the only age group to have a
majority unfavorable opinion of the National Rifle Association,
according to the poll.

When the massacre in El Paso took place, the Democratic presidential
race was just beginning to heat up, and gun control again appeared
poised to become a central focus.

The presidential candidate Beto O'Rourke, who had spent six years
representing El Paso in Congress, suspended his campaign and visited the
site of the shooting. During a presidential debate a month later, he
pledged to put in place a national gun registry and a mandatory buyback
program for assault rifles, declaring: ``Hell, yes, we're going to take
your AR-15, your AK-47.''

But today, with a different youth-protest movement sweeping the country,
millennials and young adults in Generation Z are more likely to
explicitly name racial justice as their top political concern. A
\href{https://static.foxnews.com/foxnews.com/content/uploads/2020/07/Fox_July-12-15-2020_Complete_National_Topline_July-19-Release.pdf}{Fox
News poll} last month found that voters under 30 were three times as
likely as those 45 and over to call race-related issues their No. 1
policy priority. Just three percent of the youngest voters named gun
violence as their top concern.

And in a sense, this is not an entirely new phenomenon. Less than a
month after the Parkland shooting, a
\href{https://news.gallup.com/poll/229562/preference-stricter-gun-laws-highest-1993.aspx}{Gallup
survey} found that race relations was tied with gun control as the
most-cited issue of concern among Americans under 30.

Charlie Kelly, Everytown's senior political adviser, said that
emphasizing the links between racial justice and gun policy could be
essential to driving home a message that resonates this year. ``These
issues are inextricably linked,'' he said. ``When we put out a call to
our supporters to support the
\href{https://www.nytimes.com/2020/06/25/us/politics/house-police-overhaul-bill.html}{George
Floyd Justice in Policing Act}, we saw students take action at twice the
rate of any previous action.''

Mr. Kelly said that Everytown's own research this year had shown that
gun violence remains an issue organizers think they can win on, and one
with particular appeal to young voters. ``Gun safety is especially
powerful and persuasive at mobilizing young voters, communities of
color, suburban women,'' Mr. Kelly said, pointing to internal polling
and message-testing that the organization recently undertook in
battleground states, including Texas.

``Gun safety messaging was the most effective and resonant among young
voters, 18 to 34, and independents,'' he added.

The midterms in 2018 drew high levels of participation across the board,
but the spike was especially large for young people. Among voters under
30, turnout doubled from 2014 to 2018, according to the
\href{http://www.electproject.org/home/voter-turnout/demographics}{United
States Elections Project at the University of Florida}. Not since the
1980s had young people made up so big a share of the midterm electorate.

The highest-profile Democrats who put gun control at the center of their
campaigns --- such as Andrew Gillum, the candidate for Florida governor,
and Mr. O'Rourke, then running for Senate in Texas ---~lost. But amid a
Democratic surge in congressional races, advocates reported broad
success: In 43 federal races in which Everytown and the N.R.A. endorsed
opposing candidates, the Everytown candidate won 33 of them.

``This isn't the end of the race, this is permission to start,'' David
Hogg, a Parkland student who had by then become a spokesman for the
movement,
\href{https://www.sun-sentinel.com/local/broward/parkland/florida-school-shooting/fl-ne-whats-next-march-for-our-lives-20181107-story.html}{said}
after the midterms. ``The shooting at Stoneman Douglas has all been
training for us on how to get corrupt politicians out of power.''

A similar cycle played out~in 2019, when voters went to the polls in
Virginia just three months after the El Paso massacre. With gun control
most frequently cited
\href{https://www.washingtonpost.com/context/washington-post-schar-school-virginia-poll-sept-25-30-2019/26848165-b676-4baa-9c70-cad966571930/?itid=lk_inline_manual_1}{in
statewide polls} as their top issue, voters handed Democrats control of
both houses in the State Legislature for the first time in 25 years.

Advertisement

\protect\hyperlink{after-bottom}{Continue reading the main story}

\hypertarget{site-index}{%
\subsection{Site Index}\label{site-index}}

\hypertarget{site-information-navigation}{%
\subsection{Site Information
Navigation}\label{site-information-navigation}}

\begin{itemize}
\tightlist
\item
  \href{https://help.nytimes.com/hc/en-us/articles/115014792127-Copyright-notice}{©~2020~The
  New York Times Company}
\end{itemize}

\begin{itemize}
\tightlist
\item
  \href{https://www.nytco.com/}{NYTCo}
\item
  \href{https://help.nytimes.com/hc/en-us/articles/115015385887-Contact-Us}{Contact
  Us}
\item
  \href{https://www.nytco.com/careers/}{Work with us}
\item
  \href{https://nytmediakit.com/}{Advertise}
\item
  \href{http://www.tbrandstudio.com/}{T Brand Studio}
\item
  \href{https://www.nytimes.com/privacy/cookie-policy\#how-do-i-manage-trackers}{Your
  Ad Choices}
\item
  \href{https://www.nytimes.com/privacy}{Privacy}
\item
  \href{https://help.nytimes.com/hc/en-us/articles/115014893428-Terms-of-service}{Terms
  of Service}
\item
  \href{https://help.nytimes.com/hc/en-us/articles/115014893968-Terms-of-sale}{Terms
  of Sale}
\item
  \href{https://spiderbites.nytimes.com}{Site Map}
\item
  \href{https://help.nytimes.com/hc/en-us}{Help}
\item
  \href{https://www.nytimes.com/subscription?campaignId=37WXW}{Subscriptions}
\end{itemize}
