Sections

SEARCH

\protect\hyperlink{site-content}{Skip to
content}\protect\hyperlink{site-index}{Skip to site index}

\href{https://www.nytimes.com/section/well/family}{Family}

\href{https://myaccount.nytimes.com/auth/login?response_type=cookie\&client_id=vi}{}

\href{https://www.nytimes.com/section/todayspaper}{Today's Paper}

\href{/section/well/family}{Family}\textbar{}The Benefits of Talking to
Strangers

\url{https://nyti.ms/3i2Stgs}

\begin{itemize}
\item
\item
\item
\item
\item
\item
\end{itemize}

\href{https://www.nytimes.com/news-event/coronavirus?action=click\&pgtype=Article\&state=default\&region=TOP_BANNER\&context=storylines_menu}{The
Coronavirus Outbreak}

\begin{itemize}
\tightlist
\item
  live\href{https://www.nytimes.com/2020/08/03/world/coronavirus-covid-19.html?action=click\&pgtype=Article\&state=default\&region=TOP_BANNER\&context=storylines_menu}{Latest
  Updates}
\item
  \href{https://www.nytimes.com/interactive/2020/us/coronavirus-us-cases.html?action=click\&pgtype=Article\&state=default\&region=TOP_BANNER\&context=storylines_menu}{Maps
  and Cases}
\item
  \href{https://www.nytimes.com/interactive/2020/science/coronavirus-vaccine-tracker.html?action=click\&pgtype=Article\&state=default\&region=TOP_BANNER\&context=storylines_menu}{Vaccine
  Tracker}
\item
  \href{https://www.nytimes.com/2020/08/02/us/covid-college-reopening.html?action=click\&pgtype=Article\&state=default\&region=TOP_BANNER\&context=storylines_menu}{College
  Reopening}
\item
  \href{https://www.nytimes.com/live/2020/08/03/business/stock-market-today-coronavirus?action=click\&pgtype=Article\&state=default\&region=TOP_BANNER\&context=storylines_menu}{Economy}
\end{itemize}

Advertisement

\protect\hyperlink{after-top}{Continue reading the main story}

Supported by

\protect\hyperlink{after-sponsor}{Continue reading the main story}

Personal Health

\hypertarget{the-benefits-of-talking-to-strangers}{%
\section{The Benefits of Talking to
Strangers}\label{the-benefits-of-talking-to-strangers}}

Casual connections with people we encounter in the course of daily life
can give us the sense of belonging to a community.

\includegraphics{https://static01.nyt.com/images/2020/08/04/science/03BRODY-STRANGERS-illo/03BRODY-STRANGERS-illo-articleLarge.jpg?quality=75\&auto=webp\&disable=upscale}

\href{https://www.nytimes.com/by/jane-e-brody}{\includegraphics{https://static01.nyt.com/images/2018/06/12/multimedia/jane-e-brody/jane-e-brody-thumbLarge.png}}

By \href{https://www.nytimes.com/by/jane-e-brody}{Jane E. Brody}

\begin{itemize}
\item
  Aug. 3, 2020
\item
  \begin{itemize}
  \item
  \item
  \item
  \item
  \item
  \item
  \end{itemize}
\end{itemize}

I'm a lifelong extrovert who readily establishes and relishes casual
contacts with people I encounter during daily life: while walking my
dog, shopping for groceries, working out at the Y, even sweeping my
sidewalk. These ephemeral connections add variety to my life, are a
source of useful information and often provide needed emotional and
physical support. Equally important, they nearly always leave me with a
smile on my face (although now hidden under a mask!).

In recent months, under stay-at-home orders because of the coronavirus
pandemic, many people lost such daily encounters. I, on the other hand,
have done my best to maintain as many of them as possible while striving
to remain safe. With in-person time with family and close friends now
limited by a mutual desire to avoid exposure to Covid-19, the brief
socially distant contacts with people in my neighborhood, both those
I've known casually for years and others I just met, have been crucial
to my emotional and practical well-being and maybe even my health.

The benefits I associate with my casual connections were reinforced
recently by a fortuitous find. During a Covid-inspired cleanup I
stumbled upon a book in my library called ``Consequential Strangers: The
Power of People Who Don't Seem to Matter \ldots{} \emph{But Really
Do.''} Published 11 years ago, this enlightening tome was written by
Melinda Blau, a science writer, and Karen L. Fingerman, currently a
professor of psychology at the University of Texas, Austin, who studies
the nature and effects of so-called weak ties that people have with
others in their lives: the barista who fetches their coffee, the person
who cuts their hair, the proprietor of the local market, the folks they
see often at the gym or train station.

In an interview, Dr. Fingerman noted that casual connections with people
encountered in the course of daily life can give people a feeling that
they belong to a community, which she described as ``a basic human
need.''

As she and Ms. Blau wrote in their book, consequential strangers ``are
as vital to our well-being, growth, and day-to-day existence as family
and close friends. Consequential strangers anchor us in the world and
give us a sense of being plugged into something larger. They also
enhance and enrich our lives and offer us opportunities for novel
experiences and information that is beyond the purview of our inner
circles. They are vital social connections --- people who help you get
through the day and make life more interesting.''

\hypertarget{latest-updates-global-coronavirus-outbreak}{%
\section{\texorpdfstring{\href{https://www.nytimes.com/2020/08/03/world/coronavirus-covid-19.html?action=click\&pgtype=Article\&state=default\&region=MAIN_CONTENT_1\&context=storylines_live_updates}{Latest
Updates: Global Coronavirus
Outbreak}}{Latest Updates: Global Coronavirus Outbreak}}\label{latest-updates-global-coronavirus-outbreak}}

Updated 2020-08-04T07:33:06.428Z

\begin{itemize}
\tightlist
\item
  \href{https://www.nytimes.com/2020/08/03/world/coronavirus-covid-19.html?action=click\&pgtype=Article\&state=default\&region=MAIN_CONTENT_1\&context=storylines_live_updates\#link-4547638f}{Fauci
  defends Birx after she is criticized by Trump.}
\item
  \href{https://www.nytimes.com/2020/08/03/world/coronavirus-covid-19.html?action=click\&pgtype=Article\&state=default\&region=MAIN_CONTENT_1\&context=storylines_live_updates\#link-15e7f995}{Trump
  derides Democrats as lawmakers and administration officials try to
  break stimulus impasse.}
\item
  \href{https://www.nytimes.com/2020/08/03/world/coronavirus-covid-19.html?action=click\&pgtype=Article\&state=default\&region=MAIN_CONTENT_1\&context=storylines_live_updates\#link-e5a2cda}{The
  deadline for 2020 census counting has been moved up by a month.}
\end{itemize}

\href{https://www.nytimes.com/2020/08/03/world/coronavirus-covid-19.html?action=click\&pgtype=Article\&state=default\&region=MAIN_CONTENT_1\&context=storylines_live_updates}{See
more updates}

More live coverage:
\href{https://www.nytimes.com/live/2020/08/03/business/stock-market-today-coronavirus?action=click\&pgtype=Article\&state=default\&region=MAIN_CONTENT_1\&context=storylines_live_updates}{Markets}

My tendency to ``chat up'' total strangers I meet in the course of just
living has resulted in a slew of acquaintances who have filled my days
with pleasantries, advice, information, needed assistance and, most
important of all during this time of enforced semi-isolation, a valuable
sense of connections to people who share my environment.

Covid-19 lockdowns have reminded so many of us of how important our
relationships are to our quality of life --- not only relationships with
the friends and family members we love and know well and who know us
well, but also with more casual ones that help us maintain a positive
outlook during dark and distressing times.

Dr. Fingerman's research has also shown that people who are more
socially integrated are also more active physically. ``Being sedentary
kills you,'' she said. ``You have to get up and move to be with the
people you run into when exercising.'' Consequential strangers also help
your brain, she said, because ``conversations are more stimulating than
with people you know well.''

A fellow researcher in the field, Katherine L. Fiori, chairwoman of
undergraduate psychology at Adelphi University who studies social
networks of older adults, has found that activities that foster ``weaker
ties'' than are formed with family and close friends foster greater life
satisfaction and better emotional and physical health.

``The greater the number of weaker ties, the stronger the association
with positive feelings and fewer depressed feelings,'' Dr. Fiori said in
an interview. ``It's clearly not the case that close ties are all that
older adults need.''

And not just older adults, all adults. Dr. Fingerman said research has
shown that, in general, ``people do better when they have a more diverse
group of people in their lives.'' But as Dr. Fiori observed,
``Unfortunately, Covid has severely curtailed our ability to maintain
weaker ties. It can take a lot more effort to do this online.''

\href{https://www.nytimes.com/news-event/coronavirus?action=click\&pgtype=Article\&state=default\&region=MAIN_CONTENT_3\&context=storylines_faq}{}

\hypertarget{the-coronavirus-outbreak-}{%
\subsubsection{The Coronavirus Outbreak
›}\label{the-coronavirus-outbreak-}}

\hypertarget{frequently-asked-questions}{%
\paragraph{Frequently Asked
Questions}\label{frequently-asked-questions}}

Updated August 3, 2020

\begin{itemize}
\item ~
  \hypertarget{im-a-small-business-owner-can-i-get-relief}{%
  \paragraph{I'm a small-business owner. Can I get
  relief?}\label{im-a-small-business-owner-can-i-get-relief}}

  \begin{itemize}
  \tightlist
  \item
    The
    \href{https://www.nytimes.com/article/small-business-loans-stimulus-grants-freelancers-coronavirus.html?action=click\&pgtype=Article\&state=default\&region=MAIN_CONTENT_3\&context=storylines_faq}{stimulus
    bills enacted in March} offer help for the millions of American
    small businesses. Those eligible for aid are businesses and
    nonprofit organizations with fewer than 500 workers, including sole
    proprietorships, independent contractors and freelancers. Some
    larger companies in some industries are also eligible. The help
    being offered, which is being managed by the Small Business
    Administration, includes the Paycheck Protection Program and the
    Economic Injury Disaster Loan program. But lots of folks have
    \href{https://www.nytimes.com/interactive/2020/05/07/business/small-business-loans-coronavirus.html?action=click\&pgtype=Article\&state=default\&region=MAIN_CONTENT_3\&context=storylines_faq}{not
    yet seen payouts.} Even those who have received help are confused:
    The rules are draconian, and some are stuck sitting on
    \href{https://www.nytimes.com/2020/05/02/business/economy/loans-coronavirus-small-business.html?action=click\&pgtype=Article\&state=default\&region=MAIN_CONTENT_3\&context=storylines_faq}{money
    they don't know how to use.} Many small-business owners are getting
    less than they expected or
    \href{https://www.nytimes.com/2020/06/10/business/Small-business-loans-ppp.html?action=click\&pgtype=Article\&state=default\&region=MAIN_CONTENT_3\&context=storylines_faq}{not
    hearing anything at all.}
  \end{itemize}
\item ~
  \hypertarget{what-are-my-rights-if-i-am-worried-about-going-back-to-work}{%
  \paragraph{What are my rights if I am worried about going back to
  work?}\label{what-are-my-rights-if-i-am-worried-about-going-back-to-work}}

  \begin{itemize}
  \tightlist
  \item
    Employers have to provide
    \href{https://www.osha.gov/SLTC/covid-19/standards.html}{a safe
    workplace} with policies that protect everyone equally.
    \href{https://www.nytimes.com/article/coronavirus-money-unemployment.html?action=click\&pgtype=Article\&state=default\&region=MAIN_CONTENT_3\&context=storylines_faq}{And
    if one of your co-workers tests positive for the coronavirus, the
    C.D.C.} has said that
    \href{https://www.cdc.gov/coronavirus/2019-ncov/community/guidance-business-response.html}{employers
    should tell their employees} -\/- without giving you the sick
    employee's name -\/- that they may have been exposed to the virus.
  \end{itemize}
\item ~
  \hypertarget{should-i-refinance-my-mortgage}{%
  \paragraph{Should I refinance my
  mortgage?}\label{should-i-refinance-my-mortgage}}

  \begin{itemize}
  \tightlist
  \item
    \href{https://www.nytimes.com/article/coronavirus-money-unemployment.html?action=click\&pgtype=Article\&state=default\&region=MAIN_CONTENT_3\&context=storylines_faq}{It
    could be a good idea,} because mortgage rates have
    \href{https://www.nytimes.com/2020/07/16/business/mortgage-rates-below-3-percent.html?action=click\&pgtype=Article\&state=default\&region=MAIN_CONTENT_3\&context=storylines_faq}{never
    been lower.} Refinancing requests have pushed mortgage applications
    to some of the highest levels since 2008, so be prepared to get in
    line. But defaults are also up, so if you're thinking about buying a
    home, be aware that some lenders have tightened their standards.
  \end{itemize}
\item ~
  \hypertarget{what-is-school-going-to-look-like-in-september}{%
  \paragraph{What is school going to look like in
  September?}\label{what-is-school-going-to-look-like-in-september}}

  \begin{itemize}
  \tightlist
  \item
    It is unlikely that many schools will return to a normal schedule
    this fall, requiring the grind of
    \href{https://www.nytimes.com/2020/06/05/us/coronavirus-education-lost-learning.html?action=click\&pgtype=Article\&state=default\&region=MAIN_CONTENT_3\&context=storylines_faq}{online
    learning},
    \href{https://www.nytimes.com/2020/05/29/us/coronavirus-child-care-centers.html?action=click\&pgtype=Article\&state=default\&region=MAIN_CONTENT_3\&context=storylines_faq}{makeshift
    child care} and
    \href{https://www.nytimes.com/2020/06/03/business/economy/coronavirus-working-women.html?action=click\&pgtype=Article\&state=default\&region=MAIN_CONTENT_3\&context=storylines_faq}{stunted
    workdays} to continue. California's two largest public school
    districts --- Los Angeles and San Diego --- said on July 13, that
    \href{https://www.nytimes.com/2020/07/13/us/lausd-san-diego-school-reopening.html?action=click\&pgtype=Article\&state=default\&region=MAIN_CONTENT_3\&context=storylines_faq}{instruction
    will be remote-only in the fall}, citing concerns that surging
    coronavirus infections in their areas pose too dire a risk for
    students and teachers. Together, the two districts enroll some
    825,000 students. They are the largest in the country so far to
    abandon plans for even a partial physical return to classrooms when
    they reopen in August. For other districts, the solution won't be an
    all-or-nothing approach.
    \href{https://bioethics.jhu.edu/research-and-outreach/projects/eschool-initiative/school-policy-tracker/}{Many
    systems}, including the nation's largest, New York City, are
    devising
    \href{https://www.nytimes.com/2020/06/26/us/coronavirus-schools-reopen-fall.html?action=click\&pgtype=Article\&state=default\&region=MAIN_CONTENT_3\&context=storylines_faq}{hybrid
    plans} that involve spending some days in classrooms and other days
    online. There's no national policy on this yet, so check with your
    municipal school system regularly to see what is happening in your
    community.
  \end{itemize}
\item ~
  \hypertarget{is-the-coronavirus-airborne}{%
  \paragraph{Is the coronavirus
  airborne?}\label{is-the-coronavirus-airborne}}

  \begin{itemize}
  \tightlist
  \item
    The coronavirus
    \href{https://www.nytimes.com/2020/07/04/health/239-experts-with-one-big-claim-the-coronavirus-is-airborne.html?action=click\&pgtype=Article\&state=default\&region=MAIN_CONTENT_3\&context=storylines_faq}{can
    stay aloft for hours in tiny droplets in stagnant air}, infecting
    people as they inhale, mounting scientific evidence suggests. This
    risk is highest in crowded indoor spaces with poor ventilation, and
    may help explain super-spreading events reported in meatpacking
    plants, churches and restaurants.
    \href{https://www.nytimes.com/2020/07/06/health/coronavirus-airborne-aerosols.html?action=click\&pgtype=Article\&state=default\&region=MAIN_CONTENT_3\&context=storylines_faq}{It's
    unclear how often the virus is spread} via these tiny droplets, or
    aerosols, compared with larger droplets that are expelled when a
    sick person coughs or sneezes, or transmitted through contact with
    contaminated surfaces, said Linsey Marr, an aerosol expert at
    Virginia Tech. Aerosols are released even when a person without
    symptoms exhales, talks or sings, according to Dr. Marr and more
    than 200 other experts, who
    \href{https://academic.oup.com/cid/article/doi/10.1093/cid/ciaa939/5867798}{have
    outlined the evidence in an open letter to the World Health
    Organization}.
  \end{itemize}
\end{itemize}

When Covid-19 descended with a fury on New York City, many people I knew
who had second homes ``escaped'' the city in hopes of avoiding the
virus. I, on the other hand, chose to stay in my Brooklyn neighborhood
where everyday I encountered people I knew casually as well as others in
my extended network of friends and acquaintances I'd made at the Y, in
local stores and when walking and cycling in Prospect Park.

In my country house, especially during the dark cold days of early
spring, I would have been far more isolated. Yes, I could walk my dog
and ride my bike without having to wear a mask because I would have met
almost no one else on route. But I would also have been deprived of
conversations with the many ``consequential strangers'' I encountered
daily during my outdoor excursions in Brooklyn, including the 7 p.m.
``shout-out'' in support of our essential workers.

To counter the loneliness and maintain her many casual connections, one
of my Y buddies started a group email that not only filled in for the
daily conversations she was missing but also gave her an ongoing support
system when faced with an injury and struggling with doom-and-gloom
isolation.

In their book, Ms. Blau and Dr. Fingerman emphasize the importance of
creating and being in environments that foster relationships with
consequential strangers. Decades ago when The New York Times erected
cubicles for its writers and editors, it destroyed an environment that
was conducive to sharing information and fostering camaraderie,
prompting me to work from home most days and save the time and effort
needed to dress for work and commute. I suspect that when Covid
limitations are finally lifted, many more office workers will do the
same and sacrifice casual work-based relationships.

As the authors wrote, ``Where we live, work, shop, and mingle has
everything to do with the weak ties we cultivate, and therefore our
quality of life.'' As they described a central theme of their book,
``Casual acquaintances inspire us to venture beyond our comfort zones.''
And until we do, we'll never know what we might gain from relationships
with ``people who don't seem to matter.''

Advertisement

\protect\hyperlink{after-bottom}{Continue reading the main story}

\hypertarget{site-index}{%
\subsection{Site Index}\label{site-index}}

\hypertarget{site-information-navigation}{%
\subsection{Site Information
Navigation}\label{site-information-navigation}}

\begin{itemize}
\tightlist
\item
  \href{https://help.nytimes.com/hc/en-us/articles/115014792127-Copyright-notice}{©~2020~The
  New York Times Company}
\end{itemize}

\begin{itemize}
\tightlist
\item
  \href{https://www.nytco.com/}{NYTCo}
\item
  \href{https://help.nytimes.com/hc/en-us/articles/115015385887-Contact-Us}{Contact
  Us}
\item
  \href{https://www.nytco.com/careers/}{Work with us}
\item
  \href{https://nytmediakit.com/}{Advertise}
\item
  \href{http://www.tbrandstudio.com/}{T Brand Studio}
\item
  \href{https://www.nytimes.com/privacy/cookie-policy\#how-do-i-manage-trackers}{Your
  Ad Choices}
\item
  \href{https://www.nytimes.com/privacy}{Privacy}
\item
  \href{https://help.nytimes.com/hc/en-us/articles/115014893428-Terms-of-service}{Terms
  of Service}
\item
  \href{https://help.nytimes.com/hc/en-us/articles/115014893968-Terms-of-sale}{Terms
  of Sale}
\item
  \href{https://spiderbites.nytimes.com}{Site Map}
\item
  \href{https://help.nytimes.com/hc/en-us}{Help}
\item
  \href{https://www.nytimes.com/subscription?campaignId=37WXW}{Subscriptions}
\end{itemize}
