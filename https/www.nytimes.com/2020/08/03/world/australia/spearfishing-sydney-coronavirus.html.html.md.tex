Sections

SEARCH

\protect\hyperlink{site-content}{Skip to
content}\protect\hyperlink{site-index}{Skip to site index}

\href{/section/world/australia}{Australia}\textbar{}Taking a Spear Into
the Sea, and Washing Anxiety Away

\url{https://nyti.ms/3i6iNpF}

\begin{itemize}
\item
\item
\item
\item
\item
\end{itemize}

\href{https://www.nytimes.com/spotlight/at-home?action=click\&pgtype=Article\&state=default\&region=TOP_BANNER\&context=at_home_menu}{At
Home}

\begin{itemize}
\tightlist
\item
  \href{https://www.nytimes.com/2020/08/03/well/family/the-benefits-of-talking-to-strangers.html?action=click\&pgtype=Article\&state=default\&region=TOP_BANNER\&context=at_home_menu}{Talk:
  To Strangers}
\item
  \href{https://www.nytimes.com/2020/08/01/at-home/coronavirus-make-pizza-on-a-grill.html?action=click\&pgtype=Article\&state=default\&region=TOP_BANNER\&context=at_home_menu}{Make:
  Grilled Pizza}
\item
  \href{https://www.nytimes.com/2020/07/31/arts/television/goldbergs-abc-stream.html?action=click\&pgtype=Article\&state=default\&region=TOP_BANNER\&context=at_home_menu}{Watch:
  'The Goldbergs'}
\item
  \href{https://www.nytimes.com/interactive/2020/at-home/even-more-reporters-editors-diaries-lists-recommendations.html?action=click\&pgtype=Article\&state=default\&region=TOP_BANNER\&context=at_home_menu}{Explore:
  Reporters' Google Docs}
\end{itemize}

\includegraphics{https://static01.nyt.com/images/2020/07/31/world/00spearfishing-dispatch-1/merlin_174989472_4a4fd234-dfc0-40c6-b035-8f8d070f9c16-articleLarge.jpg?quality=75\&auto=webp\&disable=upscale}

Sydney Dispatch

\hypertarget{taking-a-spear-into-the-sea-and-washing-anxiety-away}{%
\section{Taking a Spear Into the Sea, and Washing Anxiety
Away}\label{taking-a-spear-into-the-sea-and-washing-anxiety-away}}

I kept seeing people in Sydney carry spearguns to and from the ocean. To
understand why, I held my breath and dived in.

Emma Shearman spearfishing off Manly, Australia, in
July.Credit...Michaela Skovranova for The New York Times

Supported by

\protect\hyperlink{after-sponsor}{Continue reading the main story}

\href{https://www.nytimes.com/by/damien-cave}{\includegraphics{https://static01.nyt.com/images/2018/10/08/multimedia/author-damien-cave/author-damien-cave-thumbLarge.png}}

By \href{https://www.nytimes.com/by/damien-cave}{Damien Cave}

\begin{itemize}
\item
  Aug. 3, 2020
\item
  \begin{itemize}
  \item
  \item
  \item
  \item
  \item
  \end{itemize}
\end{itemize}

SYDNEY, Australia --- Emma Shearman held her speargun and focused on her
breathing. In, out, relax, she thought. Deep and steady, as rhythmic as
the waves.

She plunged into the cold Pacific off Sydney's rocky coast, holding her
breath until she reached a depth of about 30 feet. Quiet and calm, she
lifted the gun, aimed and fired --- spearing a
\href{https://www.dpi.nsw.gov.au/fishing/fish-species/species-list/red-morwong}{red
morwong} through its middle.

It was the second catch of the day. Her friend Tim Charody, who taught
her to spearfish during Australia's coronavirus lockdown, had already
caught another morwong, a common fish in these waters. But this was Ms.
Shearman's deepest dive, and she emerged proud, holding her prey by the
gills.

``There's a real courage and confidence to know that I can go out and
catch my own food and provide, and still do womanly things --- go salsa
dancing and wear heels,'' she said when we were all on land.

``It's so challenging,'' she added, ``but also meditative.''

I'd joined them early one morning out of curiosity. For months now,
since the first coronavirus lockdown, I've been seeing more and more
people carrying spearguns to and from the waters around Sydney. One day
I nearly collided with a spear-toting dad lugging Australian salmon into
our suburb, at which point I started to wonder what was going on with
all the Poseidons.

\includegraphics{https://static01.nyt.com/images/2020/07/31/world/00spearfishing-dispatch-2/merlin_174989355_09f63290-2f5a-40be-9776-ac42e92f7d1f-articleLarge.jpg?quality=75\&auto=webp\&disable=upscale}

Image

``It's so challenging,'' said Emma Shearman of spearfishing, with Mr.
Charody.Credit...Michaela Skovranova for The New York Times

Image

A school of fish at Inscription Point in Botany Bay National Park in
Sydney.Credit...Michaela Skovranova for The New York Times

Sydney has long been a city of surfboards in hatchbacks and sandy toes
on sidewalks. The ocean here is like a neighbor you see everywhere. It
appears around unexpected corners from the craggy coast and for miles
inland along a harbor shaped like an oak leaf --- as Mark Twain
\href{https://musingsofaliterarydilettante.wordpress.com/2010/03/28/the-wayward-tourist-mark-twains-adventures-in-australia-by-mark-twain/}{pointed
out} in 1897 when he called the sheets of blue ``superbly beautiful.''

All those spear guns seemed to be introducing a deeper and darker vibe.
Or so I thought.

In fact, during a time of rising unemployment and restrictions on group
sports and social gatherings, spearfishing has become an increasingly
popular escape for people seeking calm, control and sustenance far from
the anxieties of land. Spear gear has been selling out at dive shops up
and down Australia's east coast since March. Young and old, men and
women: They are all finding something for their stomachs and souls in an
act that is ancient and elemental.

``It's all about living off the ocean,'' said Robert Cooley, a lifelong
spearfisherman and the leader of the Gamay Rangers, an Aboriginal group
that helps manage and protect Botany Bay on Sydney's southern edge.
``It's a catch-your-breath type of thing, far away from the big city.''

Mr. Cooley, 53, tall, talkative and full of local lore, said his team of
a half-dozen rangers had already put their spearfishing skills to good
use. During Sydney's peak lockdown period in April, their underwater
hunting became community service. Between fish, lobsters and abalone,
they caught 3,000 meals to distribute to neighbors in need.

Image

Mr. Charody spearfishing.Credit...Michaela Skovranova for The New York
Times

Image

Ms. Shearman with a catch.Credit...Michaela Skovranova for The New York
Times

Image

Mr. Charody resurfacing.Credit...Michaela Skovranova for The New York
Times

``It was critical work,'' Mr. Cooley said. ``Some of our elders live
alone. Others couldn't leave the house.''

One day at dawn, I met him by the bay where he had speared his first
flathead as a boy. Later that morning, we joined a few rangers at the
point where James Cook landed in 1770, bringing together European and
Aboriginal cultures for the first time.

Mr. Cooley put on his wet suit around the corner from a statue of
humpback whales, an important animal for local Indigenous groups, at a
site across the bay from Sydney's commercial port with its towering
cranes.

The industrial and the traditional --- on the surface, they pressed
against each other. In the water, they disappeared. A soft blanket of
blue covered sea grass that swayed like slow-motion dancers on the stage
of a sandstone reef.

Mr. Cooley and two other rangers dived head first, using their long fins
and weight belts to help them search between rocks and under the layered
shelves of the coastal ledge.

In many places, spearfishing with scuba gear is allowed. In Australia,
it's considered cheating. The skill and joy of the sport come with a
stretching of the lungs.

Most people learn with friends, but I had taken a spearfishing class
with two 14-year-olds. Fábio Leitão, a pony-tailed instructor originally
from the Azores, taught me that if I did not suck in a huge breath or
hyperventilate, I'd be able to hold my breath for longer.

Image

Robert Cooley, left, leader of the Gamay Rangers, an Aboriginal group
that helps manage and protect Botany Bay in Sydney, with Bryce Liddell,
another ranger.~Credit...Michaela Skovranova for The New York Times

Image

Ms. Shearman gutting a successful catch.Credit...Michaela Skovranova for
The New York Times

Image

The Dive Spear and Sport shop in Sydney.Credit...Matthew Abbott for The
New York Times

So when one of the rangers waved me over to a lobster, I was ready. I
steadied my breathing and pushed myself down, holding as long as I
could.

The little guy had lodged himself in a tight spot. I tried to grab him
and failed --- there'd be no
\href{https://www.instagram.com/p/CDGWKvipsWg/}{Instagram bragging} from
me, I caught nothing on my reporting trips --- but with a few more
dives, the rangers pried him loose.

``My supermarket is safer than the one you're going to,'' Mr. Cooley
told me.

In terms of the coronavirus, he was right, of course. But spearfishing
is hardly risk free.

Sharks are lazy bullies that grab fish after they've been shot. Shallow
water blackouts --- fainting underwater --- can lead to drowning if
there is no one around to help. That seems to be what happened to
\href{https://au.sports.yahoo.com/alex-chumpy-pullin-death-shallow-water-blackout-explained-020035708.html}{Alex
``Chumpy'' Pullin}, 32, an Australian Olympic snowboarder who died while
spearfishing alone in early July.

And yet, along with such dangers come benefits. In Sydney, edible fish
can be found just a few feet down, and spearfishing is the most
sustainable form of fishing, with no lures left behind and no by-catch
from nets. Many ``spearos,'' as they are called, take great pride in
being able to feed their families with their kill.

At Adreno in Sydney, Australia's largest spearfishing retailer, one of
the employees, Jayden Nightingale, 22, told me he goes out three times a
week and would soon be in the water to catch a feast for his brother's
birthday. ``My mother asked for octopus,'' he said.

Image

Fabio Leitao, center, a professional spearfishing instructor, giving a
spearfishing~ lesson to~ Clay Warner, left, and Heath Jones at Dive
Spear and Sport.Credit...Matthew Abbott for The New York Times

Image

Mr. Leitao, center, holding the two students in place during a breathing
and relaxation lesson.~Credit...Matthew Abbott for The New York Times

Image

Mr. Leitao teaching an emergency survival technique.Credit...Matthew
Abbott for The New York Times

Feeding others, he added, was only part of the appeal. Sitting beside me
as I tried on fins, he dialed down his sales clerk ebullience.

``I was in a coma for three months,'' he said. ``When I came out of it,
all I wanted to do was get in the ocean.''

He described a bad car accident, a head injury --- he put my hand to the
wound --- then a bout of depression that only water could cure.

Kimi Werner, a champion spearfisher from Hawaii,
\href{https://waterpeoplepodcast.com/2019/07/09/kimi-werner-flipping-your-instincts/}{often
speaks} about feeling hugged by the ocean --- the pressure on her chest,
the peace she feels looking up to the sun from deep below. Mr.
Nightingale told me the entire experience amounted to therapy.

``The ocean is like a different world,'' he said, surrounded by shelves
picked clean by the recent surge of spearfishing interest. ``It's
relaxing because you get to be one with nature.''

Ms. Shearman, 25, described a similar feeling, with a twist --- a
bending of time.

We'd met at sunrise at the top of a cliff in Manly, a seaside suburb to
the north, where our little group hiked down a treacherous path to a
rugged outcrop. The water was chilly, the swell was large, and we saw a
few small sharks along with a
\href{http://www.dpi.nsw.gov.au/__data/assets/pdf_file/0004/264775/Identifying-sharks-and-rays.pdf}{bull
ray} big enough to cover a king-size bed. By the time we got out of the
water, more than three hours had passed in a blur.

Asked later what she thinks about during all that time in the water, Ms.
Shearman replied: ``I actually don't think about anything. It's not like
running, where you think about ideas or things you want to do --- you're
just there.''

In such uncertain times on land, that alone draws many of us to the sea.

Image

A spearfishing spot at Inscription Point, Botany Bay, in
Sydney.Credit...Michaela Skovranova for The New York Times

Advertisement

\protect\hyperlink{after-bottom}{Continue reading the main story}

\hypertarget{site-index}{%
\subsection{Site Index}\label{site-index}}

\hypertarget{site-information-navigation}{%
\subsection{Site Information
Navigation}\label{site-information-navigation}}

\begin{itemize}
\tightlist
\item
  \href{https://help.nytimes.com/hc/en-us/articles/115014792127-Copyright-notice}{©~2020~The
  New York Times Company}
\end{itemize}

\begin{itemize}
\tightlist
\item
  \href{https://www.nytco.com/}{NYTCo}
\item
  \href{https://help.nytimes.com/hc/en-us/articles/115015385887-Contact-Us}{Contact
  Us}
\item
  \href{https://www.nytco.com/careers/}{Work with us}
\item
  \href{https://nytmediakit.com/}{Advertise}
\item
  \href{http://www.tbrandstudio.com/}{T Brand Studio}
\item
  \href{https://www.nytimes.com/privacy/cookie-policy\#how-do-i-manage-trackers}{Your
  Ad Choices}
\item
  \href{https://www.nytimes.com/privacy}{Privacy}
\item
  \href{https://help.nytimes.com/hc/en-us/articles/115014893428-Terms-of-service}{Terms
  of Service}
\item
  \href{https://help.nytimes.com/hc/en-us/articles/115014893968-Terms-of-sale}{Terms
  of Sale}
\item
  \href{https://spiderbites.nytimes.com}{Site Map}
\item
  \href{https://help.nytimes.com/hc/en-us}{Help}
\item
  \href{https://www.nytimes.com/subscription?campaignId=37WXW}{Subscriptions}
\end{itemize}
