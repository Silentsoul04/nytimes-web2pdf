Sections

SEARCH

\protect\hyperlink{site-content}{Skip to
content}\protect\hyperlink{site-index}{Skip to site index}

\href{https://www.nytimes.com/section/world}{World}

\href{https://myaccount.nytimes.com/auth/login?response_type=cookie\&client_id=vi}{}

\href{https://www.nytimes.com/section/todayspaper}{Today's Paper}

\href{/section/world}{World}\textbar{}Fauci Supports Birx's Coronavirus
Assessment After Trump Criticizes Her

\url{https://nyti.ms/3hZXNRC}

\begin{itemize}
\item
\item
\item
\item
\item
\item
\end{itemize}

\href{https://www.nytimes.com/news-event/coronavirus?action=click\&pgtype=Article\&state=default\&region=TOP_BANNER\&context=storylines_menu}{The
Coronavirus Outbreak}

\begin{itemize}
\tightlist
\item
  live\href{https://www.nytimes.com/2020/08/04/world/coronavirus-covid-19.html?action=click\&pgtype=Article\&state=default\&region=TOP_BANNER\&context=storylines_menu}{Latest
  Updates}
\item
  \href{https://www.nytimes.com/interactive/2020/us/coronavirus-us-cases.html?action=click\&pgtype=Article\&state=default\&region=TOP_BANNER\&context=storylines_menu}{Maps
  and Cases}
\item
  \href{https://www.nytimes.com/interactive/2020/science/coronavirus-vaccine-tracker.html?action=click\&pgtype=Article\&state=default\&region=TOP_BANNER\&context=storylines_menu}{Vaccine
  Tracker}
\item
  \href{https://www.nytimes.com/2020/08/02/us/covid-college-reopening.html?action=click\&pgtype=Article\&state=default\&region=TOP_BANNER\&context=storylines_menu}{College
  Reopening}
\item
  \href{https://www.nytimes.com/live/2020/08/03/business/stock-market-today-coronavirus?action=click\&pgtype=Article\&state=default\&region=TOP_BANNER\&context=storylines_menu}{Economy}
\end{itemize}

Advertisement

\protect\hyperlink{after-top}{Continue reading the main story}

Supported by

\protect\hyperlink{after-sponsor}{Continue reading the main story}

\hypertarget{fauci-supports-birxs-coronavirus-assessment-after-trump-criticizes-her}{%
\section{Fauci Supports Birx's Coronavirus Assessment After Trump
Criticizes
Her}\label{fauci-supports-birxs-coronavirus-assessment-after-trump-criticizes-her}}

Counting for the 2020 census will end on Sept. 30, a month earlier than
previously announced, the Census Bureau said. Some U.S. schools have
begun to reopen, with fraught results.

\begin{itemize}
\item
  Published Aug. 3, 2020Updated Aug. 4, 2020, 5:36 a.m. ET
\item
  \begin{itemize}
  \item
  \item
  \item
  \item
  \item
  \item
  \end{itemize}
\end{itemize}

This briefing has ended. Read live coronavirus updates
\href{https://www.nytimes.com/2020/08/04/world/coronavirus-covid-19.html}{here}.

\hypertarget{heres-what-you-need-to-know}{%
\subsubsection{Here's what you need to
know:}\label{heres-what-you-need-to-know}}

\begin{itemize}
\tightlist
\item
  \protect\hyperlink{link-4547638f}{Fauci defends Birx after she is
  criticized by Trump.}
\item
  \protect\hyperlink{link-15e7f995}{Trump derides Democrats as lawmakers
  and administration officials try to break stimulus impasse.}
\item
  \protect\hyperlink{link-e5a2cda}{The deadline for 2020 census counting
  has been moved up by a month.}
\item
  \protect\hyperlink{link-4c85ed64}{As some students and teachers go
  back to school in the U.S., they're bringing the virus with them.}
\item
  \protect\hyperlink{link-5ccc012}{Hurricane Isaias makes landfall in
  North Carolina, as officials across the Southeast scramble.}
\item
  \protect\hyperlink{link-6bfd36ea}{Italian sex workers face poverty and
  illness during the pandemic.}
\item
  \protect\hyperlink{link-bfeb498}{New Zealand newlyweds, stranded in
  the Falkland Islands, went home on a fishing boat.}
\end{itemize}

\includegraphics{https://static01.nyt.com/images/2020/08/03/us/politics/03virus-briefing-fauci/merlin_175155375_a565045c-e0d8-4c45-96ec-718c6bf140c0-articleLarge.jpg?quality=75\&auto=webp\&disable=upscale}

\hypertarget{fauci-defends-birx-after-she-is-criticized-by-trump}{%
\subsection{Fauci defends Birx after she is criticized by
Trump.}\label{fauci-defends-birx-after-she-is-criticized-by-trump}}

Dr. Anthony S. Fauci, the nation's top infectious disease specialist,
agreed on Monday with his colleague Dr. Deborah Birx that the United
States has entered a ``new phase'' of the coronavirus pandemic, in which
the virus is now spreading uncontrolled in some states by asymptomatic
people --- comments that drew fire from President Trump.

Dr. Fauci said Dr. Birx had been referring to the ``inherent community
spread'' that is occurring in some states, adding: ``When you have
community spread, it's much more difficult to get your arms around that
and contain it.''

Speaking during a news conference with Gov. Ned Lamont of Connecticut,
Dr. Fauci called the community spread ``insidious'' and noted that it
was happening outside of confined spaces like nursing homes and prisons.

In backing up Dr. Birx, the Trump administration's coronavirus response
coordinator, Dr. Fauci indirectly put himself at odds with the
president. Earlier on Monday, Mr. Trump had called Dr. Birx ``pathetic''
on Twitter and suggested that her comments about a ``new phase'' were an
effort to curry favor with Speaker Nancy Pelosi.

At an evening news conference, Mr. Trump appeared to temper his comments
about Dr. Birx. ``She's a person I have a lot of respect for,'' he said,
while defending his administration's response to the virus.

Former Vice President Joseph R. Biden Jr.
\href{https://twitter.com/JoeBiden/status/1290350721515139072}{responded
to Mr. Trump in a tweet} on Monday afternoon. ``It's hard to believe
this has to be said, but if I'm elected president, I'll spend my Monday
mornings working with our nation's top experts to control this virus ---
not insulting them on Twitter,'' Mr. Biden said.

But other Republicans piled on. ``Dr. Birx, like Dr. Fauci, has been
wrong much more than she has been right on COVID-19, \& their
destructive prescriptions have led to the devastation of countless
American lives,'' Representative Andy Biggs, Republican of Arizona,
\href{https://twitter.com/RepAndyBiggsAZ/status/1290297517582610433?s=20}{wrote
on Twitter}.

Dr. Birx had warned during an appearance on the CNN program ``State of
the Union'' on Sunday that the United States was entering a ``new
phase'' in its fight against the pandemic, and that rural communities
would not be spared. ``It is extraordinarily widespread,'' she said.

On Monday morning, shortly after Mr. Trump tweeted about her, Dr. Birx
told governors on a weekly briefing call that a lack of masks at large
gatherings in homes was ``a critical issue,'' pointing to spikes in many
Southern states.

Mr. Trump has also criticized Dr. Fauci, despite his claims that the two
have a ``very good relationship.'' In
\href{https://twitter.com/realDonaldTrump/status/1289633359681839105}{a
tweet on Saturday} responding to news reports that Dr. Fauci had linked
the recent surge in cases to inadequate lockdowns, Mr. Trump tweeted:
``Wrong!''

In a
\href{https://www.youtube.com/watch?v=8PgmAWgiL1A\&feature=youtu.be}{livestreamed
conversation} on Monday with Dr. Howard Bauchner, the editor in chief of
The Journal of the American Medical Association, Dr. Fauci said that as
communities around the country struggle to decide whether and how to
reopen schools, scientists ``really need to be humble.''

There are significant gaps in knowledge about how likely children are to
contract the coronavirus, become ill and transmit the disease, Dr. Fauci
said. A
\href{https://www.nytimes.com/2020/07/30/health/coronavirus-children.html?searchResultPosition=1}{recent
study} found that young children carried high levels of the virus in
their noses and throats, for example, but did not prove they were
contagious.

Dr. Fauci said he hoped some questions about the risks to children and
their role in transmission would be answered by a new
\href{https://www.nih.gov/news-events/news-releases/study-determine-incidence-novel-coronavirus-infection-us-children-begins}{government
study} that involves 6,000 people and seeks to find the rate of
infection in children and their families.

\href{https://www.nytimes.com/interactive/2020/us/coronavirus-us-cases.html}{Tracking
the Coronavirus~›}

\href{https://www.nytimes.com/interactive/2020/us/coronavirus-us-cases.html}{}

\hypertarget{where-cases-are-rising-fastest}{%
\subsubsection{\texorpdfstring{Where cases are \textbf{rising}
fastest}{Where cases are rising fastest}}\label{where-cases-are-rising-fastest}}

\href{https://www.nytimes.com/interactive/2020/us/hawaii-coronavirus-cases.html}{}

Hawaii

\href{https://www.nytimes.com/interactive/2020/us/rhode-island-coronavirus-cases.html}{}

R.I.

\href{https://www.nytimes.com/interactive/2020/us/new-jersey-coronavirus-cases.html}{}

N.J.

\href{https://www.nytimes.com/interactive/2020/us/massachusetts-coronavirus-cases.html}{}

Mass.

\href{https://www.nytimes.com/interactive/2020/us/nebraska-coronavirus-cases.html}{}

Neb.

\href{https://www.nytimes.com/interactive/2020/us/missouri-coronavirus-cases.html}{}

Mo.

\href{https://www.nytimes.com/interactive/2020/us/alaska-coronavirus-cases.html}{}

Alaska

\href{https://www.nytimes.com/interactive/2020/us/oklahoma-coronavirus-cases.html}{}

Okla.

\href{https://www.nytimes.com/interactive/2020/us/south-dakota-coronavirus-cases.html}{}

S.D.

\href{https://www.nytimes.com/interactive/2020/us/new-hampshire-coronavirus-cases.html}{}

N.H.

\href{https://www.nytimes.com/interactive/2020/us/illinois-coronavirus-cases.html}{}

Ill.

\href{https://www.nytimes.com/interactive/2020/us/montana-coronavirus-cases.html}{}

Mont.

\href{https://www.nytimes.com/interactive/2020/us/coronavirus-us-cases.html}{}

\hypertarget{us-hot-spots-}{%
\subsubsection{U.S. hot spots~›}\label{us-hot-spots-}}

\includegraphics{https://static01.nyt.com/newsgraphics/2020/03/16/coronavirus-maps/65bf0c6883e2d9d68afa51d181e1fa9779cd7420/images/orphan_usa-threeByTwoSmallAt2X.png}

\href{https://www.nytimes.com/interactive/2020/world/coronavirus-maps.html}{}

\hypertarget{worldwide-}{%
\subsubsection{Worldwide~›}\label{worldwide-}}

\includegraphics{https://static01.nyt.com/newsgraphics/2020/03/16/coronavirus-maps/65bf0c6883e2d9d68afa51d181e1fa9779cd7420/images/orphan_world-threeByTwoSmallAt2X.png}

\hypertarget{trump-derides-democrats-as-lawmakers-and-administration-officials-try-to-break-stimulus-impasse}{%
\subsection{Trump derides Democrats as lawmakers and administration
officials try to break stimulus
impasse.}\label{trump-derides-democrats-as-lawmakers-and-administration-officials-try-to-break-stimulus-impasse}}

Image

President Trump shows a map of the coronavirus outbreak in the United
States during an executive order signing ceremony on hiring Americans at
the White House on Monday.Credit...Doug Mills/The New York Times

Mr. Trump on Monday hurled insults at Democratic leaders who were
huddling with his top advisers in search of a compromise economic
recovery package, threatening to act on his own to ban evictions as he
again undercut negotiations to reach a broader deal.

Mr. Trump floated the possibility of using an executive order to address
an expired federal moratorium on evictions, even though a \$1 trillion
Republican aid proposal did not include such a pause. He said he
remained ``totally involved'' in stimulus talks, even though he wasn't
``over there with Crazy Nancy,'' a reference to Speaker Nancy Pelosi of
California.

But the president has been notably absent from the negotiations on a
sweeping economic stabilization package, even as tens of millions of
Americans have been cut off from enhanced jobless benefits they have
depended on for months during the coronavirus pandemic.

At the same moment that Mr. Trump was blasting her, Ms. Pelosi met on
Capitol Hill with Senator Chuck Schumer of New York, the minority
leader, Mark Meadows, the White House chief of staff, and Steven
Mnuchin, the Treasury secretary, in search of a compromise. It was the
fifth such meeting in eight days, following a staff policy call on
Sunday and a rare Saturday session with the four negotiators.

At the White House, Mr. Trump accused Democrats of being single-mindedly
focused on getting ``bailout money'' for states controlled by Democrats,
and unconcerned with extending unemployment benefits.

``All they're really interested in is bailout money to bail out radical
left governors and radical left mayors like in Portland and places that
are so badly run --- Chicago, New York City,'' Mr. Trump said.

Democrats have proposed providing more than \$900 billion to
cash-strapped states and cities whose budgets have been devastated in
the recession, but it is Republicans who have proposed slashing the
jobless aid. Democrats have refused to do so, feeding the stalemate.

While White House officials and Democratic leaders reported some
progress over the weekend in their talks, they
\href{https://www.nytimes.com/2020/08/02/us/politics/coronavirus-jobless-aid.html}{still
have substantial differences}. Democrats are pushing a \$3 trillion
rescue plan that would include restoring \$600-per-week jobless aid
payments that expired on Friday and extending them through January,
while Republicans have proposed a \$1 trillion package that would slash
the unemployment payments considerably.

More than 47,800 new cases and more than 600 new deaths were reported in
the United States on Monday.

\hypertarget{the-deadline-for-2020-census-counting-has-been-moved-up-by-a-month}{%
\subsection{The deadline for 2020 census counting has been moved up by a
month.}\label{the-deadline-for-2020-census-counting-has-been-moved-up-by-a-month}}

Image

Census data is used to divvy up trillions of dollars in federal
aid.Credit...Gabriele Holtermann/Sipa, via Associated Press

Counting for the 2020 census will end on Sept. 30, a month earlier than
previously scheduled, the Census Bureau said in a statement on Monday.

\hypertarget{latest-updates-global-coronavirus-outbreak}{%
\section{\texorpdfstring{\href{https://www.nytimes.com/2020/08/04/world/coronavirus-covid-19.html?action=click\&pgtype=Article\&state=default\&region=MAIN_CONTENT_1\&context=storylines_live_updates}{Latest
Updates: Global Coronavirus
Outbreak}}{Latest Updates: Global Coronavirus Outbreak}}\label{latest-updates-global-coronavirus-outbreak}}

Updated 2020-08-04T09:15:14.275Z

\begin{itemize}
\tightlist
\item
  \href{https://www.nytimes.com/2020/08/04/world/coronavirus-covid-19.html?action=click\&pgtype=Article\&state=default\&region=MAIN_CONTENT_1\&context=storylines_live_updates\#link-6b644638}{`Long
  days, long nights': Washington prepares for a prolonged fight over
  virus relief.}
\item
  \href{https://www.nytimes.com/2020/08/04/world/coronavirus-covid-19.html?action=click\&pgtype=Article\&state=default\&region=MAIN_CONTENT_1\&context=storylines_live_updates\#link-7af9fca0}{Israel's
  rocky reopening of its schools may be a lesson for the U.S.}
\item
  \href{https://www.nytimes.com/2020/08/04/world/coronavirus-covid-19.html?action=click\&pgtype=Article\&state=default\&region=MAIN_CONTENT_1\&context=storylines_live_updates\#link-33bf9168}{Hurricane
  Isaias arrives in North Carolina as officials along the East Coast
  scramble.}
\end{itemize}

\href{https://www.nytimes.com/2020/08/04/world/coronavirus-covid-19.html?action=click\&pgtype=Article\&state=default\&region=MAIN_CONTENT_1\&context=storylines_live_updates}{See
more updates}

More live coverage:
\href{https://www.nytimes.com/live/2020/08/03/business/stock-market-today-coronavirus?action=click\&pgtype=Article\&state=default\&region=MAIN_CONTENT_1\&context=storylines_live_updates}{Markets}

The census is constitutionally required to count all residents of the
United States every 10 years, but the 2020 effort has
\href{https://www.nytimes.com/2020/04/18/us/coronavirus-census.html}{faltered}
amid the pandemic. In recent weeks, the Trump administration and Senate
Republicans
\href{https://www.nytimes.com/2020/07/28/us/trump-census.html}{appeared
to signal that they wanted the census finished well ahead of schedule}.

Census data is enormously important. It is used to reapportion all 435
House seats and thousands of state and local districts, as well as divvy
up trillions of dollars in federal aid.

``Under this plan, the Census Bureau intends to meet a similar level of
household responses as collected in prior censuses, including outreach
to hard-to-count communities,'' the Census Bureau said in its
\href{https://www.census.gov/newsroom/press-releases/2020/delivering-complete-accurate-count.html}{statement}.

Critics said the move was pushed by the White House and motivated by
partisanship.

``We're dealing with a census that's been really challenged by
Covid-19,'' said Vanita Gupta, a former head of the Justice Department's
civil rights division who is now the
\href{https://civilrights.org/about/our-staff/vanita-gupta/}{president
of the Leadership Conference on Civil and Human Rights}. ``And in the
middle of this pandemic, the administration has tried to sabotage the
census for partisan gain, to move its anti-immigrant agenda and to
silence communities of color.''

She added that rural communities could be badly hurt by an undercount.

On Monday night, the White House referred questions to the Commerce
Department, which oversees the Census Bureau. It did not immediately
respond to a request for comment.

In 2010, census takers worked from May to August to count hard-to-find
households. This spring, the bureau said it was pushing back the start
to August, ending on Oct. 31.

The population totals, required to reapportion the House of
Representatives, are traditionally delivered to the president on Dec.
31, but this year the bureau had asked Congress for a
\href{https://www.nytimes.com/2020/04/13/us/census-coronavirus-delay.html?searchResultPosition=9}{four-month
extension of the statutory deadline}. The White House backed the
extension at the time. The House approved the delay; the Senate has not.

Congress could still act to extend census statutory deadlines as part of
the next
\href{https://www.nytimes.com/2020/08/02/us/politics/coronavirus-jobless-aid.html}{coronavirus
relief package}.

Education Roundup

\hypertarget{as-some-students-and-teachers-go-back-to-school-in-the-us-theyre-bringing-the-virus-with-them}{%
\subsection{As some students and teachers go back to school in the U.S.,
they're bringing the virus with
them.}\label{as-some-students-and-teachers-go-back-to-school-in-the-us-theyre-bringing-the-virus-with-them}}

Image

Students arrive to Dallas Elementary School for the first day of school
in Dallas, Ga., on Monday.Credit...Brynn Anderson/Associated Press

The new academic year is underway in some parts of the United States,
with the first few days of school showing just how fraught reopening
classrooms can be. Already in some states, schools that decided to open
for in-person classes are quarantining staff members and students, and
even closing temporarily as positive cases are found.

Traditionally,
\href{https://www.pewresearch.org/fact-tank/2019/08/14/back-to-school-dates-u-s/}{about
14 percent of the nation's children} go back to school by the second
week of August, mostly in the South and Midwest, although this year,
some districts in those areas have postponed classes by a week or two,
or plan to start the year online.

Many schools in Indiana started on Thursday. On Saturday, the
superintendent of the Elwood Community School Corporation in the central
part of the state sent a note thanking students and parents for
\href{https://www.facebook.com/ElwoodCommunitySchools/photos/pcb.1875944365880857/1875944205880873/?type=3\&theater}{``a
great first two days of school!''}

But the optimistic tone quickly gave way: Staff members had tested
positive, and the high school was forced to close its doors and move all
students in seventh through 12th grades to online learning for at least
a week.

And similar developments occurred across the country. Just hours into
the first day of classes at
\href{https://www.nytimes.com/2020/08/01/us/schools-reopening-indiana-coronavirus.html}{Greenfield
Central Junior High School}, also in Indiana, the county health
department notified the school that a student had tested positive. The
student was isolated, and others who had been in proximity were forced
to quarantine for two weeks.

At a high school in Corinth, Miss.,
\href{https://www.facebook.com/corinthschooldistrict/?tn-str=k*F}{someone
also tested positive} during the first week back, and exposed students
there were asked to stay home for 14 days. And in the Atlanta area,
\href{https://www.ajc.com/news/atlanta-news/covid-cases-exposure-have-260-gwinnett-school-employees-not-working/RVZP4UFBPFHDNJJ73MNUFIKEPY/}{more
than 200 employees of a single school district} in Gwinnett County
tested positive or were in quarantine last week before classes even
resumed.

Gwinnett County Public Schools is the largest school system in Georgia,
with more than 180,000 students. Teachers returned to work last
Wednesday, in preparation for starting classes remotely on Aug. 12. But
as of Thursday, about 260 employees had been excluded from work because
they tested positive or had potentially been exposed to the virus.

Other key developments in education:

\begin{itemize}
\item
  Gov. Larry Hogan of \textbf{Maryland} on Monday
  \href{https://twitter.com/GovLarryHogan/status/1290330304830246912}{issued
  an emergency order} counteracting Montgomery County's health
  department, which on Friday said that all private schools needed to
  \href{https://www.washingtonpost.com/local/education/montgomery-county-health-officials-tell-private-schools-to-start-school-online/2020/08/01/64552b9e-d3fd-11ea-9038-af089b63ac21_story.html}{start
  the year remotely} in the fall, just as public schools in the region
  plan to.
\item
  \textbf{In New Jersey}, face coverings will be required for all
  students inside a school building, unless doing so would adversely
  affect a student's health, the governor said.
\item
  \textbf{The University of North Carolina at Chapel Hill} is planning
  to fully reopen next week, but 30 tenured faculty members wrote an
  open letter to students published Friday in The Charlotte Observer
  pushing for virtual learning and encouraging students to stay home.
\end{itemize}

\hypertarget{hurricane-isaias-makes-landfall-in-north-carolina-as-officials-across-the-southeast-scramble}{%
\subsection{Hurricane Isaias makes landfall in North Carolina, as
officials across the Southeast
scramble.}\label{hurricane-isaias-makes-landfall-in-north-carolina-as-officials-across-the-southeast-scramble}}

Image

Damage in Southport, N.C., early Tuesday as Hurricane Isaias moved
through the state.Credit...WECT-TV, via Associated Press

Hurricane Isaias made landfall in the Carolinas on Monday night, at a
time when many people in the Southeast are already beleaguered by the
coronavirus outbreak.

Disruptions in testing, in transporting samples and supplies, in and
staffing labs could complicate efforts to gauge virus transmission in
states that have struggled to contain it.

In Florida, for example, 43 state-run
\href{https://floridadisaster.org/covid19/testing-sites/}{testing sites}
were forced to close on Friday as Isaias, then a tropical storm, headed
for the state's east coast. The number of lab results
\href{http://ww11.doh.state.fl.us/comm/_partners/covid19_report_archive/state_reports_latest.pdf}{received
daily by Florida's Health Department}, which had generally been in the
90,000 range in the past two weeks, fell on Sunday to about 61,000.

Isaias made landfall in southern North Carolina late Monday night after
strengthening into a Category 1 hurricane and was expected to travel up
the East Coast. Testing sites have been closed
\href{https://patch.com/maryland/baltimore/tropical-storm-isaias-closes-14-md-coronavirus-testing-sites}{as
far north as Maryland} in anticipation.

Officials have told residents in the storm's projected path to prepare
themselves, and businesses are concerned about how much damage it will
bring.

``It's a wait-and-see game,'' said Jay Slevin, the manager of a pizzeria
about a mile from the shore in Myrtle Beach, S.C., southwest of where
Isaias made landfall.

Officials are also changing how they run shelters and advising residents
to regard them as a last resort,
\href{https://www.nytimes.com/2020/07/26/us/virus-texas-storm.html}{out
of fear that the virus could spread} in crowded indoor spaces.

At a
\href{https://www.ncdps.gov/storm-update?fbclid=IwAR3gtINKPXqQdsJixuf1kwBKbAtEMz62wzyoDHlu7FDBN1HTarDlw8FlVwQ}{briefing}
on Monday afternoon, Gov. Roy Cooper of North Carolina advised residents
in vulnerable areas to stay with friends or family or to go to a hotel.
But he added that shelters would open for those who need them, with
health screenings, social distancing and cleaning protocols.

``I know that North Carolinians have had to dig deep in recent months to
tap into our strength and resilience during the pandemic, and that
hasn't been easy,'' he said. ``But with this storm on the way, we have
to dig a little deeper.''

\hypertarget{italian-sex-workers-face-poverty-and-illness-during-the-pandemic}{%
\subsection{Italian sex workers face poverty and illness during the
pandemic.}\label{italian-sex-workers-face-poverty-and-illness-during-the-pandemic}}

In Italy, prostitution is not illegal, nor is it regulated as an
official occupation, making the country's 70,000 sex workers largely
ineligible to receive economic relief. Many have been forced to take
their chances by returning to work in order to avoid poverty.

In May, organizations that promote the rights of Italian sex workers
sought to draw the government's attention and get support, arguing that
the pandemic showed the harm of forcing sex work underground.

In March, Regina Satariano, a 60-year-old sex worker in Tuscany, started
hearing about colleagues who hadn't eaten and a landlord who had
threatened to evict a group of 17 housemates, all sex workers who were
out of work because of the pandemic.

Ms. Satariano put together her savings and bought bags of pasta, tomato
sauce, chicken and soap to distribute to her colleagues. But without
support from the state, she said, many sex workers will continue to go
hungry. If officials don't change things now, she added, ``they never
will.''

In other European countries, such as the Netherlands and Germany, sex
workers can enter formal contracts with their clients. During the
lockdown, those who were officially registered with the government were
eligible for economic relief.

Scotland also included sex workers in its relief programs. In Greece,
where prostitution is legal and regulated, brothels were allowed to
reopen on June 15, provided that sex workers kept their clients' names
and contact details for four weeks for tracing purposes.

In Italy,
\href{https://www.produzionidalbasso.com/project/covid19-nessuna-da-sola-solidarieta-immediata-alle-lavoratrici-sessuali-piu-colpite-dall-emergenza/}{various
charities and associations have raised money} for groceries, medicines,
bills and rent to benefit the country's sex workers. But for the most
part, Italian sex workers, who are often from immigrant communities,
have had to fend for themselves.

A recent report by the Sex Workers' Rights Advocacy Network and the
International Committee on the Rights of Sex Workers in Europe showed
that many sex workers defied lockdown rules in order to work, putting
both themselves and their clients at risk.

\hypertarget{new-zealand-newlyweds-stranded-in-the-falkland-islands-went-home-on-a-fishing-boat}{%
\subsection{New Zealand newlyweds, stranded in the Falkland Islands,
went home on a fishing
boat.}\label{new-zealand-newlyweds-stranded-in-the-falkland-islands-went-home-on-a-fishing-boat}}

Image

Feeonaa and Neville Clifton in late July, aboard the Antarctic fishing
vessel that took them home to New Zealand from the Falkland Islands.

A newly married couple from New Zealand who were stranded for months in
the remote Falkland Islands have managed to return home --- by catching
a ride for more than 5,000 nautical miles on an Antarctic fishing boat.

The couple, Feeonaa and Neville Clifton, were honeymooning in the south
Atlantic archipelago, about 300 miles off the coast of Argentina, as
South America's coronavirus epidemic began to escalate in March. After
their flights home via Brazil were canceled, they remained in lockdown
with an aunt in the Falklands, where Mr. Clifton was born.

The couple have been together more than 25 years and raised three
children, but decided only recently to marry and take a honeymoon. Ms.
Clifton said they spent their time in lockdown rekindling old hobbies,
like playing card games.

``I think maybe I fell in love with him just a little bit more,'' she
added of her husband.

That was the easy part.

Ms. Clifton, 48, said that when they began planning their escape from
the Falklands, one of their only options was a military transport
through Africa and Britain.

``At the time we were being told Latam might fly next month, and then
again the next month after that,'' she said, referring to Latam
Airlines, a major carrier in the region. ``Unfortunately the deadline
kept getting further and further pushed back.''

They explored other travel options, but each seemed complex and likely
to put them at increased risk of contracting the virus.

Eventually, they settled on the San Aotea II, a fishing boat that was
heading their way. The only catch was that the journey would take 29
days and traverse the notoriously treacherous Southern Ocean.

But Ms. Clifton, who had never spent a night on a boat, said the trip
was surprisingly calm, and that the crew helped pass the time by playing
cards with them.

The couple arrived in New Zealand on Tuesday morning after testing
negative for the virus. Ms. Clifton said in a telephone interview a few
hours later that they still felt ``extremely wobbly'' --- to the point
where a shopkeeper they came upon during the drive home thought they
were dancing.

``We were just trying to stand up straight,'' she said.

\hypertarget{despite-the-pandemic-facebook-leases-more-office-space-in-manhattan}{%
\subsection{Despite the pandemic, Facebook leases more office space in
Manhattan.}\label{despite-the-pandemic-facebook-leases-more-office-space-in-manhattan}}

Image

The Farley building, where Facebook has leased all the office space, was
once the main post office building in Manhattan.~Credit...Hiroko
Masuike/The New York Times

Facebook on Monday
\href{https://www.nytimes.com/2020/08/03/nyregion/facebook-nyc-office-farley-building.html}{agreed
to lease all the office space} in the mammoth James A. Farley Building
in Midtown Manhattan, reaffirming its commitment to an office-centric
urban culture despite the continued spread of the coronavirus.

The timing of the announcement was somewhat of a surprise because
Facebook has given most of its employees the
\href{https://www.nytimes.com/2020/05/21/technology/facebook-remote-work-coronavirus.html}{option
of working from home}. Even after the pandemic subsides, Facebook has
said that within the next 10 years, up to half of its roughly 52,200
employees across the country would work from home.

New York's economy has been cratered by the outbreak. The city is slowly
reopening, but many companies have told their employees not to return to
their offices until early next year, if not later. Much of Manhattan's
business district remains a virtual ghost town.

\href{https://www.nytimes.com/news-event/coronavirus?action=click\&pgtype=Article\&state=default\&region=MAIN_CONTENT_3\&context=storylines_faq}{}

\hypertarget{the-coronavirus-outbreak-}{%
\subsubsection{The Coronavirus Outbreak
›}\label{the-coronavirus-outbreak-}}

\hypertarget{frequently-asked-questions}{%
\paragraph{Frequently Asked
Questions}\label{frequently-asked-questions}}

Updated August 3, 2020

\begin{itemize}
\item ~
  \hypertarget{im-a-small-business-owner-can-i-get-relief}{%
  \paragraph{I'm a small-business owner. Can I get
  relief?}\label{im-a-small-business-owner-can-i-get-relief}}

  \begin{itemize}
  \tightlist
  \item
    The
    \href{https://www.nytimes.com/article/small-business-loans-stimulus-grants-freelancers-coronavirus.html?action=click\&pgtype=Article\&state=default\&region=MAIN_CONTENT_3\&context=storylines_faq}{stimulus
    bills enacted in March} offer help for the millions of American
    small businesses. Those eligible for aid are businesses and
    nonprofit organizations with fewer than 500 workers, including sole
    proprietorships, independent contractors and freelancers. Some
    larger companies in some industries are also eligible. The help
    being offered, which is being managed by the Small Business
    Administration, includes the Paycheck Protection Program and the
    Economic Injury Disaster Loan program. But lots of folks have
    \href{https://www.nytimes.com/interactive/2020/05/07/business/small-business-loans-coronavirus.html?action=click\&pgtype=Article\&state=default\&region=MAIN_CONTENT_3\&context=storylines_faq}{not
    yet seen payouts.} Even those who have received help are confused:
    The rules are draconian, and some are stuck sitting on
    \href{https://www.nytimes.com/2020/05/02/business/economy/loans-coronavirus-small-business.html?action=click\&pgtype=Article\&state=default\&region=MAIN_CONTENT_3\&context=storylines_faq}{money
    they don't know how to use.} Many small-business owners are getting
    less than they expected or
    \href{https://www.nytimes.com/2020/06/10/business/Small-business-loans-ppp.html?action=click\&pgtype=Article\&state=default\&region=MAIN_CONTENT_3\&context=storylines_faq}{not
    hearing anything at all.}
  \end{itemize}
\item ~
  \hypertarget{what-are-my-rights-if-i-am-worried-about-going-back-to-work}{%
  \paragraph{What are my rights if I am worried about going back to
  work?}\label{what-are-my-rights-if-i-am-worried-about-going-back-to-work}}

  \begin{itemize}
  \tightlist
  \item
    Employers have to provide
    \href{https://www.osha.gov/SLTC/covid-19/standards.html}{a safe
    workplace} with policies that protect everyone equally.
    \href{https://www.nytimes.com/article/coronavirus-money-unemployment.html?action=click\&pgtype=Article\&state=default\&region=MAIN_CONTENT_3\&context=storylines_faq}{And
    if one of your co-workers tests positive for the coronavirus, the
    C.D.C.} has said that
    \href{https://www.cdc.gov/coronavirus/2019-ncov/community/guidance-business-response.html}{employers
    should tell their employees} -\/- without giving you the sick
    employee's name -\/- that they may have been exposed to the virus.
  \end{itemize}
\item ~
  \hypertarget{should-i-refinance-my-mortgage}{%
  \paragraph{Should I refinance my
  mortgage?}\label{should-i-refinance-my-mortgage}}

  \begin{itemize}
  \tightlist
  \item
    \href{https://www.nytimes.com/article/coronavirus-money-unemployment.html?action=click\&pgtype=Article\&state=default\&region=MAIN_CONTENT_3\&context=storylines_faq}{It
    could be a good idea,} because mortgage rates have
    \href{https://www.nytimes.com/2020/07/16/business/mortgage-rates-below-3-percent.html?action=click\&pgtype=Article\&state=default\&region=MAIN_CONTENT_3\&context=storylines_faq}{never
    been lower.} Refinancing requests have pushed mortgage applications
    to some of the highest levels since 2008, so be prepared to get in
    line. But defaults are also up, so if you're thinking about buying a
    home, be aware that some lenders have tightened their standards.
  \end{itemize}
\item ~
  \hypertarget{what-is-school-going-to-look-like-in-september}{%
  \paragraph{What is school going to look like in
  September?}\label{what-is-school-going-to-look-like-in-september}}

  \begin{itemize}
  \tightlist
  \item
    It is unlikely that many schools will return to a normal schedule
    this fall, requiring the grind of
    \href{https://www.nytimes.com/2020/06/05/us/coronavirus-education-lost-learning.html?action=click\&pgtype=Article\&state=default\&region=MAIN_CONTENT_3\&context=storylines_faq}{online
    learning},
    \href{https://www.nytimes.com/2020/05/29/us/coronavirus-child-care-centers.html?action=click\&pgtype=Article\&state=default\&region=MAIN_CONTENT_3\&context=storylines_faq}{makeshift
    child care} and
    \href{https://www.nytimes.com/2020/06/03/business/economy/coronavirus-working-women.html?action=click\&pgtype=Article\&state=default\&region=MAIN_CONTENT_3\&context=storylines_faq}{stunted
    workdays} to continue. California's two largest public school
    districts --- Los Angeles and San Diego --- said on July 13, that
    \href{https://www.nytimes.com/2020/07/13/us/lausd-san-diego-school-reopening.html?action=click\&pgtype=Article\&state=default\&region=MAIN_CONTENT_3\&context=storylines_faq}{instruction
    will be remote-only in the fall}, citing concerns that surging
    coronavirus infections in their areas pose too dire a risk for
    students and teachers. Together, the two districts enroll some
    825,000 students. They are the largest in the country so far to
    abandon plans for even a partial physical return to classrooms when
    they reopen in August. For other districts, the solution won't be an
    all-or-nothing approach.
    \href{https://bioethics.jhu.edu/research-and-outreach/projects/eschool-initiative/school-policy-tracker/}{Many
    systems}, including the nation's largest, New York City, are
    devising
    \href{https://www.nytimes.com/2020/06/26/us/coronavirus-schools-reopen-fall.html?action=click\&pgtype=Article\&state=default\&region=MAIN_CONTENT_3\&context=storylines_faq}{hybrid
    plans} that involve spending some days in classrooms and other days
    online. There's no national policy on this yet, so check with your
    municipal school system regularly to see what is happening in your
    community.
  \end{itemize}
\item ~
  \hypertarget{is-the-coronavirus-airborne}{%
  \paragraph{Is the coronavirus
  airborne?}\label{is-the-coronavirus-airborne}}

  \begin{itemize}
  \tightlist
  \item
    The coronavirus
    \href{https://www.nytimes.com/2020/07/04/health/239-experts-with-one-big-claim-the-coronavirus-is-airborne.html?action=click\&pgtype=Article\&state=default\&region=MAIN_CONTENT_3\&context=storylines_faq}{can
    stay aloft for hours in tiny droplets in stagnant air}, infecting
    people as they inhale, mounting scientific evidence suggests. This
    risk is highest in crowded indoor spaces with poor ventilation, and
    may help explain super-spreading events reported in meatpacking
    plants, churches and restaurants.
    \href{https://www.nytimes.com/2020/07/06/health/coronavirus-airborne-aerosols.html?action=click\&pgtype=Article\&state=default\&region=MAIN_CONTENT_3\&context=storylines_faq}{It's
    unclear how often the virus is spread} via these tiny droplets, or
    aerosols, compared with larger droplets that are expelled when a
    sick person coughs or sneezes, or transmitted through contact with
    contaminated surfaces, said Linsey Marr, an aerosol expert at
    Virginia Tech. Aerosols are released even when a person without
    symptoms exhales, talks or sings, according to Dr. Marr and more
    than 200 other experts, who
    \href{https://academic.oup.com/cid/article/doi/10.1093/cid/ciaa939/5867798}{have
    outlined the evidence in an open letter to the World Health
    Organization}.
  \end{itemize}
\end{itemize}

But Facebook now has more than 4,000 employees in its offices in
Manhattan, up from about 2,900 employees at the beginning of the year.
The company leased office space at Hudson Yards --- which is also in
Midtown --- in November, and it has expressed interest in the
107-year-old Farley Building for months. With the addition of 730,000
square feet there, Facebook
\href{https://www.nytimes.com/2020/01/05/nyregion/nyc-tech-facebook-amazon-google.html}{has
acquired more than 2.2 million square feet} of office space in Midtown
Manhattan in less than a year, enough for thousands of employees.

Apple, Amazon and Google all lease space in the same area,
\href{https://www.nytimes.com/2020/01/05/nyregion/nyc-tech-facebook-amazon-google.html}{an
emerging tech corridor}. A Facebook spokeswoman said it was too soon to
estimate how many employees will end up at the Manhattan properties,
given the uncertainties of the outbreak.

\hypertarget{st-louis-cardinals-outbreak-grows-to-13-as-major-league-baseball-season-teeters}{%
\subsection{St. Louis Cardinals' outbreak grows to 13 as Major League
Baseball season
teeters.}\label{st-louis-cardinals-outbreak-grows-to-13-as-major-league-baseball-season-teeters}}

Image

The St. Louis Cardinals waiting for the start of a game in Pittsburgh
last week. The team's most recent 4-game series at Detroit was postponed
after more players tested positive.Credit...Jeff Roberson/Associated
Press

The St. Louis Cardinals' outbreak has swelled to at least 13 players and
staff members, in yet another blow to the rocky start of the Major
League Baseball season. The Cardinals' outbreak comes after the Miami
Marlins reported an outbreak last week: 18 players and two coaches.

The Cardinals have been quarantined since Thursday at their hotel in
Milwaukee, where their three-game series with the Brewers was postponed
last weekend after St. Louis's first cases were confirmed.

``I think everyone is trying to look for someone or something to blame,
and there isn't one person or one thing to blame,'' said Derek Jeter,
the Marlins' chief executive. ``This is a health crisis that we're all
dealing with --- a health crisis that not only our country is dealing
with, but our world is dealing with.''

Baseball wants to insulate itself from that world, but its 30 teams are
traveling throughout the United States to stage a 60-game season. The
league determined that a so-called bubble approach was impractical, and
the areas it considered months ago to carry out a season in a contained
environment --- Arizona, Texas and Florida --- have since become hot
spots for the virus. Yet road trips have increased the risk of
infection.

Mr. Jeter said the Marlins had been unfairly maligned for playing in
Philadelphia on July 26 after they learned of four positive tests; in
fact, he said, the Phillies and M.L.B. were also aware of those test
results. He also disputed that the Marlins had acted recklessly in
Atlanta, where they played two exhibitions.

Mostly, Mr. Jeter said, the Marlins were careless, failing to adhere
strictly to mask wearing and social distancing. While there was ``no
salacious activity'' in Atlanta, he said, some players did leave the
hotel for coffee or shopping.

\hypertarget{white-house-staff-will-be-randomly-tested-for-the-virus}{%
\subsection{White House staff will be randomly tested for the
virus.}\label{white-house-staff-will-be-randomly-tested-for-the-virus}}

White House officials have been told they will be randomly screened for
the coronavirus starting on Monday, according to a person who received
the email.

The new policy is a change for the White House, where the testing
requirement had previously been only for people in proximity to Mr.
Trump.

It was unclear what exactly prompted the change. An employee in the
White House complex
\href{https://www.nytimes.com/2020/07/22/us/politics/white-house-employee-covid-19.html}{cafeterias
recently tested positive} for the virus, prompting the closing of the
dining halls. And last week, Mr. Trump's national security adviser,
\href{https://www.nytimes.com/2020/07/27/us/politics/robert-obrien-virus.html}{Robert
C. O'Brien, tested positive} after experiencing minor symptoms. Mr.
O'Brien is the most senior White House official known to have contracted
the virus. He typically works from an office steps away from the Oval
Office.

\href{https://www.nytimes.com/2020/08/02/health/dr-birx-coronavirus-phase.html}{Public
health experts} have been concerned about the high levels of
asymptomatic transmission across the country. Even employees who do not
work in proximity to the president could be seeding chains of infection.

The email made clear that anyone who does not comply with assignments
for testing would be seen as refusing to be tested, according to the
person who received it.

\hypertarget{is-telemedicine-here-to-stay}{%
\subsection{Is telemedicine here to
stay?}\label{is-telemedicine-here-to-stay}}

Image

Dr. Meeta Shah taking telemedicine calls at Rush University Medical
Center in Chicago in March.Credit...Danielle Scruggs for The New York
Times

Over the past few months, millions of people have relied on video or
telephone calls to talk to their doctors. But how long will the moment
last?

The answer largely depends on
\href{https://www.nytimes.com/2020/08/03/health/covid-telemedicine-congress.html}{whether
Medicare and private health insurers will adequately cover virtual
doctor visits} once coronavirus outbreaks subside.

Medicare's coverage of a broad range of services is slated to end when
the coronavirus no longer poses a public health emergency. Private
insurers, which followed the federal government's lead, could revert to
paying doctors for virtual visits at a fraction of the cost for
traditional visits, if anything at all.

``The concern everyone in the industry has is that reimbursement is in
jeopardy,'' said Dr. Mia Levy, the director of the cancer center at Rush
University Medical Center in Chicago, which treated patients virtually
during the height of the pandemic.

On Monday, President Trump described telehealth as a ``very, very big
priority'' after he signed an executive order aimed at making permanent
some of the changes in Medicare policy that his administration adopted
during the pandemic. The order focuses in part on finding ways to ensure
access to medical care for people in rural areas.

While there is broad bipartisan support for telehealth coverage,
Congress would have to pass specific legislation to make some of
Medicare's changes permanent. Some lawmakers favor permanently expanding
Medicare payment for a broad range of telemedicine services, but others
are concerned about the technology's cost and potential for fraud.

Many patients enjoy the convenience of telemedicine. And it is
particularly valuable for those vulnerable to the virus, like Susan
Varak, 45, who has breast cancer. ``I don't think it's absolutely
necessary to be face-to-face every couple of weeks,'' she said.

But for some patients, telemedicine is not a substitute for in-person
care. Jorge Cueto, who is in his mid-20s, said a virtual visit is often
an additional step before going to the doctor's office for, say, a sore
throat.

``It's another fee, it's another gating mechanism,'' he said.

Dr. Ateev Mehrotra, a professor of health care policy at Harvard Medical
School, argues that the goal of telemedicine should not be to lower
health care costs over all. One of its main benefits, he said, is
improving patients' access to care, adding that it would be foolish to
expect savings if more people also get treatment. ``Those don't
reconcile,'' he said.

U.S. ROUNDUP

\hypertarget{as-cases-rise-new-jersey-limits-indoor-gatherings-again}{%
\subsection{As cases rise, New Jersey limits indoor gatherings
again.}\label{as-cases-rise-new-jersey-limits-indoor-gatherings-again}}

Image

A waiter takes an order at a boardwalk restaurant in Wildwood, N.J., in
July.Credit...Mark Makela/Getty Images

New Jersey will again restrict indoor gatherings as cases have risen in
the state, Gov. Philip D. Murphy said Monday. Gatherings will be limited
to 25 percent of a room's usual occupancy limit, with a maximum of 25
people, down from 100 people.

The new guidance will not apply to weddings, funerals, or memorial
services, he said, nor will it affect religious or political activities
protected under the First Amendment. Those events will still be capped
at 100 people, or 25 percent maximum occupancy.

The governor reiterated that officials believed that indoor house
parties and other gatherings were contributing to the resurgence of the
virus in New Jersey, which made significant progress battling its
outbreak in April and May.

The rate of virus transmission, a rough measure of the spread of
infection, in the state rose significantly in July, Mr. Murphy said. As
of Saturday, the first day of August,
\href{https://twitter.com/GovMurphy/status/1290335172446097409/photo/2}{the
rate was at 1.48}, according to the state, meaning that each person with
the virus infected an average of 1.48 people. Just one month ago, Mr.
Murphy said, the rate was at 0.87. The rate is a rough estimate, because
many infections are undetected for a variety of reasons, including
asymptomatic infections and testing issues.

``I don't think we ever graduated out of the first wave,'' he said. ``As
the clock has gone on, folks have begun to, a little bit, fall off the
wagon,'' in terms of indoor gatherings that violated the state's
social-distancing guidance.

Elsewhere in the United States:

\includegraphics{https://static01.nyt.com/images/2020/08/03/world/03virus-briefing-ca/03virus-briefing-ca-videoSixteenByNineJumbo1600.jpg}

\begin{itemize}
\item
  Declaring that ``lives are at stake,'' Mayor Sylvester Turner of
  \textbf{Houston} announced that residents of the nation's fourth
  largest city will face citations and fines of up to \$250 for failing
  to wear masks in public. He said police officers and firefighters
  would be authorized to issue citations to anyone in public without a
  mask after first issuing a warning. Though there have been signs of
  improvement, Houston and surrounding Harris County remain one of the
  nation's worst hot spots with 50,896 cases and 478 deaths since the
  start of the pandemic.
\item
  Gov. Gavin Newsom of \textbf{California} had encouraging news on
  Monday for a state whose residents
  \href{https://slack-redir.net/link?url=https\%3A\%2F\%2Fwww.nytimes.com\%2F2020\%2F07\%2F23\%2Fus\%2Fcalifornia-covid-19-cases.html}{have
  been whipsawed} by early successes against the virus and then a
  disastrous reopening of the economy in early summer. After surging for
  most of July, the average number of new cases and intensive care
  admissions have decreased in the state, the governor reported. Data
  compiled by The New York Times show the seven-day average of cases
  down 19 percent from a peak on July 25 in California. Mr. Newsom
  cautioned that the virus is still surging in some parts of the state.
\end{itemize}

\begin{itemize}
\item
  More than two million Americans who have lost ground economically
  during the pandemic have also lost health insurance recently, with
  African-Americans and low-wage workers the hardest hit, according to a
  \href{https://familiesusa.org/resources/americas-coverage-crisis-deepens-new-survey-data-show-millions-of-adults-became-uninsured-starting-in-late-june/}{new
  analysis} of census data by the advocacy group Families U.S.A. ``It's
  part of a bigger story --- a tale of two pandemics, where some of us
  are doing fine, or even doing really well, and others are really
  suffering,'' said Stan Dorn, the author of the study.
\item
  In \textbf{New York City}, the mayor said Monday that he plans to
  bring back a program that allows restaurants to serve patrons in
  outdoor dining areas on city streets, next year on June 1. He said the
  program had helped more than 9,000
  \href{https://www.nytimes.com/2020/08/03/nyregion/nyc-small-businesses-closing-coronavirus.html}{restaurants
  reopen for outdoor dining}, allowing an estimated 80,000 workers to
  return to their jobs.
\end{itemize}

GLOBAL ROUNDUP

\hypertarget{mexicos-television-and-radio-networks-to-broadcast-classes-for-students}{%
\subsection{Mexico's television and radio networks to broadcast classes
for
students}\label{mexicos-television-and-radio-networks-to-broadcast-classes-for-students}}

Image

Workers prepare a classroom with protective measures in Mexico City last
month. Schools will reopen when authorities determine that new and
active infections decline enough for a safe return.Credit...Pedro
Pardo/Agence France-Presse --- Getty Images

Students in Mexico will exclusively take classes broadcast on television
or the radio when the school year begins later this month, in an effort
to avoid further coronavirus outbreaks, the government announced on
Monday. Schools will only reopen when authorities determine that new and
active infections, which remain high across the nation, decline enough
for a safe return to the classroom.

``We would like to return to face-to-face classes, but it is neither
possible nor prudent,'' said the education minister, Esteban Moctezuma
Barragán, in a news conference. Mr. Moctezuma said he wanted to avoid
the fates of ``Israel, South Korea, the United Kingdom, France, to name
a few examples, who reopened their schools and had to close them
again.''

The announcement is one of the first signs of caution in the nation's
approach to reopening the economy in the face of the pandemic.
Restaurants, hotels and factories have all been allowed to restart
operations, even though the country has not managed to get the virus
under control. Over the weekend, Mexico surpassed the United Kingdom to
become the country with the third highest coronavirus deaths worldwide.

Four television companies will air classes to 30 million students on
broadcast channels, in Spanish, sign language and 20 Indigenous
languages, the education minister said. The government will distribute
free textbooks to students who can't afford them.

In other global news:

\begin{itemize}
\tightlist
\item
  The head of the World Health Organization said that while there was
  great progress in the global search for a vaccine for the coronavirus,
  people should not expect the crisis to end anytime soon. ``A number of
  vaccines are now in Phase 3 clinical trials and we all hope to have a
  number of effective vaccines that can help prevent people from
  infection,'' \href{https://twitter.com/DrTedros}{Tedros Adhanom
  Ghebreyesus}, the W.H.O.'s director general, told reporters on Monday.
  ``However, there's no silver bullet at the moment and there might
  never be.''
\end{itemize}

\begin{itemize}
\item
  A Norwegian cruise ship line halted all trips and apologized Monday
  after a coronavirus outbreak on one ship infected at least five
  passengers and 36 crew, The Associated Press reported. Health
  authorities said they feared the ship also could have spread the virus
  to dozens of communities along Norway's western coast. The Hurtigruten
  cruise line was one of the first companies to resume sailing during
  the pandemic.
\item
  President Rodrigo Duterte of the Philippines on Sunday ordered Manila
  and its suburbs to re-enter lockdown for two weeks as the health
  department reported 5,032 new cases of the coronavirus. Infections
  spiked after the government eased
  \href{https://www.nytimes.com/2020/04/15/world/asia/manila-coronavirus-lockdown-slum.html}{lockdown
  rules} and gradually opened up in an effort to jump-start the economy.
  Hospitals have been overwhelmed, and doctors have warned they are
  reaching a breaking point.
\end{itemize}

\hypertarget{federal-aid-for-us-small-businesses-may-not-be-enough-to-keep-many-afloat}{%
\subsection{Federal aid for U.S. small businesses may not be enough to
keep many
afloat.}\label{federal-aid-for-us-small-businesses-may-not-be-enough-to-keep-many-afloat}}

Image

Caroline Keefer's clothing business, River + Sky, lost nearly \$700,000
in orders when the virus hit, but her emergency loan from the Small
Business Administration was capped at \$150,000.\\

Credit...Nolwen Cifuentes for The New York Times

For nearly 70 years, the Small Business Administration's disaster relief
program has helped companies recover from catastrophe. But it has never
faced anything like this.

Besieged by more than eight million applicants --- and operating in the
shadow of the hastily assembled
\href{https://www.nytimes.com/2020/04/26/business/ppp-small-business-loans.html}{Paycheck
Protection Program} --- the disaster relief effort has given out more
money in the past few months than it had in its entire history.

But the demand has created a problem that is hobbling hundreds of
thousands of applicants: The agency, afraid of running out of cash,
capped its coronavirus loans at a fraction of what companies can
normally borrow --- even though the program has handed out less than
half the \$360 billion it can lend.

The cap has left many borrowers with loans that they fear will not be
enough to keep their businesses afloat. Nearly 400,000 businesses have
run into the \$150,000 limit, according to
\href{https://www.sba.gov/funding-programs/loans/coronavirus-relief-options/economic-injury-disaster-loans\#section-header-5}{the
agency's data}. S.B.A. representatives declined to comment on the cap or
why it was imposed.

``Without the extra capital, it will be very difficult for us to
survive,'' Caroline Keefer, a clothing designer in Los Angeles, wrote in
an appeal to the agency after her loan was capped.

The cap has been just one problem with the program, officially called
the Economic Injury Disaster Loan program. Applicants faced
\href{https://www.nytimes.com/2020/04/09/business/smallbusiness/small-business-disaster-loans-coronavirus.html}{long
delays}, confusing procedures and communication lapses. And on Tuesday,
the agency's internal watchdog said that hundreds of millions of dollars
handed out through the program
\href{https://www.nytimes.com/live/2020/07/28/business/stock-market-today-coronavirus\#thieves-are-targeting-small-business-relief-programs-a-watchdog-says}{may
have been fraudulently obtained}.

In New York City, an expanding universe of distinctive small businesses
--- from coffee shops to dry cleaners to hardware stores --- that give
its neighborhoods their unique personalities and are key to the city's
economy
\href{https://www.nytimes.com/2020/08/03/nyregion/nyc-small-businesses-closing-coronavirus.html}{are
starting to topple}. When the pandemic eventually subsides, roughly
one-third of the city's 240,000 small businesses may never reopen,
\href{https://pfnyc.org/wp-content/uploads/2020/07/actionandcollaboration.pdf}{according
to a report} by the Partnership for New York City, an influential
business group. So far, those businesses have shed 520,000 jobs.

On Monday, more than 100 current and former chief executives called for
more aid to small businesses across the country in a letter sent to
Treasury Secretary Steven Mnuchin.

``Allowing small businesses to fail will turn temporary job losses into
permanent ones,'' states
\href{https://www.howardschultz.com/lettertocongress/}{the letter},
which was organized by the former Starbucks chief Howard Schultz with
the support of Senators Michael Bennet, a Democrat, and Todd Young, a
Republican. It was signed by the likes of Walmart's Doug McMillon,
Alphabet's Sundar Pichai and Disney's Bob Chapek.

Reporting was contributed by Reed Abelson, Livia Albeck-Ripka, Peter
Baker, Emma Bubola, Benedict Carey, Emily Cochrane, Jill Cowan, Stacy
Cowley, Jacey Fortin, Thomas Fuller, Michael Gold, Denise Grady, Jason
Gutierrez, Matthew Haag, Maggie Haberman, Javier C. Hernández, Annie
Karni, Tyler Kepner, Sarah Kliff, Andrew E. Kramer, Sharon LaFraniere,
Dan Levin, Apoorva Mandavilli, Sarah Mervosh, Azi Paybarah, Daniel E.
Slotnik, Eileen Sullivan, Sheryl Gay Stolberg, Jim Tankersley, Katie
Thomas, Noah Weiland, Michael Wines, Sameer Yasir and Karen Zraick.

Advertisement

\protect\hyperlink{after-bottom}{Continue reading the main story}

\hypertarget{site-index}{%
\subsection{Site Index}\label{site-index}}

\hypertarget{site-information-navigation}{%
\subsection{Site Information
Navigation}\label{site-information-navigation}}

\begin{itemize}
\tightlist
\item
  \href{https://help.nytimes.com/hc/en-us/articles/115014792127-Copyright-notice}{©~2020~The
  New York Times Company}
\end{itemize}

\begin{itemize}
\tightlist
\item
  \href{https://www.nytco.com/}{NYTCo}
\item
  \href{https://help.nytimes.com/hc/en-us/articles/115015385887-Contact-Us}{Contact
  Us}
\item
  \href{https://www.nytco.com/careers/}{Work with us}
\item
  \href{https://nytmediakit.com/}{Advertise}
\item
  \href{http://www.tbrandstudio.com/}{T Brand Studio}
\item
  \href{https://www.nytimes.com/privacy/cookie-policy\#how-do-i-manage-trackers}{Your
  Ad Choices}
\item
  \href{https://www.nytimes.com/privacy}{Privacy}
\item
  \href{https://help.nytimes.com/hc/en-us/articles/115014893428-Terms-of-service}{Terms
  of Service}
\item
  \href{https://help.nytimes.com/hc/en-us/articles/115014893968-Terms-of-sale}{Terms
  of Sale}
\item
  \href{https://spiderbites.nytimes.com}{Site Map}
\item
  \href{https://help.nytimes.com/hc/en-us}{Help}
\item
  \href{https://www.nytimes.com/subscription?campaignId=37WXW}{Subscriptions}
\end{itemize}
