Sections

SEARCH

\protect\hyperlink{site-content}{Skip to
content}\protect\hyperlink{site-index}{Skip to site index}

\href{https://www.nytimes.com/section/world/europe}{Europe}

\href{https://myaccount.nytimes.com/auth/login?response_type=cookie\&client_id=vi}{}

\href{https://www.nytimes.com/section/todayspaper}{Today's Paper}

\href{/section/world/europe}{Europe}\textbar{}Turkish Aggression Is
NATO's `Elephant in the Room'

\url{https://nyti.ms/3kf9G8n}

\begin{itemize}
\item
\item
\item
\item
\item
\end{itemize}

Advertisement

\protect\hyperlink{after-top}{Continue reading the main story}

Supported by

\protect\hyperlink{after-sponsor}{Continue reading the main story}

\hypertarget{turkish-aggression-is-natos-elephant-in-the-room}{%
\section{Turkish Aggression Is NATO's `Elephant in the
Room'}\label{turkish-aggression-is-natos-elephant-in-the-room}}

Despite being a NATO member, Turkey has bought Russian air defense. And
a recent push into Libya and its energy ambitions nearly led to armed
conflicts with France and Greece.

\includegraphics{https://static01.nyt.com/images/2020/07/30/world/02turkey-nato01/merlin_174864534_36e61b4e-7cae-4a2c-af12-0808106015a8-articleLarge.jpg?quality=75\&auto=webp\&disable=upscale}

\href{https://www.nytimes.com/by/steven-erlanger}{\includegraphics{https://static01.nyt.com/images/2018/10/10/multimedia/author-steven-erlanger/author-steven-erlanger-thumbLarge.png}}

By \href{https://www.nytimes.com/by/steven-erlanger}{Steven Erlanger}

\begin{itemize}
\item
  Aug. 3, 2020
\item
  \begin{itemize}
  \item
  \item
  \item
  \item
  \item
  \end{itemize}
\end{itemize}

BRUSSELS --- The warships were escorting a vessel suspected of smuggling
weapons into Libya, violating a United Nations arms embargo. Challenged
by a French naval frigate, the warships went to battle alert.
Outnumbered and outgunned, the French frigate withdrew.

But this mid-June naval showdown in the Mediterranean was not a
confrontation of enemies. The antagonists were France and Turkey, fellow
members of NATO, sworn to protect one another.

A similarly hostile encounter between Turkey and a fellow NATO member
happened just two weeks ago, when Turkish warplanes buzzed an area near
the Greek island of Rhodes after Greek warships went on alert over
Turkey's intent to drill for undersea natural gas there.

Turkey --- increasingly assertive, ambitious and authoritarian --- has
become ``the elephant in the room'' for NATO, European diplomats say.
But it is a matter, they say, that few want to discuss.

A NATO member since 1952, Turkey is too big, powerful and strategically
important --- it is the crossroads of Europe and Asia --- to allow an
open confrontation, alliance officials suggest.

Turkey has dismissed any criticism of its behavior as unjustified. But
some NATO ambassadors believe that Turkey now represents an open
challenge to the group's democratic values and its collective defense.

A more aggressive, nationalist and religious Turkey is increasingly at
odds with its Western allies over Libya, Syria, Iraq, Russia and the
energy resources of the eastern Mediterranean. Turkey's tilt toward
strongman rule after 17 years with President Recep Tayyip Erdogan at the
helm also has unsettled other NATO members.

``It's getting hard to describe Turkey as an ally of the U.S.,'' said
\href{https://www.cfr.org/expert/philip-h-gordon}{Philip H. Gordon}, a
foreign policy adviser and former assistant secretary of state who dealt
with Turkey during the Obama administration.

\includegraphics{https://static01.nyt.com/images/2020/07/30/world/02turkey-nato06/merlin_165409620_e2731be2-3596-4db5-8f90-b581fdc0a9cb-articleLarge.jpg?quality=75\&auto=webp\&disable=upscale}

Despite that, Turkey is getting a kind of free pass, analysts say, its
path having been cleared by a lack of consistent U.S. leadership,
exacerbated by President Trump's contempt for NATO and his clear
admiration for Mr. Erdogan.

``You can't say what U.S. policy on Turkey is, and you can't even see
where Trump is,'' Mr. Gordon said. ``It's a big dilemma for U.S. policy,
where we seem to disagree strategically on nearly every issue.''

Those strategic divides are proliferating. They include Turkey's support
for different armed groups in Syria; its 2019 purchase of a
\href{https://www.nytimes.com/2019/07/12/world/europe/turkey-russia-missiles.html?searchResultPosition=1}{sophisticated
Russian antiaircraft system over fierce objections} by the United States
and other NATO members; its violation of the arms embargo in Libya; its
aggressive drilling in the eastern Mediterranean; its constant
demonization of Israel; and its increasing use of state-sponsored
disinformation.

But NATO officials' general meekness in standing up to Turkey has not
helped, analysts say, pointing to the group's secretary-general, Jens
Stoltenberg, whose job is to keep the 30-nation alliance together, but
who is considered excessively tolerant of both American and Turkish
misbehavior.

The last serious discussion of Turkey's policies among NATO ambassadors
was late last year, despite the purchase of the antiaircraft system,
\href{https://www.nytimes.com/2019/07/12/world/russia-turkey-missile-explain.html?searchResultPosition=1}{the
S-400}.

Image

A Russian military cargo plane unloading parts of the S-400 antiaircraft
system in Ankara, in a photograph released by the Turkish Defense
Ministry last year.Credit...Turkish Defence Ministry, via Agence
France-Presse --- Getty Images

Other countries, like Hungary and Poland, also fall short on the values
scale, argued Nicholas Burns, a former NATO ambassador now at Harvard.
But only Turkey blocks key alliance business.

NATO operates by consensus, so Turkish objections can stall nearly any
policy, and its diplomats are both diligent and knowledgeable, ``on top
of every ball,'' as one NATO official said. France has also used its
effective veto to pursue national interests, but never to undermine
collective defense, NATO ambassadors say. But Turkey has blocked NATO
partnerships for countries it dislikes, like Israel, Armenia, Egypt and
the United Arab Emirates.

More seriously, for many months Turkey blocked a NATO plan for the
defense of Poland and the Baltic nations, which all border Russia. And
Turkey wanted NATO to list various armed Kurdish groups, which have
fought for their independence, as terrorist groups --- something that
NATO does not do.

Some of these same Kurdish groups are also Washington's best allies in
its fight against Islamic State and Al Qaeda in Syria and Iraq.

A deal was supposedly worked out at the last NATO summit meeting in
December in London, but Turkey created bureaucratic complications, and
it was only in late June that Turkey relented --- after considerable
pressure from official Washington, which has lost patience with Mr.
Erdogan and is infuriated by his insistence on buying the S-400.

If deployed, the S-400 would put Russian engineers inside a NATO air
defense system, giving them valuable insights into the alliance's
strengths while threatening to diminish the capability of the expensive
fifth-generation fighter, the F-35.

The assumption is that Mr. Erdogan, who has grown significantly more
suspicious since a
\href{https://www.nytimes.com/interactive/2016/07/16/world/europe/turkey-coup-photos.html}{failed
2016 coup} against him, wants to be able to shoot down American and
Israeli planes like the ones his own air force used in the coup attempt.

``Every time we discuss Russia'' in NATO, ``everyone thinks of the S-400
and no one says anything,'' said one European diplomat, who spoke on the
condition of anonymity to discuss a sensitive matter. ``It's a major
breach in NATO air defense, and it's not even discussed.''

Instead, NATO assumes that talks between Washington and Ankara will
somehow handle the problem. But Washington is divided, and Mr.
\href{https://www.nytimes.com/2020/06/10/world/europe/erdogan-trump-turkey-libya-syria.html}{Erdogan
talks only to Mr. Trump}.

Yet the confusion is not simply Washington's, said
\href{https://ash.harvard.edu/people/amanda-sloat}{Amanda Sloat}, a
former deputy assistant secretary of state who dealt with Turkey in the
Obama State Department and wrote
\href{https://www.foreignaffairs.com/articles/turkey/2020-01-10/dangerous-unraveling-us-turkish-alliance}{a
recent essay}with Mr. Gordon. The European Union and the United Nations
also have no clear policy on Turkey or Libya, she said.

Turkey has pursued its own national interests in northern Syria, where
it now has more than 10,000 troops, and in Libya, where its
\href{https://www.nytimes.com/2020/01/02/world/europe/erdogan-turkey-libya.html}{military
support for a failing government} helped turn the tide in return for a
share in Libya's rich energy resources.

Image

Turkish soldiers in Idlib Province, Syria, in May. Turkey has more than
10,000 troops in northern Syria.Credit...Ghaith Alsayed/Associated Press

It was near Libya in June that three Turkish warships confronted the
French frigate.

While the European Union has a mission to help enforce the arms embargo
on Libya, NATO does not. The frigate, the Courbet, was engaged in a
different NATO mission aimed at migration flows, but since Turkey and
France support different sides in the Libyan civil war, the
confrontation between NATO allies was troubling.

Turkey said the ship was carrying aid rather than arms, and has denied
harassing the Courbet. NATO officials say that its military committee is
investigating and that the evidence is not as clear-cut as the French
suggest.

Still, President Emmanuel Macron of France has used the clash as
\href{https://www.euronews.com/2020/06/23/emmanuel-macron-turkey-is-playing-a-dangerous-game-in-libya}{another
moment to assert} that NATO is nearing ``brain death,'' because it seems
incapable of reining in Turkey or acting in a coordinated political way.

His first accusation also involved Turkey, when Mr. Trump,
\href{https://www.foxnews.com/politics/trump-phone-call-erdogan-turkey-syria}{after
a call} with Mr. Erdogan last October, unilaterally decided to pull U.S.
troops out of northern Syria, where NATO is fighting the Islamic State,
leaving the French and other allies exposed. Ultimately, the Pentagon
persuaded Mr. Trump to leave some American troops there.

French-Turkish tensions at NATO date to the 2011 decision to intervene
against Col. Muammar el-Qaddafi in Libya, noted Ivo Daalder, who was
then the American ambassador to NATO.

France, with its policy of secularism, fears that Mr. Erdogan's
reinsertion of Islam into politics will spread in North Africa,
encourage Islamist militias and damage France's ``sphere of influence,''
said Soner Cagaptay, the director of the Turkish Research Program at the
Washington Institute for Near East Policy. ``They are quite worried.''

The latest flash point is over Turkey's demand to share in discoveries
of natural gas made in 2015 in the eastern Mediterranean, which led to
deals and alliances among Greece, Cyprus, Israel and Egypt.

Image

A Turkish drilling vessel in the eastern Mediterranean off Cyprus last
year.Credit...Murad Sezer/Reuters

Maritime claims are disputed, and Mr. Erdogan complained in June that
``their aim was to imprison our country, which has the longest coastline
in the Mediterranean, into a coastal strip from which you can only catch
fish with a rod.''

He then sent survey and drilling ships to explore off Cyprus,
\href{https://www.nytimes.com/2019/07/15/world/europe/eu-turkey-cyprus.html}{prompting
European sanctions}, and said he would do the same near Rhodes, bringing
the Greeks to threaten warfare. Last week, Chancellor Angela Merkel of
Germany got Mr. Erdogan to hold off while talks proceed.

While many looked to Turkey as a moderate democratic model during the
Arab spring a decade ago, Turkey is a different country under Mr.
Erdogan, who has mobilized the more religious voters in the countryside.

A devout Muslim, Mr. Erdogan has become more nationalist and
authoritarian, especially in the aftermath of the 2016 coup attempt,
when he purged and jailed many Turkish secularists, judges, journalists
and military commanders.

He has broken definitively with Turkish secularism, symbolized by his
\href{https://www.nytimes.com/2020/07/24/world/europe/turkey-hagia-sophia-mosque-prayers.html?searchResultPosition=3}{recent
decision to turn Hagia Sophia} from a museum back into a mosque. He has
pushed hard into the region with a neo-Ottoman ambition, downgrading
older alliances to press Turkish interests.

Image

Mr. Erdogan recently made the decision to change Hagia Sophia from a
secular museum into a working mosque.Credit...Burak Kara/Getty Images

For NATO, Ms. Sloat said, ``The question now is whether Turkey is still
a Western country and shares our values.''

Ibrahim Kalin, Mr. Erdogan's spokesman,
\href{http://www.epc.eu/en/past-events/Turkish-foreign-policy-in-an-age-of-uncertainty~35d928}{brushes
off criticism} and says Mr. Trump and Mr. Macron are the ones
questioning NATO's value.

``I guess Macron is trying to assert some sort of leadership in North
Africa, the kind he doesn't have in Europe,'' Mr. Kalin said. ``He
called Turkey criminal, and it is incredible for France to call that to
another NATO member.''

As for Brussels, Mr. Kalin said, ``the E.U. should look into the
mirror.'' Greece ``uses E.U. membership as a way to pressure Turkey, but
this language of sanctions will not work,'' he said, arguing that Turkey
wants only ``an equitable and fair sharing of energy resources.''

The public American position is essentially to urge Turkey ``to halt
operations that raise tensions,'' Philip T. Reeker, the acting U.S.
assistant secretary of state for Europe and Eurasia, said of the eastern
Mediterranean.

``We want our friends and allies --- and let's remember, we're all,
Turkey, Greece and the United States, all NATO allies --- we want
friends and allies in the region to approach these issues in a spirit of
cooperation,'' he said.

``There is a big conversation to have about what to do about Turkey,'' a
senior European diplomat said. ``But it's not for now.''

Carlotta Gall contributed reporting from Istanbul, and Matina
Stevis-Gridneff from Brussels.

Advertisement

\protect\hyperlink{after-bottom}{Continue reading the main story}

\hypertarget{site-index}{%
\subsection{Site Index}\label{site-index}}

\hypertarget{site-information-navigation}{%
\subsection{Site Information
Navigation}\label{site-information-navigation}}

\begin{itemize}
\tightlist
\item
  \href{https://help.nytimes.com/hc/en-us/articles/115014792127-Copyright-notice}{©~2020~The
  New York Times Company}
\end{itemize}

\begin{itemize}
\tightlist
\item
  \href{https://www.nytco.com/}{NYTCo}
\item
  \href{https://help.nytimes.com/hc/en-us/articles/115015385887-Contact-Us}{Contact
  Us}
\item
  \href{https://www.nytco.com/careers/}{Work with us}
\item
  \href{https://nytmediakit.com/}{Advertise}
\item
  \href{http://www.tbrandstudio.com/}{T Brand Studio}
\item
  \href{https://www.nytimes.com/privacy/cookie-policy\#how-do-i-manage-trackers}{Your
  Ad Choices}
\item
  \href{https://www.nytimes.com/privacy}{Privacy}
\item
  \href{https://help.nytimes.com/hc/en-us/articles/115014893428-Terms-of-service}{Terms
  of Service}
\item
  \href{https://help.nytimes.com/hc/en-us/articles/115014893968-Terms-of-sale}{Terms
  of Sale}
\item
  \href{https://spiderbites.nytimes.com}{Site Map}
\item
  \href{https://help.nytimes.com/hc/en-us}{Help}
\item
  \href{https://www.nytimes.com/subscription?campaignId=37WXW}{Subscriptions}
\end{itemize}
