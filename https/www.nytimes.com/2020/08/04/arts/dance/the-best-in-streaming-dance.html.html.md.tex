Sections

SEARCH

\protect\hyperlink{site-content}{Skip to
content}\protect\hyperlink{site-index}{Skip to site index}

\href{https://www.nytimes.com/section/arts/dance}{Dance}

\href{https://myaccount.nytimes.com/auth/login?response_type=cookie\&client_id=vi}{}

\href{https://www.nytimes.com/section/todayspaper}{Today's Paper}

\href{/section/arts/dance}{Dance}\textbar{}The Vail All-Stars, San
Francisco Dreaming and Black Lives Matter

\url{https://nyti.ms/2XqTLKl}

\begin{itemize}
\item
\item
\item
\item
\item
\end{itemize}

\href{https://www.nytimes.com/spotlight/at-home?action=click\&pgtype=Article\&state=default\&region=TOP_BANNER\&context=at_home_menu}{At
Home}

\begin{itemize}
\tightlist
\item
  \href{https://www.nytimes.com/2020/08/03/well/family/the-benefits-of-talking-to-strangers.html?action=click\&pgtype=Article\&state=default\&region=TOP_BANNER\&context=at_home_menu}{Talk:
  To Strangers}
\item
  \href{https://www.nytimes.com/2020/08/01/at-home/coronavirus-make-pizza-on-a-grill.html?action=click\&pgtype=Article\&state=default\&region=TOP_BANNER\&context=at_home_menu}{Make:
  Grilled Pizza}
\item
  \href{https://www.nytimes.com/2020/07/31/arts/television/goldbergs-abc-stream.html?action=click\&pgtype=Article\&state=default\&region=TOP_BANNER\&context=at_home_menu}{Watch:
  'The Goldbergs'}
\item
  \href{https://www.nytimes.com/interactive/2020/at-home/even-more-reporters-editors-diaries-lists-recommendations.html?action=click\&pgtype=Article\&state=default\&region=TOP_BANNER\&context=at_home_menu}{Explore:
  Reporters' Google Docs}
\end{itemize}

Advertisement

\protect\hyperlink{after-top}{Continue reading the main story}

Supported by

\protect\hyperlink{after-sponsor}{Continue reading the main story}

Watching and Moving

\hypertarget{the-vail-all-stars-san-francisco-dreaming-and-black-lives-matter}{%
\section{The Vail All-Stars, San Francisco Dreaming and Black Lives
Matter}\label{the-vail-all-stars-san-francisco-dreaming-and-black-lives-matter}}

This week's standouts in streaming dance, plus something to get the kids
involved and moving.

\includegraphics{https://static01.nyt.com/images/2020/08/04/arts/04watching-moving-1/merlin_175268094_4768a1a2-7273-42c8-9be6-9f4dea386647-articleLarge.jpg?quality=75\&auto=webp\&disable=upscale}

\href{https://www.nytimes.com/by/brian-seibert}{\includegraphics{https://static01.nyt.com/images/2019/04/03/multimedia/author-brian-seibert/author-brian-seibert-thumbLarge.png}}

By \href{https://www.nytimes.com/by/brian-seibert}{Brian Seibert}

\begin{itemize}
\item
  Aug. 4, 2020
\item
  \begin{itemize}
  \item
  \item
  \item
  \item
  \item
  \end{itemize}
\end{itemize}

\hypertarget{watch-high-altitude-all-stars}{%
\subsection{Watch: High-Altitude
All-Stars}\label{watch-high-altitude-all-stars}}

Situated in a Rocky Mountain resort town,
\href{https://vaildance.org/}{the Vail Dance Festival} has never been
the most easily accessible event. But in this summer of digital or
(\href{https://www.nytimes.com/2020/07/29/arts/dance/kaatsbaan-dance-festival-stella-abrera.html}{nearly})
nothing, it's available to all, on YouTube through Aug. 15. Since this
year's four programs are mostly a selection of performances filmed in
recent years, they offer a chance to discover what, apart from its
alpine setting, has made the festival distinct.

Vail is a bit like fantasy football or a bunch of all-star games. New
York City Ballet luminaries join with big names from American Ballet
Theater, alongside performers like the tap dance leader Michelle
Dorrance and the Memphis jookin prodigy Lil Buck. And unlike in similar
gala situations, the teams really mingle --- shuffling rosters, swapping
repertory, collectively contributing to commissioned novelties. The new
combinations and collaborations are sometimes rough or superficial,
sometimes fresh and exceptional.

The ``Now: Premieres'' program, debuting on Tuesday, features two new
made-for-the moment videos. In Robert Fairchild's ``A Summer Place,'' he
does a dreamy song and dance on his roof. In Bobbi Jene Smith's
``Mercy,'' she and Melissa Toogood and Calvin Royal III writhe
attractively at the seashore. But the festival's core spirit is best
captured in the final selection, dropping on Friday: Ms. Dorrance's 2017
``we seem to be more than one,'' in which the motley masters of many
disciplines are adroitly woven together with rhythm and with Bill Irwin
reciting Samuel Beckett. Forget all-star game: This is the ultimate
dance camp finale.

\hypertarget{watch-stages-of-grief}{%
\subsection{Watch: Stages of Grief}\label{watch-stages-of-grief}}

\includegraphics{https://static01.nyt.com/images/2020/08/05/arts/04watching-moving-2NEW/merlin_175266327_2f561bde-2945-456d-9d9a-e55483014009-articleLarge.jpg?quality=75\&auto=webp\&disable=upscale}

As chipper as Mr. Fairchild appears up on the roof, he's gone through a
lot of changes lately: leaving City Ballet for Broadway and movies,
ending his marriage. He's had to let go of who he used to be, and he
\href{https://www.dancemagazine.com/robbie-fairchild-short-film-2639173287.html?rebelltitem=5\#rebelltitem5}{drew
on that experience} for the 2019 dance short ``In This Life,'' which is
streaming on
\href{https://allarts.org/programs/all-arts-performance-selects/life-fvtylu/}{the
website of WNET All Arts} starting Wednesday.

The 11-minute film, directed by Bat-Sheva Guez and written by Ms. Guez
and Mr. Fairchild, is structured around the five stages of grief, each
one imagined by a different choreographer in a different striking
location. For ``Bargaining,'' Mr. Fairchild gets soaked in the ocean,
baptized in Andrea Miller moves. For ``Depression,'' he does a clingy
Christopher Wheeldon pas de deux with a masked figure in a restaurant
bathroom. ``In This Life'' is a little horror movie and what holds it
together is the versatility and presence of its leading man.

\hypertarget{watch-a-touch-of-vertigo}{%
\subsection{Watch: A Touch of Vertigo}\label{watch-a-touch-of-vertigo}}

Image

Ellen Rose Hummel and Daniel Deivison-Oliveira in ``Dance of Dreams,''
directed by Benjamin Millepied in San Francisco.Credit...via San
Francisco Ballet

The coronavirus has cooped up dancers, but it has also given rise to a
spate of short films in which performers who normally ply their trade on
stages exult in spreading their limbs outdoors. On Aug. 13,
\href{https://www.sfballet.org/sf-ballet-home/}{San Francisco Ballet} is
debuting another: ``Dance of Dreams,'' directed by Benjamin Millepied.

The dream here is one of free motion in space, and, in two duets, of
physical connection. But the six-minute short is equally a celebration
of San Francisco as a grand location for film. The choreography is by
Justin Peck, Christopher Wheeldon, Janie Taylor and Dwight Rhoden, and
the excellent dancers look liberated. But oh, the settings: the cliffs
of Sausalito, the Golden Gate shrouded in mist. The film ends in the
Palace of Fine Arts, where Hitchcock shot some of ``Vertigo,'' and that
movie memory is also in the music: the rich ``Scene d'Amour'' from
Bernard Herrmann's score.

\hypertarget{watch-words-and-moves-for-black-lives}{%
\subsection{Watch: Words and Moves for Black
Lives}\label{watch-words-and-moves-for-black-lives}}

Image

A photo of Kaldi Makutike, a dancer from South Africa, overlaid on
footage of downtown New Orleans, in a post on
\#MOVEforBLACKLIVES.Credit...Screengrab

This has been the season of very short dance films, often grouped in
series. Amid a deluge of videos inspired mainly by the lack of anything
else to do, the Instagram series
\href{https://www.instagram.com/explore/tags/moveforblacklives/}{\#MOVEforBLACKLIVES}
stands out for its clarity of purpose.

Initiated and produced by the New York company
\href{http://mathetadance.com/}{Matheta Dance}, it's a fund-raising
project for Black Lives Matter organizations and bailout collections.
Donors were invited to a choose a word from a list --- ``rise,''
``breathe,'' ``heal,'' ``protest.'' Then a dancer, responding to that
word, improvised on video, and Matheta's company manager, Terri Ayanna
Wright, edited the footage into a one-minute physical statement.

Since the project began, on Juneteenth, dozens of videos have accrued.
You might \href{https://www.instagram.com/p/CCmnTQMMm_g/}{start with
``Fear.''} As Kendrick Lamar, in his track by the same title, calmly
lists the ways he might die, the Alvin Ailey dancer
\href{https://www.dancemagazine.com/on-the-rise-chalvar-monteiro-2366853607.html}{Chalvar
Monteiro} dances with tenderness and beauty in the shadow of a fire
escape. Fear is present, but not only fear.

\hypertarget{watch-and-move-children-and-art}{%
\subsection{Watch and Move: Children and
Art}\label{watch-and-move-children-and-art}}

Image

This week's installment of the New Victory Dance program features the
tap queen Dormeshia.Credit...via New Victory

Back in spring, when parents were going nuts trying to occupy
stuck-inside children, the family friendly New Victory Theater provided
a great service with its
\href{https://newvictory.org/stories/category/family-engagement/new-victory-arts-break/}{``Arts
Break''} series of activity videos.

This summer, the theater has adapted its
\href{https://newvictory.org/virtual-events/new-victory-dance-2020/}{New
Victory Dance} program,
\href{https://www.nytimes.com/2014/07/31/arts/dance/dancing-and-hoping-to-win-fans-for-life.html?searchResultPosition=2}{which
normally provides free dance performances} to day camps and summer
schools, into an online series for ages 8 and up. Each 20-minute episode
focuses on a company or choreographer; there's a terrifically diverse
array, from the Afro-Mexican fandango of Ballet Nepantla to the same-sex
tango of Kate Weare.

Hosted by Patrick Ferreri, who has the manner of a dance-world Mister
Rogers, these casual videos allow the choreographers to introduce
themselves and share a performance excerpt. Then Mr. Ferreri strikes up
a conversation about the dance with young New Victory staff members and
leads a brief dance class based in that episode's idiom. This all
culminates in a dance party, with the cast demonstrating how it's done
from their kitchens and living rooms. The end, arrived at
unintimidatingly, is education.

This week's installment is particularly inspiring. It features
\href{https://www.nytimes.com/2019/11/22/arts/Dormeshia-tap-dancing.html?searchResultPosition=4}{the
tap queen Dormeshia} and part of
\href{https://www.nytimes.com/2019/12/04/arts/dance/dormeshia-and-still-you-must-swing-review.html?searchResultPosition=3}{her
knockout show ``And Still You Must Swing.''}

Advertisement

\protect\hyperlink{after-bottom}{Continue reading the main story}

\hypertarget{site-index}{%
\subsection{Site Index}\label{site-index}}

\hypertarget{site-information-navigation}{%
\subsection{Site Information
Navigation}\label{site-information-navigation}}

\begin{itemize}
\tightlist
\item
  \href{https://help.nytimes.com/hc/en-us/articles/115014792127-Copyright-notice}{©~2020~The
  New York Times Company}
\end{itemize}

\begin{itemize}
\tightlist
\item
  \href{https://www.nytco.com/}{NYTCo}
\item
  \href{https://help.nytimes.com/hc/en-us/articles/115015385887-Contact-Us}{Contact
  Us}
\item
  \href{https://www.nytco.com/careers/}{Work with us}
\item
  \href{https://nytmediakit.com/}{Advertise}
\item
  \href{http://www.tbrandstudio.com/}{T Brand Studio}
\item
  \href{https://www.nytimes.com/privacy/cookie-policy\#how-do-i-manage-trackers}{Your
  Ad Choices}
\item
  \href{https://www.nytimes.com/privacy}{Privacy}
\item
  \href{https://help.nytimes.com/hc/en-us/articles/115014893428-Terms-of-service}{Terms
  of Service}
\item
  \href{https://help.nytimes.com/hc/en-us/articles/115014893968-Terms-of-sale}{Terms
  of Sale}
\item
  \href{https://spiderbites.nytimes.com}{Site Map}
\item
  \href{https://help.nytimes.com/hc/en-us}{Help}
\item
  \href{https://www.nytimes.com/subscription?campaignId=37WXW}{Subscriptions}
\end{itemize}
