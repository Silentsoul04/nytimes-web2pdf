Sections

SEARCH

\protect\hyperlink{site-content}{Skip to
content}\protect\hyperlink{site-index}{Skip to site index}

\href{https://www.nytimes.com/section/books/review}{Book Review}

\href{https://myaccount.nytimes.com/auth/login?response_type=cookie\&client_id=vi}{}

\href{https://www.nytimes.com/section/todayspaper}{Today's Paper}

\href{/section/books/review}{Book Review}\textbar{}The Reporter Who Told
the World About the Bomb

\url{https://nyti.ms/2DaW6Ce}

\begin{itemize}
\item
\item
\item
\item
\item
\end{itemize}

Advertisement

\protect\hyperlink{after-top}{Continue reading the main story}

Supported by

\protect\hyperlink{after-sponsor}{Continue reading the main story}

nonfiction

\hypertarget{the-reporter-who-told-the-world-about-the-bomb}{%
\section{The Reporter Who Told the World About the
Bomb}\label{the-reporter-who-told-the-world-about-the-bomb}}

\includegraphics{https://static01.nyt.com/images/2020/08/04/books/review/04Langewiesche1/merlin_146255937_1fd6a1ce-d2ea-4b50-b28e-c47530cecc2d-articleLarge.jpg?quality=75\&auto=webp\&disable=upscale}

Buy Book ▾

\begin{itemize}
\tightlist
\item
  \href{https://www.amazon.com/gp/search?index=books\&tag=NYTBSREV-20\&field-keywords=Fallout\%3A+The+Hiroshima+Cover-Up+and+the+Reporter+Who+Revealed+It+to+the+World+Lesley+M.M.+Blume}{Amazon}
\item
  \href{https://du-gae-books-dot-nyt-du-prd.appspot.com/buy?title=Fallout\%3A+The+Hiroshima+Cover-Up+and+the+Reporter+Who+Revealed+It+to+the+World\&author=Lesley+M.M.+Blume}{Apple
  Books}
\item
  \href{https://www.anrdoezrs.net/click-7990613-11819508?url=https\%3A\%2F\%2Fwww.barnesandnoble.com\%2Fw\%2F\%3Fean\%3D9781982128531}{Barnes
  and Noble}
\item
  \href{https://www.anrdoezrs.net/click-7990613-35140?url=https\%3A\%2F\%2Fwww.booksamillion.com\%2Fp\%2FFallout\%253A\%2BThe\%2BHiroshima\%2BCover-Up\%2Band\%2Bthe\%2BReporter\%2BWho\%2BRevealed\%2BIt\%2Bto\%2Bthe\%2BWorld\%2FLesley\%2BM.M.\%2BBlume\%2F9781982128531}{Books-A-Million}
\item
  \href{https://bookshop.org/a/3546/9781982128531}{Bookshop}
\item
  \href{https://www.indiebound.org/book/9781982128531?aff=NYT}{Indiebound}
\end{itemize}

When you purchase an independently reviewed book through our site, we
earn an affiliate commission.

By William Langewiesche

\begin{itemize}
\item
  Aug. 4, 2020, 5:00 a.m. ET
\item
  \begin{itemize}
  \item
  \item
  \item
  \item
  \item
  \end{itemize}
\end{itemize}

\textbf{FALLOUT}\\
\textbf{The Hiroshima Cover-Up and the Reporter Who Revealed It}
\textbf{to the World}\\
By Lesley M. M. Blume

Seventy-five years ago, on the bright clear morning of Aug. 6, 1945, the
United States dropped an atomic bomb on Hiroshima, immediately killing
70,000 people, and so grievously crushing, burning and irradiating
another 50,000 that they too soon died. The numbers are necessarily
approximate, but even from within the deadliest conflict in history,
such devastation from a single, airdropped device raised the stakes of
war from conquest into the realm of human annihilation.

For a moment the Japanese had no idea what had hit them. But President
Harry S. Truman soon provided an explanation. Returning from the Potsdam
Conference, and broadcasting mid-Atlantic from the U.S.S. Augusta, a
battle-weary cruiser, he said: ``Sixteen hours ago an American airplane
dropped one bomb on Hiroshima, an important Japanese army base. That
bomb had more power than 20,000 tons of TNT. \ldots{} It is an atomic
bomb. It is a harnessing of the basic power of the universe. The force
from which the sun draws its power has been loosed against those who
brought war to the Far East.''

Three days after Hiroshima the United States dropped additional evidence
on Nagasaki, and Japan surrendered. Afterward, as part of a clampdown on
information --- an extension of routine wartime censorship --- little
mention of realities on the ground was allowed by American authorities
beyond the obvious fact that with one bomb each, two cities had been
smashed. And so what? In the United States the hatred for the Japanese
far exceeded that of the hatred for the Germans; racism aside, the
Japanese had dared to bomb Americans on American territory. Days after
the bombings a Gallup poll found that 85 percent of Americans approved
of the attacks, and another survey, made after the war, indicated that
23 percent wished that more such weapons had been dropped before the
Japanese surrender.

\includegraphics{https://static01.nyt.com/images/2020/08/04/books/review/04Langewiesche2/04Langewiesche2-articleLarge.jpg?quality=75\&auto=webp\&disable=upscale}

Among those harboring no love for the enemy was a reporter named John
Hersey, who had covered the war in Europe and the Pacific, and had
described the Japanese as ``stunted physically'' and as ``a swarm of
intelligent little animals.'' Hersey was over 6 feet tall, lanky,
handsome, a graduate of Hotchkiss and Yale, and a modest, retiring man.
He lived in New York, and was a rising star in the city's publishing
circles. When the war ended he was 31, had recently returned from a
posting in Moscow and had just won a Pulitzer Prize for ``A Bell for
Adano,'' a war novel set in Sicily. Preferring fiction over straight
reporting, he spent much of his subsequent life writing novels.

But first there was this matter of the atomic bombs. Hersey despaired
when he heard Truman's Hiroshima announcement on the radio: He
understood the ominous implications for humanity. At the same time, he
felt relieved. The bombing, he guessed, would end the war; one such hit
would prove to be plenty. He was outraged therefore when three days
later the United States nuked Nagasaki; he called that second bombing a
criminal action.

For weeks afterward little was known about the consequences in Hiroshima
and Nagasaki beyond reports of impressive physical devastation. When
word of widespread radiation sickness began to circulate in occupied
Japan and the first Western press reports slipped by the censors, the
accounts were categorically denied. In late August 1945, The New York
Times ran a United Press dispatch from Hiroshima, but only after
deleting nearly all references to radiation poisoning; as published, the
article asserted that victims were succumbing solely to the sort of
injuries that one would expect from a conventional bombing. An
accompanying editorial note stated, ``United States scientists say the
atomic bomb will not have any lingering aftereffects in the devastated
area.''

Less than two months earlier, a group of United States scientists had
worried that the world's first nuclear explosion, the ultrasecret
Trinity test in New Mexico, might ignite the atmosphere. That did not
happen. Yet in a narrow sense, the scientists were right about lingering
effects at the blast site: Surprisingly soon after the bombings, the
residual radiation in Hiroshima and Nagasaki dropped to levels that
allowed the cities to begin to recover.

Image

John Hersey, on assignment in China, circa 1946. After the United States
dropped the atomic bombs, Hersey wrote that if civilization was to mean
anything, people had to acknowledge the humanity of their
enemies.Credit...Dmitri Kessel/The LIFE Picture Collection, via Getty
Images

But that was only half the radiation story. The other half consisted of
tens of thousands of people who had absorbed dangerous doses on the
mornings of the bombings and were now sickening and in some cases dying.
The U.S. Army officer who had directed the atomic bomb program, Lt. Gen.
Leslie Groves, dismissed reports of dangerous radiation as propaganda.
``I think our best answer to anyone who doubts this is that we did not
start the war, and if they don't like the way we ended it, to remember
who started it.'' This was obviously a non sequitur. By the fall of 1945
accounts of radiation sickness had become indisputable even by Groves.
Called to testify before a Senate committee on atomic energy, he
resorted to claiming that radiation poisoning ``is a very pleasant way
to die.''

Hatred blinds people. Hatred makes people stupid. John Hersey was
different. He was a New England sophisticate who had attended his
exalted schools on scholarships, and now stood as evidence that if
imbued with discipline and a deep education in the humanities,
patricians can be molded as well as bred. He was physically brave. As a
war correspondent he had willingly exposed himself to great danger. The
Army formally commended him for having rescued a wounded G.I. on
Guadalcanal. Characteristically, he explained that helping the man to
safety was the best way he knew to remove himself from the fight. No one
believed it. War correspondents move forward into fights. Hersey moved
forward a lot. But he was not a Hollywood tough guy. He was quiet,
self-effacing and empathetic. Throughout his experience with battle, and
despite the slurs he had written about the Japanese, he distinguished
between the idea of a hated enemy --- the Japanese as a swarm --- and
the reality of whatever individual was currently bringing him under
fire. ``Was he from Hakone, perhaps Hokkaido? What food was in his
knapsack? What private hopes had his conscription snatched from him?''

After the United States dropped the atomic bombs, Hersey wrote that if
civilization was to mean anything, people had to acknowledge the
humanity of their enemies. As the months passed he realized that this
was the element still lacking in descriptions of the devastation. It was
a failing of journalism, and an opportunity for him. With the backing of
The New Yorker --- specifically of the magazine's founder and editor,
Harold Ross, and his colleague William Shawn --- he flew in early 1946
to China, and from there found his way into Japan, where he managed to
obtain permission to visit Hiroshima. He was there for two weeks before
returning to New York to escape the censors and beginning to write. The
result was an austere, 30,000-word reportorial masterpiece that
described the experiences of six survivors of the atomic attack. That
August, The New Yorker devoted an entire issue to it. It made a huge
sensation. Knopf then published the story in book form as ``Hiroshima.''
It was translated into many languages. Millions of copies were sold
worldwide.

Today it exists as something of an artifact, a stunning work that
nonetheless has lost the power to engage largely because the stories it
contains have permeated our consciousness of nuclear war. Few people
read the original source anymore. That is unfortunate, but now --- 74
years after the book's publication, and 27 years after Hersey's death
--- help has arrived in the form of a tightly focused new book,
``Fallout,'' that unpacks the full story of the making of ``Hiroshima.''
The author is Lesley M. M. Blume, a tireless researcher and beautiful
writer, who moves through her narrative with seeming effortlessness ---
a trick that belies the skill and hard labor required to produce such
prose. Her previous nonfiction book, ``Everybody Behaves Badly,'' was a
purely literary work about the background of Hemingway's first novel,
``The Sun Also Rises''; though Blume's attributes as a writer were fully
apparent, the book suffered from requiring readers to care about
Hemingway and his narcissistic excesses.

Image

Lt. Gen. Leslie Groves, who directed the United States' atomic bomb
program, in 1945. He told a Senate committee that radiation poisoning
was ``a very pleasant way to die.''Credit...Associated Press

Such burdens are absent from ``Fallout.'' The subject of nuclear war is
too important not to fascinate, and though we have avoided it for 75
years, the possibility now looms closer than before. ``Fallout'' is a
warning without being a polemic. In the introduction Blume writes:
``Recently, climate change has been dominating headlines and
conversations as \emph{the} existential threat to human survival; yet
nuclear weapons continue to pose the other great existential threat ---
and that threat is accelerating. Climate change promises to rework the
world violently yet gradually. Nuclear war could spell instantaneous
global destruction, with little or no advance warning.''

Blume reminds us that Hersey's work still best describes what that would
look like on an intimate level; like his original reporting, ``Fallout''
is a book of serious intent that is nonetheless pleasant to read. There
are knowable reasons for this, including Blume's flawless paragraphs;
her clear narrative structure; her compelling stories, subplots and
insights; her descriptions of two great magazine editors establishing
the standards of integrity that continue at The New Yorker and other
high-end magazines today; the oddball characters like General Groves who
keep popping up; and most of all, the attractive qualities of her
protagonist, John Hersey. In a world sick with selfies, Hersey's
asceticism still stands out.

``Fallout'' does suffer from two flaws. The first is the claim that the
United States mounted an important cover-up to hide the realities of
radiation sickness from public knowledge. Blume's publisher chose to
hype this claim in the subtitle --- a mistake --- and then, in a letter
accompanying the advance proof, went so far as to describe the cover-up
as the biggest of the century and a ``cloak and dagger tale.'' It must
be embarrassing for Blume. It's obvious to anyone who has been around
the U.S. Army that whatever ineffective obfuscation occurred during the
months following the atomic bombings resulted from the same old stuff
--- a mixture of authentic ignorance, reflexive secrecy and incompetent
military spin. The book's second flaw is the unnecessary claim that
Hersey's work altered the course of history, changed attitudes toward
the arms race, and has helped the world avoid nuclear war ever since.
This is just silly, though there are indications that Hersey himself may
have believed some of it in his old age. If so, given his contributions
to humanity he may be excused. But what altered the course of history
was the acquisition of nuclear weapons by countries other than the
United States --- particularly the Soviet Union in 1948 --- and the
certainty of retaliation should ever a nuclear weapon be used again.
Were it not for that threat it seems likely that the United States would
have struck again against other foes --- North Korea, Russia, China,
North Vietnam, Cuba, somewhere in the Middle East? --- despite the
suffering described so powerfully in Hersey's ``Hiroshima.''

But against the scale of the subject these are quibbles, and do not
detract from the excellence of Blume's work. She ends the book with an
exhortation that connects with our time: ``The greatest tragedy of the
21st century may be that we have learned so little from the greatest
tragedies of the 20th century. Apparently catastrophe lessons need to be
experienced firsthand by each generation. So, here are some refreshers:
Nuclear conflict may mean the end of life on this planet. Mass
dehumanization can lead to genocide. The death of an independent press
can lead to tyranny and render a population helpless to protect itself
against a government that disdains law and conscience.'' She continues
in a similar vein, finishing with the optimistic assertion that the
opportunity to learn from history's tragedies has not yet passed. To
which an appreciative reader can only think: We'll see.

Advertisement

\protect\hyperlink{after-bottom}{Continue reading the main story}

\hypertarget{site-index}{%
\subsection{Site Index}\label{site-index}}

\hypertarget{site-information-navigation}{%
\subsection{Site Information
Navigation}\label{site-information-navigation}}

\begin{itemize}
\tightlist
\item
  \href{https://help.nytimes.com/hc/en-us/articles/115014792127-Copyright-notice}{©~2020~The
  New York Times Company}
\end{itemize}

\begin{itemize}
\tightlist
\item
  \href{https://www.nytco.com/}{NYTCo}
\item
  \href{https://help.nytimes.com/hc/en-us/articles/115015385887-Contact-Us}{Contact
  Us}
\item
  \href{https://www.nytco.com/careers/}{Work with us}
\item
  \href{https://nytmediakit.com/}{Advertise}
\item
  \href{http://www.tbrandstudio.com/}{T Brand Studio}
\item
  \href{https://www.nytimes.com/privacy/cookie-policy\#how-do-i-manage-trackers}{Your
  Ad Choices}
\item
  \href{https://www.nytimes.com/privacy}{Privacy}
\item
  \href{https://help.nytimes.com/hc/en-us/articles/115014893428-Terms-of-service}{Terms
  of Service}
\item
  \href{https://help.nytimes.com/hc/en-us/articles/115014893968-Terms-of-sale}{Terms
  of Sale}
\item
  \href{https://spiderbites.nytimes.com}{Site Map}
\item
  \href{https://help.nytimes.com/hc/en-us}{Help}
\item
  \href{https://www.nytimes.com/subscription?campaignId=37WXW}{Subscriptions}
\end{itemize}
