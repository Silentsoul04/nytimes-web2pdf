Sections

SEARCH

\protect\hyperlink{site-content}{Skip to
content}\protect\hyperlink{site-index}{Skip to site index}

\href{https://www.nytimes.com/section/books/review}{Book Review}

\href{https://myaccount.nytimes.com/auth/login?response_type=cookie\&client_id=vi}{}

\href{https://www.nytimes.com/section/todayspaper}{Today's Paper}

\href{/section/books/review}{Book Review}\textbar{}`Looking for Miss
America' Tells the History of the Legendary Pageant

\url{https://nyti.ms/3i9uf4b}

\begin{itemize}
\item
\item
\item
\item
\item
\end{itemize}

Advertisement

\protect\hyperlink{after-top}{Continue reading the main story}

Supported by

\protect\hyperlink{after-sponsor}{Continue reading the main story}

nonfiction

\hypertarget{looking-for-miss-america-tells-the-history-of-the-legendary-pageant}{%
\section{`Looking for Miss America' Tells the History of the Legendary
Pageant}\label{looking-for-miss-america-tells-the-history-of-the-legendary-pageant}}

\includegraphics{https://static01.nyt.com/images/2020/08/04/books/review/04Fischer1/merlin_105032884_440a458c-3575-495e-8312-a890f5ab2649-articleLarge.jpg?quality=75\&auto=webp\&disable=upscale}

Buy Book ▾

\begin{itemize}
\tightlist
\item
  \href{https://www.amazon.com/gp/search?index=books\&tag=NYTBSREV-20\&field-keywords=Looking+for+Miss+America\%3A+A+Pageant\%27s+100-Year+Quest+to+Define+Womanhood+Margot+Mifflin}{Amazon}
\item
  \href{https://du-gae-books-dot-nyt-du-prd.appspot.com/buy?title=Looking+for+Miss+America\%3A+A+Pageant\%27s+100-Year+Quest+to+Define+Womanhood\&author=Margot+Mifflin}{Apple
  Books}
\item
  \href{https://www.anrdoezrs.net/click-7990613-11819508?url=https\%3A\%2F\%2Fwww.barnesandnoble.com\%2Fw\%2F\%3Fean\%3D9781640092235}{Barnes
  and Noble}
\item
  \href{https://www.anrdoezrs.net/click-7990613-35140?url=https\%3A\%2F\%2Fwww.booksamillion.com\%2Fp\%2FLooking\%2Bfor\%2BMiss\%2BAmerica\%253A\%2BA\%2BPageant\%2527s\%2B100-Year\%2BQuest\%2Bto\%2BDefine\%2BWomanhood\%2FMargot\%2BMifflin\%2F9781640092235}{Books-A-Million}
\item
  \href{https://bookshop.org/a/3546/9781640092235}{Bookshop}
\item
  \href{https://www.indiebound.org/book/9781640092235?aff=NYT}{Indiebound}
\end{itemize}

When you purchase an independently reviewed book through our site, we
earn an affiliate commission.

By Molly Fischer

\begin{itemize}
\item
  Aug. 4, 2020, 5:00 a.m. ET
\item
  \begin{itemize}
  \item
  \item
  \item
  \item
  \item
  \end{itemize}
\end{itemize}

\textbf{LOOKING FOR MISS AMERICA}\\
\textbf{A Pageant's 100-Year Quest to Define Womanhood}\\
By Margot Mifflin

First of all, in case you wondered, Donald Trump does not own, nor has
he ever owned, the Miss America pageant. He owned the other one --- Miss
USA. Margot Mifflin makes this clarification a few times in the course
of her book ``Looking for Miss America: A Pageant's 100-Year Quest to
Define Womanhood.''

``When I tell people the topic of this book, 90 percent respond by
saying it's timely because of Trump,'' Mifflin explains in one
parenthetical aside. The president has bragged about surprising teenage
pageant contestants in their dressing room and once ``famously
fat-shamed a Miss Universe.'' In the minds of many, including Mifflin's
interlocutors, ``this was all Miss America scandal,'' she writes --- but
no.

The need to draw the distinction is revealing. Today, the Miss America
pageant is culturally marginal enough for the average person to possess
only a blurry awareness that it persists. This average person isn't
keeping it straight from rival pageants,
\href{https://www.hollywoodreporter.com/live-feed/miss-america-2020-tv-ratings-thursday-dec-19-2019-1264411\#:~:text=For\%20the\%20second\%20straight\%20year,rating\%20in\%20adults\%2018\%2D49.}{much
less reliably tuning in}. But at the same time, for its partisans,
there's also long been a need to hold Miss America apart from other
pageants. In the words of Lenora Slaughter, the woman who gave Miss
America its enduring shape and served as the pageant's director for 32
years: ``It couldn't just be a beauty contest.''

\includegraphics{https://static01.nyt.com/images/2020/07/01/books/review/Fischer2/Fischer2-articleLarge.jpg?quality=75\&auto=webp\&disable=upscale}

The challenge Mifflin sets herself in ``Looking for Miss America'' is
articulating what exactly makes this nearly 100-year-old institution
something more. For Slaughter, distinguishing Miss America meant
offering a talent competition and scholarships; today, the pageant touts
itself as ``one of the nation's largest providers of scholarship
assistance to young women,'' with \$3 million awarded annually. This
branding sidesteps the question of what scholarships have to do with
swimsuits --- Miss America maintained its swimsuit competition until
2018 --- and thus tends to inspire a certain amount of skepticism. (``It
is \emph{not} a beauty pageant; it is a scholarship program!'' Sandra
Bullock snaps, of the fictional pageant in which she competes in ``Miss
Congeniality.'')

Mifflin is no Miss America apologist. She's cleareyed about the
pageant's many hypocrisies and failures, which include a legacy of
racial exclusion that endured long after a rule requiring contestants to
be ``in good health and of the white race'' was scrapped in the 1950s.
But Mifflin, too, is invested in the pageant's sense of specialness;
she's mining Miss America for meaning, which requires making the case
that there's meaning to be found. ``The pageant has wormed its way into
our national subconscious,'' she writes. A different kind of book might
be written about the subculture that has sprung up around Miss America,
about the feeder pageants and local traditions that make up the lived
experience of the pageant for most of the thousands of women who
compete. (Today, they number around 4,000; in the 1980s, the figure was
more like 80,000.) ``Looking for Miss America'' focuses instead on the
pageant's mass-culture significance --- the stage it has offered and the
kind of public figures it has produced.

Who is Miss America? She's not quite a first lady and not quite a
Playboy Bunny, but she shares some qualifications and job
responsibilities with both. She plays a ceremonial role that's patriotic
without being democratic, simultaneously quasi-royal and girl-next-door,
and also, on occasion, under-clothed. She represents some unstable
combination of qualities that Americans might want to salute, feel up or
be.

Mifflin tracks the evolution of that peculiar role alongside the
shifting expectations and ideals of femininity in America, from flappers
to Rosie the Riveter to Helen Gurley Brown to ``empowerment''
doublespeak. The marks she hits are largely familiar, and her galloping
pace through a century of pop culture --- 310 pages pass swiftly ---
produces some moments of Wikipedia on speed. (``In the 1970s, punk music
channeled white-knuckled anger and nihilistic despair, and `Saturday
Night Live' lampooned celebrities and politicians.'') ``Looking for Miss
America'' is at its best when Mifflin pauses this sweeping summary to
tell the stories of individual contestants. The pageant's tensions and
ambiguities emerge most vividly through the way particular women
understood them in the context of their particular time.

Image

Vanessa Williams, Miss America, 1984. The only winner to resign the
title, Williams went on to become more successful than any Miss America
had ever been.Credit...Danny Drake/Associated Press

For Yolande Betbeze --- the 1951 winner, and one of Mifflin's most
affectionate portraits --- Miss America was a ticket out of Mobile,
Ala., to New York City, where she studied philosophy at the New School
and acting with Stella Adler. Betbeze got involved with the civil rights
movement and started an Off Broadway theater; in later life, she became
a fixture of D.C. society, maintaining a long-term affair with an
Algerian resistance fighter turned diplomat.

She became a vocal critic of Miss America, particularly its white-bread
homogeneity, but her own experiences gained literary immortality: She
advised Philip Roth on his portrayal of an ex-Miss New Jersey in
``American Pastoral*.*'' And she also managed to rewrite Miss America's
job description by refusing to model swimsuits during her reign. (The
pageant's swimwear sponsor was sufficiently aggrieved that it abandoned
Miss America and launched a pageant of its own --- Miss USA, eventually
acquired by Trump.)

As much as the history of Miss America is about womanhood, it's also
about celebrity, and the developing American attitude toward fame. In
the pageant's earliest years, seeking to capitalize on the event's
publicity was considered unsavory; in the modern era, the contest has
been matter-of-factly regarded as a steppingstone to public life.
Perhaps the pageant's most illuminating winner is Vanessa Williams: In
1984, Williams became the first Black woman crowned Miss America, and
was applauded as a barrier-breaker even by some pageant skeptics ---
while also becoming the object of death threats from strangers and
racist jokes from Johnny Carson. Then, 10 months into her reign,
Williams learned Penthouse planned to publish nude photos obtained
without her permission. When the magazine came out, she was given 72
hours to resign her title. Mifflin writes, ``Williams was the pageant's
own Hester Prynne, the first and only winner to be dethroned, whose
transgression only intensified her aura.'' Indeed, a funny thing
happened after Williams stopped being Miss America: She became more
successful than any Miss America had ever been. Her debut album went
platinum; she collected Grammy, Tony and Emmy nominations; she's enjoyed
a long career onstage and onscreen. Her ascendancy to the status of
actual famous person made winning Miss America seem provincial in
comparison.

In 1970, the pageant had 22 million viewers; by 2000, viewership had
declined to 8.8 million. The 21st century grew increasingly removed from
the years in which Miss America had been a prime target for feminist
protest, when a critic might plausibly write of ``turning on the TV to
watch at least one or two beauty contests each year,'' as Pauline Kael
did in 1975. Reality TV has usurped its appeal as entertainment, and
ambitious young women who want to capitalize on looks and charisma don't
need an organized competition; they've got Instagram.

The Miss America pageant's earliest origins, back in 1921, lay in local
business owners' desire to extend the Atlantic City tourist season from
the summer months and into early fall. After the 2018 pageant, the
city's Casino Reinvestment Development Authority pulled its subsidies
and ousted Miss America from its traditional venue at Boardwalk Hall ---
a mighty symbolic blow to an institution already struggling to find its
footing in a changed world. (Gretchen Carlson, Miss America 1989,
assumed a leadership role in the aftermath of \#MeToo, but her tenure
was controversial and short-lived.) The commercial promise that saw the
pageant through shifting winds of feminism and fame would seem, at
present, to have mostly disappeared. Mifflin's lively book reads as an
obituary.

Advertisement

\protect\hyperlink{after-bottom}{Continue reading the main story}

\hypertarget{site-index}{%
\subsection{Site Index}\label{site-index}}

\hypertarget{site-information-navigation}{%
\subsection{Site Information
Navigation}\label{site-information-navigation}}

\begin{itemize}
\tightlist
\item
  \href{https://help.nytimes.com/hc/en-us/articles/115014792127-Copyright-notice}{©~2020~The
  New York Times Company}
\end{itemize}

\begin{itemize}
\tightlist
\item
  \href{https://www.nytco.com/}{NYTCo}
\item
  \href{https://help.nytimes.com/hc/en-us/articles/115015385887-Contact-Us}{Contact
  Us}
\item
  \href{https://www.nytco.com/careers/}{Work with us}
\item
  \href{https://nytmediakit.com/}{Advertise}
\item
  \href{http://www.tbrandstudio.com/}{T Brand Studio}
\item
  \href{https://www.nytimes.com/privacy/cookie-policy\#how-do-i-manage-trackers}{Your
  Ad Choices}
\item
  \href{https://www.nytimes.com/privacy}{Privacy}
\item
  \href{https://help.nytimes.com/hc/en-us/articles/115014893428-Terms-of-service}{Terms
  of Service}
\item
  \href{https://help.nytimes.com/hc/en-us/articles/115014893968-Terms-of-sale}{Terms
  of Sale}
\item
  \href{https://spiderbites.nytimes.com}{Site Map}
\item
  \href{https://help.nytimes.com/hc/en-us}{Help}
\item
  \href{https://www.nytimes.com/subscription?campaignId=37WXW}{Subscriptions}
\end{itemize}
