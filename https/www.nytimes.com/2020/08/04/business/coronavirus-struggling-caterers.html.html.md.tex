Sections

SEARCH

\protect\hyperlink{site-content}{Skip to
content}\protect\hyperlink{site-index}{Skip to site index}

\href{https://www.nytimes.com/section/business}{Business}

\href{https://myaccount.nytimes.com/auth/login?response_type=cookie\&client_id=vi}{}

\href{https://www.nytimes.com/section/todayspaper}{Today's Paper}

\href{/section/business}{Business}\textbar{}`Closing Isn't Even an
Option': With No Events, Caterers Rush to Adjust

\url{https://nyti.ms/2XqueRe}

\begin{itemize}
\item
\item
\item
\item
\item
\item
\end{itemize}

\href{https://www.nytimes.com/spotlight/at-home?action=click\&pgtype=Article\&state=default\&region=TOP_BANNER\&context=at_home_menu}{At
Home}

\begin{itemize}
\tightlist
\item
  \href{https://www.nytimes.com/2020/08/03/well/family/the-benefits-of-talking-to-strangers.html?action=click\&pgtype=Article\&state=default\&region=TOP_BANNER\&context=at_home_menu}{Talk:
  To Strangers}
\item
  \href{https://www.nytimes.com/2020/08/01/at-home/coronavirus-make-pizza-on-a-grill.html?action=click\&pgtype=Article\&state=default\&region=TOP_BANNER\&context=at_home_menu}{Make:
  Grilled Pizza}
\item
  \href{https://www.nytimes.com/2020/07/31/arts/television/goldbergs-abc-stream.html?action=click\&pgtype=Article\&state=default\&region=TOP_BANNER\&context=at_home_menu}{Watch:
  'The Goldbergs'}
\item
  \href{https://www.nytimes.com/interactive/2020/at-home/even-more-reporters-editors-diaries-lists-recommendations.html?action=click\&pgtype=Article\&state=default\&region=TOP_BANNER\&context=at_home_menu}{Explore:
  Reporters' Google Docs}
\end{itemize}

Advertisement

\protect\hyperlink{after-top}{Continue reading the main story}

Supported by

\protect\hyperlink{after-sponsor}{Continue reading the main story}

\hypertarget{closing-isnt-even-an-option-with-no-events-caterers-rush-to-adjust}{%
\section{`Closing Isn't Even an Option': With No Events, Caterers Rush
to
Adjust}\label{closing-isnt-even-an-option-with-no-events-caterers-rush-to-adjust}}

The pandemic has devastated the industry. But some entrepreneurs are
finding creative ways to keep their businesses afloat.

\includegraphics{https://static01.nyt.com/images/2020/07/31/business/28virus-caterers-sub1/merlin_174904206_96c765ea-75dd-4f9c-bd6f-7ad40ea3e112-articleLarge.jpg?quality=75\&auto=webp\&disable=upscale}

By \href{https://www.nytimes.com/by/julie-creswell}{Julie Creswell}

\begin{itemize}
\item
  Aug. 4, 2020
\item
  \begin{itemize}
  \item
  \item
  \item
  \item
  \item
  \item
  \end{itemize}
\end{itemize}

On a recent Saturday, petite lobster rolls on toasted brioche and
coconut shrimp with mango aioli were to be passed among the guests at a
210-person wedding. A bar mitzvah party for 180 was going to conclude
with torched s'mores and a chocolate fountain.

For David Cingari of David's Soundview Catering in Stamford, Conn., the
events, along with food for an anniversary party, should have brought in
roughly \$6,600 in profits.

Instead, he was dashing about, serving lobster rolls, blackened
mahi-mahi tacos and smashburgers alongside cocktails like the Painkiller
to socially distanced diners at a pop-up restaurant he opened in
mid-June.

His take? About \$600.

The restaurant, David's at the Landing, is the third iteration of Mr.
Cingari's catering business since
\href{https://www.nytimes.com/news-event/coronavirus}{the coronavirus
pandemic} struck, bringing his \$7 million-a-year company to a sudden
stop.

``We were going to do \$300,000 in graduation parties this spring,'' he
said. ``That's just gone.''

The pandemic has the nation's caterers --- roughly 12,000 individuals or
companies with annual revenues of more than \$60 billion --- reeling.
Many caterers say they expect their business to be down between 80 and
90 percent this year. Corporate cafeterias that they provide food and
staff to remain closed. Events like graduation and anniversary parties,
bar mitzvahs, charity dinners and weddings have been canceled or pushed
into next year.

And the ones that took place were on a decidedly smaller scale.

``We did one 50-person wedding,'' Mr. Cingari said. ``It was a clambake
in the backyard. That was supposed to be a 250-person wedding.''

The collapse of the catering industry this year directly affects
bartenders, wait staff and others who typically work these events as
part-time employees.

\includegraphics{https://static01.nyt.com/images/2020/07/31/opinion/27virus-caterers2/merlin_174904932_46fa1c28-c79b-44cc-9b1c-0d638cd5a397-articleLarge.jpg?quality=75\&auto=webp\&disable=upscale}

The industry --- a collection of large corporations like Aramark and
Compass Group and thousands of smaller companies owned by individuals
--- is not tracking how many caterers have permanently closed because of
the pandemic, but they say it will happen.

``If I look locally at South Jersey, I know of a few caterers and some
venues that are severely struggling,'' said Doug Quattrini, the
president of the National Association for Catering and Events and an
event producer at Sensational Host in Maple Shade, N.J.

While caterers say they are taking a financial beating, many feel better
situated than those in the restaurant business. (Not surprisingly, many
caterers worked in restaurants before switching jobs.) Instead of paying
often expensive rent in desirable locations like most restaurants,
caterers typically pay less for large kitchens that can be off the
beaten track.

Moreover, caterers tend to be a nimble group of entrepreneurs, adept at
providing finicky couples with their every heart's whim and overcoming
the oddest of logistical challenges. Those traits have helped them
during the pandemic.

``We have huge logistical expertise,'' said Peter Callahan of Peter
Callahan Catering, whose clients include some of New York's wealthiest
financiers and whose specialty is mini food like one-bite cheeseburgers
and tiny grilled cheese sandwiches. ``When you're an off-premise
caterer, you might be doing an event that requires barges to get to a
private island with no vehicles.

``We're creative thinkers, and right now people are thinking about how
to shape their businesses for the need at hand,'' he added.

As the traditional bar-mitzvah-and-wedding circuit collapsed, caterers
began to think about different ways to make money.

``It's the year of the pivot,'' said Holly Sheppard, who spent years
working as a line cook at New York City restaurants before starting her
Brooklyn catering business, Fig \& Pig, in 2011.

Image

``I'm going to be a female pitmaster on the roadside in upstate New York
until the weddings come back,'' said Holly Sheppard, who started her
Brooklyn catering business, Fig \& Pig, in 2011.Credit...Amr Alfiky/The
New York Times

Ms. Sheppard was in the middle of preparing a meal for 600 people in
mid-March when the client called, canceling the event. The food, which
had already been paid for, was donated.

After that, Ms. Sheppard said, the cancellations and postponements
rolled in. Of the 47 weddings she had scheduled for this year, 40 have
been pushed into next year. The others were canceled outright.

With her calendar now largely empty through the fall, Ms. Sheppard gave
up the lease on her apartment in Brooklyn, worked out a deal with the
landlord for her kitchen to pay what she can now and make it up next
year, and moved to her house in Tillson, N.Y.

There, she bought a smoker and is honing her skills, planning to add
barbecue to her catering options.

``I'm going to be a female pitmaster on the roadside in upstate New York
until the weddings come back,'' Ms. Sheppard said. ``I'm going to make
it through all of this. Closing isn't even an option. I'm a scrapper.''

Mr. Cingari has been hustling in the food industry for four decades and
has no intention of letting the coronavirus end his business.

After working as a hotel chef at the Grand Hyatt in New York, Mr.
Cingari opened a restaurant, David's American Food and Drink, in
Stamford in 1987.

But after a decade of long hours, constant staff turnover and long
nights worrying about paying his \$13,000-a-month lease, Mr. Cingari,
whose family owns ShopRite grocery stores in the area, decided to close
the restaurant in 1997 and focus solely on his catering operation.

The business took off, and by the end of the year, David's Soundview
Catering had 85 employees preparing food out of a 6,000-square-foot
commissary kitchen.

About 80 percent of the business came from delivering breakfast and
lunches for corporate meetings and from preparing food for and staffing
more than a dozen corporate cafeterias in the area. On weekends, Mr.
Cingari's calendar was filled with weddings, anniversary parties and bar
mitzvahs.

Image

Mr. Cingari's business has been through several iterations during the
pandemic.Credit...Amr Alfiky/The New York Times

The first inkling Mr. Cingari received that this year was going to be
anything but normal came in late February when he was notified that the
employees of a Japanese-based company in one of the buildings where he
managed the cafeteria would be working from home as part of an emergency
response trial. A week later, a large international bank said it would
be doing the same thing.

``It was like wildfire,'' he said. ``Within three weeks, every one of
the cafeterias were closed and any event we had on the books was
canceled.''

Mr. Cingari said he had received money from the federal
\href{https://www.nytimes.com/article/small-business-loans-stimulus-grants-freelancers-coronavirus.html}{Paycheck
Protection Program} to cover around 80 of his employees.

As companies shut down and people began staying at home in mid-March,
Mr. Cingari shifted his business. He had noticed how people were raising
money on social media to provide meals to hospitals and emergency
medical workers, so he did the same. The money donated through the
social media outreach paid for the cost of food and supplies.

``Since we had this large commissary kitchen, we could do huge numbers
of meals,'' he said, though he made no profit from it. ``So we started
making a few thousand meals a day for several weeks to feed hospital
workers and others.''

That effort began to dry up as coronavirus cases declined in Connecticut
in the late spring.

So Mr. Cingari shifted again, this time providing groceries,
hard-to-find household items like toilet paper and Clorox disinfecting
wipes, and take-home meals for \$50 that could feed a family of four. In
early June he would sell close to 60 meals on a Saturday night, he said.

``It didn't even come close to what we were making before,'' he said,
``but it was something.''

But that business petered out when the state allowed outdoor dining. On
the final weekend of that iteration of his business, Mr. Cingari sold
five take-home meals.

So in early July, he shifted again. Through one of the buildings in a
corporate office park where he manages the cafeteria, he had access to
an indoor dining area and outdoor patio space overlooking the harbor in
Stamford. He had used the space in the past for weekend events like
birthday parties and bar mitzvahs.

Image

``I can't believe I'm back in the restaurant business,'' Mr. Cingari
said. ``Shoot me.''Credit...Amr Alfiky/The New York Times

Now, on that outdoor patio, Mr. Cingari has started a pop-up restaurant,
David's at the Landing. The restaurant is open Thursday through Saturday
nights and serves a limited menu of appetizers, five entrees, cocktails,
wines and beers. On a recent Saturday evening, the wait time for a table
at the restaurant, which seats 65 with social distancing, was nearly two
hours, he said.

``I can't believe I'm back in the restaurant business,'' Mr. Cingari
said. ``Shoot me. Still, the business is covering costs and making a
little bit of money for the eight people who are working there.''

This latest incarnation will also be short-lived, likely to close in
mid-September as the weather in Connecticut turns cooler.

Mr. Cingari had hoped the corporate cafeteria side of his business would
come back at least a little bit by the fall. But with
\href{https://www.nytimes.com/interactive/2020/us/coronavirus-us-cases.html}{coronavirus
cases spiking} in different parts of the country, he now has his doubts
about that.

``It's all I think about all day and all night,'' he said. ``I just hope
that another pivot comes to mind by mid-September that will hold us
until January. There has to be some way. I have too many good people and
too much wisdom under my belt to not be able to figure this out.''

Advertisement

\protect\hyperlink{after-bottom}{Continue reading the main story}

\hypertarget{site-index}{%
\subsection{Site Index}\label{site-index}}

\hypertarget{site-information-navigation}{%
\subsection{Site Information
Navigation}\label{site-information-navigation}}

\begin{itemize}
\tightlist
\item
  \href{https://help.nytimes.com/hc/en-us/articles/115014792127-Copyright-notice}{©~2020~The
  New York Times Company}
\end{itemize}

\begin{itemize}
\tightlist
\item
  \href{https://www.nytco.com/}{NYTCo}
\item
  \href{https://help.nytimes.com/hc/en-us/articles/115015385887-Contact-Us}{Contact
  Us}
\item
  \href{https://www.nytco.com/careers/}{Work with us}
\item
  \href{https://nytmediakit.com/}{Advertise}
\item
  \href{http://www.tbrandstudio.com/}{T Brand Studio}
\item
  \href{https://www.nytimes.com/privacy/cookie-policy\#how-do-i-manage-trackers}{Your
  Ad Choices}
\item
  \href{https://www.nytimes.com/privacy}{Privacy}
\item
  \href{https://help.nytimes.com/hc/en-us/articles/115014893428-Terms-of-service}{Terms
  of Service}
\item
  \href{https://help.nytimes.com/hc/en-us/articles/115014893968-Terms-of-sale}{Terms
  of Sale}
\item
  \href{https://spiderbites.nytimes.com}{Site Map}
\item
  \href{https://help.nytimes.com/hc/en-us}{Help}
\item
  \href{https://www.nytimes.com/subscription?campaignId=37WXW}{Subscriptions}
\end{itemize}
