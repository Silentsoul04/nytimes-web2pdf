Sections

SEARCH

\protect\hyperlink{site-content}{Skip to
content}\protect\hyperlink{site-index}{Skip to site index}

\href{https://www.nytimes.com/section/business/media}{Media}

\href{https://myaccount.nytimes.com/auth/login?response_type=cookie\&client_id=vi}{}

\href{https://www.nytimes.com/section/todayspaper}{Today's Paper}

\href{/section/business/media}{Media}\textbar{}McClatchy, Family-Run
News Chain, Is Set to Go to Hedge Fund in Bankruptcy Sale

\url{https://nyti.ms/3ic1QdL}

\begin{itemize}
\item
\item
\item
\item
\item
\end{itemize}

Advertisement

\protect\hyperlink{after-top}{Continue reading the main story}

Supported by

\protect\hyperlink{after-sponsor}{Continue reading the main story}

\hypertarget{mcclatchy-family-run-news-chain-is-set-to-go-to-hedge-fund-in-bankruptcy-sale}{%
\section{McClatchy, Family-Run News Chain, Is Set to Go to Hedge Fund in
Bankruptcy
Sale}\label{mcclatchy-family-run-news-chain-is-set-to-go-to-hedge-fund-in-bankruptcy-sale}}

A federal judge is expected to sign off on a \$312 million deal for the
prestigious newspaper company to Chatham Asset Management at a hearing
Tuesday.

\includegraphics{https://static01.nyt.com/images/2020/08/04/business/04Mcclatchy-01/merlin_174636087_f90cc924-bf21-4019-bd5a-e6d1c79dbab5-articleLarge.jpg?quality=75\&auto=webp\&disable=upscale}

\href{https://www.nytimes.com/by/marc-tracy}{\includegraphics{https://static01.nyt.com/images/2018/02/20/multimedia/author-marc-tracy/author-marc-tracy-thumbLarge.jpg}}

By \href{https://www.nytimes.com/by/marc-tracy}{Marc Tracy}

\begin{itemize}
\item
  Aug. 4, 2020Updated 2:41 p.m. ET
\item
  \begin{itemize}
  \item
  \item
  \item
  \item
  \item
  \end{itemize}
\end{itemize}

The history of the newspaper business will be on vivid display at a
hearing scheduled for Tuesday, when a federal bankruptcy judge is
expected to confirm the sale of the
\href{https://www.nytimes.com/2020/07/12/business/media/hedge-fund-mcclatchy-newspapers.html}{McClatchy
Company}, a newspaper chain run by the same family since 1857, to a New
Jersey hedge fund in a deal valued at \$312 million.

The sale of McClatchy, the owner of The Sacramento Bee, The Miami Herald
and more than two dozen other news outlets in 14 states, to Chatham
Asset Management, a fund that controls more than \$4 billion in assets,
has been in the works since February, when McClatchy filed for Chapter
11 bankruptcy protection after more than a decade of losses and
cutbacks.

The expected deal, which will move a prestigious news publisher from
family control to an investment company, is in keeping with a broader
trend that has alarmed many journalists and press advocates, who argue
that finance firms are imperfect stewards of an industry built on the
watchdog work of chronicling government and commerce.

At the 3 p.m. hearing, Judge Michael E. Wiles will be asked to approve
the result of an auction held last month in which Chatham emerged as the
winning bidder for the distressed company.

The hedge fund offered to convert the more than \$262 million it owns in
McClatchy debt into equity in a Chatham-owned version of the company. It
will also throw in roughly \$49 million in cash, a court filing said,
and has agreed to pay additional costs, including employee payroll, that
McClatchy incurred since filing for bankruptcy.

After the sale becomes official --- probably before October --- Chatham
is expected to become the owner, with the publicly traded McClatchy
going private. Its chairman, Kevin S. McClatchy, the
great-great-grandson of the company founder, James McClatchy, and its
chief executive, Craig Forman, said they planned to depart once the deal
closes.

\includegraphics{https://static01.nyt.com/images/2020/08/04/business/04mcclatchy-02/merlin_175293861_bc21fcc9-28b9-4d59-8a63-924b476f13cf-articleLarge.jpg?quality=75\&auto=webp\&disable=upscale}

The company will not be split up, according to the terms of the
agreement. In a July 24 news release, McClatchy said Chatham's bid would
allow it to retain most of its employees and honor its collective
bargaining agreements.

The runner-up bidder,
\href{https://www.nytimes.com/2020/07/02/business/media/tribune-alden-board-seat.html}{Alden
Global Capital}, a New York hedge fund that controls more than 200 news
outlets through MediaNews Group, lost out after having offered what
amounted to approximately \$250 million, or approximately \$100 million
less than the value of Chatham's offer, according to a court filing. The
offer included \$40 million in cash, along with \$170 million in new
debt, plus some payroll costs and a tax adjustment. (An unspecified
third party made a bid that was deemed insufficient, according to a
filing.)

Chatham does not have as much of a hold on the news industry as Alden,
but it is getting there. In 2016, the fund took a majority stake in
Postmedia, one of Canada's largest newspaper companies. Since that deal
went through,
\href{https://www.nytimes.com/2020/07/16/business/media/hedge-fund-chatham-mcclatchy-postmedia-newspapers.html}{1,600
Postmedia employees} have been laid off, and more than 30 of its
publications have been shut down.

Chatham is also the principal owner of American Media Inc., the parent
company of The National Enquirer and other supermarket tabloids. The
fund has tried to unload The Enquirer. In 2018, American Media
\href{https://www.nytimes.com/2019/04/18/business/media/national-enquirer-james-cohen-hudson-news.html}{announced}
the sale of The Enquirer, which came under federal scrutiny for
\href{https://www.nytimes.com/2018/04/12/us/national-enquirer-doorman-trump.html}{its
role} in the 2016 presidential election, to the family that founded
Hudson News, a chain of newspaper and magazine shops. That deal has not
closed.

By taking over McClatchy, a consistent winner of prestigious journalism
awards, Chatham will acquire 31 news outlets in the United States. In
addition to the McClatchy flagship paper, The Bee --- which was founded
in the wake of the California gold rush --- the chain includes The
Charlotte Observer, The Kansas City Star and the news agency
McClatchyDC.

The mayors of several cities with McClatchy dailies, including
Sacramento and Lexington, Ky., filed letters with the bankruptcy court
urging civic-minded local ownership. Sree Sreenivasan, a professor of
digital innovation at Stony Brook University's School of Journalism,
noted with dismay the absence of bidders who were not part of the
finance world.

``It's a sad moment, because that tells you that people who
traditionally might have supported local journalism --- including people
with local connections, local stakeholders --- were not there,'' Mr.
Sreenivasan said.

The completion of the sale would extend the finance industry's huge sway
over local news coverage. The nation's largest newspaper chain,
\href{https://www.nytimes.com/2019/11/19/business/media/gannett-gatehouse-merger.html}{Gannett},
the publisher of USA Today and some 250 other dailies, owes significant
debt to one private equity fund, Apollo Global Management, and is
controlled by another private equity fund, Fortress Investment Group,
which is owned by the Japanese conglomerate SoftBank.

Alden, in addition to its roughly 200 news outlets, has a 32 percent
stake in another major chain,
\href{https://www.nytimes.com/2020/07/02/business/media/tribune-alden-board-seat.html}{Tribune
Publishing}. It also owns a large stake in Lee Enterprises, which
publishes The Buffalo News and roughly 75 other dailies in 26 states.

\href{https://www.nytimes.com/2006/03/13/business/media/newspaper-chain-agrees-to-a-sale-for-45-billion.html}{McClatchy's
troubles} can be traced to 2006, when it bought its much larger rival,
Knight Ridder, for \$4.5 billion, plus the assumption of \$2 billion in
debt. From shortly after the merger to the end of 2018, McClatchy cut
its work force from more than 15,000 full-time employees to around
3,300, according to public filings.

Chatham started investing in McClatchy in 2009 on its way to becoming
the chain's largest creditor. That put Chatham in a strong position to
strike a deal once McClatchy filed for bankruptcy, citing its inability
to meet obligations that were part of a \$1.4 billion pension plan meant
to provide money to more than 24,000 current and future retirees.

Chatham is led by Anthony Melchiorre, a Chicago-area native who earned a
reputation on Wall Street as a tough negotiator during stints at elite
firms like Goldman Sachs and Morgan Stanley. In 2002, he was
\href{https://www.fnlondon.com/articles/morgan-stanley-cuts-leveraged-finance-as-bank-cull-hits-2500-20021121}{let
go} from Morgan Stanley as part of a round of layoffs. Soon after, he
set up
\href{https://www.sec.gov/Archives/edgar/data/915802/000091580217000002/redmontprochathamsupplement0.htm}{his
own} hedge fund in Chatham, N.J. Some of its clients are listed under a
Cayman Islands address, where more favorable tax rates apply.

Another hedge fund that holds McClatchy debt, Brigade Capital
Management, had been listed as a Chatham partner in earlier filings, but
it will not have an ownership role under the agreement, according to
public filings.

Hedge funds often find bargains by taking on properties that may seem
unattractive. The newspaper industry has been struggling for years as
the rise of digital media has cut deeply into advertising and
circulation revenue. Roughly a quarter of the newspapers in the United
States, most of them weeklies, were shut down between 2004 and 2019, and
about 50 percent of newspaper jobs were eliminated in that time.

Making the outlook grimmer is the economic slowdown imposed by the
coronavirus pandemic --- meaning the new owner of McClatchy is in for a
challenge.

``We know the outlook is unpredictable in every single metric,'' Mr.
Sreenivasan said. ``How do even the best-intentioned of owners plan for
looking ahead?''

Advertisement

\protect\hyperlink{after-bottom}{Continue reading the main story}

\hypertarget{site-index}{%
\subsection{Site Index}\label{site-index}}

\hypertarget{site-information-navigation}{%
\subsection{Site Information
Navigation}\label{site-information-navigation}}

\begin{itemize}
\tightlist
\item
  \href{https://help.nytimes.com/hc/en-us/articles/115014792127-Copyright-notice}{©~2020~The
  New York Times Company}
\end{itemize}

\begin{itemize}
\tightlist
\item
  \href{https://www.nytco.com/}{NYTCo}
\item
  \href{https://help.nytimes.com/hc/en-us/articles/115015385887-Contact-Us}{Contact
  Us}
\item
  \href{https://www.nytco.com/careers/}{Work with us}
\item
  \href{https://nytmediakit.com/}{Advertise}
\item
  \href{http://www.tbrandstudio.com/}{T Brand Studio}
\item
  \href{https://www.nytimes.com/privacy/cookie-policy\#how-do-i-manage-trackers}{Your
  Ad Choices}
\item
  \href{https://www.nytimes.com/privacy}{Privacy}
\item
  \href{https://help.nytimes.com/hc/en-us/articles/115014893428-Terms-of-service}{Terms
  of Service}
\item
  \href{https://help.nytimes.com/hc/en-us/articles/115014893968-Terms-of-sale}{Terms
  of Sale}
\item
  \href{https://spiderbites.nytimes.com}{Site Map}
\item
  \href{https://help.nytimes.com/hc/en-us}{Help}
\item
  \href{https://www.nytimes.com/subscription?campaignId=37WXW}{Subscriptions}
\end{itemize}
