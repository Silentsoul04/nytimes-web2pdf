\href{/section/business}{Business}\textbar{}For Robots, It's a Time to
Shine (and Maybe Disinfect)

\url{https://nyti.ms/3fvB4eq}

\begin{itemize}
\item
\item
\item
\item
\item
\end{itemize}

\href{https://www.nytimes.com/news-event/coronavirus?action=click\&pgtype=Article\&state=default\&region=TOP_BANNER\&context=storylines_menu}{The
Coronavirus Outbreak}

\begin{itemize}
\tightlist
\item
  live\href{https://www.nytimes.com/2020/08/04/world/coronavirus-cases.html?action=click\&pgtype=Article\&state=default\&region=TOP_BANNER\&context=storylines_menu}{Latest
  Updates}
\item
  \href{https://www.nytimes.com/interactive/2020/us/coronavirus-us-cases.html?action=click\&pgtype=Article\&state=default\&region=TOP_BANNER\&context=storylines_menu}{Maps
  and Cases}
\item
  \href{https://www.nytimes.com/interactive/2020/science/coronavirus-vaccine-tracker.html?action=click\&pgtype=Article\&state=default\&region=TOP_BANNER\&context=storylines_menu}{Vaccine
  Tracker}
\item
  \href{https://www.nytimes.com/2020/08/02/us/covid-college-reopening.html?action=click\&pgtype=Article\&state=default\&region=TOP_BANNER\&context=storylines_menu}{College
  Reopening}
\item
  \href{https://www.nytimes.com/live/2020/08/04/business/stock-market-today-coronavirus?action=click\&pgtype=Article\&state=default\&region=TOP_BANNER\&context=storylines_menu}{Economy}
\end{itemize}

\includegraphics{https://static01.nyt.com/images/2020/08/05/business/04Virus-Robots-01/04Virus-Robots-01-articleLarge.jpg?quality=75\&auto=webp\&disable=upscale}

Sections

\protect\hyperlink{site-content}{Skip to
content}\protect\hyperlink{site-index}{Skip to site index}

Square Feet

\hypertarget{for-robots-its-a-time-to-shine-and-maybe-disinfect}{%
\section{For Robots, It's a Time to Shine (and Maybe
Disinfect)}\label{for-robots-its-a-time-to-shine-and-maybe-disinfect}}

The pandemic has turned cleaning and other mundane building tasks into a
challenge, stoking interest in machines as cost-effective solutions.

The Neo floor-scrubbing robot at Cincinnati/Northern Kentucky
International Airport.~Demand for the robot has shot up 100 percent
since March. Credit...Ty Wright for The New York Times

Supported by

\protect\hyperlink{after-sponsor}{Continue reading the main story}

By Lisa Prevost

\begin{itemize}
\item
  Aug. 4, 2020Updated 12:32 p.m. ET
\item
  \begin{itemize}
  \item
  \item
  \item
  \item
  \item
  \end{itemize}
\end{itemize}

The Neo is a four-foot-tall, 1,000-pound robot floor scrubber. The
high-tech machine can cruise large commercial buildings on its own, with
no human supervision required.

Since its introduction in 2016, Neo's sales have roughly doubled each
year, said Faizan Sheikh, the chief executive and a co-founder of
\href{https://www.avidbots.com/}{Avidbots}, the Canadian start-up that
created the robot. This year, however, demand has shot up 100 percent
just since the pandemic-induced shutdown in March. Suddenly, the need
for thorough, reliable and frequent cleaning is front and center.

``Before, a top executive at a big company would not really have known
how their facilities got cleaned,'' Mr. Sheikh said. ``They would have
outsourced it to a facilities management company, who might outsource it
out again.''

Now, company leaders are showing more interest, asking questions about
the cleaning process and schedule, as well as safety and effectiveness.
``That can lead to interest in automation,'' he said.

Indeed, cleaning robots are having a moment in commercial real estate.
Their creators are promoting the machines as cost-effective solutions to
the cleaning challenges posed by the pandemic. They can be put to
frequent use without requiring more paid labor hours, they are always
compliant, and some can even provide the data to prove that they have
scoured every inch assigned.

\includegraphics{https://static01.nyt.com/images/2020/08/05/business/04Virus-Robots-05/merlin_163092192_85320fdf-442f-454e-b7d8-9a452d49ea70-articleLarge.jpg?quality=75\&auto=webp\&disable=upscale}

The autonomous robots available now are primarily for cleaning floors
and carpets, but companies are busy developing other cleaning
applications. Boston Dynamics, a robotics design company in Waltham,
Mass., for example, is in a partnership to develop a disinfecting
solution that can be mounted atop its four-legged
\href{https://www.bostondynamics.com/spot}{Spot robot}, a company
spokeswoman said.

Robotics are also being used to relieve humans of repetitive back-office
tasks like accounting, according to
\href{https://www2.deloitte.com/global/en/pages/real-estate/articles/robotics-real-estate-services.html}{a
2018 report} from Deloitte. As more buildings incorporate smart
technology, data collection and conversion will become increasingly
important.

\href{http://getsomatic.com/}{Somatic}, a start-up in New York, is
working on a robot that can clean bathrooms using a spray technology,
said Michael Levy, the chief executive. Removing a human cleaner from
the bathroom makes the area safer because of the reduced risk of
spreading germs, Mr. Levy said. And the robot will always do the job
exactly as it is programmed to do.

``You have to let the chemicals set to do their job, but compliance is
tough in the industry,'' Mr. Levy said. ``If you tell a robot to leave
the chemicals for 36 seconds, they leave the chemicals for 36 seconds
every single time.''

\hypertarget{latest-updates-economy}{%
\section{\texorpdfstring{\href{https://www.nytimes.com/live/2020/08/04/business/stock-market-today-coronavirus?action=click\&pgtype=Article\&state=default\&region=MAIN_CONTENT_1\&context=storylines_live_updates}{Latest
Updates:
Economy}}{Latest Updates: Economy}}\label{latest-updates-economy}}

\href{https://www.nytimes.com/live/2020/08/04/business/stock-market-today-coronavirus?action=click\&pgtype=Article\&state=default\&region=MAIN_CONTENT_1\&context=storylines_live_updates\#the-ad-giant-publicis-has-parted-ways-with-an-executive-over-his-virus-tweets}{11m
ago}

\href{https://www.nytimes.com/live/2020/08/04/business/stock-market-today-coronavirus?action=click\&pgtype=Article\&state=default\&region=MAIN_CONTENT_1\&context=storylines_live_updates\#the-ad-giant-publicis-has-parted-ways-with-an-executive-over-his-virus-tweets}{The
ad giant Publicis has `parted ways' with an executive over his virus
tweets.}

\href{https://www.nytimes.com/live/2020/08/04/business/stock-market-today-coronavirus?action=click\&pgtype=Article\&state=default\&region=MAIN_CONTENT_1\&context=storylines_live_updates\#nbcuniversal-to-cut-about-10-percent-of-its-work-force}{1h
ago}

\href{https://www.nytimes.com/live/2020/08/04/business/stock-market-today-coronavirus?action=click\&pgtype=Article\&state=default\&region=MAIN_CONTENT_1\&context=storylines_live_updates\#nbcuniversal-to-cut-about-10-percent-of-its-work-force}{NBCUniversal
to cut about 10 percent of its work force.}

\href{https://www.nytimes.com/live/2020/08/04/business/stock-market-today-coronavirus?action=click\&pgtype=Article\&state=default\&region=MAIN_CONTENT_1\&context=storylines_live_updates\#loans-are-harder-to-get-even-as-interest-rates-are-low}{3h
ago}

\href{https://www.nytimes.com/live/2020/08/04/business/stock-market-today-coronavirus?action=click\&pgtype=Article\&state=default\&region=MAIN_CONTENT_1\&context=storylines_live_updates\#loans-are-harder-to-get-even-as-interest-rates-are-low}{Loans
are harder to get, even as interest rates are low.}

\href{https://www.nytimes.com/live/2020/08/04/business/stock-market-today-coronavirus?action=click\&pgtype=Article\&state=default\&region=MAIN_CONTENT_1\&context=storylines_live_updates}{See
more updates}

More live coverage:
\href{https://www.nytimes.com/2020/08/04/world/coronavirus-cases.html?action=click\&pgtype=Article\&state=default\&region=MAIN_CONTENT_1\&context=storylines_live_updates}{Global}

The idea of robotic cleaning is not new. The first attempts were in the
1970s, Mr. Sheikh said, but the technology was not up to the task, and
the machines were ``extremely cost prohibitive.''

Image

The cameras on the Neo robot can be monitored through a mobile
app.Credit...Ty Wright for The New York Times

The Neo is sophisticated enough to create its own maps of a facility
after being walked through it a single time, he said. The customer then
works with Avidbots to develop cleaning plans, which may vary depending
on the day of the week.

``After a human selects a cleaning plan, you press start and walk
away,'' Mr. Sheikh said. ``The robot figures out its own path.''

Designed for facilities of at least 80,000 square feet, Neos sell for
\$50,000, plus \$300 a month for software that tracks cleaning
performance. At that price, the break-even point for the buyer is 12 to
18 months, Mr. Sheikh said.

They can also be rented for \$2,500 a month, including maintenance and
software, on a minimum three-year contract.

Cincinnati/Northern Kentucky International Airport deploys its Neo three
or four times a day to clean the hundreds of thousands of square feet of
tiled floor, said Brian Cobb, the airport's chief innovation officer.

``Neo has the artificial intelligence capability where, as it's moving
along its original path, if it sees something in its way, it will go
around it,'' Mr. Cobb said. ``If the obstacle is there the next day, Neo
will incorporate it into its map.''

Image

Neo cleans the floor at Cincinnati/Northern Kentucky International
Airport three or four times a day.Credit...Ty Wright for The New York
Times

Before Neo's activation in January, the airport had three workers
cleaning floors every night, amounting to an average 24 labor hours per
day, Mr. Cobb said. The Neo has taken over a portion of that, though
workers are still needed to do heavier floor maintenance, like
burnishing and recoating. It also frees cleaning staff to focus on
making sure that ``high-touch'' areas of the airport are cleaned more
frequently during the pandemic, he said.

SoftBank, the Japanese multinational conglomerate, introduced the
\href{https://13779usreg20181102.com/us/whiz}{Whiz} autonomous carpet
cleaner through its robotics unit in November, said Kass Dawson, the
vice president of brand strategy and brand communications at SoftBank
Robotics. Already, more than 10,000 compact Whiz robots have been
deployed around the globe

They caught the attention of Jeff Tingley, the president of Sparkle
Services, a cleaning company in Enfield, Conn., that works in large
commercial facilities throughout Connecticut, New Jersey and New York.
He said he had long been interested in robotic cleaning but had not
found the technology to be advanced enough or cost effective.

``Vacuuming is one of the most time-consuming processes in cleaning.
With Whiz, you can essentially wipe out 90 percent of the vac time
required,'' Mr. Tingley said. ``You still need humans with backpack vacs
for under desks and chairs, but we've gained a lot of hours.''

The Whiz leases for \$500 to \$550 a month, which includes maintenance
and data collection that provides clients with ``the confirmed clean,''
Mr. Dawson said.

\includegraphics{https://static01.nyt.com/images/2020/08/04/autossell/04Virus-Robots-vid-still-2/04Virus-Robots-vid-still-2-videoSixteenByNineJumbo1600.jpg}

The robot's software was
\href{https://www.nytimes.com/2020/04/10/business/coronavirus-workplace-automation.html}{developed
by Brain Corp}, a San Diego company that teams up with outside
manufacturers mainly in cleaning and warehousing industries. Brain
Corp's autonomous technology, BrainOS, is also in robots made by
Tennant, Minuteman, Kärcher and others.

In the second quarter this year, retailers' use of BrainOS-powered
robots climbed 24 percent from a year earlier, said Chris Wright, Brain
Corp's vice president of sales. Median daily use rose 20 percent, to
2.58 hours from 2.15, he said.

He noted that much of the increase was during daytime hours, signaling a
major shift in cleaning schedules.

``Cleaning is now coming to the first shift because it's becoming
important to companies' image,'' Mr. Wright said. ``Everyone's a little
tentative when they walk into buildings now. One of the things that will
immediately put people at ease is when they see cleaning happening.''

Mr. Tingley has seen it when the Whiz is moving around an office floor.
It's a ``friendly machine'' that stops if you walk in front of it and
uses a blinker to signal when it's turning, and people seem to like it,
he said.

``During this fearful period, the folks in buildings have blank looks or
even unhappy frowns,'' he said. ``When the Whiz passes by, it brings a
smile to their face. It's almost like a pet --- everybody wants to name
it.''

Advertisement

\protect\hyperlink{after-bottom}{Continue reading the main story}

\hypertarget{site-index}{%
\subsection{Site Index}\label{site-index}}

\hypertarget{site-information-navigation}{%
\subsection{Site Information
Navigation}\label{site-information-navigation}}

\begin{itemize}
\tightlist
\item
  \href{https://help.nytimes.com/hc/en-us/articles/115014792127-Copyright-notice}{©~2020~The
  New York Times Company}
\end{itemize}

\begin{itemize}
\tightlist
\item
  \href{https://www.nytco.com/}{NYTCo}
\item
  \href{https://help.nytimes.com/hc/en-us/articles/115015385887-Contact-Us}{Contact
  Us}
\item
  \href{https://www.nytco.com/careers/}{Work with us}
\item
  \href{https://nytmediakit.com/}{Advertise}
\item
  \href{http://www.tbrandstudio.com/}{T Brand Studio}
\item
  \href{https://www.nytimes.com/privacy/cookie-policy\#how-do-i-manage-trackers}{Your
  Ad Choices}
\item
  \href{https://www.nytimes.com/privacy}{Privacy}
\item
  \href{https://help.nytimes.com/hc/en-us/articles/115014893428-Terms-of-service}{Terms
  of Service}
\item
  \href{https://help.nytimes.com/hc/en-us/articles/115014893968-Terms-of-sale}{Terms
  of Sale}
\item
  \href{https://spiderbites.nytimes.com}{Site Map}
\item
  \href{https://help.nytimes.com/hc/en-us}{Help}
\item
  \href{https://www.nytimes.com/subscription?campaignId=37WXW}{Subscriptions}
\end{itemize}
