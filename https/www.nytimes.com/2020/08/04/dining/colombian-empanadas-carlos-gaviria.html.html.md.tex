Sections

SEARCH

\protect\hyperlink{site-content}{Skip to
content}\protect\hyperlink{site-index}{Skip to site index}

\href{https://www.nytimes.com/section/food}{Food}

\href{https://myaccount.nytimes.com/auth/login?response_type=cookie\&client_id=vi}{}

\href{https://www.nytimes.com/section/todayspaper}{Today's Paper}

\href{/section/food}{Food}\textbar{}A Colombian Chef Shares His Secret
to Better Empanadas

\url{https://nyti.ms/2DmdUu7}

\begin{itemize}
\item
\item
\item
\item
\item
\item
\end{itemize}

Advertisement

\protect\hyperlink{after-top}{Continue reading the main story}

Supported by

\protect\hyperlink{after-sponsor}{Continue reading the main story}

\hypertarget{a-colombian-chef-shares-his-secret-to-better-empanadas}{%
\section{A Colombian Chef Shares His Secret to Better
Empanadas}\label{a-colombian-chef-shares-his-secret-to-better-empanadas}}

For even better flavor and texture in his masa, J. Kenji López-Alt
borrowed a technique from the chef Carlos Gaviria: milling popcorn.

\includegraphics{https://static01.nyt.com/images/2020/08/05/dining/03Kenji4/merlin_174846099_dd3040d6-88f8-4e4c-b38e-3bc5247c1482-articleLarge.jpg?quality=75\&auto=webp\&disable=upscale}

By \href{https://www.nytimes.com/by/j-kenji-lopez-alt}{J. Kenji
López-Alt}

\begin{itemize}
\item
  Aug. 4, 2020, 11:46 a.m. ET
\item
  \begin{itemize}
  \item
  \item
  \item
  \item
  \item
  \item
  \end{itemize}
\end{itemize}

I showed up at the chef Carlos Gaviria's apartment in Bogotá, Colombia,
fully expecting to bite into something surprisingly delicious, but I did
not expect to have my mind blown.

Earlier that month, he had invited my family via Instagram direct
message to cook with him during our most recent trip to my wife, Adri's,
native Colombia.

The chef, a professor of food studies at the University of La Sabana, is
the author of two scholarly books on Colombian cuisine. The first,
``Técnicas Profesionales de Cocina Colombiana'' (Professional Techniques
of Colombian Cuisine), is easily the most in-depth guide to the
country's regional cuisine I've ever read, while the second, ``Arepas
Colombianas,'' is a deep dive into the variations of corn cakes, the
country's staple food.

That surprisingly delicious thing came in the form of a simple fried
arepa made from freshly milled dough, with deep corn flavor and crunch.
But the part that left me speechless was the secret ingredient that
provided that extraordinary taste and texture: popcorn.

But, before we get to that, let's talk about the dough, the element that
makes a truly great arepa or empanada --- just I would argue as the key
to a great taco is the tortilla, or a great pizza is in the crust. In
Colombia, more often than not, that dough is made with masarepa: corn
that is parcooked, dried and ground into a flour.

Masarepa is enticing. All you need to do is add warm water, knead it a
touch, let it rest and you've got masa, ready to shape into arepas or
empanadas. I've spied bags of P.A.N., the most popular brand in
Colombia, in virtually every home kitchen I've seen across the country
(and in a good deal of restaurants, too).

But the best masa is made from dried corn that is boiled and milled
fresh (either whole kernel, or maíz pelao, corn that is treated with
alkaline lime and peeled, known as nixtamal in Mexico). Unlike the
uniform quality of masarepa, the texture of freshly milled corn is
coarser and more diverse, resulting in masa that fries up with more
surface area, more crunch and more corny flavor.

Mr. Gaviria's popcorn revelation came when he realized that, while dried
dent or flour corn may be difficult for an American like me to find,
popcorn is readily available. So he tried boiling regular supermarket
popcorn in a pressure cooker before passing it through a hand-crank
grain mill and kneading it with a bit of water into masa. The flavor and
texture was outstanding.

Image

The texture of freshly milled corn is coarser and more diverse, for masa
that fries up with greater crunch and corny flavor.Credit...Johnny
Miller for The New York Times. Food Stylist: Rebecca Jurkevich

Surely, I thought to myself, this works only with Colombian popcorn. The
corn varieties we see in the United States are different from those
available in Colombia, and popcorn, while similar to the flint or dent
corn typically used for arepas and empanadas, is a distinct variety.

When I returned to the United States, the first thing I did was order a
Colombian-made hand-crank grain mill (Victoria brand, which you can find
online for around \$50) and try it out with standard American popcorn,
straight from the grocery shelf.

To my delight, it worked perfectly, and we were treated to the best
Colombian empanadas we've eaten outside Colombia.

Since then, I've tested this method using a few different brands of
popcorn, with and without the pressure cooker. (To do it without, soak
the popcorn overnight in enough water to allow for it to triple in
volume, then boil it for about two hours, until the kernels are cracked
open.)

\includegraphics{https://static01.nyt.com/images/2020/08/05/dining/03Kenji1/merlin_174846111_3e7feaa1-319f-454e-b03b-581e8eb9b992-articleLarge.jpg?quality=75\&auto=webp\&disable=upscale}

I've also successfully formed masa by grinding the corn in a food
processor. The texture is not quite as interesting as what you'd get
from a grain mill, but it's still leaps and bounds better than dough
made from masarepa. Not that masarepa makes bad dough --- I still keep a
bag of it on hand for when Adri and I need our empanada fix in a hurry
--- but compared with freshly milled masa, its crispness is fleeting,
its corn flavor shallow.

The only moderately tricky part, whether using popcorn or masarepa, is
getting the water content right. After cooking the popcorn kernels, I
drain them, grind them, then add water to the masa a few tablespoons at
a time.

Kneading corn masa is a little different from kneading a wheat
flour-based dough. Wheat flour forms gluten, the protein network that
gives bread its chewiness and elasticity. Corn masa does not. That means
that kneading masa is more like smearing, and less like folding and
stretching. I place the dough on a cutting board, then press it with the
heel of my hand, smearing it out, regathering it, and repeating. If
you're familiar with the fraisage technique used for making flaky
pastry, you get the idea.

The dough is ready when you can form a golf ball-size mass and press it
between two sheets of plastic into a circle about three inches in
diameter, without the masa cracking or breaking around the edges
excessively. (A zip-top bag with its sides split and a heavy skillet or
small cutting board work well for this.)

Compared with a wheat-based dough, corn masa is, thankfully, easy to
work with. It behaves much like Play-Doh. If you accidentally form a
hole in it, no problem. You can squeeze it back together easily.

Image

There are as many types of ají as there are regions in Colombia, but
this version is heavy on cilantro and onions.Credit...Johnny Miller for
The New York Times. Food Stylist: Rebecca Jurkevich

Empanadas can come filled with a variety of flavors: the tiny,
half-dollar-sized empanadas de pipián of Boyacá, served with a spicy
peanut sauce; hefty empanadas de arroz in Bogotá, filled with rice and
meat; or the most common variety, ground meat and potatoes seasoned with
hogao, a cooked mixture of onion and tomato. Both the filling and masa
can be made several days in advance, which makes day-of preparation as
easy as stuffing empanadas, and frying them.

To fry, I prefer to use a wide wok. The flared sides keep your stovetop
clean from spatter, and the wide shape gives you plenty of room to
maneuver a metal spider or slotted spoon under the empanadas as they
fry. A Dutch oven is your next best bet.

As with all fried foods, the best way to prevent burns is to lower the
empanadas into the fryer with a slotted spoon, or if adding by hand,
gently lower the empanadas into the oil, dropping them only when your
hand is nearly touching the hot surface. Dropping from high up is how
you end up with hot oil splashes on your arms.

In Colombia, empanadas are typically a street snack, and the best are
fried to order and served immediately. I recommend cooking them only
when you're ready to serve, and taking a casual approach, letting
everyone gather around the kitchen and eat them as soon as they are cool
enough to handle.

Colombian food is not spicy by default, but at almost every meal you'll
find a small dish of ají, the Colombian Spanish term for both hot chiles
and any sauce made with them. This version of ají is made primarily with
cilantro, onions and chiles. As a general rule, ají tends to be soupier
in texture than similar fresh Mexican salsas you may be used to. Vinegar
or citrus juice are not uncommon ingredients, but it's also just as
common to see ají made with nothing but aromatics, salt and water, as I
do here.

When I eat empanadas, I like to take the first little bite as-is,
exposing the filling, then spooning in some ají and a little squeeze of
lime before each bite. But before biting in, be wary of two things: The
potato and meat filling will be hot, and the crisp crust may just ruin
you for all other empanadas.

Recipes:
\textbf{\href{https://cooking.nytimes.com/recipes/1021300-colombian-beef-and-potato-empanadas}{Colombian
Beef and Potato Empanadas}} \textbar{}
\textbf{\href{https://cooking.nytimes.com/recipes/1021299-aji-colombian-style-fresh-salsa}{Ají
(Colombian-Style Fresh Salsa)}} \textbar{}
\textbf{\href{https://cooking.nytimes.com/recipes/1021301-popcorn-masa-for-empanadas}{Popcorn
Masa for Empanadas}} \textbar{}
\textbf{\href{https://cooking.nytimes.com/recipes/1021302-standard-masa-for-empanadas}{Standard
Masa for Empanadas}}

\hypertarget{and-to-drink-}{%
\subsection{And to Drink \ldots{}}\label{and-to-drink-}}

Creaky conventional wisdom has it that you should choose wine
regionally, from near to where the food originates. But when the palette
of cuisines spreads far beyond the reach of wine production, that no
longer holds. In a sense, that's liberating. You can choose whatever you
like without fear of transgression. (To be honest, you can do that no
matter what the cuisine.) With this savory dish, I would opt for a
fresh, lively red with little in the way of oak or tannins. California
makes some terrific reds from the
\href{https://www.nytimes.com/2018/11/26/dining/drinks/wine-school-carignan.html}{carignan
grape} that would work beautifully here. You could drink frappato from
Sicily, Beaujolais, Bairradas from Portugal, inexpensive blaufränkisch
from Austria or any thirst-quenching natural wine. For a white, try a
dry Austrian riesling or a chenin blanc from the Loire Valley.
\textbf{ERIC ASIMOV}

\emph{Follow} \href{https://twitter.com/nytfood}{\emph{NYT Food on
Twitter}} \emph{and}
\href{https://www.instagram.com/nytcooking/}{\emph{NYT Cooking on
Instagram}}\emph{,}
\href{https://www.facebook.com/nytcooking/}{\emph{Facebook}}\emph{,}
\href{https://www.youtube.com/nytcooking}{\emph{YouTube}} \emph{and}
\href{https://www.pinterest.com/nytcooking/}{\emph{Pinterest}}\emph{.}
\href{https://www.nytimes.com/newsletters/cooking}{\emph{Get regular
updates from NYT Cooking, with recipe suggestions, cooking tips and
shopping advice}}\emph{.}

Advertisement

\protect\hyperlink{after-bottom}{Continue reading the main story}

\hypertarget{site-index}{%
\subsection{Site Index}\label{site-index}}

\hypertarget{site-information-navigation}{%
\subsection{Site Information
Navigation}\label{site-information-navigation}}

\begin{itemize}
\tightlist
\item
  \href{https://help.nytimes.com/hc/en-us/articles/115014792127-Copyright-notice}{©~2020~The
  New York Times Company}
\end{itemize}

\begin{itemize}
\tightlist
\item
  \href{https://www.nytco.com/}{NYTCo}
\item
  \href{https://help.nytimes.com/hc/en-us/articles/115015385887-Contact-Us}{Contact
  Us}
\item
  \href{https://www.nytco.com/careers/}{Work with us}
\item
  \href{https://nytmediakit.com/}{Advertise}
\item
  \href{http://www.tbrandstudio.com/}{T Brand Studio}
\item
  \href{https://www.nytimes.com/privacy/cookie-policy\#how-do-i-manage-trackers}{Your
  Ad Choices}
\item
  \href{https://www.nytimes.com/privacy}{Privacy}
\item
  \href{https://help.nytimes.com/hc/en-us/articles/115014893428-Terms-of-service}{Terms
  of Service}
\item
  \href{https://help.nytimes.com/hc/en-us/articles/115014893968-Terms-of-sale}{Terms
  of Sale}
\item
  \href{https://spiderbites.nytimes.com}{Site Map}
\item
  \href{https://help.nytimes.com/hc/en-us}{Help}
\item
  \href{https://www.nytimes.com/subscription?campaignId=37WXW}{Subscriptions}
\end{itemize}
