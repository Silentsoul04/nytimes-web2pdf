Sections

SEARCH

\protect\hyperlink{site-content}{Skip to
content}\protect\hyperlink{site-index}{Skip to site index}

\href{https://www.nytimes.com/section/fashion/weddings}{Love}

\href{https://myaccount.nytimes.com/auth/login?response_type=cookie\&client_id=vi}{}

\href{https://www.nytimes.com/section/todayspaper}{Today's Paper}

\href{/section/fashion/weddings}{Love}\textbar{}Weddings as a
Coronavirus Super-Spreader Worry

\url{https://nyti.ms/3gvrr0H}

\begin{itemize}
\item
\item
\item
\item
\item
\end{itemize}

Advertisement

\protect\hyperlink{after-top}{Continue reading the main story}

Supported by

\protect\hyperlink{after-sponsor}{Continue reading the main story}

\hypertarget{weddings-as-a-coronavirus-super-spreader-worry}{%
\section{Weddings as a Coronavirus Super-Spreader
Worry}\label{weddings-as-a-coronavirus-super-spreader-worry}}

Despite precautions, the coronavirus has swept through a number of
weddings, large and small, infecting guests and vendors.

\includegraphics{https://static01.nyt.com/images/2020/07/28/fashion/00CovidSpreadingWeddings1/00CovidSpreadingWeddings1-articleLarge.jpg?quality=75\&auto=webp\&disable=upscale}

By Alyson Krueger

\begin{itemize}
\item
  Published Aug. 4, 2020Updated Aug. 5, 2020, 2:02 a.m. ET
\item
  \begin{itemize}
  \item
  \item
  \item
  \item
  \item
  \end{itemize}
\end{itemize}

Jo Ellen Chism, 57, a retiree who lives in The Woodlands, Texas, about
an hour outside Houston, was nervous about attending her stepson's
wedding on June 20.

``They were going to postpone it, but then the Catholic church decided
they would open and would have up to 75 people,'' she said. ``75 people
seemed like a pretty big gathering to me during this Covid time.''

She went to support her family. She was inside the church for an
hourlong service that included a processional and communion. At the
reception, at Haak Winery, she sat indoors at a round table with other
guests, some of whom were from out of town. While everyone started the
day in masks, they took them off for photos and never replaced them.

Her symptoms started four days later. With a runny nose, sore throat and
bad headache, it could have been a sinus infection. Two days later she
tested positive for Covid-19 along with 12 other guests, including her
10-year-old grandson and the groom's 76-year-old grandfather. He is
still recovering after a trip to the emergency room with double
pneumonia. She said 13 additional guests had symptoms but didn't get
tested.

Ms. Chism's oldest son kept track of all the sick guests through the
seating chart, on which he marked who was positive, negative and
untested. Still, like most super-spreader events, without sophisticated
contact tracing, it's impossible to identify patient zero.

``I could just kick myself because I probably shouldn't have gone to
that wedding,'' she said. ``I am really thankful I was not terribly
ill.'' (She missed the birth of two grandchildren because of her need to
isolate.)

\hypertarget{vendors-helpless-at-controlling-guests-behavior}{%
\subsubsection{Vendors Helpless at Controlling Guests'
Behavior}\label{vendors-helpless-at-controlling-guests-behavior}}

After a brief pause, wedding season is back in full swing across the
country. Couples are working within the
\href{https://web.csg.org/covid19/state-reopen-plans/}{confines of state
laws}to carry out their nuptials during the pandemic. But despite
precautions coronavirus has swept into many of these events, both large
and small, infecting guests and vendors.

The situation is so dire, some wedding planners are self-quarantining
after events and even subcontracting their duties at the reception, the
part of the weddings where people mingle more closely. Some brides and
grooms are having guests sign liability forms upon arrival. Others say
they are losing sleep for two weeks after their wedding, wondering what
unintentional harm they might have caused to people they love.

In June, a wedding planner in Arkansas who wished to remain anonymous to
protect her business predicted weddings would become the next
super-spreader events.

``Weddings are so different from going into a store or sitting in a
restaurant for 45 minutes,'' she said. ``These receptions last for
three, four hours, and everyone is in an indoor space, breathing the
air. They aren't wearing masks and they are dancing. And when they start
drinking, it's like there is no pandemic.''

Six months ago her anxieties were about the weather or tight schedules.
Now they are much heavier. ``I am scared there is going to be an
outbreak at one of my weddings and someone is going to die.''

The problem, she said, is that she, along with other vendors, are
helpless at controlling guests' behavior at a private party. ``All the
vendors are masked up, and I am cracking the whip on the vendors, but I
can't do anything with the guests,'' she said.

That vendor, despite her nervousness, pointed out that she is
contractually obligated to carry out terms of the contract signed with
the couple.

Sarah Bett, a wedding planner in Houston, said even if vendors had power
to rein in rowdy guests, the bride and groom could just move their event
to a less strict venue. ``Some venues make the bride wear masks, while
others say those walking down the aisle are exempt,'' she said. ``It's a
little lawless down here.''

Without universal standards she is at the mercy of her clients, many of
whom want their festivities indoors, without masks, with out-of-town
friends and with dancing. ``I have a grandmother who is 90 who I am
around a lot,'' she said. ``I haven't had my first wedding yet this
summer, but when I do, I am going to self-quarantine after.''

\hypertarget{rules-and-regulations-vary-by-state}{%
\subsubsection{\texorpdfstring{\textbf{Rules and Regulations Vary by
State}}{Rules and Regulations Vary by State}}\label{rules-and-regulations-vary-by-state}}

State laws vary when it comes to weddings. Some wedding spaces are
governed by the same rules as restaurants, meaning they can accommodate
a certain percentage of their overall capacity. In Arkansas, for
example, you can fill venues to 66 percent capacity. So an event in a
1,000-person ballroom can legally host 666 guests. In other states
events are limited to the size of the group. In parts of New York, for
example, gatherings are limited to 50 people regardless of the space.

Ms. Bett said many of her clients feel safer with smaller affairs. ``I
have clients doing private, intimate ceremonies, because no one is
making a big stink about those,'' she said. ``No one wants to be the new
epicenter of the outbreak.''

But even weddings with the tightest guest list aren't immune to the
coronavirus.

Sunshine Borrer, 26, a veterinary technician in Houston, attended her
sister-in-law's wedding in Crockett, Texas, which has a population of
6,000. ``It was a real small town,'' she said. ``Covid wasn't something
I was super concerned about.'' The 30-person wedding was held outdoors,
but the after party was in a small bar area of an indoor restaurant.

It took about a week for her symptoms to develop. She tested positive
for Covid-19, along with the bride and groom, another couple, and the
bride's daughter. Fortunately all cases were mild.

She noticed there is no etiquette for how to communicate a coronavirus
outbreak to wedding guests. ``The bride and groom maybe told the people
they were living with, but that was it,'' she said. ``They told one of
my other sisters-in-law, and she is a nurse, so she took it upon herself
to tell people.''

Ms. Chism said it was her oldest son, not the bride and groom, who
alerted wedding guests to the virus exposure. ``If it were me I would
have been on the phone calling every single person,'' she said. ``But it
wasn't me.''

Pre-wedding events are risky as well. In July, Kathleen Oglesby, 66,
hosted a tea-party bridal shower at her home in Aubrey, Texas, for her
daughter-in-law. The 10 guests wore big, Kentucky Derby-style hats and
ate mini Bundt cakes. Days after the event the entire guest list went
into a two-week quarantine after a guest tested positive for the virus.

``She was a friend of my daughter-in-law's, and she helped me so much
with the bridal shower that I went to her house and brought her a wreath
as a thank you,'' she said. ``I'm so lucky I didn't get it, because I
probably wouldn't make it.'' Ms. Oglesby has an underlying heart
condition.

``It was really scary,'' she added. ``My mind was running wild.''

\hypertarget{some-are-concerned-about-risks}{%
\subsubsection{Some Are Concerned About
Risks}\label{some-are-concerned-about-risks}}

Some couples are acutely aware of the fact that their wedding could turn
into a super-spreader event.

Kate, 31, a social worker for the state of New York, married her
husband, a 30-year-old engineer, in a boutique hotel in central New York
during the July 4 weekend. She didn't want to give her full name,
because ``there's a lot of judgment for people who went through with
weddings, even with precautions.''

The event had less than 50 attendees, including vendors. Masks were on
the entire time even outside and in photographs. There was no dancing
--- not even a first dance for the bride and groom. ``We didn't want to
leave room for interpretation,'' she said.

Still, she spent her wedding night in the honeymoon suite of the
boutique hotel worrying. ``I was hit with the thought, `What did we just
do? What if everyone gets sick?''' she said. ``I didn't sleep more than
10 minutes that whole night.''

She checked in with guests regularly, making sure no one had symptoms.
Only on Day 14 could she begin thinking about her wedding with joy. ``My
husband and I needed those two weeks to pass so the memories weren't
tainted by anything terrible,'' she said. ``It was a long two weeks.''

Some couples are turning to waivers to protect themselves from liability
in case of an outbreak.

The wedding planner in Arkansas said she uses her clients' fears about
liability to drive them toward more protective measures. ``I tell them,
`Listen, we don't know where liability is going to fall, and you are the
host of this event,''' she said. ``You want to say at the end of the day
you did everything you could possible to keep your guests safe.''

Ms. Bett said, ``I tell my clients, `If you really feel you have to push
this form, why are we having this wedding in the first place?'''

Then there are the newlyweds who feel little responsibility for wedding
guests getting infected.

Ms. Chism's stepson, a 27-year-old engineer in Houston who didn't want
to be named because of the topic's sensitivity, believes his guests
exercised free will when attending his wedding.

``My wife felt bad and said, `I feel like it's all our fault,''' he
recalled. ``I said, `Look, they took a chance on coming, they knew the
risk. People could have come or they didn't have to come.'''

When asked whether he would make the same decision again, his answer was
absolutely: ``The day was very memorable, it felt like a normal wedding.
Minus the part about people getting sick.''

Advertisement

\protect\hyperlink{after-bottom}{Continue reading the main story}

\hypertarget{site-index}{%
\subsection{Site Index}\label{site-index}}

\hypertarget{site-information-navigation}{%
\subsection{Site Information
Navigation}\label{site-information-navigation}}

\begin{itemize}
\tightlist
\item
  \href{https://help.nytimes.com/hc/en-us/articles/115014792127-Copyright-notice}{©~2020~The
  New York Times Company}
\end{itemize}

\begin{itemize}
\tightlist
\item
  \href{https://www.nytco.com/}{NYTCo}
\item
  \href{https://help.nytimes.com/hc/en-us/articles/115015385887-Contact-Us}{Contact
  Us}
\item
  \href{https://www.nytco.com/careers/}{Work with us}
\item
  \href{https://nytmediakit.com/}{Advertise}
\item
  \href{http://www.tbrandstudio.com/}{T Brand Studio}
\item
  \href{https://www.nytimes.com/privacy/cookie-policy\#how-do-i-manage-trackers}{Your
  Ad Choices}
\item
  \href{https://www.nytimes.com/privacy}{Privacy}
\item
  \href{https://help.nytimes.com/hc/en-us/articles/115014893428-Terms-of-service}{Terms
  of Service}
\item
  \href{https://help.nytimes.com/hc/en-us/articles/115014893968-Terms-of-sale}{Terms
  of Sale}
\item
  \href{https://spiderbites.nytimes.com}{Site Map}
\item
  \href{https://help.nytimes.com/hc/en-us}{Help}
\item
  \href{https://www.nytimes.com/subscription?campaignId=37WXW}{Subscriptions}
\end{itemize}
