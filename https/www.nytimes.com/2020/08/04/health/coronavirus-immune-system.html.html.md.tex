Sections

SEARCH

\protect\hyperlink{site-content}{Skip to
content}\protect\hyperlink{site-index}{Skip to site index}

\href{https://www.nytimes.com/section/health}{Health}

\href{https://myaccount.nytimes.com/auth/login?response_type=cookie\&client_id=vi}{}

\href{https://www.nytimes.com/section/todayspaper}{Today's Paper}

\href{/section/health}{Health}\textbar{}Scientists Uncover Biological
Signatures of the Worst Covid-19 Cases

\url{https://nyti.ms/3frWSHS}

\begin{itemize}
\item
\item
\item
\item
\item
\end{itemize}

\href{https://www.nytimes.com/news-event/coronavirus?action=click\&pgtype=Article\&state=default\&region=TOP_BANNER\&context=storylines_menu}{The
Coronavirus Outbreak}

\begin{itemize}
\tightlist
\item
  live\href{https://www.nytimes.com/2020/08/04/world/coronavirus-cases.html?action=click\&pgtype=Article\&state=default\&region=TOP_BANNER\&context=storylines_menu}{Latest
  Updates}
\item
  \href{https://www.nytimes.com/interactive/2020/us/coronavirus-us-cases.html?action=click\&pgtype=Article\&state=default\&region=TOP_BANNER\&context=storylines_menu}{Maps
  and Cases}
\item
  \href{https://www.nytimes.com/interactive/2020/science/coronavirus-vaccine-tracker.html?action=click\&pgtype=Article\&state=default\&region=TOP_BANNER\&context=storylines_menu}{Vaccine
  Tracker}
\item
  \href{https://www.nytimes.com/2020/08/02/us/covid-college-reopening.html?action=click\&pgtype=Article\&state=default\&region=TOP_BANNER\&context=storylines_menu}{College
  Reopening}
\item
  \href{https://www.nytimes.com/live/2020/08/04/business/stock-market-today-coronavirus?action=click\&pgtype=Article\&state=default\&region=TOP_BANNER\&context=storylines_menu}{Economy}
\end{itemize}

Advertisement

\protect\hyperlink{after-top}{Continue reading the main story}

Supported by

\protect\hyperlink{after-sponsor}{Continue reading the main story}

\hypertarget{scientists-uncover-biological-signatures-of-the-worst-covid-19-cases}{%
\section{Scientists Uncover Biological Signatures of the Worst Covid-19
Cases}\label{scientists-uncover-biological-signatures-of-the-worst-covid-19-cases}}

Studies of patients with severe cases of Covid-19 show the immune system
lacks its usual coordinated response.

\includegraphics{https://static01.nyt.com/images/2020/08/03/science/03VIRUS-IMMUNITY/03VIRUS-IMMUNITY-articleLarge.jpg?quality=75\&auto=webp\&disable=upscale}

By \href{https://www.nytimes.com/by/katherine-j--wu}{Katherine J. Wu}

\begin{itemize}
\item
  Aug. 4, 2020, 12:17 p.m. ET
\item
  \begin{itemize}
  \item
  \item
  \item
  \item
  \item
  \end{itemize}
\end{itemize}

Scientists are beginning to untangle one of the most complex biological
mysteries of the coronavirus pandemic: Why do some people get severely
sick, whereas others quickly recover?

In certain patients, according to a
\href{https://www.nature.com/articles/s41586-020-2588-y}{flurry} of
\href{https://science.sciencemag.org/content/early/2020/07/15/science.abc8511}{recent}
\href{https://immunology.sciencemag.org/content/5/49/eabd7114}{studies},
the virus appears to make the immune system go haywire.

Unable to marshal the right cells and molecules to fight off the
invader, the bodies of the infected instead launch an entire arsenal of
weapons --- a misguided barrage that can wreak havoc on healthy tissues,
experts said.

``We are seeing some crazy things coming up at various stages of
infection,'' said Akiko Iwasaki, an immunologist at Yale University who
led one of the new studies.

Researchers studying these unusual responses are finding patterns that
distinguish patients on the path to recovery from those who fare far
worse. Insights gleaned from the data might
\href{https://www.nytimes.com/interactive/2020/science/coronavirus-drugs-treatments.html}{help
tailor treatments to individuals}, easing symptoms or perhaps even
vanquishing the virus before it has a chance to
\href{https://www.nytimes.com/2020/04/01/health/coronavirus-cytokine-storm-immune-system.html}{push
the immune system too far}.

``A lot of these data are telling us that we need to be acting pretty
early in this process,'' said John Wherry, an immunologist at the
University of Pennsylvania who recently published a study of these
telltale immune signatures. As more findings come out, researchers may
be able to begin testing the idea that ``we can change the trajectory of
disease,'' he said.

\href{https://www.nytimes.com/interactive/2020/science/coronavirus-drugs-treatments.html}{}

\includegraphics{https://static01.nyt.com/images/2020/07/14/us/coronavirus-drugs-treatments-promo-1594761806092/coronavirus-drugs-treatments-promo-1594761806092-articleLarge-v12.png}

\hypertarget{coronavirus-drug-and-treatment-tracker}{%
\subsection{Coronavirus Drug and Treatment
Tracker}\label{coronavirus-drug-and-treatment-tracker}}

An updated list of potential treatments for Covid-19.

When a more familiar respiratory infection, like a flu virus, tries to
gain a foothold in the body, the immune response launches a defense in
two orchestrated acts. First, a cavalry of fast-acting fighters flocks
to the site of infection and tries to corral the invader, buying the
rest of the immune system time to mount a more tailored attack.

Much of the early response depends on
\href{https://www.nytimes.com/2020/04/01/health/coronavirus-cytokine-storm-immune-system.html}{signaling
molecules called cytokines} that are produced in response to a virus.
Like microscopic alarms, cytokines can mobilize reinforcements from
elsewhere in the body, triggering a round of inflammation.

Eventually, these cells and molecules leading the initial charge will
stand down, making way for
\href{https://www.nytimes.com/2020/07/26/health/coronvirus-antibody-tests.html}{antibodies
and T cells} --- specialized assassins built to home in on the virus and
the cells it has infected.

\hypertarget{latest-updates-global-coronavirus-outbreak}{%
\section{\texorpdfstring{\href{https://www.nytimes.com/2020/08/04/world/coronavirus-cases.html?action=click\&pgtype=Article\&state=default\&region=MAIN_CONTENT_1\&context=storylines_live_updates}{Latest
Updates: Global Coronavirus
Outbreak}}{Latest Updates: Global Coronavirus Outbreak}}\label{latest-updates-global-coronavirus-outbreak}}

Updated 2020-08-04T19:32:24.665Z

\begin{itemize}
\tightlist
\item
  \href{https://www.nytimes.com/2020/08/04/world/coronavirus-cases.html?action=click\&pgtype=Article\&state=default\&region=MAIN_CONTENT_1\&context=storylines_live_updates\#link-4825b93}{Public
  and private schools in Maryland and elsewhere are divided over
  in-person instruction.}
\item
  \href{https://www.nytimes.com/2020/08/04/world/coronavirus-cases.html?action=click\&pgtype=Article\&state=default\&region=MAIN_CONTENT_1\&context=storylines_live_updates\#link-4d1eafa8}{N.Y.C.'s
  health commissioner resigns after clashing with the mayor over the
  virus.}
\item
  \href{https://www.nytimes.com/2020/08/04/world/coronavirus-cases.html?action=click\&pgtype=Article\&state=default\&region=MAIN_CONTENT_1\&context=storylines_live_updates\#link-6b644638}{`Long
  days, long nights': Washington prepares for a prolonged fight over
  virus relief.}
\end{itemize}

\href{https://www.nytimes.com/2020/08/04/world/coronavirus-cases.html?action=click\&pgtype=Article\&state=default\&region=MAIN_CONTENT_1\&context=storylines_live_updates}{See
more updates}

More live coverage:
\href{https://www.nytimes.com/live/2020/08/04/business/stock-market-today-coronavirus?action=click\&pgtype=Article\&state=default\&region=MAIN_CONTENT_1\&context=storylines_live_updates}{Markets}

But this coordinated handoff seems to break down in people with severe
Covid-19.

Rather than bowing out gracefully, the cytokines that drive the first
surge never stop sounding the alarm, even after antibodies and T cells
arrive on the scene. That means the wildfire response of inflammation
may never get snuffed out, even when it's no longer needed.

``It's normal to develop inflammation during a viral infection,'' said
Catherine Blish, a viral immunologist at Stanford University. ``The
problem comes when you can't resolve it.''

This sustained signaling may result in part from the body's inability to
keep the virus in check, Dr. Iwasaki said. Many who struggle to recover
from their illness seem to harbor the pathogen long after other patients
have purged it, perhaps goading the immune system into prolonging its
frantic inflammatory siege.

\href{https://www.nytimes.com/2020/07/31/opinion/coronavirus-antibodies-immunity.html}{Plenty
of other viruses}, including those that cause AIDS and herpes, have
evolved tricks to elude the immune system. Recent evidence hints that
the new coronavirus might have a way of delaying or muffling interferon,
one of the
\href{https://www.biorxiv.org/content/10.1101/2020.05.11.088179v1}{earliest
cytokine defenses} the body mounts.

The failure of this first line of defense may dupe the immune system
into sounding its alarm bells even louder, dragging out the response
into something destructive. ``It's an enigma,'' said Avery August, an
immunologist at Cornell University. ``You have this raging immune
response, but the virus continues to replicate.''

And the quality of these cytokines may matter as much as the quantity.
In a \href{https://www.nature.com/articles/s41586-020-2588-y}{paper
published last week in Nature Medicine}, Dr. Iwasaki and her colleagues
showed that patients with severe Covid-19 appear to be churning out
signals that are better suited to subduing pathogens that aren't
viruses.

Although the delineations aren't always clear-cut, the immune system's
responses to pathogens can be
\href{https://www.jacionline.org/article/S0091-6749(14)01585-1/fulltext}{roughly
grouped into three categories}: type 1, which is directed against
viruses and certain bacteria that infiltrate our cells; type 2, which
fights parasites like worms that don't invade cells; and type 3, which
goes after fungi and bacteria that can survive outside of cells. Each
branch uses different cytokines to rouse different subsets of molecular
fighters.

\href{https://www.nytimes.com/news-event/coronavirus?action=click\&pgtype=Article\&state=default\&region=MAIN_CONTENT_3\&context=storylines_faq}{}

\hypertarget{the-coronavirus-outbreak-}{%
\subsubsection{The Coronavirus Outbreak
›}\label{the-coronavirus-outbreak-}}

\hypertarget{frequently-asked-questions}{%
\paragraph{Frequently Asked
Questions}\label{frequently-asked-questions}}

Updated August 4, 2020

\begin{itemize}
\item ~
  \hypertarget{i-have-antibodies-am-i-now-immune}{%
  \paragraph{I have antibodies. Am I now
  immune?}\label{i-have-antibodies-am-i-now-immune}}

  \begin{itemize}
  \tightlist
  \item
    As of right
    now,\href{https://www.nytimes.com/2020/07/22/health/covid-antibodies-herd-immunity.html?action=click\&pgtype=Article\&state=default\&region=MAIN_CONTENT_3\&context=storylines_faq}{that
    seems likely, for at least several months.} There have been
    frightening accounts of people suffering what seems to be a second
    bout of Covid-19. But experts say these patients may have a
    drawn-out course of infection, with the virus taking a slow toll
    weeks to months after initial exposure. People infected with the
    coronavirus typically
    \href{https://www.nature.com/articles/s41586-020-2456-9}{produce}
    immune molecules called antibodies, which are
    \href{https://www.nytimes.com/2020/05/07/health/coronavirus-antibody-prevalence.html?action=click\&pgtype=Article\&state=default\&region=MAIN_CONTENT_3\&context=storylines_faq}{protective
    proteins made in response to an
    infection}\href{https://www.nytimes.com/2020/05/07/health/coronavirus-antibody-prevalence.html?action=click\&pgtype=Article\&state=default\&region=MAIN_CONTENT_3\&context=storylines_faq}{.
    These antibodies may} last in the body
    \href{https://www.nature.com/articles/s41591-020-0965-6}{only two to
    three months}, which may seem worrisome, but that's perfectly normal
    after an acute infection subsides, said Dr. Michael Mina, an
    immunologist at Harvard University. It may be possible to get the
    coronavirus again, but it's highly unlikely that it would be
    possible in a short window of time from initial infection or make
    people sicker the second time.
  \end{itemize}
\item ~
  \hypertarget{im-a-small-business-owner-can-i-get-relief}{%
  \paragraph{I'm a small-business owner. Can I get
  relief?}\label{im-a-small-business-owner-can-i-get-relief}}

  \begin{itemize}
  \tightlist
  \item
    The
    \href{https://www.nytimes.com/article/small-business-loans-stimulus-grants-freelancers-coronavirus.html?action=click\&pgtype=Article\&state=default\&region=MAIN_CONTENT_3\&context=storylines_faq}{stimulus
    bills enacted in March} offer help for the millions of American
    small businesses. Those eligible for aid are businesses and
    nonprofit organizations with fewer than 500 workers, including sole
    proprietorships, independent contractors and freelancers. Some
    larger companies in some industries are also eligible. The help
    being offered, which is being managed by the Small Business
    Administration, includes the Paycheck Protection Program and the
    Economic Injury Disaster Loan program. But lots of folks have
    \href{https://www.nytimes.com/interactive/2020/05/07/business/small-business-loans-coronavirus.html?action=click\&pgtype=Article\&state=default\&region=MAIN_CONTENT_3\&context=storylines_faq}{not
    yet seen payouts.} Even those who have received help are confused:
    The rules are draconian, and some are stuck sitting on
    \href{https://www.nytimes.com/2020/05/02/business/economy/loans-coronavirus-small-business.html?action=click\&pgtype=Article\&state=default\&region=MAIN_CONTENT_3\&context=storylines_faq}{money
    they don't know how to use.} Many small-business owners are getting
    less than they expected or
    \href{https://www.nytimes.com/2020/06/10/business/Small-business-loans-ppp.html?action=click\&pgtype=Article\&state=default\&region=MAIN_CONTENT_3\&context=storylines_faq}{not
    hearing anything at all.}
  \end{itemize}
\item ~
  \hypertarget{what-are-my-rights-if-i-am-worried-about-going-back-to-work}{%
  \paragraph{What are my rights if I am worried about going back to
  work?}\label{what-are-my-rights-if-i-am-worried-about-going-back-to-work}}

  \begin{itemize}
  \tightlist
  \item
    Employers have to provide
    \href{https://www.osha.gov/SLTC/covid-19/standards.html}{a safe
    workplace} with policies that protect everyone equally.
    \href{https://www.nytimes.com/article/coronavirus-money-unemployment.html?action=click\&pgtype=Article\&state=default\&region=MAIN_CONTENT_3\&context=storylines_faq}{And
    if one of your co-workers tests positive for the coronavirus, the
    C.D.C.} has said that
    \href{https://www.cdc.gov/coronavirus/2019-ncov/community/guidance-business-response.html}{employers
    should tell their employees} -\/- without giving you the sick
    employee's name -\/- that they may have been exposed to the virus.
  \end{itemize}
\item ~
  \hypertarget{should-i-refinance-my-mortgage}{%
  \paragraph{Should I refinance my
  mortgage?}\label{should-i-refinance-my-mortgage}}

  \begin{itemize}
  \tightlist
  \item
    \href{https://www.nytimes.com/article/coronavirus-money-unemployment.html?action=click\&pgtype=Article\&state=default\&region=MAIN_CONTENT_3\&context=storylines_faq}{It
    could be a good idea,} because mortgage rates have
    \href{https://www.nytimes.com/2020/07/16/business/mortgage-rates-below-3-percent.html?action=click\&pgtype=Article\&state=default\&region=MAIN_CONTENT_3\&context=storylines_faq}{never
    been lower.} Refinancing requests have pushed mortgage applications
    to some of the highest levels since 2008, so be prepared to get in
    line. But defaults are also up, so if you're thinking about buying a
    home, be aware that some lenders have tightened their standards.
  \end{itemize}
\item ~
  \hypertarget{what-is-school-going-to-look-like-in-september}{%
  \paragraph{What is school going to look like in
  September?}\label{what-is-school-going-to-look-like-in-september}}

  \begin{itemize}
  \tightlist
  \item
    It is unlikely that many schools will return to a normal schedule
    this fall, requiring the grind of
    \href{https://www.nytimes.com/2020/06/05/us/coronavirus-education-lost-learning.html?action=click\&pgtype=Article\&state=default\&region=MAIN_CONTENT_3\&context=storylines_faq}{online
    learning},
    \href{https://www.nytimes.com/2020/05/29/us/coronavirus-child-care-centers.html?action=click\&pgtype=Article\&state=default\&region=MAIN_CONTENT_3\&context=storylines_faq}{makeshift
    child care} and
    \href{https://www.nytimes.com/2020/06/03/business/economy/coronavirus-working-women.html?action=click\&pgtype=Article\&state=default\&region=MAIN_CONTENT_3\&context=storylines_faq}{stunted
    workdays} to continue. California's two largest public school
    districts --- Los Angeles and San Diego --- said on July 13, that
    \href{https://www.nytimes.com/2020/07/13/us/lausd-san-diego-school-reopening.html?action=click\&pgtype=Article\&state=default\&region=MAIN_CONTENT_3\&context=storylines_faq}{instruction
    will be remote-only in the fall}, citing concerns that surging
    coronavirus infections in their areas pose too dire a risk for
    students and teachers. Together, the two districts enroll some
    825,000 students. They are the largest in the country so far to
    abandon plans for even a partial physical return to classrooms when
    they reopen in August. For other districts, the solution won't be an
    all-or-nothing approach.
    \href{https://bioethics.jhu.edu/research-and-outreach/projects/eschool-initiative/school-policy-tracker/}{Many
    systems}, including the nation's largest, New York City, are
    devising
    \href{https://www.nytimes.com/2020/06/26/us/coronavirus-schools-reopen-fall.html?action=click\&pgtype=Article\&state=default\&region=MAIN_CONTENT_3\&context=storylines_faq}{hybrid
    plans} that involve spending some days in classrooms and other days
    online. There's no national policy on this yet, so check with your
    municipal school system regularly to see what is happening in your
    community.
  \end{itemize}
\end{itemize}

People with moderate cases of Covid-19 take what seems like the most
sensible approach, concentrating on type 1 responses, Dr. Iwasaki's team
found. Patients struggling to recover, on the other hand, seem to be
pouring an unusual number of resources into type 2 and type 3 responses,
which is kind of ``wacky,'' Dr. Iwasaki said. ``As far as we know, there
is no parasite involved.''

It's almost as if the immune system is struggling to ``pick a lane,''
Dr. Wherry said.

This disorientation also seems to extend into the realm of B cells and T
cells --- two types of immune fighters that usually need to stay in
conversation to coordinate their attacks. Certain types of T cells, for
instance, are crucial for coaxing B cells into manufacturing
disease-fighting antibodies.

Last month, Dr. Wherry and his colleagues
\href{https://science.sciencemag.org/content/early/2020/07/15/science.abc8511}{published
a paper in Science} finding that, in many patients with severe Covid-19,
the virus had somehow driven a wedge between these two close-knit
cellular communities. It's too soon to tell for sure, but perhaps
something about the coronavirus is preventing B and T cells from
``talking to each other,'' he said.

These studies suggest that treating bad cases of Covid-19 might require
an immunological reset --- drugs that could, in theory, restore the
balance in the body and resurrect lines of communication between
bamboozled cells. Such therapies could even be focused on specific
subsets of patients whose bodies are responding bizarrely to the virus,
Dr. Blish said: ``the ones who have deranged cytokines from the
beginning.''

But that's easier said than done. ``The challenge here is trying to
blunt the response, without completely suppressing it, and getting the
right types of responses,'' Dr. August said. ``It's hard to fine-tune
that.''

\href{https://www.nationalgeographic.com/science/2020/05/how-quieting-cytokine-storms-could-be-key-to-treating-severe-cvd/}{Timing}
is also crucial. Dose a patient too early with a drug that tempers
immune signaling, and they may not respond strongly enough; give it too
late, and the worst of the damage may have already been done. The same
goes for treatments intended to shore up the initial immune response
against the coronavirus, like
\href{https://www.nytimes.com/2020/07/20/world/covid-19-treatment-synairgen-interferon-beta.html}{interferon-based
therapies}, Dr. Blish said. These could stamp out the pathogen if
\href{https://www.cell.com/cell-host-microbe/fulltext/S1931-3128(20)30401-7}{given
shortly after infection} --- or run roughshod over the body if
\href{https://science.sciencemag.org/content/early/2020/06/10/science.abc3545}{administered
after too long of a delay}.

So far, treatments that block the effects of
\href{https://www.nytimes.com/interactive/2020/science/coronavirus-drugs-treatments.html}{one
cytokine at a time} have yielded mixed or lackluster results --- perhaps
because researchers haven't yet identified the right combinations of
signals that drive disease, said Donna Farber, an immunologist at
Columbia University.

Steroids like dexamethasone, on the other hand, are like ``big hammers''
that can curb the activity of multiple cytokines at once, Dr. Farber
said. Early clinical trials have hinted at dexamethasone's benefits
against severe cases of the coronavirus, and more are underway. Such
\href{https://www.ncbi.nlm.nih.gov/pmc/articles/PMC7196557/}{broad-acting
treatments have their downsides}. But, she added, ``it seems that's a
good strategy, until we know more.''

Advertisement

\protect\hyperlink{after-bottom}{Continue reading the main story}

\hypertarget{site-index}{%
\subsection{Site Index}\label{site-index}}

\hypertarget{site-information-navigation}{%
\subsection{Site Information
Navigation}\label{site-information-navigation}}

\begin{itemize}
\tightlist
\item
  \href{https://help.nytimes.com/hc/en-us/articles/115014792127-Copyright-notice}{©~2020~The
  New York Times Company}
\end{itemize}

\begin{itemize}
\tightlist
\item
  \href{https://www.nytco.com/}{NYTCo}
\item
  \href{https://help.nytimes.com/hc/en-us/articles/115015385887-Contact-Us}{Contact
  Us}
\item
  \href{https://www.nytco.com/careers/}{Work with us}
\item
  \href{https://nytmediakit.com/}{Advertise}
\item
  \href{http://www.tbrandstudio.com/}{T Brand Studio}
\item
  \href{https://www.nytimes.com/privacy/cookie-policy\#how-do-i-manage-trackers}{Your
  Ad Choices}
\item
  \href{https://www.nytimes.com/privacy}{Privacy}
\item
  \href{https://help.nytimes.com/hc/en-us/articles/115014893428-Terms-of-service}{Terms
  of Service}
\item
  \href{https://help.nytimes.com/hc/en-us/articles/115014893968-Terms-of-sale}{Terms
  of Sale}
\item
  \href{https://spiderbites.nytimes.com}{Site Map}
\item
  \href{https://help.nytimes.com/hc/en-us}{Help}
\item
  \href{https://www.nytimes.com/subscription?campaignId=37WXW}{Subscriptions}
\end{itemize}
