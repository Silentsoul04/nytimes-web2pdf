Sections

SEARCH

\protect\hyperlink{site-content}{Skip to
content}\protect\hyperlink{site-index}{Skip to site index}

\href{https://www.nytimes.com/section/health}{Health}

\href{https://myaccount.nytimes.com/auth/login?response_type=cookie\&client_id=vi}{}

\href{https://www.nytimes.com/section/todayspaper}{Today's Paper}

\href{/section/health}{Health}\textbar{}Scientists Are Optimistic About
New Vaccine Studies From Novavax

\url{https://nyti.ms/2Xs4u78}

\begin{itemize}
\item
\item
\item
\item
\item
\end{itemize}

\href{https://www.nytimes.com/news-event/coronavirus?action=click\&pgtype=Article\&state=default\&region=TOP_BANNER\&context=storylines_menu}{The
Coronavirus Outbreak}

\begin{itemize}
\tightlist
\item
  live\href{https://www.nytimes.com/2020/08/04/world/coronavirus-cases.html?action=click\&pgtype=Article\&state=default\&region=TOP_BANNER\&context=storylines_menu}{Latest
  Updates}
\item
  \href{https://www.nytimes.com/interactive/2020/us/coronavirus-us-cases.html?action=click\&pgtype=Article\&state=default\&region=TOP_BANNER\&context=storylines_menu}{Maps
  and Cases}
\item
  \href{https://www.nytimes.com/interactive/2020/science/coronavirus-vaccine-tracker.html?action=click\&pgtype=Article\&state=default\&region=TOP_BANNER\&context=storylines_menu}{Vaccine
  Tracker}
\item
  \href{https://www.nytimes.com/2020/08/02/us/covid-college-reopening.html?action=click\&pgtype=Article\&state=default\&region=TOP_BANNER\&context=storylines_menu}{College
  Reopening}
\item
  \href{https://www.nytimes.com/live/2020/08/04/business/stock-market-today-coronavirus?action=click\&pgtype=Article\&state=default\&region=TOP_BANNER\&context=storylines_menu}{Economy}
\end{itemize}

Advertisement

\protect\hyperlink{after-top}{Continue reading the main story}

Supported by

\protect\hyperlink{after-sponsor}{Continue reading the main story}

\hypertarget{scientists-are-optimistic-about-new-vaccine-studies-from-novavax}{%
\section{Scientists Are Optimistic About New Vaccine Studies From
Novavax}\label{scientists-are-optimistic-about-new-vaccine-studies-from-novavax}}

The company has received a \$1.6 billion grant from the government's
Operation Warp Speed to have 100 million doses ready by early 2021.

\includegraphics{https://static01.nyt.com/images/2020/08/04/science/04VIRUS-NOVAVAX1/merlin_170789967_c125706d-4398-4dc3-bb8e-b58346bc6984-articleLarge.jpg?quality=75\&auto=webp\&disable=upscale}

By \href{https://www.nytimes.com/by/carl-zimmer}{Carl Zimmer} and
\href{https://www.nytimes.com/by/katie-thomas}{Katie Thomas}

\begin{itemize}
\item
  Aug. 4, 2020
\item
  \begin{itemize}
  \item
  \item
  \item
  \item
  \item
  \end{itemize}
\end{itemize}

Novavax, the little-known Maryland company that
\href{https://www.nytimes.com/2020/07/16/health/coronavirus-vaccine-novavax.html}{received
a \$1.6 billion} deal from the federal government for its experimental
coronavirus vaccine, announced encouraging results in two preliminary
studies on Tuesday.

In one study, 56 volunteers produced a high level of antibodies against
the virus without any dangerous side effects. In the other, researchers
found that the vaccine strongly protected monkeys from coronavirus
infections.

Although it's not possible to directly compare the data from clinical
trials of different coronavirus vaccines, John Moore, a virologist at
Weill Cornell Medicine who was not involved in the studies, said the
Novavax results were the most impressive he had seen so far.

``This is the first one I'm looking at and saying, `Yeah, I'd take
that,''' Dr. Moore said.

Angela Rasmussen, a virologist at Columbia University who was not
involved in the studies, called them ``encouraging preliminary
results,'' but cautioned that it won't be possible to say whether the
vaccine is safe and effective until Novavax conducts a large-scale study
--- known as Phase 3 --- comparing people who get vaccinated to people
who get a placebo.

The studies are being submitted to scientific journals to be reviewed
for publication, said Dr. Gregory Glenn, the president of research and
development at Novavax.

The company has said that if its vaccine is shown to be effective, it
can produce 100 million doses by the beginning of next year, or enough
to give to 50 million people if administered in two doses. Under its
deal with the federal government, the company will also receive money to
undertake large-scale manufacturing of millions more doses if the
vaccine is shown to work.

Novavax's vaccine is
\href{https://www.nytimes.com/interactive/2020/science/coronavirus-vaccine-tracker.html}{one
of more than two dozen products} to have entered the first round of
safety tests in people, known as Phase 1 trials. Five other coronavirus
vaccines are already in Phase 3 trials, in which thousands of people are
tested to see if a vaccine works.

But Novavax, which has never brought a vaccine to market in its 33-year
history, uses a formula that's different from all the other vaccines
that have produced results in humans so far.

\includegraphics{https://static01.nyt.com/images/2020/07/15/science/04VIRUS-NOVAVAX2/merlin_174348666_fe62efc5-626b-4509-8fa5-09ccd9f87bfe-articleLarge.jpg?quality=75\&auto=webp\&disable=upscale}

Its vaccines contain a coronavirus protein that prompts a response from
the immune system. Protein-based vaccines have a longer track record
than some of the newer approaches used by competing coronavirus
vaccines, such as those
\href{https://www.nytimes.com/2020/07/28/health/coronavirus-moderna-vaccine-monkeys.html}{based
on viral genes} or so-called
\href{https://www.nytimes.com/2020/07/20/world/covid-coronavirus-vaccine.html}{adenoviruses}.

Protein-based vaccines are licensed for diseases such as hepatitis B and
shingles. Novavax successfully completed a Phase 3 trial for a
protein-based vaccine for influenza earlier this year and has done
research on other diseases, such as MERS.

\hypertarget{latest-updates-global-coronavirus-outbreak}{%
\section{\texorpdfstring{\href{https://www.nytimes.com/2020/08/04/world/coronavirus-cases.html?action=click\&pgtype=Article\&state=default\&region=MAIN_CONTENT_1\&context=storylines_live_updates}{Latest
Updates: Global Coronavirus
Outbreak}}{Latest Updates: Global Coronavirus Outbreak}}\label{latest-updates-global-coronavirus-outbreak}}

Updated 2020-08-05T07:58:24.076Z

\begin{itemize}
\tightlist
\item
  \href{https://www.nytimes.com/2020/08/04/world/coronavirus-cases.html?action=click\&pgtype=Article\&state=default\&region=MAIN_CONTENT_1\&context=storylines_live_updates\#link-762df92}{As
  talks drag on, McConnell signals openness to jobless aid extension,
  and negotiators agree on a deadline.}
\item
  \href{https://www.nytimes.com/2020/08/04/world/coronavirus-cases.html?action=click\&pgtype=Article\&state=default\&region=MAIN_CONTENT_1\&context=storylines_live_updates\#link-1228a480}{Novavax
  sees encouraging results from two studies of its experimental
  vaccine.}
\item
  \href{https://www.nytimes.com/2020/08/04/world/coronavirus-cases.html?action=click\&pgtype=Article\&state=default\&region=MAIN_CONTENT_1\&context=storylines_live_updates\#link-794484ed}{Mississippians
  must now wear masks in public, governor says.}
\end{itemize}

\href{https://www.nytimes.com/2020/08/04/world/coronavirus-cases.html?action=click\&pgtype=Article\&state=default\&region=MAIN_CONTENT_1\&context=storylines_live_updates}{See
more updates}

More live coverage:
\href{https://www.nytimes.com/live/2020/08/04/business/stock-market-today-coronavirus?action=click\&pgtype=Article\&state=default\&region=MAIN_CONTENT_1\&context=storylines_live_updates}{Markets}

Novavax's technology turns moth cells into factories for a
\href{https://www.nytimes.com/2020/07/28/health/coronavirus-mutation-spike-treatment.html}{coronavirus
protein called spike}, which studs the surface of coronaviruses. Its
vaccine combines several of the spike proteins in a nanoparticle.

To improve the performance of the vaccine, Novavax mixed the spike
proteins with a compound called an adjuvant. Studies on mice had
previously shown that the adjuvant stimulates immune cells so that they
develop a potent response to the virus.

The researchers gave the protein and adjuvant to monkeys in different
combinations of doses. The monkeys began making high levels of
antibodies that could specifically block the coronavirus.

\href{https://www.nytimes.com/interactive/2020/science/coronavirus-vaccine-tracker.html}{}

\includegraphics{https://static01.nyt.com/images/2020/06/09/us/coronavirus-vaccine-tracker-promo-1591728041922/coronavirus-vaccine-tracker-promo-1591728041922-articleLarge-v34.png}

\hypertarget{coronavirus-vaccine-tracker}{%
\subsection{Coronavirus Vaccine
Tracker}\label{coronavirus-vaccine-tracker}}

A look at all the vaccines that have reached trials in humans.

When the monkeys were infected, some versions of the vaccine left them
with no trace of the vaccine in their lungs or noses.

``That's pretty remarkable,'' said Akiko Iwasaki, an immunologist at
Yale University. She noted that the Novavax vaccine provided stronger
protection for monkeys than have other coronavirus vaccines, such as
Moderna's messenger RNA vaccine.

In May, Novavax started a Phase 1 human trial on 134 volunteers. Some of
the people who received the vaccine experienced tenderness at the spot
where they got injected. But the researchers found no serious side
effects.

The researchers extracted serum from the vaccinated volunteers and mixed
it with coronaviruses and cells. This showed that the volunteers
produced high levels of antibodies that prevented the viruses from
infecting cells. The vaccine produced more antibodies in the volunteers
than in patients who had recovered from Covid-19 on their own.

\href{https://www.nytimes.com/news-event/coronavirus?action=click\&pgtype=Article\&state=default\&region=MAIN_CONTENT_3\&context=storylines_faq}{}

\hypertarget{the-coronavirus-outbreak-}{%
\subsubsection{The Coronavirus Outbreak
›}\label{the-coronavirus-outbreak-}}

\hypertarget{frequently-asked-questions}{%
\paragraph{Frequently Asked
Questions}\label{frequently-asked-questions}}

Updated August 4, 2020

\begin{itemize}
\item ~
  \hypertarget{i-have-antibodies-am-i-now-immune}{%
  \paragraph{I have antibodies. Am I now
  immune?}\label{i-have-antibodies-am-i-now-immune}}

  \begin{itemize}
  \tightlist
  \item
    As of right
    now,\href{https://www.nytimes.com/2020/07/22/health/covid-antibodies-herd-immunity.html?action=click\&pgtype=Article\&state=default\&region=MAIN_CONTENT_3\&context=storylines_faq}{that
    seems likely, for at least several months.} There have been
    frightening accounts of people suffering what seems to be a second
    bout of Covid-19. But experts say these patients may have a
    drawn-out course of infection, with the virus taking a slow toll
    weeks to months after initial exposure. People infected with the
    coronavirus typically
    \href{https://www.nature.com/articles/s41586-020-2456-9}{produce}
    immune molecules called antibodies, which are
    \href{https://www.nytimes.com/2020/05/07/health/coronavirus-antibody-prevalence.html?action=click\&pgtype=Article\&state=default\&region=MAIN_CONTENT_3\&context=storylines_faq}{protective
    proteins made in response to an
    infection}\href{https://www.nytimes.com/2020/05/07/health/coronavirus-antibody-prevalence.html?action=click\&pgtype=Article\&state=default\&region=MAIN_CONTENT_3\&context=storylines_faq}{.
    These antibodies may} last in the body
    \href{https://www.nature.com/articles/s41591-020-0965-6}{only two to
    three months}, which may seem worrisome, but that's perfectly normal
    after an acute infection subsides, said Dr. Michael Mina, an
    immunologist at Harvard University. It may be possible to get the
    coronavirus again, but it's highly unlikely that it would be
    possible in a short window of time from initial infection or make
    people sicker the second time.
  \end{itemize}
\item ~
  \hypertarget{im-a-small-business-owner-can-i-get-relief}{%
  \paragraph{I'm a small-business owner. Can I get
  relief?}\label{im-a-small-business-owner-can-i-get-relief}}

  \begin{itemize}
  \tightlist
  \item
    The
    \href{https://www.nytimes.com/article/small-business-loans-stimulus-grants-freelancers-coronavirus.html?action=click\&pgtype=Article\&state=default\&region=MAIN_CONTENT_3\&context=storylines_faq}{stimulus
    bills enacted in March} offer help for the millions of American
    small businesses. Those eligible for aid are businesses and
    nonprofit organizations with fewer than 500 workers, including sole
    proprietorships, independent contractors and freelancers. Some
    larger companies in some industries are also eligible. The help
    being offered, which is being managed by the Small Business
    Administration, includes the Paycheck Protection Program and the
    Economic Injury Disaster Loan program. But lots of folks have
    \href{https://www.nytimes.com/interactive/2020/05/07/business/small-business-loans-coronavirus.html?action=click\&pgtype=Article\&state=default\&region=MAIN_CONTENT_3\&context=storylines_faq}{not
    yet seen payouts.} Even those who have received help are confused:
    The rules are draconian, and some are stuck sitting on
    \href{https://www.nytimes.com/2020/05/02/business/economy/loans-coronavirus-small-business.html?action=click\&pgtype=Article\&state=default\&region=MAIN_CONTENT_3\&context=storylines_faq}{money
    they don't know how to use.} Many small-business owners are getting
    less than they expected or
    \href{https://www.nytimes.com/2020/06/10/business/Small-business-loans-ppp.html?action=click\&pgtype=Article\&state=default\&region=MAIN_CONTENT_3\&context=storylines_faq}{not
    hearing anything at all.}
  \end{itemize}
\item ~
  \hypertarget{what-are-my-rights-if-i-am-worried-about-going-back-to-work}{%
  \paragraph{What are my rights if I am worried about going back to
  work?}\label{what-are-my-rights-if-i-am-worried-about-going-back-to-work}}

  \begin{itemize}
  \tightlist
  \item
    Employers have to provide
    \href{https://www.osha.gov/SLTC/covid-19/standards.html}{a safe
    workplace} with policies that protect everyone equally.
    \href{https://www.nytimes.com/article/coronavirus-money-unemployment.html?action=click\&pgtype=Article\&state=default\&region=MAIN_CONTENT_3\&context=storylines_faq}{And
    if one of your co-workers tests positive for the coronavirus, the
    C.D.C.} has said that
    \href{https://www.cdc.gov/coronavirus/2019-ncov/community/guidance-business-response.html}{employers
    should tell their employees} -\/- without giving you the sick
    employee's name -\/- that they may have been exposed to the virus.
  \end{itemize}
\item ~
  \hypertarget{should-i-refinance-my-mortgage}{%
  \paragraph{Should I refinance my
  mortgage?}\label{should-i-refinance-my-mortgage}}

  \begin{itemize}
  \tightlist
  \item
    \href{https://www.nytimes.com/article/coronavirus-money-unemployment.html?action=click\&pgtype=Article\&state=default\&region=MAIN_CONTENT_3\&context=storylines_faq}{It
    could be a good idea,} because mortgage rates have
    \href{https://www.nytimes.com/2020/07/16/business/mortgage-rates-below-3-percent.html?action=click\&pgtype=Article\&state=default\&region=MAIN_CONTENT_3\&context=storylines_faq}{never
    been lower.} Refinancing requests have pushed mortgage applications
    to some of the highest levels since 2008, so be prepared to get in
    line. But defaults are also up, so if you're thinking about buying a
    home, be aware that some lenders have tightened their standards.
  \end{itemize}
\item ~
  \hypertarget{what-is-school-going-to-look-like-in-september}{%
  \paragraph{What is school going to look like in
  September?}\label{what-is-school-going-to-look-like-in-september}}

  \begin{itemize}
  \tightlist
  \item
    It is unlikely that many schools will return to a normal schedule
    this fall, requiring the grind of
    \href{https://www.nytimes.com/2020/06/05/us/coronavirus-education-lost-learning.html?action=click\&pgtype=Article\&state=default\&region=MAIN_CONTENT_3\&context=storylines_faq}{online
    learning},
    \href{https://www.nytimes.com/2020/05/29/us/coronavirus-child-care-centers.html?action=click\&pgtype=Article\&state=default\&region=MAIN_CONTENT_3\&context=storylines_faq}{makeshift
    child care} and
    \href{https://www.nytimes.com/2020/06/03/business/economy/coronavirus-working-women.html?action=click\&pgtype=Article\&state=default\&region=MAIN_CONTENT_3\&context=storylines_faq}{stunted
    workdays} to continue. California's two largest public school
    districts --- Los Angeles and San Diego --- said on July 13, that
    \href{https://www.nytimes.com/2020/07/13/us/lausd-san-diego-school-reopening.html?action=click\&pgtype=Article\&state=default\&region=MAIN_CONTENT_3\&context=storylines_faq}{instruction
    will be remote-only in the fall}, citing concerns that surging
    coronavirus infections in their areas pose too dire a risk for
    students and teachers. Together, the two districts enroll some
    825,000 students. They are the largest in the country so far to
    abandon plans for even a partial physical return to classrooms when
    they reopen in August. For other districts, the solution won't be an
    all-or-nothing approach.
    \href{https://bioethics.jhu.edu/research-and-outreach/projects/eschool-initiative/school-policy-tracker/}{Many
    systems}, including the nation's largest, New York City, are
    devising
    \href{https://www.nytimes.com/2020/06/26/us/coronavirus-schools-reopen-fall.html?action=click\&pgtype=Article\&state=default\&region=MAIN_CONTENT_3\&context=storylines_faq}{hybrid
    plans} that involve spending some days in classrooms and other days
    online. There's no national policy on this yet, so check with your
    municipal school system regularly to see what is happening in your
    community.
  \end{itemize}
\end{itemize}

``There's no way to know yet what level leads to protection,'' said Matt
Frieman, a virologist at the University of Maryland School of Medicine
and a co-author of the Phase 1 study. ``But all of these pieces point to
it being quite effective.''

Volunteers who only received the spike protein in their vaccine did not
make a lot of antibodies, the researchers found. ``The adjuvant is
critical,'' Dr. Glenn of Novavax said.

Dr. Moore said that the volunteers' strong response to the vaccine
``does not surprise me in the slightest.'' In June, he and his Weill
Cornell colleague P. J. Klasse predicted protein-based vaccines would
produce a strong antibody response in a
\href{https://jvi.asm.org/content/early/2020/06/26/JVI.01083-20}{review}
they published in the Journal of Virology.

Image

Novavax, like Sanofi, manufactures its vaccine using moth cells, which
allows it to be produced more quickly.Credit...Guo Cheng/Xinhua, via
Getty Images

Although Novavax is the first company to share clinical results on the
immune response of a protein vaccine, it is not the only one testing
this technology. Three other protein-based vaccines for Covid-19 ---
from Clover Biopharmaceuticals, the University of Queensland and Vaxine
--- are also in Phase 1 trials.

The World Health Organization lists more than 50 protein-based
coronavirus vaccines in preclinical trials. One of them is being
developed by Sanofi and GlaxoSmithKline, a partnership that last month
was
\href{https://www.nytimes.com/2020/07/31/health/covid-19-vaccine-sanofi-gsk.html}{granted
\$2.1 billion from the federal government} for 100 million doses. Sanofi
expects to start its Phase 1 trial next month.

Sanofi and Novavax both manufacture their vaccines inside the cells of
the fall armyworm moth, which allows them to be produced more quickly
than older methods that use mammal cells. This technique is one reason
Novavax's vaccine candidate has gotten so much attention --- in addition
to its deal with the U.S. government, the company has also secured up to
\$388 million from the nonprofit Coalition for Epidemic Preparedness
Innovations, which seeks to make vaccines available outside of the
United States. Sanofi's Flublok vaccine, which is already on the market,
uses this technology.

Immunologists have been debating the importance of antibodies for
fighting the coronavirus. It's possible that another part of the immune
system, T cells, is also needed to fend off the virus. Dr. Moore
speculated that the best vaccine against Covid-19 might marshal both
types of responses.

A protein vaccine could provide strong antibody protection, while
another vaccine --- perhaps one based on messenger RNA or an adenovirus
--- could enlist T cells.

``If these first-generation vaccines are safe but they're just not
potent enough, then you would certainly want to look at combining
them,'' Dr. Moore said.

Advertisement

\protect\hyperlink{after-bottom}{Continue reading the main story}

\hypertarget{site-index}{%
\subsection{Site Index}\label{site-index}}

\hypertarget{site-information-navigation}{%
\subsection{Site Information
Navigation}\label{site-information-navigation}}

\begin{itemize}
\tightlist
\item
  \href{https://help.nytimes.com/hc/en-us/articles/115014792127-Copyright-notice}{©~2020~The
  New York Times Company}
\end{itemize}

\begin{itemize}
\tightlist
\item
  \href{https://www.nytco.com/}{NYTCo}
\item
  \href{https://help.nytimes.com/hc/en-us/articles/115015385887-Contact-Us}{Contact
  Us}
\item
  \href{https://www.nytco.com/careers/}{Work with us}
\item
  \href{https://nytmediakit.com/}{Advertise}
\item
  \href{http://www.tbrandstudio.com/}{T Brand Studio}
\item
  \href{https://www.nytimes.com/privacy/cookie-policy\#how-do-i-manage-trackers}{Your
  Ad Choices}
\item
  \href{https://www.nytimes.com/privacy}{Privacy}
\item
  \href{https://help.nytimes.com/hc/en-us/articles/115014893428-Terms-of-service}{Terms
  of Service}
\item
  \href{https://help.nytimes.com/hc/en-us/articles/115014893968-Terms-of-sale}{Terms
  of Sale}
\item
  \href{https://spiderbites.nytimes.com}{Site Map}
\item
  \href{https://help.nytimes.com/hc/en-us}{Help}
\item
  \href{https://www.nytimes.com/subscription?campaignId=37WXW}{Subscriptions}
\end{itemize}
