Behrouz Boochani Just Wants to Be Free

\url{https://nyti.ms/2DgOKNq}

\begin{itemize}
\item
\item
\item
\item
\item
\item
\end{itemize}

\includegraphics{https://static01.nyt.com/images/2020/08/09/magazine/09mag-Boochani-03/09mag-Boochani-03-articleLarge-v6.jpg?quality=75\&auto=webp\&disable=upscale}

Sections

\protect\hyperlink{site-content}{Skip to
content}\protect\hyperlink{site-index}{Skip to site index}

feature

\hypertarget{behrouz-boochani-just-wants-to-be-free}{%
\section{Behrouz Boochani Just Wants to Be
Free}\label{behrouz-boochani-just-wants-to-be-free}}

He fled Iran's Revolutionary Guard. He exposed Australia's offshore
detention camps --- from the inside. He survived, stateless, for seven
years. What's next?

Behrouz Boochani's book, ``No Friend but the Mountains,'' won the
prestigious Victorian Prize for Literature in 2019 while he was still
detained on Manus Island.Credit...Birgit Krippner for The New York Times

Supported by

\protect\hyperlink{after-sponsor}{Continue reading the main story}

By Megan K. Stack

\begin{itemize}
\item
  Aug. 4, 2020Updated 11:57 a.m. ET
\item
  \begin{itemize}
  \item
  \item
  \item
  \item
  \item
  \item
  \end{itemize}
\end{itemize}

\hypertarget{listen-to-this-article}{%
\subsubsection{Listen to This Article}\label{listen-to-this-article}}

Audio Recording by Audm

\emph{To hear more audio stories from publishers like The New York
Times, download}
\href{https://www.audm.com/?utm_source=nytmag\&utm_medium=embed\&utm_campaign=solitary_soul_stack}{\emph{Audm
for iPhone or Android}}\emph{.}

It was hard, in the end, to figure out what to take and what to leave.
Spread over the linoleum floor of Behrouz Boochani's motel room were
drifts of clothing, books in Persian and ashtrays overflowing with
cigarette stubs. It was a November morning last year in Port Moresby,
the capital of Papua New Guinea; outside, roosters screamed under a
stinging equatorial sun. Boochani's room was cramped; the door propped
open by a wastebasket stuffed with the remains of chicken dinners.
Everything he owned, all the objects and talismans gathered during six
and a half years of imprisonment, were crammed into this small room.
Boochani had been an Iranian dissident and a boat person; a detainee and
a refugee. In the morning he would strike out again, hoping to reach yet
another new life. It didn't matter, really, what stuff he carried along.
``I don't care about these books,'' he said suddenly, though many of
them contained Boochani's own work.

The motel loomed around him, a sealed, somber spot in the bustle of the
port town. Everyone staying in Lodge 10 --- every guest, although that's
the wrong word --- was a refugee awaiting resettlement. These men were
brought into the country against their will for the noncrime of seeking
political asylum in Australia. They were among hundreds of migrants
locked up in an old naval base on Manus Island, which lies off the
northeast coast of mainland Papua New Guinea. Now they had been moved to
this motel with its shared toilets and atmosphere of stultified trauma.
Some of the refugees hardly stirred from bed; medical contractors dosed
them with sleeping pills and psychiatric drugs. They had survived Manus
only to find themselves floundering like castaways in Port Moresby, one
of the world's most dangerous cities, notorious for armed robberies,
gang violence and rape. Days, weeks, months slipped away while they
waited for news of resettlement. Meanwhile, they were stuck. Or, to be
precise, everyone but Boochani was stuck.

All the men had started out together in the shared misery of detention,
but then Boochani did something extraordinary: Letter by letter, pecked
out on contraband telephones while locked up on Manus, he wrote his
first book. ``No Friend but the Mountains'' was published in 2018,
electrifying readers with its harrowing and deeply humanistic rendering
of life in the secretive and little-understood camp. The book was an
award-winning best seller; its beleaguered author became a cause
célèbre. Now Boochani was armed with priceless paperwork: an invitation
from a literary organization in New Zealand, a
\href{https://www.nytimes.com/2019/11/14/world/australia/behrouz-boochani-refugee.html}{one-month
visa} to cross the border and a ticket on a morning flight.

But for now Boochani was troubled and chain-smoking. He stayed up late
the night before, glumly replaying his own comments from an earlier
interview. He wished he hadn't described himself as independent; he
regretted saying that he admired his own work. He was vexed by the
awkwardness of becoming a subject after all the times he'd written about
others. ``I feel like I'm a selfish person,'' he said. With his haunting
gaze, unshaven jawline and mane of hair, photographs of Boochani tend to
draw comparisons to Jesus. In person, though, his swagger is
unmistakably modern. In a polo shirt and tapered pants, Ray-Bans perched
to hold dark locks off his face, he looked as if he belonged at a
sidewalk cafe in Rome. He seized a wheeled duffel. ``This bag is all
right?'' he asked. ``It's OK?'' The suitcase was old; a fading splash of
paint stained its side along with a label: MEG45 --- Boochani's serial
number at Manus. Nobody, I said, would pay attention to his bag. He
nodded, unconvinced.

Boochani didn't have a passport, just a refugee travel document with his
name spelled wrong --- there was no guarantee that he would even make it
to New Zealand. He needed to be careful, and also lucky. That's why it
was so startling when morning came and Boochani was late to the airport.
Everybody arrived before him: the TV crew filming his departure for a
documentary, friends who came to see him off, the other passengers
booked on the flight. By the time Boochani ambled into the departure
hall, bleary-eyed and still wearing yesterday's clothes, only an hour
remained before takeoff. It was cavalier; it was incomprehensible. How
could Boochani be late to this flight, with its promise of long-elusive
escape? It was such an elaborate display of insouciance that it was
somehow wondrous.

But this is the alchemy of Boochani's persona: an impervious, unbroken
spirit that defies his oppressors and also, at times, his would-be
supporters. He projects the image of an undaunted bohemian and lets you
forget --- maybe he hopes you won't notice --- that he has also been a
displaced and vulnerable man.

As it turned out, Boochani was late to the airport for the most obvious
and unbelievable reason: He didn't know what time he needed to show up.
He had flown commercially exactly once before, when he fled political
persecution in Iran. At the check-in counter, understanding his mistake,
a look of unease came into his eyes. And when he finally --- after tense
debates and a flurry of phone calls --- got his boarding pass, he jogged
to the gate as if something were chasing him.

Once onboard, he collapsed into a window seat and squinted into the
rising sun spilling across the runway. ``I'm so tired of this country,''
he erupted, voice sharp and loud in the hushed plane. ``It's a very
strange country.'' The plane lifted off and rose on a run of sky. The
earth fell away below --- green hills veined with red dirt roads, small
islands speckling the vast spread of sea. Papua New Guinea vanished from
sight.

A few hours later we landed in Manila for a 21-hour layover. Because he
had no passport, Boochani was forced to pass the time in a dim, drab
lounge with stiff chairs, a water dispenser and a TV blaring ceaseless
loops of Philippine Airlines commercials. Smoking was forbidden inside,
and no, Boochani was told, he didn't have permission to step outside. He
slammed his bag to the floor and cursed. Boochani was still muttering
about cigarettes when he was approached by two Afghan refugees from
Manus --- they were headed for resettlement in the United States. The
men exchanged pleasantries in Persian, but Boochani soon moved away and
stared grimly at the floor. ``Seeing these guys here, it made me so
depressed,'' he said quietly. ``Even I come here, I see refugees.''

The day ground past in slow circles on the wall clock. One hour, another
hour. The room kept getting colder. With a flimsy airline blanket draped
over his shoulders, Boochani looked like a kid playing superhero. He
sipped cup after cup of plain hot water and talked elaborately about
time. He had lost time in Manus. Literally misplaced two years. He was
sent to Manus at age 28. Now he was 36. He'd been there for only six
years. Explain that. You can't explain that. There must be a mistake.
There must be, but there isn't. Boochani blamed the ocean. He nearly
drowned trying to reach Australia and, as he flailed in the sea, he
sensed a kind of wild power in the water, enough to casually swallow a
chunk of time or leave a memory dull as sea glass. In the end, he simply
accepted that the loss of this time would never be explained nor
reconciled. It would linger as the cost of his imprisonment. And even if
the ocean had swallowed time, Boochani had survived the ocean. He
mentioned often the feat of overcoming the sea. In his mind full of
metaphors, it was more than a factual account of near drowning. He faced
death and madness but emerged, somehow, still intact.

I had planned to stay with Boochani until he boarded the plane to New
Zealand, but in the morning, airport personnel marched him through
immigration and blocked me from following. ``They still didn't let me
smoke,'' he groaned when he called from the plane. We talked and texted
intermittently until he took off. He was ready. And then he was gone.

\textbf{The cellphone was} everything on Manus. Boochani and the other
detainees hoarded their cigarettes for weeks to barter for phones with
the detention center's local employees. Once acquired, the phones had to
be hidden from the guards, who conducted surprise dawn inspections to
hunt for contraband. Boochani's phone was confiscated twice; each time,
there was no recourse but to start over again, one sacrificed smoke at a
time.

The phones quickly became the only tool successful at breaking through
the shroud of secrecy that Australia tried to throw over the migrants'
detention. Locked up in the disused rooms of the old naval base, the
asylum seekers were called by serial numbers instead of names.
Communications were tightly restricted. Under Australian law, workers
who spoke publicly about what they saw or heard at the detention sites
faced up to two years in prison. But official documents and accounts
from survivors and whistle blowers gradually leaked out, along with
accusations of sexual and physical abuse. Asylum seekers sought solace
in self-harm as their mental and physical health crumbled under the
strain of prolonged and uncertain detention.

In his quest for refuge, Boochani had landed in a dystopian enclosure
administered by a crazy collection of bureaucrats and guards and
contractors. A solitary soul, he was tormented in the camp by the
constant presence of so many other people. He yearned for a paper and
pen. The only way to fight off a creeping madness, he concluded, was to
work. Boochani had been a journalist in Iran; now he started texting
information about Manus to journalists. As he grew more bold, he moved
on to writing his own dispatches in publications including The Guardian
and giving speeches and interviews via livestream. He co-directed a
documentary, using his phone to shoot intimate footage and interviews
within the detention center's walls. Editors at Picador in Australia
approached Boochani about writing a memoir; Boochani replied that he was
already working on a more genre-bending book.

Boochani wrote ``No Friend but the Mountains'' in Persian, sending texts
of ideas and descriptive fragments to nonexistent WhatsApp numbers that
he used to organize his thoughts. Once satisfied with a passage, he sent
it to Moones Mansoubi, a translator in Sydney, who organized the
material into chapters before sending it along to Omid Tofighian, an
Iranian-Australian philosophy professor. Slowly, haltingly, Boochani and
Tofighian texted back and forth about how best to translate and arrange
the passages into a draft. Together they blended poetry and prose into a
genre Tofighian calls ``horrific surrealism.''

The book chronicles the early months of the detention center, starting
with Boochani's desperate 2013 boat voyage from Indonesia to Australia
and ending with the first riot on Manus the following year. Boochani
describes the story as autobiographical and true, but most of the
characters in the book are composites with nicknames: the Prime
Minister, the Cow, the Man With the Thick Moustache, the Cunning Young
Man. The only exceptions are Boochani himself and his friend Reza
Barati, the first detainee to be killed at Manus.

Boochani wrote feverishly, finishing the first draft in six months, and
with a single ambition: He was desperate to make people believe that the
asylum seekers on Manus were being tortured. Not mistreated or deprived
of human rights, but tortured. It troubled him that even his
sympathizers pushed back against this description, asking whether it
wasn't melodramatic or sensationalized. Boochani insists the systematic
use of psychological torment and dehumanization was meant to destroy the
men altogether.

\includegraphics{https://static01.nyt.com/images/2020/08/09/magazine/09mag-Boochani-04/09Boochani-04-articleLarge.jpg?quality=75\&auto=webp\&disable=upscale}

``I said, `Behrouz, the quality is amazing, the nuances, the
techniques,''' Tofighian recalled. ``He said: `Omid, Omid, that's not
what I'm asking. Will people understand systematic torture?'''

Boochani made international headlines in 2019 when the book won the
prestigious
\href{https://www.nytimes.com/2019/01/31/world/australia/behrouz-boochani-victorian-prize-manus-island.html}{Victorian
Prize for Literature} --- the most cash-rich award in Australian letters
--- while he was still detained on Manus. His immigration status made
him technically ineligible, but his publisher argued that, as a refugee
living and writing under Australian custody, Boochani had no other
homeland in which to be judged. ``Even if I don't go to Australia, I
will be a part of Australia,'' he told me. ``They don't want to
recognize they did this crime, because it makes them feel shame, but it
is a part of Australian history.''

The award was a validation of Boochani's artistry, but it also served as
a rebuke to those who supported the ``P.N.G. solution'' --- a policy
that had divided Australians bitterly. ``I've been attending these
literary events for years, and I've never seen anything like it,'' Jane
Novak, Boochani's agent, said. ``Everyone was in tears.'' Novak stayed
up all night after the ceremony, wading through hundreds of emails. When
she agreed to represent Boochani, she had warned him that reception to
his work would be ``death or glory.'' That night, her doubts were
erased. ``Suddenly I had this army of true believers all over the
world.''

First-person narratives that paint historical events from the
perspective of the persecuted have proven powerful and enduring. These
stories are subversive; the images slip into a reader's mind and create
empathy where there was little before. They can permanently alter the
way history is recorded and understood.

Boochani's book challenges readers to acknowledge that we are living in
the age of camps. The camps lie scattered throughout the Middle East,
cluster on Greek islands and stretch like an ugly tattoo along the
U.S.-Mexican border. Camps sprawl through Bangladesh, Chad and Colombia.
People are suspended in a stateless and extralegal limbo on the tiny
Pacific island nation Nauru, in Guantánamo and in the Syrian town of
al-Hawl. At no time since humans first drew borders have there been more
migrants and refugees than today. Countless individual lives weave into
a collective panorama of displacement and statelessness and detention.
These truncated journeys are a defining experience of our times.

As for Boochani, he refuses to cede the story of his hardships to
third-party observers. He criticizes journalists who depict refugees as
faceless victims. He bristles at perceived condescension from academics
or activists who benefit from what he describes as an industry built
around the plight of refugees. When Kristina Keneally, a prominent
center-left senator in Australia, sent a tweet supporting Boochani, he
tweeted in anger: ``Such a rediclilius {[}sic{]} and unacceptable
statement by Labor Party. You exiled me to Manus and you have supported
this exile policy for years.''

``No Friend but the Mountains'' had become powerful --- and sometimes
its author chafed against that power. ``People just know me as a person
who wrote a book,'' he said. ``This book is only a small part of my
work.'' He didn't want to get stuck forever writing and talking about
Manus Island and refugees. He wanted the world to view him as a writer
who had been, for a time, a refugee.

He was other things before; he wanted the freedom to change again, and
keep changing.

\textbf{Boochani was the} second of five children born to illiterate
Kurdish farmers. He grew up on the outer fringe of a small village where
the Zagros Mountains ripple toward the Iraqi border. The bloody slog of
the Iran-Iraq war raged in the surrounding countryside throughout his
childhood, filling Boochani's earliest memories with warplanes and fear.
The family sometimes went hungry, so he climbed oak trees to gather
pigeon eggs. When people in his village needed money, they stood on the
edge of the road and waited for someone to come looking for labor crews
or construction workers.

Because he is a Kurd, Boochani inherited a legacy of bigotry and
official repression in Iran, but his upbringing also gave him a mind-set
that would eventually prove invaluable: The conviction that he, a
descendant of perpetually put-upon warriors, could withstand even
extreme hardship with his dignity intact. Boochani was better at sports
than school, and so he sat for university entrance examinations with
little hope. He could only afford to apply for a free slot at a public
university. He was competing against more privileged students all over
the country --- teenagers who grew up with books and highbrow
conversation and tutors. Boochani took the exam and tried to forget
about it.

Image

Boochani~criticizes journalists who depict refugees as faceless
victims.Credit...Birgit Krippner for The New York Times

High school completed, he joined a work crew to dig out a building
foundation. The dirt was hard; the progress slow; the work exhausting.
On the third day he rode home in a funk. ``This is how the rest of my
life will be,'' he recalled thinking. As the bus pulled into the
village, he caught sight of a pack of friends and cousins waiting on the
roadside. Jubilant, waving a letter they'd torn open, they shouted the
news: Boochani had earned a seat at Tarbiat Moallem University in
Tehran.

At university, one of Boochani's closest friends was Toomas Askarian,
who still recalls their freewheeling discussions of ``everything:
European football, philosophy, people.'' They took rambling walks and
honed their novelistic skills by dreaming up elaborate back stories for
their fellow students. They had little interest in the formal niceties
of academia. Rather than spend money on texts, Boochani would borrow the
books to cram the night before exams. Once again, his raw intellect
carried him. He completed his undergraduate degree as well as a master's
in geopolitics. (Askarian, by contrast, was asked to leave the
university without a degree.)

In Tehran, Boochani wrote dispatches for a Kurdish magazine and quietly
taught Kurdish language lessons. Advocacy of Kurdish culture is
considered subversive by Shiite rulers who view Kurdish nationalism as a
threat, but Boochani was unfazed. ``We were working just to keep the
Kurdish language alive,'' he said. ``When you see a system denying your
identity or planning to destroy your culture, you react.''

After graduation, Boochani stayed in the capital. Journalism and
activism paid little, and he struggled for cash. He drifted around,
crashing with friends. Meanwhile, the political danger was growing.

In 2013, the Revolutionary Guard raided the magazine and jailed some of
his colleagues. Boochani went into hiding. ``They were listening to my
phone,'' he said. ``They knew everything about me. They were following
me. It was too much pressure.''

He scraped together \$5,000 to be smuggled through a notoriously
dangerous refugee route to Australia. He would fly to Indonesia and then
sail hundreds of miles to the Australian territory of Christmas Island,
where he would ask for political asylum. Plenty of migrants had drowned
on this voyage. But Boochani imagined Australia as a prosperous country
that protected human rights and so, he decided, the journey was worth
the risk.

On his first attempted crossing, the boat sank before clearing
Indonesian waters. Thrashing in the dark sea, Boochani prepared to die,
but fishermen hauled the migrants aboard and turned them over to the
Indonesian police. Back on dry land, Boochani escaped from jail and then
spent time hiding in a hotel basement, where he ran out of money and
began to starve. He dreaded going back to sea, but there was no choice
--- having fled, he couldn't go back to Iran. ``To return to the point
from which I started would be a death sentence,'' he later wrote in his
book.

The second craft was rickety and overcrowded; storms crashed; the boat
got lost and nearly sank. But the migrants reached Australian waters; a
naval ship took them to Christmas Island. At that point, Boochani
assumed, one of two things would happen: Either he would be sent back to
Indonesia or his asylum case would be heard.

But as Boochani was enduring his desperate escape, a harsh new migration
policy was announced in Australia. Prime Minister Kevin Rudd declared,
the same week that Boochani landed on Christmas Island, that anybody
trying to reach Australia by boat without a visa ``will never be settled
in Australia'' and would instead be shipped off to Papua New Guinea.

\textbf{Manus Regional Processing} Center doesn't exist anymore. Four
years after Boochani arrived on the island, he saw bulldozers razing the
decrepit buildings. Foundering in debt, rife with corruption and stunted
by a legacy of Australian colonialism, Papua New Guinea had agreed to
host the camp in exchange for about \$300 million. But backlash from the
international community was immediate and scathing. Pilloried by
criticism from home and abroad, Papua New Guinea soured on the deal, and
in 2016, the country's Supreme Court declared the detention of asylum
seekers unlawful and ordered the camp closed.

Manus remains, however, as a cultural identity shared by hundreds of
asylum seekers who survived its barracks. They have their own history
and iconography; they carry a collective grief for the seven men, at
least, who were killed in flares of violence,
\href{https://www.nytimes.com/2019/06/26/world/australia/australia-manus-suicide.html}{died
by suicide} or succumbed to medical negligence.

It was on Christmas Island that Australian officials began to taunt the
asylum seekers with lurid tales of cannibals and malaria-tainted
mosquitoes. Boochani's book describes the strange day he was moved to
Manus: ``Guards came in like debt collectors and heaved us out of bed,''
he wrote. The men were strip-searched and dressed in ill-fitting
clothes, marched past news photographers and loaded onto an airplane.

The miseries of offshore detention were meant to pressure migrants to
abandon their asylum claims so they could legally be sent back whence
they came and --- more crucial --- to create a spectacle so chilling
that ``boat people'' would stop coming to Australia altogether. That was
the first and last point of this byzantine enterprise.

Boats ferried the first white prisoners to Australia in 1788, and today
they float in the national imagination as symbols of unchecked
immigration and demographic change. ``Sometimes I feel that Manus and
Nauru are like a mirror,'' Boochani said. ``Australia sees its real face
on that mirror, and they hate it. Because we are boat people. They call
us boat people. But you are boat people, too.''

Arriving in Manus, Boochani found himself among tents and rough
buildings of lime and dirt that shed white powder onto the ground,
sticking to everyone's feet. Drain pipes poked from bathrooms and the
kitchen, dripping ``a potion of rotting excrement, the perfect
fertilizer for the tropical plants.'' The generator whose failures
paralyzed the cooling fans was a never-seen, godlike presence, ``a mind
made of machinery and wires \ldots{} that takes pleasure in throwing the
prison into disarray.'' The harsh sun was ``in cahoots with the prison
to intensify the misery,'' but when the sun set, the darkness was worse:
``We are all transformed into dark shadows scavenging for scraps of
light,'' he wrote.

The asylum seekers at first stuck with the people they met on the sea
voyage, but gradually, in what Boochani described as ``a kind of
internal migration,'' the men regrouped along ethnic and national lines:
Afghan, Sri Lankan, Sudanese, Lebanese, Iranian, Somali, Pakistani,
Rohingya, Iraqi, Kurdish. They struggled against traumatic memories and
boredom --- even playing cards was banned. Somebody found a marker and
drew a backgammon board onto a plastic table; bottle caps were gathered
as checkers. But guards defaced the board, scrawling ``Games
Prohibited'' over the table, leaving the men ``just staring at each
other in distress,'' Boochani wrote.

The camp was suffused with a dark, existential uncertainty. Nobody knew
how long they would be held or what fate awaited them afterward. The men
were pressed to go home or stay in Papua New Guinea for good. Asylum
cases were seldom and sporadically heard. The detainees didn't
understand whether Australia would eventually relent and accept them
and, if so, how long they needed to hold out. Meanwhile, they'd been
transported against their will over an international border and held
without trial or even the suggestion of a crime. Imprisoned, Boochani
thought. Taken hostage.

These problems were enormous and unanswerable, but in the daily slog of
camp life, small objects and petty interactions dominated. Boochani
wrote of the irrational rush of euphoria and pride he felt one night
when, sleepless and miserable with a toothache, he climbed onto the
camp's roof and reached a mango tree coveted by the detainees. ``I have
made it up here, up into the ether, up on top of the prison,'' he wrote.
``Witnessing the spectacle, witnessing the jungle and the ocean,
observing as I evaporate into the darkness.''

Image

Boochani outside the abandoned naval base on Manus island in June
2018.Credit...Jonas Gratzer/LightRocket, via Getty Images

``Humans are like this, after all,'' he wrote. ``Even in unexpected
situations they become gripped by wonder.''

The men fought for a spot near the front of the line at meals, which
left Boochani a ``frail fox,'' because he was always at the back,
subsisting on the last and worst of the food. He loathed having to
greet, over and over, the people who recurred constantly in the crammed
yards. ``The distress caused by saying `hi' is so intense that when
prisoners pass each other they pretend that they don't see anyone. It is
like shadows.''

As weeks slid past, paranoia clouded their minds. Boochani was haunted
by a fear that the Australians might suddenly, one day, load all the men
onto a ship and push them out to sea to die. ``For years I thought,
Anytime, it's possible --- I always imagined this --- if a war happens,
they'll put us all on a ship,'' he told me. ``They could do this. People
would talk about a Third World War. I thought, They'll kill everyone.''

Self-harm provided a much-craved airing of dark emotion. People
swallowed razor blades; sliced their wrists; hanged themselves; sewed
their lips together. Detainees hurt themselves in reaction to even minor
shifts or suggestions: a dawn inspection, a change in Australian
politics, a rumor.

After six months of misery and unanswered questions, immigration
officials appeared at the camp and warned asylum seekers that they would
be stuck in Manus for a long time yet. Enraged detainees rioted that
night, lunging at the guards and hurling chairs. Local police and Manus
residents rushed into the compound to quell the unrest. Dozens of
detainees were injured, some suffering broken bones and severe
lacerations. One man lost an eye; another's throat was slashed,
reportedly by a guard. Barati, Boochani's close friend, was viciously
attacked by a group that included an employee of the Salvation Army,
which had a \$50 million contract from the Australian government to
provide counseling to the asylum seekers. The assailants killed Barati
by dropping a heavy rock onto his head. He was the first detainee to die
on Manus.

As the years passed, the terms of the men's confinement changed, but
freedom never came. When they heard of the court order to close the
camp, they were jubilant with the assumption that Australia would
finally have to let them in --- but this, too, was a false hope. They
could come and go from the camp, but without travel documents, the men
were still stuck on the island. They swam in the ocean, met women and
played soccer by the water --- but they couldn't leave. ``Our prison
became bigger,'' Boochani said. Tofighian, the translator, traveled to
Manus with proofs so the pair could finalize the book in person. Novak,
his agent, came to meet him, too.

In the end, most of the men
\href{https://www.nytimes.com/2017/11/02/world/australia/manus-island-refugees.html}{clung}
to the camp. Traumatized and depleted, they balked at moving to another
detention center or starting life in an unfamiliar land. Medical
services and food were withdrawn. Electricity and water lines were cut.
The police and guards attacked them. Finally, the last holdouts were
forcibly relocated to temporary lodgings in and around the nearby town
of Lorengau.

In 2019, most of the asylum seekers were moved to motels in Port Moresby
because, it seemed, nobody knew what else to do with them. The Manus
Regional Processing Center was closed for good; the buildings and fences
mostly erased from the landscape. Now the United States and Australia
have new ambitions for the site: A joint naval base to counter Chinese
influence in the South China Sea.

\textbf{The first time} I saw Boochani, he was still being detained on
Manus Island. It was a chilly, wind-scraped morning in 2019. Boochani
was discussing his book via video link at the annual writers festival in
Byron Bay, Australia. When his face flickered onto the screen, the
overflowing crowd that jammed the seaside auditorium gasped and burst
into applause. Boochani looked haggard and detached; dangling hair
framed his craggy features. ``Oh, God,'' said a woman near me. ``He
looks so alone.''

Image

The clock in Boochani's kitchen in Christchurch stopped working. He
prefers to keep it there unmoving, suspended in time.Credit...Birgit
Krippner for The New York Times

Once the clapping died down, Boochani spoke with the urgency of a man
who knows he might vanish at any moment. He told the Australian crowd
that their government had lied to them about Manus. He described the
years he spent trying to get Australian readers to pay attention.
``People didn't listen to me,'' he said. ``This is part of my struggle:
to get my identity back.'' The audience listened with a mood that
approached gratitude. Some wept softly; others set their mouths and
nodded grimly. Invited to ask questions, several audience members
apologized to Boochani.

Australian politicians sometimes cast the problem of the boats in
humanitarian terms: Ruthless people smugglers, they say, must be starved
out of business. At other times, the boats are discussed as a security
threat, carrying an unchecked flow of strange and potentially dangerous
foreigners. Often these two strands of thought --- we don't want those
people here, nor do we want them to drown --- are woven together so
tightly they are impossible to separate.

At the same time, politicians have taken pains to deflect attention from
the human beings aboard the boats. Former military spokespeople have
said they were expressly forbidden to humanize the asylum seekers or
present them as relatable to the Australian public. Politicians scorn
boat arrivals as ``queue jumpers'' who have greedily taken the spots of
rule-abiding migrants seeking to come to Australia ``the right way.'' In
truth, there is no queue to jump; governments are not obliged to
consider wait time when choosing people for resettlement. Most refugees
will never get the fresh start they seek; they are far more likely to
return to their home country or stay in limbo until they die.

Peter Dutton, Australia's home affairs minister, frequently says the
asylum seekers in Papua New Guinea include men ``of bad character'' ---
``Labour's mess'' that he has been forced to ``clean up.'' Pauline
Hanson, a right-wing populist senator, called the men ``rapists'' on the
floor of Parliament this past winter. ``These people are thugs,'' she
said. ``They don't belong here in Australia.''

If any of this sounds familiar, that's not a coincidence. The practice
of ``offshore processing'' can be traced to Guantánamo Bay, where the
United States housed tens of thousands of asylum seekers who fled by
boat from Haiti and Cuba in the 1990s. Daniel Ghezelbash, an Australian
legal academic who wrote a book about links between American and
Australian refugee policy, has documented decades of advice and
influence exchanged between the two governments. ``The goal is the same:
Creating extralegal spaces which you can exert control over but not be
responsible for,'' Ghezelbash said. ``Or ostensibly deny legal
responsibility for what goes on there.''

President Obama, during his final months in office, agreed that hundreds
of detainees from Manus Island and Nauru could resettle in the United
States. As part of the deal, Australia was expected to grant asylum to
an unspecified number of refugees from Central America and Africa.
Ghezelbash calls the swapping of politically inconvenient people
``refugee laundering.''

When President Trump heard about the trade-off he had inherited, he
famously grumbled that it was a ``dumb deal,'' but he didn't stop it.
Gradually, quietly, refugees from Manus flew off to America. At least
785 people from Manus and Nauru have settled in the United States; more
are expected to arrive.

All told, Australia has locked up thousands of desperate people,
including children, in de facto prisons on Manus and Nauru. The
detentions have been harsh but effective, officials say: The flow of
boats slowed and eventually stopped. Asylum seekers are still stuck on
Nauru; until last year, they included children. The Australian
government recently spent about \$130 million to reopen the detention
center on Christmas Island --- despite the lack of new arrivals to lock
up. In other words, the policy is still unapologetically intact, ready
and waiting for any boats that make it to Australian waters.

\textbf{It was a brilliant January} day in Christchurch, New Zealand.
Screeching gulls wheeled in off the Pacific; swollen roses bobbed in the
breeze. In the hydrangea-fringed garden of a spare, tidy house, Boochani
sat smoking. He couldn't smoke inside because the house wasn't exactly
his; it was on loan from the University of Canterbury. Boochani's
neighborhood looked as if Beatrix Potter had painted it in watercolors:
prim, ivy-laced cottages and tidy beds of hollyhocks and lavender. It
was nice, Boochani conceded. Too nice, sometimes. ``It's too much, you
know?'' he said. ``It's too much peace and too much beauty. It's hard to
deal with this. It's like you go from a very cold place to a very hot
place.''

Boochani had landed in New Zealand without a credit card or bank
account; he had no idea what his book earnings were worth in real terms.
The Christchurch mayor and local Maori representatives welcomed him as
he stepped off the plane. He appeared before a rapt and sold-out crowd
at an event organized by Word Christchurch, the group that had invited
him to the country. He was constantly surrounded by people offering
help. Somebody took him to buy clothes; somebody else drove him on a run
for hair gel. He was shown to a room in an upscale hotel, then later
moved to a vacant apartment. The memories of detention were still fresh,
and Boochani struggled to adapt himself to an unfamiliar place and
lifestyle. He kept signing up for grocery-store discount cards, then
losing them. His sleep was crowded with nightmares; his days were full
of meetings and public appearances. He had an idea to write a new novel,
a contemporary Kurdish love story. He talked with friends about starting
a literary journal. More often than not, he drifted around in a kind of
daze. ``I feel empty,'' he said. ``Like I never read a book. But I'm OK
with that. And, I think, it will come.''

During these early and disorienting weeks, Boochani got word that it was
finally time to begin the final steps to resettle in the United States.
He'd been awaiting this news for months, but when his chance came, he
backed out. Reports of tensions between the U.S. and Iran, immigration
crackdowns and political tumult had eroded his eagerness. ``I don't feel
safe in America now,'' he said simply. ``I don't mean that someone would
kill me. But I don't trust the American system. It's like chaos there
now.''

Instead, Boochani took a bold gamble: He applied for asylum in New
Zealand. He accepted a fellowship with the university's Ngai Tahu
Research Center, which specializes in Maori and Indigenous studies --- a
nod to his Kurdish identity --- although the post would remain a secret
while his application to stay in New Zealand was pending. Neither his
whereabouts nor his plans were public knowledge. Conservative
politicians in both New Zealand and Australia were calling for Boochani
to be turned out. What would he do then, where would he go? He shrugged;
he didn't answer; instead, he began to roll another cigarette. The right
to smoke had become a kind of index by which Boochani took stock of his
own liberty. By that measure he was almost free, but not quite. He
dreamed of owning a house and smoking with impunity. ``I'll put up a
sign that says, `Smoking is free.' I'll even say, `If you don't smoke,
don't come.'''

When he set eyes on this pretty cottage, with its two bedrooms and
sensible kitchenette, he called one of his new friends in amazement:
``You've got to see this place, you're not going to believe it!'' The
friend, an Iranian-New Zealander named Donna Miles-Mojab, laughed as she
recounted this story later. She rushed over, imagining a swimming pool,
gleaming marble floors, elaborate gardens.

In those early months, Miles-Mojab became a kind of guide and cultural
translator to Boochani. Speaking in Farsi, she explained things to him
in a gentle, almost maternal tone. Boochani, she felt, was oblivious to
his own celebrity and the double takes he provoked around town. For his
part, he was offended by small and seemingly insignificant exchanges,
and he told these stories one night over whiskey in a pub. ``Something
happened to me here,'' he leaned forward and lowered his voice. ``If
this happened to you, you would cry.'' He was walking alone late one
night, he said, when he crossed paths with a group of young people.
Someone called out to him, ``Aren't you that writer who was on TV?''
Boochani replied, ``Yeah.'' He paused. A tense silence gathered over the
table, heavy with all the violent and demeaning conclusions this story
might reach.

``He took out his wallet,'' Boochani concluded dolefully, ``and he tried
to give me \$200.'' There was a beat of silence. Miles-Mojab's face
twitched with smothered laughter. ``Did you take it?'' she asked.

``No. I said: `Keep your money. Probably I am richer than you. My book
is in many countries.'''

This is the complication and the delicacy of Boochani: His most famous
work was derived from the considerable suffering he endured at the hands
of the state. He is proud, even cocky at times. And yet this pride must
wrestle with the dehumanization he has endured. His existence was
controlled by a hostile bureaucracy for years; now his days were
arranged by benevolent well-wishers.

Arriving in New Zealand just before the earliest coronavirus infections
emerged in China, Boochani had freed himself as people around the world
were shut into quarantines. He sometimes thought that he had finally
escaped detention and accidentally spread it all over the world. He
wondered, too, whether this taste of extreme isolation might help people
imagine more clearly the horror of being locked away. ``People should
understand now that life is not only food or having a bed,'' he said.
``We are nothing without people. Absolutely nothing, you know?''

The months slid past. Wait a few more weeks, Boochani was told. And then
a few more weeks, and still more. Boochani wrote some short stories.
Bought some new clothes. Took up biking.

Then, on July 23, Boochani's birthday, he finally got word from his
lawyer: His application had been
\href{https://www.nytimes.com/2020/07/24/world/australia/behrouz-boochani-asylum-new-zealand.html}{accepted}.
Boochani could stay in New Zealand. He was free. On the phone, he let
out a wild and incredulous laugh. Of course! When else? It had been his
birthday, too, the day he was lifted from the sea and taken into
Australian custody. Hearing him laugh like that, I remembered one of his
stories: When he was born, his parents asked a visiting cousin who knew
how to read to choose a name for the baby. The cousin opened a book and
poked his finger onto the page at random, striking the word ``Behrouz''
--- Farsi for ``fortunate.'' Literally, ``good day.''

Boochani rode his bike from his house to the sea. He looked at the
expanse of ocean, these waters that had almost killed him, the sea he
suspected of absconding with years of his life, the waves that crashed
now on the mineral grains of this new land he called home. He looked at
the ocean, at all of that past and all of that future, the churn of time
and destiny, and he smoked a cigarette. Just one cigarette. One
cigarette and the sea in his eyes. And then he rode home again.

\begin{center}\rule{0.5\linewidth}{\linethickness}\end{center}

\textbf{Megan K. Stack} is an author and a journalist living in
Washington. Her most recent book is ``Women's Work: A Personal Reckoning
With Labor, Motherhood, and Privilege.'' She last wrote for the magazine
about the coronavirus outbreak in Singapore.

Advertisement

\protect\hyperlink{after-bottom}{Continue reading the main story}

\hypertarget{site-index}{%
\subsection{Site Index}\label{site-index}}

\hypertarget{site-information-navigation}{%
\subsection{Site Information
Navigation}\label{site-information-navigation}}

\begin{itemize}
\tightlist
\item
  \href{https://help.nytimes.com/hc/en-us/articles/115014792127-Copyright-notice}{©~2020~The
  New York Times Company}
\end{itemize}

\begin{itemize}
\tightlist
\item
  \href{https://www.nytco.com/}{NYTCo}
\item
  \href{https://help.nytimes.com/hc/en-us/articles/115015385887-Contact-Us}{Contact
  Us}
\item
  \href{https://www.nytco.com/careers/}{Work with us}
\item
  \href{https://nytmediakit.com/}{Advertise}
\item
  \href{http://www.tbrandstudio.com/}{T Brand Studio}
\item
  \href{https://www.nytimes.com/privacy/cookie-policy\#how-do-i-manage-trackers}{Your
  Ad Choices}
\item
  \href{https://www.nytimes.com/privacy}{Privacy}
\item
  \href{https://help.nytimes.com/hc/en-us/articles/115014893428-Terms-of-service}{Terms
  of Service}
\item
  \href{https://help.nytimes.com/hc/en-us/articles/115014893968-Terms-of-sale}{Terms
  of Sale}
\item
  \href{https://spiderbites.nytimes.com}{Site Map}
\item
  \href{https://help.nytimes.com/hc/en-us}{Help}
\item
  \href{https://www.nytimes.com/subscription?campaignId=37WXW}{Subscriptions}
\end{itemize}
