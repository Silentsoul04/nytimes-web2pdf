Sections

SEARCH

\protect\hyperlink{site-content}{Skip to
content}\protect\hyperlink{site-index}{Skip to site index}

\href{https://www.nytimes.com/section/movies}{Movies}

\href{https://myaccount.nytimes.com/auth/login?response_type=cookie\&client_id=vi}{}

\href{https://www.nytimes.com/section/todayspaper}{Today's Paper}

\href{/section/movies}{Movies}\textbar{}`An American Pickle' Review:
Keeping It Kosher

\url{https://nyti.ms/3guLQmG}

\begin{itemize}
\item
\item
\item
\item
\item
\end{itemize}

\href{https://www.nytimes.com/spotlight/at-home?action=click\&pgtype=Article\&state=default\&region=TOP_BANNER\&context=at_home_menu}{At
Home}

\begin{itemize}
\tightlist
\item
  \href{https://www.nytimes.com/2020/08/03/well/family/the-benefits-of-talking-to-strangers.html?action=click\&pgtype=Article\&state=default\&region=TOP_BANNER\&context=at_home_menu}{Talk:
  To Strangers}
\item
  \href{https://www.nytimes.com/2020/08/01/at-home/coronavirus-make-pizza-on-a-grill.html?action=click\&pgtype=Article\&state=default\&region=TOP_BANNER\&context=at_home_menu}{Make:
  Grilled Pizza}
\item
  \href{https://www.nytimes.com/2020/07/31/arts/television/goldbergs-abc-stream.html?action=click\&pgtype=Article\&state=default\&region=TOP_BANNER\&context=at_home_menu}{Watch:
  'The Goldbergs'}
\item
  \href{https://www.nytimes.com/interactive/2020/at-home/even-more-reporters-editors-diaries-lists-recommendations.html?action=click\&pgtype=Article\&state=default\&region=TOP_BANNER\&context=at_home_menu}{Explore:
  Reporters' Google Docs}
\end{itemize}

Advertisement

\protect\hyperlink{after-top}{Continue reading the main story}

Supported by

\protect\hyperlink{after-sponsor}{Continue reading the main story}

\hypertarget{an-american-pickle-review-keeping-it-kosher}{%
\section{`An American Pickle' Review: Keeping It
Kosher}\label{an-american-pickle-review-keeping-it-kosher}}

A time-travel farce plays as a Jewish joke about an old-world immigrant
and his millennial great-grandson, both played by Seth Rogen.

\includegraphics{https://static01.nyt.com/images/2020/08/05/arts/05americanpickle2/merlin_175171032_fb69ff1d-73f9-44f8-ae33-d60453977f93-articleLarge.jpg?quality=75\&auto=webp\&disable=upscale}

\href{https://www.nytimes.com/by/a-o--scott}{\includegraphics{https://static01.nyt.com/images/2018/02/20/multimedia/author-a-o-scott/author-a-o-scott-thumbLarge.jpg}}

By \href{https://www.nytimes.com/by/a-o--scott}{A.O. Scott}

\begin{itemize}
\item
  Aug. 4, 2020
\item
  \begin{itemize}
  \item
  \item
  \item
  \item
  \item
  \end{itemize}
\end{itemize}

\begin{itemize}
\tightlist
\item
  An American Pickle\\
  Directed by Brandon Trost Comedy PG-13 1h 30m
\end{itemize}

\href{https://www.imdb.com/showtimes/title/tt9059704?ref_=ref_ext_NYT}{Find
Tickets}

When you purchase a ticket for an independently reviewed film through
our site, we earn an affiliate commission.

``An American Pickle,'' a time-travel farce directed by Brandon Trost
and adapted from a New Yorker story by Simon Rich, marinates crisp
almost-timeliness in the mild brine of nostalgia. It's not too salty or
too sour, and it's neither self-consciously artisanal nor aggressively,
weirdly authentic. The subject, more or less, is what it means to be
Jewish, and given how contentious that topic can become --- can I get an
oy vey? --- the movie finds an agreeable, occasionally touching vein of
humor.

The setup for most of the jokes is that, in 1919, an impoverished
immigrant named Herschel Greenbaum, recently arrived in Brooklyn from a
fictitious, Cossack-ridden
anti-\href{https://www.youtube.com/watch?v=F9E_PTTHvgI}{Anatevka} called
Schlupsk, falls into a vat of saltwater and cucumbers. He leaves behind
a pregnant wife, Sarah (Sarah Snook). She has a son, who has a son,
whose son, in 2019, is a sad-sack tech guy named Ben. When Herschel is
fished out of his century-long bath, alive and perfectly preserved, he
goes to live with Ben, his only known relative, setting up a
cross-generational odd-couple situation brimming with comic potential.

All the more so because both Herschel and Ben are played by Seth Rogen,
who does the bewhiskered Yiddish thing and the diffident millennial
thing with equal craftiness. While the characters are recognizable types
--- from popular culture if nowhere else --- Rogen brings more than mere
shtick to the performances. Herschel is neither a sentimental schlemiel
nor a twinkly old-world grandpa, but rather an impatient, sometimes
intolerant striver with a violent streak. His pre-pickling experience of
the world was hard and bitter, leavened only by the hope that future
generations of Greenbaums would be better off.

Which is just what happened, of course. Herschel once confessed to Sarah
that he hoped to taste seltzer water before he died, and Ben has a gizmo
in his apartment that makes it on demand. He's even less of a caricature
than his great-grandpa --- not a hipster or a nerd so much as a smart
guy with a deep streak of melancholy. It turns out that what connects
him to Herschel isn't just genetics: it's also grief. Ben's parents are
dead, and Herschel's accident robbed him of the pleasures and
consolations of family.

That's some pretty heavy stuff, but ``An American Pickle'' is swift and
nimble enough to avoid weighing itself down with schmaltz. It's almost
too thin to sustain its premise for the running time --- a scant 90
minutes --- and sometimes feels more like a stretched-out sketch than a
fully developed feature.

The century that separates Herschel from Ben allows the story to
leapfrog over quite a lot of history, including the Holocaust, Israel,
socialism, and the complicated process of upward mobility, acculturation
and self-preservation that is the movie's very condition of possibility.
The drama of Jewish male selfhood that preoccupied so many in the middle
generations --- the whole
\href{https://slate.com/culture/2001/08/the-estranged-twins.html}{Philip
Roth-Woody Allen} megillah --- is all but erased. Herschel had his
beloved Sarah. Ben has no apparent sexual or romantic interests, or even
any friends that we know about. There's no room for women in this pickle
jar.

But the flimsiness of the movie's conceit also works to its benefit. At
its best, it's a brisk, silly plucking of some low-hanging contemporary
fruit. Food trends. Social media. Unpaid internships. The inevitable
conflict between Herschel and Ben turns a family squabble into a
culture-war skirmish, a conflict played out in a way that feels both
satirically sharp and oddly comforting.

And pickles can be comfort food. Not too filling, good for the
digestion, noisy and a little sloppy rather than artful or exquisite or
challenging. This one, as I've said, isn't bad, and even allows a
soupçon of profundity into its formula. The tough, pious ancestor and
his sensitive, secular descendant have almost nothing in common, and the
imaginative challenge is to find an identity that can include them both
more or less as they are. What makes them both Jews? The answer turns
out to be simple and, at least for this conflicted 21st-century Jew,
persuasive: the shared obligation
\href{https://www.myjewishlearning.com/article/text-of-the-mourners-kaddish/}{to
mourn the dead}.

\textbf{An American Pickle}\\
Rated PG-13. A little violence, a little swearing. Running time: 1 hour
30 minutes. Watch on \href{https://www.hbomax.com/}{HBO Max.}

Advertisement

\protect\hyperlink{after-bottom}{Continue reading the main story}

\hypertarget{site-index}{%
\subsection{Site Index}\label{site-index}}

\hypertarget{site-information-navigation}{%
\subsection{Site Information
Navigation}\label{site-information-navigation}}

\begin{itemize}
\tightlist
\item
  \href{https://help.nytimes.com/hc/en-us/articles/115014792127-Copyright-notice}{©~2020~The
  New York Times Company}
\end{itemize}

\begin{itemize}
\tightlist
\item
  \href{https://www.nytco.com/}{NYTCo}
\item
  \href{https://help.nytimes.com/hc/en-us/articles/115015385887-Contact-Us}{Contact
  Us}
\item
  \href{https://www.nytco.com/careers/}{Work with us}
\item
  \href{https://nytmediakit.com/}{Advertise}
\item
  \href{http://www.tbrandstudio.com/}{T Brand Studio}
\item
  \href{https://www.nytimes.com/privacy/cookie-policy\#how-do-i-manage-trackers}{Your
  Ad Choices}
\item
  \href{https://www.nytimes.com/privacy}{Privacy}
\item
  \href{https://help.nytimes.com/hc/en-us/articles/115014893428-Terms-of-service}{Terms
  of Service}
\item
  \href{https://help.nytimes.com/hc/en-us/articles/115014893968-Terms-of-sale}{Terms
  of Sale}
\item
  \href{https://spiderbites.nytimes.com}{Site Map}
\item
  \href{https://help.nytimes.com/hc/en-us}{Help}
\item
  \href{https://www.nytimes.com/subscription?campaignId=37WXW}{Subscriptions}
\end{itemize}
