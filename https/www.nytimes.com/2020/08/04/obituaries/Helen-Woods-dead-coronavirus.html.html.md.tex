Sections

SEARCH

\protect\hyperlink{site-content}{Skip to
content}\protect\hyperlink{site-index}{Skip to site index}

\href{https://www.nytimes.com/section/obituaries}{Obituaries}

\href{https://myaccount.nytimes.com/auth/login?response_type=cookie\&client_id=vi}{}

\href{https://www.nytimes.com/section/todayspaper}{Today's Paper}

\href{/section/obituaries}{Obituaries}\textbar{}Helen Jones Woods,
Member of an All-Female Jazz Group, Dies at 96

\url{https://nyti.ms/2DjRoSC}

\begin{itemize}
\item
\item
\item
\item
\item
\end{itemize}

\href{https://www.nytimes.com/news-event/coronavirus?action=click\&pgtype=Article\&state=default\&region=TOP_BANNER\&context=storylines_menu}{The
Coronavirus Outbreak}

\begin{itemize}
\tightlist
\item
  live\href{https://www.nytimes.com/2020/08/04/world/coronavirus-cases.html?action=click\&pgtype=Article\&state=default\&region=TOP_BANNER\&context=storylines_menu}{Latest
  Updates}
\item
  \href{https://www.nytimes.com/interactive/2020/us/coronavirus-us-cases.html?action=click\&pgtype=Article\&state=default\&region=TOP_BANNER\&context=storylines_menu}{Maps
  and Cases}
\item
  \href{https://www.nytimes.com/interactive/2020/science/coronavirus-vaccine-tracker.html?action=click\&pgtype=Article\&state=default\&region=TOP_BANNER\&context=storylines_menu}{Vaccine
  Tracker}
\item
  \href{https://www.nytimes.com/2020/08/02/us/covid-college-reopening.html?action=click\&pgtype=Article\&state=default\&region=TOP_BANNER\&context=storylines_menu}{College
  Reopening}
\item
  \href{https://www.nytimes.com/live/2020/08/04/business/stock-market-today-coronavirus?action=click\&pgtype=Article\&state=default\&region=TOP_BANNER\&context=storylines_menu}{Economy}
\end{itemize}

Advertisement

\protect\hyperlink{after-top}{Continue reading the main story}

Supported by

\protect\hyperlink{after-sponsor}{Continue reading the main story}

Those We've Lost

\hypertarget{helen-jones-woods-member-of-an-all-female-jazz-group-dies-at-96}{%
\section{Helen Jones Woods, Member of an All-Female Jazz Group, Dies at
96}\label{helen-jones-woods-member-of-an-all-female-jazz-group-dies-at-96}}

She played trombone in the multiracial International Sweethearts of
Rhythm, but later put down her horn forever. She died of the
coronavirus.

\includegraphics{https://static01.nyt.com/images/2020/08/05/obituaries/03Woods/merlin_175276422_757cded5-0e31-468c-849c-48f773003b53-articleLarge.jpg?quality=75\&auto=webp\&disable=upscale}

\href{https://www.nytimes.com/by/john-leland}{\includegraphics{https://static01.nyt.com/images/2018/02/20/multimedia/author-john-leland/author-john-leland-thumbLarge.jpg}}

By \href{https://www.nytimes.com/by/john-leland}{John Leland}

\begin{itemize}
\item
  Aug. 4, 2020
\item
  \begin{itemize}
  \item
  \item
  \item
  \item
  \item
  \end{itemize}
\end{itemize}

\emph{This obituary is part of a series about people who have died in
the coronavirus pandemic. Read about others}
\href{https://www.nytimes.com/interactive/2020/obituaries/people-died-coronavirus-obituaries.html}{\emph{here}}\emph{.}

Helen Jones Woods was an African-American jazz musician who toured the
country, including the Jim Crow South, in the 1930s and '40s. This could
be the start of a familiar story of racism on the road. But Ms. Woods's
journey has some distinctive wrinkles.

Ms. Woods played trombone in the
\href{https://www.oxfordamerican.org/magazine/item/297-the-international-sweethearts-of-rhythm}{International
Sweethearts of Rhythm}, an all-female, multiracial ensemble so anomalous
that the white members had to wear blackface in the South to avoid
trouble.

When the group split up in 1949 --- bruised by the road and feeling
exploited financially --- Ms. Woods found the classical world no less
racist. After her first performance with the Omaha Symphony, her father,
who did not share her light complexion, picked her up, tipping off the
orchestra that she was not white.

``They fired her,'' said Ms. Woods's daughter
\href{http://cathyhughes.com/}{Cathy Hughes}, a founder and chairperson
of Urban One, a media company that focuses on Black culture. ``She never
touched her horn again.''

Ms. Woods died on July 25 of the coronavirus in a hospital in Sarasota,
Fla., her daughter said. She was 96.

Helen Elizabeth Jones was born on either Oct. 9 or Nov. 14, 1923 (family
documents differ), and spent some of her earliest days in an orphanage
for white children in Meridian, Miss. Upon realizing she was not white,
the orphanage no longer wanted her, and she was adopted by Dr.
\href{https://en.wikipedia.org/wiki/Laurence_C._Jones}{Laurence Clifton
Jones} and his wife, Grace. Dr. Jones was the founder of the
\href{https://www.pineywoods.org/}{Piney Woods Country Life School} (now
the Piney Woods School), a Black boarding school; Ms. Jones's
grandmother, Ms. Hughes said, lectured with Frederick Douglass and may
have worked on the Underground Railroad with Harriet Tubman.

Piney Woods had a strong musical bent and was the birthplace of the
gospel group the
\href{https://en.wikipedia.org/wiki/Five_Blind_Boys_of_Mississippi}{Five
Blind Boys of Mississippi}. To raise money for the school, Grace Jones
started male and female quartets called the
\href{https://kihm4.wordpress.com/2013/05/20/the-cotton-blossom-singers/}{Cotton
Blossom Singers}, precursors to the International Sweethearts.

From an early age, Helen was fascinated by the slide motion of the
trombone. She played in the girls' band at Piney Woods and at 13 or 14
became an original member of the Swinging Rays of Rhythm, a school band
modeled on a white all-female group popular on the radio at the time.

The Rays were a hit. They lived on a dollar a day each for food, plus a
dollar a week allowance, and they toured in two school buses that had
been retrofitted --- one with bunk beds, the other as a mobile
classroom.

When they met a smooth-talking manager in Washington who promised the
girls diamond rings if they would abandon the school and turn pro, they
commandeered the buses and absconded to Virginia, renaming themselves
the \href{https://youtu.be/WczP3PyHt20}{International Sweethearts of
Rhythm}, a nod to the diverse races and nationalities of the musicians.

Success followed: performances at the Apollo Theater in New York and
Wrigley Field in Chicago; bills with
\href{https://www.nytimes.com/1993/01/07/arts/dizzy-gillespie-who-sounded-some-of-modern-jazz-s-earliest-notes-dies-at-75.html}{Dizzy
Gillespie},
\href{https://www.nytimes.com/1959/07/18/archives/billie-holiday-dies-here-at-14-jazz-singer-had-wide-influence.html}{Billie
Holiday} and
\href{https://www.nytimes.com/1996/06/16/nyregion/ella-fitzgerald-the-voice-of-jazz-dies-at-79.html}{Ella
Fitzgerald}. DownBeat magazine in 1944 called them ``America's number
one all-female orchestra.''

Despite their popularity, money never seemed to flow their way, nor
diamond rings. ``The girls were ripped off, arrested, harassed and
bullied,'' Ms. Hughes said. ``But they loved the music, so they
continued.'' They recorded only a few songs, but recordings of some of
their radio performances have survived, and they were the subject of a
short
\href{https://jezebelproductions.org/international-sweethearts-of-rhythm/}{documentary
film} made in 1986.

In the late 1940s, Helen Jones met and married William Alfred Woods.
They made a home in Omaha and had four children. In addition to Ms.
Hughes, Ms. Woods is survived by her sons, William and Robert, and
another daughter, Jacquelyn Marie Woods Williams.

The couple separated but never divorced. Mr. Woods died at 45.

After leaving the Omaha orchestra, Ms. Woods returned to school,
studying at Creighton University and the University of Nebraska, and
then worked for 30 years as a registered nurse and social worker.

She never returned to music. When the surviving members of the
International Sweethearts gave a reunion concert in 1980, their first
performance in 30 years, she could not bring herself to take part, Ms.
Hughes said. Instead, she said, ``She just started crying and walked out
of the ballroom.''

At a \href{https://www.youtube.com/watch?v=_Cjmg8Jepvw}{2011 discussion}
organized by the Smithsonian Institution's National Museum of American
History, Ms. Woods was asked whether the hard work of being a musician
was worth it. She resisted romanticizing a tough past.

``I don't know if it paid off,'' she said. ``I didn't get enough
money.''

But her music, however few its artifacts, carries on.

\href{https://www.nytimes.com/interactive/2020/obituaries/people-died-coronavirus-obituaries.html?action=click\&pgtype=Article\&state=default\&region=BELOW_MAIN_CONTENT\&context=covid_obits_promo}{}

\hypertarget{those-weve-lost}{%
\section{Those We've Lost}\label{those-weve-lost}}

The coronavirus pandemic has taken an incalculable death toll. This
series is designed to put names and faces to the numbers.

Read more

\includegraphics{https://static01.nyt.com/images/2020/08/05/obituaries/03Woods/03Woods-square640.jpg}

\hypertarget{helen-jones-woods}{%
\section{Helen Jones Woods}\label{helen-jones-woods}}

d. Sarasota, Fla.

Musician in all-female, multi-racial jazz band

\includegraphics{https://static01.nyt.com/images/2020/08/05/obituaries/30Pedro/30Pedro-square640.jpg}

\hypertarget{bernaldina-josuxe9-pedro}{%
\section{Bernaldina José Pedro}\label{bernaldina-josuxe9-pedro}}

d. Boa Vista, Brazil

Leader among the Indigenous Macuxi

\includegraphics{https://static01.nyt.com/images/2020/08/05/obituaries/31Swing/merlin_175167783_8913bc90-0d64-43f3-a655-1bb1bf1601c9-square640.jpg}

\hypertarget{john-eric-swing}{%
\section{John Eric Swing}\label{john-eric-swing}}

d. Fountain Valley, Calif.

Champion of Filipino-Americans

\includegraphics{https://static01.nyt.com/images/2020/07/27/obituaries/27Victor/merlin_175001436_38b11f8e-227a-4e2c-9821-7618af9b2524-square640.jpg}

\hypertarget{victor-victor}{%
\section{Victor Victor}\label{victor-victor}}

d. Santo Domingo, Dominican Republic

Beloved musician of the Dominican Republic

\includegraphics{https://static01.nyt.com/images/2020/08/05/obituaries/31Negron/merlin_175160169_516322ae-fd23-4969-b6b2-193ced371105-square640.jpg}

\hypertarget{dr-eddie-negruxf3n}{%
\section{Dr. Eddie Negrón}\label{dr-eddie-negruxf3n}}

d. Fort Walton Beach, Fla.

Internist on Florida's Emerald Coast

\includegraphics{https://static01.nyt.com/images/2020/07/30/obituaries/30Dobson/merlin_175115928_f6b9271c-8f05-4fe1-a38a-5ca4a58f8935-square640.jpg}

\hypertarget{dobby-dobson}{%
\section{Dobby Dobson}\label{dobby-dobson}}

d. Coral Springs, Fla.

Jamaican singer and songwriter

Advertisement

\protect\hyperlink{after-bottom}{Continue reading the main story}

\hypertarget{site-index}{%
\subsection{Site Index}\label{site-index}}

\hypertarget{site-information-navigation}{%
\subsection{Site Information
Navigation}\label{site-information-navigation}}

\begin{itemize}
\tightlist
\item
  \href{https://help.nytimes.com/hc/en-us/articles/115014792127-Copyright-notice}{©~2020~The
  New York Times Company}
\end{itemize}

\begin{itemize}
\tightlist
\item
  \href{https://www.nytco.com/}{NYTCo}
\item
  \href{https://help.nytimes.com/hc/en-us/articles/115015385887-Contact-Us}{Contact
  Us}
\item
  \href{https://www.nytco.com/careers/}{Work with us}
\item
  \href{https://nytmediakit.com/}{Advertise}
\item
  \href{http://www.tbrandstudio.com/}{T Brand Studio}
\item
  \href{https://www.nytimes.com/privacy/cookie-policy\#how-do-i-manage-trackers}{Your
  Ad Choices}
\item
  \href{https://www.nytimes.com/privacy}{Privacy}
\item
  \href{https://help.nytimes.com/hc/en-us/articles/115014893428-Terms-of-service}{Terms
  of Service}
\item
  \href{https://help.nytimes.com/hc/en-us/articles/115014893968-Terms-of-sale}{Terms
  of Sale}
\item
  \href{https://spiderbites.nytimes.com}{Site Map}
\item
  \href{https://help.nytimes.com/hc/en-us}{Help}
\item
  \href{https://www.nytimes.com/subscription?campaignId=37WXW}{Subscriptions}
\end{itemize}
