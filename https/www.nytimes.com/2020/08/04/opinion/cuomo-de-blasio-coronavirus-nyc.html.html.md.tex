Sections

SEARCH

\protect\hyperlink{site-content}{Skip to
content}\protect\hyperlink{site-index}{Skip to site index}

\href{https://myaccount.nytimes.com/auth/login?response_type=cookie\&client_id=vi}{}

\href{https://www.nytimes.com/section/todayspaper}{Today's Paper}

\href{/section/opinion}{Opinion}\textbar{}New York Needs Less Bickering,
More Teamwork

\url{https://nyti.ms/2Pm55CU}

\begin{itemize}
\item
\item
\item
\item
\item
\item
\end{itemize}

Advertisement

\protect\hyperlink{after-top}{Continue reading the main story}

\href{/section/opinion}{Opinion}

Supported by

\protect\hyperlink{after-sponsor}{Continue reading the main story}

\hypertarget{new-york-needs-less-bickering-more-teamwork}{%
\section{New York Needs Less Bickering, More
Teamwork}\label{new-york-needs-less-bickering-more-teamwork}}

Residents are nervous and exhausted. It's not too much to ask for a
unified leadership from the mayor and the governor.

By
\href{https://www.nytimes.com/interactive/opinion/editorialboard.html}{The
Editorial Board}

The editorial board is a group of opinion journalists whose views are
informed by expertise, research, debate and certain longstanding ****
\href{https://www.nytimes.com/interactive/2018/opinion/editorialboard.html}{values}.
It is separate from the newsroom.

\begin{itemize}
\item
  Aug. 4, 2020
\item
  \begin{itemize}
  \item
  \item
  \item
  \item
  \item
  \item
  \end{itemize}
\end{itemize}

\includegraphics{https://static01.nyt.com/images/2020/08/04/opinion/04feud-editorial/04feud-editorial-articleLarge.jpg?quality=75\&auto=webp\&disable=upscale}

Gov. Andrew Cuomo and Mayor Bill de Blasio have before them some of the
most difficult challenges any holder of their respective offices has
ever faced. Both men are often forced to choose between the least worst
of many bad options. So, you'd think this great crisis would be an
opportunity for the mayor and the governor to set aside their
long-running feud and work together. The prognosis isn't great.

Even as New Yorkers face their most wrenching, and consequential,
decision --- whether to send their children into schools during a
pandemic --- Mr. Cuomo and Mr. de Blasio quibble and often put the
personal over the professional to a dispiriting degree.

On Friday, the city submitted a 32-page reopening plan in which most
children would go to school two or three days a week and continue online
instruction on the other days. The plan, in some ways, established even
more careful measures of safety than the governor had suggested, calling
for schools to be shut if more than 3 percent of coronavirus tests in
the city proved positive. The governor had suggested that 5 percent be
the standard.

Rather than collaborating with city officials to ensure that the plan
could have the confidence of New York's parents and school employees,
the governor's initial public response was to slap it down.

``Just because a school district says `we're open' does not mean
students are going to go,'' Mr. Cuomo
\href{https://www.nydailynews.com/news/politics/ny-cuomo-de-blasio-schools-reopen-20200802-s3bdp5yix5he7mrxr2uez2cwfu-story.html}{said}
on Sunday.

``We'll accomplish nothing if we open the schools, but a significant
number of parents decide to keep their children home,'' he added.

One of Mr. Cuomo's advisers had already
\href{https://nypost.com/2020/08/02/senior-cuomo-aide-slams-de-blasios-reopening-plan-as-an-outline/}{called}
the proposal ``an outline'' rather than a plan and noted that a school
reopening proposal in Yonkers, a much smaller city, was roughly 50 pages
longer.

On Monday, the mayor said he was ``past the point of irritation'' with
the governor. He's not alone.

Mr. Cuomo is right that parents' trust is necessary. New York City's
teachers' union already has expressed skepticism about the reopening
plan. The city's rollout of online instruction left little reason for
confidence in its plan to resume in-class instruction, nor does the
\href{https://www.nytimes.com/2020/08/04/nyregion/oxiris-barbot-health-commissioner-resigns.html?referringSource=articleShare}{resignation}
of the city's health commissioner on Tuesday. And the governor's warning
also applied to other districts in the state.

Still, this political Punch and Judy show has grown tiresome. It began
shortly after the mayor's first inauguration, in 2014, when the governor
balked at Mr. de Blasio's signature initiative, providing
prekindergarten to all of the city's families. Ultimately the governor
made universal pre-K a statewide program, to the benefit of all, but
their rivalry had begun.

It continued with bickering over responsibility for the collapse of the
subway system --- remember when \emph{that} was an existential urban
crisis?

The feuding continued over who should take responsibility for the
deterioration of the city's public housing and who was to blame for
Amazon's decision to abandon plans for a major development in Queens.

It continued as the coronavirus pandemic loomed. Before the city shut
down schools, restaurants and bars in mid-March, the mayor and the
governor squabbled over
\href{https://www.nytimes.com/2020/03/17/nyregion/coronavirus-nyc-shelter-in-place.html}{who
had the authority} to make such decisions.

Shutting schools as the pandemic surged was unavoidable. The failure of
the online alternative makes reopening them urgent. Doing so safely will
be a huge challenge, though, and many questions remain. How great a
threat is the coronavirus to children, and how easily do they transmit
it? How can schools be made safe to staff members and the communities of
adults that surround them? How can staff members be effectively tested
for infection? Schools that have reopened elsewhere in the country
already have seen outbreaks.

New York City and New York State made their share of early missteps in
fighting the pandemic and are trying to avoid making more. Yet a corner
has been turned, and the threat is being contained.

The mutual loathing between mayor and governor impedes that success. Mr.
Cuomo and Mr. de Blasio can't let it get in the way of educating
children and keeping their constituents healthy.

\emph{The Times is committed to publishing}
\href{https://www.nytimes.com/2019/01/31/opinion/letters/letters-to-editor-new-york-times-women.html}{\emph{a
diversity of letters}} \emph{to the editor. We'd like to hear what you
think about this or any of our articles. Here are some}
\href{https://help.nytimes.com/hc/en-us/articles/115014925288-How-to-submit-a-letter-to-the-editor}{\emph{tips}}\emph{.
And here's our email:}
\href{mailto:letters@nytimes.com}{\emph{letters@nytimes.com}}\emph{.}

\emph{Follow The New York Times Opinion section on}
\href{https://www.facebook.com/nytopinion}{\emph{Facebook}}\emph{,}
\href{http://twitter.com/NYTOpinion}{\emph{Twitter (@NYTopinion)}}
\emph{and}
\href{https://www.instagram.com/nytopinion/}{\emph{Instagram}}\emph{.}

Advertisement

\protect\hyperlink{after-bottom}{Continue reading the main story}

\hypertarget{site-index}{%
\subsection{Site Index}\label{site-index}}

\hypertarget{site-information-navigation}{%
\subsection{Site Information
Navigation}\label{site-information-navigation}}

\begin{itemize}
\tightlist
\item
  \href{https://help.nytimes.com/hc/en-us/articles/115014792127-Copyright-notice}{©~2020~The
  New York Times Company}
\end{itemize}

\begin{itemize}
\tightlist
\item
  \href{https://www.nytco.com/}{NYTCo}
\item
  \href{https://help.nytimes.com/hc/en-us/articles/115015385887-Contact-Us}{Contact
  Us}
\item
  \href{https://www.nytco.com/careers/}{Work with us}
\item
  \href{https://nytmediakit.com/}{Advertise}
\item
  \href{http://www.tbrandstudio.com/}{T Brand Studio}
\item
  \href{https://www.nytimes.com/privacy/cookie-policy\#how-do-i-manage-trackers}{Your
  Ad Choices}
\item
  \href{https://www.nytimes.com/privacy}{Privacy}
\item
  \href{https://help.nytimes.com/hc/en-us/articles/115014893428-Terms-of-service}{Terms
  of Service}
\item
  \href{https://help.nytimes.com/hc/en-us/articles/115014893968-Terms-of-sale}{Terms
  of Sale}
\item
  \href{https://spiderbites.nytimes.com}{Site Map}
\item
  \href{https://help.nytimes.com/hc/en-us}{Help}
\item
  \href{https://www.nytimes.com/subscription?campaignId=37WXW}{Subscriptions}
\end{itemize}
