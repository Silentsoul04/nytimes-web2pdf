Sections

SEARCH

\protect\hyperlink{site-content}{Skip to
content}\protect\hyperlink{site-index}{Skip to site index}

\href{https://myaccount.nytimes.com/auth/login?response_type=cookie\&client_id=vi}{}

\href{https://www.nytimes.com/section/todayspaper}{Today's Paper}

\href{/section/opinion}{Opinion}\textbar{}Is Microsoft Sure It Wants to
Buy TikTok?

\url{https://nyti.ms/3kbDthU}

\begin{itemize}
\item
\item
\item
\item
\item
\end{itemize}

Advertisement

\protect\hyperlink{after-top}{Continue reading the main story}

\href{/section/opinion}{Opinion}

Supported by

\protect\hyperlink{after-sponsor}{Continue reading the main story}

\hypertarget{is-microsoft-sure-it-wants-to-buy-tiktok}{%
\section{Is Microsoft Sure It Wants to Buy
TikTok?}\label{is-microsoft-sure-it-wants-to-buy-tiktok}}

Running a social media service brings headaches and scrutiny.

By Greg Bensinger

Mr. Bensinger is a member of the editorial board.

\begin{itemize}
\item
  Aug. 4, 2020
\item
  \begin{itemize}
  \item
  \item
  \item
  \item
  \item
  \end{itemize}
\end{itemize}

\includegraphics{https://static01.nyt.com/images/2020/08/05/opinion/05besigner_tiktok/05besigner_tiktok-mediumSquareAt3X.jpg}

\href{https://cn.nytimes.com/opinion/20200805/tiktok-microsoft-trump/}{阅读简体中文版}\href{https://cn.nytimes.com/opinion/20200805/tiktok-microsoft-trump/zh-hant/}{閱讀繁體中文版}

Microsoft has emerged in recent days as the leading suitor for the
fast-growing but embattled social media app TikTok. What the company may
be buying is a big headache.

The tech giant said it would continue negotiating toward a purchase of
the short-video-sharing service amid President Trump's attempts to ban
from the United States or wrest control of TikTok, which is owned by a
Chinese company. The president's concerns ostensibly revolve around
security, particularly the threat --- real or imagined --- that the
Chinese government may gain access to TikTok's valuable data on American
citizens.

Mr. Trump on Monday
\href{https://www.nytimes.com/2020/08/03/technology/trump-tiktok-microsoft.html}{blessed
Microsoft's pursuit} of some of TikTok's assets, including versions of
the app that are available in the United States, Australia, Canada and
New Zealand. If Microsoft or another ``very American'' company fails to
reach a deal to bring TikTok in the U.S. under local control by Sept.
15, Mr. Trump said, he will force it to shut down.

TikTok offers a steady stream of user-made videos, featuring everything
from dance moves to cooking tips to piano-playing cats to movie
re-enactments, most clocking in at less than a minute. The app is wildly
popular with teenagers and young adults and is backed by software that
delivers videos to users matching their interests. It's great fun.

But what would staid old Microsoft, purveyor of Windows and Excel, get
out of this deal? Instant access to a demographic that has largely
bypassed it.

The company would, though, also be taking on all of the hassles and
dangers that come with running a social media service: hate speech,
misinformation, trolling, nudity, copyright infringement. Interspersed
with TikTok's largely anodyne content are growing
\href{https://www.huffpost.com/entry/far-right-tiktok-gen-z_n_5cb63040e4b082aab08da0d3}{far-right
communities},
\href{https://www.vice.com/en_us/article/yw74gy/tiktok-neo-nazis-white-supremacy}{white
supremacists} and
\href{https://www.rollingstone.com/culture/culture-features/tiktok-conspiracy-theories-bill-gates-microchip-vaccine-996394/}{Covid-19
misinformation}.

TikTok has said it removed
\href{https://www.theverge.com/2020/7/9/21317832/tiktok-content-violations-videos-removed-49-million-2h2019-transparency-report}{49
million videos} for various reasons in the last half of 2019, compared
with about 15 million taken down by its much larger rival YouTube. Like
Facebook, TikTok has formed
\href{https://techcrunch.com/2020/03/18/tiktok-brings-in-outside-experts-to-help-it-craft-moderation-and-content-policies/}{an
outside group} to help police its content.

``Microsoft is buying itself real aggravation,'' said Gigi Sohn, a
former Federal Communications Commission senior adviser and a fellow at
the Georgetown Law Institute for Technology Law \& Policy. ``There's no
reason hate speech and misinformation won't just keep growing.''

Whether or not TikTok funnels personal information to the Chinese
government, its collection of data like locations, internet addresses
and private messages rivals that of Facebook,
\href{https://www.washingtonpost.com/technology/2020/07/13/tiktok-privacy/}{The
Washington Post has reported}. India
\href{https://www.nytimes.com/2020/06/30/technology/india-china-tiktok.html}{banned}
the app over privacy concerns, and Secretary of State Mike Pompeo has
raised similar concerns. In part to quell that unease, TikTok has hired
an American chief executive and says data is stored only in the United
States and Singapore.

Microsoft so far has declined to comment on its interest in TikTok
beyond
\href{https://blogs.microsoft.com/blog/2020/08/02/microsoft-to-continue-discussions-on-potential-tiktok-purchase-in-the-united-states/}{a
blog post} over the weekend assuring users that information collected
from TikTok would be kept in the United States, and data housed
elsewhere would be deleted.

The company would need to do much more than that to protect users.
Though Facebook, YouTube and Twitter have deployed artificial
intelligence software and hired thousands of content moderators to take
down vile or misleading posts, dangerous content inevitably slips
through. Recently, a
\href{https://www.nytimes.com/2020/07/28/technology/virus-video-trump.html}{video
spreading bogus claims} about the value of using hydroxychloroquine to
treat Covid-19 racked up millions of views on YouTube, Facebook and
Twitter and was shared by members of the Trump family before being taken
down.

Microsoft has little experience in the tricky and subjective art of
corralling a social media site. It
\href{https://www.nytimes.com/2016/12/08/technology/with-linkedin-microsoft-looks-to-avoid-past-acquisition-busts.html}{bought}
LinkedIn for about \$26 billion in 2016, but that's a reliably sober
platform. If the company is successful in snapping up TikTok's U.S.
operations, and its roughly 100 million American users, Microsoft should
immediately beef up TikTok's policies with a clear, aggressive code of
conduct of its own --- and then actually enforce that policy.

If YouTube, Facebook and Twitter are any indication, Microsoft would
have to defend itself against significant criticism from both sides of
the political aisle about unevenly enforced policies. Does the company
have the wherewithal to ramp up content removal or blocking of users
when it might anger a key political or business ally --- or a TikTok
star?

TikTok users already have demonstrated their might, including
\href{https://www.nytimes.com/2020/06/21/style/tiktok-trump-rally-tulsa.html}{claims}
to registering en masse for tickets to President Trump's Tulsa, Okla.,
rally that they never redeemed.

There are, of course, serious matters of international diplomacy to
consider in this deal. There's also the concern that, despite vehement
denials by TikTok's parent, ByteDance, the Chinese government may now or
in the future collect and analyze the reams of data that the app culls
from its roughly 800 million users. In the United States,
\href{https://www.nytimes.com/reuters/2020/07/22/technology/22reuters-usa-tiktok-vote.html}{senators},
Mr. Trump's cabinet and, apparently,
\href{https://www.nytimes.com/2020/07/10/technology/tiktok-amazon-security-risk.html}{Amazon}
all have developed a mistrust of the service.

``While TikTok isn't immune to the challenges that all platforms face,
the fact that our users come to express their creative sides immediately
makes TikTok a much more uplifting environment than one might experience
on other platforms,'' a company spokeswoman, Jamie Favazza, said in a
statement. Ms. Favazza noted that TikTok forbids posting misleading
information about elections and removes content that violates its
policies, and she reiterated that ByteDance doesn't share data with the
Chinese government.

Microsoft has a perfectly
\href{https://www.wsj.com/articles/microsoft-revenue-surges-thoughazure-cloud-growth-slows-11595449687}{profitable
business} without TikTok, meaning the potential scrutiny must be worth
it to score tech's most coveted prize: data, data, data. The investment
firm Cowen Group estimated the coveted 18- to 24-year-old demographic
averages nearly an hour a day on TikTok, compared with 44 minutes on
Instagram and 36 minutes on Snapchat. And TikTok could be worth
\href{https://markets.businessinsider.com/news/stocks/microsoft-tiktok-stock-add-value-billion-deal-wall-street-analyst-2020-8-1029463140\#}{\$200
billion} to Microsoft in just three years, estimates Wedbush Securities.

Microsoft has
\href{https://www.nytimes.com/2020/08/03/technology/tiktok-microsoft-tweens.html}{longed}
to be one of the cool kids of tech. Last week it was left out of a House
antitrust subcommittee hearing where the chief executives of Facebook,
Google, Amazon and Apple were peppered with questions about a host of
issues, including bias, post removals and their ever-shifting policies.
Cool or not, buying TikTok just might get Microsoft into that hot seat.

\emph{The Times is committed to publishing}
\href{https://www.nytimes.com/2019/01/31/opinion/letters/letters-to-editor-new-york-times-women.html}{\emph{a
diversity of letters}} \emph{to the editor. We'd like to hear what you
think about this or any of our articles. Here are some}
\href{https://help.nytimes.com/hc/en-us/articles/115014925288-How-to-submit-a-letter-to-the-editor}{\emph{tips}}\emph{.
And here's our email:}
\href{mailto:letters@nytimes.com}{\emph{letters@nytimes.com}}\emph{.}

\emph{Follow The New York Times Opinion section on}
\href{https://www.facebook.com/nytopinion}{\emph{Facebook}}\emph{,}
\href{http://twitter.com/NYTOpinion}{\emph{Twitter (@NYTopinion)}}
\emph{and}
\href{https://www.instagram.com/nytopinion/}{\emph{Instagram}}\emph{.}

Advertisement

\protect\hyperlink{after-bottom}{Continue reading the main story}

\hypertarget{site-index}{%
\subsection{Site Index}\label{site-index}}

\hypertarget{site-information-navigation}{%
\subsection{Site Information
Navigation}\label{site-information-navigation}}

\begin{itemize}
\tightlist
\item
  \href{https://help.nytimes.com/hc/en-us/articles/115014792127-Copyright-notice}{©~2020~The
  New York Times Company}
\end{itemize}

\begin{itemize}
\tightlist
\item
  \href{https://www.nytco.com/}{NYTCo}
\item
  \href{https://help.nytimes.com/hc/en-us/articles/115015385887-Contact-Us}{Contact
  Us}
\item
  \href{https://www.nytco.com/careers/}{Work with us}
\item
  \href{https://nytmediakit.com/}{Advertise}
\item
  \href{http://www.tbrandstudio.com/}{T Brand Studio}
\item
  \href{https://www.nytimes.com/privacy/cookie-policy\#how-do-i-manage-trackers}{Your
  Ad Choices}
\item
  \href{https://www.nytimes.com/privacy}{Privacy}
\item
  \href{https://help.nytimes.com/hc/en-us/articles/115014893428-Terms-of-service}{Terms
  of Service}
\item
  \href{https://help.nytimes.com/hc/en-us/articles/115014893968-Terms-of-sale}{Terms
  of Sale}
\item
  \href{https://spiderbites.nytimes.com}{Site Map}
\item
  \href{https://help.nytimes.com/hc/en-us}{Help}
\item
  \href{https://www.nytimes.com/subscription?campaignId=37WXW}{Subscriptions}
\end{itemize}
