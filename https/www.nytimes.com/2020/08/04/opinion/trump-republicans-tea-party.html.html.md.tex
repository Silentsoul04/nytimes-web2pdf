Sections

SEARCH

\protect\hyperlink{site-content}{Skip to
content}\protect\hyperlink{site-index}{Skip to site index}

\href{https://myaccount.nytimes.com/auth/login?response_type=cookie\&client_id=vi}{}

\href{https://www.nytimes.com/section/todayspaper}{Today's Paper}

\href{/section/opinion}{Opinion}\textbar{}Republicans Are Ready for the
Don Draper Method

\href{https://nyti.ms/3i9uvjF}{https://nyti.ms/3i9uvjF}

\begin{itemize}
\item
\item
\item
\item
\item
\item
\end{itemize}

Advertisement

\protect\hyperlink{after-top}{Continue reading the main story}

\href{/section/opinion}{Opinion}

Supported by

\protect\hyperlink{after-sponsor}{Continue reading the main story}

\hypertarget{republicans-are-ready-for-the-don-draper-method}{%
\section{Republicans Are Ready for the Don Draper
Method}\label{republicans-are-ready-for-the-don-draper-method}}

The coronavirus relief debate has Republicans falling back into
pre-Trump grooves.

\href{https://www.nytimes.com/by/ross-douthat}{\includegraphics{https://static01.nyt.com/images/2018/04/03/opinion/ross-douthat/ross-douthat-thumbLarge.png}}

By \href{https://www.nytimes.com/by/ross-douthat}{Ross Douthat}

Opinion Columnist

\begin{itemize}
\item
  Aug. 4, 2020
\item
  \begin{itemize}
  \item
  \item
  \item
  \item
  \item
  \item
  \end{itemize}
\end{itemize}

\includegraphics{https://static01.nyt.com/images/2020/08/04/opinion/04douthatSub/04douthatSub-articleLarge.jpg?quality=75\&auto=webp\&disable=upscale}

``This never happened. It will \emph{shock} you how much it never
happened.''

This is a Don Draper line from ``Mad Men,''
\href{https://www.youtube.com/watch?v=kEMe3wj-QuM}{delivered} as advice
he earned the hard way, when he stole another man's identity and left
his own behind.

It's also the way that many Republican senators hope to deal with the
memory of the Trump era, assuming that they wake up on Nov. 4 (or
whenever the ballot counting ends) and discover that the president has
not been re-elected.

Acting as if Trumpism ``never happened'' doesn't just mean they want to
blot out their memories of Trump himself, his Twitter feed, their
unwilling ring-kissing, all the rest. It means that many of them believe
that Trump's election was essentially an accident, a fluke, a temporary
hiatus from the kind of conservative politics they're comfortable
practicing, and so if he loses there's no reason the Republican Party
can't go back to the way things used to be.

One of the last times I was in Washington, in days when it was still
normal to hop a plane to our nation's capital, a smart Republican
staffer remarked to me that out of his entire caucus, only a small group
of senators thought the G.O.P. had something significant to learn from
Trump's ascent.

The rest were ready for the Draper method.

You can see that readiness at work already in the internal Republican
debates about the latest round of coronavirus relief. These debates are
somewhat mystifying if you believe that the party has been remade in
Trump's populist image, or alternatively if you just believe that the
G.O.P. is full of cynics who attack deficits under Democrats but happily
spend whatever it takes to stay in power. Neither theory explains the
Republican determination to dramatically underbid the Democrats on
relief spending three months before an election, nor the emergence of a
faction within the Senate Republicans that doesn't want to spend more
money on relief at all.

But these developments are easier to understand if you see the
Republican Senate, in what feels like the twilight of the Trump
presidency, instinctively returning to its pre-Trump battle lines. The
anti-relief faction, with its sudden warnings about deficits, is eager
to revive the Tea Party spirit, and its would-be leaders are ur-Tea
Partyers like Rand Paul and Ted Cruz. The faction that wants to spend
less than the Democrats but ultimately wants to strike a deal is playing
the same beleaguered-establishmentarian role that John Boehner and Mitch
McConnell played in the pre-Trump party --- and of course McConnell is
still leading it. And the fact that neither approach seems responsive to
the actual crisis unfolding in America right now doesn't matter: The old
Tea Party-establishment battle --- a battle over \emph{whether to cut a
deal at all}, more than \emph{what should be in it} --- is still the
Republican comfort zone, and the opportunity to slip back into that
groove is just too tempting to resist.

Of course there is cynicism as well as ideological comfort at work. Some
of the Republicans rediscovering deficit hawkishness --- including
non-senators like
\href{https://medium.com/@nikkihaley/a-day-of-reckoning-is-coming-with-the-national-debt-c30296bffe50}{Nikki
Haley} --- are taking a Joe Biden presidency for granted and positioning
themselves as the foes of a big-government liberalism before it even
takes power, in the hopes of becoming the leaders of the post-2020
opposition.

But it's not clear that self-interest rightly understood would incline
an ambitious Republican to bring back the old Tea Party spirit. After
all, the lesson of 2016 was that Ted Cruz didn't win, and instead True
Conservatism as defined by the right's ideological enforcers got
thrashed by a real-estate mogul who promised big, beautiful health care
and infrastructure and a whole bunch of things that it turned out
Republican voters favored even if their party's activists did not. So if
running the Tea Party play again reflects cynicism, then it's a highly
motivated cynicism --- with the motivation being the palpable desire of
most Republican senators to look back on the Trump experience and recite
the Draper catechism: \emph{This never happened}.

Most, but not all: There is also that group my staffer friend mentioned,
the senators who accept that Trumpism really happened, and who envision
a different party on the other side.

You can identify the members of this group both by their willingness to
spend money in the current crisis and by their interest in how it might
be spent. That means Marco Rubio
\href{https://www.rubio.senate.gov/public/index.cfm/2020/7/rubio-collins-introduce-phase-iv-small-business-relief-package}{spearheading}
the small business relief bill. It means Josh Hawley
\href{https://www.hawley.senate.gov/sites/default/files/2020-04/Getting-America-Back-to-Work_0.pdf}{pushing}
for the federal government to pre-empt layoffs by paying a chunk of
worker salaries. It means Tom Cotton
\href{https://www.washingtonpost.com/politics/2020/07/22/daily-202-cruz-vs-cotton-clash-coronavirus-deficits-may-preview-2024-contest-post-trump-gop/}{defending}
crisis spending against Cruz's attack. It means Mitt Romney
\href{https://www.romney.senate.gov/romney-cassidy-daines-rubio-call-family-focused-economic-impact-payments-coronavirus-relief}{leading}
a push to put more of the federal stimulus payments in the hands of
families with kids.

Notably, all of these figures have had differing approaches to Trump the
man: Romney famously in opposition, Cotton and Hawley fully on-side,
Rubio somewhere in between. And the same diversity shows up among the
born-again deficit hawks, a group that includes not just reliable Trump
allies but also the 2016 Never Trumper
\href{https://www.sasse.senate.gov/public/index.cfm/2020/7/sasse-statement-on-intra-democratic-mnuchin-pelosi-negotiations}{Ben
Sasse}.

So Republican divisions over Trump himself are somewhat different from
Republican divisions over what to learn from Trumpism. A figure like
Romney is anti-Trump, but he might be friendlier to post-Trump populism,
while Cruz and Paul have ended up pro-Trump but will probably revert to
their libertarian roots once he's gone.

Or, I should say, if he ever goes. Because the trouble with both the
Draper method and the ``this happened, let's learn from it'' approaches
to the Trump experience is that they assume not only that Trump will
lose (a strong bet but of course not a certain one) but also that in
defeat he will recede sufficiently to be willfully forgotten, or allow a
more robust nationalism to supplant his ersatz, personalized version.

Will he? I don't know. No politician's mystique is permanent; maybe a
sweeping defeat will really be the end of Trump's. But nobody should be
surprised if the desires that are so palpable among Republican senators
right now --- both the yearning for a simple return to the status quo
ante and the hope for a better, smarter populism --- will have to
contend, across a Biden presidency, with an alternative embodied either
by a scion or by the man himself: the dream of a Trump Restoration.

\emph{The Times is committed to publishing}
\href{https://www.nytimes.com/2019/01/31/opinion/letters/letters-to-editor-new-york-times-women.html}{\emph{a
diversity of letters}} \emph{to the editor. We'd like to hear what you
think about this or any of our articles. Here are some}
\href{https://help.nytimes.com/hc/en-us/articles/115014925288-How-to-submit-a-letter-to-the-editor}{\emph{tips}}\emph{.
And here's our email:}
\href{mailto:letters@nytimes.com}{\emph{letters@nytimes.com}}\emph{.}

\emph{Follow The New York Times Opinion section on}
\href{https://www.facebook.com/nytopinion}{\emph{Facebook}}\emph{,}
\href{http://twitter.com/NYTOpinion}{\emph{Twitter (@NYTOpinion)}}
\emph{and}
\href{https://www.instagram.com/nytopinion/}{\emph{Instagram}}\emph{,
join the Facebook political discussion group,}
\href{https://www.facebook.com/groups/votingwhilefemale/}{\emph{Voting
While Female}}\emph{.}

Advertisement

\protect\hyperlink{after-bottom}{Continue reading the main story}

\hypertarget{site-index}{%
\subsection{Site Index}\label{site-index}}

\hypertarget{site-information-navigation}{%
\subsection{Site Information
Navigation}\label{site-information-navigation}}

\begin{itemize}
\tightlist
\item
  \href{https://help.nytimes.com/hc/en-us/articles/115014792127-Copyright-notice}{©~2020~The
  New York Times Company}
\end{itemize}

\begin{itemize}
\tightlist
\item
  \href{https://www.nytco.com/}{NYTCo}
\item
  \href{https://help.nytimes.com/hc/en-us/articles/115015385887-Contact-Us}{Contact
  Us}
\item
  \href{https://www.nytco.com/careers/}{Work with us}
\item
  \href{https://nytmediakit.com/}{Advertise}
\item
  \href{http://www.tbrandstudio.com/}{T Brand Studio}
\item
  \href{https://www.nytimes.com/privacy/cookie-policy\#how-do-i-manage-trackers}{Your
  Ad Choices}
\item
  \href{https://www.nytimes.com/privacy}{Privacy}
\item
  \href{https://help.nytimes.com/hc/en-us/articles/115014893428-Terms-of-service}{Terms
  of Service}
\item
  \href{https://help.nytimes.com/hc/en-us/articles/115014893968-Terms-of-sale}{Terms
  of Sale}
\item
  \href{https://spiderbites.nytimes.com}{Site Map}
\item
  \href{https://help.nytimes.com/hc/en-us}{Help}
\item
  \href{https://www.nytimes.com/subscription?campaignId=37WXW}{Subscriptions}
\end{itemize}
