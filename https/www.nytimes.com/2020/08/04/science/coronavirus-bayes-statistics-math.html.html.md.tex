Sections

SEARCH

\protect\hyperlink{site-content}{Skip to
content}\protect\hyperlink{site-index}{Skip to site index}

\href{https://www.nytimes.com/section/science}{Science}

\href{https://myaccount.nytimes.com/auth/login?response_type=cookie\&client_id=vi}{}

\href{https://www.nytimes.com/section/todayspaper}{Today's Paper}

\href{/section/science}{Science}\textbar{}How to Think Like an
Epidemiologist

\url{https://nyti.ms/31hdb5h}

\begin{itemize}
\item
\item
\item
\item
\item
\item
\end{itemize}

\href{https://www.nytimes.com/news-event/coronavirus?action=click\&pgtype=Article\&state=default\&region=TOP_BANNER\&context=storylines_menu}{The
Coronavirus Outbreak}

\begin{itemize}
\tightlist
\item
  live\href{https://www.nytimes.com/2020/08/04/world/coronavirus-cases.html?action=click\&pgtype=Article\&state=default\&region=TOP_BANNER\&context=storylines_menu}{Latest
  Updates}
\item
  \href{https://www.nytimes.com/interactive/2020/us/coronavirus-us-cases.html?action=click\&pgtype=Article\&state=default\&region=TOP_BANNER\&context=storylines_menu}{Maps
  and Cases}
\item
  \href{https://www.nytimes.com/interactive/2020/science/coronavirus-vaccine-tracker.html?action=click\&pgtype=Article\&state=default\&region=TOP_BANNER\&context=storylines_menu}{Vaccine
  Tracker}
\item
  \href{https://www.nytimes.com/2020/08/02/us/covid-college-reopening.html?action=click\&pgtype=Article\&state=default\&region=TOP_BANNER\&context=storylines_menu}{College
  Reopening}
\item
  \href{https://www.nytimes.com/live/2020/08/04/business/stock-market-today-coronavirus?action=click\&pgtype=Article\&state=default\&region=TOP_BANNER\&context=storylines_menu}{Economy}
\end{itemize}

Advertisement

\protect\hyperlink{after-top}{Continue reading the main story}

Supported by

\protect\hyperlink{after-sponsor}{Continue reading the main story}

\hypertarget{how-to-think-like-an-epidemiologist}{%
\section{How to Think Like an
Epidemiologist}\label{how-to-think-like-an-epidemiologist}}

Don't worry, a little Bayesian analysis won't hurt you.

\includegraphics{https://static01.nyt.com/images/2020/08/04/science/04BAYES-illo/04BAYES-illo-articleLarge.jpg?quality=75\&auto=webp\&disable=upscale}

By Siobhan Roberts

\begin{itemize}
\item
  Aug. 4, 2020
\item
  \begin{itemize}
  \item
  \item
  \item
  \item
  \item
  \item
  \end{itemize}
\end{itemize}

There is a statistician's rejoinder --- sometimes offered as wry
criticism, sometimes as honest advice --- that could hardly be a better
motto for our times: ``Update your priors!''

In stats lingo, ``priors'' are your prior knowledge and beliefs,
inevitably fuzzy and uncertain, before seeing evidence. Evidence prompts
an updating; and then more evidence prompts further updating, so forth
and so on. This iterative process hones greater certainty and generates
a coherent accumulation of knowledge.

In the early pandemic era, for instance, airborne transmission of
Covid-19 was not considered likely, but in early July the World Health
Organization, with mounting scientific evidence,
\href{https://www.nytimes.com/2020/07/30/opinion/coronavirus-aerosols.html?surface=most-popular\&fellback=false\&req_id=877412152\&algo=top_conversion\&imp_id=115726583\&action=click\&module=Most\%20Popular\&pgtype=Homepage}{conceded}
that it is a factor, especially indoors. The W.H.O. updated its priors,
and changed its advice.

This is the heart of Bayesian analysis, named after Thomas Bayes, an
18th-century Presbyterian minister who did math on the side. It captures
uncertainty in terms of probability: Bayes's theorem, or rule, is a
device for rationally updating your prior beliefs and uncertainties
based on observed evidence.

Reverend Bayes set out his ideas in ``An Essay Toward Solving a Problem
in the Doctrine of Chances,'' published posthumously in 1763; it was
refined by the preacher and mathematician Richard Price and included
\href{https://www.nytimes.com/2011/08/07/books/review/the-theory-that-would-not-die-by-sharon-bertsch-mcgrayne-book-review.html?searchResultPosition=7}{Bayes's
theorem}. A couple of centuries later, Bayesian frameworks and methods,
powered by computation, are at the heart of various models in
epidemiology and other scientific fields

As Marc Lipsitch, an infectious disease epidemiologist at Harvard,
\href{https://twitter.com/mlipsitch/status/1257858402186940421}{noted on
Twitter}, Bayesian reasoning comes awfully close to his working
definition of rationality. ``As we learn more, our
\href{https://twitter.com/CT_Bergstrom/status/1276742731948158976}{beliefs}
should change,'' Dr. Lipsitch said in an interview. ``One extreme is to
decide what you think and be impervious to new information. Another
extreme is to over-privilege the last thing you learned. In rough terms,
Bayesian reasoning is a principled way to integrate what you previously
thought with what you have learned and come to a conclusion that
incorporates them both, giving them appropriate weights.''

With a new disease like Covid-19 and all the uncertainties it brings,
there is intense interest in nailing down the parameters for models:
What is the basic reproduction number, the rate at which new cases
arise? How deadly is it? What is the infection fatality rate, the
proportion of people with the virus that it kills?

But there is little point in trying to establish fixed numbers, said
Natalie Dean, an assistant professor of biostatistics at the University
of Florida.

``We should be less focused on finding the single `truth' and more
focused on establishing a reasonable range, recognizing that the true
value may vary across populations,'' Dr. Dean said. ``Bayesian analyses
allow us to include this variability in a clear way, and then propagate
this uncertainty through the model.''

A textbook application of Bayes's theorem is serology testing for
Covid-19, which looks for the presence of antibodies to the virus. All
tests are imperfect, and the accuracy of an antibody test turns on many
factors including,
\href{https://www.scientificamerican.com/article/coronavirus-antibody-tests-have-a-mathematical-pitfall/}{critically},
the rarity or prevalence of the disease.

\hypertarget{latest-updates-global-coronavirus-outbreak}{%
\section{\texorpdfstring{\href{https://www.nytimes.com/2020/08/04/world/coronavirus-cases.html?action=click\&pgtype=Article\&state=default\&region=MAIN_CONTENT_1\&context=storylines_live_updates}{Latest
Updates: Global Coronavirus
Outbreak}}{Latest Updates: Global Coronavirus Outbreak}}\label{latest-updates-global-coronavirus-outbreak}}

Updated 2020-08-05T07:58:24.076Z

\begin{itemize}
\tightlist
\item
  \href{https://www.nytimes.com/2020/08/04/world/coronavirus-cases.html?action=click\&pgtype=Article\&state=default\&region=MAIN_CONTENT_1\&context=storylines_live_updates\#link-762df92}{As
  talks drag on, McConnell signals openness to jobless aid extension,
  and negotiators agree on a deadline.}
\item
  \href{https://www.nytimes.com/2020/08/04/world/coronavirus-cases.html?action=click\&pgtype=Article\&state=default\&region=MAIN_CONTENT_1\&context=storylines_live_updates\#link-1228a480}{Novavax
  sees encouraging results from two studies of its experimental
  vaccine.}
\item
  \href{https://www.nytimes.com/2020/08/04/world/coronavirus-cases.html?action=click\&pgtype=Article\&state=default\&region=MAIN_CONTENT_1\&context=storylines_live_updates\#link-794484ed}{Mississippians
  must now wear masks in public, governor says.}
\end{itemize}

\href{https://www.nytimes.com/2020/08/04/world/coronavirus-cases.html?action=click\&pgtype=Article\&state=default\&region=MAIN_CONTENT_1\&context=storylines_live_updates}{See
more updates}

More live coverage:
\href{https://www.nytimes.com/live/2020/08/04/business/stock-market-today-coronavirus?action=click\&pgtype=Article\&state=default\&region=MAIN_CONTENT_1\&context=storylines_live_updates}{Markets}

The first SARS-CoV-2 antibody test approved by the F.D.A., in April,
seemed to be wrong as often as it was right. With Bayes's theorem, you
can calculate what you really want to know: the probability that the
test result is correct. As one
\href{https://twitter.com/Riderius/status/1246172832071135236}{commenter}
on Twitter put it: ``Understanding Bayes's theorem is a matter of life
and death right now.''

\hypertarget{the-logic-of-uncertainty}{%
\subsection{The logic of uncertainty}\label{the-logic-of-uncertainty}}

Joseph Blitzstein, a statistician at Harvard, delves into the utility of
Bayesian analysis in his popular course
``\href{https://www.youtube.com/playlist?list=PL2SOU6wwxB0uwwH80KTQ6ht66KWxbzTIo}{Statistics
110: Probability}.'' For a primer, in lecture one, he says: ``Math is
the logic of certainty, and statistics is the logic of uncertainty.
Everyone has uncertainty. If you have 100 percent certainty about
everything, there is something wrong with you.''

By the end of lecture four, he arrives at Bayes's theorem --- his
favorite theorem because it is mathematically simple yet conceptually
powerful.

``Literally, the proof is just one line of algebra,'' Dr. Blitzstein
said. The theorem essentially reduces to a fraction; it expresses the
probability P of some event A happening given the occurrence of another
event B.

\includegraphics{https://static01.nyt.com/images/2020/08/04/science/04SCI-BAYES-equation1/04SCI-BAYES-equation1-articleLarge-v3.jpg?quality=75\&auto=webp\&disable=upscale}

``Naïvely, you would think, How much could you get from that?'' Dr.
Blitzstein said. ``It turns out to have incredibly deep consequences and
to be applicable to just about every field of inquiry'' --- from finance
and genetics to political science and historical studies. The Bayesian
approach is applied in
\href{https://www.nature.com/articles/s41562-020-0858-1}{analyzing
racial disparities in policing} (in the assessment of officer decisions
to search drivers during a traffic stop) and
\href{https://www.nytimes.com/2014/09/30/science/the-odds-continually-updated.html}{search-and-rescue
operations} (the search area narrows as new data is added). Cognitive
scientists ask, `Is the brain Bayesian?' Philosophers of science posit
that science as a whole is a Bayesian process --- as is common sense.

Take diagnostic testing. In this scenario, the setup of Bayes's theorem
might use events labeled ``T'' for a positive test result --- and ``C''
for the presence of Covid-19 antibodies:

Image

Now suppose the prevalence of cases is 10 percent (that was so in
\href{https://dash.harvard.edu/bitstream/handle/1/42665370/Kissler_etal_NYC_mobility.pdf?sequence=1\&isAllowed=y}{New
York City in the spring}), and you have a positive result from a test
with accuracy of 87.5 percent sensitivity and 97.5 percent specificity.
Running numbers through the Bayesian gears, the probability that the
result is correct, and that you do indeed have antibodies is 79.5\%.
Decent odds,
\href{https://www.nytimes.com/2020/07/26/health/coronvirus-antibody-tests.html?searchResultPosition=1}{all
things considered}. If you want more certainty, get a second opinion.
And continue to be cautious.

An international \href{https://arxiv.org/abs/2007.13847}{collaboration}
of researchers, doctors and developers created another Bayesian
strategy, pairing the test result with a
\href{http://homecovidtests.org}{questionnaire} to produce a better
estimate of whether the result might be a false negative or a false
positive. The tool, which has won two hackathons, collects contextual
information: Did you go to work during lockdown? What did you do to
avoid catching Covid-19? Has anyone in your household had Covid-19?

``It's a little akin to having two `medical experts,''' said Claire
Donnat, who recently finished her Ph.D. in statistics at Stanford and
was part of the team. One expert has access to the patient's symptoms
and background, the other to the test; the two diagnoses are combined to
produce a more precise score, and more reliable immunity estimates. The
priors are updated with an aggregation of information.

``As new information comes in, we update our priors all the time,'' said
Susan Holmes, a Stanford statistician, via unstable internet from rural
Portugal, where she unexpectedly pandemicked for 105 days, while
visiting her mother.

\href{https://www.nytimes.com/news-event/coronavirus?action=click\&pgtype=Article\&state=default\&region=MAIN_CONTENT_3\&context=storylines_faq}{}

\hypertarget{the-coronavirus-outbreak-}{%
\subsubsection{The Coronavirus Outbreak
›}\label{the-coronavirus-outbreak-}}

\hypertarget{frequently-asked-questions}{%
\paragraph{Frequently Asked
Questions}\label{frequently-asked-questions}}

Updated August 4, 2020

\begin{itemize}
\item ~
  \hypertarget{i-have-antibodies-am-i-now-immune}{%
  \paragraph{I have antibodies. Am I now
  immune?}\label{i-have-antibodies-am-i-now-immune}}

  \begin{itemize}
  \tightlist
  \item
    As of right
    now,\href{https://www.nytimes.com/2020/07/22/health/covid-antibodies-herd-immunity.html?action=click\&pgtype=Article\&state=default\&region=MAIN_CONTENT_3\&context=storylines_faq}{that
    seems likely, for at least several months.} There have been
    frightening accounts of people suffering what seems to be a second
    bout of Covid-19. But experts say these patients may have a
    drawn-out course of infection, with the virus taking a slow toll
    weeks to months after initial exposure. People infected with the
    coronavirus typically
    \href{https://www.nature.com/articles/s41586-020-2456-9}{produce}
    immune molecules called antibodies, which are
    \href{https://www.nytimes.com/2020/05/07/health/coronavirus-antibody-prevalence.html?action=click\&pgtype=Article\&state=default\&region=MAIN_CONTENT_3\&context=storylines_faq}{protective
    proteins made in response to an
    infection}\href{https://www.nytimes.com/2020/05/07/health/coronavirus-antibody-prevalence.html?action=click\&pgtype=Article\&state=default\&region=MAIN_CONTENT_3\&context=storylines_faq}{.
    These antibodies may} last in the body
    \href{https://www.nature.com/articles/s41591-020-0965-6}{only two to
    three months}, which may seem worrisome, but that's perfectly normal
    after an acute infection subsides, said Dr. Michael Mina, an
    immunologist at Harvard University. It may be possible to get the
    coronavirus again, but it's highly unlikely that it would be
    possible in a short window of time from initial infection or make
    people sicker the second time.
  \end{itemize}
\item ~
  \hypertarget{im-a-small-business-owner-can-i-get-relief}{%
  \paragraph{I'm a small-business owner. Can I get
  relief?}\label{im-a-small-business-owner-can-i-get-relief}}

  \begin{itemize}
  \tightlist
  \item
    The
    \href{https://www.nytimes.com/article/small-business-loans-stimulus-grants-freelancers-coronavirus.html?action=click\&pgtype=Article\&state=default\&region=MAIN_CONTENT_3\&context=storylines_faq}{stimulus
    bills enacted in March} offer help for the millions of American
    small businesses. Those eligible for aid are businesses and
    nonprofit organizations with fewer than 500 workers, including sole
    proprietorships, independent contractors and freelancers. Some
    larger companies in some industries are also eligible. The help
    being offered, which is being managed by the Small Business
    Administration, includes the Paycheck Protection Program and the
    Economic Injury Disaster Loan program. But lots of folks have
    \href{https://www.nytimes.com/interactive/2020/05/07/business/small-business-loans-coronavirus.html?action=click\&pgtype=Article\&state=default\&region=MAIN_CONTENT_3\&context=storylines_faq}{not
    yet seen payouts.} Even those who have received help are confused:
    The rules are draconian, and some are stuck sitting on
    \href{https://www.nytimes.com/2020/05/02/business/economy/loans-coronavirus-small-business.html?action=click\&pgtype=Article\&state=default\&region=MAIN_CONTENT_3\&context=storylines_faq}{money
    they don't know how to use.} Many small-business owners are getting
    less than they expected or
    \href{https://www.nytimes.com/2020/06/10/business/Small-business-loans-ppp.html?action=click\&pgtype=Article\&state=default\&region=MAIN_CONTENT_3\&context=storylines_faq}{not
    hearing anything at all.}
  \end{itemize}
\item ~
  \hypertarget{what-are-my-rights-if-i-am-worried-about-going-back-to-work}{%
  \paragraph{What are my rights if I am worried about going back to
  work?}\label{what-are-my-rights-if-i-am-worried-about-going-back-to-work}}

  \begin{itemize}
  \tightlist
  \item
    Employers have to provide
    \href{https://www.osha.gov/SLTC/covid-19/standards.html}{a safe
    workplace} with policies that protect everyone equally.
    \href{https://www.nytimes.com/article/coronavirus-money-unemployment.html?action=click\&pgtype=Article\&state=default\&region=MAIN_CONTENT_3\&context=storylines_faq}{And
    if one of your co-workers tests positive for the coronavirus, the
    C.D.C.} has said that
    \href{https://www.cdc.gov/coronavirus/2019-ncov/community/guidance-business-response.html}{employers
    should tell their employees} -\/- without giving you the sick
    employee's name -\/- that they may have been exposed to the virus.
  \end{itemize}
\item ~
  \hypertarget{should-i-refinance-my-mortgage}{%
  \paragraph{Should I refinance my
  mortgage?}\label{should-i-refinance-my-mortgage}}

  \begin{itemize}
  \tightlist
  \item
    \href{https://www.nytimes.com/article/coronavirus-money-unemployment.html?action=click\&pgtype=Article\&state=default\&region=MAIN_CONTENT_3\&context=storylines_faq}{It
    could be a good idea,} because mortgage rates have
    \href{https://www.nytimes.com/2020/07/16/business/mortgage-rates-below-3-percent.html?action=click\&pgtype=Article\&state=default\&region=MAIN_CONTENT_3\&context=storylines_faq}{never
    been lower.} Refinancing requests have pushed mortgage applications
    to some of the highest levels since 2008, so be prepared to get in
    line. But defaults are also up, so if you're thinking about buying a
    home, be aware that some lenders have tightened their standards.
  \end{itemize}
\item ~
  \hypertarget{what-is-school-going-to-look-like-in-september}{%
  \paragraph{What is school going to look like in
  September?}\label{what-is-school-going-to-look-like-in-september}}

  \begin{itemize}
  \tightlist
  \item
    It is unlikely that many schools will return to a normal schedule
    this fall, requiring the grind of
    \href{https://www.nytimes.com/2020/06/05/us/coronavirus-education-lost-learning.html?action=click\&pgtype=Article\&state=default\&region=MAIN_CONTENT_3\&context=storylines_faq}{online
    learning},
    \href{https://www.nytimes.com/2020/05/29/us/coronavirus-child-care-centers.html?action=click\&pgtype=Article\&state=default\&region=MAIN_CONTENT_3\&context=storylines_faq}{makeshift
    child care} and
    \href{https://www.nytimes.com/2020/06/03/business/economy/coronavirus-working-women.html?action=click\&pgtype=Article\&state=default\&region=MAIN_CONTENT_3\&context=storylines_faq}{stunted
    workdays} to continue. California's two largest public school
    districts --- Los Angeles and San Diego --- said on July 13, that
    \href{https://www.nytimes.com/2020/07/13/us/lausd-san-diego-school-reopening.html?action=click\&pgtype=Article\&state=default\&region=MAIN_CONTENT_3\&context=storylines_faq}{instruction
    will be remote-only in the fall}, citing concerns that surging
    coronavirus infections in their areas pose too dire a risk for
    students and teachers. Together, the two districts enroll some
    825,000 students. They are the largest in the country so far to
    abandon plans for even a partial physical return to classrooms when
    they reopen in August. For other districts, the solution won't be an
    all-or-nothing approach.
    \href{https://bioethics.jhu.edu/research-and-outreach/projects/eschool-initiative/school-policy-tracker/}{Many
    systems}, including the nation's largest, New York City, are
    devising
    \href{https://www.nytimes.com/2020/06/26/us/coronavirus-schools-reopen-fall.html?action=click\&pgtype=Article\&state=default\&region=MAIN_CONTENT_3\&context=storylines_faq}{hybrid
    plans} that involve spending some days in classrooms and other days
    online. There's no national policy on this yet, so check with your
    municipal school system regularly to see what is happening in your
    community.
  \end{itemize}
\end{itemize}

That was the base from which Dr. Holmes refined a
\href{https://arxiv.org/abs/2004.05272}{preprint paper}, co-authored
with Dr. Donnat, that provides another example of Bayesian analysis,
broadly speaking. Observing early research in March about how the
pandemic might evolve, they noticed that classic epidemiological models
tend to use fixed parameters, or constants, for the reproduction number
--- for instance, with an R0 of 2.0.

But in reality, the reproduction number depends on random, uncertain
factors: viral loads and susceptibility, behavior and social networks,
culture and socioeconomic class, weather, air conditioning and unknowns.

With a Bayesian perspective, the uncertainty is encoded into randomness.
The researchers began by supposing that the reproductive number had
various distributions (the priors). Then they modeled the uncertainty
using a random variable that fluctuates, taking on a range of values as
small as 0.6 and as large as 2.2 or 3.5. In something of a nesting
process, the random variable itself has parameters that fluctuate
randomly; and those parameters, too, have random parameters
(hyper-parameters), etcetera. The effects accumulate into a ``Bayesian
hierarchy'' --- ``turtles all the way down,'' Dr. Holmes said.

The effects of all these up-and-down random fluctuations multiply, like
compound interest. As a result, the study found that using random
variables for reproductive numbers more realistically predicts the risky
tail events, the rarer but
\href{https://www.nytimes.com/2020/06/30/science/how-coronavirus-spreads.html}{more
significant superspreader events}.

Humans on their own, however, without a Bayesian model for a compass,
are \href{https://en.wikipedia.org/wiki/Prospect_theory}{notoriously
bad} at fathoming individual
\href{https://twitter.com/xkcdComic/status/1283437923421937666/photo/1}{risk}.

``People, including very young children, can and do use Bayesian
inference unconsciously,'' said Alison Gopnik, a psychologist at the
University of California, Berkeley. ``But they need direct evidence
about the frequency of events to do so.''

Much of the information that guides our behavior in the context of
Covid-19 is probabilistic. For example, by
\href{https://www.nature.com/articles/d41586-020-01738-2}{some
estimates}, if you get infected with the coronavirus, there is a 1
percent chance you will die; but in reality an individual's odds can
vary by a
\href{https://medium.com/wintoncentre/how-much-normal-risk-does-covid-represent-4539118e1196}{thousandfold}
or more, depending on age and other factors. ``For something like an
illness, most of the evidence is usually indirect, and people are very
bad at dealing with explicit probabilistic information,'' Dr. Gopnik
said.

\hypertarget{modeling-humility}{%
\subsection{Modeling humility}\label{modeling-humility}}

Even with evidence, revising beliefs isn't easy. The scientific
community struggled to update its priors about the asymptomatic
transmission of Covid-19, even when evidence emerged that it is a factor
and that masks are a helpful preventive measure. This arguably
contributed to the
\href{https://www.nytimes.com/2020/06/27/world/europe/coronavirus-spread-asymptomatic.html?action=click\&module=RelatedLinks\&pgtype=Article}{world's
sluggish response to the virus}.

``The problems come when we don't update,'' said David Spiegelhalter, a
statistician and chair of the Winton Centre for Risk and Evidence
Communication at the University of Cambridge. ``You can interpret
confirmation bias, and so many of the ways in which we react badly, by
being too slow to revise our beliefs.''

There are techniques that compensate for Bayesian shortcomings. Dr.
Spiegelhalter is fond of an approach called
\href{https://understandinguncertainty.org/node/97}{Cromwell's law}.
``It's heaven,'' he said. In 1650, Oliver Cromwell, Lord Protector of
the Commonwealth of England, wrote in a letter to the Church of
Scotland: ``I beseech you, in the bowels of Christ, think it possible
you may be mistaken.''

In the Bayesian world, Cromwell's law means you should always ``keep a
bit back --- with a little bit of probability, a little tiny bit --- for
the fact that you may be wrong,'' Dr. Spiegelhalter said. ``Then if new
evidence comes along that totally contradicts your main prior belief,
you can quickly ditch what you thought before and lurch over to that new
way of thinking.''

``In other words, keep an open mind,'' said Dr. Spiegelhalter. ``That's
a very powerful idea. And it doesn't necessarily have to be done
technically or formally; it can just be in the back of your mind as an
idea. Call it `modeling humility.' You may be wrong.''

\textbf{\emph{{[}}\href{http://on.fb.me/1paTQ1h}{\emph{Like the Science
Times page on Facebook.}}} ****** \emph{\textbar{} Sign up for the}
\textbf{\href{http://nyti.ms/1MbHaRU}{\emph{Science Times
newsletter.}}\emph{{]}}}

Advertisement

\protect\hyperlink{after-bottom}{Continue reading the main story}

\hypertarget{site-index}{%
\subsection{Site Index}\label{site-index}}

\hypertarget{site-information-navigation}{%
\subsection{Site Information
Navigation}\label{site-information-navigation}}

\begin{itemize}
\tightlist
\item
  \href{https://help.nytimes.com/hc/en-us/articles/115014792127-Copyright-notice}{©~2020~The
  New York Times Company}
\end{itemize}

\begin{itemize}
\tightlist
\item
  \href{https://www.nytco.com/}{NYTCo}
\item
  \href{https://help.nytimes.com/hc/en-us/articles/115015385887-Contact-Us}{Contact
  Us}
\item
  \href{https://www.nytco.com/careers/}{Work with us}
\item
  \href{https://nytmediakit.com/}{Advertise}
\item
  \href{http://www.tbrandstudio.com/}{T Brand Studio}
\item
  \href{https://www.nytimes.com/privacy/cookie-policy\#how-do-i-manage-trackers}{Your
  Ad Choices}
\item
  \href{https://www.nytimes.com/privacy}{Privacy}
\item
  \href{https://help.nytimes.com/hc/en-us/articles/115014893428-Terms-of-service}{Terms
  of Service}
\item
  \href{https://help.nytimes.com/hc/en-us/articles/115014893968-Terms-of-sale}{Terms
  of Sale}
\item
  \href{https://spiderbites.nytimes.com}{Site Map}
\item
  \href{https://help.nytimes.com/hc/en-us}{Help}
\item
  \href{https://www.nytimes.com/subscription?campaignId=37WXW}{Subscriptions}
\end{itemize}
