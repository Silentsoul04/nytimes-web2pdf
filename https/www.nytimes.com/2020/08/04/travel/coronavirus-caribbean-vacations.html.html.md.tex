Sections

SEARCH

\protect\hyperlink{site-content}{Skip to
content}\protect\hyperlink{site-index}{Skip to site index}

\href{https://www.nytimes.com/section/travel}{Travel}

\href{https://myaccount.nytimes.com/auth/login?response_type=cookie\&client_id=vi}{}

\href{https://www.nytimes.com/section/todayspaper}{Today's Paper}

\href{/section/travel}{Travel}\textbar{}The Caribbean Dilemma

\url{https://nyti.ms/2EGtssO}

\begin{itemize}
\item
\item
\item
\item
\item
\item
\end{itemize}

\href{https://www.nytimes.com/news-event/coronavirus?action=click\&pgtype=Article\&state=default\&region=TOP_BANNER\&context=storylines_menu}{The
Coronavirus Outbreak}

\begin{itemize}
\tightlist
\item
  live\href{https://www.nytimes.com/2020/08/04/world/coronavirus-cases.html?action=click\&pgtype=Article\&state=default\&region=TOP_BANNER\&context=storylines_menu}{Latest
  Updates}
\item
  \href{https://www.nytimes.com/interactive/2020/us/coronavirus-us-cases.html?action=click\&pgtype=Article\&state=default\&region=TOP_BANNER\&context=storylines_menu}{Maps
  and Cases}
\item
  \href{https://www.nytimes.com/interactive/2020/science/coronavirus-vaccine-tracker.html?action=click\&pgtype=Article\&state=default\&region=TOP_BANNER\&context=storylines_menu}{Vaccine
  Tracker}
\item
  \href{https://www.nytimes.com/2020/08/02/us/covid-college-reopening.html?action=click\&pgtype=Article\&state=default\&region=TOP_BANNER\&context=storylines_menu}{College
  Reopening}
\item
  \href{https://www.nytimes.com/live/2020/08/04/business/stock-market-today-coronavirus?action=click\&pgtype=Article\&state=default\&region=TOP_BANNER\&context=storylines_menu}{Economy}
\end{itemize}

Advertisement

\protect\hyperlink{after-top}{Continue reading the main story}

Supported by

\protect\hyperlink{after-sponsor}{Continue reading the main story}

\hypertarget{the-caribbean-dilemma}{%
\section{The Caribbean Dilemma}\label{the-caribbean-dilemma}}

Many islands are open to American travelers. Going could mean bringing
coronavirus to places ill prepared to deal with it. Not going could mean
deepening economic woes. How do you choose?

\includegraphics{https://static01.nyt.com/images/2020/08/04/travel/04caribbean-dilemma/merlin_174773829_3f39dfce-3c4f-4752-9d03-e8176083a50c-articleLarge.jpg?quality=75\&auto=webp\&disable=upscale}

By Nina Burleigh

\begin{itemize}
\item
  Aug. 4, 2020
\item
  \begin{itemize}
  \item
  \item
  \item
  \item
  \item
  \item
  \end{itemize}
\end{itemize}

Last year, more than 31 million people visited the Caribbean, more than
half of them from the United States. I was one of them. Together, we
contributed \$59 billion to the region's 2019 gross domestic product ---
accounting for a whopping 50 to 90 percent of the G.D.P. for most of the
countries, according to the International Monetary Fund.

I admit that in moments of pandemic weariness I have been one of those
people eyeing cheap tickets to the Caribbean, wondering when I might
feel ready to jump on a flight.

Now, though, our business comes with a mortal threat --- that for the
sake of a vacation we will bring the coronavirus to islands that are ill
prepared to handle a major outbreak. But staying home could be equally
ruinous. The Covid-19 lockdown --- and the severity of the epidemic in
the United States --- has been a disaster beyond any hurricane for the
Caribbean economy. The pandemic has closed airports and cruise ship
docks, shut down restaurants and dive shops and deprived the Caribbean
of tens of billions of dollars.

``To not have visitors arriving for any period of time, but particularly
for an extended period of time, has brought immense hardship to a number
of people throughout the Caribbean,'' said Hugh Riley, the former head
of the Caribbean Tourism Office, and a partner with Portfolio Marketing
Group, which represents some islands. ``Caribbean countries face an
important dilemma: Try to hermetically seal their borders from visitors
until there's an effective vaccine, or tackle the risks of restarting
tourism now. It is the classic risk/reward decision,'' he said.

As of August 3, 22 islands in the region have
\href{https://docs.google.com/document/d/1ytW37gjS3WeVhN-k4ZK-N5fwgrksljMIEsd6ToGSOB8/edit}{reopened
to tourism,} with 14 allowing visitors from the United States --- with
negative Covid-19 tests and, usually, periods of quarantine. It has not
always gone smoothly: The Bahamas allowed Americans to visit beginning
in July, slammed the door shut as coronavirus cases surged in that
nation, reopened and then shut down again, indicative of the efforts to
manage a moving crisis. Puerto Rico opened to Americans from the
mainland on July 15, but pushed that date back to August 15 after a
weekend of viral videos showing incoming visitors ignoring mask and
social distancing rules. On the other end of the spectrum, Barbados is
offering a 12-month visa to any American interested in moving a
work-from-home office to the island.

Tourists looking to escape to a coronavirus-free tropical island have a
responsibility to weigh the risks and take precautions.

So does the airline industry, says Allen Chastanet, the prime minister
of St. Lucia and a former airline executive nominated by CARICOM, the
20-nation Caribbean consortium, and the Organization of Eastern
Caribbean States to develop recommendations for reopening the region.
Mr. Chastanet has been urging the airlines to push for the development
and implementation of rapid preboarding airport testing for all
passengers.

``You have to have testing sites, the way you have a Dunkin' Donuts
kiosk in every airport,'' he said. ``The airlines in many ways acted
like they had ostrich syndrome, and said it is somebody else's problem,
but ultimately it is their problem. They have to use their advocacy
strength to make it happen.''

Tourism has always been a two-edged sword for the region. It brought
money for some, but also brought corruption, environmental degradation
and unchecked development. No tourist who steps outside an ``inclusive''
resort can fail to notice the incredible disparity of wealth on the
islands: palatial walled estates are often a stone's throw from cement
block shacks. Crime is such a problem on some Caribbean islands that
\href{https://www.tripsavvy.com/safest-and-most-dangerous-caribbean-islands-4157732}{websites}
are devoted to statistics to help worried travelers shop for the safest
destinations. (I can attest to this problem, having been burglarized in
Tobago and Vieques.) The BBC once
\href{http://www.bbc.co.uk/caribbean/news/story/2006/01/060103_murderlist.shtml}{called}
Jamaica ``the murder capital of the world,'' to howls of outrage from
the Jamaicans.

As Caribbean tourism exploded and got cheaper, local tour operators
raked in money, but faced unexpected problems. Tropical infrastructure,
local police and medical systems were overwhelmed on some islands even
before the virus. One island friend, a divemaster at a major site, who
asked that his name not be used for fear of losing his job, told me he
has seen increasingly obese, relatively unhealthy American tourists who
feel entitled to be squished into neoprene suits and taken to the depths
as cruise lines and cheap tours market scuba diving --- once reserved
for scientists, Navy SEALs and the ultrawealthy and sporty --- to all.

\hypertarget{latest-updates-global-coronavirus-outbreak}{%
\section{\texorpdfstring{\href{https://www.nytimes.com/2020/08/04/world/coronavirus-cases.html?action=click\&pgtype=Article\&state=default\&region=MAIN_CONTENT_1\&context=storylines_live_updates}{Latest
Updates: Global Coronavirus
Outbreak}}{Latest Updates: Global Coronavirus Outbreak}}\label{latest-updates-global-coronavirus-outbreak}}

Updated 2020-08-05T05:55:41.927Z

\begin{itemize}
\tightlist
\item
  \href{https://www.nytimes.com/2020/08/04/world/coronavirus-cases.html?action=click\&pgtype=Article\&state=default\&region=MAIN_CONTENT_1\&context=storylines_live_updates\#link-762df92}{As
  talks drag on, McConnell signals openness to jobless aid extension,
  and negotiators agree on a deadline.}
\item
  \href{https://www.nytimes.com/2020/08/04/world/coronavirus-cases.html?action=click\&pgtype=Article\&state=default\&region=MAIN_CONTENT_1\&context=storylines_live_updates\#link-1228a480}{Novavax
  sees encouraging results from two studies of its experimental
  vaccine.}
\item
  \href{https://www.nytimes.com/2020/08/04/world/coronavirus-cases.html?action=click\&pgtype=Article\&state=default\&region=MAIN_CONTENT_1\&context=storylines_live_updates\#link-794484ed}{Mississippians
  must now wear masks in public, governor says.}
\end{itemize}

\href{https://www.nytimes.com/2020/08/04/world/coronavirus-cases.html?action=click\&pgtype=Article\&state=default\&region=MAIN_CONTENT_1\&context=storylines_live_updates}{See
more updates}

More live coverage:
\href{https://www.nytimes.com/live/2020/08/04/business/stock-market-today-coronavirus?action=click\&pgtype=Article\&state=default\&region=MAIN_CONTENT_1\&context=storylines_live_updates}{Markets}

The Caribbean is the biggest source of business for the global cruise
industry, which is notoriously callous about the environment. Cruise
lines were the
\href{https://www.nytimes.com/2020/03/19/travel/coronavirus-cruise-costa-luminosa.html}{first
global heralds} of the coronavirus disaster and will likely be
\href{https://www.nytimes.com/2020/06/26/travel/coronavirus-cruises-reopening.html}{the
last travel industry} to come back once the virus is under control.

The cruise industry always had the upper hand on the islands. When a
cruise ship docks and thousands of people are disgorged, the impression
of prosperity is illusory. Most of the islands pay a per head fee to the
cruise lines for each passenger who disembarks, the cruise ships are
notoriously bad for reefs, and they have a stranglehold on the
discretionary dollars their passengers are spending.

``Everything that can be sold on board is already sold, and anyplace on
the island that could benefit has already made arrangements with the
cruise company,'' said Noel Mignott, a former deputy director of tourism
for Jamaica and a founding partner of Portfolio Marketing Group. ``If
one good thing could come of Covid, I would be encouraged to see
governments take this opportunity to renegotiate the relationship with
the cruise lines. And if I was a cruise line, I would wave that green
flag and try to be as good as I can to the environment --- if only to
say we are not dumping our garbage in the ocean two miles off Ocho
Rios.''

The Dutch island of Bonaire is one of the ports of call for behemoth and
often super-discounted cruise ships plying the Caribbean. In the last
few years, two building-size ships have daily disgorged up to 4,000
passengers at a time during the cruising season. The ships have
sometimes sparked food shortages by taking up dock space needed for
cargo.

Now, in the pandemic lull, tour providers, officials and some citizens
have been quietly discussing what to do about the ships when they
return. Facebook groups like
\href{https://www.facebook.com/groups/BonaireFutureForum/}{Bonaire
Future Forum: Opportunity From Crisis} are debating whether the island
should limit access to specific ships that cost more and are therefore
more selective in their choice of passenger.

The island has one of the most pristine reefs in the Caribbean, and
animal behavior has changed since the number of daily human divers
dropped from thousands to the single digits. Local divers are noticing
animals come closer, and the elusive seahorse has been a common sight
these last months.

The pandemic has already changed life by necessity. The Caribbean has a
``ridiculously high'' food import bill because of an assumption that
tourists don't want to eat local food, Mr. Riley said. The pandemic may
change that. ``We have been laboring under the misconception that
tourists want something other than what we have. We think people want
hamburgers and hot dogs. Now that we are consuming what we have, I think
this will lead to an increased variety in what we produce locally,'' he
said.

Sven Olof Lindblad, the chief executive of Lindblad Expeditions, which
offers high-end, small-ship, environmentally conscious cruises around
the world, sees the pandemic as a moment in which destinations can seize
control of the downside of overtourism and demand changes. ``This
clearly is a time to rethink --- but it won't be led by businesses who
are, by and large, too fat and happy with the way it is. Create working
groups to totally rethink the relationship of tourism focused on value
--- and not just financial value.''

\hypertarget{selling-sun-sand-and-sea}{%
\subsection{Selling ``sun, sand and
sea''}\label{selling-sun-sand-and-sea}}

Stepping out of an aluminum tube in the dead of winter and into a
blanket of tropical humidity is, in my view, one of life's singular
pleasures. And I've endured many a discount middle seat to get some
``last-minute'' sun and sand in the Caribbean.

But these jaunts have sometimes come with a measure of self-loathing.
Quaffing wintertime margaritas poolside at an inclusive Jamaican resort
next to my fellow pasty North Americans while our sunburned kids went
sugar-mad refilling plastic cups at a Willy Wonka-style eternal soda
fountain is not a look I'm proud of.

More, I can never fully repress the awareness that these trips are not
ecologically friendly. Even before flight-shaming, the rampant
construction of resorts, the ribbons of new roads and the abomination of
air conditioning all struck me as a blight on the natural beauty of the
islands.

\href{https://www.nytimes.com/news-event/coronavirus?action=click\&pgtype=Article\&state=default\&region=MAIN_CONTENT_3\&context=storylines_faq}{}

\hypertarget{the-coronavirus-outbreak-}{%
\subsubsection{The Coronavirus Outbreak
›}\label{the-coronavirus-outbreak-}}

\hypertarget{frequently-asked-questions}{%
\paragraph{Frequently Asked
Questions}\label{frequently-asked-questions}}

Updated August 4, 2020

\begin{itemize}
\item ~
  \hypertarget{i-have-antibodies-am-i-now-immune}{%
  \paragraph{I have antibodies. Am I now
  immune?}\label{i-have-antibodies-am-i-now-immune}}

  \begin{itemize}
  \tightlist
  \item
    As of right
    now,\href{https://www.nytimes.com/2020/07/22/health/covid-antibodies-herd-immunity.html?action=click\&pgtype=Article\&state=default\&region=MAIN_CONTENT_3\&context=storylines_faq}{that
    seems likely, for at least several months.} There have been
    frightening accounts of people suffering what seems to be a second
    bout of Covid-19. But experts say these patients may have a
    drawn-out course of infection, with the virus taking a slow toll
    weeks to months after initial exposure. People infected with the
    coronavirus typically
    \href{https://www.nature.com/articles/s41586-020-2456-9}{produce}
    immune molecules called antibodies, which are
    \href{https://www.nytimes.com/2020/05/07/health/coronavirus-antibody-prevalence.html?action=click\&pgtype=Article\&state=default\&region=MAIN_CONTENT_3\&context=storylines_faq}{protective
    proteins made in response to an
    infection}\href{https://www.nytimes.com/2020/05/07/health/coronavirus-antibody-prevalence.html?action=click\&pgtype=Article\&state=default\&region=MAIN_CONTENT_3\&context=storylines_faq}{.
    These antibodies may} last in the body
    \href{https://www.nature.com/articles/s41591-020-0965-6}{only two to
    three months}, which may seem worrisome, but that's perfectly normal
    after an acute infection subsides, said Dr. Michael Mina, an
    immunologist at Harvard University. It may be possible to get the
    coronavirus again, but it's highly unlikely that it would be
    possible in a short window of time from initial infection or make
    people sicker the second time.
  \end{itemize}
\item ~
  \hypertarget{im-a-small-business-owner-can-i-get-relief}{%
  \paragraph{I'm a small-business owner. Can I get
  relief?}\label{im-a-small-business-owner-can-i-get-relief}}

  \begin{itemize}
  \tightlist
  \item
    The
    \href{https://www.nytimes.com/article/small-business-loans-stimulus-grants-freelancers-coronavirus.html?action=click\&pgtype=Article\&state=default\&region=MAIN_CONTENT_3\&context=storylines_faq}{stimulus
    bills enacted in March} offer help for the millions of American
    small businesses. Those eligible for aid are businesses and
    nonprofit organizations with fewer than 500 workers, including sole
    proprietorships, independent contractors and freelancers. Some
    larger companies in some industries are also eligible. The help
    being offered, which is being managed by the Small Business
    Administration, includes the Paycheck Protection Program and the
    Economic Injury Disaster Loan program. But lots of folks have
    \href{https://www.nytimes.com/interactive/2020/05/07/business/small-business-loans-coronavirus.html?action=click\&pgtype=Article\&state=default\&region=MAIN_CONTENT_3\&context=storylines_faq}{not
    yet seen payouts.} Even those who have received help are confused:
    The rules are draconian, and some are stuck sitting on
    \href{https://www.nytimes.com/2020/05/02/business/economy/loans-coronavirus-small-business.html?action=click\&pgtype=Article\&state=default\&region=MAIN_CONTENT_3\&context=storylines_faq}{money
    they don't know how to use.} Many small-business owners are getting
    less than they expected or
    \href{https://www.nytimes.com/2020/06/10/business/Small-business-loans-ppp.html?action=click\&pgtype=Article\&state=default\&region=MAIN_CONTENT_3\&context=storylines_faq}{not
    hearing anything at all.}
  \end{itemize}
\item ~
  \hypertarget{what-are-my-rights-if-i-am-worried-about-going-back-to-work}{%
  \paragraph{What are my rights if I am worried about going back to
  work?}\label{what-are-my-rights-if-i-am-worried-about-going-back-to-work}}

  \begin{itemize}
  \tightlist
  \item
    Employers have to provide
    \href{https://www.osha.gov/SLTC/covid-19/standards.html}{a safe
    workplace} with policies that protect everyone equally.
    \href{https://www.nytimes.com/article/coronavirus-money-unemployment.html?action=click\&pgtype=Article\&state=default\&region=MAIN_CONTENT_3\&context=storylines_faq}{And
    if one of your co-workers tests positive for the coronavirus, the
    C.D.C.} has said that
    \href{https://www.cdc.gov/coronavirus/2019-ncov/community/guidance-business-response.html}{employers
    should tell their employees} -\/- without giving you the sick
    employee's name -\/- that they may have been exposed to the virus.
  \end{itemize}
\item ~
  \hypertarget{should-i-refinance-my-mortgage}{%
  \paragraph{Should I refinance my
  mortgage?}\label{should-i-refinance-my-mortgage}}

  \begin{itemize}
  \tightlist
  \item
    \href{https://www.nytimes.com/article/coronavirus-money-unemployment.html?action=click\&pgtype=Article\&state=default\&region=MAIN_CONTENT_3\&context=storylines_faq}{It
    could be a good idea,} because mortgage rates have
    \href{https://www.nytimes.com/2020/07/16/business/mortgage-rates-below-3-percent.html?action=click\&pgtype=Article\&state=default\&region=MAIN_CONTENT_3\&context=storylines_faq}{never
    been lower.} Refinancing requests have pushed mortgage applications
    to some of the highest levels since 2008, so be prepared to get in
    line. But defaults are also up, so if you're thinking about buying a
    home, be aware that some lenders have tightened their standards.
  \end{itemize}
\item ~
  \hypertarget{what-is-school-going-to-look-like-in-september}{%
  \paragraph{What is school going to look like in
  September?}\label{what-is-school-going-to-look-like-in-september}}

  \begin{itemize}
  \tightlist
  \item
    It is unlikely that many schools will return to a normal schedule
    this fall, requiring the grind of
    \href{https://www.nytimes.com/2020/06/05/us/coronavirus-education-lost-learning.html?action=click\&pgtype=Article\&state=default\&region=MAIN_CONTENT_3\&context=storylines_faq}{online
    learning},
    \href{https://www.nytimes.com/2020/05/29/us/coronavirus-child-care-centers.html?action=click\&pgtype=Article\&state=default\&region=MAIN_CONTENT_3\&context=storylines_faq}{makeshift
    child care} and
    \href{https://www.nytimes.com/2020/06/03/business/economy/coronavirus-working-women.html?action=click\&pgtype=Article\&state=default\&region=MAIN_CONTENT_3\&context=storylines_faq}{stunted
    workdays} to continue. California's two largest public school
    districts --- Los Angeles and San Diego --- said on July 13, that
    \href{https://www.nytimes.com/2020/07/13/us/lausd-san-diego-school-reopening.html?action=click\&pgtype=Article\&state=default\&region=MAIN_CONTENT_3\&context=storylines_faq}{instruction
    will be remote-only in the fall}, citing concerns that surging
    coronavirus infections in their areas pose too dire a risk for
    students and teachers. Together, the two districts enroll some
    825,000 students. They are the largest in the country so far to
    abandon plans for even a partial physical return to classrooms when
    they reopen in August. For other districts, the solution won't be an
    all-or-nothing approach.
    \href{https://bioethics.jhu.edu/research-and-outreach/projects/eschool-initiative/school-policy-tracker/}{Many
    systems}, including the nation's largest, New York City, are
    devising
    \href{https://www.nytimes.com/2020/06/26/us/coronavirus-schools-reopen-fall.html?action=click\&pgtype=Article\&state=default\&region=MAIN_CONTENT_3\&context=storylines_faq}{hybrid
    plans} that involve spending some days in classrooms and other days
    online. There's no national policy on this yet, so check with your
    municipal school system regularly to see what is happening in your
    community.
  \end{itemize}
\end{itemize}

Everyone I talked to about a post-Covid Caribbean mentioned one thing: a
hope that the pandemic might result in a different kind of tourist: a
\emph{traveler,} not necessarily richer in money, but more conscious,
more of an explorer and less of a sybarite.
\href{https://www.nytimes.com/2020/07/02/travel/venice-coronavirus-tourism.html}{It
is a hope shared} by many overtouristed spots around the globe, from
Venice to the beaches of southern Thailand. For the Caribbean, a long
history of being seen as a playground for visitors from the mainland
United States might make things harder.

The tourist industry itself trained Americans to think of the Caribbean
as ``sun, sand and sea,'' and to think of the diverse islands as
interchangeable, Mr. Mignott said. Other than the sea they share, the
islands are different, each with a unique geological and human history.
The older islands to the west, including Cuba, are formed of limestone
and billions of shells and skeletons of ancient marine life, while the
black cliffs and crags of the younger islands along the eastern edge ---
where the Caribbean and the Atlantic tectonic plates grind against each
other --- are relics of violent prehistoric volcanic events.

In my years exploring the Caribbean, I've visited Guadeloupe, Bonaire,
St. John, Vieques, Jamaica and Tobago, and met people who have in common
that they were born with the sound of the sea in their ears, but
otherwise possess unique traditions, history, language and culture, that
reward visitors with a little curiosity.

The Caribbean tourism industry could take this opportunity to
differentiate the islands, and maybe even put responsibility on
travelers to go beyond the resort walls or cruise ship all-inclusives
and explore local food and culture.

Can it happen? As airlines and cruise ships reduce capacity, and the
tourist industry consolidates, the islands need to act deliberately,
said Mr. Riley. ``Are we going to leave it to happenstance or are we
going to plan for more socially responsible tourism and put policies in
place that redress and undo damage to the environment?'' he asked.

The premier of the island of Nevis, Mark Brantley, said the pandemic has
taught the Caribbean that overreliance on tourism is not the best model
and that Covid-19 could mark the end of the era of cheap tourism and
mega cruises. ``Jurisdictions are going to pivot to more tourism pitched
at the luxury market, with smaller numbers of people and arguably a
better yield,'' he said. Additionally, he predicted that local
industries, especially agriculture and agri-processing, will become more
important sectors of the Caribbean economies. ``Countries will be trying
to diversify, where tourism continues to be important, but not the only
game in town anymore.''

Mr. Chastanet said that when the pandemic struck, St. Lucia was already
midway into a national program to promote what he called ``village
tourism,'' sprucing up hamlets with new infrastructure and training and
providing seed money for resort workers and hotel chefs to open up their
own small-scale, boutique operations. ``The things we were doing just
got reinforced by Covid,'' he said.

``We really hope if one good thing happens from the pandemic, it will be
that travel is more thoughtful, and travelers are more conscious about
the environment,'' said Mr. Mignott, the former deputy tourism director
for Jamaica. ``We don't think people are just going to go back like
Covid never happened. We really think it will be different.''

\hypertarget{a-different-kind-of-tourism}{%
\subsection{A different kind of
tourism}\label{a-different-kind-of-tourism}}

I will regret the end of cheap, mass Caribbean tourism, if it comes, but
I understood its downside long before the coronavirus. I have also been
another kind of island traveler --- a temporary resident. I spent most
of my seventh month of one pregnancy floating like a turtle in the sea
outside an old-time resort called Arnos Vale in Tobago, traditionally
known as a destination for birders. We couldn't afford to lodge there,
but we swam on the beach and spent time under the slow flapping porch
fan where a talking parrot held court.

A year later, we moved our family into a Tobago rental for six weeks. We
lived simply on peanut butter sandwiches, the daily fish catch and Betty
Crocker box cakes. Every day I rode my bike past a ruined pink
plantation and through a hilltop hamlet impossibly named ``Whim.''

In Whim, Tobagoans lived in simple wood shacks perched on cliffs
overlooking crashing surf, poor in money, but the owners of stupendous,
million-dollar views.

When we returned to the island a few years later we found newly paved
roads, traffic jams, and a new mood --- the hum and honk of progress
drowning out the hummingbirds and the cackle of the national bird, the
cocrico, at dawn. I know Whim is still on the map, but I wonder who owns
those little shacks.

\begin{center}\rule{0.5\linewidth}{\linethickness}\end{center}

\emph{\textbf{Follow New York Times Travel}}
\emph{on}\href{https://www.instagram.com/nytimestravel/}{\emph{Instagram}}\emph{,}\href{https://twitter.com/nytimestravel}{\emph{Twitter}}
\emph{and}\href{https://www.facebook.com/nytimestravel/}{\emph{Facebook}}\emph{.
And}\href{https://www.nytimes.com/newsletters/traveldispatch}{\emph{sign
up for our weekly Travel Dispatch newsletter}} \emph{to receive expert
tips on traveling smarter and inspiration for your next vacation.}

Advertisement

\protect\hyperlink{after-bottom}{Continue reading the main story}

\hypertarget{site-index}{%
\subsection{Site Index}\label{site-index}}

\hypertarget{site-information-navigation}{%
\subsection{Site Information
Navigation}\label{site-information-navigation}}

\begin{itemize}
\tightlist
\item
  \href{https://help.nytimes.com/hc/en-us/articles/115014792127-Copyright-notice}{©~2020~The
  New York Times Company}
\end{itemize}

\begin{itemize}
\tightlist
\item
  \href{https://www.nytco.com/}{NYTCo}
\item
  \href{https://help.nytimes.com/hc/en-us/articles/115015385887-Contact-Us}{Contact
  Us}
\item
  \href{https://www.nytco.com/careers/}{Work with us}
\item
  \href{https://nytmediakit.com/}{Advertise}
\item
  \href{http://www.tbrandstudio.com/}{T Brand Studio}
\item
  \href{https://www.nytimes.com/privacy/cookie-policy\#how-do-i-manage-trackers}{Your
  Ad Choices}
\item
  \href{https://www.nytimes.com/privacy}{Privacy}
\item
  \href{https://help.nytimes.com/hc/en-us/articles/115014893428-Terms-of-service}{Terms
  of Service}
\item
  \href{https://help.nytimes.com/hc/en-us/articles/115014893968-Terms-of-sale}{Terms
  of Sale}
\item
  \href{https://spiderbites.nytimes.com}{Site Map}
\item
  \href{https://help.nytimes.com/hc/en-us}{Help}
\item
  \href{https://www.nytimes.com/subscription?campaignId=37WXW}{Subscriptions}
\end{itemize}
