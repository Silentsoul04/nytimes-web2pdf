Sections

SEARCH

\protect\hyperlink{site-content}{Skip to
content}\protect\hyperlink{site-index}{Skip to site index}

\href{https://myaccount.nytimes.com/auth/login?response_type=cookie\&client_id=vi}{}

\href{https://www.nytimes.com/section/todayspaper}{Today's Paper}

\href{/section/upshot}{The Upshot}\textbar{}How Ballot Initiatives
Changed the Game on Medicaid Expansion

\url{https://nyti.ms/2XkQ3BR}

\begin{itemize}
\item
\item
\item
\item
\item
\item
\end{itemize}

Advertisement

\protect\hyperlink{after-top}{Continue reading the main story}

Upshot

Supported by

\protect\hyperlink{after-sponsor}{Continue reading the main story}

\hypertarget{how-ballot-initiatives-changed-the-game-on-medicaid-expansion}{%
\section{How Ballot Initiatives Changed the Game on Medicaid
Expansion}\label{how-ballot-initiatives-changed-the-game-on-medicaid-expansion}}

Missouri is the latest state where a nonprofit has helped put the issue
before voters, bypassing Republican officials. And the vote is today.

By \href{https://www.nytimes.com/by/sarah-kliff}{Sarah Kliff}

\begin{itemize}
\item
  Aug. 4, 2020, 5:00 a.m. ET
\item
  \begin{itemize}
  \item
  \item
  \item
  \item
  \item
  \item
  \end{itemize}
\end{itemize}

\includegraphics{https://static01.nyt.com/images/2020/08/04/upshot/04up-medicaid-ballots/04up-medicaid-ballots-articleLarge.jpg?quality=75\&auto=webp\&disable=upscale}

It was the middle of 2016, and Obamacare supporters were stuck.

Nineteen states were refusing to participate in the health law's
Medicaid expansion, which provides health coverage to low-income
Americans. States run by Democrats eagerly signed up for the program,
lured in part by generous federal funding.

Most Republican governors and legislatures had little interest in
expanding the reach of the Affordable Care Act, and declined the money.

``People were frustrated,'' said Chris Jennings, a longtime health care
strategist who served in the Clinton and Obama administrations. ``We
were left with either doing nothing or finding a new solution. And then
these guys came up with this referendum strategy.''

``These guys'' are the Fairness Project, a nonprofit created by a
California health workers union. Their strategy: ask voters to expand
Medicaid with state ballot initiatives.

In a few years, the Fairness Project's ballot campaigns have gone from
an untested tactic to the main approach for expanding the Affordable
Care Act's reach. Five states have expanded Medicaid through ballot
initiatives since President Trump's inauguration. A sixth, Virginia, did
so after Democrats gained control of the state legislature.

These efforts have extended Medicaid eligibility to nearly one million
low-income Americans living in states where governors or legislatures
have opposed the program. ``Some of us were a little skeptical at
first,'' said Mr. Jennings, who has since become an informal adviser to
the Fairness Project. ``We thought there would be extraordinary
resources waged against them. But they're taking on hard issues in hard
states, and they're a lot more successful than some of us can say.''

Missouri will vote on a Medicaid
\href{https://ballotpedia.org/Missouri_Amendment_2,_Medicaid_Expansion_Initiative_(August_2020)}{ballot
initiative} today. If passed, it would extend coverage to an estimated
217,000 people. Gov. Mike Parson, a Republican, opposes the ballot
initiative and
\href{https://www.kshb.com/news/local-news/missouri-medicaid-expansion-set-for-august-election}{has
argued} that it will harm a state budget that is already under strain
because of the pandemic.

The financial impact of Medicaid expansion is uncertain and could range
from \$200 million in extra costs to \$1 billion in additional annual
savings, according to an
\href{https://www.stltoday.com/news/local/state-and-regional/petition-seeks-to-put-medicaid-expansion-on-missouri-ballot/article_5cd960fe-be37-59a4-8391-f907f5f19352.html}{estimate}
prepared by the state's auditor, Nicole Galloway, a Democrat who is
running for governor this fall.

The Fairness Project grew out of a memo that a California union leader
wrote in 2014, warning that steep declines in union membership could
leave workers unprotected with fewer benefits.

``Unionism is in decline, and there is no end to that in sight,'' Dave
Regan, president of United Healthcare Workers West, said recently. His
group represents 95,000 hospital workers in California. ``But we still
need to give regular people the opportunity to have positive change in
their lives.''

In his memo, Mr. Regan proposed creating a nonprofit that would use the
ballot initiative process to secure policies that would benefit workers,
like increased access to health coverage and a higher minimum wage.

``Ballots are an opportunity to put a question, in its undiluted form,
in front of millions of people,'' he said. ``As opposed to traditional
legislative work, where things get watered down to get out of committee,
you end up with what you actually want when you use the ballot.''

Not all of his union members were enthusiastic about the project. Some
questioned why dues paid in California would be spent running campaigns
in the Midwest. But the initiative had enough support that the United
Healthcare Workers West executive board approved its funding, and has
continued to do so each year since.

The Fairness Project began in 2016, starting with ballot initiative
campaigns for increasing the minimum wage in California and Maine. The
next year, it came back to Maine to support the country's first Medicaid
expansion referendum.

The Maine legislature had already passed bills to expand Medicaid five
times, only to have each vetoed by Gov. Paul LePage.

``We kept falling a vote or two shy of overriding the governor's veto,''
said Robyn Merrill, executive director of Maine Equal Justice. ``It felt
like this was a huge problem that many wanted to fix, and that we had to
find a way to make it happen.''

Ms. Merrill's group quickly gathered enough signatures to secure a spot
on the 2017 ballot. The Fairness Project joined the campaign shortly
afterward, providing financial support for advertising and data about
which voters to target and how to reach them.

``The way we ensure that we win is by running these campaigns like
gubernatorial or Senate races,'' said Jonathan Schleifer, a former
congressional staffer who now leads the Fairness Project. ``They have
modeling, they have research, they have a diverse coalition; we have
that, too.''

The Maine campaign
\href{https://www.nytimes.com/elections/results/maine-ballot-measure-medicaid-expansion}{succeeded},
with 59 percent of voters supporting Medicaid expansion. That caught the
attention of supporters in other states, who were similarly struggling
to enact the program.

``I got in touch with the Fairness Project and basically asked: What
would it take to have you come to Nebraska,'' said State Senator Adam
Morfeld, who had spent years introducing legislation to create the
program. ``They immediately did a poll in December 2017, and saw there
was a path to victory.''

In 2018, the Fairness Project ran successful Medicaid expansion
campaigns in Nebraska, Utah and Idaho. A fourth ballot campaign, to
continue funding Montana's already-existing Medicaid expansion, failed,
but the state legislature ultimately stepped in to pay for the program.

The Fairness Project does not disclose a list of its donors, and
declined to provide one to The New York Times (Mr. Schleifer did
identify United Healthcare Workers West as the group's ``most
significant'' supporter). This lack of transparency has led to
\href{https://www.idahostatesman.com/news/politics-government/state-politics/article228258029.html}{some
criticism} of the group because it makes it harder for voters to know
who is supporting and organizing the ballot measures.

The ballot initiative method has its drawbacks. After the initiatives
pass, governors often delay, alter or outright refuse implementation.
Some have even
\href{https://www.idahostatesman.com/news/politics-government/state-politics/article227828559.html}{changed
the ballot initiative process}, making it harder to secure spots, after
seeing a Medicaid ballot succeed.

Governor LePage of Maine said he
``\href{https://apnews.com/b4ccacffb7e445c08f31c4fc444c2d85/LePage-says-he'd-risk-jail-before-Medicaid-puts-Maine-in-red}{would
go to jail}'' before expanding Medicaid. The program did not start
enrolling members until the state elected a new governor, Janet Mills, a
Democrat, in 2018.

Utah's government added a provision that Medicaid enrollees had to work,
volunteer or search for work to secure coverage, a restriction not in
the original ballot.

Nebraska began enrolling patients into its Medicaid expansion only this
month, nearly two years after the ballot passed. It also added a work
requirement, although both it and Utah have suspended those rules during
the pandemic.

``Even after the people pass it, they still fight it,'' Senator Morfeld
said.

Such post-ballot resistance prompted the Fairness Project to revise its
strategy for its 2020 Medicaid campaign in Oklahoma. The ballot asked
voters to approve a constitutional amendment, which could be altered
only by another statewide referendum.
\href{https://www.nytimes.com/2020/07/01/upshot/oklahoma-obamacare-Republican-voters-expand.html}{The
referendum passed} in June. The Fairness Project is employing the same
method in Missouri.

A constitutional ballot typically requires more work and stronger
support. In Oklahoma, for example, ballot organizers can pursue
statutory or constitutional initiatives. The constitutional initiatives
have more staying power, but also require gathering
\href{https://ballotpedia.org/Laws_governing_the_initiative_process_in_Oklahoma}{twice}
as many signatures.

Not all states allow ballot initiatives. Of the 12 remaining states that
have not expanded Medicaid (excluding Missouri, where voters will decide
today), only four have referendum processes: Florida, Mississippi, South
Dakota and Wyoming.

Florida, which has 2.7 million uninsured residents, is the Fairness
Project's next major focus.

``We've been working in Florida for about two years now,'' Mr. Schleifer
said. ``It's such a massive undertaking, but the number impacted would
be the same as everywhere else combined, close to 800,000.''

He has targeted 2022 as the earliest date the Fairness Project could run
a campaign there.

``One of the things we've learned is there is a huge benefit to building
a long runway,'' he said. ``You need a substantial campaign entity,
especially in Florida where every region is like its own state. We need
the grass-roots infrastructure, and we need to disrupt the messaging and
all that has been invested in attacking this policy.''

Advertisement

\protect\hyperlink{after-bottom}{Continue reading the main story}

\hypertarget{site-index}{%
\subsection{Site Index}\label{site-index}}

\hypertarget{site-information-navigation}{%
\subsection{Site Information
Navigation}\label{site-information-navigation}}

\begin{itemize}
\tightlist
\item
  \href{https://help.nytimes.com/hc/en-us/articles/115014792127-Copyright-notice}{©~2020~The
  New York Times Company}
\end{itemize}

\begin{itemize}
\tightlist
\item
  \href{https://www.nytco.com/}{NYTCo}
\item
  \href{https://help.nytimes.com/hc/en-us/articles/115015385887-Contact-Us}{Contact
  Us}
\item
  \href{https://www.nytco.com/careers/}{Work with us}
\item
  \href{https://nytmediakit.com/}{Advertise}
\item
  \href{http://www.tbrandstudio.com/}{T Brand Studio}
\item
  \href{https://www.nytimes.com/privacy/cookie-policy\#how-do-i-manage-trackers}{Your
  Ad Choices}
\item
  \href{https://www.nytimes.com/privacy}{Privacy}
\item
  \href{https://help.nytimes.com/hc/en-us/articles/115014893428-Terms-of-service}{Terms
  of Service}
\item
  \href{https://help.nytimes.com/hc/en-us/articles/115014893968-Terms-of-sale}{Terms
  of Sale}
\item
  \href{https://spiderbites.nytimes.com}{Site Map}
\item
  \href{https://help.nytimes.com/hc/en-us}{Help}
\item
  \href{https://www.nytimes.com/subscription?campaignId=37WXW}{Subscriptions}
\end{itemize}
