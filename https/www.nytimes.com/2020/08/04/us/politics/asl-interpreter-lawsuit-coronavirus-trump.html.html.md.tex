Sections

SEARCH

\protect\hyperlink{site-content}{Skip to
content}\protect\hyperlink{site-index}{Skip to site index}

\href{https://www.nytimes.com/section/politics}{Politics}

\href{https://myaccount.nytimes.com/auth/login?response_type=cookie\&client_id=vi}{}

\href{https://www.nytimes.com/section/todayspaper}{Today's Paper}

\href{/section/politics}{Politics}\textbar{}Lawsuit Demands Sign
Language at White House Virus Briefings

\url{https://nyti.ms/3i7d4Qx}

\begin{itemize}
\item
\item
\item
\item
\item
\end{itemize}

\href{https://www.nytimes.com/news-event/coronavirus?action=click\&pgtype=Article\&state=default\&region=TOP_BANNER\&context=storylines_menu}{The
Coronavirus Outbreak}

\begin{itemize}
\tightlist
\item
  live\href{https://www.nytimes.com/2020/08/04/world/coronavirus-cases.html?action=click\&pgtype=Article\&state=default\&region=TOP_BANNER\&context=storylines_menu}{Latest
  Updates}
\item
  \href{https://www.nytimes.com/interactive/2020/us/coronavirus-us-cases.html?action=click\&pgtype=Article\&state=default\&region=TOP_BANNER\&context=storylines_menu}{Maps
  and Cases}
\item
  \href{https://www.nytimes.com/interactive/2020/science/coronavirus-vaccine-tracker.html?action=click\&pgtype=Article\&state=default\&region=TOP_BANNER\&context=storylines_menu}{Vaccine
  Tracker}
\item
  \href{https://www.nytimes.com/2020/08/02/us/covid-college-reopening.html?action=click\&pgtype=Article\&state=default\&region=TOP_BANNER\&context=storylines_menu}{College
  Reopening}
\item
  \href{https://www.nytimes.com/live/2020/08/04/business/stock-market-today-coronavirus?action=click\&pgtype=Article\&state=default\&region=TOP_BANNER\&context=storylines_menu}{Economy}
\end{itemize}

Advertisement

\protect\hyperlink{after-top}{Continue reading the main story}

Supported by

\protect\hyperlink{after-sponsor}{Continue reading the main story}

\hypertarget{lawsuit-demands-sign-language-at-white-house-virus-briefings}{%
\section{Lawsuit Demands Sign Language at White House Virus
Briefings}\label{lawsuit-demands-sign-language-at-white-house-virus-briefings}}

The lack of American Sign Language interpreters at coronavirus briefings
violates the First Amendment, the National Association of the Deaf says.

By \href{https://www.nytimes.com/by/aimee-ortiz}{Aimee Ortiz}

\begin{itemize}
\item
  Aug. 4, 2020
\item
  \begin{itemize}
  \item
  \item
  \item
  \item
  \item
  \end{itemize}
\end{itemize}

\includegraphics{https://static01.nyt.com/images/2020/08/04/us/4xp-interpreter/4xp-interpreter-articleLarge-v2.jpg?quality=75\&auto=webp\&disable=upscale}

The National Association of the Deaf and five deaf Americans have sued
the White House, arguing that the lack of a sign language interpreter at
President Trump's coronavirus briefings violates the First Amendment.

The association is seeking to force Mr. Trump and other White House
officials to use American Sign Language, or A.S.L., interpreters during
``television broadcasts of their coronavirus press conferences and
briefings to make them accessible to deaf and hard-of-hearing people.''

The lawsuit contends that the refusal to provide in-frame sign language
prevents the plaintiffs from accessing the communications provided by
their elected representatives, thus violating their First Amendment
rights.

``Deaf and hard-of-hearing people are affected by the pandemic, just
like everyone else, but we're not getting the same access to
information, resources, and updates as others,'' Howard A. Rosenblum,
the association's chief executive, said in an email. ``Accurate
captioning helps anyone and everyone. Appropriately assigned
interpreters at press briefings avoids possible misunderstandings.''

Reached for a response to the lawsuit, which was filed on Monday in the
United States District Court for the District of Columbia, the White
House referred inquiries to the Justice Department, which declined to
comment.

The N.A.D.
\href{https://www.nad.org/2020/08/03/nad-sues-white-house/}{said in a
statement} that deaf and hard-of-hearing Americans ``are often left
behind with the latest updates and actions the U.S. government has taken
to address this pandemic.'' Live captioning on television can be
unreliable, the association said, especially for those whose primary
language is A.S.L.

In the United States, more than 37 million adults,
\href{https://www.nidcd.nih.gov/health/statistics/quick-statistics-hearing}{or
about 15 percent of people 18 and over}, report some trouble hearing,
according to the National Institutes of Health. For many who are deaf or
hard of hearing, the
\href{https://www.nytimes.com/2020/06/04/us/coronavirus-deaf-culture-challenges.html}{pandemic
has already made life more complicated}.

The lawsuit states that ``A.S.L. is a complete and complex language
distinct from English, with its own vocabulary and rules for grammar and
syntax --- it is not simply English in hand signals.''

The governors of all 50 states have provided in-frame sign language
interpretation for their public briefings, according to the lawsuit.
``All but a small handful continue to do so consistently,'' as have
other world leaders and the mayors of major cities across the nation,
the lawsuit says.

``President Trump, however, does not,'' the lawsuit says. ``He now
stands alone in holding televised briefings regarding the Covid-19
pandemic without ever having provided any A.S.L. interpretation.''

Mr. Rosenblum noted that up until May 13, Gov. Andrew M. Cuomo of New
York used a sign language interpreter only in the online streams of his
daily coronavirus briefings, not in the television broadcasts.

``This rendered the television broadcast inaccessible for those who rely
on A.S.L. and do not have internet access,'' Mr. Rosenblum said.

Mr. Cuomo added the interpreter to the broadcasts only after he was
compelled to do so by a federal judge in response to a lawsuit filed by
Disability Rights New York, an advocacy group.

As far as the White House briefings are concerned, Mr. Rosenblum said on
Tuesday that ``every day there is a delay puts more and more deaf and
hard of hearing people at risk.''

``Access cannot be an afterthought," he said. ``Especially not when
lives are at risk.''

Advertisement

\protect\hyperlink{after-bottom}{Continue reading the main story}

\hypertarget{site-index}{%
\subsection{Site Index}\label{site-index}}

\hypertarget{site-information-navigation}{%
\subsection{Site Information
Navigation}\label{site-information-navigation}}

\begin{itemize}
\tightlist
\item
  \href{https://help.nytimes.com/hc/en-us/articles/115014792127-Copyright-notice}{©~2020~The
  New York Times Company}
\end{itemize}

\begin{itemize}
\tightlist
\item
  \href{https://www.nytco.com/}{NYTCo}
\item
  \href{https://help.nytimes.com/hc/en-us/articles/115015385887-Contact-Us}{Contact
  Us}
\item
  \href{https://www.nytco.com/careers/}{Work with us}
\item
  \href{https://nytmediakit.com/}{Advertise}
\item
  \href{http://www.tbrandstudio.com/}{T Brand Studio}
\item
  \href{https://www.nytimes.com/privacy/cookie-policy\#how-do-i-manage-trackers}{Your
  Ad Choices}
\item
  \href{https://www.nytimes.com/privacy}{Privacy}
\item
  \href{https://help.nytimes.com/hc/en-us/articles/115014893428-Terms-of-service}{Terms
  of Service}
\item
  \href{https://help.nytimes.com/hc/en-us/articles/115014893968-Terms-of-sale}{Terms
  of Sale}
\item
  \href{https://spiderbites.nytimes.com}{Site Map}
\item
  \href{https://help.nytimes.com/hc/en-us}{Help}
\item
  \href{https://www.nytimes.com/subscription?campaignId=37WXW}{Subscriptions}
\end{itemize}
