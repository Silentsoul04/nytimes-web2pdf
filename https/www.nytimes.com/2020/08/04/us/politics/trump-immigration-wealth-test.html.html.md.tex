Sections

SEARCH

\protect\hyperlink{site-content}{Skip to
content}\protect\hyperlink{site-index}{Skip to site index}

\href{https://www.nytimes.com/section/politics}{Politics}

\href{https://myaccount.nytimes.com/auth/login?response_type=cookie\&client_id=vi}{}

\href{https://www.nytimes.com/section/todayspaper}{Today's Paper}

\href{/section/politics}{Politics}\textbar{}Appeals Court Blocks
Immigrant Wealth Test in the Northeast

\url{https://nyti.ms/3kdkIec}

\begin{itemize}
\item
\item
\item
\item
\item
\end{itemize}

Advertisement

\protect\hyperlink{after-top}{Continue reading the main story}

Supported by

\protect\hyperlink{after-sponsor}{Continue reading the main story}

\hypertarget{appeals-court-blocks-immigrant-wealth-test-in-the-northeast}{%
\section{Appeals Court Blocks Immigrant Wealth Test in the
Northeast}\label{appeals-court-blocks-immigrant-wealth-test-in-the-northeast}}

The federal court ruled that the Trump administration's so-called public
charge rule for green card applicants could not go into effect in New
York, Connecticut and Vermont.

\includegraphics{https://static01.nyt.com/images/2020/08/04/us/politics/04dc-publiccharge/04dc-publiccharge-articleLarge.jpg?quality=75\&auto=webp\&disable=upscale}

\href{https://www.nytimes.com/by/zolan-kanno-youngs}{\includegraphics{https://static01.nyt.com/images/2019/12/13/reader-center/author-zolan-kanno-youngs/author-zolan-kanno-youngs-thumbLarge.png}}

By \href{https://www.nytimes.com/by/zolan-kanno-youngs}{Zolan
Kanno-Youngs}

\begin{itemize}
\item
  Aug. 4, 2020
\item
  \begin{itemize}
  \item
  \item
  \item
  \item
  \item
  \end{itemize}
\end{itemize}

WASHINGTON --- A federal appeals court on Tuesday blocked the Trump
administration's efforts to deny permanent residency to legal immigrants
who make even limited use of public benefits like Medicaid, food stamps
or housing vouchers, but restricted the injunction to New York,
Connecticut and Vermont.

The 114-page ruling by the U.S. Court of Appeals for the Second Circuit
affirmed
\href{https://www.nytimes.com/2020/07/30/us/trump-green-card.html}{a
decision last week} by Judge George B. Daniels of the U.S. District
Court in Manhattan, who said the wealth test could discourage residents
from seeking medical care during the coronavirus pandemic. The so-called
public charge rule that was introduced last year expanded the number of
federal support programs whose enrollment would disqualify applicants
from green cards.

Immigration groups have argued that the rule, even before it took
effect, had discouraged immigrants in the country legally from seeking
medical treatment or financial support.

In the past, only substantial and sustained monetary help or long-term
institutionalization counted against immigrants applying for green
cards, and fewer than 1 percent of applicants were disqualified on
public-charge grounds.

Department of Homeland Security officials have criticized the issuance
of nationwide injunctions by district judges, and the three-judge panel
of the appeals court indicated it shared the concern that the lower
court's nationwide block would be ``imposing its view of the law within
the geographic jurisdiction of courts that have reached contrary
conclusions.''

The judges also noted that the rule had been the subject of multiple
legal challenges, including one that has reached the Supreme Court. The
court
\href{https://www.nytimes.com/2020/01/27/us/supreme-court-trump-green-cards.html}{ruled
in January} that the Trump administration could move forward with the
rule as the court system heard substantive arguments for and against it.
At that time, Justices Neil M. Gorsuch and Clarence Thomas issued a
concurring opinion saying that such nationwide blocks caused ``chaos for
litigants, the government, courts and all those affected by these
conflicting decisions.''

The states covered by the new injunction are within the federal appeals
court's jurisdiction.

``We see no need for a broader injunction at this point, particularly in
light of the somewhat unusual posture of this case, namely that the
preliminary injunction has already been stayed by the Supreme Court,''
wrote Judge Gerard E. Lynch, who was appointed by President Barack
Obama.

The plaintiffs in one of the two lawsuits considered by the court
included New York City, Connecticut and Vermont. Immigrant rights groups
brought the second case.

The Department of Homeland Security did not immediately respond to
requests for comment.

Department officials have argued that the wealth test would prevent the
admission of immigrants who would not be able to support themselves in
the United States. After announcing the policy,
\href{https://www.nytimes.com/2019/09/05/us/politics/ken-cuccinelli-immigration-trump.html}{Kenneth
T. Cuccinelli}, the department's acting deputy secretary and a defendant
in the case, promoted the rule by
\href{https://www.nytimes.com/2019/08/14/us/cuccinelli-statue-liberty-poem.html}{revising
the iconic sonnet} on the Statue of Liberty by saying the United States
would welcome those ``who can stand on their own two feet.''

He added that the verses, written by Emma Lazarus, referred to ``people
coming from Europe where they had class-based societies.''

In his ruling, Judge Lynch challenged the argument from homeland
security officials.

``D.H.S. goes too far in assuming that all those who participate in
noncash benefits programs would be otherwise unable to meet their needs
and that they can thus be categorically considered `public charged,'''
Judge Lynch wrote.

Advertisement

\protect\hyperlink{after-bottom}{Continue reading the main story}

\hypertarget{site-index}{%
\subsection{Site Index}\label{site-index}}

\hypertarget{site-information-navigation}{%
\subsection{Site Information
Navigation}\label{site-information-navigation}}

\begin{itemize}
\tightlist
\item
  \href{https://help.nytimes.com/hc/en-us/articles/115014792127-Copyright-notice}{©~2020~The
  New York Times Company}
\end{itemize}

\begin{itemize}
\tightlist
\item
  \href{https://www.nytco.com/}{NYTCo}
\item
  \href{https://help.nytimes.com/hc/en-us/articles/115015385887-Contact-Us}{Contact
  Us}
\item
  \href{https://www.nytco.com/careers/}{Work with us}
\item
  \href{https://nytmediakit.com/}{Advertise}
\item
  \href{http://www.tbrandstudio.com/}{T Brand Studio}
\item
  \href{https://www.nytimes.com/privacy/cookie-policy\#how-do-i-manage-trackers}{Your
  Ad Choices}
\item
  \href{https://www.nytimes.com/privacy}{Privacy}
\item
  \href{https://help.nytimes.com/hc/en-us/articles/115014893428-Terms-of-service}{Terms
  of Service}
\item
  \href{https://help.nytimes.com/hc/en-us/articles/115014893968-Terms-of-sale}{Terms
  of Sale}
\item
  \href{https://spiderbites.nytimes.com}{Site Map}
\item
  \href{https://help.nytimes.com/hc/en-us}{Help}
\item
  \href{https://www.nytimes.com/subscription?campaignId=37WXW}{Subscriptions}
\end{itemize}
