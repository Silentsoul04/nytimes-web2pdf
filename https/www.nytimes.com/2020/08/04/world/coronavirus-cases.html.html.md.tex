Sections

SEARCH

\protect\hyperlink{site-content}{Skip to
content}\protect\hyperlink{site-index}{Skip to site index}

\href{https://www.nytimes.com/section/world}{World}

\href{https://myaccount.nytimes.com/auth/login?response_type=cookie\&client_id=vi}{}

\href{https://www.nytimes.com/section/todayspaper}{Today's Paper}

\href{/section/world}{World}\textbar{}Coronavirus Live Updates:
McConnell Signals Openness to Jobless Aid Extension

\url{https://nyti.ms/31pgRCi}

\begin{itemize}
\item
\item
\item
\item
\item
\item
\end{itemize}

\href{https://www.nytimes.com/news-event/coronavirus?action=click\&pgtype=Article\&state=default\&region=TOP_BANNER\&context=storylines_menu}{The
Coronavirus Outbreak}

\begin{itemize}
\tightlist
\item
  live\href{https://www.nytimes.com/2020/08/04/world/coronavirus-cases.html?action=click\&pgtype=Article\&state=default\&region=TOP_BANNER\&context=storylines_menu}{Latest
  Updates}
\item
  \href{https://www.nytimes.com/interactive/2020/us/coronavirus-us-cases.html?action=click\&pgtype=Article\&state=default\&region=TOP_BANNER\&context=storylines_menu}{Maps
  and Cases}
\item
  \href{https://www.nytimes.com/interactive/2020/science/coronavirus-vaccine-tracker.html?action=click\&pgtype=Article\&state=default\&region=TOP_BANNER\&context=storylines_menu}{Vaccine
  Tracker}
\item
  \href{https://www.nytimes.com/2020/08/02/us/covid-college-reopening.html?action=click\&pgtype=Article\&state=default\&region=TOP_BANNER\&context=storylines_menu}{College
  Reopening}
\item
  \href{https://www.nytimes.com/live/2020/08/04/business/stock-market-today-coronavirus?action=click\&pgtype=Article\&state=default\&region=TOP_BANNER\&context=storylines_menu}{Economy}
\end{itemize}

Advertisement

\protect\hyperlink{after-top}{Continue reading the main story}

Supported by

\protect\hyperlink{after-sponsor}{Continue reading the main story}

LIVE UPDATES

Updated~

Aug. 5, 2020, 3:44 a.m. ET

Aug. 5, 2020, 3:44 a.m. ET

\hypertarget{coronavirus-live-updates-mcconnell-signals-openness-to-jobless-aid-extension}{%
\section{Coronavirus Live Updates: McConnell Signals Openness to Jobless
Aid
Extension}\label{coronavirus-live-updates-mcconnell-signals-openness-to-jobless-aid-extension}}

Tens of millions of Americans have lost crucial jobless benefits, and
lawmakers still can't seem to agree on a relief measure. Israel's
troubled school reopenings could be a lesson for the U.S.

Right Now

Gov. Tate Reeves of Mississippi, a Republican, said that masks would be
mandatory in public and retail spaces statewide starting Tuesday.

\hypertarget{heres-what-you-need-to-know}{%
\subsubsection{Here's what you need to
know:}\label{heres-what-you-need-to-know}}

\begin{itemize}
\tightlist
\item
  \protect\hyperlink{link-762df92}{As talks drag on, McConnell signals
  openness to jobless aid extension, and negotiators agree on a
  deadline.}
\item
  \protect\hyperlink{link-1228a480}{Novavax sees encouraging results
  from two studies of its experimental vaccine.}
\item
  \protect\hyperlink{link-794484ed}{Mississippians must now wear masks
  in public, governor says.}
\item
  \protect\hyperlink{link-30928a04}{The top U.S. health official will
  visit Taiwan, lauded for its coronavirus response.}
\item
  \protect\hyperlink{link-50f7386d}{The United Nations calls on
  policymakers to `plan thoroughly for school reopenings.'}
\item
  \protect\hyperlink{link-567c6115}{A small border hospital in Texas is
  overwhelmed by a surge in cases.}
\item
  \protect\hyperlink{link-4d1eafa8}{N.Y.C.'s health commissioner resigns
  after clashing with the mayor over the virus.}
\end{itemize}

\includegraphics{https://static01.nyt.com/images/2020/08/04/us/politics/04virus-briefing-mcconnell-sub2/merlin_175303407_bfb3e8f7-40ce-40b5-9789-05b0b021935f-articleLarge.jpg?quality=75\&auto=webp\&disable=upscale}

\subsection{}

As talks drag on, McConnell signals openness to jobless aid extension,
and negotiators agree on a deadline.

Negotiators on Capitol Hill reported little progress on Tuesday toward
reaching an agreement over an economic recovery package. But the top
Senate Republican signaled that he might be willing to reverse course
and accept the extension of \$600-per-week jobless-aid payments that
many in his party oppose if it would yield a compromise, and the White
House and congressional Democrats agreed to an end-of-the-week deadline
to seal a deal.

``The American people, in the end, need help,'' Senator Mitch McConnell,
Republican of Kentucky and the majority leader, told reporters. ``And
wherever this thing settles between the president of the United States
and his team that has to sign it into law and the Democrat
not-insignificant minority in the Senate and majority in the House is
something I am prepared to support, even if I have some problems with
certain parts of it.''

Democrats
\href{https://www.nytimes.com/2020/08/02/us/politics/coronavirus-jobless-aid.html}{have
rejected narrow proposals} extending the expired benefits, insisting
that the problem must be dealt with in a broader package of relief
measures. They also want aid for states and cities whose budgets have
been crippled.

Mr. McConnell's comments came after he and other Republicans huddled
privately over lunch with Mark Meadows, the White House chief of staff,
and Steven Mnuchin, the Treasury secretary. Afterward, Republican
senators who have largely sat out the talks sounded downbeat about
striking a deal before they are scheduled to begin a monthlong recess on
Friday.

But later, after a meeting with Mr. Meadows and Mr. Mnuchin, top
Democrats indicated there had been progress.

``They made some concessions, which we appreciated; we made some
concessions, which they appreciated,'' Senator Chuck Schumer of New
York, the Democratic leader, said after the 90-minute meeting, which
Speaker Nancy Pelosi of California hosted in her Capitol Hill suite.
``We're still far away on a lot of the important issues, but we're
continuing to go at it.''

Tens of millions of Americans have lost crucial
\href{https://slack-redir.net/link?url=https\%3A\%2F\%2Fwww.nytimes.com\%2F2020\%2F07\%2F30\%2Fbusiness\%2Funemployment-payments-change.html}{unemployment
benefits} that formally expired on Friday, and economists warn that
permanent damage could be wrought on the economy without action.

Republican leaders have put forward their own plan to extend the weekly
benefit at a significantly lower level. But many of their own
rank-and-file members oppose even that, giving them little leverage
against the united Democrats.

At the White House, Mr. Trump continued to dangle the possibility that
he could circumvent Congress and take executive action to halt evictions
nationwide and suspend the payroll tax. It is far from clear that the
president has the power to do either of these unilaterally, but his
deputies appeared to be using the possibility as a negotiating tactic
with Democrats --- and to get around the objections even within Mr.
Trump's own party on the payroll-tax issue.

``We want to take care of the eviction problem,'' Mr. Trump said at a
news conference. ``People are being evicted unfairly. It's not their
fault. It's China's fault.''

The president blamed the Democrats for rejecting White House offers to
pass a short-term extension of the expired unemployment benefits and
said the only thing Democrats ``really want to do is bail out states
that have been poorly managed by Democrats.''

\hypertarget{-1}{%
\subsection{}\label{-1}}

Novavax sees encouraging results from two studies of its experimental
vaccine.

Image

Vaccine research at a Novavax laboratory in Maryland earlier this year.
The company has said that if its vaccine is shown to be effective, it
can produce 100 million doses by the beginning of next
year.Credit...Andrew Caballero-Reynolds/Agence France-Presse --- Getty
Images

Novavax, the little-known Maryland company that
\href{https://www.nytimes.com/2020/07/16/health/coronavirus-vaccine-novavax.html}{received
a \$1.6 billion} deal from the federal government for its experimental
coronavirus vaccine, announced encouraging results in two preliminary
studies on Tuesday.

In one study, 56 volunteers produced a high level of antibodies against
the virus without any dangerous side effects. In the other, researchers
found that the vaccine strongly protected monkeys from coronavirus
infections.

Although it's not possible to directly compare the data from clinical
trials of different coronavirus vaccines, John Moore, a virologist at
Weill Cornell Medicine who was not involved in the studies, said the
Novavax results were the most impressive he had seen so far.

``This is the first one I'm looking at and saying, `Yeah, I'd take
that,''' Dr. Moore said.

Angela Rasmussen, a virologist at Columbia University who was not
involved in the studies, called them ``encouraging preliminary
results,'' but cautioned that it won't be possible to say whether the
vaccine is safe and effective until Novavax conducts a large-scale study
--- known as Phase III --- comparing people who get vaccinated with
people who get a placebo.

The company, which has never brought a vaccine to market in its 33-year
history, has said that if its vaccine is shown to be effective, it can
produce 100 million doses by the beginning of next year, or enough to
give to 50 million people if administered in two doses. Under its deal
with the federal government, the company will also receive money to
undertake large-scale manufacturing of millions more doses if the
vaccine is shown to work.

Novavax's vaccine is
\href{https://www.nytimes.com/interactive/2020/science/coronavirus-vaccine-tracker.html}{one
of more than two dozen products} to have entered the first round of
safety tests in people, known as Phase I trials. Five other coronavirus
vaccines are already in Phase III trials, in which thousands of people
are tested to see if a vaccine works.

\href{https://www.nytimes.com/interactive/2020/science/coronavirus-vaccine-tracker.html}{}

\includegraphics{https://static01.nyt.com/images/2020/06/09/us/coronavirus-vaccine-tracker-promo-1591728041922/coronavirus-vaccine-tracker-promo-1591728041922-articleLarge-v34.png}

\hypertarget{coronavirus-vaccine-tracker}{%
\subsection{Coronavirus Vaccine
Tracker}\label{coronavirus-vaccine-tracker}}

A look at all the vaccines that have reached trials in humans.

U.S. roundup

\hypertarget{-2}{%
\subsection{}\label{-2}}

Mississippians must now wear masks in public, governor says.

Image

People wearing masks inside a restaurant in Oxford, Miss. Gov. Tate
Reeves said he was ``implementing a statewide mask
mandate.''Credit...Timothy Ivy for The New York Times

In Mississippi, masks are now mandatory in public and retail spaces
statewide, the governor announced Tuesday.

Gov. Tate Reeves, a Republican, said at a news conference that he was
``implementing a statewide mask mandate today.'' He also said that all
students and teachers would be mandated to wear masks when schools open
in the fall, unless they have a medical reason not to. And he said that
he was pushing back the start of the school year in eight counties that
have been hit hardest by the coronavirus.

``I know that I want to see college football in the fall,'' he added.
``The best way for that to occur is for us all to recognize that wearing
a mask, as irritating as it can be --- and I promise you, I hate it more
than anybody watching today --- it is critical.''

Previously, masks had been mandated in 37 of Mississippi's 82 counties.
At the news conference, Mr. Reeves noted that his ``piecemeal approach''
had been criticized ``by an awful lot of people.''

Mr. Reeves has also been criticized for
\href{https://www.nytimes.com/2020/03/26/us/mississippi-coronavirus-essential-businesses-tate-reeves.html}{failing
to encourage many businesses to shut down} during the early months of
the pandemic. And in the months that followed, he had been eager to lift
restrictions that were stalling Mississippi's economy, hoping to have
the whole state open by July 1.

According to
\href{https://www.nytimes.com/interactive/2020/us/mississippi-coronavirus-cases.html}{a
New York Times database}, at least 8 new coronavirus deaths and 572 new
cases were reported in Mississippi on Monday. Over the past week, there
have been an average of 1,167 cases per day, an increase of 13 percent
from the average two weeks earlier.

On Monday, Mr. Reeves said the state was ``starting to turn a corner.''

``Things are improving here,'' he said. ``But that does not mean that we
can declare victory and take a step back.''

Elsewhere in the U.S:

\begin{itemize}
\item
  \textbf{Students in Ohio} will also wear face coverings when they
  return to school in the fall, Gov. Mike DeWine, a Republican, said
  Tuesday. The order will apply to students from kindergarten through
  Grade 12, with exemptions for children who have developmental delays
  or who cannot remove their masks without assistance. Mr. DeWine
  \href{https://twitter.com/GovMikeDeWine/status/1290715998102880259}{cited
  recommendations} from the Ohio Children's Hospital Association and the
  state's chapter of the American Academy of Pediatrics, which said in a
  letter Tuesday that children returning to school should wear masks.
  ``We are going to make that an order from the health department,'' Mr.
  DeWine said. ``I have great confidence that the teachers will work
  this out with kids.''
\item
  Public and private schools in \textbf{Maryland} are divided over
  in-person instruction. An emergency order
  \href{https://twitter.com/GovLarryHogan/status/1290330304830246912}{issued
  Monday} by Gov. Larry Hogan countermanded a Montgomery County Health
  Department directive instructing private schools to
  \href{https://www.montgomerycountymd.gov/OPI/Resources/Files/pdf/2020/NonPublicSchools_07-31-20.pdf}{start
  the year teaching remotely}, as every public school district in the
  Washington area has already decided to do. Mr. Hogan, a Republican,
  said that county health officers didn't have the authority to stop
  private schools from reopening. A similar dynamic is
  \href{https://www.nytimes.com/2020/07/16/upshot/coronavirus-school-reopening-private-public-gap.html}{playing
  out in some other parts of the country}, where public schools are
  opening remotely while private schools are planning in-person or
  various hybrid models.
\item
  A
  \href{https://www.nytimes.com/2020/08/03/us/school-closing-coronavirus.html}{rash
  of positive cases} during the first week of school in some
  \textbf{parts of the United States} foreshadows a stop-and-start year
  in which students and staff members may have to bounce between
  instruction in the classroom and remotely at home because of
  infections and quarantines.
\item
  In his first report to Congress since being appointed by Mr. Trump in
  June, \textbf{Brian D. Miller}, the inspector general overseeing the
  Treasury Department's \$500 billion pandemic recovery fund, said some
  individuals and companies have been able to draw from multiple pots of
  federal pandemic relief money at the same time, a practice he warned
  could lead to an increase in fraud and abuse. Mr. Miller, a former
  White House lawyer, offered a broad overview of his responsibilities
  in his report on Tuesday and provided status updates on the lending
  programs that Treasury is managing as part of the \$2.2 trillion law
  enacted in March.
\item
  Counting for \textbf{the 2020 census} will end on Sept. 30, a month
  earlier than previously scheduled, the Census Bureau
  \href{https://www.census.gov/newsroom/press-releases/2020/delivering-complete-accurate-count.html}{said
  in a statement on Monday}. In recent weeks, the Trump administration
  and Senate Republicans appeared to signal that they
  \href{https://www.nytimes.com/2020/07/28/us/trump-census.html}{wanted
  the census finished well ahead of schedule}.
\item
  Isaias pounded a large swath of the Atlantic Coast on Tuesday,
  unleashing heavy rains and winds as fast as 70 miles per hour as the
  storm swept through the Carolinas and into the Northeast. Shelters had
  prepared to deal with a dual threat from severe weather and the virus
  by screening for symptoms, socially distancing people and distributing
  protective gear. The storm has also closed \textbf{testing centers}
  from Florida to Maryland, which could complicate efforts to gauge
  virus transmission.
\end{itemize}

\hypertarget{tracking-the-coronavirus-}{%
\subsection{\texorpdfstring{\href{https://www.nytimes.com/interactive/2020/us/coronavirus-us-cases.html}{Tracking
the Coronavirus
›}}{Tracking the Coronavirus ›}}\label{tracking-the-coronavirus-}}

\href{https://www.nytimes.com/interactive/2020/us/coronavirus-us-cases.html}{}

\hypertarget{where-cases-are-rising-fastest}{%
\subsubsection{\texorpdfstring{Where cases are \textbf{rising}
fastest}{Where cases are rising fastest}}\label{where-cases-are-rising-fastest}}

\href{https://www.nytimes.com/interactive/2020/us/hawaii-coronavirus-cases.html}{}

Hawaii
\href{https://www.nytimes.com/interactive/2020/us/rhode-island-coronavirus-cases.html}{}

R.I.
\href{https://www.nytimes.com/interactive/2020/us/new-jersey-coronavirus-cases.html}{}

N.J.
\href{https://www.nytimes.com/interactive/2020/us/massachusetts-coronavirus-cases.html}{}

Mass.
\href{https://www.nytimes.com/interactive/2020/us/nebraska-coronavirus-cases.html}{}

Neb.
\href{https://www.nytimes.com/interactive/2020/us/missouri-coronavirus-cases.html}{}

Mo.
\href{https://www.nytimes.com/interactive/2020/us/south-dakota-coronavirus-cases.html}{}

S.D.
\href{https://www.nytimes.com/interactive/2020/us/illinois-coronavirus-cases.html}{}

Ill.
\href{https://www.nytimes.com/interactive/2020/us/oklahoma-coronavirus-cases.html}{}

Okla.
\href{https://www.nytimes.com/interactive/2020/us/alaska-coronavirus-cases.html}{}

Alaska
\href{https://www.nytimes.com/interactive/2020/us/puerto-rico-coronavirus-cases.html}{}

P.R.
\href{https://www.nytimes.com/interactive/2020/us/montana-coronavirus-cases.html}{}

Mont.

\href{https://www.nytimes.com/interactive/2020/us/coronavirus-us-cases.html}{}

\hypertarget{us-hot-spots-}{%
\subsubsection{U.S. hot spots ›}\label{us-hot-spots-}}

\includegraphics{https://static01.nyt.com/newsgraphics/2020/03/16/coronavirus-maps/20139a6bad1057ff5cc9c0ed4110308185896386/images/orphan_usa-threeByTwoSmallAt2X.png}
\href{https://www.nytimes.com/interactive/2020/world/coronavirus-maps.html}{}

\hypertarget{worldwide-}{%
\subsubsection{Worldwide ›}\label{worldwide-}}

\includegraphics{https://static01.nyt.com/newsgraphics/2020/03/16/coronavirus-maps/20139a6bad1057ff5cc9c0ed4110308185896386/images/orphan_world-threeByTwoSmallAt2X.png}

\hypertarget{-3}{%
\subsection{}\label{-3}}

The top U.S. health official will visit Taiwan, lauded for its
coronavirus response.

Image

Alex M. Azar II, the health and human services secretary, praised Taiwan
as ``a model of transparency and cooperation in global health during the
Covid-19.''Credit...Al Drago for The New York Times

The United States' top health official, Alex M. Azar II, will lead a
delegation on a trip to Taiwan, a rare high-level visit by an American
official to the island that has won praise for its success in battling
the coronavirus.

Despite the likelihood that the visit will anger China and further fray
ties between Beijing and Washington, officials billed it as an
opportunity to strengthen economic and public health cooperation between
the United States and Taiwan, a self-ruled territory that is claimed by
Beijing.

As of Tuesday, the island of 23 million people just off the coast of
southern China had reported 476 coronavirus cases and seven deaths.
Officials in Taiwan have tried to turn their relative success in
battling the coronavirus at home
\href{https://www.nytimes.com/2020/04/22/world/asia/coronavirus-china-taiwan.html?searchResultPosition=4}{into
a geopolitical victory.} The island has sent millions of masks,
emblazoned with the words ``made in Taiwan,'' to the United States,
Italy and other countries devastated by the coronavirus.

No date was given for the visit. The trip by Mr. Azar, the secretary of
health and human services, will be the first by a U.S. health secretary
and the first in six years by a U.S. cabinet member, the department said
in a
\href{https://www.hhs.gov/about/news/2020/08/04/hhs-secretary-alex-azar-lead-delegation-taiwan-in-first-visit-by-us-hhs-secretary.html}{statement}
on Tuesday. He is scheduled to meet with senior Taiwanese counterparts
to discuss Taiwan's role as a supplier of medical equipment and critical
technology, among other issues, the health department said.

``Taiwan has been a model of transparency and cooperation in global
health during the Covid-19 pandemic and long before it,'' Mr. Azar said
in the department's statement. ``I look forward to conveying President
Trump's support for Taiwan's global health leadership and underscoring
our shared belief that free and democratic societies are the best model
for protecting and promoting health.''

\hypertarget{-4}{%
\subsection{}\label{-4}}

The United Nations calls on policymakers to `plan thoroughly for school
reopenings.'

\includegraphics{https://static01.nyt.com/images/2020/08/04/world/04virus-briefing-un/merlin_174338310_a5db87e5-45e5-4b44-a65b-b846a42948cf-videoSixteenByNine3000.jpg}

The United Nations on Tuesday called for the world's schools to make
plans to reopen safely --- but only after countries suppress
transmission of the virus and control outbreaks.

``With the combined effect of the pandemic's worldwide economic impact
and the school closures, the learning crisis could turn into a
generational catastrophe,''
\href{https://www.un.org/sites/un2.un.org/files/sg_policy_brief_covid-19_and_education_august_2020.pdf}{a
U.N. policy brief} from U.N. Secretary General António Guterres said.

The announcement argued that the adjustments made by schools worldwide
to closing orders made in response to the coronavirus --- providing
lessons online, over radio, even on television in some places ---
highlighted inequalities among students and school districts and left
many children behind, including those with disabilities.

``Parents, especially women, have been forced to assume heavy care
burdens in the home,'' the brief said.

In many countries around the world, including the United States, school
districts planning to reopen are considering various measures, including
holding classes in shifts or outdoors, mask wearing and so-called
blended classes, in which students supplement in-person lessons with
virtual ones.

\hypertarget{-5}{%
\subsection{}\label{-5}}

A small border hospital in Texas is overwhelmed by a surge in cases.

Image

Medical staff at Starr County Memorial move a patient to a helicopter to
be transported to a larger hospital in San Antonio.Credit...Christopher
Lee for The New York Times

Nearly every day, a crew at
\href{https://www.nytimes.com/2020/08/04/us/texas-coronavirus-rio-grande-valley-starr-county.html}{Starr
County Memorial, a small rural hospital on the Mexican border}, prepares
a patient whom its doctors are unable to help, loads the gurney into a
helicopter and stands back as the aircraft roars into the country sky.

``Very, very unfortunately, of all of the patients we have transferred,
none have come back alive,'' said Dr. Jose Vazquez, the health authority
in Starr County, a remote section of the Rio Grande Valley in Texas that
before the coronavirus outbreak did not have a single I.C.U. bed.

Starr County Memorial's 45 beds were once sufficient for the roughly
65,000 people spread out along the border near Tamaulipas, Mexico. But
the new wave of coronavirus infections has been merciless, with more
than 2,110 cases in the county and nearly 70 deaths that are suspected
of being linked to Covid-19, local officials said.

The surge was slow to arrive. After neighboring counties began reporting
an explosion of infections in the spring, 21 days passed before a single
case was detected in Starr County, Dr. Vazquez said.

But when the state reopened its economy in May, the virus began
spreading rapidly through nearby Hidalgo and Cameron Counties, fueled by
poverty and chronic disease.
\href{https://www.nytimes.com/2020/07/14/us/coronavirus-texas-rio-grande-valley-border.html}{Large
family outbreaks} occurred as soon as people were allowed to leave their
homes freely, health officials said.

Starr County,
\href{https://www.usatoday.com/story/money/2019/01/25/poorest-counties-in-the-us-median-household-income/38870175/}{one
of the poorest in the nation}, is not alone. A
\href{https://www.healthaffairs.org/doi/10.1377/hlthaff.2020.00581}{study
published this week in the journal Health Affairs}, warning of stark
disparities in the availability of critical care facilities, found that
nearly half of the nation's communities with a median income of \$35,000
or less had no intensive care beds at all.

NEW YORK ROUNDUP

\hypertarget{-6}{%
\subsection{}\label{-6}}

N.Y.C.'s health commissioner resigns after clashing with the mayor over
the virus.

Image

New York City's health commissioner, Dr. Oxiris Barbot, in March. Her
resignation on Tuesday could renew questions about the mayor's handling
of the response to the outbreak.Credit...Gabriela Bhaskar for The New
York Times

New York City's health commissioner, Dr. Oxiris Barbot,
\href{https://www.nytimes.com/2020/08/04/nyregion/oxiris-barbot-health-commissioner-resigns.html}{resigned
on Tuesday in protest over her ``deep disappointment''} with Mayor Bill
de Blasio's handling of the coronavirus outbreak and subsequent efforts
to keep the outbreak in check.

Her departure came after
\href{https://www.nytimes.com/2020/05/14/nyregion/coronavirus-de-blasio-mitchell-katz.html}{escalating
tensions} between City Hall and top city health department officials,
which had begun at the start of the outbreak in March, burst into public
view and raised concerns that the feuding was undermining crucial public
health policies.

``I leave my post today with deep disappointment that during the most
critical public health crisis in our lifetime, that the health
department's incomparable disease control expertise was not used to the
degree it could have been,'' she said in her resignation email sent to
Mr. de Blasio, a copy of which was shared with The New York Times.

``Our experts are world renowned for their epidemiology, surveillance
and response work. The city would be well served by having them at the
strategic center of the response not in the background.''

Mr. de Blasio reacted to her resignation by defending his handling of
the outbreak, which devastated the city in the spring, killing more than
20,000 residents, even as it has largely subsided in recent weeks.

Still, the turnover in the Department of Health and Mental Hygiene comes
at a pivotal moment: Public schools are scheduled to partially open next
month --- which could be crucial for the city's recovery --- and fears
are growing that the outbreak could surge again when the weather cools.

``It had been clear in recent days that it was time for a change,'' Mr.
de Blasio said in a hastily called news conference. ``We need an
atmosphere of unity. We need an atmosphere of common purpose.''

The mayor moved quickly to replace Dr. Barbot, immediately announcing
the appointment of a new health commissioner, Dr. Dave A. Chokshi, a
former senior leader at Health + Hospitals, the city's public hospital
system. The speed of the appointment and the robustness of the
announcement --- Mr. de Blasio had lined up
\href{https://wayback.archive-it.org/4765/20170106172109/https:/www.hhs.gov/about/leadership/vadm-vivek-murthy/index.html}{a
former surgeon general} to speak highly of Dr. Chokshi --- suggested
that Dr. Barbot's resignation had not occurred in a vacuum. One city
official said she had done so on Tuesday because she believed she was
going to be fired.

Elsewhere in the New York area:

\begin{itemize}
\item
  The governors of \textbf{New York} and
  \textbf{\href{https://twitter.com/GovNedLamont/status/1290658706003238913}{Connecticut}}
  said Tuesday their states would now require travelers from Rhode
  Island
  \href{https://coronavirus.health.ny.gov/covid-19-travel-advisory}{to
  quarantine for 14 days}, an addition to a list of 33 other states and
  Puerto Rico. The weekly update also saw Delaware and Washington, D.C.,
  removed from the list. The travelers can otherwise face fines,
  \href{https://portal.ct.gov/Coronavirus/Travel}{with some exceptions}.
\item
  \textbf{\href{https://twitter.com/GovMurphy/status/1290656319578484737}{New
  Jersey}} also said travelers from those 35 places were subject to a
  14-day quarantine,
  \href{https://covid19.nj.gov/faqs/nj-information/travel-information/which-states-are-on-the-travel-advisory-list-are-there-travel-restrictions-to-or-from-new-jersey}{though
  complying is voluntary}. Many states across the country
  \href{https://www.nytimes.com/2020/07/10/travel/state-travel-restrictions.html?referringSource=articleShare}{have
  added travel restrictions}.
\end{itemize}

\begin{itemize}
\tightlist
\item
  \textbf{In New York City}, the 2020 holiday production of ``Christmas
  Spectacular Starring the Radio City Rockettes''
  \href{https://www.nytimes.com/2020/08/04/theater/radio-city-rockettes-christmas-canceled.html}{has
  been canceled because of the pandemic}, Madison Square Garden
  Entertainment, which manages the show, announced on Tuesday. The
  Madison Square Garden Company plans to lay off 350 people, a
  spokeswoman said.
\end{itemize}

Global Roundup

\hypertarget{-7}{%
\subsection{}\label{-7}}

Tanzania's president says the country is virus-free, but others are
skeptical.

Image

Members of Tanzania's ruling party had their temperature checked and
sanitized their hands as a precaution against the virus in
July.Credit...Associated Press

More than
\href{https://www.who.int/docs/default-source/coronaviruse/situation-reports/20200803-covid-19-sitrep-196-cleared.pdf?sfvrsn=8a8a3ca4_6}{87
days have passed} since **** Tanzania reported even a single new virus
case --- far longer than any other African country. Tanzania's president
has declared the scourge ``absolutely finished'' and encouraged tourists
to come back.

But outside the country, people are skeptical, and inside, few dare
stand up to the president, John Magufuli, who has become increasingly
autocratic since he was elected. Mr. Magufuli has said that the power of
prayer helped purge the virus from Tanzania, even as the African
continent is expected this week to cross the threshold of one million
reported cases.

The Tanzanian president has promoted an unproven herbal tea from
Madagascar as a cure. He has disparaged social distancing and mask
wearing. And his government has not disseminated any recent data to the
World Health Organization. The group last heard from Tanzania on April
29, when the country reported 509 cases and 21 deaths from Covid-19.

Mr. Magufuli's handling of the pandemic ``has been nothing short of an
irresponsible disaster,'' said Tundu Lissu, an opposition leader who
\href{https://www.nytimes.com/aponline/2020/07/27/world/africa/ap-af-tanzania-opposition-leader-returns.html}{fled
the country in 2017 but recently returned} to run for president. ``His
attitude has been Covid-19 will somehow go away if we all stop talking
about it.''

In neighboring Kenya, lawmakers have also
\href{https://www.youtube.com/watch?v=K1PeaFeAxGM}{expressed concern}
about Tanzania's virus response. The Kenyan authorities denied entry to
dozens of Tanzanian truck drivers who had tested positive at border
points.

Health experts warn that Mr. Magufuli's denial around the virus could be
calamitous.

``With no testing data or clinical surveillance information, Tanzania
will be late in detecting and dealing with a potentially delayed
explosion of severe clinical cases,'' said Frank Minja, a Tanzanian
doctor who is an associate professor of radiology and biomedical imaging
at the Yale School of Medicine.

October could be a ``make-or-break election'' in Tanzania's history, Mr.
Lissu said. ``We stand on the brink of disaster,'' he added. ``But we
are also on the brink of a miracle.''

Elsewhere around the world:

\begin{itemize}
\item
  Many hospitals in \textbf{Lebanon}, already dealing with virus
  patients, are now overwhelmed after
  \href{https://www.nytimes.com/2020/08/04/world/middleeast/beirut-explosion-blast.html}{an
  immense explosion in Beirut} injured thousands. The country has
  reported 5,062 cases and 65 deaths from the virus, while lockdown
  measures have aggravated an already
  \href{https://www.nytimes.com/2020/07/12/world/middleeast/beirut-lebanon-economic-crisis.html}{dire
  economic crisis}.
\item
  After Moscow announced that it would begin widespread vaccination of
  its population in October with a vaccine that had not yet been fully
  tested in clinical trials, the \textbf{World Health Organization} on
  Tuesday urged caution, recommending that the country follow
  established guidelines for producing safe vaccines. Russia is moving
  ahead with several vaccine prototypes, its officials said, and at
  least one effort, developed by the Gamaleya Institute in Moscow, is in
  advanced stages of testing and has reportedly has been tested to some
  extent on soldiers.
\end{itemize}

\begin{itemize}
\item
  Prime Minister Hubert Alexander Minnis of \textbf{the Bahamas}
  \href{https://www.youtube.com/watch?v=ZzwLiboOXe0}{announced Monday}
  that the country would resume a national lockdown ``for a minimum of
  two weeks,'' starting at 10 p.m. on Tuesday. ``Nearing the end of this
  period, we will assess the health data and advise whether a further
  lockdown period is necessary,'' he said. The Bahamas previously
  instituted a strict 24-hour lockdown for residents, which if broken
  could result in a \$10,000 fine or 18 months in prison. Virus cases
  have
  \href{https://www.nytimes.com/interactive/2020/world/coronavirus-maps.html}{skyrocketed
  there recently}, with almost 44 percent of the total 679 cases being
  reported in the past seven days.
\item
  The state of Victoria in \textbf{Australia}, which has had a
  resurgence of the coronavirus and has
  \href{https://www.nytimes.com/2020/08/04/world/australia/coronavirus-melbourne-lockdown.html}{enforced
  among the strictest lockdown measures in the world}, reported 725 new
  cases and 15 deaths from the coronavirus on Wednesday, the highest
  numbers since the pandemic began. New curfews and restrictions in the
  state mean essential workers must now carry a permit before leaving
  home.
\item
  Students in \textbf{Mexico} will exclusively take classes broadcast on
  television or the radio when the school year begins later this month,
  in an effort to avoid further coronavirus outbreaks, the government
  announced on Monday. Schools will only reopen when authorities
  determine that new and active infections, which remain high across the
  nation, decline enough for a safe return to the classroom.
\item
  \textbf{Israel} reopened schools in May, and within days
  \href{https://www.nytimes.com/2020/08/04/world/middleeast/coronavirus-israel-schools-reopen.html}{infections
  were reported} at a Jerusalem high school. The virus rippled out to
  the students' homes and then to other schools and neighborhoods,
  ultimately infecting hundreds of students, teachers and relatives.
  Other outbreaks forced hundreds of schools to close, and across the
  country, tens of thousands of students and teachers were quarantined.
  As countries consider back-to-school strategies for the fall, the
  outbreaks there
  \href{https://www.nytimes.com/2020/08/04/world/middleeast/coronavirus-israel-schools-reopen.html}{illustrate
  the dangers of moving too precipitously}.
\end{itemize}

\hypertarget{-8}{%
\subsection{}\label{-8}}

Trump addresses the death toll: `It is what it is.'

A day before the United States surpassed
\href{https://www.nytimes.com/2020/07/29/us/coronavirus-deaths-150000.html}{150,000
deaths from the coronavirus}, Mr. Trump appeared resigned to the toll,
saying in an interview, ``It is what it is.''

``They are dying. That's true,'' Mr. Trump told
\href{https://www.axios.com/full-axios-hbo-interview-donald-trump-cd5a67e1-6ba1-46c8-bb3d-8717ab9f3cc5.html}{Axios
in an interview} recorded on July 28 and released in its entirety on
Monday. ``It is what it is. But that doesn't mean we aren't doing
everything we can. It's under control as much as you can control it.''

The president's critics say he could have done much more to keep the
virus from spreading to the extent it has, including encouraging states
to be more cautious in reopening instead of encouraging them.

The country's death toll, currently nearly 156,000, is far from the
total of ``75, 80 to 100,000'' deaths that
\href{https://www.nytimes.com/2020/05/03/us/politics/trump-coronavirus.html}{Mr.
Trump predicted in early May} when he credited himself with preventing
the toll from being worse.

Even after his predictions proved wrong, Mr. Trump has continued to
credit himself for the United States not being even worse off.

``One person's too much,'' Mr. Trump told Axios. ``And those people that
really understand it, that really understand it, they said it's an
incredible job that we've done.''

SCIENCE AND TECHNOLOGY ROUNDUP

\hypertarget{-9}{%
\subsection{}\label{-9}}

Six U.S. governors join forces to buy 3 million tests.

Image

Gov. Larry Hogan of Maryland, who created the new
compact.Credit...Jonathan Ernst/Reuters

Governors of six states said on Tuesday that they were partnering to
purchase millions of
\href{https://www.nytimes.com/2020/08/04/us/virus-testing-delays.html}{virus
tests} and expand their testing capability as many states continue to
struggle to keep up with the demand for tests.

The governors of Louisiana, Maryland, Massachusetts, Michigan, Ohio and
Virginia are negotiating a purchase of three million antigen tests ---
500,000 per state --- as part of the new compact, which was created by
Gov. Larry Hogan of Maryland, a Republican and the outgoing chair of the
National Governors Association.

Members of the compact hope that it will show companies that there is
``significant demand'' to create more tests, according to a statement
from Mr. Hogan's office, something made apparent by the long lines that
continue to plague virus testing sites across the country. The governors
--- three Republicans and three Democrats --- also hoped the compact
would help states buy tests in a more ``cost-effective manner.'' More
states and local governments may join the group.

\href{https://www.nytimes.com/2020/07/06/health/fast-coronavirus-tests.html}{Antigen
tests, the type the states would buy, can provide results} in less than
an hour, but scientists have said that they fear the tests will
frequently miss infections. The governors are negotiating to purchase
the three million tests from two medical companies --- Becton, Dickinson
\& Company and the
\href{https://www.nytimes.com/2020/05/09/health/antigen-testing-fda-coronavirus.html}{Quidel
Corporation} --- whose tests
\href{https://www.nytimes.com/2020/07/06/health/fast-coronavirus-tests.html}{could
produce false negative results} between 15 and 20 percent of the time.
The companies were
\href{https://www.nytimes.com/2020/07/06/health/fast-coronavirus-tests.html}{the
first to receive emergency authorization} from the Food and Drug
Administration for their coronavirus antigen tests.

The Rockefeller Foundation, a philanthropic organization in New York, is
also part of the compact between the governors and said it was ready to
help find sources of funding for the testing operation.

``With severe shortages and delays in testing and the federal
administration attempting to cut funding for testing, the states are
banding together to acquire millions of faster tests to help save lives
and slow the spread of Covid-19,'' Mr. Hogan said in a statement.

\begin{itemize}
\item
  The National Institutes of Health is now recruiting patients for two
  studies to test possible \textbf{antibody treatments} for Covid-19.
  The studies, which are now in their second phase, are testing drugs
  called monoclonal antibodies produced by Eli Lilly and its partner,
  Abcellera Biologics in Vancouver. The process began in March, and has
  progressed at ``record speed,'' said Daniel Skovronsky, chief
  scientific officer at Eli Lilly. Researchers hope to have results in
  October or November.
\item
  Researchers are studying why in certain patients, according to a
  flurry of recent studies, the virus appears to
  \href{https://www.nytimes.com/2020/08/04/health/coronavirus-immune-system.html}{make}\textbf{\href{https://www.nytimes.com/2020/08/04/health/coronavirus-immune-system.html}{the
  immune
  system}}\href{https://www.nytimes.com/2020/08/04/health/coronavirus-immune-system.html}{go
  haywire}. Unable to marshal the right cells and molecules to fight off
  the invader, the bodies of the infected instead launch an entire
  arsenal of weapons --- a misguided barrage that can wreak havoc on
  healthy tissues, experts said. Researchers studying these unusual
  responses are finding patterns that distinguish patients on the path
  to recovery from those who fare far worse. Insights gleaned from the
  data might
  \href{https://www.nytimes.com/interactive/2020/science/coronavirus-drugs-treatments.html}{help
  tailor treatments to individuals}, easing symptoms or perhaps even
  vanquishing the virus before it has a chance to
  \href{https://www.nytimes.com/2020/04/01/health/coronavirus-cytokine-storm-immune-system.html}{push
  the immune system too far}.
\item
  Thousands of Covid-19 patients have been
  \href{https://www.nytimes.com/2020/08/04/health/blood-plasma-covid-19.html}{treated
  with}\textbf{\href{https://www.nytimes.com/2020/08/04/health/blood-plasma-covid-19.html}{blood
  plasma}} outside of rigorous clinical trials --- hampering research
  that would have shown whether the therapy worked. Doctors and
  hospitals desperate to save the sickest patients have been eager to
  try a therapy that is safe and might work. Many patients and their
  doctors --- knowing they could get the treatment under a government
  program --- have been unwilling to join clinical trials that might
  provide them with a placebo instead of the plasma. The unexpected
  demand for plasma has inadvertently undercut the research that could
  prove that it works.
\end{itemize}

Reporting was contributed by Livia Albeck-Ripka, Pam Belluck, Nicholas
Bogel-Burroughs**,** Emma Bubola, Benedict Carey, Julia Carmel, Damien
Cave, Emily Cochrane, Abdi Latif Dahir, Jacey Fortin, Nicholas Fandos,
Michael Gold, J. David Goodman, Maggie Haberman, Mike Ives, Juliana Kim,
Isabel Kershner, Gina Kolata, Giulia McDonnell Nieto del Rio, Emily
Palmer, Amy Qin, Marc Stein, Eileen Sullivan, Jim Tankersley, Michael
Wines, Will Wright and Karen Zraick.

Advertisement

\protect\hyperlink{after-bottom}{Continue reading the main story}

\hypertarget{site-index}{%
\subsection{Site Index}\label{site-index}}

\hypertarget{site-information-navigation}{%
\subsection{Site Information
Navigation}\label{site-information-navigation}}

\begin{itemize}
\tightlist
\item
  \href{https://help.nytimes.com/hc/en-us/articles/115014792127-Copyright-notice}{©~2020~The
  New York Times Company}
\end{itemize}

\begin{itemize}
\tightlist
\item
  \href{https://www.nytco.com/}{NYTCo}
\item
  \href{https://help.nytimes.com/hc/en-us/articles/115015385887-Contact-Us}{Contact
  Us}
\item
  \href{https://www.nytco.com/careers/}{Work with us}
\item
  \href{https://nytmediakit.com/}{Advertise}
\item
  \href{http://www.tbrandstudio.com/}{T Brand Studio}
\item
  \href{https://www.nytimes.com/privacy/cookie-policy\#how-do-i-manage-trackers}{Your
  Ad Choices}
\item
  \href{https://www.nytimes.com/privacy}{Privacy}
\item
  \href{https://help.nytimes.com/hc/en-us/articles/115014893428-Terms-of-service}{Terms
  of Service}
\item
  \href{https://help.nytimes.com/hc/en-us/articles/115014893968-Terms-of-sale}{Terms
  of Sale}
\item
  \href{https://spiderbites.nytimes.com}{Site Map}
\item
  \href{https://help.nytimes.com/hc/en-us}{Help}
\item
  \href{https://www.nytimes.com/subscription?campaignId=37WXW}{Subscriptions}
\end{itemize}
