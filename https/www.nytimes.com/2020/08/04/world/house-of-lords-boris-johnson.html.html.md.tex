Sections

SEARCH

\protect\hyperlink{site-content}{Skip to
content}\protect\hyperlink{site-index}{Skip to site index}

\href{https://www.nytimes.com/section/world}{World}

\href{https://myaccount.nytimes.com/auth/login?response_type=cookie\&client_id=vi}{}

\href{https://www.nytimes.com/section/todayspaper}{Today's Paper}

\href{/section/world}{World}\textbar{}New Nominations to U.K. House of
Lords Raise Old Concerns of Cronyism

\url{https://nyti.ms/3km5io2}

\begin{itemize}
\item
\item
\item
\item
\item
\end{itemize}

Advertisement

\protect\hyperlink{after-top}{Continue reading the main story}

Supported by

\protect\hyperlink{after-sponsor}{Continue reading the main story}

\hypertarget{new-nominations-to-uk-house-of-lords-raise-old-concerns-of-cronyism}{%
\section{New Nominations to U.K. House of Lords Raise Old Concerns of
Cronyism}\label{new-nominations-to-uk-house-of-lords-raise-old-concerns-of-cronyism}}

Critics say Prime Minister Boris Johnson's nominations for lifetime
legislative posts continued a pattern of patronage that undermines the
credibility of a long-troubled institution.

\includegraphics{https://static01.nyt.com/images/2020/08/04/world/04uk-lords/merlin_160177563_fe404045-8372-4199-a525-23dde7c89dd6-articleLarge.jpg?quality=75\&auto=webp\&disable=upscale}

\href{https://www.nytimes.com/by/mark-landler}{\includegraphics{https://static01.nyt.com/images/2019/10/22/reader-center/author-mark-landler/author-mark-landler-thumbLarge-v3.png}}

By \href{https://www.nytimes.com/by/mark-landler}{Mark Landler}

\begin{itemize}
\item
  Published Aug. 4, 2020Updated Aug. 5, 2020, 12:36 a.m. ET
\item
  \begin{itemize}
  \item
  \item
  \item
  \item
  \item
  \end{itemize}
\end{itemize}

LONDON --- One is a Russian-born British newspaper baron whose father
was once a K.G.B. officer. Another is a retired cricket player who goes
by the nickname Beefy, and yet another is the prime minister's younger
brother.

With a collection of names like that, it was perhaps little wonder that
Prime Minister Boris Johnson decided to release his first
\href{https://www.nytimes.com/aponline/2020/07/31/world/europe/ap-eu-britain-house-of-lords-1st-ld-writethru.html}{list
of appointments} to the House of Lords late on a Friday afternoon last
week, with Parliament in recess and the public lulled into a tropical
haze on the hottest day of the year so far.

But Downing Street's apparent effort to bury the news seemed, in the
end, unnecessary. The handing out of peerages, as lifetime appointments
to the House of Lords are called, is one of Britain's most
\href{https://www.nytimes.com/2015/12/18/world/europe/britain-house-of-lords-changes.html}{predictable
displays of patronage} and cronyism --- so reliably unsavory, regardless
of the prime minister or party in power, that even Mr. Johnson's critics
found it hard to get too wound up about it.

``Shameless,''
\href{https://www.theguardian.com/commentisfree/2020/aug/03/boris-johnsons-list-of-lords-is-a-disgrace-corruption-westminster}{wrote
Simon Jenkins}, a columnist for The Guardian, on Monday.

Reached later by phone, Mr. Jenkins said he was actually more concerned
about a new government proposal to overhaul Britain's planning laws,
which date back to 1947. He said it would strip local councils of
control over real estate development in the name of Mr. Johnson's drive
to ``build, build, build.''

\includegraphics{https://static01.nyt.com/images/2020/08/04/world/04uk-lords2/merlin_174787905_d0e79b41-71e8-4875-b036-49cd938d63d4-articleLarge.jpg?quality=75\&auto=webp\&disable=upscale}

The
\href{https://www.nytimes.com/2015/08/23/world/europe/a-british-house-overflowing-with-lords-draws-scorn.html}{degradation
of the House of Lords}, by comparison, has been going on for 400 years.
``To be honest, since James I,'' Mr. Jenkins said, referring to the
first Stuart king of England, whose sale of peerages, during his reign
from 1603 to 1625, was so brazen that it turned the landed gentry
against the crown.

Still, Mr. Jenkins and other critics said Mr. Johnson's appointments
broke new ground in ways that could further tarnish the credibility of
the House of Lords. At its best, the British Parliament's ancient upper
chamber serves as a check on the more unruly House of Commons, debating
and amending legislation, if with less power than the lower chamber. In
recent decades, though, it has become known mainly as
\href{https://www.nytimes.com/2009/01/29/world/europe/29iht-lords.1.19775638.html}{a
sinecure for wealthy donors} and other well-connected types.

Defying a commitment to fight bloat, the prime minister created 36 new
peers, the second highest number in more than two decades, swelling the
chamber to nearly 800 members. Among legislative bodies worldwide, only
the Chinese National People's Congress is larger, with nearly 3,000
seats. The House of Commons, which does most of the legislative heavy
lifting, is capped at 650 elected members.

``This just hangs around the neck of the House of Lords, and to the
extent that its reputation is damaged, it is weakened,'' said Meg
Russell, a professor of politics at University College London and an
expert on the institution. ``It becomes more expensive, less efficient
and effective, and more open to ridicule.''

Mr. Johnson also broke with custom by nominating peers from the
opposition Labour Party, usually the prerogative of the party's leader.
He chose people who supported his Brexit campaign, a thumb in the eye to
Labour, which was split by Brexit in the last election.

That will buttress the pro-Brexit contingent in the House of Lords but
it also creates a politically nonaligned faction --- members who are not
welcome in their own party but do not belong to Mr. Johnson's
Conservative Party either --- which could make it more unpredictable in
its voting.

Image

The 36 new peers created by Mr. Johnson bring the House of Lords to
nearly 800 members.Credit...Pool photo by Kirsty Wigglesworth

Queen Elizabeth II confers peerages and the prime minister's nominations
are vetted by a House of Lords Appointments Commission. In rare
instances, the commission can refuse to endorse them on grounds of
propriety. This time, a person with knowledge of the process said, the
government brushed aside the commission on certain cases, adding to the
lack of transparency.

``The House of Lords is completely at the mercy of the prime minister,''
Professor Russell said. ``He can discredit the institution by putting in
an inappropriate number of people and inappropriate types of people.''

Some experts said Mr. Johnson's nominations were in keeping with a
government that holds the House of Lords in contempt. He has suggested
\href{https://www.bbc.co.uk/news/uk-politics-53432776}{moving the
chamber to York}, in the north of England, to make it more attuned to
the interests of ordinary voters. The lords, many of whom live in or
near London, are unsurprisingly reluctant.

Even though Mr. Johnson appointed dozens of new peers, a spokesman for
10 Downing Street said the prime minister remained committed to reducing
the size of the House of Lords. He did not comment on individual
nominees, beyond saying they were ``nominated in recognition of their
contribution to society.''

But there is no shortage of red flags on Mr. Johnson's list, starting
with
\href{https://www.nytimes.com/2015/01/01/style/the-rise-of-evgeny-lebedev.html}{Evgeny
Lebedev}, who owns The London Evening Standard and is a close friend of
Mr. Johnson's. Mr. Lebedev's father and bankroller, Alexander, is an
oligarch who once worked for the K.G.B. and has financed Novaya Gazeta,
a liberal-leaning paper disliked by the Kremlin.

Image

Evgeny Lebedev is the owner of The London Evening Standard, which his
family purchased in 2009.Credit...David M. Benett/Getty Images

The younger Mr. Lebedev is known for throwing bacchanals at his
converted castle in Italy. Mr. Johnson was photographed in 2018 at an
airport, returning from one of them. Fellow passengers
\href{https://www.theguardian.com/politics/2019/jul/26/boris-johnson-security-evgeny-lebedev-perugia-party}{told
The Guardian} that Mr. Johnson, then serving as foreign secretary,
looked as though ``he had slept in his clothes.''

There is no suggestion that Mr. Lebedev is acting as an agent of the
Russian government. But the timing of his peerage was awkward, coming a
week after a parliamentary committee
\href{https://www.nytimes.com/2020/07/21/world/europe/uk-russia-report-brexit-interference.html}{released
a long-awaited report documenting how Russian money had corrupted
British politics}. The report said several other members of the House of
Lords, whom it did not name, had business interests linked to Russia or
worked for companies with Russian ties.

Two vocal Brexiteers on Mr. Johnson's list also attracted notice: Ian
Botham, a colorful and charismatic retired cricket player widely known
as Beefy, and Claire Fox, a writer and politician who began as a
Communist and migrated to the right over the years, joining the Brexit
Party and serving in the European Parliament.

Ms. Fox herself is no fan of the institution she is now joining, having
praised others for turning down peerages. On Twitter, Ms. Fox said she
still favored abolishing the Lords but that as long as it existed, she
would be happy to argue for its mothballing while standing in its
majestic chamber.

There was less debate about Mr. Johnson's decision to recommend a
peerage for his brother, Jo Johnson. While the younger Mr. Johnson has a
good reputation --- he served in Parliament and the cabinet before
resigning to protest his brother's handling of Brexit --- most viewed it
as an open-and-shut case of nepotism.

Image

Jo Johnson, brother of the prime minister, made it onto the list of
appointees.Credit...Daniel Leal-Olivas/Agence France-Presse --- Getty
Images

And then there was the conspicuous snub of John Bercow, the former
speaker of the House of Commons, whose stentorian calls for order during
the emotional debates over Brexit
\href{https://www.nytimes.com/2019/10/31/world/europe/british-parliament-speaker-john-bercow-resigns.html}{turned
him briefly into a celebrity}.

The Labour Party's former leader, Jeremy Corbyn, nominated Mr. Bercow in
keeping with a long tradition of elevating retired speakers. But Mr.
Johnson left him off the list, with aides suggesting his candidacy had
been sunk by allegations that he bullied subordinates. Others said it
was revenge for Mr. Bercow's readiness to let backbenchers harangue the
government on its Brexit policy.

To some critics, penalizing Mr. Bercow for bad behavior seemed almost
comically arbitrary, given the other names on Mr. Johnson's list.

``It's as if he invited his mates for a drunken weekend in Prague and
offered them all seats in the House of Lords afterward,'' Mr. Jenkins
said.

Advertisement

\protect\hyperlink{after-bottom}{Continue reading the main story}

\hypertarget{site-index}{%
\subsection{Site Index}\label{site-index}}

\hypertarget{site-information-navigation}{%
\subsection{Site Information
Navigation}\label{site-information-navigation}}

\begin{itemize}
\tightlist
\item
  \href{https://help.nytimes.com/hc/en-us/articles/115014792127-Copyright-notice}{©~2020~The
  New York Times Company}
\end{itemize}

\begin{itemize}
\tightlist
\item
  \href{https://www.nytco.com/}{NYTCo}
\item
  \href{https://help.nytimes.com/hc/en-us/articles/115015385887-Contact-Us}{Contact
  Us}
\item
  \href{https://www.nytco.com/careers/}{Work with us}
\item
  \href{https://nytmediakit.com/}{Advertise}
\item
  \href{http://www.tbrandstudio.com/}{T Brand Studio}
\item
  \href{https://www.nytimes.com/privacy/cookie-policy\#how-do-i-manage-trackers}{Your
  Ad Choices}
\item
  \href{https://www.nytimes.com/privacy}{Privacy}
\item
  \href{https://help.nytimes.com/hc/en-us/articles/115014893428-Terms-of-service}{Terms
  of Service}
\item
  \href{https://help.nytimes.com/hc/en-us/articles/115014893968-Terms-of-sale}{Terms
  of Sale}
\item
  \href{https://spiderbites.nytimes.com}{Site Map}
\item
  \href{https://help.nytimes.com/hc/en-us}{Help}
\item
  \href{https://www.nytimes.com/subscription?campaignId=37WXW}{Subscriptions}
\end{itemize}
