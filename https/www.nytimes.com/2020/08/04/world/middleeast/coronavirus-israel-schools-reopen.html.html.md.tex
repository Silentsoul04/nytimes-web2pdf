Sections

SEARCH

\protect\hyperlink{site-content}{Skip to
content}\protect\hyperlink{site-index}{Skip to site index}

\href{https://www.nytimes.com/section/world/middleeast}{Middle East}

\href{https://myaccount.nytimes.com/auth/login?response_type=cookie\&client_id=vi}{}

\href{https://www.nytimes.com/section/todayspaper}{Today's Paper}

\href{/section/world/middleeast}{Middle East}\textbar{}When Covid
Subsided, Israel Reopened Its Schools. It Didn't Go Well.

\url{https://nyti.ms/3k7S8eb}

\begin{itemize}
\item
\item
\item
\item
\item
\item
\end{itemize}

\href{https://www.nytimes.com/news-event/coronavirus?action=click\&pgtype=Article\&state=default\&region=TOP_BANNER\&context=storylines_menu}{The
Coronavirus Outbreak}

\begin{itemize}
\tightlist
\item
  live\href{https://www.nytimes.com/2020/08/04/world/coronavirus-cases.html?action=click\&pgtype=Article\&state=default\&region=TOP_BANNER\&context=storylines_menu}{Latest
  Updates}
\item
  \href{https://www.nytimes.com/interactive/2020/us/coronavirus-us-cases.html?action=click\&pgtype=Article\&state=default\&region=TOP_BANNER\&context=storylines_menu}{Maps
  and Cases}
\item
  \href{https://www.nytimes.com/interactive/2020/science/coronavirus-vaccine-tracker.html?action=click\&pgtype=Article\&state=default\&region=TOP_BANNER\&context=storylines_menu}{Vaccine
  Tracker}
\item
  \href{https://www.nytimes.com/2020/08/02/us/covid-college-reopening.html?action=click\&pgtype=Article\&state=default\&region=TOP_BANNER\&context=storylines_menu}{College
  Reopening}
\item
  \href{https://www.nytimes.com/live/2020/08/04/business/stock-market-today-coronavirus?action=click\&pgtype=Article\&state=default\&region=TOP_BANNER\&context=storylines_menu}{Economy}
\end{itemize}

Advertisement

\protect\hyperlink{after-top}{Continue reading the main story}

Supported by

\protect\hyperlink{after-sponsor}{Continue reading the main story}

\hypertarget{when-covid-subsided-israel-reopened-its-schools-it-didnt-go-well}{%
\section{When Covid Subsided, Israel Reopened Its Schools. It Didn't Go
Well.}\label{when-covid-subsided-israel-reopened-its-schools-it-didnt-go-well}}

As countries consider back-to-school strategies for the fall, a
coronavirus outbreak at a Jerusalem high school offers a cautionary
tale.

\includegraphics{https://static01.nyt.com/images/2020/07/30/world/xxvirus-israel7/merlin_175107777_4e8b4d3c-4cff-4662-9a50-ed36b51f50d8-articleLarge.jpg?quality=75\&auto=webp\&disable=upscale}

\href{https://www.nytimes.com/by/isabel-kershner}{\includegraphics{https://static01.nyt.com/images/2018/10/12/multimedia/author-isabel-kershner/author-isabel-kershner-thumbLarge.png}}\href{https://www.nytimes.com/by/pam-belluck}{\includegraphics{https://static01.nyt.com/images/2018/02/16/multimedia/author-pam-belluck/author-pam-belluck-thumbLarge-v2.png}}

By \href{https://www.nytimes.com/by/isabel-kershner}{Isabel Kershner}
and \href{https://www.nytimes.com/by/pam-belluck}{Pam Belluck}

\begin{itemize}
\item
  Aug. 4, 2020
\item
  \begin{itemize}
  \item
  \item
  \item
  \item
  \item
  \item
  \end{itemize}
\end{itemize}

JERUSALEM --- As the United States and other countries anxiously
consider how to reopen schools, Israel, one of the first countries to do
so, illustrates the dangers of moving too precipitously.

Confident it had beaten the coronavirus and desperate to reboot a
devastated economy, the Israeli government invited the entire student
body back in late May.

Within days, infections were reported at a Jerusalem high school, which
quickly mushroomed into the largest outbreak in a single school in
Israel, possibly the world.

The virus rippled out to the students' homes and then to other schools
and neighborhoods, ultimately infecting hundreds of students, teachers
and relatives.

Other outbreaks forced hundreds of schools to close. Across the country,
tens of thousands of students and teachers were quarantined.

Israel's advice for other countries?

``They definitely should not do what we have done,'' said Eli Waxman, a
professor at the Weizmann Institute of Science and chairman of the team
advising Israel's National Security Council on the pandemic. ``It was a
major failure.''

The lesson, experts say, is that even communities that have gotten the
spread of the virus under control need to take strict precautions when
reopening schools. Smaller classes, mask wearing, keeping desks six feet
apart and providing adequate ventilation, they say, are likely to be
crucial until a vaccine is available.

``If there is a low number of cases, there is an illusion that the
disease is over,'' said Dr. Hagai Levine, a professor of epidemiology at
Hebrew University-Hadassah School of Public Health. ``But it's a
complete illusion.''

``The mistake in Israel,'' he said, ``is that you can open the education
system, but you have to do it gradually, with certain limits, and you
have to do it in a very careful way.''

The United States is facing similar pressures to fully reopen schools,
and President Trump has threatened to withhold funding for districts
that don't reopen. But the U.S. is in a far worse position than Israel
was in May: Israel had fewer than 100 new infections a day then. The
U.S. is now averaging more than 60,000 new cases a day, and some states
continue to set alarming records.

Israel's handling of the pandemic was considered successful at first.
The country of nine million quickly closed its borders, shuttered
schools in mid-March and introduced remote learning for its two million
students. In April,
\href{https://www.nytimes.com/2020/04/07/world/middleeast/coronavirus-passover-israel.html?searchResultPosition=1}{Passover}
and
\href{https://www.nytimes.com/2020/05/15/world/middleeast/ramadan-coronavirus-al-aqsa.html?searchResultPosition=1}{Ramadan}
were celebrated under lockdown.

\includegraphics{https://static01.nyt.com/images/2020/07/30/world/xxvirus-israel-schools3/merlin_171622530_d27665e9-db14-4460-a60f-749bbfa67a76-articleLarge.jpg?quality=75\&auto=webp\&disable=upscale}

By early May, infection rates had fallen from more than 750 confirmed
cases a day to double digits. The youngest students, grades three and
under, and older students taking final exams returned in small groups,
splitting the week to take turns using classrooms.

Then, emboldened by the dropping infection rates, the government
completely reopened schools on May 17, the day a new government was
sworn in.

\hypertarget{latest-updates-global-coronavirus-outbreak}{%
\section{\texorpdfstring{\href{https://www.nytimes.com/2020/08/04/world/coronavirus-cases.html?action=click\&pgtype=Article\&state=default\&region=MAIN_CONTENT_1\&context=storylines_live_updates}{Latest
Updates: Global Coronavirus
Outbreak}}{Latest Updates: Global Coronavirus Outbreak}}\label{latest-updates-global-coronavirus-outbreak}}

Updated 2020-08-05T07:58:24.076Z

\begin{itemize}
\tightlist
\item
  \href{https://www.nytimes.com/2020/08/04/world/coronavirus-cases.html?action=click\&pgtype=Article\&state=default\&region=MAIN_CONTENT_1\&context=storylines_live_updates\#link-762df92}{As
  talks drag on, McConnell signals openness to jobless aid extension,
  and negotiators agree on a deadline.}
\item
  \href{https://www.nytimes.com/2020/08/04/world/coronavirus-cases.html?action=click\&pgtype=Article\&state=default\&region=MAIN_CONTENT_1\&context=storylines_live_updates\#link-1228a480}{Novavax
  sees encouraging results from two studies of its experimental
  vaccine.}
\item
  \href{https://www.nytimes.com/2020/08/04/world/coronavirus-cases.html?action=click\&pgtype=Article\&state=default\&region=MAIN_CONTENT_1\&context=storylines_live_updates\#link-794484ed}{Mississippians
  must now wear masks in public, governor says.}
\end{itemize}

\href{https://www.nytimes.com/2020/08/04/world/coronavirus-cases.html?action=click\&pgtype=Article\&state=default\&region=MAIN_CONTENT_1\&context=storylines_live_updates}{See
more updates}

More live coverage:
\href{https://www.nytimes.com/live/2020/08/04/business/stock-market-today-coronavirus?action=click\&pgtype=Article\&state=default\&region=MAIN_CONTENT_1\&context=storylines_live_updates}{Markets}

In his inaugural speech, Prime Minister Benjamin Netanyahu
\href{https://www.nytimes.com/2020/05/17/world/middleeast/israel-netanyahu-gantz-government.html}{promised
a new budget} that would deliver three things: ``Jobs, jobs, jobs.'' His
new education minister, Yoav Gallant,
\href{https://www.facebook.com/YoavGallant/photos/a.621695154643450/2149523848527232/?type=3\&theater}{said}
that the school system's ``immediate mission'' was to allow parents to
return to work with peace of mind.

Inna Zaltsman, an Education Ministry official, said administrators also
wanted ``to return the children to routine as much as possible, for
their emotional and pedagogic well-being.''

Shopping malls, outdoor markets and gyms had already reopened, and soon
houses of worship, restaurants, bars, hotels and wedding halls did too.
Mr. Netanyahu told Israelis to grab a beer and, while taking
precautions, ``Go out and have a good time.''

In hindsight, that advice
\href{https://www.nytimes.com/2020/07/24/world/middleeast/israel-virus-protests-netanyahu.html?searchResultPosition=2}{was
wildly premature}.

That same day, a mother phoned a teacher at Jerusalem's historic
Gymnasia Ha'ivrit high school. Her son, a seventh-grade student there,
had tested positive for the virus.

By the next day, the school confirmed another case in the ninth grade.
Ultimately, Israeli officials said, 154 students and 26 staff members
were found to be infected.

Image

As Israel began to relax its restrictions in May, street life returned
to Tel Aviv.Credit...Dan Balilty for The New York Times

``There was a general euphoria among the public, a sense that we had
dealt with the first wave well and that it was behind us,'' said Danniel
Leibovitch, Gymnasia's principal. ``Of course, that wasn't true.''

The Education Ministry had issued safety instructions: Masks were to be
worn by students in fourth grade and higher, windows kept open, hands
washed frequently and students kept six feet apart whenever possible.

But in many Israeli schools, where up to 38 children squeeze into
classrooms of about 500 square feet, physical distancing proved
impossible.

Unable to comply with the rules, some local authorities ignored them or
simply decided not to reopen at full capacity.

Then a heat wave hit. Parents complained that it was inhumane to make
children wear masks in steaming classrooms where open windows nullified
the air conditioning.

In response, the government exempted everyone from wearing masks for
four days, and schools shut the windows.

That decision proved disastrous, experts say.

``Instead of canceling school in those days, they just told the kids
`OK, well you have to stay in the class with the air conditioning on and
take your masks off,' so you have no ventilation really,'' said Dr.
Ronit Calderon-Margalit, a professor of epidemiology at Hebrew
University-Hadassah Braun School of Public Health. ``You have the ideal
circumstances for an outbreak.''

The Gymnasia became a petri dish for Covid-19.

Image

In Gymnasia's 90-year-old building, an average of 33 to 34 students
packed each classroom. Credit...Dan Balilty for The New York Times

When the first case was discovered, the student's classmates, teachers
and other contacts were quarantined. After the second case, which was
not directly linked to the first, the school was closed and everyone was
instructed to quarantine for two weeks. All students and staff were
tested, often waiting in line for hours.

About
\href{https://www.eurosurveillance.org/content/10.2807/1560-7917.ES.2020.25.29.2001352?mc_source=MTExMDY2Ojo6OTgxM2NkZDM4OGRjNGFlM2JhY2RhNWIyZTNlODhkOTE6OnYzOjoxNTk2NDc1MjIzOjox\#html_fulltext}{60
percent of infected students were asymptomatic}. Teachers, some of whom
had been teaching multiple classes, suffered the most and a few were
hospitalized, the principal said.

Parents were furious. Oz Arbel told Israel's Army Radio that for a
school project, his daughter's classmates sat at a table and passed
around a cellphone with a teacher who was showing symptoms. His daughter
and wife became infected.

Image

Books wrapped in plastic bags wait in Gymnasia's library for
September.Credit...Dan Balilty for The New York Times

One Gymnasia student, Ofek Amzaleg, told Kan public radio that a teacher
who coughed in class and joked that he didn't have coronavirus was among
those who tested positive. Ofek also became infected.

\href{https://www.nytimes.com/news-event/coronavirus?action=click\&pgtype=Article\&state=default\&region=MAIN_CONTENT_3\&context=storylines_faq}{}

\hypertarget{the-coronavirus-outbreak-}{%
\subsubsection{The Coronavirus Outbreak
›}\label{the-coronavirus-outbreak-}}

\hypertarget{frequently-asked-questions}{%
\paragraph{Frequently Asked
Questions}\label{frequently-asked-questions}}

Updated August 4, 2020

\begin{itemize}
\item ~
  \hypertarget{i-have-antibodies-am-i-now-immune}{%
  \paragraph{I have antibodies. Am I now
  immune?}\label{i-have-antibodies-am-i-now-immune}}

  \begin{itemize}
  \tightlist
  \item
    As of right
    now,\href{https://www.nytimes.com/2020/07/22/health/covid-antibodies-herd-immunity.html?action=click\&pgtype=Article\&state=default\&region=MAIN_CONTENT_3\&context=storylines_faq}{that
    seems likely, for at least several months.} There have been
    frightening accounts of people suffering what seems to be a second
    bout of Covid-19. But experts say these patients may have a
    drawn-out course of infection, with the virus taking a slow toll
    weeks to months after initial exposure. People infected with the
    coronavirus typically
    \href{https://www.nature.com/articles/s41586-020-2456-9}{produce}
    immune molecules called antibodies, which are
    \href{https://www.nytimes.com/2020/05/07/health/coronavirus-antibody-prevalence.html?action=click\&pgtype=Article\&state=default\&region=MAIN_CONTENT_3\&context=storylines_faq}{protective
    proteins made in response to an
    infection}\href{https://www.nytimes.com/2020/05/07/health/coronavirus-antibody-prevalence.html?action=click\&pgtype=Article\&state=default\&region=MAIN_CONTENT_3\&context=storylines_faq}{.
    These antibodies may} last in the body
    \href{https://www.nature.com/articles/s41591-020-0965-6}{only two to
    three months}, which may seem worrisome, but that's perfectly normal
    after an acute infection subsides, said Dr. Michael Mina, an
    immunologist at Harvard University. It may be possible to get the
    coronavirus again, but it's highly unlikely that it would be
    possible in a short window of time from initial infection or make
    people sicker the second time.
  \end{itemize}
\item ~
  \hypertarget{im-a-small-business-owner-can-i-get-relief}{%
  \paragraph{I'm a small-business owner. Can I get
  relief?}\label{im-a-small-business-owner-can-i-get-relief}}

  \begin{itemize}
  \tightlist
  \item
    The
    \href{https://www.nytimes.com/article/small-business-loans-stimulus-grants-freelancers-coronavirus.html?action=click\&pgtype=Article\&state=default\&region=MAIN_CONTENT_3\&context=storylines_faq}{stimulus
    bills enacted in March} offer help for the millions of American
    small businesses. Those eligible for aid are businesses and
    nonprofit organizations with fewer than 500 workers, including sole
    proprietorships, independent contractors and freelancers. Some
    larger companies in some industries are also eligible. The help
    being offered, which is being managed by the Small Business
    Administration, includes the Paycheck Protection Program and the
    Economic Injury Disaster Loan program. But lots of folks have
    \href{https://www.nytimes.com/interactive/2020/05/07/business/small-business-loans-coronavirus.html?action=click\&pgtype=Article\&state=default\&region=MAIN_CONTENT_3\&context=storylines_faq}{not
    yet seen payouts.} Even those who have received help are confused:
    The rules are draconian, and some are stuck sitting on
    \href{https://www.nytimes.com/2020/05/02/business/economy/loans-coronavirus-small-business.html?action=click\&pgtype=Article\&state=default\&region=MAIN_CONTENT_3\&context=storylines_faq}{money
    they don't know how to use.} Many small-business owners are getting
    less than they expected or
    \href{https://www.nytimes.com/2020/06/10/business/Small-business-loans-ppp.html?action=click\&pgtype=Article\&state=default\&region=MAIN_CONTENT_3\&context=storylines_faq}{not
    hearing anything at all.}
  \end{itemize}
\item ~
  \hypertarget{what-are-my-rights-if-i-am-worried-about-going-back-to-work}{%
  \paragraph{What are my rights if I am worried about going back to
  work?}\label{what-are-my-rights-if-i-am-worried-about-going-back-to-work}}

  \begin{itemize}
  \tightlist
  \item
    Employers have to provide
    \href{https://www.osha.gov/SLTC/covid-19/standards.html}{a safe
    workplace} with policies that protect everyone equally.
    \href{https://www.nytimes.com/article/coronavirus-money-unemployment.html?action=click\&pgtype=Article\&state=default\&region=MAIN_CONTENT_3\&context=storylines_faq}{And
    if one of your co-workers tests positive for the coronavirus, the
    C.D.C.} has said that
    \href{https://www.cdc.gov/coronavirus/2019-ncov/community/guidance-business-response.html}{employers
    should tell their employees} -\/- without giving you the sick
    employee's name -\/- that they may have been exposed to the virus.
  \end{itemize}
\item ~
  \hypertarget{should-i-refinance-my-mortgage}{%
  \paragraph{Should I refinance my
  mortgage?}\label{should-i-refinance-my-mortgage}}

  \begin{itemize}
  \tightlist
  \item
    \href{https://www.nytimes.com/article/coronavirus-money-unemployment.html?action=click\&pgtype=Article\&state=default\&region=MAIN_CONTENT_3\&context=storylines_faq}{It
    could be a good idea,} because mortgage rates have
    \href{https://www.nytimes.com/2020/07/16/business/mortgage-rates-below-3-percent.html?action=click\&pgtype=Article\&state=default\&region=MAIN_CONTENT_3\&context=storylines_faq}{never
    been lower.} Refinancing requests have pushed mortgage applications
    to some of the highest levels since 2008, so be prepared to get in
    line. But defaults are also up, so if you're thinking about buying a
    home, be aware that some lenders have tightened their standards.
  \end{itemize}
\item ~
  \hypertarget{what-is-school-going-to-look-like-in-september}{%
  \paragraph{What is school going to look like in
  September?}\label{what-is-school-going-to-look-like-in-september}}

  \begin{itemize}
  \tightlist
  \item
    It is unlikely that many schools will return to a normal schedule
    this fall, requiring the grind of
    \href{https://www.nytimes.com/2020/06/05/us/coronavirus-education-lost-learning.html?action=click\&pgtype=Article\&state=default\&region=MAIN_CONTENT_3\&context=storylines_faq}{online
    learning},
    \href{https://www.nytimes.com/2020/05/29/us/coronavirus-child-care-centers.html?action=click\&pgtype=Article\&state=default\&region=MAIN_CONTENT_3\&context=storylines_faq}{makeshift
    child care} and
    \href{https://www.nytimes.com/2020/06/03/business/economy/coronavirus-working-women.html?action=click\&pgtype=Article\&state=default\&region=MAIN_CONTENT_3\&context=storylines_faq}{stunted
    workdays} to continue. California's two largest public school
    districts --- Los Angeles and San Diego --- said on July 13, that
    \href{https://www.nytimes.com/2020/07/13/us/lausd-san-diego-school-reopening.html?action=click\&pgtype=Article\&state=default\&region=MAIN_CONTENT_3\&context=storylines_faq}{instruction
    will be remote-only in the fall}, citing concerns that surging
    coronavirus infections in their areas pose too dire a risk for
    students and teachers. Together, the two districts enroll some
    825,000 students. They are the largest in the country so far to
    abandon plans for even a partial physical return to classrooms when
    they reopen in August. For other districts, the solution won't be an
    all-or-nothing approach.
    \href{https://bioethics.jhu.edu/research-and-outreach/projects/eschool-initiative/school-policy-tracker/}{Many
    systems}, including the nation's largest, New York City, are
    devising
    \href{https://www.nytimes.com/2020/06/26/us/coronavirus-schools-reopen-fall.html?action=click\&pgtype=Article\&state=default\&region=MAIN_CONTENT_3\&context=storylines_faq}{hybrid
    plans} that involve spending some days in classrooms and other days
    online. There's no national policy on this yet, so check with your
    municipal school system regularly to see what is happening in your
    community.
  \end{itemize}
\end{itemize}

Mr. Leibovitch, the principal, said he had no knowledge of any teacher
coming in with symptoms.

Seeking to contain the contagion, the Education Ministry vowed to shut
any school with even one Covid-19 case. It ultimately closed more than
240 schools and quarantined more than 22,520 teachers and students.

When the school year ended in late June, the ministry said, 977 pupils
and teachers had contracted Covid-19.

But the Health Ministry, lacking the infrastructure and resources, did
not make contact tracing a priority. In the Gymnasia case, Professor
Waxman said, nobody even identified which buses the students had ridden
on to school.

Proms were canceled, but graduating seniors in the central city of
Ra'anana held an underground prom party anyway. Dozens contracted the
virus.

A nursery school teacher, Shalva Zalfreund, 64, sent
a\href{https://www.facebook.com/photo.php?fbid=2694147267485782\&set=a.1636227159944470\&type=3\&theater}{note
to parents} saying she believed she had been infected in her school,
where some parents had sent their children from homes with cases of the
virus. She died in July.

Image

Israeli health officials are divided as to whether the outbreak in
schools spurred Israel's second wave of infections.Credit...Gil
Cohen-Magen/Agence France-Presse --- Getty Images

Outside school walls, the coronavirus returned with a vengeance. Covid
wards that had closed with festive ceremonies in late April began
filling again, with confirmed infections spiraling to about 800 a day by
late June and more than 2,000 a day by late July.

Some blamed the hasty school reopening as a major factor in the second
wave. Siegal Sadetzki, who
\href{https://www.timesofisrael.com/top-health-ministry-official-quits-warns-israel-heading-down-dangerous-path/}{resigned
in frustration} last month as Israel's director of public health
services, wrote that insufficient safety precautions in schools, as well
as large gatherings like weddings, fueled a ``significant portion'' of
second-wave infections.

But others said singling out schools was unfair when the real problem
was that everything reopened too quickly.

``The single super-spreader event in the Gymnasia just happened to be in
a school,'' said Dr. Ran Balicer, an Israeli health care official and
adviser to the prime minister on the pandemic. ``It could have happened
in any other setting.''

Now Israel is confronting the same questions as other countries, trying
to learn from its mistakes in planning for the school year that begins
Sept. 1.

Public health experts worldwide have coalesced around a set of
guidelines for reopening schools.

A major recommendation is to create groups of 10 to 15 students who stay
together in classrooms, at recess and lunchtime, with teachers assigned
to only one group. Each group has minimal contact with other groups,
limiting any spread of infection. And if a case of Covid-19 emerges, one
group can be quarantined at home while others can continue at school.

Other key recommendations include staggering schedules or teaching older
students online, keeping desks several feet apart, sanitizing classrooms
more frequently, providing ventilation and opening windows if possible,
and requiring masks for staff and students old enough to wear them
properly.

Image

Students keep their distance at an elementary school in the central
Israeli town of Pardes Hanna-Karkur in May.Credit...Jack Guez/Agence
France-Presse --- Getty Images

Israel has already moved in that direction.

The government recently appointed a coronavirus czar, Dr. Ronni Gamzu,
who transferred responsibility for virus testing and investigation from
the Health Ministry to the military. ``This is an operation, not
medicine,'' he declared.

On Sunday, the government approved plans for returning only grades two
and lower to school in full-size classes in the fall. Younger children
are less likely to become seriously ill, and some studies have suggested
that they are less likely than adults and teenagers to transmit the
virus to others.

The plans also call for splitting older students into capsules of 18 and
for mostly online instruction for grades five and above. Principals will
have flexibility to adjust their school's policies based on local
conditions.

Even those measures may not be enough.

Menashe Levy, president of the Israeli High School Principals
Association, arranged desks six feet apart in a standard classroom. It
could accommodate 14 students, not 18.

But Israel is plunging ahead. Only one option has been ruled out:
closing the schools.

``This is a long-term pandemic,'' said Dr. Nadav Davidovitch, a pandemic
policy adviser to the government. ``We cannot close schools for a
year.''

Isabel Kershner reported from Jerusalem, and Pam Belluck from the United
States.

Advertisement

\protect\hyperlink{after-bottom}{Continue reading the main story}

\hypertarget{site-index}{%
\subsection{Site Index}\label{site-index}}

\hypertarget{site-information-navigation}{%
\subsection{Site Information
Navigation}\label{site-information-navigation}}

\begin{itemize}
\tightlist
\item
  \href{https://help.nytimes.com/hc/en-us/articles/115014792127-Copyright-notice}{©~2020~The
  New York Times Company}
\end{itemize}

\begin{itemize}
\tightlist
\item
  \href{https://www.nytco.com/}{NYTCo}
\item
  \href{https://help.nytimes.com/hc/en-us/articles/115015385887-Contact-Us}{Contact
  Us}
\item
  \href{https://www.nytco.com/careers/}{Work with us}
\item
  \href{https://nytmediakit.com/}{Advertise}
\item
  \href{http://www.tbrandstudio.com/}{T Brand Studio}
\item
  \href{https://www.nytimes.com/privacy/cookie-policy\#how-do-i-manage-trackers}{Your
  Ad Choices}
\item
  \href{https://www.nytimes.com/privacy}{Privacy}
\item
  \href{https://help.nytimes.com/hc/en-us/articles/115014893428-Terms-of-service}{Terms
  of Service}
\item
  \href{https://help.nytimes.com/hc/en-us/articles/115014893968-Terms-of-sale}{Terms
  of Sale}
\item
  \href{https://spiderbites.nytimes.com}{Site Map}
\item
  \href{https://help.nytimes.com/hc/en-us}{Help}
\item
  \href{https://www.nytimes.com/subscription?campaignId=37WXW}{Subscriptions}
\end{itemize}
