Sections

SEARCH

\protect\hyperlink{site-content}{Skip to
content}\protect\hyperlink{site-index}{Skip to site index}

\href{https://www.nytimes.com/section/world/middleeast}{Middle East}

\href{https://myaccount.nytimes.com/auth/login?response_type=cookie\&client_id=vi}{}

\href{https://www.nytimes.com/section/todayspaper}{Today's Paper}

\href{/section/world/middleeast}{Middle East}\textbar{}Blasts Rock
Beirut, Killing Dozens and Wounding Thousands

\url{https://nyti.ms/3fnBUtO}

\begin{itemize}
\item
\item
\item
\item
\item
\end{itemize}

Advertisement

\protect\hyperlink{after-top}{Continue reading the main story}

Supported by

\protect\hyperlink{after-sponsor}{Continue reading the main story}

\hypertarget{blasts-rock-beirut-killing-dozens-and-wounding-thousands}{%
\section{Blasts Rock Beirut, Killing Dozens and Wounding
Thousands}\label{blasts-rock-beirut-killing-dozens-and-wounding-thousands}}

The cause remained unclear hours later. But officials in the Lebanese
capital said 2,750 tons of highly explosive ammonium nitrate had been
stored in a depot at the center of the explosion.

\includegraphics{https://static01.nyt.com/images/2020/09/04/world/04lebanon-ledeall-1sub/merlin_175322382_cb808571-ba69-485d-9750-68044814a161-articleLarge.jpg?quality=75\&auto=webp\&disable=upscale}

\href{https://www.nytimes.com/by/ben-hubbard}{\includegraphics{https://static01.nyt.com/images/2018/10/10/multimedia/author-ben-hubbard/author-ben-hubbard-thumbLarge.png}}

By \href{https://www.nytimes.com/by/ben-hubbard}{Ben Hubbard}

\begin{itemize}
\item
  Published Aug. 4, 2020Updated Aug. 5, 2020, 2:26 a.m. ET
\item
  \begin{itemize}
  \item
  \item
  \item
  \item
  \item
  \end{itemize}
\end{itemize}

BEIRUT, Lebanon --- The blasts came within seconds of each other.

First, an explosion in Beirut's port, possibly from a fireworks
warehouse, sent a plume of smoke billowing over the capital skyline
early Tuesday evening.

Then a much larger explosion from a building nearby shot a chrysanthemum
of orange and red smoke into the air followed by a massive shock wave of
whitish dust and debris that rose hundreds of feet and spread out for
blocks.

The seaside capital rocked like an earthquake. Cars tumbled upside down
and bricks rained down from apartment buildings. Glass flew out of
windows miles away and roofs collapsed.

\href{https://www.nytimes.com/interactive/2020/08/04/world/middleeast/beirut-explosion-damage.html}{}

\includegraphics{https://static01.nyt.com/images/2020/08/04/us/beirut-explosion-damage-promo-1596586440536/beirut-explosion-damage-promo-1596586440536-articleLarge-v2.jpg}

\hypertarget{mapping-the-damage-from-the-beirut-explosions}{%
\subsection{Mapping the Damage From the Beirut
Explosions}\label{mapping-the-damage-from-the-beirut-explosions}}

Damage was seen at least two miles from the explosions, encompassing an
area with more than 750,000 residents.

The wounded stumbled through debris-choked streets to hospitals, only to
be turned away in some cases because the hospitals, already reeling from
the coronavirus pandemic, were overwhelmed.

By late evening, the Health Ministry said, more than 70 people were dead
and at least 3,000 wounded in the worst carnage to hit the city in more
than a decade. For many of Lebanon's 5.2 million people, the images that
ricocheted through social media recalled the scenes of urban destruction
from the long-troubled country's decades of war.

\includegraphics{https://static01.nyt.com/images/2020/08/04/world/04lebanon-ledeall2/merlin_175301955_99c28328-5328-4492-b6d6-de2dfbaac092-articleLarge.jpg?quality=75\&auto=webp\&disable=upscale}

It was unclear exactly what caused the explosions, but Prime Minister
Hassan Diab said an estimated 2,750 tons of highly explosive ammonium
nitrate, commonly used in fertilizer and bombs, had been stored in a
depot at the port for six years.

``As head of the government, I will not relax until we find the
responsible party for what happened, hold it accountable and apply the
most serious punishments against it,'' Mr. Diab said.

Maj. Gen. Abbas Ibrahim, the head of Lebanon's general security service,
told the state-run news agency that ``highly explosive materials'' had
been seized by the government years ago and were stored near the blast
site. Although the thought of an attack was in the front of everyone's
mind, he warned against getting ``ahead of the investigation'' and
speculating about a terrorist act.

Image

Smoke rising from the scene of an explosion in Beirut on
Tuesday.Credit...Anwar Amro/Agence France-Presse --- Getty Images

In a televised statement, Mr. Diab hinted that neglect had led to the
blast and said the government would hold those responsible to account.

``Facts on this dangerous depot, which has existed since 2014 or the
past six years, will be announced,'' Mr. Diab said. ``Those responsible
will pay a price for this catastrophe.''

Mr. Diab said that Wednesday would be a national day of mourning. The
governor of Beirut, Marwan Abboud, speaking on television, called it ``a
national catastrophe'' and burst into tears.

At a briefing in Washington, President Trump suggested the explosion was
the result of an attack. He said he consulted with military generals and
that ``they seem to think it's an attack, a bomb of some kind.''

However, a senior U.S. official said, ``Everything I'm seeing thus far
points to a tragic accident.''

The explosion was the latest in a string of events in recent months that
have plunged Lebanon, a sectarian-based democracy with a long history of
civil strife, into simultaneous political and economic crises.

Since last fall, waves of protests calling for the ouster of the
country's political class for decades of mismanagement and corruption
have shut down cities and towns across the country, and a severe
financial crisis has eroded the value of the Lebanese pound by 80
percent, plunging many Lebanese into poverty.

More recently, the number of new coronavirus cases has begun to rise
quickly, raising fears that a new government-imposed lockdown could
further damage the economy. Many of the country's hospitals were already
on the verge of capacity.

Image

Wreckage along a road in Beirut.Credit...Hassan Ammar/Associated Press

Lebanon's last major war was in 2006, between Israel and Hezbollah, the
Iranian-backed militant group and political party that remains committed
to the destruction of the Jewish state. In recent years, Israel has
launched frequent airstrikes on Hezbollah targets in neighboring Syria,
but has mostly avoided bombing it in Lebanon to avoid setting off a
cycle of retaliation that could lead to a new war.

Tensions between Hezbollah and Israel have flared lately on Lebanon's
southern border, leading many Lebanese to speculate that Israel had
targeted materials connected to Hezbollah and hidden in Beirut's port.

An Israeli official said that Israel ``had nothing to do with the
incident'' on Tuesday.

The blasts emanated from Beirut's port but were felt as far away as
Cyprus, more than 180 miles to the west. They ravaged Beirut's downtown
business district, a nearby waterfront full of restaurants and
nightclubs, and a number of crowded residential neighborhoods in the
city's eastern and predominantly Christian half.

Nearly all the windows along one popular commercial strip had been blown
out and the street was littered with glass, rubble and cars that had
slammed into each other after the blast.

Abbas Saleh, a 28-year-old driver, was in his car when he saw a flash
and heard a boom, and his windshield shattered.

``You would never think it was an explosion,'' he said. ``More like
missiles coming down on us.''

He ran out of his car and began helping Red Cross workers carry the dead
and wounded.

All around, families struggled to get wounded relatives out of their
buildings so they could be piled into ambulances or onto the backs of
motor scooters. The Lebanese Red Cross said that every available
ambulance from North Lebanon, Bekaa and South Lebanon was dispatched to
Beirut, but so many roads had been rendered impassible that many of the
wounded had to walk to the hospital themselves.

Space, medics and supplies were lacking. Hospitals in the hardest-hit
areas were heavily damaged, with at least one shutting down altogether
and others treating bleeding patients in their parking lots.

St. George Hospital in central Beirut, one of the city's biggest, was so
severely damaged that it had to send patients elsewhere.

``My friends, my friends,'' Dr. Joseph Haddad, the hospital's director
of intensive care, said in a phone call. ``This is Joseph Haddad calling
you from St. George Hospital. There is no St. George Hospital anymore.
It's fallen, it's on the floor,'' Dr. Haddad says, as broken glass is
heard crackling underfoot. ``It's all destroyed. All of it. Pray to God,
pray to God.''

At Bikhazi Medical Group hospital in the center of Beirut, wounded
patients streamed into a damaged hospital.

``The door to the entrance of the hospital is completely shattered,''
said Rima Azar, the hospital director and co-owner. ``The full ceiling
fell on some patients in some rooms. The pressure was horrific. We heard
a boom, then everything was shaking. There was a second blow that was
super loud. Everything was falling from desks, from shelves.''

The 60-bed hospital treated 500 patients in the hours after the blast,
she said.

Another hospital farther out received so many patients that medics lined
them on the floor and in hallways. Those with non-life-threatening
injuries had them cleaned and stapled shut before being sent on their
way.

Image

Emergency workers and civilians at the site of the explosion in Beirut
on Tuesday.Credit...Anwar Amro/Agence France-Presse --- Getty Images

It was unclear how the disaster would affect the country's tense
political situation. Many Lebanese are already fed up with a political
class they feel has looted the country for years, leaving it virtually
bankrupt and with a collapsing currency. Greater anger would likely
follow should it turn out that the blast was yet another example of
governmental neglect.

When the explosion struck, meetings were in full swing less than a mile
away, at the hillside headquarters of the Kataeb Party, a Christian
political group that was once one of Lebanon's most powerful.

The blast shook the building so badly that party members thought a bomb
had gone off inside. The party's general secretary, Nazar Najarian, was
killed by falling debris.

``He had been through explosions, assassination attempts, wars with the
Palestinians and Syrians,'' said
\href{https://www.facebook.com/EliasHankach2018/}{Elias Hankach}, a
Kataeb member of Parliament. ``Our headquarters looks like a bomb went
off inside. The inside is a mess, it's madness.''

He said the party was waiting for clarity on whether the blast was an
attack, the kind of crude tool used for decades to shape Lebanon's
political landscape, or just an accident resulting from mismanagement.
If it turned out to be accidental, he said, then the disaster is not
particularly surprising, the product of ``cumulative nonchalance at all
levels.''

``Whether you talk about the economy, safety standards, the port, the
corruption --- none of the country's issues have had a serious attempt
at resolution,'' Mr. Hankach said. ``We are living in this doomed
management of the country.''

Image

A highway near the blast site on Wednesday.Credit...Anwar Amro/Agence
France-Presse --- Getty Images

Hwaida Saad contributed reporting from Beirut, Nada Rashwan and Mona
El-Naggar from Cairo, Maria Abi-Habib from Los Angeles, Alan Yuhas from
Philadelphia, Adam Rasgon and Ronen Bergman from Tel Aviv, Rick
Gladstone from Eastham, Mass., Eric Schmitt in Washington, and Richard
Pérez-Peña from New York.

Advertisement

\protect\hyperlink{after-bottom}{Continue reading the main story}

\hypertarget{site-index}{%
\subsection{Site Index}\label{site-index}}

\hypertarget{site-information-navigation}{%
\subsection{Site Information
Navigation}\label{site-information-navigation}}

\begin{itemize}
\tightlist
\item
  \href{https://help.nytimes.com/hc/en-us/articles/115014792127-Copyright-notice}{©~2020~The
  New York Times Company}
\end{itemize}

\begin{itemize}
\tightlist
\item
  \href{https://www.nytco.com/}{NYTCo}
\item
  \href{https://help.nytimes.com/hc/en-us/articles/115015385887-Contact-Us}{Contact
  Us}
\item
  \href{https://www.nytco.com/careers/}{Work with us}
\item
  \href{https://nytmediakit.com/}{Advertise}
\item
  \href{http://www.tbrandstudio.com/}{T Brand Studio}
\item
  \href{https://www.nytimes.com/privacy/cookie-policy\#how-do-i-manage-trackers}{Your
  Ad Choices}
\item
  \href{https://www.nytimes.com/privacy}{Privacy}
\item
  \href{https://help.nytimes.com/hc/en-us/articles/115014893428-Terms-of-service}{Terms
  of Service}
\item
  \href{https://help.nytimes.com/hc/en-us/articles/115014893968-Terms-of-sale}{Terms
  of Sale}
\item
  \href{https://spiderbites.nytimes.com}{Site Map}
\item
  \href{https://help.nytimes.com/hc/en-us}{Help}
\item
  \href{https://www.nytimes.com/subscription?campaignId=37WXW}{Subscriptions}
\end{itemize}
