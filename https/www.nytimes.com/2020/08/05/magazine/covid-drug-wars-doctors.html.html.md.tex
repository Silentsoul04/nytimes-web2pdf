Sections

SEARCH

\protect\hyperlink{site-content}{Skip to
content}\protect\hyperlink{site-index}{Skip to site index}

The Covid Drug Wars That Pitted Doctor vs. Doctor

\href{https://nyti.ms/3a18XTw}{https://nyti.ms/3a18XTw}

\begin{itemize}
\item
\item
\item
\item
\item
\item
\end{itemize}

\href{https://www.nytimes.com/news-event/coronavirus?action=click\&pgtype=Article\&state=default\&region=TOP_BANNER\&context=storylines_menu}{The
Coronavirus Outbreak}

\begin{itemize}
\tightlist
\item
  live\href{https://www.nytimes.com/2020/08/08/world/coronavirus-updates.html?action=click\&pgtype=Article\&state=default\&region=TOP_BANNER\&context=storylines_menu}{Latest
  Updates}
\item
  \href{https://www.nytimes.com/interactive/2020/us/coronavirus-us-cases.html?action=click\&pgtype=Article\&state=default\&region=TOP_BANNER\&context=storylines_menu}{Maps
  and Cases}
\item
  \href{https://www.nytimes.com/interactive/2020/science/coronavirus-vaccine-tracker.html?action=click\&pgtype=Article\&state=default\&region=TOP_BANNER\&context=storylines_menu}{Vaccine
  Tracker}
\item
  \href{https://www.nytimes.com/interactive/2020/world/coronavirus-tips-advice.html?action=click\&pgtype=Article\&state=default\&region=TOP_BANNER\&context=storylines_menu}{F.A.Q.}
\item
  \href{https://www.nytimes.com/live/2020/08/07/business/stock-market-today-coronavirus?action=click\&pgtype=Article\&state=default\&region=TOP_BANNER\&context=storylines_menu}{Markets
  \& Economy}
\end{itemize}

\includegraphics{https://static01.nyt.com/images/2020/08/09/magazine/09mag-Doctors-05/09mag-Doctors-05-articleLarge.jpg?quality=75\&auto=webp\&disable=upscale}

\hypertarget{the-covid-drug-wars-that-pitted-doctor-vs-doctor}{%
\section{The Covid Drug Wars That Pitted Doctor vs.
Doctor}\label{the-covid-drug-wars-that-pitted-doctor-vs-doctor}}

How much freedom should front-line clinicians have in treating Covid-19
patients with unproven drugs? The question opened up a civil war in some
hospitals.

In the absence of conclusive research for Covid-19 treatments, many
doctors are having to rely on their experience to make judgment calls
about medications.Credit...Adam Ferguson for The New York Times

Supported by

\protect\hyperlink{after-sponsor}{Continue reading the main story}

By \href{https://www.nytimes.com/by/susan-dominus}{Susan Dominus}

\begin{itemize}
\item
  Published Aug. 5, 2020Updated Aug. 8, 2020, 11:46 a.m. ET
\item
  \begin{itemize}
  \item
  \item
  \item
  \item
  \item
  \item
  \end{itemize}
\end{itemize}

\hypertarget{listen-to-this-article}{%
\subsubsection{Listen to This Article}\label{listen-to-this-article}}

Audio Recording by Audm

\emph{To hear more audio stories from publishers like The New York
Times, download}
\href{https://www.audm.com/?utm_source=nytmag\&utm_medium=embed\&utm_campaign=doctor_vs_doctor_dominus}{\emph{Audm
for iPhone or Android}}\emph{.}

Mangala Narasimhan, an intensive-care-unit doctor, started feeling
impatient soon after the start of a meeting she attended at Long Island
Jewish Medical Center on May 13. She wanted to get back to the unit, but
instead she was sitting in a conference room with about a dozen
colleagues. By then, the surge of Covid-19 cases, the waves of suffering
that had crashed down on her hospital for months, was beginning,
miraculously, to recede. The throngs of out-of-town health care workers
who had come to New York City to help were also diminishing, heading
home to regions whose own times would come. Narasimhan and her team now
had fewer hands to oversee new patients coming in and the long-suffering
ones on ventilators who were still in need of meticulous care. Long
Island Jewish, in Queens, had, at the time, treated more Covid-19
patients than any other hospital in the country; the doctors there were
still weary, still battered, their energy and time in need of careful
rationing.

Narasimhan, who was in charge of more than 20 I.C.U.s across the
Northwell Health system, knew heading into the meeting that it might be
tense. Adey Tsegaye, a pulmonary-critical-care doctor who was calling in
remotely, shared some of Narasimhan's concerns. The meeting's agenda
included time for remarks from Alex Spyropoulos, a lead researcher at
the Feinstein Institutes for Medical Research --- the research arm of
Northwell --- who was running a clinical trial. The research was trying
to determine whether a standard dose of an anticoagulant or a higher
dose yielded better outcomes for Covid-19 patients who were already on
oxygen or a ventilator and were at high risk of organ failure and
clotting.

A doctor on Narasimhan's unit had recently been at odds with a member of
Spyropoulos's research team. Stella Hahn, a pulmonary-critical-care
doctor, arrived at work the day before the meeting to find that a
Covid-19 patient had gone into cardiac arrest. She knew that the patient
was enrolled in the clinical trial and had been randomly assigned to
receive either the standard dose of the anticoagulant or the higher one.
As is always the case in the most rigorous trials, neither the patient
nor Hahn was supposed to know to which group this woman belonged.
Double-blind, randomized, controlled trials --- R.C.T.s --- are
considered the gold standard in research because they do not allow
findings to be muddied by any individual doctor's biases or assumptions.
But Hahn believed that the patient's condition now called for the higher
dose, which could potentially require the patient's removal from the
trial.

\includegraphics{https://static01.nyt.com/images/2020/08/09/magazine/09mag-Doctors-06/09mag-Doctors-06-articleLarge.jpg?quality=75\&auto=webp\&disable=upscale}

Word made it back to a doctor working with Spyropoulos, and that doctor
called Hahn to urge her to reconsider, or at least to get more tests
before acting. They exchanged heated words, as the colleague implored
her to stay the course. Hahn pushed back: She had to rely on her
clinical judgment and believed that it was unethical to wait for more
information. How could researchers dictate care to a doctor right there
at the bedside, especially when a patient's condition was so dire?

The point of contention would be discussed at the May 13 meeting. Dozens
of doctors from the Northwell system videoconferenced in, including
Spyropoulos, who was seated in his home in Westchester. Hahn's
colleagues, a tightknit unit who had seen one another through so much,
sat together in the conference room, occasionally checking their phones
or exchanging glances as the meeting went on. As Spyropoulos recalls, he
talked to the group about the importance of high-quality, randomized
trials in making scientific progress, and the risks of trying
experimental treatments without them. ``I stressed to the group that we
should not abandon this principle, even in the very stressful
environment of a pandemic that was overwhelming our hospitals at
Northwell,'' he said. Relying on gut instinct rather than evidence, he
told them, was essentially ``witchcraft.''

For Tsegaye, the word landed like a blow. ``There was a chill in the
air,'' said Tsegaye, who registered it even by videoconference.
``Followed by rapid backpedaling.'' Spyropoulos quickly explained that
he had so much respect for what those doctors had done --- he had not
been in those critical-care units, in the emergency room, which he knew
were unlike any other he had ever experienced. ``But it was like a
retraction sent to the newspaper the next day,'' Tsegaye said. ``The
headline says it all. The retraction the next day? It doesn't have the
same impact.''

In the days to come, whenever Tsegaye thought about what Spyropoulos
said in that meeting, she felt appalled all over again. She knew that
she had never extended herself on behalf of her patients the way she had
since March. She kept flashing back to a day when she was told that a
ventilated patient's endotracheal tube had fallen out, a situation that
can be fatal for the patient and is also dangerous for the physician:
Replacing it requires the doctor to come into close contact with the
patient's breath. Tsegaye was putting on her N95 mask to enter the
patient's room when its elastic snapped in two. There was no time to go
to the supply area to get a new mask. What was the right thing to do?
With a sense of dread, she found her feet and moved toward the patient's
room. As she prepared to enter, one of her fellows, whose mask was
intact, told her to leave --- she could manage it on her own.

Looking back, Tsegaye felt that the agony of making those kinds of
decisions all day long compounded the grief she felt while treating so
many patients she could not help. ``These are the decisions we have had
to face,'' Tsegaye said. ``For someone like me, who had been in that
situation, to have someone tell you that you have been practicing
witchcraft is kind of giving no value to the sacrifice that I have made
--- that my colleagues have made.''

Image

Stella Hahn, a pulmonary-critical-care doctor at Long Island Jewish
Medical Center.Credit...Adam Ferguson for The New York Times

Image

Mangala Narasimhan, a doctor who is in charge of intensive-care units
throughout the Northwell Health system.Credit...Adam Ferguson for The
New York Times

\textbf{As doctors face new} spikes of Covid-19 cases around the
country, they are also confronting a harsh reality: The virus's deadly
secrets remain largely intact. The medical community now has some
research-backed drug treatments --- remdesivir, an antiviral drug found
to shorten hospital stays, and dexamethasone, a cheap, readily available
steroid that seems to cut deaths of patients on ventilators by a third.
But six months after the first patient tested positive on the West
Coast, there is still no treatment that reliably slows progression of
the illness, much less a cure. In July, the number of patients dying in
this country topped 1,000 five days in a row, according to the Covid
Tracking Project.

In these early months, doctors have faced two unknowns in trying to
fight the devastation. The first is the virus itself: deadly, contagious
and entirely novel. The standard of care for most intractable illnesses
develops over years, as doctors build a body of research that tests
various theories, compares and contrasts dosages, measures one drug's
power against another. Here doctors were starting from scratch: Any
treatment protocol beyond supportive care --- oxygen, hydration,
antibiotics and ventilation --- was conjecture. The second, equally
novel challenge has been the sheer scale of the outbreak. Few doctors in
this country had encountered the overwhelming volume of patients, the
sense of helplessness, the exhaustion and the desperation to save lives.
Hospital administrators found themselves plunging headlong into making
difficult decisions in the absence of strong, unifying federal guidance.
Most did so without the benefit of perfectly parallel case studies or
personal experience in hospitals so overrun by suffering.

When there is no precedent, when there is an information vacuum,
decisions are inevitably subject to challenge. In an already heated
environment, some of the worst of the tensions played out between
research-oriented doctors and those who saw themselves primarily as
clinicians. Many treating patients on the floor considered it axiomatic
that, with so many dying so fast and so little to go on, they would rely
on their experience to make judgment calls about treatment options. They
would try using medications that had been approved for other illnesses
but not yet for this one --- what the medical community calls off-label
uses --- if they felt they had good reasons to do so. They would take
into consideration any information that was available: the observations
of doctors in Milan and China, conversations among doctors in WhatsApp
group texts and in Covid-19 physician Facebook groups, tidbits of
research that made medical sense but had not yet been peer-reviewed.

Image

Remdesivir, an antiviral drug found to shorten hospital
stays.Credit...Adam Ferguson for The New York Times

Other clinicians, and especially doctors more heavily involved in
research, were frustrated that many of their colleagues were not
sufficiently invested in the importance of empirical research to figure
out which treatments worked best and were safest. Kevin Tracey,
president of the Feinstein Institutes, tried to emphasize to the doctors
affiliated with the Northwell hospital system that if they were going to
try drugs off-label, they should always be doing so in the context of a
clinical research trial: The drug might help some patients but could
hurt even more of them. If that was the case, it was better to know than
to operate out of a mix of hope and conviction. He understood, he said,
the impulse for doctors to try drugs off-label out of compassion --- and
the ``raw emotion of humans trying to help each other survive and not
knowing what to do.'' But he did not approve of it. ``Emotions cannot
carry the day,'' he said. ``You need evidence-based medicine, and you
need clinical trials. You don't make an exception in the middle of a
pandemic.''

\hypertarget{latest-updates-the-coronavirus-outbreak}{%
\section{\texorpdfstring{\href{https://www.nytimes.com/2020/08/07/world/covid-19-news.html?action=click\&pgtype=Article\&state=default\&region=MAIN_CONTENT_1\&context=storylines_live_updates}{Latest
Updates: The Coronavirus
Outbreak}}{Latest Updates: The Coronavirus Outbreak}}\label{latest-updates-the-coronavirus-outbreak}}

Updated 2020-08-08T12:04:28.992Z

\begin{itemize}
\tightlist
\item
  \href{https://www.nytimes.com/2020/08/07/world/covid-19-news.html?action=click\&pgtype=Article\&state=default\&region=MAIN_CONTENT_1\&context=storylines_live_updates\#link-1f86d03a}{As
  the U.S. relief talks falter again, Trump says he is prepared to act
  on his own.}
\item
  \href{https://www.nytimes.com/2020/08/07/world/covid-19-news.html?action=click\&pgtype=Article\&state=default\&region=MAIN_CONTENT_1\&context=storylines_live_updates\#link-3f64a70a}{Cuomo
  says N.Y. schools can reopen in-person but leaves it up to districts
  to determine if, when and how.}
\item
  \href{https://www.nytimes.com/2020/08/07/world/covid-19-news.html?action=click\&pgtype=Article\&state=default\&region=MAIN_CONTENT_1\&context=storylines_live_updates\#link-14e70066}{Thousands
  of cases went unreported in California when a computer server failed.}
\end{itemize}

\href{https://www.nytimes.com/2020/08/07/world/covid-19-news.html?action=click\&pgtype=Article\&state=default\&region=MAIN_CONTENT_1\&context=storylines_live_updates}{See
more updates}

More live coverage:
\href{https://www.nytimes.com/live/2020/08/07/business/stock-market-today-coronavirus?action=click\&pgtype=Article\&state=default\&region=MAIN_CONTENT_1\&context=storylines_live_updates}{Markets}

Ethan Weiss, a cardiologist at the University of California, San
Francisco, who specializes in metabolic research, spent two weeks
treating patients at a hospital in New York and was also distressed by
how quickly doctors were trying untested therapies outside clinical
trials. ``I mean, it felt like it wasn't even World War I medicine,'' he
said. ``It was almost like Civil War-level medicine.'' He asked that the
name of that New York hospital be withheld out of respect for his
colleagues, whom he knows were not only risking their lives but were
also overwhelmed by their clinical demands and had no research to rely
on. He nonetheless was surprised to see many of them making decisions
``based on the sort of opinion or written protocol of one or a couple of
people that was based on kind of nothing that I could see, other than
just, `This seems like a good idea.'''

Many clinicians on the ground felt the urgency of treating the hundreds
of patients dying in front of them; researchers, with their literal and
intellectual distance from the I.C.U., were pressing them to think about
the thousands of patients who were sure to follow --- to slow down long
enough to build a body of evidence that they knew with more certainty
could help. The tensions between these two ways of thinking about
medicine have always existed. But during the early months of the
pandemic, the disagreements --- what one critical-care doctor called, on
his well-read blog, the profession's ``intellectual food fight'' ---
provided another layer of painful stress to some doctors already near
their limits. ``It became like Republicans and Democrats,'' said Pierre
Kory, a critical-care doctor who faced that tension himself at the
University of Wisconsin Hospital and Clinics. ``The two sides can't talk
to each other.''

Image

Dexamethasone is a cheap, readily available steroid that seems to cut
deaths of patients on ventilators by a third.Credit...Adam Ferguson for
The New York Times

\textbf{As they prepare} to evaluate a given medication or procedure,
researchers are expected to approach their task with a certain neutral
mind-set. The official term for that stance sounds both scientific and
strangely poetic: ``clinical equipoise.'' It's a point at which a
doctor's curiosity is greater than her conviction that any one result is
the most likely one. Clinical equipoise is an elegant characterization
of a humble admission: \emph{I have no idea which of these two choices
is better.}

Equipoise gave way to unbridled enthusiasm among some physicians at
Lenox Hill Hospital on the Upper East Side of New York in April when the
city was in the thick of the surge. Many doctors there believed they
were seeing great results by providing tocilizumab, an anti-inflammatory
drug that tamps down the autoimmune response and is used for rheumatoid
arthritis. The doctors were prescribing the drug, sometimes in
conjunction with a steroid, to Covid-19 patients, particularly those who
were not yet on ventilators but whose blood tests suggested that they
were about to take a turn for the worse. In using it, doctors hoped to
stave off
\href{https://www.nytimes.com/2020/04/01/health/coronavirus-cytokine-storm-immune-system.html}{what's
known as a cytokine storm}, a potentially deadly immune-system
overreaction in which a torrent of cytokines --- proteins that can
trigger infection-fighting forces --- is released. Tocilizumab, which
blocks the pathway of a cytokine called IL-6, might prevent that deadly
storm from gathering force. But any anti-inflammatory carries risk,
because in fighting inflammation, it can also hamper the body's ability
to clear the primary infection or others that follow; tocilizumab is
also thought to carry some elevated risk of anaphylactic shock and
lower-intestinal perforation.

Image

Tocilizumab, which might help prevent a potentially deadly immune-system
overreaction but carries the risk of hampering the body's ability to
clear infections.Credit...Adam Ferguson for The New York Times

Patients with extreme flu in the I.C.U. sometimes received tocilizumab;
it is also used to treat cytokine storms that some cancer patients
experience as a side effect of treatment. Doctors at Lenox Hill did not
believe it was a leap to think that the drug could address cytokine
storms in Covid-19 patients. They knew that doctors in Milan were
leaning heavily on the drug; they were in conversation with doctors at
Yale New Haven Health, considered a fortress of research-heavy medicine,
which also incorporated tocilizumab into their protocol. In addition,
some small studies showed support for the drug's effectiveness, though
none were randomized, controlled trials.

``I understand that it has never been trialed,'' John Boockvar, a
neurosurgeon at Lenox Hill who is affiliated with the Feinstein
Institutes, told me in late April. (Boockvar is one of the doctors
featured on the documentary series ``Lenox Hill.'') ``But there is
clearly enough data to support its use.'' The doctors at Lenox Hill had
also briefly participated in a randomized, controlled trial for another
drug with a similar mechanism, called sarilumab. But to Boockvar,
enrolling a patient in that trial, which might result in a patient
receiving a placebo, posed an ethical challenge when he could simply
prescribe tocilizumab --- doctors refer to it as toci --- instead. In
April, he learned that Massachusetts General Hospital was starting a
randomized, controlled trial for tocilizumab. ``If that was \emph{my}
loved one,'' he said, imagining a family member who might receive a
placebo in that trial, ``I'd be upset. I'd think, Why am I doing this?
If it's an off-label use with an approved drug --- give the damn drug to
everybody.''

At Long Island Jewish, some doctors who were hearing about the drug from
colleagues at Lenox Hill, a part of the Northwell Health consortium,
started clamoring for liberal access to it. And yet sometimes, when
doctors placed orders with the hospital pharmacist, their prescriptions
were declined; those patients didn't meet criteria Northwell had
established for administering tocilizumab, which was in short supply.
Physicians were frustrated that patients who they believed would benefit
from the drug could not receive it. Northwell wanted to be conservative
about the off-label use of drugs outside clinical trials. ``There's no
proof that \emph{anything} works!'' Tsegaye thought at the time.
``\emph{Everything} is experimental!'' As for enrolling patients in a
trial, as overwhelmed as she was, she hardly felt she was in a position
to take that on.

Tsegaye's supervisor, Narasimhan, also knew researchers were concerned
that in prescribing tocilizumab so readily, physicians were possibly
hampering enrollment in the trial underway at her hospital for sarilumab
--- a patient who received tocilizumab could not also receive sarilumab.
She and her team did not prioritize the trials, she said; they wanted to
provide the drugs they thought were needed. ``We've always been allowed
to choose treatment, right or wrong, based on what we thought was
best,'' Narasimhan said in May. ``And that was gone. It was hard.''

In addition to fighting resistance from their administrators, the
doctors were sometimes also at odds with their colleagues, especially
infectious-disease doctors, many of whom believed that
anti-inflammatories like tocilizumab and steroids could do more harm
than good. ``You're killing these patients,'' one infectious-disease
doctor told Hahn at Long Island Jewish.

Image

Adey Tsegaye, a pulmonary-critical-care doctor at Long Island Jewish
Medical Center in Queens.Credit...Adam Ferguson for The New York Times

In the Mount Sinai Health System, tocilizumab was also in demand.
Administrators felt the stress of making decisions in the absence of
clear data. Judith Aberg, the chief of the division of infectious
diseases for Mount Sinai, fielded demands from doctors working on wards
who wanted to use tocilizumab, early and often. ``I have to give her
credit; she was single-handedly fighting off a lot of pressure from
hematologists,'' said Keren Osman, a Mount Sinai oncologist and
hematologist who was on some of those calls. As experts in blood cancers
and diseases, hematologists had experience working with tocilizumab to
treat cytokine storms that were a side effect of some cancer treatments.
``She was saying, `I'm not comfortable just giving patients willy-nilly
anything we have --- we don't know.'''

Patients and their families, who heard through the news media about the
drug, also started to demand it, even for Covid-19 patients whose
inflammatory markers were normal. ``People were calling for us to give
it, just to give it, because there were no other therapies,'' Aberg
said. At first, a medical team that included Aberg agreed to put some
patients who were on ventilators on the drug --- in those patients, it
was obvious that systemic inflammation was already evident; also, the
closer the patient was to dying, the more the risk seemed justified.
Eventually, the thinking at Aberg's hospitals and at others evolved to
favor use of the drug earlier, before systemic inflammation did so much
damage that the patient was already clinging to life.

By May, doctors at Long Island Jewish and Mount Sinai had stopped
pressing for tocilizumab --- if it was effective, it was not such a
miracle drug that they could see its effects clearly. Many had started
to pin their hopes instead on convalescent plasma, another experimental
treatment in which sick patients are given plasma from recovered
patients with antibodies, though its effectiveness is still unknown.
``We did rush,'' Aberg says now. ``I mean, we were pushed. We were
grasping for anything that we could possibly do.''

In early July, the drug company studying sarilumab, the drug similar to
tocilizumab, announced that it was halting its trial; researchers found,
as Aberg put it, ``nada.'' A few weeks later, the pharmaceutical company
Roche announced preliminary results of a tocilizumab trial that was run
on Covid patients with pneumonia. The drug's effects were no better than
a placebo. By then, Narasimhan was also starting to see preliminary
reports of other research that showed the drug could, in fact, be
dangerous, increasing the risk of fatal secondary and fungal infections.

``My take-home is that I wish we had done more randomized, controlled
trials so we could have some real answers, so that we could tell Florida
and Texas, `This works, and this doesn't work,''' Narasimhan, who is now
in charge of intensive-care units throughout the Northwell Health
system, told me in July. ``We could have had so many more answers in a
way that was meaningful. We had this fixation that all these drugs were
curative. And they weren't.''

Image

Hydroxychloroquine, the drug that President Trump claimed was a ``game
changer'' in mid-March.Credit...Adam Ferguson for The New York Times

\textbf{The story of} hydroxychloroquine will most likely be recalled as
a classic medical parable of the pandemic. It was a drug that seemed so
promising that physicians were desperate to use it, and researchers were
equally driven to see if it actually delivered the hoped-for results. In
the end, the enthusiasm of the first camp most likely slowed the speed
with which the second could study the drug --- only to find that the
enthusiasm was never really justified in the first place.

In mid-March, Steven Libutti, director of the Rutgers Cancer Institute
of New Jersey, read about a small hydroxychloroquine trial in France
that was generating attention, having found that the anti-malarial might
be effective in the treatment of Covid-19. ``It looked interesting,
exciting, promising, but it looked very far from convincing,'' Libutti
said. Although his specialty is cancer, he wanted to bring his extensive
research knowledge to bear on the pressing question of the drug's
effectiveness. He wrote a proposal for a randomized, controlled trial
that would measure the effectiveness of hydroxychloroquine on a
patient's viral load. (He was comparing the effect of the drug alone
with placebo, as well as with the drug when administered with another
drug called azithromycin.)

The Food and Drug Administration and the ethical review board at the
Rutgers Cancer Institute approved his trial in record time, as has been
typical for many proposed drug trials during the pandemic. He enrolled
the first patient on April 1, hoping he could easily reach 150, calling
on doctors to recruit patients at six hospitals in New Jersey.

By then President Trump had claimed in mid-March that the drug was a
``game changer.'' Some doctors in New York were quietly taking it
prophylactically. That month, the F.D.A. authorized hydroxychloroquine
for emergency use, a special dispensation that facilitated doctors'
access to the drug even outside the context of a trial. Many New York
hospitals' standard treatment protocols encouraged doctors to consider
hydroxychloroquine for patients, even though the evidence that it worked
remained slim and reports were emerging that in some patients it was
causing heart problems.

Thousands of patients were pouring through those six New Jersey
hospitals, but Libutti waited, for weeks, with great frustration as only
a handful of patients were enrolled in his trial each day. Typically, in
clinical trials, after a patient is admitted to the hospital, a doctor
or nurse, often affiliated with the research, talks to the patient about
the possibility of enrolling in a clinical trial. But Libutti's team was
finding that by the time a nurse could begin the conversation with the
patient, that person had already been administered hydroxychloroquine
--- which meant the researchers could not get a baseline reading of that
patient's viral load. Patient after patient was disqualified from the
study. They had ``been handed hydroxychloroquine along with their
toothbrush and slippers when they got to the emergency room,'' Libutti
told me. ``They were giving it out like dinner mints.'' The researcher
said he ``was shocked by the number of folks whom I thought were
incredibly well-read, knowledgeable physicians but were just
panic-prescribing hydroxychloroquine. I've never seen anything like it.
It just shows how lost in the storm folks were.'' (Michael Steinberg,
who helps oversee trials as well as clinical care at Robert Wood Johnson
University Hospital, which was involved in Libutti's trial, said that
although physicians use their clinical judgment to make decisions about
treatment, they strongly encourage doctors to use evidence-based
criteria.)

Other doctors shared Libutti's experience. Arthur Caplan, a bioethicist
at New York University's medical school, said he is aware of three
medical centers where researchers trying to study hydroxychloroquine
felt that the early ardor for the drug among doctors and patients made
it difficult for them to recruit subjects --- to determine, essentially,
whether the embrace of the drug was at all justified. Caplan and a
colleague argued, in an article published online in April in The Journal
of Clinical Investigation, that ``panicked rhetoric about right-to-try
must be aggressively discouraged in order for scientists to learn what
regimens or vaccines actually work.'' Communicating directly with
doctors at various hospitals who were making the drug part of the
official protocol, he used more plain language: ``This is nuts!''

Image

Michelle Ng Gong, the director of critical-care research for the
Montefiore Health System in New York.Credit...Adam Ferguson for The New
York Times

Image

Steven Libutti, the director of the Rutgers Cancer Institute of New
Jersey.Credit...Adam Ferguson for The New York Times

The Montefiore Health System in New York was one of the many that
included hydroxychloroquine as an option in its treatment protocol,
starting in late March. Michelle Ng Gong, the director of critical-care
research, did not actively fight to have the drug removed from the
protocol. But when she was working in her capacity as a critical-care
doctor, she does not recall ever prescribing the medication, and she
sometimes took patients who had received it in the emergency room off
it. ``When so many people are dying, you want to do something,'' she
said. But very sick patients are more susceptible to adverse events.
``The problem is that we know from critical-care literature, as well as
trials in the past, that we can always do more harm.''

\href{https://www.nytimes.com/news-event/coronavirus?action=click\&pgtype=Article\&state=default\&region=MAIN_CONTENT_3\&context=storylines_faq}{}

\hypertarget{the-coronavirus-outbreak-}{%
\subsubsection{The Coronavirus Outbreak
›}\label{the-coronavirus-outbreak-}}

\hypertarget{frequently-asked-questions}{%
\paragraph{Frequently Asked
Questions}\label{frequently-asked-questions}}

Updated August 6, 2020

\begin{itemize}
\item ~
  \hypertarget{why-are-bars-linked-to-outbreaks}{%
  \paragraph{Why are bars linked to
  outbreaks?}\label{why-are-bars-linked-to-outbreaks}}

  \begin{itemize}
  \tightlist
  \item
    Think about a bar. Alcohol is flowing. It can be loud, but it's
    definitely intimate, and you often need to lean in close to hear
    your friend. And strangers have way, way fewer reservations about
    coming up to people in a bar. That's sort of the point of a bar.
    Feeling good and close to strangers. It's no surprise, then, that
    \href{https://www.nytimes.com/2020/07/02/us/coronavirus-bars.html?action=click\&pgtype=Article\&state=default\&region=MAIN_CONTENT_3\&context=storylines_faq}{bars
    have been linked to outbreaks in several states.} Louisiana health
    officials have tied
    \href{https://www.nytimes.com/2020/06/22/us/new-coronavirus-phase.html?action=click\&pgtype=Article\&state=default\&region=MAIN_CONTENT_3\&context=storylines_faq}{at
    least 100 coronavirus cases} to bars in the Tigerland nightlife
    district in Baton Rouge. Minnesota has traced 328 recent cases to
    bars across the state.
    \href{https://www.boisestatepublicradio.org/post/bars-large-venues-close-ada-county-after-surge-coronavirus-prompts-rollback\#stream/0}{In
    Idaho}, health officials shut down bars in Ada County after
    reporting clusters of infections among young adults who had visited
    several bars in downtown Boise. Governors in
    \href{https://www.nytimes.com/2020/07/01/us/california-coronavirus-reopening.html?action=click\&pgtype=Article\&state=default\&region=MAIN_CONTENT_3\&context=storylines_faq}{California},
    \href{https://www.nytimes.com/2020/06/14/us/coronavirus-united-states.html?action=click\&pgtype=Article\&state=default\&region=MAIN_CONTENT_3\&context=storylines_faq}{Texas
    and Arizona}, where coronavirus cases are soaring, have ordered
    hundreds of newly reopened bars to shut down. Less than two weeks
    after Colorado's bars reopened at limited capacity, Gov. Jared Polis
    \href{https://www.denverpost.com/2020/06/30/colorado-bars-closed-coronavirus/}{ordered
    them to close}.
  \end{itemize}
\item ~
  \hypertarget{i-have-antibodies-am-i-now-immune}{%
  \paragraph{I have antibodies. Am I now
  immune?}\label{i-have-antibodies-am-i-now-immune}}

  \begin{itemize}
  \tightlist
  \item
    As of right now,
    \href{https://www.nytimes.com/2020/07/22/health/covid-antibodies-herd-immunity.html?action=click\&pgtype=Article\&state=default\&region=MAIN_CONTENT_3\&context=storylines_faq}{that
    seems likely, for at least several months.} There have been
    frightening accounts of people suffering what seems to be a second
    bout of Covid-19. But experts say these patients may have a
    drawn-out course of infection, with the virus taking a slow toll
    weeks to months after initial exposure. People infected with the
    coronavirus typically
    \href{https://www.nature.com/articles/s41586-020-2456-9}{produce}
    immune molecules called antibodies, which are
    \href{https://www.nytimes.com/2020/05/07/health/coronavirus-antibody-prevalence.html?action=click\&pgtype=Article\&state=default\&region=MAIN_CONTENT_3\&context=storylines_faq}{protective
    proteins made in response to an
    infection}\href{https://www.nytimes.com/2020/05/07/health/coronavirus-antibody-prevalence.html?action=click\&pgtype=Article\&state=default\&region=MAIN_CONTENT_3\&context=storylines_faq}{.
    These antibodies may} last in the body
    \href{https://www.nature.com/articles/s41591-020-0965-6}{only two to
    three months}, which may seem worrisome, but that's perfectly normal
    after an acute infection subsides, said Dr. Michael Mina, an
    immunologist at Harvard University. It may be possible to get the
    coronavirus again, but it's highly unlikely that it would be
    possible in a short window of time from initial infection or make
    people sicker the second time.
  \end{itemize}
\item ~
  \hypertarget{im-a-small-business-owner-can-i-get-relief}{%
  \paragraph{I'm a small-business owner. Can I get
  relief?}\label{im-a-small-business-owner-can-i-get-relief}}

  \begin{itemize}
  \tightlist
  \item
    The
    \href{https://www.nytimes.com/article/small-business-loans-stimulus-grants-freelancers-coronavirus.html?action=click\&pgtype=Article\&state=default\&region=MAIN_CONTENT_3\&context=storylines_faq}{stimulus
    bills enacted in March} offer help for the millions of American
    small businesses. Those eligible for aid are businesses and
    nonprofit organizations with fewer than 500 workers, including sole
    proprietorships, independent contractors and freelancers. Some
    larger companies in some industries are also eligible. The help
    being offered, which is being managed by the Small Business
    Administration, includes the Paycheck Protection Program and the
    Economic Injury Disaster Loan program. But lots of folks have
    \href{https://www.nytimes.com/interactive/2020/05/07/business/small-business-loans-coronavirus.html?action=click\&pgtype=Article\&state=default\&region=MAIN_CONTENT_3\&context=storylines_faq}{not
    yet seen payouts.} Even those who have received help are confused:
    The rules are draconian, and some are stuck sitting on
    \href{https://www.nytimes.com/2020/05/02/business/economy/loans-coronavirus-small-business.html?action=click\&pgtype=Article\&state=default\&region=MAIN_CONTENT_3\&context=storylines_faq}{money
    they don't know how to use.} Many small-business owners are getting
    less than they expected or
    \href{https://www.nytimes.com/2020/06/10/business/Small-business-loans-ppp.html?action=click\&pgtype=Article\&state=default\&region=MAIN_CONTENT_3\&context=storylines_faq}{not
    hearing anything at all.}
  \end{itemize}
\item ~
  \hypertarget{what-are-my-rights-if-i-am-worried-about-going-back-to-work}{%
  \paragraph{What are my rights if I am worried about going back to
  work?}\label{what-are-my-rights-if-i-am-worried-about-going-back-to-work}}

  \begin{itemize}
  \tightlist
  \item
    Employers have to provide
    \href{https://www.osha.gov/SLTC/covid-19/standards.html}{a safe
    workplace} with policies that protect everyone equally.
    \href{https://www.nytimes.com/article/coronavirus-money-unemployment.html?action=click\&pgtype=Article\&state=default\&region=MAIN_CONTENT_3\&context=storylines_faq}{And
    if one of your co-workers tests positive for the coronavirus, the
    C.D.C.} has said that
    \href{https://www.cdc.gov/coronavirus/2019-ncov/community/guidance-business-response.html}{employers
    should tell their employees} -\/- without giving you the sick
    employee's name -\/- that they may have been exposed to the virus.
  \end{itemize}
\item ~
  \hypertarget{what-is-school-going-to-look-like-in-september}{%
  \paragraph{What is school going to look like in
  September?}\label{what-is-school-going-to-look-like-in-september}}

  \begin{itemize}
  \tightlist
  \item
    It is unlikely that many schools will return to a normal schedule
    this fall, requiring the grind of
    \href{https://www.nytimes.com/2020/06/05/us/coronavirus-education-lost-learning.html?action=click\&pgtype=Article\&state=default\&region=MAIN_CONTENT_3\&context=storylines_faq}{online
    learning},
    \href{https://www.nytimes.com/2020/05/29/us/coronavirus-child-care-centers.html?action=click\&pgtype=Article\&state=default\&region=MAIN_CONTENT_3\&context=storylines_faq}{makeshift
    child care} and
    \href{https://www.nytimes.com/2020/06/03/business/economy/coronavirus-working-women.html?action=click\&pgtype=Article\&state=default\&region=MAIN_CONTENT_3\&context=storylines_faq}{stunted
    workdays} to continue. California's two largest public school
    districts --- Los Angeles and San Diego --- said on July 13, that
    \href{https://www.nytimes.com/2020/07/13/us/lausd-san-diego-school-reopening.html?action=click\&pgtype=Article\&state=default\&region=MAIN_CONTENT_3\&context=storylines_faq}{instruction
    will be remote-only in the fall}, citing concerns that surging
    coronavirus infections in their areas pose too dire a risk for
    students and teachers. Together, the two districts enroll some
    825,000 students. They are the largest in the country so far to
    abandon plans for even a partial physical return to classrooms when
    they reopen in August. For other districts, the solution won't be an
    all-or-nothing approach.
    \href{https://bioethics.jhu.edu/research-and-outreach/projects/eschool-initiative/school-policy-tracker/}{Many
    systems}, including the nation's largest, New York City, are
    devising
    \href{https://www.nytimes.com/2020/06/26/us/coronavirus-schools-reopen-fall.html?action=click\&pgtype=Article\&state=default\&region=MAIN_CONTENT_3\&context=storylines_faq}{hybrid
    plans} that involve spending some days in classrooms and other days
    online. There's no national policy on this yet, so check with your
    municipal school system regularly to see what is happening in your
    community.
  \end{itemize}
\end{itemize}

In the end, the biggest randomized, controlled trial on
hydroxychloroquine came out of Britain in June, and preliminary results
found that the drug was not an effective treatment for Covid-19. In
contrast to American doctors whose access to the use of the drug, even
outside trials, had been eased by a federal agency, British physicians
were given the opposite message. On April 1, the highest medical
officials in England, Wales, Northern Ireland and Scotland each sent a
letter to every hospital in their respective countries, urging doctors
not to prescribe medications off-label outside trials. Instead they
encouraged doctors to enroll their patients in large, multicenter,
randomized, controlled trials, like a study run by the University of
Oxford called Recovery, which looked at the efficacy of
hydroxychloroquine, tocilizumab, convalescent plasma, dexamethasone and
two other treatments. At some hospitals in Britain, as many as about 60
percent of patients were enrolled in Recovery trials; even the Northwell
system, which is committed to research, was able to enroll, at its most
trial-driven hospital, North Shore University Hospital, around only 20
percent of its patients in clinical trials.

A flood of patients all with the same illness presents logistical
challenges to trials, but also the perfect conditions for them; that the
American medical system could not harness more of those patients into
randomized, controlled trials, said Peter Horby, one of the two chief
investigators for Oxford's Recovery trials, represents a lost
opportunity. Whether or not convalescent plasma actually helps patients,
for example, has not yet been resolved by a randomized, controlled trial
despite the tens of thousands of doses that American patients have
received, numbers that dwarf those in Britain. Given those numbers,
American researchers ``could have nailed it by now,'' said Horby, whose
own trial on convalescent plasma is still underway.

Caplan, the N.Y.U. bioethicist, acknowledges that doctors in the United
States did manage to enroll more patients in trials more quickly than
ever. But even still, he believes that the commitment to long-shot
efforts to rescue patients was stronger than the commitment to science,
which slowed results and possibly cost more lives. ``We did a lot,'' he
said. ``But we could have gone faster and resolved questions sooner.''

Image

Convalescent plasma, another experimental treatment in which sick
patients are given plasma from recovered patients with antibodies,
though its effectiveness is still unknown.Credit...Adam Ferguson for The
New York Times

\textbf{If researchers see} hydroxychloroquine's failure as a cautionary
tale about the perils of acting without evidence, Pierre Kory, the
Wisconsin critical-care doctor, sees a different medical lesson emerging
from the pandemic: that the emphasis on randomized, controlled trials
can get in the way of doctors' providing common-sense, lifesaving
treatments.

In April, supportive care alone was considered the best option for
patients with Covid-19, given that there was no evidence yet to back
other treatments. Kory, who was then the chief of critical-care service
at the University of Wisconsin Hospital and Clinics, believed instead
that medications commonly used in critical care would most likely help
critically ill Covid-19 patients, too. That month, at a well-attended
meeting with fellows, residents and leadership, including Lynn Schnapp,
the chair of the department of medicine at the University of Wisconsin
medical school, Kory suggested an approach that went beyond supportive
care. He had been consulting with senior hematologists at the hospital
and had observed alarming blood clotting in Covid-19 patients. He and
the hematologists proposed that the hospital consider administering an
aggressive dose of anticoagulants to patients whose blood tests showed
elevated risks for clotting. (Many medical-society guidelines that once
called for only supportive care now recommend the use of anticoagulants
in Covid-19 patients, but not in doses as aggressive as those that Kory
and specialists at the hospital had proposed.)

Image

Heparin, an anticoagulant now recommended by many medical-society
guidelines in the treatment of Covid-19.Credit...Adam Ferguson for The
New York Times

The meeting among Kory and his colleagues took an adversarial turn. ``No
one else is doing this,'' said Lynn Schnapp, as Kory recalls. (She
denies saying that, although a former colleague of Kory's who attended
the meeting confirmed Kory's account.) ``There is no evidence,'' a
fellow I.C.U. doctor said more than once, her voice raised. Kory, who
pointed out at the meeting that his suggestion was based on the opinion
of the hospital's own experts, says he fired back with equal intensity.
``And this is Wisconsin,'' he told me. ``People don't yell here.'' Other
colleagues who were supposed to jump off the call to attend another
meeting later confided to Kory that they couldn't bring themselves to
leave, for fear of missing out on this unusual hospital drama.

At a subsequent, smaller meeting, Kory brought up with Nizar Jarjour, a
division chief, the possibility of giving steroids, commonly used on
critical-care patients, to Covid-19 patients in the I.C.U. ``I don't
want to talk about it,'' Jarjour said.

In a lengthy email Jarjour later sent me, he explained that open
discussion was welcome during that period of time; he also sympathized
with the sentiments of the I.C.U. colleague who was urging caution while
facing a novel virus.

Corticosteroids have a complicated and controversial history in
critical-care medicine. Numerous trials over the past 50 years have been
conducted on their efficacy in patients with acute respiratory distress
syndrome, or ARDS, a diagnosis for patients who have reached a stage of
perilous respiratory failure. Because many of those patients at that
stage of illness have confounding factors, findings are far from
definitive. But based largely on some meta-analyses, including those
looking at how patients with MERS and SARS fared, the World Health
Organization advised, early in the pandemic in this country, against the
use of steroids in Covid-19 patients experiencing ARDS, which is to say,
most patients on ventilators.

Kory and several colleagues at hospitals around the country noted that
the studies that the W.H.O. cited, for example, were largely not
randomized and controlled; other relevant institutions like the Society
of Critical Care Medicine, whose doctors treat the most ill patients,
and the European Society of Intensive Care Medicine did recommend the
use of steroids for ventilated Covid-19 patients with ARDS. Also, in
Kory's own clinical experience, corticosteroids could be lifesavers. He
did not see them as a wild-card drug for this disease, like
hydroxychloroquine; he used them for non-Covid-19 patients who were
facing cytokine storms or ARDS. He was surprised by the heat with which
colleagues challenged him when he made the recommendation, and he
believed that his own leadership role in conference calls subsequently
diminished. He and Jarjour, he said, had more disagreements in three
days than they had in the previous five years.

On April 7, Kory's colleague Ellie Golestanian sent an email to Kory and
others, at 1:32 a.m., in response to another colleague's call for the
use of corticosteroids and anticoagulants: ``In patients with severe
Covid-19, we are fumbling in the dark, clutching at anything that might
work. But as you are well aware, just because a therapy `should' work,
or we desperately `want' it to work --- it does not follow that it
`will' work.''

``When I hear stuff like corticosteroids described as experimental and
unproven, I want to jump out a window,'' Kory told me later that month.
``They make it sound like we are experimenting on people. I want to be
respectful of my colleagues, but I feel like they are getting it 100
percent wrong. I've never seen smarter people get a problem more wrong.
Because they are running hypotheses in a lab and so many of them fail,
they think when I approach a patient, I am testing out a hypothesis.
It's not like a hypothesis, but more like a problem, and I have to
figure out how to fix it with a couple of decades of experience to back
me up. It's a stretch to call it a hypothesis. It's just me doctoring.''

Kory was so frustrated about the hospital's approach that in May he
resigned, taking a job instead at Aurora St. Luke's Medical Center in
Milwaukee. ``Our differences were so far apart, I felt I couldn't be a
part of it,'' said Kory, who foresaw, in April, a ``catastrophe'' if
doctors at any hospital could apply only supportive care. A colleague of
his in New York, an I.C.U. doctor affiliated with a major medical
center, confirmed that he, too, resigned from his hospital, in part
because of tensions around his decision to try an F.D.A.-approved
medication off-label and outside a trial. In May, Kory, following his
disagreement in Wisconsin, spent several weeks in New York treating
patients, often with steroids.

In June, Oxford posted a preliminary report for its Recovery trial of
more than 6,000 patients who received either standard care or
dexamethasone, a steroid similar to the ones that Kory and other I.C.U.
doctors had been advocating. At least when administered to patients who
were already on oxygen or ventilators, the drug saved lives.

Kory sees, in the Oxford results, a story of triumph. He believes that
he successfully treated patients with steroids and that the Recovery
trial results prove it. And yet if patients did not respond, he would go
further, increasing the dose, in a few instances, to a level 10 times as
strong as that in the trial. Did the higher dosage increase the risk?
The answer to that question, a research purist like Kevin Tracey would
point out, is still unknown. Despite the enthusiasm for the Recovery
trial, Tracey maintains that even one stellar randomized, controlled
trial does not settle the question of the use of steroids for patients
with Covid-19. ``It needs to be replicated,'' he said. Given the long,
complicated history of steroid studies, he predicts that sometime down
the road another statistically powerful randomized, controlled trial
will yield contradictory findings. In Tracey's reservations, Kory sees
not rational evaluation but bias. ``That's a 6,000-person trial he's
discrediting,'' Kory said. ``That's a person who will never be
convinced.''

Kory is also part of a group of critical-care doctors who widely
disseminated a protocol for treating Covid-19 that includes
anticoagulants and steroids but also other treatments --- including
Pepcid and intravenous vitamin C --- whose efficacy is hotly contested
among doctors.

Should Kory and his colleagues have been administering steroids when
they did? Were they right? Kory thinks so. But Eric Rubin, the editor of
The New England Journal of Medicine, thinks it's not so clear-cut. ``You
could also say he was lucky,'' Rubin said.

At times, over the course of several conversations, Rubin defended the
bond between doctors and patients, the need for physician autonomy, the
necessity of making judgments in the absence of evidence, especially
when mortality rates were so high; at other times he seemed frustrated
that doctors were still relying on treatments for which there was no
evidence, concerned that a lack of equipoise had possibly muddled the
course of research. ``I know I seem to be saying opposite things,'' he
admitted. ``And I agree with myself.''

Rubin is an infectious-disease doctor at Brigham and Women's Hospital in
Boston, and like many doctors in that specialty, he had strong
reservations about steroids. He advised colleagues against using them.
``And I was wrong,'' he said. He also acknowledges that he was less
opposed to the use of tocilizumab, although that, too, was untested and
could also increase the risk of infection. I asked him whether perhaps
there was something about tocilizumab's novelty, even its scarcity, its
high price, that may have given it a sheen of credibility? In the
absence of evidence, we are all susceptible to predictable irrational
biases. ``We know less about toci,'' is how Rubin said he thought about
it at the time: It left open the possibility that it could be helpful,
although of course that means it also left open the possibility that it
could do more harm than supportive care, more harm even than steroids.
Unlike steroids, tocilizumab was not yet the devil he knew. Rubin could
see, months later, his decision-making process with humility. ``I can't
argue that I was super rational either,'' he finally said.

\textbf{At the peak} of the surge, Rubin was far from the only doctor
whose usual commitment to evidence faltered. ``There would be a
physician that would have said, `No, no, no,' and all of the sudden,
it's his mother,'' Aberg, the Mount Sinai doctor, told me. ``All of a
sudden, let me tell you, they wanted everything. We had some of our own
physicians admitted, and people were just crazed about what they wanted
to do to those individuals.'' Kevin Tracey, the Feinstein researcher,
says that overwhelming uncertainty was driving people's reactions: ``We
just lived through a plague. It was life and death. Fear. Ignorance. You
were seeing raw human behavior in survival mode, a classic reaction to
threat.''

In recent months, a relative calm has set in. Since that tense mid-May
meeting between the researcher Spyropoulos and his colleagues on the
ground at Long Island Jewish Medical Center, clinicians and researchers
have forged more compromise through a series of conversations. Looking
back at that time, one critical-care doctor mentioned that Spyropoulos,
when he called in by videoconference that day, seemed tired and
stressed; perhaps, the critical-care doctor thought, that accounted for
why Spyropoulos had spoken so harshly to the group.

Spyropoulos, the director of the anticoagulation program at Northwell,
had in fact been working 19-hour days to try to get the trial up and
running at breakneck speed. He also mentioned to me in passing, during
an interview, that both he and his wife had been sick with Covid-19, his
wife more so than him. Perhaps, like many doctors, he was laboring under
the additional stress of having not just a professional but a deeply
personal struggle with the power of the disease. In response to a text I
sent asking about the extent of his wife's illness, Spyropoulos wrote
back on July 11. ``My wife was very sick and bedridden for 3 days (I was
mildly symptomatic) but she responded well to hydroxychloroquine,'' he
wrote.

His casual certainty about the cause and effect of her recovery was
surprising. His wife had been ill, taken hydroxychloroquine and
recovered. Whether the second event caused the third was at best unknown
but statistically unlikely, given the now-significant body of research
showing that the drug does not help. As if even Spyropoulos recognized
that his comment was less than rational, he went on to rehearse the
argument he made to his colleagues at Long Island Jewish about the
importance of relying on research. Anything else, he repeated, was
witchcraft.

\begin{center}\rule{0.5\linewidth}{\linethickness}\end{center}

Advertisement

\protect\hyperlink{after-bottom}{Continue reading the main story}

\hypertarget{site-index}{%
\subsection{Site Index}\label{site-index}}

\hypertarget{site-information-navigation}{%
\subsection{Site Information
Navigation}\label{site-information-navigation}}

\begin{itemize}
\tightlist
\item
  \href{https://help.nytimes.com/hc/en-us/articles/115014792127-Copyright-notice}{©~2020~The
  New York Times Company}
\end{itemize}

\begin{itemize}
\tightlist
\item
  \href{https://www.nytco.com/}{NYTCo}
\item
  \href{https://help.nytimes.com/hc/en-us/articles/115015385887-Contact-Us}{Contact
  Us}
\item
  \href{https://www.nytco.com/careers/}{Work with us}
\item
  \href{https://nytmediakit.com/}{Advertise}
\item
  \href{http://www.tbrandstudio.com/}{T Brand Studio}
\item
  \href{https://www.nytimes.com/privacy/cookie-policy\#how-do-i-manage-trackers}{Your
  Ad Choices}
\item
  \href{https://www.nytimes.com/privacy}{Privacy}
\item
  \href{https://help.nytimes.com/hc/en-us/articles/115014893428-Terms-of-service}{Terms
  of Service}
\item
  \href{https://help.nytimes.com/hc/en-us/articles/115014893968-Terms-of-sale}{Terms
  of Sale}
\item
  \href{https://spiderbites.nytimes.com}{Site Map}
\item
  \href{https://help.nytimes.com/hc/en-us}{Help}
\item
  \href{https://www.nytimes.com/subscription?campaignId=37WXW}{Subscriptions}
\end{itemize}
