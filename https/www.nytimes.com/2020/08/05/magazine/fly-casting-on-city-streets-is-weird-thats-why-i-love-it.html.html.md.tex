Sections

SEARCH

\protect\hyperlink{site-content}{Skip to
content}\protect\hyperlink{site-index}{Skip to site index}

\href{https://myaccount.nytimes.com/auth/login?response_type=cookie\&client_id=vi}{}

\href{https://www.nytimes.com/section/todayspaper}{Today's Paper}

Fly Casting on City Streets Is Weird. That's Why I Love It.

\begin{itemize}
\item
\item
\item
\item
\item
\end{itemize}

Advertisement

\protect\hyperlink{after-top}{Continue reading the main story}

Supported by

\protect\hyperlink{after-sponsor}{Continue reading the main story}

\href{/column/letter-of-recommendation}{Letter of Recommendation}

\hypertarget{fly-casting-on-city-streets-is-weird-thats-why-i-love-it}{%
\section{Fly Casting on City Streets Is Weird. That's Why I Love
It.}\label{fly-casting-on-city-streets-is-weird-thats-why-i-love-it}}

\includegraphics{https://static01.nyt.com/images/2020/08/09/magazine/09LOR-mag/09LOR-mag-articleLarge.jpg?quality=75\&auto=webp\&disable=upscale}

By Jon Gluck

\begin{itemize}
\item
  Aug. 5, 2020
\item
  \begin{itemize}
  \item
  \item
  \item
  \item
  \item
  \end{itemize}
\end{itemize}

I live in New York City, in downtown Manhattan, on the seventh floor of
a 13-story apartment building. Two or three times a week, I wake up
early, ride the elevator down to my lobby and say good morning to my
doorman, in the custom of millions of city dwellers everywhere.

But on the particular days I'm describing, my next move isn't so
familiar: I plant myself in the middle of West 12th Street and commence
fly-casting --- essentially fly-fishing without the fish --- slinging 30
or 40 feet of thin nylon line behind me and in front of me, over and
over again while stepping in and out of the street in sync with the
traffic-light cycles to avoid passing cars, like some kind of
bastardized urban version of Brad Pitt in ``A River Runs Through It,''
God and Norman Maclean forgive me.

I've been practicing this peculiar ritual for years. Some time ago, I
was looking to shake off the rust and get my arm in shape to prepare for
an upcoming fishing trip to Wyoming, but living where I do, I didn't
have a suitable place to do so. Or I thought I didn't, anyway. But then
it occurred to me that a city street --- long, straight and, in my case,
relatively free of traffic --- is actually quite suitable. Pretty great,
even. \emph{Peculiar} is in the eye of the beholder.

This year, street-casting has taken on a new urgency. I typically fish
20 or so days a year, everywhere from the Catskills to the Bahamas, but
because of Covid-19, I haven't managed to get out on the water at all.
And yet, like many of us these days, I'm desperate to find pockets of
joy wherever I can. Some people bake bread; others do jigsaw puzzles. I
cast a fly rod on West 12th Street. For now, it's not a way for me to
prepare for a trip --- it \emph{is} the trip.

While street-casting, per se, may not be an actual thing, fly-casting
definitely is. The sport dates back some 150 years and was popular
enough in the first half of the 20th century that competitions were held
at Madison Square Garden. Today the pursuit is mostly centered on local
clubs, with various associations hosting distance and accuracy
competitions around the world. Fly-casting's undisputed GOAT,
63-year-old Steve Rajeff, won the American Casting Association's
all-around championship 46 years in a row and has taken first place at
the World Casting Championship 14 times. Its newest superstar is Maxine
McCormick, a 16-year-old who took up casting at age 9 and notched two
world titles by the time she was 14. (She has been called
\href{https://www.nytimes.com/2018/08/19/sports/maxine-mccormick.html}{the
Mozart of fly-casting.})

There's a simple Zen pleasure in the metronomic rhythms of fly-casting,
and it's a pretty cool experiment in applied physics. The trick is to
``load'' the line on the back cast, then transfer the coiled energy on
the forward cast, stopping the rod at precisely the right moment to
shoot the line forward with maximum speed. As with a golf swing, a
million things can go wrong. But when you get it right, it's magic.

In some ways, casting in the street isn't all that different from
casting on a river. For safety reasons, I cut the hook off the fly, and
I practice my accuracy by aiming for things like street signs and
manholes. They're not exactly rising trout, but they do. Any distance
constraints the street presents aren't really an issue, at least not for
me. Championship casters regularly shoot line well over 200 feet --- the
current U.S. record, held by Rajeff, stands at an astonishing 243 feet
--- but I'm more of a 30-to-40-feet guy.

The casting itself is only part of the appeal. I also find myself
reveling in the particular pleasures of doing something weird.

Just about everyone who passes by on the sidewalk stops, gawks or
comments. Roughly half of them say, ``Catch anything?'' The more
self-conscious among them note that I probably get that all the time.
(For the record, that does not make the question any less awkward.)

At the same time, a certain kind of blithe New Yorker will affect a ``no
big deal'' attitude when they see me, as if the strange tableau they've
come upon is something they've beheld a thousand times before. (Most of
these people are men.)

People will often try to surreptitiously take a picture or shoot a
video. They aren't as clever as they think they are (and are sometimes a
little creepier than they probably imagine). On the other hand, there's
something warm, even life-affirming, about people who ask me if I'd
mind.

Tourists under age 35 who stumble upon me tend to act as though they've
witnessed an Instagram-age miracle. I can practically hear them
composing their caption: \emph{Dude fly-fishing in downtown Manhattan!}
(With three Edvard Munch ``The Scream'' emojis.)

Who can blame them? There's no denying fly-fishing in the middle of a
Manhattan street isn't exactly ``normal.'' Then again, what is normal
right now? This is a time to do whatever we can to find our moments of
peace and contentment, no matter how strange a form they may take.

A few weeks ago, on a Sunday morning, a woman who looked to be at least
90 walked past me on the sidewalk without so much as slowing down.
``I've lived in this neighborhood my whole life,'' she said, as much to
the universe as to anyone in particular. ``\emph{That} I have never
seen.''

Advertisement

\protect\hyperlink{after-bottom}{Continue reading the main story}

\hypertarget{site-index}{%
\subsection{Site Index}\label{site-index}}

\hypertarget{site-information-navigation}{%
\subsection{Site Information
Navigation}\label{site-information-navigation}}

\begin{itemize}
\tightlist
\item
  \href{https://help.nytimes.com/hc/en-us/articles/115014792127-Copyright-notice}{©~2020~The
  New York Times Company}
\end{itemize}

\begin{itemize}
\tightlist
\item
  \href{https://www.nytco.com/}{NYTCo}
\item
  \href{https://help.nytimes.com/hc/en-us/articles/115015385887-Contact-Us}{Contact
  Us}
\item
  \href{https://www.nytco.com/careers/}{Work with us}
\item
  \href{https://nytmediakit.com/}{Advertise}
\item
  \href{http://www.tbrandstudio.com/}{T Brand Studio}
\item
  \href{https://www.nytimes.com/privacy/cookie-policy\#how-do-i-manage-trackers}{Your
  Ad Choices}
\item
  \href{https://www.nytimes.com/privacy}{Privacy}
\item
  \href{https://help.nytimes.com/hc/en-us/articles/115014893428-Terms-of-service}{Terms
  of Service}
\item
  \href{https://help.nytimes.com/hc/en-us/articles/115014893968-Terms-of-sale}{Terms
  of Sale}
\item
  \href{https://spiderbites.nytimes.com}{Site Map}
\item
  \href{https://help.nytimes.com/hc/en-us}{Help}
\item
  \href{https://www.nytimes.com/subscription?campaignId=37WXW}{Subscriptions}
\end{itemize}
