Sections

SEARCH

\protect\hyperlink{site-content}{Skip to
content}\protect\hyperlink{site-index}{Skip to site index}

\href{https://www.nytimes.com/section/opinion/sunday}{Sunday Review}

\href{https://myaccount.nytimes.com/auth/login?response_type=cookie\&client_id=vi}{}

\href{https://www.nytimes.com/section/todayspaper}{Today's Paper}

\href{/section/opinion/sunday}{Sunday Review}\textbar{}Why Edmund Burke
Still Matters

\href{https://nyti.ms/3gF4054}{https://nyti.ms/3gF4054}

\begin{itemize}
\item
\item
\item
\item
\item
\item
\end{itemize}

Advertisement

\protect\hyperlink{after-top}{Continue reading the main story}

\href{/section/opinion}{Opinion}

Supported by

\protect\hyperlink{after-sponsor}{Continue reading the main story}

\hypertarget{why-edmund-burke-still-matters}{%
\section{Why Edmund Burke Still
Matters}\label{why-edmund-burke-still-matters}}

He reminds us it's hard to respect democratic political institutions
while disdaining the founders of those institutions.

\href{https://www.nytimes.com/by/bret-stephens}{\includegraphics{https://static01.nyt.com/images/2017/08/27/insider/bretstephens/bretstephens-thumbLarge-v6.png}}

By \href{https://www.nytimes.com/by/bret-stephens}{Bret Stephens}

Opinion Columnist

\begin{itemize}
\item
  Aug. 5, 2020
\item
  \begin{itemize}
  \item
  \item
  \item
  \item
  \item
  \item
  \end{itemize}
\end{itemize}

\includegraphics{https://static01.nyt.com/images/2020/08/09/opinion/sunday/07Stephens/07Stephens-articleLarge.jpg?quality=75\&auto=webp\&disable=upscale}

Had it not been for the revolution in France, Edmund Burke would likely
have been remembered, a bit vaguely, as an 18th-century
philosopher-statesman of extravagant rhetorical gifts but frustratingly
ambivalent views. The Irish-born member of the British Parliament was
sympathetic to the grievances of the American colonies but not (like his
onetime friend Thomas Paine) an enthusiastic champion of their
independence; an acerbic critic of George III but a firm defender of
monarchy; a staunch opponent of English rapacity in India but a
supporter of British Empire; an advocate for the gradual emancipation of
at least some slaves, but no believer in equality.

He was also an unabashed snob. ``The occupation of a hairdresser,'' he
wrote, ``cannot be a matter of honor to any person.''

Burke's name endures because of his uncompromising opposition to the
French Revolution --- a view he laid out as some of Britain's more
liberal thinkers thought it represented humanity's best hopes.
``Reflections on the Revolution in France'' was published in November
1790, more than a year after the fall of the Bastille but before the
Reign of Terror, when it still seemed possible that Louis XVI would
survive as a constitutional monarch and the country wouldn't descend
into a blood bath.

Burke foresaw, more accurately than most of his great contemporaries,
what the revolution would bring: the executions of Louis and Marie
Antoinette; the ineffectuality of moderate revolutionary leaders (``a
sort of people who affect to proceed as if they thought that men may
deceive without fraud, rob without injustice, and overturn every thing
without violence''); the rise of a military dictator in the mold of
Napoleon; and a long European war in which the ``Republic of Regicide''
would seek to subjugate the world in the name of liberating it.

How did Burke get it right about the ultimate course of events in France
--- and, by extension, so many subsequent revolutions that aimed to
establish morally enlightened societies and wound up producing despotism
and terror? The question is worth pondering in light of two main
ideological currents of today: the tear-it-all-down populism that has
swept so much of the right in the past five years and the
tear-it-all-down progressivism that threatens to sweep the left.

At the core of Burke's view of the revolution is a profound
understanding of how easily things can be shattered in the name of moral
betterment, national purification and radical political transformation.
States, societies and personal consciences are not Lego-block
constructions to be disassembled and reassembled with ease. They are
more like tapestries, passed from one generation to the next, to be
carefully mended at one edge, gracefully enlarged on the other and
otherwise handled with caution lest a single pulled thread unravel the
entire pattern. ``The nature of man is intricate; the objects of society
are of the greatest possible complexity,'' Burke wrote. ``And therefore
no simple disposition or direction of power can be suitable either to
man's nature, or to the quality of his affairs.''

Burke's objection to the French revolutionaries is that they paid so
little attention to this complexity: They were men of theory, not
experience. Men of experience tend to be cautious about gambling what
they have painstakingly gained. Men of theory tend to be reckless with
what they've inherited but never earned. ``They have wrought underground
a mine that will blow up, at one grand explosion, all examples of
antiquity, all precedents, charters, and acts of parliament. They have
`the rights of men.' Against these there can be no prescriptions.''

Not that Burke was against rights per se. The usual caricature of Burke
is that he is the conservative's conservative, a man for whom any type
of change was dangerous in practice and anathema on principle. That view
of him would have astonished his contemporaries, who knew him as a
champion of Catholic emancipation --- the civil rights movement of his
day --- and other reformist (and usually unpopular) causes.

A fairer reading of Burke would describe him as either a near-liberal or
a near-conservative --- a man who defied easy categorization in his time
and defies it again in ours. He believed in limited government, gradual
reform, parliamentary sovereignty and, with caveats and qualifications,
individual rights. But he also believed that to secure rights, it wasn't
enough simply to declare them on paper, codify them in law and claim
them as entitlements from a divine being or the general will. The
conditions of liberty had to be nurtured through prudent statesmanship,
moral education, national and local loyalties, attention to circumstance
and a healthy respect for the ``latent wisdom'' of long-established
customs and beliefs. If Burke lacked Thomas Jefferson's clarity and
idealism, he never suffered from his hypocrisy.

All of this may sound suspicious to modern readers, especially
progressive ones. But consider what Burke might have made of Trump and
Trumpism. He would have been bemused by the phrase ``drain the swamp'':
To take the metaphor seriously, one would end up destroying all the life
within the swamp, leaving only mud. He would have been revolted by the
Trump family's self-dealing: Among the great causes of Burke's life was
his role in the impeachment of Warren Hastings, the de facto governor
general of India, for corrupt and cruel administration.

\includegraphics{https://static01.nyt.com/images/2020/08/03/opinion/sunday/07Stephens4/merlin_175070103_ba2a528c-b545-4a2c-bbde-0c9750dd574e-articleLarge.jpg?quality=75\&auto=webp\&disable=upscale}

Above all, Burke would have been disgusted by Trump's manners. ``Manners
are of more importance than laws,''
\href{https://quod.lib.umich.edu/e/ecco/004903102.0001.000/1:4?rgn=div1;view=fulltext}{he
wrote}.

``The law touches us but here and there, and now and then. Manners are
what vex or soothe, corrupt or purify, exalt or debase, barbarize or
refine us \ldots. They give their whole form and color to our lives.
According to their quality, they aid morals, they supply them, or they
totally destroy them.''

Burke's understanding of the centrality of manners to norms, of norms to
morals, of morals to culture and of culture to the health of the
political order means that he would have been unimpressed by claims that
Trump had scored policy ``wins,'' like appointing conservative judges or
cutting the corporate tax rate. Those would have been baubles floating
in befouled waters.

Trump's real legacy, in Burke's eyes, would be his relentless debasement
of political culture: of personal propriety; of respect for
institutions; of care for tradition; of trust between citizens and civil
authority; of a society that believes --- and has reason to believe ---
in its own essential decency. ``To make us love our country,'' he wrote,
`` our country ought to be lovely.''

Then again, Burke would have been no less withering in his views of the
far left. ``You began ill,'' he said of the French revolutionaries,
``because you began by despising everything that belonged to you.''

For Burke, the materials of successful social change had to be found in
what the country already provided --- historically, culturally,
institutionally --- not in what it lacked. Britain became the most
liberal society of its day, Burke argued, because it held fast to what
he called ``our \emph{ancient}, indisputable laws and liberties,''
handed down ``as an inheritance from our forefathers.'' Inheritance, he
added, ``furnishes a sure principle of transmission; without at all
excluding a principle of improvement.''

Image

A vandalized statue of George Washington in Washington Square Park in
New York.Credit...Jason Szenes/EPA, via Shutterstock

The people now pulling down statues of Thomas Jefferson and George
Washington and
\href{https://www.nytimes.com/2020/06/26/opinion/statues-protests.html}{spray-painting
``1619''} on them may believe they are striking a blow against the
racial hypocrisy of the founding fathers. But if Burke were alive now,
he would likely note that people who trade ancient liberties --- freedom
of speech, for instance --- for newfangled rights (freedom \emph{from}
speech) could soon wind up with neither. He'd observe that it may not be
easy to teach respect for democratic political institutions while
inculcating contempt for the founders of those institutions. He'd
suggest that if protesters want to make the case for fuller equality for
all Americans, better to enlist the memory of the founders in their
cause than hand them over to their political opponents to champion. He'd
caution that destructiveness toward property tends to lead to violence
toward people.

And he'd warn that the damage being done --- to civil order, public
property and, most of all perhaps, to the values demonstrators claim to
champion --- may not be easy to undo. ``Rage and phrensy will pull down
more in half an hour, than prudence, deliberation and foresight can
build up in a hundred years.''

Because Burke champions a different concept of liberty than the one most
Americans cherish, it may be easy to dismiss his teachings as
interesting but ultimately irrelevant. George Will, in his magnum opus
``The Conservative Sensibility,'' speaks of Burke as a
``throne-and-altar'' conservative of little relevance to American
experience. Whatever else might be said of events in places like
Portland or Seattle, it is not the storming of the Bastille, and
wokeness isn't Jacobinism --- at least not yet. The time to write
``Reflections on the Revolutions in America'' is still a ways off.

A ways off --- but ever more visible on the horizon. To read and admire
Burke does not require us to embrace his views, much less treat him as a
prophet. But it's an opportunity to learn something from a man who saw,
more clearly than most, how ``very plausible schemes, with very pleasing
commencements, have often shameful and lamentable conclusions.''

\emph{The Times is committed to publishing}
\href{https://www.nytimes.com/2019/01/31/opinion/letters/letters-to-editor-new-york-times-women.html}{\emph{a
diversity of letters}} \emph{to the editor. We'd like to hear what you
think about this or any of our articles. Here are some}
\href{https://help.nytimes.com/hc/en-us/articles/115014925288-How-to-submit-a-letter-to-the-editor}{\emph{tips}}\emph{.
And here's our email:}
\href{mailto:letters@nytimes.com}{\emph{letters@nytimes.com}}\emph{.}

\emph{Follow The New York Times Opinion section on}
\href{https://www.facebook.com/nytopinion}{\emph{Facebook}}\emph{,}
\href{http://twitter.com/NYTOpinion}{\emph{Twitter (@NYTopinion)}}
\emph{and}
\href{https://www.instagram.com/nytopinion/}{\emph{Instagram}}\emph{.}

Advertisement

\protect\hyperlink{after-bottom}{Continue reading the main story}

\hypertarget{site-index}{%
\subsection{Site Index}\label{site-index}}

\hypertarget{site-information-navigation}{%
\subsection{Site Information
Navigation}\label{site-information-navigation}}

\begin{itemize}
\tightlist
\item
  \href{https://help.nytimes.com/hc/en-us/articles/115014792127-Copyright-notice}{©~2020~The
  New York Times Company}
\end{itemize}

\begin{itemize}
\tightlist
\item
  \href{https://www.nytco.com/}{NYTCo}
\item
  \href{https://help.nytimes.com/hc/en-us/articles/115015385887-Contact-Us}{Contact
  Us}
\item
  \href{https://www.nytco.com/careers/}{Work with us}
\item
  \href{https://nytmediakit.com/}{Advertise}
\item
  \href{http://www.tbrandstudio.com/}{T Brand Studio}
\item
  \href{https://www.nytimes.com/privacy/cookie-policy\#how-do-i-manage-trackers}{Your
  Ad Choices}
\item
  \href{https://www.nytimes.com/privacy}{Privacy}
\item
  \href{https://help.nytimes.com/hc/en-us/articles/115014893428-Terms-of-service}{Terms
  of Service}
\item
  \href{https://help.nytimes.com/hc/en-us/articles/115014893968-Terms-of-sale}{Terms
  of Sale}
\item
  \href{https://spiderbites.nytimes.com}{Site Map}
\item
  \href{https://help.nytimes.com/hc/en-us}{Help}
\item
  \href{https://www.nytimes.com/subscription?campaignId=37WXW}{Subscriptions}
\end{itemize}
