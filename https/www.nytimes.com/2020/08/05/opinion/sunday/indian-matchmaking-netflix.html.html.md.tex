Sections

SEARCH

\protect\hyperlink{site-content}{Skip to
content}\protect\hyperlink{site-index}{Skip to site index}

\href{https://www.nytimes.com/section/opinion/sunday}{Sunday Review}

\href{https://myaccount.nytimes.com/auth/login?response_type=cookie\&client_id=vi}{}

\href{https://www.nytimes.com/section/todayspaper}{Today's Paper}

\href{/section/opinion/sunday}{Sunday Review}\textbar{}We Need to Talk
About `Indian Matchmaking'

\href{https://nyti.ms/2F6DrYV}{https://nyti.ms/2F6DrYV}

\begin{itemize}
\item
\item
\item
\item
\item
\end{itemize}

Advertisement

\protect\hyperlink{after-top}{Continue reading the main story}

\href{/section/opinion}{Opinion}

Supported by

\protect\hyperlink{after-sponsor}{Continue reading the main story}

\hypertarget{we-need-to-talk-about-indian-matchmaking}{%
\section{We Need to Talk About `Indian
Matchmaking'}\label{we-need-to-talk-about-indian-matchmaking}}

The Netflix show is controversial. But it tells awkward truths about my
community.

By Sanjena Sathian

Ms. Sathian is a novelist.

\begin{itemize}
\item
  Aug. 5, 2020
\item
  \begin{itemize}
  \item
  \item
  \item
  \item
  \item
  \end{itemize}
\end{itemize}

\includegraphics{https://static01.nyt.com/images/2020/08/06/opinion/sunday/06sathian/05sathian-articleLarge.jpg?quality=75\&auto=webp\&disable=upscale}

Five years ago, I met with a matchmaker. I was reporting a feature on
India's \$50-billion marriage-industrial complex --- which includes
everything from the dating app Dil Mil to the lavish wedding of Priyanka
Chopra and Nick Jonas.

I went in scornful. Like many of my progressive South Asian peers, I
denounced arranged marriage as offensive and regressive.

But when the matchmaker recited her lengthy questionnaire, I grasped, if
just for a beat, why people did things this way.

\emph{Do you believe in a higher power?} (No idea.)

\emph{Should your partner share your creative interests?} (Must read,
though preferably not write, novels.)

\emph{Do you want children?} (Not particularly.)

By the time we'd worked through the list of questions, I could almost
imagine that someone out there would meet all my ``criteria,'' as
matchmakers put it. I felt a similar empathy when I switched on ``Indian
Matchmaking,'' Netflix's new, controversial docu-series that follows
Sima Taparia, a desi yenta who is paid to marry off clients in India and
the United States.

The show has
\href{https://www.npr.org/sections/goatsandsoda/2020/07/26/895008997/netflixs-indian-matchmaking-is-the-talk-of-india-and-not-in-a-good-way}{received
sharp criticism} --- some well deserved --- among progressive South
Asians, including
\href{https://www.theatlantic.com/culture/archive/2020/08/netflix-indian-matchmaking-and-the-shadow-of-caste/614863/}{Dalit
writers}, for normalizing the casteist, sexist and colorist elements of
Indian society.

But that doesn't mean we should dismiss the positive ways ``Indian
Matchmaking'' complicates and advances depictions of South Asian life.
It explores the fact that many
\href{https://www.cnn.com/2018/03/21/world/arranged-marriage-christiane-amanpour-sex-love-around-world-delhi/index.html}{Indian
millennials} and their diaspora kin still opt for match-made marriage.
The show reveals conversations that take place behind closed doors,
making desis confront our biases and assumptions, while inviting
non-desis to better understand our culture.

The series, which was produced by the Oscar-nominated documentary
filmmaker Smriti Mundhra, presents people who want to find a middle way
between parentally arranged marriage and contemporary dating. American
career women hire Ms. Taparia of their own accord; relatives bully rich,
hapless Mumbai boys into meeting her.

Ms. Taparia (often just ``Sima Auntie'') married at 19 after speaking to
her husband for 20 minutes. She's a product of the old world and is
serving the new one. That dynamic drives the show. She finds young
people inflexible --- they want partners who are affluent, improbably
tall, well traveled and acceptable to Mom. (One man-child just wants a
clone of his mother.)

There is more nuance to this depiction of arranged marriage than what's
been shown in other films and TV shows featuring South Asians, which
have long disdained match-made partnerships. On the sitcom ``New Girl,''
Cece Parekh and her parent-approved betrothed narrowly escaped their
union, instead finding love with white people. In
``\href{https://www.nytimes.com/2017/07/23/movies/the-big-sick-south-asian-identity-and-marriage.html}{The
Big Sick}'' and
``\href{https://www.nytimes.com/2015/09/11/movies/review-in-meet-the-patels-a-son-submits-to-a-marriage-quest.html}{Meet
the Patels},'' matchmaking served as the obstacle to South Asian men's
sexual liberty. Even Bollywood prefers meet-cutes.

In fact, Western viewers rarely get to see South Asians in romantic
partnerships with one another. Hollywood deserves blame for this --- for
too long, one brown person on screen was revolution enough; two boggled
producers' minds.
``\href{https://www.nytimes.com/2003/03/12/movies/film-review-her-mom-may-kick-but-a-girl-plays-to-win.html}{Bend
It Like Beckham}'' and
``\href{https://www.nytimes.com/1992/02/05/movies/review-film-indian-immigrants-in-a-black-and-white-milieu.html}{Mississippi
Masala}'' featured Indian women dating outside the race. (``Masala''
deserves praise for tackling anti-Blackness among South Asians.) On
``\href{https://www.nytimes.com/watching/recommendations/watching-tv-master-of-none}{Master
of None}'' and
``\href{https://www.nytimes.com/watching/recommendations/watching-tv-the-mindy-project}{The
Mindy Project},'' the protagonists generally dated white people.

But by 2020, South Asians have arrived on screens in more formats. Hasan
Minhaj is the new Jon Stewart on
``\href{https://www.nytimes.com/2018/10/18/arts/television/hasan-minhaj-netflix-patriot-act.html}{Patriot
Act}''; Bravo's deliciously tawdry
``\href{https://www.nytimes.com/2019/10/29/style/bravo-real-housewives-race.html}{Family
Karma}'' showcases rich Indian Americans in Miami. Netflix and Amazon
are
\href{https://www.nytimes.com/2019/12/30/arts/television/indian-tv-amazon-netflix.html}{investing
in stories} for Indian viewers.

Now, desi creators can portray ourselves dating and marrying brown.
``Family Karma'' sees Indians courting (and sniping) within the
community. Mindy Kaling's comedy
``\href{https://www.nytimes.com/2020/04/27/arts/television/never-have-i-ever-normal-people.html}{Never
Have I Ever}'' subverts familiar narratives: A woman trying to avoid a
family setup ends up actually liking the guy.

``Matchmaking'' also reveals more textured dynamics within the
community. A Sindhi woman bonds with a Sindhi man over their shared love
of business --- playing on a stereotype that Sindhis are good
businesspeople. A Guyanese woman's quest to meet a man who understands
her family's heritage --- as laborers who left India in the 19th century
--- points to
\href{https://press.uchicago.edu/ucp/books/book/chicago/C/bo13393932.html}{a
rarely depicted migration history}, which unfortunately goes unexplored
in the episode.

The series stops short of being revolutionary, and tacitly accepts a
caste system that can have
\href{https://www.nytimes.com/2010/07/10/world/asia/10honor.html}{fatal}
consequences for those who cross lines.

``By coding caste in harmless phrases such as `similar backgrounds,'
`shared communities' and `respectable families,'''
\href{https://www.theatlantic.com/culture/archive/2020/08/netflix-indian-matchmaking-and-the-shadow-of-caste/614863/}{Yashica
Dutt} wrote in The Atlantic, ``the show does exactly what many
upper-caste Indian families tend to do when discussing this fraught
subject: It makes caste invisible.''

However, ``Matchmaking'' does compellingly examine the challenges faced
by desi women who want a relationship with their culture \emph{and} an
equal partnership. The most poignant motif of the series involves the
common Indian English mantra of ``adjustment.'' A Delhi entrepreneur
says families think an independent woman ``won't know how to adjust.'' A
Mumbai mom says girls, not boys, must adjust. And yet Ms. Taparia's
``adjustment'' advice also helps a pessimistic lawyer be more positive
about her love life.

The show asks us to consider whether ``adjustment'' connotes
open-mindedness, or gender imbalance.

The unsettling answer seems to be that it's both. We should be able to
hold multiple truths about the ``Matchmaking'' subjects ---
understanding why someone might want a partner who speaks the same
language, eats the same comfort food and shares the same religious
beliefs, while also seeing how such worldviews are connected to a
hierarchical and discriminatory system.

It's easy to applaud stories about rejecting old customs in favor of
modern ideals. It's harder, yet worthwhile, to sit with the subtler
tension between tradition and modernity. This is what the great marriage
plots have always considered: a mannered society, and how to live within
it.

Sanjena Sathian
(\href{https://twitter.com/sanjenasathian}{@sanjenasathian}) is the
author of the forthcoming novel ``Gold Diggers.''

\emph{The Times is committed to publishing}
\href{https://www.nytimes.com/2019/01/31/opinion/letters/letters-to-editor-new-york-times-women.html}{\emph{a
diversity of letters}} \emph{to the editor. We'd like to hear what you
think about this or any of our articles. Here are some}
\href{https://help.nytimes.com/hc/en-us/articles/115014925288-How-to-submit-a-letter-to-the-editor}{\emph{tips}}\emph{.
And here's our email:}
\href{mailto:letters@nytimes.com}{\emph{letters@nytimes.com}}\emph{.}

\emph{Follow The New York Times Opinion section on}
\href{https://www.facebook.com/nytopinion}{\emph{Facebook}}\emph{,}
\href{http://twitter.com/NYTOpinion}{\emph{Twitter (@NYTopinion)}}
\emph{and}
\href{https://www.instagram.com/nytopinion/}{\emph{Instagram}}\emph{.}

Advertisement

\protect\hyperlink{after-bottom}{Continue reading the main story}

\hypertarget{site-index}{%
\subsection{Site Index}\label{site-index}}

\hypertarget{site-information-navigation}{%
\subsection{Site Information
Navigation}\label{site-information-navigation}}

\begin{itemize}
\tightlist
\item
  \href{https://help.nytimes.com/hc/en-us/articles/115014792127-Copyright-notice}{©~2020~The
  New York Times Company}
\end{itemize}

\begin{itemize}
\tightlist
\item
  \href{https://www.nytco.com/}{NYTCo}
\item
  \href{https://help.nytimes.com/hc/en-us/articles/115015385887-Contact-Us}{Contact
  Us}
\item
  \href{https://www.nytco.com/careers/}{Work with us}
\item
  \href{https://nytmediakit.com/}{Advertise}
\item
  \href{http://www.tbrandstudio.com/}{T Brand Studio}
\item
  \href{https://www.nytimes.com/privacy/cookie-policy\#how-do-i-manage-trackers}{Your
  Ad Choices}
\item
  \href{https://www.nytimes.com/privacy}{Privacy}
\item
  \href{https://help.nytimes.com/hc/en-us/articles/115014893428-Terms-of-service}{Terms
  of Service}
\item
  \href{https://help.nytimes.com/hc/en-us/articles/115014893968-Terms-of-sale}{Terms
  of Sale}
\item
  \href{https://spiderbites.nytimes.com}{Site Map}
\item
  \href{https://help.nytimes.com/hc/en-us}{Help}
\item
  \href{https://www.nytimes.com/subscription?campaignId=37WXW}{Subscriptions}
\end{itemize}
