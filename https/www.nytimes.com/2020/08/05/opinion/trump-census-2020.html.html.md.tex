Sections

SEARCH

\protect\hyperlink{site-content}{Skip to
content}\protect\hyperlink{site-index}{Skip to site index}

\href{https://myaccount.nytimes.com/auth/login?response_type=cookie\&client_id=vi}{}

\href{https://www.nytimes.com/section/todayspaper}{Today's Paper}

\href{/section/opinion}{Opinion}\textbar{}Trump Is Plotting Against the
Census. Here's Why.

\href{https://nyti.ms/30x5HvQ}{https://nyti.ms/30x5HvQ}

\begin{itemize}
\item
\item
\item
\item
\item
\item
\end{itemize}

Advertisement

\protect\hyperlink{after-top}{Continue reading the main story}

\href{/section/opinion}{Opinion}

Supported by

\protect\hyperlink{after-sponsor}{Continue reading the main story}

\hypertarget{trump-is-plotting-against-the-census-heres-why}{%
\section{Trump Is Plotting Against the Census. Here's
Why.}\label{trump-is-plotting-against-the-census-heres-why}}

By his calculations, the fewer people of color and noncitizens who are
counted, the better.

By
\href{https://www.nytimes.com/interactive/opinion/editorialboard.html}{The
Editorial Board}

The editorial board is a group of opinion journalists whose views are
informed by expertise, research, debate and certain longstanding ****
\href{https://www.nytimes.com/interactive/2018/opinion/editorialboard.html}{values}.
It is separate from the newsroom.

\begin{itemize}
\item
  Aug. 5, 2020
\item
  \begin{itemize}
  \item
  \item
  \item
  \item
  \item
  \item
  \end{itemize}
\end{itemize}

\includegraphics{https://static01.nyt.com/images/2020/08/08/opinion/08census/05census-articleLarge-v6.jpg?quality=75\&auto=webp\&disable=upscale}

The Census Bureau hasn't offered a clear explanation for its
\href{https://www.nytimes.com/2020/08/04/us/2020-census-ending-early.html}{decision}
this week to bring an early end to the decennial enumeration of the
nation's population, but the reason is clear enough: The Trump
administration doesn't want a complete count, as the law requires.

This is not a secret plot. Mr. Trump has been trying to whitewash the
census since the moment he took office. First his administration tried
to add a question about citizenship in an effort to depress the response
rate of noncitizens. As one longtime Republican strategist
\href{https://www.nytimes.com/2019/05/30/us/census-citizenship-question-hofeller.html}{concluded
in a 2015} analysis, excluding noncitizens from the census would ``be
advantageous to Republicans and non-Hispanic whites.'' After the Supreme
Court
\href{https://www.nytimes.com/2019/06/27/us/politics/census-citizenship-question-supreme-court.html}{poured
cold water} on that plan last year, Mr. Trump
\href{https://www.nytimes.com/2020/07/21/us/politics/trump-immigrants-census-redistricting.html}{directed
the government} last month not to count undocumented immigrants for the
purposes of reapportioning seats in the House of Representatives.
(That's
\href{https://www.motherjones.com/politics/2020/07/trumps-new-census-directive-is-almost-certainly-unconstitutional/}{almost
certainly unconstitutional}.)

The latest gambit is broader: Ending the crucial in-person canvass one
month early will ensure a significant undercount of minorities, as well
as rural populations and other groups.

Even in the best of times, counting roughly 330 million people is a
monumental task. In the midst of a pandemic, it becomes incalculably
harder. The bureau anticipated this back in April, during the first wave
of the coronavirus, when
\href{https://www.nytimes.com/2020/04/13/us/census-coronavirus-delay.html?searchResultPosition=12}{it
requested} from Congress a four-month extension to deliver its data.
Current federal law requires the data to be turned in by Dec. 31; the
extension would have run through April 2021. As part of its request, the
bureau said it would continue knocking on doors, trying to reach every
person in the country, through the end of October.

The House of Representatives approved that request in May; the Senate
has not acted on it. Instead of pressing harder, the census director,
Steven Dillingham,
\href{https://www.census.gov/newsroom/press-releases/2020/delivering-complete-accurate-count.html}{said
Monday} that door-to-door data collection would end September 30, a
month earlier than previously planned, to meet the Dec. 31 deadline.

``We are committed to a complete and accurate 2020 census,'' Mr.
Dillingham said.

It's hard to see how. Just last month, the census's associate director,
Albert Fontenot Jr., said, ``we are past the window of being able to get
those counts'' by the end of the year.

Why does an accurate and complete census matter? Because it is the
anchor of representative democracy. The Constitution's framers made a
national head count the first job of the federal government for a
reason. Based on this count, we make some of our most consequential
decisions as a society, from the states' representation in Congress to
the distribution of more than \$1.5 trillion in annual funding for a
wide range of public programs. Businesses rely on the data to plan
investments. School districts rely on it to decide how many teachers
they need. Researchers use it to analyze the patterns of American life.

The financial ramifications of any mistakes in the census count for
state and local governments are particularly significant. Research by
Andrew Reamer, a professor at George Washington University, provides a
partial picture of the impact of undercounting. For each person missed
by the 2010 census, he calculates that in the 2015 fiscal year, that
person's state lost about
\href{https://gwipp.gwu.edu/sites/g/files/zaxdzs2181/f/downloads/GWIPP\%20Reamer\%20Fiscal\%20Impacts\%20of\%20Census\%20Undercount\%20on\%20FMAP-based\%20Programs\%2003-19-18.pdf}{\$1,091
in federal funding} for Medicaid and child welfare programs. Those
programs comprise just a quarter of federal funding tied to the census.

That's why it is essential for the census to be as precise and as
comprehensive as possible. ``Like the military, the census, the nation's
largest peacetime mobilization, cannot fail,'' a former Census Bureau
director
\href{https://www.nytimes.com/2020/03/11/opinion/contributors/census-coronavirus.html}{wrote
in The New York Times} this year. ``The stakes are too high, its numbers
too consequential.''

To date, just under
\href{https://www.census.gov/newsroom/press-releases/2020/delivering-complete-accurate-count.html}{63
percent} of American households have responded to the census. In normal
years, census workers would knock on as many doors as possible from the
other 37 percent of homes, many in poorer and rural areas of the
country. The arrival of the pandemic, only weeks before the start of the
count on April 1, disrupted those plans. Census officials were hoping
that the virus would fade by now, allowing for more in-person data
collection. But the outbreak hasn't abated, and could get even worse
this fall when the traditional flu season begins. Add to that the many
Americans who have been displaced by the virus, either temporarily or
permanently, and you have the ingredients for
\href{https://www.censushardtocountmaps2020.us/?latlng=40.00000\%2C-98.09000\&z=4\&promotedfeaturetype=states\&baselayerstate=3\&rtrYear=sR2020latest\&infotab=info-rtrselfresponse\&filterQuery=false}{a
major distortion} in the count.

The president ought to do everything in his power to ameliorate that
distortion. Instead, Mr. Trump and his Republican allies have repeatedly
tried to exacerbate it. By their calculations, the fewer people of color
and noncitizens who are counted, the better.

It's true that people of color, who are more likely to be poor or
marginalized than white people, are less likely to be counted in the
census,
\href{https://www.npr.org/2019/06/04/728034176/2020-census-could-lead-to-worst-undercount-of-black-latinx-people-in-30-years\#:~:text=The\%20nonpartisan\%20think\%20tank\%20found,in\%20the\%20U.S.\%20since\%201990}{perhaps
more so this year} than in decades. But the irony is that a rush to
finish the counting process could hurt Mr. Trump's own voters, too.
That's because the poorest states, which depend the most on federal
funding, also tend to have lower census response rates. In West
Virginia, federal funding from programs tied to the census accounted for
17 percent of economic activity in 2017, according to Mr. Reamer's
calculations. The state has one of the lowest
\href{https://2020census.gov/en/response-rates.html}{census response
rates}.

And because so much federal funding is allocated to states based at
least in part on census population estimates, an inaccurate census
doesn't just harm people in undercounted communities. It harms
\href{https://www.nytimes.com/2018/04/03/opinion/trump-census-citizenship-question.html}{everyone}
who lives in the same state.

Whatever happens in the election, the effects of the census will be with
the country for at least another decade --- a legacy that will long
outlive this administration.

Congress can intervene. The deadline for delivery of the final count
needs to be extended to April 30, 2021, as the Census Bureau initially
requested. That would force states to delay the process of drawing new
legislative maps, and in some cases could make it impossible to meet
deadlines written into state law. But the necessary adjustments are a
small price to pay for 10 years of a fairer and more accurate democracy.

Four former census directors, from Democratic and Republican
administrations, called in
\href{https://www.documentcloud.org/documents/7013550-Aug-4-2020-Statement-By-Former-U-S-Census-Bureau.html}{a
statement} this week for Congress to commission outside experts to
establish criteria for evaluating the accuracy of the final count.
That's a good idea, too.

State and local governments also have an important role to play, by
using all available means to urge people to complete their census forms.

If you haven't filled out your own census form yet,
\href{https://2020census.gov/en.html}{what are you waiting for}?

\emph{The Times is committed to publishing}
\href{https://www.nytimes.com/2019/01/31/opinion/letters/letters-to-editor-new-york-times-women.html}{\emph{a
diversity of letters}} \emph{to the editor. We'd like to hear what you
think about this or any of our articles. Here are some}
\href{https://help.nytimes.com/hc/en-us/articles/115014925288-How-to-submit-a-letter-to-the-editor}{\emph{tips}}\emph{.
And here's our email:}
\href{mailto:letters@nytimes.com}{\emph{letters@nytimes.com}}\emph{.}

\emph{Follow The New York Times Opinion section on}
\href{https://www.facebook.com/nytopinion}{\emph{Facebook}}\emph{,}
\href{http://twitter.com/NYTOpinion}{\emph{Twitter (@NYTopinion)}}
\emph{and}
\href{https://www.instagram.com/nytopinion/}{\emph{Instagram}}\emph{.}

Advertisement

\protect\hyperlink{after-bottom}{Continue reading the main story}

\hypertarget{site-index}{%
\subsection{Site Index}\label{site-index}}

\hypertarget{site-information-navigation}{%
\subsection{Site Information
Navigation}\label{site-information-navigation}}

\begin{itemize}
\tightlist
\item
  \href{https://help.nytimes.com/hc/en-us/articles/115014792127-Copyright-notice}{©~2020~The
  New York Times Company}
\end{itemize}

\begin{itemize}
\tightlist
\item
  \href{https://www.nytco.com/}{NYTCo}
\item
  \href{https://help.nytimes.com/hc/en-us/articles/115015385887-Contact-Us}{Contact
  Us}
\item
  \href{https://www.nytco.com/careers/}{Work with us}
\item
  \href{https://nytmediakit.com/}{Advertise}
\item
  \href{http://www.tbrandstudio.com/}{T Brand Studio}
\item
  \href{https://www.nytimes.com/privacy/cookie-policy\#how-do-i-manage-trackers}{Your
  Ad Choices}
\item
  \href{https://www.nytimes.com/privacy}{Privacy}
\item
  \href{https://help.nytimes.com/hc/en-us/articles/115014893428-Terms-of-service}{Terms
  of Service}
\item
  \href{https://help.nytimes.com/hc/en-us/articles/115014893968-Terms-of-sale}{Terms
  of Sale}
\item
  \href{https://spiderbites.nytimes.com}{Site Map}
\item
  \href{https://help.nytimes.com/hc/en-us}{Help}
\item
  \href{https://www.nytimes.com/subscription?campaignId=37WXW}{Subscriptions}
\end{itemize}
