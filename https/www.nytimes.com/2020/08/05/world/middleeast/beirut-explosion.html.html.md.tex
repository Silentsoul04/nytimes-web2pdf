Sections

SEARCH

\protect\hyperlink{site-content}{Skip to
content}\protect\hyperlink{site-index}{Skip to site index}

\href{https://www.nytimes.com/section/world/middleeast}{Middle East}

\href{https://myaccount.nytimes.com/auth/login?response_type=cookie\&client_id=vi}{}

\href{https://www.nytimes.com/section/todayspaper}{Today's Paper}

\href{/section/world/middleeast}{Middle East}\textbar{}After Deadly
Beirut Explosion, a Search for Answers and Survivors: Live Updates

\url{https://nyti.ms/31jGLHp}

\begin{itemize}
\item
\item
\item
\item
\item
\end{itemize}

Advertisement

\protect\hyperlink{after-top}{Continue reading the main story}

Supported by

\protect\hyperlink{after-sponsor}{Continue reading the main story}

LIVE UPDATES

Updated~

Aug. 5, 2020, 3:58 a.m. ET

Aug. 5, 2020, 3:58 a.m. ET

\hypertarget{after-deadly-beirut-explosion-a-search-for-answers-and-survivors-live-updates}{%
\section{After Deadly Beirut Explosion, a Search for Answers and
Survivors: Live
Updates}\label{after-deadly-beirut-explosion-a-search-for-answers-and-survivors-live-updates}}

Rescue workers dug through rubble searching for survivors as fires
continued to burn in the Lebanese capital, where an explosion killed
more than 100 and injured thousands.

Right Now

``We need everything to hospitalize the victims, and there is an acute
shortage of everything,'' Lebanon's health minister told reporters.

\hypertarget{heres-what-you-need-to-know}{%
\subsubsection{Here's what you need to
know:}\label{heres-what-you-need-to-know}}

\begin{itemize}
\tightlist
\item
  \protect\hyperlink{link-480981c6}{Search is on for survivors after
  blast kills more than 100.}
\item
  \protect\hyperlink{link-a31c6f4}{Even as hospitals were destroyed and
  staffers killed, doctors and nurses raced to help.}
\item
  \protect\hyperlink{link-5d384d85}{The science behind the blast: Why
  fertilizer packs a punch.}
\item
  \protect\hyperlink{link-1fdae9ed}{I was bloodied and dazed. Beirut
  strangers treated me like a friend.}
\item
  \protect\hyperlink{link-1ba3fa3b}{In maps: A two-mile radius around
  the blast was flattened.}
\end{itemize}

\includegraphics{https://static01.nyt.com/images/2020/08/05/world/05lebanon-briefing-1sub/merlin_175321077_5b30b6db-7a82-407b-9bf0-064c6b78641d-articleLarge.jpg?quality=75\&auto=webp\&disable=upscale}

\subsection{}

Search is on for survivors after blast kills more than 100.

Rescue workers still struggling to treat thousands of people wounded in
an enormous explosion that rocked Beirut turned their attention on
Wednesday morning to the desperate search for survivors.

The blast, so powerful it could be felt more than 150 miles away in
Cyprus, leveled whole sections of the city near the port of Beirut,
leaving nothing but twisted metal and debris for blocks in Beirut's
downtown business district.

The waterfront neighborhood normally full of restaurants and nightclubs
was essentially flattened. A number of crowded residential neighborhoods
in the city's eastern and predominantly Christian half were also
ravaged.

Nearly all the windows along one popular commercial strip had been blown
out and the street was littered with glass, rubble and cars that had
slammed into each other after the blast. The buildings that remained
standing in the blast area looked as if they had been skinned, leaving
only hulking skeletons.

The death toll rose to over 100 and with an untold number still missing
and officials expected that figure to rise. More than 4,000 people were
injured, overwhelming the city's hospitals.

``What we are witnessing is a huge catastrophe,'' the head of Lebanon's
Red Cross George Kettani told the news network Mayadeen. ``There are
victims and casualties everywhere.''

With electricity out in most of the city, emergency workers were limited
in what they could do until the sun rose.

Emergency workers joined residents scattered across the wreckage digging
through the rubble even fires still smoldered around them.

``There are many people missing until now. People are asking the
emergency department about their loved ones and it is difficult to
search at night because there is no electricity,'' health minister Hamad
Hasan told Reuters.

``We need everything to hospitalize the victims, and there is an acute
shortage of everything,'' Mr. Hassan told local news stations on
Wednesday morning.

Officials said it appeared the blast was caused by the detonation of
more than 2,700 tons of ammonium nitrate, a chemical commonly used in
fertilizer and bombs, which had been stored in a warehouse at the port
since it was confiscated from a cargo ship in 2014.

``As head of the government, I will not relax until we find the
responsible party for what happened, hold it accountable and apply the
most serious punishments against it,'' Prime Minister Hassan Diab said.

Maj. Gen. Abbas Ibrahim, the head of Lebanon's general security service,
told the state-run news agency that ``highly explosive materials'' had
been seized by the government years ago and were stored near the blast
site. Although the possibility that the explosives had been
intentionally set off was being probed, he warned against getting
``ahead of the investigation'' and speculating that it was a terrorist
act.

The Lebanese Red Cross said that every available ambulance from North
Lebanon, Bekaa and South Lebanon was being dispatched to Beirut to help
patients and engaged in search-and-rescue operations.

\includegraphics{https://static01.nyt.com/images/2020/08/05/world/04lebanon-vidcover/04lebanon-vidcover-videoSixteenByNine3000.jpg}

\hypertarget{-1}{%
\subsection{}\label{-1}}

Even as hospitals were destroyed and staffers killed, doctors and nurses
raced to help.

Image

Hospitals in Beirut were overwhelmed. As staff members coped with their
own injuries, patients came flooding in.~Credit...Nabil Mounzer/EPA, via
Shutterstock

At least four large hospitals in Beirut were so severely damaged by the
explosion that they were unable to admit patients, doctors said. Health
care workers were injured and killed in the blast, and a warehouse
storing much of the country's vaccine supply was believed to have been
razed.

An official at American University Hospital in Beirut, the country's
most prestigious and largest private hospital, said they were sending
noncritical patients to hospitals outside the capital.

At least four nurses died and five doctors were wounded at St. George
Hospital, one of the hardest hit, according to Dr. Joseph Haddad, the
director of the hospital's intensive care unit. One nurse scooped up
three premature infants from the natal intensive care unit, where glass
was blown in and the ceiling partially collapsed, and screamed for help
as she held their fragile bodies to her chest.

Dr. Haddad had just finished his rounds and was walking home when the
explosion struck. He rushed to check on his family and found his
apartment completely destroyed.

He then returned to the hospital to get to work, expecting to be busy
stitching up patients injured in the blast and saving lives. But he
discovered that the hospital, too, was in rubble.

``The patients were coming down the stairs, the elevators weren't
working. They were walking down from as high as nine floors up,'' Dr.
Haddad said. ``It was the deepest hell of an apocalypse. When I went
back to my home an hour later, people were crying in the streets.''

``Every floor of the hospital is damaged. I didn't see this even during
the war. It's a catastrophe,'' said Dr. Peter Noun, the head of St.
George Hospital's Pediatric Hematology and Oncology Department. ``The
damage is extremely bad. All the rooms are damaged. All the parents and
their children were in their rooms. Everything just fell down, the
windows destroyed, the ceiling in pieces.''

In addition to taking out some of the capital's most important
hospitals, worries mounted over hundreds of thousands of vaccines and
medications that are stored at the Ministry of Public Health-run central
medical warehouse at Karantina, located a half a mile from the port
where the explosion took place.

The vaccines and medications stored at the warehouse are used to prevent
infectious diseases in children under 5 years old and to treat acute
sicknesses as well as cancer and autoimmune diseases.

\hypertarget{-2}{%
\subsection{}\label{-2}}

The science behind the blast: Why fertilizer packs a punch.

Image

A helicopter at the scene of an explosion on Wednesday.Credit...Issam
Abdallah/Reuters

When an explosive compound detonates, it releases gas that rapidly
expands. This ``shock wave'' is essentially a wall of dense air that can
cause damage, and it dissipates as it spreads further out. A mass of
exploding ammonium nitrate produces a blast that moves at many times the
speed of sound, and this wave can reflect and bounce as it moves ---
especially in an urban area like the Beirut waterfront --- destroying
some buildings while leaving others relatively undamaged

The explosive power of ammonium nitrate can be difficult to quantify in
absolute terms, given its age and the conditions in which it has been
stored. However, it could be as high as about 40 percent of the power of
TNT.

At 40 percent the power of TNT, the detonation of 2,750 tons of ammonium
nitrate could produce 1 p.s.i. of overpressure --- defined as the
pressure caused by a shock wave over and above normal atmospheric
pressure --- as far as 6,600 feet away. The same explosion would produce
27 p.s.i. at a range of 793 feet away, which would destroy most
buildings, and kill people either through direct trauma or being struck
by debris.

Accidental detonation of ammonium nitrate has caused a number of deadly
industrial accidents, including the worst in United States history: In
1947, a ship carrying an estimated 2,000 tons of ammonium nitrate caught
fire and exploded in the harbor of Texas City, Texas, starting a chain
reaction of blasts and blazes that killed 581 people.

The chemical has also been the primary ingredient in bombs used in
several terrorist attacks, including the destruction of the federal
office building in Oklahoma City in 1995, which killed 168 people. That
bomb contained about two tons of ammonium nitrate.

\hypertarget{-3}{%
\subsection{}\label{-3}}

I was bloodied and dazed. Beirut strangers treated me like a friend.

Image

Injured people being evacuated on Tuesday.Credit...Hassan
Ammar/Associated Press

\emph{Vivian Yee, a correspondent for The New York Times, was at home in
Beiru}t \emph{when two explosions convulsed the city. This is her
first-person account of what happened.}

I was just about to look at a video a friend had sent me on Tuesday
afternoon --- ``the port seems to be burning,'' she said --- when my
whole building shook. Uneasily, naïvely, I ran to the window, then back
to my desk to check for news.

Then came a much bigger boom, and the sound itself seemed to splinter.
There was shattered glass flying everywhere. Not thinking but moving, I
ducked under my desk.

When the world stopped cracking open, I couldn't see at first because of
the blood running down my face. After blinking the blood from my eyes, I
tried to take in the sight of my apartment turned into a demolition
site. My yellow front door had been hurled on top of my dining table. I
couldn't find my passport, or sturdy shoes.

Later, someone would tell me that Beirutis of her generation, raised
during Lebanon's 15-year civil war, instinctively ran into their
hallways as soon as they heard the first blast, to escape the glass they
knew would break.

I was not so well trained, but the Lebanese who would help me in the
hours to come had the steadiness that comes from having lived through
countless previous disasters. Nearly all were strangers, yet they
treated me like a friend.

When I got downstairs, someone passing on a motorbike saw my bloody face
and told me to hop on.

Everyone on the street seemed to be either bleeding from open gashes or
swathed in makeshift bandages --- all except one woman in a chic,
backless top leading a small dog on a leash. Only an hour before, we had
all been walking dogs or checking email or grocery shopping. Only an
hour before, there had been no blood.

\hypertarget{-4}{%
\subsection{}\label{-4}}

In maps: A two-mile radius around the blast was flattened.

\href{https://www.nytimes.com/interactive/2020/08/04/world/middleeast/beirut-explosion-damage.html}{}

\includegraphics{https://static01.nyt.com/images/2020/08/04/us/beirut-explosion-damage-promo-1596586440536/beirut-explosion-damage-promo-1596586440536-articleLarge-v2.jpg}

\hypertarget{mapping-the-damage-from-the-beirut-explosions}{%
\subsection{Mapping the Damage From the Beirut
Explosions}\label{mapping-the-damage-from-the-beirut-explosions}}

Damage was seen at least two miles from the explosions, encompassing an
area with more than 750,000 residents.

Reporting was contributed by Ben Hubbard, Vivian Yee, Hwaida Saad, Maria
Abi-Habib, John Ismay, Russell Goldman and Marc Santora.

Advertisement

\protect\hyperlink{after-bottom}{Continue reading the main story}

\hypertarget{site-index}{%
\subsection{Site Index}\label{site-index}}

\hypertarget{site-information-navigation}{%
\subsection{Site Information
Navigation}\label{site-information-navigation}}

\begin{itemize}
\tightlist
\item
  \href{https://help.nytimes.com/hc/en-us/articles/115014792127-Copyright-notice}{©~2020~The
  New York Times Company}
\end{itemize}

\begin{itemize}
\tightlist
\item
  \href{https://www.nytco.com/}{NYTCo}
\item
  \href{https://help.nytimes.com/hc/en-us/articles/115015385887-Contact-Us}{Contact
  Us}
\item
  \href{https://www.nytco.com/careers/}{Work with us}
\item
  \href{https://nytmediakit.com/}{Advertise}
\item
  \href{http://www.tbrandstudio.com/}{T Brand Studio}
\item
  \href{https://www.nytimes.com/privacy/cookie-policy\#how-do-i-manage-trackers}{Your
  Ad Choices}
\item
  \href{https://www.nytimes.com/privacy}{Privacy}
\item
  \href{https://help.nytimes.com/hc/en-us/articles/115014893428-Terms-of-service}{Terms
  of Service}
\item
  \href{https://help.nytimes.com/hc/en-us/articles/115014893968-Terms-of-sale}{Terms
  of Sale}
\item
  \href{https://spiderbites.nytimes.com}{Site Map}
\item
  \href{https://help.nytimes.com/hc/en-us}{Help}
\item
  \href{https://www.nytimes.com/subscription?campaignId=37WXW}{Subscriptions}
\end{itemize}
