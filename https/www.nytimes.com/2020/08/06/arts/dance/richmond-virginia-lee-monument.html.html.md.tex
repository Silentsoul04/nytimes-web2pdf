Sections

SEARCH

\protect\hyperlink{site-content}{Skip to
content}\protect\hyperlink{site-index}{Skip to site index}

\href{https://www.nytimes.com/section/arts/dance}{Dance}

\href{https://myaccount.nytimes.com/auth/login?response_type=cookie\&client_id=vi}{}

\href{https://www.nytimes.com/section/todayspaper}{Today's Paper}

\href{/section/arts/dance}{Dance}\textbar{}In Richmond, Black Dance
Claims a Space Near Robert E. Lee

\href{https://nyti.ms/3i9njDS}{https://nyti.ms/3i9njDS}

\begin{itemize}
\item
\item
\item
\item
\item
\item
\end{itemize}

Advertisement

\protect\hyperlink{after-top}{Continue reading the main story}

Supported by

\protect\hyperlink{after-sponsor}{Continue reading the main story}

\hypertarget{in-richmond-black-dance-claims-a-space-near-robert-e-lee}{%
\section{In Richmond, Black Dance Claims a Space Near Robert E.
Lee}\label{in-richmond-black-dance-claims-a-space-near-robert-e-lee}}

Black dancers, musicians and artists have invigorated what had long been
considered a whites-only space.

\includegraphics{https://static01.nyt.com/images/2020/08/07/arts/07richmond-dance-3/merlin_174979377_6d5d3a40-9792-4ab2-9387-b3b7445dc835-articleLarge.jpg?quality=75\&auto=webp\&disable=upscale}

By Rebecca J. Ritzel

\begin{itemize}
\item
  Published Aug. 6, 2020Updated Aug. 7, 2020
\item
  \begin{itemize}
  \item
  \item
  \item
  \item
  \item
  \item
  \end{itemize}
\end{itemize}

RICHMOND, Va. --- Janine Bell lived in Richmond for 35 years without
visiting Monument Avenue. But that changed in July, when Ms. Bell threw
a gathering honoring Emmett Till under the shadow of a statue of Robert
E. Lee.

Standing at the base of the three-story pedestal supporting the
Confederate general's likeness, Ms. Bell, the artistic director of the
Elegba Folklore Society, welcomed a small sea of drummers, dancers and
bystanders banging on plastic buckets to an event she called the
Reclamation Drum Circle.

``We are not playing today,'' she said, and invited all present to move
and sway to the music. And so began an extended jam session at a park
long considered a whites-only space. The drum circle, held on what would
have been Emmett Till's 79th birthday, was the latest in a series of
dance happenings --- some spontaneous, some thoughtfully choreographed
--- drawing Black dancers to the Lee statue.

\includegraphics{https://static01.nyt.com/images/2020/08/07/arts/07richmond-dance-2/merlin_174979359_bb41ed08-829d-4be2-a895-836867fec908-articleLarge.jpg?quality=75\&auto=webp\&disable=upscale}

``My grandfather never could have imagined this,'' a sweaty Lito
Raymondo said after performing a solo in the circle's center. ``This is
a revelation.''

The gathering united a disparate group of dancers: community organizers
who take African dance classes, modern dancers and self-taught dancers
like Mr. Raymondo, whose style fuses African, hip-hop and the martial
arts. He said he regularly comes out to ``do his part'' with the
Folklore Society, a group that promotes African culture in a city with a
robust Black dance community.

The festivities have been going on since early June, when Richmond's
mayor and Virginia's governor vowed to take down the huge statues of
Civil War leaders erected along Monument Avenue. Four of those statues
are now being stored at the city's wastewater treatment plant. But
multiple lawsuits and court injunctions have prevented the bronze Robert
E. Lee and his horse Traveler from joining them.

While the judges deliberate, Black artists and residents have been
invigorating the space. ``Whether it's Black people playing basketball
or musicians or dancers, life is happening,'' Ms. Bell said. ``And when
life happens, there is optimism for the future.''

Image

Janine Bell, the Elegba society's artistic director, says: ``Whether
it's Black people playing basketball or musicians or dancers, life is
happening. And when life happens, there is optimism for the
future.''Credit...Brian Palmer for The New York Times

Some dancers go to make political statements; some want memorable
photos. Maggie Small, a longtime star of Richmond Ballet, said dancers
were drawn to the general's shadow because they are living in a time
when ``articulating your thoughts with words'' could be overwhelming. So
they are using the vocabulary they have, because ``dance is a universal
form of communication, of expression and of catharsis.''

It was a dance moment that went viral: Photos of two young dancers, Ava
Holloway and Kennedy George, both 14, turned out and on their toes, each
raising a fist against the backdrop of the statue's graffiti-covered
pedestal. Among those who reposted on Instagram: Beyoncé's mother.
``This is art,'' the Black activist and author Shaun King said in an
Instagram post, accompanied by a fire emoji.

Ms. Holloway and Ms. George, who study at the Central Virginia Dance
Academy, had run into each other while posing at the monument for family
photos. At the request of Marcus Ingram, a photographer in Richmond,
they returned to the statue the next day, on June 5, for a more formal
shoot, which was also captured by a freelance photojournalist.

The girls became famous beyond the James River, accepting appearance
requests from, among others, the ``Today" Show and a John Legend music
video. Both said they remain crushed that they had to miss out on their
eighth grade graduations, final dance competitions and spring recitals.
Instead they got horrible blisters from running barefoot on asphalt
while ``Today'' show cameras rolled. (``I thought I'd never dance
again,'' Ms. George said, pulling out her phone to display a photo of a
giant purple welt on her foot.)

Image

Kennedy George and Ava Holloway, both 14, at the monument.Credit...Julia
Rendleman/Reuters

They said they understood why pictures of them balancing on point became
symbols of the Black Lives Matter movement, and why other dancers want
to be photographed at the site. The words scrawled on the monument
reflect a world ``that is tough and hard and scary,'' Ms. George said.
``But it's reality, and people have to deal with it.''

Among the copycats who have won their approval: Morgan Bullock, a
20-year-old Richmonder who does Irish dance, and who last year became
one of the first Black dancers to finish in the top 50 at the World
Irish Dance Championships. The Guardian photographed Ms. Bullock jumping
off the Lee statue's pedestal, arms at her side and hair flying, her
white blouse and billowy leggings in sharp contrast to the colorful
expletives graffitied on the plinth behind her.

``She is the very definition of an angel,'' Ms. George said. Ms.
Holloway added, ``It's like she's floating.''

When Ira Lunetter White, a dancer in Richmond Ballet, visited the
statue, he wore a white T-shirt and black pants, similar to the classic
uniform of a male dancer in a
\href{https://www.nytimes.com/2013/09/26/arts/design/new-york-city-ballets-balanchine-black-and-white.html}{``black-and-white''
ballet by George Balanchine}, the founding choreographer of New York
City Ballet. Mr. White, who has performed several of those works in
Richmond, traversed the statue platform adopting signature Balanchine
positions. He and the photographer Meghan McSweeney called their series
``Ode to Arthur Mitchell,'' in honor of
\href{https://www.nytimes.com/2013/09/26/arts/design/new-york-city-ballets-balanchine-black-and-white.html}{City
Ballet's first Black principal dancer}.

In one of Ms. McSweeney's favorite images, the words ``Uplift Black
Voices'' appear beneath Mr. White's feet. ``That is literally what Ira
has been trying to do his entire life,'' she said. Mr. White, 27, was
introduced to dance through Minds in Motion, a program that sends
Richmond Ballet ambassadors into fourth-grade classrooms. He's now in
his sixth season with the senior company, one of five dancers of color
out of 17. He's always been fortunate, he said, to have Black mentors
and colleagues, but recognizes that in ballet beyond Richmond that's not
always the case.

Image

Ira Lunetter White dancing his ``Ode to Arthur
Mitchell.''Credit...Meghan McSweeney

``Now is when we need more voices, more faces being seen and being
heard,'' he said.

Chief among local role models is Ms. Small, a biracial dancer who became
Richmond Ballet's first Black Clara in ``The Nutcracker'' 23 years ago,
and went on to have a long career with the company.

Ms. Small retired from Richmond Ballet last year, at 34, and now serves
as the company's grant writer. Last fall she sent out an email offering
to visit Virginia dance studios as a master class teacher, and was
shocked when every single school said yes. ``So much for finally having
weekends off,'' she said, with a laugh.

A critically lauded dancer who landed on the cover of Dance Magazine,
Ms. Small never made race her calling card. ``There is not a single
narrative to capture what it is to be a Black dancer,'' she said. ``I
was homegrown; that was my narrative.''

It's wrong, Ms. Small said, to assume that the Black dancers at regional
companies remain there because they aren't good enough for bigger
companies in New York or Europe. Over summers Ms. Small made it a point
to seek out-of-town opportunities, including at the National
Choreographer's Initiative in California and with Jessica Lang Dance in
New York, but always came out thinking, ``Richmond was the place that
fed my soul,'' she said. ``I felt comfortable to be the dancer I wanted
to be.''

Image

The area around the statue had long been considered a whites-only
space.Credit...Brian Palmer for The New York Times

And it's not lost on her that in this particular moment of history,
dancers from her hometown have become symbols of a national movement.
Ms. George and Ms. Holloway, both honors students, aren't sure yet if
they'll pursue professional careers in dance. But they are proud to
train at a supportive, diverse studio in a city that elevates Black
dancers.

``Richmond,'' Ms. Holloway said, shaking her head. ``If Richmond can do
it, in our city of Confederate statues, than any other city can, too.''

Advertisement

\protect\hyperlink{after-bottom}{Continue reading the main story}

\hypertarget{site-index}{%
\subsection{Site Index}\label{site-index}}

\hypertarget{site-information-navigation}{%
\subsection{Site Information
Navigation}\label{site-information-navigation}}

\begin{itemize}
\tightlist
\item
  \href{https://help.nytimes.com/hc/en-us/articles/115014792127-Copyright-notice}{©~2020~The
  New York Times Company}
\end{itemize}

\begin{itemize}
\tightlist
\item
  \href{https://www.nytco.com/}{NYTCo}
\item
  \href{https://help.nytimes.com/hc/en-us/articles/115015385887-Contact-Us}{Contact
  Us}
\item
  \href{https://www.nytco.com/careers/}{Work with us}
\item
  \href{https://nytmediakit.com/}{Advertise}
\item
  \href{http://www.tbrandstudio.com/}{T Brand Studio}
\item
  \href{https://www.nytimes.com/privacy/cookie-policy\#how-do-i-manage-trackers}{Your
  Ad Choices}
\item
  \href{https://www.nytimes.com/privacy}{Privacy}
\item
  \href{https://help.nytimes.com/hc/en-us/articles/115014893428-Terms-of-service}{Terms
  of Service}
\item
  \href{https://help.nytimes.com/hc/en-us/articles/115014893968-Terms-of-sale}{Terms
  of Sale}
\item
  \href{https://spiderbites.nytimes.com}{Site Map}
\item
  \href{https://help.nytimes.com/hc/en-us}{Help}
\item
  \href{https://www.nytimes.com/subscription?campaignId=37WXW}{Subscriptions}
\end{itemize}
