Sections

SEARCH

\protect\hyperlink{site-content}{Skip to
content}\protect\hyperlink{site-index}{Skip to site index}

\href{https://www.nytimes.com/section/business}{Business}

\href{https://myaccount.nytimes.com/auth/login?response_type=cookie\&client_id=vi}{}

\href{https://www.nytimes.com/section/todayspaper}{Today's Paper}

\href{/section/business}{Business}\textbar{}As Jobs Report Looms, White
House and Congress Say Stark Divisions Remain Over Stimulus Plan

\href{https://nyti.ms/2PsgD7E}{https://nyti.ms/2PsgD7E}

\begin{itemize}
\item
\item
\item
\item
\item
\end{itemize}

Advertisement

\protect\hyperlink{after-top}{Continue reading the main story}

Supported by

\protect\hyperlink{after-sponsor}{Continue reading the main story}

\hypertarget{as-jobs-report-looms-white-house-and-congress-say-stark-divisions-remain-over-stimulus-plan}{%
\section{As Jobs Report Looms, White House and Congress Say Stark
Divisions Remain Over Stimulus
Plan}\label{as-jobs-report-looms-white-house-and-congress-say-stark-divisions-remain-over-stimulus-plan}}

As senators left Washington without a deal on an economic rescue
package, economists warned of new signs of a lagging recovery

\includegraphics{https://static01.nyt.com/images/2020/08/06/us/politics/06dc-virus-cong/merlin_175381380_6f64b881-cec9-4d9c-b357-28f6a520be2c-articleLarge.jpg?quality=75\&auto=webp\&disable=upscale}

\href{https://www.nytimes.com/by/emily-cochrane}{\includegraphics{https://static01.nyt.com/images/2018/11/28/multimedia/author-emily-cochrane/author-emily-cochrane-thumbLarge-v3.png}}\href{https://www.nytimes.com/by/jim-tankersley}{\includegraphics{https://static01.nyt.com/images/2018/10/19/multimedia/author-jim-tankersley/author-jim-tankersley-thumbLarge.png}}

By \href{https://www.nytimes.com/by/emily-cochrane}{Emily Cochrane} and
\href{https://www.nytimes.com/by/jim-tankersley}{Jim Tankersley}

\begin{itemize}
\item
  Aug. 6, 2020
\item
  \begin{itemize}
  \item
  \item
  \item
  \item
  \item
  \end{itemize}
\end{itemize}

WASHINGTON --- Top Democrats and the White House clashed anew on
Thursday over an economic recovery package as a jobs report loomed over
stalled negotiations on the plan, raising the stakes of an agreement
even as a compromise appeared to be nowhere in sight.

Grasping for leverage, President Trump threatened to act on his own if
no bipartisan deal could be reached, telling reporters that he could
move as soon as Friday or Saturday to sign executive orders to forestall
evictions, suspend payroll tax collection and provide unemployment aid
and student loan relief. But it was not clear that he had the power to
do so without Congress, which controls spending, or that any set of
executive actions could stabilize an economy devastated by the pandemic.

After more than three hours of talks in the Capitol Hill offices of
Speaker Nancy Pelosi, negotiators emerged without an agreement and said
stark divisions remained. Ms. Pelosi, of California, described a
``consequential meeting'' where ``we could see the difference in values
that we bring to the table.''

``We're very far apart; it's most unfortunate,'' she added after the
meeting. Senator Chuck Schumer of New York, the minority leader, said he
urged the officials negotiating on behalf of the administration, Steven
Mnuchin, the Treasury secretary, and Mark Meadows, the White House chief
of staff, to ``meet us in the middle,'' and work to settle significant
policy divisions, even as lawmakers conceded that they were not close to
such a resolution.

Ms. Pelosi accused Mr. Meadows of slamming the table during the meeting,
though Mr. Meadows denied doing so.

``We're still a considerable amount apart in terms of a compromise that
could be signed into law,'' Mr. Meadows said after the meeting, adding,
``We're willing to stay engaged, but I can tell you that the differences
are still significant.''

Democrats, who
\href{https://www.nytimes.com/2020/05/15/us/politics/house-simulus-vote.html}{are
pressing a \$3.4 trillion package}, assailed the Republicans, saying
\href{https://www.nytimes.com/interactive/2020/07/30/upshot/coronavirus-stimulus-bill.html}{their
offers} had not come close to meeting the needs of Americans struggling
through historic economic and public health crises.

In an interview with CNBC on Thursday, the normally genteel Ms. Pelosi
said of Republicans, ``Perhaps you mistook them for somebody who gives a
damn.''

``Why are we holding America's working families who are struggling, who
have children to care for, senior elders to care for and the rest, and
make it as if they're --- my goodness --- they're not worthy of this?''
she told reporters later at the Capitol.

With no agreement at hand, lawmakers in both chambers have now left
Washington, with the promise of 24 hours' notice before any vote on a
recovery package, which negotiators had hoped to reach before the end of
the week.

Lobbyists and some congressional staff members increasingly fear the
developments are raising the possibility that lawmakers will be unable
to bridge
\href{https://www.nytimes.com/2020/08/05/us/politics/congress-coronavirus-stimulus.html}{the
yawning policy divide} on a new economic stimulus bill. Some expressed
worries on Thursday that Democrats would abandon negotiations if Mr.
Trump chose to follow through with his threatened orders, some of which
Democrats have called illegal and which they could challenge in court.

But a White House official said lawyers there believed the president
would be on solid ground to act on his own to repurpose funding provided
in the last stimulus measure.

On his way to board Air Force One for a flight to Ohio on Thursday, the
president told reporters that he expected to sign the orders ``probably
tomorrow afternoon'' or Saturday morning, though he left open the
possibility of a bipartisan deal instead.

\includegraphics{https://static01.nyt.com/images/2020/08/06/us/politics/06dc-virus-cong2/merlin_175392303_5649181b-05c5-466f-b7c3-3307add4dbca-articleLarge.jpg?quality=75\&auto=webp\&disable=upscale}

A breakdown in negotiations, even one that ends with Mr. Trump taking
unilateral action, could particularly hurt small businesses --- which
have largely run through the aid lawmakers approved for them this year,
with
\href{https://www.nytimes.com/2020/07/06/us/ppp-small-business-loans.html}{a
loan program} to assist them
\href{https://www.nytimes.com/2020/08/06/business/small-businesses-relief-program-ending.html}{slated
to lapse at week's end} --- and state and local government workers, who
could face mass layoffs as budget shortfalls widen.

Looming over the talks on Thursday was the anticipation of a new monthly
jobs report that could influence the trajectory of negotiations.

The Labor Department will report Friday morning on how many jobs the
economy created in July, as the United States climbs back from the
depths of the pandemic recession. Forecasters expect
\href{https://www.nytimes.com/2020/06/05/business/economy/jobs-report.html}{a
slowdown from May}, when the nascent recovery added 2.7 million jobs,
and June, when
\href{https://www.nytimes.com/2020/07/02/business/economy/jobs-unemployment-coronavirus.html}{it
added 4.8 million}. That is because
\href{https://www.nytimes.com/interactive/2020/us/coronavirus-us-cases.html}{the
resurgence of the coronavirus} has cooled off growth in consumer
spending and business activity for much of the summer.

If Friday's report shows a drastic slowdown in job creation, while the
economy
\href{https://www.nytimes.com/2020/04/02/business/economy/coronavirus-unemployment-claims.html}{remains
down more than 10 million jobs} from its prepandemic peak in February,
pressure will rise on Mr. Trump and congressional leaders to cut a deal
to provide additional aid for struggling small businesses, laid-off
workers and state and local governments that are facing large shortfalls
in tax revenue.

``It's so clear that we should do something, and we should do something
big, and we should do it in a way that is bipartisan as we have done
every other bill,'' Ms. Pelosi said after the meeting.

Republicans, for their part, blamed Democrats for what they described as
an unwillingness to compromise on a number of critical fronts, like
agreeing to liability protections for businesses or accepting a lower
level of funding for schools that are already starting the academic
year. They remained bitterly opposed to Democrats' demands for hundreds
of billions of dollars for food aid, election security and
\href{https://www.nytimes.com/2020/07/31/us/politics/trump-usps-mail-delays.html}{the
Postal Service}.

``A lot of Americans' hopes --- a lot of American lives --- are riding
on the Democrats' endless talk,'' said Senator Mitch McConnell,
Republican of Kentucky and the majority leader, vowing to remain in
Washington in anticipation of an agreement. ``I hope they are not
disappointed.''

It is all but guaranteed that a popular small-business loan program will
stop accepting applications at the end of the week, becoming yet another
casualty of the faltering negotiations. And it appeared likely that the
talks would stretch into next week. Mr. Meadows said he would host a
daily conference call next week with Republican senators to keep them
updated on the progress --- or lack thereof --- of negotiations.

``I was hoping that maybe you wouldn't have that call after Friday
because we'd have a deal,'' Senator Roy Blunt, Republican of Missouri,
told reporters. ``I do think at some point, everybody has to make a
decision either we're going to do this or not, and if we're not, we're
not.''

The persisting impasse has prompted the president and his lieutenants to
double down on the threat of unilateral executive action, including
addressing a lapsed federal unemployment benefit and Mr. Trump's demands
for a payroll tax cut. (At least one Republican senator, Charles E.
Grassley, the chairman of the Finance Committee, expressed skepticism
about whether a payroll tax cut was warranted with millions of Americans
unemployed.)

Image

Mark Meadows, the White House chief of staff, said he would host a daily
conference call next week with Republican senators to keep them updated
on the progress --- or lack thereof --- of
negotiations.Credit...Gabriella Demczuk for The New York Times

Democrats could challenge some of those actions, though Ms. Pelosi said
on Thursday that she would welcome an eviction moratorium order,
provided there was additional rental and housing assistance attached.
Speaking to reporters after the meeting, Mr. Schumer cautioned that
executive orders ``will be litigated in court, and be awkward and
difficult to implement.''

Mr. Mnuchin said the president, who checked in with his deputies three
times during the meeting, viewed the executive orders as a last resort
and instructed them to work toward a deal that could become law.

But a better-than-expected jobs number on Friday could sway Mr. Trump
--- who has repeatedly said that the rebound from the recession is well
underway, and that the economy will rapidly return to its precrisis
state --- against agreeing to any more of Democrats' demands on issues
like reviving the now-expired \$600-per-week federal supplement for
unemployed workers. Both Ms. Pelosi and Mr. Schumer have repeatedly
rejected any Republican proposals that would curtail that benefit in
favor of a new, likely more complex system or an overall lower weekly
sum.

It could also embolden the faction of the Senate Republican caucus that
is pushing for no additional federal deficit spending. But some analysts
in Washington say even a particularly brutal jobs report could
complicate negotiations, because Republicans may cite it as a sign that
the additional unemployment benefits that had been in place through the
end of July were so generous that they deterred laid-off Americans from
returning to work.

A string of recent studies have found the opposite: that the additional
income from the benefits has propped up consumer spending and bolstered
the economy, without discouraging workers from taking jobs if offered.

Analysts have raised warnings about a possible letdown in the jobs
report this week, particularly after a sharp drop in the private-sector
job growth that the private payrolls firm ADP reported on Wednesday.

``We believe the labor market reached an inflection point in July,''
economists at Nomura wrote this week in a research note in which they
forecast a gain of 550,000 jobs in July, ``starting what will likely be
a slower phase of recovery.''

Mr. Trump, speaking to Fox News on Thursday, predicted a ``big number''
from Friday's report. Presidents have the ability to see jobs numbers a
day ahead of their release, but an administration official said Mr.
Trump had not seen the report before making his remarks.

Annie Karni contributed reporting.

Advertisement

\protect\hyperlink{after-bottom}{Continue reading the main story}

\hypertarget{site-index}{%
\subsection{Site Index}\label{site-index}}

\hypertarget{site-information-navigation}{%
\subsection{Site Information
Navigation}\label{site-information-navigation}}

\begin{itemize}
\tightlist
\item
  \href{https://help.nytimes.com/hc/en-us/articles/115014792127-Copyright-notice}{©~2020~The
  New York Times Company}
\end{itemize}

\begin{itemize}
\tightlist
\item
  \href{https://www.nytco.com/}{NYTCo}
\item
  \href{https://help.nytimes.com/hc/en-us/articles/115015385887-Contact-Us}{Contact
  Us}
\item
  \href{https://www.nytco.com/careers/}{Work with us}
\item
  \href{https://nytmediakit.com/}{Advertise}
\item
  \href{http://www.tbrandstudio.com/}{T Brand Studio}
\item
  \href{https://www.nytimes.com/privacy/cookie-policy\#how-do-i-manage-trackers}{Your
  Ad Choices}
\item
  \href{https://www.nytimes.com/privacy}{Privacy}
\item
  \href{https://help.nytimes.com/hc/en-us/articles/115014893428-Terms-of-service}{Terms
  of Service}
\item
  \href{https://help.nytimes.com/hc/en-us/articles/115014893968-Terms-of-sale}{Terms
  of Sale}
\item
  \href{https://spiderbites.nytimes.com}{Site Map}
\item
  \href{https://help.nytimes.com/hc/en-us}{Help}
\item
  \href{https://www.nytimes.com/subscription?campaignId=37WXW}{Subscriptions}
\end{itemize}
