Sections

SEARCH

\protect\hyperlink{site-content}{Skip to
content}\protect\hyperlink{site-index}{Skip to site index}

\href{https://www.nytimes.com/section/business/economy}{Economy}

\href{https://myaccount.nytimes.com/auth/login?response_type=cookie\&client_id=vi}{}

\href{https://www.nytimes.com/section/todayspaper}{Today's Paper}

\href{/section/business/economy}{Economy}\textbar{}New Unemployment
Claims Decline, but Remain `Alarmingly High'

\href{https://nyti.ms/31v8WmZ}{https://nyti.ms/31v8WmZ}

\begin{itemize}
\item
\item
\item
\item
\item
\end{itemize}

\href{https://www.nytimes.com/news-event/coronavirus?action=click\&pgtype=Article\&state=default\&region=TOP_BANNER\&context=storylines_menu}{The
Coronavirus Outbreak}

\begin{itemize}
\tightlist
\item
  live\href{https://www.nytimes.com/2020/08/08/world/coronavirus-updates.html?action=click\&pgtype=Article\&state=default\&region=TOP_BANNER\&context=storylines_menu}{Latest
  Updates}
\item
  \href{https://www.nytimes.com/interactive/2020/us/coronavirus-us-cases.html?action=click\&pgtype=Article\&state=default\&region=TOP_BANNER\&context=storylines_menu}{Maps
  and Cases}
\item
  \href{https://www.nytimes.com/interactive/2020/science/coronavirus-vaccine-tracker.html?action=click\&pgtype=Article\&state=default\&region=TOP_BANNER\&context=storylines_menu}{Vaccine
  Tracker}
\item
  \href{https://www.nytimes.com/interactive/2020/world/coronavirus-tips-advice.html?action=click\&pgtype=Article\&state=default\&region=TOP_BANNER\&context=storylines_menu}{F.A.Q.}
\item
  \href{https://www.nytimes.com/live/2020/08/07/business/stock-market-today-coronavirus?action=click\&pgtype=Article\&state=default\&region=TOP_BANNER\&context=storylines_menu}{Markets
  \& Economy}
\end{itemize}

Advertisement

\protect\hyperlink{after-top}{Continue reading the main story}

Supported by

\protect\hyperlink{after-sponsor}{Continue reading the main story}

\hypertarget{new-unemployment-claims-decline-but-remain-alarmingly-high}{%
\section{New Unemployment Claims Decline, but Remain `Alarmingly
High'}\label{new-unemployment-claims-decline-but-remain-alarmingly-high}}

Nearly 1.2 million filed for state benefits last week, the lowest total
since March, as economic readings offer only limited encouragement.

\includegraphics{https://static01.nyt.com/images/2020/08/06/business/06markets-brf-preview2/merlin_175356201_fd4aca9e-aea3-4ca5-b997-87448183975f-articleLarge.jpg?quality=75\&auto=webp\&disable=upscale}

\href{https://www.nytimes.com/by/patricia-cohen}{\includegraphics{https://static01.nyt.com/images/2018/02/16/multimedia/author-patricia-cohen/author-patricia-cohen-thumbLarge.jpg}}

By \href{https://www.nytimes.com/by/patricia-cohen}{Patricia Cohen}

\begin{itemize}
\item
  Aug. 6, 2020
\item
  \begin{itemize}
  \item
  \item
  \item
  \item
  \item
  \end{itemize}
\end{itemize}

The \href{https://oui.doleta.gov/press/2020/080620.pdf}{government
reported} on Thursday that nearly 1.2 million workers filed new claims
for state unemployment benefits last week. It was the lowest weekly
total since March, but signaled the continuing damage that the pandemic
is inflicting on the labor market.

An additional 656,000 claims were filed by freelancers, part-time
workers and others who do not qualify for regular state jobless aid but
are eligible for benefits under a separate federal unemployment
insurance program, the Labor Department announced. Unlike the state
figures, that number is not seasonally adjusted.

``Over all, the data was modestly better than we expected, a surprising
improvement,'' said Kathy Bostjancic, chief U.S. financial economist at
Oxford Economics. There were declines across nearly all the states, even
those where the virus is resurgent.

But jobless claims ``remain at alarmingly high levels,'' she said, and
the stubbornly high number of people collecting unemployment ---
estimated by economists at 30 million --- suggests that ``temporary
layoffs are becoming permanent.''

Although the number of new claims is down from the stratospheric levels
reached in the early days of the pandemic, the million-plus tallies that
have continued for 20 weeks in a row are still extraordinarily high by
historical standards.

And now that emergency federal supplemental benefits have expired, the
newest entrants to join the ranks of unemployed will not be receiving
the extra \$600 a week that has helped jobless workers pay bills through
the spring and early summer.

\hypertarget{job-postings-are-picking-up-but-more-layoffs-are-expected}{%
\subsection{Job postings are picking up, but more layoffs are
expected.}\label{job-postings-are-picking-up-but-more-layoffs-are-expected}}

\includegraphics{https://static01.nyt.com/images/2020/08/06/business/06markets-brf-hiring/merlin_175342185_3dd9a16d-0cd6-4d26-a6de-b6c71d4a7b2b-articleLarge.jpg?quality=75\&auto=webp\&disable=upscale}

While the elevated levels of
\href{https://www.nytimes.com/live/2020/08/06/business/stock-market-today-coronavirus\#new-state-jobless-claims-decline-but-exceed-one-million-for-the-20th-week}{jobless
claims} show that businesses are still struggling to keep employees on
the payroll, there has been some pickup in hiring. After drooping, job
postings at the online jobs site ZipRecruiter rose by 7.4 percent in
July and are still climbing, said Julia Pollak, the company's labor
economist.

But the latest economic data is mixed, she cautioned. Surveys from the
\href{https://www.ismworld.org/}{Institute for Supply Management}, for
instance, showed that business activity in service industries expanded
last month, but that the
\href{https://www.nytimes.com/live/2020/08/06/business/stock-market-today-coronavirus\#as-unemployment-benefits-began-to-run-out-a-freelance-job-came-just-in-time}{employment
index} declined, an indication that many companies are still not
bringing back workers.

There were steep increases in joblessness related to the performing arts
and other live events in July, Ms. Pollak said.

\hypertarget{latest-updates-the-coronavirus-outbreak-and-the-economy}{%
\section{\texorpdfstring{\href{https://www.nytimes.com/live/2020/08/07/business/stock-market-today-coronavirus?action=click\&pgtype=Article\&state=default\&region=MAIN_CONTENT_1\&context=storylines_live_updates}{Latest
Updates: The Coronavirus Outbreak and the
Economy}}{Latest Updates: The Coronavirus Outbreak and the Economy}}\label{latest-updates-the-coronavirus-outbreak-and-the-economy}}

\href{https://www.nytimes.com/live/2020/08/07/business/stock-market-today-coronavirus?action=click\&pgtype=Article\&state=default\&region=MAIN_CONTENT_1\&context=storylines_live_updates\#wealthy-families-are-throwing-a-lifeline-to-distressed-businesses}{14h
ago}

\href{https://www.nytimes.com/live/2020/08/07/business/stock-market-today-coronavirus?action=click\&pgtype=Article\&state=default\&region=MAIN_CONTENT_1\&context=storylines_live_updates\#wealthy-families-are-throwing-a-lifeline-to-distressed-businesses}{Wealthy
families are throwing a lifeline to distressed businesses.}

\href{https://www.nytimes.com/live/2020/08/07/business/stock-market-today-coronavirus?action=click\&pgtype=Article\&state=default\&region=MAIN_CONTENT_1\&context=storylines_live_updates\#the-publisher-of-the-onion-jezebel-and-other-websites-lays-off-15-employees}{15h
ago}

\href{https://www.nytimes.com/live/2020/08/07/business/stock-market-today-coronavirus?action=click\&pgtype=Article\&state=default\&region=MAIN_CONTENT_1\&context=storylines_live_updates\#the-publisher-of-the-onion-jezebel-and-other-websites-lays-off-15-employees}{The
publisher of The Onion, Jezebel and other websites lays off 15
employees.}

\href{https://www.nytimes.com/live/2020/08/07/business/stock-market-today-coronavirus?action=click\&pgtype=Article\&state=default\&region=MAIN_CONTENT_1\&context=storylines_live_updates\#canada-outlines-its-response-to-the-new-us-aluminum-tariff}{20h
ago}

\href{https://www.nytimes.com/live/2020/08/07/business/stock-market-today-coronavirus?action=click\&pgtype=Article\&state=default\&region=MAIN_CONTENT_1\&context=storylines_live_updates\#canada-outlines-its-response-to-the-new-us-aluminum-tariff}{Canada
outlines its response to the new U.S. aluminum tariff.}

\href{https://www.nytimes.com/live/2020/08/07/business/stock-market-today-coronavirus?action=click\&pgtype=Article\&state=default\&region=MAIN_CONTENT_1\&context=storylines_live_updates}{See
more updates}

More live coverage:
\href{https://www.nytimes.com/2020/08/07/world/covid-19-news.html?action=click\&pgtype=Article\&state=default\&region=MAIN_CONTENT_1\&context=storylines_live_updates}{Global}

And announcements of impending layoffs continue to pile in. Ms. Pollak
has been tracking plant closings and layoffs that the government
requires to be announced in advance. ``They are showing that new layoffs
are still taking place at an alarming rate,'' she said. ``Plenty of
layoffs are scheduled for August, September and October, as well.''

``Many companies are realizing now that the effects will be much longer
than expected,'' she said.

\hypertarget{the-july-jobs-report-is-likely-to-reflect-lost-momentum}{%
\subsection{The July jobs report is likely to reflect lost
momentum.}\label{the-july-jobs-report-is-likely-to-reflect-lost-momentum}}

On Friday morning, the Labor Department will offer another gauge of the
pandemic's impact: the employment report for July. Economists' forecasts
vary widely, with a consensus pointing to a gain of 1.5 million jobs but
some expecting a net loss.

In any case, the figure is expected to be far less auspicious than the
June gain of 4.8 million. And even an addition of 1.5 million jobs would
be a small fraction of the 22 million lost in March and April, when all
but essential businesses closed.

There was a burst of hiring after the lockdown orders were lifted, and
it seemed as if the economy might rebound sharply in the late spring.
But a coronavirus surge in large states like California, Florida and
Texas, and the reintroduction of restrictions, has dimmed those hopes.

``There is a lot of uncertainty this time around,'' said Lydia Boussour,
senior U.S. economist with Oxford Economics, whose firm estimates that
employment dropped last month by 280,000. ``The labor market has
definitely lost momentum in recent weeks.''

\hypertarget{for-many-workers-getting-traction-in-the-job-market-may-require-new-skills}{%
\subsection{For many workers, getting traction in the job market may
require new
skills.}\label{for-many-workers-getting-traction-in-the-job-market-may-require-new-skills}}

Image

People awaiting help with unemployment claims at an event last month in
Tulsa, Okla.Credit...Joseph Rushmore for The New York Times

With rising concerns that temporary layoffs are turning into permanent
job losses, economists worry what this will mean for workers at the
bottom rungs of the labor market --- those with the fewest skills and
the lowest pay.

Workers in low-skill industries like restaurants and bars will need
retraining to be hired in sectors like manufacturing, construction or
technology, said Rubeela Farooqi, chief U.S. economist at High Frequency
Economics.

``It's not easy to switch,'' she said. ``We are at risk of structural
damage to the labor market.''

Ms. Farooqi also warned that the mounting number of school closings
would make it difficult for parents --- particularly mothers --- to
re-enter the work force, causing more lasting damage to the labor
market.

\hypertarget{as-unemployment-benefits-began-to-run-out-a-freelance-job-came-just-in-time}{%
\subsection{As unemployment benefits began to run out, a freelance job
came just in
time.}\label{as-unemployment-benefits-began-to-run-out-a-freelance-job-came-just-in-time}}

For Curtis Hoover, the freelance designing gig came just in time. His
regular state unemployment benefits had run out, as had the weekly \$600
supplement that Congress approved to help jobless workers make it
through the pandemic. He was still eligible for payments under an
emergency extension of benefits for 13 weeks, but the clock was ticking
on that assistance as well.

``It couldn't have come at a better time,'' said Mr. Hoover, who got his
first assignment this week. ``I'm very grateful that I can work in my
safe environment, although it's odd jumping in as a team member when you
have never met the team face to face.''

Mr. Hoover, who is 57 and lives in Reading, Pa., lost his job as a
graphic designer last year. His search for new work got off to a slow
start. He had an interview the week before the shutdowns --- and
remembers debating whether he should shake hands at the meeting --- but
it went nowhere. Two other interviews were canceled in the following
weeks.

Last month, as the expiration of the \$600 supplement loomed, he
prepared for the steep cut in income. He pared his spending, canceling
Netflix, ending his gym membership, and shopping more carefully at the
supermarket.

``I'm in a fortunate position because I paid off my house several years
ago,'' Mr. Hoover said. ``If I had a mortgage, I'd be in deep trouble by
now.''

Nelson D. Schwartz and Ben Casselman contributed reporting.

Advertisement

\protect\hyperlink{after-bottom}{Continue reading the main story}

\hypertarget{site-index}{%
\subsection{Site Index}\label{site-index}}

\hypertarget{site-information-navigation}{%
\subsection{Site Information
Navigation}\label{site-information-navigation}}

\begin{itemize}
\tightlist
\item
  \href{https://help.nytimes.com/hc/en-us/articles/115014792127-Copyright-notice}{©~2020~The
  New York Times Company}
\end{itemize}

\begin{itemize}
\tightlist
\item
  \href{https://www.nytco.com/}{NYTCo}
\item
  \href{https://help.nytimes.com/hc/en-us/articles/115015385887-Contact-Us}{Contact
  Us}
\item
  \href{https://www.nytco.com/careers/}{Work with us}
\item
  \href{https://nytmediakit.com/}{Advertise}
\item
  \href{http://www.tbrandstudio.com/}{T Brand Studio}
\item
  \href{https://www.nytimes.com/privacy/cookie-policy\#how-do-i-manage-trackers}{Your
  Ad Choices}
\item
  \href{https://www.nytimes.com/privacy}{Privacy}
\item
  \href{https://help.nytimes.com/hc/en-us/articles/115014893428-Terms-of-service}{Terms
  of Service}
\item
  \href{https://help.nytimes.com/hc/en-us/articles/115014893968-Terms-of-sale}{Terms
  of Sale}
\item
  \href{https://spiderbites.nytimes.com}{Site Map}
\item
  \href{https://help.nytimes.com/hc/en-us}{Help}
\item
  \href{https://www.nytimes.com/subscription?campaignId=37WXW}{Subscriptions}
\end{itemize}
