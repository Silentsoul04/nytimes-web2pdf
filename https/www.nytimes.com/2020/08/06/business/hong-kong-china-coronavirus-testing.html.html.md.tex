Sections

SEARCH

\protect\hyperlink{site-content}{Skip to
content}\protect\hyperlink{site-index}{Skip to site index}

\href{https://www.nytimes.com/section/business}{Business}

\href{https://myaccount.nytimes.com/auth/login?response_type=cookie\&client_id=vi}{}

\href{https://www.nytimes.com/section/todayspaper}{Today's Paper}

\href{/section/business}{Business}\textbar{}China's Offer to Help With
Virus Testing Spooks Hong Kong

\href{https://nyti.ms/33wAP0h}{https://nyti.ms/33wAP0h}

\begin{itemize}
\item
\item
\item
\item
\item
\item
\end{itemize}

\href{https://www.nytimes.com/news-event/coronavirus?action=click\&pgtype=Article\&state=default\&region=TOP_BANNER\&context=storylines_menu}{The
Coronavirus Outbreak}

\begin{itemize}
\tightlist
\item
  live\href{https://www.nytimes.com/2020/08/08/world/coronavirus-updates.html?action=click\&pgtype=Article\&state=default\&region=TOP_BANNER\&context=storylines_menu}{Latest
  Updates}
\item
  \href{https://www.nytimes.com/interactive/2020/us/coronavirus-us-cases.html?action=click\&pgtype=Article\&state=default\&region=TOP_BANNER\&context=storylines_menu}{Maps
  and Cases}
\item
  \href{https://www.nytimes.com/interactive/2020/science/coronavirus-vaccine-tracker.html?action=click\&pgtype=Article\&state=default\&region=TOP_BANNER\&context=storylines_menu}{Vaccine
  Tracker}
\item
  \href{https://www.nytimes.com/interactive/2020/world/coronavirus-tips-advice.html?action=click\&pgtype=Article\&state=default\&region=TOP_BANNER\&context=storylines_menu}{F.A.Q.}
\item
  \href{https://www.nytimes.com/live/2020/08/07/business/stock-market-today-coronavirus?action=click\&pgtype=Article\&state=default\&region=TOP_BANNER\&context=storylines_menu}{Markets
  \& Economy}
\end{itemize}

Advertisement

\protect\hyperlink{after-top}{Continue reading the main story}

Supported by

\protect\hyperlink{after-sponsor}{Continue reading the main story}

\hypertarget{chinas-offer-to-help-with-virus-testing-spooks-hong-kong}{%
\section{China's Offer to Help With Virus Testing Spooks Hong
Kong}\label{chinas-offer-to-help-with-virus-testing-spooks-hong-kong}}

Infections have surged in the city, and its labs have been going at full
speed. But wariness of the Chinese Communist Party runs deep.

\includegraphics{https://static01.nyt.com/images/2020/08/06/world/06hk-testing-1/merlin_174899484_db9ceba3-2da2-49ae-b73d-ccdf288af1e2-articleLarge.jpg?quality=75\&auto=webp\&disable=upscale}

\href{https://www.nytimes.com/by/sui-lee-wee}{\includegraphics{https://static01.nyt.com/images/2018/07/13/multimedia/author-sui-lee-wee/author-sui-lee-wee-thumbLarge.png}}\href{https://www.nytimes.com/by/tiffany-may}{\includegraphics{https://static01.nyt.com/images/2019/12/04/reader-center/author-tiffany-may/author-tiffany-may-thumbLarge.png}}

By \href{https://www.nytimes.com/by/sui-lee-wee}{Sui-Lee Wee} and
\href{https://www.nytimes.com/by/tiffany-may}{Tiffany May}

\begin{itemize}
\item
  Published Aug. 6, 2020Updated Aug. 7, 2020
\item
  \begin{itemize}
  \item
  \item
  \item
  \item
  \item
  \item
  \end{itemize}
\end{itemize}

\href{https://cn.nytimes.com/china/20200807/hong-kong-china-coronavirus-testing/}{阅读简体中文版}\href{https://cn.nytimes.com/china/20200807/hong-kong-china-coronavirus-testing/zh-hant/}{閱讀繁體中文版}

The offer was presented as a favor to Hong Kong, a city struggling with
a surge in
\href{https://www.nytimes.com/2020/08/07/us/covid-test-accuracy-governor-dewine-ohio.html}{coronavirus}
infections: a team of 60 medical officials from mainland China who would
help expand testing across the city.

But it is being viewed with skepticism by some residents, who worry
about the growing reach of the Chinese Communist Party and the testing
project's potential implications for their privacy.

Hong Kong could use the help. The largest wave of coronavirus infections
to hit the semiautonomous city has overwhelmed its isolation wards and
testing facilities in recent weeks.

To reopen schools and lift restrictions on public gatherings and
businesses, the local government needs an effective system of
coronavirus testing that can help contain the outbreak. The problem is,
the city is short of workers who can conduct testing, and the
government's labs are already running at maximum capacity.

By mid-July, labs were operating around the clock, processing 10,000
tests a day, a rate that is unsustainable, according to Carrie Lam, Hong
Kong's top leader. The government has had to limit access to testing in
recent days, saying that it would allocate tests only to people with
symptoms, or who had been in close contact with confirmed cases.

The ability to provide testing for all who need or want it is a
challenge for many cities and countries. That is where China comes in.

``If you want to have a quantum jump in terms of the number of tests
done per day, then we definitely need some help from other countries, or
the mainland government,'' said **** Leo Poon, head of the division of
public health laboratory sciences at the University of Hong Kong.

\includegraphics{https://static01.nyt.com/images/2020/08/06/world/06hk-testing-2/merlin_175223937_218f24cd-c9a8-4a57-a505-0fd686317e19-articleLarge.jpg?quality=75\&auto=webp\&disable=upscale}

When it comes to conducting widespread testing, China is in a league of
its own. The Chinese government takes pride in its ability to marshal
the resources needed for mass testing, citing it as an advantage of the
Communist Party's system of centralized control.

When officials in the central Chinese city of Wuhan, where the virus
emerged, were confronted with a new outbreak in May, they
\href{https://www.nytimes.com/2020/05/26/world/asia/coronavirus-wuhan-tests.html}{tested
11 million people in roughly two weeks.} In Beijing, the government
mobilized close to 100,000 community workers in June
\href{https://www.nytimes.com/2020/06/19/world/asia/coronavirus-china-beijing.html}{to
test roughly 2.3 million residents in about a week}, as it tried to
stamp out a new outbreak of its own.

\hypertarget{latest-updates-the-coronavirus-outbreak-and-the-economy}{%
\section{\texorpdfstring{\href{https://www.nytimes.com/live/2020/08/07/business/stock-market-today-coronavirus?action=click\&pgtype=Article\&state=default\&region=MAIN_CONTENT_1\&context=storylines_live_updates}{Latest
Updates: The Coronavirus Outbreak and the
Economy}}{Latest Updates: The Coronavirus Outbreak and the Economy}}\label{latest-updates-the-coronavirus-outbreak-and-the-economy}}

\href{https://www.nytimes.com/live/2020/08/07/business/stock-market-today-coronavirus?action=click\&pgtype=Article\&state=default\&region=MAIN_CONTENT_1\&context=storylines_live_updates\#wealthy-families-are-throwing-a-lifeline-to-distressed-businesses}{25h
ago}

\href{https://www.nytimes.com/live/2020/08/07/business/stock-market-today-coronavirus?action=click\&pgtype=Article\&state=default\&region=MAIN_CONTENT_1\&context=storylines_live_updates\#wealthy-families-are-throwing-a-lifeline-to-distressed-businesses}{Wealthy
families are throwing a lifeline to distressed businesses.}

\href{https://www.nytimes.com/live/2020/08/07/business/stock-market-today-coronavirus?action=click\&pgtype=Article\&state=default\&region=MAIN_CONTENT_1\&context=storylines_live_updates\#the-publisher-of-the-onion-jezebel-and-other-websites-lays-off-15-employees}{26h
ago}

\href{https://www.nytimes.com/live/2020/08/07/business/stock-market-today-coronavirus?action=click\&pgtype=Article\&state=default\&region=MAIN_CONTENT_1\&context=storylines_live_updates\#the-publisher-of-the-onion-jezebel-and-other-websites-lays-off-15-employees}{The
publisher of The Onion, Jezebel and other websites lays off 15
employees.}

\href{https://www.nytimes.com/live/2020/08/07/business/stock-market-today-coronavirus?action=click\&pgtype=Article\&state=default\&region=MAIN_CONTENT_1\&context=storylines_live_updates\#canada-outlines-its-response-to-the-new-us-aluminum-tariff}{31h
ago}

\href{https://www.nytimes.com/live/2020/08/07/business/stock-market-today-coronavirus?action=click\&pgtype=Article\&state=default\&region=MAIN_CONTENT_1\&context=storylines_live_updates\#canada-outlines-its-response-to-the-new-us-aluminum-tariff}{Canada
outlines its response to the new U.S. aluminum tariff.}

\href{https://www.nytimes.com/live/2020/08/07/business/stock-market-today-coronavirus?action=click\&pgtype=Article\&state=default\&region=MAIN_CONTENT_1\&context=storylines_live_updates}{See
more updates}

More live coverage:
\href{https://www.nytimes.com/2020/08/07/world/covid-19-news.html?action=click\&pgtype=Article\&state=default\&region=MAIN_CONTENT_1\&context=storylines_live_updates}{Global}

In June, Guo Yanhong, a senior official with China's National Health
Commission, said that China had been able to triple its nationwide
testing capacity to 3.8 million tests a day from three months earlier,
according to a government statement. Based on this rate, testing 7.5
million, the entire population of Hong Kong, ``should not be a
problem,'' the local pro-Beijing newspaper Wen Wei Po reported.

Beijing dispatched seven medical experts to Hong Kong on Sunday to help
with testing, Chinese state media reported. Yu Dewen, a health official
from the southern province of Guangdong who is in charge of the team,
said that even with the help of third-party laboratories, Hong Kong
could only process 20,000 to 30,000 tests a day, according to Southern
Metropolis Daily, a state-run Chinese newspaper. He said the team's goal
was roughly 200,000 samples a day.

The seven experts who arrived on Sunday were laying the groundwork for a
larger team of technicians who would cross the border and work with
three laboratories to ramp up testing. The labs are Hong Kong
subsidiaries of mainland companies: Sunrise Diagnostic Center,
established by the Chinese genomics giant BGI; Kingmed Diagnostics; and
Hong Kong Molecular Pathology Diagnostic Center,
\href{https://www.scmp.com/news/hong-kong/health-environment/article/3096037/hong-kong-third-wave-three-labs-picked-help}{according
to The South China Morning Post}.

BGI is one of the biggest companies conducting coronavirus testing in
China. In just three days, it built an ``air capsule'' testing lab in
Beijing that was capable of running 100,000 tests a day.

Image

Commuters in Hong Kong last month.Credit...Anthony Wallace/Agence
France-Presse --- Getty Images

Its unit, Sunrise Diagnostic Center, could help build similar makeshift
testing facilities in Hong Kong if needed, according to The Post. The
firm would be able to process up to 30,000 tests a day by the end of
this week. That capacity could be increased by five times --- to 150,000
a day --- by pooling five samples in one tube.

But many Hong Kong health experts are skeptical of universal testing,
seeing it as a waste of resources and hard to achieve in a short time.

Instead, they say the government needs to test more people who are
deemed to be at greater risk of infection, such as nursing home
residents and public transportation workers.

Arisina Ma, president of the Public Doctors Association, a large union
representing physicians in public hospitals and the Department of
Health, said doctors were surprised by the government's decision to
invite Chinese experts.

Dr. Ma said the process was ``very problematic'' because of its lack of
transparency. She noted that a top health official who heads the
communicable diseases branch in the city's health protection bureau had
been unable to provide any information about what the Chinese team was
doing in the city.

``It felt like a higher-level political decision,'' Dr. Ma said.

Hong Kong's medical facilities have the capacity to increase testing but
have not been told by the government to do so, Dr. Ma said. She wondered
if Beijing's help was even necessary.

``We are not so devastated that we need to ask for help,'' she said.
``Even when hospitals in America and the United Kingdom are devastated,
they just try to mobilize their own personnel. In Hong Kong, we have
sufficient and adequate personnel to do it.''

Some establishment-backed critics have argued that it is doctors who
have made it difficult or prohibitively expensive for people to get
tested.

Regina Ip, a pro-Beijing lawmaker in Hong Kong, said the city had not
been able to do nearly enough testing because of what she described as
the medical profession's grip over the process. ``I think it is
protectionism --- `you have to do it our way,''' she said of the Hong
Kong health care sector.

The cost of taking a coronavirus test is also exacerbating the problem.
At most private hospitals, tests may only be done after a consultation
with a doctor. That means one test can cost about \$200 --- a price tag
that adds up for families.

Image

Lining up for free Covid-19 test kits outside a government clinic in the
Sham Shui Po district of Hong Kong last month.Credit...Anthony
Wallace/Agence France-Presse --- Getty Images

But for some residents, the prospect of more readily available tests was
overshadowed by concern that the outreach by Beijing was only the
Communist Party's latest intrusion into their lives.

They found it especially unnerving in the wake of the sweeping
\href{https://www.nytimes.com/2020/07/31/world/asia/hong-kong-election-national-security-law.html}{national
security law that Beijing imposed} on June 30 to quash dissent in Hong
Kong. Police officers investigating alleged subversion crimes under the
new law have been
\href{https://www.nytimes.com/2020/07/05/world/asia/hong-kong-security-law.html}{collecting
DNA samples from people arrested} at protests.

The Hong Kong government has not said who it plans to test, but it has
pledged that DNA samples will not be transported to the mainland.

But the local government's lack of transparency about the move to invite
Chinese experts and the involvement of Chinese testing companies have
raised alarm bells, activists say. Compounding such fears, the Hong Kong
government said it was looking into the potential criminality of
``spreading rumors'' that the testing program could lead to the
harvesting of DNA samples.

The chairman of Sunrise Diagnostic Center, Hu Dingxu, has said that
samples would not be sent to the mainland, according to Wen Wei Po, the
pro-Beijing newspaper.

Danny Yeung, the chief executive and co-founder of Prenetics, a firm
that has been working with the Hong Kong government to process
coronavirus samples, said his company had no access to the private
information of anyone being tested because they were identified only by
a bar code. Samples would be sent to a biohazard waste company for
disposal after seven days, he said in an interview on Wednesday.

The activists' concerns stem from the fact that the cells collected in
the nasal and throat swabs used for coronavirus testing can also be used
to generate DNA profiles.

Many worry that their data could live on beyond the pandemic, especially
if the samples are stored in a DNA database, similar to the kinds that
the Chinese government has set up for collecting
\href{https://www.nytimes.com/2019/02/21/business/china-xinjiang-uighur-dna-thermo-fisher.html}{DNA
samples from Uighurs} in the far western region of Xinjiang and from
\href{https://www.nytimes.com/2020/06/17/world/asia/China-DNA-surveillance.html}{men
and boys all across China.}

``People in Hong Kong fear they will be subject to similar methods of
control like in Xinjiang,'' said Maya Wang, a senior China researcher at
Human Rights Watch who has studied biometric surveillance in China.

``A large part of that is the mass collection of biometrics,'' she said.

Image

A temporary field hospital in Hong Kong last week.Credit...Anthony
Kwan/Getty Images

Keith Bradsher contributed reporting and Liu Yi contributed research.

Advertisement

\protect\hyperlink{after-bottom}{Continue reading the main story}

\hypertarget{site-index}{%
\subsection{Site Index}\label{site-index}}

\hypertarget{site-information-navigation}{%
\subsection{Site Information
Navigation}\label{site-information-navigation}}

\begin{itemize}
\tightlist
\item
  \href{https://help.nytimes.com/hc/en-us/articles/115014792127-Copyright-notice}{©~2020~The
  New York Times Company}
\end{itemize}

\begin{itemize}
\tightlist
\item
  \href{https://www.nytco.com/}{NYTCo}
\item
  \href{https://help.nytimes.com/hc/en-us/articles/115015385887-Contact-Us}{Contact
  Us}
\item
  \href{https://www.nytco.com/careers/}{Work with us}
\item
  \href{https://nytmediakit.com/}{Advertise}
\item
  \href{http://www.tbrandstudio.com/}{T Brand Studio}
\item
  \href{https://www.nytimes.com/privacy/cookie-policy\#how-do-i-manage-trackers}{Your
  Ad Choices}
\item
  \href{https://www.nytimes.com/privacy}{Privacy}
\item
  \href{https://help.nytimes.com/hc/en-us/articles/115014893428-Terms-of-service}{Terms
  of Service}
\item
  \href{https://help.nytimes.com/hc/en-us/articles/115014893968-Terms-of-sale}{Terms
  of Sale}
\item
  \href{https://spiderbites.nytimes.com}{Site Map}
\item
  \href{https://help.nytimes.com/hc/en-us}{Help}
\item
  \href{https://www.nytimes.com/subscription?campaignId=37WXW}{Subscriptions}
\end{itemize}
