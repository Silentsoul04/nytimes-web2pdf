Sections

SEARCH

\protect\hyperlink{site-content}{Skip to
content}\protect\hyperlink{site-index}{Skip to site index}

\href{https://www.nytimes.com/section/movies}{Movies}

\href{https://myaccount.nytimes.com/auth/login?response_type=cookie\&client_id=vi}{}

\href{https://www.nytimes.com/section/todayspaper}{Today's Paper}

\href{/section/movies}{Movies}\textbar{}What Megan Fox Taught Me About
the Power of Subversive Girls

\href{https://nyti.ms/3gMduvi}{https://nyti.ms/3gMduvi}

\begin{itemize}
\item
\item
\item
\item
\item
\end{itemize}

Advertisement

\protect\hyperlink{after-top}{Continue reading the main story}

Supported by

\protect\hyperlink{after-sponsor}{Continue reading the main story}

\hypertarget{what-megan-fox-taught-me-about-the-power-of-subversive-girls}{%
\section{What Megan Fox Taught Me About the Power of Subversive
Girls}\label{what-megan-fox-taught-me-about-the-power-of-subversive-girls}}

The horror comedy ``Jennifer's Body,'' starring Fox as a demon who eats
boys, showed me the value of teen scream queens when I needed it most.

\includegraphics{https://static01.nyt.com/images/2020/08/08/arts/06SUMMER-WILSON-lead/merlin_174777018_6c4bd042-0af9-4803-91a0-a19449483bd5-articleLarge.jpg?quality=75\&auto=webp\&disable=upscale}

By Lena Wilson

\begin{itemize}
\item
  Aug. 6, 2020
\item
  \begin{itemize}
  \item
  \item
  \item
  \item
  \item
  \end{itemize}
\end{itemize}

One freshman biology lab, I was grouped with an OK guy and the only out
girl in my grade. We were supposed to be dissecting worms.

On that particular day, the guy was playing against type.

``I bet \emph{you} want to see
`\href{https://www.nytimes.com/2009/09/18/movies/18jennifer.html}{Jennifer's
Body},''' he said to the girl suggestively. We had all seen the ads for
the movie, which featured a scantily clad Megan Fox.

By 2009, Megan Fox was not just a sex symbol, she was \emph{the} sex
symbol --- a universal barometer of hotness. And she had recently come
out as bisexual in
\href{https://classic.esquire.com/article/2009/6/1/good-morning-megan}{Esquire}.

At that point in my life, I was coping with my own closeted lesbianism
by pretending homosexuality did not exist. I wasn't seeking out teen
horror led by Sapphic sexpots. I looked down at my worm and prepared to
slice it down the middle.

It turns out the best time to get into horror movies is after you
yourself have been bisected like a lab worm.

When I was 16, I spent most of summer 2011 on the couch, recovering from
spinal fusion surgery. One day, I happened upon a cable TV showing of
``Jennifer's Body'' halfway through, at the film's girl-on-girl make-out
scene. I was intrigued and effectively alone while my mother worked from
her bedroom. I caught the whole movie later that day.

``Jennifer's Body'' was Diablo Cody's next screenplay after she won the
Oscar for ``Juno'' in 2008. The film, directed by Karyn Kusama, follows
the best friends Jennifer (Fox) and Needy (Amanda Seyfried) through the
severing of their toxic bond. Jennifer is a demon who has to eat boys to
remain beautiful. Needy would prefer she not do that. Bloodshed ensues.

\includegraphics{https://static01.nyt.com/images/2020/08/06/arts/06SUMMER-WILSON2/merlin_29512285_3245598c-4b62-41c8-bf1d-6836acb2c62f-articleLarge.jpg?quality=75\&auto=webp\&disable=upscale}

Until ``Jennifer's Body,'' I had approached horror movies with cautious
interest at best. But this film was different. With its references to
emo music and late-aughts pop culture, it seemed like a comedic time
capsule of my own life, so its protagonists, though Hollywood beautiful,
felt real to me. ``Jennifer's Body'' put horror's great assets ---
social transgression, complex female characters and bloodthirsty
vengeance --- in the hands of two contemporary teenage girls. I have
been obsessed with monstrous women like Jennifer ever since.

By the film's end, Needy and Jennifer are shells of their
yearbook-picture-perfect selves. But their path to oblivion is oddly
liberating, as both girls forgo stereotypical feminine docility to don
the roles of hero (Needy) and villain (Jennifer). After an indie band
murders Jennifer in an erroneous virgin sacrifice, she is reborn as a
monster with a taste for male blood. Mild-mannered Needy must save her
helpless boyfriend from Jennifer --- and by the end Needy hunts down and
kills the band that started it all. Such subversive female derangement
is mostly possible in horror films, where bullied, bloody girls burn
down their schools and
\href{https://www.nytimes.com/2020/02/12/movies/the-lodge-bad-mothers.html}{passive
mothers sacrifice their children}. That is why it is my favorite genre,
and one I return to
\href{https://www.nytimes.com/2019/12/16/opinion/black-christmas-movie.html}{over}
and
\href{https://slate.com/culture/2018/06/metoo-produces-more-horror-movies-where-young-women-bite-back.html}{over}
and
\href{https://seventh-row.com/2020/02/24/isabelle-adjani-possession/}{over}
again for novel representations of women.

``Jennifer's Body'' satirizes gendered tropes. It is one of the few
horror movies where a teenage girl's promiscuity actually saves her from
her untimely end --- if Jennifer really \emph{had} been a virgin, there
would be no movie. The film also plays the ``wanton'' Jennifer and
``virginal'' Needy against each other to farcical extremes. Jennifer and
Needy are both sexually active throughout the film, despite Needy's
mousy affect.

Though it was written with a female audience in mind, sexist
expectations marginalized the movie. After I saw ``Jennifer's Body'' at
16, I searched for it on the review aggregator site Rotten Tomatoes,
expecting to see my jubilance reflected back at me. The film was
certified rotten.

Image

Jennifer tries to seduce her next meal: Chip (Johnny Simmons), Needy's
boyfriend.Credit...Doane Gregory/20th Century Fox

In 2009, Kusama and Fox were wounded by critical acrimony and a
\href{https://www.indiewire.com/2018/12/karyn-kusama-jennifers-body-marketing-misogynistic-1202026860/}{sexist
marketing campaign} pegged to horny male viewers. (That is why I had not
seen the film in theaters --- I did not think it was ``for'' me.) Their
detractors, many of whom were men, seemed to have expected an
objectifying chiller. Instead, they saw an intentionally subversive,
campy film and called it a failure. The experience spurred Kusama to
\href{https://www.nytimes.com/2018/12/20/magazine/destroyer-movie-karyn-kusama.html}{leave
the studio system altogether}. Fox, already controversial for her
outspoken criticism of Michael Bay, who had directed her in the
``Transformers'' movies, was written off as a star.

In the last few years, female fans have reclaimed ``Jennifer's Body''
and consider it a pre-\#MeToo classic. For my 21st birthday at Smith
College, my friends and I commandeered a classroom and projected the
movie. The virgin-sacrifice scene, which had barely registered to my
teenage brain, now stole all the air from the room. It was 2015, and it
seemed the whole country was waking up to college rape culture. I had
helped carry a mattress across campus in solidarity with
\href{https://www.nytimes.com/2014/09/22/arts/design/in-a-mattress-a-fulcrum-of-art-and-political-protest.html?searchResultPosition=8}{Emma
Sulkowicz} the year before.

When rocker boys sacrifice Jennifer to Satan, the scene is absurd and
chock-full of Cody's signature quips, but it is also oppressively dark.
The band's frontman, Nikolai (Adam Brody), stabs Jennifer repeatedly
while merrily singing. Jennifer is betrayed by the very artist she
worships. And he victimizes her specifically because she is female.

The violence is heavily sexualized --- Jennifer worries aloud in the
band's van that the members might be rapists, and there's a longstanding
symbolic relationship between stabbing and sexual penetration.
Jennifer's is a pain many women understand. It is especially jarring to
learn that the musician (or comedian, chef or actor) you once admired
could see you as little more than a means to an end.

While Jennifer is sacrificed because of, well, her body, society scorns
Needy --- the only character who knows the truth about Jennifer ---
because of her mind. She first appears in a psychiatric hospital, where
she kicks a doctor and spits in her face. As a teenager, before I was
wheeled into surgery, I had a panic attack so strong I was dosed with
what felt like enough Ativan to fell a hippopotamus. Watching
``Jennifer's Body'' with a foot-long incision healing on my back, I was
as drawn to Needy's wretched, anti-medical mania as I was to Jennifer's
emo-worship.

Image

Needy, played by Amanda Seyfried, after Jennifer attempts to eat her
soul.Credit...Doane Gregory/20th Century Fox

As the violence escalates, sweet Needy drops her first F-bomb --- and
finally consummates her ``totally lesbi-gay'' friendship with Jennifer
in that make-out scene, which has inspired lesbians and bisexual women
to likewise
\href{https://www.them.us/story/jennifers-body-film-cult-status}{reclaim
the film}.

This, I learned at 16, is where the true beauty of the horror genre
lies. In horror, girls and women do not have to be pretty, polite,
chaste or even heterosexual --- in fact, these characters are so
terrifying \emph{because} they willfully eschew gendered assumptions.
Teenage girls --- their emotions too often dismissed as hormonal
hysteria --- can finally lose their cool. Jennifer and Needy have joined
the likes of Regan MacNeil (``The Exorcist''), Carrie White (``Carrie'')
and Brigitte and Ginger Fitzgerald (``Ginger Snaps''), and live on in
more recent unhinged young women like Dani Ardor (``Midsommar'') and
Justine (``Raw'').

The summer of my surgery, I was incorrigibly sad and in too much agony
to eat, sleep or shower independently, much less dress up, wear makeup
or smile. Now, as an adult, I still do not wear feminine clothes or
makeup. I have realized that this is simply how I feel most comfortable
as a woman and an out lesbian.

A lot of things got me here, but ``Jennifer's Body'' first showed me the
messy, risky rapture that could await me if I learned to be female on my
own terms.

Advertisement

\protect\hyperlink{after-bottom}{Continue reading the main story}

\hypertarget{site-index}{%
\subsection{Site Index}\label{site-index}}

\hypertarget{site-information-navigation}{%
\subsection{Site Information
Navigation}\label{site-information-navigation}}

\begin{itemize}
\tightlist
\item
  \href{https://help.nytimes.com/hc/en-us/articles/115014792127-Copyright-notice}{©~2020~The
  New York Times Company}
\end{itemize}

\begin{itemize}
\tightlist
\item
  \href{https://www.nytco.com/}{NYTCo}
\item
  \href{https://help.nytimes.com/hc/en-us/articles/115015385887-Contact-Us}{Contact
  Us}
\item
  \href{https://www.nytco.com/careers/}{Work with us}
\item
  \href{https://nytmediakit.com/}{Advertise}
\item
  \href{http://www.tbrandstudio.com/}{T Brand Studio}
\item
  \href{https://www.nytimes.com/privacy/cookie-policy\#how-do-i-manage-trackers}{Your
  Ad Choices}
\item
  \href{https://www.nytimes.com/privacy}{Privacy}
\item
  \href{https://help.nytimes.com/hc/en-us/articles/115014893428-Terms-of-service}{Terms
  of Service}
\item
  \href{https://help.nytimes.com/hc/en-us/articles/115014893968-Terms-of-sale}{Terms
  of Sale}
\item
  \href{https://spiderbites.nytimes.com}{Site Map}
\item
  \href{https://help.nytimes.com/hc/en-us}{Help}
\item
  \href{https://www.nytimes.com/subscription?campaignId=37WXW}{Subscriptions}
\end{itemize}
