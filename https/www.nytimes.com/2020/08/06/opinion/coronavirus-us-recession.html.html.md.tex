Sections

SEARCH

\protect\hyperlink{site-content}{Skip to
content}\protect\hyperlink{site-index}{Skip to site index}

\href{https://myaccount.nytimes.com/auth/login?response_type=cookie\&client_id=vi}{}

\href{https://www.nytimes.com/section/todayspaper}{Today's Paper}

\href{/section/opinion}{Opinion}\textbar{}Coming Next: The Greater
Recession

\href{https://nyti.ms/30BSyBT}{https://nyti.ms/30BSyBT}

\begin{itemize}
\item
\item
\item
\item
\item
\item
\end{itemize}

Advertisement

\protect\hyperlink{after-top}{Continue reading the main story}

\href{/section/opinion}{Opinion}

Supported by

\protect\hyperlink{after-sponsor}{Continue reading the main story}

\hypertarget{coming-next-the-greater-recession}{%
\section{Coming Next: The Greater
Recession}\label{coming-next-the-greater-recession}}

The suspension of federal benefits would create damage almost as
terrifying as the economic effects of the coronavirus.

\href{https://www.nytimes.com/by/paul-krugman}{\includegraphics{https://static01.nyt.com/images/2018/04/02/opinion/paul-krugman/paul-krugman-thumbLarge.png}}

By \href{https://www.nytimes.com/by/paul-krugman}{Paul Krugman}

Opinion Columnist

\begin{itemize}
\item
  Aug. 6, 2020
\item
  \begin{itemize}
  \item
  \item
  \item
  \item
  \item
  \item
  \end{itemize}
\end{itemize}

\includegraphics{https://static01.nyt.com/images/2020/08/08/opinion/06krugmanWeb/06krugmanWeb-articleLarge.jpg?quality=75\&auto=webp\&disable=upscale}

One pretty good forecasting rule for the coronavirus era has been to
take whatever Trump administration officials are saying and assume that
the opposite will happen. When President Trump
\href{https://www.cbsnews.com/news/timeline-president-donald-trump-changing-statements-on-coronavirus/}{declared}
in February that the number of cases would soon go close to zero, you
knew that a huge pandemic was coming. When Vice President Mike Pence
\href{https://www.wsj.com/articles/there-isnt-a-coronavirus-second-wave-11592327890}{insisted}
in mid-June that ``there isn't a coronavirus `second wave,''' a giant
surge in new cases and deaths was clearly imminent.

And when Larry Kudlow, the administration's chief economist,
\href{https://www.foxbusiness.com/economy/kudlow-maintains-v-shaped-economic-recovery-still-intact-despite-coronavirus-resurgence}{declared}
just last week that a ``V-shaped recovery'' was still on track, it was
predictable that the economy would stall.

On Friday, we'll get an official employment report for July. But a
variety of private indicators, like the monthly report from the
data-processing firm \href{https://adpemploymentreport.com/}{ADP},
already suggest that the rapid employment gains of May and June were a
dead-cat bounce and that job growth has at best
\href{https://fred.stlouisfed.org/series/NPPTTL}{slowed to a crawl}.

ADP's number was at least positive --- some
\href{https://twitter.com/ernietedeschi/status/1290971597508169729}{other
indicators} suggest that employment is actually falling. But even if the
small reported job gains were right, at this rate we won't be back to
precoronavirus employment until \ldots{} 2027.

Also, both ADP and the forthcoming official report will be old news ---
basically snapshots of the economy in the second week of July. Since
then much of the country has either
\href{https://www.nytimes.com/interactive/2020/us/states-reopen-map-coronavirus.html}{paused
or reversed} economic reopening, and there are
\href{https://d3n8a8pro7vhmx.cloudfront.net/prosperousamerica/pages/5561/attachments/original/1596512047/Cornell-JQI-RIWI_Poll_Report_-_Second_Wave_of_Layoffs_Well_Under_Way_-_080420_FINAL.pdf?1596512047}{indications}
that many workers rehired during the abortive recovery of May and June
have been laid off again.

But things could get much worse. In fact, they probably will get much
worse unless Republicans get serious about another economic relief
package, and do it very soon.

I'm not sure how many people realize just how much deeper the
coronavirus recession of 2020 could have been. Obviously it was
terrible: Employment plunged, and
\href{https://fred.stlouisfed.org/graph/fredgraph.png?g=tN5v}{real
G.D.P.} fell by around 10 percent. Almost all of that, however,
reflected the direct effects of the pandemic, which forced much of the
economy into lockdown.

What \emph{didn't} happen was a major second round of job losses driven
by plunging consumer demand. Millions of workers lost their regular
incomes; without federal aid, they would have been forced to slash
spending, causing millions more to lose their jobs. Luckily Congress
stepped up to the plate with special aid to the unemployed, which
sustained consumer spending and kept the nonquarantined parts of the
economy afloat.

Now that aid has expired. Democrats offered a plan months ago to
maintain benefits, but Republicans can't even agree among themselves on
a counteroffer. Even if an agreement is hammered out --- and there's no
sign that this is imminent --- it will be weeks before the money is
flowing again.

The suffering among cut-off families will be immense, but there will
also be broad damage to the economy as a whole. How big will this damage
be? I've been doing the math, and it's terrifying.

Unlike affluent Americans, the mostly low-wage workers whose benefits
have just been terminated can't blunt the impact by drawing on savings
or borrowing against assets. So their spending will fall by a lot.
Evidence on the initial effects of emergency aid suggests that the end
of benefits will push overall consumer spending --- the main driver of
the economy --- down by
\href{https://twitter.com/p_ganong/status/1289213387830960128}{more than
4 percent}.

Furthermore, evidence from
\href{https://www.imf.org/external/pubs/ft/wp/2013/wp1301.pdf}{austerity
policies} a decade ago suggests a substantial ``multiplier'' effect, as
spending cuts lead to falling incomes, leading to further spending cuts.

Put it all together and the expiration of emergency aid could produce a
4 percent to 5 percent fall in G.D.P. But wait, there's more. States and
cities are in dire straits and are already planning harsh spending cuts;
but Republicans refuse to provide aid, with Trump insisting, falsely,
that local fiscal crises have
\href{https://twitter.com/kaylatausche/status/1291129767832543233}{nothing
to do} with Covid-19.

Bear in mind that the coronavirus itself --- a shock that came out of
the blue, though the United States mishandled it terribly --- reduced
G.D.P. by ``only'' around 10 percent. What we're looking at now may be
another shock, a sort of economic second wave, almost as severe in
monetary terms as the first. And unlike the pandemic, this shock will be
entirely self-generated, brought on by the fecklessness of President
Trump and --- let's give credit where it's due --- Mitch McConnell, the
Senate majority leader.

The question is, how can this be happening? The 2008 financial crisis
and the sluggish recovery that followed weren't that long ago, and they
taught us valuable lessons directly relevant to our current plight.
Above all, experience in that slump demonstrated both that economic
depressions are no time to obsess over debt and that slashing spending
in the face of mass unemployment is a terrible mistake.

But nobody in the White House or on the G.O.P. side of Capitol Hill
seems to have learned anything from that experience. In fact, \emph{not}
having learned anything from the last crisis almost seems to be a
requirement for Republican economic advisers.

So at the moment we seem to be headed for a Greater Recession --- a
worse slump than 2007-2009, overlaid on the coronavirus slump. MAGA!

\emph{The Times is committed to publishing}
\href{https://www.nytimes.com/2019/01/31/opinion/letters/letters-to-editor-new-york-times-women.html}{\emph{a
diversity of letters}} \emph{to the editor. We'd like to hear what you
think about this or any of our articles. Here are some}
\href{https://help.nytimes.com/hc/en-us/articles/115014925288-How-to-submit-a-letter-to-the-editor}{\emph{tips}}\emph{.
And here's our email:}
\href{mailto:letters@nytimes.com}{\emph{letters@nytimes.com}}\emph{.}

\emph{Follow The New York Times Opinion section on}
\href{https://www.facebook.com/nytopinion}{\emph{Facebook}}\emph{,}
\href{http://twitter.com/NYTOpinion}{\emph{Twitter (@NYTopinion)}}
\emph{and}
\href{https://www.instagram.com/nytopinion/}{\emph{Instagram}}\emph{.}

Advertisement

\protect\hyperlink{after-bottom}{Continue reading the main story}

\hypertarget{site-index}{%
\subsection{Site Index}\label{site-index}}

\hypertarget{site-information-navigation}{%
\subsection{Site Information
Navigation}\label{site-information-navigation}}

\begin{itemize}
\tightlist
\item
  \href{https://help.nytimes.com/hc/en-us/articles/115014792127-Copyright-notice}{©~2020~The
  New York Times Company}
\end{itemize}

\begin{itemize}
\tightlist
\item
  \href{https://www.nytco.com/}{NYTCo}
\item
  \href{https://help.nytimes.com/hc/en-us/articles/115015385887-Contact-Us}{Contact
  Us}
\item
  \href{https://www.nytco.com/careers/}{Work with us}
\item
  \href{https://nytmediakit.com/}{Advertise}
\item
  \href{http://www.tbrandstudio.com/}{T Brand Studio}
\item
  \href{https://www.nytimes.com/privacy/cookie-policy\#how-do-i-manage-trackers}{Your
  Ad Choices}
\item
  \href{https://www.nytimes.com/privacy}{Privacy}
\item
  \href{https://help.nytimes.com/hc/en-us/articles/115014893428-Terms-of-service}{Terms
  of Service}
\item
  \href{https://help.nytimes.com/hc/en-us/articles/115014893968-Terms-of-sale}{Terms
  of Sale}
\item
  \href{https://spiderbites.nytimes.com}{Site Map}
\item
  \href{https://help.nytimes.com/hc/en-us}{Help}
\item
  \href{https://www.nytimes.com/subscription?campaignId=37WXW}{Subscriptions}
\end{itemize}
