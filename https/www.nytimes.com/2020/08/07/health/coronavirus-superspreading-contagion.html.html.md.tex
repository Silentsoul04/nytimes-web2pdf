Sections

SEARCH

\protect\hyperlink{site-content}{Skip to
content}\protect\hyperlink{site-index}{Skip to site index}

\href{https://www.nytimes.com/section/health}{Health}

\href{https://myaccount.nytimes.com/auth/login?response_type=cookie\&client_id=vi}{}

\href{https://www.nytimes.com/section/todayspaper}{Today's Paper}

\href{/section/health}{Health}\textbar{}Why the Coronavirus Is More
Likely to `Superspread' Than the Flu

\href{https://nyti.ms/3kpxWVi}{https://nyti.ms/3kpxWVi}

\begin{itemize}
\item
\item
\item
\item
\item
\end{itemize}

\href{https://www.nytimes.com/news-event/coronavirus?action=click\&pgtype=Article\&state=default\&region=TOP_BANNER\&context=storylines_menu}{The
Coronavirus Outbreak}

\begin{itemize}
\tightlist
\item
  live\href{https://www.nytimes.com/2020/08/08/world/coronavirus-updates.html?action=click\&pgtype=Article\&state=default\&region=TOP_BANNER\&context=storylines_menu}{Latest
  Updates}
\item
  \href{https://www.nytimes.com/interactive/2020/us/coronavirus-us-cases.html?action=click\&pgtype=Article\&state=default\&region=TOP_BANNER\&context=storylines_menu}{Maps
  and Cases}
\item
  \href{https://www.nytimes.com/interactive/2020/science/coronavirus-vaccine-tracker.html?action=click\&pgtype=Article\&state=default\&region=TOP_BANNER\&context=storylines_menu}{Vaccine
  Tracker}
\item
  \href{https://www.nytimes.com/interactive/2020/world/coronavirus-tips-advice.html?action=click\&pgtype=Article\&state=default\&region=TOP_BANNER\&context=storylines_menu}{F.A.Q.}
\item
  \href{https://www.nytimes.com/live/2020/08/07/business/stock-market-today-coronavirus?action=click\&pgtype=Article\&state=default\&region=TOP_BANNER\&context=storylines_menu}{Markets
  \& Economy}
\end{itemize}

Advertisement

\protect\hyperlink{after-top}{Continue reading the main story}

Supported by

\protect\hyperlink{after-sponsor}{Continue reading the main story}

\hypertarget{why-the-coronavirus-is-more-likely-to-superspread-than-the-flu}{%
\section{Why the Coronavirus Is More Likely to `Superspread' Than the
Flu}\label{why-the-coronavirus-is-more-likely-to-superspread-than-the-flu}}

Most people won't spread the virus widely. The few who do are probably
in the wrong place at the wrong time in their infection, new models
suggest.

\includegraphics{https://static01.nyt.com/images/2020/08/07/science/07VIRUS-SUPERSPREADING/07VIRUS-SUPERSPREADING-articleLarge.jpg?quality=75\&auto=webp\&disable=upscale}

By \href{https://www.nytimes.com/by/katherine-j--wu}{Katherine J. Wu}

\begin{itemize}
\item
  Aug. 7, 2020
\item
  \begin{itemize}
  \item
  \item
  \item
  \item
  \item
  \end{itemize}
\end{itemize}

For a spiky sphere just 120 nanometers wide, the coronavirus can be a
remarkably cosmopolitan traveler.

Spewed from the nose or mouth, it can rocket across a room and splatter
onto surfaces; it can waft into poorly ventilated spaces and
\href{https://www.nytimes.com/2020/07/09/health/virus-aerosols-who.html}{linger
in the air for hours}. At its most intrepid, the virus can spread from a
single individual to dozens of others, perhaps even a hundred or more at
once, proliferating through packed crowds in what is called a
\href{https://www.nytimes.com/2020/06/30/science/how-coronavirus-spreads.html}{superspreading
event}.

Such scenarios, which have been traced to
\href{https://wwwnc.cdc.gov/eid/article/26/8/20-1274_article}{call
centers},
\href{https://www.nytimes.com/2020/04/15/us/coronavirus-south-dakota-meat-plant-refugees.html}{meat
processing facilities},
\href{https://wwwnc.cdc.gov/eid/article/26/9/20-1469_article?deliveryName=USCDC_333-DM28908}{weddings}
\href{https://www.nytimes.com/2020/03/30/us/coronavirus-funeral-albany-georgia.html}{and
more}, have
\href{https://www.nytimes.com/2020/06/02/opinion/coronavirus-superspreaders.html}{helped
propel a pandemic} that, in the span of eight months, has reached nearly
every corner of the globe. And yet, while some people seem particularly
apt to spread the coronavirus, others barely pass it on.

``There's this small percentage of people who appear to infect a lot of
people,'' said Dr. Joshua Schiffer, a physician and mathematical
modeling expert who studies infectious diseases at the Fred Hutchinson
Cancer Research Center in Seattle. Estimates vary from population to
population, but they consistently show a striking skew:
\href{https://wellcomeopenresearch.org/articles/5-67}{Between 10} and
\href{https://europepmc.org/article/ppr/ppr165671}{20 percent} of
coronavirus cases may seed 80 percent of new infections. Other
respiratory diseases, like the flu, are far more egalitarian in their
spread.

Figuring out what drives coronavirus superspreading events could be key
to stopping them, and expediting an end to the pandemic. ``That's the
million dollar question,'' said Ayesha Mahmud, who studies infectious
disease dynamics at the University of California, Berkeley.

In a paper
\href{https://www.medrxiv.org/content/10.1101/2020.08.07.20169920v1.full.pdf}{posted
Friday to the website medRxiv} that has
\href{https://www.nytimes.com/2020/04/14/science/coronavirus-disinformation.html}{not
yet been through peer review}, Dr. Schiffer and his colleagues reported
that coronavirus superspreading events were most likely to happen at the
intersection where bad timing and poor placement collide: a person who
has reached the point in their infection when they are shedding large
amounts of virus, and are doing so in a setting where there are plenty
of other people around to catch it.

According to a model built by Dr. Schiffer's team, the riskiest window
for such transmission may be extremely brief --- a one- to two-day
period in the week or so after a person is infected, when coronavirus
levels are at their highest.

\hypertarget{latest-updates-the-coronavirus-outbreak}{%
\section{\texorpdfstring{\href{https://www.nytimes.com/2020/08/07/world/covid-19-news.html?action=click\&pgtype=Article\&state=default\&region=MAIN_CONTENT_1\&context=storylines_live_updates}{Latest
Updates: The Coronavirus
Outbreak}}{Latest Updates: The Coronavirus Outbreak}}\label{latest-updates-the-coronavirus-outbreak}}

Updated 2020-08-08T12:04:28.992Z

\begin{itemize}
\tightlist
\item
  \href{https://www.nytimes.com/2020/08/07/world/covid-19-news.html?action=click\&pgtype=Article\&state=default\&region=MAIN_CONTENT_1\&context=storylines_live_updates\#link-1f86d03a}{As
  the U.S. relief talks falter again, Trump says he is prepared to act
  on his own.}
\item
  \href{https://www.nytimes.com/2020/08/07/world/covid-19-news.html?action=click\&pgtype=Article\&state=default\&region=MAIN_CONTENT_1\&context=storylines_live_updates\#link-3f64a70a}{Cuomo
  says N.Y. schools can reopen in-person but leaves it up to districts
  to determine if, when and how.}
\item
  \href{https://www.nytimes.com/2020/08/07/world/covid-19-news.html?action=click\&pgtype=Article\&state=default\&region=MAIN_CONTENT_1\&context=storylines_live_updates\#link-14e70066}{Thousands
  of cases went unreported in California when a computer server failed.}
\end{itemize}

\href{https://www.nytimes.com/2020/08/07/world/covid-19-news.html?action=click\&pgtype=Article\&state=default\&region=MAIN_CONTENT_1\&context=storylines_live_updates}{See
more updates}

More live coverage:
\href{https://www.nytimes.com/live/2020/08/07/business/stock-market-today-coronavirus?action=click\&pgtype=Article\&state=default\&region=MAIN_CONTENT_1\&context=storylines_live_updates}{Markets}

The virus can still ** spread outside this window, and individuals
outside it should not let up on measures like mask-wearing and physical
distancing, Dr. Schiffer said. But the longer an infection drags on, the
less likely a person is to be contagious --- an idea that might help
experts advise
\href{https://www.nytimes.com/2020/07/22/health/coronavirus-isolation-testing.html}{when
to end self-isolation}, or how to allocate resources to those most in
need, said Dr. Mahmud, who was not involved in the study.

Catching and containing a person at their most infectious is another
matter, however. Some people stricken with the coronavirus start to feel
unwell within a couple days, whereas others take weeks, and many never
end up experiencing symptoms. The length of the so-called
\href{https://advances.sciencemag.org/content/early/2020/08/07/sciadv.abc1202}{incubation
period}, which spans the time between infection and the onset of
symptoms, can be so variable that some people who catch the virus fall
ill before the person who gave it to them does. That rarely happens with
the flu, which reliably rouses a spate of symptoms within a couple days
of infection.

If the coronavirus reaches a peak in the body before symptoms appear ---
if symptoms appear at all --- that increase might be
\href{https://www.nytimes.com/2020/07/19/health/coronavirus-testing-viral-spread.html}{very
tough to identify} without
\href{https://www.nytimes.com/2020/08/06/health/rapid-Covid-tests.html}{frequent
and proactive testing}. Symptom-free spikes in virus load appear to
happen very often, which ``really distorts our ability to tell when
somebody is contagious,'' Dr. Schiffer said. That, in turn, makes it all
too easy for people to obliviously shed the pathogen.

``It really is about opportunity,'' said Shweta Bansal, an infectious
disease ecologist at Georgetown University who was not involved in the
study. ``These processes really come together when you are not only
infected, but you also don't know you're infected because you don't feel
crummy.'' Some of these unwitting coronavirus chauffeurs, emboldened to
go out in public, may end up causing a superspreading event that sends
the pathogen blazing through a new population.

This confluence of factors --- a person in the wrong place at the wrong
point in their infection --- sets the stage for ``explosive
transmission,'' Dr. Bansal said.

The team's model also pointed to another important variable: the
remarkable resilience of the coronavirus when it is aloft.

A growing body of evidence now suggests that the coronavirus
\href{https://www.nytimes.com/2020/07/09/health/virus-aerosols-who.html}{can
be airborne in crowded, poorly ventilated indoor environments}, where it
may encounter many people at once. The virus also travels in larger,
heavier droplets, but these quickly fall to the ground after they are
expelled from the airway and do not have the same reach or longevity as
their smaller counterparts. Dr. Schiffer said he thought the coronavirus
might be more amenable to superspreading than flu viruses because it is
better at
\href{https://www.nejm.org/doi/full/10.1056/NEJMc2004973}{persisting} in
contagious clouds, which can ferry pathogens over relatively long
distances.

``It's a spatial phenomenon,'' he said. ``People further away from the
transmitter may be more likely to be infected.''

\href{https://www.nytimes.com/news-event/coronavirus?action=click\&pgtype=Article\&state=default\&region=MAIN_CONTENT_3\&context=storylines_faq}{}

\hypertarget{the-coronavirus-outbreak-}{%
\subsubsection{The Coronavirus Outbreak
›}\label{the-coronavirus-outbreak-}}

\hypertarget{frequently-asked-questions}{%
\paragraph{Frequently Asked
Questions}\label{frequently-asked-questions}}

Updated August 6, 2020

\begin{itemize}
\item ~
  \hypertarget{why-are-bars-linked-to-outbreaks}{%
  \paragraph{Why are bars linked to
  outbreaks?}\label{why-are-bars-linked-to-outbreaks}}

  \begin{itemize}
  \tightlist
  \item
    Think about a bar. Alcohol is flowing. It can be loud, but it's
    definitely intimate, and you often need to lean in close to hear
    your friend. And strangers have way, way fewer reservations about
    coming up to people in a bar. That's sort of the point of a bar.
    Feeling good and close to strangers. It's no surprise, then, that
    \href{https://www.nytimes.com/2020/07/02/us/coronavirus-bars.html?action=click\&pgtype=Article\&state=default\&region=MAIN_CONTENT_3\&context=storylines_faq}{bars
    have been linked to outbreaks in several states.} Louisiana health
    officials have tied
    \href{https://www.nytimes.com/2020/06/22/us/new-coronavirus-phase.html?action=click\&pgtype=Article\&state=default\&region=MAIN_CONTENT_3\&context=storylines_faq}{at
    least 100 coronavirus cases} to bars in the Tigerland nightlife
    district in Baton Rouge. Minnesota has traced 328 recent cases to
    bars across the state.
    \href{https://www.boisestatepublicradio.org/post/bars-large-venues-close-ada-county-after-surge-coronavirus-prompts-rollback\#stream/0}{In
    Idaho}, health officials shut down bars in Ada County after
    reporting clusters of infections among young adults who had visited
    several bars in downtown Boise. Governors in
    \href{https://www.nytimes.com/2020/07/01/us/california-coronavirus-reopening.html?action=click\&pgtype=Article\&state=default\&region=MAIN_CONTENT_3\&context=storylines_faq}{California},
    \href{https://www.nytimes.com/2020/06/14/us/coronavirus-united-states.html?action=click\&pgtype=Article\&state=default\&region=MAIN_CONTENT_3\&context=storylines_faq}{Texas
    and Arizona}, where coronavirus cases are soaring, have ordered
    hundreds of newly reopened bars to shut down. Less than two weeks
    after Colorado's bars reopened at limited capacity, Gov. Jared Polis
    \href{https://www.denverpost.com/2020/06/30/colorado-bars-closed-coronavirus/}{ordered
    them to close}.
  \end{itemize}
\item ~
  \hypertarget{i-have-antibodies-am-i-now-immune}{%
  \paragraph{I have antibodies. Am I now
  immune?}\label{i-have-antibodies-am-i-now-immune}}

  \begin{itemize}
  \tightlist
  \item
    As of right now,
    \href{https://www.nytimes.com/2020/07/22/health/covid-antibodies-herd-immunity.html?action=click\&pgtype=Article\&state=default\&region=MAIN_CONTENT_3\&context=storylines_faq}{that
    seems likely, for at least several months.} There have been
    frightening accounts of people suffering what seems to be a second
    bout of Covid-19. But experts say these patients may have a
    drawn-out course of infection, with the virus taking a slow toll
    weeks to months after initial exposure. People infected with the
    coronavirus typically
    \href{https://www.nature.com/articles/s41586-020-2456-9}{produce}
    immune molecules called antibodies, which are
    \href{https://www.nytimes.com/2020/05/07/health/coronavirus-antibody-prevalence.html?action=click\&pgtype=Article\&state=default\&region=MAIN_CONTENT_3\&context=storylines_faq}{protective
    proteins made in response to an
    infection}\href{https://www.nytimes.com/2020/05/07/health/coronavirus-antibody-prevalence.html?action=click\&pgtype=Article\&state=default\&region=MAIN_CONTENT_3\&context=storylines_faq}{.
    These antibodies may} last in the body
    \href{https://www.nature.com/articles/s41591-020-0965-6}{only two to
    three months}, which may seem worrisome, but that's perfectly normal
    after an acute infection subsides, said Dr. Michael Mina, an
    immunologist at Harvard University. It may be possible to get the
    coronavirus again, but it's highly unlikely that it would be
    possible in a short window of time from initial infection or make
    people sicker the second time.
  \end{itemize}
\item ~
  \hypertarget{im-a-small-business-owner-can-i-get-relief}{%
  \paragraph{I'm a small-business owner. Can I get
  relief?}\label{im-a-small-business-owner-can-i-get-relief}}

  \begin{itemize}
  \tightlist
  \item
    The
    \href{https://www.nytimes.com/article/small-business-loans-stimulus-grants-freelancers-coronavirus.html?action=click\&pgtype=Article\&state=default\&region=MAIN_CONTENT_3\&context=storylines_faq}{stimulus
    bills enacted in March} offer help for the millions of American
    small businesses. Those eligible for aid are businesses and
    nonprofit organizations with fewer than 500 workers, including sole
    proprietorships, independent contractors and freelancers. Some
    larger companies in some industries are also eligible. The help
    being offered, which is being managed by the Small Business
    Administration, includes the Paycheck Protection Program and the
    Economic Injury Disaster Loan program. But lots of folks have
    \href{https://www.nytimes.com/interactive/2020/05/07/business/small-business-loans-coronavirus.html?action=click\&pgtype=Article\&state=default\&region=MAIN_CONTENT_3\&context=storylines_faq}{not
    yet seen payouts.} Even those who have received help are confused:
    The rules are draconian, and some are stuck sitting on
    \href{https://www.nytimes.com/2020/05/02/business/economy/loans-coronavirus-small-business.html?action=click\&pgtype=Article\&state=default\&region=MAIN_CONTENT_3\&context=storylines_faq}{money
    they don't know how to use.} Many small-business owners are getting
    less than they expected or
    \href{https://www.nytimes.com/2020/06/10/business/Small-business-loans-ppp.html?action=click\&pgtype=Article\&state=default\&region=MAIN_CONTENT_3\&context=storylines_faq}{not
    hearing anything at all.}
  \end{itemize}
\item ~
  \hypertarget{what-are-my-rights-if-i-am-worried-about-going-back-to-work}{%
  \paragraph{What are my rights if I am worried about going back to
  work?}\label{what-are-my-rights-if-i-am-worried-about-going-back-to-work}}

  \begin{itemize}
  \tightlist
  \item
    Employers have to provide
    \href{https://www.osha.gov/SLTC/covid-19/standards.html}{a safe
    workplace} with policies that protect everyone equally.
    \href{https://www.nytimes.com/article/coronavirus-money-unemployment.html?action=click\&pgtype=Article\&state=default\&region=MAIN_CONTENT_3\&context=storylines_faq}{And
    if one of your co-workers tests positive for the coronavirus, the
    C.D.C.} has said that
    \href{https://www.cdc.gov/coronavirus/2019-ncov/community/guidance-business-response.html}{employers
    should tell their employees} -\/- without giving you the sick
    employee's name -\/- that they may have been exposed to the virus.
  \end{itemize}
\item ~
  \hypertarget{what-is-school-going-to-look-like-in-september}{%
  \paragraph{What is school going to look like in
  September?}\label{what-is-school-going-to-look-like-in-september}}

  \begin{itemize}
  \tightlist
  \item
    It is unlikely that many schools will return to a normal schedule
    this fall, requiring the grind of
    \href{https://www.nytimes.com/2020/06/05/us/coronavirus-education-lost-learning.html?action=click\&pgtype=Article\&state=default\&region=MAIN_CONTENT_3\&context=storylines_faq}{online
    learning},
    \href{https://www.nytimes.com/2020/05/29/us/coronavirus-child-care-centers.html?action=click\&pgtype=Article\&state=default\&region=MAIN_CONTENT_3\&context=storylines_faq}{makeshift
    child care} and
    \href{https://www.nytimes.com/2020/06/03/business/economy/coronavirus-working-women.html?action=click\&pgtype=Article\&state=default\&region=MAIN_CONTENT_3\&context=storylines_faq}{stunted
    workdays} to continue. California's two largest public school
    districts --- Los Angeles and San Diego --- said on July 13, that
    \href{https://www.nytimes.com/2020/07/13/us/lausd-san-diego-school-reopening.html?action=click\&pgtype=Article\&state=default\&region=MAIN_CONTENT_3\&context=storylines_faq}{instruction
    will be remote-only in the fall}, citing concerns that surging
    coronavirus infections in their areas pose too dire a risk for
    students and teachers. Together, the two districts enroll some
    825,000 students. They are the largest in the country so far to
    abandon plans for even a partial physical return to classrooms when
    they reopen in August. For other districts, the solution won't be an
    all-or-nothing approach.
    \href{https://bioethics.jhu.edu/research-and-outreach/projects/eschool-initiative/school-policy-tracker/}{Many
    systems}, including the nation's largest, New York City, are
    devising
    \href{https://www.nytimes.com/2020/06/26/us/coronavirus-schools-reopen-fall.html?action=click\&pgtype=Article\&state=default\&region=MAIN_CONTENT_3\&context=storylines_faq}{hybrid
    plans} that involve spending some days in classrooms and other days
    online. There's no national policy on this yet, so check with your
    municipal school system regularly to see what is happening in your
    community.
  \end{itemize}
\end{itemize}

Since the start of the pandemic, many comparisons have been drawn
between Covid-19 and the flu, both of which are diseases caused by
viruses that attack the respiratory tract. But
\href{https://www.cdc.gov/flu/symptoms/flu-vs-covid19.htm}{plenty of
differences exist}, and in many ways the coronavirus is more formidable.
``This study adds yet another layer to how it's different from
influenza,'' said Olivia Prosper, a researcher at the University of
Tennessee, Knoxville who uses mathematical models to study infectious
diseases but was not involved in the study. ``It's not just about how
sick it makes you, but also its ability to transmit.''

Moreover, certain people may be predisposed to be more generous
transmitters of the coronavirus, although the details are ``still a
mystery,'' Dr. Schiffer said.

But when a superspreading event occurs, it likely has
\href{https://www.nytimes.com/2020/06/30/science/how-coronavirus-spreads.html}{more
to do with the circumstances} than with a single person's biology, Dr.
Schiffer said. Even someone carrying a lot of the coronavirus can stave
off mass transmission by avoiding large groups, thus depriving the germ
of conduits to travel.

``A superspreading event is a function of what somebody's viral load is
and if they're in a crowded space,'' he said. ``If those are the two
levers, you can control the crowding bit.''

Both Dr. Mahmud and Dr. Prosper noted that not everyone has the means to
practice physical distancing. Some people work essential jobs in packed
environments, for instance, and are left more vulnerable to the
consequences of superspreading events.

That makes it all the more important for those who can participate in
control measures like mask-wearing and physical distancing to remain
vigilant about their behavior, Dr. Mahmud said.

``That's what we should be doing,'' she said. ``Not just to protect
ourselves, but to protect others.''

\textbf{\emph{{[}}\href{http://on.fb.me/1paTQ1h}{\emph{Like the Science
Times page on Facebook.}}} ****** \emph{\textbar{} Sign up for the}
\textbf{\href{http://nyti.ms/1MbHaRU}{\emph{Science Times
newsletter.}}\emph{{]}}}

Advertisement

\protect\hyperlink{after-bottom}{Continue reading the main story}

\hypertarget{site-index}{%
\subsection{Site Index}\label{site-index}}

\hypertarget{site-information-navigation}{%
\subsection{Site Information
Navigation}\label{site-information-navigation}}

\begin{itemize}
\tightlist
\item
  \href{https://help.nytimes.com/hc/en-us/articles/115014792127-Copyright-notice}{©~2020~The
  New York Times Company}
\end{itemize}

\begin{itemize}
\tightlist
\item
  \href{https://www.nytco.com/}{NYTCo}
\item
  \href{https://help.nytimes.com/hc/en-us/articles/115015385887-Contact-Us}{Contact
  Us}
\item
  \href{https://www.nytco.com/careers/}{Work with us}
\item
  \href{https://nytmediakit.com/}{Advertise}
\item
  \href{http://www.tbrandstudio.com/}{T Brand Studio}
\item
  \href{https://www.nytimes.com/privacy/cookie-policy\#how-do-i-manage-trackers}{Your
  Ad Choices}
\item
  \href{https://www.nytimes.com/privacy}{Privacy}
\item
  \href{https://help.nytimes.com/hc/en-us/articles/115014893428-Terms-of-service}{Terms
  of Service}
\item
  \href{https://help.nytimes.com/hc/en-us/articles/115014893968-Terms-of-sale}{Terms
  of Sale}
\item
  \href{https://spiderbites.nytimes.com}{Site Map}
\item
  \href{https://help.nytimes.com/hc/en-us}{Help}
\item
  \href{https://www.nytimes.com/subscription?campaignId=37WXW}{Subscriptions}
\end{itemize}
