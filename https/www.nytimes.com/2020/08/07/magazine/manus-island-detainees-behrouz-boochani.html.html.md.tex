Sections

SEARCH

\protect\hyperlink{site-content}{Skip to
content}\protect\hyperlink{site-index}{Skip to site index}

\href{https://myaccount.nytimes.com/auth/login?response_type=cookie\&client_id=vi}{}

\href{https://www.nytimes.com/section/todayspaper}{Today's Paper}

The Men Australia Detained in a Secretive Detention Camp

\href{https://nyti.ms/3gBETjP}{https://nyti.ms/3gBETjP}

\begin{itemize}
\item
\item
\item
\item
\item
\end{itemize}

Advertisement

\protect\hyperlink{after-top}{Continue reading the main story}

Supported by

\protect\hyperlink{after-sponsor}{Continue reading the main story}

At War

\hypertarget{the-men-australia-detained-in-a-secretive-detention-camp}{%
\section{The Men Australia Detained in a Secretive Detention
Camp}\label{the-men-australia-detained-in-a-secretive-detention-camp}}

While profiling Behrouz Boochani, the story of the detention where he
and others were held was underpinned by a sinister and sorrowful mood
that ran through every person I interviewed.

\includegraphics{https://static01.nyt.com/images/2020/08/09/magazine/09mag-boochani-07/09mag-boochani-07-articleLarge.jpg?quality=75\&auto=webp\&disable=upscale}

By Megan K. Stack

\begin{itemize}
\item
  Aug. 7, 2020
\item
  \begin{itemize}
  \item
  \item
  \item
  \item
  \item
  \end{itemize}
\end{itemize}

\emph{\textbf{You're reading this week's At War newsletter.}}
\textbf{\href{https://www.nytimes.com/newsletters/at-war}{\emph{Sign up
here}}} \emph{\textbf{to get it delivered to your inbox every Friday.
Email us at}}
\textbf{\href{https://mail.google.com/mail/u/0/?view=cm\&fs=1\&tf=1\&source=mailto\&to=atwar@nytimes.com}{\emph{atwar@nytimes.com}}\emph{.}}

I was interviewing Behrouz Boochani over a cafe table in Port Moresby,
the capital of Papua New Guinea, when he reached across, tore a sheet
out of my notebook and began to draw. Hunched over the paper, he labeled
a crooked square ``MANUS PRISON.'' He ran a hand along the page's edge
--- here is the ocean. A road stretched along a spindly line to end in a
circle: Lorengau, the nearest town.

``You have to be able to see it,'' he muttered as he drew. ``This is
important.''

Studying the lines upside down, I tried to fill them out, in my
imagination, with Boochani's descriptions of Australia's detention camp
on Manus Island, which has since been demolished. I'd read about the
camp in Boochani's book, ``No Friend But The Mountains,'' and studied
footage shot there, but the map added another dimension. Meanwhile,
Boochani was talking about hunger strike, solitary confinement and the
complex social structures of detained men. Movement on the table caught
my eye: Boochani's pen was still sawing over the square marked
``prison.'' Back and forth, over and over --- black ink on black ink.
Boochani sat backlit by the brilliant tapestry of a tropical sunset,
oblivious to the compulsive movements of his own hand: black, black,
black over the square of the Manus detention center. The sheet thinned;
finally the pen broke through. Feeling the catch and tear, Boochani
looked down in surprise and then glanced at me; had I noticed? I said
nothing. His picture had become so black it had fallen apart altogether.

Boochani's hand accidentally told the truth that day. The story of Manus
was underpinned by a sinister and sorrowful mood that ran through every
interview I did. The things that happened at Manus, the events
themselves, can be explained: arrests and protests; mealtimes and
interviews. But as I interviewed the refugees --- men who had fled war,
persecution and ethnic cleansing in their home countries --- I kept
uncovering a bleak and existential aspect to the story: They believed
they had suffered a state-sponsored attempt to destroy their psyches.

I heard a lot of stories from the refugees in Port Moresby. None of
these tales was cheerful. ``This darkness Australia opened up here'' an
Iranian refugee named Ali Fardmavini told me. ``It will never go away.''
A compact, athletic figure who spoke with intense dignity and sometimes
seemed to quiver on the brink of tears that never came, Fardmavini
apologized repeatedly as we spoke: After years on Manus, he said, his
memory had degraded and his mind didn't function as well as it had
before.

Fardmavini told me about a man he called the Doctor. Sayed Mirwais
Rohani, a young Afghan physician held at Manus, had asked camp officials
for permission to volunteer in a local hospital. Presumably, like the
other refugees, he was desperate to give his days structure and meaning.
But Rohani was denied. And then, Fardmavini told me, something strange
happened: Rohani started walking around without clothes. Perhaps the
nakedness was a protest, a relinquishment of human norms, a cry for
help. Certainly it was an advertisement of his vulnerability and,
according to Fardmavini, it drew vicious attacks.

Rohani's father, himself a refugee settled in London, came to Papua New
Guinea to check on his son and, noting his condition, begged the
Australian government for mercy. When at last Rohani was transferred to
Australia on medical grounds, he was held in ``community detention,'' a
program designed as a humanitarian concession but which has,
nevertheless, been criticized for the meager and heavily stigmatized
lifestyle it offers. Two years after reaching Australia, Rohani threw
himself out a 22nd floor window.

``They took this guy's life,'' Fardmavini concluded. ``They burned it
like a candle.''

This story stunned me. My first instinct was to dissect the symbolism of
a man who ended his life by smashing himself against the same piece of
earth that wouldn't accept him. I thought about Australia, a promise
that had drawn the asylum seekers, heads full of bright daydreams. I
thought about how land is dirt, but also a nation and an idea, about how
the land Rohani desired had become the instrument of his undoing.

Then I stopped. The truth is, these men are not symbols or myths or
characters of fiction. They are human beings, or they were. Some of them
have been destroyed altogether. Some are still detained in Port Moresby,
their fates undetermined. Some of them are walking around, trying to
start their lives anew.

I couldn't properly express the darkness of Manus in a simple story. It
must be inferred from the people themselves, their stories and eyes; or
you can read it in the book, which is steeped in ominous mood. And what
about the other unseen places --- is there a Boochani languishing in ICE
detention? The Boochanis of statelessness, of extralegal limbo, of
geopolitical fallout --- these writers will not always be comfortable to
study. When it comes to this kind of story, we --- the citizens of
governments that signed the conventions and pledged to protect human
rights --- used to be the sympathizers and the saviors. Now, it seems,
we're running the camps.

\emph{Megan K. Stack is an author and a journalist living in Washington.
Her most recent book is ``Women's Work: A Personal Reckoning With Labor,
Motherhood, and Privilege.''}

\begin{center}\rule{0.5\linewidth}{\linethickness}\end{center}

\hypertarget{from-beyond-the-world-war-ii-we-know}{%
\subsection{From Beyond the World War II We
Know}\label{from-beyond-the-world-war-ii-we-know}}

\includegraphics{https://static01.nyt.com/images/2020/08/06/multimedia/06ww2-bombing-ogawa-01/merlin_175276872_dfcc7576-6b4a-46d4-9658-02bc79839fd2-articleLarge.jpg?quality=75\&auto=webp\&disable=upscale}

The Japanese novelist Yoko Ogawa reflects on the literature unleashed by
the atomic bombings. ``With the help of literature, the words of the
dead may be gathered and placed carefully aboard their small boat, to
flow on to join the stream of reality.''
{[}\textbf{\href{https://www.nytimes.com/2020/08/06/magazine/hiroshima-nagasaki-japan-literature.html}{Read
the article.}}{]}

Image

Claude Eatherly speaking with a news reporter in a Dallas city jail
after he was arrested for attempted armed robbery in March
1959.Credit...FK/Associated Press

We remember Claude Eatherly, the one pilot who expressed remorse for his
involvement in the Hiroshima bombing. His life fell apart, probably
because of it, but he also became a symbol for antinuclear activism.
{[}\textbf{\href{https://www.nytimes.com/2020/08/06/magazine/hiroshima-claude-eatherly-antinuclear.html}{Read
the article.}}{]}

\emph{\textbf{More from the 75th anniversary of the bombing of
Hiroshima:}}

\begin{itemize}
\item
  \textbf{\href{https://www.nytimes.com/2020/08/06/world/asia/hiroshima-japan-setsuko-thurlow.html}{Witnessing
  Nuclear Carnage, Then Devoting Her Life to Peace}}
\item
  \textbf{\href{https://www.nytimes.com/2020/08/06/world/asia/hiroshima-nagasaki-japan-photos.html}{After
  Atomic Bombings, These Photographers Worked Under Mushroom Clouds}}
\item
  \textbf{\href{https://www.nytimes.com/2020/08/03/books/review/unconditional-marc-gallicchio.html}{Why
  the U.S. Dropped Atomic Bombs on Japan}}
\item
  \textbf{\href{https://www.nytimes.com/2020/08/05/world/asia/hiroshima-japan-75th-anniversary.html}{Hiroshima
  75th Anniversary: Preserving Survivors' Message of Peace}}
\end{itemize}

\begin{center}\rule{0.5\linewidth}{\linethickness}\end{center}

\hypertarget{afghan-war-casualty-report-august-2020}{%
\subsection{Afghan War Casualty Report: August
2020}\label{afghan-war-casualty-report-august-2020}}

Image

Afghan soldiers walk past debris near the main entrance of a prison
after a raid in Jalalabad on Aug. 3.Credit...Noorullah Shirzada/Agence
France-Presse --- Getty Images

At least 42 pro-government forces and 41 civilians have been killed in
Afghanistan in August so far.
{[}\textbf{\href{https://www.nytimes.com/2020/08/06/magazine/afghan-war-casualty-report-august-2020.html}{Read
the report.}}{]}

\begin{center}\rule{0.5\linewidth}{\linethickness}\end{center}

\hypertarget{editors-picks}{%
\subsection{Editor's Picks}\label{editors-picks}}

\emph{\textbf{Here are four articles from The Times that you might have
missed.}}

Image

A military accessory shop in Schwerin whose owner was part of the
Nordkreuz group.Credit...Gordon Welters for The New York Times

\textbf{``I fear we've only seen the tip of the iceberg.''} Germany has
woken up to a problem of far-right extremism in its elite special
forces. But the threat of neo-Nazi infiltration of state institutions is
much broader.
{[}\href{https://www.nytimes.com/2020/08/01/world/europe/germany-nazi-infiltration.html}{Read
the article.}{]}

\textbf{``We're going down to 4,000, we're negotiating right now.''}
President Trump said that there would be fewer than 5,000 American
troops in Afghanistan by Election Day in November, signaling that the
United States would continue to withdraw troops from the country despite
limited progress toward the start of peace negotiations between the
Afghan government and the Taliban.
{[}\href{https://www.nytimes.com/2020/08/04/world/asia/us-troops-afghanistan.html}{Read
the article.}{]}

\textbf{``Journalists don't always need years of schooling in a subject
to be able to report effectively on it.''} The At War reporter John
Ismay writes about how his military background helped The Times cover
the U.S.S. Bonhomme Richard fire, the beating of the Navy veteran
Christopher David in Portland, Ore., and the explosions on Beirut's
waterfront.
{[}\href{https://www.nytimes.com/2020/08/06/insider/bomb-training-beirut-explosions.html}{Read
the article.}{]}

\textbf{``It's getting hard to describe Turkey as an ally of the U.S.''}
Turkey --- increasingly assertive, ambitious and authoritarian --- has
become ``the elephant in the room'' for NATO, European diplomats say.
But it is a matter, they say, that few want to discuss.
{[}\href{https://www.nytimes.com/2020/08/03/world/europe/turkey-nato.html}{Read
the article.}{]}

\begin{center}\rule{0.5\linewidth}{\linethickness}\end{center}

\textbf{How are we doing?}

\emph{We'd love your feedback on this newsletter. Please email thoughts
and suggestions to}
\href{mailto:atwar@nytimes.com?subject=Newsletter\%20Feedback}{\emph{atwar@nytimes.com}}\emph{.
Or invite someone to subscribe through}
\href{https://www.nytimes.com/newsletters/at-war}{\emph{this link.}}
\emph{Read more from At War}
\href{https://www.nytimes.com/atwar}{\emph{here.}}

\emph{Follow us on}
\href{https://twitter.com/NYTimesAtWar}{\emph{Twitter}} \emph{for more
from At War.}

Advertisement

\protect\hyperlink{after-bottom}{Continue reading the main story}

\hypertarget{site-index}{%
\subsection{Site Index}\label{site-index}}

\hypertarget{site-information-navigation}{%
\subsection{Site Information
Navigation}\label{site-information-navigation}}

\begin{itemize}
\tightlist
\item
  \href{https://help.nytimes.com/hc/en-us/articles/115014792127-Copyright-notice}{©~2020~The
  New York Times Company}
\end{itemize}

\begin{itemize}
\tightlist
\item
  \href{https://www.nytco.com/}{NYTCo}
\item
  \href{https://help.nytimes.com/hc/en-us/articles/115015385887-Contact-Us}{Contact
  Us}
\item
  \href{https://www.nytco.com/careers/}{Work with us}
\item
  \href{https://nytmediakit.com/}{Advertise}
\item
  \href{http://www.tbrandstudio.com/}{T Brand Studio}
\item
  \href{https://www.nytimes.com/privacy/cookie-policy\#how-do-i-manage-trackers}{Your
  Ad Choices}
\item
  \href{https://www.nytimes.com/privacy}{Privacy}
\item
  \href{https://help.nytimes.com/hc/en-us/articles/115014893428-Terms-of-service}{Terms
  of Service}
\item
  \href{https://help.nytimes.com/hc/en-us/articles/115014893968-Terms-of-sale}{Terms
  of Sale}
\item
  \href{https://spiderbites.nytimes.com}{Site Map}
\item
  \href{https://help.nytimes.com/hc/en-us}{Help}
\item
  \href{https://www.nytimes.com/subscription?campaignId=37WXW}{Subscriptions}
\end{itemize}
