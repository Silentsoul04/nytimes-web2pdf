Sections

SEARCH

\protect\hyperlink{site-content}{Skip to
content}\protect\hyperlink{site-index}{Skip to site index}

\href{https://myaccount.nytimes.com/auth/login?response_type=cookie\&client_id=vi}{}

\href{https://www.nytimes.com/section/todayspaper}{Today's Paper}

\href{/section/opinion}{Opinion}\textbar{}The Presidential Debates
Debate

\href{https://nyti.ms/2DO3Rh5}{https://nyti.ms/2DO3Rh5}

\begin{itemize}
\item
\item
\item
\item
\item
\end{itemize}

Advertisement

\protect\hyperlink{after-top}{Continue reading the main story}

\href{/section/opinion}{Opinion}

Supported by

\protect\hyperlink{after-sponsor}{Continue reading the main story}

letters

\hypertarget{the-presidential-debates-debate}{%
\section{The Presidential Debates
Debate}\label{the-presidential-debates-debate}}

Does this forum make sense for evaluating the candidates? Readers weigh
in.

Aug. 7, 2020

\begin{itemize}
\item
\item
\item
\item
\item
\end{itemize}

\includegraphics{https://static01.nyt.com/images/2020/08/06/opinion/06drew3-inyt/merlin_75655960_325e36f8-8421-4c4e-b014-1b44f7d93169-articleLarge.jpg?quality=75\&auto=webp\&disable=upscale}

\textbf{To the Editor:}

Re Elizabeth Drew's Op-Ed article
``\href{https://www.nytimes.com/2020/08/03/opinion/trump-biden-presidential-debates-2020.html?searchResultPosition=1}{Scrap
the Presidential Debates}'' (Aug. 4): Let's not.

While Ms. Drew's criticisms are fair, they miss a bigger point. For six
decades, the singular benefit of the televised debates has been that
they let voters see candidates talk to each other face to face ---
something that happens nowhere else in the course of a presidential
election.

Particularly in our modern campaigns, saturated as they are in dark
money and social media advertising, don't Americans deserve some genuine
interaction between the candidates on a national platform?

The debates are the only time in a modern campaign when voters see
candidates think on their feet and speak at length and extemporaneously,
without the benefit of script or consultants, armed with nothing but
their character and intellect. The debates give voters multiple
opportunities to see how candidates handle pressure.

And the televised debate is a feature of the American presidential
campaign that other countries admire. More than 90 countries now have
some kind of leader debates, and most seek guidance on organizing them
from the Commission on Presidential Debates, on which I am a board
member.

I've been involved one way or another in every televised presidential
debate and know full well that such debates are not perfect --- never
have been, never will be. Winston Churchill believed that democracy was
the worst form of government \ldots{} except for all the others. The
same truth applies to the debates.

Newton N. Minow\\
Chicago

\textbf{To the Editor:}

Debates are like job interviews. I once heard an executive search
professional say that the worst indicator of future job performance is
the job interview; the best indicator of future job performance is past
job performance.

Yes, do away with presidential sideshows, and tell the electorate the
real story.

Susan Maggiotto\\
Hastings-on-Hudson, N.Y.

\textbf{To the Editor:}

Here's an idea: Instead of scrapping the presidential ``debates,'' why
don't we actually make the candidates debate?

Candidates should make their case using the accepted debate formalisms
(point/rebuttal/cross-examination \ldots) to minimize interruptions and
digressions. There should be several debates, each about one hot policy
topic that the candidates clearly disagree on.

Each candidate gets to assert two or three affirmative positions to
favor, and each must rebut two or three positions asserted by the
opponent. Each should bring a favored policy expert to help make the
case. I care less about the eloquence of the candidates than I do about
the solidity of the advice they are getting and the solidity of their
decision-making process.

Augustus P. Lowell\\
Durham, N.H.

\textbf{To the Editor:}

I support Elizabeth Drew's proposal that we scrap presidential debates.
Commentators and observers focus on winners and losers, which tends only
to deepen the nation's partisan animosities.

I would expand her scope and promote scrapping all debates, in politics
and in our schools. After all, the strategy of debate is to avoid
supporting any reasonable observations by your opponent. It's an
exercise in being unreasonable.

Instead, I suggest a forum of ``common dialogue.'' In this setting, the
candidates sit at a round table. The moderator presents a problem that
needs the attention of leaders. The candidates have 45 minutes to reach
consensus on a solution. Candidates can comment often, but each comment
would be limited to two minutes.

Audience members would gain firsthand evidence of candidates'
intelligence, biases and ability to lead toward consensus. There are no
losers. The community wins.

Tad Dunne\\
Adrian, Mich.\\
\emph{The writer is a philosophy professor at Siena Heights University.}

\textbf{To the Editor:}

Elizabeth Drew has a point: The presidential debates are essentially
worthless. If anything, they're worse than that: They give an undeserved
edge to show-business types, to out-to-lunch types who sincerely believe
nonsense and to seasoned liars, none of whom deserve to be president of
the United States of America.

The most powerful elected office in the world shouldn't go to the
candidate who looks best on TV, sounds best on the radio or is quickest
with an irrelevant quip in front of a microphone.

But most people don't seem to want to pay attention to the presidential
campaign over the long haul, the ``better way'' Ms. Drew suggests.
They'd rather decide early and cruise to the voting booth on autopilot.

Democracy works when the voters actually pay attention to what the
candidates say and think carefully about whether it's actually true, or
even plausible. That didn't happen in 2016, when real questions about
Donald Trump's ignorance and dishonesty were submerged beneath cries of
``But her emails!'' We've been paying for that ever since.

Eric B. Lipps\\
Staten Island

\textbf{To the Editor:}

Nervous Democrats are drawing the wagons around their presidential
nominee.

The only inference to be drawn from the likes of Elizabeth Drew's
column, and by Democrats urging Joe Biden not to debate President Trump,
is that doing so will reveal Mr. Biden's lack of mental agility, of
which Mr. Biden has already provided numerous examples. If Mr. Biden
refuses to debate Donald Trump, millions of voters will deem him a
coward. And they will be right.

Gerald Katz\\
Edwards, Colo.

\textbf{To the Editor:}

If Joe Biden enters that debate circus, the bully ringmaster in chief,
Donald Trump, will interrupt, insult, lie to and badger him incessantly.

As a candidate, Mr. Biden has maintained his dignity and continues to
deliver quiet, measured messages to America, and he could hold his own
in a debate. But why should he subject himself to that onslaught? And
why give President Trump another bully pulpit?

Merritt H. Cohen\\
East Hanover, N.J.

\textbf{To the Editor:}

Elizabeth Drew's characterization of the debates as ``professional
wrestling matches'' is unfortunately true. But as Thomas L. Friedman
states in his column
``\href{https://www.nytimes.com/2020/07/07/opinion/biden-trump-debate.html}{Biden
Should Not Debate Trump Unless \ldots{}}'' (July 8), two conditions
should be enforced if there are to be presidential debates this year:
The candidates must reveal their tax statements (``Biden has already
done so''), and fact-checking teams must be present to reveal false
statements made by either candidate.

I would add a third condition: \emph{no} live audience, a must in these
pandemic times, and certainly less conducive to a wrestling match or
circuslike atmosphere.

Joan Berglund\\
Orient, N.Y.

\textbf{To the Editor:}

I completely agree with Elizabeth Drew about scrapping the debates. Any
pretense that the president of the United States alone can ``fix any
problem'' should be scrapped as well.

The power to make laws is vested in Congress. The duty of the president
is to enforce the laws made by Congress. The duty of the judiciary is to
interpret those laws. This is basic U. S. constitutional law, of which
every candidate, every voter and everyone in the media should be aware.

Jane Langseth\\
Colts Neck, N.J.

Advertisement

\protect\hyperlink{after-bottom}{Continue reading the main story}

\hypertarget{site-index}{%
\subsection{Site Index}\label{site-index}}

\hypertarget{site-information-navigation}{%
\subsection{Site Information
Navigation}\label{site-information-navigation}}

\begin{itemize}
\tightlist
\item
  \href{https://help.nytimes.com/hc/en-us/articles/115014792127-Copyright-notice}{©~2020~The
  New York Times Company}
\end{itemize}

\begin{itemize}
\tightlist
\item
  \href{https://www.nytco.com/}{NYTCo}
\item
  \href{https://help.nytimes.com/hc/en-us/articles/115015385887-Contact-Us}{Contact
  Us}
\item
  \href{https://www.nytco.com/careers/}{Work with us}
\item
  \href{https://nytmediakit.com/}{Advertise}
\item
  \href{http://www.tbrandstudio.com/}{T Brand Studio}
\item
  \href{https://www.nytimes.com/privacy/cookie-policy\#how-do-i-manage-trackers}{Your
  Ad Choices}
\item
  \href{https://www.nytimes.com/privacy}{Privacy}
\item
  \href{https://help.nytimes.com/hc/en-us/articles/115014893428-Terms-of-service}{Terms
  of Service}
\item
  \href{https://help.nytimes.com/hc/en-us/articles/115014893968-Terms-of-sale}{Terms
  of Sale}
\item
  \href{https://spiderbites.nytimes.com}{Site Map}
\item
  \href{https://help.nytimes.com/hc/en-us}{Help}
\item
  \href{https://www.nytimes.com/subscription?campaignId=37WXW}{Subscriptions}
\end{itemize}
