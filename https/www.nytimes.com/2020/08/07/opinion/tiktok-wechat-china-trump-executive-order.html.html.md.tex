Sections

SEARCH

\protect\hyperlink{site-content}{Skip to
content}\protect\hyperlink{site-index}{Skip to site index}

\href{https://myaccount.nytimes.com/auth/login?response_type=cookie\&client_id=vi}{}

\href{https://www.nytimes.com/section/todayspaper}{Today's Paper}

\href{/section/opinion}{Opinion}\textbar{}So What Does Trump Have
Against TikTok?

\href{https://nyti.ms/3fFHFDc}{https://nyti.ms/3fFHFDc}

\begin{itemize}
\item
\item
\item
\item
\item
\end{itemize}

Advertisement

\protect\hyperlink{after-top}{Continue reading the main story}

\href{/section/opinion}{Opinion}

Supported by

\protect\hyperlink{after-sponsor}{Continue reading the main story}

\hypertarget{so-what-does-trump-have-against-tiktok}{%
\section{So What Does Trump Have Against
TikTok?}\label{so-what-does-trump-have-against-tiktok}}

His foolish and dangerous edicts suggest that the United States, like
China, no longer believes in a global internet.

\includegraphics{https://static01.nyt.com/images/2020/01/07/opinion/josephine-wolff/josephine-wolff-thumbLarge.png}

By Josephine Wolff

Dr. Wolff is an assistant professor of cybersecurity policy at Tufts
University and a contributing opinion writer.

\begin{itemize}
\item
  Aug. 7, 2020
\item
  \begin{itemize}
  \item
  \item
  \item
  \item
  \item
  \end{itemize}
\end{itemize}

\includegraphics{https://static01.nyt.com/images/2020/08/07/opinion/07Wolff/07Wolff-articleLarge.jpg?quality=75\&auto=webp\&disable=upscale}

The one thing my students all invariably know about China is that you
can't use Facebook there, or YouTube or Google. For at least a decade,
China has maintained strict control over the internet and aggressively
blocked foreign tech platforms within its borders.

So when President Trump
\href{https://www.nytimes.com/2020/08/06/technology/trump-wechat-tiktok-china.html}{issued
two executive orders Thursday night} that all but ban two Chinese social
media networks --- the video app TikTok and the messaging app WeChat ---
from operating in the United States, citing national security concerns,
the decision seemed straight out of China's own playbook.

The executive orders and
\href{https://blogs.microsoft.com/blog/2020/08/02/microsoft-to-continue-discussions-on-potential-tiktok-purchase-in-the-united-states/}{Microsoft's
interest} in buying TikTok's American business echo what happened in
2017, when China's cybersecurity law went into effect and required
foreign companies to store data about Chinese customers within China.
Some American companies, including
\href{https://www.wsj.com/articles/amazon-to-sell-its-china-cloud-computing-business-1510628802}{Amazon},
had to sell the hardware components of their cloud computing services in
China to Chinese companies in order to continue operating there.

The United States government's approach to cybersecurity is now looking
more and more like China's. If that meant only limiting access to
humorous video apps then it would be merely unfortunate. But it's a
deeply misguided and unproductive way to try to secure data and computer
networks --- one that relies on the profoundly untrue assumption that
data stored within a country's own borders is more secure than data
stored in other places.

No one knows better than the United States government that the data kept
within its borders is highly vulnerable to Chinese cyberespionage. In
2015,
\href{https://www.nytimes.com/2015/07/10/us/office-of-personnel-management-hackers-got-data-of-millions.html}{Chinese
hackers stole personal information} belonging to more than 21 million
people from the federal government's Office of Personnel Management. In
2017, members of the Chinese military managed to steal records belonging
to 145 million Americans from the U.S. credit bureau Equifax, according
to
\href{https://www.justice.gov/opa/press-release/file/1246891/download}{charges
filed by the Department of Justice} earlier this year.

Any number of lessons could be drawn from these incidents, including the
importance of vetting outside vendors and the need to carefully monitor
outbound data. But deciding that information is more secure because it
is collected and stored by American companies is precisely the wrong
conclusion.

In January, the Department of Defense
\href{https://www.nytimes.com/2020/01/04/us/tiktok-pentagon-military-ban.html}{announced}
that military personnel would be required to remove TikTok from their
government-issued smartphones. Even absent any evidence that ByteDance
was sharing data with the Chinese government, that decision made sense
for smartphones that were being used by military officers given the
sensitive nature of their work. But for the government to expand that
ban to the phones of civilians in the United States, it needs to show
some clearer indication that the app poses a real risk to its users.
Otherwise, this just looks like an anti-competitive decision made to
disadvantage a Chinese tech firm in the name of strengthening security.

It's not clear whether the Trump administration regards either TikTok's
or WeChat's data, or their parent companies, as particularly pernicious
or dangerous, but it has not released any evidence that these companies
are distributing compromised software to their users via the apps or
sharing any data about their American customers with the Chinese
government.

But make no mistake: the president's executive orders are not about
cybersecurity --- they are a retaliatory jab in the ongoing tensions
between China and the United States. In fact, the ban's greatest impact
will probably not be on the bottom lines of TikTok and WeChat's parent
companies, but instead on promoting a fundamentally Chinese view of ****
internet security.

For years, the American government has championed the idea of an open
and global internet, in which the same online content and services are
available worldwide, regardless of where users live. Tech companies
could operate internationally, moving data freely between their data
centers across the globe. But if the government now believes that the
only safe data and computer networks are within its own borders --- as
the animus toward TikTok and WeChat suggests --- then, **** like China,
the United States fundamentally does not believe in a global internet.
That's a terrible mistake for a country whose tech industry depends
heavily on companies that do business all over the world. It's also a
mistake from a security perspective.

To protect Americans' data, the federal government needs to set clearer
and more rigorous standards for how that data is protected and what the
consequences are for failing to meet those standards. By pretending that
restricting the use of TikTok and WeChat could possibly serve the same
--- or even a similar --- purpose, the government is failing to engage
with the hard questions around liability for cybersecurity breaches.
Instead, it is buying into China's belief that the only way to secure
the internet is to keep international influences and services offline.

\emph{The Times is committed to publishing}
\href{https://www.nytimes.com/2019/01/31/opinion/letters/letters-to-editor-new-york-times-women.html}{\emph{a
diversity of letters}} \emph{to the editor. We'd like to hear what you
think about this or any of our articles. Here are some}
\href{https://help.nytimes.com/hc/en-us/articles/115014925288-How-to-submit-a-letter-to-the-editor}{\emph{tips}}\emph{.
And here's our email:}
\href{mailto:letters@nytimes.com}{\emph{letters@nytimes.com}}\emph{.}

\emph{Follow The New York Times Opinion section on}
\href{https://www.facebook.com/nytopinion}{\emph{Facebook}}\emph{,}
\href{http://twitter.com/NYTOpinion}{\emph{Twitter (@NYTopinion)}}
\emph{and}
\href{https://www.instagram.com/nytopinion/}{\emph{Instagram}}\emph{.}

Advertisement

\protect\hyperlink{after-bottom}{Continue reading the main story}

\hypertarget{site-index}{%
\subsection{Site Index}\label{site-index}}

\hypertarget{site-information-navigation}{%
\subsection{Site Information
Navigation}\label{site-information-navigation}}

\begin{itemize}
\tightlist
\item
  \href{https://help.nytimes.com/hc/en-us/articles/115014792127-Copyright-notice}{©~2020~The
  New York Times Company}
\end{itemize}

\begin{itemize}
\tightlist
\item
  \href{https://www.nytco.com/}{NYTCo}
\item
  \href{https://help.nytimes.com/hc/en-us/articles/115015385887-Contact-Us}{Contact
  Us}
\item
  \href{https://www.nytco.com/careers/}{Work with us}
\item
  \href{https://nytmediakit.com/}{Advertise}
\item
  \href{http://www.tbrandstudio.com/}{T Brand Studio}
\item
  \href{https://www.nytimes.com/privacy/cookie-policy\#how-do-i-manage-trackers}{Your
  Ad Choices}
\item
  \href{https://www.nytimes.com/privacy}{Privacy}
\item
  \href{https://help.nytimes.com/hc/en-us/articles/115014893428-Terms-of-service}{Terms
  of Service}
\item
  \href{https://help.nytimes.com/hc/en-us/articles/115014893968-Terms-of-sale}{Terms
  of Sale}
\item
  \href{https://spiderbites.nytimes.com}{Site Map}
\item
  \href{https://help.nytimes.com/hc/en-us}{Help}
\item
  \href{https://www.nytimes.com/subscription?campaignId=37WXW}{Subscriptions}
\end{itemize}
