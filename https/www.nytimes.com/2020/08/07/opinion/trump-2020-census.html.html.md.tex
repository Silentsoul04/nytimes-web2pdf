Sections

SEARCH

\protect\hyperlink{site-content}{Skip to
content}\protect\hyperlink{site-index}{Skip to site index}

\href{https://myaccount.nytimes.com/auth/login?response_type=cookie\&client_id=vi}{}

\href{https://www.nytimes.com/section/todayspaper}{Today's Paper}

\href{/section/opinion}{Opinion}\textbar{}Trump and His Allies Think
They Know Who Counts

\href{https://nyti.ms/2PAvRr6}{https://nyti.ms/2PAvRr6}

\begin{itemize}
\item
\item
\item
\item
\item
\item
\end{itemize}

Advertisement

\protect\hyperlink{after-top}{Continue reading the main story}

\href{/section/opinion}{Opinion}

Supported by

\protect\hyperlink{after-sponsor}{Continue reading the main story}

\hypertarget{trump-and-his-allies-think-they-know-who-counts}{%
\section{Trump and His Allies Think They Know Who
Counts}\label{trump-and-his-allies-think-they-know-who-counts}}

But history has a way of confounding those who think they can control
it.

\href{https://www.nytimes.com/column/jamelle-bouie}{\includegraphics{https://static01.nyt.com/images/2019/01/24/opinion/jamelle-bouie/jamelle-bouie-thumbLarge-v3.png}}

By \href{https://www.nytimes.com/column/jamelle-bouie}{Jamelle Bouie}

Opinion Columnist

\begin{itemize}
\item
  Aug. 7, 2020
\item
  \begin{itemize}
  \item
  \item
  \item
  \item
  \item
  \item
  \end{itemize}
\end{itemize}

\includegraphics{https://static01.nyt.com/images/2020/08/08/opinion/08bouie_print/merlin_153869544_fe018d11-1ded-4eef-909b-30ab36d22516-articleLarge.jpg?quality=75\&auto=webp\&disable=upscale}

Over the last few years, the Trump administration has fought to shape
the 2020 census to its political benefit and the benefit of the
Republican Party. In 2018, it sought to
\href{https://www.scotusblog.com/2019/07/trump-administration-ends-effort-to-include-citizenship-question-on-2020-census/}{introduce}
a citizenship question on the census itself, to reduce response rates
among immigrant communities. Then, after that was rebuffed by the
Supreme Court, it tried to
\href{https://www.vox.com/policy-and-politics/2020/7/21/21328714/trump-executive-order-immigration-census-2020-redistricting}{exclude}
unauthorized immigrants altogether, in direct conflict with the
Constitution, which calls on Congress to count ``the whole number of
persons in each State.'' Now it wants to
\href{https://www.npr.org/2020/08/03/898548910/census-cut-short-a-month-rushes-to-finish-all-counting-efforts-by-sept-30}{cut}
the census short and deliver it uncompleted --- a last-ditch effort to
rig the nation's politics for the sake of its exclusionary political
vision.

The goal is to freeze political representation in place as much as
possible; to keep demographic change --- the growing share of Americans
who are Black, Hispanic and Asian-American --- from swamping the
Republican Party's ability to win national elections with a white,
heavily rural minority.

The census, as Trump and his allies correctly understand, is a critical
source of dynamism within the American political system. A political
majority (or in Trump's case, a
\href{https://www.nytimes.com/2020/08/04/opinion/trump-2020-electoral-college.html}{minority})
can try to insulate itself from demographic shifts and transformations,
but the fact of mandatory reapportionment makes that difficult. New
people --- whether immigrants or Americans moving from place to place
--- will always mean new politics.

It is ironic, then, that the origin of the census lies less in
principles of democratic representation, and more in the interests of
slaveholders, who wanted political recognition of their slave wealth,
with constitutional assurance that this peculiar interest would always
weigh on future apportionment. But in a perfect example of unintended
consequences, the slaveholders' push for a census would help lay the
groundwork for the end of the institution itself.

The decennial federal census comes out of the fight over congressional
representation at the Constitutional Convention. Upon gathering in
Philadelphia in 1787, the delegates agreed quickly that the United
States should have a bicameral legislature, in keeping with the Virginia
Plan, James Madison's blueprint for a new national government. They
agreed, too, that the lower house of Congress should be directly elected
by voters, with the upper house chosen indirectly. But they disagreed,
sharply, over apportionment.

Madison's plan called for apportioning representation in both chambers
of the national legislature
\href{https://avalon.law.yale.edu/18th_century/vatexta.asp}{according}
to ``the quotas of contribution, or to the number of free inhabitants,
as the one or the other rule may seem best in different cases.''
Proportional representation, he thought, would lead larger states like
Pennsylvania and his native Virginia to join the union, since they would
have greater say in government. As would the smaller states of the lower
South --- North Carolina, South Carolina and Georgia --- which were
expected to experience rapid growth as a result of new migrants and the
``\href{https://www.monticello.org/thomas-jefferson/jefferson-slavery/the-business-of-slavery-at-monticello/}{natural
increase}'' of slaves.

Of course, it wouldn't be so easy. Under the original Articles of
Confederation, each state claimed equal representation in Congress.
Small state delegates like those from Delaware and Connecticut liked
that arrangement and sought to preserve it as much as possible. Against
supporters of population-based apportionment --- who
\href{https://avalon.law.yale.edu/18th_century/debates_629.asp}{noted}
it was ``the rights of the people composing'' the states who deserved
representation --- small state delegates argued that the federal
government was to be formed for states ``in their political capacity, as
well as for the individuals composing them.'' Besides, they continued,
larger states would dominate the government if the convention abandoned
the principle of equal representation.

The solution, as most Americans know, was the ``Great Compromise,'' in
which equal state voting would survive in the Senate and proportional
representation would prevail in the House of Representatives. This was a
momentous decision, not just because it kept the convention from falling
apart, although it did, but because it dictated the shape of the
compromise over \emph{how} to actually proportion representation.

At the time, Michael J. Klarman, a legal historian, noted in
``\href{https://global.oup.com/academic/product/the-framers-coup-9780199942039?facet_narrowbyreleaseDate_facet=Released\%20this\%20month\&facet_narrowbybinding_facet=Ebook\&facet_narrowbytype_facet=General\%20Interest\&lang=en\&cc=us}{The
Framers' Coup}: The Making of the United States Constitution,'' that
``most elite statesmen believed that political representation ought to
reflect wealth as well as population'' and ``several state constitutions
provided for legislative apportionment based partly on wealth.'' As
Charles Cotesworth Pinckney of South Carolina argued, the South's
``superior wealth'' should have ``its due weight in the government.''
And northern delegates like Rufus King of Massachusetts sympathized with
this view, confessing he ``had always expected that as the southern
states are the richest, they would not league themselves with the
northern unless some respect were paid to their superior wealth.''

If equal state representation --- which disregarded the size and wealth
of each state --- was the rule for the Senate, then proportional
representation in the House had to factor in wealth, including the
ownership of slaves, the major economic interest for the South. This led
us to the three-fifths clause, based off a proposed ``federal ratio''
for taxation under the Articles, which ensured slave wealth
representation. ``The three-fifths clause,'' the historian George
William Van Cleve writes in
``\href{https://press.uchicago.edu/ucp/books/book/chicago/S/bo9270100.html}{A
Slaveholders' Union}: Slavery, Politics, and the Constitution in the
early American Republic,'' ``was the explicitly chosen
political-security foundation for the constitutional bargain protecting
the political economy of the slave states.''

Even still, in its initial apportionment of the House, the committee
responsible gave the eight northern states a modest seven-seat advantage
over the five southern states, \href{https://d.pr/f/hDu7Kf}{36 to 29}.
More important, as the historian Jack N. Rakove explains in
``\href{https://www.penguinrandomhouse.com/books/137447/original-meanings-by-jack-n-rakove/}{Original
Meanings}: Politics and Ideas in the Making of the Constitution,'' the
committee left reapportionment up to the discretion of Congress. ``The
Atlantic States having the government in their own hands, may take care
of their own interest,'' explained Nathaniel Gorham of Massachusetts,
``by dealing out the right of Representation in safe proportions to the
Western States.''

This was a problem for the Southerners, who were already unhappy with
their initial minority status in the Legislature. Discretionary
reapportionment gave the northern majority control over the political
future of the region. As I said earlier, there was broad expectation of
rapid growth in the South and its western lands, including among
enslaved people. Would a northern majority account for slave growth in
its reapportionment? Would it give equal political representation to the
migrants of the Southwest? Or would it entrench itself against
demographic change? ``Those who have power in their hands,''
\href{https://avalon.law.yale.edu/18th_century/debates_711.asp}{warned}
George Mason of Virginia, ``will not give it up while they can retain
it.''

The solution was to take reapportionment out of the hands of Congress.
``According to the present population of America,'' Mason
\href{https://teachingamericanhistory.org/resources/ratification/elliot/vol5/0711_1787/}{declared},
``the northern part of it had a right to preponderate, and he could not
deny it. But he wished it not to preponderate hereafter when the reason
no longer continued.''

Northern delegates resisted, but they lost. ``The apportionment of
representatives in the future,'' Klarman writes, ``would be based on a
census, which the Constitution would require Congress to undertake
within three years of its first meeting and then again once every
decade.'' And slaves would be counted on the same three-fifths basis as
they were for the initial apportionment of the House. To assuage a
northern public that might object to representation for enslaved people,
a Pennsylvania delegate, Gouverneur Morris, proposed a clause to tie
representation to taxation, which had not yet been under discussion.

Instead of saying outright that enslaved people would count for
representation, they would link representation to
``\href{https://avalon.law.yale.edu/18th_century/debates_913.asp}{direct
taxes}'' (which no delegate \href{https://d.pr/n/B8wyIJ}{expected} the
federal government to ever impose) and link \emph{that} to a population
that included slaves. ``The delegates could pretend that they were not
doing what they were actually doing,'' the historian Robin L. Einhorn
explains in
``\href{https://books.google.com/books/about/American_Taxation_American_Slavery.html?id=97qls5TNm_8C\&source=kp_book_description}{American
Taxation, American Slavery}.'' She quotes delegate James Wilson of
Pennsylvania making this exact point: ``Less umbrage would perhaps be
taken'' against ``an admission of the slaves into the rule of
representation, if it should be so expressed as to make them indirectly
only an ingredient in the rule, by saying they should enter into the
rule of taxation: and as representation was to be according to taxation,
the end would be equally attained.''

In other words, as with so much of the Constitution of 1787, the census
is wrapped up in slavery as an institution of significant political and
economic influence. And the slaveholder gambit worked, for a time. As
slavery grew to new heights in the first decades of the 19th century,
mandatory reapportionment gave greater influence to the slaveholding
South, providing it with a strong grip on the federal government.

But what no one at the time of the founding could have anticipated was
mass immigration to the Northern states and its territories. Millions of
immigrants --- the bulk arriving from Germany, Ireland and Britain ---
reached American shores between 1830 and 1860. Rather than settle in the
South to compete with enslaved Africans, they remained in the North,
moving to cities like New York and Boston or going west to states like
Ohio, Michigan, Missouri and Wisconsin.

These immigrants changed the face of American politics. Germans, in
particular, would play a significant role in the mass antislavery
politics of the 1850s. ``German émigrés joined existing radical
movements, the labor movement, land reform, and abolition while others
became free soilers,'' the historian Manisha Sinha writes in
``\href{https://yalebooks.yale.edu/book/9780300227116/slaves-cause}{The
Slave's Cause}: A History of Abolition.'' German refugees from the
failed revolutions of 1848 ``formed alliances with abolitionists and
brought a substantial section of the German immigrant population into
the Republican Party.''

And the census, of course, helped ensure that these demographic and
cultural and ideological changes would make their way into Congress. The
decade before the Civil War saw an influx of antislavery congressmen
into the House of Representatives, first as Free Soilers, then as
Republicans. Indeed, it is the rise of a popular antislavery politics
that sets up the legislative confrontations and political realignments
of the 1850s that culminated in the election of Abraham Lincoln in 1860.

The census is completely unassuming. Almost no one outside of
politicians, bureaucrats and the professionally interested thinks about
it, or about reapportionment. But these provisions are quietly powerful
parts of our constitutional order. Their creation, Van Cleve notes,
meant acceptance of the idea that the political majority ``should be
continuously represented in government, no matter where that majority
was found within the nation's expanding boundaries.'' It meant that no
existing political majority could ever fully insulate itself from the
winds of change.

Southern slaveholders were, among the delegates to Philadelphia, the
least committed to popular government. South Carolina, to use one
example, would be a planter oligarchy until after the Civil War. But in
their drive to protect their political and economic interests, they
introduced a mechanism for population representation that eventually
helped fuel the forces of abolition.

None of this was inevitable, and it was certainly unintended. If there
is a greater lesson here, it has everything to do with chance and
circumstance and the contingency of human affairs. It's a reminder that
in the political realm there are no final victories or permanent
defeats. At a time when just such dreams and fears are pushing our
politics to dangerous places, this is very much something worth
remembering.

\emph{The Times is committed to publishing}
\href{https://www.nytimes.com/2019/01/31/opinion/letters/letters-to-editor-new-york-times-women.html}{\emph{a
diversity of letters}} \emph{to the editor. We'd like to hear what you
think about this or any of our articles. Here are some}
\href{https://help.nytimes.com/hc/en-us/articles/115014925288-How-to-submit-a-letter-to-the-editor}{\emph{tips}}\emph{.
And here's our email:}
\href{mailto:letters@nytimes.com}{\emph{letters@nytimes.com}}\emph{.}

\emph{Follow The New York Times Opinion section on}
\href{https://www.facebook.com/nytopinion}{\emph{Facebook}}\emph{,}
\href{http://twitter.com/NYTOpinion}{\emph{Twitter (@NYTopinion)}}
\emph{and}
\href{https://www.instagram.com/nytopinion/}{\emph{Instagram}}\emph{.}

Advertisement

\protect\hyperlink{after-bottom}{Continue reading the main story}

\hypertarget{site-index}{%
\subsection{Site Index}\label{site-index}}

\hypertarget{site-information-navigation}{%
\subsection{Site Information
Navigation}\label{site-information-navigation}}

\begin{itemize}
\tightlist
\item
  \href{https://help.nytimes.com/hc/en-us/articles/115014792127-Copyright-notice}{©~2020~The
  New York Times Company}
\end{itemize}

\begin{itemize}
\tightlist
\item
  \href{https://www.nytco.com/}{NYTCo}
\item
  \href{https://help.nytimes.com/hc/en-us/articles/115015385887-Contact-Us}{Contact
  Us}
\item
  \href{https://www.nytco.com/careers/}{Work with us}
\item
  \href{https://nytmediakit.com/}{Advertise}
\item
  \href{http://www.tbrandstudio.com/}{T Brand Studio}
\item
  \href{https://www.nytimes.com/privacy/cookie-policy\#how-do-i-manage-trackers}{Your
  Ad Choices}
\item
  \href{https://www.nytimes.com/privacy}{Privacy}
\item
  \href{https://help.nytimes.com/hc/en-us/articles/115014893428-Terms-of-service}{Terms
  of Service}
\item
  \href{https://help.nytimes.com/hc/en-us/articles/115014893968-Terms-of-sale}{Terms
  of Sale}
\item
  \href{https://spiderbites.nytimes.com}{Site Map}
\item
  \href{https://help.nytimes.com/hc/en-us}{Help}
\item
  \href{https://www.nytimes.com/subscription?campaignId=37WXW}{Subscriptions}
\end{itemize}
