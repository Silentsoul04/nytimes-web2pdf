\href{/podcasts/the-daily}{The Daily}\textbar{}Jack Dorsey on Twitter's
Mistakes

\href{https://nyti.ms/2XCscxL}{https://nyti.ms/2XCscxL}

\begin{itemize}
\item
\item
\item
\item
\item
\end{itemize}

\includegraphics{https://static01.nyt.com/images/2020/08/10/podcasts/10daily-dorsey/10daily-dorsey-articleLarge-v2.jpg?quality=75\&auto=webp\&disable=upscale}

Sections

\protect\hyperlink{site-content}{Skip to
content}\protect\hyperlink{site-index}{Skip to site index}

\hypertarget{jack-dorsey-on-twitters-mistakes}{%
\section{Jack Dorsey on Twitter's
Mistakes}\label{jack-dorsey-on-twitters-mistakes}}

The social network's C.E.O. has plans to improve the platform. But will
the changes be enough?

Credit...Getty Images

Supported by

\protect\hyperlink{after-sponsor}{Continue reading the main story}

\textbf{This episode of ``The Daily'' was hosted by Michael Barbaro,
produced by Andy Mills and Rachel Quester, and edited by Lisa Tobin.}

This article was written by Lauren Jackson and Desiree Ibekwe.

\emph{\textbf{Listen and subscribe to our podcast from your mobile
device:}}
\textbf{\href{https://itunes.apple.com/us/podcast/the-daily/id1200361736?mt=2}{\emph{Via
Apple Podcasts}}} \emph{\textbf{\textbar{}}}
\textbf{\href{https://open.spotify.com/show/3IM0lmZxpFAY7CwMuv9H4g?si=SfuMSC55R1qprFsRZU3_zw}{\emph{Via
Spotify}}} \emph{\textbf{\textbar{}}}
\textbf{\href{http://www.stitcher.com/podcast/the-new-york-times/the-daily-10}{\emph{Via
Stitcher}}}

\includegraphics{https://static01.nyt.com/images/2017/01/29/podcasts/the-daily-album-art/the-daily-album-art-articleInline-v2.jpg?quality=75\&auto=webp\&disable=upscale}

\hypertarget{listen-to-the-daily-jack-dorsey-on-twitters-mistakes}{%
\subsubsection{Listen to `The Daily': Jack Dorsey on Twitter's
Mistakes}\label{listen-to-the-daily-jack-dorsey-on-twitters-mistakes}}

``Do you believe that you are one of the most powerful people on Earth
right now?''

Jack Dorsey, Twitter's C.E.O., was quick to deflect. ``No,'' he laughed.
``Everything that has made Twitter powerful has come from the people
using it.''

That was one of the first questions that Michael asked Mr. Dorsey in a
wide-ranging interview on the role of his platform in shaping public
discourse and beliefs in America. The conversation that followed probed
the limits, dangers and responsibilities of that role --- revealing just
how complicated the struggle against misinformation and polarization on
social media can be.

In the four years since the 2016 election revealed social media's role
in the American electoral process, governments around the world have
\href{https://www.nytimes.com/2020/08/04/technology/europe-big-tech.html}{grappled
with how to regulate} the scale and scope of tech giants. Some social
networks, like Facebook, have been slower to acknowledge a need for
change, but Twitter has embraced the challenge, acknowledging that the
company made mistakes in the past. Still,
\href{https://www.nytimes.com/2020/05/30/technology/twitter-trump-dorsey.html}{Twitter
itself is divided} over how to design products and policies that assign
visibility to users working in the public interest.

With three months to go until the 2020 election, we asked Mr. Dorsey to
reflect on his regrets and on his platform's flaws --- and on what he
hopes to change as he forges a new path for his company.

\hypertarget{three-key-moments-from-our-interview-with-twitters-ceo}{%
\subsection{Three key moments from our interview with Twitter's
C.E.O.}\label{three-key-moments-from-our-interview-with-twitters-ceo}}

\hypertarget{what-the-platform-should-have-done-differently-from-the-start}{%
\subsubsection{What the platform should have done differently from the
start}\label{what-the-platform-should-have-done-differently-from-the-start}}

Twitter was founded without a plan, Mr. Dorsey said. ``It wasn't
something we really invented, it was something we discovered. And we
kept pulling the thread on it.''

The unraveling was ``electric,'' he said, as the small, localized
platform he built for friends to share updates on their lives morphed
into a global social network. In the process, though, Mr. Dorsey said he
now believes that he made a critical mistake: not hiring experts to help
him understand the potentially far-reaching importance of apparently
small design choices.

``The disciplines that we were lacking in the company in the early days,
that I wish we would have understood and hired for,'' he said, were ``a
game theorist to just really understand the ramifications of tiny
decisions that we make, such as what happens with retweet versus retweet
with comment and what happens when you put a count next to a like
button?''

Without this expertise, he said he thought that the company had built
incentives into the app that encouraged users and media outlets to write
tweets and headlines that appealed to sensationalism instead of
accuracy. At the time, he noted, he struggled to envision the app's
potential social implications --- and what those design decisions might
mean for ``how people interrelate with one another, how people converse
with one another.''

Still, he said he believed that his company had played a correlative,
not a causal role in shaping public discourse --- amplifying trends that
existed ``in parallel'' to the platform.

Abuse and harassment did not start after this polarization or the
political dialogue coming on Twitter,'' he said. ``It's been on the
internet forever.''

\hypertarget{building-nuance-and-openness-into-the-platforms-design}{%
\subsubsection{Building nuance and openness into the platform's
design}\label{building-nuance-and-openness-into-the-platforms-design}}

Mr. Dorsey said that he, and his company, intended to learn from past
mistakes.

``It would be silly for us not to change Twitter,'' he said. To Mr.
Dorsey, the company ``should become irrelevant if it doesn't change, if
it doesn't constantly evolve and if it doesn't recognize gaps and
opportunities to get better.''

He said he also hoped to build that openness, and admission of
wrongdoing, into the platform's discourse. ``It's important that we
continue to allow the space for people to express their past and their
history in context,'' he said, responding to the critique that Twitter,
with its limited character count and incentives for pith, promotes
intolerance.

``If we can't express that, we can't learn from it, and then we can't
really progress,'' he said, ``or improve as a culture, or as individuals
either.''

To do this, Mr. Dorsey said that he was considering alterations to how
Twitter worked. In some iterations of the platform's algorithm, he said,
``the most salacious or controversial tweets will naturally rise to the
top because those are the things that people naturally click on or share
without thinking about it or reply to.''

Solving this issue, he said, requires demystifying the code that governs
social networks --- an issue the industry has generally shied away from.
``They are way too much of a black box,'' he said.

``We need to open up and be transparent around how our algorithms work
and how they're used, and maybe even enable people to choose their own
algorithms to rank the content or to create their own algorithms, to
rank it. To be that open, I think, would be pretty incredible.''

\hypertarget{should-president-trump-be-censored}{%
\subsubsection{Should President Trump be
censored?}\label{should-president-trump-be-censored}}

Since he first sat behind the Resolute Desk, iPhone in hand, President
Trump has
\href{https://www.nytimes.com/interactive/2019/11/02/us/politics/trump-twitter-presidency.html}{reshaped
the presidency} --- and the nation --- with the help of more than 11,000
tweets. His account is a forum for early-morning musings, personal
vendettas and off-the-cuff policy decisions. The platform has also
facilitated digital connections between the president and
\href{https://www.nytimes.com/interactive/2019/11/02/us/politics/trump-twitter-disinformation.html}{extremists,
impostors and spies}.

While Mr. Dorsey acknowledged that President Trump had leveraged
Twitter's algorithms to create visibility ``to great effect,'' he
challenged the assertion that Twitter gave the president an unmediated
platform to share his views --- and misinformation.

``I think it's important that we do recognize, number one, that these
annotations are happening by the crowd in real time all the time,'' he
said, noting that through replies, comments and retweets with added
context, users around the world have the opportunity to qualify,
challenge or engage with the president.

However, he said he did believe that there were particular areas in
which Twitter had a responsibility to intervene --- specifically with
regard to language that encourages violence or voter suppression or that
challenges electoral integrity.

Twitter has recently risked provoking Mr. Trump's ire after
\href{https://www.nytimes.com/2020/06/23/technology/trump-twitter-label-seattle.html}{placing
some of the president's tweets} behind a warning label saying that they
violated the company's policies forbidding abusive behavior. Mr. Dorsey
said that the platform was working to make the decision-making process
about when to apply such warnings ``as tight as possible'' and that he
hoped to ``make those interventions as infrequent as possible.''

Still, he added, the company ``won't hesitate'' to take action on
accounts that violate Twitter's terms of service.

\textbf{Background reading:}

\begin{itemize}
\item
  A
  \href{https://www.nytimes.com/2020/08/02/technology/florida-teenager-twitter-hack.html}{17-year-old
  in Florida} was recently responsible for one of the worst hacking
  attacks in Twitter's history --- successfully breaching the accounts
  of some of the world's most famous people, including Barack Obama,
  Kanye West and Elon Musk.
  \href{https://www.nytimes.com/2020/07/16/us/politics/twitter-hack.html}{But
  did the teenager do the country a favor}?
\item
  \href{https://www.nytimes.com/2020/08/04/technology/europe-big-tech.html}{Twitter
  is in hot water} with the government for sharing with advertisers
  phone numbers given to the company for personal security purposes.
\end{itemize}

\emph{Tune in, and tell us what you think. Email us at}
\href{mailto:thedaily@nytimes.com}{\emph{thedaily@nytimes.com}}\emph{.
Follow Michael Barbaro on Twitter:}
\href{https://twitter.com/mikiebarb}{\emph{@mikiebarb}}\emph{. And if
you're interested in advertising with ``The Daily,'' write to us at}
\href{mailto:thedaily-ads@nytimes.com}{\emph{thedaily-ads@nytimes.com}}\emph{.}

``The Daily'' is made by Theo Balcomb, Andy Mills, Lisa Tobin, Rachel
Quester, Lynsea Garrison, Annie Brown, Clare Toeniskoetter, Paige
Cowett, Michael Simon Johnson, Brad Fisher, Larissa Anderson, Wendy
Dorr, Chris Wood, Jessica Cheung, Stella Tan, Alexandra Leigh Young,
Lisa Chow, Eric Krupke, Marc Georges, Luke Vander Ploeg, Kelly Prime,
Julia Longoria, Sindhu Gnanasambandan, M.J. Davis Lin, Austin Mitchell,
Neena Pathak, Dan Powell, Dave Shaw, Sydney Harper, Daniel Guillemette,
Hans Buetow, Robert Jimison, Mike Benoist, Bianca Giaever, Liz O. Baylen
and Asthaa Chaturvedi. Our theme music is by Jim Brunberg and Ben
Landsverk of Wonderly. Special thanks to Sam Dolnick, Mikayla Bouchard,
Lauren Jackson, Julia Simon, Mahima Chablani, Nora Keller and Desiree
Ibekwe.

Advertisement

\protect\hyperlink{after-bottom}{Continue reading the main story}

\hypertarget{site-index}{%
\subsection{Site Index}\label{site-index}}

\hypertarget{site-information-navigation}{%
\subsection{Site Information
Navigation}\label{site-information-navigation}}

\begin{itemize}
\tightlist
\item
  \href{https://help.nytimes.com/hc/en-us/articles/115014792127-Copyright-notice}{©~2020~The
  New York Times Company}
\end{itemize}

\begin{itemize}
\tightlist
\item
  \href{https://www.nytco.com/}{NYTCo}
\item
  \href{https://help.nytimes.com/hc/en-us/articles/115015385887-Contact-Us}{Contact
  Us}
\item
  \href{https://www.nytco.com/careers/}{Work with us}
\item
  \href{https://nytmediakit.com/}{Advertise}
\item
  \href{http://www.tbrandstudio.com/}{T Brand Studio}
\item
  \href{https://www.nytimes.com/privacy/cookie-policy\#how-do-i-manage-trackers}{Your
  Ad Choices}
\item
  \href{https://www.nytimes.com/privacy}{Privacy}
\item
  \href{https://help.nytimes.com/hc/en-us/articles/115014893428-Terms-of-service}{Terms
  of Service}
\item
  \href{https://help.nytimes.com/hc/en-us/articles/115014893968-Terms-of-sale}{Terms
  of Sale}
\item
  \href{https://spiderbites.nytimes.com}{Site Map}
\item
  \href{https://help.nytimes.com/hc/en-us}{Help}
\item
  \href{https://www.nytimes.com/subscription?campaignId=37WXW}{Subscriptions}
\end{itemize}
