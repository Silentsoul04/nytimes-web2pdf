Sections

SEARCH

\protect\hyperlink{site-content}{Skip to
content}\protect\hyperlink{site-index}{Skip to site index}

\href{/section/sports}{Sports}\textbar{}`When You Say Badminton, You Say
Indonesia'

\href{https://nyti.ms/3fDCa80}{https://nyti.ms/3fDCa80}

\begin{itemize}
\item
\item
\item
\item
\item
\end{itemize}

\includegraphics{https://static01.nyt.com/images/2020/08/08/sports/08indonesia-badminton-02/08indonesia-badminton-02-articleLarge.jpg?quality=75\&auto=webp\&disable=upscale}

\hypertarget{when-you-say-badminton-you-say-indonesia}{%
\section{`When You Say Badminton, You Say
Indonesia'}\label{when-you-say-badminton-you-say-indonesia}}

Badminton is part of the national identity. It is the only sport in
which Indonesia has won an Olympic gold medal, and the country had
expected to add another this month in Tokyo.

There are 800 badminton clubs spread across Indonesia. Games --- both
organized and informal --- are common sights around the
country.Credit...

Supported by

\protect\hyperlink{after-sponsor}{Continue reading the main story}

Photographs and Text by James Hill

\begin{itemize}
\item
  Aug. 7, 2020
\item
  \begin{itemize}
  \item
  \item
  \item
  \item
  \item
  \end{itemize}
\end{itemize}

JAKARTA --- Raja Sapta Oktohari, the muscular and youthful president of
the Indonesian Olympic Committee, was finding it hard to contain his
enthusiasm.

Badminton, he explained in an interview early this year, is more than a
casual pastime in his country. It is part of the country's social
fabric, a game played by families in backyards and cramped public spaces
and by shop workers waiting for clients.

``When you say badminton, you say Indonesia,'' Oktohari declared. ``That
is how important it is.''

So it was a blow to Indonesia's sporting culture when the Tokyo Olympics
were postponed earlier this year because of the growing coronavirus
pandemic. Badminton is the only sport in which Indonesia has won an
Olympic gold medal, a feat its players have achieved seven times. It is,
in an Olympic year, the only sport that matters here.

The coronavirus has tested that commitment since The Times visited in
February to document badminton's place in Indonesian life, but it hasn't
dimmed it a bit. Slowly but surely, the game and its players are
emerging from lockdown. For months, training centers and courts in
Jakarta have been closed, but any easing of rules will revive familiar
routines, even if coaching instructions will have to come from behind
masks and face shields.

The Olympics, rescheduled for next year, are never far from the players'
minds. The national squad recently held an internal tournament ``so they
could not feel bored and can measure the results of training programs''
during the lockdown, one official said.

Coaches and players hope Jakarta's clubs will rumble back to life soon,
too, bringing the sport --- and its future --- out of its temporary
hibernation.

It is in those smaller gyms and neighborhoods where the sport that has
been nurtured for decades by mentors like Christian Hadinata, a
70-year-old former world champion. In regular times, Hadinata could be
found each weekday morning at 6 a.m. on the courts of the Djarum
Badminton Club in Jakarta, waiting for his students to arrive.

Uncle Chris, as Hadinata is known to the junior players, sees his
contributions as paying back a debt to his sport, and his country, by
passing on the learning of his lifetime. It is an obligation that
Hadinata says he has felt since the Munich Olympics of 1972, when
badminton was first presented as a demonstration sport. He won the men's
doubles that summer, but it was a victory without a medal or an anthem
and one, he said, that was quickly ``overwhelmed by the tragedy caused
by Black September.''

\includegraphics{https://static01.nyt.com/images/2020/08/08/sports/08indonesia-badminton-01/merlin_168908754_6f59595f-1109-43ef-89a0-0975695c3bee-articleLarge.jpg?quality=75\&auto=webp\&disable=upscale}

Image

A technician restrings a badminton racquet. At the ACI Sports store in
Jakarta, it is not uncommon to repair 20 a day.

Image

Shuttlecocks litter courts during morning practice at the Djarum
Badminton Club in Jakarta, one of dozens like it in the capital.

When badminton was introduced two decades later as an official sport at
the Barcelona Games, Indonesia won five medals. Susi Susanti, now the
director of performance for the national team, became the first player
to win gold for Indonesia, in the women's singles. As the Indonesian
flag was raised during the medals ceremony, television cameras focused
tightly on her as \href{https://youtu.be/AeWa6qlGFas?t=42}{tears rolled
down her face}. Her boyfriend Alan Budikusuma, now her husband, won the
men's singles competition a few days later.

Only when they returned home, though, where they were greeted by huge
crowds, did they understand how much their victories had meant to the
country. They have since taken on the task of molding the country's next
generation of champions.

If anyone knows about the long path to success it is Susanti. In her
early teens, she left home to move to Jakarta to live and train at one
of the capital's powerhouse badminton clubs. It is a path still followed
by many of the players who reach the national team.

Liliyana Natsir, a four-time world champion who won gold in
mixed-doubles at the 2016 Rio Games, was born in Manado, a port on the
island of Sulawesi, and came to the Tangkas club in Jakarta at age 12.
Though her parents did not play badminton, she said, her mother was a
passionate follower of the sport, a woman who binge-watched games late
at night while pregnant with Liliyana. ``She told me,'' Natsir said,
``that I must have been watching, too.''

Image

Susi Susanti and Alan Budikusuma won Indonesia's first Olympic golds in
badminton at the 1992 Barcelona Games.

Rudy Hartono, one of the country's greatest singles players and a
dominant force in international badminton in the 1970s, said that
Indonesia's deep love for the sport stemmed from the fact that it has
always been a backyard game for Indonesian families. ``When you go to
small villages,'' he said, ``you can see in the evening, often from 6
till midnight, people gathering to play badminton.''

But the game's popular appeal also has also been a ``unifying force,''
according to Yuppy Suhandinata, the owner of the Tangkas club, because
it blends players from different ethnicities, different religions and
different backgrounds. While Indonesia is the largest Muslim nation in
the world by population, its badminton players --- including many with
Chinese heritage --- come from all religions.

Image

On any normal day in the Indonesian capital, Jakarta, the sport's
enormous popularity is visible. During early morning rush hour earlier
in the year, mechanics practiced while waiting for clients.

Image

A mother and her three sons playing near Jakarta's central stadium.

Before each practice at the Tangkas club, the players are invited to say
a prayer according to their religion. It is a tradition that is carried
up to the highest levels, even at the training sessions of the national
squad.

The origin of the nation's love for the game is unclear. Badminton's
rules were formalized in England at the end of the 19th century, and
spread to Asia --- initially in India and Malaysia --- through British
influence. Indonesia now has more than a million active club players,
according to Achmad Budiharto, the secretary general of the country's
national badminton association.

Rudy Hartono argued it was Indonesia's first victory in the Thomas Cup,
the international men's team competition, in 1958, that helped
popularize the game. Hartono, still trim and elegant at the age of 70,
said that victory inspired him to pursue a career in badminton; the
game, he said, became ``my daily breakfast.'' He grew up to become a
world champion.

Image

Members of the national team practice at a facility run by the Badminton
Association of Indonesia, where portraits of the Association's secretary
generals hang on the wall.

Image

Girls gathered for morning practice at the Djarum Badminton Club.

That level of success, though, has meant enormous pressure upon each
successive generation of Indonesian players. Now that the four-year
Olympic cycle is being extended a year, there is a new weight upon them.

Marcus Fernaldi Gideon is a member of the world's leading doubles team,
and he and his partner were widely considered to have been Indonesia's
strongest chance for a gold medal in Tokyo. Now he, and everyone else,
must find a way to stay motivated as the pressure continues to build.

''Everyone expects us to win,'' he said, ``because this is badminton and
Indonesia.''

Image

Credit...Members of the Indonesian national badminton squad prayed
before practice.~

Advertisement

\protect\hyperlink{after-bottom}{Continue reading the main story}

\hypertarget{site-index}{%
\subsection{Site Index}\label{site-index}}

\hypertarget{site-information-navigation}{%
\subsection{Site Information
Navigation}\label{site-information-navigation}}

\begin{itemize}
\tightlist
\item
  \href{https://help.nytimes.com/hc/en-us/articles/115014792127-Copyright-notice}{©~2020~The
  New York Times Company}
\end{itemize}

\begin{itemize}
\tightlist
\item
  \href{https://www.nytco.com/}{NYTCo}
\item
  \href{https://help.nytimes.com/hc/en-us/articles/115015385887-Contact-Us}{Contact
  Us}
\item
  \href{https://www.nytco.com/careers/}{Work with us}
\item
  \href{https://nytmediakit.com/}{Advertise}
\item
  \href{http://www.tbrandstudio.com/}{T Brand Studio}
\item
  \href{https://www.nytimes.com/privacy/cookie-policy\#how-do-i-manage-trackers}{Your
  Ad Choices}
\item
  \href{https://www.nytimes.com/privacy}{Privacy}
\item
  \href{https://help.nytimes.com/hc/en-us/articles/115014893428-Terms-of-service}{Terms
  of Service}
\item
  \href{https://help.nytimes.com/hc/en-us/articles/115014893968-Terms-of-sale}{Terms
  of Sale}
\item
  \href{https://spiderbites.nytimes.com}{Site Map}
\item
  \href{https://help.nytimes.com/hc/en-us}{Help}
\item
  \href{https://www.nytimes.com/subscription?campaignId=37WXW}{Subscriptions}
\end{itemize}
