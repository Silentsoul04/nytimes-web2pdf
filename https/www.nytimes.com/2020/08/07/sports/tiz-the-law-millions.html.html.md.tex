Sections

SEARCH

\protect\hyperlink{site-content}{Skip to
content}\protect\hyperlink{site-index}{Skip to site index}

\href{https://www.nytimes.com/section/sports}{Sports}

\href{https://myaccount.nytimes.com/auth/login?response_type=cookie\&client_id=vi}{}

\href{https://www.nytimes.com/section/todayspaper}{Today's Paper}

\href{/section/sports}{Sports}\textbar{}For a Small Stable, Big Prizes
Could Lie Ahead

\href{https://nyti.ms/2DFkRGn}{https://nyti.ms/2DFkRGn}

\begin{itemize}
\item
\item
\item
\item
\item
\end{itemize}

Advertisement

\protect\hyperlink{after-top}{Continue reading the main story}

Supported by

\protect\hyperlink{after-sponsor}{Continue reading the main story}

On Horse Racing

\hypertarget{for-a-small-stable-big-prizes-could-lie-ahead}{%
\section{For a Small Stable, Big Prizes Could Lie
Ahead}\label{for-a-small-stable-big-prizes-could-lie-ahead}}

Tiz the Law has won the first leg of this year's rejiggered Triple
Crown, and his stallion rights have the potential to generate millions
for Sackatoga Stable in New York.

\includegraphics{https://static01.nyt.com/images/2020/08/08/sports/07horses2-print/merlin_175373931_845ebb67-3ec7-4992-9ffa-9a9b0e4fc284-articleLarge.jpg?quality=75\&auto=webp\&disable=upscale}

\href{https://www.nytimes.com/by/joe-drape}{\includegraphics{https://static01.nyt.com/images/2018/12/10/multimedia/author-joe-drape/author-joe-drape-thumbLarge.png}}

By \href{https://www.nytimes.com/by/joe-drape}{Joe Drape}

\begin{itemize}
\item
  Aug. 7, 2020
\item
  \begin{itemize}
  \item
  \item
  \item
  \item
  \item
  \end{itemize}
\end{itemize}

How lucky is Jack Knowlton? Lucky enough to have the even-money
favorite, Tiz the Law, in the 151st running of the Travers Stakes on
Saturday in Saratoga Springs, N.Y., and with no real pressure to win it.

In gambling parlance, Knowlton and his Sackatoga Stable have a free
roll: whether his colt wins or loses the Travers, also known as the
Midsummer Derby, Tiz the Law will remain the only horse with a chance to
sweep the sport's Holy Grail, the Triple Crown.

The \href{https://www.nytimes.com/news-event/coronavirus}{coronavirus
pandemic} has reshuffled the seasons of every sport, but it truly
upended thoroughbred racing. Instead of the Kentucky Derby serving as
the first leg of the Triple Crown on the first Saturday in May, the
Belmont Stakes did so for the first time in history on a Saturday in
June.

After Tiz the Law rolled to an emphatic four-length victory at Belmont
Park on Long Island, Knowlton took advantage of the scrambled calendar
and chose an ambitious schedule for his colt, one that mirrored some of
the sport's greatest horses.

The Triple Crown champions Sir Barton (1919), Omaha (1935), Whirlaway
(1941), Count Fleet (1943) and Citation (1948) competed in races between
the Preakness and the Belmont, typically the final two legs. All but
Omaha were victorious.

\includegraphics{https://static01.nyt.com/images/2020/08/08/sports/07horse3-print/merlin_175373880_573ce7db-c71b-4fd1-b2d3-c0527eb99b70-articleLarge.jpg?quality=75\&auto=webp\&disable=upscale}

``Why not?'' asked Knowlton, the managing partner of Sackatoga Stable.
``With the pandemic turning the Triple Crown upside down, and the fact
that we haven't been able to watch his last two victories in person, we
have almost a month to get ready for Kentucky. It is worth a shot.''

Even better, Saratoga Race Course is Knowlton's home track and the
meet's premier race means the world to him. In fact, he has suffered a
broken heart at the hands of the Travers before.

Seventeen years ago, a gelding by the name of Funny Cide was scratched
the day before the race. Funny Cide, the 2003 Kentucky Derby and
Preakness winner, had made Knowlton and his partnership semifamous.
Sackatoga Stable was born in Sackets Harbor, N.Y.,
\href{https://www.nytimes.com/2003/06/02/sports/horse-racing-hicks-from-sticks-now-racing-s-elite.html}{when
six old high school buddies sat on the front porch} of the village's
former mayor, acknowledged they were approaching midlife crises and
decided they needed to get in the horse business.

They added more partners and captured the imagination of sports fans by
arriving at each Triple Crown race in an old yellow school bus and with
coolers full of beer. But at the Belmont, Empire Maker upset Funny Cide,
who finished third, and dashed their Triple Crown dreams.

Image

Funny Cide racing at Belmont Park in 2005. The horse won the first two
legs of the Triple Crown in 2003 for Sackatoga Stable.Credit...John Dunn
for The New York Times

Image

Jack Knowlton, right, with his wife, Dorothy, and a partner, Dave Mahan,
in 2003 at Churchill Downs. They became known for arriving to Triple
Crown races in school buses with coolers full of beer.Credit...Garry
Jones/Associated Press

The Travers was supposed to make them feel better. Instead, Funny Cide
got sick and the stable's trainer, Barclay Tagg, pulled him from the
race.

``To tell you the truth, that was almost as disappointing as not
sweeping the Triple Crown,'' Knowlton said. ``Small stables like ours
don't get a lot of chances.''

Home run horses are scarce, and Knowlton and his partners were grateful
Funny Cide had walked into their shedrow.

They had paid \$75,000 for him, a blue-plate special price that ended up
paying off blue-blood dividends. Funny Cide won 11 races, earned more
than \$3.5 million in purses and etched his name in the history books as
the first New York-bred horse to win the Kentucky Derby.

Still, Knowlton did not deviate from his stable's playbook --- he buys
only modestly bred New York horses and spreads the risk among a lot of
partners who pay a small amount for shares. They have had some success,
winning more than 50 races and campaigning nine horses that have won
between \$95,000 to \$200,000.

Then, Tiz the Law came along.

Image

Barclay Tagg, the trainer for the stable, first saw Tiz the Law in
2018.Credit...Cindy Schultz for The New York Times

Tagg, who also advises on what horses to buy, liked the yearling as soon
as he saw him at the 2018 Fasig-Tipton New York-bred Yearling Sale. Like
Funny Cide, he was the son of a first crop, or unproven sire, by the
name of Constitution. His mother, Tizfiz, had won a Graded Stakes race
at 1 ⅛ miles, suggesting class and stamina.

``We thought we'd get him for \$100,000,'' Knowlton said. ``It's a good
thing we raised our hand one more time and got him at \$110,000.''

Yes, it was very good for Knowlton and his 34 partners. All Tiz the Law
has done is win five out of his six starts for purse earnings of nearly
\$1.5 million.

Even better, however, for them was that the colt's stallion rights were
sold in an eight-figure deal to Coolmore America's Ashford Stud after
his Belmont victory. Knowlton would not reveal the exact number, but he
did acknowledge that Sackatoga will earn bonuses if Tiz the Law wins the
Travers, the Derby, the Preakness and the Breeders' Cup Classic, which
is scheduled for November.

Image

Knowlton, right, has 34 partners, including Bruce Cerone. Knowlton
spreads the risk among a lot of partners, who pay a small amount for
shares.Credit...Cindy Schultz for The New York Times

Ashford Stud structured a bonus deal for the 2015 Triple Crown Champion
American Pharoah that promised his owner, Ahmed Zayat, \$3 million for
winning the Kentucky Derby, \$2 million each for victories in the
Preakness, Belmont, Travers Stakes and Breeders' Cup Classic as well as
a \$2 million bonus for being named 3-year-old male champion.

Do the math: A similar deal would mean a potential \$13 million for
Sackatoga to roll up this year.

``I drew a line in the sand --- no one was going to own his racing
rights other than us and we intended to run him throughout his
4-year-old year,'' Knowlton said.

Lucky or not, Knowlton and his partners are taking a free roll with
found money.

Advertisement

\protect\hyperlink{after-bottom}{Continue reading the main story}

\hypertarget{site-index}{%
\subsection{Site Index}\label{site-index}}

\hypertarget{site-information-navigation}{%
\subsection{Site Information
Navigation}\label{site-information-navigation}}

\begin{itemize}
\tightlist
\item
  \href{https://help.nytimes.com/hc/en-us/articles/115014792127-Copyright-notice}{©~2020~The
  New York Times Company}
\end{itemize}

\begin{itemize}
\tightlist
\item
  \href{https://www.nytco.com/}{NYTCo}
\item
  \href{https://help.nytimes.com/hc/en-us/articles/115015385887-Contact-Us}{Contact
  Us}
\item
  \href{https://www.nytco.com/careers/}{Work with us}
\item
  \href{https://nytmediakit.com/}{Advertise}
\item
  \href{http://www.tbrandstudio.com/}{T Brand Studio}
\item
  \href{https://www.nytimes.com/privacy/cookie-policy\#how-do-i-manage-trackers}{Your
  Ad Choices}
\item
  \href{https://www.nytimes.com/privacy}{Privacy}
\item
  \href{https://help.nytimes.com/hc/en-us/articles/115014893428-Terms-of-service}{Terms
  of Service}
\item
  \href{https://help.nytimes.com/hc/en-us/articles/115014893968-Terms-of-sale}{Terms
  of Sale}
\item
  \href{https://spiderbites.nytimes.com}{Site Map}
\item
  \href{https://help.nytimes.com/hc/en-us}{Help}
\item
  \href{https://www.nytimes.com/subscription?campaignId=37WXW}{Subscriptions}
\end{itemize}
