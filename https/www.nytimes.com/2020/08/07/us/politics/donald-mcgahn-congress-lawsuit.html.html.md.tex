Sections

SEARCH

\protect\hyperlink{site-content}{Skip to
content}\protect\hyperlink{site-index}{Skip to site index}

\href{https://www.nytimes.com/section/politics}{Politics}

\href{https://myaccount.nytimes.com/auth/login?response_type=cookie\&client_id=vi}{}

\href{https://www.nytimes.com/section/todayspaper}{Today's Paper}

\href{/section/politics}{Politics}\textbar{}House Can Sue to Force
Testimony From McGahn, Appeals Court Rules

\href{https://nyti.ms/3a9SPPp}{https://nyti.ms/3a9SPPp}

\begin{itemize}
\item
\item
\item
\item
\item
\end{itemize}

Advertisement

\protect\hyperlink{after-top}{Continue reading the main story}

Supported by

\protect\hyperlink{after-sponsor}{Continue reading the main story}

\hypertarget{house-can-sue-to-force-testimony-from-mcgahn-appeals-court-rules}{%
\section{House Can Sue to Force Testimony From McGahn, Appeals Court
Rules}\label{house-can-sue-to-force-testimony-from-mcgahn-appeals-court-rules}}

A court endorsed the House Judiciary Committee's ability to pursue
testimony from the former White House counsel as part of its oversight
responsibilities.

\includegraphics{https://static01.nyt.com/images/2020/08/07/us/politics/07dc-mcgahn/07dc-mcgahn-articleLarge.jpg?quality=75\&auto=webp\&disable=upscale}

By \href{https://www.nytimes.com/by/hailey-fuchs}{Hailey Fuchs}

\begin{itemize}
\item
  Aug. 7, 2020
\item
  \begin{itemize}
  \item
  \item
  \item
  \item
  \item
  \end{itemize}
\end{itemize}

WASHINGTON --- The House Judiciary Committee can sue to force the former
White House counsel Donald F. McGahn II to testify before Congress, a
federal appeals court ruled on Friday.

The United States Court of Appeals for the District of Columbia Circuit
said in a 7-to-2 decision that enforcement of congressional subpoenas
was crucial to its oversight duties over the executive branch and
remanded to a panel of judges other issues Mr. McGahn raised in the
case. Mr. McGahn is unlikely to appear before Congress ahead of the
election, but the decision endorsed strong congressional oversight
powers and Congress's ability to take the White House to court if an
administration fails to comply with its subpoenas.

``Effective functioning of the legislative branch critically depends on
the legislative prerogative to obtain information, and constitutional
structure and historical practice support judicial enforcement of
congressional subpoenas when necessary,'' Judge Judith Rogers wrote for
the court's majority. ``And it cannot undertake impeachment proceedings
without knowing how the official in question has discharged his or her
constitutional responsibilities.''

The two judges on the court appointed by President Trump, Gregory G.
Katsas and Neomi J. Rao, recused themselves from the case. Both had
served in the administration before joining the bench, and Judge Katsas
had served as deputy White House counsel under Mr. McGahn.

The House Judiciary Committee
\href{https://www.nytimes.com/2019/04/22/us/politics/mcgahn-trump-attacks.html}{subpoenaed
Mr. McGahn} in April 2019 as part of its investigation into possible
obstruction of justice by Mr. Trump. He was a
\href{https://www.nytimes.com/interactive/2019/04/19/us/politics/mueller-report-citations.html}{key
witness for the inquiry} conducted by the former special counsel Robert
S. Mueller III into the possible obstruction of justice and Russian
interference in the 2016 election. Mr. McGahn told the special counsel
that the president
\href{https://www.nytimes.com/2018/01/25/us/politics/trump-mueller-special-counsel-russia.html}{ordered
him} to have the Justice Department dismiss Mr. Mueller, and when he
refused and threatened to quit, Mr. Trump backed off. Later, the
president ordered him to deny that he had ever asked and to issue a memo
saying as such. He threatened to fire Mr. McGahn if he failed to comply.

The committee
\href{https://www.nytimes.com/2019/08/07/us/politics/don-mcgahn-subpoena.html}{sued
Mr. McGahn}, who
\href{https://www.nytimes.com/interactive/2018/03/16/us/politics/all-the-major-firings-and-resignations-in-trump-administration.html}{left
the White House} in 2018, when the administration directed him not to
appear, asking the court to quash the claims that Mr. Trump's aides are
``absolutely immune'' from its subpoenas.

The decision was a major loss for the Trump administration, which
\href{https://www.nytimes.com/2019/04/24/us/politics/donald-trump-subpoenas.html}{has
sought to stonewall subpoenas} issued by Congress since Democrats
assumed control of the House in 2019. The lawsuit against Mr. McGahn was
the first of several last year in which Congress asked the courts to
compel the administration to cooperate with its oversight requests.
Although the Senate acquitted Mr. Trump of the House's impeachment
charges in February, the House has persisted in its subpoena lawsuits.

Judge Rogers wrote that presidents have long cooperated with subpoena
enforcement, but Mr. Trump had taken an ``unprecedented categorical
direction'' when his administration refused to cooperate with the
impeachment investigation. Enforcement lawsuits may be ``an essential
tool in keeping the executive branch at the negotiating table,'' she
said.

In the wake of the court's decision, congressional Democrats celebrated
the ruling. Speaker Nancy Pelosi called it ``a victory for the rule of
law.'' In a statement, the chairman of the Judiciary Committee,
Representative Jerrold Nadler, said it was ``a blow against the wall of
impunity that President Trump has tried to build for himself.''

Still, Mr. McGahn, who has returned to private practice, is unlikely to
testify before Congress in the near future. The court remanded other
issues in the case to the three-judge panel, and the Justice Department
said that it would continue to fight the subpoena in court.

The circuit court also ruled in a separate case on Friday, concerning
the administration's ability to divert funds appropriated by Congress to
the border wall. The House had sued the administration, arguing that it
usurped legislative powers by identifying more money than Congress had
allocated. The decision in Mr. McGahn's case established that the
judiciary could intervene in this case as well, the court ruled.

The Justice Department said it also planned to fight the border wall
suit.

``While we strongly disagree with the standing ruling in McGahn, the en
banc court properly recognized that we have additional threshold grounds
for dismissal of both cases, and we intend to vigorously press those
arguments before the panels hearing those cases,'' the department's
spokeswoman, Kerri Kupec, wrote in a statement.

In 2019, a lower court
\href{https://www.nytimes.com/2019/11/25/us/politics/mcgahn-testimony-ruling.html}{had
ordered} Mr. McGahn to comply with the subpoena and issued a scathing
dismissal of the administration's arguments. Judge Ketanji Brown Jackson
of the United States District Court in Washington called them
``fiction,'' adding, ``presidents are not kings.''

A three-judge panel from the court of appeals
\href{https://www.nytimes.com/2020/02/28/us/mcgahn-subpoena-trump.html}{later
reversed} that decision, ruling that the judiciary could not intervene
in the matter. The decision on Friday from the en banc court revived the
lawsuit and found that the House Judiciary Committee had standing.

The decision fell along ideological lines. The judges in the majority
were Democratic-appointees, and the dissenting judges were selected by
Republican presidents. Those two judges, Judge Thomas B. Griffith and
Judge Karen L. Henderson, were the majority on the panel that dismissed
the lawsuit.

In his dissent on Friday, Judge Griffith warned that involving the
courts in interbranch disputes could risk transforming the judiciary
into a political referee. There is a ``vanishingly slim'' chance that
Congress would benefit from the majority's decision anytime soon, at the
cost of the public trust in the judiciary, he said.

Historically, congressional subpoenas have expired once a new Congress
convenes. If Mr. McGahn does not testify before January, the case itself
may become moot.

``The majority's opinion is a Pyrrhic victory for Congress,'' Judge
Griffith wrote. ``If we venture into this increasingly politicized
territory, we risk undermining that neutrality and losing the public's
trust.''

Advertisement

\protect\hyperlink{after-bottom}{Continue reading the main story}

\hypertarget{site-index}{%
\subsection{Site Index}\label{site-index}}

\hypertarget{site-information-navigation}{%
\subsection{Site Information
Navigation}\label{site-information-navigation}}

\begin{itemize}
\tightlist
\item
  \href{https://help.nytimes.com/hc/en-us/articles/115014792127-Copyright-notice}{©~2020~The
  New York Times Company}
\end{itemize}

\begin{itemize}
\tightlist
\item
  \href{https://www.nytco.com/}{NYTCo}
\item
  \href{https://help.nytimes.com/hc/en-us/articles/115015385887-Contact-Us}{Contact
  Us}
\item
  \href{https://www.nytco.com/careers/}{Work with us}
\item
  \href{https://nytmediakit.com/}{Advertise}
\item
  \href{http://www.tbrandstudio.com/}{T Brand Studio}
\item
  \href{https://www.nytimes.com/privacy/cookie-policy\#how-do-i-manage-trackers}{Your
  Ad Choices}
\item
  \href{https://www.nytimes.com/privacy}{Privacy}
\item
  \href{https://help.nytimes.com/hc/en-us/articles/115014893428-Terms-of-service}{Terms
  of Service}
\item
  \href{https://help.nytimes.com/hc/en-us/articles/115014893968-Terms-of-sale}{Terms
  of Sale}
\item
  \href{https://spiderbites.nytimes.com}{Site Map}
\item
  \href{https://help.nytimes.com/hc/en-us}{Help}
\item
  \href{https://www.nytimes.com/subscription?campaignId=37WXW}{Subscriptions}
\end{itemize}
