Sections

SEARCH

\protect\hyperlink{site-content}{Skip to
content}\protect\hyperlink{site-index}{Skip to site index}

\href{https://www.nytimes.com/section/politics}{Politics}

\href{https://myaccount.nytimes.com/auth/login?response_type=cookie\&client_id=vi}{}

\href{https://www.nytimes.com/section/todayspaper}{Today's Paper}

\href{/section/politics}{Politics}\textbar{}Russia Continues Interfering
in Election to Try to Help Trump, U.S. Intelligence Says

\href{https://nyti.ms/2PA1EJ6}{https://nyti.ms/2PA1EJ6}

\begin{itemize}
\item
\item
\item
\item
\item
\end{itemize}

\begin{itemize}
\item
  \href{https://www.nytimes.com/2020/08/07/us/elections/biden-vs-trump.html?action=click\&pgtype=Article\&state=default\&region=TOP_BANNER\&context=storylines_menu}{Election
  Updates}
\item
  \href{https://www.nytimes.com/interactive/2020/08/06/us/elections/results-tennessee-primary-elections.html?action=click\&pgtype=Article\&state=default\&region=TOP_BANNER\&context=storylines_menu}{Tennessee
  Results}
\item
  \href{https://www.nytimes.com/article/biden-vice-president-2020.html?action=click\&pgtype=Article\&state=default\&region=TOP_BANNER\&context=storylines_menu}{Biden's
  V.P. Search}
\item
  \href{https://www.nytimes.com/interactive/2019/us/politics/2020-presidential-candidates.html?action=click\&pgtype=Article\&state=default\&region=TOP_BANNER\&context=storylines_menu}{The
  Candidates}
\item
  \href{https://www.nytimes.com/newsletters/politics?action=click\&pgtype=Article\&state=default\&region=TOP_BANNER\&context=storylines_menu}{Politics
  Newsletter}
\end{itemize}

Advertisement

\protect\hyperlink{after-top}{Continue reading the main story}

Supported by

\protect\hyperlink{after-sponsor}{Continue reading the main story}

\hypertarget{russia-continues-interfering-in-election-to-try-to-help-trump-us-intelligence-says}{%
\section{Russia Continues Interfering in Election to Try to Help Trump,
U.S. Intelligence
Says}\label{russia-continues-interfering-in-election-to-try-to-help-trump-us-intelligence-says}}

But a new assessment says China would prefer to see the president
defeated, though it is not clear Beijing is doing much to meddle in the
2020 campaign to help Joseph R. Biden Jr.

\includegraphics{https://static01.nyt.com/images/2020/08/09/us/politics/09DC-INTEL/merlin_175047585_bb745441-bc29-4907-be99-56a1b4493a5d-articleLarge.jpg?quality=75\&auto=webp\&disable=upscale}

\href{https://www.nytimes.com/by/julian-e-barnes}{\includegraphics{https://static01.nyt.com/images/2019/12/13/reader-center/author-julian-barnes/author-julian-barnes-thumbLarge.png}}

By \href{https://www.nytimes.com/by/julian-e-barnes}{Julian E. Barnes}

\begin{itemize}
\item
  Published Aug. 7, 2020Updated Aug. 8, 2020, 11:05 a.m. ET
\item
  \begin{itemize}
  \item
  \item
  \item
  \item
  \item
  \end{itemize}
\end{itemize}

WASHINGTON --- Russia is using a range of techniques to denigrate Joseph
R. Biden Jr., American intelligence officials said Friday in their first
public assessment that Moscow continues to try to interfere in the 2020
campaign to help President Trump.

At the same time, the officials said China preferred that Mr. Trump be
defeated in November and was weighing whether to take more aggressive
action in the election.

But officials briefed on the intelligence said that Russia was the far
graver, and more immediate, threat. While China seeks to gain influence
in American politics, its leaders have not yet decided to wade directly
into the presidential contest, however much they may dislike Mr. Trump,
the officials said.

The assessment, included in a
\href{https://www.dni.gov/index.php/newsroom/press-releases/item/2139-statement-by-ncsc-director-william-evanina-election-threat-update-for-the-american-public}{statement}
released by William R. Evanina, the director of the National
Counterintelligence and Security Center, suggested the intelligence
community was treading carefully, reflecting the political heat
generated by previous findings.

\href{https://www.nytimes.com/2020/08/08/magazine/us-russia-intelligence.html?}{The
White House has objected in the past} to conclusions that Moscow is
working to help Mr. Trump, and Democrats on Capitol Hill have expressed
growing concern that the intelligence agencies are not being forthright
enough about Russia's preference for him and that the agencies are
introducing China's anti-Trump stance to balance the scales.

The assessment appeared to draw a distinction between what it called the
``range of measures'' being deployed by Moscow to influence the election
and its conclusion that China prefers that Mr. Trump be defeated.

It cited efforts coming out of pro-Russia forces in Ukraine to damage
Mr. Biden and Kremlin-linked figures who ``are also seeking to boost
President Trump's candidacy on social media and Russian television.''

China, it said, has so far signaled its position mostly through
increased public criticism of the administration's tough line on China
on a variety of fronts.

An American official briefed on the intelligence said it was wrong to
equate the two countries. Russia, the official said, is a tornado,
capable of inflicting damage on American democracy now. China is more
like climate change, the official said: The threat is real and grave,
but more long term.

Democratic lawmakers made the same point about the report, which also
found that Iran was seeking ``to undermine U.S. democratic institutions,
President Trump, and to divide the country'' ahead of the general
election.

``Unfortunately, today's statement still treats three actors of
differing intent and capability as equal threats to our democratic
elections,'' Speaker Nancy Pelosi and Representative Adam B. Schiff, the
chairman of the House Intelligence Committee, said in a joint statement.

Asked about the report during a news conference on Friday night at his
golf club in New Jersey, Mr. Trump said, ``The last person Russia wants
to see in office is Donald Trump because nobody's been tougher on Russia
than I have.'' He said that if Mr. Biden won the presidency, ``China
would own our country.''

Aides and allies of Mr. Biden assailed Mr. Trump, saying that he had
repeatedly sided with President Vladimir V. Putin on whether Russia had
intervened to help him in 2016 and that he had been impeached by the
House for trying to pressure Ukraine into helping him undercut Mr.
Biden.

``Donald Trump has publicly and repeatedly invited, emboldened and even
tried to coerce foreign interference in American elections,'' said Tony
Blinken, a senior adviser to the former vice president.

\includegraphics{https://static01.nyt.com/images/2020/08/07/us/politics/07dc-intel01/merlin_175407054_9a9b0d69-0170-45e3-b16e-925039855ed2-articleLarge.jpg?quality=75\&auto=webp\&disable=upscale}

It is not clear how much China is doing to interfere directly in the
presidential election. Intelligence officials have briefed Congress in
recent days that much of Beijing's focus is on state and local races.
But Mr. Evanina's statement on Friday suggested China was on weighing an
increased effort.

``Although China will continue to weigh the risks and benefits of
aggressive action, its public rhetoric over the past few months has
grown increasingly critical of the current administration's Covid-19
response, closure of China's Houston Consulate and actions on other
issues,'' Mr. Evanina said.

Mr. Evanina pointed to growing tensions over territorial claims in the
South China Sea, Hong Kong autonomy, the TikTok app and other issues.
China, officials have said, has also tried to collect information on the
presidential campaigns, as it has in previous contests.

The release on Friday was short on specifics, but that was largely
because the intelligence community is intent on trying to protect its
sources of information, said Senator Angus King, the Maine independent
who caucuses with the Democrats.

``The director has basically put the American people on notice that
Russia in particular, also China and Iran, are going to be trying to
meddle in this election and undermine our democratic system,'' said Mr.
King, a member of the Senate Intelligence Committee.

Intelligence officials said there was no way to avoid political
criticism when releasing information about the election. An official
with the Office of the Director of National Intelligence said that the
goal was not to rank order threats and that Russia, China and Iran all
pose a danger to the election.

Fighting over the intelligence reports, the official said, only benefits
adversaries trying to sow divisions.

While both Beijing and Moscow have a preference, the Chinese and Russian
influence campaigns are very different, officials said.

Outside of a few scattered examples, it is hard to find much evidence of
intensifying Chinese influence efforts that could have a national
effect.

Much of what China is doing currently amounts to using its economic
might to influence local politics, officials said. But that is hardly
new. Beijing is also using a variety of means to push back on various
Trump administration policies, including tariffs and bans on Chinese
tech companies, but those efforts are not covert and it is unclear if
they would have an effect on presidential politics.

Russia, but not China, is trying to ``actively influence'' the outcome
of the 2020 election, said the American official briefed on the
underlying intelligence.

``The fact that adversaries like China or Iran don't like an American
president's policies is normal fare,'' said Jeremy Bash, a former Obama
administration official. ``What's abnormal, disturbing and dangerous is
that an adversary like Russia is actively trying to get Trump
re-elected.''

Russia tried to use influence campaigns during 2018 midterm voting to
try to sway public opinion, but it did not successfully tamper with
voting infrastructure.

Mr. Evanina said it would be difficult for adversarial countries to try
to manipulate voting results on a large scale. But nevertheless, the
countries could try to interfere in the voting process or take steps
aimed at ``calling into question the validity of the election results.''

The new release comes on the heels of congressional briefings that have
alarmed lawmakers, particularly Democrats. Those briefings have
described a stepped-up Chinese pressure campaign, as well as efforts by
Moscow to paint Mr. Biden as corrupt.

``Ahead of the 2020 U.S. elections, foreign states will continue to use
covert and overt influence measures in their attempts to sway U.S.
voters' preferences and perspectives, shift U.S. policies, increase
discord in the United States, and undermine the American people's
confidence in our democratic process,'' Mr. Evanina said in a statement.

The statement called out Andriy Derkach, a pro-Russia member of
Ukraine's Parliament who has been involved in releasing information
about Mr. Biden. Intelligence officials said he had ties to Russian
intelligence.

Intelligence officials have briefed Congress in recent weeks on details
of the Russian efforts to tarnish Mr. Biden as corrupt, prompting
\href{https://www.nytimes.com/2020/07/20/us/politics/congress-disinformation-biden-russia-ukraine.html}{senior
Democrats to request} more information.

A Senate committee led by Senator Ron Johnson, Republican of Wisconsin,
has been leading an investigation of Mr. Biden's son Hunter Biden and
his work for Burisma, a Ukrainian energy firm. Some intelligence
officials have said that a witness the committee was seeking to call was
a witting or unwitting agent of Russian disinformation.

Democrats had pushed intelligence officials to release more information
to the public, arguing that only a broad declassification of the foreign
interference attempts can inoculate voters against attempts by Russia,
China or other countries to try to influence voting.

\href{https://www.nytimes.com/2020/07/24/us/politics/election-interference-russia-china-iran.html}{In
meetings on Capitol Hill}, Mr. Evanina and other intelligence officials
have expanded their warnings beyond Russia and have included China and
Iran, as well. This year, the Office of the Director of National
Intelligence put Mr. Evanina in charge of election security briefings to
Congress and the campaigns.

Intelligence and other officials in recent days have been stepping up
their
\href{https://www.nytimes.com/2020/07/24/us/politics/election-interference-russia-china-iran.html}{releases
of information} about foreign interference efforts, and the
\href{https://www.nytimes.com/2020/08/06/us/politics/election-meddling-texts-russia-iran.html}{State
Department has sent texts} to cellphones around the world advertising a
\$10 million reward for information on would-be election hackers.

How effective China's campaign or Russia's efforts to smear Mr. Biden as
corrupt have been is not clear. Intelligence agencies focus their work
on the intentions of foreign governments, and steer clear of assessing
if those efforts have had an effect on American voters.

The first reactions from Capitol Hill to the release of the assessment
were positive. A joint statement by the Republican and Democratic
leaders of the Senate Intelligence Committee praised it, and asked
colleagues to refrain from politicizing Mr. Evanina's statement.

Senator Marco Rubio of Florida, the acting Republican chairman of the
committee, and Senator Mark Warner of Virginia, the Democratic vice
chairman, said they hoped Mr. Evanina continued to make more information
available to the public. But they praised him for responding to calls
for more information.

``Evanina's statement highlights some of the serious and ongoing threats
to our election from China, Russia, and Iran,'' the two men's joint
statement said. ``Everyone --- from the voting public, local officials,
and members of Congress --- needs to be aware of these threats.''

Maggie Haberman contributed reporting from New York.

\hypertarget{our-2020-election-guide}{%
\section{Our 2020 Election Guide}\label{our-2020-election-guide}}

Updated Aug. 7, 2020

\begin{itemize}
\item
  \begin{center}\rule{0.5\linewidth}{\linethickness}\end{center}

  \hypertarget{the-latest}{%
  \subsection{The Latest}\label{the-latest}}

  \begin{itemize}
  \tightlist
  \item
    \href{https://www.nytimes.com/2020/08/07/us/politics/russia-china-trump-biden-election-interference.html?action=click\&pgtype=Article\&state=default\&region=BELOW_MAIN_CONTENT\&context=storylines_guide}{Russia
    is using a range of techniques to denigrate Joe Biden}, American
    intelligence officials said, declaring that Moscow continues to try
    to interfere in the 2020 campaign to help President Trump.
  \end{itemize}
\item
  \begin{center}\rule{0.5\linewidth}{\linethickness}\end{center}

  \hypertarget{bidens-vp-search}{%
  \subsection{Biden's V.P. Search}\label{bidens-vp-search}}

  \begin{itemize}
  \tightlist
  \item
    \href{https://www.nytimes.com/article/biden-vice-president-2020.html?action=click\&pgtype=Article\&state=default\&region=BELOW_MAIN_CONTENT\&context=storylines_guide}{Here
    are 13 women} who have been under consideration to be Joe Biden's
    running mate, and why each might be chosen --- and might not be.
  \end{itemize}
\item
  \begin{center}\rule{0.5\linewidth}{\linethickness}\end{center}

  \hypertarget{keep-up-with-our-coverage}{%
  \subsection{Keep Up With Our
  Coverage}\label{keep-up-with-our-coverage}}

  \begin{itemize}
  \tightlist
  \item
    Get an
    \href{https://www.nytimes.com/newsletters/politics?action=click\&pgtype=Article\&state=default\&region=BELOW_MAIN_CONTENT\&context=storylines_guide}{email}
    recapping the day's news
  \end{itemize}

  \begin{itemize}
  \tightlist
  \item
    Download our mobile app on
    \href{https://apps.apple.com/us/app/nytimes/id284862083?ls=1\&mat_click_id=5c79ae7455014fd1bd66b5610c05b8f2-20191112-16948\&referrer=mat_click_id\%3D5c79ae7455014fd1bd66b5610c05b8f2-20191112-16948\%26link_click_id\%3D722930677036718082}{iOS}
    and
    \href{http://a.localytics.com/android?id=com.nytimes.android\&referrer=utm_source\%3Dother_nyt_mobile_web\%26utm_medium\%3DWeb\%2520page\%26utm_term\%3DGeneral\%2520Mobile\%2520Page\%26utm_campaign\%3DNYT\%2520Mobile\%2520General\%2520Page}{Android}
    and turn on Breaking News and Politics alerts
  \end{itemize}
\end{itemize}

Advertisement

\protect\hyperlink{after-bottom}{Continue reading the main story}

\hypertarget{site-index}{%
\subsection{Site Index}\label{site-index}}

\hypertarget{site-information-navigation}{%
\subsection{Site Information
Navigation}\label{site-information-navigation}}

\begin{itemize}
\tightlist
\item
  \href{https://help.nytimes.com/hc/en-us/articles/115014792127-Copyright-notice}{©~2020~The
  New York Times Company}
\end{itemize}

\begin{itemize}
\tightlist
\item
  \href{https://www.nytco.com/}{NYTCo}
\item
  \href{https://help.nytimes.com/hc/en-us/articles/115015385887-Contact-Us}{Contact
  Us}
\item
  \href{https://www.nytco.com/careers/}{Work with us}
\item
  \href{https://nytmediakit.com/}{Advertise}
\item
  \href{http://www.tbrandstudio.com/}{T Brand Studio}
\item
  \href{https://www.nytimes.com/privacy/cookie-policy\#how-do-i-manage-trackers}{Your
  Ad Choices}
\item
  \href{https://www.nytimes.com/privacy}{Privacy}
\item
  \href{https://help.nytimes.com/hc/en-us/articles/115014893428-Terms-of-service}{Terms
  of Service}
\item
  \href{https://help.nytimes.com/hc/en-us/articles/115014893968-Terms-of-sale}{Terms
  of Sale}
\item
  \href{https://spiderbites.nytimes.com}{Site Map}
\item
  \href{https://help.nytimes.com/hc/en-us}{Help}
\item
  \href{https://www.nytimes.com/subscription?campaignId=37WXW}{Subscriptions}
\end{itemize}
