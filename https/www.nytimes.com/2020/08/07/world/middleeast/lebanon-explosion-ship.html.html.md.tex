Sections

SEARCH

\protect\hyperlink{site-content}{Skip to
content}\protect\hyperlink{site-index}{Skip to site index}

\href{/section/world/middleeast}{Middle East}\textbar{}Ship Cited in
Beirut Blast Hasn't Sailed in 7 Years. We Found It.

\href{https://nyti.ms/3a4TFgo}{https://nyti.ms/3a4TFgo}

\begin{itemize}
\item
\item
\item
\item
\item
\end{itemize}

\includegraphics{https://static01.nyt.com/images/2020/08/06/video/beirut-ship-a/beirut-ship-a-articleLarge.jpg?quality=75\&auto=webp\&disable=upscale}

visual investigations

\hypertarget{ship-cited-in-beirut-blast-hasnt-sailed-in-7-years-we-found-it}{%
\section{Ship Cited in Beirut Blast Hasn't Sailed in 7 Years. We Found
It.}\label{ship-cited-in-beirut-blast-hasnt-sailed-in-7-years-we-found-it}}

The explosive material responsible for the Aug. 4 blast in Beirut was
carried there by the ailing cargo ship Rhosus in 2013. Here's what
happened next.

Credit...Anthony Vrailas, via MarineTraffic.com

Supported by

\protect\hyperlink{after-sponsor}{Continue reading the main story}

By \href{https://www.nytimes.com/by/christoph-koettl}{Christoph Koettl}

Videos by \href{https://www.nytimes.com/by/drew-jordan}{Drew Jordan}

\begin{itemize}
\item
  Published Aug. 7, 2020Updated Aug. 8, 2020, 11:44 a.m. ET
\item
  \begin{itemize}
  \item
  \item
  \item
  \item
  \item
  \end{itemize}
\end{itemize}

On Nov. 21, 2013, at 11:27 a.m., a leaky ship arrived in Beirut's port.
It never left, and its volatile cargo would lead to tragedy in the city
almost seven years later. The Rhosus was loaded with 2,750 tons of
ammonium nitrate, which is believed to have
\href{https://www.nytimes.com/2020/08/05/video/beirut-explosion-footage.html}{blown
up most of the port} and damaged large parts of the city when it ignited
in a warehouse on Aug. 4.

The former captain of the ship, Boris Prokoshev, told The New York Times
that he heard from other sailors that the Rhosus sank
\href{https://www.nytimes.com/2020/08/05/world/middleeast/beirut-explosion-ship.html}{in
2015 or 2016}. This time frame turned out to be incorrect. Using
satellite imagery analysis and ship tracking data, our Visual
Investigations unit went back in time to follow the ship that brought
the disastrous cargo to Beirut. We found its exact location, where it
remains hidden a short distance from Beirut's ground zero.

The timeline and location of the Rhosus in Beirut gained new relevance
on Friday as Lebanon's president, Michael Aoun,
\href{https://www.nytimes.com/reuters/2020/08/07/world/middleeast/07reuters-lebanon-security-blast-president.html}{said}
that an investigation into the incident will also focus on how the
explosive materials entered and were stored in the area.

\hypertarget{final-voyage}{%
\subsection{Final Voyage}\label{final-voyage}}

The Rhosus left for its last journey from Batumi, Georgia, in September
2013. Its cargo was destined for Mozambique, but the captain was ordered
to make an unscheduled stop in Beirut to take on additional freight.
Captain Prokoshev said they needed to make extra cash to pay for their
passage through the Suez Canal. Lawyers for the ship's creditors said
the additional cargo was supposed to be transported to Jordan.

\includegraphics{https://static01.nyt.com/images/2020/08/06/autossell/01_GFX-SHIP-PATH_v3/01_GFX-SHIP-PATH_v3-videoSixteenByNineJumbo1600.jpg}

\hypertarget{seizure}{%
\subsection{Seizure}\label{seizure}}

In Beirut, port authorities impounded the 27-year-old ship after they
found multiple deficiencies, according to a maritime news
\href{https://www.fleetmon.com/maritime-news/2014/4194/crew-kept-hostages-floating-bomb-mv-rhosus-beirut/}{article}
from 2014. The captain and some of his crew
\href{https://www.nytimes.com/2020/08/05/world/middleeast/beirut-explosion-ship.html}{were
ordered} to stay on board.

A photo from 2014 shows him with some of the 2,750 bags of
\href{https://www.nytimes.com/2020/08/05/world/middleeast/beirut-explosion-ammonium-nitrate.html}{ammonium
nitrate} in the port. These bags match the bags photographed later at
the warehouse that eventually blew up,
\href{https://twitter.com/DimaSadek/status/1291486030458224640}{posted
by a Lebanese journalist on Twitter} after the incident. The warehouse
pictures also show the name of the company: Rustavi Azot L.L.C. of
Georgia, which is also listed on the 2013 shipping documents for the
voyage.

\includegraphics{https://static01.nyt.com/images/2020/07/08/multimedia/beirut-ship-d-image/beirut-ship-d-image-videoSixteenByNineJumbo1600.jpg}

The ship's transponder sent its last position on Aug. 7, 2014, the same
month the crew was released.

\hypertarget{abandoned}{%
\subsection{Abandoned}\label{abandoned}}

The Rhosus was left abandoned, and Lebanese authorities transferred its
cargo to a warehouse in the port. In 2015, the ship was moved 1,000 feet
up the pier where it remained for about three years. Satellite images
show a ship matching the dimensions of the Rhosus, with open cargo bays,
indicating it is empty.

It can also be seen in the background of
\href{https://archive.vn/6xqjc}{a picture posted on Twitter in March
2016}, clearly recognizable through the yellowish color of the ship's
bridge.

\includegraphics{https://static01.nyt.com/images/2020/08/07/video/beirut-ship-c-pic/beirut-ship-c-pic-articleLarge.jpg?quality=75\&auto=webp\&disable=upscale}

\hypertarget{disrepair}{%
\subsection{Disrepair}\label{disrepair}}

The Rhosus was leaking badly and began sinking in February 2018. Within
days, the ship was fully submerged. Stephen Wood, a satellite image
expert at space technology company Maxar, analyzed an image from Feb.
18, 2018, for The Times. He used multi-spectral imagery that can
penetrate water and better reveal submerged objects and features. The
resulting image shows the ship in great detail, despite being
underwater.

Image

The Rhosus sank between Feb. 16 and 18, 2018. Image date: Feb. 18,
2018.Credit...Maxar Technologies

\hypertarget{forgotten}{%
\subsection{Forgotten}\label{forgotten}}

Authorities never removed the shipwreck on the northern edge of the
port, where it did not seem to obstruct ship traffic. The Rhosus
remained out of sight and forgotten until now. So was its explosive
cargo, stored a mere 1,500 feet away in a warehouse. On Aug. 4, a fire
ignited the ammonium nitrate, killing more than 150 and injuring
thousands.

\includegraphics{https://static01.nyt.com/images/2020/08/06/autossell/01_SHIP_BLAST_PROXIMITY_v1/01_SHIP_BLAST_PROXIMITY_v1-videoSixteenByNineJumbo1600.jpg}

Lebanon's president said that multiple administrations since 2013
received warnings about the materials from the Rhosus. An investigation
is ongoing and 20 port officials have been detained.

Evan Hill, Stella Cooper, Christiaan Triebert, Declan Walsh and Andrew
Higgins contributed reporting. David Botti contributed production.

Advertisement

\protect\hyperlink{after-bottom}{Continue reading the main story}

\hypertarget{site-index}{%
\subsection{Site Index}\label{site-index}}

\hypertarget{site-information-navigation}{%
\subsection{Site Information
Navigation}\label{site-information-navigation}}

\begin{itemize}
\tightlist
\item
  \href{https://help.nytimes.com/hc/en-us/articles/115014792127-Copyright-notice}{©~2020~The
  New York Times Company}
\end{itemize}

\begin{itemize}
\tightlist
\item
  \href{https://www.nytco.com/}{NYTCo}
\item
  \href{https://help.nytimes.com/hc/en-us/articles/115015385887-Contact-Us}{Contact
  Us}
\item
  \href{https://www.nytco.com/careers/}{Work with us}
\item
  \href{https://nytmediakit.com/}{Advertise}
\item
  \href{http://www.tbrandstudio.com/}{T Brand Studio}
\item
  \href{https://www.nytimes.com/privacy/cookie-policy\#how-do-i-manage-trackers}{Your
  Ad Choices}
\item
  \href{https://www.nytimes.com/privacy}{Privacy}
\item
  \href{https://help.nytimes.com/hc/en-us/articles/115014893428-Terms-of-service}{Terms
  of Service}
\item
  \href{https://help.nytimes.com/hc/en-us/articles/115014893968-Terms-of-sale}{Terms
  of Sale}
\item
  \href{https://spiderbites.nytimes.com}{Site Map}
\item
  \href{https://help.nytimes.com/hc/en-us}{Help}
\item
  \href{https://www.nytimes.com/subscription?campaignId=37WXW}{Subscriptions}
\end{itemize}
