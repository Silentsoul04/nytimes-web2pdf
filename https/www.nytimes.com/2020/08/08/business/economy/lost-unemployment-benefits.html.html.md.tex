Sections

SEARCH

\protect\hyperlink{site-content}{Skip to
content}\protect\hyperlink{site-index}{Skip to site index}

\href{https://www.nytimes.com/section/business/economy}{Economy}

\href{https://myaccount.nytimes.com/auth/login?response_type=cookie\&client_id=vi}{}

\href{https://www.nytimes.com/section/todayspaper}{Today's Paper}

\href{/section/business/economy}{Economy}\textbar{}Without \$600 Weekly
Benefit, Unemployed Face Bleak Choices

\href{https://nyti.ms/3gInbeg}{https://nyti.ms/3gInbeg}

\begin{itemize}
\item
\item
\item
\item
\item
\item
\end{itemize}

\href{https://www.nytimes.com/news-event/coronavirus?action=click\&pgtype=Article\&state=default\&region=TOP_BANNER\&context=storylines_menu}{The
Coronavirus Outbreak}

\begin{itemize}
\tightlist
\item
  live\href{https://www.nytimes.com/2020/08/08/world/coronavirus-updates.html?action=click\&pgtype=Article\&state=default\&region=TOP_BANNER\&context=storylines_menu}{Latest
  Updates}
\item
  \href{https://www.nytimes.com/interactive/2020/us/coronavirus-us-cases.html?action=click\&pgtype=Article\&state=default\&region=TOP_BANNER\&context=storylines_menu}{Maps
  and Cases}
\item
  \href{https://www.nytimes.com/interactive/2020/science/coronavirus-vaccine-tracker.html?action=click\&pgtype=Article\&state=default\&region=TOP_BANNER\&context=storylines_menu}{Vaccine
  Tracker}
\item
  \href{https://www.nytimes.com/interactive/2020/world/coronavirus-tips-advice.html?action=click\&pgtype=Article\&state=default\&region=TOP_BANNER\&context=storylines_menu}{F.A.Q.}
\item
  \href{https://www.nytimes.com/live/2020/08/07/business/stock-market-today-coronavirus?action=click\&pgtype=Article\&state=default\&region=TOP_BANNER\&context=storylines_menu}{Markets
  \& Economy}
\end{itemize}

Advertisement

\protect\hyperlink{after-top}{Continue reading the main story}

Supported by

\protect\hyperlink{after-sponsor}{Continue reading the main story}

\hypertarget{without-600-weekly-benefit-unemployed-face-bleak-choices}{%
\section{Without \$600 Weekly Benefit, Unemployed Face Bleak
Choices}\label{without-600-weekly-benefit-unemployed-face-bleak-choices}}

A federal supplement to jobless pay was a lifeline for millions and for
the economy. Its cutoff, even if temporary, may have lasting
consequences.

\includegraphics{https://static01.nyt.com/images/2020/08/06/business/00virus-cliff1c/00virus-cliff1c-articleLarge-v2.jpg?quality=75\&auto=webp\&disable=upscale}

By \href{https://www.nytimes.com/by/ben-casselman}{Ben Casselman} and
Gillian Friedman

\begin{itemize}
\item
  Aug. 8, 2020Updated 6:20 p.m. ET
\item
  \begin{itemize}
  \item
  \item
  \item
  \item
  \item
  \item
  \end{itemize}
\end{itemize}

When Latrish Oseko lost her job last spring, government aid helped
prevent a crisis from becoming a catastrophe.

A \$1,700 federal stimulus payment meant that when her 26-year-old car
broke down, she could replace it. The \$600 a week in extra unemployment
benefits from the federal government allowed her to pay rent and buy
food. When her day care provider closed, she was able to get her
4-year-old daughter a subscription to ABCmouse, an online learning app.

But the federal money ran out at the end of July, and politicians in
Washington have been unable to agree on how to replace it.

So Ms. Oseko, 39, is spending much of her time sitting in the Delaware
hotel room where she has lived since her landlord kicked her out at the
end of July, applying for jobs on her phone while watching the debate
play out on the local news.

``I'm glued to it because I want to know, is there going to be hope for
me?'' she said. ``They're fighting, and I have to watch them fight, but
they have a place to sleep at night.''

On Saturday, with negotiations in Congress stalled and on the verge of
collapse, President Trump signed four directives aimed at providing
economic assistance, including financial help to the unemployed. But it
was unclear if Mr. Trump had the authority to act on his own on matters
requiring federal spending, or how long it would take for money to start
flowing if he did.

Congress may yet agree on a new emergency spending bill that would
include extra unemployment benefits, perhaps including retroactive
payments for the period when the program lapsed.

\hypertarget{latest-updates-the-coronavirus-outbreak-and-the-economy}{%
\section{\texorpdfstring{\href{https://www.nytimes.com/live/2020/08/07/business/stock-market-today-coronavirus?action=click\&pgtype=Article\&state=default\&region=MAIN_CONTENT_1\&context=storylines_live_updates}{Latest
Updates: The Coronavirus Outbreak and the
Economy}}{Latest Updates: The Coronavirus Outbreak and the Economy}}\label{latest-updates-the-coronavirus-outbreak-and-the-economy}}

\href{https://www.nytimes.com/live/2020/08/07/business/stock-market-today-coronavirus?action=click\&pgtype=Article\&state=default\&region=MAIN_CONTENT_1\&context=storylines_live_updates\#wealthy-families-are-throwing-a-lifeline-to-distressed-businesses}{24h
ago}

\href{https://www.nytimes.com/live/2020/08/07/business/stock-market-today-coronavirus?action=click\&pgtype=Article\&state=default\&region=MAIN_CONTENT_1\&context=storylines_live_updates\#wealthy-families-are-throwing-a-lifeline-to-distressed-businesses}{Wealthy
families are throwing a lifeline to distressed businesses.}

\href{https://www.nytimes.com/live/2020/08/07/business/stock-market-today-coronavirus?action=click\&pgtype=Article\&state=default\&region=MAIN_CONTENT_1\&context=storylines_live_updates\#the-publisher-of-the-onion-jezebel-and-other-websites-lays-off-15-employees}{25h
ago}

\href{https://www.nytimes.com/live/2020/08/07/business/stock-market-today-coronavirus?action=click\&pgtype=Article\&state=default\&region=MAIN_CONTENT_1\&context=storylines_live_updates\#the-publisher-of-the-onion-jezebel-and-other-websites-lays-off-15-employees}{The
publisher of The Onion, Jezebel and other websites lays off 15
employees.}

\href{https://www.nytimes.com/live/2020/08/07/business/stock-market-today-coronavirus?action=click\&pgtype=Article\&state=default\&region=MAIN_CONTENT_1\&context=storylines_live_updates\#canada-outlines-its-response-to-the-new-us-aluminum-tariff}{30h
ago}

\href{https://www.nytimes.com/live/2020/08/07/business/stock-market-today-coronavirus?action=click\&pgtype=Article\&state=default\&region=MAIN_CONTENT_1\&context=storylines_live_updates\#canada-outlines-its-response-to-the-new-us-aluminum-tariff}{Canada
outlines its response to the new U.S. aluminum tariff.}

\href{https://www.nytimes.com/live/2020/08/07/business/stock-market-today-coronavirus?action=click\&pgtype=Article\&state=default\&region=MAIN_CONTENT_1\&context=storylines_live_updates}{See
more updates}

More live coverage:
\href{https://www.nytimes.com/2020/08/07/world/covid-19-news.html?action=click\&pgtype=Article\&state=default\&region=MAIN_CONTENT_1\&context=storylines_live_updates}{Global}

But for many of the 30 million Americans relying on unemployment
benefits, it could already be too late to prevent lasting financial
harm. Without a federal supplement, they will need to get by on regular
state unemployment benefits, which often total a few hundred dollars a
week or less. For many families, that will not be enough to pay the
rent, stave off hunger or avoid mounting debt that will make it harder
to climb out of the hole.

Households and the broader economy are particularly vulnerable at this
moment.
\href{https://www.nytimes.com/2020/08/07/business/economy/housing-economy-eviction-renters.html}{Eviction
moratoriums} are expiring or have expired in much of the country. The
Paycheck Protection Program, which helped thousands of small businesses
to retain workers, ended Saturday.

There are already signs that the economy has slowed down this summer as
virus cases have surged in much of the country. On Friday, the
\href{https://www.nytimes.com/2020/08/07/business/economy/july-jobs-report.html}{Labor
Department reported a net gain of 1.8 million jobs} in July, a smaller
increase than in May or June. Many economists warn that layoffs could
begin rising again without more government support. Food banks say they
are bracing for a new wave of demand.

Before the pandemic, Ms. Oseko and her family were making ends meet,
albeit with little margin for error. She earned \$15 an hour as a
contractor doing data entry. Her boyfriend earned a bit less cleaning
dormitories at the University of Delaware. They were able to rent a
two-bedroom house near a park where their daughter could play.

When the pandemic hit, Ms. Oseko's hours were cut and her boyfriend was
furloughed. Then, in May, she lost her job altogether. In the midst of
that crisis, another one appeared: Their landlord sold her building and
gave them 60 days to leave. They moved out at the end of July and are
burning through their meager savings at a rate of \$76 a night at a
Delaware motel that is filling up with families in the same predicament.

Image

After losing her day care provider, Ms. Oseko put some of her
unemployment benefits toward a learning app for her
daughter.Credit...Hannah Yoon for The New York Times

Image

Without an apartment, it has been hard to find a job. ``The jobs that I
am qualified for want me to work from home, but I have no home,'' she
said.Credit...Hannah Yoon for The New York Times

Without a job, Ms. Oseko hasn't been able to find a new apartment;
without an apartment, it has been hard to find a job.

``The jobs that I am qualified for want me to work from home, but I have
no home,'' she said.

The economic crisis caused by the pandemic has disproportionately
affected low-wage workers like Ms. Oseko who have little in savings.
\href{https://www.aeaweb.org/articles?id=10.1257/aer.20170537}{Research
from the last recession} found that when unemployment benefits ran out,
people cut their spending on food, medicine and other necessities,
suggesting they were able to do little to prepare for the drop in
income.

The more generous benefits offered during this recession may have
allowed families to save some money, but those savings won't last long,
particularly when food prices are rising at the fastest pace in years.

As a result, families are being forced to make decisions with lasting
consequences.

When Jason Depretis and his fiancée lost their Florida restaurant jobs
in early March, they started falling behind on their rent and their car
payment. The \$600 unemployment supplement was a lifeline, allowing them
to hold on to their home and their car. But on July 28, that lifeline
snapped: The repo man showed up for the car on the day that their
landlord delivered a three-day notice of eviction.

With the extra \$600 a week, Mr. Depretis, 42, would probably have been
able to pay enough to hold off both creditors. Without it, he had to
choose. He paid his landlord \$650 to stave off eviction, and watched
the car be towed away.

But it was a terrible time to lose the car. He had found a job starting
in September at a restaurant, but it is 45 minutes away, and there is no
bus service that corresponds with his hours. The closest food bank is 30
minutes away, and he can't get there without a vehicle. He said he
didn't know how he and his fiancée would put food on the table for
themselves and their two children.

``Without the \$600, there's absolutely no way that my family's going to
make it,'' he said.

For families like Mr. Depretis's, even a temporary loss of income can be
the start of a downward spiral, said Elizabeth Ananat, a Barnard College
economist who has been studying the pandemic's impact on low-wage
workers. Wealthier families may be able to draw on savings to get
through until Congress reaches a deal. But for lower-income households,
even a temporary lapse in benefits can have lasting consequences. An
eviction can make it hard to rent in the future. Having a car
repossessed can make it hard to find another job. And for children,
periods of hunger, homelessness and stress can have long-term effects on
development and learning.

``Children cannot smooth their eating over the year,'' Ms. Ananat said.
``Families that do not have access to credit cannot smooth their food,
their electricity, any of their necessities.''

Many Republicans argue that the extra benefits were keeping recipients
from looking for work, especially because many were getting more on
unemployment than they had made on the job. Business owners have
complained that they are struggling to fill positions.

But
\href{https://papers.ssrn.com/sol3/papers.cfm?abstract_id=3664265}{several}
\href{https://www.dropbox.com/s/q0kcoix35jxt1u4/UI_Employment_HPS.pdf?dl=0}{studies}
have found
\href{https://news.yale.edu/2020/07/27/yale-study-finds-expanded-jobless-benefits-did-not-reduce-employment}{no
evidence} that the supplement was discouraging job hunting, and many
workers
\href{https://twitter.com/ernietedeschi/status/1283832188434362368}{appear
to be accepting jobs} even when the pay is less than their unemployment
benefits. And by injecting billions of dollars into the economy each
week, the benefits almost certainly prevented even more layoffs.

The lapse in benefits will push some people to return to work. But that
decision, too, can carry costs.

\includegraphics{https://static01.nyt.com/images/2020/08/06/business/00virus-cliff2a/00virus-cliff2a-articleLarge.jpg?quality=75\&auto=webp\&disable=upscale}

When the pandemic hit, Enrique Guzman, a fleet service clerk at Los
Angeles International Airport, was given the choice: to keep working or
to stay home and receive a portion of his income, the equivalent of 10
hours a week.

Mr. Guzman, 27, decided to stay home. He has asthma, which puts him at a
higher risk of complications if he were to catch the coronavirus, and he
lives with his girlfriend and her mother, whose age, 51, makes her
vulnerable to the virus. Between unemployment benefits and the partial
paychecks from the airline, he was able to bring in \$1,050 a week ---
less than he earned working full time, but enough to support his
girlfriend and her mother.

But without the extra money, Mr. Guzman can no longer afford the \$1,875
rent for their two-bedroom apartment in Montebello, Calif., plus the
cost of utilities, food, and his student and car loan payments.

On Monday, with a sinking feeling in his stomach, he put on his uniform
and returned to the airport for his first shift since the pandemic
started. Mr. Guzman said he had no other choice.

``It wasn't something that I wanted to do, but I'm the only income in my
household now and I needed to go back to work so we can afford to pay
our rent, afford to pay our bills,'' he said. ``I'm putting myself at
risk so that we can afford to stay afloat.''

Advertisement

\protect\hyperlink{after-bottom}{Continue reading the main story}

\hypertarget{site-index}{%
\subsection{Site Index}\label{site-index}}

\hypertarget{site-information-navigation}{%
\subsection{Site Information
Navigation}\label{site-information-navigation}}

\begin{itemize}
\tightlist
\item
  \href{https://help.nytimes.com/hc/en-us/articles/115014792127-Copyright-notice}{©~2020~The
  New York Times Company}
\end{itemize}

\begin{itemize}
\tightlist
\item
  \href{https://www.nytco.com/}{NYTCo}
\item
  \href{https://help.nytimes.com/hc/en-us/articles/115015385887-Contact-Us}{Contact
  Us}
\item
  \href{https://www.nytco.com/careers/}{Work with us}
\item
  \href{https://nytmediakit.com/}{Advertise}
\item
  \href{http://www.tbrandstudio.com/}{T Brand Studio}
\item
  \href{https://www.nytimes.com/privacy/cookie-policy\#how-do-i-manage-trackers}{Your
  Ad Choices}
\item
  \href{https://www.nytimes.com/privacy}{Privacy}
\item
  \href{https://help.nytimes.com/hc/en-us/articles/115014893428-Terms-of-service}{Terms
  of Service}
\item
  \href{https://help.nytimes.com/hc/en-us/articles/115014893968-Terms-of-sale}{Terms
  of Sale}
\item
  \href{https://spiderbites.nytimes.com}{Site Map}
\item
  \href{https://help.nytimes.com/hc/en-us}{Help}
\item
  \href{https://www.nytimes.com/subscription?campaignId=37WXW}{Subscriptions}
\end{itemize}
