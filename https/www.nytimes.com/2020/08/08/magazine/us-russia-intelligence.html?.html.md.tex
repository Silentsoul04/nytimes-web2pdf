Unwanted Truths: Inside Trump's Battles With U.S. Intelligence Agencies

\href{https://nyti.ms/30FcvYc}{https://nyti.ms/30FcvYc}

\begin{itemize}
\item
\item
\item
\item
\item
\item
\end{itemize}

\begin{itemize}
\item
  \href{https://www.nytimes.com/2020/08/07/us/elections/biden-vs-trump.html?action=click\&pgtype=Article\&state=default\&region=TOP_BANNER\&context=storylines_menu}{Election
  Updates}
\item
  \href{https://www.nytimes.com/interactive/2020/08/08/us/elections/results-hawaii-primary-elections.html?action=click\&pgtype=Article\&state=default\&region=TOP_BANNER\&context=storylines_menu}{Hawaii
  Results}
\item
  \href{https://www.nytimes.com/article/biden-vice-president-2020.html?action=click\&pgtype=Article\&state=default\&region=TOP_BANNER\&context=storylines_menu}{Biden's
  V.P. Search}
\item
  \href{https://www.nytimes.com/interactive/2019/us/politics/2020-presidential-candidates.html?action=click\&pgtype=Article\&state=default\&region=TOP_BANNER\&context=storylines_menu}{The
  Candidates}
\item
  \href{https://www.nytimes.com/newsletters/politics?action=click\&pgtype=Article\&state=default\&region=TOP_BANNER\&context=storylines_menu}{Politics
  Newsletter}
\end{itemize}

\includegraphics{https://static01.nyt.com/images/2020/08/16/magazine/16mag-intelligence/16mag-intelligence-articleLarge-v3.jpg?quality=75\&auto=webp\&disable=upscale}

Sections

\protect\hyperlink{site-content}{Skip to
content}\protect\hyperlink{site-index}{Skip to site index}

Feature

\hypertarget{unwanted-truths-inside-trumps-battles-with-us-intelligence-agencies}{%
\section{Unwanted Truths: Inside Trump's Battles With U.S. Intelligence
Agencies}\label{unwanted-truths-inside-trumps-battles-with-us-intelligence-agencies}}

Last year, intelligence officials gathered to write a classified report
on Russia's interest in the 2020 election. An investigation from the
magazine uncovered what happened next.

Credit...Doug Mills/The New York Times

Supported by

\protect\hyperlink{after-sponsor}{Continue reading the main story}

By \href{https://www.nytimes.com/by/robert-draper}{Robert Draper}

\begin{itemize}
\item
  Aug. 8, 2020Updated 11:05 a.m. ET
\item
  \begin{itemize}
  \item
  \item
  \item
  \item
  \item
  \item
  \end{itemize}
\end{itemize}

In early July of last year, the first draft of a classified document
known as a National Intelligence Estimate circulated among key members
of the agencies making up the U.S. intelligence community. N.I.E.s are
intended to be that community's most authoritative class of top-secret
document, reflecting its consensus judgment on national-security matters
ranging from Iran's nuclear capabilities to global terrorism. The draft
of the July 2019 N.I.E. ran to about 15 pages, with another 10 pages of
appendices and source notes.

According to multiple officials who saw it, the document discussed
Russia's ongoing efforts to influence U.S. elections: the 2020
presidential contest and 2024's as well. It was compiled by a working
group consisting of about a dozen senior analysts, led by Christopher
Bort, a veteran national intelligence officer with nearly four decades
of experience, principally focused on Russia and Eurasia. The N.I.E.
began by enumerating the authors' ``key judgments.'' Key Judgment 2 was
that in the 2020 election, Russia favored the current president: Donald
Trump.

The intelligence provided to the N.I.E.'s authors indicated that in the
lead-up to 2020, Russia worked in support of the Democratic presidential
candidate Bernie Sanders as well. But Bort explained to his colleagues,
according to notes taken by one participant in the process, that this
reflected not a genuine preference for Sanders but rather an effort ``to
weaken that party and ultimately help the current U.S. president.'' To
allay any speculation that Putin's interest in Trump had cooled, Key
Judgment 2 was substantiated by current information from a highly
sensitive foreign source described by someone who read the N.I.E. as
``100 percent reliable.''

On its face, Key Judgment 2 was not a contentious assertion. In 2017,
the Office of the Director of National Intelligence, the umbrella entity
supervising the 16 other U.S. intelligence agencies, released a
\href{https://www.nytimes.com/interactive/2017/01/06/us/politics/document-russia-hacking-report-intelligence-agencies.html}{report}
drawing on intelligence from the C.I.A., the F.B.I. and the National
Security Agency that found Russia had interfered in the 2016
presidential election and aspired to help Trump. At a
\href{https://www.nytimes.com/2018/07/16/world/europe/trump-putin-election-intelligence.html}{news
conference with Trump} in Helsinki in July 2018, President Vladimir
Putin of Russia denied interfering in the election. But when asked by a
reporter if he had wanted Trump to win, he replied bluntly: ``Yes, I
did.''

Yet Trump never accepted this and often actively disputed it, judging
officials who expressed such a view to be disloyal. As a former senior
adviser to Trump, speaking on the condition of anonymity, told me, ``You
couldn't have any conversation about Russia and the election without the
president assuming you were calling his election into question. Everyone
in the White House knew that, and so you just didn't talk about that
with him.'' According to this former adviser, both John Bolton and Mick
Mulvaney, who were Trump's national security adviser and acting chief of
staff in 2019, went to considerable lengths to keep the subject of
Russian election interference off the president's agenda. (Bolton and
Mulvaney declined to comment for this article.)

The president's displeasure with any suggestion that he was Putin's
favorite factored into the discussion over the N.I.E. that summer, in
particular the ``back and forth,'' as Dan Coats, then the director of
national intelligence, put it, over the assessment that Russia favored
Trump in 2020. Eventually, this debate made it to Coats's desk. ``I can
affirm that one of my staffers who was aware of the controversy
requested that I modify that assessment,'' Coats told me recently. ``But
I said, `No, we need to stick to what the analysts have said.'''

{[}\href{https://www.nytimes.com/2020/08/08/us/politics/trump-russia.html?action=click\&module=Top\%20Stories\&pgtype=Homepage}{Read
the key takeaways from this investigation.}{]}

Coats had been director of national intelligence since early in Trump's
presidency, but his tenure had been rocky at times, and earlier that
year, he and Trump agreed to part ways; Coats expected to resign near
the end of September. So it surprised him when on July 28, not long
after he was approached about the change to the N.I.E., Trump announced
via Twitter that Coats's last day in office would be Aug. 15. In the
days to come, Coats's regular meetings with Trump on intelligence
matters continued. During those conversations, Coats told me, the
president never explained what prompted his sudden decision.

Coats's interim successor would be retired Vice Adm. Joseph Maguire, who
at the time was director of the National Counterterrorism Center.
Maguire had served under eight presidents in a military or government
capacity. Within the intelligence community, his appointment elicited
relief but also worry: ``From the very beginning,'' one former senior
intelligence official told me, ``there was a lot of consternation over
not getting Maguire fired.'' One issue looming over the new acting
director was the fact that the N.I.E., which had yet to be finalized,
contained a conclusion that the president had often railed against.

One of the intelligence officials most directly acquainted with Trump's
opinions on the agencies' work was Beth Sanner. A veteran of the C.I.A.,
Sanner now serves as the O.D.N.I.'s deputy director for mission
integration. Her responsibilities include delivering the president's
daily brief, the regular presentation of new intelligence findings of
pressing importance that Trump, like his predecessors, receives.

Delivering the P.D.B., as it is known, requires an astute understanding
of the briefer's audience. Sanner, who earlier in her C.I.A. career was
flagged for promotions by managers who viewed her as an exceptional
talent, was tough but also outgoing. In a
\href{https://www.youtube.com/watch?v=OZhW72KH9lk}{rare public
appearance} at an online conference hosted by the nonprofit Intelligence
\& National Security Alliance last month, Sanner offered a window onto
her experience as Trump's briefer. ``I think that fear for us is the
most debilitating thing that we face in our personal or professional
lives,'' she said. ``And if every time I went in and talked with the
president I was afraid, I would never get anything done. You might be
afraid right before you get there. But then you're there; let it go. You
are there because you're good.'' She had learned over time how to put
Trump at ease with self-deprecating humor. Encountering the limits of
his attention, she once said (according to someone familiar with this
particular briefing), ``OK, I can see you're not interested ---
\emph{I'm} not interested, I don't even know why I brought this up ---
so let's move on.''

In early September, an email went out from an O.D.N.I. official to the
N.I.E.'s reviewers with the latest version attached --- which, according
to the email, ``includes edits from D.M.I. Beth Sanner. We have
highlighted the major changes in yellow; they make some of the KJ
language clearer and highlight \ldots{} Russia's motivation for its
influence activities.''

No longer did Key Judgment 2 clearly state that Russia favored the
current president, according to an individual who compared the two
versions of the N.I.E. side by side. Instead, in the words of a written
summary of the document that I obtained, the new version concluded that
``Russian leaders probably assess that chances to improve relations with
the U.S. will diminish under a different U.S. president.'' The National
Intelligence Board approved the final version at a meeting on the
afternoon of Sept. 26, 2019.

Such a change, a former senior intelligence official said, would amount
to ``a distinction without a difference and a way to make sure Maguire
doesn't get fired.'' But the distinction was in fact both real and
important. A document intended to explain Russia's playbook for the
upcoming elections no longer included an explanation of what Russia's
immediate goal was. Omitting that crucial detail would later allow the
White House to question the credibility of the testimony of intelligence
and law-enforcement officials who informed lawmakers of Russia's
interest in Trump's re-election in a closed-door congressional committee
briefing early this year. It would also set in motion Maguire's own
departure, in spite of the efforts to protect him.

Relationships between presidents and the intelligence agencies they
command are often testy, and Trump is hardly the first president to
\href{https://www.nytimes.com/2020/07/16/magazine/colin-powell-iraq-war.html}{ignore
or mischaracterize intelligence}. But the alarm in the intelligence
community over Russian interference on behalf of Trump's election in
2016, and Trump's reciprocal suspicion of the intelligence community,
immediately marked their relationship as categorically different from
those with past presidents. ``Trump's first encounter with the
intelligence community as president-elect was in meetings with James
Comey, John Brennan and James Clapper, all of whom turned out to be
involved with spying on President Trump's campaign,'' Kayleigh McEnany,
the White House press secretary, said in a statement responding to a
list of factual queries for this article. The investigation of Trump's
campaign, McEnany said, was ``the greatest political scandal and crime
in U.S. history.'' (Although the F.B.I. investigated links between Trump
campaign associates and Russian officials, a
\href{https://www.nytimes.com/2019/12/09/us/politics/fbi-ig-report-russia-investigation.html}{2019
report} by the Justice Department's inspector general found no evidence
that it had tried to place informants inside the campaign. No claims of
spying on the campaign by other American intelligence agencies have ever
been substantiated.)

The depth of Trump's animosity has been known since before his
inauguration. What has not been known is the full extent of how this
suspicion has reshaped the intelligence community and the personal and
professional calculations of its members, forcing officials to walk a
fine line between serving the president and maintaining the integrity of
their work. The brunt of Trump's discontent has been borne by those who
work in the Office of the Director of National Intelligence, which was
established in late 2004 at the recommendation of the 9/11 Commission to
facilitate better communication among the intelligence agencies. The
O.D.N.I.'s directors and briefers, like Sanner, have been the
community's most direct point of contact with the president. In the
past, that proximity was straightforward. A briefing would be given, and
then the briefer would leave the Oval Office so that the president could
discuss policy options with his advisers.

Under Trump, intelligence officials have been placed in the unusual
position of being pressured to justify the importance of their work,
protect their colleagues from political retribution and demonstrate
fealty to a president. Though intelligence officials have been loath to
admit it publicly, the cumulative result has been devastating.
Representative Sean Patrick Maloney, a Democrat on the House
Intelligence Committee, compared the O.D.N.I.'s decline under Trump to
that of the Justice Department, where ``they have, step by step, set out
to destroy one of the crown jewels of the American government,'' he told
me. ``And they're using the same playbook with the intelligence
community.''

The O.D.N.I.'s erosion has in turn shaped the information that flows out
of the intelligence community to the White House --- or doesn't. The
softening of Key Judgment 2 signified a sobering new development of the
Trump era: the intelligence community's willingness to change what it
would otherwise say straightforwardly so as not to upset the president.
``To its credit, the intelligence community resisted during the earlier
part of the president's term,'' Representative Adam Schiff, the
Democratic chairman of the House Intelligence Committee, told me. ``But
by casting out Dan Coats and then Maguire, and replacing them with
loyalists, I think over time it's had the effect of wearing the
intelligence community down, making them less willing to speak truth to
power.''

This ``wearing down'' has extended well beyond the dismissal of a few
top intelligence officials whom the president perceived to be disloyal.
It has also meant that those who remain in the community are acutely
mindful of the risks of challenging Trump's ``alternative facts,'' as
the White House counselor Kellyanne Conway once memorably described them
--- with consequences that are substantive, if often hidden from view.

That concern was palpable among nearly all of the 40 current and former
intelligence officials, lawmakers and congressional staff with whom I
spoke --- among them more than 15 people who worked in, or closely with,
the intelligence community throughout Trump's presidency. Though these
people would discuss their experiences only in exchange for anonymity
out of fear of reprisal or dismissal, the unusual fact of their
willingness to discuss them at all --- and the extent to which their
stories could be confirmed by multiple sources, and in many cases by
contemporaneous documents --- itself was a testament to how profoundly
Trump has reordered their world and their work. As one of them told me:
``The problem is that when you've been treated the way the intelligence
community has, they become afraid of their own shadow. The most
dangerous thing now is the churn --- the not knowing who's going to be
fired, and what it is you might say that could cost you your job. It's
trying to put out something and not get creamed for it.''

\includegraphics{https://static01.nyt.com/images/2020/08/16/magazine/16mag-intelligence-02/16mag-intelligence-02-articleLarge-v2.jpg?quality=75\&auto=webp\&disable=upscale}

\textbf{Like the rest} of America, the thousands of people making up the
U.S. intelligence community were divided by the election of Donald
Trump. Many were wary of a candidate who pledged to bring back
waterboarding and assassinate families of ISIS members, who praised
WikiLeaks and played down Putin's extrajudicial assassinations by
observing, ``What, you think our country's so innocent?'' Three weeks
after beginning to receive his first intelligence briefings as a
candidate, Trump publicly offered the dubious claim that his briefers
``were not happy'' that President Obama and his administration ``did not
follow what they were recommending.'' Listening to Trump throughout the
campaign, Michael Hayden, who directed the C.I.A. under both George W.
Bush and Obama, told me, ``I was really scared for my country.'' But
others in the community were rankled by what they saw as Obama's
passivity in global affairs and were receptive to the prospect of a
change.

On Jan. 21, 2017, his first full day in office, Trump
\href{https://www.nytimes.com/video/us/politics/100000004886212/watch-live-trump-at-the-cia.html}{addressed
an audience} of agency employees at C.I.A. headquarters in Langley, Va.
Standing in front of the agency's Memorial Wall, an austere slab of
marble engraved with more than a hundred stars commemorating the agency
officers who died in service to their country --- three C.I.A.
paramilitary officers had recently been killed in Afghanistan --- he
proceeded to unleash one of his stream-of-consciousness diatribes.
``Probably almost everybody in this room voted for me,'' he declared. He
complimented himself on his pick for secretary of agriculture and
admonished the Bush administration for not having seized Iraq's oil
after invading the country. He bragged about his inauguration speech and
repeated his false claims about the mammoth crowd it attracted and his
record number of appearances on the cover of Time magazine. He
questioned the judgment of whoever it was who had chosen to build the
C.I.A. headquarters lobby with so many columns.

``I was literally in tears,'' one senior agency official at the time
told me, ``as I watched him standing in the most hallowed place we have
--- so disconnected, talking about himself, asking why our building had
columns.'' A second agency veteran angrily characterized Trump's speech
as ``a near-desecration of the wall,'' adding: ``I'm tearing up now just
thinking about it.''

Trump bragged to the C.I.A. audience that he would be the agency's most
lavish supporter: ``You're going to get so much backing. Maybe you're
going to say, `Please don't give us so much backing.''' But in truth, he
already had reservations about the intelligence community. The C.I.A.
director John Brennan and the former director Hayden had publicly
criticized various statements he made during the campaign. The former
acting director Michael Morell, who advised Hillary Clinton's campaign,
had described Trump in an op-ed as ``an unwitting agent of the Russian
Federation.'' At Langley headquarters before his speech, Trump met with
several of the C.I.A.'s top officials and, according to someone familiar
with the conversation, asked several of them individually whether they
had voted for him.

Two weeks before his inauguration, the president-elect and his senior
aides received a briefing at Trump Tower led by the departing director
of national intelligence, James Clapper, outlining the intelligence
community's assessment of Russia's interference in the 2016 election.
Trump was friendly and attentive but also dismissive. ``Anybody's going
to tell you what they think you want to hear,'' Trump told them,
according to Clapper.

Toward the end of the briefing, Trump's new chief of staff, Reince
Priebus, began to discuss drafting a press statement. Priebus, Clapper
recalled, ``wanted to include language in it that we said Russian
interference had no impact on the outcome of the election. Well, we
didn't have the authority to make that judgment. The only thing we said
was that we saw no evidence of tampering with the votes.''

As the briefing concluded, James Comey, director of the F.B.I., spoke
with Trump alone. There was another matter to disclose: a dossier
compiled by the former British intelligence officer Christopher Steele,
which discussed Russia's entanglements with Trump's campaign and the
candidate himself. (Many of these claims were never substantiated or
were later disproved outright.) Fusion GPS, the research firm that was
involved in producing the dossier, had confidentially organized
briefings on Steele's findings for a handful of reporters. But when
BuzzFeed published the dossier four days after Comey's briefing, the
president-elect blamed intelligence officials. ``Intelligence agencies
should never have allowed this fake news to `leak' out into the
public,'' he tweeted the following morning. ``One last shot at me. Are
we living in Nazi Germany?''

Clapper spoke with Trump that afternoon and defended the intelligence
community. Trump did not apologize, and he instead asked Clapper to
release a statement refuting the dossier's claims. Clapper declined to
do so.

Trump's hostility was not purely a matter of self-interest. As a
candidate, he often railed against the foreign policies of his
predecessors, Democrat and Republican alike --- in particular the Iraq
war, a debacle that was inseparable from the failures of the
intelligence community. After it was reported in December 2016 that the
C.I.A. had concluded that Russia interfered with the 2016 election on
Trump's behalf, his transition team released a press statement
declaring, ``These are the same people that said Saddam Hussein had
weapons of mass destruction.'' Once Trump was in the White House, a
former Trump-administration official recalls: ``I cannot tell you how
many times he randomly raised the Iraq war. Like it morally offended
him. He believed the intelligence community purposely made it all up.''

But the gross intelligence failures in the run-up to the Iraq war
offered a subtler cautionary tale too. The Bush administration had a
tendency to see only what it wished to see of that intelligence, to
contort and mischaracterize semi-educated guesses as unassailable facts
--- a tendency that, in Trump, was compulsive to a nearly pathological
degree. As one intelligence veteran who occasionally briefed Trump told
me: ``On a visceral level, his view was, `You all are supposed to be
helping me.' But when you'd bring in evidence that Russia interfered,
that's what he'd refer to as not helpful. Or when he's wanting to turn
the screws on NATO, we'd come in with a warning of the consequences of
NATO falling apart. And he'd say, `You never do things for me.'''

\textbf{Historically, the C.I.A.} has learned to accommodate the
individual presidents it serves, though always with the tacit
understanding that the ``first customer'' would not abuse the courtesy.
Bill Clinton's famously fluid schedule made it difficult for him to
commit to daily one-on-one briefings. (When a man in a stolen Cessna 150
plane crashed it into the South Lawn of the White House in 1994, the
mordant joke around the C.I.A. was that it was the agency's director,
Jim Woolsey, trying to get a meeting with the president.)

Still, Clinton read his briefing material. George W. Bush, whose father
had been a C.I.A. director, faithfully took his briefings six mornings a
week --- though it famously did not result in his heeding the August
2001 briefing titled ``Bin Ladin Determined to Strike in U.S.'' Obama,
too, took daily briefings for most of his presidency; Lisa Monaco, his
homeland-security adviser, earned the presidential nickname Dr. Doom for
her grim counterterrorism updates. The briefings were a ritual through
which the intelligence community implicitly made the case for itself as
something that transcended partisanship and operated on a time scale
beyond mere presidencies.

It was inevitable that some adjustments would prove necessary for Trump,
novice as he was to government. The new president's interests were
primarily economic, a field that was never the intelligence community's
strong suit. Under Trump, intelligence officials learned to ``up our
econ briefings game,'' as one of them told me.

But the culture clash posed more serious problems too. Trump was
accustomed to cutting deals and sharing gossip on his private cellphone,
often loudly. He enjoyed being around billionaires, to whom he would
``show off about some of the stuff he thought was cool --- the
capabilities of different weapons systems,'' one former senior
administration official recalled. ``These were superrich guys who
wouldn't give him the time of day before he became president. He'd use
that stuff as currency he had that they didn't, not understanding the
implications.'' Trump also stocked his President's Intelligence Advisory
Board with wealthy businesspeople who, when briefed by one intelligence
official, ``would sometimes make you uncomfortable'' because on
occasion, ``their questions were related to their business dealings,''
this individual recalled.

The chairman of that advisory board, Stephen Feinberg, is co-chief
executive of Cerberus Capital Management, which owns DynCorp, a major
defense contractor that has won several lucrative military contracts.
Feinberg was a friend of the president's son-in-law, Jared Kushner,
whose expansive role in the new administration also created unease
within the intelligence community. ``His attitude,'' one former
intelligence official recalled of Kushner, ``is like that of his
father-in-law, who always thought that people who weren't trying to be
wealthy but instead went into public service were lesser.'' There were
obvious security issues that seemed not to have occurred to Kushner, who
``would have the Chinese ambassador and his minions wandering around the
West Wing unescorted,'' recalled one former senior administration
official. (The White House disputes this. ``No foreign nationals are
allowed to roam freely in the West Wing,'' McEnany said in a statement.)

Early in the administration, Kushner and an aide showed up to Langley
headquarters --- conspicuous in their fitted suits --- for a meeting to
learn how the C.I.A. functions. The agency accommodated them, but
afterward, according to one participant in the meeting, concern
developed within the agency about Kushner's potential conflicts. His
complicated international business interests, as well as his evolving
friendship with Crown Prince Mohammed bin Salman of Saudi Arabia, had
raised serious concerns among officials responsible for awarding
security credentials. A further concern, another former senior
intelligence official said, ``was just his cavalier and arrogant
attitude that `I know what I'm doing,' without any cultural
understanding of why things are classified, that would put our
intelligence at risk.''

Trump publicly claimed to know little about Kushner's security-clearance
problem. But in fact, the president ``made a huge deal of it and tried
to pull all sorts of strings and go around the system,'' one former
official recalled. Another former official said, ``I'd hear the
president say, `Just do it, just give it to him.' I'm not sure he
understood what it actually meant. He made it sound like Jared was just
trying to join a club.''

Some of Trump's intelligence advisers feared that his carelessness would
inevitably get him in trouble when dealing one on one with cannier
foreign leaders. ``When you're a president, any slip can be used,'' one
former national-security aide said. Because of Trump's indiscretion, one
former senior intelligence official told me, the intelligence office of
at least one foreign country --- a NATO ally that had sent troops to
Afghanistan --- was discouraged by that country's president from
interacting with its American counterparts, for fear that Trump would be
briefed on the information and subsequently blurt it out to the
Russians. The president
\href{https://www.nytimes.com/2017/05/15/us/politics/trump-russia-classified-information-isis.html}{did
precisely that} four months into his tenure, sharing sensitive
intelligence about ISIS with the Russian foreign minister and ambassador
during a meeting in the Oval Office, reportedly exposing a source of
Israeli intelligence in the process. Two years later, Trump would
\href{https://www.nytimes.com/2019/08/30/world/middleeast/trump-iran-missile-explosion-satellite-image.html}{tweet
a surveillance photograph} of a damaged space facility in Iran, a
sensitive image that almost certainly came from a U.S. drone or
satellite.

Image

Credit...Tom Williams/CQ Roll Call, via Getty Images

\textbf{Trump's indiscretion} wasn't the only issue. Officials came to
realize that his lack of interest and tendency toward distraction posed
their own concerns. His briefers, a former senior administration
official said, ``were stunned and miffed that he had no real interest in
the P.D.B. And it wasn't just the P.D.B.; it was almost anything
generated by his N.S.C.'' --- Trump's National Security Council. ``He
kind of likes the military details but just doesn't read briefing
materials. They'd put all this time and effort into these briefing
papers, and he'd literally throw it aside.''

Recognizing that Trump responded to visual material, his aides for a
time tried to compose briefs out of photos, charts and a limited number
of captions, until it became evident that such a presentation would not
convey all that a president needed to know. But it remained a challenge
to engage Trump, a former adviser said: ``Anyone who's ever briefed him
wouldn't get more than three or four minutes into it, and then the
president would go off on tangents.'' Such tangents, a former
intelligence briefer said, would include Trump's standing in the polls,
Hillary Clinton's email server and the prospect of holding a military
parade in the United States.

For one briefing that concerned an adversarial nation's weapons system,
the C.I.A. briefer arrived with a prop: a portable model of the weapon
in question. ``Trump held it in his hands, and it's all he paid
attention to,'' a former senior intelligence official recalled. ``The
briefer would be talking about range and deployment, and all the
president wanted to know was: `What's this made of? What's this part
here?'''

From the 2016 campaign to early 2019, Trump's principal briefer was Ted
Gistaro, a much-respected C.I.A. veteran whom the president called ``my
Ted.'' Sometime in the spring of 2019, Gistaro accepted a posting
overseas, though not before unburdening himself to a former colleague.
``I knew you've heard how bad it is,'' the colleague recalled him
saying. ``Believe me, it's worse than that.'' (The O.D.N.I. declined
requests for an interview with Gistaro.)

By that spring, Trump was souring on Gistaro's boss, Dan Coats. A
77-year-old former Republican senator who was once in the running to be
George W. Bush's defense secretary, Coats had denounced Trump during his
candidacy for his ``totally inappropriate and disgusting'' comments in
the ``Access Hollywood'' tape. He had not expressed interest in the job
of director of national intelligence, and Trump had not even bothered to
interview him for it. It was Vice President Mike Pence, a friend from
Indiana, who extended the offer on Trump's behalf and who later swore
him in.

Shortly after nominating Coats for the director job, Trump invited him
to a dinner gathering at the White House residence. According to the
special prosecutor Robert Mueller's report on his investigation into
Russian election interference in 2016 and Coats's testimony before the
House Intelligence Committee, Trump asked his guests what they thought
of James Comey. When Trump asked if anyone knew Comey personally, Coats
replied that Comey had been a good F.B.I. director and advised the
president to get to know him better.

According to the same report and testimony, barely a week into Coats's
tenure as director of national intelligence, he was asked by Trump to
publicly clear the president of Russia-related wrongdoing. Coats
carefully replied that it was not in his purview to do so.

The president repeated his request in an evening phone call. Coats, an
avid college-basketball fan, was watching the Final Four N.C.A.A.
semi-finals at the time. He was struck by the abjectness of the new
president, alone in the White House on a Saturday night, talking to a
near-stranger while his family remained in New York. But he did not
buckle. He advised Trump to let the investigation run its course. ``I
made sure that if the information in the briefing was exact and true, it
had to be presented to him, regardless of what the consequences might
be,'' Coats told me. ``And I kept reminding people putting together the
P.D.B. that they could in no way modify anything for political
purposes.''

This was especially perilous when the subject was Russia. In
\href{https://www.nytimes.com/2020/06/17/books/review-room-where-it-happened-john-bolton-memoir.html}{``The
Room Where It Happened,''} John Bolton's recently published memoir of
his ill-fated stint as Trump's national security adviser from April 2018
to September 2019, Bolton recalled watching the president chafe over
sanctions on Russia. In 2018, the U.S. government initiated a
cyberattack against the
\href{https://www.nytimes.com/2015/06/07/magazine/the-agency.html}{Internet
Research Agency}, a Russian troll farm singled out by Mueller for its
efforts to influence the 2016 election. Although the Trump
administration would later point to this as proof of the president's
toughness on Russia, three individuals who had real-time knowledge of
the attack told me that Trump did not specifically order it.

In March 2018, Secretary of Homeland Security Kirstjen Nielsen warned a
gathering of foreign diplomats that there would be harsh consequences
for meddling in the 2018 midterm elections --- at which point the
Russian representative stormed out of the meeting. The White House
communications office subsequently complained privately to the
Department of Homeland Security that Nielsen's remarks were off-message.
That July, at an N.S.C. meeting convened for the express purpose of
discussing election security, Nielsen got only five minutes into her
opening presentation before Trump interrupted her with a barrage of
questions relating to the wall he wanted built along the Mexico border.

Coats, too, was at the N.S.C. meeting. He had received a more public
snubbing on the subject just a few days earlier, when President Trump,
standing alongside Putin at the news conference in Helsinki, responded
to a question about Russian meddling in the 2016 election by saying,
``Dan Coats came to me and some others, they said they think it's
Russia.'' But, Trump went on, ``President Putin was extremely strong and
powerful in his denial today.'' Coats responded later that day with a
statement reaffirming ``our assessments of Russian meddling in the 2016
election.'' Coats's defense ``added fuel to the fire,'' Bolton later
wrote.

\textbf{Despite the president's} aggressive indifference on the subject
--- or because of it --- some of his cabinet officials remained
concerned that Russia could throw the upcoming elections into turmoil
and perhaps even disrupt the results. To them, the intelligence relating
to Putin's aims was indisputable. So was the president's intransigence.
As Bolton would write, ``Trump believed that acknowledging Russia's
meddling in U.S. politics, or in that of many other countries in Europe
or elsewhere, would implicitly acknowledge that he had colluded with
Russia in his 2016 campaign.''

It was against this backdrop that Coats, Nielsen, Secretary of State
Mike Pompeo and Secretary of Defense Jim Mattis worked together to write
an executive order in the summer of 2018 that would enable sanctions on
foreign countries trying to interfere with the American electoral
process. Trump wasn't briefed on these efforts, because, as one
individual involved in the process recalled, ``there was a belief that
such a meeting would go sideways.'' Instead, according to Bolton's book,
on Sept. 12, 2018, as several aides gathered with the president to
discuss the border wall, Bolton seized the moment and held out the
executive order for Trump to sign. Suspiciously, the president asked
whose idea the executive order was. Bolton volunteered that it was his.
``Oh,'' Trump said, and he signed it.

Among other things, the executive order set in motion the process of
drafting the intelligence assessment that Coats would be asked by a
subordinate to change 10 months later. But by the time the order was
signed, the fraying relationship between the president and his director
of national intelligence was already on the verge of unraveling
altogether. On Jan. 29, 2019, Coats and other intelligence-agency
leaders presented the intelligence community's annual threat assessment
to the Senate Intelligence Committee. As had now become customary for
many public statements that might contradict Trump's own, the O.D.N.I.'s
senior staff labored over the draft of the director's opening statement
and then cleared it with the N.S.C. staff. Still, its stark depictions
of Russia's ongoing election meddling, North Korea's determination to
maintain its nuclear arsenal and the resilience of ISIS amounted to a
sweeping rebuttal to the president's claimed foreign-policy
accomplishments.

Trump tweeted his displeasure the following day, writing, ``Perhaps
Intelligence should go back to school!'' Two days after their testimony,
Coats and Gina Haspel, the C.I.A. director, met with the president, with
Bolton in attendance as well. Later, Trump tweeted: ``Just concluded a
great meeting with my Intel team in the Oval Office who told me that
what they said on Tuesday at the Senate Hearing was mischaracterized by
the media. \ldots{} We are all on the same page!''

That was far from the truth, Coats told me. ``We basically said this is
what we said, and it had already been presented to White House personnel
because we knew it was sensitive. The president was not happy that Gina
and I pushed back on that and that it was approved by the White House.
He said, `How did this happen?'''

But, Coats added, ``when he made the remarks about going back to school,
I knew my time was coming to an end.'' Behind his back, Trump was
referring to Coats as old, lazy, ignorant and, Bolton wrote, ``an
idiot.''

Coats was not going to become another
\href{https://www.nytimes.com/2020/06/30/magazine/jeff-sessions.html}{Jeff
Sessions}, the attorney general who spent nearly two years twisting in
the wind and weathering scorn until the president finally fired him. He
prepared a letter of resignation. Trump rejected it, but only because of
its timing: He didn't want Coats to leave while Mueller's investigation
was ongoing. Coats agreed to wait, figuring that a departure date near
the end of the fiscal year, Sept. 30, made sense. He also began
suggesting potential replacements to the White House.

A federal statute stipulated that should the position of director become
vacant, it should be filled on an acting basis by the O.D.N.I.'s deputy
director. In this case, that was Sue Gordon, a well-respected former
C.I.A. official and onetime deputy director of the National
Geospatial-Intelligence Agency. When Coats recruited Gordon to be his
deputy and introduced her to Trump in 2017, he informed the president
that she had been a captain on the Duke women's basketball team. Trump
commented on her height and then, without discussing Gordon's
qualifications for the job, asked her a series of basketball-related
questions, concluding by asking Gordon who was likely to win the
N.C.A.A. tournament.

A few months after her initial meeting with Trump, Gordon appeared
onstage at an intelligence forum with four former directors of the
C.I.A., including Brennan and Hayden. The unprecedented war of words
between a sitting president and the two former intelligence czars had
continued (and would only intensify a year later, when Trump declared
that he had revoked Brennan's security clearance). On this panel,
Brennan said that Trump had ``undermined'' the intelligence community by
refusing to accept its assessment of Russia's election meddling. Hayden
asserted that ``the most disruptive element in the world today is the
United States.'' Gordon, the panel's moderator, kept the conversation
moving.

This would be enough to brand Gordon as disloyal to some in Trump's
inner circle, putting her in the same camp as her boss, Coats, who had
won over the intelligence community's senior officials by protecting
their work from the pressures coming from the White House. By contrast,
both of Trump's C.I.A. directors seemed more willing to accommodate the
president. His first director, Mike Pompeo, aggressively worked to
develop a close relationship with Trump. At the Aspen Security Forum in
the summer of 2017, Pompeo said that Russia had interfered in the 2016
election --- and ``the one before that and the one before that.'' A year
later, when British intelligence officials requested assistance from the
C.I.A. in investigating the apparent poisoning of a double agent by
Russian operatives, Pompeo was initially disinclined to offer
assistance, saying to a roomful of subordinates, according to someone
with knowledge of the conversation, that because Britain had done
nothing to help the United States when it came to Iran, he saw no reason
the United States should help on this matter.

Haspel, who replaced Pompeo after he was
\href{https://www.nytimes.com/2019/02/26/magazine/mike-pompeo-translates-trump.html}{tapped}
to run the State Department, had previously overseen one of the C.I.A.'s
notorious overseas interrogation facilities known as ``black sites'' ---
a fact that endeared her to Trump, according to one former intelligence
official. ``He loved that Gina is a badass,'' the official said. ``He
loved her involvement in the prisons.'' Still, the director also felt
obliged to show her supportiveness in ways that others in the agency
found inappropriate, from applauding during Trump's State of the Union
address to saying publicly of his North Korea policy, ``After years of
failure, I do think that President Trump has shown a lot of wisdom in
reaching out his hand to the North Korean leader.''

Coats exhibited no such pretenses of fealty. ``What we were standing up
for was the integrity of the intelligence,'' he told me. That included
the intelligence community's N.I.E. assessing Russia's interference
campaign. ``There was a lot of back and forth on that assessment''
relating to Russia's preference for Trump, Coats acknowledged to me.
Still, the director held firm by not modifying the assessment. It would
be one of his last acts as director of national intelligence.

On Sunday, July 28, Trump announced via Twitter that Coats would be
replaced by Representative John Ratcliffe of Texas, a Republican and an
outspoken Trump defender. Just four days earlier, while questioning
Mueller at a House Judiciary Committee hearing regarding the special
prosecutor's report, Ratcliffe argued that while Trump shouldn't be
above the law, he ``damn sure shouldn't be below the law, which is where
Volume 2 of this report puts him.'' Some speculated at the time that
Ratcliffe's performance was a job audition.

But Ratcliffe's nomination for director was immediately stalled by
accusations that he had inflated his résumé. In the interim, Adam
Schiff, by now one of Trump's
\href{https://www.nytimes.com/2019/11/05/magazine/adam-schiff-impeachment.html}{most
prominent congressional critics}, suggested that Sue Gordon would be
``superbly qualified'' for acting director. Trump's son Donald Jr.
promptly tweeted: ``If Adam Schiff wants her in there, the rumors about
her being besties with Brennan and the rest of the clown cadre must be
100\% true.'' Gordon elected to resign.

Joseph Maguire was named acting director instead --- a relief to those
in the intelligence community who had recoiled at the thought of a Trump
loyalist like Ratcliffe overseeing them. But Trump himself made clear
that their relief would be temporary. Explaining to the White House
press corps why Ratcliffe was his preference, he said: ``I think we need
somebody like that that's strong and can really rein it in. As you've
all learned, the intelligence agencies have run amok. They've run
amok.''

Image

Credit...Jussi Nukari/Xinhua, via Getty Images

\textbf{On July 19, 2019,} nine days before Trump announced Coats's
departure, Coats created a new post within the intelligence community:
election-threats executive. He awarded the job to an analyst named
Shelby Pierson, who had worked in the community for over two decades,
most recently as a Russia issues manager, before Coats asked her in 2018
to serve as the O.D.N.I.'s crisis manager for election security.

Less than a month later, a C.I.A. whistle-blower reported to the
O.D.N.I. inspector general that Trump and members of his administration
had pressured Volodymyr Zelensky, the recently elected president of
Ukraine, to investigate the activities of Joe Biden, by then the likely
Democratic presidential nominee, and his son Hunter. The nation was soon
consumed with the impeachment proceedings against Trump over the Ukraine
affair. Beneath the din, Pierson and other senior intelligence officials
continued to meet and review Russia's influence campaign, past and
present. They learned that in the 2016 election, Russian cyberattacks
compromised voter-registration databases in Illinois and Florida and
hacked a Florida-based election-software vendor. They learned as well
that Russia would be focusing its 2020 efforts on the battleground
states. It was during this same period that the N.I.E. was finalized. In
early February of this year, Pierson and other intelligence officials
gave a classified briefing on prospective election threats to the Senate
Intelligence Committee. Nothing about the contents of this briefing made
its way into the press.

On the morning of Feb. 13, Pierson testified before the House
Intelligence Committee in the secure hearing room beneath the Capitol
Visitor Center that the committee uses for classified briefings. The
committee had recently held hearings on the grounds for Trump's
impeachment; tempers were raw and partisan confrontations inevitable.
The day before the hearing, a White House official called the committee
staff to ask whether someone from the West Wing could sit in on the
top-secret hearing. Denied permission to do so, an employee from the
White House Office of Legal Counsel nonetheless showed up that morning
and was denied entry.

The conference room was full, and nearly every committee member was
present. Pierson sat at the witness table, alongside senior officials
from the F.B.I., the C.I.A., the N.S.A. and the Department of Homeland
Security. Upward of two dozen support staff sat behind them. Pierson
began with a routine prepared statement about Russia's ongoing efforts.

After she finished, Schiff pointedly asked Pierson if the available
intelligence suggested whether Russia had a preference in this
November's outcome. Pierson replied that it did, and that Russia's
preference was for the current president. This was in keeping with Key
Judgment 2 of the previous July's N.I.E. draft --- the finding that was
softened in the final version issued five months before the hearing.
Pierson turned to the F.B.I. official seated beside her at the witness
table. The bureau official concurred with Pierson's assessment.

The
\href{https://www.nytimes.com/2020/02/20/us/politics/russian-interference-trump-democrats.html}{congressional
questioning} that followed ``was very contentious,'' one attendee
recalled. A number of Republican members of Congress vehemently objected
to Pierson's assertion that Putin favored Trump. Representative Will
Hurd of Texas, a former C.I.A. case officer, expressed doubt about the
sourcing of Pierson's assessment. Asked by one of the Republicans about
the Democratic candidate Bernie Sanders, Pierson acknowledged that there
was recent evidence in the primaries of pro-Sanders activity from
Russian trolls and bots. Still, as Coats had, Pierson stood behind the
intelligence community's original judgment. The hearing was adjourned
before noon.

Pierson reported to Maguire that the briefing had been heated. Indeed,
sometime later that day, according to a former senior intelligence
official with knowledge of the events, the House committee's ranking
minority member, Representative Devin Nunes, relayed to Trump what
Pierson said in her testimony. The following day, Feb. 14, Trump was
given a routine intelligence briefing on election security. Three
subject-matter briefers, along with Haspel, Beth Sanner and Maguire,
were in attendance.

In the middle of the briefing, according to one participant, Trump
interrupted and said to Maguire: ``Hey, Joe, I understand that you
briefed Adam Schiff and that you told him that Russia prefers me. Why
did you tell that to Schiff?'' Trump went on to say that he heard this
from several members of the committee and wanted to know why Maguire had
not informed Trump.

Maguire tried to explain that it was another intelligence official who
had given the testimony, during a routine bipartisan hearing. But Trump
continued to question Maguire, and the meeting then broke up. According
to the participant, as they were leaving, Sanner said: ``Mr. President,
Joe is not out to undermine you.''

Maguire left the Oval Office knowing that he would soon be fired. On the
evening of Feb. 19, he was informed by Robert O'Brien, who succeeded
Bolton as national security adviser, that Maguire's likely replacement
would need to be let into O.D.N.I. headquarters the following morning.
That morning, Maguire greeted his successor, wished him well and left
the building for good.

The new acting director was Richard Grenell, Trump's ambassador to
Germany. A 53-year-old former United Nations ambassador's spokesman,
media consultant and Fox News commentator with no previous experience in
the intelligence community, Grenell was best known as a pugnacious Trump
loyalist who made undiplomatic comments about his host country's
unwillingness to contribute more to NATO.

Grenell assured Pierson that her job was safe, as Pierson herself later
acknowledged to The Times and other media outlets. At the same time,
Pierson would have to sit by in silence as administration officials
insisted to the media that in the Feb. 13 briefing, she had
misrepresented the U.S. intelligence community's assessments about
Russia's preference for president. On ABC's ``This Week'' three days
after Maguire's departure, O'Brien told the host, George Stephanopoulos,
``I haven't seen any evidence that Russia is doing anything to get
President Trump re-elected.''

Instead, O'Brien said --- echoing a talking point Trump delivered at a
rally two days beforehand, and which Pence's chief of staff, Marc Short,
would also use that same morning on NBC's ``Meet the Press'' --- that
Russia's likely preference would be Bernie Sanders, a socialist who
``honey-mooned in Moscow.'' (Sanders visited Russia around the time of
his wedding, though not on a honeymoon.) Unnamed ``people familiar with
the matter'' leaked to The Washington Post a classified briefing that
took place over a month earlier on Jan. 8, in which the F.B.I. informed
Sanders that Russia appeared to be aiding his campaign --- omitting the
N.I.E. authors' view that the aid was seen in Moscow as a means to the
end of re-electing Trump.

Grenell's staff, meanwhile, instructed Maguire's chief of staff, Viraj
Mirani, to clear out his office. Other departures would follow during
Grenell's tenure: the O.D.N.I.'s principal deputy, Andrew Hallman; its
chief of operations, Dierdre Walsh; its inspector general, Michael
Atkinson, who had delivered the Ukraine whistle-blower's complaint to
the House Intelligence Committee after Maguire declined to do so; and
Russell Travers, Maguire's acting replacement as director of the
National Counterterrorism Center. An adviser assigned to Grenell, the
former Nunes protégé and Trump N.S.C. staff member Kashyap Patel,
undertook a thorough reorganization of the O.D.N.I. Even Grenell was
wary of Patel, who had expectations of being the acting director's
deputy and who while on Nunes's staff reportedly shared dubious
information about Ukraine with Trump, though that was not his field of
expertise. (Patel has denied this.)

With Coats and Maguire both gone, Patel set about fulfilling a White
House request to cut the O.D.N.I.'s staff, according to someone familiar
with the events. The concern within the intelligence community was that
downsizing could offer a pretext for purging individuals like the
anonymous C.I.A. analyst who filed the Ukraine whistle-blower complaint.
As Sean Patrick Maloney of the House Intelligence Committee told me,
``It seems pretty clear to me that in the wake of the whistle-blower
complaint, he'd put a bunch of political hacks in charge, so that he'd
never have to worry about the truth getting out from the intelligence
community.''

In May, Ratcliffe was confirmed as director in spite of the earlier
concerns about his résumé. Grenell returned to Germany. In response to
detailed questions regarding this article, Grenell offered a statement
blasting ``the typical Washington types that hate the fact that Donald
Trump is a Washington outsider unwilling to play the Washington game.''
Trump ``won't just let the system do its thing and give us another Iraq
W.M.D.-style assessment,'' continued Grenell, who served as a spokesman
in the State Department during George W. Bush's presidency.

\textbf{Schiff believes that} the decision by Joseph Maguire, an
apolitical official with the respect of the intelligence community's
rank and file, not to forward the Ukraine whistle-blower's complaint to
Congress was an instructive moment. ``Looking back on Director Maguire's
decision to withhold the complaint,'' he told me, ``I don't think that
would have been done, but for being aware that the administration would
have been unhappy had he not.''

The options faced by the intelligence community during Trump's
presidency have been stark: avoid infuriating the president but
compromise the agencies' ostensible independence, or assert that
independence and find yourself replaced with a more sycophantic
alternative.

But Schiff argues that this is a false choice. For Maguire,
``Withholding it was not enough to keep his job,'' Schiff said. ``And I
think people need to understand this about Donald Trump: It will never
be enough when you attempt to do his bidding. He'll bring in personnel
who are more malleable, and the result is a degradation in the quality
of the information. Maguire is now an object lesson for those in the
intelligence community.''

I spoke with Schiff on Friday, July 24. Earlier that day, the O.D.N.I.
released
\href{https://www.nytimes.com/2020/07/24/us/politics/election-interference-russia-china-iran.html}{an
official statement about election security threats} by William Evanina,
director of the National Counterintelligence and Security Center and a
Trump appointee. ``At this time,'' Evanina's statement said, ``we're
primarily concerned with China, Russia and Iran --- although other
nation states and nonstate actors could also do harm to our electoral
process.''

Once again, the compromise was small but hardly meaningless: As several
retired intelligence officials pointed out to me, it conflated the
aboveboard ``influence'' campaign conducted by China --- pressuring
politicians, countering criticism --- with the clandestine
``interference'' efforts by Russia to subvert the voting process. A week
later during a classified briefing, Nancy Pelosi, the speaker of the
House,
\href{https://www.politico.com/news/2020/07/31/nancy-pelosi-william-evanina-russia-meddling-389847}{upbraided
Evanina} for his misleading statement.

Just as this article was going to press --- and shortly after I
submitted a list of questions to the O.D.N.I. relating to its struggle
to avoid becoming politically compromised ---
\href{https://www.nytimes.com/2020/08/07/us/politics/russia-china-trump-biden-election-interference.html}{Evanina
put out a new statement}. In it, the O.D.N.I. at last acknowledged
publicly that Russia ``is using a range of measures to primarily
denigrate former Vice President Biden and what it sees as an anti-Russia
`establishment.''' In the same statement, however, Evanina also asserted
for the first time that both China and Iran were hoping to defeat Trump.
As with the preceding statement, the O.D.N.I. made no distinction
between Russia's sophisticated election-disrupting capabilities and the
less insidious influence campaigns of the two supposedly anti-Trump
countries. Like its predecessor, the statement seemed to be tortured
with political calculation --- an implicit declaration of anguish rather
than of independence.

It called to mind something the former C.I.A. acting director Michael
Morell said several months before, when we were discussing Russia's
interference in the 2016 election. ``This is the only time in American
history when we've been attacked by a foreign country and not come
together as a nation,'' Morell said. ``In fact, it split us further
apart. It was an inexpensive, relatively easy to carry out covert
mission. It deepened our divisions. I'm absolutely convinced that those
Russian intelligence officers who put together and managed the attack on
our democracy in 2016 all received medals personally from Vladimir
Putin.''

\hypertarget{our-2020-election-guide}{%
\section{Our 2020 Election Guide}\label{our-2020-election-guide}}

Updated Aug. 8, 2020

\begin{itemize}
\item
  \begin{center}\rule{0.5\linewidth}{\linethickness}\end{center}

  \hypertarget{the-latest}{%
  \subsection{The Latest}\label{the-latest}}

  \begin{itemize}
  \tightlist
  \item
    With 160 lawsuits filed over voting rules and President Trump's
    baseless claims of fraud, Election Day in America
    \href{https://www.nytimes.com/2020/08/08/us/politics/voting-nov-3-election.html?action=click\&pgtype=Article\&state=default\&region=BELOW_MAIN_CONTENT\&context=storylines_guide}{could
    become Election Month}.
  \end{itemize}
\item
  \begin{center}\rule{0.5\linewidth}{\linethickness}\end{center}

  \hypertarget{bidens-vp-search}{%
  \subsection{Biden's V.P. Search}\label{bidens-vp-search}}

  \begin{itemize}
  \tightlist
  \item
    \href{https://www.nytimes.com/article/biden-vice-president-2020.html?action=click\&pgtype=Article\&state=default\&region=BELOW_MAIN_CONTENT\&context=storylines_guide}{Here
    are 13 women} who have been under consideration to be Joe Biden's
    running mate, and why each might be chosen --- and might not be.
  \end{itemize}
\item
  \begin{center}\rule{0.5\linewidth}{\linethickness}\end{center}

  \hypertarget{keep-up-with-our-coverage}{%
  \subsection{Keep Up With Our
  Coverage}\label{keep-up-with-our-coverage}}

  \begin{itemize}
  \tightlist
  \item
    Get an
    \href{https://www.nytimes.com/newsletters/politics?action=click\&pgtype=Article\&state=default\&region=BELOW_MAIN_CONTENT\&context=storylines_guide}{email}
    recapping the day's news
  \end{itemize}

  \begin{itemize}
  \tightlist
  \item
    Download our mobile app on
    \href{https://apps.apple.com/us/app/nytimes/id284862083?ls=1\&mat_click_id=5c79ae7455014fd1bd66b5610c05b8f2-20191112-16948\&referrer=mat_click_id\%3D5c79ae7455014fd1bd66b5610c05b8f2-20191112-16948\%26link_click_id\%3D722930677036718082}{iOS}
    and
    \href{http://a.localytics.com/android?id=com.nytimes.android\&referrer=utm_source\%3Dother_nyt_mobile_web\%26utm_medium\%3DWeb\%2520page\%26utm_term\%3DGeneral\%2520Mobile\%2520Page\%26utm_campaign\%3DNYT\%2520Mobile\%2520General\%2520Page}{Android}
    and turn on Breaking News and Politics alerts
  \end{itemize}
\end{itemize}

Advertisement

\protect\hyperlink{after-bottom}{Continue reading the main story}

\hypertarget{site-index}{%
\subsection{Site Index}\label{site-index}}

\hypertarget{site-information-navigation}{%
\subsection{Site Information
Navigation}\label{site-information-navigation}}

\begin{itemize}
\tightlist
\item
  \href{https://help.nytimes.com/hc/en-us/articles/115014792127-Copyright-notice}{©~2020~The
  New York Times Company}
\end{itemize}

\begin{itemize}
\tightlist
\item
  \href{https://www.nytco.com/}{NYTCo}
\item
  \href{https://help.nytimes.com/hc/en-us/articles/115015385887-Contact-Us}{Contact
  Us}
\item
  \href{https://www.nytco.com/careers/}{Work with us}
\item
  \href{https://nytmediakit.com/}{Advertise}
\item
  \href{http://www.tbrandstudio.com/}{T Brand Studio}
\item
  \href{https://www.nytimes.com/privacy/cookie-policy\#how-do-i-manage-trackers}{Your
  Ad Choices}
\item
  \href{https://www.nytimes.com/privacy}{Privacy}
\item
  \href{https://help.nytimes.com/hc/en-us/articles/115014893428-Terms-of-service}{Terms
  of Service}
\item
  \href{https://help.nytimes.com/hc/en-us/articles/115014893968-Terms-of-sale}{Terms
  of Sale}
\item
  \href{https://spiderbites.nytimes.com}{Site Map}
\item
  \href{https://help.nytimes.com/hc/en-us}{Help}
\item
  \href{https://www.nytimes.com/subscription?campaignId=37WXW}{Subscriptions}
\end{itemize}
