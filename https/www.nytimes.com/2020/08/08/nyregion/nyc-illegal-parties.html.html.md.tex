Sections

SEARCH

\protect\hyperlink{site-content}{Skip to
content}\protect\hyperlink{site-index}{Skip to site index}

\href{https://www.nytimes.com/section/nyregion}{New York}

\href{https://myaccount.nytimes.com/auth/login?response_type=cookie\&client_id=vi}{}

\href{https://www.nytimes.com/section/todayspaper}{Today's Paper}

\href{/section/nyregion}{New York}\textbar{}Rave Under the Kosciuszko
Bridge: Are Illicit Parties Endangering N.Y.C.?

\href{https://nyti.ms/3gE3DId}{https://nyti.ms/3gE3DId}

\begin{itemize}
\item
\item
\item
\item
\item
\end{itemize}

\href{https://www.nytimes.com/news-event/coronavirus?action=click\&pgtype=Article\&state=default\&region=TOP_BANNER\&context=storylines_menu}{The
Coronavirus Outbreak}

\begin{itemize}
\tightlist
\item
  live\href{https://www.nytimes.com/2020/08/08/world/coronavirus-updates.html?action=click\&pgtype=Article\&state=default\&region=TOP_BANNER\&context=storylines_menu}{Latest
  Updates}
\item
  \href{https://www.nytimes.com/interactive/2020/us/coronavirus-us-cases.html?action=click\&pgtype=Article\&state=default\&region=TOP_BANNER\&context=storylines_menu}{Maps
  and Cases}
\item
  \href{https://www.nytimes.com/interactive/2020/science/coronavirus-vaccine-tracker.html?action=click\&pgtype=Article\&state=default\&region=TOP_BANNER\&context=storylines_menu}{Vaccine
  Tracker}
\item
  \href{https://www.nytimes.com/interactive/2020/world/coronavirus-tips-advice.html?action=click\&pgtype=Article\&state=default\&region=TOP_BANNER\&context=storylines_menu}{F.A.Q.}
\item
  \href{https://www.nytimes.com/live/2020/08/07/business/stock-market-today-coronavirus?action=click\&pgtype=Article\&state=default\&region=TOP_BANNER\&context=storylines_menu}{Markets
  \& Economy}
\end{itemize}

Advertisement

\protect\hyperlink{after-top}{Continue reading the main story}

Supported by

\protect\hyperlink{after-sponsor}{Continue reading the main story}

\hypertarget{rave-under-the-kosciuszko-bridge-are-illicit-parties-endangering-nyc}{%
\section{Rave Under the Kosciuszko Bridge: Are Illicit Parties
Endangering
N.Y.C.?}\label{rave-under-the-kosciuszko-bridge-are-illicit-parties-endangering-nyc}}

Videos and photos posted on social media of a number of parties show few
guests abiding by social-distancing guidelines.

\includegraphics{https://static01.nyt.com/images/2020/08/04/nyregion/00nyvirus-parties/00nyvirus-parties-articleLarge.jpg?quality=75\&auto=webp\&disable=upscale}

\href{https://www.nytimes.com/by/mihir-zaveri}{\includegraphics{https://static01.nyt.com/images/2018/07/18/multimedia/author-mihir-zaveri/author-mihir-zaveri-thumbLarge.png}}

By \href{https://www.nytimes.com/by/mihir-zaveri}{Mihir Zaveri}

\begin{itemize}
\item
  Aug. 8, 2020Updated 3:59 p.m. ET
\item
  \begin{itemize}
  \item
  \item
  \item
  \item
  \item
  \end{itemize}
\end{itemize}

On a humid Saturday night, under a segment of the Kosciuszko Bridge,
which connects Brooklyn and Queens, hundreds of people at an illicit
gathering danced and swayed to the thumps of hip-hop and electronic
music. Some wore masks. Many did not.

Many were attending their first party in months, since the pandemic had
forced many venues to close.

``People were just enjoying themselves,'' said one of the attendees,
Jimmy Escobar, 30, of Brooklyn. ``People got locked up for so long, and
they finally got to go out.''

New Yorkers, by and large, have adhered to rules mandating social
distancing and mask wearing. The diligence has helped keep the
coronavirus under control in the city even as outbreaks have raged
across the country, primarily in the South and the West.

As the summer wears on, however, mounting reports of parties, concerts
and other social events, like the rave attended by Mr. Escobar, are
raising fears that New York's hard-earned stability may be tenuous.

Over the last few weeks, videos and photos posted on social media --- at
bars, at beaches, at warehouses, at pools, at hotels --- show densely
packed, mask-free crowds, similar to the Memorial Day weekend gatherings
at
\href{https://www.nytimes.com/2020/05/26/us/lake-of-the-ozarks-coronavirus.html}{Lake
of the Ozarks in Missouri} and in states like California and Florida
that are now reeling from virus outbreaks.

The images contrast sharply with the memories of a brutal spring in New
York that left tens of thousands dead, disproportionately ravaging
low-income communities and neighborhoods with high numbers of Black and
Latino people.

``It's disrespectful,'' Gov. Andrew M. Cuomo said at a recent news
conference about the partying. ``It's illegal. It not only violates
public health, but it violates human decency.''

Many of the images show how a segment of the nightlife industry, a
crucial piece of New York City's culture, is desperately trying to
revive itself after having been shut down when the pandemic hit.

But other events, which charge for tickets, drinks or other amenities,
perhaps illustrate how some people are looking to capitalize on the
public's restlessness.

\hypertarget{latest-updates-the-coronavirus-outbreak}{%
\section{\texorpdfstring{\href{https://www.nytimes.com/2020/08/07/world/covid-19-news.html?action=click\&pgtype=Article\&state=default\&region=MAIN_CONTENT_1\&context=storylines_live_updates}{Latest
Updates: The Coronavirus
Outbreak}}{Latest Updates: The Coronavirus Outbreak}}\label{latest-updates-the-coronavirus-outbreak}}

Updated 2020-08-08T12:04:28.992Z

\begin{itemize}
\tightlist
\item
  \href{https://www.nytimes.com/2020/08/07/world/covid-19-news.html?action=click\&pgtype=Article\&state=default\&region=MAIN_CONTENT_1\&context=storylines_live_updates\#link-1f86d03a}{As
  the U.S. relief talks falter again, Trump says he is prepared to act
  on his own.}
\item
  \href{https://www.nytimes.com/2020/08/07/world/covid-19-news.html?action=click\&pgtype=Article\&state=default\&region=MAIN_CONTENT_1\&context=storylines_live_updates\#link-3f64a70a}{Cuomo
  says N.Y. schools can reopen in-person but leaves it up to districts
  to determine if, when and how.}
\item
  \href{https://www.nytimes.com/2020/08/07/world/covid-19-news.html?action=click\&pgtype=Article\&state=default\&region=MAIN_CONTENT_1\&context=storylines_live_updates\#link-14e70066}{Thousands
  of cases went unreported in California when a computer server failed.}
\end{itemize}

\href{https://www.nytimes.com/2020/08/07/world/covid-19-news.html?action=click\&pgtype=Article\&state=default\&region=MAIN_CONTENT_1\&context=storylines_live_updates}{See
more updates}

More live coverage:
\href{https://www.nytimes.com/live/2020/08/07/business/stock-market-today-coronavirus?action=click\&pgtype=Article\&state=default\&region=MAIN_CONTENT_1\&context=storylines_live_updates}{Markets}

Osvaldo Chance Jimenez, 44, who has helped organize underground parties
in New York City in the past, said the growing number of events risked
seeding future outbreaks, which would likely disproportionately affect
communities of color, particularly in Brooklyn and Queens where many of
the gatherings are taking place.

Mr. Jimenez has used \href{https://www.instagram.com/hilovenewyork/}{his
Instagram account, @hilovenewyork}, to draw attention to what he sees as
reckless behavior. He pointed to yacht parties where organizers are
selling tickets for up to \$100, and a requirement at a day club at the
Ravel Hotel in Long Island City, Queens, that guests pay between \$35
and \$50 to take a rapid coronavirus test on site, as examples of what
he called ``vulture capitalism.''

``It is the arrogance of money,'' he said. ``These people do not care.''

Lauren Flax, a D.J., producer and artist based in Brooklyn, said people
should not be partying yet. Ms. Flax has lost work and is living off
unemployment checks and other government assistance. But she said that
even thinking about holding a dance party would be irresponsible before
there was a better understanding of the virus and better testing
technology.

``I don't think any of us should be thinking about our career right
now,'' she said.

Asked this week about the rave under the Kosciuszko Bridge, and an
illegal boat party with more than 170 guests that was held over the
weekend, \href{https://www.youtube.com/watch?v=SaaJUcexvog}{Mayor Bill
de Blasio said} the authorities had moved quickly to address a few cases
of wrongdoing and that most people were following the rules.

``But where we see something wrong, we got to go in and stop it
immediately,'' the mayor said.

The New York City Sheriff's Office, one of the city agencies tasked with
enforcing the social distancing rules during the pandemic, has responded
to several reports of illegal parties since the pandemic began,
including the boat party, Sheriff Joseph Fucito said.

He said the office was trying to take a more proactive approach to stop
the parties before they take place.

Dr. Jay Varma, Mr. de Blasio's senior adviser for public health, said
the city had not ``seen any large clusters specifically associated with
any of these events.'' But about 15 percent of people who test positive
in the city, and are interviewed by contact tracers, reported being at
some sort of gathering outside of their home, city officials said.

\href{https://www1.nyc.gov/site/doh/covid/covid-19-data.page}{City data}
indicates that while the parties and other gatherings are on the rise,
the outbreak does not appear to be worsening. The number of new
hospitalizations in a day has not reached 50 in weeks --- in March and
April, it was routinely higher than 1,000 per day, according to city
data.

An uptick in cases also did not materialize after thousands of
protesters, many of them wearing masks, gathered for weeks during Black
Lives Matter demonstrations across the city.

But the fear, Mr. Jimenez and others say, is that younger people who get
infected while attending a party, and who may be less likely to be
severely affected by the virus, will spread it to more vulnerable
people.

``Things are only going to get worse,'' he said. ``I feel for my city. I
pray that I'm wrong.''

Andrew Rigie, the executive director of the New York City Hospitality
Alliance, said the parties should not come as a surprise.

``People will socialize in nightlife spaces, whether you like it or
not,'' Mr. Rigie said. ``You could have unsafe, unregulated nightlife.
Or you can do everything you can to have safe, regulated nightlife. But
we can't act as if it's not going to exist.''

Mr. Rigie said that local, state and federal governments should be
providing financial assistance, like rent support, to these businesses
and others affected by the nightlife shutdown, and that there should be
better guidance and research on how to operate safely.

Some businesses, like the Ravel Hotel, have tried to figure that out
themselves, with mixed results. The hotel's day club, Profundo, which
has an outdoor rooftop pool, reopened in late June at 50 percent
capacity and required guests to be tested for the virus on site.

\href{https://www.nytimes.com/news-event/coronavirus?action=click\&pgtype=Article\&state=default\&region=MAIN_CONTENT_3\&context=storylines_faq}{}

\hypertarget{the-coronavirus-outbreak-}{%
\subsubsection{The Coronavirus Outbreak
›}\label{the-coronavirus-outbreak-}}

\hypertarget{frequently-asked-questions}{%
\paragraph{Frequently Asked
Questions}\label{frequently-asked-questions}}

Updated August 6, 2020

\begin{itemize}
\item ~
  \hypertarget{why-are-bars-linked-to-outbreaks}{%
  \paragraph{Why are bars linked to
  outbreaks?}\label{why-are-bars-linked-to-outbreaks}}

  \begin{itemize}
  \tightlist
  \item
    Think about a bar. Alcohol is flowing. It can be loud, but it's
    definitely intimate, and you often need to lean in close to hear
    your friend. And strangers have way, way fewer reservations about
    coming up to people in a bar. That's sort of the point of a bar.
    Feeling good and close to strangers. It's no surprise, then, that
    \href{https://www.nytimes.com/2020/07/02/us/coronavirus-bars.html?action=click\&pgtype=Article\&state=default\&region=MAIN_CONTENT_3\&context=storylines_faq}{bars
    have been linked to outbreaks in several states.} Louisiana health
    officials have tied
    \href{https://www.nytimes.com/2020/06/22/us/new-coronavirus-phase.html?action=click\&pgtype=Article\&state=default\&region=MAIN_CONTENT_3\&context=storylines_faq}{at
    least 100 coronavirus cases} to bars in the Tigerland nightlife
    district in Baton Rouge. Minnesota has traced 328 recent cases to
    bars across the state.
    \href{https://www.boisestatepublicradio.org/post/bars-large-venues-close-ada-county-after-surge-coronavirus-prompts-rollback\#stream/0}{In
    Idaho}, health officials shut down bars in Ada County after
    reporting clusters of infections among young adults who had visited
    several bars in downtown Boise. Governors in
    \href{https://www.nytimes.com/2020/07/01/us/california-coronavirus-reopening.html?action=click\&pgtype=Article\&state=default\&region=MAIN_CONTENT_3\&context=storylines_faq}{California},
    \href{https://www.nytimes.com/2020/06/14/us/coronavirus-united-states.html?action=click\&pgtype=Article\&state=default\&region=MAIN_CONTENT_3\&context=storylines_faq}{Texas
    and Arizona}, where coronavirus cases are soaring, have ordered
    hundreds of newly reopened bars to shut down. Less than two weeks
    after Colorado's bars reopened at limited capacity, Gov. Jared Polis
    \href{https://www.denverpost.com/2020/06/30/colorado-bars-closed-coronavirus/}{ordered
    them to close}.
  \end{itemize}
\item ~
  \hypertarget{i-have-antibodies-am-i-now-immune}{%
  \paragraph{I have antibodies. Am I now
  immune?}\label{i-have-antibodies-am-i-now-immune}}

  \begin{itemize}
  \tightlist
  \item
    As of right now,
    \href{https://www.nytimes.com/2020/07/22/health/covid-antibodies-herd-immunity.html?action=click\&pgtype=Article\&state=default\&region=MAIN_CONTENT_3\&context=storylines_faq}{that
    seems likely, for at least several months.} There have been
    frightening accounts of people suffering what seems to be a second
    bout of Covid-19. But experts say these patients may have a
    drawn-out course of infection, with the virus taking a slow toll
    weeks to months after initial exposure. People infected with the
    coronavirus typically
    \href{https://www.nature.com/articles/s41586-020-2456-9}{produce}
    immune molecules called antibodies, which are
    \href{https://www.nytimes.com/2020/05/07/health/coronavirus-antibody-prevalence.html?action=click\&pgtype=Article\&state=default\&region=MAIN_CONTENT_3\&context=storylines_faq}{protective
    proteins made in response to an
    infection}\href{https://www.nytimes.com/2020/05/07/health/coronavirus-antibody-prevalence.html?action=click\&pgtype=Article\&state=default\&region=MAIN_CONTENT_3\&context=storylines_faq}{.
    These antibodies may} last in the body
    \href{https://www.nature.com/articles/s41591-020-0965-6}{only two to
    three months}, which may seem worrisome, but that's perfectly normal
    after an acute infection subsides, said Dr. Michael Mina, an
    immunologist at Harvard University. It may be possible to get the
    coronavirus again, but it's highly unlikely that it would be
    possible in a short window of time from initial infection or make
    people sicker the second time.
  \end{itemize}
\item ~
  \hypertarget{im-a-small-business-owner-can-i-get-relief}{%
  \paragraph{I'm a small-business owner. Can I get
  relief?}\label{im-a-small-business-owner-can-i-get-relief}}

  \begin{itemize}
  \tightlist
  \item
    The
    \href{https://www.nytimes.com/article/small-business-loans-stimulus-grants-freelancers-coronavirus.html?action=click\&pgtype=Article\&state=default\&region=MAIN_CONTENT_3\&context=storylines_faq}{stimulus
    bills enacted in March} offer help for the millions of American
    small businesses. Those eligible for aid are businesses and
    nonprofit organizations with fewer than 500 workers, including sole
    proprietorships, independent contractors and freelancers. Some
    larger companies in some industries are also eligible. The help
    being offered, which is being managed by the Small Business
    Administration, includes the Paycheck Protection Program and the
    Economic Injury Disaster Loan program. But lots of folks have
    \href{https://www.nytimes.com/interactive/2020/05/07/business/small-business-loans-coronavirus.html?action=click\&pgtype=Article\&state=default\&region=MAIN_CONTENT_3\&context=storylines_faq}{not
    yet seen payouts.} Even those who have received help are confused:
    The rules are draconian, and some are stuck sitting on
    \href{https://www.nytimes.com/2020/05/02/business/economy/loans-coronavirus-small-business.html?action=click\&pgtype=Article\&state=default\&region=MAIN_CONTENT_3\&context=storylines_faq}{money
    they don't know how to use.} Many small-business owners are getting
    less than they expected or
    \href{https://www.nytimes.com/2020/06/10/business/Small-business-loans-ppp.html?action=click\&pgtype=Article\&state=default\&region=MAIN_CONTENT_3\&context=storylines_faq}{not
    hearing anything at all.}
  \end{itemize}
\item ~
  \hypertarget{what-are-my-rights-if-i-am-worried-about-going-back-to-work}{%
  \paragraph{What are my rights if I am worried about going back to
  work?}\label{what-are-my-rights-if-i-am-worried-about-going-back-to-work}}

  \begin{itemize}
  \tightlist
  \item
    Employers have to provide
    \href{https://www.osha.gov/SLTC/covid-19/standards.html}{a safe
    workplace} with policies that protect everyone equally.
    \href{https://www.nytimes.com/article/coronavirus-money-unemployment.html?action=click\&pgtype=Article\&state=default\&region=MAIN_CONTENT_3\&context=storylines_faq}{And
    if one of your co-workers tests positive for the coronavirus, the
    C.D.C.} has said that
    \href{https://www.cdc.gov/coronavirus/2019-ncov/community/guidance-business-response.html}{employers
    should tell their employees} -\/- without giving you the sick
    employee's name -\/- that they may have been exposed to the virus.
  \end{itemize}
\item ~
  \hypertarget{what-is-school-going-to-look-like-in-september}{%
  \paragraph{What is school going to look like in
  September?}\label{what-is-school-going-to-look-like-in-september}}

  \begin{itemize}
  \tightlist
  \item
    It is unlikely that many schools will return to a normal schedule
    this fall, requiring the grind of
    \href{https://www.nytimes.com/2020/06/05/us/coronavirus-education-lost-learning.html?action=click\&pgtype=Article\&state=default\&region=MAIN_CONTENT_3\&context=storylines_faq}{online
    learning},
    \href{https://www.nytimes.com/2020/05/29/us/coronavirus-child-care-centers.html?action=click\&pgtype=Article\&state=default\&region=MAIN_CONTENT_3\&context=storylines_faq}{makeshift
    child care} and
    \href{https://www.nytimes.com/2020/06/03/business/economy/coronavirus-working-women.html?action=click\&pgtype=Article\&state=default\&region=MAIN_CONTENT_3\&context=storylines_faq}{stunted
    workdays} to continue. California's two largest public school
    districts --- Los Angeles and San Diego --- said on July 13, that
    \href{https://www.nytimes.com/2020/07/13/us/lausd-san-diego-school-reopening.html?action=click\&pgtype=Article\&state=default\&region=MAIN_CONTENT_3\&context=storylines_faq}{instruction
    will be remote-only in the fall}, citing concerns that surging
    coronavirus infections in their areas pose too dire a risk for
    students and teachers. Together, the two districts enroll some
    825,000 students. They are the largest in the country so far to
    abandon plans for even a partial physical return to classrooms when
    they reopen in August. For other districts, the solution won't be an
    all-or-nothing approach.
    \href{https://bioethics.jhu.edu/research-and-outreach/projects/eschool-initiative/school-policy-tracker/}{Many
    systems}, including the nation's largest, New York City, are
    devising
    \href{https://www.nytimes.com/2020/06/26/us/coronavirus-schools-reopen-fall.html?action=click\&pgtype=Article\&state=default\&region=MAIN_CONTENT_3\&context=storylines_faq}{hybrid
    plans} that involve spending some days in classrooms and other days
    online. There's no national policy on this yet, so check with your
    municipal school system regularly to see what is happening in your
    community.
  \end{itemize}
\end{itemize}

The rapid tests used by the hotel were made by Abbott Laboratories,
\href{https://gothamist.com/arts-entertainment/profundo-ravel-covid-test-rooftop-pandemic-pool-parties-rage-lic}{according
to Gothamist}. The tests, authorized by the Food and Drug Administration
for emergency use, were designed to provide results within minutes.

\includegraphics{https://static01.nyt.com/images/2020/08/07/nyregion/00nyvirus-parties2/merlin_175350612_aa2f111c-3296-4fc9-9373-3012dbc1db3b-articleLarge.jpg?quality=75\&auto=webp\&disable=upscale}

But the F.D.A. had said in May that the tests might be delivering false
negative results, raising questions about safety at Profundo.

Then, on July 4, videos and photos posted on social media showed several
poolside guests at the hotel not wearing masks, and standing and dancing
close together. In a statement, the hotel acknowledged that those guests
were ``not following directions,'' but said that it had ``retrained
security and staff on how to handle these situations moving forward''
with the help of the city.

An F.D.A. spokeswoman said this week that the agency had since worked
with Abbott to improve the test. A spokeswoman for the hotel said this
week that the tests were no longer being administered, but would not
answer questions about why.

Seth Levine, an owner of the Ravel Hotel, said in a separate statement
that the site also provides guests with hand sanitizer, masks and
printed rules about social distancing. A security team makes sure guests
wear masks when moving around the property, he said.

Some guests said they believed that the tests and other measures had
adequately lowered the risk of infection or transmission. Joey Sutera,
who works in marketing, said he went to the rooftop pool on July 4 with
30 friends.

``We look at New York and there's zero deaths,'' he said, referencing a
day last month in which the city
\href{https://www.cnn.com/2020/07/13/health/new-york-city-coronavirus-zero-deaths/index.html}{did
not initially report} any coronavirus-related fatalities. ``And it's
like, New York has it under control. So, is the reward greater than the
risk when we're young and been locked up for so long in solitary
confinement? I think people are willing to take that risk.''

A D.J. \href{https://www.instagram.com/p/CDbmjr1Delp/}{performed at
Profundo on Monday}, and the venue was
\href{https://www.instagram.com/p/CDggXIED4Yn/}{beckoning guests to
gather} there this weekend. One of its posts on Instagram offered a free
bottle of rosé to some guests who missed celebrating their birthday
because of the pandemic. A poolside table was listed at \$500 on the
hotel's website.

Mr. Escobar, who attended the party under the bridge, said he was not
worried about contracting or transmitting the virus. He did not have
symptoms, he said, and a sign at the party told people to wear a face
covering. He saw people handing out hand sanitizer and masks.

``If people want to go out and enjoy themselves, regardless of risk, let
them do it,'' he said.

But Kristina Alaniesse, 36, who has worked as an event promoter and now
posts images on Instagram of reckless behavior at parties, said the
danger was not only for the partygoers, but the people they interact
with later.

``It's not a time to celebrate,'' she said. ``We're not out of the
woods.''

Advertisement

\protect\hyperlink{after-bottom}{Continue reading the main story}

\hypertarget{site-index}{%
\subsection{Site Index}\label{site-index}}

\hypertarget{site-information-navigation}{%
\subsection{Site Information
Navigation}\label{site-information-navigation}}

\begin{itemize}
\tightlist
\item
  \href{https://help.nytimes.com/hc/en-us/articles/115014792127-Copyright-notice}{©~2020~The
  New York Times Company}
\end{itemize}

\begin{itemize}
\tightlist
\item
  \href{https://www.nytco.com/}{NYTCo}
\item
  \href{https://help.nytimes.com/hc/en-us/articles/115015385887-Contact-Us}{Contact
  Us}
\item
  \href{https://www.nytco.com/careers/}{Work with us}
\item
  \href{https://nytmediakit.com/}{Advertise}
\item
  \href{http://www.tbrandstudio.com/}{T Brand Studio}
\item
  \href{https://www.nytimes.com/privacy/cookie-policy\#how-do-i-manage-trackers}{Your
  Ad Choices}
\item
  \href{https://www.nytimes.com/privacy}{Privacy}
\item
  \href{https://help.nytimes.com/hc/en-us/articles/115014893428-Terms-of-service}{Terms
  of Service}
\item
  \href{https://help.nytimes.com/hc/en-us/articles/115014893968-Terms-of-sale}{Terms
  of Sale}
\item
  \href{https://spiderbites.nytimes.com}{Site Map}
\item
  \href{https://help.nytimes.com/hc/en-us}{Help}
\item
  \href{https://www.nytimes.com/subscription?campaignId=37WXW}{Subscriptions}
\end{itemize}
