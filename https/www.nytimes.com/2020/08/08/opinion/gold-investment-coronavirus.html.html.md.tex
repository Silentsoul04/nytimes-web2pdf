Sections

SEARCH

\protect\hyperlink{site-content}{Skip to
content}\protect\hyperlink{site-index}{Skip to site index}

\href{https://myaccount.nytimes.com/auth/login?response_type=cookie\&client_id=vi}{}

\href{https://www.nytimes.com/section/todayspaper}{Today's Paper}

\href{/section/opinion}{Opinion}\textbar{}Why Is Everyone Buying Gold?

\href{https://nyti.ms/31Aeooj}{https://nyti.ms/31Aeooj}

\begin{itemize}
\item
\item
\item
\item
\item
\end{itemize}

Advertisement

\protect\hyperlink{after-top}{Continue reading the main story}

\href{/section/opinion}{Opinion}

Supported by

\protect\hyperlink{after-sponsor}{Continue reading the main story}

\hypertarget{why-is-everyone-buying-gold}{%
\section{Why Is Everyone Buying
Gold?}\label{why-is-everyone-buying-gold}}

It's one of the best performing assets in the world this year. That's
not a great sign.

\href{https://www.nytimes.com/by/ruchir-sharma}{\includegraphics{https://static01.nyt.com/images/2018/04/02/opinion/ruchir-sharma/ruchir-sharma-thumbLarge.png}}

By \href{https://www.nytimes.com/by/ruchir-sharma}{Ruchir Sharma}

Mr. Sharma is a global investor and contributing Opinion writer.

\begin{itemize}
\item
  Aug. 8, 2020, 11:00 a.m. ET
\item
  \begin{itemize}
  \item
  \item
  \item
  \item
  \item
  \end{itemize}
\end{itemize}

\includegraphics{https://static01.nyt.com/images/2020/08/08/opinion/08Sharma/08Sharma-articleLarge.jpg?quality=75\&auto=webp\&disable=upscale}

Gold bugs --- investors perpetually bullish on gold --- have long been
seen as a paranoid fringe of the financial world, holding the shiny
asset as a hedge against a disaster they always think is near. But
lately, they appear to be on to something. This year, gold is the best
performing traditional asset in the world. Its price just topped \$2,000
an ounce for the first time. From serious investors to newly minted day
traders, everyone is talking up its virtues.

A
\href{https://www.magnifymoney.com/blog/news/gold-and-cryptocurrency-gain-popularity-amid-covid-19/}{recent
survey} of 1,000 people found that one in six Americans bought gold or
other precious metals in the last three months, and about one in four
were seriously thinking about it. On Robinhood, the popular online
trading platform, the number of users holding two of its largest gold
funds has tripled since January.

It seems we're all gold bugs now.

It's tempting to attribute the vogue for gold to a desire for a safe
haven during the pandemic --- a kind of financial panic reflex that will
release as the crisis abates. But the gold mania is also driven by a
hunch that the easy money pouring out of central banks and government
stimulus programs could trigger inflation, which makes it a more
worrisome economic omen.

Serious investors have in the past dismissed gold as an asset that for
the most part just sits there yielding nothing. In many ways, gold is
like oil or iron ore or any other commodity people dig out of the
ground. Most commodity prices rise and fall in cycles, gaining nothing
in value over time.

Owing to its image as a stable store of value when others are shaky,
gold has held up better than other commodities, but it still hasn't been
a dynamic investment. Over the past century, the price of gold, adjusted
for inflation, has risen by an average of just 1.1 percent a year,
compared with 6.5 percent for U.S. stocks. Even the 10-year U.S.
Treasury bond, considered the most risk free asset in the world, has
produced higher annual returns.

Gold has shone mainly in hard luck moments. It surged amid the
stagflation of the 1970s, rising more than sevenfold over the course of
that decade to peak at \$850 in early 1980. It surged again after the
global financial crisis of 2008, peaking at \$1,900 in 2011, but then it
slid backward over much of the subsequent decade.

In 2019, after the Federal Reserve signaled that it was suspending plans
to push interest rates higher, gold mounted another ascent.
Historically, gold has done best when interest rates fall below the rate
of inflation. As the inflation-adjusted return on bonds turns negative,
investors feel comfortable owning gold as a store of value, even if it
yields nothing.

That is what has been happening over the past few months. With bond
yields near zero in the United States and negative in Europe and Japan,
investors have driven up the price of gold more than 30 percent this
year after a gain of nearly 20 percent last year. In recent weeks, that
surge has been turbocharged by growing expectations that all the money
governments are pumping into their economies will reignite inflation.

In addition, with valuations of stocks well above their long-term
average, gold appears relatively cheap. And with central banks printing
money hand over fist, some see gold as a stable alternative to the
dollar and other major currencies. (Gold is also pulling up the price of
its less glamorous relative, silver, which is rising from an unusually
depressed level because people see it as a cheaper play on the same
trends.)

For gold to keep rallying, expectations of inflation will have to keep
rising. Anticipating higher inflation has been a losing bet for a large
part of the last four decades, but the odds appear better now. Most
nations are doling out record levels of stimulus at a time when forces
like globalization, which kept inflation in check, are weakening.
Normally, if inflation looms, central banks can be relied on to raise
interest rates, but Fed officials have
\href{https://www.cnbc.com/2020/06/10/fed-meeting-decision-interest-rates.html}{signaled}
that they aren't ``thinking about thinking about raising rates,'' and do
not expect to move before 2022.

This is not a healthy turn. When interest rates are this low, money is
virtually free, encouraging speculation in assets of no value to
society, beyond what the seller can get for them. Gold is the prime
example just now. The wider risk is that this kind of purely financial
speculation undermines the economy by sucking capital away from
industries that will put it to more productive use.

As an investment, gold has none of the virtues I admire, like innovation
and dynamism, and many of the vices I despise, including the ``rent
seeking'' mind-set typical of extractive industries. But these times
aren't normal. Unless a vaccine emerges quickly, central banks stop
printing money frantically and real interest rates start rising again,
it is difficult not to be a gold bug now.

Ruchir Sharma is the chief global strategist at Morgan Stanley
Investment Management, the author, most recently, of ``The Ten Rules of
Successful Nations'' and a contributing Opinion writer. This essay
reflects his opinions alone.

\emph{The Times is committed to publishing}
\href{https://www.nytimes.com/2019/01/31/opinion/letters/letters-to-editor-new-york-times-women.html}{\emph{a
diversity of letters}} \emph{to the editor. We'd like to hear what you
think about this or any of our articles. Here are some}
\href{https://help.nytimes.com/hc/en-us/articles/115014925288-How-to-submit-a-letter-to-the-editor}{\emph{tips}}\emph{.
And here's our email:}
\href{mailto:letters@nytimes.com}{\emph{letters@nytimes.com}}\emph{.}

\emph{Follow The New York Times Opinion section on}
\href{https://www.facebook.com/nytopinion}{\emph{Facebook}}\emph{,}
\href{http://twitter.com/NYTOpinion}{\emph{Twitter (@NYTopinion)}}
\emph{and}
\href{https://www.instagram.com/nytopinion/}{\emph{Instagram}}\emph{.}

Advertisement

\protect\hyperlink{after-bottom}{Continue reading the main story}

\hypertarget{site-index}{%
\subsection{Site Index}\label{site-index}}

\hypertarget{site-information-navigation}{%
\subsection{Site Information
Navigation}\label{site-information-navigation}}

\begin{itemize}
\tightlist
\item
  \href{https://help.nytimes.com/hc/en-us/articles/115014792127-Copyright-notice}{©~2020~The
  New York Times Company}
\end{itemize}

\begin{itemize}
\tightlist
\item
  \href{https://www.nytco.com/}{NYTCo}
\item
  \href{https://help.nytimes.com/hc/en-us/articles/115015385887-Contact-Us}{Contact
  Us}
\item
  \href{https://www.nytco.com/careers/}{Work with us}
\item
  \href{https://nytmediakit.com/}{Advertise}
\item
  \href{http://www.tbrandstudio.com/}{T Brand Studio}
\item
  \href{https://www.nytimes.com/privacy/cookie-policy\#how-do-i-manage-trackers}{Your
  Ad Choices}
\item
  \href{https://www.nytimes.com/privacy}{Privacy}
\item
  \href{https://help.nytimes.com/hc/en-us/articles/115014893428-Terms-of-service}{Terms
  of Service}
\item
  \href{https://help.nytimes.com/hc/en-us/articles/115014893968-Terms-of-sale}{Terms
  of Sale}
\item
  \href{https://spiderbites.nytimes.com}{Site Map}
\item
  \href{https://help.nytimes.com/hc/en-us}{Help}
\item
  \href{https://www.nytimes.com/subscription?campaignId=37WXW}{Subscriptions}
\end{itemize}
