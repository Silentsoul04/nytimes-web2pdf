Sections

SEARCH

\protect\hyperlink{site-content}{Skip to
content}\protect\hyperlink{site-index}{Skip to site index}

\href{/section/sports/football}{Pro Football}\textbar{}Bear Hugs and
Bubbles: Why Some N.F.L. Players Opted Out

\href{https://nyti.ms/2PwsAt0}{https://nyti.ms/2PwsAt0}

\begin{itemize}
\item
\item
\item
\item
\item
\end{itemize}

\includegraphics{https://static01.nyt.com/images/2020/08/08/sports/08nfl-optouts-1/merlin_175405419_33dfb7cf-fa3d-40d2-9094-3dc79fc00fc7-articleLarge.jpg?quality=75\&auto=webp\&disable=upscale}

\hypertarget{bear-hugs-and-bubbles-why-some-nfl-players-opted-out}{%
\section{Bear Hugs and Bubbles: Why Some N.F.L. Players Opted
Out}\label{bear-hugs-and-bubbles-why-some-nfl-players-opted-out}}

``I'm literally bear-hugging another creature on the other side of the
ball every single play,'' Leo Koloamatangi, a Jets offensive lineman,
said. ``If that guy has any symptoms, I'm going to get them.''

Credit...Christian Monterrosa for The New York Times

Supported by

\protect\hyperlink{after-sponsor}{Continue reading the main story}

\href{https://www.nytimes.com/by/ben-shpigel}{\includegraphics{https://static01.nyt.com/images/2018/02/20/multimedia/author-ben-shpigel/author-ben-shpigel-thumbLarge.jpg}}

By \href{https://www.nytimes.com/by/ben-shpigel}{Ben Shpigel}

\begin{itemize}
\item
  Aug. 8, 2020Updated 10:40 a.m. ET
\item
  \begin{itemize}
  \item
  \item
  \item
  \item
  \item
  \end{itemize}
\end{itemize}

When his daughter, Marae, is old enough, Philadelphia Eagles receiver
Marquise Goodwin will teach her to put family first. He will say that
she should prioritize the people she loves most when making decisions.
He will share his own choice, made five months after she was born: He
will sit out the 2020 N.F.L. season.

Picking family over football during the coronavirus pandemic, Goodwin
was one of 68 players who the N.F.L. has listed as having opted out by
Thursday's deadline even as the league, despite surging transmission
rates around the country, contends that the season will begin, as
scheduled, on Sept. 10.

The players who opted out represent a microcosm of N.F.L. rosters:
rookies and veterans, practice-squadders and starters, all of whom
determined after careful consideration to lessen one risk while
absorbing another. In order to keep themselves and their families safer,
they will sacrifice the chance to compete for a Super Bowl, forgo
showcasing themselves for more lucrative contracts and, in some cases,
cede starting jobs and roster spots that may or may not be there next
season.

As part of an
\href{https://www.nytimes.com/2020/07/24/sports/football/nfl-players-regular-season-start.html}{agreement
between the N.F.L. and the Players Association,} players with one of the
15 medical conditions that the league regards as high risk for
contracting the virus could earn \$350,000 this year, while players who
decided not to play will receive a \$150,000 advance toward next year's
salary.

Half of the players who opted out are offensive and defensive linemen,
who are in closest contact with other players during practices and
games. Leo Koloamatangi, an offensive lineman on the Jets who opted out,
said he was resigned to contracting the virus had he chosen to play.

``Where I play, I'm literally bear-hugging another creature on the other
side of the ball every single play,'' Koloamatangi, 26, said in an
interview. ``If that guy has any symptoms, I'm going to get them.'' He
added, ``For myself, I couldn't take those chances.''

Neither could Goodwin, 29,
\href{https://www.espn.com/nfl/story/_/id/29301718/philadelphia-eagles-receiver-marquise-goodwin-wife-morgan-persevere-pain-promise-family}{whose
family bore a string of personal tragedies}. He would not permit himself
to perhaps cause another. His wife, Morgan, twice endured pregnancy
complications, losing a prematurely born son in November 2017, and then,
in November 2018, twin boys.

The first time, Goodwin elected to play the same day, Nov. 12, for the
San Francisco 49ers, and after catching an 83-yard touchdown pass he
blew a kiss to the heavens. He was with the team for a game in Tampa,
Fla., the next year when Morgan woke up with contractions. She suggested
he come home but never explicitly said she needed him to, knowing how
seriously he took his career. He flew home, skipping the game, to be
with her.

``I told myself at some point that I've got to hold it down for my
family,'' Goodwin said in an interview. ``I can't let work and the check
and the money dictate decisions that I truly want to make.''

Goodwin had been expecting training camp to be pushed back and when it
wasn't he grew stressed about leaving his family, outside Dallas. His
mother, Tamina, takes care of his younger sister, Deja, who has cerebral
palsy, and Morgan's mother looks after a niece and a nephew. If either
Morgan or Marae fell ill there wouldn't be a family member who could
care for them. Then he heard that Kansas City Chiefs guard
\href{https://www.nytimes.com/2018/05/30/sports/laurent-duvernay-tardif-nfl-doctor-chiefs.html}{Laurent
Duvernay-Tardif}, the only medical doctor playing in the N.F.L., was
opting out. Other players followed, and Goodwin felt more at peace with
joining them.

\includegraphics{https://static01.nyt.com/images/2020/08/08/sports/08nfl-optout-2/merlin_175405431_2805b43b-c256-4edb-9224-c6e32bf4aeb4-articleLarge.jpg?quality=75\&auto=webp\&disable=upscale}

``I'm always hesitant to make any serious moves because you never know
how the N.F.L. will treat you, you know?'' Goodwin said. ``I was super
excited because it was the first time in my life that I made a decision
I'm comfortable living with the result of, as far as work is
concerned.''

\hypertarget{the-coronavirus-outbreak}{%
\subsubsection{The Coronavirus
Outbreak}\label{the-coronavirus-outbreak}}

\hypertarget{sports-and-the-virus}{%
\paragraph{Sports and the Virus}\label{sports-and-the-virus}}

Updated Aug. 7, 2020

Here's what's happening as the world of sports slowly comes back to
life:

\begin{itemize}
\item
  \begin{itemize}
  \tightlist
  \item
    Baseball
    \href{https://www.nytimes.com/2020/08/06/sports/baseball/mlb-safety-protocols.html?action=click\&pgtype=Article\&state=default\&region=MAIN_CONTENT_2\&context=storylines_keepup}{tightened
    its virus protocols} again: Players and staff members must wear
    masks in more places and cannot visit ``bars, lounges or malls''
    when they are home.
  \item
    With no live crowd noise as a buffer at baseball games,
    \href{https://www.nytimes.com/2020/08/06/sports/baseball/mlb-swearing.html?action=click\&pgtype=Article\&state=default\&region=MAIN_CONTENT_2\&context=storylines_keepup}{on-field
    sounds are easy to hear} on broadcasts --- and it's not all rated
    PG.
  \item
    The University of Connecticut
    \href{https://www.nytimes.com/2020/08/05/sports/ncaafootball/coronavirus-uconn-cancels-football.html?action=click\&pgtype=Article\&state=default\&region=MAIN_CONTENT_2\&context=storylines_keepup}{canceled
    its football season}, and Divisions II and III scrapped all of their
    fall championships.
  \end{itemize}
\end{itemize}

For Koloamatangi, choosing to opt out was easy in one sense: He wanted
to protect his 9-month-old daughter, Aurora, and his stepfather, Sele,
who has heart problems, from the virus that killed two close relatives
and infected another.

But other, more complex components factored into what he called ``the
hardest decision of my life.'' Koloamatangi has spent his three N.F.L.
seasons bouncing between the practice squad and active rosters of the
Detroit Lions and Jets but has yet to appear in a regular season game.
He and his agent assessed the professional impact and Koloamatangi
deliberated with his wife, Athena, over the financial burden the family
would assume if he didn't play. By taking the opt out, he would, in
effect, be making about a fraction of his \$750,000 salary --- ``an
uncomfortable difference.''

``I had to take my losses and look my wife in the eyes,'' Koloamatangi
said. ``I did it to ensure the safety in my home.''

Koloamatangi and his family have been sheltering in place in California
since March. The rising infection rates in New Jersey, where the Jets
train and play, prompted the team to
\href{https://www.nytimes.com/2020/07/20/sports/football/jets-giants-rutgers-fans-metlife-stadium.html}{announce
that it would play} regular season home games without fans in MetLife
Stadium or at training camp.

Koloamatangi said he had spoken with a union rep every day since
mid-March, lobbing questions that he wanted answered. He knew it was
unfeasible for the N.F.L. to enter a so-called ``bubble,'' as the
\href{https://www.nytimes.com/2020/07/22/sports/basketball/nba-bubble-practice.html}{N.B.A.}
and
\href{https://www.nytimes.com/2020/07/06/sports/hockey/nhl-playoffs.html}{N.H.L.}
have. But as he and Athena debated their options, he wondered why the
N.F.L. refused to push back camp and the season, or introduce additional
safety measures --- such as gloves or helmets with masks --- that would
further mitigate his risk of infection. As it stands, the N.F.L.'s
testing protocol calls for players to be tested every day for the first
two weeks of training camp, and then every other day after that.

Ultimately, Koloamatangi said, he didn't feel confident enough to risk
the travel and contact that come with playing the game he loves.

``I'm happy my workplace will be safe, but what about when I have to go
out and perform my job?'' he said. ``What are you doing to ensure that
when I make full contact with the guy next to me, I'm not going to
contract the virus? Imagine going through an entire summer understanding
that you're going to have to go to work at some point, but your job
doesn't say anything about your work conditions until two weeks ago.''

Kyle Peko, a defensive tackle on the Denver Broncos, reached a similar
conclusion. Peko, 27, has moderate to severe asthma, among the medical
conditions the league regards as high risk. He has two young children
and a wife, Giuliana, who he said has been cancer-free for seven months.

Image

Kyle Peko, a defensive tackle with the Denver Broncos, had his bags
packed and his truck gassed up for the 15-hour drive to training camp
when he got an email from the union that detailed his
options.Credit...Christian Monterrosa for The New York Times

Their every discussion on opting out focused on the same question: How
could he play football without putting himself at risk?

Undrafted out of Oregon State in 2016, Peko has lived on the margins the
last four seasons, playing in 13 games. Normally he treasures this time
--- rejoining teammates, preparing for camp, battling for a spot. On
July 26, two days before he was to report to camp, Peko had his bags
packed and his truck gassed up for the 15-hour drive from their home in
La Habra, Calif.

That day, he said, he received an email from the union detailing his
options, and when he realized he could keep his family safe without
losing his job, he did not hesitate. He spoke to Broncos officials and
coaches, all of whom, he said, respected his choice.

``Trying to go back and play football during this pandemic,'' Peko said
in an interview, ``it was just hard to wrap my head around putting my
family at risk when I could do my part in trying to put this pandemic to
rest.''

Instead, Peko will be taking the 12 credits he needs to complete his
college degree. Koloamatangi said he can concentrate on his two ventures
in Hawaii --- \href{https://www.hawaiitowardszero.org/}{a nonprofit}
that provides resources for people and business affected by the
pandemic, and a grocery delivery platform for older adults. Goodwin,
meantime, can't wait to bond more with Marae.

Sometimes, safe at home, when she sneezes or coughs, he startles. Then
he remembers what he did, and he thinks to himself, ``Dang, I'm glad I
didn't put her in that situation.''

Advertisement

\protect\hyperlink{after-bottom}{Continue reading the main story}

\hypertarget{site-index}{%
\subsection{Site Index}\label{site-index}}

\hypertarget{site-information-navigation}{%
\subsection{Site Information
Navigation}\label{site-information-navigation}}

\begin{itemize}
\tightlist
\item
  \href{https://help.nytimes.com/hc/en-us/articles/115014792127-Copyright-notice}{©~2020~The
  New York Times Company}
\end{itemize}

\begin{itemize}
\tightlist
\item
  \href{https://www.nytco.com/}{NYTCo}
\item
  \href{https://help.nytimes.com/hc/en-us/articles/115015385887-Contact-Us}{Contact
  Us}
\item
  \href{https://www.nytco.com/careers/}{Work with us}
\item
  \href{https://nytmediakit.com/}{Advertise}
\item
  \href{http://www.tbrandstudio.com/}{T Brand Studio}
\item
  \href{https://www.nytimes.com/privacy/cookie-policy\#how-do-i-manage-trackers}{Your
  Ad Choices}
\item
  \href{https://www.nytimes.com/privacy}{Privacy}
\item
  \href{https://help.nytimes.com/hc/en-us/articles/115014893428-Terms-of-service}{Terms
  of Service}
\item
  \href{https://help.nytimes.com/hc/en-us/articles/115014893968-Terms-of-sale}{Terms
  of Sale}
\item
  \href{https://spiderbites.nytimes.com}{Site Map}
\item
  \href{https://help.nytimes.com/hc/en-us}{Help}
\item
  \href{https://www.nytimes.com/subscription?campaignId=37WXW}{Subscriptions}
\end{itemize}
