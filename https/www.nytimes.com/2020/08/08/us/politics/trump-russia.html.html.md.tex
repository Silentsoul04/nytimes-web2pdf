Sections

SEARCH

\protect\hyperlink{site-content}{Skip to
content}\protect\hyperlink{site-index}{Skip to site index}

\href{https://www.nytimes.com/section/politics}{Politics}

\href{https://myaccount.nytimes.com/auth/login?response_type=cookie\&client_id=vi}{}

\href{https://www.nytimes.com/section/todayspaper}{Today's Paper}

\href{/section/politics}{Politics}\textbar{}Trump, Russia and an
Intelligence Document: Key Moments

\href{https://nyti.ms/3a7SxbL}{https://nyti.ms/3a7SxbL}

\begin{itemize}
\item
\item
\item
\item
\item
\end{itemize}

\begin{itemize}
\item
  \href{https://www.nytimes.com/2020/08/07/us/elections/biden-vs-trump.html?action=click\&pgtype=Article\&state=default\&region=TOP_BANNER\&context=storylines_menu}{Election
  Updates}
\item
  \href{https://www.nytimes.com/interactive/2020/08/08/us/elections/results-hawaii-primary-elections.html?action=click\&pgtype=Article\&state=default\&region=TOP_BANNER\&context=storylines_menu}{Hawaii
  Results}
\item
  \href{https://www.nytimes.com/article/biden-vice-president-2020.html?action=click\&pgtype=Article\&state=default\&region=TOP_BANNER\&context=storylines_menu}{Biden's
  V.P. Search}
\item
  \href{https://www.nytimes.com/interactive/2019/us/politics/2020-presidential-candidates.html?action=click\&pgtype=Article\&state=default\&region=TOP_BANNER\&context=storylines_menu}{The
  Candidates}
\item
  \href{https://www.nytimes.com/newsletters/politics?action=click\&pgtype=Article\&state=default\&region=TOP_BANNER\&context=storylines_menu}{Politics
  Newsletter}
\end{itemize}

Advertisement

\protect\hyperlink{after-top}{Continue reading the main story}

Supported by

\protect\hyperlink{after-sponsor}{Continue reading the main story}

\hypertarget{trump-russia-and-an-intelligence-document-key-moments}{%
\section{Trump, Russia and an Intelligence Document: Key
Moments}\label{trump-russia-and-an-intelligence-document-key-moments}}

Officials told The New York Times Magazine that the draft of a
classified document reporting that Russia favored President Trump in the
2020 election was changed to soften its assessment.

\includegraphics{https://static01.nyt.com/images/2020/08/07/multimedia/07xp-magazine-takeaways-pix1/merlin_117034598_b8cece40-4b8c-4de9-be5c-fb1f3de18ff6-articleLarge.jpg?quality=75\&auto=webp\&disable=upscale}

By Alan Yuhas

\begin{itemize}
\item
  Aug. 8, 2020Updated 8:11 a.m. ET
\item
  \begin{itemize}
  \item
  \item
  \item
  \item
  \item
  \end{itemize}
\end{itemize}

A little more than a year ago, American intelligence agencies drafted a
classified document reporting that the Russian government favored
President Trump in the 2020 presidential election, a finding that fit
with their consensus that the Kremlin tried to help him in 2016.

The director of national intelligence was asked to modify the assessment
--- he did not --- and not long afterward, Mr. Trump declared the
director was out.

Soon after the new acting director arrived, an intelligence official
changed the document, softening the claim that President Vladimir V.
Putin of Russia wanted Mr. Trump to win,
\href{https://www.nytimes.com/2020/08/08/magazine/us-russia-intelligence.html?}{according
to an article published on Saturday} by The New York Times Magazine. The
investigation includes details not previously reported about the fears
of officials in U.S. intelligence agencies under the Trump
administration, who described struggling to brief the president without
provoking his anger or losing their jobs.

\href{https://www.nytimes.com/2020/08/08/magazine/us-russia-intelligence.html}{Read
the entire article from The New York Times Magazine here.}

Following are some key takeaways, based on the reporter Robert Draper's
conversations with 40 current and former intelligence officials,
lawmakers and congressional staff.

\hypertarget{russia-favored-trump-and-helped-sanders-draft-says}{%
\subsection{Russia favored Trump and helped Sanders, draft
says}\label{russia-favored-trump-and-helped-sanders-draft-says}}

The early draft of the classified document assembled last year, a
National Intelligence Estimate, touched on a chronic sore point between
intelligence officials and the White House.

Among other things, the draft concerned Russia's efforts to influence
American elections in 2020 and 2024, according to multiple officials who
saw it.

A ``key judgment'' of the document was that in the 2020 election, Russia
favored the current president. To allay any speculation that Mr. Putin's
interest in Mr. Trump had cooled, the judgment was supported by
information from a highly sensitive foreign source described as ``100
percent reliable'' by someone who read the draft.

The intelligence used by the analysts also indicated that Russia had
worked in support of Senator Bernie Sanders, then running for the
Democratic nomination for president. A veteran national intelligence
officer explained to his colleagues, according to notes taken by one
participant in the process, that this did not reflect a genuine
preference for Mr. Sanders, but instead an effort ``to weaken that party
and ultimately help the current U.S. president.''

\hypertarget{an-intelligence-chief-is-out-and-the-draft-is-revised}{%
\subsection{An intelligence chief is out, and the draft is
revised}\label{an-intelligence-chief-is-out-and-the-draft-is-revised}}

\includegraphics{https://static01.nyt.com/images/2020/08/07/multimedia/07xp-magazine-takeaways-pix2/merlin_142672992_5a9f68a7-18ba-4bc4-a34d-e0610ea19848-articleLarge.jpg?quality=75\&auto=webp\&disable=upscale}

Later, a suggestion was made to the director of national intelligence at
the time, Dan Coats, that the draft be modified. Coats, who recalled the
request coming from a staff member, refused. On July 28, Mr. Trump
announced that Mr. Coats's last day in office would be Aug. 15, over a
month before he had expected to resign.

In September, a new version of the document was circulated with edits.
It no longer clearly said that Russia favored the current president.
Instead, in a summary, it said, ``Russian leaders probably assess that
chances to improve relations with the U.S. will diminish under a
different U.S. president.''

The changes reflected what Mr. Draper calls ``a sobering development of
the Trump era'' that has alarmed some current and former officials,
lawmakers and congressional staff members: ``the intelligence
community's willingness to change what it would otherwise say
straightforwardly so as not to upset the president.''

By firing top officials and replacing them with loyalists, said
Representative Adam Schiff, the Democratic chairman of the House
Intelligence Committee, ``it's had the effect of wearing the
intelligence community down, making them less willing to speak truth to
power.''

\hypertarget{a-second-intelligence-director-is-shown-the-door}{%
\subsection{A second intelligence director is shown the
door}\label{a-second-intelligence-director-is-shown-the-door}}

Image

Joseph Maguire, who replaced Mr. Coats as the director of national
intelligence, was forced out early this year.Credit...Erin Schaff/The
New York Times

On Feb. 13, Shelby Pierson, an analyst for the Office of the Director of
National Intelligence, testified in a classified hearing to the House
Intelligence Committee that Russia preferred the current president to
win in the 2020 election.

A number of Republicans objected, and Ms. Pierson's testimony was
relayed to Mr. Trump. The next day, on Feb. 14, he interrupted a routine
briefing on election security, according to one of the meeting's
participants. He asked the director of national intelligence at the
time, retired Vice Adm. Joseph Maguire: ``Hey, Joe, I understand that
you briefed Adam Schiff and that you told him that Russia prefers me.
Why did you tell that to Schiff?''

Although Mr. Maguire tried to explain that it was another official, Mr.
Trump continued to question him and the meeting broke up. On Feb. 19,
Mr. Maguire was informed that his likely replacement should be let into
his office's headquarters the following morning.

Mr. Trump named his replacement as Richard Grenell, the ambassador to
Germany and a former United Nations ambassador's spokesman, media
consultant and Fox News commentator.

\hypertarget{anger-and-anxiety-from-day-1}{%
\subsection{Anger and anxiety from Day
1}\label{anger-and-anxiety-from-day-1}}

Mr. Trump's speech on the first day of his presidency, in front of the
C.I.A.'s Memorial Wall, a tribute to agency officers killed in service,
drew intense anger for some in the agency. At the event, Mr. Trump
repeated false claims about the size of the crowd at his inauguration,
attacked the news media and asked why the lobby of C.I.A. headquarters
had so many columns.

One agency veteran called the speech ``a near desecration of the wall.''

The president's penchant for bargaining and gossiping on his private
cellphone, and for inviting billionaires into his circle, created
anxiety in the intelligence agencies. Intelligence officials of at least
one country, a NATO ally, were discouraged by their president from
interacting with American counterparts for fear that Mr. Trump would
blurt out information to Russians, one former senior intelligence
official said.

Mr. Trump also stocked the President's Intelligence Advisory Board with
wealthy businesspeople who, when briefed, ``would sometimes make you
uncomfortable,'' because at times ``their questions were related to
their business dealings,'' one intelligence official said.

Under Mr. Grenell, fears grew that, under the pretext of downsizing, the
services might be purged of people like the C.I.A. analyst who
\href{https://www.nytimes.com/2019/11/26/us/politics/trump-whistle-blower-complaint-ukraine.html}{filed
the Ukraine whistle-blower complaint} last year.

``It seems pretty clear to me that, in the wake of the whistle-blower
complaint, he'd put a bunch of political hacks in charge, so that he'd
never have to worry about the truth getting out from the intelligence
community,'' said Representative Sean Patrick Maloney, a Democrat on the
House Intelligence Committee.

\hypertarget{our-2020-election-guide}{%
\section{Our 2020 Election Guide}\label{our-2020-election-guide}}

Updated Aug. 8, 2020

\begin{itemize}
\item
  \begin{center}\rule{0.5\linewidth}{\linethickness}\end{center}

  \hypertarget{the-latest}{%
  \subsection{The Latest}\label{the-latest}}

  \begin{itemize}
  \tightlist
  \item
    With 160 lawsuits filed over voting rules and President Trump's
    baseless claims of fraud, Election Day in America
    \href{https://www.nytimes.com/2020/08/08/us/politics/voting-nov-3-election.html?action=click\&pgtype=Article\&state=default\&region=BELOW_MAIN_CONTENT\&context=storylines_guide}{could
    become Election Month}.
  \end{itemize}
\item
  \begin{center}\rule{0.5\linewidth}{\linethickness}\end{center}

  \hypertarget{bidens-vp-search}{%
  \subsection{Biden's V.P. Search}\label{bidens-vp-search}}

  \begin{itemize}
  \tightlist
  \item
    \href{https://www.nytimes.com/article/biden-vice-president-2020.html?action=click\&pgtype=Article\&state=default\&region=BELOW_MAIN_CONTENT\&context=storylines_guide}{Here
    are 13 women} who have been under consideration to be Joe Biden's
    running mate, and why each might be chosen --- and might not be.
  \end{itemize}
\item
  \begin{center}\rule{0.5\linewidth}{\linethickness}\end{center}

  \hypertarget{keep-up-with-our-coverage}{%
  \subsection{Keep Up With Our
  Coverage}\label{keep-up-with-our-coverage}}

  \begin{itemize}
  \tightlist
  \item
    Get an
    \href{https://www.nytimes.com/newsletters/politics?action=click\&pgtype=Article\&state=default\&region=BELOW_MAIN_CONTENT\&context=storylines_guide}{email}
    recapping the day's news
  \end{itemize}

  \begin{itemize}
  \tightlist
  \item
    Download our mobile app on
    \href{https://apps.apple.com/us/app/nytimes/id284862083?ls=1\&mat_click_id=5c79ae7455014fd1bd66b5610c05b8f2-20191112-16948\&referrer=mat_click_id\%3D5c79ae7455014fd1bd66b5610c05b8f2-20191112-16948\%26link_click_id\%3D722930677036718082}{iOS}
    and
    \href{http://a.localytics.com/android?id=com.nytimes.android\&referrer=utm_source\%3Dother_nyt_mobile_web\%26utm_medium\%3DWeb\%2520page\%26utm_term\%3DGeneral\%2520Mobile\%2520Page\%26utm_campaign\%3DNYT\%2520Mobile\%2520General\%2520Page}{Android}
    and turn on Breaking News and Politics alerts
  \end{itemize}
\end{itemize}

Advertisement

\protect\hyperlink{after-bottom}{Continue reading the main story}

\hypertarget{site-index}{%
\subsection{Site Index}\label{site-index}}

\hypertarget{site-information-navigation}{%
\subsection{Site Information
Navigation}\label{site-information-navigation}}

\begin{itemize}
\tightlist
\item
  \href{https://help.nytimes.com/hc/en-us/articles/115014792127-Copyright-notice}{©~2020~The
  New York Times Company}
\end{itemize}

\begin{itemize}
\tightlist
\item
  \href{https://www.nytco.com/}{NYTCo}
\item
  \href{https://help.nytimes.com/hc/en-us/articles/115015385887-Contact-Us}{Contact
  Us}
\item
  \href{https://www.nytco.com/careers/}{Work with us}
\item
  \href{https://nytmediakit.com/}{Advertise}
\item
  \href{http://www.tbrandstudio.com/}{T Brand Studio}
\item
  \href{https://www.nytimes.com/privacy/cookie-policy\#how-do-i-manage-trackers}{Your
  Ad Choices}
\item
  \href{https://www.nytimes.com/privacy}{Privacy}
\item
  \href{https://help.nytimes.com/hc/en-us/articles/115014893428-Terms-of-service}{Terms
  of Service}
\item
  \href{https://help.nytimes.com/hc/en-us/articles/115014893968-Terms-of-sale}{Terms
  of Sale}
\item
  \href{https://spiderbites.nytimes.com}{Site Map}
\item
  \href{https://help.nytimes.com/hc/en-us}{Help}
\item
  \href{https://www.nytimes.com/subscription?campaignId=37WXW}{Subscriptions}
\end{itemize}
