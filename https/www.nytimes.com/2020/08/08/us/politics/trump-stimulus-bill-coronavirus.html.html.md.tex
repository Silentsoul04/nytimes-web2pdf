Sections

SEARCH

\protect\hyperlink{site-content}{Skip to
content}\protect\hyperlink{site-index}{Skip to site index}

\href{https://www.nytimes.com/section/politics}{Politics}

\href{https://myaccount.nytimes.com/auth/login?response_type=cookie\&client_id=vi}{}

\href{https://www.nytimes.com/section/todayspaper}{Today's Paper}

\href{/section/politics}{Politics}\textbar{}Sidestepping Congress, Trump
Signs Executive Measures for Pandemic Relief

\href{https://nyti.ms/3aakK1M}{https://nyti.ms/3aakK1M}

\begin{itemize}
\item
\item
\item
\item
\item
\end{itemize}

\href{https://www.nytimes.com/news-event/coronavirus?action=click\&pgtype=Article\&state=default\&region=TOP_BANNER\&context=storylines_menu}{The
Coronavirus Outbreak}

\begin{itemize}
\tightlist
\item
  live\href{https://www.nytimes.com/2020/08/08/world/coronavirus-updates.html?action=click\&pgtype=Article\&state=default\&region=TOP_BANNER\&context=storylines_menu}{Latest
  Updates}
\item
  \href{https://www.nytimes.com/interactive/2020/us/coronavirus-us-cases.html?action=click\&pgtype=Article\&state=default\&region=TOP_BANNER\&context=storylines_menu}{Maps
  and Cases}
\item
  \href{https://www.nytimes.com/interactive/2020/science/coronavirus-vaccine-tracker.html?action=click\&pgtype=Article\&state=default\&region=TOP_BANNER\&context=storylines_menu}{Vaccine
  Tracker}
\item
  \href{https://www.nytimes.com/interactive/2020/world/coronavirus-tips-advice.html?action=click\&pgtype=Article\&state=default\&region=TOP_BANNER\&context=storylines_menu}{F.A.Q.}
\item
  \href{https://www.nytimes.com/live/2020/08/07/business/stock-market-today-coronavirus?action=click\&pgtype=Article\&state=default\&region=TOP_BANNER\&context=storylines_menu}{Markets
  \& Economy}
\end{itemize}

Advertisement

\protect\hyperlink{after-top}{Continue reading the main story}

Supported by

\protect\hyperlink{after-sponsor}{Continue reading the main story}

\hypertarget{sidestepping-congress-trump-signs-executive-measures-for-pandemic-relief}{%
\section{Sidestepping Congress, Trump Signs Executive Measures for
Pandemic
Relief}\label{sidestepping-congress-trump-signs-executive-measures-for-pandemic-relief}}

President Trump signed the directives after two weeks of stalemate with
Democrats over a recovery package, using legally dubious measures to try
to restore lapsed benefits.

\includegraphics{https://static01.nyt.com/images/2020/08/08/business/08virus-briefing-trumplede/08virus-briefing-trumplede-videoSixteenByNine3000.jpg}

\href{https://www.nytimes.com/by/maggie-haberman}{\includegraphics{https://static01.nyt.com/images/2018/07/12/multimedia/author-maggie-haberman/author-maggie-haberman-thumbLarge.png}}\href{https://www.nytimes.com/by/emily-cochrane}{\includegraphics{https://static01.nyt.com/images/2018/11/28/multimedia/author-emily-cochrane/author-emily-cochrane-thumbLarge-v3.png}}\href{https://www.nytimes.com/by/jim-tankersley}{\includegraphics{https://static01.nyt.com/images/2018/10/19/multimedia/author-jim-tankersley/author-jim-tankersley-thumbLarge.png}}

By \href{https://www.nytimes.com/by/maggie-haberman}{Maggie Haberman},
\href{https://www.nytimes.com/by/emily-cochrane}{Emily Cochrane} and
\href{https://www.nytimes.com/by/jim-tankersley}{Jim Tankersley}

\begin{itemize}
\item
  Aug. 8, 2020Updated 8:25 p.m. ET
\item
  \begin{itemize}
  \item
  \item
  \item
  \item
  \item
  \end{itemize}
\end{itemize}

President Trump took executive action on Saturday to circumvent Congress
and try to extend an array of federal pandemic relief, resorting to a
legally dubious set of edicts whose impact was unclear, as negotiations
over
\href{https://www.nytimes.com/2020/08/08/world/coronavirus-updates.html}{an
economic recovery package} appeared on the brink of collapse.

It was not clear what authority Mr. Trump had to act on his own on the
measures or what immediate effect, if any, they would have, given that
Congress controls federal spending. But his decision to sign the
measures --- billed as a federal eviction ban, a payroll tax suspension,
and relief for student borrowers and the unemployed --- reflected the
failure of two weeks of talks between White House officials and top
congressional Democrats to strike a deal on a broad relief plan as
crucial benefits have expired with no resolution in sight.

Mr. Trump's move also illustrated the heightened concern of a president
staring down re-election in the middle of a historic recession and a
pandemic, and determined to show voters that he was doing something to
address the crises. But despite Mr. Trump's assertions on Saturday that
his actions ``will take care of this entire situation,'' the orders also
leave a number of critical bipartisan funding proposals unaddressed,
including providing assistance to small businesses, billions of dollars
to schools ahead of the new school year, aid to states and cities and a
second round of \$1,200 stimulus checks to Americans.

``Nancy Pelosi and Chuck Schumer have chosen to hold this vital
assistance hostage,'' Mr. Trump said, savaging the two top Democrats
during a news conference at his private golf club in New Jersey, his
second in two days. A few dozen club guests were in attendance, and the
president appeared to revel in their laughter at his jokes denouncing
his political rivals.

``We've had it,'' he added, repeatedly referring to his directives as
``bills,'' a term reserved for legislation passed by Congress. He
accused the Democrats of holding up negotiations with demands for
provisions that appeared to had little to do with the pandemic, though
he made little mention of similar, seemingly unrelated items
---~\href{https://www.nytimes.com/2020/07/28/us/politics/republicans-trump-fbi-building-virus-relief-bill.html}{including
money for a new building for the F.B.I.} --- in the \$1 trillion
proposal Republicans
\href{https://www.nytimes.com/2020/07/27/us/politics/republicans-jobless-aid.html}{unveiled
last month}.

Democrats have refused to agree to that plan, pressing instead for a far
more expansive economic relief package, at least twice as large, that
would extend \$600-per-week enhanced federal jobless aid payments the
Republicans are seeking to reduce, if revived, and provide billions more
for schools, states and cities and food aid.

It was unclear whether the effort to bypass Congress would kill the
already-stalled negotiations altogether, though Mr. Trump told reporters
he would be open to continuing the discussions and Democratic leaders
responded by demanding that the talks resume.

\hypertarget{latest-updates-the-coronavirus-outbreak}{%
\section{\texorpdfstring{\href{https://www.nytimes.com/2020/08/07/world/covid-19-news.html?action=click\&pgtype=Article\&state=default\&region=MAIN_CONTENT_1\&context=storylines_live_updates}{Latest
Updates: The Coronavirus
Outbreak}}{Latest Updates: The Coronavirus Outbreak}}\label{latest-updates-the-coronavirus-outbreak}}

Updated 2020-08-08T12:04:28.992Z

\begin{itemize}
\tightlist
\item
  \href{https://www.nytimes.com/2020/08/07/world/covid-19-news.html?action=click\&pgtype=Article\&state=default\&region=MAIN_CONTENT_1\&context=storylines_live_updates\#link-1f86d03a}{As
  the U.S. relief talks falter again, Trump says he is prepared to act
  on his own.}
\item
  \href{https://www.nytimes.com/2020/08/07/world/covid-19-news.html?action=click\&pgtype=Article\&state=default\&region=MAIN_CONTENT_1\&context=storylines_live_updates\#link-3f64a70a}{Cuomo
  says N.Y. schools can reopen in-person but leaves it up to districts
  to determine if, when and how.}
\item
  \href{https://www.nytimes.com/2020/08/07/world/covid-19-news.html?action=click\&pgtype=Article\&state=default\&region=MAIN_CONTENT_1\&context=storylines_live_updates\#link-14e70066}{Thousands
  of cases went unreported in California when a computer server failed.}
\end{itemize}

\href{https://www.nytimes.com/2020/08/07/world/covid-19-news.html?action=click\&pgtype=Article\&state=default\&region=MAIN_CONTENT_1\&context=storylines_live_updates}{See
more updates}

More live coverage:
\href{https://www.nytimes.com/live/2020/08/07/business/stock-market-today-coronavirus?action=click\&pgtype=Article\&state=default\&region=MAIN_CONTENT_1\&context=storylines_live_updates}{Markets}

``We're disappointed that instead of putting in the work to solve
Americans' problems, the president instead chose to stay on his luxury
golf course to announce unworkable, weak and narrow policy announcements
to slash the unemployment benefits that millions desperately need and
endanger seniors' Social Security and Medicare,'' Speaker Nancy Pelosi
and Senator Chuck Schumer, Democrat of New York and the minority leader,
said in a statement. They called on Republicans to ``return to the
table'' to continue negotiating.

It was unclear whether the aid would even materialize if lawsuits are
filed challenging their legality. Mr. Trump walked away from the lectern
after just a few questions from reporters about his claim that he had
the ability to circumvent Congress.

For Mr. Trump, signing the orders was a familiar tactic from a president
who has portrayed himself as the ultimate deal-maker, but in practice
has shown little interest in or skill for negotiating with Congress,
bristling against the limitations of his power. It recalled his decision
in 2018 to
\href{https://www.nytimes.com/2018/12/21/us/politics/trump-shutdown-border-wall.html}{shut
down the government} over his demand for funding for a wall on the
southwestern border, his signature campaign promise, in an effort to
force Democrats to agree to the money. They never did, and the president
ultimately declared a national emergency to divert other federal money
to fund it himself, a move that drew legal challenges.

Shortly after the event on Saturday, the White House released texts of
the measures --- one executive order and three memorandums --- which
included several flourishes that read like political documents. One
invoked the Stafford Act, a federal disaster relief statute, to divert
money from a homeland security fund and allow states to use money
already allocated by Congress to help people who have been laid off amid
the coronavirus pandemic, effectively allowing them to apply for
disaster relief to cover lost wages. The mechanism would pull from the
same fund that covers natural disasters in the middle of what is
expected to be a highly active hurricane season.

Mr. Trump claimed that the action would provide \$400 weekly in enhanced
unemployment benefits, \$200 less than laid-off workers had been
receiving under benefits that lapsed at the end of July. But with states
being directed to pick up \$100 of that aid, the federal amount would be
no more than \$300 a week.

It was unclear how quickly states, whose unemployment systems had
already been overburdened by the record numbers of new jobless claims,
would be able to adjust to a new system, or whether they will have the
resources to supplement an additional benefit.

``If they don't, they don't --- that's going to be their problem,'' Mr.
Trump said.

He also retroactively signed a memorandum suspending the payroll tax
from Aug. 1 through the end of 2020, though the order would just defer
the payment of the taxes without congressional action. (Mr. Trump vowed
that if re-elected in November, he would extend the deferral and the
payments.)

If Mr. Trump tried to make a payroll tax cut permanent, it would have
drastic effect on the funding of Social Security, which he has
previously vowed not to cut.

The memorandum that Mr. Trump called a moratorium on evictions did not
revive the expired moratorium that was part of the \$2.2 trillion
stimulus law. Instead, it said that federal policy was to minimize
evictions during the pandemic and that officials should identify
statutory ways to help homeowners and renters.

Long before taking office, Mr. Trump
\href{https://twitter.com/realdonaldtrump/status/222739756105207808}{criticized
Barack Obama} for what he described as an overreliance on executive
orders to accomplish policy goals that had been blocked by Congress, but
in acting unilaterally, Mr. Trump was vastly expanding the use of such
measures.

\href{https://www.nytimes.com/news-event/coronavirus?action=click\&pgtype=Article\&state=default\&region=MAIN_CONTENT_3\&context=storylines_faq}{}

\hypertarget{the-coronavirus-outbreak-}{%
\subsubsection{The Coronavirus Outbreak
›}\label{the-coronavirus-outbreak-}}

\hypertarget{frequently-asked-questions}{%
\paragraph{Frequently Asked
Questions}\label{frequently-asked-questions}}

Updated August 6, 2020

\begin{itemize}
\item ~
  \hypertarget{why-are-bars-linked-to-outbreaks}{%
  \paragraph{Why are bars linked to
  outbreaks?}\label{why-are-bars-linked-to-outbreaks}}

  \begin{itemize}
  \tightlist
  \item
    Think about a bar. Alcohol is flowing. It can be loud, but it's
    definitely intimate, and you often need to lean in close to hear
    your friend. And strangers have way, way fewer reservations about
    coming up to people in a bar. That's sort of the point of a bar.
    Feeling good and close to strangers. It's no surprise, then, that
    \href{https://www.nytimes.com/2020/07/02/us/coronavirus-bars.html?action=click\&pgtype=Article\&state=default\&region=MAIN_CONTENT_3\&context=storylines_faq}{bars
    have been linked to outbreaks in several states.} Louisiana health
    officials have tied
    \href{https://www.nytimes.com/2020/06/22/us/new-coronavirus-phase.html?action=click\&pgtype=Article\&state=default\&region=MAIN_CONTENT_3\&context=storylines_faq}{at
    least 100 coronavirus cases} to bars in the Tigerland nightlife
    district in Baton Rouge. Minnesota has traced 328 recent cases to
    bars across the state.
    \href{https://www.boisestatepublicradio.org/post/bars-large-venues-close-ada-county-after-surge-coronavirus-prompts-rollback\#stream/0}{In
    Idaho}, health officials shut down bars in Ada County after
    reporting clusters of infections among young adults who had visited
    several bars in downtown Boise. Governors in
    \href{https://www.nytimes.com/2020/07/01/us/california-coronavirus-reopening.html?action=click\&pgtype=Article\&state=default\&region=MAIN_CONTENT_3\&context=storylines_faq}{California},
    \href{https://www.nytimes.com/2020/06/14/us/coronavirus-united-states.html?action=click\&pgtype=Article\&state=default\&region=MAIN_CONTENT_3\&context=storylines_faq}{Texas
    and Arizona}, where coronavirus cases are soaring, have ordered
    hundreds of newly reopened bars to shut down. Less than two weeks
    after Colorado's bars reopened at limited capacity, Gov. Jared Polis
    \href{https://www.denverpost.com/2020/06/30/colorado-bars-closed-coronavirus/}{ordered
    them to close}.
  \end{itemize}
\item ~
  \hypertarget{i-have-antibodies-am-i-now-immune}{%
  \paragraph{I have antibodies. Am I now
  immune?}\label{i-have-antibodies-am-i-now-immune}}

  \begin{itemize}
  \tightlist
  \item
    As of right now,
    \href{https://www.nytimes.com/2020/07/22/health/covid-antibodies-herd-immunity.html?action=click\&pgtype=Article\&state=default\&region=MAIN_CONTENT_3\&context=storylines_faq}{that
    seems likely, for at least several months.} There have been
    frightening accounts of people suffering what seems to be a second
    bout of Covid-19. But experts say these patients may have a
    drawn-out course of infection, with the virus taking a slow toll
    weeks to months after initial exposure. People infected with the
    coronavirus typically
    \href{https://www.nature.com/articles/s41586-020-2456-9}{produce}
    immune molecules called antibodies, which are
    \href{https://www.nytimes.com/2020/05/07/health/coronavirus-antibody-prevalence.html?action=click\&pgtype=Article\&state=default\&region=MAIN_CONTENT_3\&context=storylines_faq}{protective
    proteins made in response to an
    infection}\href{https://www.nytimes.com/2020/05/07/health/coronavirus-antibody-prevalence.html?action=click\&pgtype=Article\&state=default\&region=MAIN_CONTENT_3\&context=storylines_faq}{.
    These antibodies may} last in the body
    \href{https://www.nature.com/articles/s41591-020-0965-6}{only two to
    three months}, which may seem worrisome, but that's perfectly normal
    after an acute infection subsides, said Dr. Michael Mina, an
    immunologist at Harvard University. It may be possible to get the
    coronavirus again, but it's highly unlikely that it would be
    possible in a short window of time from initial infection or make
    people sicker the second time.
  \end{itemize}
\item ~
  \hypertarget{im-a-small-business-owner-can-i-get-relief}{%
  \paragraph{I'm a small-business owner. Can I get
  relief?}\label{im-a-small-business-owner-can-i-get-relief}}

  \begin{itemize}
  \tightlist
  \item
    The
    \href{https://www.nytimes.com/article/small-business-loans-stimulus-grants-freelancers-coronavirus.html?action=click\&pgtype=Article\&state=default\&region=MAIN_CONTENT_3\&context=storylines_faq}{stimulus
    bills enacted in March} offer help for the millions of American
    small businesses. Those eligible for aid are businesses and
    nonprofit organizations with fewer than 500 workers, including sole
    proprietorships, independent contractors and freelancers. Some
    larger companies in some industries are also eligible. The help
    being offered, which is being managed by the Small Business
    Administration, includes the Paycheck Protection Program and the
    Economic Injury Disaster Loan program. But lots of folks have
    \href{https://www.nytimes.com/interactive/2020/05/07/business/small-business-loans-coronavirus.html?action=click\&pgtype=Article\&state=default\&region=MAIN_CONTENT_3\&context=storylines_faq}{not
    yet seen payouts.} Even those who have received help are confused:
    The rules are draconian, and some are stuck sitting on
    \href{https://www.nytimes.com/2020/05/02/business/economy/loans-coronavirus-small-business.html?action=click\&pgtype=Article\&state=default\&region=MAIN_CONTENT_3\&context=storylines_faq}{money
    they don't know how to use.} Many small-business owners are getting
    less than they expected or
    \href{https://www.nytimes.com/2020/06/10/business/Small-business-loans-ppp.html?action=click\&pgtype=Article\&state=default\&region=MAIN_CONTENT_3\&context=storylines_faq}{not
    hearing anything at all.}
  \end{itemize}
\item ~
  \hypertarget{what-are-my-rights-if-i-am-worried-about-going-back-to-work}{%
  \paragraph{What are my rights if I am worried about going back to
  work?}\label{what-are-my-rights-if-i-am-worried-about-going-back-to-work}}

  \begin{itemize}
  \tightlist
  \item
    Employers have to provide
    \href{https://www.osha.gov/SLTC/covid-19/standards.html}{a safe
    workplace} with policies that protect everyone equally.
    \href{https://www.nytimes.com/article/coronavirus-money-unemployment.html?action=click\&pgtype=Article\&state=default\&region=MAIN_CONTENT_3\&context=storylines_faq}{And
    if one of your co-workers tests positive for the coronavirus, the
    C.D.C.} has said that
    \href{https://www.cdc.gov/coronavirus/2019-ncov/community/guidance-business-response.html}{employers
    should tell their employees} -\/- without giving you the sick
    employee's name -\/- that they may have been exposed to the virus.
  \end{itemize}
\item ~
  \hypertarget{what-is-school-going-to-look-like-in-september}{%
  \paragraph{What is school going to look like in
  September?}\label{what-is-school-going-to-look-like-in-september}}

  \begin{itemize}
  \tightlist
  \item
    It is unlikely that many schools will return to a normal schedule
    this fall, requiring the grind of
    \href{https://www.nytimes.com/2020/06/05/us/coronavirus-education-lost-learning.html?action=click\&pgtype=Article\&state=default\&region=MAIN_CONTENT_3\&context=storylines_faq}{online
    learning},
    \href{https://www.nytimes.com/2020/05/29/us/coronavirus-child-care-centers.html?action=click\&pgtype=Article\&state=default\&region=MAIN_CONTENT_3\&context=storylines_faq}{makeshift
    child care} and
    \href{https://www.nytimes.com/2020/06/03/business/economy/coronavirus-working-women.html?action=click\&pgtype=Article\&state=default\&region=MAIN_CONTENT_3\&context=storylines_faq}{stunted
    workdays} to continue. California's two largest public school
    districts --- Los Angeles and San Diego --- said on July 13, that
    \href{https://www.nytimes.com/2020/07/13/us/lausd-san-diego-school-reopening.html?action=click\&pgtype=Article\&state=default\&region=MAIN_CONTENT_3\&context=storylines_faq}{instruction
    will be remote-only in the fall}, citing concerns that surging
    coronavirus infections in their areas pose too dire a risk for
    students and teachers. Together, the two districts enroll some
    825,000 students. They are the largest in the country so far to
    abandon plans for even a partial physical return to classrooms when
    they reopen in August. For other districts, the solution won't be an
    all-or-nothing approach.
    \href{https://bioethics.jhu.edu/research-and-outreach/projects/eschool-initiative/school-policy-tracker/}{Many
    systems}, including the nation's largest, New York City, are
    devising
    \href{https://www.nytimes.com/2020/06/26/us/coronavirus-schools-reopen-fall.html?action=click\&pgtype=Article\&state=default\&region=MAIN_CONTENT_3\&context=storylines_faq}{hybrid
    plans} that involve spending some days in classrooms and other days
    online. There's no national policy on this yet, so check with your
    municipal school system regularly to see what is happening in your
    community.
  \end{itemize}
\end{itemize}

Mark Meadows, the White House chief of staff and a vicious critic of Mr.
Obama's actions while a North Carolina congressman, was among those who
recommended that Mr. Trump issue the orders, even as he conceded that an
agreement with lawmakers would be more potent for the American economy.

``This is not a perfect answer --- we'll be the first ones to say
that,'' Mr. Meadows said on Friday, after he and Steven Mnuchin, the
Treasury secretary, emerged from another meeting with congressional
Democrats with no deal. ``But it is all that we can do and all the
president can do within the confines of his executive power and we're
going to encourage him to do it.''

Mr. Trump had told reporters on Friday evening that he would probably
sign executive orders to provide economic relief next week if no
compromise could be reached with Democrats, but by Saturday morning,
officials were already drafting them and planning an afternoon news
conference.

After signing the measures, Mr. Trump handed out the black Sharpies he
had used, embossed with his name, to members of his golf club standing
at the back of the room.

The moves could face legal challenges. And they are unlikely to add much
additional fuel to the economic recovery, which has slowed in the summer
months as
\href{https://www.nytimes.com/interactive/2020/us/coronavirus-us-cases.html}{infections
surged again} in large pockets of the country.

The Labor Department reported on Friday that the economy created 1.8
million jobs in July, a sharp slowdown from May and June, and economic
forecasters expect further slowing in August. Many economists have
warned that supplemental unemployment benefits, which expired at the end
of July, had been propping up consumer spending at a time when about 30
million Americans are unemployed.

Mr. Trump's memorandum seeking to repurpose other money to cover lost
wages is unlikely to deliver additional cash to laid-off workers any
time soon. And while some of Mr. Trump's outside advisers, including the
conservative economists Arthur B. Laffer and Stephen Moore, have urged
him to suspend payroll tax collections, other economists say the move is
unlikely to bolster workers' paychecks because it is only a delay in tax
liability. Many companies are likely to continue withholding the taxes
in order to remit them next year on workers' behalf, they say.

It is an idea that Republican lawmakers have also resisted.

It remains unclear what effect Mr. Trump's actions will have on the
negotiations, either scuttling them altogether or serving as an
accelerant toward ending the impasse, which has lasted for more than two
weeks.

``They don't make that much difference,'' Ms. Pelosi told members of the
Democratic caucus on a private call Saturday afternoon, according to
three people familiar with her remarks. She questioned whether Mr. Trump
genuinely wanted to reach an agreement, telling lawmakers that the
president cared about the markets and having his ``name on the letter
when those direct payments go out.''

Luke Broadwater contributed reporting.

Advertisement

\protect\hyperlink{after-bottom}{Continue reading the main story}

\hypertarget{site-index}{%
\subsection{Site Index}\label{site-index}}

\hypertarget{site-information-navigation}{%
\subsection{Site Information
Navigation}\label{site-information-navigation}}

\begin{itemize}
\tightlist
\item
  \href{https://help.nytimes.com/hc/en-us/articles/115014792127-Copyright-notice}{©~2020~The
  New York Times Company}
\end{itemize}

\begin{itemize}
\tightlist
\item
  \href{https://www.nytco.com/}{NYTCo}
\item
  \href{https://help.nytimes.com/hc/en-us/articles/115015385887-Contact-Us}{Contact
  Us}
\item
  \href{https://www.nytco.com/careers/}{Work with us}
\item
  \href{https://nytmediakit.com/}{Advertise}
\item
  \href{http://www.tbrandstudio.com/}{T Brand Studio}
\item
  \href{https://www.nytimes.com/privacy/cookie-policy\#how-do-i-manage-trackers}{Your
  Ad Choices}
\item
  \href{https://www.nytimes.com/privacy}{Privacy}
\item
  \href{https://help.nytimes.com/hc/en-us/articles/115014893428-Terms-of-service}{Terms
  of Service}
\item
  \href{https://help.nytimes.com/hc/en-us/articles/115014893968-Terms-of-sale}{Terms
  of Sale}
\item
  \href{https://spiderbites.nytimes.com}{Site Map}
\item
  \href{https://help.nytimes.com/hc/en-us}{Help}
\item
  \href{https://www.nytimes.com/subscription?campaignId=37WXW}{Subscriptions}
\end{itemize}
