Sections

SEARCH

\protect\hyperlink{site-content}{Skip to
content}\protect\hyperlink{site-index}{Skip to site index}

\href{https://www.nytimes.com/section/politics}{Politics}

\href{https://myaccount.nytimes.com/auth/login?response_type=cookie\&client_id=vi}{}

\href{https://www.nytimes.com/section/todayspaper}{Today's Paper}

\href{/section/politics}{Politics}\textbar{}Trump Wants U.S. to Get Cut
of Any TikTok Deal. No One Knows How That'd Work.

\href{https://nyti.ms/3a92ztu}{https://nyti.ms/3a92ztu}

\begin{itemize}
\item
\item
\item
\item
\item
\end{itemize}

Advertisement

\protect\hyperlink{after-top}{Continue reading the main story}

Supported by

\protect\hyperlink{after-sponsor}{Continue reading the main story}

White House MEmo

\hypertarget{trump-wants-us-to-get-cut-of-any-tiktok-deal-no-one-knows-how-thatd-work}{%
\section{Trump Wants U.S. to Get Cut of Any TikTok Deal. No One Knows
How That'd
Work.}\label{trump-wants-us-to-get-cut-of-any-tiktok-deal-no-one-knows-how-thatd-work}}

The president has repeatedly called for a big payment to the Treasury
Department if a Chinese company sells the app to Microsoft, but there is
no provision in the law for that.

\includegraphics{https://static01.nyt.com/images/2020/08/08/us/politics/08dc-trump-deal01/merlin_175385274_5c253a96-ea25-4174-8386-27b256225651-articleLarge.jpg?quality=75\&auto=webp\&disable=upscale}

\href{https://www.nytimes.com/by/michael-d-shear}{\includegraphics{https://static01.nyt.com/images/2018/06/13/multimedia/author-michael-d-shear/author-michael-d-shear-thumbLarge-v2.png}}\href{https://www.nytimes.com/by/alan-rappeport}{\includegraphics{https://static01.nyt.com/images/2018/06/12/multimedia/author-alan-rappeport/author-alan-rappeport-thumbLarge-v2.png}}\href{https://www.nytimes.com/by/ana-swanson}{\includegraphics{https://static01.nyt.com/images/2018/12/10/multimedia/author-ana-swanson/author-ana-swanson-thumbLarge.png}}

By \href{https://www.nytimes.com/by/michael-d-shear}{Michael D. Shear},
\href{https://www.nytimes.com/by/alan-rappeport}{Alan Rappeport} and
\href{https://www.nytimes.com/by/ana-swanson}{Ana Swanson}

\begin{itemize}
\item
  Aug. 8, 2020Updated 9:52 a.m. ET
\item
  \begin{itemize}
  \item
  \item
  \item
  \item
  \item
  \end{itemize}
\end{itemize}

WASHINGTON --- If he were still running casinos in rough-and-tumble
Atlantic City, N.J., President Trump's demand about Microsoft's possible
purchase of TikTok might be translated this way: I want a piece of the
action.

In exchange for
\href{https://www.nytimes.com/2020/08/03/technology/trump-tiktok-microsoft.html}{blessing
Microsoft's acquisition} of the Chinese-owned social media platform, Mr.
Trump has said the United States Treasury should receive a ``very big
proportion'' of the sale price. If he follows through, it would signal
an effort to carve out an entirely new role for the federal government
in exerting its powers to approve or thwart business deals with national
security considerations.

In essence, the president is promising to orchestrate the kind of
pay-to-play bounty that the United States prohibits companies from
making to governments of other countries under the Foreign Corrupt
Practices Act.

And he is playing a role that is common among the autocratic leaders on
whom he has often heaped praise: using the sheer power of his office to
influence the private marketplace without clear legal or regulatory
authority.

``It's protection money. It's not what the government of the United
States should do,'' said Avery Gardiner, the general counsel for the
Center for Democracy \& Technology, a nonprofit advocacy organization
focused on digital rights, privacy and an open internet.

``It's scary to think that it might apply in some parts of business and
not in others,'' she added. ``It becomes a special tax if your business
is involved in social media. but you can only do the deal if you pay the
protection money. That's even worse.''

Numerous legal experts said they knew of no provision in United States
law that would allow the president, or anyone else in the government, to
force two private companies to make a substantial payment to consummate
a merger or an acquisition.

And even the president's own top economic adviser played down the idea,
conceding that it was not well thought out.

``I don't know if that's a key stipulation,'' Larry Kudlow, the director
of the National Economic Council, told reporters this week. ``It may be
that the president was thinking, because the Treasury has had to do so
much work on this, there are a lot of options here. I'm not sure it's a
specific concept that will be followed through.''

But for Mr. Trump --- who has repeated his demand no fewer than four
times in the last 10 days --- the instinct fits
\href{https://www.nytimes.com/2020/08/03/business/economy/trump-tiktok-china-business.html}{a
longstanding pattern of behavior} that has always challenged his party's
usual free-market philosophy.

The president berates or inflates companies with his Twitter feed,
seeking to interfere in the free market. He wields his office like an
economic cudgel, threatening tariffs against friend and foe alike and
demanding that government contracts be renegotiated. And he frequently
muses aloud about a presidency in which he can run the world as he ran
his company --- without the guardrails established by law, regulation,
customs or norms.

\includegraphics{https://static01.nyt.com/images/2020/08/08/us/politics/08dc-trump-deal02/merlin_175327311_9188ea4d-3bb1-4222-a029-14c2fdbd4315-articleLarge.jpg?quality=75\&auto=webp\&disable=upscale}

``Very simple,'' he explained to reporters this week about his approach
to a TikTok deal. ``I mean we have all the cards because, without us,
you can't come into the United States. It's like if you're a landlord,
and you have a tenant. The tenant's business needs a rent; it needs a
lease. And so what I said to them is, `Whatever the price is, a very big
proportion of that price would have to go to the Treasury of the United
States.'''

Mr. Trump added: ``And they understood that. And actually, they agreed
with me. I mean, I think they agreed with me very much.''

A spokesman for Microsoft declined to comment. But in a statement
\href{https://blogs.microsoft.com/blog/2020/08/02/microsoft-to-continue-discussions-on-potential-tiktok-purchase-in-the-united-states/}{issued
last Sunday}, the company offered a vague promise that it was committed
to ``providing proper economic benefits to the United States, including
the United States Treasury.''

If Microsoft ends up buying TikTok, or a part of its business, the
combined company would be subject to existing laws that could increase
local and federal tax revenue. The company could promise to bring
additional jobs to the United States, which could increase economic
activity and generate revenue. And there are small filing fees
associated with the national security review that the two companies are
undergoing.

Legal experts said there is also no law that explicitly prohibits
companies from voluntarily offering a gift to the government, as long as
it is not made under duress and the gift does not benefit any particular
individual government officials.

But they also warn that extracting a large cash payment as a condition
of a TikTok sale would undermine the integrity of a legal process that
operates with specific, objective standards. That could set a precedent
that deters similar deals in the future by injecting uncertainty into
the prospect of any big business deal.

That appears to be exactly what Mr. Trump wants.

The federal government has a role to play in a potential arrangement
between Microsoft and TikTok because of concerns that the Chinese-owned
video app could pose a national security threat by funneling personal
information about United States citizens to the Chinese government.

In an executive order issued late Thursday, Mr. Trump banned the app
from operating in the United States, but said the ban would take effect
in 45 days, apparently to give Microsoft time to explore a possible
purchase.

``This data collection threatens to allow the Chinese Communist Party
access to Americans' personal and proprietary information ---
potentially allowing China to track the locations of Federal employees
and contractors, build dossiers of personal information for blackmail,
and conduct corporate espionage,'' Mr. Trump said in the order.

Those concerns have also prompted a review by a special government panel
that examines national security threats, the Committee on Foreign
Investment in the United States, also known as CFIUS.

In the past, the committee has required companies to take tangible steps
to reduce the risk that their products or services could threaten the
security of the United States. But experts in the process said that
there were no provisions in the law that would justify Mr. Trump
demanding a cash payment to mitigate security issues.

The review process has no mechanism, they said, for ``side payments,''
however labeled, as a condition of a sale to Microsoft. Several said the
mere proposal could deter foreign investment in the United States.

``It would be deleterious to the process,'' said Aimen Mir, a former
deputy assistant secretary for investment security at the Treasury
Department. ``One of the strengths of CFIUS, in the eyes of investors
and companies from allied nations that are sometimes the subject of
CFIUS orders or mitigation agreements, is clarity that CFIUS focuses on
national security and national security alone.''

It is unclear how Mr. Trump got the idea of a cash payment in the first
place.

Some of the president's advisers had objected to the potential sale of
TikTok to an American company like Microsoft in part because such a deal
would end up funneling American dollars to China. Why should China get
paid for posing a security threat to the United States?

To Mr. Trump, ever the negotiator, there appeared to be a simple
solution to that problem: The United States would demand its cut, too.

''The president does use the power of the federal government against
individual companies in ways that are different than ever before,'' Ms.
Gardiner said. ``It's very antidemocratic.''

Advertisement

\protect\hyperlink{after-bottom}{Continue reading the main story}

\hypertarget{site-index}{%
\subsection{Site Index}\label{site-index}}

\hypertarget{site-information-navigation}{%
\subsection{Site Information
Navigation}\label{site-information-navigation}}

\begin{itemize}
\tightlist
\item
  \href{https://help.nytimes.com/hc/en-us/articles/115014792127-Copyright-notice}{©~2020~The
  New York Times Company}
\end{itemize}

\begin{itemize}
\tightlist
\item
  \href{https://www.nytco.com/}{NYTCo}
\item
  \href{https://help.nytimes.com/hc/en-us/articles/115015385887-Contact-Us}{Contact
  Us}
\item
  \href{https://www.nytco.com/careers/}{Work with us}
\item
  \href{https://nytmediakit.com/}{Advertise}
\item
  \href{http://www.tbrandstudio.com/}{T Brand Studio}
\item
  \href{https://www.nytimes.com/privacy/cookie-policy\#how-do-i-manage-trackers}{Your
  Ad Choices}
\item
  \href{https://www.nytimes.com/privacy}{Privacy}
\item
  \href{https://help.nytimes.com/hc/en-us/articles/115014893428-Terms-of-service}{Terms
  of Service}
\item
  \href{https://help.nytimes.com/hc/en-us/articles/115014893968-Terms-of-sale}{Terms
  of Sale}
\item
  \href{https://spiderbites.nytimes.com}{Site Map}
\item
  \href{https://help.nytimes.com/hc/en-us}{Help}
\item
  \href{https://www.nytimes.com/subscription?campaignId=37WXW}{Subscriptions}
\end{itemize}
