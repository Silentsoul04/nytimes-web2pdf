Sections

SEARCH

\protect\hyperlink{site-content}{Skip to
content}\protect\hyperlink{site-index}{Skip to site index}

\href{https://www.nytimes.com/section/world}{World}

\href{https://myaccount.nytimes.com/auth/login?response_type=cookie\&client_id=vi}{}

\href{https://www.nytimes.com/section/todayspaper}{Today's Paper}

\href{/section/world}{World}\textbar{}Coronavirus Live Updates: Weeks
Before Classes Start, Colleges Make Reopening Plans

\href{https://nyti.ms/30EIgRt}{https://nyti.ms/30EIgRt}

\begin{itemize}
\item
\item
\item
\item
\item
\end{itemize}

\href{https://www.nytimes.com/news-event/coronavirus?action=click\&pgtype=Article\&state=default\&region=TOP_BANNER\&context=storylines_menu}{The
Coronavirus Outbreak}

\begin{itemize}
\tightlist
\item
  live\href{https://www.nytimes.com/2020/08/08/world/coronavirus-updates.html?action=click\&pgtype=Article\&state=default\&region=TOP_BANNER\&context=storylines_menu}{Latest
  Updates}
\item
  \href{https://www.nytimes.com/interactive/2020/us/coronavirus-us-cases.html?action=click\&pgtype=Article\&state=default\&region=TOP_BANNER\&context=storylines_menu}{Maps
  and Cases}
\item
  \href{https://www.nytimes.com/interactive/2020/science/coronavirus-vaccine-tracker.html?action=click\&pgtype=Article\&state=default\&region=TOP_BANNER\&context=storylines_menu}{Vaccine
  Tracker}
\item
  \href{https://www.nytimes.com/interactive/2020/world/coronavirus-tips-advice.html?action=click\&pgtype=Article\&state=default\&region=TOP_BANNER\&context=storylines_menu}{F.A.Q.}
\item
  \href{https://www.nytimes.com/live/2020/08/07/business/stock-market-today-coronavirus?action=click\&pgtype=Article\&state=default\&region=TOP_BANNER\&context=storylines_menu}{Markets
  \& Economy}
\end{itemize}

Advertisement

\protect\hyperlink{after-top}{Continue reading the main story}

Supported by

\protect\hyperlink{after-sponsor}{Continue reading the main story}

LIVE UPDATES

Updated~

Aug. 8, 2020, 4:07 p.m. ET

Aug. 8, 2020, 4:07 p.m. ET

\hypertarget{coronavirus-live-updates-weeks-before-classes-start-colleges-make-reopening-plans}{%
\section{Coronavirus Live Updates: Weeks Before Classes Start, Colleges
Make Reopening
Plans}\label{coronavirus-live-updates-weeks-before-classes-start-colleges-make-reopening-plans}}

Talks on a new U.S. pandemic relief package are stalled and its crucial
benefits are expiring. President Trump said he would use executive
orders to provide aid, but it is unclear whether he has the power to do
so.

Right Now

President Trump plans to hold a news conference, where he's expected to
sign a handful of executive orders addressing the economic fallout from
the coronavirus.

\hypertarget{heres-what-you-need-to-know}{%
\subsubsection{Here's what you need to
know:}\label{heres-what-you-need-to-know}}

\begin{itemize}
\tightlist
\item
  \protect\hyperlink{link-7bd2f2ea}{Universities make reopening plans,
  and parents see tough choices no matter what.}
\item
  \protect\hyperlink{link-182e4a8e}{As U.S. relief talks falter again,
  Trump says he's prepared to act on his own.}
\item
  \protect\hyperlink{link-6d42ce45}{Motorcycles fill the streets of
  Sturgis, S.D., for a 10-day rally expected to attract 250,000 people.}
\item
  \protect\hyperlink{link-57c61e05}{A C.D.C. report on children shows
  hundreds were sent to intensive care for a syndrome connected to
  Covid-19.}
\item
  \protect\hyperlink{link-bb0ce1a}{The blockaded Gaza Strip is nearly
  untouched, except for tough new limits on movement.}
\item
  \protect\hyperlink{link-1f7e24cf}{Low-wage and unemployed workers find
  themselves in limbo as stimulus measures expire.}
\item
  \protect\hyperlink{link-5bad68cc}{Here's how to regulate indoor air
  when summer weather coincides with a pandemic.}
\end{itemize}

\includegraphics{https://static01.nyt.com/images/2020/08/08/business/08virus-briefing-college/merlin_175377363_8014a0f4-9a2d-44df-9db4-db595c604847-articleLarge.jpg?quality=75\&auto=webp\&disable=upscale}

\hypertarget{section}{%
\subsection{}\label{section}}

Universities make reopening plans, and parents see tough choices no
matter what.

The usual parental worries about college-bound children --- whether they
will be happy, or productive, or find a suitable major leading to a
stable career --- are getting sidelined this fall by one overwhelming
concern: With coronavirus cases
\href{https://www.nytimes.com/interactive/2020/us/coronavirus-us-cases.html}{spiking
in many parts of the country}, will students be safe at school?

\href{https://collegecrisis.shinyapps.io/dashboard/}{More than a quarter
of U.S. colleges} plan to begin fall instruction fully or mostly online,
but many are still opening up their dorms. And at many schools,
upperclassmen are returning to off-campus apartments, or fraternity or
sorority houses. That leaves parents with the choice of forcing their
20-year-olds to stay home against their will, or allowing them to leave
and join their friends, knowing the infection data may not be in their
favor.

``This is a situation where you have to pray for the best and be ready
for the worst,'' said Kelly Hutchison, a retired firefighter and
\href{https://books.apple.com/us/book/book-title/id1291492202?ls=1}{single
father} in Chicago whose daughter, Katelyn, is a student at Ithaca
College.

Some parents are still debating whether their child should take the year
off entirely. For schools on the semester system, tuition bills for
thousands, or even tens of thousands of dollars, are due this month. But
up until those due dates, colleges are trying to be flexible. In many
cases, ``you can defer admission, or you can take an academic leave, and
they'll allow you to come back,'' said Lynn Pasquerella, the president
of the Association of American Colleges and Universities.

Taking such a break, however, may not be realistic, said Jill
Schwitzgebel, a college counselor in Celebration, Fla. ``What is your
child going to do with a gap year?'' she said. ``Getting a job is tough.
Flying overseas is not happening.''

Other updates from around the U.S.:

\begin{itemize}
\item
  \textbf{Princeton}
  \href{https://www.princeton.edu/news/2020/08/07/fall-2020-update-undergraduate-education-be-fully-remote}{announced}
  Friday that all undergraduate classes would be held online during the
  fall semester. In a statement, the university's president said that
  the pandemic ``prevents a genuinely meaningful on-campus experience
  for undergraduates.'' On Monday, the university also said it would cut
  tuition by 10 percent for all undergraduates during the 2020-21 school
  year.
\item
  \textbf{Johns Hopkins University}
  \href{https://hub.jhu.edu/2020/08/06/university-moves-undergraduate-instruction-online/}{made
  a similar announcement} on Thursday, moving to remote learning and
  reducing undergraduate tuition by 10 percent for the fall term.
\item
  Gov. Gavin Newsom of California released
  \href{https://files.covid19.ca.gov/pdf/guidance-higher-education--en.pdf}{guidance}
  on Friday for colleges and universities that plan to reopen. For
  schools in counties that are flagged by the state for elevated
  transmission for three consecutive days, the guidance would prohibit
  indoor classes. Many of the campuses of \textbf{California State
  University}, the nation's largest four-year public university system,
  have
  \href{https://www.nytimes.com/2020/05/12/us/cal-state-online-classes.html}{already
  committed to remote learning} for the fall.
\item
  On Thursday, the \textbf{University of Massachusetts, Amherst}
  \href{https://www.umass.edu/coronavirus/news/significant-changes-our-fall-reopening-plan}{backtracked}
  on a previous plan to let students enrolled in online classes live on
  campus. Just weeks before the semester is scheduled to begin, the
  university said only a small subset of students ``enrolled in
  essential face-to-face classes'' would be allowed into dorms and
  dining halls.
\item
  Officials at \textbf{Harvard}
  \href{https://www.fas.harvard.edu/fas-decision-2020-2021-academic-year}{said}
  on Thursday that they planned to allow up to 40 percent of
  undergraduates, including the entire freshman class, to return to
  campus for the fall, but that all instruction would be delivered
  online. The university has not offered discounted tuition.
\end{itemize}

\hypertarget{tracking-the-coronavirus-}{%
\subsection{\texorpdfstring{\href{https://www.nytimes.com/interactive/2020/us/coronavirus-us-cases.html}{Tracking
the Coronavirus
›}}{Tracking the Coronavirus ›}}\label{tracking-the-coronavirus-}}

\href{https://www.nytimes.com/interactive/2020/us/coronavirus-us-cases.html}{}

\hypertarget{where-cases-are-rising-fastest}{%
\subsubsection{\texorpdfstring{Where cases are \textbf{rising}
fastest}{Where cases are rising fastest}}\label{where-cases-are-rising-fastest}}

\href{https://www.nytimes.com/interactive/2020/us/oklahoma-coronavirus-cases.html}{}

Okla.
\href{https://www.nytimes.com/interactive/2020/us/puerto-rico-coronavirus-cases.html}{}

P.R.
\href{https://www.nytimes.com/interactive/2020/us/virginia-coronavirus-cases.html}{}

Va.
\href{https://www.nytimes.com/interactive/2020/us/illinois-coronavirus-cases.html}{}

Ill.
\href{https://www.nytimes.com/interactive/2020/us/hawaii-coronavirus-cases.html}{}

Hawaii
\href{https://www.nytimes.com/interactive/2020/us/south-dakota-coronavirus-cases.html}{}

S.D.
\href{https://www.nytimes.com/interactive/2020/us/rhode-island-coronavirus-cases.html}{}

R.I.
\href{https://www.nytimes.com/interactive/2020/us/massachusetts-coronavirus-cases.html}{}

Mass.

\href{https://www.nytimes.com/interactive/2020/us/coronavirus-us-cases.html}{}

\hypertarget{us-hot-spots-}{%
\subsubsection{U.S. hot spots ›}\label{us-hot-spots-}}

\includegraphics{https://static01.nyt.com/newsgraphics/2020/03/16/coronavirus-maps/73f1a29b653c7065c40929a24ba018b33e7d99dc/images/orphan_usa-threeByTwoSmallAt2X.png}
\href{https://www.nytimes.com/interactive/2020/world/coronavirus-maps.html}{}

\hypertarget{worldwide-}{%
\subsubsection{Worldwide ›}\label{worldwide-}}

\includegraphics{https://static01.nyt.com/newsgraphics/2020/03/16/coronavirus-maps/73f1a29b653c7065c40929a24ba018b33e7d99dc/images/orphan_world-threeByTwoSmallAt2X.png}

\hypertarget{section-1}{%
\subsection{}\label{section-1}}

As U.S. relief talks falter again, Trump says he's prepared to act on
his own.

\includegraphics{https://static01.nyt.com/images/2020/09/07/business/07vid-trump/07vid-trump-videoSixteenByNine3000.jpg}

Crisis negotiations between the White House and top Democrats teetered
on the brink of collapse on Friday, as both sides said they remained
deeply divided on an economic recovery package and President Trump
indicated that he was prepared to act on his own to provide relief,
although it was unclear whether he has the authority to do so.

On Saturday afternoon, Mr. Trump plans to sign a handful of executive
orders related to virus relief. The White House declined to describe the
substance of the orders in advance. One person familiar with what was
expected said they related to a halt in the payroll tax, an eviction
moratorium, unemployment benefits and student loan relief.

\href{https://www.nytimes.com/2020/08/07/us/politics/trump-news-conference-bedminster.html}{At
a news conference on Friday} evening at his golf resort in Bedminster,
N.J.,
\href{https://www.nytimes.com/video/us/politics/100000007279339/trump-says-he-will-act-on-his-own-if-congress-doesnt-agree-on-relief.html}{Mr.
Trump said} that if an aid agreement with congressional Democrats could
not be reached, he would sign executive orders reinstating a national
moratorium on evictions, deferring student loan interest and payments
``until further notice,'' and ``enhancing unemployment benefits''
through the end of the year.

He also said he would defer payroll taxes, retroactive from July 1
through the end of the year.

The president **** did not specify how the deferral would work, and it
was unclear whether he had the authority to take such an action without
approval from Congress. The move, which would not help unemployed
workers, faces opposition from both Democrats and Republicans in
Congress.

The news conference came after a meeting between administration
officials and Democratic leaders that ended with no agreement and no
additional talks scheduled.

Democrats, who had earlier said they would be willing to lower their
spending demands to \$2 trillion from \$3.4 trillion, said the White
House needed to return with a higher overall price tag after Mr. Trump's
negotiators declined to accept that offer. Republicans have proposed a
\$1 trillion plan.

Treasury Secretary Steven Mnuchin and Mark Meadows, the White House
chief of staff, called for Democrats to lower the amount of aid for
state and local governments and to provide more specifics on how they
proposed to revive lapsed unemployment benefits.

While the executive orders have not been finalized, Mr. Meadows said it
was likely that action would come over the weekend.

\hypertarget{section-2}{%
\subsection{}\label{section-2}}

Motorcycles fill the streets of Sturgis, S.D., for a 10-day rally
expected to attract 250,000 people.

\includegraphics{https://static01.nyt.com/images/2020/08/07/us/07VIRUS-STURGIS/07VIRUS-STURGIS-videoSixteenByNine3000.jpg}

Tens of thousands of
\href{https://www.nytimes.com/2020/08/07/us/sturgis-motorcyle-rally.html}{motorcyclists
swarmed} the streets of Sturgis, S.D., on Saturday for an annual rally
despite
\href{https://www.nytimes.com/2020/08/06/us/sturgis-motorcyle-rally-coronavirus.html}{objections
from residents} --- and with little regard for the coronavirus.

The herds of people driving recreational vehicles, bikes and classic
cars overran every street in town, making no effort to keep six feet
apart. Few masks could be seen, and free bandannas being passed out were
mostly folded, or wrapped around people's heads.

With temperatures in the low 80s and not much cloud cover, many people
crowded under shopping tents where ``Screw Covid'' shirts were sold,
seeking shade.

The Sturgis Motorcycle Rally, a 10-day affair that began Friday, is
expected to attract roughly 250,000 enthusiasts this year --- about half
the number who attended last year, but a figure that puts it on track to
be among the country's largest public gatherings since the first
coronavirus cases emerged.

South Dakota is one of several states that did not impose a lockdown,
and state officials have not required residents to wear masks.

Health experts say the coronavirus is less likely to spread outdoors,
especially when people wear masks and socially distance. But large
gatherings like the motorcycle rally also increase the number of
visitors inside restaurants and stores.

A few businesses in Sturgis put up signs limiting the number of
customers who could enter, but most did not.

Over the past week, South Dakota has reported an
\href{https://www.nytimes.com/interactive/2020/us/south-dakota-coronavirus-cases.html}{average
of 87 coronavirus cases per day}. At least two new virus deaths and 106
new cases were reported on Saturday.

\hypertarget{section-3}{%
\subsection{}\label{section-3}}

A C.D.C. report on children shows hundreds were sent to intensive care
for a syndrome connected to Covid-19.

Image

Children cooling off in a fountain in New York City last
month.Credit...Jeenah Moon/Reuters

Hundreds of children in America, most of them previously healthy, have
experienced an inflammatory syndrome associated with Covid-19, and most
became so ill that they needed intensive care, according to
\href{https://www.cdc.gov/mmwr/volumes/69/wr/mm6932e2.htm?s_cid=mm6932e2_w\#T1_down}{a
new report} from the Centers for Disease Control and Prevention.

The syndrome, which can be deadly, has rattled parents and education
officials as schools across the United States struggle with the prospect
of reopening in the fall and the coronavirus continues its spread.

The researchers said that from early March to late July, the C.D.C.
received reports of 570 young people --- ranging from infants to age 20
--- who met the definition of the new condition, called
\href{https://www.nytimes.com/2020/05/17/health/coronavirus-multisystem-fnflammatory-syndrome-children-teenagers.html}{Multisystem
Inflammatory Syndrome in Children} or MIS-C. The reports came from
health departments in 40 states, as well as New York City and
Washington, D.C.

The patients were disproportionately people of color, echoing a pattern
in adults who have been struck by the respiratory disease caused by the
virus. About 40 percent were Hispanic or Latino, 33 percent were Black,
and 13 percent were white, the report said. The median age was 8. About
25 percent of the patients had obesity before becoming sick.

MIS-C was first recognized in May as a condition linked to Covid-19 that
appears to occur in children and young people who often had not
developed any of the respiratory symptoms that are the primary way the
virus attacks adults.

The syndrome, which can include a fever, rash, pinkeye, stomach
distress, confusion, bluish lips, muscle weakness, racing heart rate and
cardiac shock, appears to emerge days or weeks after the initial viral
infection, and experts believe it may be the result of a revved-up
immune system response to defeating the virus's first assault.

The C.D.C. reported that about two-thirds of the patients had no
previous underlying medical conditions, and most experienced
complications that involved four or more organ systems, especially the
heart. Ten died. Nearly two-thirds were admitted to intensive care units
for a median of five days.

\hypertarget{section-4}{%
\subsection{}\label{section-4}}

The blockaded Gaza Strip is nearly untouched, except for tough new
limits on movement.

Image

Neveen Zanon, center, at her home in Gaza, has now been able to visit
her father in the West Bank, where he is coping with esophageal
cancer.Credit...Shbair Fatima for The New York Times

The blockaded Gaza Strip might be among the few places in the world
where no cases of community transmission of the coronavirus have been
recorded --- a phenomenon attributed to the coastal enclave's isolation
as well as to swift measures taken by its militant Hamas rulers.

But the pandemic has not left Gaza untouched.

Citing a need to combat the virus, the authorities that control Gaza's
borders have imposed new restrictions on movement outside the territory.
That has exacerbated an already challenging situation for Palestinians
who say they urgently need to travel to Israel and the West Bank.

In March, fearing an outbreak in Gaza, the Hamas authorities ordered all
travelers returning to the territory by way of Israel and Egypt to enter
quarantine facilities for three weeks. They could not leave quarantine
until they had passed two virus tests.

The system seems to have succeeded. All 78 known infections in the
territory were detected at quarantine facilities.

Still, experts did not rule out the possibility of the pandemic
penetrating into the area's densely populated cities and towns.

``All it takes is one small mistake,'' said Gerald Rockenschaub, the
head of the World Health Organization's mission to the Palestinians.
``There's no guarantee the virus won't get inside.''

Mr. Rockenschaub warned that Gaza lacked the resources to deal with a
widespread outbreak, noting that medical institutions had only about 100
adult ventilators, most of which were already in use.

\hypertarget{section-5}{%
\subsection{}\label{section-5}}

Low-wage and unemployed workers find themselves in limbo as stimulus
measures expire.

Image

Since her recent eviction, Latrish Oseko and her daughter have been
staying at a Delaware hotel. She said she was following the debate over
emergency relief, wondering, ``Is there going to be hope for
me?''Credit...Hannah Yoon for The New York Times

Before the coronavirus hobbled the U.S. economy, many low-wage workers
were already struggling to make ends meet.

After
\href{https://www.nytimes.com/interactive/2020/08/05/upshot/us-unemployment-maps-coronavirus.html}{mass
layoffs} and a deep recession followed in the early months of the
pandemic, millions of workers found themselves faced with evictions,
late car payments, and crushing medical bills. For many, the main solace
through the worst months of the crisis was a broad range of stimulus
measures, including \$600 per week in extra unemployment benefits.

But with those measures expiring, and
\href{https://www.nytimes.com/2020/08/07/us/politics/trump-congress-stimulus.html}{no
clear indication of whether new ones} will replace them, many unemployed
workers now find themselves in limbo, struggling to find work in an
economy that remains significantly weakened.

\href{https://www.nytimes.com/2020/08/07/business/economy/housing-economy-eviction-renters.html}{Eviction
moratoriums} are expiring or have expired in much of the country, and
\href{https://nlihc.org/sites/default/files/The_Eviction_Crisis_080720.pdf}{a
report released Friday} warned that 30 million to 40 million tenants
\href{https://www.nytimes.com/2020/08/07/business/economy/housing-economy-eviction-renters.html}{risk
losing their homes in the coming months}. The Paycheck Protection
Program, which helped thousands of small businesses to retain workers,
also ends this week.

\href{https://www.aeaweb.org/articles?id=10.1257/aer.20170537}{Research
from the last recession} found that when unemployment benefits ran out,
people cut their spending on food, medicine and other necessities,
suggesting they were able to do little to prepare for the drop in
income.

While wealthier families may be able to draw on savings to get by until
Congress strikes a deal to prolong the stimulus, lower-income households
face serious long-term consequences from even a temporary lapse in
income. An eviction can make it hard to rent in the future. Having a car
repossessed can make it hard to find another job. And for children,
periods of hunger, homelessness and stress can have long-term effects on
development and learning.

While the U.S. economy has
\href{https://www.nytimes.com/live/2020/08/07/business/stock-market-today-coronavirus}{slowly
added back some jobs} that vanished at the beginning of the pandemic,
the unemployment rate still stands at over 10 percent. For those who may
not return to work for some time, the loss of protections has only added
to uncertainty about the future.

\hypertarget{section-6}{%
\subsection{}\label{section-6}}

Here's how to regulate indoor air when summer weather coincides with a
pandemic.

Image

Window air-conditioning units are typically designed for comfort, not
health.Credit...Gleb Garanich/Reuters

Even as the coronavirus continues to spread widely, and public health
officials have urged people to move activities outside as much as
possible, the summer heat still tends to demand a great deal of time
spent indoors.

For those who regularly share home or office spaces with others for
extended periods, this may raise questions about indoor air quality. A
growing number of scientists are convinced that significant coronavirus
transmission can occur through the air indoors, and that poor
ventilation magnifies the risk. But the options available for increasing
airflow or filtering out are not all created equal.

Experts have a few recommendations.

If the temperature outside is tolerable, consider opening a few windows
to let outdoor air in. This can be amplified by blowing air inside with
a box fan.

``The more outside air you have, the more you dilute the virus,'' said
Jose-Luis Jimenez, an aerosol scientist at the University of Colorado
Boulder.

In hotter climates, some air-conditioners can be used safely if they
cool and circulate both outdoor and indoor air. But be wary of certain
models that only recirculate the air inside.

Those looking to be especially cautious may consider using air filters.
But as with air-conditioners, to derive any real benefit consumers
should look to those that meet specifications to filter out virus
particles that are far smaller than other airborne particles like dust
or pollen.

Above all, experts caution that airflow patterns are difficult to
predict. The best way to prevent spreading the virus inside may be to
avoid holding indoor gatherings altogether.

\hypertarget{section-7}{%
\subsection{}\label{section-7}}

The political parties in Belgium miss another deadline to form a
government, and virus cases are increasing.

Image

From left, Bart De Wever, the leader of the conservative Flemish
separatist party known as the N-VA, King Philippe and Paul Magnette, the
Socialist party leader in the French-speaking Walloon region, in
Brussels on Saturday.Credit...Olivier Hoslet/EPA, via Shutterstock

Even a pandemic could not bring Belgium's fractious political parties
together.

Party leaders blew through a Saturday afternoon deadline to form a new
government, more than a year and a half after the last one collapsed.
The country has been operating with an emergency minority coalition
throughout the coronavirus epidemic.

But the crisis has
\href{https://www.nytimes.com/2020/08/08/world/europe/coronavirus-nursing-homes-elderly.html}{exposed
the weaknesses in a bureaucratic political system} --- it has among the
highest Covid-19 death
\href{https://www.nytimes.com/interactive/2020/world/coronavirus-maps.html}{rates
in the world}. Belgium has nine health ministers who answer to six
parliaments. Officials have acknowledged being slow to respond to the
outbreak as they haggled over who was responsible for what.

Making ambitious change to the political system or taking up an
aggressive economic stimulus package would most likely require a
full-fledged majority government, something that has eluded Belgium
since December 2018. Leaders of the two largest parties --- the
conservative Flemish separatist party known as the N-VA and the
French-speaking Socialists --- are seeking a majority coalition with
smaller parties.

But party leaders said Saturday that they were unable to meet the
deadline set by King Philippe, the Belgian head of state. The king
extended the deadline, once again, to Aug. 17.

The country is polarized along regional and linguistic lines, making
governing perpetually difficult. This is now the longest period without
a formal government in Belgian history.

``I hope to form a government as soon as possible,'' said Paul Magnette,
the head of the French-speaking Socialists. ``Our country needs it to
effectively combat the epidemic, which sadly is rising again.''

\hypertarget{section-8}{%
\subsection{}\label{section-8}}

The C.D.C. closes some offices in Atlanta after discovering dangerous
bacteria in the water.

Image

The headquarters of the Centers for Disease Control and Prevention in
Atlanta.Credit...Audra Melton for The New York Times

The nation's foremost public health agency is learning that it is not
immune to the complex effects of the coronavirus pandemic.

Recently, the Centers for Disease Control and Prevention told employees
that some office space it leases in the Atlanta area would be closed
again after property managers of the buildings discovered Legionella,
the bacteria that causes Legionnaires' disease, in water sources at the
sites. No employees were sickened. The announcement was
\href{https://www.cnn.com/2020/08/07/health/coronavirus-legionnaires-water-cdc/index.html}{reported
on Friday by CNN}.

That the C.D.C. is contending with this problem highlights the
seriousness of Legionella in the aftermath of coronavirus lockdowns, and
how complicated it can be to prevent it.

The C.D.C. itself \href{https://www.cdc.gov/legionella/index.html}{warns
that Legionnaires' disease}, a respiratory illness, can be fatal in 1 in
10 cases. Since various jurisdictions in the United States have put in
effect lockdowns to contain the spread of the new coronavirus, some
experts have been warning of the
\href{https://www.nytimes.com/2020/05/20/health/coronavirus-legionnaires-offices.html}{risk
of Legionnaires' outbreaks} when people return to buildings left
unoccupied for months.

The bacteria that causes the illness, Legionella pneumophila, can form
in warm, stagnant water that is not properly disinfected. When sinks are
turned on or toilets flushed, the bacteria can then be sent through the
air and inhaled.

While most earlier research focused on the growth of Legionella during
weekends and short holiday periods, scientists are only beginning to
learn about how the bacteria proliferates during periods of long-term
stagnation, and which methods are most effective to protect against it.

\hypertarget{section-9}{%
\subsection{}\label{section-9}}

Are illicit parties endangering New York City?

Image

At a party under a segment of the Kosciuszko Bridge that spans Brooklyn
and Queens, many people did not wear masks.Credit...Jimmy Escobar

New Yorkers, by and large, have adhered to rules mandating social
distancing and mask wearing. The diligence has helped keep the
coronavirus under control in the city even as outbreaks have raged
across the United States, primarily in the South and the West.

As the summer wears on, however, mounting reports of parties, concerts
and other social events, like a recent rave under the Kosciuszko Bridge,
are raising fears that New York's hard-earned stability may be tenuous.

Over the last few weeks, videos and photos posted on social media have
shown densely packed, mask-free crowds.

``It's illegal,'' Gov. Andrew M. Cuomo said at a recent news conference,
referring to the partying. ``It not only violates public health, but it
violates human decency.''

The images contrast sharply with the
\href{https://www.nytimes.com/2020/08/07/style/coronavirus-nyc-historic-season.html}{memories}
of
\href{https://www.nytimes.com/2020/04/04/nyregion/coronavirus-hospital-brooklyn.html}{a
brutal spring} in New York that left tens of thousands dead,
disproportionately ravaging low-income communities and neighborhoods
with high numbers of Black and Latino people.

Illegal raves are growing in popularity in Europe, including in Berlin,
in London and near Paris, as coronavirus lockdowns are eased across the
continent but most nightclubs remain closed.

Outdoor events for hundreds --- in some cases, thousands --- organized
via social media and messaging apps, are in full swing each weekend,
causing headaches for police forces and lawmakers, and stirring public
debate and news media panic.

Worries that nightlife activity would fuel the spread of the virus have
in the meantime led Curaçao, the Caribbean island, to close its bars and
clubs for at least two weeks since Friday, according to the Dutch
newswire ANP. The nearby island Aruba was reported to have almost 300
confirmed cases over the last five days.

\hypertarget{section-10}{%
\subsection{}\label{section-10}}

As Myanmar's tourism collapses, horses become too expensive to keep
alive.

Image

Praying outside the closed Shwedagon Pagoda in Yangon, Myanmar, last
week. The Buddhist pagoda is one of the country's most popular tourist
sites.~~Credit...Sai Aung Main/Agence France-Presse --- Getty Images

When a coronavirus lockdown
\href{https://www.nytimes.com/2020/03/27/world/asia/coronavirus-myanmar-jobs-china.html?searchResultPosition=7}{sealed
Myanmar's borders} in March, the tourism industry was devastated, even
if the country was spared from disease.

Now, in the hill town of Pyin Oo Lwin, owners of horse carts that used
to clip-clop through streets laden with visitors are sending their
animals to slaughterhouses because they can no longer afford to keep
them alive.

``I feel sad about selling the horse, because he is like a family
member,'' said U Maung Win, a horse cart owner. ``He worked so hard to
save our lives, and I could not save his life.''

For months now, no tourists have come to ride through the town, with its
cool breezes and pretty gardens, Mr. Maung Win said, but the horses
still needed to be fed, at a cost of a couple dollars a day. The
slaughterhouses paid about \$500 per animal.

Mr. Maung Win, who supports a family of six, now works as a mason and is
paid less than \$10 a week.

``It's better than nothing,'' he said.

With his horse and a cart painted like a fairy-tale stagecoach, Mr.
Maung Win could pull in \$10 in a single day, delivering tourists to the
botanical gardens or cafes offering fresh strawberries. Couples posed
for wedding pictures in the carriages, holding the bell-adorned reins in
their intertwined hands.

Two-thirds of the 100 or so horse carts in town are now gone, Mr. Maung
Win said.

``I tried not to sell the horse to the slaughterhouse, but I had no
choice,'' he said. ``I still feel sad talking about this.''

Lucky friends, he said, had two horses. But he owned only one.

\hypertarget{section-11}{%
\subsection{}\label{section-11}}

When the virus hit, nursing home residents in some countries were left
to die.

Image

The Christalain nursing home in Brussels in June. An owner of the home
said hospitals had rejected residents who had Covid-19, even when beds
were available.Credit...Mauricio Lima for The New York Times

Of all the missteps by governments during the pandemic, few have had
such an immediate and devastating impact as the failure to protect
nursing homes. Tens of thousands of older people have died ---
casualties not only of the virus, but of more than a decade of ignored
warnings that nursing homes were vulnerable.

Public health officials around the world excluded nursing homes from
their pandemic preparedness plans and omitted residents from the
mathematical models used to guide their responses.

In recent months, as the United States has blundered its way into the
world's largest death toll, about 40 percent of those fatalities
\href{https://www.kff.org/health-costs/issue-brief/state-data-and-policy-actions-to-address-coronavirus/\#stateleveldata}{have
been linked} to long-term care centers. Yet European countries still
lead the world in deaths per capita, in part because of what happened
inside their nursing homes.

Spanish prosecutors
\href{https://www.nytimes.com/2020/03/25/world/europe/Spain-coronavirus-nursing-homes.html}{are
investigating cases} in which residents were abandoned to die. In
Sweden, overwhelmed emergency doctors have acknowledged
\href{https://www.dn.se/nyheter/sverige/overlakare-logn-att-patienter-inte-prioriterats-bort/}{turning
away elderly patients}. In Britain, the
\href{https://www.independent.co.uk/news/health/coronavirus-care-homes-nhs-hospital-discharges-deaths-a9544671.html}{government
ordered thousands of older hospital patients} --- including some with
Covid-19 --- back to nursing homes to make room for an expected crush of
virus cases. (Similar policies were in effect in
\href{https://www.nytimes.com/2020/04/24/us/nursing-homes-coronavirus.html}{some
U.S. states}.)

The response in Belgium has offered a gruesome twist: Paramedics and
hospitals sometimes flatly denied care to elderly people, even as
hospital beds sat unused.

``Paramedics had been instructed by their referral hospital not to take
patients over a certain age, often 75 but sometimes as low as 65,'' the
charity Doctors Without Borders said in a July
\href{https://www.msf.org/sites/msf.org/files/2020-07/Left\%20behind\%20-\%20MSF\%20care\%20homes\%20in\%20Belgium\%20report.pdf}{report}.

More than 5,700 residents of nursing homes in the country have died,
\href{https://www.medrxiv.org/content/10.1101/2020.06.20.20136234v1.full.pdf}{according
to newly published data}. During the peak of the crisis, from March
through mid-May, residents accounted for two out of every three
coronavirus deaths.

Reporting was contributed by Iyad Abuheweila, Matt Apuzzo, Makr Walker,
Hannah Beech, Pam Belluck, Conor Dougherty, Alex Marshall, Constant
Méheut, Zach Montague, Heather Murphy, Julia Echikson, Max Horberry,
Claire Moses, Monika Pronczuk, Adam Rasgon, Thomas Rogers, Constance
Sommer, Matina Stevis-Gridneff, Katherine J. Wu and Mihir Zaveri.

Advertisement

\protect\hyperlink{after-bottom}{Continue reading the main story}

\hypertarget{site-index}{%
\subsection{Site Index}\label{site-index}}

\hypertarget{site-information-navigation}{%
\subsection{Site Information
Navigation}\label{site-information-navigation}}

\begin{itemize}
\tightlist
\item
  \href{https://help.nytimes.com/hc/en-us/articles/115014792127-Copyright-notice}{©~2020~The
  New York Times Company}
\end{itemize}

\begin{itemize}
\tightlist
\item
  \href{https://www.nytco.com/}{NYTCo}
\item
  \href{https://help.nytimes.com/hc/en-us/articles/115015385887-Contact-Us}{Contact
  Us}
\item
  \href{https://www.nytco.com/careers/}{Work with us}
\item
  \href{https://nytmediakit.com/}{Advertise}
\item
  \href{http://www.tbrandstudio.com/}{T Brand Studio}
\item
  \href{https://www.nytimes.com/privacy/cookie-policy\#how-do-i-manage-trackers}{Your
  Ad Choices}
\item
  \href{https://www.nytimes.com/privacy}{Privacy}
\item
  \href{https://help.nytimes.com/hc/en-us/articles/115014893428-Terms-of-service}{Terms
  of Service}
\item
  \href{https://help.nytimes.com/hc/en-us/articles/115014893968-Terms-of-sale}{Terms
  of Sale}
\item
  \href{https://spiderbites.nytimes.com}{Site Map}
\item
  \href{https://help.nytimes.com/hc/en-us}{Help}
\item
  \href{https://www.nytimes.com/subscription?campaignId=37WXW}{Subscriptions}
\end{itemize}
