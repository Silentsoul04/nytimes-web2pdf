Sections

SEARCH

\protect\hyperlink{site-content}{Skip to
content}\protect\hyperlink{site-index}{Skip to site index}

\href{https://www.nytimes.com/section/world}{World}

\href{https://myaccount.nytimes.com/auth/login?response_type=cookie\&client_id=vi}{}

\href{https://www.nytimes.com/section/todayspaper}{Today's Paper}

\href{/section/world}{World}\textbar{}Coronavirus Live Updates: U.S.
Surpasses 5 Million Coronavirus Cases

\href{https://nyti.ms/30EIgRt}{https://nyti.ms/30EIgRt}

\begin{itemize}
\item
\item
\item
\item
\item
\end{itemize}

\href{https://www.nytimes.com/news-event/coronavirus?action=click\&pgtype=Article\&state=default\&region=TOP_BANNER\&context=storylines_menu}{The
Coronavirus Outbreak}

\begin{itemize}
\tightlist
\item
  live\href{https://www.nytimes.com/2020/08/08/world/coronavirus-updates.html?action=click\&pgtype=Article\&state=default\&region=TOP_BANNER\&context=storylines_menu}{Latest
  Updates}
\item
  \href{https://www.nytimes.com/interactive/2020/us/coronavirus-us-cases.html?action=click\&pgtype=Article\&state=default\&region=TOP_BANNER\&context=storylines_menu}{Maps
  and Cases}
\item
  \href{https://www.nytimes.com/interactive/2020/science/coronavirus-vaccine-tracker.html?action=click\&pgtype=Article\&state=default\&region=TOP_BANNER\&context=storylines_menu}{Vaccine
  Tracker}
\item
  \href{https://www.nytimes.com/interactive/2020/world/coronavirus-tips-advice.html?action=click\&pgtype=Article\&state=default\&region=TOP_BANNER\&context=storylines_menu}{F.A.Q.}
\item
  \href{https://www.nytimes.com/live/2020/08/07/business/stock-market-today-coronavirus?action=click\&pgtype=Article\&state=default\&region=TOP_BANNER\&context=storylines_menu}{Markets
  \& Economy}
\end{itemize}

Advertisement

\protect\hyperlink{after-top}{Continue reading the main story}

Supported by

\protect\hyperlink{after-sponsor}{Continue reading the main story}

LIVE UPDATES

Updated~

Aug. 8, 2020, 8:32 p.m. ET

Aug. 8, 2020, 8:32 p.m. ET

\hypertarget{coronavirus-live-updates-us-surpasses-5-million-coronavirus-cases}{%
\section{Coronavirus Live Updates: U.S. Surpasses 5 Million Coronavirus
Cases}\label{coronavirus-live-updates-us-surpasses-5-million-coronavirus-cases}}

Brazil ranks second after the U.S. and also reached a milestone: more
than 100,000 deaths. Tens of thousands of motorcyclists gathered in
Sturgis, S.D., despite objections from residents.

Right Now

President Trump signed executive actions on economic aid as stimulus
talks stalled. Their impact may be limited, and legal challenges are
expected.

\hypertarget{heres-what-you-need-to-know}{%
\subsubsection{Here's what you need to
know:}\label{heres-what-you-need-to-know}}

\begin{itemize}
\tightlist
\item
  \protect\hyperlink{link-697eb3e1}{At 5 million cases, the U.S. has
  passed another coronavirus milestone.}
\item
  \protect\hyperlink{link-5b7b4fa2}{With Congress at an impasse, Trump
  signs actions for another round of economic aid.}
\item
  \protect\hyperlink{link-680eccee}{Brazil surpasses 100,000 virus
  deaths a month earlier than health officials predicted.}
\item
  \protect\hyperlink{link-7bd2f2ea}{Universities make reopening plans,
  and parents see tough choices no matter what.}
\item
  \protect\hyperlink{link-6d42ce45}{Motorcycles fill the streets of
  Sturgis, S.D., for a 10-day rally expected to attract 250,000 people.}
\item
  \protect\hyperlink{link-458f8def}{Parents in the U.S. are suing
  schools, demanding they teach children in person.}
\item
  \protect\hyperlink{link-57c61e05}{A C.D.C. report on children shows
  hundreds were sent to intensive care for a syndrome connected to
  Covid-19.}
\end{itemize}

\includegraphics{https://static01.nyt.com/images/2020/08/07/world/07virus-briefing-5million-sub/merlin_175093896_a95684f2-9a09-4ead-a7fd-cca08f985db0-articleLarge.jpg?quality=75\&auto=webp\&disable=upscale}

\hypertarget{section}{%
\subsection{}\label{section}}

At 5 million cases, the U.S. has passed another coronavirus milestone.

While politicians wrangled over a pandemic relief package and schools
struggled over whether to open their doors to students, the United
States passed another milestone on Saturday: more than five million
known coronavirus infections.

No other country has reported as many cases. Brazil ranks second, with
about three million, and India is third with two million. (In cases per
capita, the United States ranks eighth, between Oman and Peru.)

The data, from
\href{https://www.nytimes.com/interactive/2020/us/coronavirus-us-cases.html}{a
New York Times database}, is based on reports of known cases from
federal, state and local officials. Public health experts have warned
that the actual number of people infected is far greater.

Cases are trending upward in seven states, as well as in Puerto Rico,
Guam and the U.S. Virgin Islands, and decreasing in 17, according to The
Times database. In the past week, Louisiana, Mississippi and Florida had
the most new cases relative to population.

The United States reported its millionth case on April 28, more than
three months after the first reported case. The country passed two
million cases on June 10, three million on July 7 and four million on
July 23.

The United States now tests roughly 720,000 people a day, according to
\href{https://covidtracking.com/data/us-daily}{data from the Covid
Tracking Project.}

The number of new coronavirus cases daily peaked on July 16, with
75,697. It has been slowly tapering off since then, to a seven-day
average of around 54,000 per day.

The seven-day average daily death toll is hovering around 1,000. That is
down from a peak of more than 2,200 on a single day in mid-April, when
bigger cities like New York and Seattle were hit the hardest. (The most
deadly single day was April 15, with 2,752.)

At least 161,000 people have died since the pandemic began. But the
seven-day average daily death toll is now significantly higher than it
was in early July, when it was around 500. Cases have surged since then
---
\href{https://www.nytimes.com/2020/06/14/us/coronavirus-united-states.html}{particularly
in the Sun Belt states} and in communities where officials moved quickly
to
\href{https://www.nytimes.com/interactive/2020/07/09/us/coronavirus-cases-reopening-trends.html}{reopen}.
Many of the places with the most cases per capita have been smaller
cities and rural communities in the South and the Midwest.

\includegraphics{https://static01.nyt.com/images/2020/08/08/business/08virus-briefing-trumplede/08virus-briefing-trumplede-videoSixteenByNine3000.jpg}

\hypertarget{section-1}{%
\subsection{}\label{section-1}}

With Congress at an impasse, Trump signs actions for another round of
economic aid.

President Trump took executive action on Saturday to circumvent Congress
and try to extend an array of federal pandemic relief, resorting to a
legally dubious set of edicts whose impact was unclear, as negotiations
over
\href{https://www.nytimes.com/2020/08/08/world/coronavirus-updates.html}{an
economic recovery package} appeared on the brink of collapse.

It was not clear what authority Mr. Trump had to act on his own on the
measures or what immediate effect, if any, they would have, given that
Congress controls federal spending. But his decision to sign the
measures --- billed as a federal eviction ban, a payroll tax suspension,
and relief for student borrowers and the unemployed --- reflected the
failure of two weeks of talks between White House officials and top
congressional Democrats to strike a deal on a broad relief plan as
crucial benefits have expired with no resolution in sight.

Mr. Trump's move also illustrated the heightened concern of a president
staring down re-election in the middle of a historic recession and a
pandemic, and determined to show voters that he was doing something to
address the crises. But despite Mr. Trump's assertions on Saturday that
his actions ``will take care of this entire situation,'' the orders also
leave a number of critical bipartisan funding proposals unaddressed,
including providing assistance to small businesses, billions of dollars
to schools ahead of the new school year, aid to states and cities and a
second round of \$1,200 stimulus checks to Americans.

``Nancy Pelosi and Chuck Schumer have chosen to hold this vital
assistance hostage,'' Mr. Trump said, savaging the two top Democrats
during a news conference at his private golf club in New Jersey, his
second in two days. A few dozen club guests were in attendance, and the
president appeared to revel in their laughter at his jokes denouncing
his political rivals.

\hypertarget{tracking-the-coronavirus-}{%
\subsection{\texorpdfstring{\href{https://www.nytimes.com/interactive/2020/us/coronavirus-us-cases.html}{Tracking
the Coronavirus
›}}{Tracking the Coronavirus ›}}\label{tracking-the-coronavirus-}}

\href{https://www.nytimes.com/interactive/2020/us/coronavirus-us-cases.html}{}

\hypertarget{where-cases-are-rising-fastest}{%
\subsubsection{\texorpdfstring{Where cases are \textbf{rising}
fastest}{Where cases are rising fastest}}\label{where-cases-are-rising-fastest}}

\href{https://www.nytimes.com/interactive/2020/us/oklahoma-coronavirus-cases.html}{}

Okla.
\href{https://www.nytimes.com/interactive/2020/us/puerto-rico-coronavirus-cases.html}{}

P.R.
\href{https://www.nytimes.com/interactive/2020/us/virginia-coronavirus-cases.html}{}

Va.
\href{https://www.nytimes.com/interactive/2020/us/illinois-coronavirus-cases.html}{}

Ill.
\href{https://www.nytimes.com/interactive/2020/us/hawaii-coronavirus-cases.html}{}

Hawaii
\href{https://www.nytimes.com/interactive/2020/us/south-dakota-coronavirus-cases.html}{}

S.D.
\href{https://www.nytimes.com/interactive/2020/us/rhode-island-coronavirus-cases.html}{}

R.I.
\href{https://www.nytimes.com/interactive/2020/us/massachusetts-coronavirus-cases.html}{}

Mass.

\href{https://www.nytimes.com/interactive/2020/us/coronavirus-us-cases.html}{}

\hypertarget{us-hot-spots-}{%
\subsubsection{U.S. hot spots ›}\label{us-hot-spots-}}

\includegraphics{https://static01.nyt.com/newsgraphics/2020/03/16/coronavirus-maps/cd3f3bc6b8c8089803b9023f79b7e90a4d168062/images/orphan_usa-threeByTwoSmallAt2X.png}
\href{https://www.nytimes.com/interactive/2020/world/coronavirus-maps.html}{}

\hypertarget{worldwide-}{%
\subsubsection{Worldwide ›}\label{worldwide-}}

\includegraphics{https://static01.nyt.com/newsgraphics/2020/03/16/coronavirus-maps/cd3f3bc6b8c8089803b9023f79b7e90a4d168062/images/orphan_world-threeByTwoSmallAt2X.png}

\hypertarget{section-2}{%
\subsection{}\label{section-2}}

Brazil surpasses 100,000 virus deaths a month earlier than health
officials predicted.

Image

Julio Cesar Ramos and his cousin Eduardo Magela mourning at the funeral
of Maria das Dores, Ramos' mother, in Brasilia, Brazil, last month. Ms.
Dores died after contracting the virus.Credit...Andre Sousa Borges/EPA,
via Shutterstock

Five months after its first case of Covid-19, Brazil on Saturday passed
the bleak milestones of 100,000 deaths and three million cases,
\href{https://www.nytimes.com/interactive/2020/world/americas/brazil-coronavirus-cases.html}{according
to a New York Times database}.

President Jair Bolsonaro has repeatedly denied the severity of Brazil's
coronavirus crisis even as the death count has risen more quickly than
the government anticipated.

Mr. Bolsonaro's refusal to support social distancing measures pushed two
health ministers out, leaving the country's response to the virus to be
led by a general with no experience in public health.

The ministry has yet to reach an agreement with city and state
officials, who have been scrambling to respond with varying levels of
success, on how to combat the pandemic.

In early March, officials at Brazil's Ministry of Health predicted the
virus would kill at least 100,000 of the country's citizens. But they
estimated that number would only be reached in September, said Julio
Croda, who then headed the ministry's department overseeing immunization
and transmissible diseases.

``The presidency wouldn't believe in these numbers,'' he said. ``It's
one month ahead of schedule because the social distancing measures
fell.''

Since June, Brazil has frequently reported more than 1,000 new deaths a
day, as the number of new infections and deaths plateaued at a high
level. Dr. Croda believes the country will continue on this trajectory
for some weeks, adding tens of thousands of deaths to its toll in the
coming months.

The numbers, he believes, will eventually fall --- as they have begun to
do in severely hit states such as Amazonas --- when a large number of
Brazilians acquire immunity to the virus.

But that ``has nothing to do with the government,'' Dr. Croda said. ``It
is a consequence of tragedy.''

\hypertarget{section-3}{%
\subsection{}\label{section-3}}

Universities make reopening plans, and parents see tough choices no
matter what.

Image

Katelyn Hutchison, a member of her schools track team, and her father,
Kelly Hutchison. Seeing her disappointment at a national track
championship meet being canceled because of the coronavirus ``was one of
the most painful things I've ever experienced,'' Mr. Hutchison
said.Credit...Nolis Anderson for The New York Times

The usual parental worries about college-bound children --- whether they
will be happy, or productive, or find a suitable major leading to a
stable career --- are getting sidelined this fall by one overwhelming
concern: With coronavirus cases
\href{https://www.nytimes.com/interactive/2020/us/coronavirus-us-cases.html}{spiking
in many parts of the country}, will students be safe at school?

\href{https://collegecrisis.shinyapps.io/dashboard/}{More than a quarter
of U.S. colleges} plan to begin fall instruction fully or mostly online,
but many are still opening up their dorms. And at many schools,
upperclassmen are returning to off-campus apartments, or fraternity or
sorority houses. That leaves parents with the choice of forcing their
20-year-olds to stay home against their will, or allowing them to leave
and join their friends, knowing the infection data may not be in their
favor.

``This is a situation where you have to pray for the best and be ready
for the worst,'' said Kelly Hutchison, a retired firefighter and
\href{https://books.apple.com/us/book/book-title/id1291492202?ls=1}{single
father} in Chicago whose daughter, Katelyn, is a student at Ithaca
College.

Some parents are still debating whether their child should take the year
off entirely. For schools on the semester system, tuition bills for
thousands, or even tens of thousands of dollars, are due this month. But
up until those due dates, colleges are trying to be flexible. In many
cases, ``you can defer admission, or you can take an academic leave, and
they'll allow you to come back,'' said Lynn Pasquerella, the president
of the Association of American Colleges and Universities.

Taking such a break, however, may not be realistic, said Jill
Schwitzgebel, a college counselor in Celebration, Fla. ``What is your
child going to do with a gap year?'' she said. ``Getting a job is tough.
Flying overseas is not happening.''

Other updates from around the U.S.:

\begin{itemize}
\item
  \textbf{Princeton}
  \href{https://www.princeton.edu/news/2020/08/07/fall-2020-update-undergraduate-education-be-fully-remote}{announced}
  Friday that all undergraduate classes would be held online during the
  fall semester. In a statement, the university's president said that
  the pandemic ``prevents a genuinely meaningful on-campus experience
  for undergraduates.'' On Monday, the university also said it would cut
  tuition by 10 percent for all undergraduates during the 2020-21 school
  year.
\item
  \textbf{Johns Hopkins University}
  \href{https://hub.jhu.edu/2020/08/06/university-moves-undergraduate-instruction-online/}{made
  a similar announcement} on Thursday, moving to remote learning and
  reducing undergraduate tuition by 10 percent for the fall term.
\item
  Gov. Gavin Newsom of California released
  \href{https://files.covid19.ca.gov/pdf/guidance-higher-education--en.pdf}{guidance}
  on Friday for colleges and universities that plan to reopen. For
  schools in counties that are flagged by the state for elevated
  transmission for three consecutive days, the guidance would prohibit
  indoor classes. Many of the campuses of \textbf{California State
  University}, the nation's largest four-year public university system,
  have
  \href{https://www.nytimes.com/2020/05/12/us/cal-state-online-classes.html}{already
  committed to remote learning} for the fall.
\item
  On Thursday, the \textbf{University of Massachusetts, Amherst}
  \href{https://www.umass.edu/coronavirus/news/significant-changes-our-fall-reopening-plan}{backtracked}
  on a previous plan to let students enrolled in online classes live on
  campus. Just weeks before the semester is scheduled to begin, the
  university said only a small subset of students ``enrolled in
  essential face-to-face classes'' would be allowed into dorms and
  dining halls.
\item
  Officials at \textbf{Harvard}
  \href{https://www.fas.harvard.edu/fas-decision-2020-2021-academic-year}{said}
  on Thursday that they planned to allow up to 40 percent of
  undergraduates, including the entire freshman class, to return to
  campus for the fall, but that all instruction would be delivered
  online. The university has not offered discounted tuition.
\end{itemize}

\hypertarget{section-4}{%
\subsection{}\label{section-4}}

Motorcycles fill the streets of Sturgis, S.D., for a 10-day rally
expected to attract 250,000 people.

\includegraphics{https://static01.nyt.com/images/2020/08/07/us/07VIRUS-STURGIS/07VIRUS-STURGIS-videoSixteenByNine3000.jpg}

Tens of thousands of
\href{https://www.nytimes.com/2020/08/07/us/sturgis-motorcyle-rally.html}{motorcyclists
swarmed} the streets of Sturgis, S.D., on Saturday for an annual rally
despite
\href{https://www.nytimes.com/2020/08/06/us/sturgis-motorcyle-rally-coronavirus.html}{objections
from residents} --- and with little regard for the coronavirus.

The herds of people driving recreational vehicles, bikes and classic
cars overran every street in town, making no effort to keep six feet
apart. Few masks could be seen, and free bandannas being passed out were
mostly folded, or wrapped around people's heads.

With temperatures in the low 80s and not much cloud cover, many people
crowded under shopping tents where ``Screw Covid'' shirts were sold,
seeking shade.

The Sturgis Motorcycle Rally, a 10-day affair that began Friday, is
expected to attract roughly 250,000 enthusiasts this year --- about half
the number who attended last year, but a figure that puts it on track to
be among the country's largest public gatherings since the first
coronavirus cases emerged.

South Dakota is one of several states that did not impose a lockdown,
and state officials have not required residents to wear masks.

Health experts say the coronavirus is less likely to spread outdoors,
especially when people wear masks and socially distance. But large
gatherings like the motorcycle rally also increase the number of
visitors inside restaurants and stores.

A few businesses in Sturgis put up signs limiting the number of
customers who could enter, but most did not.

Over the past week, South Dakota has reported an
\href{https://www.nytimes.com/interactive/2020/us/south-dakota-coronavirus-cases.html}{average
of 87 coronavirus cases per day}. At least two new virus deaths and 106
new cases were reported on Saturday.

\hypertarget{section-5}{%
\subsection{}\label{section-5}}

Parents in the U.S. are suing schools, demanding they teach children in
person.

Image

Parents and children who want schools to reopen protest outside a
meeting of the Hillsborough County school board in Tampa, Fla., on
Thursday.Credit...Octavio Jones/Getty Images

Two parents sued the school board and health department in Franklin
County, Ohio, this week demanding that their son's high school provide
in-person classes to start the school year later this month. The lawsuit
claims that remote learning, which the district plans to provide to all
students until at least Sept. 21, does not meet their son's educational
needs.

Similar lawsuits have been filed in other parts of the country,
including Springfield, Mo., where
\href{https://www.ky3.com/2020/07/31/springfield-attorney-files-lawsuit-against-springfield-public-schools-over-reopening-plan/}{three
families are demanding} five days a week of in-person classes, and
California, where
\href{https://www.ocregister.com/2020/08/07/parents-sue-gov-newsom-other-state-officials-demanding-in-person-instruction/}{more
than a dozen parents} are seeking to overturn an order by Gov. Gavin
Newsom that prevents schools from immediately reopening classrooms in
most of the state.

Parents of private school students in Maryland
\href{https://www.nytimes.com/2020/08/05/us/schools-reopening-private-public.html}{also
sued this week} to block a Montgomery County order requiring private
schools to teach remotely. The order was
\href{https://www.baltimoresun.com/coronavirus/bs-md-republicans-private-schools-20200807-3eiwqeyfgfh5nnppen3z4o7vba-story.html}{rescinded
on Friday} after a battle of authority between the county and the
governor.

``Distance learning has been proved to be largely ineffective,'' said
Rex Elliott, the lawyer representing the Ohio parents suing the Upper
Arlington Board of Education and the Franklin County Health Department.
``That is devastating to their educational growth in the face of a virus
that, in this age group, simply is not a dangerous or lethal concern.''

Public health experts
\href{https://www.nytimes.com/2020/07/30/health/coronavirus-children.html}{continue
to debate the evidence} over how easily children contract or spread the
virus. It is also unclear how often they develop a
\href{https://www.nytimes.com/2020/06/29/well/family/caring-for-children-with-multisystem-inflammatory-syndrome.html}{rare
inflammatory condition} that has been linked to Covid-19.

\hypertarget{section-6}{%
\subsection{}\label{section-6}}

A C.D.C. report on children shows hundreds were sent to intensive care
for a syndrome connected to Covid-19.

Image

Children cooling off in a fountain in New York City last
month.Credit...Jeenah Moon/Reuters

Hundreds of children in America, most of them previously healthy, have
experienced an inflammatory syndrome associated with Covid-19, and most
became so ill that they needed intensive care, according to
\href{https://www.cdc.gov/mmwr/volumes/69/wr/mm6932e2.htm?s_cid=mm6932e2_w\#T1_down}{a
new report} from the Centers for Disease Control and Prevention.

The syndrome, which can be deadly, has rattled parents and education
officials as schools across the United States struggle with the prospect
of reopening in the fall and the coronavirus continues its spread.

The researchers said that from early March to late July, the C.D.C.
received reports of 570 young people --- ranging from infants to age 20
--- who met the definition of the new condition, called
\href{https://www.nytimes.com/2020/05/17/health/coronavirus-multisystem-fnflammatory-syndrome-children-teenagers.html}{Multisystem
Inflammatory Syndrome in Children} or MIS-C. The reports came from
health departments in 40 states, as well as New York City and
Washington, D.C.

The patients were disproportionately people of color, echoing a pattern
in adults who have been struck by the respiratory disease caused by the
virus. About 40 percent were Hispanic or Latino, 33 percent were Black,
and 13 percent were white, the report said. The median age was 8. About
25 percent of the patients had obesity before becoming sick.

MIS-C was first recognized in May as a condition linked to Covid-19 that
appears to occur in children and young people who often had not
developed any of the respiratory symptoms that are the primary way the
virus attacks adults.

The syndrome, which can include a fever, rash, pinkeye, stomach
distress, confusion, bluish lips, muscle weakness, racing heart rate and
cardiac shock, appears to emerge days or weeks after the initial viral
infection, and experts believe it may be the result of a revved-up
immune system response to defeating the virus's first assault.

The C.D.C. reported that about two-thirds of the patients had no
previous underlying medical conditions, and most experienced
complications that involved four or more organ systems, especially the
heart. Ten died. Nearly two-thirds were admitted to intensive care units
for a median of five days.

\hypertarget{section-7}{%
\subsection{}\label{section-7}}

The blockaded Gaza Strip is nearly untouched, except for tough new
limits on movement.

Image

Neveen Zanon, center, at her home in Gaza, has now been able to visit
her father in the West Bank, where he is coping with esophageal
cancer.Credit...Shbair Fatima for The New York Times

The blockaded Gaza Strip might be among the few places in the world
where no cases of community transmission of the coronavirus have been
recorded --- a phenomenon attributed to the coastal enclave's isolation
as well as to swift measures taken by its militant Hamas rulers.

But the pandemic has not left Gaza untouched.

Citing a need to combat the virus, the authorities that control Gaza's
borders have imposed new restrictions on movement outside the territory.
That has exacerbated an already challenging situation for Palestinians
who say they urgently need to travel to Israel and the West Bank.

In March, fearing an outbreak in Gaza, the Hamas authorities ordered all
travelers returning to the territory by way of Israel and Egypt to enter
quarantine facilities for three weeks. They could not leave quarantine
until they had passed two virus tests.

The system seems to have succeeded. All 78 known infections in the
territory were detected at quarantine facilities.

Still, experts did not rule out the possibility of the pandemic
penetrating into the area's densely populated cities and towns.

``All it takes is one small mistake,'' said Gerald Rockenschaub, the
head of the World Health Organization's mission to the Palestinians.
``There's no guarantee the virus won't get inside.''

Mr. Rockenschaub warned that Gaza lacked the resources to deal with a
widespread outbreak, noting that medical institutions had only about 100
adult ventilators, most of which were already in use.

\hypertarget{section-8}{%
\subsection{}\label{section-8}}

Low-wage and unemployed workers find themselves in limbo as stimulus
measures expire.

Image

Since her recent eviction, Latrish Oseko and her daughter have been
staying at a Delaware hotel. She said she was following the debate over
emergency relief, wondering, ``Is there going to be hope for
me?''Credit...Hannah Yoon for The New York Times

Before the coronavirus hobbled the U.S. economy, many low-wage workers
were already struggling to make ends meet.

After
\href{https://www.nytimes.com/interactive/2020/08/05/upshot/us-unemployment-maps-coronavirus.html}{mass
layoffs} and a deep recession followed in the early months of the
pandemic, millions of workers found themselves faced with evictions,
late car payments, and crushing medical bills. For many, the main solace
through the worst months of the crisis was a broad range of stimulus
measures, including \$600 per week in extra unemployment benefits.

But with those measures expiring, and
\href{https://www.nytimes.com/2020/08/07/us/politics/trump-congress-stimulus.html}{no
clear indication of whether new ones} will replace them, many unemployed
workers now find themselves in limbo, struggling to find work in an
economy that remains significantly weakened.

\href{https://www.nytimes.com/2020/08/07/business/economy/housing-economy-eviction-renters.html}{Eviction
moratoriums} are expiring or have expired in much of the country, and
\href{https://nlihc.org/sites/default/files/The_Eviction_Crisis_080720.pdf}{a
report released Friday} warned that 30 million to 40 million tenants
\href{https://www.nytimes.com/2020/08/07/business/economy/housing-economy-eviction-renters.html}{risk
losing their homes in the coming months}. The Paycheck Protection
Program, which helped thousands of small businesses to retain workers,
also ends this week.

\href{https://www.aeaweb.org/articles?id=10.1257/aer.20170537}{Research
from the last recession} found that when unemployment benefits ran out,
people cut their spending on food, medicine and other necessities,
suggesting they were able to do little to prepare for the drop in
income.

While wealthier families may be able to draw on savings to get by until
Congress strikes a deal to prolong the stimulus, lower-income households
face serious long-term consequences from even a temporary lapse in
income. An eviction can make it hard to rent in the future. Having a car
repossessed can make it hard to find another job. And for children,
periods of hunger, homelessness and stress can have long-term effects on
development and learning.

While the U.S. economy has
\href{https://www.nytimes.com/live/2020/08/07/business/stock-market-today-coronavirus}{slowly
added back some jobs} that vanished at the beginning of the pandemic,
the unemployment rate still stands at over 10 percent. For those who may
not return to work for some time, the loss of protections has only added
to uncertainty about the future.

\hypertarget{section-9}{%
\subsection{}\label{section-9}}

Here's how to regulate indoor air when summer weather coincides with a
pandemic.

Image

Window air-conditioning units are typically designed for comfort, not
health.Credit...Gleb Garanich/Reuters

Even as the coronavirus continues to spread widely, and public health
officials have urged people to move activities outside as much as
possible, the summer heat still tends to demand a great deal of time
spent indoors.

For those who regularly share home or office spaces with others for
extended periods, this may raise questions about indoor air quality. A
growing number of scientists are convinced that significant coronavirus
transmission can occur through the air indoors, and that poor
ventilation magnifies the risk. But the options available for increasing
airflow or filtering out are not all created equal.

Experts have a few recommendations.

If the temperature outside is tolerable, consider opening a few windows
to let outdoor air in. This can be amplified by blowing air inside with
a box fan.

``The more outside air you have, the more you dilute the virus,'' said
Jose-Luis Jimenez, an aerosol scientist at the University of Colorado
Boulder.

In hotter climates, some air-conditioners can be used safely if they
cool and circulate both outdoor and indoor air. But be wary of certain
models that only recirculate the air inside.

Those looking to be especially cautious may consider using air filters.
But as with air-conditioners, to derive any real benefit consumers
should look to those that meet specifications to filter out virus
particles that are far smaller than other airborne particles like dust
or pollen.

Above all, experts caution that airflow patterns are difficult to
predict. The best way to prevent spreading the virus inside may be to
avoid holding indoor gatherings altogether.

\hypertarget{section-10}{%
\subsection{}\label{section-10}}

The political parties in Belgium miss another deadline to form a
government, and virus cases are increasing.

Image

From left, Bart De Wever, the leader of the conservative Flemish
separatist party known as the N-VA, King Philippe and Paul Magnette, the
Socialist party leader in the French-speaking Walloon region, in
Brussels on Saturday.Credit...Olivier Hoslet/EPA, via Shutterstock

Even a pandemic could not bring Belgium's fractious political parties
together.

Party leaders blew through a Saturday afternoon deadline to form a new
government, more than a year and a half after the last one collapsed.
The country has been operating with an emergency minority coalition
throughout the coronavirus epidemic.

But the crisis has
\href{https://www.nytimes.com/2020/08/08/world/europe/coronavirus-nursing-homes-elderly.html}{exposed
the weaknesses in a bureaucratic political system} --- it has among the
highest Covid-19 death
\href{https://www.nytimes.com/interactive/2020/world/coronavirus-maps.html}{rates
in the world}. Belgium has nine health ministers who answer to six
parliaments. Officials have acknowledged being slow to respond to the
outbreak as they haggled over who was responsible for what.

Making ambitious change to the political system or taking up an
aggressive economic stimulus package would most likely require a
full-fledged majority government, something that has eluded Belgium
since December 2018. Leaders of the two largest parties --- the
conservative Flemish separatist party known as the N-VA and the
French-speaking Socialists --- are seeking a majority coalition with
smaller parties.

But party leaders said Saturday that they were unable to meet the
deadline set by King Philippe, the Belgian head of state. The king
extended the deadline, once again, to Aug. 17.

The country is polarized along regional and linguistic lines, making
governing perpetually difficult. This is now the longest period without
a formal government in Belgian history.

``I hope to form a government as soon as possible,'' said Paul Magnette,
the head of the French-speaking Socialists. ``Our country needs it to
effectively combat the epidemic, which sadly is rising again.''

\hypertarget{section-11}{%
\subsection{}\label{section-11}}

Are illicit parties endangering New York City?

Image

At a party under a segment of the Kosciuszko Bridge that spans Brooklyn
and Queens, many people did not wear masks.Credit...Jimmy Escobar

New Yorkers, by and large, have adhered to rules mandating social
distancing and mask wearing. The diligence has helped keep the
coronavirus under control in the city even as outbreaks have raged
across the United States, primarily in the South and the West.

As the summer wears on, however, mounting reports of parties, concerts
and other social events, like a recent rave under the Kosciuszko Bridge,
are raising fears that New York's hard-earned stability may be tenuous.

Over the last few weeks, videos and photos posted on social media have
shown densely packed, mask-free crowds.

``It's illegal,'' Gov. Andrew M. Cuomo said at a recent news conference,
referring to the partying. ``It not only violates public health, but it
violates human decency.''

The images contrast sharply with the
\href{https://www.nytimes.com/2020/08/07/style/coronavirus-nyc-historic-season.html}{memories}
of
\href{https://www.nytimes.com/2020/04/04/nyregion/coronavirus-hospital-brooklyn.html}{a
brutal spring} in New York that left tens of thousands dead,
disproportionately ravaging low-income communities and neighborhoods
with high numbers of Black and Latino people.

Illegal raves are growing in popularity in Europe, including in Berlin,
in London and near Paris, as coronavirus lockdowns are eased across the
continent but most nightclubs remain closed.

Outdoor events for hundreds --- in some cases, thousands --- organized
via social media and messaging apps, are in full swing each weekend,
causing headaches for police forces and lawmakers, and stirring public
debate and news media panic.

Worries that nightlife activity would fuel the spread of the virus have
in the meantime led Curaçao, the Caribbean island, to close its bars and
clubs for at least two weeks since Friday, according to the Dutch
newswire ANP. The nearby island Aruba was reported to have almost 300
confirmed cases over the last five days.

Reporting was contributed by Iyad Abuheweila, Sarah Almukhtar, Manuela
Andreoni, Matt Apuzzo, Hannah Beech, Pam Belluck, Julia Calderone, Emily
Cochrane, Conor Dougherty, Jacey Fortin, Maggie Haberman, Alex Marshall,
Giulia McDonnell Nieto del Rio, Constant Méheut, Zach Montague, Heather
Murphy, Julia Echikson, Max Horberry, Claire Moses, Monika Pronczuk,
Adam Rasgon, Thomas Rogers, Constance Sommer, Matina Stevis-Gridneff,
Jim Tankersley, Derrick Taylor, Mark Walker, Katherine J. Wu and Mihir
Zaveri.

Advertisement

\protect\hyperlink{after-bottom}{Continue reading the main story}

\hypertarget{site-index}{%
\subsection{Site Index}\label{site-index}}

\hypertarget{site-information-navigation}{%
\subsection{Site Information
Navigation}\label{site-information-navigation}}

\begin{itemize}
\tightlist
\item
  \href{https://help.nytimes.com/hc/en-us/articles/115014792127-Copyright-notice}{©~2020~The
  New York Times Company}
\end{itemize}

\begin{itemize}
\tightlist
\item
  \href{https://www.nytco.com/}{NYTCo}
\item
  \href{https://help.nytimes.com/hc/en-us/articles/115015385887-Contact-Us}{Contact
  Us}
\item
  \href{https://www.nytco.com/careers/}{Work with us}
\item
  \href{https://nytmediakit.com/}{Advertise}
\item
  \href{http://www.tbrandstudio.com/}{T Brand Studio}
\item
  \href{https://www.nytimes.com/privacy/cookie-policy\#how-do-i-manage-trackers}{Your
  Ad Choices}
\item
  \href{https://www.nytimes.com/privacy}{Privacy}
\item
  \href{https://help.nytimes.com/hc/en-us/articles/115014893428-Terms-of-service}{Terms
  of Service}
\item
  \href{https://help.nytimes.com/hc/en-us/articles/115014893968-Terms-of-sale}{Terms
  of Sale}
\item
  \href{https://spiderbites.nytimes.com}{Site Map}
\item
  \href{https://help.nytimes.com/hc/en-us}{Help}
\item
  \href{https://www.nytimes.com/subscription?campaignId=37WXW}{Subscriptions}
\end{itemize}
