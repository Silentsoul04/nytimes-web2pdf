Sections

SEARCH

\protect\hyperlink{site-content}{Skip to
content}\protect\hyperlink{site-index}{Skip to site index}

\href{https://www.nytimes.com/section/world/middleeast}{Middle East}

\href{https://myaccount.nytimes.com/auth/login?response_type=cookie\&client_id=vi}{}

\href{https://www.nytimes.com/section/todayspaper}{Today's Paper}

\href{/section/world/middleeast}{Middle East}\textbar{}Clashes Erupt in
Beirut at Blast Protest as Lebanon's Anger Boils Over

\href{https://nyti.ms/2XFhwyn}{https://nyti.ms/2XFhwyn}

\begin{itemize}
\item
\item
\item
\item
\item
\end{itemize}

Beirut Explosion

\begin{itemize}
\tightlist
\item
  \href{https://www.nytimes.com/2020/08/05/world/middleeast/beirut-explosion-what-happened.html?action=click\&pgtype=Article\&state=default\&region=TOP_BANNER\&context=storylines_menu}{What
  We Know}
\item
  \href{https://www.nytimes.com/2020/08/05/video/beirut-explosion-footage.html?action=click\&pgtype=Article\&state=default\&region=TOP_BANNER\&context=storylines_menu}{Footage
  of the Blast}
\item
  \href{https://www.nytimes.com/2020/08/05/world/middleeast/beirut-explosion-ammonium-nitrate.html?action=click\&pgtype=Article\&state=default\&region=TOP_BANNER\&context=storylines_menu}{What
  is Ammonium Nitrate?}
\item
  \href{https://www.nytimes.com/interactive/2020/08/04/world/middleeast/beirut-explosion-damage.html?action=click\&pgtype=Article\&state=default\&region=TOP_BANNER\&context=storylines_menu}{Mapping
  the Damage}
\end{itemize}

Advertisement

\protect\hyperlink{after-top}{Continue reading the main story}

Supported by

\protect\hyperlink{after-sponsor}{Continue reading the main story}

\hypertarget{clashes-erupt-in-beirut-at-blast-protest-as-lebanons-anger-boils-over}{%
\section{Clashes Erupt in Beirut at Blast Protest as Lebanon's Anger
Boils
Over}\label{clashes-erupt-in-beirut-at-blast-protest-as-lebanons-anger-boils-over}}

The demonstrations were fueled by fury over the corruption and
negligence of the country's ruling elite. Security forces fired tear gas
to push back the protesters.

\includegraphics{https://static01.nyt.com/images/2020/08/08/world/08Lebanon01/08Lebanon01-videoSixteenByNine3000.jpg}

\href{https://www.nytimes.com/by/ben-hubbard}{\includegraphics{https://static01.nyt.com/images/2018/10/10/multimedia/author-ben-hubbard/author-ben-hubbard-thumbLarge.png}}\href{https://www.nytimes.com/by/mona-el-naggar}{\includegraphics{https://static01.nyt.com/images/2018/06/13/multimedia/author-mona-el-naggar/author-mona-el-naggar-thumbLarge.jpg}}

By \href{https://www.nytimes.com/by/ben-hubbard}{Ben Hubbard} and
\href{https://www.nytimes.com/by/mona-el-naggar}{Mona El-Naggar}

\begin{itemize}
\item
  Aug. 8, 2020Updated 3:37 p.m. ET
\item
  \begin{itemize}
  \item
  \item
  \item
  \item
  \item
  \end{itemize}
\end{itemize}

BEIRUT, Lebanon --- Violent clashes between demonstrators and security
forces transformed much of central Beirut into a battle zone of flying
rocks, swinging batons and clouds of tear gas on Saturday, as the fury
over a huge explosion in Beirut's port this week fueled attacks on
government buildings.

By nightfall, angry protesters had stormed three government ministries,
a handful of legislators had resigned, and the prime minister had called
for early elections, the first major signs that the blast could shake up
the country's political system, widely derided as dysfunctional.

Many Lebanese considered
\href{https://www.nytimes.com/2020/08/04/world/middleeast/lebanon-explosion.html}{the
blast}, which sent a shock wave through the capital that destroyed
entire neighborhoods and killed at least 154 people, as only the latest
and most dangerous manifestation of the corruption and negligence of the
country's political elite.

The clashes on Saturday erupted across broad swaths of the city's
center, with demonstrators yanking down barricades blocking access to
the Parliament and throwing rocks at the security forces, who flooded
the area with tear gas.

``Haven't they quenched their thirst for blood? We came here peacefully,
and they do this?'' Rasha Habbal, a 21-year-old student who had come to
protest with her 57-year-old mother, said of the security forces. Both
women had been tear-gassed.

``Either they go and we stay, or they stay and we leave,'' Ms. Habbal
said of the country's leaders.

Elsewhere in the city, about 200 protesters, including a group of
retired military officers, took over the Foreign Ministry building for a
number of hours. They hung red banners with a raised fist from the
building, which had been damaged in the blast, and proclaimed Beirut a
``disarmed'' city. The group left the building after the army arrived.

Throughout the day, many thousands of people gathered to demonstrate in
the central Martyrs' Square, which is not far from the blast site and is
surrounded by high-priced office buildings and an upscale pedestrian
shopping mall, both of which were damaged in the blast.

The square, which is also close to the Parliament, has been the central
site of protests that have
\href{https://www.nytimes.com/2019/10/23/world/middleeast/lebanon-protests.html}{flared
since last fall} demanding the removal of the country's top politicians.
Many of Saturday's protesters said it was anger at what they had lost in
the blast that had driven them back into the streets.

``I lost my house, my car, my job, I lost friends,'' said a protester,
Eddy Gabriel, who carried a photo of two neighbors who had died in the
blast. ``There is nothing to be afraid of. Everything is gone.''

Lebanon was already grappling with an array of crises before this week's
explosion, as the
\href{https://www.nytimes.com/2020/07/12/world/middleeast/beirut-lebanon-economic-crisis.html}{economy
has sunk}, banks have refused to give depositors access to their money,
and unemployment and inflation have soared. In the weeks before the
blast, the number of coronavirus cases reported daily had begun to
spike, and many parts of the country were suffering from lengthy power
cuts.

But the explosion, and indications that it was
\href{https://www.nytimes.com/2020/08/05/world/middleeast/beirut-explosion-lebanon.html}{rooted
in governmental neglect}, have pushed tensions to the boiling point.

Lebanese officials have said the explosion on Tuesday happened when
2,750 tons of
\href{https://www.nytimes.com/2020/08/05/world/middleeast/beirut-explosion-ammonium-nitrate.html}{ammonium
nitrate}, a compound often used to make fertilizer and bombs, combusted,
perhaps because of a fire started by welders working nearby. The
industrial chemical
\href{https://www.nytimes.com/2020/08/05/world/middleeast/beirut-explosion-ship.html}{had
been stored in the port} since 2014.

The dead included 43 Syrians, the Syrian state news agency said on
Saturday. Lebanon hosts about one million Syrian refugees, and many
other Syrians live and work in the country.

The blast injured some 5,000 people and pushed at least 250,000 from
their homes. The prime minister has vowed to investigate it and hold all
those who were behind it accountable, but doubts that justice will be
done abound in a country with a long history of civil strife and
assassinations whose perpetrators were never prosecuted.

\includegraphics{https://static01.nyt.com/images/2020/08/08/world/08Lebanon02sub/merlin_175432008_52a5b340-0074-4705-a366-0b5ea8795ed0-articleLarge.jpg?quality=75\&auto=webp\&disable=upscale}

President Michel Aoun on Friday said the blast could have been caused by
a bomb or ``foreign interference,'' without providing details or
evidence.

In a televised speech, Hassan Nasrallah, the secretary-general of
Hezbollah, the powerful militant group and political party, denied his
group had any connection to the chemicals, the blast or the port.

Hezbollah, which is backed by Iran and has sent fighters to help keep
President Bashar al-Assad of Syria in power, is widely believed to use
the port to smuggle and store weapons. But no evidence has surfaced
linking the group to the chemicals or the explosion.

Anger at the country's top politicians was tangible at the protests,
where demonstrators erected gallows and conducted ceremonial hangings of
cardboard cutouts of Mr. Aoun, Nabih Berri, the speaker of Parliament,
and Mr. Nasrallah of Hezbollah.

But the fury targeted not just specific figures, but also the political
system itself, in which everything from top governmental posts to civil
service jobs are allocated according to a complex sectarian system. The
protesters consider that system, and the power brokers who use it to
enrich themselves and channel patronage to their supporters, to be the
source of many of the country's problems.

``It's a corrupt government, they have to be held accountable,'' said
Marilyn Kallas, 21, wielding a broom she used to help clean up a damaged
neighborhood before coming to the protest. ``Hopefully they will
resign.''

Over the course of Saturday's protests, some demonstrators broke into
the Economy Ministry, where they sent papers raining down onto the
sidewalk, and others made it into the Energy Ministry. On the wall of
the \href{https://www.abl.org.lb/english/home}{Association of Banks in
Lebanon}, someone had spray painted ``fallen'' in Arabic.

Siding with the protesters, four members of Parliament resigned on
Saturday. Sami Gemayel, the head of Kataeb, a Christian opposition
party, said its three legislators had quit and called on others to
resign for the ``birth of a new Lebanon.''

Paula Yacoubian, an independent member of Parliament, also resigned, she
confirmed in a text message.

In a televised speech, Prime Minister Hassan Diab said he would ask his
cabinet on Monday to approve early parliamentary elections.

But those moves fell well short of the sweeping changes to how the
country is run that protesters have demanded.

While government assistance to the blast victims has been minimal,
foreign aid has streamed in, along with technicians and medics who are
helping identify buildings at risk of collapsing and treating the
wounded.

The office of President Emmanuel Macron of France announced that an
international aid summit will be held by video conference on Sunday,
co-hosted by France and the United Nations.

Mr. Macron was the first foreign leader to visit Lebanon since the
blast, and he
\href{https://www.nytimes.com/2020/08/06/world/middleeast/beirut-explosion.html}{walked
through some of the hardest hit areas} to speak with residents,
something that Lebanon's own president and prime minister have not done,
likely to avoid becoming the targets of public anger.

The United States is providing more than \$15 million in aid, and
President Trump said on Friday that he would join Sunday's
videoconference.

Ahmed Aboul Gheit, the head of the Arab League, said on Saturday that he
would seek to mobilize support from Arab countries after meeting with
President Aoun.

``We are ready to help with all our means,'' Mr. Aboul Gheit said.

Image

The site of the explosion seen from the window of a damaged
building.~Credit...Diego Ibarra Sanchez for The New York Times

Despite drawing large numbers of people, the protest movement has so far
failed to make significant progress toward putting a new governing
system in place.

Many of the country's top politicians and party leaders are former
militia commanders from Lebanon's 15-year civil war, which ended in
1990, and Lebanese accuse them of looting the country while failing to
ensure basic services, like regular electricity and drinkable water.

``It had become clear that this regime could not deliver, but now it has
become clear that it can kill and obliterate an entire neighborhood,''
said Sami Atallah, director of the Lebanese Center for Policy Studies.
``The question to me is, is this going to be a game changer, and what
does it mean to have a game changer?''

Georgi Azar and Kareem Chehayeb contributed reporting.

Advertisement

\protect\hyperlink{after-bottom}{Continue reading the main story}

\hypertarget{site-index}{%
\subsection{Site Index}\label{site-index}}

\hypertarget{site-information-navigation}{%
\subsection{Site Information
Navigation}\label{site-information-navigation}}

\begin{itemize}
\tightlist
\item
  \href{https://help.nytimes.com/hc/en-us/articles/115014792127-Copyright-notice}{©~2020~The
  New York Times Company}
\end{itemize}

\begin{itemize}
\tightlist
\item
  \href{https://www.nytco.com/}{NYTCo}
\item
  \href{https://help.nytimes.com/hc/en-us/articles/115015385887-Contact-Us}{Contact
  Us}
\item
  \href{https://www.nytco.com/careers/}{Work with us}
\item
  \href{https://nytmediakit.com/}{Advertise}
\item
  \href{http://www.tbrandstudio.com/}{T Brand Studio}
\item
  \href{https://www.nytimes.com/privacy/cookie-policy\#how-do-i-manage-trackers}{Your
  Ad Choices}
\item
  \href{https://www.nytimes.com/privacy}{Privacy}
\item
  \href{https://help.nytimes.com/hc/en-us/articles/115014893428-Terms-of-service}{Terms
  of Service}
\item
  \href{https://help.nytimes.com/hc/en-us/articles/115014893968-Terms-of-sale}{Terms
  of Sale}
\item
  \href{https://spiderbites.nytimes.com}{Site Map}
\item
  \href{https://help.nytimes.com/hc/en-us}{Help}
\item
  \href{https://www.nytimes.com/subscription?campaignId=37WXW}{Subscriptions}
\end{itemize}
