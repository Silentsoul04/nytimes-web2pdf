Sections

SEARCH

\protect\hyperlink{site-content}{Skip to
content}\protect\hyperlink{site-index}{Skip to site index}

\href{https://www.nytimes.com/section/world/middleeast}{Middle East}

\href{https://myaccount.nytimes.com/auth/login?response_type=cookie\&client_id=vi}{}

\href{https://www.nytimes.com/section/todayspaper}{Today's Paper}

\href{/section/world/middleeast}{Middle East}\textbar{}As Lebanon Reels,
Long-Awaited Hariri Assassination Verdicts Loom

\href{https://nyti.ms/33E4y7J}{https://nyti.ms/33E4y7J}

\begin{itemize}
\item
\item
\item
\item
\item
\end{itemize}

Beirut Explosion

\begin{itemize}
\tightlist
\item
  \href{https://www.nytimes.com/2020/08/05/world/middleeast/beirut-explosion-what-happened.html?action=click\&pgtype=Article\&state=default\&region=TOP_BANNER\&context=storylines_menu}{What
  We Know}
\item
  \href{https://www.nytimes.com/2020/08/05/video/beirut-explosion-footage.html?action=click\&pgtype=Article\&state=default\&region=TOP_BANNER\&context=storylines_menu}{Footage
  of the Blast}
\item
  \href{https://www.nytimes.com/2020/08/05/world/middleeast/beirut-explosion-ammonium-nitrate.html?action=click\&pgtype=Article\&state=default\&region=TOP_BANNER\&context=storylines_menu}{What
  is Ammonium Nitrate?}
\item
  \href{https://www.nytimes.com/interactive/2020/08/04/world/middleeast/beirut-explosion-damage.html?action=click\&pgtype=Article\&state=default\&region=TOP_BANNER\&context=storylines_menu}{Mapping
  the Damage}
\end{itemize}

Advertisement

\protect\hyperlink{after-top}{Continue reading the main story}

Supported by

\protect\hyperlink{after-sponsor}{Continue reading the main story}

\hypertarget{as-lebanon-reels-long-awaited-hariri-assassination-verdicts-loom}{%
\section{As Lebanon Reels, Long-Awaited Hariri Assassination Verdicts
Loom}\label{as-lebanon-reels-long-awaited-hariri-assassination-verdicts-loom}}

A U.N.-backed court will soon pronounce verdicts in a 15-year-old
bombing in Beirut that roiled the Middle East. But critics say the
court's protracted deliberations and huge expense have undermined its
original purpose.

\includegraphics{https://static01.nyt.com/images/2020/08/07/world/07hariri2/merlin_175381473_a6263c94-9fa7-4b84-9434-e88ab2bbd0cf-articleLarge.jpg?quality=75\&auto=webp\&disable=upscale}

By \href{https://www.nytimes.com/by/marlise-simons}{Marlise Simons} and
\href{https://www.nytimes.com/by/vivian-yee}{Vivian Yee}

\begin{itemize}
\item
  Aug. 8, 2020Updated 4:13 a.m. ET
\item
  \begin{itemize}
  \item
  \item
  \item
  \item
  \item
  \end{itemize}
\end{itemize}

The blast ripped off balconies along the Mediterranean, smashed windows
blocks away and echoed across Beirut, leaving a city shattered by the
immeasurable loss.

It happened 15 years, five months and three weeks ago, when Rafik
Hariri, Lebanon's former prime minister,
\href{https://www.nytimes.com/2005/02/14/world/africa/former-prime-minister-rafik-hariri-killed-in-explosion.html}{was
assassinated along with 21 others} by a suicide bomber in an
explosives-packed van that devastated the waterfront of the Lebanese
capital and roiled the Middle East.

Now, as Lebanon's 6.8 million people grapple with the trauma of the
enormous explosions on Tuesday that killed more than 150 people and
leveled wide stretches of Beirut, they are also bracing for the verdicts
in Mr. Hariri's assassination from \href{https://www.stl-tsl.org/en}{a
special U.N.-backed court} in the Netherlands.

But just as few people in Lebanon trust their government to hold
officials to account for this week's blasts, almost no one is expecting
the full truth about the massacre of Mr. Hariri and his entourage on
Valentine's Day in 2005.

Already, in the aftermath of the latest explosions, political factions
are bickering over whether to call for an international investigation
along the lines of the one into Mr. Hariri's assassination.

The Hariri proceedings cost nearly \$700 million, took many years and
became a virtual industry unto itself, with a staff of nearly 400 and 11
full-time judges --- all for a trial never even attended by the four
defendants. They are all low-level operatives of Hezbollah, the militant
Lebanese Shiite political organization. Their whereabouts is unknown and
they were tried in absentia.

Even more fundamentally, prosecutors have not addressed the basic
underlying question of who --- or which government, if any --- ordered
the attack and why.

\includegraphics{https://static01.nyt.com/images/2020/08/07/world/07hariri10/07hariri10-articleLarge.jpg?quality=75\&auto=webp\&disable=upscale}

The case, much like the blasts that devastated Beirut this week, is a
searing example of the debilitating lack of accountability,
\href{https://www.nytimes.com/2020/08/05/world/middleeast/beirut-explosion-lebanon.html}{government
dysfunction} and volatile political divisions that have long plagued
Lebanon.

Even before the explosions on Tuesday, the country had been reeling from
enormous debts, a
\href{https://www.nytimes.com/2020/05/10/world/middleeast/lebanon-economic-crisis.html}{precipitous
economic crisis}, corruption, the coronavirus pandemic and the burden of
absorbing more than a million war refugees from Syria.

Then came the tremendous shock wave that swept across the city.
Officials have attributed its terrifying force to a giant stockpile of
highly explosive material that the government
\href{https://www.nytimes.com/2020/08/05/world/middleeast/beirut-explosion-lebanon.html}{had
neglected} for years, allowing it to sit in a dense urban area despite
the obvious risks.

President Michel Aoun said the authorities would examine ``whether the
explosion was a result of negligence or an accident'' and ``the
possibility that there was external interference,'' including a bomb or
other deliberate act.

But, just as with the Hariri assassination, this week's tragedy has
inflamed Lebanon's deep political divides. On Friday, Hassan Nasrallah,
the secretary-general of Hezbollah, angrily denied speculation that the
blasts may have been caused by a weapons cache belonging to the group.

``Several factions who oppose Hezbollah have started spreading lies that
the hangar is a weapons, missile or ammunitions depot,'' Mr. Nasrallah
said, saying the intent was to ``terrorize the Lebanese people and paint
Hezbollah as responsible for the disaster that befell them.''

The same kind of divisions have loomed over the Hariri case since the
beginning. Hezbollah has dismissed the court as a tool of its enemies,
Israel and the United States. Its leader, Mr. Nasrallah, warned against
cooperating with the tribunal and threatened to go after any followers
who did.

The court had been planning to announce verdicts on Friday but it
delayed them until Aug. 18 because of the explosions. But whatever the
outcome, it will fail to solve one of the most important cases in the
nation's recent history.

At the time of his assassination, Mr. Hariri, a billionaire businessman
and former prime minister with high-placed friends in France and Saudi
Arabia, was clashing with President Bashar al-Assad of Syria, whose
country's military had been occupying Lebanon for nearly three decades.

Mr. Hariri had been seeking to end Syria's domination. He also disliked
Hezbollah's close links with Syria and Iran.

Parliamentary elections were looming and Mr. Hariri, the country's
dominant Sunni Muslim politician, had been likely to return as prime
minister.

Immediately after the assassination, suspicion fell on Syria. An early
United Nations investigation pointed to the involvement of Syrian high
officials and their Lebanese associates.

Image

Visitors paying their respects at the tomb of Mr. Hariri in
2010.Credit...Bryan Denton for The New York Times

Under enormous international pressure,
\href{https://www.nytimes.com/2005/04/25/world/africa/syria-troops-end-29-years-of-presence-in-lebanon.html?searchResultPosition=8}{Syria
withdrew from Lebanon} two months later. But the possible roles of Syria
and Iran in the assassination were enormously difficult to prove and
were not examined at the trial, a lapse that was widely criticized.

``The most shocking thing about the case is how little was invested into
finding out who ordered and planned the assassination and who had an
interest in killing Hariri,'' said
\href{https://www.doughtystreet.co.uk/barristers/dr-guenael-mettraux}{Guénaël
Mettraux}, a jurist appointed by the court as a defense lawyer. ``There
is a murder but no one with a motive.''

Instead, the prosecution focused on the activities of the four low-level
defendants: Salim Jamil Ayyash, Hassan Habib Merhi, Hussein Hassan
Oneissi and Assad Hassan Sabra. All were linked by investigators to
Hezbollah. Records of their mobile telephone data placed the defendants
close to the bombing. Their phones went silent immediately afterward.

A fifth suspect, the highest ranking, was dropped from the indictment
after he was killed in Syria. The suspect, Mustafa Badreddine, was the
head of Hezbollah's military wing and a close confidant of Mr.
Nasrallah.

Fear among Lebanese officials that a trial could not be held safely in
Beirut led to the creation of the court, known as the Special Tribunal
for Lebanon, formed in 2009 under a resolution of the United Nations
Security Council.

Image

Anti-Syria protesters in Martys' Square in Beirut, one month after the
assassination of Mr. Hariri.Credit...Norbert Schiller for The New York
Times

With a mandate to investigate crimes of terrorism based on Lebanese law,
the court was assigned a mixed Lebanese and international staff. Not
being a U.N. body, half its budget was paid by Lebanon and half by
mostly Western governments, including France and the United States,
which had supported the creation of the court.

The difficulties were apparent from the start. Investigators sent by the
United Nations had to work under heavy security in a country where
bombings were routine. Witnesses feared testifying; some recanted or
disappeared. Detlev Mehlis, a German prosecutor sent by the United
Nations soon after the killing, reported that his work had been
frustrated by the Syrian authorities, who had denied any involvement.

Mr. Mehlis identified close to 20 suspects, including
\href{https://www.nytimes.com/2005/09/02/world/4-highly-placed-lebanese-are-charged-in-killing-of-former-premier.html}{four
senior Lebanese security officers} and top Syrian officials. But then
Mr. Mehlis left the investigation, having been warned that U.N.
officials could no longer guarantee his security. Some suspected that
his inquiry, rather than resolving the case, risked inflaming conflict
among Lebanon's Shiite and Sunni factions.

Mr. Mehlis's successors were increasingly focused, it seemed, on crime
scene forensics.

As the tribunal opened 11 years ago, lawyers close to the prosecution
said that evidence about the role of senior Lebanese or Syrian
officials, though widely reported, had not risen to the level required
at trial. A pretrial judge shocked many by
\href{https://www.nytimes.com/2009/04/30/world/middleeast/30lebanon.html?searchResultPosition=5}{ordering
the release of the four high-ranking Lebanese security officers
implicated by Mr. Mehlis, citing a lack of evidence.}

Image

A man shouting for help for the wounded at the car bombing
scene.Credit...Mohamed Azakir/Reuters

``I believe there was a desire not to get to the bottom of the killing
for political reasons,'' said
\href{https://carnegie-mec.org/experts/1258}{Michael Young}, a senior
editor with the Carnegie Middle East Center in Beirut who wrote about
the assassination. ``Important information was available in the first
few years.''

Norman Farrell, the current prosecutor, a Canadian, has said he hoped to
bring some form of justice, perhaps ``incomplete justice'' even without
defendants present.

Asked why the prosecution had not determined who was behind the killing,
Wajed Ramadan, a spokesman for the tribunal, replied in an email: ``A
judicial institution can only try people based on evidence that can
stand up in court.''

The trial focused overwhelmingly on technical evidence. Prosecutors
produced elaborate maps of when and where calls from the defendants'
cellphones had been made, showing a systematic tracking of Mr. Hariri's
movements. The prosecution even conducted a re-enactment
\href{https://www.france24.com/en/20101020-hariri-murder-blast-reconstructed-france-court-labanon-assasination}{of
the explosion} on a military base in southern France.

Image

The international tribunal viewing a replica of the assassination
scene.Credit...Toussaint Kluiters/Agence France-Presse --- Getty Images

Mr. Mettraux, who now teaches law at the
\href{http://www.nuigalway.ie/irish-centre-human-rights/}{Irish Center
for Human Rights} at Galway, said the underlying goal of a court
prosecution in the Hariri case was unrealistic.

``We defense lawyers contributed to making it look like a real trial,''
he said. ``We had to argue, but we had no real evidence the accused were
even alive.''

Early supporters of the tribunal had said that its aim had been to
empower the judiciary and introduce a new era of accountability in a
country, and a region, with a history of settling political disputes by
assassination. But opinions quickly divided as opponents of Mr. Hariri
denounced it as a tool to attack Syria and Iran.

\href{https://www.aub.edu.lb/fas/soam/soan/Pages/Sari-Hanafi.aspx}{Dr.
Sari Hanafi,} a sociologist at the American University of Beirut who has
studied Lebanese perceptions of the tribunal, said the polarization
surrounding the trial partly reflected Lebanon's failure to address the
trauma of its 1975-1990 civil war.

Massacres and disappearances were never fully investigated; warlords
were never prosecuted. That led many Lebanese to question the
international push to achieve justice for Mr. Hariri --- a wealthy and
privileged pro-West politician --- rather than resolve the crimes
committed during the war.

``The issue was, `Why Hariri, and no one before Hariri?''' he said.

Holding the trial far away in the Netherlands, with many protected
witnesses testifying behind closed doors, may have further dimmed its
reputation among Lebanese.

For some critics of the expanding field of international justice, the
Lebanon Tribunal has thrown fresh doubts on the efficacy of creating
costly special institutions to deal with distant and complex crimes.

In this case, an inconclusive outcome was produced and the defendants
were absent.

Image

Saad Hariri, Mr. Hariri's son, and then the prime minister of Lebanon,
outside the tribunal in 2018.Credit...Pool photo by Bas Czerwinski

``This was a disproportionate use of resources, given the small group of
people killed, compared to atrocities elsewhere in the world,'' said
\href{https://www.mdx.ac.uk/about-us/our-people/staff-directory/profile/schabas-william}{William
Schabas,} a law professor at Middlesex University in London. ``It will
ultimately be symbolic because no one found guilty can be punished. And
if they are found, they will have to be tried all over again.''

Image

A billboard of Mr. Hariri in Beirut in 2011.Credit...Grace
Kassab/Associated Press

Advertisement

\protect\hyperlink{after-bottom}{Continue reading the main story}

\hypertarget{site-index}{%
\subsection{Site Index}\label{site-index}}

\hypertarget{site-information-navigation}{%
\subsection{Site Information
Navigation}\label{site-information-navigation}}

\begin{itemize}
\tightlist
\item
  \href{https://help.nytimes.com/hc/en-us/articles/115014792127-Copyright-notice}{©~2020~The
  New York Times Company}
\end{itemize}

\begin{itemize}
\tightlist
\item
  \href{https://www.nytco.com/}{NYTCo}
\item
  \href{https://help.nytimes.com/hc/en-us/articles/115015385887-Contact-Us}{Contact
  Us}
\item
  \href{https://www.nytco.com/careers/}{Work with us}
\item
  \href{https://nytmediakit.com/}{Advertise}
\item
  \href{http://www.tbrandstudio.com/}{T Brand Studio}
\item
  \href{https://www.nytimes.com/privacy/cookie-policy\#how-do-i-manage-trackers}{Your
  Ad Choices}
\item
  \href{https://www.nytimes.com/privacy}{Privacy}
\item
  \href{https://help.nytimes.com/hc/en-us/articles/115014893428-Terms-of-service}{Terms
  of Service}
\item
  \href{https://help.nytimes.com/hc/en-us/articles/115014893968-Terms-of-sale}{Terms
  of Sale}
\item
  \href{https://spiderbites.nytimes.com}{Site Map}
\item
  \href{https://help.nytimes.com/hc/en-us}{Help}
\item
  \href{https://www.nytimes.com/subscription?campaignId=37WXW}{Subscriptions}
\end{itemize}
