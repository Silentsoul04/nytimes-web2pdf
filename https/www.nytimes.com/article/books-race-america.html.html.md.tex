Sections

SEARCH

\protect\hyperlink{site-content}{Skip to
content}\protect\hyperlink{site-index}{Skip to site index}

\href{https://www.nytimes.com/section/books}{Books}

\href{https://myaccount.nytimes.com/auth/login?response_type=cookie\&client_id=vi}{}

\href{https://www.nytimes.com/section/todayspaper}{Today's Paper}

\href{/section/books}{Books}\textbar{}`Every Work of American Literature
Is About Race': Writers on How We Got Here

\url{https://nyti.ms/3eFgj0w}

\begin{itemize}
\item
\item
\item
\item
\item
\item
\end{itemize}

\href{https://www.nytimes.com/spotlight/at-home?action=click\&pgtype=Article\&state=default\&region=TOP_BANNER\&context=at_home_menu}{At
Home}

\begin{itemize}
\tightlist
\item
  \href{https://www.nytimes.com/2020/08/03/well/family/the-benefits-of-talking-to-strangers.html?action=click\&pgtype=Article\&state=default\&region=TOP_BANNER\&context=at_home_menu}{Talk:
  To Strangers}
\item
  \href{https://www.nytimes.com/2020/08/01/at-home/coronavirus-make-pizza-on-a-grill.html?action=click\&pgtype=Article\&state=default\&region=TOP_BANNER\&context=at_home_menu}{Make:
  Grilled Pizza}
\item
  \href{https://www.nytimes.com/2020/07/31/arts/television/goldbergs-abc-stream.html?action=click\&pgtype=Article\&state=default\&region=TOP_BANNER\&context=at_home_menu}{Watch:
  'The Goldbergs'}
\item
  \href{https://www.nytimes.com/interactive/2020/at-home/even-more-reporters-editors-diaries-lists-recommendations.html?action=click\&pgtype=Article\&state=default\&region=TOP_BANNER\&context=at_home_menu}{Explore:
  Reporters' Google Docs}
\end{itemize}

Advertisement

\protect\hyperlink{after-top}{Continue reading the main story}

Supported by

\protect\hyperlink{after-sponsor}{Continue reading the main story}

\hypertarget{every-work-of-american-literature-is-about-race-writers-on-how-we-got-here}{%
\section{`Every Work of American Literature Is About Race': Writers on
How We Got
Here}\label{every-work-of-american-literature-is-about-race-writers-on-how-we-got-here}}

Amid the most profound social upheaval since the 1960s, these novelists,
historians, poets, comedians and activists take a moment to look back to
the literature.

\includegraphics{https://static01.nyt.com/images/2020/07/05/books/review/05OnRace-Combo-03/05OnRace-Combo-03-articleLarge.jpg?quality=75\&auto=webp\&disable=upscale}

Compiled by \href{https://www.nytimes.com/by/lauren-christensen}{Lauren
Christensen}

\begin{itemize}
\item
  June 30, 2020
\item
  \begin{itemize}
  \item
  \item
  \item
  \item
  \item
  \item
  \end{itemize}
\end{itemize}

\emph{Almost 100 years ago, responding to the public outcry over the
violent drowning of a Black boy by a white mob at a public beach on Lake
Michigan, a citywide (multiracial but white-led) commission published
``The Negro in Chicago: A Study of Race Relations and a Race Riot.''}

\emph{``Centuries of the Negro slave trade and of slavery as an
institution \ldots{} placed a stamp upon the relations of the two races
which it will require many years to erase,'' the nearly 700-page 1922
report began. ``The past is of value only as it aids in understanding
the present; and an understanding of the facts of the problem \ldots{}
is the first step toward its solution.''}

\emph{As protests spread across America once more, bringing to front
pages and the forefronts of our minds ugly truths about our country that
shouldn't have been forgotten in the first place, we turn again to the
written record, to the literature. In an effort to deepen our
understanding of race and racism in America, we asked writers to share
with us the texts that have done the most to deepen theirs. Together
these histories, novels and verses have helped shape our collective
consciousness of a subject that is irreducible, and universal.}

\hypertarget{gabriel-bump-novelist}{%
\subsection{\texorpdfstring{Gabriel Bump,
\emph{novelist}}{Gabriel Bump, novelist}}\label{gabriel-bump-novelist}}

Appreciating social movements in hindsight is a complicated endeavor.
Leaders like Martin Luther King Jr. and Harriet Tubman are often
whitewashed to appease modern sensibilities. Some, like
\href{https://www.nytimes.com/1987/08/25/obituaries/bayard-rustin-is-dead-at-75-pacifist-and-a-rights-activist.html}{Bayard
Rustin}, are almost forgotten entirely.

I came across John D'Emilio's
\textbf{\href{https://www.nytimes.com/2003/11/09/books/the-organizer.html}{LOST
PROPHET}} (2003) by chance, in my local bookstore in college. I don't
often take notes while reading for pleasure, but this time I made a
detailed index on the back flap, marking pages and lines I wanted to
save for future reference.

For example, did you know Rustin introduced Gandhian tactics of
nonviolent protest to Dr. King? Did you know he helped organize the
first Freedom Rides, the March on Washington for Jobs and Freedom, a
boycott of segregated New York City public schools?

As a young Black man with a temper, I credit Rustin with strengthening
my faith in pacifism. He was a Black queer Quaker from Pennsylvania who
went to jail for his sexuality and his refusal to fight in American
wars. When I myself came face to face with armored cops a few weeks ago,
I felt a powerful calm. I didn't want to lash out. I wanted them to look
me in the eye and see I wasn't afraid; I wasn't going to move. They
could attack me, arrest me, shoot me with gas and rubber, and still, I
wasn't going to move.

\hypertarget{chimamanda-ngozi-adichie-novelist-and-essayist}{%
\subsection{\texorpdfstring{Chimamanda Ngozi Adichie, \emph{novelist and
essayist}}{Chimamanda Ngozi Adichie, novelist and essayist}}\label{chimamanda-ngozi-adichie-novelist-and-essayist}}

\textbf{\href{https://www.nytimes.com/2019/02/07/books/review/richard-gergel-unexampled-courage.html}{UNEXAMPLED
COURAGE}} (2019), by Richard Gergel, is a remarkable book. In clear and
elegant prose, Gergel --- a United States district judge in South
Carolina --- strips legal cases of jargon and presents them as what they
essentially are: human drama. The result is intellectually and
emotionally satisfying. The author's loving admiration of Judge J.
Waties Waring --- who descended from slave-owning Confederates, but
turned his back on ``the doctrine of white supremacy'' in his courtroom
--- is obvious, but the evidence for it is rendered so dispassionately
that it feels apt. As I finished it I felt deeply moved all around, but
mostly by the incredible courage of the ordinary Black Americans in the
1951 Briggs v. Elliott case, over school segregation. Meticulously
researched and full of heart, this book is important at this time when
the United States is confronting its ever-present past.

\hypertarget{clint-smith-poet}{%
\subsection{\texorpdfstring{Clint Smith,
\emph{poet}}{Clint Smith, poet}}\label{clint-smith-poet}}

Ira Katznelson's
\textbf{\href{https://www.nytimes.com/2005/08/28/books/review/when-affirmative-action-was-white-uncivil-rights.html}{WHEN
AFFIRMATIVE ACTION WAS WHITE}} (2005) was one of the first books that
helped me concretely understand how racism was embedded into federal
policy. In my American history classes growing up, the New Deal had been
celebrated as the great catalyst of intergenerational opportunity and
wealth for millions across the country. And it was. What I had not been
taught, however, was how New Deal legislation was intentionally crafted
to prevent millions of Black Americans from having access to its
benefits. As Katznelson outlines, in the 1930s, 75 percent of Black
workers in the South were employed as either maids or farmworkers.
People in those professions were excluded for decades from social
programs that set the minimum wage, regulated work hours, created labor
unions and Social Security --- which is to say, the programs that were
the economic bedrock for millions of White Americans.

\hypertarget{sandra-cisneros-novelist}{%
\subsection{\texorpdfstring{Sandra Cisneros,
\emph{novelist}}{Sandra Cisneros, novelist}}\label{sandra-cisneros-novelist}}

Roxanne Dunbar-Ortiz's \textbf{AN INDIGENOUS PEOPLES' HISTORY OF THE
UNITED STATES} (2014) helped me clarify my place in this country. It
confirmed what had been told to me by my ancestors: that Indigenous
peoples, from the North Pole to the South, have been here since before
the world was known as round. As a conquering nation, the United States
has rewritten history to make people of the U.S. forget our past as
natives to this land. This is especially apparent in the Mexi-phobic,
immigrant-phobic policies of our time.

If every politician were to read and understand this book --- which
should be required in every high school curriculum --- we would have a
reconsideration of our current border policies and our practice of
detaining human beings in cages. Our present is made of a past of
genocide and colonialism. This book is necessary reading if we are to
move into a more humane future.

\hypertarget{reginald-dwayne-betts-poet}{%
\subsection{\texorpdfstring{Reginald Dwayne Betts,
\emph{poet}}{Reginald Dwayne Betts, poet}}\label{reginald-dwayne-betts-poet}}

Right now I'm rereading Marlon James's
\textbf{\href{https://www.nytimes.com/2019/01/31/books/review/black-leopard-red-wolf-marlon-james.html}{BLACK
LEOPARD, RED WOLF}} (2019). Wild that fantasy is where I turn to think
about race and these states. But the book's first line, ``The child is
dead,'' begins all our troubles these days. George Floyd, still his
mother's son, calling for her with his last breath. James's book follows
in the tradition of Octavia Butler and Toni Morrison: In his world, like
our own, everyone is complicit. A few pages in, I remember that the one
telling us what has happened is in a cell --- death and prison, aren't
these the sources of this nation's discontent? Name a story more
American than that. And still, the story gives me hope. On the first
page is tragedy; the next 600 explore, challenge and critique an
encyclopedia of phobias and isms, in language so rich and a story so
compelling that you aren't even aware the reading is making you wiser
until it already has.

\includegraphics{https://static01.nyt.com/images/2020/07/05/books/review/05OnRace-Combo/05OnRace-Combo-articleLarge.jpg?quality=75\&auto=webp\&disable=upscale}

\hypertarget{desus-nice--the-kid-mero-late-night-co-hosts-and-authors}{%
\subsection{\texorpdfstring{Desus Nice \& The Kid Mero, \emph{late night
co-hosts and
authors}}{Desus Nice \& The Kid Mero, late night co-hosts and authors}}\label{desus-nice--the-kid-mero-late-night-co-hosts-and-authors}}

\textbf{Desus:} As a child, I attended a program for extremely gifted
children. We were so advanced we read Richard Wright's \textbf{BLACK
BOY} (1945) in third grade. I still have my copy and it's in shambles,
because that became my favorite book. The vivid description of Richard's
life in the South fascinated me --- and, as a young boy in the Bronx, I
could still relate. His depiction of racist situations, and being called
the ``N-word,'' hit me; because at that age I too had already been
called the ``N-word.'' That book showed me Black people have been going
through the struggles I'd been going through forever.

\textbf{Mero:} I read \textbf{TO BE A SLAVE} (1968), by
\href{https://www.nytimes.com/2018/01/19/obituaries/julius-lester-chronicler-of-black-america-is-dead-at-78.html}{Julius
Lester}, illustrated by Tom Feelings, in around seventh grade. My
teacher, shout-out to Mr. Adeghe, was from Ghana and very passionate.
All the ``American History'' classes I took prior to that were all about
Pilgrims and Native Americans shaking hands and eating turkey. This book
was like a nuclear bomb of knowledge that made me connect even more with
my Afro-Latino identity and roots.

\hypertarget{jill-leovy-journalist}{%
\subsection{\texorpdfstring{Jill Leovy,
\emph{journalist}}{Jill Leovy, journalist}}\label{jill-leovy-journalist}}

For understanding the Jim Crow South, I always recommend \textbf{AFTER
FREEDOM} (1939), based on the anthropologist
\href{https://www.nytimes.com/1970/06/17/archives/hortense-powdermaker-is-dead-an-authority-on-varied-cultures.html}{Hortense
Powdermaker}'s hard-won observations in Mississippi and the best of the
great fieldwork studies from that era. Less often read, although it
ought to be more, is Mark Schultz's \textbf{THE RURAL FACE OF WHITE
SUPREMACY: Beyond Jim Crow} (2005). Schultz spent years on this oral
history project, capturing the fascinating personal stories of elderly
black and white residents of a Georgia county who spoke candidly about
race relations in the first half of the 20th century. For contemporary
inner-city politics and violence issues, check out Cid Martinez's
\textbf{THE NEIGHBORHOOD HAS ITS OWN RULES} (2016), an astute analysis
of activist politics in Los Angeles. Although it is not about America,
\textbf{INFORMAL JUSTICE IN DIVIDED SOCIETIES} (2002), by Colin Knox and
Rachel Monaghan, helped me place our domestic race issues in a global
context. Americans tend to view this country's racial situation as
singular and distinct, but in the streets of Watts ring echoes of
Belfast and Cape Town.

\hypertarget{darryl-pinckney-novelist-and-essayist}{%
\subsection{\texorpdfstring{Darryl Pinckney, \emph{novelist and
essayist}}{Darryl Pinckney, novelist and essayist}}\label{darryl-pinckney-novelist-and-essayist}}

Among the books I have gone back to in this historic moment are:
\textbf{DARKWATER: Voices From Within the Veil} (1920), by W.E.B. Du
Bois, because of his thoughtful insights 100 years ago into the very
matters that now call people into the streets at some risk; and
\textbf{DARKNESS OVER GERMANY: A Warning From History} (1943), by E. Amy
Buller, because of what she tells us about the mass psychology of
fascism.

\hypertarget{tressie-mcmillan-cottom-sociologist-and-essayist}{%
\subsection{\texorpdfstring{Tressie McMillan Cottom, \emph{sociologist
and
essayist}}{Tressie McMillan Cottom, sociologist and essayist}}\label{tressie-mcmillan-cottom-sociologist-and-essayist}}

Race is a living, breathing thing that morphs across time and context
and even our own understanding; so the most important books that have
shaped my understanding of race are tied to who I was at the time that I
read them. The list will change as I change, and that is as it should
be.

I read Anne Moody's \textbf{COMING OF AGE IN MISSISSIPPI} (1969) as a
child, before I knew what memoir was. But still the book resonated with
me, as I already understood what it meant to be a Black girl in a world
where race and gender circumscribed who we could become. As a young
adult, I read
\textbf{\href{https://www.nytimes.com/1996/06/23/books/an-accidental-family.html}{A
FINE BALANCE}} (1996), by Rohinton Mistry, and for the first time
understood that racism in the United States has genealogies other than
the global slave trade. (I immediately signed up for courses on South
Asian studies at my historically Black college.) As an adult, I think of
Derrick Bell's science-fiction story \textbf{THE SPACE TRADERS} (1992)
at least once a week, mostly wishing everyone else had also read it so
that we could stop reliving its message. Finally, there is no book more
important to understanding the underpinnings of race, racism and
uprisings right now than a new book by William A. Darity Jr. and A.
Kirsten Mullen, \textbf{FROM HERE TO EQUALITY} (2020). ** Part history
and part social policy, it takes economic reparations for Black
Americans seriously. I wish we could say the same for America.

\hypertarget{henry-louis-gates-jr-historian-and-literary-critic}{%
\subsection{\texorpdfstring{Henry Louis Gates Jr., \emph{historian and
literary
critic}}{Henry Louis Gates Jr., historian and literary critic}}\label{henry-louis-gates-jr-historian-and-literary-critic}}

Few reading experiences on the history of race in America have been as
profound for me as the works of Eric Foner. From
\textbf{\href{https://www.nytimes.com/1988/05/22/books/a-moment-of-terrifying-promise.html}{RECONSTRUCTION}}
(1988), ** his definitive study of the era, to last year's tour de force
on the trio of constitutional amendments that established
\textbf{\href{https://www.nytimes.com/2019/09/18/books/review/the-second-founding-eric-foner.html}{THE
SECOND FOUNDING}} after the Civil War, no one has done more since W.E.B.
Du Bois's \textbf{BLACK RECONSTRUCTION IN AMERICA} (1935) to refute the
racist fabrications of previous generations of Lost Cause ``scholars.''
In rescuing the facts about the promise and violent overthrow of our
country's most thrilling experiment in interracial democracy, Foner has
proved that no one set of historians has the final word. ``The slave
went free; stood a brief moment in the sun; then moved back again toward
slavery,'' Du Bois wrote --- succinctly, poetically and so very sadly
--- of the period of Reconstruction's nakedly racist rollback
(perversely named ``Redemption'') that ushered in nearly a century of
Jim Crow.

I'm also inspired by a new generation of scholars --- from Kimberlé
Crenshaw's \textbf{CRITICAL RACE THEORY} (1995) to Martha Jones's
\textbf{VANGUARD} (2020) --- who are shining a light on this crucial
chapter in our story, pointing out its harbingers in earlier efforts to
circumvent the 13th, 14th and 15th Amendments (especially voter
suppression). As dark and unsettling forces attempt to roll back the
gains of what historians sometimes call ``The Second Reconstruction'' of
the 1960s, and as tyrannical impulses seek to curtail our most
foundational and sacred constitutional rights, let us look to these
examples of great scholarship, which preserve the noble tale of the
triumphant determination of black people to rise undiminished out of the
ashes of racial repression, violence and lynching.

Image

Clockwise from left: Derrick Bell, the first tenured black professor at
Harvard Law School, took an unpaid leave of absence in 1990 to protest
the school's racist hiring practices; the author Zora Neale Hurston,
November 1934; Black Lives Matter protesters march the length of
Manhattan, June 3, 2020.Credit...Steve Liss/The LIFE Images Collection,
via Getty Images; Beinecke Library, Yale University, Van Vechten Trust;
Ashley Gilbertson for The New York Times

\hypertarget{danzy-senna-novelist}{%
\subsection{\texorpdfstring{Danzy Senna,
\emph{novelist}}{Danzy Senna, novelist}}\label{danzy-senna-novelist}}

Every work of American literature is about race, whether the writer
knows it or not. That said, these are some nonfiction books that have
given me necessary tools to think about our culture. In college I read
both
\href{https://www.nytimes.com/2019/02/28/books/bell-hooks-min-jin-lee-aint-i-a-woman.html}{bell
hooks}'s \textbf{BLACK LOOKS} (1992) ** and Donald Bogle's
\textbf{\href{https://www.nytimes.com/1973/06/09/archives/greater-expectations-books-of-the-times-liberation-from-illusions.html}{TOMS,
COONS, MULATTOES, MAMMIES \& BUCKS}} (1973), and was never the same.
Toni Morrison's
\textbf{\href{https://www.nytimes.com/1992/04/05/books/the-clearest-eye.html}{PLAYING
IN THE DARK}} (1992) is utter genius, revealing through literary
analysis how whiteness doesn't exist without blackness.
\href{https://www.nytimes.com/2001/02/18/books/on-writers-and-writing-authentic-american.html}{Nella
Larsen}'s \textbf{QUICKSAND} (1928) and \textbf{PASSING} (1929), both
published during the Harlem Renaissance, feel just as contemporary and
lucid today in their portrayal of mixed-race women and the perils of
white passing. More recently, I have been enamored by the brilliance of
both Hilton Als's
\textbf{\href{https://www.nytimes.com/2013/11/10/books/review/white-girls-by-hilton-als.html}{WHITE
GIRLS}} (2013) and Margo Jefferson's
\textbf{\href{https://www.nytimes.com/2015/09/20/books/review/margo-jeffersons-negroland-a-memoir.html}{NEGROLAND}}
(2015)\emph{.}

\hypertarget{mitchell-duneier-sociologist}{%
\subsection{\texorpdfstring{Mitchell Duneier,
\emph{sociologist}}{Mitchell Duneier, sociologist}}\label{mitchell-duneier-sociologist}}

Gunnar Myrdal's massive sociological study \textbf{AN AMERICAN DILEMMA}
(1944) saw ``the Negro problem'' as something that could never be
understood through data about black living conditions alone, but as a
phenomenon of the majority's power. It was a moral situation in which
conflicting values were held both within the white population and,
importantly, within white individuals themselves.

If change did not come about, Myrdal predicted uprisings. ``America can
never more regard its Negroes as a patient, submissive minority,'' he
writes. ``They will organize for defense and offense. \ldots{} They have
the advantage that they can fight wholeheartedly.''

Myrdal perceptively noted that the average white Northerner did not
understand racism as something in which he or she was taking part every
day. But he also argued that whites were deeply troubled by the
contradiction between their egalitarian principles and their attitude
toward black citizens. This was the ``American dilemma.''

There is much value in this big book, but even more to be learned today
from Myrdal's naïveté. By the time of the civil rights movement, it had
become clear to a new generation of critics that the whites Myrdal had
interviewed --- perhaps like many today who are rushing to issue public
statements or participating in multiracial rallies --- were still
perfectly capable of compartmentalizing words and deeds, living with
moral dissonance.

\hypertarget{valeria-luiselli-novelist-and-essayist}{%
\subsection{\texorpdfstring{Valeria Luiselli, \emph{novelist and
essayist}}{Valeria Luiselli, novelist and essayist}}\label{valeria-luiselli-novelist-and-essayist}}

So-called third-world problems --- hunger, poverty, violence --- are
often explained as a result of particular ``political cultures'' endemic
to specific nations. But while partly true, that narrative is also
reductionist. All countries interact with other countries, and most
``developing'' nations have to sustain unequal relations with larger
powers that systematically abuse them through military interventions,
economic sanctions or unequal treatises. The borderlands between Mexico
and the United States are a clear and poignant example of this
interrelatedness. In \textbf{THE FEMICIDE MACHINE} (2012), Sergio
González Rodríguez focuses on Ciudad Juárez, across the border from El
Paso, Texas, where the rate of femicides started escalating drastically
in the early 1990s, after decades of mutually accorded industrialization
programs that resulted in NAFTA. ``The Femicide Machine'' discusses the
politics of killing women for being women not within the oversimplifying
framework of Mexican culture alone, but as a result of the economic
interactions between Mexico and the U.S., and the geopolitical
conditions that fuel them.

\hypertarget{eddie-s-glaude-jr-professor}{%
\subsection{\texorpdfstring{Eddie S. Glaude Jr.,
\emph{professor}}{Eddie S. Glaude Jr., professor}}\label{eddie-s-glaude-jr-professor}}

I find myself these days reaching for James Baldwin's \textbf{NO NAME IN
THE STREET} (1972), his first book after Dr. King's assassination, which
broke him. Shadowed by grief and trauma, this memoir is as fragmented as
Baldwin's memories. ``Much, much, much has been blotted out,'' he
writes, ``coming back only lately in bewildering and untrustworthy
flashes.'' The book is also Baldwin's attempt to come to terms with
America's latest betrayal of Black people, and his effort to muster the
energy and the faith to keep fighting --- to give Black people the
language to keep fighting. The prose is angry, because Baldwin is
profoundly wounded. If ``The Fire Next Time'' (1963) was prophetic, ``No
Name in the Street'' was the reckoning.

\hypertarget{albert-woodfox-activist}{%
\subsection{\texorpdfstring{Albert Woodfox,
\emph{activist}}{Albert Woodfox, activist}}\label{albert-woodfox-activist}}

I read \textbf{THE NATURE OF PREJUDICE} (1954), by Gordon W. Allport,
sometime in the '70s, while I was in prison. This book had the greatest
impact on my ability to understand the difference between prejudice and
racism. Prejudice is a normal reaction to the unknown. Racism is a
premeditated sickness.

\hypertarget{kerri-greenidge-historian}{%
\subsection{\texorpdfstring{Kerri Greenidge,
\emph{historian}}{Kerri Greenidge, historian}}\label{kerri-greenidge-historian}}

Harriet Jacobs's \textbf{INCIDENTS IN THE LIFE OF A SLAVE GIRL} (1861),
for its sophisticated critique of slavery, 19th-century feminism and the
gendered nature of white supremacy. Paula Giddings's \textbf{WHEN AND
WHERE I ENTER} (1984), for the groundbreaking nature of her research,
and because in the post-Kerner Commission era, when black women were
blamed for ``the state of the black family'' and stereotyped as
``welfare queens,'' Giddings provided historical context for
understanding the black women I knew as a child. Robin D.G. Kelley's
\textbf{HAMMER AND HOE} (1990) managed to contextualize black radical
politics within a rural, Southern and Marxist framework, challenging the
liberal argument that civil rights was concerned only with integrated
lunch counters, not the dismantling of global racial capitalism.

\hypertarget{carol-anderson-historian}{%
\subsection{\texorpdfstring{Carol Anderson,
\emph{historian}}{Carol Anderson, historian}}\label{carol-anderson-historian}}

Jesmyn Ward's
\textbf{\href{https://www.nytimes.com/2017/09/05/books/review-sing-unburied-sing-jesmyn-ward.html}{SING,
UNBURIED, SING}} (2017); Isabel Wilkerson's
\textbf{\href{https://www.nytimes.com/2010/09/05/books/review/Oshinsky-t.html}{THE
WARMTH OF OTHER SUNS}} (2010); Kiese Laymon's
\textbf{\href{https://www.nytimes.com/2018/11/13/books/review/kiese-laymon-heavy.html}{HEAVY}}
(2018); David Oshinsky's \textbf{WORSE THAN SLAVERY} (1996); Claudia
Rankine's
\textbf{\href{https://www.nytimes.com/2014/12/28/books/review/claudia-rankines-citizen.html}{CITIZEN}}
(2014); J. Mills Thornton's \textbf{DIVIDING LINES} (2002); John W.
Dower's \textbf{WAR WITHOUT MERCY} (1986); Patrick Phillips's
\textbf{\href{https://www.nytimes.com/2016/09/15/books/review-blood-at-the-root-a-tale-of-racial-cleansing-close-to-home.html}{BLOOD
AT THE ROOT}} (2016); Françoise Hamlin's \textbf{CROSSROADS AT
CLARKSDALE} (2012); Joshua Bloom and Waldo E. Martin Jr.'s \textbf{BLACK
AGAINST EMPIRE} (2013).

Each speaks, in some way, to the power of racism, and sometimes just
sheer, raw, unadulterated anti-blackness, in destroying millions upon
millions of lives. Each also lays out the power of the refusal to accept
subjugation. And that the subsequent and ongoing battles between
anti-blackness and freedom are messy.

Image

Protesters gather for a silent march outside the Brooklyn Museum, June
14, 2020.Credit...Demetrius Freeman for The New York Times

\hypertarget{morgan-jerkins-essayist-and-memoirist}{%
\subsection{\texorpdfstring{Morgan Jerkins, \emph{essayist and
memoirist}}{Morgan Jerkins, essayist and memoirist}}\label{morgan-jerkins-essayist-and-memoirist}}

\textbf{TELL MY HORSE: Voodoo and Life in Haiti and Jamaica} (1938), by
Zora Neale Hurston (1938): Determined to tell the stories of Black
people outside of distant, scientific analysis, Hurston writes of her
experiences of spiritual practices in these two Caribbean nations.

\textbf{SING, UNBURIED, SING,} by Jesmyn Ward: This is one of the best
novels I've ever read. Set in Mississippi, it involves an odyssey to the
notorious Parchman Farm penitentiary to pick up a lover; a ghost who
haunts an elderly former inmate at said prison; and a young boy who
observes it all.

\textbf{\href{https://www.nytimes.com/2020/02/17/books/review-minor-feelings-cathy-park-hong.html}{MINOR
FEELINGS}} (2020), by Cathy Park Hong: This wonderfully crafted essay
collection is a necessary read for those who want to understand
Asian-American experiences --- as well as immigration and migration,
intergenerational trauma and even anti-Blackness.

\textbf{\href{https://www.nytimes.com/2019/02/12/books/review/thick-tressie-mcmillan-cottom.html}{THICK}}
(2019), by Tressie McMillan Cottom: In her second book, the sociologist,
a savant and wordsmith, addresses the intersections of race, gender and
class with enviable grace and confidence.

\textbf{\href{https://www.nytimes.com/2018/11/30/t-magazine/black-women-writers.html}{CANNIBAL}}
(2016), by Safiya Sinclair: One of my favorite poetry collections.
Sinclair covers so much ground: her Jamaican background, spirituality,
womanhood, America, race relations. She laces words together in a
beautiful tapestry, full of history, life, death and, most of all,
renewal.

\hypertarget{natalie-diaz-poet}{%
\subsection{\texorpdfstring{Natalie Diaz,
\emph{poet}}{Natalie Diaz, poet}}\label{natalie-diaz-poet}}

Reading is a way of practicing the imagination necessary to broaden our
capacities to understand ourselves and others. These four books
constellate conversations that have long been held separate from one
another, lest their accumulation create an energy perilous to the
colony: Simone Browne's \textbf{DARK MATTERS} (2017), Dina
Gilio-Whitaker's \textbf{AS LONG AS GRASS GROWS} (2019), Dolores
Dorantes's \textbf{STYLE} (2016) and Mahmoud Darwish's \textbf{JOURNAL
OF AN ORDINARY GRIEF} (1973). Each book exists separately within its own
conditions, while taking on exponential meaning in relation to one
another.

\hypertarget{david-treuer-novelist-and-historian}{%
\subsection{\texorpdfstring{David Treuer, \emph{novelist and
historian}}{David Treuer, novelist and historian}}\label{david-treuer-novelist-and-historian}}

\textbf{\href{https://archive.nytimes.com/www.nytimes.com/books/98/01/11/home/14013.html?mcubz=3}{BELOVED}}
(1987), by Toni Morrison: I read this when I was 19. No book, no matter
the intelligence behind it, can put the reader into the position of
unfreedom in which African-Americans lived as enslaved people. Morrison,
I think, knew this. What ``Beloved'' taught me to see and to feel was
what it might be like to have the things we think of as universally
human --- in this case, a mother's love for her children --- twisted and
deformed by the institution and experience of slavery.

\textbf{\href{https://www.nytimes.com/1969/11/09/archives/custer-died-for-your-sins-an-indian-manifesto-by-vine-deloria-jr.html}{CUSTER
DIED FOR YOUR SINS}} (1969): In this essay collection, the lawyer and
activist Vine Deloria Jr. shouts, chides, teases and preaches about the
pain and absurdity of being Native American in a modern world.

\textbf{NOTES OF A NATIVE SON} (1958), by James Baldwin: No other writer
has written as lucidly, powerfully and productively about what it means
to be black in America --- and, as a result, what this country means.

\textbf{\href{https://www.nytimes.com/1970/03/08/archives/one-hundred-years-of-solitude-memory-and-prophecy-illusion-and.html}{ONE
HUNDRED YEARS OF SOLITUDE}} (1967), by Gabriel García Márquez: Billed as
a Latin American fantasy, the Colombian-Mexican author's magical-realist
epic is as much an American fantasy, about the lives caught in the web
of 19th- and 20th-century colonialism.

\hypertarget{thomas-chatterton-williams-memoirist-and-critic}{%
\subsection{\texorpdfstring{Thomas Chatterton Williams, \emph{memoirist
and
critic}}{Thomas Chatterton Williams, memoirist and critic}}\label{thomas-chatterton-williams-memoirist-and-critic}}

There are several dozen books explicitly about race in America that have
left lifelong marks on me, but only two have reversed the course of my
own thought. The first is
\href{https://www.nytimes.com/2013/08/20/books/albert-murray-essayist-who-challenged-the-conventional-dies-at-97.html}{Albert
Murray}'s \textbf{THE OMNI-AMERICANS} (1970). ** Murray's argument is
simple but profound: America is a mongrel nation, both culturally and in
its DNA. Though we may come up with all kinds of methods to obscure this
basic truth, ``any fool can see,'' he writes, ``that the white people
are not really white, and that black people are not black.''

The second is \textbf{RACECRAFT} (2012), by Barbara J. Fields and Karen
Elise Fields. The Fields sisters prove with witty, withering brilliance
that racism --- and the ideology of white supremacy, rooted in economic
exploitation --- creates race, and not the other way around.

Finally, though it's trans-Atlantic in scope, the British sociologist
\href{https://www.nytimes.com/2019/03/14/arts/paul-gilroy-holberg-prize.html}{Paul
Gilroy}'s monumental work \textbf{AGAINST RACE} (2000) argues that race
is not something intrinsic and immutable but something fluid, illusory
and imposed, ``an afterimage --- a lingering effect of looking too
casually into the damaging glare emanating from colonial conflicts at
home and abroad.''

All three books convinced me that we will never transcend racism so long
as we continue to reify the illusory, inherently hierarchical color
categories that it gives us.

\hypertarget{richard-rothstein-historian}{%
\subsection{\texorpdfstring{Richard Rothstein,
\emph{historian}}{Richard Rothstein, historian}}\label{richard-rothstein-historian}}

Whites may find it challenging to confront stereotypes and comprehend
the deeply embedded legacies of slavery and Jim Crow, but even harder is
embracing remedies, because seemingly race-neutral policies perpetuate
our racial caste system. With engaging profiles of housing advocates and
the opposition they face, Conor Dougherty's
\textbf{\href{https://www.nytimes.com/2020/02/14/books/review/golden-gates-housing-conor-dougherty.html}{GOLDEN
GATES}} (2020) focuses on California, but has lessons for all
metropolitan areas. Smugly deeming itself racially progressive, the
state allows high-wage employment (mostly for whites and educated
immigrants) to grow faster than housing supply, ensuring that priced-out
black and Hispanic families will suffer greater homelessness and
displacement to job-starved distant suburbs. Segregation increases as
voters enact local zoning codes to prevent new home-building, but those
in desperate need of housing can't register to vote in the no-growth
towns that ban them. That's structural racism.

\emph{Follow New York Times Books on}
\href{https://www.facebook.com/nytbooks/}{\emph{Facebook}}\emph{,}
\href{https://twitter.com/nytimesbooks}{\emph{Twitter}} \emph{and}
\href{https://www.instagram.com/nytbooks/}{\emph{Instagram}}\emph{, sign
up for}
\href{https://www.nytimes.com/newsletters/books-review}{\emph{our
newsletter}} \emph{or}
\href{https://www.nytimes.com/interactive/2017/books/books-calendar.html}{\emph{our
literary calendar}}\emph{. And listen to us on the}
\href{https://www.nytimes.com/column/book-review-podcast}{\emph{Book
Review podcast}}\emph{.}

Advertisement

\protect\hyperlink{after-bottom}{Continue reading the main story}

\hypertarget{site-index}{%
\subsection{Site Index}\label{site-index}}

\hypertarget{site-information-navigation}{%
\subsection{Site Information
Navigation}\label{site-information-navigation}}

\begin{itemize}
\tightlist
\item
  \href{https://help.nytimes.com/hc/en-us/articles/115014792127-Copyright-notice}{©~2020~The
  New York Times Company}
\end{itemize}

\begin{itemize}
\tightlist
\item
  \href{https://www.nytco.com/}{NYTCo}
\item
  \href{https://help.nytimes.com/hc/en-us/articles/115015385887-Contact-Us}{Contact
  Us}
\item
  \href{https://www.nytco.com/careers/}{Work with us}
\item
  \href{https://nytmediakit.com/}{Advertise}
\item
  \href{http://www.tbrandstudio.com/}{T Brand Studio}
\item
  \href{https://www.nytimes.com/privacy/cookie-policy\#how-do-i-manage-trackers}{Your
  Ad Choices}
\item
  \href{https://www.nytimes.com/privacy}{Privacy}
\item
  \href{https://help.nytimes.com/hc/en-us/articles/115014893428-Terms-of-service}{Terms
  of Service}
\item
  \href{https://help.nytimes.com/hc/en-us/articles/115014893968-Terms-of-sale}{Terms
  of Sale}
\item
  \href{https://spiderbites.nytimes.com}{Site Map}
\item
  \href{https://help.nytimes.com/hc/en-us}{Help}
\item
  \href{https://www.nytimes.com/subscription?campaignId=37WXW}{Subscriptions}
\end{itemize}
