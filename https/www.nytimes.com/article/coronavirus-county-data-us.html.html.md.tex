Sections

SEARCH

\protect\hyperlink{site-content}{Skip to
content}\protect\hyperlink{site-index}{Skip to site index}

\href{https://www.nytimes.com/section/us}{U.S.}

\href{https://myaccount.nytimes.com/auth/login?response_type=cookie\&client_id=vi}{}

\href{https://www.nytimes.com/section/todayspaper}{Today's Paper}

\href{/section/us}{U.S.}\textbar{}We're Sharing Coronavirus Case Data
for Every U.S. County

\url{https://nyti.ms/2yeqKHZ}

\begin{itemize}
\item
\item
\item
\item
\item
\end{itemize}

\href{https://www.nytimes.com/news-event/coronavirus?action=click\&pgtype=Article\&state=default\&region=TOP_BANNER\&context=storylines_menu}{The
Coronavirus Outbreak}

\begin{itemize}
\tightlist
\item
  live\href{https://www.nytimes.com/2020/08/02/world/coronavirus-updates.html?action=click\&pgtype=Article\&state=default\&region=TOP_BANNER\&context=storylines_menu}{Latest
  Updates}
\item
  \href{https://www.nytimes.com/interactive/2020/us/coronavirus-us-cases.html?action=click\&pgtype=Article\&state=default\&region=TOP_BANNER\&context=storylines_menu}{Maps
  and Cases}
\item
  \href{https://www.nytimes.com/interactive/2020/science/coronavirus-vaccine-tracker.html?action=click\&pgtype=Article\&state=default\&region=TOP_BANNER\&context=storylines_menu}{Vaccine
  Tracker}
\item
  \href{https://www.nytimes.com/interactive/2020/07/29/us/schools-reopening-coronavirus.html?action=click\&pgtype=Article\&state=default\&region=TOP_BANNER\&context=storylines_menu}{What
  School May Look Like}
\item
  \href{https://www.nytimes.com/live/2020/07/31/business/stock-market-today-coronavirus?action=click\&pgtype=Article\&state=default\&region=TOP_BANNER\&context=storylines_menu}{Economy}
\end{itemize}

Advertisement

\protect\hyperlink{after-top}{Continue reading the main story}

Supported by

\protect\hyperlink{after-sponsor}{Continue reading the main story}

\hypertarget{were-sharing-coronavirus-case-data-for-every-us-county}{%
\section{We're Sharing Coronavirus Case Data for Every U.S.
County}\label{were-sharing-coronavirus-case-data-for-every-us-county}}

With no detailed government database on where the thousands of
coronavirus cases have been reported, a team of New York Times
journalists is attempting to track every case.

By The New York Times

\begin{itemize}
\item
  March 28, 2020
\item
  \begin{itemize}
  \item
  \item
  \item
  \item
  \item
  \end{itemize}
\end{itemize}

Download county-level data for coronavirus cases in the United States
from The New York Times
\href{https://github.com/nytimes/covid-19-data}{on GitHub}.

As the coronavirus has spread across the United States, killing hundreds
of people and sickening tens of thousands more, comprehensive data on
the extent of the outbreak has been difficult to come by.

No single agency has provided the public with an accurate, up-to-date
record of coronavirus cases, tracked to the county level. To fill the
gap, The New York Times has launched a round-the-clock effort to tally
every known coronavirus case in the United States. The data, which The
Times will continue to track, is being made available to the public on
Friday.

Individual states and counties have tracked their own cases and
presented them to the public with varying degrees of speed and accuracy,
but those tallies provide only limited snapshots of the nation's
outbreak. A
\href{https://www.cdc.gov/coronavirus/2019-ncov/cases-updates/cases-in-us.html}{publicly
available tracker from the federal Centers for Disease Control and
Prevention}, updated five times a week, includes only state-level data.
Other entities have made efforts, including
\href{https://www.arcgis.com/apps/opsdashboard/index.html\#/bda7594740fd40299423467b48e9ecf6}{a
notable one by Johns Hopkins University}, to track cases worldwide or
\href{https://coronavirus.1point3acres.com/en}{within the United
States}.

In late January, not long after the first known case was reported in
Washington State, The Times began tracking each known U.S. case as
counties and states began reporting results of testing. Such testing,
which had been delayed, gradually became more widely available. For the
last eight weeks, a team of Times journalists has recorded an array of
details --- locations, dates, ages and conditions, when possible ---
about newly confirmed cases reported by state and local officials.

By Friday morning, The Times had tracked more than 85,000 cases in all
50 states, the District of Columbia and three U.S. territories. There
have been
\href{https://www.nytimes.com/2020/03/26/health/usa-coronavirus-cases.html}{more
known cases in the United States than in China, Italy or any other
country}, and more than 1,200 people have died in the United States.
Researchers, scientists, government officials and business executives
have requested access to the information. The Times is releasing its
data publicly in an effort to broaden understanding of the virus's toll.

``We hope the data set can help inform the ongoing public health
response to the pandemic and ultimately, save lives,'' said Dean Baquet,
the executive editor of The Times. ``We believe the data may help reveal
how Covid-19 has spread through communities and clusters; which
geographic areas may be hit the hardest; and how its spread in hard-hit
areas may offer clues for regions that could face wider outbreaks in the
future.''

The tracking has shown how quickly a single known case can mushroom into
an uncontrolled outbreak, as has happened in
\href{https://www.nytimes.com/2020/03/26/us/coronavirus-louisiana-new-orleans.html}{Louisiana}.

The database has shown how the detection of a small cluster in one area,
like New Rochelle, N.Y., can precede the discovery of thousands more
cases in nearby cities and states.

And it has shown, with tragic frequency, how vulnerable older adults are
to the worst of the virus. Public health officials have linked at least
37 deaths to the
\href{https://www.nytimes.com/2020/03/21/us/coronavirus-nursing-home-kirkland-life-care.html}{Life
Care nursing facility in Kirkland, Wash}. The Times has also tracked
outbreaks at other nursing and senior living facilities in Washington
State, Colorado, Florida and Louisiana.

The goal of the tracking was to compile a historical record,
\href{https://www.nytimes.com/interactive/2020/03/21/us/coronavirus-us-cases-spread.html}{with
as much detail about individual patients as can be obtained,}of the
largest public health crisis in modern American history. By collecting
the data continuously, and from multiple levels of government, The Times
has been able to
\href{https://www.nytimes.com/interactive/2020/us/coronavirus-us-cases.html}{map
the spread of the virus}, with updated information published several
times a day.

The Times's database has already informed academic research that showed
\href{https://www.nytimes.com/interactive/2020/03/20/us/coronavirus-model-us-outbreak.html}{how
the virus might be slowed}, and
\href{https://www.gainesville.com/opinion/20200325/stephen-j-hagen-and-peter-j-hirschfeld-florida-must-be-locked-down-now}{what
might happen} if it is not. In the weeks and months ahead, it is our
hope that the data can continue to inform researchers, policymakers and
journalists as they seek to understand how this pandemic evaded
containment, how it still might be mitigated and how similar disasters
might be avoided in the future.

The tracking effort grew from a handful of Times correspondents to a
large team of journalists that includes experts in data and graphics,
staff news assistants and freelance reporters, as well as journalism
students from Northwestern University, the University of Missouri and
the University of Nebraska-Lincoln. The reporting continues nearly all
day and night, seven days a week, across U.S. time zones, to record as
many details as possible about every case in real time. The Times is
committed to collecting as much data as possible in connection with the
outbreak and is collaborating with the University of California,
Berkeley, on an effort in that state.

\href{https://www.nytimes.com/news-event/coronavirus?action=click\&pgtype=Article\&state=default\&region=MAIN_CONTENT_3\&context=storylines_faq}{}

\hypertarget{the-coronavirus-outbreak-}{%
\subsubsection{The Coronavirus Outbreak
›}\label{the-coronavirus-outbreak-}}

\hypertarget{frequently-asked-questions}{%
\paragraph{Frequently Asked
Questions}\label{frequently-asked-questions}}

Updated July 27, 2020

\begin{itemize}
\item ~
  \hypertarget{should-i-refinance-my-mortgage}{%
  \paragraph{Should I refinance my
  mortgage?}\label{should-i-refinance-my-mortgage}}

  \begin{itemize}
  \tightlist
  \item
    \href{https://www.nytimes.com/article/coronavirus-money-unemployment.html?action=click\&pgtype=Article\&state=default\&region=MAIN_CONTENT_3\&context=storylines_faq}{It
    could be a good idea,} because mortgage rates have
    \href{https://www.nytimes.com/2020/07/16/business/mortgage-rates-below-3-percent.html?action=click\&pgtype=Article\&state=default\&region=MAIN_CONTENT_3\&context=storylines_faq}{never
    been lower.} Refinancing requests have pushed mortgage applications
    to some of the highest levels since 2008, so be prepared to get in
    line. But defaults are also up, so if you're thinking about buying a
    home, be aware that some lenders have tightened their standards.
  \end{itemize}
\item ~
  \hypertarget{what-is-school-going-to-look-like-in-september}{%
  \paragraph{What is school going to look like in
  September?}\label{what-is-school-going-to-look-like-in-september}}

  \begin{itemize}
  \tightlist
  \item
    It is unlikely that many schools will return to a normal schedule
    this fall, requiring the grind of
    \href{https://www.nytimes.com/2020/06/05/us/coronavirus-education-lost-learning.html?action=click\&pgtype=Article\&state=default\&region=MAIN_CONTENT_3\&context=storylines_faq}{online
    learning},
    \href{https://www.nytimes.com/2020/05/29/us/coronavirus-child-care-centers.html?action=click\&pgtype=Article\&state=default\&region=MAIN_CONTENT_3\&context=storylines_faq}{makeshift
    child care} and
    \href{https://www.nytimes.com/2020/06/03/business/economy/coronavirus-working-women.html?action=click\&pgtype=Article\&state=default\&region=MAIN_CONTENT_3\&context=storylines_faq}{stunted
    workdays} to continue. California's two largest public school
    districts --- Los Angeles and San Diego --- said on July 13, that
    \href{https://www.nytimes.com/2020/07/13/us/lausd-san-diego-school-reopening.html?action=click\&pgtype=Article\&state=default\&region=MAIN_CONTENT_3\&context=storylines_faq}{instruction
    will be remote-only in the fall}, citing concerns that surging
    coronavirus infections in their areas pose too dire a risk for
    students and teachers. Together, the two districts enroll some
    825,000 students. They are the largest in the country so far to
    abandon plans for even a partial physical return to classrooms when
    they reopen in August. For other districts, the solution won't be an
    all-or-nothing approach.
    \href{https://bioethics.jhu.edu/research-and-outreach/projects/eschool-initiative/school-policy-tracker/}{Many
    systems}, including the nation's largest, New York City, are
    devising
    \href{https://www.nytimes.com/2020/06/26/us/coronavirus-schools-reopen-fall.html?action=click\&pgtype=Article\&state=default\&region=MAIN_CONTENT_3\&context=storylines_faq}{hybrid
    plans} that involve spending some days in classrooms and other days
    online. There's no national policy on this yet, so check with your
    municipal school system regularly to see what is happening in your
    community.
  \end{itemize}
\item ~
  \hypertarget{is-the-coronavirus-airborne}{%
  \paragraph{Is the coronavirus
  airborne?}\label{is-the-coronavirus-airborne}}

  \begin{itemize}
  \tightlist
  \item
    The coronavirus
    \href{https://www.nytimes.com/2020/07/04/health/239-experts-with-one-big-claim-the-coronavirus-is-airborne.html?action=click\&pgtype=Article\&state=default\&region=MAIN_CONTENT_3\&context=storylines_faq}{can
    stay aloft for hours in tiny droplets in stagnant air}, infecting
    people as they inhale, mounting scientific evidence suggests. This
    risk is highest in crowded indoor spaces with poor ventilation, and
    may help explain super-spreading events reported in meatpacking
    plants, churches and restaurants.
    \href{https://www.nytimes.com/2020/07/06/health/coronavirus-airborne-aerosols.html?action=click\&pgtype=Article\&state=default\&region=MAIN_CONTENT_3\&context=storylines_faq}{It's
    unclear how often the virus is spread} via these tiny droplets, or
    aerosols, compared with larger droplets that are expelled when a
    sick person coughs or sneezes, or transmitted through contact with
    contaminated surfaces, said Linsey Marr, an aerosol expert at
    Virginia Tech. Aerosols are released even when a person without
    symptoms exhales, talks or sings, according to Dr. Marr and more
    than 200 other experts, who
    \href{https://academic.oup.com/cid/article/doi/10.1093/cid/ciaa939/5867798}{have
    outlined the evidence in an open letter to the World Health
    Organization}.
  \end{itemize}
\item ~
  \hypertarget{what-are-the-symptoms-of-coronavirus}{%
  \paragraph{What are the symptoms of
  coronavirus?}\label{what-are-the-symptoms-of-coronavirus}}

  \begin{itemize}
  \tightlist
  \item
    Common symptoms
    \href{https://www.nytimes.com/article/symptoms-coronavirus.html?action=click\&pgtype=Article\&state=default\&region=MAIN_CONTENT_3\&context=storylines_faq}{include
    fever, a dry cough, fatigue and difficulty breathing or shortness of
    breath.} Some of these symptoms overlap with those of the flu,
    making detection difficult, but runny noses and stuffy sinuses are
    less common.
    \href{https://www.nytimes.com/2020/04/27/health/coronavirus-symptoms-cdc.html?action=click\&pgtype=Article\&state=default\&region=MAIN_CONTENT_3\&context=storylines_faq}{The
    C.D.C. has also} added chills, muscle pain, sore throat, headache
    and a new loss of the sense of taste or smell as symptoms to look
    out for. Most people fall ill five to seven days after exposure, but
    symptoms may appear in as few as two days or as many as 14 days.
  \end{itemize}
\item ~
  \hypertarget{does-asymptomatic-transmission-of-covid-19-happen}{%
  \paragraph{Does asymptomatic transmission of Covid-19
  happen?}\label{does-asymptomatic-transmission-of-covid-19-happen}}

  \begin{itemize}
  \tightlist
  \item
    So far, the evidence seems to show it does. A widely cited
    \href{https://www.nature.com/articles/s41591-020-0869-5}{paper}
    published in April suggests that people are most infectious about
    two days before the onset of coronavirus symptoms and estimated that
    44 percent of new infections were a result of transmission from
    people who were not yet showing symptoms. Recently, a top expert at
    the World Health Organization stated that transmission of the
    coronavirus by people who did not have symptoms was ``very rare,''
    \href{https://www.nytimes.com/2020/06/09/world/coronavirus-updates.html?action=click\&pgtype=Article\&state=default\&region=MAIN_CONTENT_3\&context=storylines_faq\#link-1f302e21}{but
    she later walked back that statement.}
  \end{itemize}
\end{itemize}

In tracking the cases, the reporting process is labor-intensive but
straightforward much of the time. But with dozens of states and hundreds
of local health departments using their own reporting methods --- and
sometimes moving patients from county to county or state to state with
no explanation --- judgment calls have sometimes been required.

When the federal government arranged flights to the United States for
Americans exposed to the coronavirus in China and Japan, The Times
recorded their cases in the states where they subsequently were treated,
even though local health departments generally did not.

When a
\href{https://www.nytimes.com/2020/03/22/us/coronavirus-deaths-united-states.html}{resident
of Florida died in Los Angeles}, we recorded her death as having
occurred in California, though officials in Florida counted her case in
their own records. When officials in some states reported new cases
without immediately identifying where they were being treated, our team
attempted to add county information later, once it became available.

This accounting is an attempt to record how and where the coronavirus
has spread across the United States. But it is a product of a fragmented
American public health system in which overwhelmed officials at the
state or local level have sometimes struggled to report accurately and
consistently.

On several occasions, officials have corrected information hours or days
after they first reported it. Or cases have disappeared from a local
database. Or states have created confusing, sometimes duplicative ways
of tracking the cases of people who became ill while traveling. In those
instances, which have become more common as caseloads have increased,
The Times has sought to reflect the most current, accurate information
while ensuring that every known case,
\href{https://www.nytimes.com/2020/03/22/us/coronavirus-deaths-united-states.html}{each
of them representing someone's loved one}, is counted.

Advertisement

\protect\hyperlink{after-bottom}{Continue reading the main story}

\hypertarget{site-index}{%
\subsection{Site Index}\label{site-index}}

\hypertarget{site-information-navigation}{%
\subsection{Site Information
Navigation}\label{site-information-navigation}}

\begin{itemize}
\tightlist
\item
  \href{https://help.nytimes.com/hc/en-us/articles/115014792127-Copyright-notice}{©~2020~The
  New York Times Company}
\end{itemize}

\begin{itemize}
\tightlist
\item
  \href{https://www.nytco.com/}{NYTCo}
\item
  \href{https://help.nytimes.com/hc/en-us/articles/115015385887-Contact-Us}{Contact
  Us}
\item
  \href{https://www.nytco.com/careers/}{Work with us}
\item
  \href{https://nytmediakit.com/}{Advertise}
\item
  \href{http://www.tbrandstudio.com/}{T Brand Studio}
\item
  \href{https://www.nytimes.com/privacy/cookie-policy\#how-do-i-manage-trackers}{Your
  Ad Choices}
\item
  \href{https://www.nytimes.com/privacy}{Privacy}
\item
  \href{https://help.nytimes.com/hc/en-us/articles/115014893428-Terms-of-service}{Terms
  of Service}
\item
  \href{https://help.nytimes.com/hc/en-us/articles/115014893968-Terms-of-sale}{Terms
  of Sale}
\item
  \href{https://spiderbites.nytimes.com}{Site Map}
\item
  \href{https://help.nytimes.com/hc/en-us}{Help}
\item
  \href{https://www.nytimes.com/subscription?campaignId=37WXW}{Subscriptions}
\end{itemize}
