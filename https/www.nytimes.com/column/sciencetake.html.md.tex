Sections

SEARCH

\protect\hyperlink{site-content}{Skip to
content}\protect\hyperlink{site-index}{Skip to site index}

\href{https://www.nytimes.com/column/sciencetake}{ScienceTake}

\href{https://myaccount.nytimes.com/auth/login?response_type=cookie\&client_id=vi}{}

\href{https://www.nytimes.com/section/todayspaper}{Today's Paper}

Advertisement

\protect\hyperlink{after-top}{Continue reading the main story}

Supported by

\protect\hyperlink{after-sponsor}{Continue reading the main story}

\hypertarget{science}{%
\subsubsection{\texorpdfstring{\href{/section/science}{Science}}{Science}}\label{science}}

\includegraphics{https://static01.nyt.com/images/2018/02/16/multimedia/author-james-gorman/author-james-gorman-thumbLarge.jpg}

\hypertarget{sciencetake}{%
\section{ScienceTake}\label{sciencetake}}

\hypertarget{sciencetake-combines-cutting-edge-research-from-the-world-of-science-with-stunning-footage-of-the-natural-world-in-action-more}{%
\subsection{ScienceTake combines cutting-edge research from the world of
science with stunning footage of the natural world in action.
More**}\label{sciencetake-combines-cutting-edge-research-from-the-world-of-science-with-stunning-footage-of-the-natural-world-in-action-more}}

\hypertarget{sciencetake-combines-cutting-edge-research-from-the-world-of-science-with-stunning-footage-of-the-natural-world-in-action-more-1}{%
\subsection{ScienceTake combines cutting-edge research from the world of
science with stunning footage of the natural world in action.
More**}\label{sciencetake-combines-cutting-edge-research-from-the-world-of-science-with-stunning-footage-of-the-natural-world-in-action-more-1}}

James Gorman is a science writer at large for The New York Times and the
host and writer of the regular video
series~\href{https://www.nytimes.com/video/sciencetake}{``ScienceTake.''}
He has been at the Times since 1993, as an editor on The New York Times
Magazine, deputy science editor, editor of a personal technology
section, outdoors columnist, science columnist and editor of Science
Times.

Over the course of his career at the Times and elsewhere, Mr. Gorman~has
written about ~everything from the invention of flea collars to the
nature of consciousness. Most recently he has covered neuroscience and
the lives of animals in and out of scientific research.

Before joining The Times, Mr. Gorman wrote books on penguins, dinosaurs,
the Southern Ocean and hypochondria. His most recent book is ``How to
Build a Dinosaur,'' 2009, written with the paleontologist Jack Horner.

He also writes humor, which he has contributed to The New Yorker, The
Atlantic, the New York Times Magazine and other publications.

He has taught science writing at New York University, Fordham University
and online in Stanford University's Continuing Studies program. In the
fall of 2011, he was the McGraw Visiting Professor of Writing at
Princeton University.

Mr. Gorman graduated from Princeton in 1971 with a bachelor's degree in
English literature.

\begin{itemize}
\tightlist
\item
  \protect\hyperlink{stream-panel}{Latest}
\item
  Search
\end{itemize}

\begin{enumerate}
\def\labelenumi{\arabic{enumi}.}
\item
  \href{/2019/03/26/science/water-droplets-dance.html}{}

  \includegraphics{https://static01.nyt.com/images/2019/03/27/autossell/27a3_videorail/droplet_image_final-thumbWide.jpg?quality=75\&auto=webp\&disable=upscale}

  \hypertarget{this-water-drop-its-the-greatest-dancer}{%
  \subsection{This Water Drop, It's the Greatest
  Dancer}\label{this-water-drop-its-the-greatest-dancer}}

  You've probably never seen water do this.

  By Nicholas St. Fleur
\item
  \href{/2019/03/05/science/microwave-grapes-plasma.html}{}

  \includegraphics{https://static01.nyt.com/images/2019/03/06/autossell/GrapeBeadPlasma-3/GrapeBeadPlasma-3-thumbWide.jpg?quality=75\&auto=webp\&disable=upscale}

  \hypertarget{when-plasma-becomes-another-fruit-of-the-vine}{%
  \subsection{When Plasma Becomes Another Fruit of the
  Vine}\label{when-plasma-becomes-another-fruit-of-the-vine}}

  A parlor trick with grapes leads to new findings about water and
  microwaves.

  By James Gorman
\item
  \href{/2019/02/19/science/spittlebugs-bubble-home.html}{}

  \includegraphics{https://static01.nyt.com/images/2019/02/19/autossell/adult-5-025/adult-5-025-thumbWide.jpg?quality=75\&auto=webp\&disable=upscale}

  \hypertarget{inside-the-spittlebugs-bubble-home}{%
  \subsection{Inside the Spittlebug's Bubble
  Home}\label{inside-the-spittlebugs-bubble-home}}

  Those foamy eruptions on garden plants protect a slow and steady sap
  drinker that is growing into a froghopper. But it has to stick its
  hind end out to breathe.

  By James Gorman
\item
  \href{/2019/02/05/science/hummingbirds-science-take.html}{}

  \includegraphics{https://static01.nyt.com/images/2019/02/05/science/05SCI-TAKE1/05SCI-TAKE1-thumbWide.jpg?quality=75\&auto=webp\&disable=upscale}

  \hypertarget{the-hummingbird-as-warrior-evolution-of-a-fierce-and-furious-beak}{%
  \subsection{The Hummingbird as Warrior: Evolution of a Fierce and
  Furious
  Beak}\label{the-hummingbird-as-warrior-evolution-of-a-fierce-and-furious-beak}}

  Winsomely captured in poems and song, the birds are yielding new
  secrets about their astounding beaks and penchant for violence.

  By James Gorman
\item
  \href{/2019/01/22/science/ants-navigate-scent.html}{}

  \includegraphics{https://static01.nyt.com/images/2019/01/23/science/22SCI-TAKE/22SCI-TAKE-thumbWide.jpg?quality=75\&auto=webp\&disable=upscale}

  \hypertarget{how-ants-sniff-out-the-right-path}{%
  \subsection{How Ants Sniff Out the Right
  Path}\label{how-ants-sniff-out-the-right-path}}

  They may seem like automatons, but ants are surprisingly sophisticated
  in their navigational strategies.

  By James Gorman
\item
  \href{/2019/01/08/science/volcanos-explosions-lava.html}{}

  \includegraphics{https://static01.nyt.com/images/2019/01/09/science/08SCI-TAKE-promo/08SCI-TAKE-promo-thumbWide-v2.jpg?quality=75\&auto=webp\&disable=upscale}

  \hypertarget{watch-scientists-brew-their-own-lava}{%
  \subsection{Watch Scientists Brew Their Own
  Lava}\label{watch-scientists-brew-their-own-lava}}

  In controlled experiments, high-speed cameras caught video of
  explosions that occur when water hits hot liquid rock.

  By Nicholas St. Fleur

  \href{https://www.nytimes.com/es/2019/01/12/volcanes-explosiones-lava/}{Leer
  en español}
\item
  \href{/2018/12/11/science/geckos-running-water.html}{}

  \includegraphics{https://static01.nyt.com/images/2018/12/12/autossell/Gecko/Gecko-thumbWide.jpg?quality=75\&auto=webp\&disable=upscale}

  \hypertarget{geckos-can-run-on-water}{%
  \subsection{Geckos Can Run on Water}\label{geckos-can-run-on-water}}

  A small lizard is among the elite group of animals that race across
  the surface of water.

  By James Gorman
\item
  \href{/2018/11/27/science/cockroach-kick-wasp.html}{}

  \includegraphics{https://static01.nyt.com/images/2018/12/04/autossell/Figure-9/Figure-9-thumbWide.jpg?quality=75\&auto=webp\&disable=upscale}

  \hypertarget{the-wasp-wants-a-zombie-the-cockroach-says-no-with-a-karate-kick}{%
  \subsection{The Wasp Wants a Zombie. The Cockroach Says `No' With a
  Karate
  Kick.}\label{the-wasp-wants-a-zombie-the-cockroach-says-no-with-a-karate-kick}}

  Scientists documented the fancy footwork that helps some cockroaches
  fend off a wasp's paralyzing sting.

  By Nicholas St. Fleur
\item
  \href{/2018/11/06/science/spider-vision.html}{}

  \includegraphics{https://static01.nyt.com/images/2018/11/06/autossell/Spider-with-a-hat-1-of-1/Spider-with-a-hat-1-of-1--thumbWide-v3.jpg?quality=75\&auto=webp\&disable=upscale}

  \hypertarget{how-the-jumping-spider-sees-its-prey}{%
  \subsection{How the Jumping Spider Sees Its
  Prey}\label{how-the-jumping-spider-sees-its-prey}}

  Researchers looked deep into the eyes of a predatory spider to learn
  what it was looking at.

  By James Gorman
\item
  \href{/2018/10/31/science/spiders-halloween.html}{}

  \includegraphics{https://static01.nyt.com/images/2014/10/31/multimedia/science-take-spiders/science-take-spiders-thumbWide.jpg?quality=75\&auto=webp\&disable=upscale}

  \hypertarget{this-halloween-consider-the-unappreciated-beauty-of-spiders}{%
  \subsection{This Halloween, Consider the Unappreciated Beauty of
  Spiders}\label{this-halloween-consider-the-unappreciated-beauty-of-spiders}}

  Arachnids get a bad rap, particularly around Halloween, but they're
  actually quite lovely in their own, deadly, leaping, eight-legged,
  cannibalistic way.

  By James Gorman
\end{enumerate}

Show More

Advertisement

\protect\hyperlink{after-mid1}{Continue reading the main story}

Advertisement

\protect\hyperlink{after-mktg}{Continue reading the main story}

\hypertarget{site-index}{%
\subsection{Site Index}\label{site-index}}

\hypertarget{site-information-navigation}{%
\subsection{Site Information
Navigation}\label{site-information-navigation}}

\begin{itemize}
\tightlist
\item
  \href{https://help.nytimes.com/hc/en-us/articles/115014792127-Copyright-notice}{©~2020~The
  New York Times Company}
\end{itemize}

\begin{itemize}
\tightlist
\item
  \href{https://www.nytco.com/}{NYTCo}
\item
  \href{https://help.nytimes.com/hc/en-us/articles/115015385887-Contact-Us}{Contact
  Us}
\item
  \href{https://www.nytco.com/careers/}{Work with us}
\item
  \href{https://nytmediakit.com/}{Advertise}
\item
  \href{http://www.tbrandstudio.com/}{T Brand Studio}
\item
  \href{https://www.nytimes.com/privacy/cookie-policy\#how-do-i-manage-trackers}{Your
  Ad Choices}
\item
  \href{https://www.nytimes.com/privacy}{Privacy}
\item
  \href{https://help.nytimes.com/hc/en-us/articles/115014893428-Terms-of-service}{Terms
  of Service}
\item
  \href{https://help.nytimes.com/hc/en-us/articles/115014893968-Terms-of-sale}{Terms
  of Sale}
\item
  \href{https://spiderbites.nytimes.com}{Site Map}
\item
  \href{https://help.nytimes.com/hc/en-us}{Help}
\item
  \href{https://www.nytimes.com/subscription?campaignId=37WXW}{Subscriptions}
\end{itemize}
