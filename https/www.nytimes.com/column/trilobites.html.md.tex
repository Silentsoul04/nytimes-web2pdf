Sections

SEARCH

\protect\hyperlink{site-content}{Skip to
content}\protect\hyperlink{site-index}{Skip to site index}

\href{https://www.nytimes.com/column/trilobites}{Trilobites}

\href{https://myaccount.nytimes.com/auth/login?response_type=cookie\&client_id=vi}{}

\href{https://www.nytimes.com/section/todayspaper}{Today's Paper}

Advertisement

\protect\hyperlink{after-top}{Continue reading the main story}

Supported by

\protect\hyperlink{after-sponsor}{Continue reading the main story}

\hypertarget{science}{%
\subsubsection{\texorpdfstring{\href{/section/science}{Science}}{Science}}\label{science}}

\hypertarget{trilobites}{%
\section{Trilobites}\label{trilobites}}

\hypertarget{unearthing-fascinating-morsels-of-science}{%
\subsection{Unearthing fascinating morsels of
science.}\label{unearthing-fascinating-morsels-of-science}}

\hypertarget{unearthing-fascinating-morsels-of-science-1}{%
\subsection{Unearthing fascinating morsels of
science.}\label{unearthing-fascinating-morsels-of-science-1}}

\begin{itemize}
\tightlist
\item
  \protect\hyperlink{stream-panel}{Latest}
\item
  Search
\end{itemize}

\begin{enumerate}
\def\labelenumi{\arabic{enumi}.}
\item
  \href{/2020/08/03/science/beetle-frog-poop.html}{}

  \includegraphics{https://static01.nyt.com/images/2020/08/03/science/03TB-BEETLE1/03TB-BEETLE1-thumbWide.jpg?quality=75\&auto=webp\&disable=upscale}

  \hypertarget{there-are-two-ways-out-of-a-frog-this-beetle-chose-the-back-door}{%
  \subsection{There Are Two Ways Out of a Frog. This Beetle Chose the
  Back
  Door.}\label{there-are-two-ways-out-of-a-frog-this-beetle-chose-the-back-door}}

  A researcher fed beetles to frogs. The encounter did not end as
  expected.

  By Katherine J. Wu
\item
  \href{/2020/08/01/science/vines-lianas-panama.html}{}

  \includegraphics{https://static01.nyt.com/images/2020/08/04/science/00SCI-VINES1/00SCI-VINES1-thumbWide.jpg?quality=75\&auto=webp\&disable=upscale}

  \hypertarget{how-woody-vines-do-the-twist}{%
  \subsection{How Woody Vines Do the
  Twist}\label{how-woody-vines-do-the-twist}}

  Slowly, scientists are learning how lianas quickly climb.

  By Devi Lockwood
\item
  \href{/2020/07/31/science/alexandrian-glass-rome.html}{}

  \includegraphics{https://static01.nyt.com/images/2020/08/04/science/30TB-GLASS/30TB-GLASS-thumbWide.jpg?quality=75\&auto=webp\&disable=upscale}

  \hypertarget{the-romans-called-it-alexandrian-glass-where-was-it-really-from}{%
  \subsection{The Romans Called it `Alexandrian Glass.' Where Was It
  Really
  From?}\label{the-romans-called-it-alexandrian-glass-where-was-it-really-from}}

  Trace quantities of isotopes hint at the true origin of a kind of
  glass that was highly prized in the Roman Empire.

  By Katherine Kornei
\item
  \href{/2020/07/29/science/moss-quartz-biology-syntrichia.html}{}

  \includegraphics{https://static01.nyt.com/images/2020/07/28/science/00SCI-MOSS3/00SCI-MOSS3-thumbWide.jpg?quality=75\&auto=webp\&disable=upscale}

  \hypertarget{this-moss-uses-quartz-as-a-parasol}{%
  \subsection{This Moss Uses Quartz as a
  Parasol}\label{this-moss-uses-quartz-as-a-parasol}}

  In the Mojave Desert, a translucent crystal offers bryophytes
  much-needed respite from the heat of the sun.

  By Sabrina Imbler
\item
  \href{/2020/07/27/science/trees-immortality.html}{}

  \includegraphics{https://static01.nyt.com/images/2020/08/04/science/27TB-IMMORTALTREES1/27TB-IMMORTALTREES1-thumbWide.jpg?quality=75\&auto=webp\&disable=upscale}

  \hypertarget{can-trees-live-forever-new-kindling-for-an-immortal-debate}{%
  \subsection{Can Trees Live Forever? New Kindling for an Immortal
  Debate}\label{can-trees-live-forever-new-kindling-for-an-immortal-debate}}

  Some trees can live for thousands of years, but we may not be around
  long enough to really know whether they can die of old age.

  By Cara Giaimo
\item
  \href{/2020/07/22/science/vampire-bats-viruses.html}{}

  \includegraphics{https://static01.nyt.com/images/2020/08/04/science/21TB-BATS/21TB-BATS-thumbWide.jpg?quality=75\&auto=webp\&disable=upscale}

  \hypertarget{vampire-bats-self-isolate-too}{%
  \subsection{Vampire Bats Self-Isolate,
  Too}\label{vampire-bats-self-isolate-too}}

  When these mammals are ill, they have fewer interactions with family
  and friends, a new study suggests. ``It's like us,'' said one
  researcher.

  By David Waldstein
\item
  \href{/2020/07/21/science/mammals-vocal-learning.html}{}

  \includegraphics{https://static01.nyt.com/images/2020/08/04/science/19TB-VOCAL/19TB-VOCAL-thumbWide.jpg?quality=75\&auto=webp\&disable=upscale}

  \hypertarget{natures-noisiest-liars-carry-secrets-in-their-calls}{%
  \subsection{Nature's Noisiest Liars Carry Secrets in Their
  Calls}\label{natures-noisiest-liars-carry-secrets-in-their-calls}}

  Many mammals that have loud calls to deceive other animals seem to
  have a particular learning style in common.

  By Joshua Sokol
\item
  \href{/2020/07/16/science/ultra-black-fish.html}{}

  \includegraphics{https://static01.nyt.com/images/2020/07/21/science/16TB-ULTRABLACKFISH1/16TB-ULTRABLACKFISH1-thumbWide.jpg?quality=75\&auto=webp\&disable=upscale}

  \hypertarget{how-ultra-black-fish-disappear-in-the-deepest-seas}{%
  \subsection{How Ultra-Black Fish Disappear in the Deepest
  Seas}\label{how-ultra-black-fish-disappear-in-the-deepest-seas}}

  Researchers have found fish that absorb more than 99.9 percent of the
  light that hits their skin.

  By Katherine J. Wu
\item
  \href{/2020/07/15/science/hybrid-sturgeon-paddlefish.html}{}

  \includegraphics{https://static01.nyt.com/images/2020/07/15/climate/15TB-STURGEON1/15TB-STURGEON1-thumbWide.jpg?quality=75\&auto=webp\&disable=upscale}

  \hypertarget{scientists-accidentally-bred-the-fish-version-of-a-liger}{%
  \subsection{Scientists Accidentally Bred the Fish Version of a
  Liger}\label{scientists-accidentally-bred-the-fish-version-of-a-liger}}

  American paddlefish and Russian sturgeon were not supposed to be able
  to create hybrid offspring. Surprise!

  By Annie Roth
\item
  \href{/2020/07/14/science/white-fluffy-ant-wasp.html}{}

  \includegraphics{https://static01.nyt.com/images/2020/07/14/science/14SCI-ANTWASPS1/14SCI-ANTWASPS1-thumbWide.jpg?quality=75\&auto=webp\&disable=upscale}

  \hypertarget{how-a-velvet-ant-which-is-a-wasp-got-its-white-fluff}{%
  \subsection{How a Velvet Ant (Which Is a Wasp) Got Its White
  Fluff}\label{how-a-velvet-ant-which-is-a-wasp-got-its-white-fluff}}

  The thistledown velvet ant, which is actually a wasp, resembles
  creosote fuzz. But mimicry isn't the reason, a new study suggests.

  By Sabrina Imbler
\end{enumerate}

Show More

Advertisement

\protect\hyperlink{after-mid1}{Continue reading the main story}

Advertisement

\protect\hyperlink{after-mktg}{Continue reading the main story}

\hypertarget{site-index}{%
\subsection{Site Index}\label{site-index}}

\hypertarget{site-information-navigation}{%
\subsection{Site Information
Navigation}\label{site-information-navigation}}

\begin{itemize}
\tightlist
\item
  \href{https://help.nytimes.com/hc/en-us/articles/115014792127-Copyright-notice}{©~2020~The
  New York Times Company}
\end{itemize}

\begin{itemize}
\tightlist
\item
  \href{https://www.nytco.com/}{NYTCo}
\item
  \href{https://help.nytimes.com/hc/en-us/articles/115015385887-Contact-Us}{Contact
  Us}
\item
  \href{https://www.nytco.com/careers/}{Work with us}
\item
  \href{https://nytmediakit.com/}{Advertise}
\item
  \href{http://www.tbrandstudio.com/}{T Brand Studio}
\item
  \href{https://www.nytimes.com/privacy/cookie-policy\#how-do-i-manage-trackers}{Your
  Ad Choices}
\item
  \href{https://www.nytimes.com/privacy}{Privacy}
\item
  \href{https://help.nytimes.com/hc/en-us/articles/115014893428-Terms-of-service}{Terms
  of Service}
\item
  \href{https://help.nytimes.com/hc/en-us/articles/115014893968-Terms-of-sale}{Terms
  of Sale}
\item
  \href{https://spiderbites.nytimes.com}{Site Map}
\item
  \href{https://help.nytimes.com/hc/en-us}{Help}
\item
  \href{https://www.nytimes.com/subscription?campaignId=37WXW}{Subscriptions}
\end{itemize}
