Sections

SEARCH

\protect\hyperlink{site-content}{Skip to
content}\protect\hyperlink{site-index}{Skip to site index}

\href{https://www.nytimes.com/es/}{en Español}

\href{https://myaccount.nytimes.com/auth/login?response_type=cookie\&client_id=vi}{}

\href{https://www.nytimes.com/section/todayspaper}{Today's Paper}

\href{/es/}{en Español}\textbar{}Los mercados de China, epicentro de un
brote letal

\url{https://nyti.ms/2GscLPB}

\begin{itemize}
\item
\item
\item
\item
\item
\end{itemize}

\href{https://www.nytimes.com/es/spotlight/coronavirus?action=click\&pgtype=Article\&state=default\&region=TOP_BANNER\&context=storylines_menu}{El
brote de coronavirus}

\begin{itemize}
\tightlist
\item
  \href{https://www.nytimes.com/es/interactive/2020/espanol/mundo/coronavirus-en-estados-unidos.html?action=click\&pgtype=Article\&state=default\&region=TOP_BANNER\&context=storylines_menu}{Mapa
  y casos en EE. UU.}
\item
  \href{https://www.nytimes.com/es/2020/07/23/espanol/america-latina/bolivia-cloro-coronavirus-ivermectina.html?action=click\&pgtype=Article\&state=default\&region=TOP_BANNER\&context=storylines_menu}{Dióxido
  de cloro, ivermectina y más: ¿funcionan?}
\item
  \href{https://www.nytimes.com/es/interactive/2020/science/coronavirus-tratamientos-curas.html?action=click\&pgtype=Article\&state=default\&region=TOP_BANNER\&context=storylines_menu}{Fármacos
  y tratamientos}
\item
  \href{https://www.nytimes.com/es/2020/07/28/espanol/ciencia-y-tecnologia/anticuerpos-coronavirus-inmunidad.html?action=click\&pgtype=Article\&state=default\&region=TOP_BANNER\&context=storylines_menu}{Anticuerpos
  e inmunidad}
\item
  \href{https://www.nytimes.com/es/2020/04/29/espanol/estilos-de-vida/oximetro-para-que-sirve.html?action=click\&pgtype=Article\&state=default\&region=TOP_BANNER\&context=storylines_menu}{Oxímetros}
\end{itemize}

Advertisement

\protect\hyperlink{after-top}{Continue reading the main story}

Supported by

\protect\hyperlink{after-sponsor}{Continue reading the main story}

\hypertarget{los-mercados-de-china-epicentro-de-un-brote-letal}{%
\section{Los mercados de China, epicentro de un brote
letal}\label{los-mercados-de-china-epicentro-de-un-brote-letal}}

El coronavirus, propagado desde Wuhan, se ha relacionado con la venta de
fauna salvaje en un mercado que los expertos describen como una
incubadora perfecta para nuevos patógenos.

\includegraphics{https://static01.nyt.com/images/2020/01/26/world/26china-markets-ES/merlin_109975373_dfea2ef7-3f1c-4d39-aa01-edfe8ac73acc-articleLarge.jpg?quality=75\&auto=webp\&disable=upscale}

\href{https://www.nytimes.com/by/steven-lee-myers}{\includegraphics{https://static01.nyt.com/images/2018/10/15/multimedia/author-steven-lee-myers/author-steven-lee-myers-thumbLarge.png}}

Por \href{https://www.nytimes.com/by/steven-lee-myers}{Steven Lee Myers}

\begin{itemize}
\item
  27 de enero de 2020
\item
  \begin{itemize}
  \item
  \item
  \item
  \item
  \item
  \end{itemize}
\end{itemize}

\href{https://www.nytimes.com/2020/01/25/world/asia/china-markets-coronavirus-sars.html}{Read
in English}

LANGFANG, China --- El mercado típico de China tiene frutas y verduras,
cortes de res, cerdo y cordero, pollos enteros desplumados (con todo y
cabezas y picos), así como cangrejos y peces vivos, que arrojan agua de
peceras que gorgorean. Algunos mercados venden cosas más inusuales, como
serpientes vivas, tortugas y cigarras, cuyos, ratas de bambú, tejones,
erizos, nutrias, civetas de las palmeras e incluso lobeznos.

Los mercados son característicos de varias ciudades chinas y ahora, al
menos por segunda vez en dos décadas, han sido la fuente de una epidemia
que ha diseminado el miedo,
\href{https://www.nytimes.com/2020/01/25/world/asia/coronavirus-crisis-china-response.html}{agobiado
a la burocracia del Partido Comunista}y expuesto los riesgos
epidemiológicos que pueden surgir en lugares donde convergen los humanos
y la fauna silvestre.

El
\href{https://www.nytimes.com/es/2020/01/21/espanol/ciencia-y-tecnologia/coronavirus-sintomas.html}{nuevo
coronavirus}, que ha cobrado al menos 56 vidas y enfermado a más de 1370
personas en China y en todo el
mundo,\href{https://www.nytimes.com/es/2020/01/21/espanol/ciencia-y-tecnologia/coronavirus-sintomas.html}{se
cree que se propagó}precisamente desde uno de estos lugares: un mercado
de venta al mayoreo en Wuhan, una ciudad en el centro de China, donde
los vendedores comerciaban de manera legal con animales vivos en
condiciones de hacinamiento.

``Así es como surgen enfermedades nuevas y nacientes que la población
humana nunca antes ha visto'', afirmó Kevin J. Olival, biólogo y
vicepresidente de investigación en EcoHealth Alliance, una organización
de investigación sin fines de lucro, que le ha dado seguimiento a brotes
anteriores.

Si bien la trayectoria exacta del patógeno no ha sido establecida,
funcionarios del gobierno y científicos dijeron que la nueva enfermedad
tenía similitudes ominosas con el brote del SARS (por su sigla en
inglés, que en español significa síndrome respiratorio agudo grave,
SRAG), a finales de 2002, cuando murieron casi 800 personas y se
enfermaron miles más en todo el mundo.

Ahora, conforme el gobierno lucha para
\href{https://www.nytimes.com/2020/01/23/world/asia/china-coronavirus-outbreak.html}{contener
la ira del público por el brote}, debe enfrentarse a exigencias de que
haga más para regular la venta de la fauna silvestre, o incluso que la
prohíba. Asimismo, también debe responder a cada vez más preguntas sobre
por qué las cosas han cambiado tan poco en los diecisiete años desde el
brote de SARS.

\includegraphics{https://static01.nyt.com/images/2020/01/25/world/25china-markets-2-ES/merlin_167645469_de8b26dd-8269-4f9d-b671-8657ee02dfb6-articleLarge.jpg?quality=75\&auto=webp\&disable=upscale}

En suma, el
\href{https://www.nytimes.com/2003/04/27/world/the-sars-epidemic-the-path-from-china-s-provinces-a-crafty-germ-breaks-out.html}{SARS
fue rastreado} a un coronavirus que saltó de los murciélagos a las
civetas de las palmeras, una criatura con rasgos felinos considerada una
delicia en el sur de China, y luego saltó a los humanos que participaban
ahí en el comercio de vida silvestre. Según los funcionarios y
científicos, el nuevo virus también parece que se originó en los
murciélagos y luego saltó a otro mamífero, aunque aún no se sabe a cuál.

El brote más reciente ---cuyo alcance aún está por definirse--- ha
provocado reclamaciones tanto dentro de China como fuera del país para
que haya mejores regulaciones o incluso se acabe con este sentido de
aventura culinario. Aunque la tortuga y la carne de jabalí no son raras
en los restaurantes chinos, la carne de animales de caza, como la de las
civetas, serpientes o
\href{https://www.nytimes.com/es/2019/04/09/espanol/comercio-especies-pangolin.html}{pangolines}
generalmente solo se consideran exquisiteces en algunas regiones. Su
consumo está motivado tanto por el deseo de hacer alarde de riqueza como
por una mezcla de superstición y fe en los beneficios a la salud que
trae consumir fauna silvestre.

Al poco tiempo de que el Mercado Mayorista de Mariscos Huanan en Wuhan
fue identificado como la fuente más probable de este brote en diciembre,
las autoridades lo cerraron, aunque no quedó claro qué sucedió con los
animales que estaban a la venta ahí. Apenas el 22 de enero los
funcionarios anunciaron que habían prohibido la venta de animales
salvajes en toda la provincia. Dos provincias más, Henan y Mongolia
Interior, también
\href{http://m.xinhuanet.com/2020-01/23/c_1125495548.htm}{decretaron la
suspensión}de esta actividad comercial esta última semana.

El 24 de enero, funcionarios de tres agencias nacionales anunciaron
controles más estrictos, entre ellas la suspensión a nivel nacional de
la venta y el transporte de animales que quizá estén vinculados con el
nuevo coronavirus. La declaración solo especificó a tejones y ratas de
bambú, una especie de roedores del sur de China que vive en matorrales
de bambú, de los que también se alimenta. Ambos habían estado a la venta
en el mercado de Wuhan.

Image

La sección de pescado en el mercado de Langfang.Credit...Giulia Marchi
para The New York Times

La avalancha de acciones tomadas por el gobierno se dio tras profusas
manifestaciones del público en contra de la venta de animales vivos. Una
campaña en Weibo, la red social, atrajo 45 millones de vistas con la
etiqueta \#RejectGameMeat (rechaza la carne de caza).

``Comer animales de caza no cura la impotencia ni tiene propiedades
sanadoras'', escribió Jin Sichen, conductor de televisión en Nankín, una
ciudad en el sureste de China, el 22 de enero
\href{https://www.weibo.com/1863716232/IqGcXwuXm?from=page_1005051863716232_profile\&wvr=6\&mod=weibotime\&type=comment}{en
su página de Weibo}. ``La carne de estos animales no solo no cura las
enfermedades, sino que puede enfermarte a ti, a tu familia, a tus amigos
y a muchas más personas''.

``Uno tiene que estar mal de la cabeza para comer esa carne solo con el
objetivo de presumir y apantallar'', sostuvo Jin.

Un grupo de diecinueve académicos chinos también exigió que el gobierno
hiciera más para regular el comercio y al público para que dejara de
comer animales salvajes.

Image

Un mercado en PekínCredit...Lam Yik Fei para The New York Times

The Wildlife Conservation Society, una organización con sede en Nueva
York que defiende a los animales, pidió la prohibición mundial de la
venta comercial de fauna en mercados como los de China, arguyendo que
este brote reciente demostraba que era una amenaza para la salud
pública.

Christian Walzer, el director ejecutivo de salud en la organización,
dijo que la increíble diversidad de fauna silvestre en mercados como
estos, donde hay animales metidos en jaulas pequeñas en puestos de
mercados concurridos, era el laboratorio perfecto para la incubación
involuntaria de nuevos virus que pueden penetrar las células humanas.
Los virus se contagian a través de la saliva, la sangre o las heces.

``Cada animal es un paquete de patógenos'', dijo en una entrevista
telefónica.

No obstante, algunos consumidores chinos creen, debido a la medicina
tradicional, que estos animales tienen beneficios para la salud. Los
vendedores, e incluso funcionarios del Estado en los medios oficiales,
han dicho que la fauna silvestre es una fuente alternativa de proteína,
así como una fuente de ingreso en regiones empobrecidas. Un artículo de
la agencia Xinhua a finales del año pasado, por ejemplo, dijo que criar
ratas de bambú ayudaba a sacar a la gente de la pobreza en Guangxi, otra
provincia del sur.

``Es un riesgo de salud pública, no solo en China sino en todas
partes'', dijo.

En el punto más alto del brote de SARS en 2003, las autoridades
prohibieron la venta de civetas y se deshicieron de las existencias
existentes, pero en cuestión de meses
t\href{https://www.nytimes.com/2004/01/06/world/who-urges-china-to-use-caution-while-killing-civet-cats.html}{erminaron
la prohibición} y el comercio se reanudó.

``Está impulsado por intereses'', dijo del brote actual Qin Xianoa,
presidente de la Asociación Capital Animal Welfare, una organización de
defensa en Pekín. ``Mucha gente se beneficia del comercio de vida
salvaje hoy en día''.

Image

El mercado de mariscos de Huanan en Wuhan, que se ha relacionado con el
nuevo coronavirus, ha sido desinfectado y cerrado.Credit...Getty Images

El tráfico de ciertas especies está prohibido, como es el caso de los
pangolines, que están en peligro y que son muy apreciados debido a su
carne y escamas. Pero la Administración Nacional Forestal y de
Pastizales de China permite la cría de 54 animales, aves, reptiles e
insectos, entre ellos ratas almizcleras, ardillas listadas, avestruces,
emús y ciempiés.

En la popular plataforma de comercio Taobao puede encontrarse todo tipo
de animales. **** Una cría de tejón cuesta 1.300 renminbi, o 187
dólares. Un agricultor de Hunan, la provincia directamente al sur de
Hubei, vende civetas ---la fuente del SARS--- por el equivalente a 215
dólares cada una, o a 200 dólares si uno compra 500 o más.

En el extenso mercado semicerrado en Langfang, un centro de fabricación
de productos electrónicos al sur de Beijing, un vendedor anunciaba un
cocodrilo vivo (por 550 dólares) y un puercoespín (a 115).

Las autoridades de Wuhan no han dado detalles sobre todos los animales
queestaban a la venta en el mercado que ha sido vinculado al nuevo
coronavirus. Los usuarios de redes sociales compartieron una fotografía
de una lista de precios de animales que supuestamente vendía un puesto
en elmercado, pero su autenticidad no se ha verificado.

De acuerdo \href{https://mp.weixin.qq.com/s/_ZgkquIlUYnix2y7WmKPew}{con
un blog médico} publicado en WeChat, las autoridades sanitarias de Wuhan
visitaron el mercado en el mes de septiembre e inspeccionaron ocho
puestos que vendían ranas, serpientes y erizos, entre otros animales.
Todos tenían licencia para vender y no se hallaron incumplimientos.

A pesar de la propagación del virus en todo el país, una televisora de
Hong Kong, I-Cable News Channel, encontró decenas de animales salvajes
aún a la venta el miércoles en un mercado en Qingyuan, una ciudad en
Guangdong, la provincia donde se originó el SARS.

La epidemia ya ha hecho que los vendedores estén a la defensiva.

``¿Están seguros de que comer fauna silvestre es la causa de la
epidemia?'', dijo Zheng Ming, el gerente de ventas de una compañía que
vende animales en Yichang, una ciudad a 280 kilómetros de Wuhan. Hasta
la prohibición de la venta que se anunció el 22 de enero, vendía erizos,
civetas, cuyos y ratas de bambú, entre otros.

``Nos apegamos a la ley'', afirmó. ``Este es un negocio totalmente
legal''.

James Gorman colaboró con este reportaje desde Nueva York. Zoe Mou
colaboró con la investigación desde Pekín, y Claire Fu, desde Chengdú.

\begin{center}\rule{0.5\linewidth}{\linethickness}\end{center}

Advertisement

\protect\hyperlink{after-bottom}{Continue reading the main story}

\hypertarget{site-index}{%
\subsection{Site Index}\label{site-index}}

\hypertarget{site-information-navigation}{%
\subsection{Site Information
Navigation}\label{site-information-navigation}}

\begin{itemize}
\tightlist
\item
  \href{https://help.nytimes.com/hc/en-us/articles/115014792127-Copyright-notice}{©~2020~The
  New York Times Company}
\end{itemize}

\begin{itemize}
\tightlist
\item
  \href{https://www.nytco.com/}{NYTCo}
\item
  \href{https://help.nytimes.com/hc/en-us/articles/115015385887-Contact-Us}{Contact
  Us}
\item
  \href{https://www.nytco.com/careers/}{Work with us}
\item
  \href{https://nytmediakit.com/}{Advertise}
\item
  \href{http://www.tbrandstudio.com/}{T Brand Studio}
\item
  \href{https://www.nytimes.com/privacy/cookie-policy\#how-do-i-manage-trackers}{Your
  Ad Choices}
\item
  \href{https://www.nytimes.com/privacy}{Privacy}
\item
  \href{https://help.nytimes.com/hc/en-us/articles/115014893428-Terms-of-service}{Terms
  of Service}
\item
  \href{https://help.nytimes.com/hc/en-us/articles/115014893968-Terms-of-sale}{Terms
  of Sale}
\item
  \href{https://spiderbites.nytimes.com}{Site Map}
\item
  \href{https://help.nytimes.com/hc/en-us}{Help}
\item
  \href{https://www.nytimes.com/subscription?campaignId=37WXW}{Subscriptions}
\end{itemize}
