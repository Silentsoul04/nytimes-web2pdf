Sections

SEARCH

\protect\hyperlink{site-content}{Skip to
content}\protect\hyperlink{site-index}{Skip to site index}

\href{https://www.nytimes.com/es/section/ciencia-y-tecnologia}{Ciencia y
Tecnología}

\href{https://myaccount.nytimes.com/auth/login?response_type=cookie\&client_id=vi}{}

\href{https://www.nytimes.com/section/todayspaper}{Today's Paper}

\href{/es/section/ciencia-y-tecnologia}{Ciencia y
Tecnología}\textbar{}Así es la recuperación para muchos de los
sobrevivientes de la COVID-19

\url{https://nyti.ms/2YUJ43O}

\begin{itemize}
\item
\item
\item
\item
\item
\end{itemize}

\href{https://www.nytimes.com/es/spotlight/coronavirus?action=click\&pgtype=Article\&state=default\&region=TOP_BANNER\&context=storylines_menu}{El
brote de coronavirus}

\begin{itemize}
\tightlist
\item
  \href{https://www.nytimes.com/es/interactive/2020/espanol/mundo/coronavirus-en-estados-unidos.html?action=click\&pgtype=Article\&state=default\&region=TOP_BANNER\&context=storylines_menu}{Mapa
  y casos en EE. UU.}
\item
  \href{https://www.nytimes.com/es/2020/07/23/espanol/america-latina/bolivia-cloro-coronavirus-ivermectina.html?action=click\&pgtype=Article\&state=default\&region=TOP_BANNER\&context=storylines_menu}{Dióxido
  de cloro, ivermectina y más: ¿funcionan?}
\item
  \href{https://www.nytimes.com/es/interactive/2020/science/coronavirus-tratamientos-curas.html?action=click\&pgtype=Article\&state=default\&region=TOP_BANNER\&context=storylines_menu}{Fármacos
  y tratamientos}
\item
  \href{https://www.nytimes.com/es/2020/07/28/espanol/ciencia-y-tecnologia/anticuerpos-coronavirus-inmunidad.html?action=click\&pgtype=Article\&state=default\&region=TOP_BANNER\&context=storylines_menu}{Anticuerpos
  e inmunidad}
\item
  \href{https://www.nytimes.com/es/2020/04/29/espanol/estilos-de-vida/oximetro-para-que-sirve.html?action=click\&pgtype=Article\&state=default\&region=TOP_BANNER\&context=storylines_menu}{Oxímetros}
\end{itemize}

Advertisement

\protect\hyperlink{after-top}{Continue reading the main story}

Supported by

\protect\hyperlink{after-sponsor}{Continue reading the main story}

Salud

\hypertarget{asuxed-es-la-recuperaciuxf3n-para-muchos-de-los-sobrevivientes-de-la-covid-19}{%
\section{Así es la recuperación para muchos de los sobrevivientes de la
COVID-19}\label{asuxed-es-la-recuperaciuxf3n-para-muchos-de-los-sobrevivientes-de-la-covid-19}}

Dificultad persistente para respirar, debilidad muscular,
reviviscencias, confusión mental y otros síntomas podrían aquejar a los
pacientes durante mucho tiempo después.

\includegraphics{https://static01.nyt.com/images/2020/07/01/health/02virus-recovery-explainer-ES/01VIRUS-RECOVERY-EXPLAINER-articleLarge-v2.jpg?quality=75\&auto=webp\&disable=upscale}

\href{https://www.nytimes.com/by/pam-belluck}{\includegraphics{https://static01.nyt.com/images/2018/02/16/multimedia/author-pam-belluck/author-pam-belluck-thumbLarge-v2.png}}

Por \href{https://www.nytimes.com/by/pam-belluck}{Pam Belluck}

\begin{itemize}
\item
  2 de julio de 2020
\item
  \begin{itemize}
  \item
  \item
  \item
  \item
  \item
  \end{itemize}
\end{itemize}

\href{https://www.nytimes.com/2020/07/01/health/coronavirus-recovery-survivors.html}{Read
in English}

\href{https://www.nytimes.com/newsletters/el-times}{Regístrate para
recibir nuestro boletín} con lo mejor de The New York Times.

\begin{center}\rule{0.5\linewidth}{\linethickness}\end{center}

Cientos de miles de pacientes gravemente enfermos de coronavirus que
sobreviven y son dados de alta de los hospitales están enfrentando un
desafío nuevo y complicado: la recuperación. Muchos están luchando para
superar una variedad de síntomas residuales inquietantes. Algunos
problemas podrían persistir durante meses, años o incluso el resto de
sus vidas.

Los pacientes que han regresado a casa tras estar hospitalizados por
fallas respiratorias graves derivadas del virus están lidiando con
problemas físicos, neurológicos, cognitivos y emocionales.

Además, deben vivir su proceso de recuperación mientras la pandemia
continúa, con todo el estrés y la escasez de recursos que ha provocado.

``No es solo: `La pasé muy mal en el hospital pero por suerte ya estoy
en casa y todo ha vuelto a la normalidad''', explicó David Putrino,
director de innovación en la rehabilitación del Sistema de Salud Monte
Sinaí en la ciudad de Nueva York. ``Más bien es: `Acabo de pasarla muy
mal en el hospital y ¿adivinen qué? El mundo sigue envuelto en llamas.
Necesito sobrellevar eso mientras también intento recuperar el ritmo de
mi antigua vida'''.

Aún es demasiado pronto para determinar cómo será la recuperación
completa de estos pacientes. Pero a continuación ofrecemos un vistazo de
lo que han experimentado hasta ahora, lo que podemos aprender de
pacientes recuperados que han tenido experiencias médicas similares y
los retos que tal vez les depara el futuro.

\hypertarget{quuxe9-problemas-enfrentan-los-pacientes-despuuxe9s-de-salir-del-hospital}{%
\subsection{¿Qué problemas enfrentan los pacientes después de salir del
hospital?}\label{quuxe9-problemas-enfrentan-los-pacientes-despuuxe9s-de-salir-del-hospital}}

Son varios. Es posible que los pacientes salgan del hospital todavía con
cicatrices, lesiones o inflamación en los pulmones, el corazón, los
riñones, el hígado u otros órganos que no han terminado de sanar. Esto
puede causar varios problemas como complicaciones urinarias y
metabólicas, entre otros.

Zijian Chen, director médico del nuevo Centro para Cuidados Pos-COVID-19
del Sistema de Salud Monte Sinaí, comentó que el problema físico más
importante que se veía en el centro era la dificultad para respirar, lo
cual puede deberse a un daño en los pulmones o el corazón, o a un
problema de coagulación.

``Algunos tienen una tos intermitente que no cesa y que dificulta la
respiración'', mencionó. Hay quienes incluso siguen usando la cánula
nasal de oxígeno en casa, pero no les ayuda lo suficiente.

Algunos de los pacientes que estuvieron conectados a respiradores
reportan dificultades para tragar o hablar más alto que un susurro, una
consecuencia normalmente temporal de las lesiones o la inflamación que
provoca el tubo respiratorio que pasa por las cuerdas vocales.

Muchos pacientes sienten debilidad muscular después de estar acostados
tanto tiempo en una cama de hospital, dijo Dale Needham, médico de
cuidados intensivos en la Escuela de Medicina de la Universidad Johns
Hopkins y líder en el campo de la recuperación en terapia intensiva.
Como resultado, pueden tener problemas para caminar, subir escaleras o
levantar objetos.

La debilidad o el daño en los nervios también puede reducir la fuerza
muscular, afirmó Needham. Asimismo, los problemas neurológicos pueden
causar otros síntomas. Chen dijo que el Centro para Cuidados
Pos-COVID-19 del Sistema de Salud Monte Sinaí ha referido a casi el 40
por ciento de sus pacientes con neurólogos por síntomas como cansancio,
confusión y poca claridad mental.

``Algunas dolencias son muy debilitantes'', comentó. ``Tenemos pacientes
que vienen y nos dicen: `No puedo concentrarme en el trabajo. Ya me
recuperé, no tengo problemas para respirar, no siento dolor en el pecho,
pero no puedo regresar a trabajar porque no puedo concentrarme'''.

El centro refiere a algunos de estos pacientes a consultas psicológicas,
según nos dijo Chen.

``Es muy común que los pacientes recuperados tengan estrés
postraumático: pesadillas, depresión y ansiedad debido a que tienen
recuerdos de lo que pasó'', explicó Lauren Ferrante, médica de
enfermedades pulmonares y cuidados intensivos en la Escuela de Medicina
de Yale, quien estudia los resultados de recuperación tras la terapia
intensiva.

Según los expertos, los problemas emocionales pueden aumentar para los
pacientes de la COVID-19 debido a los días que pasaron hospitalizados
sin visitas de familiares y amigos.

``Esta experiencia de estar extremadamente enfermo y extremadamente solo
únicamente amplifica el trauma'', dijo Putrino, y agregó que muchos
pacientes se contactaron con su programa para solicitar servicios de
psicología de telemedicina. ``Dicen: `No soy yo mismo y necesito hablar
con alguien'''.

Para describir la amplia variedad de desafíos de recuperación, los
expertos a menudo usan un término general, acuñado hace aproximadamente
una década:
\href{https://www.sccm.org/MyICUCare/THRIVE/Post-intensive-Care-Syndrome\#:~:text=Post\%2Dintensive\%20care\%20syndrome\%2C\%20or\%20PICS\%2C\%20is\%20made\%20up,and\%20may\%20affect\%20the\%20family.}{síndrome
posterior a cuidados intensivos}, que puede incluir cualquiera de los
síntomas físicos, cognitivos y emocionales que enfrentan los pacientes.

\hypertarget{quuxe9-hace-que-algunos-sean-muxe1s-propensos-a-enfrentar-retos-para-su-recuperaciuxf3n}{%
\subsection{¿Qué hace que algunos sean más propensos a enfrentar retos
para su
recuperación?}\label{quuxe9-hace-que-algunos-sean-muxe1s-propensos-a-enfrentar-retos-para-su-recuperaciuxf3n}}

Algunos estudios realizados con personas hospitalizadas por
insuficiencia respiratoria derivada de otras causas sugieren que es
probable que la recuperación sea más difícil para las personas que
\href{https://www.ncbi.nlm.nih.gov/pmc/articles/PMC6026287/}{no gozaban
de buena salud antes} de contraer la enfermedad y para las que habían
requerido hospitalizaciones más largas, afirmó Ferrante.

No obstante, muchos otros pacientes de coronavirus ---no solo los que
son mayores o padecen otras afecciones médicas--- pasan semanas
conectados a respiradores y luego otras semanas más en el hospital
después de que les retiran los tubos respiratorios, lo cual dificulta su
proceso de recuperación.

``Estamos viendo que los periodos de permanencia en terapia intensiva
con necesidad de un respirador ahora son más prolongados que nunca'',
dijo Ferrante. ``La inquietud es que esto tenga repercusiones en las
funciones físicas y que menos gente logre recuperarse''.

Otro factor que puede extender u obstaculizar la recuperación es un
fenómeno llamado delirio hospitalario,
\href{https://www.nytimes.com/es/2020/07/01/espanol/ciencia-y-tecnologia/coronavirus-delirio-alucinaciones.html}{un
padecimiento relacionado con alucinaciones paranoicas, confusión y
ansiedad}. Es más probable que esto se dé en pacientes que pasan mucho
tiempo sedados, tienen interacciones sociales limitadas y no pueden
desplazarse, todo lo cual es común en los pacientes con la COVID-19.

Algunos estudios, entre ellos uno realizado por un equipo del Centro
Médico de la Universidad Vanderbilt, han descubierto que los pacientes
de cuidados intensivos
\href{https://www.nejm.org/doi/full/10.1056/NEJMoa1301372}{que sufren de
delirio hospitalario son más propensos a manifestar problemas
cognitivos} en los meses posteriores a su hospitalización.

\hypertarget{cuuxe1l-es-la-trayectoria-de-la-recuperaciuxf3n}{%
\subsection{¿Cuál es la trayectoria de la
recuperación?}\label{cuuxe1l-es-la-trayectoria-de-la-recuperaciuxf3n}}

Los altibajos son comunes. ``No es un proceso lineal en absoluto, y es
muy individual'', explicó Needham.

La perseverancia es importante. ``Lo que no queremos es que los
pacientes se vayan a casa y se queden acostados todo el día'', afirmó
Ferrante. ``Eso no ayudará a su recuperación y es probable que la
empeore''.

Los pacientes y sus familias deben darse cuenta de que los vaivenes del
progreso son normales.

``Habrá días en los que todo esté bien con sus pulmones, pero que las
articulaciones les duelan tanto que no puedan levantarse y hacer los
ejercicios de rehabilitación pulmonar, por lo que se estancarán un
poco'', mencionó Putrino. ``O su función pulmonar irá bien, pero la
bruma cognitiva les provocará ansiedad, así que tendrán que dejar todo
lo demás y trabajar mucho con su neuropsicólogo''.

``De verdad se siente como dar un paso adelante y dos hacia atrás'',
agregó, ``y eso está bien''.

\hypertarget{cuuxe1nto-duran-estos-efectos}{%
\subsection{¿Cuánto duran estos
efectos?}\label{cuuxe1nto-duran-estos-efectos}}

En el caso de muchas personas, los pulmones suelen recuperarse en
cuestión de meses. Sin embargo, los expertos dicen que otros problemas
pueden perdurar y algunas personas quizá nunca se recuperen del todo.

Un punto de referencia es
\href{https://www.nejm.org/doi/full/10.1056/nejmoa1011802}{un estudio
publicado en 2011 en la revista New England Journal of Medicine} y
realizado con 109 pacientes en Canadá que se habían sometido a
tratamiento por el síndrome de dificultad respiratoria aguda (SDRA), el
tipo de insuficiencia pulmonar que aqueja a muchos pacientes con la
COVID-19. Cinco años después, la mayoría había recuperado el
funcionamiento normal o casi normal de sus pulmones, pero aún lidiaba
con problemas físicos y emocionales persistentes.

En una prueba crucial, que medía cuán lejos podían caminar los pacientes
en seis minutos, su distancia promedio fue de unos 436 metros, solo tres
cuartas partes de la distancia que habían predicho los investigadores.
El rango de edad de los pacientes era de 35 a 57 años, y aunque los
pacientes más jóvenes tuvieron una tasa de recuperación física más
favorable que la de los pacientes mayores, ``a los cinco años ningún
grupo regresó a los niveles normales de condición física que se habían
calculado'', escribieron los autores.

Los pacientes del estudio tuvieron SDRA por motivos diversos, incluyendo
neumonía, septicemia, pancreatitis o quemaduras. Estuvieron
hospitalizados durante un promedio de 49 días, de los cuales pasaron 26
en terapia intensiva y 24 conectados a un respirador.

\href{https://pubmed.ncbi.nlm.nih.gov/27637716/}{La investigación
dirigida por Needham}, de Johns Hopkins, descubrió que los ``pacientes
tienen debilidad muscular prolongada que dura meses o más y que la
debilidad muscular no solo se limita a sus brazos y piernas, sino
también a sus músculos respiratorios'', dijo.

Otro \href{https://pubmed.ncbi.nlm.nih.gov/32304774/}{estudio de Needham
y sus colegas} encontró que cerca de dos tercios de los pacientes de
SDRA tenían fatiga significativa un año después.

Los síntomas psicológicos y cognitivos también pueden persistir.
Alrededor de la mitad de los pacientes en el estudio canadiense de 2011
informaron al menos un episodio de ``depresión, ansiedad o ambos,
diagnosticados por un médico, entre dos y cinco años de seguimiento''. Y
un estudio de pacientes tratados durante el
\href{https://www.who.int/mediacentre/news/releases/2003/pr56/es/}{brote
del SRAS de 2003}, otro tipo de coronavirus, descubrió que un año
después muchos tenían
``\href{https://pubmed.ncbi.nlm.nih.gov/17500304/}{preocupantes niveles
de depresión, ansiedad y síntomas postraumáticos}''.

\hypertarget{cuuxe1les-son-las-consecuencias}{%
\subsection{¿Cuáles son las
consecuencias?}\label{cuuxe1les-son-las-consecuencias}}

Entre otras cosas, es posible que los pacientes tengan dificultades para
regresar a trabajar. Un equipo liderado por Needham descubrió que casi
una tercera parte de los 64 pacientes con SDRA que monitorearon durante
cinco años
\href{https://www.ncbi.nlm.nih.gov/pmc/articles/PMC6002952/}{nunca
volvió al trabajo}.

Algunos lo intentaron, pero se dieron cuenta de que no podían desempeñar
sus labores y dejaron de trabajar por completo, relató Needham, y hubo
quienes ``tuvieron que cambiar de giro, en concreto a un empleo menos
demandante y tal vez con menor remuneración''.

Chen dijo que le preocupaba que las consecuencias a largo plazo de la
COVID-19 pudieran ser similares a los efectos crónicos en la salud de la
epidemia del sida o el atentado del 11 de septiembre de 2001 en la
ciudad de Nueva York. ``Una nueva enfermedad grave o un evento
catastrófico causan síntomas que permanecen durante mucho tiempo'',
sentenció. ``Esto se perfila para ser peor que esos dos sucesos''.

Es posible que vaya a haber ``cientos de miles de personas que padezcan
estos trastornos crónicos que pueden tardar mucho en sanar y, si no las
atendemos, esto va a ser un gran problema de salud, así como un enorme
problema económico'', concluyó Chen.

\hypertarget{quuxe9-hacen-los-hospitales-para-ayudar-a-los-pacientes-cuando-regresan-a-casa}{%
\subsection{¿Qué hacen los hospitales para ayudar a los pacientes cuando
regresan a
casa?}\label{quuxe9-hacen-los-hospitales-para-ayudar-a-los-pacientes-cuando-regresan-a-casa}}

Los programas de recuperación para pacientes de la COVID-19 están
surgiendo en el Sistema de Salud Monte Sinaí, en Yale, Johns Hopkins y
otros lugares, y ofrecen a los pacientes consultas de telemedicina y, a
veces, citas en persona.

Algunos pacientes requieren medicamentos para ayudar con la dificultad
para respirar, problemas cardíacos o coágulos en la sangre. Ferrante
dice que las personas deben consultar los medicamentos con sus doctores,
porque algunos remedios que les dieron en el hospital pueden no ser
apropiados para seguir tomándolos en casa.

Pero la medicación puede no ser necesaria, o no puede funcionar, para
muchos problemas. Practicar ejercicios de respiración y usar un
espirómetro, un dispositivo que mide cuánto aire puede respirar una
persona y qué tan rápido, puede mejorar los problemas respiratorios. La
terapia física puede ayudar a restaurar la fuerza muscular, el
movimiento y la flexibilidad. La terapia ocupacional puede ayudar a las
personas a recuperar la capacidad de realizar tareas cotidianas, como
comprar comestibles y cocinar. La terapia del lenguaje puede ayudar con
la deglución y los problemas de las cuerdas vocales.

Según los expertos, los fisiatras, médicos que se especializan en
rehabilitación física, tendrán cada vez más demanda. También los
neurólogos y los terapeutas de salud mental.

``Creo que la principal conclusión aquí es que la atención posterior a
la COVID es compleja'', dijo Putrino. ``Ya es bastante difícil
rehabilitar a alguien con una pierna rota donde solo una cosa está
mal''.

``Pero con la atención posterior a la COVID'', dijo, ``estás tratando a
personas con algunos problemas cognitivos, problemas físicos, problemas
pulmonares, problemas cardíacos, problemas renales, trauma, y todas
estas cosas tienen que ser manejadas correctamente''.

Pam Belluck es una reportera de ciencia y salud. Fue una de los siete
miembros del Times que recibieron el Premio Pulitzer en 2015 por
Cobertura Internacional, por su trabajo durante la epidemia de ébola. Es
autora de \emph{Island Practice}, sobre un peculiar doctor en Nantucket.
\href{https://twitter.com/PamBelluck}{@PamBelluck}

Advertisement

\protect\hyperlink{after-bottom}{Continue reading the main story}

\hypertarget{site-index}{%
\subsection{Site Index}\label{site-index}}

\hypertarget{site-information-navigation}{%
\subsection{Site Information
Navigation}\label{site-information-navigation}}

\begin{itemize}
\tightlist
\item
  \href{https://help.nytimes.com/hc/en-us/articles/115014792127-Copyright-notice}{©~2020~The
  New York Times Company}
\end{itemize}

\begin{itemize}
\tightlist
\item
  \href{https://www.nytco.com/}{NYTCo}
\item
  \href{https://help.nytimes.com/hc/en-us/articles/115015385887-Contact-Us}{Contact
  Us}
\item
  \href{https://www.nytco.com/careers/}{Work with us}
\item
  \href{https://nytmediakit.com/}{Advertise}
\item
  \href{http://www.tbrandstudio.com/}{T Brand Studio}
\item
  \href{https://www.nytimes.com/privacy/cookie-policy\#how-do-i-manage-trackers}{Your
  Ad Choices}
\item
  \href{https://www.nytimes.com/privacy}{Privacy}
\item
  \href{https://help.nytimes.com/hc/en-us/articles/115014893428-Terms-of-service}{Terms
  of Service}
\item
  \href{https://help.nytimes.com/hc/en-us/articles/115014893968-Terms-of-sale}{Terms
  of Sale}
\item
  \href{https://spiderbites.nytimes.com}{Site Map}
\item
  \href{https://help.nytimes.com/hc/en-us}{Help}
\item
  \href{https://www.nytimes.com/subscription?campaignId=37WXW}{Subscriptions}
\end{itemize}
