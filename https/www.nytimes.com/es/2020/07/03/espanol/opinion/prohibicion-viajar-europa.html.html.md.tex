Sections

SEARCH

\protect\hyperlink{site-content}{Skip to
content}\protect\hyperlink{site-index}{Skip to site index}

\href{https://www.nytimes.com/es/section/opinion}{Opinión}

\href{https://myaccount.nytimes.com/auth/login?response_type=cookie\&client_id=vi}{}

\href{https://www.nytimes.com/section/todayspaper}{Today's Paper}

\href{/es/section/opinion}{Opinión}\textbar{}El mundo levanta un muro
para dejar fuera a Estados Unidos

\href{https://nyti.ms/31ITBAW}{https://nyti.ms/31ITBAW}

\begin{itemize}
\item
\item
\item
\item
\item
\item
\end{itemize}

\href{https://www.nytimes.com/es/spotlight/coronavirus?action=click\&pgtype=Article\&state=default\&region=TOP_BANNER\&context=storylines_menu}{El
brote de coronavirus}

\begin{itemize}
\tightlist
\item
  \href{https://www.nytimes.com/es/interactive/2020/espanol/america-latina/coronavirus-en-mexico.html?action=click\&pgtype=Article\&state=default\&region=TOP_BANNER\&context=storylines_menu}{Mapa
  y casos en México}
\item
  \href{https://www.nytimes.com/es/2020/07/31/espanol/ciencia-y-tecnologia/ninos-contagio-coronavirus.html?action=click\&pgtype=Article\&state=default\&region=TOP_BANNER\&context=storylines_menu}{Los
  niños y el virus}
\item
  \href{https://www.nytimes.com/es/interactive/2020/science/coronavirus-tratamientos-curas.html?action=click\&pgtype=Article\&state=default\&region=TOP_BANNER\&context=storylines_menu}{Fármacos
  y tratamientos}
\item
  \href{https://www.nytimes.com/es/2020/07/06/espanol/ciencia-y-tecnologia/coronavirus-transmision-aire.html?action=click\&pgtype=Article\&state=default\&region=TOP_BANNER\&context=storylines_menu}{Cómo
  se transmite el coronavirus}
\item
  \href{https://www.nytimes.com/es/2020/07/14/espanol/estilos-de-vida/botiquin-medicina-coronavirus.html?action=click\&pgtype=Article\&state=default\&region=TOP_BANNER\&context=storylines_menu}{Prepara
  tu botiquín}
\end{itemize}

Advertisement

\protect\hyperlink{after-top}{Continue reading the main story}

\href{/es/section/opinion}{Opinión}

Supported by

\protect\hyperlink{after-sponsor}{Continue reading the main story}

Comentario

\hypertarget{el-mundo-levanta-un-muro-para-dejar-fuera-a-estados-unidos}{%
\section{El mundo levanta un muro para dejar fuera a Estados
Unidos}\label{el-mundo-levanta-un-muro-para-dejar-fuera-a-estados-unidos}}

El fracaso para enfrentar la pandemia en el país destruye la noción de
que Estados Unidos es mejor sin la gente y las ideas que nacen más allá
de sus fronteras.

\includegraphics{https://static01.nyt.com/images/2020/07/02/opinion/02manjoo1/merlin_173849757_1ff802dc-ae38-43af-bf96-1f9e58fa81e9-articleLarge.jpg?quality=75\&auto=webp\&disable=upscale}

\href{https://www.nytimes.com/by/farhad-manjoo}{\includegraphics{https://static01.nyt.com/images/2019/01/08/opinion/farhad-manjoo-opinion/farhad-manjoo-opinion-thumbLarge.png}}

Por \href{https://www.nytimes.com/by/farhad-manjoo}{Farhad Manjoo}

Es columnista de Opinión.

\begin{itemize}
\item
  3 de julio de 2020
\item
  \begin{itemize}
  \item
  \item
  \item
  \item
  \item
  \item
  \end{itemize}
\end{itemize}

\href{https://www.nytimes.com/2020/07/01/opinion/us-travel-ban-europe.html}{Read
in English}

\href{https://www.nytimes.com/newsletters/el-times}{Regístrate para
recibir nuestro boletín} con lo mejor de The New York Times.

\begin{center}\rule{0.5\linewidth}{\linethickness}\end{center}

Escucha en inglés esta columna. Para más artículos en audio, descarga
\href{https://www.audm.com/?utm_source=nytopinion\&utm_medium=embed\&utm_campaign=world_america_out}{\emph{Audm
para iPhone o Android}}\emph{.}

\hypertarget{listen-to-this-opinion-column}{%
\subsubsection{Listen to This Opinion
Column}\label{listen-to-this-opinion-column}}

Audio Recording by Audm

La situación podría considerarse poética si no fuera tan dolorosa.
Donald Trump ganó la Casa Blanca en buena medida por hacer una campaña
para cerrar las fronteras de Estados Unidos a básicamente todo aquel que
no fuera descendiente de europeos. ``¿Por qué aceptamos a todas esas
personas de países de mierda?'',
\href{https://www.washingtonpost.com/politics/trump-attacks-protections-for-immigrants-from-shithole-countries-in-oval-office-meeting/2018/01/11/bfc0725c-f711-11e7-91af-31ac729add94_story.html}{alguna
vez preguntó}, en referencia a los haitianos, los salvadoreños y los
africanos. ``Deberíamos aceptar a más gente de lugares como Noruega''.

Entonces, ¿cuál es la conclusión sobre la propia cercanía de Estados
Unidos con la letrina mundial de Trump ahora que ``lugares como
Noruega'' han decidido cerrarnos de manera indefinida \emph{sus}
fronteras?

En la
\href{https://www.consilium.europa.eu/en/press/press-releases/2020/06/30/council-agrees-to-start-lifting-travel-restrictions-for-residents-of-some-third-countries/}{lista
de naciones} a las que pronto reabrirán sus fronteras Noruega y el resto
de Europa, se encuentran tres del continente al que Trump echó por el
inodoro:
\href{https://www.nytimes.com/2020/06/30/world/europe/eu-reopening-blocks-us-travelers.html}{Argelia,
Marruecos y Ruanda}. Canadá también está en la lista. Al igual que
China, dando por cierto que corresponda el gesto.

Sin embargo, los Estados Unidos de Trump no están en la lista, porque no
estamos ni cerca de cumplir con los criterios que exige Europa para la
reducción de la propagación del coronavirus. El nivel de éxito de una
sociedad frente a una pandemia tal vez sea la medida más objetiva para
medir la capacidad nacional ---por no hablar de ``grandeza''--- y, en
este tema, como en muchos otros actualmente, Estados Unidos ronda el
fondo.

He vivido en Estados Unidos durante más de 30 años y no me viene a la
mente ningún fracaso nacional tan brutal y rotundo como este. Cuando veo
las gráficas que muestran cómo se disparan las infecciones en Estados
Unidos mientras el virus se calma
\href{https://www.nytimes.com/2020/06/29/briefing/coronavirus-mississippi-new-england-patriots-your-monday-briefing.html}{en
casi todos los otros países ricos}, siento el escozor de la derrota, la
miseria y la vergüenza.

Como inmigrante de Sudáfrica, me cuesta trabajo no considerar la
humillación europea sobre los viajes como el mejor de los merecidos para
la xenofobia de Trump. Como muchos otros ciudadanos, a veces me
encuentro con que doy por sentado el
\href{https://theweek.com/articles/654508/what-exactly-american-exceptionalism}{excepcionalismo
estadounidense}: la idea de que los ideales fundadores de Estados Unidos
nos dan una superioridad moral frente a naciones ``comunes y
corrientes'' y nos confiere una credibilidad y un entendimiento
especiales al momento de enfrentar crisis globales.

Sin embargo, el fracaso para enfrentar la pandemia en Estados Unidos
demuele la noción de que nuestro país es mejor sin la gente y las ideas
que nacen más allá de nuestras fronteras. Los últimos meses deberían
terminar de demostrar la absurda proposición según la cual Estados
Unidos disfruta de una especie de monopolio de la brillantez. No cabe la
menor duda de que no es el caso. En vez de aislarnos del planeta,
deberíamos invitar a otros a unirse al proyecto urgente de la
reconstrucción de Estados Unidos.

A menudo, menciono mi apoyo sobre este tema. Como lo he afirmado antes,
\href{https://www.nytimes.com/es/2019/01/22/espanol/opinion/fronteras-abiertas-muro-fronterizo.html}{estoy
a favor de abrir por completo las fronteras de Estados Unidos} a la
mayor parte del mundo. Mis razones principales son morales: no creo que
un país fundado sobre la idea de que todos somos iguales deba aislarse
de los miles de millones de personas con ambiciones que viven más allá
de nuestras costas.

También hay sólidos argumentos económicos y estratégicos en favor de la
apertura; el excepcionalismo estadounidense es imposible sin la
inmigración. \href{http://paulgraham.com/95.html}{La única manera} de
que un país con menos del cinco por ciento de la población mundial pueda
mantener la superioridad cultural y económica a largo plazo, a la que se
sienten con derecho muchos estadounidenses, es producir en conjunto
mucho más que el cinco por ciento de las mejores ideas del mundo.

La única forma de hacerlo es invitar al otro 95 por ciento. Pasé una
gran parte de mi carrera cubriendo Silicon Valley. Algunas de las
empresas más innovadoras del mundo ---desde Google e Intel hasta
Instagram y Stripe--- fueron fundadas por inmigrantes, y muchas personas
de la industria aseguran que
\href{https://www.nytimes.com/es/2017/02/23/espanol/por-que-silicon-valley-no-funcionaria-sin-inmigrantes.html}{nada
funcionaría en ese lugar sin la inmigración}.

No soy de esos izquierdosos que creen que Trump tiene toda la culpa de
nuestra respuesta fallida frente al virus. Aquí, el colapso fue tan
completo que
\href{https://www.theatlantic.com/magazine/archive/2020/06/underlying-conditions/610261/}{expone
males más grandes y persistentes}: nuestro tambaleante sistema de
atención médica, la crueldad de nuestra economía, nuestra red de
seguridad endeble como queso suizo y nuestra polarización política que
envenena una acción eficaz, pero destaca en suscitar guerras culturales
sin sentido.

La totalidad de nuestro fracaso es precisamente la razón para buscar el
éxito afuera\ldots{} y, sin embargo, Trump ha usado el virus como una
excusa para
\href{https://www.nytimes.com/2020/06/12/us/politics/coronavirus-trump-immigration-policies.html}{acelerar
sus restricciones a la inmigración}.

La semana pasada, Trump
\href{https://www.nytimes.com/2020/06/22/us/politics/trump-h1b-work-visas.html}{suspendió
la emisión de visas de trabajo} para cientos de miles de extranjeros,
desde personal del sector tecnológico y trabajadores estacionales en la
industria hotelera hasta niñeras y estudiantes.

Las restricciones afectan a otro grupo: los médicos.
\href{https://www.nbcnews.com/news/asian-america/fear-deportation-heightened-immigrant-doctors-h-1b-visas-amid-pandemic-n1204791}{Unos
127.000 doctores}, casi una cuarta parte de los médicos de Estados
Unidos, son inmigrantes. Muchos de ellos
\href{https://www.motherjones.com/coronavirus-updates/2020/06/immigrant-h1b-doctors-coronavirus-green-card/}{tratan
a pacientes con coronavirus} en comunidades sin suficientes
profesionales de la salud. Todo este tiempo, los médicos inmigrantes se
han tenido que preocupar no solo de la posibilidad de morir a causa del
virus mientras cuidan estadounidenses, sino también de que, si lo hacen,
podrían deportar a sus familias.

Es una locura. Y todavía hay más: si seguimos con el rechazo a los
extranjeros, ¿qué justifica nuestra suposición arrogante de que los
mejores y los más brillantes del mundo querrán venir aquí?

Por ejemplo, consideremos Ruanda, uno de los países que entró en la
lista europea. En 1994, Ruanda sufrió un genocidio, para el cual la
respuesta tristemente célebre de Estados Unidos y las Naciones Unidas
fue negarse a intervenir. Fueron asesinadas casi un millón de personas.
En los 26 años que han pasado desde ese suceso, Ruanda
\href{https://www.nytimes.com/2019/04/06/world/africa/rwanda-genocide-25-years.html}{se
ha reconstruido} y ahora puede presumir que tiene uno de los
\href{https://www.atlanticcouncil.org/blogs/africasource/rwandas-successes-and-challenges-in-response-to-covid-19/}{sistemas
médicos más capaces de África}. Los 13 millones de personas de Ruanda
tienen una cobertura casi universal de atención médica; el país usa
drones para transportar sangre y otros suministros a hospitales lejanos.

Y cuando llegó el coronavirus, gracias a que
\href{https://www.newyorker.com/news/news-desk/what-african-nations-are-teaching-the-west-about-fighting-the-coronavirus}{Ruanda}
estableció el rastreo de contactos para detener rápidamente la
propagación del virus, se convirtió en
\href{https://www.newyorker.com/news/news-desk/what-african-nations-are-teaching-the-west-about-fighting-the-coronavirus}{uno
de los varios países africanos} en sofocarlo. Hasta la fecha, solo hay
dos casos conocidos de muertes ruandesas por la COVID-19.

De verdad espero que los ruandeses y otros que son testigos de la
disfunción estadounidense no se sientan tentados a celebrar nuestra
caída. El fracaso de Estados Unidos frente al coronavirus es una pérdida
para el mundo, el cual ha dependido desde hace mucho tiempo del
liderazgo estadounidense para combatir crisis mundiales.

La lección es evidente: estamos juntos en esto. Es momento de dejar de
fingir que Estados Unidos, y los estadounidenses, tienen todas las
respuestas. Necesitamos toda la ayuda posible.

\hypertarget{horario-de-oficina-con-farhad-manjoo}{%
\subsection{Horario de oficina con Farhad
Manjoo}\label{horario-de-oficina-con-farhad-manjoo}}

\emph{Farhad quiere}
\href{https://www.nytimes.com/2019/05/16/opinion/farhad-office-hours.html?module=inline}{\emph{conversar
con los lectores}}\emph{. Si te interesa hablar (en inglés) con un
columnista de The New York Times sobre cualquier cosa, completa este
formulario. Farhad escogerá a algunos lectores para llamarlos.}

Farhad Manjoo es columnista de Opinión de The New York Times desde 2018.
Antes de eso
\href{https://www.nytimes.com/2019/07/10/opinion/pronoun-they-gender.html}{escribía}
la columna sobre tecnología
\href{https://www.nytimes.com/column/state-of-the-art}{State of the Art}
y escribió \emph{True Enough: Learning to Live in a Post-Fact Society}.
\href{https://twitter.com/fmanjoo}{@fmanjoo} •
\href{https://www.facebook.com/farhad.manjoo}{Facebook}

Advertisement

\protect\hyperlink{after-bottom}{Continue reading the main story}

\hypertarget{site-index}{%
\subsection{Site Index}\label{site-index}}

\hypertarget{site-information-navigation}{%
\subsection{Site Information
Navigation}\label{site-information-navigation}}

\begin{itemize}
\tightlist
\item
  \href{https://help.nytimes.com/hc/en-us/articles/115014792127-Copyright-notice}{©~2020~The
  New York Times Company}
\end{itemize}

\begin{itemize}
\tightlist
\item
  \href{https://www.nytco.com/}{NYTCo}
\item
  \href{https://help.nytimes.com/hc/en-us/articles/115015385887-Contact-Us}{Contact
  Us}
\item
  \href{https://www.nytco.com/careers/}{Work with us}
\item
  \href{https://nytmediakit.com/}{Advertise}
\item
  \href{http://www.tbrandstudio.com/}{T Brand Studio}
\item
  \href{https://www.nytimes.com/privacy/cookie-policy\#how-do-i-manage-trackers}{Your
  Ad Choices}
\item
  \href{https://www.nytimes.com/privacy}{Privacy}
\item
  \href{https://help.nytimes.com/hc/en-us/articles/115014893428-Terms-of-service}{Terms
  of Service}
\item
  \href{https://help.nytimes.com/hc/en-us/articles/115014893968-Terms-of-sale}{Terms
  of Sale}
\item
  \href{https://spiderbites.nytimes.com}{Site Map}
\item
  \href{https://help.nytimes.com/hc/en-us}{Help}
\item
  \href{https://www.nytimes.com/subscription?campaignId=37WXW}{Subscriptions}
\end{itemize}
