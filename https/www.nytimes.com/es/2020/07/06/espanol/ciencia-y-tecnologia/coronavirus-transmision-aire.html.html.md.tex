Sections

SEARCH

\protect\hyperlink{site-content}{Skip to
content}\protect\hyperlink{site-index}{Skip to site index}

\href{https://www.nytimes.com/es/section/ciencia-y-tecnologia}{Ciencia y
Tecnología}

\href{https://myaccount.nytimes.com/auth/login?response_type=cookie\&client_id=vi}{}

\href{https://www.nytimes.com/section/todayspaper}{Today's Paper}

\href{/es/section/ciencia-y-tecnologia}{Ciencia y
Tecnología}\textbar{}Más de doscientos expertos llegan a una misma
conclusión: el coronavirus se propaga por el aire

\url{https://nyti.ms/3gx9xKx}

\begin{itemize}
\item
\item
\item
\item
\item
\end{itemize}

\href{https://www.nytimes.com/es/spotlight/coronavirus?action=click\&pgtype=Article\&state=default\&region=TOP_BANNER\&context=storylines_menu}{El
brote de coronavirus}

\begin{itemize}
\tightlist
\item
  \href{https://www.nytimes.com/es/interactive/2020/espanol/mundo/coronavirus-en-estados-unidos.html?action=click\&pgtype=Article\&state=default\&region=TOP_BANNER\&context=storylines_menu}{Mapa
  y casos en EE. UU.}
\item
  \href{https://www.nytimes.com/es/2020/07/23/espanol/america-latina/bolivia-cloro-coronavirus-ivermectina.html?action=click\&pgtype=Article\&state=default\&region=TOP_BANNER\&context=storylines_menu}{Dióxido
  de cloro, ivermectina y más: ¿funcionan?}
\item
  \href{https://www.nytimes.com/es/interactive/2020/science/coronavirus-tratamientos-curas.html?action=click\&pgtype=Article\&state=default\&region=TOP_BANNER\&context=storylines_menu}{Fármacos
  y tratamientos}
\item
  \href{https://www.nytimes.com/es/2020/07/28/espanol/ciencia-y-tecnologia/anticuerpos-coronavirus-inmunidad.html?action=click\&pgtype=Article\&state=default\&region=TOP_BANNER\&context=storylines_menu}{Anticuerpos
  e inmunidad}
\item
  \href{https://www.nytimes.com/es/2020/04/29/espanol/estilos-de-vida/oximetro-para-que-sirve.html?action=click\&pgtype=Article\&state=default\&region=TOP_BANNER\&context=storylines_menu}{Oxímetros}
\end{itemize}

Advertisement

\protect\hyperlink{after-top}{Continue reading the main story}

Supported by

\protect\hyperlink{after-sponsor}{Continue reading the main story}

Ciencia

\hypertarget{muxe1s-de-doscientos-expertos-llegan-a-una-misma-conclusiuxf3n-el-coronavirus-se-propaga-por-el-aire}{%
\section{Más de doscientos expertos llegan a una misma conclusión: el
coronavirus se propaga por el
aire}\label{muxe1s-de-doscientos-expertos-llegan-a-una-misma-conclusiuxf3n-el-coronavirus-se-propaga-por-el-aire}}

La Organización Mundial de la Salud se ha resistido a la creciente
evidencia de que las partículas virales que flotan en interiores son
infecciosas, dicen algunos científicos. La agencia sostiene que la
investigación aún no es concluyente.

\includegraphics{https://static01.nyt.com/images/2020/07/04/science/06Virus-aire-ES-1/04virus-aerosols3-articleLarge.jpg?quality=75\&auto=webp\&disable=upscale}

Por \href{https://www.nytimes.com/by/apoorva-mandavilli}{Apoorva
Mandavilli}

\begin{itemize}
\item
  Publicado 6 de julio de 2020Actualizado 8 de julio de 2020
\item
  \begin{itemize}
  \item
  \item
  \item
  \item
  \item
  \end{itemize}
\end{itemize}

\href{https://www.nytimes.com/2020/07/04/health/239-experts-with-one-big-claim-the-coronavirus-is-airborne.html}{Read
in English}

\href{https://www.nytimes.com/newsletters/el-times}{Regístrate para
recibir nuestro boletín} con lo mejor de The New York Times.

\begin{center}\rule{0.5\linewidth}{\linethickness}\end{center}

El
\href{https://www.nytimes.com/2020/07/04/health/coronavirus-neanderthals.html}{coronavirus}
encuentra nuevas víctimas en todo el mundo, en bares y restaurantes,
oficinas, mercados y casinos, lo que ha dado lugar a preocupantes focos
de infección que confirman de manera cada vez más evidente lo que los
científicos han dicho durante meses: el virus permanece en el aire en
interiores e infecta a las personas cercanas.

Si la transmisión aérea es un factor significativo en la pandemia,
especialmente en espacios abarrotados con escasa ventilación, las
consecuencias para la contención serán importantes. Las mascarillas
podrían necesitarse en interiores, incluso en entornos con
distanciamiento social. Los trabajadores de la salud podrían necesitar
mascarillas N95, que filtran incluso las gotículas respiratorias más
pequeñas, para atender a los pacientes de coronavirus.

Los sistemas de ventilación en escuelas, asilos, residencias para
adultos mayores y negocios podrían necesitar minimizar la recirculación
de aire y agregar nuevos filtros poderosos. Se podrían necesitar luces
ultravioleta para matar partículas virales que floten en diminutas
gotículas en interiores.

La Organización Mundial de la Salud (OMS) ha sostenido que el
coronavirus se propaga principalmente por gotas respiratorias grandes
que, después de ser expulsadas por personas infectadas a través de la
tos y los estornudos, caen rápidamente al piso.

No obstante, en una carta abierta a la OMS, 239 científicos de 32 países
han descrito la evidencia que muestra que partículas más pequeñas pueden
infectar a las personas y exhortan a la agencia a que corrija sus
recomendaciones. Los investigadores planean publicar su carta en una
revista científica.

Incluso en su más reciente actualización sobre el coronavirus, difundida
el 29 de junio, la OMS dijo que la transmisión aérea del virus es
posible solo
\href{https://www.who.int/publications/i/item/WHO-2019-nCoV-IPC-2020.4}{después
de procedimientos médicos} que producen aerosoles o gotas de un tamaño
menor a los 5 micrones (un micrón o micra equivale a una millonésima
parte de un metro).

Según la OMS, una ventilación adecuada y mascarillas N95 solo son
necesarias en esas circunstancias. Su guía de control de infecciones,
antes y \href{https://www.who.int/infection-prevention/en/}{durante}
esta pandemia, más bien ha
\href{https://www.who.int/infection-prevention/campaigns/clean-hands/5may2019/en/}{promovido}
de
\href{https://www.who.int/infection-prevention/campaigns/ipc-global-survey-2019/en/}{manera
intensa} la importancia del
\href{https://www.who.int/gpsc/ipc/en/}{lavado de manos} como una
estrategia primaria de prevención, a pesar de que hay evidencia limitada
sobre la transmisión del virus a través de superficies. (Los Centros
para el Control y la Prevención de Enfermedades ---CDC, por su sigla en
inglés--- de Estados Unidos ahora dicen que es probable que las
superficies solo desempeñen un papel menor).

Benedetta Allegranzi, coordinadora de prevención y control de
infecciones de la OMS, dijo que la evidencia de que el virus se propaga
por el aire no era convincente.

``Especialmente en el último par de meses hemos expresado en diversas
ocasiones que consideramos la transmisión aérea como posible, pero
ciertamente no respaldada por evidencia sólida o por lo menos clara'',
dijo ella. ``Existe un fuerte debate sobre esto''.

No obstante, entrevistas con cerca de 20 científicos ---incluyendo a una
docena de consultores de la OMS y a varios miembros del comité que
produjo la guía--- y correos electrónicos internos pintan un panorama de
una organización que, a pesar de sus buenas intenciones, está fuera de
sintonía con la ciencia.

Ya sea que viaje en grandes gotas que se elevan por el aire después de
un estornudo o en gotículas mucho más pequeñas exhaladas que pueden
deslizarse por el aire a través de una habitación, el coronavirus se
transporta por el aire y puede infectar a las personas cuando lo
inhalan, afirman estos expertos.

La mayoría de estos expertos comprenden que la OMS tiene cada vez un
programa de trabajo más extenso y menos presupuesto, y señalaron las
difíciles relaciones políticas que tiene que gestionar, especialmente
con Estados Unidos y China. Alabaron al personal de la OMS por realizar
reuniones diarias y responder preguntas sobre la pandemia de manera
incansable.

Pero el comité de prevención y control de infecciones en particular,
según los expertos, está atado por una rígida y sobremedicalizada visión
de la evidencia científica, es lento y se empeña en evitar riesgos en
cuanto a las actualizaciones de su guía y permite que algunas voces
conservadoras acallen a quienes expresan desacuerdo.

``Todos morirán defendiendo su punto de vista'', dijo una consultora de
la OMS de larga carrera, quien no desea ser identificada debido a su
continua labor para la organización. Incluso sus simpatizantes más
leales dijeron que el comité debería
\href{https://twitter.com/JoyAgnost/status/1263802269658644480}{diversificar
sus áreas de especialidad} y relajar sus criterios de prueba,
especialmente en un brote de rápido movimiento.

``Me frustran los temas de flujo de aire y el tamaño de las partículas,
por supuesto'', dijo Mary-Louise McLaws, una integrante del comité y
epidemióloga en la Universidad de Nueva Gales del Sur en Sídney.

``Si comenzáramos a volver a analizar el flujo de aire, tendríamos que
estar preparados para cambiar mucho de lo que hacemos'', dijo. ``Pienso
que es una buena idea, una muy buena idea, pero causará una enorme
sacudida en la sociedad del control de infecciones''.

A principios de abril, un grupo de 36 expertos en calidad del aire y
aerosoles instó a la OMS a considerar la creciente evidencia sobre la
transmisión aérea del coronavirus. La agencia respondió de manera pronta
al llamar a Lidia Morawska, lideresa del grupo y consultora de la OMS
desde hace mucho tiempo, a que organizara una reunión.

No obstante, el debate fue dominado por algunos expertos que respaldaban
firmemente el lavado de manos y sentían que debía ser enfatizado más que
los aerosoles, según algunos participantes, y las recomendaciones del
comité permanecieron sin cambios.

Morawska y otros señalaron
\href{https://www.nytimes.com/2020/05/12/health/coronavirus-choir.html}{varios}
\href{https://www.nytimes.com/2020/04/20/health/airflow-coronavirus-restaurants.html}{incidentes}
que indican la
\href{https://news.sky.com/story/coronavirus-circulating-air-may-have-spread-covid-19-to-1-500-german-meat-plant-staff-12014156}{transmisión
aérea} del virus, particularmente en espacios interiores con escasa
ventilación y mucha gente. Dijeron que la OMS estaba marcando una
diferencia artificial entre los aerosoles diminutos y las gotas más
grandes, a pesar de que las personas infectadas producen ambos.

``Hemos sabido desde 1946 que toser y hablar genera aerosoles'', dijo
Linsey Marr, una experta en transmisión aérea de los virus del Instituto
Politécnico y Universidad Estatal de Virginia.

Los científicos no han sido capaces de cultivar el coronavirus de
aerosoles en el laboratorio. Sin embargo, eso no significa que los
aerosoles no infecten. Marr dijo que la mayoría de las
\href{https://www.sciencedirect.com/science/article/pii/S0013935120307143?via\%3Dihub}{muestras
en esos experimentos han provenido de cuartos de hospital} con buen
flujo de aire que diluiría los niveles virales.

En la mayoría de los edificios, dijo, ``la frecuencia de reemplazo de
aire es usualmente mucho más baja, lo que permite al virus acumularse en
el aire y representar un riesgo más alto''.

La OMS también se está basando en una definición anticuada de
transmisión aérea, dijo Marr. La agencia cree que un patógeno que se
transporta de manera aérea, como el virus del sarampión, tiene que ser
altamente infeccioso y viajar distancias largas.

La gente generalmente ``piensa y habla sobre transmisión aérea de manera
profundamente estúpida'', dijo Bill Hanage, un epidemiólogo de la
Escuela de Salud Pública T. H. Chan de la Universidad de Harvard.

``Tenemos esta idea de que transmisión aérea significa gotas suspendidas
en el aire capaces de infectarte muchas horas después, deslizándose por
las calles, introduciéndose en los buzones de correos y encontrando cómo
ingresar a los hogares en todos lugares'', dijo Hanage.

\includegraphics{https://static01.nyt.com/images/2020/07/04/science/06Virus-aire-ES-2/04virus-aerosols4-articleLarge.jpg?quality=75\&auto=webp\&disable=upscale}

Todos los expertos están de acuerdo en que el coronavirus no se comporta
de esa manera. Marr y otros dijeron que el coronavirus parece ser más
infeccioso cuando las personas estuvieron en contacto prolongado a una
distancia cercana, especialmente en interiores, y aún más en
\href{https://www.nytimes.com/es/2020/07/03/espanol/el-misterio-de-los-superpropagadores-de-coronavirus.html}{eventos
superpropagadores}, exactamente lo que los científicos esperarían de la
transmisión por aerosoles.

\hypertarget{principio-de-precauciuxf3n}{%
\subsection{Principio de precaución}\label{principio-de-precauciuxf3n}}

La OMS ha estado en desacuerdo con grupos de científicos más de una vez
durante esta pandemia.

La agencia se quedó atrás de la mayoría de sus países miembros al
\href{https://www.nytimes.com/2020/06/05/health/coronavirus-masks-who.html}{respaldar
los protectores faciales} para el público. Mientras que otras
organizaciones, incluidos los CDC, han reconocido desde hace mucho la
importancia de la transmisión
\href{https://www.nytimes.com/es/2020/06/29/espanol/mundo/coronavirus-asintomaticos.html}{por
personas asintomáticas}, la OMS aún
\href{https://www.nytimes.com/2020/06/09/health/coronavirus-asymptomatic-world-health-organization.html}{sostiene
que la transmisión asintomática es rara}.

``A nivel del país, muchos del personal técnico de la OMS se están
rascando la cabeza'', dijo un consultor en una oficina regional en el
sudeste asiático, que no deseaba ser identificado por temor a perder su
contrato. ``Esto no nos está dando credibilidad''.

El consultor recordó que los miembros del personal de la OMS en su país
fueron los únicos que siguieron sin cubrebocas después de que el
gobierno los avaló.

Muchos expertos dijeron que la OMS debería adoptar lo que algunos
llamaron un ``principio de precaución'' y otros nombraron ``necesidades
y valores'': la idea de que incluso sin evidencia definitiva, la agencia
debería asumir lo peor del virus, aplicar el sentido común y recomendar
la mejor protección posible.

``No hay una prueba incontrovertible de que el SARS-CoV-2 viaje o sea
transmitido de manera significativa por aerosoles, pero no hay evidencia
de que no sea así'', dijo Trish Greenhalgh, médica de atención primaria
en la Universidad de Oxford, en el Reino Unido.

``Así que en este momento tenemos que tomar una decisión ante la
incertidumbre y, Dios mío, será desastroso si tomamos la decisión
equivocada'', dijo. ``Así que, ¿por qué no ponernos mascarillas durante
algunas semanas? Por si acaso\ldots{}''.

Después de todo, la OMS parece dispuesta a aceptar sin mucha evidencia
la idea de que el virus podría transmitirse mediante las superficies,
mencionaron ella y otros investigadores, incluso a pesar de que otras
agencias de salud han dejado de enfatizar esa vía.

``Estoy de acuerdo en que la transmisión por fómites no está
directamente demostrada para este virus'', dijo Allegranzi, la
coordinadora de prevención y control de infecciones de la OMS, en
referencia a los objetos que podrían ser infecciosos. ``Sin embargo, se
sabe que otros coronavirus y virus respiratorios son transmitidos, y se
ha demostrado así, por contacto con fómites''.

La agencia también debe considerar las necesidades de todas las naciones
que la integran, incluidas aquellas con recursos limitados, y asegurarse
de que sus recomendaciones consideren ``la disponibilidad, la
viabilidad, el cumplimiento y las implicaciones de recursos'', dijo
ella.

Los aerosoles podrían desempeñar algún papel limitado en propagar el
virus, dijo Paul Hunter, un miembro del comité de prevención de
infecciones y profesor de Medicina en la Universidad de East Anglia en
el Reino Unido.

No obstante, si la OMS decide presionar para que se apliquen medidas de
control riguroso sin pruebas, los hospitales en países de bajos o
medianos ingresos podrían verse forzados a destinar recursos escasos de
otros programas cruciales.

``Ese es el equilibrio que una organización como la OMS tiene que
lograr'', dijo. ``Es lo más sencillo del mundo decir: `Tenemos que
seguir el principio de precaución', e ignorar los costos de oportunidad
de ello''.

En entrevistas, otros científicos criticaron este punto de vista como
paternalista. ``¿No vamos a decir lo que realmente pensamos porque
creemos que algunos países no pueden lidiar con eso? No me parece que
eso sea correcto'', dijo Don Milton, un experto en aerosoles de la
Universidad de Maryland.

Incluso los cubrebocas de tela, si toda la gente los usa, pueden reducir
de manera significativa la transmisión, y la OMS debería decirlo
claramente, agregó.

Varios expertos criticaron los mensajes de la OMS durante la pandemia, y
dijeron que el personal parece valorar la perspectiva científica por
encima de la claridad.

``Lo que dices está diseñado a ayudar a las personas a entender la
naturaleza de un problema de salud pública'', dijo William Aldis, un
antiguo colaborador de la OMS en Tailandia. ``Eso es diferente de solo
describir científicamente una enfermedad o un virus''.

La OMS tiende a describir ``una ausencia de evidencia como evidencia de
ausencia'', agregó Aldis. En abril, por ejemplo,
\href{https://www.reuters.com/article/us-health-coronavirus-who-idUSKCN2270FB}{la
OMS dijo}: ``Actualmente no hay evidencia de que las personas que se han
recuperado de la COVID-19 y tienen anticuerpos estén protegidas de una
segunda infección''.

La declaración tenía la intención de indicar incertidumbre, pero la
redacción avivó la inquietud entre el público y se ganó el reproche de
varios expertos y periodistas. Más tarde, la OMS se echó atrás en sus
comentarios.

En una instancia menos pública, la OMS dijo que ``no hay evidencia que
sugiera'' que la gente con VIH estaba en mayor riesgo por el
coronavirus. Después de que Joseph Amon, el director de salud global en
la Universidad de Drexel en Filadelfia, que había sido parte de muchos
comités de agencias, señaló que la redacción era engañosa, la OMS la
cambió para decir que el
\href{https://www.who.int/emergencies/diseases/novel-coronavirus-2019/question-and-answers-hub/q-a-detail/q-a-on-covid-19-hiv-and-antiretrovirals}{nivel
de riesgo era ``desconocido''}.

Pero el personal y algunos miembros de la OMS dijeron que los críticos
no dieron suficiente crédito a sus comités.

``Los que pueden haberse sentido frustrados pueden no ser conscientes de
cómo trabajan los comités de expertos de la OMS, y ellos trabajan lenta
y deliberadamente'', dijo McLaws.

Soumya Swaminathan, la directora científica de la OMS, dijo que los
miembros del personal de la agencia tratan de evaluar la nueva evidencia
científica lo más rápido posible, pero sin sacrificar la calidad de su
revisión. Agregó que la agencia intentará ampliar los conocimientos y
las comunicaciones de sus comités para asegurarse de que todos sean
escuchados.

``Nos tomamos en serio cuando los periodistas o los científicos o
cualquiera nos desafía y dice que podemos hacerlo mejor'', dijo.
``Definitivamente queremos hacerlo mejor''.

Apoorva Mandavilli es reportera del Times y se enfoca en ciencia y salud
global. En 2019 ganó el premio Victor Cohn a la Excelencia en Reportaje
sobre Ciencias Médicas.
\href{https://twitter.com/apoorva_nyc}{@apoorva\_nyc}

Advertisement

\protect\hyperlink{after-bottom}{Continue reading the main story}

\hypertarget{site-index}{%
\subsection{Site Index}\label{site-index}}

\hypertarget{site-information-navigation}{%
\subsection{Site Information
Navigation}\label{site-information-navigation}}

\begin{itemize}
\tightlist
\item
  \href{https://help.nytimes.com/hc/en-us/articles/115014792127-Copyright-notice}{©~2020~The
  New York Times Company}
\end{itemize}

\begin{itemize}
\tightlist
\item
  \href{https://www.nytco.com/}{NYTCo}
\item
  \href{https://help.nytimes.com/hc/en-us/articles/115015385887-Contact-Us}{Contact
  Us}
\item
  \href{https://www.nytco.com/careers/}{Work with us}
\item
  \href{https://nytmediakit.com/}{Advertise}
\item
  \href{http://www.tbrandstudio.com/}{T Brand Studio}
\item
  \href{https://www.nytimes.com/privacy/cookie-policy\#how-do-i-manage-trackers}{Your
  Ad Choices}
\item
  \href{https://www.nytimes.com/privacy}{Privacy}
\item
  \href{https://help.nytimes.com/hc/en-us/articles/115014893428-Terms-of-service}{Terms
  of Service}
\item
  \href{https://help.nytimes.com/hc/en-us/articles/115014893968-Terms-of-sale}{Terms
  of Sale}
\item
  \href{https://spiderbites.nytimes.com}{Site Map}
\item
  \href{https://help.nytimes.com/hc/en-us}{Help}
\item
  \href{https://www.nytimes.com/subscription?campaignId=37WXW}{Subscriptions}
\end{itemize}
