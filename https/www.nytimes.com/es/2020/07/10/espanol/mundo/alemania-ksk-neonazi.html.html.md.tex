Sections

SEARCH

\protect\hyperlink{site-content}{Skip to
content}\protect\hyperlink{site-index}{Skip to site index}

\href{/es/section/mundo}{Mundo}\textbar{}Alemania enfrenta `un enemigo
interno': los neonazis se infiltran en el ejército

\url{https://nyti.ms/2ZTH23c}

\begin{itemize}
\item
\item
\item
\item
\item
\item
\end{itemize}

\includegraphics{https://static01.nyt.com/images/2020/07/05/world/10germany-KSK-00/merlin_174034917_e33227ac-794e-4500-83e4-fc410dee5818-articleLarge.jpg?quality=75\&auto=webp\&disable=upscale}

Europa

\hypertarget{alemania-enfrenta-un-enemigo-interno-los-neonazis-se-infiltran-en-el-ejuxe9rcito}{%
\section{Alemania enfrenta `un enemigo interno': los neonazis se
infiltran en el
ejército}\label{alemania-enfrenta-un-enemigo-interno-los-neonazis-se-infiltran-en-el-ejuxe9rcito}}

El descubrimiento de explosivos plásticos y recuerdos nazis en casa de
un soldado de élite alemán ha encendido las alarmas sobre la
infiltración de la extrema derecha en las instituciones democráticas.

Ejercicios de tiro en la base de la KSK, las fuerzas especiales
militares en Calw, AlemaniaCredit...Laetitia Vancon para The New York
Times

Supported by

\protect\hyperlink{after-sponsor}{Continue reading the main story}

\href{https://www.nytimes.com/by/katrin-bennhold}{\includegraphics{https://static01.nyt.com/images/2018/07/13/multimedia/author-katrin-bennhold/author-katrin-bennhold-thumbLarge.png}}

Por \href{https://www.nytimes.com/by/katrin-bennhold}{Katrin Bennhold}

\begin{itemize}
\item
  10 de julio de 2020
\item
  \begin{itemize}
  \item
  \item
  \item
  \item
  \item
  \item
  \end{itemize}
\end{itemize}

\href{https://www.nytimes.com/2020/07/03/world/europe/germany-military-neo-nazis-ksk.html}{Read
in English}

\href{https://www.nytimes.com/newsletters/el-times}{Regístrate para
recibir nuestro boletín} con lo mejor de The New York Times.

\begin{center}\rule{0.5\linewidth}{\linethickness}\end{center}

CALW, Alemania --- Cuando los alemanes salían de su confinamiento por
coronavirus en mayo, unos comandos policiales pararon frente a la
propiedad rural de un sargento mayor de las fuerzas especiales, la
unidad militar secreta más entrenada del país.

Traían una excavadora.

El apodo del sargento mayor era Ovejita. Se sospechaba que era neonazi.
Enterrados en el jardín, la policía encontró dos kilogramos de
explosivos plásticos PENT, un detonador, un fusible, un AK-47, un
silenciador, dos cuchillos, una ballesta y miles de municiones, muchas
de las cuales se cree que fueron robadas al ejército alemán.

También encontraron un cancionero de las SS, 14 ediciones de una revista
para ex miembros de las Waffen SS y una gran cantidad de otros recuerdos
nazis.

``Tenía un plan'', dijo Eva Högl, comisionada parlamentaria de Alemania
para las fuerzas armadas. ``Y él no es el único''.

Alemania tiene un problema. Durante años, los políticos y los jefes de
seguridad rechazaron la noción de cualquier infiltración de extrema
derecha en los servicios de seguridad, y hablaron solo de ``casos
individuales''. La idea de que existieran redes fue descartada. Los
superiores de aquellos revelados como extremistas fueron protegidos. Las
armas y la munición desaparecían de las reservas militares sin que
hubiera una verdadera investigación.

El gobierno ahora comienza a despertar. Los casos de extremistas de
derecha en el ejército y la policía, con algunos que acumularon armas y
explosivos, se han multiplicado de manera alarmante. Los principales
funcionarios de inteligencia y altos comandantes de la nación ahora
actúan para enfrentar un problema que se ha vuelto demasiado peligroso
como para ignorarlo.

El problema se ha profundizado con el surgimiento del partido
Alternativa por Alemania, o AfD, que
\href{https://www.nytimes.com/2019/10/26/world/europe/afd-election-east-germany-hoecke.html}{legitimizó
una ideología de extrema derecha} que utilizó la llegada de más de un
millón de inmigrantes en 2015 ---y más recientemente la
\href{https://www.nytimes.com/2020/05/18/world/europe/coronavirus-germany-far-right.html}{pandemia
de coronavirus}--- para generar una sensación de crisis inminente.

\includegraphics{https://static01.nyt.com/images/2020/06/29/world/10germany-KSK-01/merlin_174042501_09225840-1504-43db-87ec-5e49ad58a061-articleLarge.jpg?quality=75\&auto=webp\&disable=upscale}

Lo más preocupante para las autoridades es que los extremistas parecen
estar concentrados en la unidad militar que se supone es la más dedicada
y de élite del estado alemán, las fuerzas especiales, conocidas por sus
sigla en alemán, KSK.

La semana pasada, la ministra de Defensa de Alemania, Annegret
Kramp-Karrenbauer, dio el drástico paso de
\href{https://www.nytimes.com/2020/07/01/world/europe/german-special-forces-far-right.html}{disolver
una compañía de combate} en las KSK a la que se consideró infestada de
extremistas. Ovejita, el sargento mayor cuyo alijo de armas fue
descubierto en mayo, era miembro.

Unos 48.000 cartuchos de municiones y 62 kilogramos de explosivos han
desaparecido por completo de la KSK, dijo.

La agencia de contrainteligencia militar de Alemania ahora investiga a
más de 600 soldados por extremismo de extrema derecha, de 184.000 que
pertenecen al ejército. Unos 20 de ellos están en las KSK, una
proporción que es cinco veces mayor que en otras unidades.

Pero a las autoridades alemanas les preocupa que el problema pueda ser
mucho mayor y que otras instituciones de seguridad también hayan sido
infiltradas. En los últimos 13 meses, terroristas de extrema derecha han
\href{https://www.nytimes.com/2019/06/17/world/europe/germany-terrorism-walter-lubcke.html}{asesinado
a un político},
\href{https://www.nytimes.com/2019/10/10/world/europe/germany-synagogue-attack.html?searchResultPosition=10}{atacado
una sinagoga} y
\href{https://www.nytimes.com/2020/02/20/world/europe/germany-hanau-shisha-bar-shooting.html?searchResultPosition=28}{matado
a tiros a nueve inmigrantes} y alemanes descendientes de inmigrantes.

Thomas Haldenwang, presidente de la agencia de inteligencia nacional de
Alemania, ha identificado el extremismo y terrorismo de extrema derecha
como el
``\href{https://www.nytimes.com/2020/02/21/world/europe/germany-shooting-terrorism.html?searchResultPosition=25}{mayor
peligro} a la democracia alemana hoy en día''.

En entrevistas realizadas a lo largo del año con agentes militares y de
inteligencia, y con miembros confesos de la extrema derecha, ellos
describieron redes nacionales de soldados y agentes de policía, tanto en
funciones como retirados, vinculadas con la extrema derecha.

En muchos casos, los soldados han usado las redes como un modo de
prepararse para cuando predicen que el orden democrático de Alemania
colapsará. Lo llaman Día X. Los funcionarios temen que sea realmente un
pretexto para incitar actos terroristas o, peor aún, un golpe de estado.

``Para los extremistas de extrema derecha, la preparación para el Día X
y su precipitación se mezclan entre sí'', me dijo Martina Renner,
legisladora del comité de seguridad nacional del Parlamento alemán.

Los lazos, dicen los funcionarios, a veces llegan hasta las viejas redes
neonazis y la escena intelectual más pulida de la llamada
\href{https://www.nytimes.com/2018/12/27/world/europe/germany-far-right-generation-identity.html}{Nueva
Derecha}. Los extremistas acumulan armas, mantienen casas de seguridad
y, en algunos casos, listas de enemigos políticos.

Este mes surgió otro caso, de un reservista, ahora suspendido, que tenía
una lista con los números celulares y las direcciones de 17 destacados
políticos, que han sido alertados. El caso condujo a, al menos, otras
nueve redadas en todo el país el viernes 3 de julio.

Algunos medios alemanes se han referido a un
\href{https://taz.de/Rechtes-Netzwerk-in-der-Bundeswehr/!5548926/}{``ejército
en la sombra''}, trazando paralelos a la década de 1920, cuando las
células nacionalistas dentro del ejército acumularon armas, planearon
golpes y conspiraron para derrocar la democracia.

La mayoría de los funcionarios aún rechazan esta analogía. Pero la
sorprendente falta de comprensión de los números involucrados, incluso
en los niveles más altos del gobierno, ha contribuido a una profunda
inquietud.

``Una vez que realmente comenzaron a buscar, encontraron muchos casos'',
dijo Konstantin von Notz, presidente adjunto del comité de supervisión
de inteligencia en el Parlamento alemán. ``Cuando tienes cientos de
casos individuales comienza a parecer que tenemos un problema
estructural. Es extremadamente preocupante''.

Von Notz señaló que Brendan Tarrant, quien masacró a 51 fieles
musulmanes el año pasado en dos mezquitas en Christchurch, Nueva
Zelanda, había viajado por Europa un año antes e incluyó una línea
ominosa en su manifesto.

Image

Un memorial para las víctimas del tiroteo en dos mezquitas el año pasado
en Christchurch, Nueva Zelanda, que mató a 51 fieles. El asesino había
escrito sobre la infiltración nacionalista de las fuerzas armadas
europeas.Credit...Adam Dean para The New York Times

``Yo estimaría que la cantidad de soldados en las fuerzas armadas
europeas que también pertenecen a grupos nacionalistas ascienden a
cientos de miles, con la misma cantidad de empleados en puestos de
fuerzas de seguridada'', escribió Tarrant.

Los investigadores, dijo von Notz, ``deben tomarse esas palabras en
serio''.

Pero investigar el problema es en sí complicado: incluso la agencia de
contrainteligencia militar, encargada de monitorear el extremismo dentro
de las fuerzas armadas, puede estar infiltrada.

Un investigador de alto rango en la unidad de extremismo fue suspendido
en junio después de compartir material confidencial de la redada de mayo
con un contacto en las KSK, quien a su vez se lo pasó a, al menos, otros
ocho soldados, avisándoles que a continuación la agencia podría dirigir
su atención a ellos.

``Si las personas que están destinadas a proteger nuestra democracia
están conspirando contra ellas, tenemos un gran problema'', dijo Stephan
Kramer, presidente de la agencia de inteligencia del estado de Turingia.
``¿Cómo los encuentras?''.

``Estos son hombres endurecidos por la batalla que saben cómo evadir la
vigilancia porque están entrenados para realizar la vigilancia ellos
mismos'', agregó.

``Aquí estamos tratando con un enemigo interno''.

\hypertarget{al-interior-de-la-casa-de-tiro}{%
\subsection{Al interior de la `casa de
tiro'}\label{al-interior-de-la-casa-de-tiro}}

El aire dentro de la ``casa de tiro'' olía a acre, de tantas balas que
se habían disparado.

Estaba de pie en el campo de tiro en las afueras de la tranquila ciudad
alemana de Calw, en la región de la Selva Negra, después de haber sido
invitada a inicios de este año para una rara visita a la base de las
KSK, la más vigilada del país.

Un soldado camuflado con un rifle de asalto G36 se agazapó a lo largo
del marco de una puerta rota. Dos sombras aparecieron. El soldado
disparó cuatro veces ---cabeza, torso, cabeza, torso--- y luego eliminó
sistemáticamente otras dos decenas de ``enemigos''. No falló una.

Image

Objetivos en la ``casa de tiro'' de la base de las KSKCredit...Laetitia
Vancon para The New York Times

Las KSK son la respuesta de Alemania a los Navy Seals. Pero en estos
días su comandante, el general Markus Kreitmayr, un bávaro afable que ha
combatido en Bosnia, Kosovo y Afganistán, es un hombre dividido entre su
lealtad hacia ellos y el reconocimiento de que tiene un grave problema
en sus manos.

El general llegó tarde a nuestra entrevista. Acababa de pasar cuatro
horas interrogando a un miembro de su unidad sobre una fiesta en la que,
se informó, media decena de soldados de las KSK habían hecho el saludo
de Hitler.

``No puedo explicar por qué supuestamente hay tantos casos de
`extremismo de extrema derecha' en el ejército'', dijo. Las KSK están
``claramente más afectadas que otros, eso parece ser un hecho''.

Nunca fue fácil ser soldado en la Alemania de la posguerra. Dada su
historia nazi y la destrucción que le impuso a Europa en la Segunda
Guerra Mundial, el país mantiene una relación conflictiva con sus
militares.

Durante décadas, Alemania intentó forjar una fuerza que representase a
una sociedad democrática y sus valores. Pero en 2011
\href{https://www.nytimes.com/2011/07/01/world/europe/01germany.html}{abolió
el servicio militar obligatorio} y pasó a ser una fuerza de voluntarios.
Como resultado, los militares son cada vez menos un reflejo de la
sociedad alemana en general, y representan a una porción más estrecha de
la misma.

El general Kreitmayr dijo que ``un gran porcentaje'' de sus soldados son
alemanes orientales, una región donde a la AfD le va
desproporcionadamente bien. Aproximadamente la mitad de los hombres en
la lista de miembros de las KSK de los cuales se sospecha ser
extremistas de extrema derecha también son del este, agregó.

Image

El general Markus Kreithmayr, en la retaguardia, ha calificado la crisis
actual en la unidad KSK ``la fase más difícil de su
historia''.Credit...Laetitia Vancon para The New York Times

El general ha llamado a la crisis actual en la unidad de ``la fase más
difícil de su historia''.

En nuestra entrevista, dijo que no podía descartar un grado
significativo de infiltración de la extrema derecha. ``No sé si hay un
ejército en la sombra en Alemania'', me dijo.

``Pero estoy preocupado'', dijo, ``y no solo como el comandante de las
KSK, sino como un ciudadano: que al final exista algo así y que tal vez
nuestra gente forme parte de ello''.

Los funcionarios hablan de un cambio perceptible ``en valores'' entre
los nuevos reclutas. En conversaciones, los propios soldados, que no
pudieron ser identificados debido a las pautas de la unidad, dijeron que
si había un punto de inflexión en la unidad, vino con la crisis
migratoria de 2015.

Cuando cientos de miles de solicitantes de asilo de Siria y Afganistán
se dirigían a Alemania, el estado de ánimo en la base era de ansiedad,
recordaron.

``Somos soldados encargados de defender este país y ellos solo abrieron
las fronteras, sin control'', recordó un oficial. ``Estábamos al
límite''.

Fue en esta atmósfera que un soldado de las KSK de 30 años de Halle, en
el este de Alemania, estableció un grupo en Telegram para soldados,
oficiales de policía y otros que compartían la creencia de que los
migrantes destruirían el país.

Se llamaba André Schmitt. Pero lo conocen con el apodo de Hannibal.

\hypertarget{la-red-de-hannibal}{%
\subsection{La red de Hannibal}\label{la-red-de-hannibal}}

En una casa en el oeste rural de Alemania, detrás de una cortina de
cadenas y más allá de una ballesta en el pasillo, una sala tipo mazmorra
bañada en luz púrpura se abre en un área de bar. La imagen de gran
tamaño de una mujer desnuda domina la pared del fondo.

En una casa en el oeste rural de Alemania, detrás de una cortina de
cadenas y más allá de una ballesta en el pasillo, una sala tipo mazmorra
bañada en luz púrpura se abre en un área de bar. La gigantografía de una
mujer desnuda domina la pared del fondo.

Fue allí donde conocí a Schmitt a principios de este año. Dio permiso
para que se usara su nombre, pero no quería que se revelara la ubicación
ni ninguna fotografía.

Abandonó el servicio activo en septiembre pasado, después de que
encontraron granadas de entrenamiento robadas en un edificio que
pertenecía a sus padres. Pero, dice, aún tiene su red: ``Fuerzas
especiales, inteligencia, ejecutivos de negocios, masones'', dijo. Se
encuentran aquí regularmente. La casa, dice, es propiedad de un
partidario rico.

Image

Un taller de entrenamiento de defensa táctica en marzo en el estado de
Renania del Norte-Westphalia organizado por Uniter, una red privada para
personal de seguridad.Credit...Laetitia Vancon para The New York Times

``Las fuerzas son como una gran familia'', me dijo Schmitt, ``todos se
conocen''.

Cuando armó sus chats de Telegram en 2015, los integró geográficamente
---norte, sur, este, oeste---, igual que el ejército alemán. En
paralelo, dirigió un grupo llamado Uniter, una organización para
profesionales relacionados con la seguridad que brinda beneficios
sociales pero también capacitación paramilitar.

Varios ex integrantes de sus chats están siendo investigados por
fiscales por tramar terrorismo. Algunos buscan bolsas para cadáveres.
Uno enfrenta juicio.

La situación de Schmitt es más compleja. Reconoció haber servido como
informante en las KSK para la agencia de contrainteligencia militar a
mediados de 2017, cuando se reunía regularmente con un oficial de
enlace. Hoy el ejército paga para que obtenga un título en negocios.

Él mismo nunca fue nombrado sospechoso. Los funcionarios alemanes
negaron que lo protegieran. Pero esta semana, la agencia de inteligencia
nacional anunció que estaba poniendo bajo vigilancia a Uniter, su red
actual.

Las autoridades dieron por primera vez con sus chats en 2017, cuando
investigaban a un soldado de la red sospechoso de organizar un complot
terrorista.

Los investigadores ahora están averiguando si los chats y Uniter fueron
el primer esqueleto de una red de extrema derecha que ha infiltrado a
instituciones estatales. Hasta el momento, no lo pueden asegurar. The
New York Times obtuvo declaraciones policiales de Schmitt y otros de su
red relacionados con el caso de 2017.

Inicialmente, dicen Schmitt y otros miembros, los chats eran para
compartir información, en gran parte de las supuestas amenazas
planteadas por los inmigrantes, que Schmitt admitió a la policía que
había inflado como un modo de ``motivar'' a las personas.

Image

Una familia de refugiados esperó para abordar un tren a Alemania en la
estación de trenes de Keleti en Budapest en 2015. Europa experimentó una
gran afluencia de personas que huían de conflictos en Siria y Afganistán
ese año.Credit...Mauricio Lima para The New York Times

``Se trataba de conmoción interna por las células dormidas y los grupos
extremistas alrededor del mundo, las formaciones de pandillas, las
amenazas terroristas'', dijo Schmitt a la policía.

Los chats eran populares entre los soldados de las KSK. Schmitt dijo que
contó a 69 de sus camaradas en la red en 2015.

Un compañero soldado en las KSK, identificado por los investigadores
como Robert P., pero conocido como Petrus, quien administraba dos de los
chats, le dijo a la policía dos años después que quizás había más del
doble que eso: ``Tengo que decir, presumiblemente la mitad de la unidad
estaba allí''.

Pronto los chats se transformaron en una plataforma para compartir
información dedicada a prepararse para el Día X. Mientras bebe agua
mineral, Schmitt describe esto como un ``juego de guerra''. Retrató una
Europa amenazada por pandillas, islamistas y Antifa. Los llamó ``tropas
enemigas en nuestro terreno''.

Su red ayudó a los miembros a prepararse para responder a lo que él
describió como un conflicto inevitable, a veces actuando por su cuenta.

``El Día X es personal'', dijo. ``Para un tipo es este día, para otro
tipo es otro día''.

``Es el día en que activas tus planes'', dijo.

Los miembros del chat se reunieron en persona, calcularon qué
provisiones y armas almacenar, y dónde tener casas de seguridad. Decenas
fueron identificadas. Uno era la propia base militar en Calw.
Practicaron cómo reconocerse entre sí, utilizando un código militar, en
los ``puntos de recogida'' donde los miembros podían reunirse el Día X.

Image

La ciudad de Calw en la región de la Selva Negra. Todos los soldados de
las KSK están estacionados en una base afuera de la
ciudad.Credit...Laetitia Vancon para The New York Times

El sentido de urgencia creció.

El 21 de marzo de 2016, un miembro del chat, identificado solo como
Matze, escribió sobre un punto de recogida cerca de Nuremberg. Había,
escribió, ``suficientes armas y municiones para luchar a su manera''.

Más tarde ese año, Schmitt envió un mensaje a otros en la red de chat.
En los últimos 18 meses, escribió, habían reunido a ``2000 personas de
ideas afines'' en Alemania y en el extranjero.

Cuando lo conocí, Schmitt lo llamó ``una hermandad global de ideas
afines''.

Niega haber planificado llevar a cabo el Día X, pero aún está convencido
de que llegará, quizás más temprano que tarde con la pandemia.

``Sabemos, gracias a nuestras fuentes en los bancos y en los servicios
de inteligencia, que a más tardar en septiembre vendrá una gran crisis
económica'', dijo en una llamada de seguimiento la semana pasada.

``Habrá insolvencia y desempleo masivo'', profetizó. ``La gente saldrá a
las calles''.

\hypertarget{cabezas-de-cerdo-y-saludos-de-hitler}{%
\subsection{Cabezas de cerdo y saludos de
Hitler}\label{cabezas-de-cerdo-y-saludos-de-hitler}}

Una noche en 2017, Ovejita, el sargento mayor cuyo alijo de armas se
descubrió en mayo, estaba entre los 70 soldados de la Segunda Compañía
de las KSK que se habían reunido en un campo de tiro militar.

Los investigadores lo han identificado solo como Philipp Sch. Él y los
demás habían organizado una fiesta especial de despedida para un
teniente coronel, un hombre reconocido como héroe de guerra por escapar
de una emboscada en Afganistán mientras disparaba y cargaba a uno de sus
hombres.

El coronel, un hombre imponente cubierto de tatuajes en cirílico que
disfruta de la lucha en jaula en su tiempo libre, tenía que completar
una carrera de obstáculos. Implicaba cortar troncos de árboles y
lanzamiento de cabezas de cerdos decapitados.

Como premio, sus hombres habían llevado a una mujer. Pero el coronel
terminó completamente borracho. La mujer, en vez de convertirse en su
trofeo, fue a la policía.

De pie junto al fuego con un puñado de soldados, los había visto cantar
letras neonazis y levantar el brazo derecho. Un hombre se destacó por su
entusiasmo, recordó ella en un
\href{https://daserste.ndr.de/panorama/archiv/2017/Hitlergruss-Ermittlungen-gegen-Kompaniechef,bundeswehr1738.html}{informe
televisado} por la emisora pública ARD. Lo llamó el ``abuelo nazi''.

Image

El Zeppelin Bar en la base de las KSK en Calw. Como hogar de la unidad
de fuerzas especiales, la base es la más vigilada de
Alemania.Credit...Laetitia Vancon para The New York Times

Aunque solo tenía 45 años, ``el abuelo nazi'' era Ovejita, quien se
había unido a las KSK en 2001.

En los tres años transcurridos desde la fiesta, el servicio de
contrainteligencia militar vigiló al sargento mayor. Pero eso no impidió
que las KSK lo promoviera al rango más alto posible de suboficial.

El manejo del caso se ajustó a un patrón, dicen soldados y funcionarios.

En junio, un soldado de las KSK dirigió una carta de 12 páginas a la
ministra de Defensa, en la que pedía una investigación sobre lo que
describió como una ``cultura tóxica de aceptación'' y una ``cultura de
miedo'' dentro de la unidad. Las pistas sobre los camaradas extremistas
eran ``colectivamente ignoradas o incluso toleradas''. Uno de sus
instructores había equiparado a las KSK con las Waffen SS, escribió el
soldado.

El instructor, un teniente coronel, estaba en el radar de las
inclinaciones hacia la extrema derecha desde 2007, cuando escribió un
correo electrónico amenazante a otro soldado. ``Estás siendo vigilado,
no, no por agencias impotentes instrumentalizadas, sino por oficiales de
una nueva generación, que actuarán cuando los tiempos lo exijan'',
decía. ``Larga vida a la santa Alemania''.

El comandante de las KSK en ese momento no suspendió al teniente.
Simplemente lo disciplinó. Le pregunté al general Kreitmayr, quien
asumió el comando en 2018, sobre el caso.

``Mira, hoy en el año 2020, con todo el conocimiento que tenemos,
miramos el correo electrónico de 2007 y decimos: `es obvio''', me dijo.

``Pero en aquella época solo pensamos: `Hombre, ¿qué le pasa? Debería
moderarse'''.

\hypertarget{el-pasillo-de-la-historia}{%
\subsection{El pasillo de la historia}\label{el-pasillo-de-la-historia}}

La puerta trasera del edificio principal en la base en Calw lleva a un
largo corredor conocido como el ``corredor de la historia'', una
colección de recuerdos reunidos durante los casi 25 años de las KSK que
incluye a un pastor alemán disecado, Kato, quien se lanzó en paracaídas
a más de 9100 metros con un equipo de comando.

Image

Un pasillo en la base de Calw muestra recuerdos de los casi 25 años de
historia de las KSK.Credit...Laetitia Vancon para The New York Times

Falta evidentemente cualquier mención a un ex comandante deshonrado de
las KSK, el general Reinhard Günzel, quien fue despedido después de
escribir una carta en 2003 en apoyo a un discurso antisemita de un
legislador conservador.

El general Günzel posteriormente publicó un libro llamado
\emph{Guerreros Secretos}. En él, ubicaba a las KSK en la tradición de
unas notorias fuerzas especiales de la época nazi que cometieron
numerosos crímenes de guerra, incluidas masacres de judíos. Ha sido un
orador popular en eventos de extrema derecha.

``Básicamente tienes a uno de los comandantes fundadores de las KSK
convertido en un destacado ideólogo de la Nueva Derecha'', dijo
Christian Weissgerber, un ex soldado que
\href{https://www.ofv.ch/sachbuch/detail/mein-vaterland-warum-ich-ein-neonazi-war/103760/}{escribió
un libro} sobre su propia experiencia de ser un neonazi en el ejército.

\href{https://www.nytimes.com/es/2019/01/01/espanol/neonazis-alemania-extrema-derecha.html}{La
Nueva Derecha}, que abarca activistas juveniles, intelectuales y a la
Afd, preocupa al general Kreitmayr.
\href{https://www.martinhohmann.de/}{El legislador}cuyos comentarios
antisemitas llevaron al despido del general Günzel hace tantos años,
ahora se sienta en el Parlamento alemán por la AfD.

``Tienes representantes destacados de partidos políticos como la AfD,
que dicen cosas que no solo son enfermizas sino que son claramente de
ideología radical de extrema derecha'', dijo el general Kreitmayr.

Los soldados no fueron inmunes a este cambio cultural en el país, dijo.
Recientemente, un compañero general se había convertido en candidato a
la alcaldía por la Afd. Varios ex soldados representan al partido en el
Parlamento.

Image

El Reichstag en Berlín, hogar del Parlamento alemán. Varios ex soldados
representan al partido de extrema derecha AfD en el
Parlamento.Credit...Emile Ducke para The New York Times

Al bajar la colina desde la casa de tiro está el Salón Verde, un cruce
entre una sala de juntas y un bar. Está dominado por una gran pintura al
óleo que representa a soldados de las KSK y sus pastores alemanes
mientras atacan con éxito un escondite talibán.

Es una escena familiar para varios soldados que se habían reunido el día
en que estuve allí. Pero los soldados con los que hablé cuestionaron la
estrategia detrás de una guerra que se ha desarrollado durante dos
décadas con pocos resultados concretos, excepto un aumento de la
migración en casa.

``Mis niñas me preguntaron el otro día: `¿Por qué tienes que ir a
Afganistán cuando hay niños de Kunduz en nuestra clase?''', relató un
oficial. ``No tenía una respuesta''.

Cuando llevó a una delegación de soldados de las KSK para reunirse con
partidos políticos en el Parlamento, les hizo la misma pregunta.
``Tampoco tenían la respuesta'', dijo.

Solo un legislador hizo una declaración clara, dijo. Era de la AfD.
``Dijo que debíamos habernos ido hace mucho tiempo'', recordó el
oficial.

Image

Material de Afganistán exhibido en la base de Calw.Credit...Laetitia
Vancon para The New York Times

\emph{Christopher F. Schuetze colaboró con este reportaje.}

\emph{Katrin Bennhold es la jefa de la corresponsalía de Berlín de The
New York Times. Anteriormente, reporteaba desde Londres y París, en
donde cubría una gama diversa de temas: desde el auge del populismo
hasta asuntos de género.}
\href{https://twitter.com/kbennhold?lang=es}{\emph{@kbennhold}}

\begin{center}\rule{0.5\linewidth}{\linethickness}\end{center}

Advertisement

\protect\hyperlink{after-bottom}{Continue reading the main story}

\hypertarget{site-index}{%
\subsection{Site Index}\label{site-index}}

\hypertarget{site-information-navigation}{%
\subsection{Site Information
Navigation}\label{site-information-navigation}}

\begin{itemize}
\tightlist
\item
  \href{https://help.nytimes.com/hc/en-us/articles/115014792127-Copyright-notice}{©~2020~The
  New York Times Company}
\end{itemize}

\begin{itemize}
\tightlist
\item
  \href{https://www.nytco.com/}{NYTCo}
\item
  \href{https://help.nytimes.com/hc/en-us/articles/115015385887-Contact-Us}{Contact
  Us}
\item
  \href{https://www.nytco.com/careers/}{Work with us}
\item
  \href{https://nytmediakit.com/}{Advertise}
\item
  \href{http://www.tbrandstudio.com/}{T Brand Studio}
\item
  \href{https://www.nytimes.com/privacy/cookie-policy\#how-do-i-manage-trackers}{Your
  Ad Choices}
\item
  \href{https://www.nytimes.com/privacy}{Privacy}
\item
  \href{https://help.nytimes.com/hc/en-us/articles/115014893428-Terms-of-service}{Terms
  of Service}
\item
  \href{https://help.nytimes.com/hc/en-us/articles/115014893968-Terms-of-sale}{Terms
  of Sale}
\item
  \href{https://spiderbites.nytimes.com}{Site Map}
\item
  \href{https://help.nytimes.com/hc/en-us}{Help}
\item
  \href{https://www.nytimes.com/subscription?campaignId=37WXW}{Subscriptions}
\end{itemize}
