Sections

SEARCH

\protect\hyperlink{site-content}{Skip to
content}\protect\hyperlink{site-index}{Skip to site index}

\href{/es/section/ciencia-y-tecnologia}{Ciencia y
Tecnología}\textbar{}Visita al laboratorio de Johnson \& Johnson que
busca la vacuna contra el coronavirus

\url{https://nyti.ms/2OLbN5c}

\begin{itemize}
\item
\item
\item
\item
\item
\end{itemize}

\href{https://www.nytimes.com/es/spotlight/coronavirus?action=click\&pgtype=Article\&state=default\&region=TOP_BANNER\&context=storylines_menu}{El
brote de coronavirus}

\begin{itemize}
\tightlist
\item
  \href{https://www.nytimes.com/es/interactive/2020/espanol/mundo/coronavirus-en-estados-unidos.html?action=click\&pgtype=Article\&state=default\&region=TOP_BANNER\&context=storylines_menu}{Mapa
  y casos en EE. UU.}
\item
  \href{https://www.nytimes.com/es/2020/07/23/espanol/america-latina/bolivia-cloro-coronavirus-ivermectina.html?action=click\&pgtype=Article\&state=default\&region=TOP_BANNER\&context=storylines_menu}{Dióxido
  de cloro, ivermectina y más: ¿funcionan?}
\item
  \href{https://www.nytimes.com/es/interactive/2020/science/coronavirus-tratamientos-curas.html?action=click\&pgtype=Article\&state=default\&region=TOP_BANNER\&context=storylines_menu}{Fármacos
  y tratamientos}
\item
  \href{https://www.nytimes.com/es/2020/07/28/espanol/ciencia-y-tecnologia/anticuerpos-coronavirus-inmunidad.html?action=click\&pgtype=Article\&state=default\&region=TOP_BANNER\&context=storylines_menu}{Anticuerpos
  e inmunidad}
\item
  \href{https://www.nytimes.com/es/2020/04/29/espanol/estilos-de-vida/oximetro-para-que-sirve.html?action=click\&pgtype=Article\&state=default\&region=TOP_BANNER\&context=storylines_menu}{Oxímetros}
\end{itemize}

\includegraphics{https://static01.nyt.com/images/2020/07/21/science/23vaccine-ES-00/00VIRUS-JNJ1-articleLarge.jpg?quality=75\&auto=webp\&disable=upscale}

\hypertarget{visita-al-laboratorio-de-johnson--johnson-que-busca-la-vacuna-contra-el-coronavirus}{%
\section{Visita al laboratorio de Johnson \& Johnson que busca la vacuna
contra el
coronavirus}\label{visita-al-laboratorio-de-johnson--johnson-que-busca-la-vacuna-contra-el-coronavirus}}

Científicos en Boston y en los Países Bajos están en una carrera para
encontrar una vacuna contra el virus que ha paralizado al mundo.

Noe Mercado, científico del Centro para la Investigación de Virus y
Vacunas en Boston, está desarrollando una vacuna contra el coronavirus
con Johnson \& Johnson.Credit...Tony Luong para The New York Times

Supported by

\protect\hyperlink{after-sponsor}{Continue reading the main story}

\href{https://www.nytimes.com/by/carl-zimmer}{\includegraphics{https://static01.nyt.com/images/2018/06/12/multimedia/author-carl-zimmer/author-carl-zimmer-thumbLarge.png}}

Por \href{https://www.nytimes.com/by/carl-zimmer}{Carl Zimmer}

\begin{itemize}
\item
  Publicado 23 de julio de 2020Actualizado 30 de julio de 2020
\item
  \begin{itemize}
  \item
  \item
  \item
  \item
  \item
  \end{itemize}
\end{itemize}

\href{https://www.nytimes.com/2020/07/17/health/coronavirus-vaccine-johnson-janssen.html}{Read
in English}

\href{https://www.nytimes.com/newsletters/el-times}{Regístrate para
recibir nuestro boletín} con lo mejor de The New York Times.

\begin{center}\rule{0.5\linewidth}{\linethickness}\end{center}

Todas las mañanas de los días hábiles de marzo, Noe Mercado conducía por
las desoladas calles de Boston hasta un edificio alto de cristal en
Blackfan Circle, en el corazón del centro de biotecnología de la ciudad.
La mayoría de los residentes se había ido a resguardar del coronavirus,
pero Mercado tenía un trabajo esencial: encontrar una
\href{https://www.nytimes.com/2020/07/20/world/covid-coronavirus-vaccine.html}{vacuna}
contra este nuevo y devastador patógeno.

Después de aparcar en el estacionamiento subterráneo, se ponía un
cubrebocas y subía por el ascensor vacío hasta el décimo piso, donde se
reunía con un equipo elemental del Centro para la Investigación de Virus
y Vacunas del Centro Médico Beth Israel Deaconess. Día tras día, Mercado
se sentaba a la mesa de su laboratorio a buscar señales del virus en
muestras nasales tomadas de decenas de monos.

Estos animales habían sido inyectados con vacunas experimentales que
Mercado había ayudado a diseñar. Los monos habían estado expuestos al
coronavirus, y ahora Mercado estaba descubriendo si alguna vacuna los
había protegido. Una mañana, después de que ingresó todo los datos en un
programa de software, una sola gráfica hizo que se le acelerara el
corazón: parecía que algunas de las vacunas
\href{https://www.nytimes.com/2020/05/20/health/coronavirus-vaccine-harvard.html}{habían
funcionado}.

Mercado corrió por el laboratorio para compartir la noticia. Dadas las
circunstancias, no hubo abrazos ni choques de manos. Tampoco se regodeó
en su triunfo por mucho tiempo. Elaborar una vacuna exige paciencia,
atención al detalle\ldots{} y tolerancia a un amargo fracaso.

``Sí, estoy emocionado, pero también estoy pensando en la siguiente
fase'', recordó Mercado después. ``¿Qué tal si luego no resulta?''.

En todo el mundo,
\href{https://www.nytimes.com/interactive/2020/world/coronavirus-maps.html}{el
coronavirus ha infectado a más de 15 millones de personas} y ha causado
la muerte de más de 600.000. Es posible que mueran millones más. La
única esperanza para contar con una protección a largo plazo y,
literalmente, la única manera de regresar a una vida normal es con una
vacuna eficaz.

En enero, los investigadores del centro dejaron todo lo que estaban
haciendo para encontrar una vacuna. La persona que encabezaba el trabajo
era el jefe de Mercado, Dan Barouch, director del centro y uno de los
creadores de vacunas más importantes del mundo.

Ahora están a punto de dar un paso importante. Janssen Pharmaceutica,
una división de Johnson \& Johnson, ha trabajado con el equipo del Beth
Israel para elaborar una vacuna contra el coronavirus que se basa en un
diseño iniciado por Barouch y sus colegas hace diez años.

Esta semana, empezarán los ensayos clínicos de la vacuna en Bélgica. El
equipo de Barouch pronto pondrá en marcha un ensayo en Boston.

Los últimos seis meses han traído una mezcolanza de semanas largas y
noches de desvelo, de medidas de seguridad estrictas y de pocos
suministros de laboratorio. ``Todos los pedidos han representado un
mayor desafío que en los tiempos anteriores a la pandemia'', señaló
Barouch.

Los investigadores de todo el mundo han trabajado en elaborar sus
propias vacunas, algunos con virus muertos, otros con fragmentos de
proteínas y cadenas de ADN. Hasta julio, hay
\href{https://www.nytimes.com/interactive/2020/science/coronavirus-vaccine-tracker.html}{más
de 135 vacunas en pruebas preclínicas y más de 30 en ensayos clínicos}
con seres humanos.

Nunca antes tantas vacunas para una enfermedad han entrado tan
rápidamente en ensayos.

Desde enero, el equipo de Barouch en Boston ha realizado experimentos en
células y monos, mientras que los investigadores de Janssen en los
Países Bajos han corrido para encontrar una receta para producir la
nueva vacuna en grandes cantidades. Ya han comenzado a producir un lote
para los ensayos clínicos.

Si se comprueba que la vacuna es segura en las pruebas iniciales,
comenzará un ensayo de su eficacia en septiembre. Si ese experimento
tiene éxito, Johnson \& Johnson fabricará cientos de millones de dosis
para su uso urgente en enero. En el transcurso del próximo año, la
compañía planea producir hasta mil millones de dosis.

Si bien Johnson \& Johnson es una de las compañías más grandes del
mundo, con una capitalización de mercado de más de 370.000 millones de
dólares, es un jugador bastante pequeño en el mercado de las vacunas. El
1 de julio, su vacuna contra el ébola recibió la aprobación de la
Comisión Europea. Las vacunas de la compañía para otras enfermedades aún
están en ensayos clínicos.

Aún así, el gobierno de Estados Unidos ha otorgado 456 millones de
dólares a Johnson \& Johnson, fondos de la
\href{https://www.hhs.gov/about/news/2020/06/16/fact-sheet-explaining-operation-warp-speed.html}{Operación
Warp Speed del gobierno de Trump}; la compañía ha invertido otros 500
millones de dólares en el proyecto de vacuna contra el coronavirus.

Barouch y sus colegas ahora terminan las pruebas de la formulación final
en monos. En los meses siguientes, empezarán a ver cómo reaccionan las
personas a la vacuna.

Es una tarea monumental desarrollar tan rápido una vacuna contra un
patógeno del que nadie había escuchado antes de este año. Pero Barouch
dijo: ``Ahora estoy incluso más optimista que hace varios meses''.

\hypertarget{cuarenta-y-un-casos}{%
\subsection{Cuarenta y un casos}\label{cuarenta-y-un-casos}}

\includegraphics{https://static01.nyt.com/images/2020/07/21/science/23vaccine-ES-01/00VIRUS-JNJ2-articleLarge.jpg?quality=75\&auto=webp\&disable=upscale}

Al caer la tarde del 10 de enero, la temperatura en Boston era de unos
10 grados Celsius, casi 11 grados por encima de lo normal. Barouch había
pasado el día a la cabeza del retiro anual del laboratorio en el último
piso del Museo de Ciencias de Boston.

Por las ventanas altas, los científicos podían ver los autos que
cruzaban el río Charles. Durante los descansos entre presentaciones, se
aglomeraron para tomar fotos grupales, con grandes sonrisas
despreocupadas.

Al final del encuentro, hablaron sobre una extraña serie de 41 casos de
neumonía en Wuhan, China. ``En ese momento, 41 casos parecían muchos'',
señaló Barouch.

Estos casos les hacía pensar en el síndrome respiratorio agudo grave
(SARS, por su sigla en inglés), una enfermedad causada por un
coronavirus, que había aparecido en China en 2002 y se había extendido a
29 países, donde enfermó a 8096 personas y mató a 773, antes de que se
detuviera. Los científicos chinos acababan de informar que otro
coronavirus andaba suelto.

``Pensamos que tal vez deberíamos hacer una vacuna para esto'', recordó
Jinyan Liu, un científico del centro. Pero sin más información sobre el
nuevo virus, no había nada que pudieran hacer.

Todo cambió esa noche. A las 9:41 p.m., Kathryn Stephenson, directora de
la unidad de ensayos clínicos del centro, le envió a Barouch un breve
correo electrónico desde su iPhone: ``Esto salió hoy, vi a alguien poner
un enlace en Twitter''.

El enlace conducía a un sitio de virología con acceso abierto donde
científicos con sede en China publicaron un archivo que
\href{https://virological.org/t/novel-2019-coronavirus-genome/319}{contenía
la secuencia genética completa del nuevo coronavirus}. ``Por favor,
siéntanse libres de descargar, compartir, usar y analizar estos datos'',
escribió Yong-Zhen Zhang, profesor de la Universidad de Fudan en
Shanghái y líder del consorcio.

Cinco minutos después, Barouch les envió un correo electrónico a Liu,
Mercado y Zhenfeng Li, un asistente de investigación del centro:
``¿Puede alguno de ustedes extraer de este archivo la secuencia del
nuevo coronavirus?''.

Pronto, los cuatro científicos estudiaban detenidamente la secuencia,
una serie de 30.000 letras genéticas que nadie había visto antes
ordenadas exactamente en este orden. ``Trabajamos viernes, sábado,
domingo, día y noche'', dijo Liu.

Cuando el fin de semana estaba por terminar, tenían una buena idea de a
qué se enfrentaban y cómo derrotarlo potencialmente. El lunes, los
científicos regresaron al laboratorio, listos para comenzar la empresa
más ambiciosa que cualquiera de ellos haya emprendido.

Pero los investigadores no tendrían que crear una vacuna desde cero.
Iban a trabajar con un manual que Barouch había estado escribiendo
durante 20 años.

Para 2004, cuando Barouch inauguró su primer laboratorio en la Escuela
de Medicina de la Universidad de Harvard, se había ganado una reputación
como un ambicioso joven investigador. De inmediato se planteó un
objetivo muy ambicioso: elaborar una vacuna contra el VIH, el virus que
causa el sida.

Ese virus se descubrió en 1983, pero en dos décadas de trabajo con las
vacunas, iban de una decepción a otra. Las formas estándar de entrenar
al sistema inmunitario para que reconozca un virus fallaron cuando se
trataba del VIH.

Barouch decidió intentar algo diferente: una vacuna elaborada con otro
virus. Eligieron el adenovirus serotipo 26 (Ad26), un virus
relativamente raro que causa resfriados leves, pero que invade las
células humanas de manera muy eficaz.

Para crear esa vacuna, trabajaron con Crucell, una empresa neerlandesa
que Johnson \& Johnson compró en 2011. Los investigadores inhabilitaron
al virus Ad26 para que solo pudiera invadir las células, pero no
multiplicarse dentro de ellas.

Image

Investigadores trabajan con un manual sobre vacunas que Barouch ha
escrito durante 20 años.Credit...Tony Luong para The New York Times

Image

La vacuna prepararía al sistema inmunitario para atacar las llamadas
proteínas espiga que cubren la superficie del nuevo
coronavirus.Credit...Tony Luong para The New York Times

Posteriormente, agregaron un gen del VIH. Las células infectadas con el
Ad26 fabricarían las proteínas del VIH que circulaban por el torrente
sanguíneo, de esta manera se prepararía al sistema inmunitario.

En los experimentos con monos, la vacuna brindó protección contra el
VIH. En las pruebas con humanos, la vacuna resultó segura y desencadenó
una fuerte respuesta inmunitaria contra el virus. Pero siguen en curso
los ensayos para ver si protege contra el virus de manera eficaz.

En 2016, en medio de la epidemia del Zika, Barouch y sus colegas
adaptaron su vacuna Ad26 para que fabricara proteínas virales del Zika.
Incluso realizaron ensayos que demostraron que la vacuna era segura para
los seres humanos y que generaba una respuesta inmunitaria, pero
suspendieron el proyecto cuando la epidemia se detuvo.

Cuando el nuevo coronavirus comenzó a propagarse en enero, el
laboratorio ya sabía cómo elaborar una vacuna para un brote repentino.
Ahora lo que necesitaban era una manera de abordar el nuevo virus.

Las investigaciones anteriores sobre el SARS y otros coronavirus
facilitaron la decisión. Prepararían el sistema inmunitario para que
atacara a las denominadas proteínas de espiga que cubren la superficie
del nuevo coronavirus.

\hypertarget{una-guerra-que-podruxedamos-ganar}{%
\subsection{`Una guerra que podríamos
ganar'}\label{una-guerra-que-podruxedamos-ganar}}

A medida que enero avanzaba, Barouch se dio cuenta de que la COVID-19
sería una amenaza mucho más grave que el SARS.

``No podríamos detener este virus mediante medidas tradicionales de
salud pública'', dijo. ``Estaba absolutamente claro que necesitábamos
una vacuna''.

Le envió un correo electrónico a Johan van Hoof, el jefe de vacunas de
Janssen. ``Te escribo hoy porque el brote de coronavirus en China se ve
mal'', escribió Barouch. ``¿Estás interesado en hacer una vacuna rápida
basada en una Ad como la que hicimos para el Zika en 2016-2017?''.

Dos minutos más tarde, Van Hoof respondió: ``¿Te puedo llamar ahora?''.
Y cuatro días después de la llamada, firmaron un acuerdo de
colaboración.

El Centro para la Investigación de Virus y Vacunas cuenta con un
personal de decenas de investigadores que incluyen médicos, científicos
importantes, investigadores de posdoctorado, estudiantes de posgrado y
asistentes recién egresados de la universidad. El equipo de Barouch dejó
los proyectos sobre el VIH y otras enfermedades y se dividió el trabajo
para diseñar una vacuna contra el coronavirus.

Mercado y sus colegas fabricaron copias del gen del coronavirus que
ordena la producción de su proteína de espiga. Obtuvieron diez variantes
para ver cuál provocaba la mejor respuesta inmunitaria.

Mientras tanto, Katherine McMahan, asistente de investigación en ese
centro, trabajaba en el equipo que diseñaba una prueba de anticuerpos
para la proteína de espiga en los animales que recibirían la vacuna.
Crearlo tomó la mayor parte de su tiempo. Algunos días, no llegaba a
almorzar sino hasta la noche.

A fines de febrero, los investigadores les inyectaron los genes de la
proteína de espiga a unos ratones y luego le enviaron a McMahan la
sangre de los animales. La prueba de McMahan confirmó que estaban
fabricando anticuerpos para el coronavirus.

McMahan estaba al borde de las lágrimas: ``Comenzaba a parecer una
guerra que podríamos ganar''.

Sin embargo, fuera del laboratorio, nadie parecía intuir que una guerra
se aproximaba. Instó a su familia y a sus amigos a abastecerse de
alimentos y otros suministros, sin mucha suerte.

``Muchos de nosotros nos sentíamos en el cuento de `Pedro y el lobo''',
dijo. ``Tú dices: `Mira, tienes que tomarte esto en serio', y te
desprecian''.

Image

En el Centro de Investigación de Virología y Vacunas, los científicos
trabajan noches y fines de semana.Credit...Tony Luong para The New York
Times

Image

Jinyan Liu, uno de los científicos que estudió detenidamente la
secuencia genética de 30.000 letras del coronavirus.Credit...Tony Luong
para The New York Times

Muy pronto, personas gravemente enfermas con la COVID-19 inundaron los
hospitales de Boston, y la ciudad comenzó a cerrar. En laboratorios muy
por encima de las calles vacías de Boston, el equipo de Barouch pasó de
los estudios en ratones a los estudios en monos.

Los hisopos nasales que examinó Mercado revelaron que algunas versiones
de las vacunas solo protegían parcialmente al mono, pero otras
funcionaban mucho mejor. Como informaron los investigadores en la
revista Science, no pudieron detectar el virus en absoluto en ocho de
los 25 monos que recibieron vacunas experimentales.

Los resultados le dieron a Barouch la esperanza de que una de las
vacunas de su grupo ---o una de aquellas desarrolladas por otro grupo---
pudiera funcionar. ``Es en serio'', dijo.

Más monos recibieron la inyección con el virus Ad26, ahora equipado para
producir el gen de la espiga. Barouch predice que esta vacuna inducirá
niveles más altos de anticuerpos que los prototipos.

El experimento también proporcionará pistas cruciales sobre cómo
responde el sistema inmunitario a la vacuna Ad26. Algunas vacunas
confieren protección principalmente al hacer que el cuerpo produzca
anticuerpos que atacan a un virus. Pero otras pueden incitar a las
células inmunes que cazan virus para unirse al ataque.

Los resultados de la última ronda de experimentos se publicarán en unas
pocas semanas.

A pesar de todo el progreso realizado por el equipo de Barouch, la
vacuna Ad26 tiene escépticos. John Moore, un virólogo de Weill Cornell
Medical College, dijo que otros tipos de vacunas probadas en animales
han producido niveles más altos de anticuerpos. Estas vacunas, hechas de
proteínas virales, serían su elección como arma contra el coronavirus.

Seis compañías ya han lanzado ensayos de seguridad en humanos de sus
vacunas proteicas. ``Eso es lo que estaría haciendo'', dijo Moore. ``Es
tremendamente obvio''.

Un inconveniente de las vacunas de proteínas virales es que tardan más
en producirse en grandes cantidades. Otras vacunas, como Ad26 de Johnson
\& Johnson, llegarán más rápidamente, y Moore reconoció que pueden
funcionar lo suficientemente bien como para brindar protección.

Si es así, puede que no sea necesaria una vacuna mejor pero más lenta.
``Si el Plan A funciona, entonces no necesitas un Plan B'', dijo Moore.

\hypertarget{una-semilla-de-virus}{%
\subsection{Una semilla de virus}\label{una-semilla-de-virus}}

Mientras Barouch y sus colegas probaban las vacunas en Estados Unidos,
un equipo de investigadores de Johnson \& Johnson estaba preparándose
para elaborarlas en los Países Bajos. Tienen experiencia con el Ad26, el
cual han usado para fabricar vacunas contra el VIH, el ébola y otros
virus.

Hacer una vacuna Ad26 requiere remodelar un adenovirus y luego crear
grandes cantidades de la nueva versión. Pero el Ad26 no puede
multiplicarse en celdas ordinarias. Debe infectar a las especialmente
diseñadas.

Los técnicos de Johnson \& Johnson producen lotes de estas células en
tanques enormes llenos de un caldo rico en nutrientes que se mantiene a
una temperatura constante y se agita para extraer oxígeno.

``Es hacer que las células se sientan felices y cómodas, hacer un
producto'', dijo Paul Ives, director sénior de desarrollo de
medicamentos de Janssen.

Una vez que un lote de estas células nutritivas ha crecido lo
suficiente, Ives y sus colegas la infectan con los virus Ad26
modificados. Cada célula produce miles de nuevos virus, que se eliminan
y purifican para que puedan usarse como vacunas.

Ives y sus colegas han estado midiendo qué tan rápido pueden
multiplicarse varias versiones de la célula Ad26 renovada. Los
científicos descubrieron que algunas se reproducen más fácilmente que
otras.

Incluso una tasa de reproducción ligeramente más lenta podría dejar a
Johnson \& Johnson con un enorme déficit en las dosis de vacunas.
``Puede significar que tienes 300 millones de vacunas o 30 millones'',
dijo Paul Stoffels, el director científico de Johnson \& Johnson.

Image

Barouch y sus colegas se preparan para inyectar su vacuna en cientos de
voluntarios en Boston a fines de julio.~Credit...Tony Luong para The New
York Times

Image

Si esos ensayos producen resultados prometedores, Johnson \& Johnson
hará uno mucho más grande en el otoño para ver si la vacuna es
efectiva.Credit...Tony Luong para The New York Times

Ives y sus colegas recientemente eligieron el mejor virus para la vacuna
y lo convirtieron en su ``semilla maestra de virus''. Crearon galones de
existencias de virus congelados. Un lote de esta semilla se convertirá
en la vacuna utilizada en los ensayos clínicos.

Y si esos ensayos muestran que la vacuna es efectiva, la fábrica usará
la misma semilla maestra de virus para elaborar un suministro de
emergencia que se distribuiría a principios de 2021. ``Teóricamente,
podemos producir 300 millones de vacunas'', dijo Stoffels.

La compañía ha formado una sociedad con un fabricante de vacunas
estadounidense y también está estableciendo dos plantas más en Asia y
Europa, ``para que podamos llegar a una capacidad de fabricación de mil
millones de vacunas'', dijo Stoffels.

Florian Krammer, un virólogo de la Escuela de Medicina Icahn School en
Mount Sinai, se pregunta si Johnson \& Johnson puede cumplir esa
promesa, dado que nunca ha logrado fabricar un Ad26 a esta escala.

``Hacer un par de millones de dosis durante varios años de ensayos
clínicos es muy diferente que producir cientos de millones de dosis en
unos meses para el mercado'', dijo.

Johnson \& Johnson ha dicho que distribuirá la vacuna sin fines de
lucro. En marzo, Stoffels declaró al periódico belga De Tijd que
\href{https://www.tijd.be/ondernemen/farma-biotech/we-rekenen-op-een-vaccin-van-10-euro-tegen-coronavirus/10217795.html}{calculó
un costo de diez dólares por vacuna}. En una entrevista de seguimiento,
dijo que el precio no se establecería hasta que la compañía terminase de
hacer un suministro inicial.

En medio de la pandemia, los críticos dicen que no se debe permitir que
Johnson \& Johnson establezca los términos. ``Si conseguimos una vacuna,
debería ser gratuita y estar disponible para todos'', dijo el reverendo
William J. Barber II, presidente de la Asociación Nacional para el
Progreso de las Personas de Color en Carolina del Norte y crítico de los
precios de los medicamentos de Johnson \& Johnson.

``¿Cómo se obtienen estos premios grandes, enormes, para producir una
vacuna sin ningún factor adicional que diga que debe usarse de manera en
que sea asequible para todos?'', preguntó.

Por el momento, nadie sabe si la vacuna en verdad tendrá éxito. Barouch
y sus colegas están preparándose para aplicar la vacuna Ad26 a cientos
de voluntarios en Boston a fines de julio. Los investigadores no solo
observarán si la vacuna es segura, sino que también analizarán los
anticuerpos que incita a producir en los voluntarios. Si esos ensayos
arrojan resultados prometedores, Johnson \& Johnson llevará a cabo uno
mucho más extenso en el otoño para ver si la vacuna es efectiva.

Al mismo tiempo, Barouch y sus colegas están planeando una tercera ronda
de experimentos con monos. Quieren inyectar a los animales anticuerpos
contra el coronavirus y luego infectarlos. Al dar a distintos monos
dosis variables, los investigadores esperan averiguar qué nivel de
anticuerpos en el cuerpo humano se requieren para prevenir la COVID-19.

Y así, incluso cuando Boston está comenzando a reabrir, Barouch y sus
colegas siguen trabajando en las noches y los fines de semana.

``Tengo una serie de notas adheridas en mi escritorio que actualizo
todos los días con el número de vidas perdidas a causa de la COVID-19'',
dijo McMahan. ``Cuando me siento agotada, miro esa cifra''.

Image

El doctor Barouch camina por el laboratorio.Credit...Tony Luong para The
New York Times

Carl Zimmer es el autor de la columna
\href{https://www.nytimes.com/column/matter}{Matter}. Ha publicado trece
libros, entre ellos S\emph{he Has Her Mother's Laugh: The Powers,
Perversions, and Potential of Heredity}.
\href{https://twitter.com/carlzimmer}{@carlzimmer} \textbar{}
\href{https://www.facebook.com/carlzimmerauthor}{Facebook}

Advertisement

\protect\hyperlink{after-bottom}{Continue reading the main story}

\hypertarget{site-index}{%
\subsection{Site Index}\label{site-index}}

\hypertarget{site-information-navigation}{%
\subsection{Site Information
Navigation}\label{site-information-navigation}}

\begin{itemize}
\tightlist
\item
  \href{https://help.nytimes.com/hc/en-us/articles/115014792127-Copyright-notice}{©~2020~The
  New York Times Company}
\end{itemize}

\begin{itemize}
\tightlist
\item
  \href{https://www.nytco.com/}{NYTCo}
\item
  \href{https://help.nytimes.com/hc/en-us/articles/115015385887-Contact-Us}{Contact
  Us}
\item
  \href{https://www.nytco.com/careers/}{Work with us}
\item
  \href{https://nytmediakit.com/}{Advertise}
\item
  \href{http://www.tbrandstudio.com/}{T Brand Studio}
\item
  \href{https://www.nytimes.com/privacy/cookie-policy\#how-do-i-manage-trackers}{Your
  Ad Choices}
\item
  \href{https://www.nytimes.com/privacy}{Privacy}
\item
  \href{https://help.nytimes.com/hc/en-us/articles/115014893428-Terms-of-service}{Terms
  of Service}
\item
  \href{https://help.nytimes.com/hc/en-us/articles/115014893968-Terms-of-sale}{Terms
  of Sale}
\item
  \href{https://spiderbites.nytimes.com}{Site Map}
\item
  \href{https://help.nytimes.com/hc/en-us}{Help}
\item
  \href{https://www.nytimes.com/subscription?campaignId=37WXW}{Subscriptions}
\end{itemize}
