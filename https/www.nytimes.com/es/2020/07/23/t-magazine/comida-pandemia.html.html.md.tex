Sections

SEARCH

\protect\hyperlink{site-content}{Skip to
content}\protect\hyperlink{site-index}{Skip to site index}

\href{https://myaccount.nytimes.com/auth/login?response_type=cookie\&client_id=vi}{}

\href{https://www.nytimes.com/section/todayspaper}{Today's Paper}

El menú de la pandemia

\url{https://nyti.ms/2CGR6F3}

\begin{itemize}
\item
\item
\item
\item
\item
\end{itemize}

\href{https://www.nytimes.com/spotlight/at-home?action=click\&pgtype=Article\&state=default\&region=TOP_BANNER\&context=at_home_menu}{At
Home}

\begin{itemize}
\tightlist
\item
  \href{https://www.nytimes.com/2020/08/03/well/family/the-benefits-of-talking-to-strangers.html?action=click\&pgtype=Article\&state=default\&region=TOP_BANNER\&context=at_home_menu}{Talk:
  To Strangers}
\item
  \href{https://www.nytimes.com/2020/08/01/at-home/coronavirus-make-pizza-on-a-grill.html?action=click\&pgtype=Article\&state=default\&region=TOP_BANNER\&context=at_home_menu}{Make:
  Grilled Pizza}
\item
  \href{https://www.nytimes.com/2020/07/31/arts/television/goldbergs-abc-stream.html?action=click\&pgtype=Article\&state=default\&region=TOP_BANNER\&context=at_home_menu}{Watch:
  'The Goldbergs'}
\item
  \href{https://www.nytimes.com/interactive/2020/at-home/even-more-reporters-editors-diaries-lists-recommendations.html?action=click\&pgtype=Article\&state=default\&region=TOP_BANNER\&context=at_home_menu}{Explore:
  Reporters' Google Docs}
\end{itemize}

Advertisement

\protect\hyperlink{after-top}{Continue reading the main story}

Supported by

\protect\hyperlink{after-sponsor}{Continue reading the main story}

\hypertarget{el-menuxfa-de-la-pandemia}{%
\section{El menú de la pandemia}\label{el-menuxfa-de-la-pandemia}}

Como en la Edad Media, el placer corporal se ha convertido en una señal,
si no de salud física, al menos de salud mental, tan importante para
sobrevivir a la pandemia del coronavirus como lo fue para sobrevivir a
la peste negra.

\includegraphics{https://static01.nyt.com/images/2020/07/14/t-magazine/25tmag-plaguefood-ES-1/14tmag-plague-articleLarge.jpg?quality=75\&auto=webp\&disable=upscale}

Por Michael Snyder

\begin{itemize}
\item
  23 de julio de 2020
\item
  \begin{itemize}
  \item
  \item
  \item
  \item
  \item
  \end{itemize}
\end{itemize}

\href{https://www.nytimes.com/2020/07/16/t-magazine/eating-food-during-plague.html}{Read
in English}

\href{https://www.nytimes.com/newsletters/el-times}{Regístrate para
recibir nuestro boletín} con lo mejor de The New York Times.

\begin{center}\rule{0.5\linewidth}{\linethickness}\end{center}

AL COMIENZO de
\href{https://www.nytimes.com/interactive/2020/07/07/magazine/decameron-project-short-story-collection.html}{\emph{El
Decamerón}}, la colección de cuentos del siglo XIV del escritor italiano
Giovanni Boccaccio, un grupo de diez jóvenes nobles ---siete mujeres y
tres hombres--- huyen de ``la mortífera peste'' que barría Florencia y
se abría camino hacia un banquete en el país a través de las colinas
toscanas. ``Usaban con gran templanza de comidas delicadísimas y óptimos
vinos, huían de los excesos'',
\href{https://www.alianzaeditorial.es/primer_capitulo/decameron.pdf}{escribe
Boccaccio} ---en inglés traducido por John Payne--- sobre su
despreocupado idilio de diez días, ``sin permitir que nadie hablase o
trajese noticias de fuera, de muerte o de enfermos, se entretenían con
la música y los placeres que podían tener''. Cenaron en ``mesas puestas
con manteles blanquísimos y con vasos que parecían de plata'',
alimentándose de acuerdo con la sabiduría médica común de la época, que
sostenía que una disposición alegre era tan necesaria para mantener a
raya la peste como el tipo de comida adecuado.

Boccaccio nunca describe estos festines en detalle, pero es fácil
adivinar lo que sus nobles podrían haber comido: ricos banquetes de aves
silvestres y ternera condimentados con pimienta, canela y nuez moscada
importados a gran costo de Asia, y pan blanco, rebanado y sin corteza,
el único tipo considerado adecuado para los ricos. Las verduras,
consideradas humildes y poco saludables, y por lo tanto aptas para los
legos, podrían haber estado ausentes de la mesa. Las dietas de la época,
tanto para ricos como para pobres, se basaban en la ciencia humoral de
los antiguos griegos, que sostenían que la desigualdad entre los cuatro
humores del cuerpo ---sangre, flema, cólera (bilis amarilla) y
melancolía (bilis negra)--- causaba cada tipo de dolencia. Una vez
consumida, se pensaba que los alimentos se convertían en sangre y luego
en carne, con el potencial de recalibrar el equilibrio humoral del
cuerpo, lo que podría afectar, o incluso transformar, la constitución de
una persona. Todos los alimentos poseían cualidades humorales ---el
hinojo era caliente y seco, el pepino era frío y mojado--- y se les
asignaba un lugar en una rígida jerarquía cósmica. Mientras que los
campesinos comían alimentos como coles y nabos que crecían cerca del
suelo, junto con panes integrales y papillas gruesas y pesadas, los
aristócratas se deleitaban con aves de aire, a veces vestidas, dice
\href{https://www.pacific.edu/academics/schools-and-colleges/college-of-the-pacific/academics/departments-and-programs/history/faculty-directory/ken-albala.html}{Ken
Albala}, historiador de la Universidad del Pacífico, ``en disfraces
completamente caprichosos e impactantes'', teñidos con colorante,
suspendidos en áspic (una invención medieval) o unidos para formar
criaturas fantásticas. Esos principios subyacentes no cambiaron en el
apogeo de la peste negra, que llegó a Europa alrededor de 1347, pero las
recomendaciones dietéticas ``se volvieron menos atrevidas'', agrega
Albala, y los médicos de la época sugirieron que ``los alimentos suaves
no se corrompen en melancolía o alteran el sistema de ninguna manera, lo
que es, casualmente, lo que las personas hacen psicológicamente en
cualquier momento de estrés''. Incluso hace siglos, los tiempos de
crisis indujeron
\href{https://www.nytimes.com/2020/04/07/business/coronavirus-processed-foods.html}{un
retorno a lo familiar}.

\includegraphics{https://static01.nyt.com/images/2020/07/14/t-magazine/25tmag-plaguefood-ES-2/14tmag-plague-02-articleLarge.jpg?quality=75\&auto=webp\&disable=upscale}

DESDE MARZO, PERIÓDICOS, revistas, sitios web de estilo de vida y, por
supuesto, las redes sociales se han henchido con imágenes de
\href{https://www.nytimes.com/2020/04/24/dining/focaccia-bread.html}{focaccia}
y
\href{https://www.nytimes.com/2020/04/11/science/sourdough-bread-starter-library.html}{pan
de masa madre},
\href{https://www.nytimes.com/2020/03/22/business/coronavirus-beans-sales.html}{frijoles}
y
\href{https://www.nytimes.com/2020/04/07/realestate/home-farming-tips-coronavirus.html}{fermentos},
\href{https://www.nytimes.com/2020/07/03/dining/the-most-delicious-chicken.html}{pollos}
de piel brillante y
\href{https://cooking.nytimes.com/recipes/1020845-slow-cooker-pot-roast}{asados}
con grasa: platos ricos y sabrosos que, para la mayor parte, Boccaccio
podría haber reconocido. Tras el cambio reciente hacia la cocina basada
en plantas y el auge de las tiendas de restricciones dietéticas ---las
ventas de productos sin gluten, por ejemplo, han crecido enormemente en
la última década, mientras que en los últimos años se han visto enormes
inversiones en reemplazos de carne impulsados por la tecnología--- estas
imágenes son sorprendentes en su aparente indiferencia hacia los dogmas
de la llamada alimentación ``limpia''. De hecho, en su flagrante
carnalidad, los alimentos reconfortantes de la crisis del nuevo
coronavirus pueden parecer prácticamente medievales, particularmente en
su descuido de las tendencias de salud a favor de la comodidad.

Estos alimentos reconfortantes, según el paradigma dominante de la
cultura alimentaria angloestadounidense, son casi siempre malos para
nosotros, bálsamos para el alma pero nunca lo que el cuerpo
\emph{necesita}, al menos no nutricionalmente. Pero hay una paradoja en
esto: en la Europa medieval, como en muchas de las culturas alimentarias
del mundo hoy, la comodidad y la salud eran inseparables; el placer y la
familiaridad se encontraban entre las guías para mantener el equilibrio
del cuerpo, una noción que persistió en el pensamiento popular incluso
cuando la ciencia médica se transformó a lo largo de los siglos.

Cuando los invasores españoles trajeron un brote catastrófico de viruela
y sarampión a las Américas en el siglo XVI, por ejemplo, algunos
colonizadores atribuyeron la crisis insondable que se produjo no a la
enfermedad, sino a las mismas carnes y vinos desconocidos introducidos
desde Europa que afirmaron ``civilizarían'' a las poblaciones nativas
(las muertes entre los suyos, mientras tanto, se atribuyeron a
ingredientes locales como el maíz y los chiles). Para los españoles,
comer alimentos desconocidos podría transformarte o matarte. A fines del
siglo XVIII, la idea de la Ilustración de que todos los cuerpos ---o al
menos todos los cuerpos masculinos blancos--- eran fundamentalmente los
mismos, hizo que la medicina humoral pareciera en gran medida obsoleta,
pero, fuera de una pequeña élite médica, la comida seguía siendo una
herramienta principal para tratar enfermedades. En el sur estadounidense
antes de la Guerra de Secesión, dice
\href{https://afamstudies.yale.edu/people/carolyn-roberts}{Carolyn
Roberts}, una historiadora de Yale centrada en la medicina y el comercio
de esclavos, los sanadores negros esclavizados siguieron siendo la
primera línea de defensa contra las enfermedades de sus comunidades, al
combinar el conocimiento médico con productos botánicos locales para
mezclar las tradiciones curativas de África y las Américas, incluso
después de que los hospitales se volvieron más comunes. En su \emph{An
Account of the Bilious Remitting Yellow Fever, as it Appeared in the
City of Philadelphia, in the Year 1793}, el médico Benjamin Rush, un
defensor de la medicina moderna, sin embargo
\href{https://collections.nlm.nih.gov/catalog/nlm:nlmuid-2569009R-bk}{prescribe}
``limonada, tamarindo, gelatina y agua de manzana cruda, tostada y
agua\ldots{} y té de manzanilla'', junto con tratamientos a base de
mercurio, durante las primeras etapas de la enfermedad y, a medida que
avanzaba la curación, un menú de ``caldos ricos, la carne de aves de
corral, ostras, cereales espesos, papilla y leche con chocolate''. Las
dietas recomendadas durante la pandemia de gripe de 1918 fueron
prácticamente idénticas, incluyendo caldos de carne y jugos cítricos
para evitar la fiebre y la avena, sopa de papas, natillas y tostadas a
medida que el paciente se recuperaba. Incluso el dicho popular de
``alimentar un resfriado, matar de hambre una fiebre'' contiene
vestigios de esa sensibilidad humoral.

Pero lo que \emph{sí} cambió fue la forma en que muchos europeos y
americanos se relacionaron con sus cuerpos fuera de la enfermedad. Los
mismos ideales de la Ilustración que produjeron revoluciones políticas,
y, por otro lado, justificaron el colonialismo sobre la base de la
superioridad europea como un supuesto imperativo biológico, más tarde
replicaron cómo cenaba la aristocracia: comidas completas, donde cada
comensal comía la misma cosa al mismo tiempo, reemplazaron los grandes
banquetes, donde todos elegían la comida que mejor se adaptaba a su
constitución. Más tarde, en el siglo XIX, los avances de la química y el
descubrimiento de los gérmenes como vectores de enfermedades convirtió a
los humanos en aglomeraciones de grasa y proteínas. ``Ya no tenías
derecho a tener opiniones sobre lo que tu cuerpo necesitaba: lo que se
requiere es un hecho científico'', dijo
\href{https://warwick.ac.uk/fac/arts/history/people/staff_index/earle/}{Rebecca
Earle}, historiadora de alimentos en la Universidad de Warwick. ``Y tu
apetito es solo un problema en lo que respecta a la ciencia
nutricional''.

Esa misma actitud autoritaria persistió en el siglo XX en forma de la
cultura de la dieta, que todavía trata el tener el cuerpo ``equivocado''
como un signo de enfermedad moral. En los primeros días de la epidemia
de VIH/sida, el ala asimilacionista de la comunidad gay se basó en una
filosofía similar, recuerda el escritor de alimentos radicado en Oakland
\href{https://www.john-birdsall.com/}{John Birdsall}, y el argumento era
que \emph{si comes bien, eso evitará la infección}. El hedonismo,
insistía la cultura en general, había llevado esta plaga a los
homosexuales; la austeridad, en forma de dietas macrobióticas sin grasa
y el naciente vegetarianismo estadounidense, podría evitarlo.

Al mismo tiempo, el lado más radical del movimiento queer insistía en
que el poder gustativo podía salvar los cuerpos queer, al igual que
reclamar el derecho al placer sexual había salvado sus almas. En su
columna de cocina de la década de 1990, ``¡Engorda, no te
mueras!''\emph{,} publicada durante casi una década en la revista de
humor negro de San Francisco
\href{https://calisphere.org/collections/22661/?fbclid=IwAR2XcihgRMuSlZePJfzGuwyfZhaKWUutxnHqZQworbMUDdoOj0wpYYNf-5s}{Diseased
Pariah News}, el activista Beowulf Thorne, que escribía bajo el
pseudónimo de Biffy Mae, prescribía comidas de cereales con crema,
budines de pan de jengibre y curry tailandés con el mismo entusiasmo que
Rush reservaba para los caldos de carne y el té diluido. Como escribió
Jonathan Kauffman en
\href{https://hazlitt.net/longreads/get-fat-dont-die}{su reciente
artículo} para Hazlitt, Thorne ``se burló de los suplementos
nutricionales comercializados para las personas con sida, y se inclinó
hacia la mezcla para hornear Bisquick, sus gustos alternadamente
cosmopolitas o de plano reconfortantes''.

Image

Una miniatura anónima titulada ``La comida'' de `El Decamerón' de
Giovanni Boccaccio (1432) en la Bibliothèque Nationale de
Francia.Credit...Album/Alamy Stock Photo

Image

Un grabado en madera de Leonhard Thurneysser (1531-1596), alrededor del
siglo XVI, que representa los cuatro elementos de la teoría humoral
griega (sangre, flema, cólera {[}bilis amarilla{]} y melancolía {[}bilis
negra{]}) en la que se basaban las dietas del siglo
XIV.Credit...Interfoto/Alamy Stock Photo

``ALTERNADAMENTE COSMOPOLITAS Y de plano reconfortantes'' encapsula más
o menos el núcleo de la
\href{https://www.nytimes.com/article/recipes-cooking-tips-coronavirus.html}{cocina
casera de la cuarentena actual}. Los alimentos que han llegado a dominar
las redes sociales ---desde la lasaña hasta el congee, omelet tamil con
curry hasta los huevos rancheros, los panqueques de masa madre al kimchi
jjigae (con kimchi casero, por supuesto)--- combinan los limitados
ingredientes que están disponibles en las tiendas con el único producto
que aún tiene amplia oferta: el tiempo. Birdsall, después de algunas
semanas de comidas elaboradas, ha vuelto en los últimos meses a la
economía y la simplicidad, imbuyendo sus cenas básicas de verduras
cocidas y hamburguesas perfectamente selladas con una atención monástica
al detalle que, dice, ``crea un halo alrededor de estos ingredientes
limitados''.
\href{https://www.wildfermentation.com/who-is-sandorkraut/}{Sandor Ellix
Katz}, cuyos libros \emph{Wild} \emph{Fermentation} (2003) y \emph{The
Art of Fermentation} (2012) ** ayudaron a impulsar el renacimiento de la
fermentación de los últimos 15 años ---y quien llegó a la fermentación a
inicios de la década de 1990--- dice que sus clases de masa madre en
línea ahora atraen hasta mil estudiantes cada sesión. En este momento de
enfermedad e incertidumbre, la fabricación de alimentos artesanales que
muchas personas habrían dejado previamente a los profesionales
---comprar su pan en la panadería, sus encurtidos en una tienda de
delicatessen, su kimchi en un supermercado coreano--- han reemplazado al
\emph{fitness} como un signo aspiracional de cuidado. El placer corporal
se ha convertido una vez más en una señal, si no de salud física, al
menos de salud mental, tan fundamental para sobrevivir a esta plaga como
lo fue para sobrevivir a la peste negra.

Pero mientras que la cocina ha traído comodidad y significado a
incontables hogares, también ha resaltado las marcadas disparidades
mundiales.
\href{https://sph.umich.edu/news/2020posts/coronavirus-pandemic-worsens-food-insecurity-for-low-income-adults.html}{Un
estudio reciente} de la Escuela de Salud Pública de la Universidad de
Michigan descubrió que el 44 por ciento de los 1500 hogares
estadounidenses de bajos ingresos encuestados a fines de marzo ya
estaban experimentando inseguridad alimentaria. En
\href{https://www.nytimes.com/es/interactive/2020/espanol/america-latina/coronavirus-en-mexico.html}{México},
donde un presidente nominalmente izquierdista ha sugerido que comer
alimentos saludables en lugar de comida chatarra podría prevenir el
contagio,
\href{https://www.washingtonpost.com/world/2020/06/21/coronavirus-mexico-city-centro-abasto-market/?arc404=true}{decenas
de comerciantes han muerto} en la Central de Abasto, el mercado de
productos más grande de América Latina. En
\href{https://www.nytimes.com/interactive/2020/world/asia/india-coronavirus-cases.html}{India},
millones mueren de hambre mientras huyen de las ciudades a las aldeas,
incluso mientras el gobierno almacena
\href{https://www.bloomberg.com/news/articles/2020-03-24/india-has-enough-food-to-feed-poor-amid-prolonged-shutdown-fears}{cantidades
sin precedentes de granos}. Al igual que las pandemias anteriores, la
COVID-19 ha matado a los pobres más rápido y en mayor número. Si los
alimentos que anhelamos y cocinamos han llegado a parecerse a un festín
medieval, tal vez sea porque nuestra sociedad siempre ha sido medieval.

Aún así, la peste bubónica, a pesar de todo su horror, no fue un
apocalipsis, y la Edad Media de Europa no fue en realidad un momento de
oscuridad o estancamiento. Las trágicas muertes de decenas de millones
en Europa generaron una escasez de mano de obra que, en el transcurso de
más de un siglo, permitió a la clase laboral exigir salarios más altos,
acumular una modesta riqueza familiar e, incluso, cambiar sus dietas,
incorporando la carne que antes había sido accesible solo a la
aristocracia. El siglo XV anunció la proliferación de los primeros
libros de cocina publicados en Europa, ya que las personas de rango
medio buscaban emular la cocina de la aristocracia, completa con
especias ---como el clavo de olor, galangal y la pimienta larga--- que
antes estaban fuera de su alcance. Las innovaciones a menudo asociadas
con el Renacimiento surgieron de revoluciones en política, educación,
arte y filosofía puestas en marcha siglos antes, a menudo inspiradas y
alimentadas por los mismos intercambios comerciales y culturales que
facilitaron la propagación de la enfermedad en primer lugar.

La pandemia de nuestra generación ha llegado con una revolución propia,
una que se ha extendido incluso más rápido que el virus. Los llamados a
la justicia y el cambio político reemplazaron las imágenes amorosas de
panes de masa madre,
\href{https://cooking.nytimes.com/recipes/7002-dan-dan-noodles}{fideos
dan dan} relucientes con aceite de chile y cuencos de khichdi manchados
de cúrcuma, los potajes de arroz y lentejas servidos en innumerables
variaciones en todo el sur de Asia como un alimento reconfortante
icónico y, en tiempos de enfermedad, un tónico. En los últimos dos
meses, hemos sido testigos del derrumbe de los bastiones de la
\href{https://www.nytimes.com/2020/06/29/dining/john-t-edge-southern-foodways-alliance.html}{cultura
de la comida blanca} junto con monumentos que conmemoran una vergonzosa
historia de racismo y colonización, un movimiento ---liderado por
\href{https://www.nytimes.com/es/2020/06/16/espanol/mundo/bipoc-que-es.html}{personas
de color}--- que exige, una vez más, el tipo de igualdad política que la
Ilustración no pudo ofrecer. También parece requerir un retorno a una
comprensión mucho más antigua de nuestros cuerpos como fluidos y
cambiantes, cada uno con su propia forma de curación, su propio tipo de
comodidad individual. Restringidos como estaban por clase y acceso, tal
vez los alimentos que se desplegaron en Instagram durante todos esos
meses fueron una visión de una cultura alimentaria que coincide con una
nueva sociedad, una que no se basa en la abnegación o la apropiación o
en nociones fáciles de unidad sino, en cambio, como un banquete medieval
refractado a través de la comodidad y el cosmopolitismo propuestos por
Thorne: una mesa interminable, un botín fantástico, con espacio para
todo tipo de cuerpo y todo tipo de deseo.

\begin{center}\rule{0.5\linewidth}{\linethickness}\end{center}

Advertisement

\protect\hyperlink{after-bottom}{Continue reading the main story}

\hypertarget{site-index}{%
\subsection{Site Index}\label{site-index}}

\hypertarget{site-information-navigation}{%
\subsection{Site Information
Navigation}\label{site-information-navigation}}

\begin{itemize}
\tightlist
\item
  \href{https://help.nytimes.com/hc/en-us/articles/115014792127-Copyright-notice}{©~2020~The
  New York Times Company}
\end{itemize}

\begin{itemize}
\tightlist
\item
  \href{https://www.nytco.com/}{NYTCo}
\item
  \href{https://help.nytimes.com/hc/en-us/articles/115015385887-Contact-Us}{Contact
  Us}
\item
  \href{https://www.nytco.com/careers/}{Work with us}
\item
  \href{https://nytmediakit.com/}{Advertise}
\item
  \href{http://www.tbrandstudio.com/}{T Brand Studio}
\item
  \href{https://www.nytimes.com/privacy/cookie-policy\#how-do-i-manage-trackers}{Your
  Ad Choices}
\item
  \href{https://www.nytimes.com/privacy}{Privacy}
\item
  \href{https://help.nytimes.com/hc/en-us/articles/115014893428-Terms-of-service}{Terms
  of Service}
\item
  \href{https://help.nytimes.com/hc/en-us/articles/115014893968-Terms-of-sale}{Terms
  of Sale}
\item
  \href{https://spiderbites.nytimes.com}{Site Map}
\item
  \href{https://help.nytimes.com/hc/en-us}{Help}
\item
  \href{https://www.nytimes.com/subscription?campaignId=37WXW}{Subscriptions}
\end{itemize}
