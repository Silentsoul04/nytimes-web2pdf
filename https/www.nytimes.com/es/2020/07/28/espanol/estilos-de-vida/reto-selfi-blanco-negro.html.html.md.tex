Sections

SEARCH

\protect\hyperlink{site-content}{Skip to
content}\protect\hyperlink{site-index}{Skip to site index}

\href{https://www.nytimes.com/es/section/estilos-de-vida}{Estilos de
Vida}

\href{https://myaccount.nytimes.com/auth/login?response_type=cookie\&client_id=vi}{}

\href{https://www.nytimes.com/section/todayspaper}{Today's Paper}

\href{/es/section/estilos-de-vida}{Estilos de
Vida}\textbar{}\#RetoAceptado: por qué algunas mujeres publican selfis
en blanco y negro

\url{https://nyti.ms/308jeK9}

\begin{itemize}
\item
\item
\item
\item
\item
\item
\end{itemize}

Advertisement

\protect\hyperlink{after-top}{Continue reading the main story}

Supported by

\protect\hyperlink{after-sponsor}{Continue reading the main story}

\hypertarget{retoaceptado-por-quuxe9-algunas-mujeres-publican-selfis-en-blanco-y-negro}{%
\section{\#RetoAceptado: por qué algunas mujeres publican selfis en
blanco y
negro}\label{retoaceptado-por-quuxe9-algunas-mujeres-publican-selfis-en-blanco-y-negro}}

Una campaña que dice ser sobre ``mujeres que apoyan a mujeres'' ha
inundado Instagram con imágenes llamativas. Sin embargo, la motivación
detrás es poco clara.

\includegraphics{https://static01.nyt.com/images/2020/07/27/fashion/28CHALLENGEACCEPTED-2-ES/27CHALLENGEACCEPTED-2-articleLarge.jpg?quality=75\&auto=webp\&disable=upscale}

\href{https://www.nytimes.com/by/taylor-lorenz}{\includegraphics{https://static01.nyt.com/images/2020/03/18/reader-center/author-taylor-lorenz/author-taylor-lorenz-thumbLarge.png}}

Por \href{https://www.nytimes.com/by/taylor-lorenz}{Taylor Lorenz}

\begin{itemize}
\item
  Publicado 28 de julio de 2020Actualizado 30 de julio de 2020
\item
  \begin{itemize}
  \item
  \item
  \item
  \item
  \item
  \item
  \end{itemize}
\end{itemize}

\href{https://www.nytimes.com/2020/07/27/style/challenge-accepted-instagram.html}{Read
in English}

\href{https://www.nytimes.com/newsletters/el-times}{Regístrate para
recibir nuestro boletín} con lo mejor de The New York Times.

\begin{center}\rule{0.5\linewidth}{\linethickness}\end{center}

En los últimos días, muchos \emph{feeds} de Instagram se han visto
invadidos por imágenes en blanco y negro de mujeres tanto famosas como
desconocidas.

Los retratos a menudo son posados y modificados con filtros, tomados
desde ángulos halagadores y acompañados de leyendas favorables sobre
``apoyo a las mujeres''.

``Me encanta esta forma sencilla de elevarnos unas a otras.
\#retoaceptado. Gracias por nominarme @vanessabryant '', publicó la
modelo Cindy Crawford el 27 de julio junto a una
\href{https://www.instagram.com/p/CDJnHlQFuQi/?igshid=1vzk7hu5crx15}{foto
en blanco y negro de sí misma paseando en una playa}que parece salida de
un anuncio de Calvin Klein.

La premisa detrás de la tendencia de ``reto aceptado'' es que estas
fotos promueven el empoderamiento femenino, y que nominar a tus amigas
para que participen en la campaña es
\href{https://twitter.com/SoniAggarwal/status/1287784564262223872}{una
forma de que las mujeres se apoyen mutuamente}.

Hasta ahora más de 3 millones de imágenes se han publicado con la
etiqueta en inglés

\#ChallengeAccepted y muchas más han aparecido sin la etiqueta o con la
etiqueta en otros idiomas, como \#RetoAceptado en español.

``La tendencia va en repunte con el uso de la etiqueta en Instagram que
se duplicó solo en el último día'', dijo el lunes una portavoz de
Instagram. ``Según las publicaciones, vemos que la mayoría de las
participantes publican notas relacionadas con la fortaleza y el apoyo a
sus comunidades''.

Muchas mujeres han incluido la etiqueta \#womensupportingwomen
(``mujeres que apoyan a mujeres'' en inglés) en sus publicaciones.
``Reto aceptado'', escribió Khloe Kardashian en
\href{https://www.instagram.com/p/CDH3xn1BB3W/}{una publicación de
Instagram el 26 de julio}. ``A todas mis reinas: difundamos el amor y
recordemos ser un poco más amables entre nosotras.
\#womensupportingwomen''.

Esta no es la primera vez que los usuarios de Instagram se valen de
selfis en blanco y negro para apoyar una causa vaga. En 2016, las fotos
en blanco y negro con la etiqueta \#ChallengeAccepted (\#RetoAceptado)
intentaban difundir un mensaje de
``\href{https://metro.co.uk/2016/08/29/what-is-the-black-and-white-photo-challenge-6097096/}{concientización
del cáncer}''. Con los años, la tendencia fotográfica también se ha
utilizado para ``difundir la positividad''.

El desafío ha circulado igual que una cadena de correo. Las
participantes nominan al menos a otra mujer (y a menudo a varias) para
publicar su propio retrato en blanco y negro. Celebridades como las
actrices Kerry Washington, Jennifer Garner, Kristen Bell y Eva Longoria
han ayudado a que la campaña gane visibilidad.

Cristine Abram, gerente de relaciones públicas e influente de marketing
para \href{https://later.com/}{Later}, una empresa de publicidad en
redes sociales, dijo que
\href{https://www.nytimes.com/2020/07/23/us/alexandria-ocasio-cortez-sexism-congress.html}{un
video} de la congresista Alexandria Ocasio-Cortez en el que habla en
contra de
\href{https://www.nytimes.com/es/2020/07/24/espanol/estados-unidos/alexandria-ocasio-cortez-insulto.html}{los
comentarios sexistas del congresista Ted Yoh}o contra ella en el pleno
del Congreso la semana pasada llevó a un aumento en publicaciones en
redes sociales sobre feminismo y empoderamiento femenino, y que lo
anterior pudo haber contribuido a la última ola de selfis en blanco y
negro.

``Esa fue la chispa que llevó al resurgimiento de la etiqueta del
desafío'', dijo Abram. ``Todo está relacionado con el empoderamiento
femenino. La etiqueta ya existía para crear conciencia sobre otros temas
importantes. Al aprovecharla las participantes pudieron ganar tracción
mucho más rápido porque el algoritmo ya conocía la etiqueta''.

Un representante de Instagram dijo que la primera publicación que la
compañía podría identificar en este ciclo actual del desafío
\href{https://www.instagram.com/p/CCxGfzTBmXP/}{la publicó hace una
semana y media} la periodista brasileña Ana Paula Padrão. Otros han
notado que las mujeres en Turquía comenzaron a compartir fotos en blanco
y negro recientemente, para
\href{https://twitter.com/imaann_patel/status/1288080743198068736?s=21}{crear
conciencia sobre el feminicidio}.

\textbf{{[}¿Participaste en el desafío? ¿Por qué?}
\textbf{\href{https://www.nytimes.com/es/2020/07/28/espanol/estilos-de-vida/reto-selfi-blanco-negro.html\#commentsContainer}{Queremos
conocer tu opinión en español{]}}}

Aunque los retratos se han extendido ampliamente, las publicaciones en
sí dicen muy poco. Al igual
que\href{https://www.nytimes.com/2020/06/02/arts/music/what-blackout-tuesday.html}{el
cuadrado negro}, que se convirtió en un símbolo de solidaridad con los
manifestantes negros, pero que exigía muy poco a quienes lo
compartieron, la selfi en blanco y negro permite a las usuarias sentir
que están expresando una postura sin decir casi nada. Los influentes y
las celebridades adoran este tipo de ``desafíos'' porque no requieren
defender nada en realidad, algo que podría alienar a ciertas facciones
de su base de seguidores.

``Señoras'',
\href{https://twitter.com/alanalevinson/status/1287818379818971136}{tuiteó
la escritora Alana Levinson el lunes}, ``en lugar de publicar esa selfi
candente en blanco y negro, ¿por qué no nos adentramos en el feminismo
con algo poco arriesgado, como alejarse de su amigo que es un
abusador?''.

\begin{center}\rule{0.5\linewidth}{\linethickness}\end{center}

Otras mujeres han hablado sobre las repercusiones negativas que han
enfrentado por criticar la tendencia. ``Ahora recibo mensajes de odio en
Instagram de desconocidos porque dije que las selfis en blanco y negro
no son una causa'',
\href{https://twitter.com/OnlineAlison/status/1287804859773677568}{tuiteó}
la \href{https://www.instagram.com/p/CDJw9mpgqc1/}{presentadora de
pódcasts Ali Segel.} ``¡Aparentemente odio a las mujeres y no me amo a
mí misma!'' ¡¡¡Me ocuparé solo de mis asuntos por el resto de mi vida
!!!!!!''

``Creo que si este `movimiento' mostrara a mujeres trans o a mujeres de
capacidades diferentes o a negocios liderados por mujeres o el logro de
las mujeres a lo largo de la historia, tendría más sentido'', explicó
después Segel por mensaje interno en Twitter. ``Pero no comprendo la
idea de que esto sea un desafío o una causa''.

Brooke Hammerling, de 46 años, fundadora de la consultora para
directores generales de empresas de tecnología New New Thing, cuestionó
la eficacia de la
tendencia\href{https://medium.com/popculturemondays/pop-culture-mondays-7-20-20-3d2276f3c16}{en
su boletín semanal de cultura pop}el lunes.

``Simplemente no sé qué significa'', dijo por teléfono. ``Prácticamente
todas a mi alrededor han hecho el desafío, muchas de mis amigas y mucha
gente que amo. Estoy 100 por ciento a favor de las mujeres que apoyan a
las mujeres y estoy agradecida con las mujeres que me nominaron, pero no
entiendo cómo una selfi de vanidad en blanco y negro logra eso. Si
pudiéramos hacerlo con retratos de las mujeres que nos inspiraron, eso
estaría un poco más en línea con lo que esto intenta lograr''.

En línea, otras mujeres
\href{https://twitter.com/itsbooyeah/status/1287807365614534661}{sugirieron}
mejor compartir fotos de libros artículos, productos y organizaciones
benéficas que ayuden a las mujeres, en lugar de una selfi en blanco y
negro. Algunas personas se preguntaron si la tendencia habría sido
\href{https://twitter.com/wolfiecomedy/status/1287813475096387584}{iniciada
por los hombres.}

Camilla Blackett, guionista de televisión, insinuó que la campaña era
poco más que un pretexto para publicar retratos atractivos. ``¿Qué
intención tiene esta cosa de
\href{https://twitter.com/hashtag/ChallengeAccepted?src=hashtag_click}{\#ChallengeAccepted}
?'' se preguntó
\href{https://twitter.com/camillard/status/1287768130140246022}{en
Twitter el lunes}. ``Acaso la gente no sabe que puedes publicar una
selfi atractiva sin motivo alguno?''.

Taylor Lorenz es una reportera de tecnología en Nueva York que cubre la
cultura de internet. Antes de
\href{https://www.nytco.com/press/taylor-lorenz-to-join-styles/}{unirse
a The New York Times}, escribía sobre tecnología y cultura en The
Atlantic y The Daily Beast.
\href{https://twitter.com/taylorlorenz}{@taylorlorenz}

Advertisement

\protect\hyperlink{after-bottom}{Continue reading the main story}

\hypertarget{site-index}{%
\subsection{Site Index}\label{site-index}}

\hypertarget{site-information-navigation}{%
\subsection{Site Information
Navigation}\label{site-information-navigation}}

\begin{itemize}
\tightlist
\item
  \href{https://help.nytimes.com/hc/en-us/articles/115014792127-Copyright-notice}{©~2020~The
  New York Times Company}
\end{itemize}

\begin{itemize}
\tightlist
\item
  \href{https://www.nytco.com/}{NYTCo}
\item
  \href{https://help.nytimes.com/hc/en-us/articles/115015385887-Contact-Us}{Contact
  Us}
\item
  \href{https://www.nytco.com/careers/}{Work with us}
\item
  \href{https://nytmediakit.com/}{Advertise}
\item
  \href{http://www.tbrandstudio.com/}{T Brand Studio}
\item
  \href{https://www.nytimes.com/privacy/cookie-policy\#how-do-i-manage-trackers}{Your
  Ad Choices}
\item
  \href{https://www.nytimes.com/privacy}{Privacy}
\item
  \href{https://help.nytimes.com/hc/en-us/articles/115014893428-Terms-of-service}{Terms
  of Service}
\item
  \href{https://help.nytimes.com/hc/en-us/articles/115014893968-Terms-of-sale}{Terms
  of Sale}
\item
  \href{https://spiderbites.nytimes.com}{Site Map}
\item
  \href{https://help.nytimes.com/hc/en-us}{Help}
\item
  \href{https://www.nytimes.com/subscription?campaignId=37WXW}{Subscriptions}
\end{itemize}
