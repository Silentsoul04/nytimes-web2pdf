\href{/es/section/cultura}{Cultura}\textbar{}La última obra de Van Gogh
esconde una pista sobre sus días finales

\url{https://nyti.ms/3hStCvp}

\begin{itemize}
\item
\item
\item
\item
\item
\item
\end{itemize}

\includegraphics{https://static01.nyt.com/images/2020/07/27/arts/29VanGogh-ES-00/merlin_175011918_65435f35-c435-488b-843e-ab1daca3c57a-articleLarge.jpg?quality=75\&auto=webp\&disable=upscale}

Sections

\protect\hyperlink{site-content}{Skip to
content}\protect\hyperlink{site-index}{Skip to site index}

\hypertarget{la-uxfaltima-obra-de-van-gogh-esconde-una-pista-sobre-sus-duxedas-finales}{%
\section{La última obra de Van Gogh esconde una pista sobre sus días
finales}\label{la-uxfaltima-obra-de-van-gogh-esconde-una-pista-sobre-sus-duxedas-finales}}

Un investigador dice que ha descubierto la ubicación precisa donde el
artista pintó ``Raíces de árbol'', que se cree fue la última obra en la
que trabajaba el día en que sufrió una herida mortal de bala.

Se cree que estas raíces en una ladera de Auvers-sur-Oise son la
inspiración para la pintura final de Van Gogh.Credit...Elliott Verdier
para The New York Times

Supported by

\protect\hyperlink{after-sponsor}{Continue reading the main story}

Por Nina Siegal

\begin{itemize}
\item
  29 de julio de 2020
\item
  \begin{itemize}
  \item
  \item
  \item
  \item
  \item
  \item
  \end{itemize}
\end{itemize}

\href{https://www.nytimes.com/2020/07/28/arts/design/vincent-van-gogh-tree-roots.html}{Read
in English}

\href{https://www.nytimes.com/newsletters/el-times}{Regístrate para
recibir nuestro boletín} con lo mejor de The New York Times.

\begin{center}\rule{0.5\linewidth}{\linethickness}\end{center}

ÁMSTERDAM --- Hace ciento treinta años, Vincent van Gogh despertó en su
habitación en una posada en Auvers-sur-Oise, Francia, y salió, como
solía hacer, con un lienzo para pintar. Esa noche, regresó a la posada
con una mortal herida de bala. Murió dos días después, el 29 de julio de
1890.

Los académicos han especulado durante mucho tiempo sobre la secuencia de
eventos el día del tiroteo, y ahora Wouter van der Veen, un investigador
en Francia, dice que ha descubierto una gran pieza del rompecabezas: la
ubicación precisa donde Van Gogh creó su última pintura,
``\href{https://www.vangoghmuseum.nl/en/collection/s0195V1962}{Raíces de
árbol}''. El hallazgo podría ayudar a comprender mejor cómo el artista
pasó su último día de trabajo.

\includegraphics{https://static01.nyt.com/images/2020/07/30/arts/29VanGogh-ES-01/merlin_175015488_5a6acbf9-926b-458f-be3a-c07a40fad72c-articleLarge.jpg?quality=75\&auto=webp\&disable=upscale}

``Ahora sabemos lo que hizo durante su último día'' antes del disparo,
dijo Van der Veen, director científico del Instituto Van Gogh, una
organización sin fines de lucro establecida para preservar la pequeña
habitación del artista en el Auberge Ravoux, la posada en
Auvers-sur-Oise. ``Sabemos que pasó todo el día pintando esta obra'',
señaló Van der Veen.

``Raíces de árbol'' fue pintada en la Rue Daubigny, una carretera
principal que atraviesa Auvers-sur-Oise, localidad ubicada unos 32
kilómetros al norte de París, descubrió Van der Veen. Las raíces y
troncos de árboles enredados y retorcidos aún se pueden ver hoy en la
ladera de una colina, a solo 152 metros del Auberge Ravoux, donde Van
Gogh pasó los últimos 70 días de su vida.

Image

Wouter van der Veen, director científico del Instituto Van Gogh,
descubrió por accidente el lugar donde el artista trabajó en ``Raíces de
árbol''.Credit...Elliott Verdier para The New York Times

Investigadores del Museo Van Gogh en Ámsterdam han respaldado el
hallazgo. El martes 28 de julio, la directora del museo, Emilie
Gordenker, asistió a una presentación donde se develó el lugar.

Louis van Tilborgh, investigador principal del Museo Van Gogh, dijo en
una entrevista que el descubrimiento era ``una interpretación, pero
parece que es verdad''.

Van der Veen dijo que llegó al hallazgo mientras miraba imágenes de
Auvers de alrededor de 1905, que le había prestado Janine Demuriez, una
mujer francesa de 94 años que ha coleccionado cientos de postales
históricas. Una de ellas muestra a un ciclista en la Rue Daubigny, de
pie junto a un terraplén empinado, donde las raíces de los árboles son
claramente visibles.

Image

Una postal de 1905 llevó a Van der Veen a su
descubrimiento.Credit...Elliott Verdier para The New York Times

Van der Veen dijo que durante el confinamiento simplemente tenía la
postal en la pantalla en su casa en Estrasburgo, Francia, cuando algo
hizo clic en su cabeza: la postal le recordaba a ``Raíces de árbol''.
Abrió una versión digital de la pintura, y las comparó lado a lado.

La postal ``no es un documento secreto que nadie puede encontrar'', dijo
Van der Veen. ``Muchas personas ya la han visto, y reconocen el tema, el
motivo de las raíces de árbol. Estaba oculto a plena vista''.

Como no podía viajar desde Estrasburgo, Van der Veen llamó a
Dominique-Charles Janssen, dueño del Instituto Van Gogh, que estaba en
Auvers, y le pidió que echara un vistazo a la zona.

``Yo diría que del 45 al 50 por ciento todavía está allí'', Janssens
dijo en una entrevista telefónica, al referirse al enredo de raíces.
``Cortaron algunos de los árboles, y estaba cubierto de hiedra, pero
eliminamos parte de eso''.

Van Gogh habría caminado a lo largo de la Rue Daubigny para llegar a la
iglesia de la ciudad, que pintó para
\href{https://www.musee-orsay.fr/en/collections/works-in-focus/painting/commentaire_id/the-church-in-auvers-sur-oise-7170.html?cHash=6d6ea9b5c1}{``La
iglesia de Auvers-sur-Oise''} en junio de 1890, y para dirigirse a los
extensos campos de trigo justo afuera de la ciudad, donde pintó
\href{https://www.vangoghmuseum.nl/en/collection/s0149V1962}{``Trigal
con cuervos''} en julio, dijo Van der Veen.

Image

El río Oise en Auvers-sur-OiseCredit...Elliott Verdier for The New York
Times

Image

La iglesia cercana que pintó Van GoghCredit...Elliott Verdier para The
New York Times

Image

~Los campos de Auvers-sur-Oise fueron otro tema favorito para Van
Gogh.Credit...Elliott Verdier para The New York Times

Durante mucho tiempo se ha debatido sobre qué pintura fue la última obra
de Van Gogh, porque el artista tendía a no fecharlas. Mucha gente cree
que fue ``Trigal con cuervos'', porque la película biográfica de 1956 de
Vincente Minnelli \href{https://www.imdb.com/title/tt0049456/}{``Sed de
vivir''} representa a Van Gogh, interpretado por Kirk Douglas, pintando
esa obra mientras enloquece, justo antes de suicidarse.

Andries Bonger, quien escribió algunos de los eventos que rodearon la
muerte de Vincent y era cuñado de Theo van Gogh, el hermano de Vincent,
señaló en una carta: ``La mañana antes de su muerte, había pintado una
escena del bosque, llena de sol y vida''.

En 2012, El Museo Van Gogh publicó un artículo de Van Tilborgh y Bert
Maes que argumentaba que la carta se refería a ``Raíces de árbol'', una
pintura inacabada en la colección del museo. Esa afirmación ahora ha
sido ampliamente aceptada por los estudiosos.

Image

``Yo diría que del 45 al 50 por ciento todavía está allí'', dijo
Janssens sobre el enredo de raíces representadas en el último trabajo de
Van Gogh.Credit...Arthénon

Debido a la forma en que se representa la luz en las raíces, Van der
Veen dice que cree que Van Gogh estaba mirando su la escena hacia el
final de la tarde, alrededor de las 5 p.m. o 6 p.m. Dice que cree que
esto significa que Van Gogh probablemente haya pasado todo ese día
pintando.

Van der Veen agregó que la nueva evidencia desafía una teoría presentada
en 2011 por Steven Naifeh y Gregory White Smith en la biografía,
\emph{Van Gogh: la vida}. Ellos argumentaron que Van Gogh no se suicidó,
sino que se habría emborrachado y discutido con dos niños pequeños,
quienes lo mataron accidentalmente, no muy lejos del Auberge Ravoux. La
investigación de Van der Veen sobre ``Raíces de árbol'' se publicó en un
libro en Francia, y también estará disponible en inglés, en
\href{http://www.arthenon.com/roots}{forma digital}.

``Ahora que sabemos que pintó durante todo el día, hubo aún menos tiempo
para que eso sucediera'', dijo Van der Veen.

Naifeh respondió que sería imposible determinar la hora de una pintura
según el ángulo de la luz. ``No es una fotografía, es una pintura'',
dijo en una entrevista telefónica. ``Van Gogh pintaba de manera algo
abstracta, y siempre presentaba muchos inventos pictóricos'', agregó,
por lo que sería difícil saber si pintaba la luz que veía con sus
propios ojos o simplemente la creaba en el lienzo.

Image

La vista desde las escaleras de la habitación de Van Gogh en la posada
Auberge Ravoux.Credit...Elliott Verdier para The New York Times

Naifeh dijo que el descubrimiento podría incluso apoyar su teoría del
asesinato. ``El hecho de que salió y pintó todo el día, no solo una
pintura promedio sino una pintura muy importante, indica que puede no
haber estado deprimido'', dijo. ``De lo contrario, fue un día normal
productivo, y eso va en contra de la idea de que él podría ir y
suicidarse''.

Van der Veen estuvo de acuerdo en un punto. ``Confirma todo lo que la
mayoría de testigos dice en ese momento, que su comportamiento fue
perfectamente normal en los últimos días'', dijo. ``No había señal de
que estuviera en crisis''.

Sin embargo, Van der Veen sostiene que Van Gogh se suicidó, lo que es
también la posición oficial del Museo Van Gogh.

Van Gogh también había hecho un dibujo de raíces de árboles cuando vivía
en La Haya en 1882. Describió la obra de arte a su hermano, Theo, en una
carta.

Él escribió que quería que el árbol ``expresara algo de la lucha de la
vida'', y que lo veía ``enraizándose frenética y fervientemente, por así
decirlo, en la tierra, e incluso siendo destruído por la tormenta''.

Van der Veen dijo que ``Raíces de árbol'' expresaba algo similar.

``Terminar su vida con esta pintura tiene mucho sentido'', dijo. ``La
pintura ilustra la lucha de la vida, y la lucha con la muerte. Eso es lo
que deja atrás. Es una nota de despedida en colores''.

Image

Tanto Vincent Van Gogh como su hermano Theo están enterrados en este
cementerio en Auvers-sur-Oise.Credit...Elliott Verdier para The New York
Times

\begin{center}\rule{0.5\linewidth}{\linethickness}\end{center}

Advertisement

\protect\hyperlink{after-bottom}{Continue reading the main story}

\hypertarget{site-index}{%
\subsection{Site Index}\label{site-index}}

\hypertarget{site-information-navigation}{%
\subsection{Site Information
Navigation}\label{site-information-navigation}}

\begin{itemize}
\tightlist
\item
  \href{https://help.nytimes.com/hc/en-us/articles/115014792127-Copyright-notice}{©~2020~The
  New York Times Company}
\end{itemize}

\begin{itemize}
\tightlist
\item
  \href{https://www.nytco.com/}{NYTCo}
\item
  \href{https://help.nytimes.com/hc/en-us/articles/115015385887-Contact-Us}{Contact
  Us}
\item
  \href{https://www.nytco.com/careers/}{Work with us}
\item
  \href{https://nytmediakit.com/}{Advertise}
\item
  \href{http://www.tbrandstudio.com/}{T Brand Studio}
\item
  \href{https://www.nytimes.com/privacy/cookie-policy\#how-do-i-manage-trackers}{Your
  Ad Choices}
\item
  \href{https://www.nytimes.com/privacy}{Privacy}
\item
  \href{https://help.nytimes.com/hc/en-us/articles/115014893428-Terms-of-service}{Terms
  of Service}
\item
  \href{https://help.nytimes.com/hc/en-us/articles/115014893968-Terms-of-sale}{Terms
  of Sale}
\item
  \href{https://spiderbites.nytimes.com}{Site Map}
\item
  \href{https://help.nytimes.com/hc/en-us}{Help}
\item
  \href{https://www.nytimes.com/subscription?campaignId=37WXW}{Subscriptions}
\end{itemize}
