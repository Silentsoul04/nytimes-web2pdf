Sections

SEARCH

\protect\hyperlink{site-content}{Skip to
content}\protect\hyperlink{site-index}{Skip to site index}

\href{https://www.nytimes.com/es/section/ciencia-y-tecnologia}{Ciencia y
Tecnología}

\href{https://myaccount.nytimes.com/auth/login?response_type=cookie\&client_id=vi}{}

\href{https://www.nytimes.com/section/todayspaper}{Today's Paper}

\href{/es/section/ciencia-y-tecnologia}{Ciencia y
Tecnología}\textbar{}Un estudio revela que los niños podrían portar
altos niveles de coronavirus

\url{https://nyti.ms/30hGuFC}

\begin{itemize}
\item
\item
\item
\item
\item
\end{itemize}

\href{https://www.nytimes.com/es/spotlight/coronavirus?action=click\&pgtype=Article\&state=default\&region=TOP_BANNER\&context=storylines_menu}{El
brote de coronavirus}

\begin{itemize}
\tightlist
\item
  \href{https://www.nytimes.com/es/interactive/2020/espanol/mundo/coronavirus-en-estados-unidos.html?action=click\&pgtype=Article\&state=default\&region=TOP_BANNER\&context=storylines_menu}{Mapa
  y casos en EE. UU.}
\item
  \href{https://www.nytimes.com/es/2020/07/23/espanol/america-latina/bolivia-cloro-coronavirus-ivermectina.html?action=click\&pgtype=Article\&state=default\&region=TOP_BANNER\&context=storylines_menu}{Dióxido
  de cloro, ivermectina y más: ¿funcionan?}
\item
  \href{https://www.nytimes.com/es/interactive/2020/science/coronavirus-tratamientos-curas.html?action=click\&pgtype=Article\&state=default\&region=TOP_BANNER\&context=storylines_menu}{Fármacos
  y tratamientos}
\item
  \href{https://www.nytimes.com/es/2020/07/28/espanol/ciencia-y-tecnologia/anticuerpos-coronavirus-inmunidad.html?action=click\&pgtype=Article\&state=default\&region=TOP_BANNER\&context=storylines_menu}{Anticuerpos
  e inmunidad}
\item
  \href{https://www.nytimes.com/es/2020/04/29/espanol/estilos-de-vida/oximetro-para-que-sirve.html?action=click\&pgtype=Article\&state=default\&region=TOP_BANNER\&context=storylines_menu}{Oxímetros}
\end{itemize}

Advertisement

\protect\hyperlink{after-top}{Continue reading the main story}

Supported by

\protect\hyperlink{after-sponsor}{Continue reading the main story}

\hypertarget{un-estudio-revela-que-los-niuxf1os-podruxedan-portar-altos-niveles-de-coronavirus}{%
\section{Un estudio revela que los niños podrían portar altos niveles de
coronavirus}\label{un-estudio-revela-que-los-niuxf1os-podruxedan-portar-altos-niveles-de-coronavirus}}

La investigación no prueba que los niños infectados sean contagiosos,
pero debería tomarse en cuenta en el debate sobre el regreso a las
escuelas, dijeron algunos expertos.

\includegraphics{https://static01.nyt.com/images/2020/08/01/science/31virus-children-ES/30VIRUS-CHILDREN1-articleLarge.jpg?quality=75\&auto=webp\&disable=upscale}

Por \href{https://www.nytimes.com/by/apoorva-mandavilli}{Apoorva
Mandavilli}

\begin{itemize}
\item
  Publicado 31 de julio de 2020Actualizado 3 de agosto de 2020
\item
  \begin{itemize}
  \item
  \item
  \item
  \item
  \item
  \end{itemize}
\end{itemize}

\href{https://www.nytimes.com/2020/07/30/health/coronavirus-children.html}{Read
in English}

\href{https://www.nytimes.com/newsletters/el-times}{Regístrate para
recibir nuestro boletín} con lo mejor de The New York Times.

\begin{center}\rule{0.5\linewidth}{\linethickness}\end{center}

Ha sido una frase reconfortante en el debate nacional sobre la
reapertura de las escuelas: la mayor parte de los niños pequeños se
libran del
\href{https://www.nytimes.com/es/interactive/2020/espanol/mundo/coronavirus-en-estados-unidos.html}{coronavirus}
y no parecen contagiar a otros, al menos no muy a menudo.

Pero el jueves 30 de julio, un estudio introdujo un giro inesperado e
indeseable a esta historia.

Según la investigación, los niños infectados
\href{https://jamanetwork.com/journals/jamapediatrics/fullarticle/2768952}{tienen
al menos la misma cantidad de coronavirus en nariz y garganta} que los
adultos infectados. De hecho, los autores descubrieron que los niños
menores de cinco años pueden albergar hasta 100 veces más virus en el
tracto respiratorio superior que los adultos.

Esa medida no necesariamente prueba que los niños transmiten el virus a
otros. Aún así, los hallazgos deberían influir en el debate sobre la
reapertura de las escuelas, dijeron varios expertos.

``La situación de la escuela es muy complicada; hay muchos matices más
allá del científico'', dijo Taylor Heald-Sargent, experta en
enfermedades infecciosas pediátricas del Hospital de Niños Ann and
Robert H. Lurie de Chicago, quien dirigió el estudio, publicado en JAMA
Pediatrics.

``Pero una conclusión es que no podemos suponer que solo porque los
niños no se están enfermando, o no se están enfermando mucho, no tienen
el virus''.

El estudio no carece de advertencias: fue pequeño y no especificó sexo u
origen étnico o racial, o si tenían condiciones subyacentes. Las pruebas
buscaron ARN viral, piezas genéticas del coronavirus, en lugar del virus
en sí. (Su material genético es ARN, no ADN).

Aún así, los expertos se alarmaron al saber que los niños pequeños
pueden portar cantidades significativas del coronavirus.

``He escuchado a muchas personas decir: `Bueno, los niños no son
susceptibles, los niños no se infectan'. Y esto claramente muestra que
no es verdad'', dijo Stacey Schultz-Cherry, viróloga del St. Jude
Children's Research Hospital.

``Creo que este es un primer paso importante, muy importante, para
entender el papel que tienen los niños en la transmisión''.

Jason Kindrachuk, virólogo de la Universidad de Manitoba, dijo: ``Ahora
que llegamos a fines de julio e intentamos abrir las escuelas el próximo
mes, esto realmente necesita tomarse en consideración''.

La prueba de diagnóstico estándar amplifica el material genético del
virus en ciclos con la señal cada vez más brillante en cada ronda.
Mientras más virus haya en el hisopo inicialmente, menos ciclos se
necesitarán para obtener un resultado claro.

Heald-Sargent, que tiene interés en la investigación de los coronavirus,
comenzó a notar que las pruebas de los niños regresaban con bajos
``umbrales de ciclo'' (CT, por su sigla en inglés), lo que sugería que
sus muestras estaban llenas de virus.

Intrigada, llamó al laboratorio del hospital un domingo y pidió que
revisaran los resultados de las pruebas de las últimas semanas. ``Ni
siquiera era algo que nos propusimos buscar'', dijo.

Ella y sus colegas analizaron muestras recolectadas con hisopos
nasofaríngeos entre el 23 de marzo y el 27 de abril en sitios de pruebas
hechas en autos en Chicago y de personas que acudieron al hospital por
cualquier motivo, incluidos los síntomas de la COVID-19.

Observaron los hisopos recabados de 145 personas: 46 niños menores de
cinco años; 51 niños de entre cinco y 17 años; y 48 adultos de entre 18
y 65 años. Para evitar las críticas de que se esperaba que los niños
realmente enfermos tuvieran muchos virus, el equipo excluyó a los niños
que necesitaron oxígeno suplementario. La mayoría de los niños en el
estudio tuvieron solo fiebre o tos, dijo Heald-Sargent.

Para comparar los grupos de manera justa, el equipo incluyó solo niños y
adultos que tenían síntomas leves a moderados y sobre los cuales había
información sobre el momento en que comenzaron los síntomas.
Heald-Sargent excluyó a las personas que no tenían síntomas y que no
recordaban cuándo habían comenzado a sentirse enfermas, así como a
aquellas que tuvieron síntomas durante más de una semana antes de la
prueba.

Los resultados confirmaron el presentimiento de Heald-Sargent: los niños
y adultos tenían CT similares, con una mediana de aproximadamente 11 y
hasta 17. Pero los niños menores de cinco años tenían un CT
significativamente más bajo, de aproximadamente 6,5. El límite superior
del rango en estos niños era un CT de 12, sin embargo, sigue siendo
comparable a los de los niños mayores y los adultos.

``Definitivamente muestra que los niños tienen niveles de virus
similares y, quizás, incluso más altos que los adultos'', dijo
Heald-Sargent. ``No sería sorprendente si pudieran arrojar'' el virus y
transmitirlo a otros.

\includegraphics{https://static01.nyt.com/images/2020/07/30/science/31virus-children-ES-02/merlin_171979788_859ad6e5-7a26-4f54-ab55-3acbf62e70b6-articleLarge.jpg?quality=75\&auto=webp\&disable=upscale}

Los resultados son consistentes con aquellos de un
\href{https://www.nytimes.com/2020/05/05/health/coronavirus-children-transmission-school.html}{estudio
alemán con 47 niños infectados} de entre uno y 11 años de edad, que
mostró que los niños que no tenían síntomas tenían cargas virales tan
altas como las de los adultos, o incluso mayores. Y un estudio reciente
de Francia halló que los niños asintomáticos tenían
\href{https://academic.oup.com/cid/article/doi/10.1093/cid/ciaa1044/5876373}{CT
de valores similares} a los de niños con síntomas.

Los valores de CT son un indicador razonable de la cantidad de
coronavirus presente, dijo Kindrachuk, quien usó esta métrica durante
los brotes de ébola en África occidental.

Aún así, él y otros dijeron que, idealmente, los investigadores deberían
cultivar virus infecciosos a partir de muestra, en lugar de analizar
solo el ARN del virus.

``Sospecho que probablemente se traducirá en que también hay más virus
reales allí, pero no podemos decir eso sin ver los datos'', dijo Juliet
Morrison, viróloga de la Universidad de California en Riverside.

Algunos virus de ARN se multiplican rápidamente y son propensos a
errores genéticos que hacen que el virus sea incapaz de infectar a las
células. Algunos ARN detectados en niños pueden representar estos virus
``defectuosos'': ``Necesitamos entender cuánto de eso son realmente
virus infecciosos'', dijo Schultz-Cherry.

(Los investigadores dijeron que no tenían acceso al tipo de laboratorio
de alta seguridad requerido para cultivar coronavirus infecciosos, pero
otros equipos \href{https://pubmed.ncbi.nlm.nih.gov/32603290/}{han
cultivado virus} de muestras de niños).

Todos los expertos enfatizaron que los hallazgos al menos indican que
los niños pueden infectarse. Aquellos que albergan una gran cantidad de
virus pueden contagiarlo a otros en sus hogares, o a maestros y otros
miembros del personal cuando las escuelas vuelvan a abrir.

Muchos distritos escolares planean proteger a los estudiantes y miembros
del personal mediante la implementación de distanciamiento físico,
cubrebocas de tela e higiene de manos. Pero no queda claro en qué medida
podrán los miembros del personal y los profesores evitar que los niños
pequeños se acerquen demasiado a otros, dijo Kindrachuk.

``Francamente, no he visto mucha discusión sobre cómo se va a controlar
ese aspecto'', dijo.

Las observaciones de escuelas en varios países han sugerido que, al
menos en lugares con brotes leves y donde se aplican medidas
preventivas, los niños no parecen transmitir el virus a otros de manera
eficiente.

Las respuestas inmunes fuertes en los niños podrían limitar la cantidad
de virus que pueden transmitir a otros y por cuánto tiempo. La salud
general de los niños, las afecciones subyacentes como la obesidad y la
diabetes, y el sexo también pueden influir en la capacidad de transmitir
el virus.

Algunos expertos han sugerido que los niños
\href{https://www.nytimes.com/2020/06/30/us/coronavirus-schools-reopening-guidelines-aap.html}{pueden
transmitir menos virus} debido a su menor capacidad pulmonar, altura u
otros aspectos físicos.

Morrison lo descartó. El virus se elimina por el tracto respiratorio
superior, no por los pulmones, destacó.

``Vamos a estar reabriendo guarderías y escuelas primarias'', dijo. Si
estos resultados se sustentan, ``entonces sí, estaría preocupada''.

Apoorva Mandavilli es reportera del Times y se enfoca en ciencia y salud
global. En 2019 ganó el premio Victor Cohn a la Excelencia en Reportaje
sobre Ciencias Médicas. @apoorva\_nyc

\begin{center}\rule{0.5\linewidth}{\linethickness}\end{center}

Advertisement

\protect\hyperlink{after-bottom}{Continue reading the main story}

\hypertarget{site-index}{%
\subsection{Site Index}\label{site-index}}

\hypertarget{site-information-navigation}{%
\subsection{Site Information
Navigation}\label{site-information-navigation}}

\begin{itemize}
\tightlist
\item
  \href{https://help.nytimes.com/hc/en-us/articles/115014792127-Copyright-notice}{©~2020~The
  New York Times Company}
\end{itemize}

\begin{itemize}
\tightlist
\item
  \href{https://www.nytco.com/}{NYTCo}
\item
  \href{https://help.nytimes.com/hc/en-us/articles/115015385887-Contact-Us}{Contact
  Us}
\item
  \href{https://www.nytco.com/careers/}{Work with us}
\item
  \href{https://nytmediakit.com/}{Advertise}
\item
  \href{http://www.tbrandstudio.com/}{T Brand Studio}
\item
  \href{https://www.nytimes.com/privacy/cookie-policy\#how-do-i-manage-trackers}{Your
  Ad Choices}
\item
  \href{https://www.nytimes.com/privacy}{Privacy}
\item
  \href{https://help.nytimes.com/hc/en-us/articles/115014893428-Terms-of-service}{Terms
  of Service}
\item
  \href{https://help.nytimes.com/hc/en-us/articles/115014893968-Terms-of-sale}{Terms
  of Sale}
\item
  \href{https://spiderbites.nytimes.com}{Site Map}
\item
  \href{https://help.nytimes.com/hc/en-us}{Help}
\item
  \href{https://www.nytimes.com/subscription?campaignId=37WXW}{Subscriptions}
\end{itemize}
