Sections

SEARCH

\protect\hyperlink{site-content}{Skip to
content}\protect\hyperlink{site-index}{Skip to site index}

\href{https://www.nytimes.com/es/section/negocios}{Negocios}

\href{https://myaccount.nytimes.com/auth/login?response_type=cookie\&client_id=vi}{}

\href{https://www.nytimes.com/section/todayspaper}{Today's Paper}

\href{/es/section/negocios}{Negocios}\textbar{}Las negociaciones de la
deuda en Argentina ponen a prueba el capitalismo amigable

\url{https://nyti.ms/30eJMJQ}

\begin{itemize}
\item
\item
\item
\item
\item
\item
\end{itemize}

\href{https://www.nytimes.com/es/spotlight/coronavirus?action=click\&pgtype=Article\&state=default\&region=TOP_BANNER\&context=storylines_menu}{El
brote de coronavirus}

\begin{itemize}
\tightlist
\item
  \href{https://www.nytimes.com/es/interactive/2020/espanol/mundo/coronavirus-en-estados-unidos.html?action=click\&pgtype=Article\&state=default\&region=TOP_BANNER\&context=storylines_menu}{Mapa
  y casos en EE. UU.}
\item
  \href{https://www.nytimes.com/es/2020/07/23/espanol/america-latina/bolivia-cloro-coronavirus-ivermectina.html?action=click\&pgtype=Article\&state=default\&region=TOP_BANNER\&context=storylines_menu}{Dióxido
  de cloro, ivermectina y más: ¿funcionan?}
\item
  \href{https://www.nytimes.com/es/interactive/2020/science/coronavirus-tratamientos-curas.html?action=click\&pgtype=Article\&state=default\&region=TOP_BANNER\&context=storylines_menu}{Fármacos
  y tratamientos}
\item
  \href{https://www.nytimes.com/es/2020/07/28/espanol/ciencia-y-tecnologia/anticuerpos-coronavirus-inmunidad.html?action=click\&pgtype=Article\&state=default\&region=TOP_BANNER\&context=storylines_menu}{Anticuerpos
  e inmunidad}
\item
  \href{https://www.nytimes.com/es/2020/04/29/espanol/estilos-de-vida/oximetro-para-que-sirve.html?action=click\&pgtype=Article\&state=default\&region=TOP_BANNER\&context=storylines_menu}{Oxímetros}
\end{itemize}

Advertisement

\protect\hyperlink{after-top}{Continue reading the main story}

Supported by

\protect\hyperlink{after-sponsor}{Continue reading the main story}

Negocios

\hypertarget{las-negociaciones-de-la-deuda-en-argentina-ponen-a-prueba-el-capitalismo-amigable}{%
\section{Las negociaciones de la deuda en Argentina ponen a prueba el
capitalismo
amigable}\label{las-negociaciones-de-la-deuda-en-argentina-ponen-a-prueba-el-capitalismo-amigable}}

BlackRock, la empresa más grande de manejo de inversiones del mundo, se
opone a un acuerdo que resolvería la deuda con Argentina, que lucha
contra la pobreza y la pandemia.

\includegraphics{https://static01.nyt.com/images/2020/07/31/business/31Argentina-Debt-ES-00/31argentinadebt-1-articleLarge.jpg?quality=75\&auto=webp\&disable=upscale}

Por \href{https://www.nytimes.com/by/peter-s-goodman}{Peter S. Goodman}
y Daniel Politi

\begin{itemize}
\item
  31 de julio de 2020
\item
  \begin{itemize}
  \item
  \item
  \item
  \item
  \item
  \item
  \end{itemize}
\end{itemize}

\href{https://www.nytimes.com/2020/07/31/business/argentina-debt.html}{Read
in English}

\href{https://www.nytimes.com/newsletters/el-times}{Regístrate para
recibir nuestro boletín} con lo mejor de The New York Times.

\begin{center}\rule{0.5\linewidth}{\linethickness}\end{center}

LONDRES--- Laurence D. Fink se presenta como la vanguardia de una forma
progresista **** de capitalismo en el que las ganancias no lo son todo:
se espera que el dinero bienpensante promueva las protecciones
ambientales y sociales.

Como director ejecutivo de BlackRock, la firma de administración de
inversiones más grande del planeta, Fink supervisa más de 7 billones de
dólares. Ha dirigido una parte de dicha fortuna hacia Argentina, un país
destrozado por la crisis, al comprar bonos del gobierno.

Pero en tanto Argentina ---en suspensión de pagos desde mayo--- busca
que se le condonen 66.000 millones de dólares en bonos, el credo
habitual de Fink, el del
\href{https://www.alainet.org/es/articulo/204355}{capitalismo de las
partes interesadas} (\emph{stakeholder capitalism}), ha chocado con los
más tradicionales imperativos de las pérdidas y ganancias. Aunque la
pobreza crece en Argentina conforme la pandemia empeora una crisis
económica, BlackRock se opone a un acuerdo propuesto por el gobierno y
anima a otros acreedores a rechazarlo mientras que aguarda un trato
marginalmente mejor.

Fink se ha involucrado en las negociaciones y hablado en dos ocasiones
con el ministro de Economía de Argentina, según tres personas con
conocimiento de las conversaciones. Los términos propuestos por el
gobierno y sus acreedores solo difieren en tres centavos de dólar.

``Los tipos de BlackRock se han puesto al teléfono con una cantidad
significativa de acreedores'', dijo Hans Humes, presidente de Greylock
Capital Management, otro acreedor en la negociación. ``Convencieron a
mucha gente de que si todos apoyábamos su acuerdo los argentinos lo
aceptarían. Ha resultado ser un enfrentamiento brutal''.

La postura de BlackRock ha enfrentado a la empresa con el Fondo
Monetario Internacional, que otorgó a Argentina
\href{https://www.nytimes.com/2018/06/07/business/argentina-imf-debt.html}{un
paquete de rescate} con valor de más de 50.000 millones de dólares hace
dos años y ha respaldado la propuesta de Argentina conforme se acerca el
plazo del 4 de agosto.

\includegraphics{https://static01.nyt.com/images/2020/07/31/business/31Argentina-Debt-ES-01/merlin_146284194_cbc130c9-6ac7-407a-8d12-7206f9904c77-articleLarge.jpg?quality=75\&auto=webp\&disable=upscale}

La directora gerente del FMI, Kristalina Georgieva, ha
\href{https://www.imf.org/es/News/Articles/2020/02/04/pr2034-statement-by-imf-managing-director-kristalina-georgieva-on-argentina}{elogiado
el enfoque de Argentina} y ha insistido en que los tenedores de bonos
deben acordar una condonación sustantiva de deuda de tal forma que el
país pueda manejar los pagos futuros. Los funcionarios del FMI han
asegurado al gobierno un nuevo rescate si Argentina no logra llegar a un
acuerdo.

La alternativa sería un incumplimiento desordenado que evitaría que
Argentina recurriera a los mercados internacionales al bloquear el
acceso de sus empresas al capital y profundizaría la recesión.

La posición de BlackRock también la enfrenta con un grupo de economistas
destacados, entre ellos un par de ganadores del Nobel, Joseph Stiglitz y
Edmund Phelps. En mayo,
\href{https://lta.reuters.com/articulo/finanzas-argentina-stiglitz-idLTAKBN22I25L-OUSLT}{publicaron
una carta} en la que alentaban a los tenedores de bonos a ponerse de
acuerdo con el gobierno.

``Argentina ha presentado una oferta responsable a los acreedores que
refleja la capacidad de pago del país'', decía la misiva, firmada por
138 economistas, entre los que se encontraba Carmen Reinhart, ahora
economista jefa en el Banco Mundial.

En un comunicado, BlackRock dijo que trabajaba diligentemente para
llegar a un acuerdo y al mismo tiempo recuperar tanto como fuera posible
para sus clientes. Alrededor de dos tercios de las inversiones que
maneja proviene de los ahorros para el retiro de trabajadores de todo el
mundo.

``En este proceso de reestructuración, los gerentes del fondo cargan la
obligación fiduciaria de tomar decisiones en interés de estos
ahorradores y al mismo tiempo reconocen las difíciles circunstancias que
enfrenta el gobierno argentino, entre ellas el desafío de la COVID-19'',
decía el comunicado.

Image

Los funcionarios argentinos dijeron que pagar más a los acreedores
equivaldría a transferir riqueza de las personas que no tenían casi nada
a los inversores internacionales.Credit...Juan Ignacio Roncoroni/EPA,
vía Shutterstock

La parálisis en Argentina refleja la complejidad de las discusiones en
torno a la deuda en una era en la que las personas comunes y corrientes
están de hecho, en la mesa de negociación. En décadas pasadas, los bonos
emitidos por los países en desarrollo eran en gran parte controlados por
los grandes bancos. Cuando los gobiernos no podían pagar, los jefes de
los bancos llegaban a un acuerdo. Hoy en día los inversores que poseen
bonos de mercados emergentes abarcan toda una gama: desde fondos
especializados con alta tolerancia al riesgo hasta fondos de pensiones
conservadores.

Que la empresa de Fink juegue un papel protagónico al presionar a
Argentina contrasta con su campaña por hacer que los negocios impulsen
el progreso social.

Hace dos años, Fink ---a quien se le menciona
\href{https://www.cnbc.com/2020/04/06/biden-donors-float-elizabeth-warren-larry-fink-others-for-key-roles.html}{en
informes noticiosos} como posible secretario del Tesoro en caso de que
\href{https://www.nytimes.com/es/interactive/2020/espanol/estados-unidos/joe-biden-elecciones.html}{Joe
Biden} llegue a la presidencia--- escribió
\href{http://www.corporance.es/wp-content/uploads/2018/01/Larry-Fink-letter-to-CEOs-2018-1.pdf}{una
carta abierta} a los directores ejecutivos de grandes corporaciones en
donde los alentaba a prestar atención a preocupaciones sociales,
laborales y medioambientales.

``Para prosperar en el tiempo, cada compañía debe mostrar que hace una
contribución positiva a la sociedad, además de lograr desempeño
financiero'', escribió.

El año pasado, Fink firmó la
\href{https://www.nytimes.com/2019/08/19/business/business-roundtable-ceos-corporations.html}{Declaración
del Propósito de una Corporación}, creada por la Business Roundtable
---Mesa Redonda de Negocios---, una asociación conformada por directores
ejecutivos estadounidenses. Se proponía ``un compromiso fundamental con
todos nuestras partes interesadas''.

En enero Fink escribió otra
\href{https://www.blackrock.com/corporate/investor-relations/larry-fink-ceo-letter}{carta
a los directores ejecutivos}, en la que advertía que las empresas que no
atendieran el cambio climático sufrirían las consecuencias en el
mercado.

BlackRock ha lanzado fondos hechos para la llamada inversión de impacto,
en la que el dinero se emplea en apoyar metas sociales y ambientales.

Image

La directora gerente del FMI, Kristalina Georgieva, con el ministro de
Economía de Argentina, Martín Guzmán, en febrero. Ella ha apoyado la
propuesta de Argentina a sus acreedores.Credit...Remo Casilli/Reuters

Argentina ahora intenta detener un alarmante aumento de la pobreza. El
que alguna vez estuvo entre los países más ricos de la tierra, ha
incumplido su deuda pública nueve veces.

La historia de Argentina ha estado dominada por gobiernos populistas que
han ganado el apoyo popular distribuyendo subsidios y efectivo a las
masas en desatención descarada a la aritmética presupuestaria, lo que ha
resultado en inflación crónica y en crisis frecuentes.

El último gobierno, encabezado por el presidente Mauricio Macri, asumió
el poder en 2015 con el mandato de restaurar la disciplina para
recuperar la confianza de los mercados internacionales y, al mismo
tiempo, mostrar compasión hacia los pobres a través del gasto social.

Entre aquellos impresionados con la misión estaba Fink. Seis meses
después de que Macri juró el cargo, el ejecutivo de BlackRock
\href{https://www.youtube.com/watch?v=TM_MC2Fj-JI}{dijo} que su
administraciónn ``realmente ha mostrado lo que un gobierno puede lograr
si se enfoca en tratar de cambiar el futuro de su país''.

Al final, Macri adquirió reputación
\href{https://www.nytimes.com/es/2019/05/14/espanol/america-latina/argentina-economia-macri-kirchnerismo.html}{por
salir del paso}, volviendo a endeudarse sin lograr crecimiento.

El año pasado, con la llegada de un nuevo presidente, Alberto Fernández,
muchos supusieron que el populismo volvía. Pero Fernández rápidamente
aseguró al FMI y a los acreedores clave que era un pragmatista resuelto
a lograr un pago viable de la deuda.

El FMI ha sido acusado desde hace mucho de esgrimir un solo instrumento
contundente para el manejo de la crisis: la austeridad. Su paquete de
rescate hace dos décadas impuso recortes paralizantes a los programas
gubernamentales argentinos, lo que cosechó un resentimiento duradero.
Georgieva, la directora gerente del fondo, se ha enfocado en proteger a
los países de deudas impagables.

Image

Una protesta en Buenos Aires contra la crisis económica. La historia de
Argentina ha estado dominada por gobiernos populistas que han repartido
subsidios sin tener en cuenta la aritmética presupuestaria.Credit...Juan
Ignacio Roncoroni/EPA, vía Shutterstock

BlackRock integra un consorcio llamado Ad Hoc Argentine Bondholder
Group, que controla aproximadamente una cuarta parte de los bonos.

El grupo Ad Hoc ha presentado un frente unificado que rechaza la más
reciente oferta del gobierno, que pagaría unos 53 centavos por dólar del
valor de los bonos. La semana pasada el grupo presentó una propuesta en
busca de mejores condiciones: más de 56 centavos por dólar.

En una carta enviada el lunes al ministro de Economía de Argentina,
Martín Guzmán, el grupo dijo que contaba con el apoyo de la mayoría de
los tenedores de bonos, lo que le confería el poder de bloquear el
acuerdo. Bajo las cláusulas de los bonos, un acuerdo que disminuya su
valor debe contar con la aprobación de los tenedores de dos terceras
partes de su valor total.

En un comunicado, el grupo Ad Hoc dijo que operaba en interés del pueblo
argentino al buscar un acuerdo que permitiría ``acceder nuevamente a los
mercados de capital e incentivar más inversión''.

Pero algunos acreedores han apoyado públicamente la propuesta del
gobierno.

``Argentina ha presentado una oferta razonable, que creo que los
acreedores deben aceptar, especialmente a la luz de la situación de
salud y de pobreza del país'', dijo Mohamed A. El-Erian, asesor
económico jefe en Allianz SE, la empresa matriz de Pacific Investment
Management Company, una de las principales administradoras de bonos del
mundo. Ha estado actuando como asesor de uno de los acreedores en la
mesa de negociación, Gramercy Funds Management LLC, que se especializa
en mercados emergentes.

Gramercy ha llegado a la conclusión de que los desacuerdos entre la
oferta del gobierno y la propuesta de Ad Hoc son triviales en
comparación con el riesgo de una moratoria extensa que terminaría por
devaluar los bonos argentinos y posiblemente sometería a los acreedores
a años de litigio y agravaría la crisis del país.

Image

Poco después de convertirse en presidente de Argentina el año pasado,
Alberto Fernández rápidamente aseguró a los acreedores clave que sería
pragmático al buscar un arreglo de deuda viable.Credit...Esteban
Collazo, vía Agence France-Presse --- Getty Images

Un alivio adicional de la deuda también mejora las probabilidades de que
Argentina sea capaz de cumplir con pagos a futuro, lo que elevaría el
valor de los bonos pendientes de pago y disminuiría el costo de
endeudarse de las empresas argentinas.

``Por tres puntos estás dispuesto a perder 20 o 30'' dijo Humes, el
presidente de Greylock. ``Es una locura. Es desafortunado cuando los
egos y la inexperiencia entorpecen una solución pragmática''.

Algunos dicen que al gobierno se le pasó la mano al enfrentarse a los
acreedores con una oferta inicial excesivamente baja de menos de 40
centavos por dólar.

``Guzmán empezó con una oferta muy baja'', dijo Siobhan Morden, analista
de bonos latinoamericanos en Amherst Pierpoint Securities, una corredora
independiente. ``Ha sido una distracción innecesaria durante meses que
se pudo haber evitado si la oferta inicial hubiera sido más razonable''.

Las negociaciones, en las que participaron decenas de acreedores, se
llevaron a cabo por Zoom. Los representantes de BlackRock chocaron con
el ministro de Economía Guzmán, un economista de 37 años que estudió
bajo la tutela de Stiglitz en la Universidad de Columbia.

Image

Elementos del ejército argentino sirvieron un guiso para los residentes
de un barrio popular en Buenos Aires. Los comedores comunitarios
atienden a más personas en la pandemia.Credit...Juan Mabromata/Agence
France-Presse --- Getty Images

En mayo, Fink hizo un llamado para que Guzmán intentara acabar con el
impasse y sugirió un acuerdo si es que el gobierno elevaba su oferta al
rango de los 50 a 55 centavos por dólar, dijeron personas con
conocimiento de las conversaciones.

En consultas privadas con BlackRock, el gobierno ofreció 50 centavos.
Pero BlackRock y su grupo Ad Hoc querían más.

Fink se quejó de que parecía injusto que los acreedores privados
asumieran todas las pérdidas, y argumentó que el FMI debería condonar
algunos de sus préstamos, un imposible.

A principios de julio
\href{https://ara.reuters.com/article/businessNews/idARL1N2ED0XB}{Guzmán
mejoró las condiciones} y ofreció 53 centavos de dólar, lo que se
granjeó el apoyo de varios acreedores, entre ellos Gramercy y Greylock.

Para entonces, la pandemia estaba agravando la recesión de Argentina y
justo cuando el gobierno requería fondos adicionales para atender la
emergencia de salud pública. Pero BlackRock comenzó una campaña tras
bambalinas para bloquear el acuerdo.

El gobierno ha insistido en que se trata de su última oferta. Con una
pobreza infantil superior al 50 por ciento, los funcionarios dicen que
pagar más a los acreedores equivaldría a quitar riqueza a las personas
que casi no tienen nada para transferirla a los inversores
internacionales.

En una mañana reciente, unas 100 familias llegaron a un comedor
comunitario a unos 40 kilómetros al oeste de Buenos Aires, más del doble
de los que acudían en marzo. Entre ellos estaba Ángel Ariel Coronel, un
plomero que vive en los alrededores con su esposa y el hijo de dos años
de ambos. La cuarentena estricta ordenada por el gobierno había
paralizado los proyectos de construcción en los que trabajaba.

``A mi mujer le daba un poco de vergüenza que vengamos aquí'', dijo
Coronel mientras esperaba su porción de lentejas humeantes. ``Pero a mí
no me importa. Necesitamos la ayuda. No he trabajado un solo día desde
que empezó todo esto''.

Image

Buenos Aires ha estado en cuarentena estricta desde
marzo.Credit...Natacha Pisarenko/Associated Press

Peter S. Goodman reportó desde Londres y Daniel Politi desde Buenos
Aires.

Peter S. Goodman es corresponsal de economía europea con sede en
Londres. Fue corresponsal económico nacional en Nueva York. También
trabajó en The Washington Post como corresponsal en China y fue editor
global en jefe del International Business Times.
\href{https://twitter.com/petersgoodman}{@petersgoodman}

\begin{center}\rule{0.5\linewidth}{\linethickness}\end{center}

Advertisement

\protect\hyperlink{after-bottom}{Continue reading the main story}

\hypertarget{site-index}{%
\subsection{Site Index}\label{site-index}}

\hypertarget{site-information-navigation}{%
\subsection{Site Information
Navigation}\label{site-information-navigation}}

\begin{itemize}
\tightlist
\item
  \href{https://help.nytimes.com/hc/en-us/articles/115014792127-Copyright-notice}{©~2020~The
  New York Times Company}
\end{itemize}

\begin{itemize}
\tightlist
\item
  \href{https://www.nytco.com/}{NYTCo}
\item
  \href{https://help.nytimes.com/hc/en-us/articles/115015385887-Contact-Us}{Contact
  Us}
\item
  \href{https://www.nytco.com/careers/}{Work with us}
\item
  \href{https://nytmediakit.com/}{Advertise}
\item
  \href{http://www.tbrandstudio.com/}{T Brand Studio}
\item
  \href{https://www.nytimes.com/privacy/cookie-policy\#how-do-i-manage-trackers}{Your
  Ad Choices}
\item
  \href{https://www.nytimes.com/privacy}{Privacy}
\item
  \href{https://help.nytimes.com/hc/en-us/articles/115014893428-Terms-of-service}{Terms
  of Service}
\item
  \href{https://help.nytimes.com/hc/en-us/articles/115014893968-Terms-of-sale}{Terms
  of Sale}
\item
  \href{https://spiderbites.nytimes.com}{Site Map}
\item
  \href{https://help.nytimes.com/hc/en-us}{Help}
\item
  \href{https://www.nytimes.com/subscription?campaignId=37WXW}{Subscriptions}
\end{itemize}
