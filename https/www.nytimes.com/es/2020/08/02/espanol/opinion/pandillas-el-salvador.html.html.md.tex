Sections

SEARCH

\protect\hyperlink{site-content}{Skip to
content}\protect\hyperlink{site-index}{Skip to site index}

\href{https://www.nytimes.com/es/section/opinion}{Opinión}

\href{https://myaccount.nytimes.com/auth/login?response_type=cookie\&client_id=vi}{}

\href{https://www.nytimes.com/section/todayspaper}{Today's Paper}

\href{/es/section/opinion}{Opinión}\textbar{}En El Salvador todos han
negociado con las pandillas

\url{https://nyti.ms/2XnsQPw}

\begin{itemize}
\item
\item
\item
\item
\item
\end{itemize}

Advertisement

\protect\hyperlink{after-top}{Continue reading the main story}

\href{/es/section/opinion}{Opinión}

Supported by

\protect\hyperlink{after-sponsor}{Continue reading the main story}

Comentario

\hypertarget{en-el-salvador-todos-han-negociado-con-las-pandillas}{%
\section{En El Salvador todos han negociado con las
pandillas}\label{en-el-salvador-todos-han-negociado-con-las-pandillas}}

Dialogar con las pandillas ha sido una realidad en el país: políticos de
todos los colores lo han hecho por casi una década, pero solo unos
cuantos han sido perseguidos por hacerlo. ¿Es una utopía transparentar
esos pactos?

\includegraphics{https://static01.nyt.com/images/2020/08/01/multimedia/01Martinez-ES/merlin_175051002_8b6c73d6-78b1-44b6-9806-b42f2b7f3923-articleLarge.jpg?quality=75\&auto=webp\&disable=upscale}

Por Óscar Martínez

Es periodista salvadoreño.

\begin{itemize}
\item
  2 de agosto de 2020
\item
  \begin{itemize}
  \item
  \item
  \item
  \item
  \item
  \end{itemize}
\end{itemize}

\href{https://www.nytimes.com/newsletters/el-times}{Regístrate para
recibir nuestro boletín} con lo mejor de The New York Times.

\begin{center}\rule{0.5\linewidth}{\linethickness}\end{center}

SAN SALVADOR --- Quien haya seguido el acontecer político de El Salvador
en la última década sabe que existió algo que pasó a la posteridad bajo
una etiqueta: la tregua del gobierno con las pandillas.

Dialogar con las pandillas ha sido una realidad en El Salvador. Negocian
los políticos, negocia el operador de energía eléctrica que debe entrar
a reparar un poste en una colonia bajo dominio pandillero y negocia la
maestra de escuela que tiene alumnos de una y otra pandilla. Los
políticos, de todos los colores, llevan casi una década haciéndolo,
según ellos, en secreto. Ha sido gracias a investigaciones periodísticas
---de, entre otros medios, El Faro, donde trabajo--- que muchas treguas
han salido a la luz pública. Incluidas las de figuras cercanas al
presidente Nayib Bukele.

Los políticos han ocultado estas treguas por dos razones: la primera,
porque saben que no es popular de cara a los electores dialogar con
mareros. La segunda, porque no quieren que se sepa lo que negocian,
porque les avergüenza reconocer que la moneda de cambio en algunos casos
fueron votos. En lugar de intentar ocultarlas es hora de
transparentarlas. El diálogo tendría que ser desde arriba, de forma
franca y pública y, sobretodo, sin buscar beneficio político. Pero
estamos muy lejos de algo así.

Pese a que se conocen muchos acuerdos, se ha aplicado particular
severidad al arquitecto de la primera tregua conocida: David Munguía
Payés, exministro de Defensa y luego de Seguridad y Justicia en el
gobierno de Mauricio Funes. Munguía Payés fue capturado y acusado de
agrupaciones ilícitas y otros delitos relacionados con aquel proceso que
arrancó en 2012. La semana pasada fue presentado ante los medios con
esposas y mascarilla. A diferencia de otros políticos que han sido
señalados por negociar con pandillas, el exministro pasó varias noches
tras rejas hasta que una jueza le
\href{https://www.laprensagrafica.com/elsalvador/Imponen-10000-de-fianza-y-arresto-domiciliario-a-Payes-20200729-0084.html}{concedió
fianza de 10.000 dólares} y arresto domiciliario por ``cuestiones de
humanidad''.

En marzo de 2012, el gobierno del expresidente Funes, el primero de
izquierda desde los acuerdos de paz de 1992,
\href{https://www.elfaro.net/es/201203/noticias/7985/}{negoció un pacto}
con las cúpulas de la Mara Salvatrucha 13 y las dos facciones del Barrio
18: beneficios carcelarios para los criminales a cambio de menos
asesinatos en las calles. Los resultados fueron contundentes: de 4371
asesinatos en 2011 se pasó a 2594 en 2012; y 2513 en 2013.

Pero esa tregua no fue la única. Y no todas fueron para disminuir la
violencia. Ha habido treguas posteriores para conseguir beneficios
electorales.

``Todos los partidos han buscado acercamiento con nosotros'',
\href{https://elfaro.net/es/201604/salanegra/18347/\%E2\%80\%9CTodos-los-partidos-han-buscado-acercamientos-con-nosotros\%E2\%80\%9D.htm}{dijo}
en abril de 2016 un líder pandillero de una facción del Barrio 18 que
negoció con varios políticos. Su frase se convierte en un mantra, no
pierde vigencia con los años.

Entonces, ¿por qué solo el creador de la tregua de 2012 y algunos de sus
colaboradores han sido capturados en los últimos años? Creo que la
respuesta está en que solo se persigue a quienes ya no están vinculados
a un partido político. Hay algunos políticos que han sido investigados
por lo mismo pero aún tienen el respaldo de partidos y hasta competirán
en las elecciones legislativas y municipales de 2021.

Fui parte del equipo de periodistas que
\href{http://www.salanegra.elfaro.net/es/201209/cronicas/9612/La-nueva-verdad-sobre-la-Tregua-entre-pandillas.htm}{descubrió
la tregua de 2012} apenas días después de que los líderes pandilleros
habían sido trasladados a prisiones de régimen normal desde donde
retomaron el control pleno de sus estructuras. Los organizadores de la
tregua no supieron cómo responder cuando publicamos el reportaje. Se
justificaron con mentiras absurdas, como que los traslados se debían a
peticiones de la Iglesia Católica.

No tenían un plan de comunicación, pero sí un objetivo: menos muertos.

El Frente Farabundo Martí para la Liberación Nacional (FMLN), el partido
que promovió aquella primera tregua, fue el mismo que la enterró en
2014. Entonces, todo parecía volver a la normalidad: otro gobierno
buscaría derrotar a las pandillas que
\href{https://www.nytimes.com/es/2018/02/18/espanol/opinion/opinion-martinez-mara-ms-13-trump.html}{llegaron
desde California} en los años noventa con balas y más balas. Pero luego
se supo que no fue del todo así. Salieron a la luz videos que
documentaban que, mientras el FMLN vendía un mensaje de odio contra las
pandillas,
\href{https://elfaro.net/es/206005/salanegra/18560/El-FMLN-hizo-alianza-con-las-pandillas-para-la-elecci\%C3\%B3n-presidencial-de-2014.htm}{también
negociaba con los líderes criminales} en las calles de cara a las
elecciones presidenciales.

A las diferentes reuniones con pandilleros asistieron efemelenistas que
no solo ofrecieron su ayuda a los pandilleros para que pudieran obtener
su documento único de identidad, sino que propusieron un
\href{https://elfaro.net/es/201610/salanegra/19473/FMLN-ofreci\%C3\%B3-a-las-pandillas-un-programa-de-cr\%C3\%A9ditos-de-10-millones-de-d\%C3\%B3lares.htm}{programa
de crédito} de 10 millones de dólares que sería administrado por los
criminales. En esta nueva tregua, figuras como los exministros Arístides
Valencia y Benito Lara no negociaban vidas, sino apoyo electoral. A
diferencia de Payés, nunca hubo orden de captura en su contra, y
enfrentan sus procesos judiciales en libertad.

En otra negociación, ocurrida en 2014, Ernesto Muyshondt, entonces
vicepresidente del derechista partido Arena, aparece en un
\href{https://elfaro.net/es/201603/video/18213/Arena-prometi\%C3\%B3-a-las-pandillas-una-nueva-tregua-si-ganaba-la-presidencia.htm}{video
filmado} por los mismos pandilleros. Ofrece una nueva tregua si su
candidato presidencial gana. Muyshondt enfrenta su proceso judicial en
libertad, es el alcalde de la capital y el candidato para repetir en el
cargo en las elecciones de 2021.

Y el recuento no ha terminado. Allegados al actual presidente de El
Salvador, Nayib Bukele, quien llama a sus adversarios ``los mismos de
siempre'', también negociaron con pandillas.

Investigaciones de El Faro prueban que cuando Bukele fue alcalde de San
Salvador, entre 2015 y 2018, algunos de sus funcionarios
\href{https://elfaro.net/es/201806/el_salvador/22148/Nayib-Bukele-tambi\%C3\%A9n-pact\%C3\%B3-con-pandillas.htm}{negociaron
con pandillas} para poder construir el Mercado Cuscatlán, una de sus
obras de infraestructura emblemáticas. Según nuestro trabajo
periodístico, cuando Bukele competía por la alcaldía en 2015, sus
delegados entregaron dinero a las pandillas para que no boicotearan su
candidatura, aseguraron fuentes pandilleras y funcionarios de su
gobierno municipal. Recientemente,
\href{https://elfaro.net/es/202007/el_salvador/24612/Nueva-informaci\%C3\%B3n-de-la-reuni\%C3\%B3n-entre-Mario-Dur\%C3\%A1n-y-Renuente-de-la-MS-13.htm}{se
publicaron} fotografías de una reunión de 2015, donde aparece el actual
ministro de Gobernación de Bukele, Mario Durán, negociando con Renuente,
un líder de la MS-13. Durán no está acusado de nada y es el candidato de
Nuevas Ideas, el partido creado bajo el ala de Bukele, para ser alcalde
de la capital en 2021.

Los principales políticos salvadoreños lo han dejado claro en la última
década: creen que es necesario negociar con los pandilleros, pero están
convencidos de que deben hacerlo en secreto. Un diálogo que debería ser
abierto, escuchando a la sociedad civil que padece a esos criminales,
parece una utopía. Sin embargo, los salvadoreños deberíamos exigirlo. Si
no, seguiremos con lo que tenemos: pactos ocultos, captura de políticos
huérfanos del poder e impunidad para los que aún mantienen sus vínculos
partidarios.

A estas alturas es muy complicado corregir el rumbo, pero es posible
crear un diálogo con interlocutores nacionales e internacionales que den
confianza al proceso. Es posible negociar la desarticulación de las
pandillas, la reinserción de sus miembros y la reparación de las
víctimas. Los políticos quieren convencernos de que no hay manera porque
prefieren seguir usando a las pandillas como una herramienta electoral.

Óscar Martínez es editor de investigaciones especiales de El Faro, autor
de \emph{Los migrantes que no importan} y \emph{Una historia de
violencia} y coautor de \emph{El Niño de Hollywood}, sobre la MS-13.

Advertisement

\protect\hyperlink{after-bottom}{Continue reading the main story}

\hypertarget{site-index}{%
\subsection{Site Index}\label{site-index}}

\hypertarget{site-information-navigation}{%
\subsection{Site Information
Navigation}\label{site-information-navigation}}

\begin{itemize}
\tightlist
\item
  \href{https://help.nytimes.com/hc/en-us/articles/115014792127-Copyright-notice}{©~2020~The
  New York Times Company}
\end{itemize}

\begin{itemize}
\tightlist
\item
  \href{https://www.nytco.com/}{NYTCo}
\item
  \href{https://help.nytimes.com/hc/en-us/articles/115015385887-Contact-Us}{Contact
  Us}
\item
  \href{https://www.nytco.com/careers/}{Work with us}
\item
  \href{https://nytmediakit.com/}{Advertise}
\item
  \href{http://www.tbrandstudio.com/}{T Brand Studio}
\item
  \href{https://www.nytimes.com/privacy/cookie-policy\#how-do-i-manage-trackers}{Your
  Ad Choices}
\item
  \href{https://www.nytimes.com/privacy}{Privacy}
\item
  \href{https://help.nytimes.com/hc/en-us/articles/115014893428-Terms-of-service}{Terms
  of Service}
\item
  \href{https://help.nytimes.com/hc/en-us/articles/115014893968-Terms-of-sale}{Terms
  of Sale}
\item
  \href{https://spiderbites.nytimes.com}{Site Map}
\item
  \href{https://help.nytimes.com/hc/en-us}{Help}
\item
  \href{https://www.nytimes.com/subscription?campaignId=37WXW}{Subscriptions}
\end{itemize}
