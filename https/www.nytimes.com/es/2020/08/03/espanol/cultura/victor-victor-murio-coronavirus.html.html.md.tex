Sections

SEARCH

\protect\hyperlink{site-content}{Skip to
content}\protect\hyperlink{site-index}{Skip to site index}

\href{https://www.nytimes.com/es/section/cultura}{Cultura}

\href{https://myaccount.nytimes.com/auth/login?response_type=cookie\&client_id=vi}{}

\href{https://www.nytimes.com/section/todayspaper}{Today's Paper}

\href{/es/section/cultura}{Cultura}\textbar{}Víctor Víctor, conocido por
el éxito `Mesita de Noche', muere a los 71 años

\url{https://nyti.ms/3k8PpB2}

\begin{itemize}
\item
\item
\item
\item
\item
\end{itemize}

\href{https://www.nytimes.com/es/spotlight/coronavirus?action=click\&pgtype=Article\&state=default\&region=TOP_BANNER\&context=storylines_menu}{El
brote de coronavirus}

\begin{itemize}
\tightlist
\item
  \href{https://www.nytimes.com/es/interactive/2020/espanol/mundo/coronavirus-en-estados-unidos.html?action=click\&pgtype=Article\&state=default\&region=TOP_BANNER\&context=storylines_menu}{Mapa
  y casos en EE. UU.}
\item
  \href{https://www.nytimes.com/es/2020/07/23/espanol/america-latina/bolivia-cloro-coronavirus-ivermectina.html?action=click\&pgtype=Article\&state=default\&region=TOP_BANNER\&context=storylines_menu}{Dióxido
  de cloro, ivermectina y más: ¿funcionan?}
\item
  \href{https://www.nytimes.com/es/interactive/2020/science/coronavirus-tratamientos-curas.html?action=click\&pgtype=Article\&state=default\&region=TOP_BANNER\&context=storylines_menu}{Fármacos
  y tratamientos}
\item
  \href{https://www.nytimes.com/es/2020/07/28/espanol/ciencia-y-tecnologia/anticuerpos-coronavirus-inmunidad.html?action=click\&pgtype=Article\&state=default\&region=TOP_BANNER\&context=storylines_menu}{Anticuerpos
  e inmunidad}
\item
  \href{https://www.nytimes.com/es/2020/04/29/espanol/estilos-de-vida/oximetro-para-que-sirve.html?action=click\&pgtype=Article\&state=default\&region=TOP_BANNER\&context=storylines_menu}{Oxímetros}
\end{itemize}

Advertisement

\protect\hyperlink{after-top}{Continue reading the main story}

Supported by

\protect\hyperlink{after-sponsor}{Continue reading the main story}

Los que hemos perdido

\hypertarget{vuxedctor-vuxedctor-conocido-por-el-uxe9xito-mesita-de-noche-muere-a-los-71-auxf1os}{%
\section{Víctor Víctor, conocido por el éxito `Mesita de Noche', muere a
los 71
años}\label{vuxedctor-vuxedctor-conocido-por-el-uxe9xito-mesita-de-noche-muere-a-los-71-auxf1os}}

El cantante, compositor y productor también llevó clases de teatro,
música y danza a comunidades desfavorecidas. Murió de COVID-19.

\includegraphics{https://static01.nyt.com/images/2020/07/27/obituaries/03Victor-ES/merlin_175001436_38b11f8e-227a-4e2c-9821-7618af9b2524-articleLarge.jpg?quality=75\&auto=webp\&disable=upscale}

\href{https://www.nytimes.com/by/sandra-e-garcia}{\includegraphics{https://static01.nyt.com/images/2020/07/10/reader-center/author-sandra-e-garcia/author-sandra-e-garcia-thumbLarge.png}}

Por \href{https://www.nytimes.com/by/sandra-e-garcia}{Sandra E. Garcia}

\begin{itemize}
\item
  3 de agosto de 2020
\item
  \begin{itemize}
  \item
  \item
  \item
  \item
  \item
  \end{itemize}
\end{itemize}

\href{https://www.nytimes.com/2020/08/01/obituaries/victor-victor-dead-coronavirus.html}{Read
in English}

\href{https://www.nytimes.com/newsletters/el-times}{Regístrate para
recibir nuestro boletín} con lo mejor de The New York Times.

\begin{center}\rule{0.5\linewidth}{\linethickness}\end{center}

\emph{Este obituario es parte de una serie sobre personas que han muerto
en la pandemia de coronavirus. Lee sobre otras}
\href{https://www.nytimes.com/interactive/2020/obituaries/people-died-coronavirus-obituaries.html}{\emph{aquí}}\emph{.}

Cuando era estudiante universitario en República Dominicana, Víctor Jose
Víctor planeaba convertirse en psicólogo, pero luego descubrió la
música.

Comenzó a interpretar sus propias canciones y pronto se hizo famoso por
ellas. En 1975 fue uno de los primeros artistas dominicanos en visitar
Cuba después de que el gobierno dominicano prohibió los viajes a ese
país.

También comenzó a escribir y producir canciones para pioneros
dominicanos de la música como Wilfrido Vargas y Juan Luis Guerra.

Su canción de 1991, ``Mesita de Noche'', fue la más conocida de todas y
lo convirtió en superestrella. A lo largo de varias giras recorrió
España y otros países hispanohablantes, así como Estados Unidos.

Víctor murió el 16 de julio de COVID-19, según dijo su hijo Ian, en un
hospital de Santo Domingo, capital de República Dominicana. Tenía 71
años.

Lo habían llevado al hospital dos semanas después de visitar un estudio
para trabajar en un especial de televisión del canal gubernamental Radio
y Televisión Dominicana a beneficio de músicos que se habían quedado sin
trabajo a causa de la pandemia, dijo su hijo en una entrevista.

Víctor Jose Víctor Rojas nació el 11 de diciembre de 1948, en Santiago
de Los Caballeros, una ciudad al norte de República Dominicana. Su
madre, Avelina Rojas de Víctor, era ama de casa; su padre, José Víctor
Arias, era dueño de una farmacia.

Estudió medicina en la Universidad Pedro Henríquez Ureña, pero se cambió
a la Universidad Autónoma de Santo Domingo para estudiar psicología,
según su prima, Panchy Cantisano. Cantisano relata que nunca se graduó
porque decidió ser músico.

Su primera canción, ``La Casita'', se lanzó en 1972.

Poco después, se unió a Vargas, su vecino en Santiago de los Caballeros
y merenguero vanguardista, y su banda, Los Beduinos. Víctor también
participó en el movimiento de oposición contra el
\href{https://www.nytimes.com/1959/07/06/archives/dominican-dictator-rafael-leonidas-trujillo-molina.html}{dictador
dominicano Rafael Trujillo} y comenzó a escribir canciones de protesta.
En su juventud, viajó por el país como integrante de las caravanas de
jóvenes que protestaban contra el gobierno; al mismo tiempo, aprendió
ritmos locales.

En 1987, comenzó a producir discos para muchos músicos dominicanos, como
Guerra, Los Hermanos Rosario y Sergio Vargas. Su hijo explica que
convertirse en productor fue la evolución natural luego de ser letrista.
Así fue como conoció a buena parte del talento nacional.

``Vivió una doble vida'', dijo Ian Víctor. ``Escribía canciones
románticas y era artista, pero también era parte del movimiento político
clandestino'', agregó.

También tuvo una tercera vida musical: escribía canciones para agencias
de publicidad. Trabajó en las campañas de empresas como Brugal y
Barceló, las principales marcas de ron dominicanas.

Durante los últimos cuatro años Víctor dirigió programas para dar clases
gratuitas de teatro, danza y música en barrios pobres.

``Ya contaban con músicos que habían alcanzado un nivel profesional'',
dijo Ian Víctor.

Además de su hijo, a Víctor le sobreviven su esposa, Zobeyda Ferreras de
Víctor; una hija, Amy; cuatro nietos; un hermano, Jorge Alfredo y una
hermana, Vilma.

``Mi padre fue un hombre generoso y feliz'', dijo Ian Víctor. ``Murió
ayudando a otros y me siento muy orgulloso de eso'', concluyó.

\begin{center}\rule{0.5\linewidth}{\linethickness}\end{center}

Advertisement

\protect\hyperlink{after-bottom}{Continue reading the main story}

\hypertarget{site-index}{%
\subsection{Site Index}\label{site-index}}

\hypertarget{site-information-navigation}{%
\subsection{Site Information
Navigation}\label{site-information-navigation}}

\begin{itemize}
\tightlist
\item
  \href{https://help.nytimes.com/hc/en-us/articles/115014792127-Copyright-notice}{©~2020~The
  New York Times Company}
\end{itemize}

\begin{itemize}
\tightlist
\item
  \href{https://www.nytco.com/}{NYTCo}
\item
  \href{https://help.nytimes.com/hc/en-us/articles/115015385887-Contact-Us}{Contact
  Us}
\item
  \href{https://www.nytco.com/careers/}{Work with us}
\item
  \href{https://nytmediakit.com/}{Advertise}
\item
  \href{http://www.tbrandstudio.com/}{T Brand Studio}
\item
  \href{https://www.nytimes.com/privacy/cookie-policy\#how-do-i-manage-trackers}{Your
  Ad Choices}
\item
  \href{https://www.nytimes.com/privacy}{Privacy}
\item
  \href{https://help.nytimes.com/hc/en-us/articles/115014893428-Terms-of-service}{Terms
  of Service}
\item
  \href{https://help.nytimes.com/hc/en-us/articles/115014893968-Terms-of-sale}{Terms
  of Sale}
\item
  \href{https://spiderbites.nytimes.com}{Site Map}
\item
  \href{https://help.nytimes.com/hc/en-us}{Help}
\item
  \href{https://www.nytimes.com/subscription?campaignId=37WXW}{Subscriptions}
\end{itemize}
