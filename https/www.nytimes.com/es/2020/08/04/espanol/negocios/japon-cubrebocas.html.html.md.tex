\href{/es/section/negocios}{Negocios}\textbar{}Llevaremos cubrebocas por
algún tiempo. ¿Por qué no hacerlos bonitos?

\url{https://nyti.ms/31kwKcU}

\begin{itemize}
\item
\item
\item
\item
\item
\item
\end{itemize}

\href{https://www.nytimes.com/es/spotlight/coronavirus?action=click\&pgtype=Article\&state=default\&region=TOP_BANNER\&context=storylines_menu}{El
brote de coronavirus}

\begin{itemize}
\tightlist
\item
  \href{https://www.nytimes.com/es/interactive/2020/espanol/america-latina/coronavirus-en-mexico.html?action=click\&pgtype=Article\&state=default\&region=TOP_BANNER\&context=storylines_menu}{Mapa
  y casos en México}
\item
  \href{https://www.nytimes.com/es/2020/07/31/espanol/ciencia-y-tecnologia/ninos-contagio-coronavirus.html?action=click\&pgtype=Article\&state=default\&region=TOP_BANNER\&context=storylines_menu}{Los
  niños y el virus}
\item
  \href{https://www.nytimes.com/es/interactive/2020/science/coronavirus-tratamientos-curas.html?action=click\&pgtype=Article\&state=default\&region=TOP_BANNER\&context=storylines_menu}{Fármacos
  y tratamientos}
\item
  \href{https://www.nytimes.com/es/2020/07/06/espanol/ciencia-y-tecnologia/coronavirus-transmision-aire.html?action=click\&pgtype=Article\&state=default\&region=TOP_BANNER\&context=storylines_menu}{Cómo
  se transmite el coronavirus}
\item
  \href{https://www.nytimes.com/es/2020/07/14/espanol/estilos-de-vida/botiquin-medicina-coronavirus.html?action=click\&pgtype=Article\&state=default\&region=TOP_BANNER\&context=storylines_menu}{Prepara
  tu botiquín}
\end{itemize}

\includegraphics{https://static01.nyt.com/images/2020/07/28/business/03Mascarillas-ES-01/00virus-japan-masks-1-articleLarge.jpg?quality=75\&auto=webp\&disable=upscale}

Sections

\protect\hyperlink{site-content}{Skip to
content}\protect\hyperlink{site-index}{Skip to site index}

\hypertarget{llevaremos-cubrebocas-por-alguxfan-tiempo-por-quuxe9-no-hacerlos-bonitos}{%
\section{Llevaremos cubrebocas por algún tiempo. ¿Por qué no hacerlos
bonitos?}\label{llevaremos-cubrebocas-por-alguxfan-tiempo-por-quuxe9-no-hacerlos-bonitos}}

Con purificadores motorizados, desinfectantes, tejidos transpirables y
estampados chic. Los protectores faciales han dejado de ser solo
productos descartables y baratos y ahora los hay para clientes más
exigentes.

Mascarilla de Donut Robotics, que funciona como una mezcla de radio de
comunicación, asistente personal y traductor.Credit...Noriko Hayashi
para The New York Times

Supported by

\protect\hyperlink{after-sponsor}{Continue reading the main story}

Por \href{https://www.nytimes.com/by/ben-dooley}{Ben Dooley} y Hisako
Ueno

\begin{itemize}
\item
  4 de agosto de 2020
\item
  \begin{itemize}
  \item
  \item
  \item
  \item
  \item
  \item
  \end{itemize}
\end{itemize}

\href{https://www.nytimes.com/2020/07/27/business/fashion-masks-coronavirus.html}{Read
in English}

\href{https://www.nytimes.com/newsletters/el-times}{Regístrate para
recibir nuestro boletín} con lo mejor de The New York Times.

\begin{center}\rule{0.5\linewidth}{\linethickness}\end{center}

TOKIO -- Rieko Kawanishi es la primera en reconocer que el cubrebocas
repleto de perlas que diseñó no es el más efectivo contra el
coronavirus. ``Está lleno de agujeros'', dijo riendo.

No obstante, su protector facial hecho a mano, que recomienda utilizar
sobre un cubrebocas normal, refleja una explosión súbita de atención
creativa en el mundo de la moda y la tecnología hacia un producto simple
que no ha sido alterado desde hace décadas.

``Después de la pandemia, hubo muchísimos lugares en los que, por
primera vez, tenías que usar cubrebocas forzosamente'', dijo Kawanishi,
una \href{https://www.instagram.com/sprbyspecialr/?hl=en}{diseñadora de
joyas} que vive en Tokio. ``Pensé: `Quiero diseñar algo elegante'''.

A medida que el virus continúa propagándose y las normas respecto al uso
de cubrebocas se endurecen en muchos lugares del mundo, los consumidores
comienzan a exigir una mayor variedad en las mascarillas que
resguardarán su aliento en lugares públicos durante el futuro próximo.

En respuesta, muchas empresas y diseñadores han inundado el mercado con
alternativas al cubrebocas quirúrgico desechable que animó a Kawanishi a
poner manos a la obra.

\includegraphics{https://static01.nyt.com/images/2020/07/23/business/03Mascarillas-ES-02/merlin_174826083_e80a398d-e9bc-4bb7-8fe5-be9c6a07b45d-articleLarge.jpg?quality=75\&auto=webp\&disable=upscale}

Los inventores han imaginado cubrebocas con
\href{https://prtimes.jp/main/html/rd/p/000000559.000032456.html}{purificadores
de aire motorizados}, bocinas con bluetooth e incluso desinfectantes que
matan bacterias calentando el protector facial (pero sin calentar el
rostro, esperemos) hasta a 93 grados Celsius. En Corea del Sur, el
gigante de la electrónica LG
\href{https://asia.nikkei.com/Business/Electronics/LG-debuts-fan-powered-mask-to-ease-breathing}{ha
creado un cubrebocas} con ventiladores que facilitan la respiración.

Las tiendas exhiben en maniquíes cubrebocas estampados a juego con ropa
de diseñador. Un empresario indio afirmó que pagó 4000 dólares por un
\href{https://www.scmp.com/news/asia/south-asia/article/3091866/coronavirus-bling-bling-indian-businessman-wears-us4000-custom}{cubrebocas
fabricado a la medida con oro}. Y una
\href{https://www.instagram.com/sophiecochevelou/}{diseñadora francesa
de vestuario} llenó su cuenta de Instagram con diseños fantasmagóricos
que tienen ilustraciones que van desde pterodáctilos hasta piernas de
muñecas.

El coronavirus ``ha provocado una rápida evolución en la tecnología de
los cubrebocas'', afirmó Yukiko Iida, experta en mascarillas del Centro
de Control Medioambiental, una consultora en Tokio.

``Cuando hay demanda, el mercado reacciona con rapidez'', afirmó. ``Las
personas los usan todo el día, todos los días, así que vemos mejoras en
aspectos como facilidad de uso y comunicación'', dijo, y mencionó
\href{https://ellebabe.com/products/hg1008-elbb1}{un cubrebocas} con la
parte frontal transparente que permite ver las expresiones faciales de
las personas.

Image

Credit...Noriko Hayashi para The New York Times

La necesidad de innovar ha sido considerable en Japón, donde los
cubrebocas ya se usaban de forma generalizada desde antes de la pandemia
para resguardar el rostro del frío o del polen, de la influenza o de la
mirada molesta de los extraños.

Aunque la mayoría de las personas sigue utilizando cubrebocas
quirúrgicos blancos y económicos, los consumidores han comenzado a dejar
de considerar a los protectores faciales como un artículo de un solo
uso, o algo que adquieres en una tienda de conveniencia para luego usar
pocas veces y tirar a la basura.

Taisuke Ono, director ejecutivo de la empresa emergente Donut Robotics,
comentó que él imaginaba un mundo en el que las personas usarían
cubrebocas en viajes al extranjero durante los próximos diez años o más.
Señaló que, si eso sucede, los consumidores exigirán que sus cubrebocas
hagan algo más que protegerlos de los virus.

Su empresa está fabricando un cubrebocas que funciona como una mezcla de
radio de comunicación, asistente personal y traductor. Puede grabar la
voz del usuario y proyectarla al teléfono inteligente de alguien más
(mucho mejor para mantener la distancia social) o traducirla del japonés
a varios idiomas.

Image

El director ejecutivo de Donut Robotics, Taisuke Ono, a la derecha, con
un colega en sus oficinasCredit...Noriko Hayashi para The New York Times

``La pandemia hizo que esto fuera posible'', dijo, y recalcó que su
prototipo había atraído la atención de los medios de comunicación y un
gran interés de inversionistas en Makuake, una versión japonesa de la
recaudadora Kickstarter. Antes, narró, ``aunque hicieras algo como esto,
nadie invertía en ello y no podías venderlo. Ahora, el mercado global ha
crecido exponencialmente''.

Aunque la pandemia terminará en algún momento, agregó, ``las personas
seguirán usando cubrebocas porque están temerosas''.

A pesar de que no está claro cómo reaccionarán los consumidores a
algunos de estos cubrebocas más ambiciosos, una innovación ha sido un
éxito evidente: los protectores faciales con telas de alta tecnología
que supuestamente ofrecen mayor comodidad o protección.

A medida que las temperaturas veraniegas se elevan, aumenta la demanda
de los cubrebocas hechos con materiales que mantengan a los usuarios
frescos. A las personas que han estado usando cubrebocas reutilizables
de tela (como los que envía el gobierno japonés a todos los hogares del
país) les cuesta soportar el calor y la humedad del verano en el centro
de Japón, ni qué decir de Singapur o Hong Kong.

Image

El cubrebocas C-mask de Donut Robotics y su aplicación para teléfono
móvilCredit...Noriko Hayashi para The New York Times

Toyoshima, una empresa comercializadora con sede en Nagoya,
\href{https://www.makuake.com/project/toyoshima-mask/}{comenzó a reunir
fondos} a mediados de abril para fabricar un cubrebocas nuevo de grado
militar con nailon. Recaudó 1,2 millones de dólares, más del 13.000 por
ciento de su meta.

Los clientes le dijeron a la empresa que querían un cubrebocas altamente
eficaz que además estuviera a la moda, afirmó Koki Yamagata, quien
dirige las iniciativas de financiamiento colectivo de la empresa.

``Muchas personas dijeron que querían que hubiera más colores'', señaló
Yamagata en una llamada por Zoom, mientras modelaba un cubrebocas blanco
que se vende en 50 dólares, aproximadamente. Los productos no han
generado muchas ganancias, sostuvo, y añadió que la empresa comenzó a
fabricarlos en parte por un sentido de responsabilidad social.

Otras compañías japonesas han seguido su ejemplo. Tadashi Yanai,
fundador de Uniqlo, la gran empresa minorista de ropa, insistió en que
su empresa no vendería cubrebocas, pero cambió de opinión después de que
los clientes clamaron por un producto hecho con la tela de alto
rendimiento y secado rápido de la marca.

Los cubrebocas se agotaron de inmediato y la compañía se ha comprometido
a fabricar 500.000 paquetes por semana, según un portavoz, Aldo Liguori,
quien dijo que la compañía ahora también planeaba venderlos en el
extranjero.

Image

Cubrebocas a la venta en una calle comercial de HarajukuCredit...Noriko
Hayashi para The New York Times

Para algunos fabricantes de indumentaria, producir cubrebocas ha sido
una necesidad, ya que las ventas al menudeo se redujeron
considerablemente debido a que los consumidores se quedan en casa.

Muchas fábricas ``han tenido poca actividad durante dos o tres meses,
así que piensan: `¿Por qué no hacemos cubrebocas de tela?''', dijo
Kensuke Kojima, consultor de productos para la industria de la moda.

Estos productores japoneses han entrado a un mercado cuyos cambios a lo
largo de las décadas han sido mínimos, como los cubrebocas de colores
diferentes o los que ofrecen revestimientos para proteger el maquillaje
y que no se corra.

Aunque los profesionales de la salud han utilizado cubrebocas de varios
tipos desde hace cientos, si no es que miles de años, los que usan
actualmente se desarrollaron por primera vez a finales del siglo XIX
para su uso en cirugías.

Se usaron para combatir epidemias por primera vez a principios del siglo
XX, cuando Wu Lien-teh, médico de ascendencia china,
\href{https://www.ncbi.nlm.nih.gov/pmc/articles/PMC4291938/}{comenzó a
promover} los cubrebocas de gasa simple como un método efectivo para
combatir un brote de peste neumónica en una zona del noreste de China
conocida como Manchuria.

Image

Cubrebocas con estampados coloridos en una tienda en
TokioCredit...Noriko Hayashi para The New York Times

Cuando llegó la gripe española en 1918, la práctica se volvió global por
primera vez.

Si bien los cubrebocas pronto cayeron en desgracia en la mayoría de los
países, el gobierno japonés continuó alentando su uso para combatir
enfermedades comunes como la gripe, dijo Christos Lynteris, antropólogo
médico de la Universidad de St. Andrews en Escocia.

La ubicuidad de las mascarillas quirúrgicas en Japón, que generalmente
están hechas de materiales sintéticos no tejidos, ha aumentado y
disminuido a lo largo de los años a medida que el país se enfrentaba a
diferentes problemas y crisis de salud.

En la década de 1990, se convirtieron en una defensa popular contra las
nubes de polen estacional creadas por los árboles de rápido crecimiento,
como el ciprés, plantado en todo el país para proporcionar una fuente de
madera barata.

En 2011, después del colapso nuclear en Fukushima, las existencias de
cubrebocas se agotaron ya que los consumidores temían las consecuencias
radiactivas. Y en los años siguientes, los aumentos drásticos de la
contaminación de China impulsaron una mayor demanda, particularmente en
el invierno.

Pero, incluso en Japón, se necesitó una pandemia para impulsar las
ventas de cubrebocas a la estratósfera; eran tan escasos al principio
que la gente hacía filas al amanecer para comprar una caja.

Meses más tarde, los cubrebocas abundan, y las tiendas en Harajuku, la
meca de la moda juvenil, las exhiben cada vez más. En la calle
Takeshita, los escaparates están llenos de cubrebocas que van desde los
modelos juguetones (con caras de animales de felpa) hasta los inspirados
en el punk (correas de cuero con estoperoles e imperdibles).

Image

Hisako Kanzaki con un cubrebocas nupcial que diseñó, dentro de su tienda
en el barrio~ Shimokitazawa de Tokio.Credit...Noriko Hayashi para The
New York Times

A pesar de que los cubrebocas de tela están a la moda, los clientes
deben tener cuidado, dijo Kazunari Onishi, experto en enfermedades
infecciosas de la Escuela de Posgrado de Salud Pública de la Universidad
Internacional de San Lucas en Tokio.

``Debes elegir un cubrebocas que cumpla con las normas
internacionales'', señaló, y agregó que ``otros tipos de mascarillas no
tienen el objetivo de prevenir contagios''.

``Si tu prioridad es evitar contagios de forma confiable, estos
cubrebocas no protegerán tu vida'', dijo, además de que, aunque uses
cubrebocas ``debes mantener la distancia social''.

Ben Dooley reporta sobre los negocios y la economía de Japón, con un
interés especial en temas sociales y el encuentro entre los negocios y
la política. \href{https://twitter.com/benjamindooley}{@benjamindooley}

\begin{center}\rule{0.5\linewidth}{\linethickness}\end{center}

Advertisement

\protect\hyperlink{after-bottom}{Continue reading the main story}

\hypertarget{site-index}{%
\subsection{Site Index}\label{site-index}}

\hypertarget{site-information-navigation}{%
\subsection{Site Information
Navigation}\label{site-information-navigation}}

\begin{itemize}
\tightlist
\item
  \href{https://help.nytimes.com/hc/en-us/articles/115014792127-Copyright-notice}{©~2020~The
  New York Times Company}
\end{itemize}

\begin{itemize}
\tightlist
\item
  \href{https://www.nytco.com/}{NYTCo}
\item
  \href{https://help.nytimes.com/hc/en-us/articles/115015385887-Contact-Us}{Contact
  Us}
\item
  \href{https://www.nytco.com/careers/}{Work with us}
\item
  \href{https://nytmediakit.com/}{Advertise}
\item
  \href{http://www.tbrandstudio.com/}{T Brand Studio}
\item
  \href{https://www.nytimes.com/privacy/cookie-policy\#how-do-i-manage-trackers}{Your
  Ad Choices}
\item
  \href{https://www.nytimes.com/privacy}{Privacy}
\item
  \href{https://help.nytimes.com/hc/en-us/articles/115014893428-Terms-of-service}{Terms
  of Service}
\item
  \href{https://help.nytimes.com/hc/en-us/articles/115014893968-Terms-of-sale}{Terms
  of Sale}
\item
  \href{https://spiderbites.nytimes.com}{Site Map}
\item
  \href{https://help.nytimes.com/hc/en-us}{Help}
\item
  \href{https://www.nytimes.com/subscription?campaignId=37WXW}{Subscriptions}
\end{itemize}
