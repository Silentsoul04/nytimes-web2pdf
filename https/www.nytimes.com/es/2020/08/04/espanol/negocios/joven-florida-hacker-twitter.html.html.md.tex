\href{/es/section/negocios}{Negocios}\textbar{}De jugador de Minecraft a
ciberpirata: la oscura carrera virtual de un chico de Florida

\url{https://nyti.ms/31nwmdE}

\begin{itemize}
\item
\item
\item
\item
\item
\end{itemize}

\includegraphics{https://static01.nyt.com/images/2020/08/02/business/04TwitterTeen-ES-00/02twitterteen-clark-copy-articleLarge.jpg?quality=75\&auto=webp\&disable=upscale}

Sections

\protect\hyperlink{site-content}{Skip to
content}\protect\hyperlink{site-index}{Skip to site index}

\hypertarget{de-jugador-de-minecraft-a-ciberpirata-la-oscura-carrera-virtual-de-un-chico-de-florida}{%
\section{De jugador de Minecraft a ciberpirata: la oscura carrera
virtual de un chico de
Florida}\label{de-jugador-de-minecraft-a-ciberpirata-la-oscura-carrera-virtual-de-un-chico-de-florida}}

El cerebro adolescente detrás del reciente ciberataque a Twitter tuvo
una vida familiar difícil y volcó su energía en los videojuegos y las
criptomonedas.

El arresto de Graham Ivan Clark planteó preguntas sobre cómo alguien tan
joven podría penetrar las defensas de una compañía tecnológica
sofisticada.Credit...

Supported by

\protect\hyperlink{after-sponsor}{Continue reading the main story}

Por \href{https://www.nytimes.com/by/nathaniel-popper}{Nathaniel
Popper}, \href{https://www.nytimes.com/by/kate-conger}{Kate Conger} y
\href{https://www.nytimes.com/by/kellen-browning}{Kellen Browning}

\begin{itemize}
\item
  4 de agosto de 2020Actualizado 19:33 ET
\item
  \begin{itemize}
  \item
  \item
  \item
  \item
  \item
  \end{itemize}
\end{itemize}

\href{https://www.nytimes.com/2020/08/02/technology/florida-teenager-twitter-hack.html}{Read
in English}

\href{https://www.nytimes.com/newsletters/el-times}{Regístrate para
recibir nuestro boletín} con lo mejor de The New York Times.

\begin{center}\rule{0.5\linewidth}{\linethickness}\end{center}

Graham Ivan Clark comenzó a hacer fechorías en línea desde pequeño.

A los 10 años era aficionado al videojuego Minecraft, en parte para
escapar de lo que, según les contaba a sus amigos, era una vida familiar
infeliz. Muchos de sus amigos señalaron que, en Minecraft, se dio a
conocer como un estafador experto con temperamento explosivo que
engañaba a la gente para quitarle dinero.

A los 15 años, se unió a un foro de ciberpiratas. A los 16, ya había
gravitado hacia el mundo del bitcóin, y al parecer se involucró en un
robo de 856.000 dólares de la criptomoneda, aunque nunca se le acusó de
ese delito, según las redes sociales y los registros legales. En sus
publicaciones de Instagram posteriores a ese suceso, apareció en tenis
de diseñador y un Rolex con diamantes incrustados.

El mal comportamiento digital del adolescente terminó el 31 de julio
cuando
\href{https://www.nytimes.com/2020/07/31/technology/twitter-hack-arrest.html}{la
policía lo arrestó} en un apartamento de Tampa, Florida. Los fiscales de
Florida afirmaron que Clark, quien ahora tiene 17 años, fue el ``autor
intelectual'' de un
\href{https://www.nytimes.com/2020/07/17/technology/twitter-hackers-interview.html}{ambicioso
ciberataque ilegal ocurrido el mes pasado}, y lo acusaron de entrar a
los sistemas de Twitter y
\href{https://www.nytimes.com/2020/07/15/technology/twitter-hack-bill-gates-elon-musk.html}{apoderarse
de las cuentas} de algunas de las personas más famosas del mundo,
incluyendo las de Barack Obama, Kanye West y Jeff Bezos.

Su arresto suscitó preguntas sobre cómo alguien tan joven pudo penetrar
las defensas de la que supuestamente es una de las compañías de
tecnología más sofisticadas de Silicon Valley. Según los fiscales, Clark
trabajó con al menos otras dos personas para burlar la seguridad de
\href{https://www.nytimes.com/2020/08/03/technology/ftc-twitter-privacy-violations.html}{Twitter},
pero lideró la operación por lo que se le acusa como adulto de 30
delitos graves.

Millones de adolescentes juegan los mismos videojuegos e interactúan en
los mismos foros en línea que Clark, pero lo que muestran las
entrevistas con más de 12 personas que lo conocen, además de los
documentos legales, la evidencia forense en línea y los historiales de
redes sociales, es la imagen de un joven que tuvo una relación tensa con
su familia y que pasó gran parte de su vida en línea adquiriendo la
habilidad de persuadir a otros de que le dieran dinero, fotografías e
información.

``Me estafó un poco de dinero cuando yo era apenas un niño'', afirmó
Colby Meeds, de 19 años, un jugador de Minecraft que asegura que Clark
le robó 50 dólares en 2016, cuando le ofreció venderle una capa digital
para un personaje de Minecraft que nunca entregó.

\includegraphics{https://static01.nyt.com/images/2020/08/03/business/04TwitterTeen-ES-01/merlin_174606657_99a08768-0e18-4af8-afd1-a79cbbedd123-articleLarge.jpg?quality=75\&auto=webp\&disable=upscale}

En una breve videollamada que realizó el domingo 2 de julio desde la
cárcel del condado de Hillsborough en Tampa, Clark apareció con una
camiseta negra sin mangas y con el cabello encima de los ojos. ``¿Cuáles
son sus preguntas?'', dijo, antes de empujar su silla hacia atrás y
colgar. El martes 4 de agosto tiene una comparecencia virtual en el
tribunal.

Clark y su hermana crecieron en Tampa con su madre, Emiliya Clark, una
inmigrante rusa que tiene certificados para trabajar como especialista
en tratamientos faciales y agente de bienes raíces. Cuando se le
contactó en su casa, su madre se negó a hacer comentarios. De acuerdo
con documentos públicos, su padre vive en Indiana, pero no respondió a
la solicitud de comentarios. Sus padres se divorciaron cuando el joven
tenía siete años.

Clark adoraba a su perro, no le gustaba la escuela y tampoco tenía
muchos amigos, dijo James Xio, quien conoció en línea a Clark hace
muchos años. Tenía el hábito de irse a los extremos en sus reacciones
emocionales y se volvía loco por las ofensas sin importancia, dijo Xio.

``Se enojaba muchísimo'', dijo Xio, de 18 años. ``Era muy poco
paciente''.

Abishek Patel, de 19 años, quien jugaba Minecraft con Clark, lo
defendió. ``Tiene un buen corazón y siempre cuida a la gente que le
importa'', dijo.

En 2016, Clark creó un canal de YouTube, según la empresa de monitoreo
de redes sociales SocialBlade. Alcanzó un público de miles de seguidores
y se dio a conocer por jugar una versión violenta de Minecraft llamada
Hardcore Factions, con nombres de usuario como ``Open'' y ``OpenHCF''.

No obstante, se volvió más popular por robarle dinero a otros jugadores
de Minecraft. En el videojuego los usuarios pueden pagar por elementos
adicionales, como accesorios para sus personajes.

Una táctica utilizada por Clark consistía en fingir que vendía nombres
de usuario atractivos para Minecraft y luego no entregarlos a los
compradores. También vendía capas para personajes de Minecraft, pero a
veces desaparecía después de que los otros jugadores le enviaban el
dinero.

Image

Algunos de los perfiles en línea vinculados con Clark fueron
deshabilitados debido a su comportamiento.

En una ocasión, Clark puso a la venta su propio nombre de usuario de
Minecraft, ``Open'', aseguró Nick Jerome, de 21 años, estudiante de la
Universidad Christopher Newport en Virginia. Los jóvenes intercambiaron
mensajes a través de Skype y Jerome, que en ese entonces tenía 17 años,
dijo que le envió alrededor de 100 dólares a cambio del nombre de
usuario porque le parecía genial. Después Clark lo bloqueó.

``Era como un adolescente tonto y, en retrospectiva, de ninguna manera
debí hacerlo'', dijo Jerome. ``¿Por qué debería haber confiado en este
tipo?''.

A finales de 2016 y principios de 2017, otros jugadores de Minecraft
produjeron videos en YouTube
\href{https://www.youtube.com/watch?v=CvrPXbk-BXw}{en los que
describieron} cómo habían perdido dinero o habían enfrentado ataques en
línea después de tener roces con ``Open'', el nombre de usuario de
Clark. En algunos de esos videos, Clark, a quien se le escucha decir
epítetos racistas y sexistas, también dijo que recibía instrucción
escolar en casa mientras ganaba 5000 dólares al mes con sus actividades
en Minecraft.

La verdadera identidad de Clark rara vez apareció en línea. En un
momento, reveló su rostro y la configuración de su juego, y algunos
jugadores lo llamaban Graham. Su nombre también fue mencionado en
\href{https://twitter.com/opengrahamclark/status/848014008632344576}{una
publicación de Twitter en 2017}.

Los intereses de Clark pronto se extendieron al videojuego
\href{https://www.nytimes.com/2018/07/25/arts/what-is-fortnite-battle-royale-nyt.html}{Fortnite}
y al lucrativo mundo de las criptomonedas. Se unió a un foro en línea
para ciberpiratas, conocido como OGUsers, con el nombre de usuario
Graham\$. Su cuenta de OGUsers fue registrada desde la misma dirección
IP en Tampa que estaba ligada a sus cuentas de Minecraft, de acuerdo con
la investigación realizada para The New York Times por Echosec, una
empresa de investigación forense en línea.

Clark se describió a sí mismo en OGUsers como un ``comerciante de
criptomonedas de tiempo completo que abandonó sus estudios'' y dijo que
estaba ``enfocado en generar dinero para todos''. Más tarde, Graham\$
fue expulsado de la comunidad, según publicaciones descubiertas por
Echosec, después de que los moderadores señalaron que no le había pagado
con bitcoines a otro usuario que ya le había enviado dinero para
concretar una transacción.

Image

~El mal comportamiento de Clark en línea terminó a finales de julio,
cuando la policía lo arrestó en su apartamento en Tampa,
Florida.Credit...Octavio Jones para The New York Times

Aun así, Clark ya había aprovechado el foro de OGUsers para abrirse
camino en una comunidad de ciberpiratas famosa por robar los números de
teléfono de las personas para acceder a todas las cuentas en línea
ligadas a esos números, un
\href{https://www.nytimes.com/2017/08/21/business/dealbook/phone-hack-bitcoin-virtual-currency.html}{ataque
conocido como intercambio de SIM}. El objetivo principal era vaciar las
cuentas de criptomonedas de las víctimas.

En 2019, los ciberpiratas asumieron a distancia el control del teléfono
de Gregg Bennett, un inversionista en tecnología en el área de Seattle.
En pocos minutos, se apoderaron de las cuentas en línea de Bennett,
incluyendo sus usuarios de Amazon y de correo electrónico, así como 164
bitcoines que valían 856.000 dólares en ese momento y que hoy valdrían
1,8 millones de dólares.

Bennett recibió enseguida una nota de extorsión, que compartió con el
Times. Estaba firmada por Scrim, otro de los alias de Clark, de acuerdo
con varios de sus amigos en línea.

``Solo queremos el resto de los fondos en Bittrex'', escribió Scrim,
refiriéndose al intercambio de bitcoines del que se habían tomado las
monedas. ``Siempre estamos un paso adelante y esta es la opción más
sencilla para ti''.

En abril, el Servicio Secreto le confiscó 100 bitcoines a Clark, según
los documentos de incautación del gobierno. Unas semanas más tarde,
Bennett recibió una carta del Servicio Secreto que decía que habían
recuperado 100 de sus bitcoines y mencionaba el mismo código que les fue
asignado a las monedas incautadas de Clark.

No está claro si hubo otras personas involucradas en el incidente o qué
fue lo que sucedió con los 64 bitcoines restantes.

Bennett afirmó en una entrevista que un agente del Servicio Secreto le
dijo que la persona con los bitcoines robados no fue arrestada por ser
menor de edad. El Servicio Secreto no respondió a una solicitud de
comentarios.

Para entonces, Clark vivía en su propio apartamento en un complejo de
condominios en Tampa. Tenía una costosa estación con equipos de
\emph{gaming setup}, un balcón y la vista a un parque cubierto de
hierba, según amigos y publicaciones en redes sociales.

Dos vecinos dijeron que Clark era muy reservado y que iba y venía a
horas inusuales y conducía un BMW Serie 3 blanco.

En una cuenta de Instagram que desde entonces ha sido retirada, @error,
Clark también compartió videos de sí mismo moviéndose al ritmo de un rap
en tenis de diseñador. Un joyero de la élite del rap lo saludó en
Instagram, con una fotografía que mostraba que Clark, como @error, se
había comprado un Rolex con gemas incrustadas.

Xio, quien se hizo amigo cercano de Clark, dijo que el encuentro en
abril con el Servicio Secreto conmocionó a Clark.

``Sabía que le habían dado una segunda oportunidad'', dijo Xio. ``Y
quería trabajar en hacer las cosas con la mayor legalidad posible''.

No obstante, menos de dos semanas después de la incautación del Servicio
Secreto, los fiscales señalaron que Clark comenzó a intentar penetrar el
sistema de Twitter. Según una declaración jurada del gobierno, Clark
convenció a un ``trabajador de Twitter de que era un colega del
departamento de informática e hizo que le proporcione la credencial de
acceso al portal de servicio al cliente''.

Para obtener ayuda, Clark encontró cómplices en OGUsers, de acuerdo con
los documentos de la acusación. Los cómplices se ofrecieron a negociar
la venta de las cuentas de Twitter que tenían nombres geniales, como @w,
mientras que Clark entraría en los sistemas de la red social y cambiaría
la propiedad de las cuentas, de acuerdo con los documentos y relatos de
los cómplices.

Image

El hackeo a Twitter se desarrolló el 15 de julio. Para obtener ayuda,
Clark encontró cómplices en OGUsers, según documentos
legales.Credit...Jim Wilson/The New York Times

La entrada al sistema de Twitter se llevó a cabo el 15 de julio. Unos
días más tarde, un cómplice, que se hacía llamar ``lol'', le
\href{https://www.nytimes.com/2020/07/17/technology/twitter-hackers-interview.html}{dijo
al Times} que la persona que conocían como el autor intelectual comenzó
a engañar a los clientes que querían comprar encubiertamente las cuentas
de Twitter. El hacker tomó el dinero y entregó la cuenta, pero luego la
recuperó rápidamente usando su acceso a los sistemas de Twitter para
expulsar al cliente. Era una reminiscencia de lo que Clark había hecho
anteriormente en Minecraft.

Cuando los conocidos en línea de Clark se enteraron de que había sido
acusado del ataque, varios dijeron que no estaban sorprendidos.

``Realmente nunca pareció preocuparse por nadie más que por sí mismo'',
dijo Connor Belcher, un jugador conocido como @iMakeMcVidz que
previamente se había asociado con Clark en un canal de YouTube antes de
convertirse en uno de sus críticos en línea.

Susan Jacobson colaboró con este reportaje desde Tampa, Florida.
Sheelagh McNeil y Jack Begg colaboraron con la investigación.

Nathaniel Popper cubre finanzas y tecnología. Es autor de \emph{Digital
Gold: Bitcoin and the Inside Story of the Misfits and Millionaires
Trying to Reinvent Money}. Anteriormente trabajó para The Los Angeles
Times y The Forward.
\href{https://twitter.com/nathanielpopper}{@nathanielpopper} •
\href{https://www.facebook.com/nathanielvpopper}{Facebook}

Kate Conger es reportera de tecnología en San Francisco, y cubre
privacidad, políticas y trabajo. Anteriormente, escribió sobre
ciberseguridad para Gizmodo y TechCrunch.
\href{https://twitter.com/kateconger}{@kateconger}

Kellen Browning es reportera de tecnología en el Área de la Bahía y
cubre la industria de los videojuegos y las noticias tecnológicas en
general. Se graduó de Pomona College.
\href{https://twitter.com/kellen_browning}{@kellen\_browning}

\begin{center}\rule{0.5\linewidth}{\linethickness}\end{center}

Advertisement

\protect\hyperlink{after-bottom}{Continue reading the main story}

\hypertarget{site-index}{%
\subsection{Site Index}\label{site-index}}

\hypertarget{site-information-navigation}{%
\subsection{Site Information
Navigation}\label{site-information-navigation}}

\begin{itemize}
\tightlist
\item
  \href{https://help.nytimes.com/hc/en-us/articles/115014792127-Copyright-notice}{©~2020~The
  New York Times Company}
\end{itemize}

\begin{itemize}
\tightlist
\item
  \href{https://www.nytco.com/}{NYTCo}
\item
  \href{https://help.nytimes.com/hc/en-us/articles/115015385887-Contact-Us}{Contact
  Us}
\item
  \href{https://www.nytco.com/careers/}{Work with us}
\item
  \href{https://nytmediakit.com/}{Advertise}
\item
  \href{http://www.tbrandstudio.com/}{T Brand Studio}
\item
  \href{https://www.nytimes.com/privacy/cookie-policy\#how-do-i-manage-trackers}{Your
  Ad Choices}
\item
  \href{https://www.nytimes.com/privacy}{Privacy}
\item
  \href{https://help.nytimes.com/hc/en-us/articles/115014893428-Terms-of-service}{Terms
  of Service}
\item
  \href{https://help.nytimes.com/hc/en-us/articles/115014893968-Terms-of-sale}{Terms
  of Sale}
\item
  \href{https://spiderbites.nytimes.com}{Site Map}
\item
  \href{https://help.nytimes.com/hc/en-us}{Help}
\item
  \href{https://www.nytimes.com/subscription?campaignId=37WXW}{Subscriptions}
\end{itemize}
