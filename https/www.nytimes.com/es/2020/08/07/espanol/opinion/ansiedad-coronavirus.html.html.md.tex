Sections

SEARCH

\protect\hyperlink{site-content}{Skip to
content}\protect\hyperlink{site-index}{Skip to site index}

\href{https://www.nytimes.com/es/section/opinion}{Opinión}

\href{https://myaccount.nytimes.com/auth/login?response_type=cookie\&client_id=vi}{}

\href{https://www.nytimes.com/section/todayspaper}{Today's Paper}

\href{/es/section/opinion}{Opinión}\textbar{}La fatiga pandémica no solo
te afecta a ti

\href{https://nyti.ms/2XDR34h}{https://nyti.ms/2XDR34h}

\begin{itemize}
\item
\item
\item
\item
\item
\end{itemize}

\href{https://www.nytimes.com/es/spotlight/coronavirus?action=click\&pgtype=Article\&state=default\&region=TOP_BANNER\&context=storylines_menu}{El
brote de coronavirus}

\begin{itemize}
\tightlist
\item
  \href{https://www.nytimes.com/es/interactive/2020/espanol/america-latina/coronavirus-en-mexico.html?action=click\&pgtype=Article\&state=default\&region=TOP_BANNER\&context=storylines_menu}{Mapa
  y casos en México}
\item
  \href{https://www.nytimes.com/es/2020/07/31/espanol/ciencia-y-tecnologia/ninos-contagio-coronavirus.html?action=click\&pgtype=Article\&state=default\&region=TOP_BANNER\&context=storylines_menu}{Los
  niños y el virus}
\item
  \href{https://www.nytimes.com/es/interactive/2020/science/coronavirus-tratamientos-curas.html?action=click\&pgtype=Article\&state=default\&region=TOP_BANNER\&context=storylines_menu}{Fármacos
  y tratamientos}
\item
  \href{https://www.nytimes.com/es/2020/07/06/espanol/ciencia-y-tecnologia/coronavirus-transmision-aire.html?action=click\&pgtype=Article\&state=default\&region=TOP_BANNER\&context=storylines_menu}{Cómo
  se transmite el coronavirus}
\item
  \href{https://www.nytimes.com/es/2020/07/14/espanol/estilos-de-vida/botiquin-medicina-coronavirus.html?action=click\&pgtype=Article\&state=default\&region=TOP_BANNER\&context=storylines_menu}{Prepara
  tu botiquín}
\end{itemize}

Advertisement

\protect\hyperlink{after-top}{Continue reading the main story}

\href{/es/section/opinion}{Opinión}

Supported by

\protect\hyperlink{after-sponsor}{Continue reading the main story}

Comentario

\hypertarget{la-fatiga-panduxe9mica-no-solo-te-afecta-a-ti}{%
\section{La fatiga pandémica no solo te afecta a
ti}\label{la-fatiga-panduxe9mica-no-solo-te-afecta-a-ti}}

Llámalo angustia por la COVID-19, cansancio de verano o como quieras,
pero tenemos un problema mucho más profundo y colectivo: como nación, no
estamos bien.

\includegraphics{https://static01.nyt.com/images/2020/08/08/opinion/07senior-ES-1/05senior2-articleLarge.jpg?quality=75\&auto=webp\&disable=upscale}

\href{https://www.nytimes.com/by/jennifer-senior}{\includegraphics{https://static01.nyt.com/images/2018/10/26/opinion/jennifer-senior/jennifer-senior-thumbLarge.png}}

Por \href{https://www.nytimes.com/by/jennifer-senior}{Jennifer Senior}

Es columnista de The New York Times.

\begin{itemize}
\item
  7 de agosto de 2020
\item
  \begin{itemize}
  \item
  \item
  \item
  \item
  \item
  \end{itemize}
\end{itemize}

\href{https://www.nytimes.com/2020/08/05/opinion/coronavirus-mental-illness-depression.html}{Read
in English}

\href{https://www.nytimes.com/newsletters/el-times}{Regístrate para
recibir nuestro boletín} con lo mejor de The New York Times.

\begin{center}\rule{0.5\linewidth}{\linethickness}\end{center}

Estoy tratando de recordar cuándo fue que me di cuenta de que todos
habíamos chocado contra un muro.

¿Fue hace dos semanas, cuando una amiga, que por lo general encarna el
papel de la esposa discreta, inició una conversación telefónica
despotricando sobre su marido?

¿Fue cuando miré a mi pareja ---probablemente una semana después--- y,
de manera calmada, le dije que todos mis problemas eran su culpa?

(Aunque eso no era así).

¿O tal vez fue cuando estaba en Twitter y vi un
\href{https://twitter.com/amandastern/status/1284639637252845570}{mensaje
de la autora Amanda Stern}, una mujer soltera que vive en Brooklyn, en
el que contaba que habían pasado 137 días desde que había dado o
recibido un abrazo? ``Hola, estoy deprimida'' eran las últimas palabras
de su tuit.

Sea lo que sea, es real, y cuantificable, y se extiende mucho más allá
de mi pequeño sistema solar de colegas, amigos y seres queridos. Llámalo
fatiga pandémica, llámalo cansancio de verano, llámalo como quieras. En
este punto, es probable que cualquier término sea trivial y desmienta lo
que en realidad es un problema mucho más profundo. Como nación, no
estamos bien.

Primero veamos los números. Según el Centro Nacional de Estadísticas de
Salud de Estados Unidos, aproximadamente
\href{https://www.cdc.gov/nchs/data/nhis/earlyrelease/ERmentalhealth-508.pdf}{1
de cada 12 adultos estadounidenses} reportaron síntomas de trastorno de
ansiedad durante esta época en 2019, ahora la tasa supera
\href{https://www.cdc.gov/nchs/covid19/pulse/mental-health.htm}{1 de
cada 3}. La semana pasada, la Kaiser Family Foundation publicó una
\href{https://www.kff.org/coronavirus-covid-19/report/kff-health-tracking-poll-july-2020/}{encuesta
de seguimiento} que muestra que, por primera vez, la mayoría de los
adultos estadounidenses (el 53 por ciento) cree que la pandemia está
afectando su salud mental.

Este número asciende al 68 por ciento, si solo nos fijamos en los
afroestadounidenses. El enorme costo que la pandemia ha cobrado en las
vidas y los medios de vida de las personas negras, debido a disparidades
estructurales que llevan siglos y que han sido agravadas por los efectos
psicológicos del racismo cotidiano, está apareciendo claramente en
nuestros datos de salud mental.

``Incluso durante lo que conocíamos como los mejores tiempos, los
adultos negros tenían más probabilidades de reportar síntomas
persistentes de angustia emocional'', me dijo Hope Hill, psicóloga
clínica y profesora asociada en el Departamento de Psicología de la
Universidad de Howard. ``Entonces, cuando me enteré de esa diferencia de
15 puntos, es algo que molesta pero no resulta sorprendente debido al
impacto del trauma a largo plazo y la desigualdad basada en la raza''.

Pero incluso los más afortunados no se han salvado. Según la Kaiser
Family Foundation, el 36 por ciento de los estadounidenses reportan que
las preocupaciones relacionadas con el coronavirus están interfiriendo
con su sueño. Un 18 por ciento dice que está perdiendo los estribos más
fácilmente. El 32 por ciento de las personas consultadas afirman que la
pandemia ha hecho que coman menos o que se alimenten en exceso.

Mi caso definitivamente entra en la segunda categoría. Resulta que los
4,5 kilos adicionales en mi cintura se han mudado y han desempacado,
aunque inicialmente esperaba que solo fuera un contrato de arrendamiento
mensual.

Entonces, ¿cómo se explica esta caída a nivel nacional en un pozo de
angustia?

La respuesta más obvia es que el coronavirus
\href{https://www.nytimes.com/es/interactive/2020/espanol/mundo/coronavirus-en-estados-unidos.html}{sigue
cobrando cientos} de vidas cada día en Estados Unidos, abriéndose paso a
través del sur y volviendo al oeste. Esto es cierto, y resulta terrible.
Pero sospecho que es más que eso.

Las vastas tasas de infección de Estados Unidos también son un
testimonio de nuestro propio fracaso nacional y, por lo tanto, una
fuente de horror existencial, de pura perversidad: ¿por qué demonios
sacrificamos tanto en estos últimos cuatro meses y medio ---nuestro
sustento, nuestras relaciones sociales, nuestra seguridad, la
escolarización de nuestros hijos, las reuniones de cumpleaños,
aniversarios y funerales--- si todo fracasa? En este punto, ¿no
esperábamos algún tipo de alivio, la reanudación de algo parecido a la
vida?

\includegraphics{https://static01.nyt.com/images/2020/08/05/opinion/07senior-ES-2/05senior1-articleLarge.jpg?quality=75\&auto=webp\&disable=upscale}

``La gente suele pensar en el trauma como un evento puntual: un incendio
o un asalto, por ejemplo'', dijo Daphne de Marneffe, autora de un
excelente libro sobre el matrimonio llamado
\href{https://www.thecut.com/2018/01/daphne-de-marneff-on-the-rough-patch.html}{\emph{The
Rough Patch}} y una de las psicólogas más astutas que conozco.

``Pero de lo que se trata realmente es de la impotencia, de sufrir las
consecuencias de fuerzas que no puedes controlar. Eso es lo que tenemos
ahora. Es como si estuviéramos en un viaje en auto infinito con un
borracho al volante. Nadie sabe cuándo cesará el dolor''.

A eso le podemos agregar que ninguno de nosotros sabe cómo será la vida
cuando esta pandemia haya cedido. ¿La economía seguirá tan afectada?
(Tengo una palabra para ti: inflación). ¿Los centros de nuestras
ciudades se convertirán en conchas rotas, arenosas y vacías? (Espero que
no). ¿El presidente Donald Trump será reelegido y transformará a la
democracia, como la conocemos, en un inquietante negativo fotográfico de
lo que era?

En su propia práctica terapéutica, De Marneffe ha notado que las
familias con tensiones y fragilidades preexistentes han empeorado: la
pandemia ha brindado más oportunidades para que las parejas en
dificultades se comuniquen mal, pongan los ojos en blanco y se proyecten
argumentos negativos (``y el matrimonio es un hervidero de chivos
expiatorios'', señaló). Los padres que apenas podían cumplir con sus
obligaciones ---mientras rezaban para que comenzara la escuela--- ahora
están llenos de desesperación, lo que arruina su falta de imaginación:
¿cómo se supone que lograrán superar otro semestre de
\href{https://www.nytimes.com/es/2020/05/01/espanol/escuela-casa-coronavirus.html}{educación
remota}?

``Los que somos padres promedio confiamos en la estructura'', me dijo.
``Necesitamos a las escuelas''.

Hace poco hojeé \emph{La peste} para ver si Albert Camus había intuido
algo sobre los ritmos del sufrimiento humano en condiciones de miedo,
enfermedad y limitaciones. Naturalmente, lo había hecho. Fue el 16 de
abril cuando el doctor Rieux sintió por primera vez que pisaba una rata
muerta en su rellano. A mediados de agosto, cuando la peste ``se lo
había tragado todo'', la emoción predominante ``era la separación y el
exilio, con lo que eso significaba de miedo y de rebeldía''.

Los que regresaron de la cuarentena comenzaron a incendiar sus hogares,
convencidos de que la plaga se había asentado en sus paredes.

Camus sintió, en otras palabras, que la impronta de esos cuatro meses se
volvió bastante extraña en Orán. Eso es más o menos lo que pasó aquí. Si
tan solo supiéramos cómo terminará.

Jennifer Senior ha sido columnista de Opinión desde septiembre de 2018.
Había sido crítica de libros y, antes de eso, pasó muchos años como
redactora de la revista New York. Su libro \emph{Todo gozo y no
diversión: la paradoja de la paternidad moderna}, ha sido traducido a 12
idiomas. \href{https://twitter.com/jenseniorny}{@JenSeniorNY}

Advertisement

\protect\hyperlink{after-bottom}{Continue reading the main story}

\hypertarget{site-index}{%
\subsection{Site Index}\label{site-index}}

\hypertarget{site-information-navigation}{%
\subsection{Site Information
Navigation}\label{site-information-navigation}}

\begin{itemize}
\tightlist
\item
  \href{https://help.nytimes.com/hc/en-us/articles/115014792127-Copyright-notice}{©~2020~The
  New York Times Company}
\end{itemize}

\begin{itemize}
\tightlist
\item
  \href{https://www.nytco.com/}{NYTCo}
\item
  \href{https://help.nytimes.com/hc/en-us/articles/115015385887-Contact-Us}{Contact
  Us}
\item
  \href{https://www.nytco.com/careers/}{Work with us}
\item
  \href{https://nytmediakit.com/}{Advertise}
\item
  \href{http://www.tbrandstudio.com/}{T Brand Studio}
\item
  \href{https://www.nytimes.com/privacy/cookie-policy\#how-do-i-manage-trackers}{Your
  Ad Choices}
\item
  \href{https://www.nytimes.com/privacy}{Privacy}
\item
  \href{https://help.nytimes.com/hc/en-us/articles/115014893428-Terms-of-service}{Terms
  of Service}
\item
  \href{https://help.nytimes.com/hc/en-us/articles/115014893968-Terms-of-sale}{Terms
  of Sale}
\item
  \href{https://spiderbites.nytimes.com}{Site Map}
\item
  \href{https://help.nytimes.com/hc/en-us}{Help}
\item
  \href{https://www.nytimes.com/subscription?campaignId=37WXW}{Subscriptions}
\end{itemize}
