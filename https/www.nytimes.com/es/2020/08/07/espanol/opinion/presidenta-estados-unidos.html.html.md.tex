Sections

SEARCH

\protect\hyperlink{site-content}{Skip to
content}\protect\hyperlink{site-index}{Skip to site index}

\href{https://www.nytimes.com/es/section/opinion}{Opinión}

\href{https://myaccount.nytimes.com/auth/login?response_type=cookie\&client_id=vi}{}

\href{https://www.nytimes.com/section/todayspaper}{Today's Paper}

\href{/es/section/opinion}{Opinión}\textbar{}Lecciones para ser
presidenta

\href{https://nyti.ms/2C71c1D}{https://nyti.ms/2C71c1D}

\begin{itemize}
\item
\item
\item
\item
\item
\end{itemize}

Advertisement

\protect\hyperlink{after-top}{Continue reading the main story}

\href{/es/section/opinion}{Opinión}

Supported by

\protect\hyperlink{after-sponsor}{Continue reading the main story}

Comentario

\hypertarget{lecciones-para-ser-presidenta}{%
\section{Lecciones para ser
presidenta}\label{lecciones-para-ser-presidenta}}

La esperanza de las mujeres de alcanzar las máximas posiciones del
gobierno en Estados Unidos lleva demasiado tiempo en suspenso.

\includegraphics{https://static01.nyt.com/images/2020/08/07/opinion/07Ramos-ES/07Ramos-articleLarge.jpg?quality=75\&auto=webp\&disable=upscale}

\includegraphics{https://static01.nyt.com/images/2019/11/08/opinion/jorge-ramos/jorege-ramos-thumbLarge.png}

Por Jorge Ramos

Es periodista y colaborador regular de opinión de The New York Times.

\begin{itemize}
\item
  7 de agosto de 2020
\item
  \begin{itemize}
  \item
  \item
  \item
  \item
  \item
  \end{itemize}
\end{itemize}

\href{https://www.nytimes.com/2020/08/07/opinion/latina-women-politics.html}{Read
in English}

\href{https://www.nytimes.com/newsletters/el-times}{Regístrate para
recibir nuestro boletín} con lo mejor de The New York Times.

\begin{center}\rule{0.5\linewidth}{\linethickness}\end{center}

MIAMI --- En febrero, antes de que el distanciamiento social fuera una
regla, tuve la oportunidad de entrevistar a Sonia Sotomayor, la jueza de
la Corte Suprema de Justicia de Estados Unidos, para mi pódcast,
\href{https://art19.com/shows/contrapoder/episodes/fd1859ec-5a1b-4b6f-ab94-6e615c9aa787}{Contrapoder}.
Sophie McLoud, una niña de 10 años, estaba ahí, escuchando nuestra
conversación, y al final le planteó una pregunta a la jueza, la primera
latina en su posición: ``¿Tú crees que una niña como yo puede ser
presidenta de Estados Unidos?''. La maravillosa respuesta de la jueza
Sotomayor a Sophie es una lección para todos. Pero empecemos con lo
último.

En los próximos días, el virtual candidato presidencial por el Partido
Demócrata,
\href{https://www.nytimes.com/es/interactive/2020/espanol/estados-unidos/joe-biden-elecciones.html}{Joe
Biden}, va a decidir quién será su candidata a la vicepresidencia.
Además de la posibilidad histórica de escoger a una afroamericana como
su compañera de fórmula, el entusiasmo también estriba en que esa mujer
podría convertirse, eventualmente, en la primera presidenta de Estados
Unidos.

La esperanza de las mujeres de alcanzar las máximas posiciones del
gobierno lleva mucho tiempo. Aún recuerdo con nostalgia una entrevista
que le hice en 1984 en Los Ángeles a Geraldine Ferraro, la primera mujer
en ser candidata a la vicepresidencia de uno de los dos partidos
políticos tradicionales de Estados Unidos. Tengo una foto con ella en la
que aparece con el puño levantado.

Ferraro y el candidato presidencial, Walter Mondale, perdieron esa
elección frente a Ronald Reagan. Pero así la recuerdo; como una
guerrera.

Tras la derrota de la campaña presidencial de Hillary Clinton en 2016 es
difícil entender cómo uno de los países más ricos y poderosos del mundo
nunca ha escogido a una mujer para la Casa Blanca. Otros países del
hemisferio ---Nicaragua, Panamá, Chile, Argentina, Brasil, Costa Rica---
han
\href{http://www.teinteresa.es/mundo/anos-mujeres-presidentas-America-Latina_0_1048695520.html}{tenido
presidentas}. No Estados Unidos.

Aunque hay cada vez más participación de las mujeres en los principales
puestos de la política estadounidense ---desde gobernadoras hasta la
presidenta de la Cámara de Representantes, Nancy Pelosi--- Estados
Unidos no figura entre las principales naciones del mundo en cuanto al
porcentaje de mujeres en su congreso local. Apenas ocupa el lugar número
83, \href{https://data.ipu.org/women-ranking?month=6\&year=2020}{según
la Unión Interparlamentaria}, una organización de parlamentos a nivel
mundial. Actualmente solo hay
\href{https://www.cawp.rutgers.edu/women-us-congress-2020}{101 mujeres}
con voto en la Cámara de Representantes de Estados Unidos (o un 23 por
ciento del total).

¿Qué se puede hacer para lograr una mayor igualdad política? ``Necesitas
leyes y necesitas una estructura para alcanzar igualdad de genero'',
\href{https://www.cnn.com/videos/tv/2020/02/07/exp-gps-0209-marin-on-gender-equality-in-usa.cnn}{dijo}
en CNN la primera ministra de Finlandia, Sanna Mirella Marin, quien a
los 34 años es la líder más joven del mundo. ``Eso no pasa por sí
mismo''. En Finlandia, por ejemplo, las leyes obligan a que haya
\href{https://thl.fi/en/web/gender-equality/gender-equality-in-finland/decision-making/gender-quotas}{al
menos un 40 por ciento de mujeres} en algunos puestos gubernamentales
que no son de elección popular.

En Estados Unidos
\href{https://thehill.com/opinion/campaign/465074-why-american-politics-needs-gender-quota}{no
hay leyes parecidas a las de Finlandia}. Pero si se
\href{https://www.britannica.com/topic/Equal-Rights-Amendment}{ratificara}
el Equal Rights Amendment (ERA),
\href{https://thewatchdogonline.com/the-equal-rights-amendment-is-almost-there-29291}{propuesto
en 1923}, habría un gran avance. Esa enmienda a la Constitución dice
simple y llanamente: ``La igualdad de derechos bajo la ley no debe ser
negada o coartada por Estados Unidos o por ninguno de los estados debido
al género''. ¿Podría un nuevo congreso en 2021 eliminar los obstáculos
legales que ha impedido que se apruebe por casi un siglo?

Y esto nos lleva a la entrevista que tuve a principios de año en Miami
con la jueza Sotomayor. Ella había publicado poco antes un libro para
niños llamado \emph{Just Ask! Be Different, Be Brave, Be You} en el que
habla de las cosas que nos hacen fuertes y distintos. Hablamos de las
experiencias que inspiraron su escritura, de su problema con la diabetes
---que la obliga a inyectarse insulina varias veces al día--- y de cómo
enfrentar el miedo.

``Cuando me nombraron para la Corte Suprema tenía yo un miedo terrible.
Es un trabajo grandísimo. ¿Quién vive la vida sin miedo?'', me dijo en
español. ``Pasé momentos diciendo: `No quiero hacer este trabajo'. No
estaba segura de que lo podía hacer bien. Y casi casi le dije que no al
presidente de Estados Unidos. Unas amistades mías oyeron que yo estaba
dudando, y una de ellas vino y me dijo: `Mira, Sonia, para de pensar en
ti. \emph{This is not about you}. Esto es de todas las nenas pequeñitas
que te van a mirar en esa posición'''.

Niñas como Sophie, quien nos escuchaba atentamente. Al final de la
entrevista y venciendo los ojos vigilantes de los adultos, se acercó a
la jueza para preguntarle si ella, una niña latina, algún día podría ser
presidenta de Estados Unidos. La jueza le dio un abrazo y luego su
respuesta: ``Sí, sí'', le dijo, en lo que se convertiría en una
verdadera lección de vida.

``Primero, una niña como tú tiene que soñar en grande siempre'', le dijo
Sotomayor.

``Segundo, nunca puedes dejar que nadie te diga que no lo puedes hacer.
Porque al minuto que te lo digan, tienes que reaccionar como yo. ¿Tú me
dices a mí que no lo puedo hacer? Yo te voy a enseñar que lo puedo
hacer. {[}\ldots{]} Tercero, tienes que estudiar, estudiar y estudiar.
Es la única manera de lograr lo que quiere hacer uno en la vida. La
educación es la llave al futuro. {[}\ldots{]} Y, cuarto, tienes que
trabajar fuerte. En la vida nadie ni nada te da algo. Tú te tienes que
ganar todo en esta vida. Y para hacer eso, entre la educación y
trabajando fuerte, son las dos maneras de ser presidenta de Estados
Unidos''.

Antes de despedirse, la jueza Sotomayor volvió a abrazar a Sophie y le
pidió un favor. ``Espero que yo esté viva cuando tú seas presidenta de
Estados Unidos'', le dijo, y le expresó su deseo de tomarle el juramento
constitucional cuando ese día llegue.

Y espero, yo también, estar frente a las dos. Pero para que eso ocurra
se necesita mucho más que buenas intenciones y un enorme trabajo
personal. Entiendo que la idea de cuotas es rechazada en un país que le
gusta pensarse a sí mismo como una meritocracia. La realidad es que si
no imponemos porcentajes mínimos de género ---como hacen en Finlandia---
en los consejos de gobierno de las ciudades y en las comisiones
estatales, será difícil romper con los prejuicios y las actuales
desigualdades. Lo que nos falta es un sentido de urgencia y nuevas
reglas que reflejen nuestra indignación.

Las mujeres latinas debe enfrentar una doble carga de obstáculos. Es por
eso que cuando una latina como la jueza Sotomayor,
\href{https://www.nytimes.com/es/2020/06/20/espanol/opinion/dreamers-suprema-corte-daca.html}{jovenes}
\href{https://www.nytimes.com/es/2020/06/20/espanol/opinion/dreamers-suprema-corte-daca.html}{\emph{dreamers}}
logran cambiar las leyes o se elige a una nueva gobernadora o senadora
hispana, ellas le abren el camino a quienes vienen detrás.

Sophie, sí, algún día podrá ser la primera presidenta latina de Estados
Unidos. Pero antes tiene que haber muchas como ella que le vayan
abriendo el camino. Y, como dijo la primera ministra Marin: ``Eso no
pasa por sí mismo''.

Jorge Ramos es periodista, conductor de los programas \emph{Noticiero
Univisión} y \emph{Al punto,} y autor del libro \emph{Stranger: El
desafío de un inmigrante latino en la era de Trump}.
\href{https://twitter.com/jorgeramosnews}{@jorgeramosnews}

Advertisement

\protect\hyperlink{after-bottom}{Continue reading the main story}

\hypertarget{site-index}{%
\subsection{Site Index}\label{site-index}}

\hypertarget{site-information-navigation}{%
\subsection{Site Information
Navigation}\label{site-information-navigation}}

\begin{itemize}
\tightlist
\item
  \href{https://help.nytimes.com/hc/en-us/articles/115014792127-Copyright-notice}{©~2020~The
  New York Times Company}
\end{itemize}

\begin{itemize}
\tightlist
\item
  \href{https://www.nytco.com/}{NYTCo}
\item
  \href{https://help.nytimes.com/hc/en-us/articles/115015385887-Contact-Us}{Contact
  Us}
\item
  \href{https://www.nytco.com/careers/}{Work with us}
\item
  \href{https://nytmediakit.com/}{Advertise}
\item
  \href{http://www.tbrandstudio.com/}{T Brand Studio}
\item
  \href{https://www.nytimes.com/privacy/cookie-policy\#how-do-i-manage-trackers}{Your
  Ad Choices}
\item
  \href{https://www.nytimes.com/privacy}{Privacy}
\item
  \href{https://help.nytimes.com/hc/en-us/articles/115014893428-Terms-of-service}{Terms
  of Service}
\item
  \href{https://help.nytimes.com/hc/en-us/articles/115014893968-Terms-of-sale}{Terms
  of Sale}
\item
  \href{https://spiderbites.nytimes.com}{Site Map}
\item
  \href{https://help.nytimes.com/hc/en-us}{Help}
\item
  \href{https://www.nytimes.com/subscription?campaignId=37WXW}{Subscriptions}
\end{itemize}
