Sections

SEARCH

\protect\hyperlink{site-content}{Skip to
content}\protect\hyperlink{site-index}{Skip to site index}

\href{https://www.nytimes.com/es/section/opinion}{Opinión}

\href{https://myaccount.nytimes.com/auth/login?response_type=cookie\&client_id=vi}{}

\href{https://www.nytimes.com/section/todayspaper}{Today's Paper}

Advertisement

\protect\hyperlink{after-top}{Continue reading the main story}

Supported by

\protect\hyperlink{after-sponsor}{Continue reading the main story}

\hypertarget{en-espauxf1ol}{%
\subsubsection{\texorpdfstring{\href{/es/}{en
Español}}{en Español}}\label{en-espauxf1ol}}

\hypertarget{opiniuxf3n}{%
\section{Opinión}\label{opiniuxf3n}}

\hypertarget{highlights}{%
\subsection{Highlights}\label{highlights}}

\begin{enumerate}
\def\labelenumi{\arabic{enumi}.}
\item
  \includegraphics{https://static01.nyt.com/images/2020/07/30/opinion/01Marr-ES-1/30Marr-jumbo.jpg}

  \hypertarget{comentario}{%
  \subsubsection{Comentario}\label{comentario}}

  \hypertarget{suxed-el-coronavirus-estuxe1-en-el-aire}{%
  \subsection{\texorpdfstring{\href{/es/2020/08/01/espanol/opinion/coronavirus-aire.html}{Sí,
  el coronavirus está en el
  aire}}{Sí, el coronavirus está en el aire}}\label{suxed-el-coronavirus-estuxe1-en-el-aire}}

  La transmisión por aerosoles es importante, y quizá sea mucho más
  relevante de lo que hemos podido comprobar hasta ahora.

  Por Linsey C. Marr
\item
  \includegraphics{https://static01.nyt.com/images/2020/07/29/multimedia/29Jimenez-ES/29Jimenez-ES-videoLarge.jpg}

  \hypertarget{comentario-1}{%
  \subsubsection{Comentario}\label{comentario-1}}

  \hypertarget{manual-de-comportamiento-para-expresidentes-insoportables}{%
  \subsection{\texorpdfstring{\href{/es/2020/07/29/espanol/opinion/espana-felipe-gonzalez-jose-maria-aznar.html}{Manual
  de comportamiento para expresidentes
  insoportables}}{Manual de comportamiento para expresidentes insoportables}}\label{manual-de-comportamiento-para-expresidentes-insoportables}}

  La incapacidad de algunos exmandatarios de España de aceptar su
  jubilación viene en parte de una falta de cultura democrática. Algunos
  países de Latinoamérica tienen el mismo problema.

  Por David Jiménez
\item
  \includegraphics{https://static01.nyt.com/images/2020/07/28/multimedia/28garcia-timerman-ES-3/28garcia-timerman-ES-3-videoLarge.jpg}

  \hypertarget{comentario-2}{%
  \subsubsection{Comentario}\label{comentario-2}}

  \hypertarget{por-quuxe9-no-explota-argentina}{%
  \subsection{\texorpdfstring{\href{/es/2020/07/28/espanol/opinion/argentina-estallido-2001-coronavirus.html}{¿Por
  qué no explota
  Argentina?}}{¿Por qué no explota Argentina?}}\label{por-quuxe9-no-explota-argentina}}

  Lecciones de 2001, una fuerte política social asistencialista y una
  grieta política potente han alejado de momento otro estallido social,
  pero solo una nueva política cooperativa podrá terminar de disipar ese
  fantasma.

  Por Marcelo J. García y Jordana Timerman
\item
  \includegraphics{https://static01.nyt.com/images/2020/07/24/opinion/00herrcher/00herrcher-videoLarge.jpg}

  \hypertarget{comentario-3}{%
  \subsubsection{Comentario}\label{comentario-3}}

  \hypertarget{las-enseuxf1anzas-de-educar-durante-la-pandemia}{%
  \subsection{\texorpdfstring{\href{/es/2020/07/27/espanol/opinion/clases-universidad-coronavirus.html}{Las
  enseñanzas de educar durante la
  pandemia}}{Las enseñanzas de educar durante la pandemia}}\label{las-enseuxf1anzas-de-educar-durante-la-pandemia}}

  He sido profesor universitario durante más de dos décadas, y estos
  meses de enseñar durante la pandemia me han hecho entender que la
  educación ha cambiado para siempre.

  Por Roberto Herrscher
\end{enumerate}

Advertisement

\protect\hyperlink{after-mid1}{Continue reading the main story}

\begin{itemize}
\tightlist
\item
  \protect\hyperlink{stream-panel}{Lo más reciente}
\item
  Buscar
\end{itemize}

\begin{enumerate}
\def\labelenumi{\arabic{enumi}.}
\item
  \href{/es/2020/08/01/espanol/opinion/trump-autoritarismo.html}{}

  \includegraphics{https://static01.nyt.com/images/2020/08/01/multimedia/01Ramos-ES/merlin_175175088_ad2b68a7-8076-4175-a205-2f0a1352507f-thumbWide.jpg?quality=75\&auto=webp\&disable=upscale}

  \hypertarget{comentario-4}{%
  \subsubsection{Comentario}\label{comentario-4}}

  \hypertarget{tentaciones-autoritarias-cuxf3mo-amuxe9rica-latina-nos-preparuxf3-para-trump}{%
  \subsection{Tentaciones autoritarias: cómo América Latina nos preparó
  para
  Trump}\label{tentaciones-autoritarias-cuxf3mo-amuxe9rica-latina-nos-preparuxf3-para-trump}}

  La democracia en Estados Unidos está a prueba. Quienes hemos vivido o
  trabajado en la región, conocemos bien de mandatarios que juegan con
  los límites de su poder. Adiós al ``excepcionalismo estadounidense''.

  Por Jorge Ramos
\item
  \href{/es/2020/07/30/espanol/opinion/usar-cubrebocas-politica.html}{}

  \includegraphics{https://static01.nyt.com/images/2020/07/28/opinion/28friedmanWeb/28friedmanWeb-thumbWide.jpg?quality=75\&auto=webp\&disable=upscale}

  \hypertarget{comentario-5}{%
  \subsubsection{Comentario}\label{comentario-5}}

  \hypertarget{si-nuestros-cubrebocas-pudieran-hablar}{%
  \subsection{Si nuestros cubrebocas pudieran
  hablar}\label{si-nuestros-cubrebocas-pudieran-hablar}}

  ¿Cómo nos volvimos tan ineficaces para combatir al coronavirus? Los
  arqueólogos del futuro que vinieran a excavar al país más rico del
  mundo, encontrarían la clave en un artefacto sencillo: la mascarilla.

  Por Thomas L. Friedman

  \href{https://www.nytimes.com/2020/07/28/opinion/coronavirus-masks.html}{Read
  in English}
\item
  \href{/es/2020/07/30/espanol/opinion/aztecas-violencia-narco-amlo.html}{}

  \includegraphics{https://static01.nyt.com/images/2020/07/29/opinion/29villoro-sub/29villoro-sub-thumbWide.jpg?quality=75\&auto=webp\&disable=upscale}

  \hypertarget{comentario-6}{%
  \subsubsection{Comentario}\label{comentario-6}}

  \hypertarget{la-tierra-en-pruxe9stamo-una-gramuxe1tica-de-la-violencia-en-muxe9xico}{%
  \subsection{La tierra en préstamo: una gramática de la violencia en
  México}\label{la-tierra-en-pruxe9stamo-una-gramuxe1tica-de-la-violencia-en-muxe9xico}}

  El hallazgo de un inmenso altar fúnebre azteca permite reflexionar
  sobre las urgencias actuales sin fantasías atávicas pero con un nítido
  sentido de la historia y los desafíos del presente.

  Por Juan Villoro
\item
  \href{/es/2020/07/30/espanol/opinion/john-lewis-derechos-civiles.html}{}

  \includegraphics{https://static01.nyt.com/images/2020/07/29/opinion/00lewis/00lewis-thumbWide.jpg?quality=75\&auto=webp\&disable=upscale}

  \hypertarget{john-lewis-juntos-ustedes-pueden-recuperar-el-alma-de-estados-unidos}{%
  \subsection{John Lewis: Juntos, ustedes pueden recuperar el alma de
  Estados
  Unidos}\label{john-lewis-juntos-ustedes-pueden-recuperar-el-alma-de-estados-unidos}}

  Aunque me haya ido, los animo a responder al llamado más elevado de su
  corazón y a defender lo que realmente creen.

  Por John Lewis

  \href{https://www.nytimes.com/2020/07/30/opinion/john-lewis-civil-rights-america.html}{Read
  in English}
\item
  \href{/es/2020/07/27/espanol/opinion/reabrir-escuelas-riesgo-covid.html}{}

  \includegraphics{https://static01.nyt.com/images/2020/07/20/opinion/27reopen-ES/20jogee-thumbWide.jpg?quality=75\&auto=webp\&disable=upscale}

  \hypertarget{comentario-7}{%
  \subsubsection{Comentario}\label{comentario-7}}

  \hypertarget{cuxf3mo-reabrir-la-economuxeda-sin-causar-la-muerte-de-padres-y-maestros}{%
  \subsection{¿Cómo reabrir la economía sin causar la muerte de padres y
  maestros?}\label{cuxf3mo-reabrir-la-economuxeda-sin-causar-la-muerte-de-padres-y-maestros}}

  Todas las clases deberían ser en línea, pero los edificios todavía
  podrían cumplir un propósito importante para los niños que más lo
  necesitan.

  Por Shardha Jogee

  \href{https://www.nytimes.com/2020/07/20/opinion/coronavirus-reopen-schools-economy.html}{Read
  in English}
\item
  \href{/es/2020/07/26/espanol/opinion/nicmer-evans-venezuela.html}{}

  \includegraphics{https://static01.nyt.com/images/2020/07/26/multimedia/26Barrera-ES/merlin_170314122_3f1b8858-af1c-45e6-bbd1-e9a4e30fba75-thumbWide.jpg?quality=75\&auto=webp\&disable=upscale}

  \hypertarget{comentario-8}{%
  \subsubsection{Comentario}\label{comentario-8}}

  \hypertarget{teoruxeda-y-pruxe1ctica-del-odio}{%
  \subsection{Teoría y práctica del
  odio}\label{teoruxeda-y-pruxe1ctica-del-odio}}

  Hace tres años, el gobierno de Nicolás Maduro instauró una ley contra
  el odio, un instrumento legal para promover la tolerancia en
  Venezuela. Pero en la práctica se convirtió en lo contrario: una forma
  eficaz para ejercer la violencia del Estado en contra de cualquier
  disidencia.

  Por Alberto Barrera Tyszka
\item
  \href{/es/2020/07/24/espanol/opinion/walter-mercado-amor-netflix.html}{}

  \includegraphics{https://static01.nyt.com/images/2020/07/24/multimedia/24toro-ES/24toro-ES-thumbWide.jpg?quality=75\&auto=webp\&disable=upscale}

  \hypertarget{comentario-9}{%
  \subsubsection{Comentario}\label{comentario-9}}

  \hypertarget{walter-mercado-despuuxe9s-del-amor}{%
  \subsection{Walter Mercado después del
  amor}\label{walter-mercado-despuuxe9s-del-amor}}

  Un documental rinde tributo a una figura tan compleja y contradictoria
  como los valores de la sociedad latinoamericana que lo idolatró y que
  encontró en sus horóscopos una fuente de esperanza. Pero también es
  una carta de amor a la nostalgia compartida de toda una región.

  Por Ana Teresa Toro
\item
  \href{/es/2020/07/23/espanol/opinion/lopez-gatell.html}{}

  \includegraphics{https://static01.nyt.com/images/2020/07/23/multimedia/23Fonseca-ES/merlin_172216524_7d574f61-d31a-4822-b54a-bd624033875b-thumbWide.jpg?quality=75\&auto=webp\&disable=upscale}

  \hypertarget{comentario-10}{%
  \subsubsection{Comentario}\label{comentario-10}}

  \hypertarget{el-factor-luxf3pez-gatell}{%
  \subsection{El factor López-Gatell}\label{el-factor-luxf3pez-gatell}}

  La fractura de la imagen del portavoz del combate al coronavirus de
  México no es solo producto de sus errores. Sus fallos son producto del
  modelo de gestión de un gobierno más proclive a la improvisación que a
  la planeación.

  Por Diego Fonseca
\item
  \href{/es/2020/07/22/espanol/opinion/portland-protestas-trump.html}{}

  \includegraphics{https://static01.nyt.com/images/2020/07/20/opinion/22Goldberg-ES/20goldbergWeb-thumbWide.jpg?quality=75\&auto=webp\&disable=upscale}

  \hypertarget{comentario-11}{%
  \subsubsection{Comentario}\label{comentario-11}}

  \hypertarget{la-ocupaciuxf3n-de-trump-de-las-ciudades-estadounidenses-ha-comenzado}{%
  \subsection{La ocupación de Trump de las ciudades estadounidenses ha
  comenzado}\label{la-ocupaciuxf3n-de-trump-de-las-ciudades-estadounidenses-ha-comenzado}}

  Manifestantes en Portland fueron detenidos en las calles sin orden
  judicial por agentes no identificados. Y se anunció que en Chicago
  podría suceder lo mismo. ¿Ya podemos llamarlo fascismo?

  Por Michelle Goldberg

  \href{https://www.nytimes.com/2020/07/20/opinion/portland-protests-trump.html}{Read
  in English}
\item
  \href{/es/2020/07/22/espanol/opinion/periodismo-brasil-favelas.html}{}

  \includegraphics{https://static01.nyt.com/images/2020/07/22/multimedia/22pires-ES-1/22pires-ES-1-thumbWide.jpg?quality=75\&auto=webp\&disable=upscale}

  \hypertarget{comentario-12}{%
  \subsubsection{Comentario}\label{comentario-12}}

  \hypertarget{desde-las-favelas-de-brasil-una-lecciuxf3n-de-periodismo}{%
  \subsection{Desde las favelas de Brasil: una lección de
  periodismo}\label{desde-las-favelas-de-brasil-una-lecciuxf3n-de-periodismo}}

  Los medios comunitarios brasileños debaten los temas más urgentes de
  nuestro tiempo: la pandemia y el racismo. Lo medios tradicionales
  pueden apoyarse en ellos para ampliar sus coberturas y el público
  acercarse a ellos para salir de sus burbujas.

  Por Carol Pires
\end{enumerate}

Ver más

Advertisement

\protect\hyperlink{after-mid2}{Continue reading the main story}

Advertisement

\protect\hyperlink{after-mktg}{Continue reading the main story}

\hypertarget{site-index}{%
\subsection{Site Index}\label{site-index}}

\hypertarget{site-information-navigation}{%
\subsection{Site Information
Navigation}\label{site-information-navigation}}

\begin{itemize}
\tightlist
\item
  \href{https://help.nytimes.com/hc/en-us/articles/115014792127-Copyright-notice}{©~2020~The
  New York Times Company}
\end{itemize}

\begin{itemize}
\tightlist
\item
  \href{https://www.nytco.com/}{NYTCo}
\item
  \href{https://help.nytimes.com/hc/en-us/articles/115015385887-Contact-Us}{Contact
  Us}
\item
  \href{https://www.nytco.com/careers/}{Work with us}
\item
  \href{https://nytmediakit.com/}{Advertise}
\item
  \href{http://www.tbrandstudio.com/}{T Brand Studio}
\item
  \href{https://www.nytimes.com/privacy/cookie-policy\#how-do-i-manage-trackers}{Your
  Ad Choices}
\item
  \href{https://www.nytimes.com/privacy}{Privacy}
\item
  \href{https://help.nytimes.com/hc/en-us/articles/115014893428-Terms-of-service}{Terms
  of Service}
\item
  \href{https://help.nytimes.com/hc/en-us/articles/115014893968-Terms-of-sale}{Terms
  of Sale}
\item
  \href{https://spiderbites.nytimes.com}{Site Map}
\item
  \href{https://help.nytimes.com/hc/en-us}{Help}
\item
  \href{https://www.nytimes.com/subscription?campaignId=37WXW}{Subscriptions}
\end{itemize}
