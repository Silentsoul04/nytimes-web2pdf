\href{https://www.nytimes.com/crosswords}{Crossword}

How to Solve The New York Times Crossword

Intro

\begin{itemize}
\tightlist
\item
  How to Get Started
\item
  A Step Further
\item
  Cracking the Harder Clues
\item
  Bonus: Crossword Themes
\end{itemize}

Open share options

Bookmark the page

Save for Later

Close your current view

Cancel

Share

\begin{itemize}
\tightlist
\item
  Facebook Icon
\item
  Twitter Icon
\item
  Pinterest Icon
\item
  Email Icon
\end{itemize}

\begin{itemize}
\tightlist
\item
  How to Get Started
\item
  A Step Further
\item
  Cracking the Harder Clues
\item
  Bonus: Crossword Themes
\end{itemize}

\hypertarget{how-to-solve-the-new-york-times-crossword}{%
\section{How to Solve The New York Times
Crossword}\label{how-to-solve-the-new-york-times-crossword}}

By \href{https://www.twitter.com/NYTimesWordplay}{Deb Amlen}

Illustrations by Elena Xausa. Animations by Lorenzo Fonda.

Share on Facebook

Share on Twitter

Share in an email

Bookmark the page

Would you like to improve your mental flexibility, learn a few
interesting things every day and establish bragging rights among your
friends? Solving crossword puzzles is like mental yoga --- both
challenging and relaxing at the same time. Plus, it's fun, especially if
you appreciate words and wordplay as much as I do. I believe that with
patience and practice anyone can learn to solve crosswords. Once you
master a few basic strategies, you'll find that puzzle-solving is not
only possible, but highly addictive. So let's get solving!

\hypertarget{how-to-get-started}{%
\subsection{How to Get Started}\label{how-to-get-started}}

\begin{quote}
``Solving crosswords eliminates worries. They make you a calmer and more
focused person.'' ~-- Will Shortz, New York Times crossword editor and
NPR puzzle master.
\end{quote}

If you've ever picked up a crossword puzzle and said to yourself, ``I am
not smart enough'' or ``I don't have a big enough vocabulary for this,''
please allow us to let you in on a little secret: ~

\begin{quote}
\emph{A crossword puzzle is not a test of intelligence, and solving is
not really about the size of your vocabulary. Becoming a good solver is
about understanding what the clues are asking you to do.}
\end{quote}

You can absolutely learn to do that. We're here to let you in on some of
the rules that most clues follow, and to teach you how to read those
clues so that they become easier to solve. It would be impossible to
cover every instance of clueing, but we can get you up and running. ~

We've even included some tips and encouragement from the puzzle pros to
help keep you motivated, like our very funny friend, Megan Amram, a
writer for television shows like ``The Simpsons'' and ``The Good
Place.'' Ms. Amram is a devoted solver and has also
\href{https://www.nytimes.com/crosswords/game/daily/2016/06/23}{made a
puzzle}that ran in The New York Times.

\begin{quote}
\emph{``I understand how intimidating starting the crossword can be, but
the bottom line is, believe in yourself. YOU ARE SMART ENOUGH TO DO THE
PUZZLE. Look at me. I do The New York Times crossword puzzle every day,
and I once tried to shoot a basket on the wrong hoop when I was on my
6th grade basketball team. Crossword puzzles are not about intelligence,
they are about keeping your mind nimble and knowing what the sneaky
trickster Will Shortz is asking of you. Show Will Shortz who's boss by
attempting the puzzle!''} ---
\href{https://twitter.com/meganamram}{Megan Amram}
\end{quote}

Got it? Now let's get started.

First, decide how you want to solve: Are you a print-only person? Do you
enjoy the extra help that comes from
\href{https://www.nytimes.com/crosswords}{playing on the web} or
\href{https://www.nytimes.com/crosswords/apps}{on-the-go with the app}?
If you subscribe, you get access to all the daily puzzles and the
archive.~ And once you log in, you can save your progress across all the
digital platforms.

\hypertarget{start-with-the-monday-puzzles}{%
\subsubsection{Start With the Monday
Puzzles}\label{start-with-the-monday-puzzles}}

The Monday New York Times Crosswords are the easiest, and the puzzles
get harder as the week goes on. Solve as many of the Mondays as you can
before pushing yourself to Tuesday puzzles. You can thank us later.

This is probably a beginning solver's most common mistake.You know what
it's like: You have some downtime on a Saturday and you look around for
something to pass the time. Your officemate keeps bragging about his
ability to finish The New York Times Crossword. You hate your
officemate.

So, not to be outdone, you pick up the paper or download our app and
turn to the Saturday puzzle. How hard could it be?

\textbf{The Saturday crossword is actually the hardest puzzle of the
week.} Mondays have the most straightforward clues and Saturday clues
are the hardest, or involve the most wordplay. Contrary to popular
belief, the Sunday puzzles are midweek difficulty, not the hardest.
They're just bigger.

A typical Monday clue will be very straightforward and drive you almost
directly to the answer. Don't believe us?~

\hypertarget{the-mini}{%
\subsubsection{The Mini}\label{the-mini}}

\href{https://www.nytimes.com/crosswords/game/special/monday-easy}{}

\hypertarget{monday-level-easy}{%
\paragraph{Monday-Level Easy}\label{monday-level-easy}}

Try this Monday-level mini crossword and show yourself what you can do.
Just click here to get your solve on.

Solve it!

Just to drive the point home, let's take a look at the difference
between a Monday clue and a late-week clue for a popular crossword
entry.~

\textbf{The Monday Clue:} \emph{``Nabisco cookie,'' ``Cookie with creme
filling'' or ```Twist, Lick, Dunk' cookie''}

\textbf{The Saturday Clue:} \emph{``Snack since 1912,'' ``It has 12
flowers on each side'' or ``Sandwich often given a twist''}

The answer to all of these clues is the same: ``OREO.'' Those delicious
sandwich cookies are so popular in crossword puzzles that they've been
dubbed by some as the ``official'' cookie of the crossword.

But we weren't kidding you. There is a \emph{big} difference between a
Monday puzzle clue and a Saturday puzzle one. Late-week clues might
require more specialized knowledge about these delicious treats.

If you're just getting started, \textbf{make your life easy and solve as
many Monday puzzles as you can}. Eventually, you'll be ready for more of
a challenge, and that's when you move on to the Tuesday puzzles.

\hypertarget{practice-makes-if-not-perfect-a-much-better-solver}{%
\subsubsection{Practice Makes, If Not Perfect, a Much Better
Solver}\label{practice-makes-if-not-perfect-a-much-better-solver}}

Once you've learned some of the shorter answers and how they are clued,
you can almost be sure you'll see them again. The brain works in weird
and wonderful ways, and when you start solving crosswords consistently,
you will feel really good when you can say, ``Hey, I know that one!''

\begin{quote}
``Do more puzzles. The more you solve, the better you'll get. It's also
useful to read \href{https://www.nytimes.com/column/wordplay}{Wordplay}
and other puzzle blogs, which helped me internalize the tricks and
tropes of crossword clues while I was learning the ropes.'' ~-\/-
\emph{Dan Feyer, seven-time champion of the}
\href{http://www.crosswordtournament.com/}{\emph{American Crossword
Puzzle Tournament}}
\end{quote}

And don't worry if you make a mistake. Everyone makes mistakes. That's
what erasers and the backspace key are for. It even happens to advanced
solvers, so don't let it get you down if you don't know something or
need to change an answer.~

\begin{quote}
``Try to solve as much as you can in each puzzle, and don't stress when
you can't finish one. For the ones you don't know, if it's something way
out of your knowledge comfort zone, look it up and read a bit more about
it. It's fun, really! There is no shame in missing an answer or not
finishing the puzzle. The key is learning what you missed. The more
puzzles you solve, the easier it gets.'' --- \emph{Howard Barkin, 2016
champion of the American Crossword Puzzle Tournament}
\end{quote}

\hypertarget{find-your-gimmes}{%
\subsubsection{Find Your `Gimmes'}\label{find-your-gimmes}}

When you start a puzzle, get comfortable, pour yourself a glass of your
favorite beverage --- it's important to stay hydrated --- and then scan
the clue list before solving.~

\textbf{Pick out the clues that are meant to be the easiest} \textbf{and
tackle them first.} See anything you definitely know? Those are your
`gimmes.' Are there any fill-in-the-blanks clues? Those are usually the
easiest.~

Trust us: There's no better boost to your solving ego than to be able to
fill in a few entries right off the bat.

You already know more than you think you do. To borrow a sports term, a
puzzle or individual clue on topics that you know well is said to be
``in your wheelhouse.'' You'll be able to find at least a few entries in
each puzzle that you know.

\begin{quote}
``Good crosswords connect to everything in life.'' \emph{-- Will Shortz}
\end{quote}

Fill-in-the-blank clues tend to be easier because they have definite
answers.~

Don't believe us? Try these clues that are designed to be easy for most
people:

1. Clue: **``Winnie-the-\_\_\_''~**

\emph{Your brain knows the answer to this: It's POOH, the
``hunny''-loving bear from the stories by A. A. Milne}

Easy clues don't even have to be fill-in-the-blanks. Your brain will
fill them in even when there is no blank.

2. Clue: \textbf{``Actor Brad of `12 Years a Slave'''}

\emph{Somewhere in your travels, your brain must have noticed that actor
Brad PITT was in the award-winning movie ``12 Years a Slave.''}

Once you have a few answers in the puzzle, sit back and congratulate
yourself. You're solving!

Ready for another mini?~

\hypertarget{the-mini-1}{%
\subsubsection{The Mini}\label{the-mini-1}}

\href{https://www.nytimes.com/crosswords/game/special/fill-in-the-blank}{}

\hypertarget{fill-in-the-blanks}{%
\paragraph{Fill-in-the-Blanks}\label{fill-in-the-blanks}}

Try this puzzle, with all fill-in-the-blank clues, and watch your brain
work in ways you might not have expected:

Solve it!

\hypertarget{use-the-crossings-luke}{%
\subsubsection{Use the Crossings, Luke}\label{use-the-crossings-luke}}

\textbf{Tip:} You can confirm whether an answer is right by solving the
entries that cross it.

Let's look at an example of why it pays to work those crossings. You
might not see this in a Monday puzzle, but say the clue is ``Black
Halloween animal,'' and you have confidently written in ``CAT.''~

Then you look at the entry that crosses the first letter of CAT and the
clue is ``Honest \_\_\_ (presidential moniker).'' The answer to that one
is ABE, so CAT must be wrong.~

Now what do you do?

Stay flexible and remember that a BAT is \emph{also} a ``black Halloween
animal.'' Yes, it's tricky, but it's doable.~

Conversely, you can also work your way through an answer that you can't
get completely by solving the crossings. Once you have enough letters
filled in, take your best guess based on the pattern of letters you've
uncovered.

\begin{quote}
``The key to solving crosswords is mental flexibility. If one answer
doesn't seem to be working out, try something else.'' \emph{-- Will
Shortz}~
\end{quote}

\hypertarget{its-not-cheating-its-learning}{%
\subsubsection{It's Not Cheating, It's
Learning}\label{its-not-cheating-its-learning}}

\textbf{Tip:} Don't be afraid to look up answers. You'll become a better
solver for it.

Let me say something that may be controversial, but it needs to be said:
It's O.K. to look something up when solving a crossword.~

Crosswords are ultimately learning tools, whether you're learning some
trivia or an interesting new word or phrase. When you look something up,
you're learning so you'll know it for next time.~

Of course, some solvers may tell you that looking up the answer to a
clue is ``cheating,'' but to us, that way lies frustration and a path to
giving up. And that's no fun. Crosswords are a game, and games are
supposed to be fun.~

Still not sure looking things up is fair? Here is outright permission:

\begin{quote}
``It's your puzzle. Solve it any way you like.'' ---
\href{https://www.nytimes.com/interactive/2017/02/14/crosswords/new-york-times-crossword-timeline.html?_r=0\&mcubz=3}{Will
Weng, the second crosswords editor of The New York Times} (1969 - 1977)
\end{quote}

\hypertarget{take-a-break-if-you-get-stuck}{%
\subsubsection{Take a Break if You Get
Stuck}\label{take-a-break-if-you-get-stuck}}

We're big fans of the brain here, especially its incredible work ethic.
But even brains get tired, so if you are stuck at some point in the
puzzle, one of the best things you can do is put it down and take a
break from it for a while.~

I'm not sure how this works, but your brain will continue working on the
clue in the background while you go about your day. When you come back
to it, you might be surprised at the ``Aha!'' moment you experience when
you thought you didn't know the answer.

\hypertarget{solve-with-a-friend}{%
\subsubsection{Solve With a Friend}\label{solve-with-a-friend}}

\textbf{Tip:} Solving with another person can work to your advantage.~

Your wheelhouse might be stuffed with
\href{https://www.nytimes.com/interactive/2017/04/01/crosswords/CROSSWORDS-baseball-terms.html}{sports}
\href{https://www.nytimes.com/interactive/2017/04/19/crosswords/CROSSWORDS-sports-names.html}{trivia}.
Your BFF's wheelhouse might be crammed to the rafters with a deep
knowledge of
\href{https://www.nytimes.com/interactive/2017/07/19/crosswords/CROSSWORDS-operas.html?rref=collection\%2Fcolumn\%2Fwordplay}{opera}.
\emph{Vive la différence}, right?

You know things your friend doesn't know, and he or she knows things
that you don't know. That's roughly twice as much stuff that you can
solve, and it's a good excuse to spend time together.~

In fact, you never know when it could lead to something more,
\href{https://www.youtube.com/watch?v=73EDetVQzCs}{like a marriage
proposal.~}

\hypertarget{sit-back-and-enjoy-your-accomplishment}{%
\subsubsection{Sit Back and Enjoy Your
Accomplishment}\label{sit-back-and-enjoy-your-accomplishment}}

Yay, you! You've started solving The New York Times Crossword!~

Solving a New York Times crossword is not easy, but it should be
satisfying. Even if you only get a few answers the first few times, keep
on solving. It just gets easier -- and better -- from there.~

But don't limit yourself. Try to master the skills you need to get
started, and then push yourself to go further into the week. That's
where all of that devious, delicious wordplay is tucked into the clues,
and where the fun in solving crosswords lies.

Are you ready to take your solving a step further?

\hypertarget{words-to-know}{%
\subsubsection{Words to Know}\label{words-to-know}}

\textbf{Clue:} A crossword clue is a hint that the solver must decipher
to find the answer that is then entered into the puzzle grid. Clues are
not necessarily dictionary definitions; they can involve puns, anagrams
and other types of wordplay.~

\textbf{Crossing:} The intersection between an Across entry and a Down
one. Crosswords are intended to play fair with solvers, so a difficult
or obscure entry will ideally cross a more ``gettable'' one.

\textbf{Entry:} The answer to a clue that solvers write into the
crossword puzzle. Entries that are part of a theme are called --- wait
for it --- theme entries. \textbf{Fun fact:} In a typical American-style
crossword, an entry must have at least three letters.

\textbf{Fill:} A general term for the words or phrases that fill a
crossword. ~

\textbf{Fill-in-the-blank:} A clue that contains a blank where the
answer goes. One of the easiest types of clues to solve.~

\textbf{Mini crossword:} A 5x5 crossword offered by The New York Times.
For comparison, the size of a Times daily crossword is 15x15 and the
Sunday crossword is 21x21.

\hypertarget{a-step-further-how-to-read-crossword-clues}{%
\subsection{A Step Further: How to Read Crossword
Clues}\label{a-step-further-how-to-read-crossword-clues}}

Some good news about crossword puzzles: A clue and answer pairing will
always be fair, even if it takes solvers a while to see it. And who
doesn't \emph{love} being set up to win, even if it's after a mental
tug-of-war?

But how \emph{are} you supposed to win? How are you supposed to beat
your braggart of an officemate in a solving race? \textbf{The key is to
learn some easy-to-remember ways to read those devilish crossword
clues}.~

To keep things fair between constructor, editor and solver, most
crossword clues follow certain predictable ``rules.'' **** We're going
to let you in on some of those rules and, if you practice using the
Minis that are included in this guide, you will get a lot further in
your solving.

\hypertarget{must-match-clues}{%
\subsubsection{"Must Match" Clues}\label{must-match-clues}}

\begin{itemize}
\tightlist
\item
  Tense
\item
  Part of Speech
\item
  Plural
\item
  Language
\end{itemize}

\begin{itemize}
\item
  \textbf{If a clue is in a certain tense (such as past tense), then the
  answer has to be in that tense as well.} This is an easy rule to start
  with that will immediately improve your solving. For example, if you
  see the past tense clue ``Adored'' in a puzzle, the answer has to be
  past tense. So if the answer is a form of the word ``love,'' the
  answer would not be LOVE, LOVES or LOVING. It would be LOVED, because
  that's the past tense form.

  Give your new knowledge a try with this mini. Some clues and answers
  will be present tense, some will be past tense. Just remember to make
  sure that the tenses of each clue and its answer match. We won't spoil
  the answer, but as an example, note that the clue for 1-Across is in
  present tense and the clue for 5-Across is in past tense. Read the
  clues carefully!~

  \hypertarget{the-mini-2}{%
  \subsubsection{The Mini}\label{the-mini-2}}

  \href{https://www.nytimes.com/crosswords/game/special/tenses}{}

  \hypertarget{tenses}{%
  \paragraph{Tenses}\label{tenses}}

  Read carefully! We've put an ``*'' by the clues that adhere to the
  `verb tense' rule.

  Solve it!
\item
  \textbf{An answer's part of speech must match the clue's part of
  speech.} If a clue is primarily a noun, the answer will be a noun. If
  the clue is primarily a verb, the answer must be a verb. And so on.

  Take the word BOOK, for example. BOOK can be \emph{both} a noun and a
  verb, so you may see a noun clue or a verb clue for the word. The
  answer will be the same, but how you get there will be very different.

  What would be the answer for these two clues?

  ``Library unit''

  ``Make reservations''

  In both cases, the answer would be BOOK. But the first would lead to
  the \emph{noun} BOOK, that bound object with pages, while the second
  clue is for the \emph{verb} BOOK, because ``Make reservations'' is a
  verb clue.

  \hypertarget{the-mini-3}{%
  \subsubsection{The Mini}\label{the-mini-3}}

  \href{https://www.nytimes.com/crosswords/game/special/parts-of-speech}{}

  \hypertarget{part-of-speech}{%
  \paragraph{Part of Speech}\label{part-of-speech}}

  How well do you know your parts of speech? Try this mini on for size.

  Solve it!
\item
  \textbf{If a clue is plural, the answer has to be plural.}~

  There are a few different ways you might see plural answers clued:

  \begin{itemize}
  \tightlist
  \item
    The plural noun answer GRAPES might be clued ``They might be sour.''
    ``They'' indicates the plural.
  \item
    The plural noun answer STATES might be clued as ``North \emph{and}
    South Dakota,'' because the ``and'' indicates more than one state.
  \end{itemize}

  \textbf{Tip}: If you are struggling with an answer that has a plural
  clue, you can make yourself feel like you've at least got a grip on it
  by dropping the letter ``S'' in at the end. Then, come back when you
  have enough letters filled in from the crossings to solve the rest of
  the entry.

  \hypertarget{the-mini-4}{%
  \subsubsection{The Mini}\label{the-mini-4}}

  \href{https://www.nytimes.com/crosswords/game/special/plurals}{}

  \hypertarget{plurals}{%
  \paragraph{Plurals}\label{plurals}}

  See if you can keep your plurals matching. We've put an ``*'' by the
  clues that adhere to the `plurals' rule.

  Solve it!
\item
  No surprise here. **If you see a foreign language clue, expect a
  foreign language answer.~**

  If there is a non-English word or phrase in the crossword, the clue
  will signal it by either including a word or phrase in the same
  language, or by connecting the answer to a place where that language
  is spoken or a person who might speak it. For the most part, foreign
  words or phrases included in puzzles they are very common words that
  most people will know, providing they paid attention in their high
  school language classes.

  For example, the Spanish word ESTA, which means ``this'' or ``it is,''
  might be clued early in the week as ``It is, en español.'' Don't
  worry, the answer is in your brain somewhere. And remember: If you
  need to, take a break and come back to the puzzle. And work those
  crossings.

  You might also see ESTA clued as ``It is, in Ibiza'' or ``This, in
  Tijuana.''

  A tantalizing glimpse at the late week wordplay you can look forward
  to when you push past the Mondays: If the entry was in a deviously
  tricky late-week puzzle, you might see the clue: ``What this is in
  Spain.''~

  And no, the puzzle constructors and editors don't hate you. They just
  want you to stretch your mind a bit.

  \hypertarget{the-mini-5}{%
  \subsubsection{The Mini}\label{the-mini-5}}

  \href{https://www.nytimes.com/crosswords/game/special/foreign-words}{}

  \hypertarget{foreign-language}{%
  \paragraph{Foreign Language}\label{foreign-language}}

  Parlez-vous français? We've put an ``*'' by the clues that adhere to
  the `foreign language' rule to help.

  Solve it!
\end{itemize}

\hypertarget{more-types-of-crossword-clues}{%
\subsubsection{More Types of Crossword
Clues}\label{more-types-of-crossword-clues}}

\begin{itemize}
\tightlist
\item
  Cross-Referenced
\item
  Partner Clues
\item
  Abbreviations
\end{itemize}

\begin{itemize}
\item
  \textbf{Some clues can be split between different entries in order to
  connect the answers.}

  You might see clues that say ``See 17-Across,'' which, on the surface,
  is not very helpful. But it's an indicator that the answers to the
  clue you are looking at and the one at 17-Across are somehow related.
  All you have to do is follow the instructions.

  \hypertarget{the-mini-6}{%
  \subsubsection{The Mini}\label{the-mini-6}}

  \href{https://www.nytimes.com/crosswords/game/special/cross-references}{}

  \hypertarget{cross-references}{%
  \paragraph{Cross-References}\label{cross-references}}

  Get some practice on these clues. We've put an ``*'' by the clues that
  adhere to the `cross-reference' rule.

  Solve it!
\item
  **A partner clue wants you to come up with a word that is typically
  partnered with another word, separated by the word ``and.''~**

  An example of this would be the clue, ``Partner of live'' for which
  the answer would be LEARN, because the popular phrase is ``Live and
  learn.'' Occasionally, the word ``and'' is not needed as a separator,
  as in the clue ``Partner of neither,'' for the answer NOR, because
  ``neither'' and ``nor'' are partnered in sentences.

  So let your mind wander and try to think of possible partners for the
  word in the clue. If you need to come up with an answer for the clue
  ``Partner of sciences,'' for example, the answer would be either
  ``Sciences and \_\_\_'' or ``\_\_\_ and sciences.'' In this case, the
  answer is ARTS, for ``arts and sciences.''

  \hypertarget{the-mini-7}{%
  \subsubsection{The Mini}\label{the-mini-7}}

  \href{https://www.nytimes.com/crosswords/game/special/partners}{}

  \hypertarget{partners}{%
  \paragraph{Partners}\label{partners}}

  Partner up with this mini. We've put an ``*'' by the clues that adhere
  to the `partner' rule.

  Solve it!
\item
  \textbf{In general, solvers will see some sort of signal that an entry
  is an abbreviation, an initialism or an acronym, although that signal
  may vary.} Some of the signals you see might include:

  \begin{itemize}
  \tightlist
  \item
    \textbf{Abbr.}, as in ``Lawyer: Abbr.'' for the answer ATTY
  \item
    \textbf{Abbreviating a word in the clue itself}, as in ``Trial
    fig.,'' short for ``Trial figure,'' also for ATTY, or ``Elephant
    grp.'' for GOP
  \item
    \textbf{For short}, as in ``Free TV spot, for short'' for PSA
  \item
    \textbf{In brief} , as in ``Individual rights defender, in brief''
    for ACLU
  \end{itemize}

  Occasionally, you will see abbreviations in the clues that have
  nothing to do with abbreviations in the answers. For reasons of
  succinctness, some words in clues are nearly always abbreviated, like
  "U.S." for United States, "U.N." for United Nations, "N.F.L." for
  National Football League, ~or "V.I.P." for very important person.

  \hypertarget{the-mini-8}{%
  \subsubsection{The Mini}\label{the-mini-8}}

  \href{https://www.nytimes.com/crosswords/game/special/abbreviations}{}

  \hypertarget{abbreviations}{%
  \paragraph{Abbreviations}\label{abbreviations}}

  Can you tacking this mini, briefly? We've put an "*" by all the clues
  that adhere to this rule.

  Solve it!

  **In general, solvers will see some sort of signal that an entry is an
  abbreviation, an initialism or an acronym, although that signal may
  vary.**Some of the signals you see might include:

  \begin{itemize}
  \tightlist
  \item
    \textbf{Abbr.}, as in ``Lawyer: Abbr.'' for the answer ATTY
  \item
    \textbf{Abbreviating a word in the clue itself}, as in ``Trial
    fig.,'' short for ``Trial figure,'' also for ATTY, or ``Elephant
    grp.'' for GOP
  \item
    \textbf{For short}, as in ``Free TV spot, for short'' for PSA
  \item
    \textbf{In brief} , as in ``Individual rights defender, in brief''
    for ACLU
  \end{itemize}

  Occasionally, you will see abbreviations in the clues that have
  nothing to do with abbreviations in the answers. For reasons of
  succinctness, some words in clues are nearly always abbreviated, like
  "U.S." for United States, "U.N." for United Nations, "N.F.L." for
  National Football League, ~or "V.I.P." for very important person.
\end{itemize}

\hypertarget{go-big}{%
\subsubsection{Go Big}\label{go-big}}

You now know enough clue secrets to get you most, if not all the way,
through an early week puzzle. That wasn't so tough, right?

Ready to try it? Will Shortz has selected 11 of his favorite Monday
puzzles from our archive for you, so you can get some practice. Don't
worry: You've got this.~

\href{https://www.nytimes.com/interactive/2017/02/14/cricticschoice/crosswords/remarkable-puzzles-will-shortz.html?_r=0}{}

\hypertarget{11-remarkable-crosswords-for-new-solvers}{%
\paragraph{11 Remarkable Crosswords for New
Solvers}\label{11-remarkable-crosswords-for-new-solvers}}

Ready to conquer The New York Times Crossword? Here's a sampling ---
hand-picked by Will Shortz --- to get you started.

Remember, don't be afraid to make mistakes. Learn from them and move on.
Above all, have fun.~

\hypertarget{words-to-know-1}{%
\subsubsection{Words to Know}\label{words-to-know-1}}

\textbf{Constructor:} The person who creates the crossword puzzle. The
constructor develops the theme if it's a themed puzzle, fills the puzzle
with interlocking words or phrases and writes the clues. In other
countries, the constructor might be called a compiler or setter.

\textbf{Cross-Reference:} Two entries whose clues are linked to each
other. In the Mini Crossword below, 1-Across and 4-Across are
cross-referenced.

\textbf{Grid:} The diagram of black and white squares that contains the
entries. Most daily puzzles are 15 squares by 15 squares and most Sunday
puzzles are 21 squares by 21 squares.

\textbf{Interlock:} The crossing of entries inside the grid. An
American-style crossword has ``all-over interlock,'' which means that no
part of the grid can be completely cut off by the black squares. In
theory, a solver should be able to solve from any section of a puzzle to
another without having to stop.

\textbf{Symmetry:} Standard crosswords have 180 degree rotational
symmetry, which means that if you turn a crossword puzzle upside down,
the black and white squares will still be in the same place.

Occasionally, left-right or mirror symmetry is used instead.

\textbf{Word count}: The word count is the number of answers in a
crossword. In a New York Times crossword, a themed 15x15 square puzzle
typically has no more than 78 answers. A 15x15 themeless puzzle has a
maximum word count of 72 answers. A 21x21 Sunday puzzle usually has no
more than 140 answers.

\hypertarget{cracking-the-harder-clues}{%
\subsection{Cracking the Harder Clues}\label{cracking-the-harder-clues}}

\begin{quote}
``It's kind of fun to be pulled out of your comfort zone. Puzzles are a
kind of nonthreatening way to remind us that there is still mystery in
the world.'' --- Jason Silva, ``Wonder junkie'' and former host of
NatGeo's ``Brain Games''
\end{quote}

The real fun and challenge of crossword solving lies in cracking the
really tough clues.

From being duped by magic tricks to pondering those brain-twisting
crossword clues, why do humans
\href{https://wordplay.blogs.nytimes.com/2013/09/12/fool-for-being-fooled/}{so
love being fooled?} Is it the rush of the ``Aha!'' moment, when their
expectations are defied?

Of course, some solvers say simply that figuring out a really tricky
clue makes them feel smart. And there is nothing wrong with that,
especially when you're learning how to decipher them.~

Here are some more clue types to conquer:~

\hypertarget{more-clue-types}{%
\subsubsection{More Clue Types}\label{more-clue-types}}

\begin{itemize}
\tightlist
\item
  Clues With a "?"
\item
  Slang
\item
  "Quotes" and {[}Brackets{]}
\item
  Veiled Capitals
\end{itemize}

\begin{itemize}
\item
  \textbf{Tip:} A question mark at the end of a clue means that it
  should not be taken at face value. The answer is likely to be a pun, a
  misdirection, or some other type of wordplay. Ask yourself if the
  words in the clues might have different meanings from the ones you
  think they do.

  These are the forehead slappers of crossword clues. The puzzle maker
  and the editor are playing around with words and phrases in a clue
  like this, so free your mind up and think about other ways the words
  in the clues might be used. \emph{Question everything.}

  Solvers will often see a ``?'' in clues that are part of a crossword
  theme involving wordplay, but any clue involving wordplay could
  conceivably have a ``?'' A clue might receive a ``?'' depending on on
  how ``stretchy'' it is --- that is, how far a clue is from being
  factually true. It may also depend on which day of the week the clue
  appears: Early week puzzles might get a ``?'' to help you, whereas
  later week clues might not.

  But the most important thing to remember is that this is when it
  really gets fun. So it's worth it to practice.

  Here are two examples:

  \begin{itemize}
  \tightlist
  \item
    \textbf{``Fitness center?''} On the surface, this clue sounds like
    it's asking for a GYM, since a gym is also known as a fitness
    center. But what if we told you that the answer is a four-letter
    word? Can you think of any other way the words in the clues can be
    used?
  \end{itemize}

  *When you are trying to stay fit, it's important to work the center of
  your body, or your CORE.~*

  \begin{itemize}
  \tightlist
  \item
    \textbf{``Current events?''} Here's another clue that misdirects you
    from a popular phrase to something totally different. On the
    surface, you might think that the clue is asking you about the
    current events you read about in the newspaper, but think: What
    \emph{other} kinds of currents do you know? The answer is five
    letters long.~
  \end{itemize}

  *The answer is TIDES, because they are events that involve currents in
  the sea.~*

  Sneaky? Maybe. But we promise you'll learn to love this, and the more
  you practice solving, the easier it will be to spot these delicious
  opportunities to play with words and language.

  \hypertarget{the-mini-9}{%
  \subsubsection{The Mini}\label{the-mini-9}}

  \href{https://www.nytimes.com/crosswords/game/special/question-mark}{}

  \hypertarget{question-marks}{%
  \paragraph{Question Marks}\label{question-marks}}

  Any questions? Show us what you've got.

  Solve it!
\item
  \textbf{Tip:} The New York Times Crossword speaks to all ages. It pays
  to learn both older and more modern slang and vernacular. And you will
  certainly learn it if you solve the crossword, dawg.

  Language is a living, evolving thing, and the entries in the crossword
  tend to reflect that. Therein lies a challenge: Older solvers have to
  keep up with our changing language and younger solvers have to learn
  words that might have fallen out of favor long before they were born.

  People who have not yet learned to enjoy The New York Times Crossword
  tend to believe that it is a stodgy pursuit for older people, but the
  truth is, there are both modern and retro references in almost every
  puzzle. So while you might see the words MASHER (slang for a man who
  makes often unwelcome advances to women) and MOOLA (slang for money),
  you will also see BAE (slang for a boyfriend or girlfriend) and
  BROMANCE (slang for a close, platonic friendship between two men).

  \hypertarget{the-mini-10}{%
  \subsubsection{The Mini}\label{the-mini-10}}

  \href{https://www.nytimes.com/crosswords/game/special/slang}{}

  \hypertarget{slang}{%
  \paragraph{Slang}\label{slang}}

  ``Keep it 100'' with this slang mini.

  Solve it!
\item
  \textbf{Tip:} Clues in quotes and brackets will make it seem like your
  puzzle is talking to you. It's only a problem if you answer.

  A clue that is in quotes can be the title of a song, a movie or a
  book. But it can also mean something else: A clue in quotes that is
  something someone might say out loud has an answer that is a synonym
  for that verbalization. Here are a few examples of clues with possible
  answers:

  \begin{itemize}
  \tightlist
  \item
    \textbf{``Stop!'':} HALT, ENOUGH, HOLD IT, FREEZE
  \item
    \textbf{``That's the way the cookie crumbles'':} C'EST LA VIE
  \item
    \textbf{``I goofed!'':} MY BAD
  \end{itemize}

  You might also see clues in brackets. A clue in brackets suggests an
  answer that might be nonverbal:~

  \begin{itemize}
  \tightlist
  \item
    \textbf{{[}That's painful{]}:} GRIMACE
  \item
    \textbf{{[}I don't care{]}:} SHRUG
  \item
    \textbf{{[}!!!!{]}:} I'M SHOCKED
  \end{itemize}

  \hypertarget{the-mini-11}{%
  \subsubsection{The Mini}\label{the-mini-11}}

  \href{https://www.nytimes.com/crosswords/game/special/quotes-and-brackets}{}

  \hypertarget{quotes-and-brackets}{%
  \paragraph{Quotes and Brackets}\label{quotes-and-brackets}}

  Ready to have a conversation with your crossword puzzle? Try this one.

  Solve it!
\item
  \textbf{Tip:} Watch out for one of the most devious cluing traps in
  crosswords: Hiding a proper name at the beginning of a clue.

  Talk about tough cluing. This type of clue requires careful reading.

  When the crossword constructor and the editors are feeling
  particularly diabolical, you might see an innocent-looking clue like
  this for a three letter entry:

  \textbf{``Brave opponent''}

  Hey, someone is calling their adversary ``brave.'' That's nice. On the
  surface, this clue looks like a compliment, doesn't it?~

  Not this time. The answer to that clue is RED. Depending on the
  puzzle, it might also be MET.

  We know. That makes no sense at first glance, but don't give up.
  There's logic behind this, we promise.

  Let's take a closer look at the clue: The word ``Brave'' has a capital
  B because it is at the beginning of the clue, but that is \emph{not
  the only reason it's capitalized}. It has a capital B because it
  \emph{also} happens to be the name of a professional baseball player,
  an Atlanta Brave. The puzzle maker and the editors put it at the
  beginning of the clue to capitalize on (sorry), or take advantage of,
  the capital letter.

  And a three letter ``Brave opponent,'' or an opponent of an Atlanta
  Brave, could be a Cincinnati RED. It might also be a New York MET.

  As we said, there may be a lot of forehead slapping as you get into
  the late week puzzles, but hang in there. We promise you'll start to
  enjoy the lengths to which the puzzle makers and editors go to twist
  your brain.

  \hypertarget{the-mini-12}{%
  \subsubsection{The Mini}\label{the-mini-12}}

  \href{https://www.nytimes.com/crosswords/game/special/veiled-capitals}{}

  \hypertarget{veiled-capitals}{%
  \paragraph{Veiled Capitals}\label{veiled-capitals}}

  Time to practice again. Watch out for those capital letters!

  Solve it!
\end{itemize}

\hypertarget{rebuses}{%
\subsubsection{Rebuses}\label{rebuses}}

\textbf{Tip:} When you find yourself cursing at the puzzle because none
of your expected answers fit, it probably has a rebus element.

Solvers either love rebuses or they hate them.~

A rebus element can be a letter, number or symbol that represents a
word, but in many crosswords, \textbf{the rebus will be a word or group
of letters that need to be written inside} \textbf{a single square}*.*
Many solvers ask if they are supposed to be warned that a rebus exists
in a puzzle, and the answer is no, that's part of the fun of solving.
It's also frustrating if you don't figure out what's going on. (That's
also why rebuses are generally reserved for Thursday and Sunday puzzles,
says Joel Fagliano, the digital puzzles editor.)

Let's say you see a clue that reads ``Do-it-yourselfer's activity,'' and
the allotted space you are given for the answer is eight squares. Maybe
you guessed right off the bat (or you used those crossings!) that the
answer is HOME REPAIR. But HOME REPAIR would need 10 squares.~

So, start to consider the theme of the puzzle. If you tighten your belt
and squeeze multiple letters into a single square, you'll end up with
the word ``air'' in one square. This puzzle maker made the word AIR a
rebus element in eight squares, where the rebus worked for both the
Across and Down clues.~

And what was the purpose of squeezing AIR into one square? To make
``compressed AIR,'' of course. Now that's a nice Thursday theme.

Part of the fun is determining where in the entry the rebus belongs.
You'll really need to work the crossings to figure that out. If you are
solving in print, of course, filling in the rebus will simply be a
matter of writing small.~

If you are solving online or in one of our apps, however,
\href{https://www.nytimes.com/2017/06/01/crosswords/yes-you-can-write-more-than-one-letter-in-a-square.html?_r=0}{there
is an easy way to enter multiple letters in a single square.}

\hypertarget{the-mini-13}{%
\subsubsection{The Mini}\label{the-mini-13}}

\href{https://www.nytimes.com/crosswords/game/special/rebus}{}

\hypertarget{rebus}{%
\paragraph{Rebus}\label{rebus}}

Read up on how to put more than one letter in a square above and then
practice with this mini puzzle.

Solve it!

\hypertarget{clues-that-use-heteronyms}{%
\subsubsection{Clues That Use
Heteronyms}\label{clues-that-use-heteronyms}}

\textbf{Tip:} We've saved the most confounding type of clue for last.
Some clues use heteronyms to misdirect you. (We swear this is fun.)~

Heteronyms are two or more words that are spelled identically but have
different pronunciations and meanings, like ``minute'' (\emph{MIN-it}),
which is a unit of time, and ``minute'' (\emph{my-NOOT}), which might
mean tiny.~

Imagine how much that unique facet of language enthralls our puzzle
makers and editors. And they use it to their advantage.This might not
seem completely fair, but if you've been learning the tricks to
understanding the clues in the rest of this guide, it's well within the
bounds of fairness. You just have to learn to think like a constructor.

How about a short quiz to help you figure this out? Here are some
particularly sneaky heteronym examples and the number of letters in
their answers. We'd like you to guess those answers. You might want to
spend some time staring at them until the heteronym reveals itself.~

Remember, question \emph{everything}. Don't allow any word to go
unexamined.

\textbf{Heteronym Clues}~

\begin{enumerate}
\def\labelenumi{\arabic{enumi}.}
\tightlist
\item
  ``Minute, to a tot'' (4 letters)
\item
  ``Nice one'' (3 letters)
\item
  ``Light shower'' (5 letters)
\item
  ``Polish person'' (5 letters)
\item
  ``Refuse work?'' (4, 3 letters)
\item
  ``Kitchen drawer?'' (5 letters)
\item
  ``Begin at the beginning?'' (8 letters)
\item
  ``Knockout number?'' (5 letters)
\item
  ``One of them does?'' (4 letters)
\item
  ``Moving supply'' (5 letters) ~\emph{(This one is Will Shortz's
  favorite. He says it fooled everyone!)}
\end{enumerate}

Are you getting the hang of this yet? These are not easy, but keep
staring, because that ``Aha!'' moment is really worth it.~

Still stuck? Here are the answers with the reasoning behind them:

\textbf{Answers}~

\begin{enumerate}
\def\labelenumi{\arabic{enumi}.}
\tightlist
\item
  \textbf{ITTY}: In this case, as in the example above, the word
  ``minute'' (pronounced `MIN-it') is not meant to be a unit of time. It
  is the ``minute'' (`my-NOOT') meant to mean ``small,'' and a tot might
  use the word ITTY.
\item
  \textbf{UNE}: ``Nice one'' is a familiar phrase, but instead of the
  word ``nice'' meaning ``good,'' this ``Nice'' is capitalized because
  it is really the city in France. And the article ``one'' in French is
  the word UNE.
\item
  \textbf{PRISM}: ~Yes, we know. Bear with us. Typically, a ``Light
  shower'' would be a mild rainstorm. In this case, the word ``shower''
  is not pronounced using the ``ow'' sound. It's pronounced using the
  ``oh'' sound, as in something that shows light. And that would be a
  PRISM.
\item
  \textbf{WAXER}: Not ``Polish'' as in someone from Poland.'' ``Polish''
  as in someone who buffs things.
\item
  \textbf{JUNK ART}: To ``Refuse work'' \emph{could} mean to turn down a
  job. Or, it could mean working with refuse or garbage, which means
  that one is creating JUNK ART.
\item
  \textbf{AROMA}: Everyone knows what a ``Kitchen drawer'' is, right?
  It's that thing you keep your utensils in. Except when it's something
  that lures people --- or draws them --- into the kitchen. And that
  ``drawer'' would be an AROMA.
\item
  \textbf{MENACHEM}: Remember when we talked about capitalized words at
  the beginnings of clues? ``Begin at the beginning'' is a common idiom,
  but in this clue the word ``Begin'' doesn't mean to start; it refers
  to the former prime minister of Israel, MENACHEM Begin. `` \ldots{} at
  the beginning'' is an indicator that the clue is looking for what goes
  \emph{before} ``Begin.''
\item
  \textbf{ETHER}: On the surface, a ``Knockout number'' might be a
  really dazzling song and dance number. But there are other ways of
  knocking someone out and there are other meanings to the word
  ``number,'' even if they are diabolical. In this case, we are knocking
  someone out and numbing them with the use of the gas ETHER.
\item
  \textbf{DEER}: ``One of them does?'' sounds like a statement about
  compliance. But when pronounced with a long ``O,'' the word ``does''
  in this clue is actually the word for female DEER.
\item
  \textbf{LITHE}: How long did you stare at the word ``supply'' before
  you got it? Not to worry, it took us a long time, too. A ``Moving
  supply'' (pronounced suh-PLY) could be a box, but we need a five
  letter word. What if we freed our minds up and thought about the word
  ``supply'' in a different way? Could it be `SUP-lee' as in a
  ``supple'' move? Sure it could. To move supply is to be LITHE.
\end{enumerate}

\hypertarget{you-can-do-it}{%
\subsubsection{You Can Do It!}\label{you-can-do-it}}

Kudos to you for hanging in there with us! If you've been practicing,
you should be able to tackle a midweek puzzle at least, if not a
later-in-the-week puzzle. All it takes is dedication to solving and
learning. Oh, and a willingness to have fun. Because as we said,
crosswords are a game, and games are meant to be fun.

\hypertarget{quiz}{%
\subsubsection{Quiz}\label{quiz}}

\href{https://www.nytimes.com/interactive/2017/11/28/crosswords/crosswordese-quiz.html?_r=0}{}

\hypertarget{how-well-do-you-know-your-crosswordese}{%
\paragraph{How Well Do You Know Your
Crosswordese?}\label{how-well-do-you-know-your-crosswordese}}

Take this quiz to familiarize yourself with some of the more common
obscure crossword entries.

Try it!

\hypertarget{bonus-crossword-themes}{%
\subsection{Bonus: Crossword Themes}\label{bonus-crossword-themes}}

\begin{quote}
``Discovering a crossword's theme is often half the fun of solving. And
once you nail one answer in a puzzle's theme, figuring out the other
long answers is usually much easier.'' --- Will Shortz
\end{quote}

There is so much more to a crossword puzzle than just a list of clues
and space for you to write in the answers. We want you to be able to
wring every drop of enjoyment out of your puzzle. That's one reason you
should know about crossword themes. A crossword theme is like bonus
content; it is an extra puzzle-within-the-puzzle for you to solve.~

Not all crosswords have themes, but for those that do, finding them will
help you be a better solver. That's another good reason to know about
them.

\hypertarget{what-is-a-theme}{%
\subsubsection{What Is a Theme?}\label{what-is-a-theme}}

Some crosswords contain a set of entries that all have something in
common. Puzzle makers have a knack for spotting oddities in our
language, and when they can put enough of the same kind of oddity
together, they have developed a theme set. As the solver, you not only
get to do the crossword puzzle, you also get to piece together the
theme.~

The number of types of themes you might see in crosswords is nearly
infinite, so we can't describe them here. But most commonly they involve
playing with words.

They can use puns, anagrams, hidden words, common elements, letters
added to familiar phrases to make new phrases, and much more. Some
puzzle themes have visual themes. Be on the lookout for these, because
they can be amazing.

\hypertarget{why-is-there-a-theme}{%
\subsubsection{Why Is There a Theme?}\label{why-is-there-a-theme}}

A theme is an extra bit of entertainment that can also be a solving
aid.~

Once you understand the theme and can guess what the other theme entries
might be, you will have a leg up on solving the rest of the puzzle.
Think of it as the frame of a house; \textbf{the crossword's theme is
the basis on which the rest of the puzzle is built.}

\hypertarget{where-is-there-a-theme}{%
\subsubsection{Where is There a Theme?}\label{where-is-there-a-theme}}

Themes can be placed anywhere in the crossword grid, depending on the
creativity of the puzzle constructor. But most commonly it will be in
the longest Across and Down entries.

Let's look at one type of easy theme you might see in a Monday puzzle.

The entries \textbf{BOULDER} DAM, \textbf{ROCK} LOBSTER, \textbf{PEBBLE}
BEACH and \textbf{DUST} JACKET all involve some sort of stone --- and
the size of the stone \textbf{(BOULDER → ROCK → PEBBLE → DUST)} gets
smaller as you go from the top of the grid to the bottom. Below is how
these answers were clued in this puzzle. The theme clues are pretty
straightforward --- fittingly for a Monday puzzle --- although the clue
for DUST JACKET is playing with us. Even so, it's not that hard,
especially if you work the crossings.~

\begin{itemize}
\tightlist
\item
  \textbf{BOULDER} DAM: ``Colorado River landmark dedicated by F.D.R.''
\item
  \textbf{ROCK} LOBSTER: ``Novelty B-52's song with the lyric `Watch out
  for that piranha''
\item
  \textbf{PEBBLE} BEACH: ``California locale of several golf U.S.
  Opens''
\item
  \textbf{DUST} JACKET: ``One covering a big story?''
\end{itemize}

Most solvers don't fill in a theme entry without first solving some of
the crossings, so don't worry if you don't know them right off the bat.
But do notice that these particular entries cover a range of topics:
Topography, pop culture, sports and, well, a pun about book covers.
There's something for nearly everyone.~

Some themes change part of a familiar word or phrase to make a pun.
Here's one that changes an `S' to an `SH' at the start of the second
word of a familiar phrase to turn it into something completely
different: At 17-Across, for example, BEST SELLER becomes BEST SHELLER.
Three more long Across answers work similarly.

In this puzzle, the theme entries were clued to make you think. They
describe the punny phrase, not the one it's based on, so it's up to you
to figure that out. Here's how this puzzled was clued:

\begin{itemize}
\tightlist
\item
  BEST \textbf{SHELLER}: ``Winner of a pea-preparing contest?''
  (Original phrase: BEST SELLER)
\item
  PICK UP \textbf{SHTICKS}: ``Lotharios' lines in a singles bar?''
  (Original phrase: PICK UP STICKS)
\item
  ALL DAY \textbf{SHUCKER}: ``One preparing corn for long hours?''
  (Original phrase: ALL DAY SUCKER)
\item
  MARRY IN \textbf{SHAM}: ``Phony wedding?'' (Original phrase: MARRYIN'
  SAM)
\end{itemize}

Now imagine opening your Sunday New York Times Magazine to the crossword
and seeing a museum come to life. Here's a Sunday puzzle from 2009 that
celebrated the 50th anniversary of the Solomon R. Guggenheim Museum in
New York City:

In this puzzle, the black squares imitate the spiral shape of the halls
of the Guggenheim Museum, and works of art that hang in the museum can
be found throughout the puzzle by artist name, along with the name of
the museum and other bonus theme content.

The Times has even run puzzles where solvers had to write the theme
\emph{outside} the grid. Talk about thinking outside the box! If you're
feeling daring,
\href{https://www.nytimes.com/crosswords/game/daily/2006/04/01}{you can
try one of those here}.~

There is so much wonderful variety in New York Times crossword themes.
These examples are just to get you started, but once you dive in and
start solving, prepare to be surprised by the incredible creativity of
the puzzle makers.

\hypertarget{about-the-author}{%
\subsection{About the Author}\label{about-the-author}}

Deb Amlen is the columnist and editor of
\href{https://www.nytimes.com/column/wordplay}{Wordplay,} the crossword
column of The New York Times. She believes that, with enough peer
pressure, anyone can learn to solve crosswords and enjoy them.

Special thanks to: Sam Ezersky, Joel Fagliano, Jason Silva, Megan Amram,
Dan Feyer, Howard Barkin and Will Shortz.~

Twitter:
\href{https://www.twitter.com/NYTimesWordplay}{@NYTimesWordplay}

\begin{itemize}
\tightlist
\item
  \href{https://www.nytimes.com}{NYTimes.com}
\item
  \href{//www.nytimes.com/spotlight/guides}{Guides}
\item
  \href{https://nyt.qualtrics.com/jfe/form/SV_7VuAQJbpWqaxzUh?AGENT_ID=}{Send
  us Feedback}
\end{itemize}

\begin{itemize}
\tightlist
\item
  \href{http://www.nytimes.com/privacy}{Privacy Policy}
\item
  \href{http://www.nytimes.com/ref/membercenter/help/agree.html}{Terms
  of Service}
\item
  \href{http://www.nytimes.com/content/help/rights/copyright/copyright-notice.html}{©
  The New York Times Company}
\end{itemize}
