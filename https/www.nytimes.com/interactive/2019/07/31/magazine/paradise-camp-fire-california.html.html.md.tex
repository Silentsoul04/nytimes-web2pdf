 **NYTimes.com no longer supports Internet Explorer 9 or earlier. Please
upgrade your browser.
\href{http://www.nytimes.com/content/help/site/ie9-support.html}{LEARN
MORE »}

**Sections

**Home

**Search

\hypertarget{the-new-york-times}{%
\subsection{\texorpdfstring{\href{http://www.nytimes.com/}{The New York
Times}}{The New York Times}}\label{the-new-york-times}}

\hypertarget{-magazine-}{%
\subsubsection{\texorpdfstring{
\href{https://www.nytimes.com/section/magazine}{Magazine}
}{ Magazine }}\label{-magazine-}}

 \href{https://www.nytimes.com/section/magazine}{Magazine} \textbar{}`We
Have Fire Everywhere': Escaping California's Deadliest Blaze

**Close search

\hypertarget{site-search-navigation}{%
\subsection{Site Search Navigation}\label{site-search-navigation}}

Search NYTimes.com

**Clear this text input

Go

\url{https://nyti.ms/319ZT9d}

\hypertarget{site-navigation}{%
\subsection{Site Navigation}\label{site-navigation}}

\hypertarget{site-mobile-navigation}{%
\subsection{Site Mobile Navigation}\label{site-mobile-navigation}}

\hypertarget{we-have-fire-everywhere-escaping-californias-deadliest-blaze}{%
\section{`We Have Fire Everywhere': Escaping California's Deadliest
Blaze}\label{we-have-fire-everywhere-escaping-californias-deadliest-blaze}}

For eight hours last fall, Paradise, Calif., became a zone at the limits
of the American imagination --- and a preview of the American future.

\hypertarget{we-have-fire-everywhere-escaping-californias-deadliest-blaze-1}{%
\section{`We Have Fire Everywhere': Escaping California's Deadliest
Blaze}\label{we-have-fire-everywhere-escaping-californias-deadliest-blaze-1}}

\hypertarget{we-have-fire-everywhere}{%
\subsection{`We Have Fire Everywhere'}\label{we-have-fire-everywhere}}

For eight hours last fall, Paradise, Calif., became a zone at the limits
of the American imagination --- and a preview of the American future.

 \textbf{By Jon MooallemPhotographs by Katy Grannan} JULY 31, 2019

\textbf{T}he fire was already growing at a rate of one football field
per second when Tamra Fisher woke up on the edge of Paradise, Calif.,
feeling that her life was no longer insurmountably strenuous or
unpleasant and that she might be up to the challenge of living it again.

She was 49 and had spent almost all of those years on the Ridge --- the
sweeping incline, in the foothills of California's Sierra Nevada, on
which Paradise and several tinier, unincorporated communities sit.
Fisher moved to the Ridge as a child, married at 16, then raised four
children of her own, working 70-hour-plus weeks caring for disabled
adults and the elderly. Paradise had attracted working-class retirees
from around California since the 1970s and was beginning to draw in
younger families for the same reasons. The town was quiet and
affordable, free of the big-box stores and traffic that addled the city
of Chico in the valley below. It still brimmed with the towering pine
trees that first made the community viable more than a century ago. The
initial settlement was poor and minuscule --- ``Poverty Ridge,'' some
called it --- until a new logging railroad was built through the town in
1904 by a company felling timber farther uphill. This was the Diamond
Match Company. The trees of Paradise made for perfect matchsticks.

Like many people who grow up in small communities, Fisher regarded her
hometown with affection but also exhaustion. All her life, she dreamed
of leaving and seeing other parts of the world, not to escape Paradise
but so that she could return with renewed appreciation for it. But as
the years wore on, she worried that she'd missed her chance. There had
been too many tribulations and not enough money. She was trapped.

Then again, who knew? That fall, Fisher was suspended in a wide-open and
recuperative limbo, having finally ended a five-year relationship with a
man who, she said, conned her financially, isolated her from her family
and seized on her diagnoses of depression and a mood disorder to make
her feel crazy and sick and insist that she go on disability. ``What I
thought was love,'' she said, ``was me trying to buy love and him
stealing from me.'' But now, a fuller, bigger life seemed possible.
She'd tried community college for a semester. And just recently, she got
together with Andy, a big-hearted baker for the Chico public-school
system, who slipped out of her bed earlier that Thursday morning to
drive down the hill to work. Fisher was feeling grounded again: happy.
It was odd to say the word, but it must have been true because there she
was, getting out of bed at 8 a.m. --- early for her --- energetically
and without resentment, to take her two miniature schnauzers and Andy's
lumbering old mutt into the yard to pee.

She stepped out in her slippers and the oversize sweatshirt she slept
in. She smelled smoke. The sky overhead was still faintly blue in spots,
but a brown fog, forced in by a hard wind, was rapidly smothering it.
``I've been here so long, it didn't even faze me,'' Fisher said. Small
wildfires erupted in the canyons on either side of Paradise every year.
But then the wind gusted sharply and a three-inch piece of burned bark
floated lazily toward her through the air like a demonic moth. Fisher
opened her hand and caught it. Bits of it crumbled in her palm like
charcoal. She took a picture and texted it to her sister Cindy
Christensen. ``WTF is happening,'' she wrote.

Cindy knew about wildfires. In fact, she'd spent every summer and fall
fixated on fire since the ``fire siege'' of 2008, when Paradise was
threatened by two blazes, one in each of the canyons alongside it. One
morning, as the Humboldt Fire approached from the east, the town ordered
more than 9,000 people to evacuate as a precaution, Cindy among them.
But when Cindy pulled out of her neighborhood, she instantly hit
gridlock. An investigation determined that it took nearly three hours
for most residents to drive the 11 miles downhill.

Sitting in traffic that morning, Cindy felt viscerally unsafe. Ever
since then, she obsessively tracked the daily indicators of high-fire
danger on the TV weather reports and with apps on her phone. ``It
consumed me,'' Cindy said. She spent many nights, unable to sleep,
listening to the wind plow out of the canyon and batter her roof. Many
days, she refused to leave home, worried a fire might blow through her
neighborhood before she could return for her pets. She didn't just sign
up to get the county's emergency alerts on her phone; she bought her own
police scanner.

It pained Tamra to see her sister fall apart every fire season; Cindy
seemed irrational --- possessed. It was hard to take her seriously.
``That's just Cindy,'' Tamra would say. Now, standing with her phone in
one hand and the charred bark in the other, Tamra needed Cindy to be
Cindy and tell her what to do.

``Evacuate,'' Cindy wrote back.

``Answer me!!'' Tamra texted again. ``It's raining ash and bark.''
Neither realized that some texts weren't being received by the other.
Then the power went out, and Tamra, who had dropped her cellular plan to
save money and could only use her phone with Wi-Fi, was cut off from
communicating with anyone.

``Leave, T. Paradise is on fire,'' Cindy was texting her. ``Leave!!''

By then, Cindy was almost off the Ridge, bawling in her car from the
stress and dread. Forty-five minutes earlier, she learned that a fire
had sparked northeast of town, and she immediately didn't like the
scenario taking shape. The relative humidity that morning, the wind
speed and direction, which would propel the fire straight toward
Paradise --- it was all very bad. ``In my mind, I pictured exactly what
happened,'' she explained. She'd spent years picturing it, in fact. She
left right away.

This time, there was no traffic; Cindy says she saw only two other cars
the whole way down. Later, she spotted her home in aerial footage of
Paradise on the local news. Her aboveground swimming pool was
unmistakable. Nearly everything else had burned into a ghostly black
smudge.

\hypertarget{--}{%
\subparagraph{❈ ❈ ❈}\label{--}}

By the time Fisher got in her yellow Volkswagen, the sky had transformed
again: It was somehow both shrouded and glowing. Many other residents
had learned to keep a ``go bag'' packed by the door, with water,
medications and copies of important documents; a woman from the local
Fire Safe Council, a volunteer known affectionately as the Bag Lady,
held frequent workshops demonstrating how to pack one. But Fisher was
indecisive and moving inefficiently. It had taken her nearly 40 minutes
to commit to leaving, wrangle the dogs and scramble to grab a few
haphazard possessions.

It was now 8:45. So many calls were being placed to 911 that a
dispatcher interrupted one man reporting a fire alongside Skyway Road
--- the busiest street in Paradise and the town's primary evacuation
route --- with a terse, ``Yeah, sir, we have fire everywhere.''
Officials had started issuing evacuation orders about an hour earlier;
Fisher's neighborhood was among those told to clear out first. Her
street was plugged with cars. A thick line of them crept forward at the
end of her driveway.

There are five routes out of Paradise. The three major ones spread south
like the legs of a tripod, passing through the heart of town and
continuing downhill toward Chico and the valley below. Fisher lived in
the northern part of town, on the easternmost leg of the tripod, Pentz
Road; she rented a bedroom from a woman who worked at a nursing home in
town. It baffled her to see that all the cars in front of her house were
heading north on Pentz, cramming themselves away from the center of
Paradise, away from the valley, and further uphill. The opposite lane,
meanwhile, was totally empty. It seemed obvious to Fisher that, if the
fire was approaching from somewhere in the canyon behind her house,
there would be plenty of Paradise left in which to safely wait it out.
So she pushed across the traffic, into the empty lane. But she barely
went 100 yards before a driver sitting in the jam alongside her rolled
down his window and explained that Pentz was blocked up ahead.

``Great,'' Fisher muttered. As she turned around and took her place in
line, she wished the man good luck.

``You, too,'' he said.

She was recording everything on her phone, compelled by some instinct
she would strain to make sense of later. She wanted people to know what
happened to her and presumed, nonsensically, that her phone would
survive even if she didn't. Maybe, too, she wanted someone to be with
her while it happened. Her phone created the illusion of an audience; it
was the best she could do.

It was suddenly much darker. Everyone had their headlights on. The sky
was blood red in places but waning into absolute black. The smoke column
was collapsing on them: The plume from the wildfire had billowed upward
until, at about 35,000 feet, it froze, became heavier, and fell
earthward again. Outside Fisher's passenger-side window, the wind
snapped an American flag in someone's yard so relentlessly that it
seemed to be rippling under the force of some machine. Then, a mammoth
gust kicked up, spattering the street with pine needles. It sounded like
a rainstorm and, when it subsided, bright orange embers appeared beside
Fisher's car: trails of pinhole lights, like fairies, skittering low
over the shoulder, chasing each other out of the dry leaves, then
capering off and vanishing in front lawns.

Fisher noticed a minivan struggling to merge just ahead --- people
weren't letting the driver in. She stopped to let it through, then
suddenly screamed: ``Oh, my God! There's a fire!'' She yelled it again,
out her window, as though she worried she were the only one seeing it:
the tremendous box of bright, anarchic flame where there used to be a
home.

It was 9:13 a.m. Fisher had been in her car for nearly half an hour and
traveled altogether nowhere; in fact, the burning house appeared to be
only a few doors down from her own. There was a second structure aflame
now. The fires were multiplying rapidly.

``I don't want to die!'' Fisher shouted. The mood had shifted. People
started honking. Fisher honked, too. She began to sob and scream, to
open her car door and lean her head out, asking what she should do.
Later, she felt embarrassed. She would see so many YouTube videos of
people calmly piloting their cars through the flames. There was
\href{https://www.youtube.com/watch?reload=9\&time_continue=166\&v=KkkxpOX8Jgc}{one
guy who went viral}, singing to his 3-year-old daughter as he honked and
swerved, commenting on the encroaching inferno as though it were an
interactive exhibit at a science museum. (``Be careful with that fire!''
the girl says adorably. The father replies, ``I'm going to stay away
from it, O.K.?'') It didn't make sense to Fisher that she would be the
only person screaming. Even the three dogs with her were silent, though
two of them were deaf and mostly blind and the third was shivering, eyes
locked open, too shocked to make a sound. ``I'm scared!'' Fisher
shouted. ``Somebody!''

``O.K., calm down,'' a voice called. The person urged her to turn around
again. She did and suddenly, still crying wildly, found herself shooting
south again, through the other, wide-open lane of Pentz, following a
white truck with a Butte County Fire Department decal on it. She tailed
the fireman intently, coasting past one burning house after another.
Some were being steadily, evenly devoured; others angrily disgorged
flames straight up from their roofs. Fisher knew the people who lived in
many of these houses --- this was her neighborhood. ``This is Pentz
Road!'' she yelled as she drove. ``These are people's homes.'' Then
added: ``I'm sorry. I am so sorry!''

When she got to the corner of Pearson Road, a major east-west artery,
she saw someone directing cars to take the right turn, where she and the
fireman found they could accelerate even more, winding along S-curves
through a wooded area that was almost entirely aflame. Fires speckled
the slopes along Pearson so that, in the dark, the hillside looked like
a lava flow. ``It's so hot,'' Fisher said. ``Keep going! Keep going!''
But then, they shot around another curve and the fireman's brake lights
came on. They had hit a wall of cars, across both lanes.

``No!'' Fisher yowled. ``What did I do?''

She was silent for a moment. Then something started beeping. It was the
low-fuel alert. She was almost out of gas, though it ultimately wouldn't
matter. Moments later, her car caught fire.

\textbf{A}fterward, you could feel your mind grinding against what
happened, desperate to whittle it down into a simple explanation of what
went wrong, who should be blamed, what could be learned. There were many
credible answers, specific mistakes to call out. But it was easy to
worry that, given the scale of this particular disaster, the principal
takeaways might be only humility and terror.

From the start, the Camp Fire was driven by an almost vengeful-seeming
confluence of circumstances, many of which had been nudged into
alignment by climate change. Paradise had prepared for disasters. But it
had prepared merely for disasters, and this was something else. In a
matter of hours, the town's roads were swamped, its emergency plans
outstripped. Nine of every 10 homes were destroyed and at least 85
people were dead. Many were elderly, some were incinerated in their cars
while trying to flee and others apparently never made it that far.

It was all more evidence that the natural world was warping, outpacing
our capacity to prepare for, or even conceive of, the magnitude of
disaster that such a disordered earth can produce. We live with an
unspoken assumption that the planet is generally survivable, that its
tantrums are infrequent and, while menacing, can be plotted along some
hazy, existentially tolerable bell curve. But the stability that
American society was built around for generations appears to be eroding.
That stability was always an illusion; wherever you live, you live with
risk --- just at some emotional and cognitive remove. Now, those risks
are ratcheting up. Nature is increasingly finding a foothold in the
unimaginable: what's not just unprecedented but also hopelessly far
beyond what we've seen. This is a realm beyond disaster, where
catastrophes live. Fisher wasn't just trapped in a fire; she was trapped
in the 21st century.

By way of analogy, Paradise's emergency-operations coordinator, Jim
Broshears, later described giving fire-safety tutorials at elementary
schools, back when he was the town's fire chief, teaching second and
third graders that if there's fire at their bedroom door, they should go
out the window, and vice versa. ``Inevitably,'' Broshears told me,
``there's the kid who goes, `What if there's fire at the door \emph{and}
the window?' '' And no matter what alternative Broshears provided, the
kid could always push the story line one step further.

``At some point, they've painted you into a corner and, well, do I tell
an 8-year-old kid, `In that case, you're going to die?' Do you tell a
community, `If this particular scenario hits, a bunch of you are going
to die?' Is that appropriate? I don't know the answer.'' He added, ``I
think that people are going to conclude that now.''

\hypertarget{---1}{%
\subparagraph{❈ ❈ ❈}\label{---1}}

Fisher saw the first flames skitter in the depression where her
windshield met the hood. She opened her car door again and leaned her
head out. Embers burned tiny holes in her leggings. She was yelling,
asking if anyone had water. A contractor in a pickup behind her
hollered: ``You don't need water. You need to get in my truck.'' He
beckoned her and all her animals over.

Fisher wedged herself among the tools and paperwork scattered on the
man's front seat, two dogs on top of her and the largest at her feet. As
they inched forward, she took a picture of her burning car and was
crushed to realize that she had just abandoned the few possessions she'd
managed to save, including the ashes of her big brother, Larry, who, 10
years earlier, died suddenly in his sleep.

``I'm Tamra,'' she told the man driving.

``I'm Larry,'' he said.

The coincidence was too much: Fisher started crying again.

Larry Laczko wore sleek, black-rimmed glasses and a San Francisco Giants
cap and seemed, to Fisher, almost preternaturally subdued, speaking with
the slow resignation of a man enduring ordinary traffic on an ordinary
Thursday morning. Laczko and his wife lived on the Ridge for 15 years,
then migrated to Chico in 2010, after raising their two kids. For years,
Laczko worked at Intel, managing 60 employees, traveling constantly.
Then, one Saturday, his wife told him to clean the windows of their home
in Paradise --- and to clean them \emph{well} this time. Laczko did some
research, geeked out a little and wound up ordering a set of
professional-grade tools from one of the oldest window-washing supply
companies in the United States. His wife was pleased. Soon, he was
washing windows every weekend, toddling around the Ridge with his tools,
getting to know his neighbors and friends of friends. ``I liked the
work, the instant gratification of a dirty window turning clean,''
Laczko explained, ``but it was the interaction with people that I
loved.'' That was 16 years ago. He quit his job and has run his own
window-washing company ever since.

Laczko was on Pearson Road by chance --- or because of his own
stupidity. In retrospect, he conceded, either assessment was fair. His
mother-in-law lived in Quail Trails Village, a nearby mobile-home and
R.V. park. She was 88 and used a walker. Laczko's wife, who was nearby
that morning, had already got her out. But Laczko wanted to be helpful.
He recently installed an automatic lift chair for his mother-in-law and
remembered how, after the 2008 evacuation, many people wound up
displaced from the Ridge for days; it would be nice for his
mother-in-law to have that chair. So he drove up the hill and cut across
on Pearson, only to be turned around by police. Backtracking, he smacked
into the traffic that had formed behind him: a blockade of cars, barely
moving and every so often, as with Fisher's Volkswagen, suddenly
sprouting into flame.

When Fisher climbed into Laczko's truck, the seriousness of his
predicament was only beginning to catch up with him. What sounded to
Fisher like extraordinary calmness was actually extraordinary focus: He
was scanning his surroundings, updating his map of everything that was
on fire around him --- that tree; the plastic fender of that S.U.V. ---
while also taking a mental inventory of the back of his pickup, gauging
how likely each item was to catch.

``We're getting out of here,'' Laczko told Fisher. He projected enough
confidence that he reassured himself, just slightly, as well.

\hypertarget{---2}{%
\subparagraph{❈ ❈ ❈}\label{---2}}

But clearly he and Fisher were stuck. Thousands of people were, on
choked roadways all over the Ridge, each sealed in his or her own saga
of agony, terror, courage or despair. It was like the 2008 evacuations,
but far more serious --- the gridlock, cinched tighter; the danger,
exponentially more acute --- and also harder to stomach, given all the
focus on avoiding those problems in the 10 years since.

After the 2008 fires, the county created a fifth route off the Ridge,
paving an old gravel road that wound through mountains to the north.
Paradise vigorously revamped and expanded the emergency plans it had in
place. The town was carved into 14 evacuation zones; these were
reorganized to better stagger the flow of cars. Paradise introduced the
idea of ``contraflow,'' whereby traffic could be sent in a single
direction across all lanes of a given street if necessary. Maps and
instructions were mailed to residents regularly. There were evacuation
drills, annual wildfire-preparedness events and other, more meticulous
layers of internal planning too. Paradise's Wildland Fire Traffic
Control Plan identified, for example, 12 ``priority intersections''
where problems might arise for drivers leaving each evacuation zone and
stipulated how many orange cones or human flaggers would ideally be
dispatched to each.

``The more you study the Camp Fire,'' says Thomas Cova, a University of
Utah geographer who has analyzed wildfire evacuations for 25 years,
``the more you think: This could have been way worse. Way worse.'' Cova
called Paradise ``one of the most prepared communities in the state.'' A
recent \href{https://apnews.com/6f621c1c54734d0b95d374556c2cf5c0}{USA
Today-California Network investigation} found that only six of
California's 27 communities at highest risk for fire had robust and
publicly available evacuation plans.

One architect of Paradise's planning was Jim Broshears, who had spent
the bulk of his 47-year career as an emergency planner and firefighter
struggling to mitigate his community's idiosyncratically high risk of
disaster. After the Camp Fire, Broshears confessed that, in his mind,
the upper limit of harrowing scenarios against which he'd been defending
Paradise was the 1991 Tunnel Fire in the Oakland hills --- a wildfire
that consumed more than 2,900 structures and killed 25 people: ``I'll be
honest,'' he told me, ``we simply didn't see it being much worse than
that.'' Recently, Broshears showed me a copy of the Traffic Control Plan
in a big, thick binder and said, with admirable directness, ``It mostly
didn't work.'' Then he clacked the binder shut and insisted, ``That is
still going to work 98 percent of the time, though.''

The Los Angeles Times and other newspapers would later dig up many
\href{https://www.latimes.com/local/lanow/la-me-paradise-fire-evacuation-system-20181120-story.html}{city
planning mistakes and communication failures} that appeared to compound
the devastation on the morning of Nov. 8. The core of the problem was
that there just wasn't any time. The fire was moving so astonishingly
fast that, only a few minutes after Paradise started evacuating its
first zones, it was obvious the entire community would have to be
cleared.

There was no plan for evacuating all 27,000 residents of Paradise at
once. ``I don't think it's physically possible,'' Paradise's mayor, Jody
Jones, told me. For a town that size to build enough additional lanes of
roadway to make it possible, she added, would have seemed preposterous
and like a waste of taxpayer money had anyone proposed it. Our
communities, as they currently exist, were planned and built primarily
to be lived in, not escaped. Fully prioritizing evacuation could mean
ripping them apart.

Paradise evolved without any genuine planning at all: Three adjacent
communities just kept expanding until they merged. This produced a town
of tangled side streets and poorly connected neighborhoods, often with a
single outlet and many dead ends. ``In towns all up and down the Sierra,
we've got the same pattern,'' says Zeke Lunder of Deer Creek Resources,
which often contracts with the state on wildfire-mapping projects. ``I
think it's inevitable that this will happen again.''

That morning in Paradise, streets were blocked by fallen trees, disabled
cars or even fire blowing crosswise across them. Flaming roadside
vegetation slowed or halted traffic on major evacuation routes like
Skyway so that many of the cross streets that fed them, like Pearson,
backed up, penning other drivers defenselessly into the side streets
that fed them.

Just ahead of Fisher and Laczko, a woman named Lorena Rodriguez watched
flames absorb the space around her car. She reached for her phone to
tell her children goodbye, but then she reconsidered, worried the memory
of her frightened voice would permit her kids to more vividly imagine
her burning alive and keep imagining it for the rest of their lives.
This enraged Rodriguez --- that she had been put in a position to have
such a thought. So she decided to run, sprinting in a pair of Danskos,
threading the lanes of idling vehicles and moving faster on foot than
all of them. She kept expecting to find some obstacle blocking the road,
a reason for the traffic, but all she saw was more cars.

Rodriguez ran for two and a half miles, all the way west on Pearson
until she reached Skyway. She says the street was bumper to bumper most
of the way, the vehicles alongside her perfectly still. It was as if
time had stopped for everyone but her.

\hypertarget{---3}{%
\subparagraph{❈ ❈ ❈}\label{---3}}

Fisher was thinking about her father, a former fire captain who was
protective to the point of pitilessness. To teach his little girls not
to play with matches, he showed them gruesome photographs of bodies
extracted from houses that burned down.

Those pictures had been flashing through Fisher's mind all morning. Now,
on Pearson Road, she sensed she was inside one. She knew there had to be
people dying around her and Laczko: good people who wanted to live just
as much as she did --- surely, who wanted it more.

Fisher inhaled deeply to rein in her crying and told Laczko: ``I gotta
say something. I've tried to kill myself multiple times, and now, I'm
scared.'' It was true. She felt guilty about it. She also knew, in that
moment, that she wanted to live.

It had been all of 10 minutes since Laczko waved Fisher into his truck.
While some people might have recoiled from a stranger making this kind
of admission, Laczko didn't pass judgment or see Fisher as a burden. As
a kid, he went to parochial school, though the faith never took; he
asked too many questions. Still, he liked the way his wife talked about
spirituality, not God so much as a form of godliness that arises
whenever two human beings connect. In that moment, he told me, his only
thought was, This person needs to talk, and I can certainly listen.

After getting turned around on Pearson, Laczko instantly felt deflated
--- and then, a little foolish too. He was starting to reprimand himself
for driving into a fire. For what? A chair?

Fisher, meanwhile, was exhausted, having so far shouldered the
responsibility for her survival alone all morning. ``I just wanted to be
with somebody,'' she explained.

For Laczko, ``Something clicked --- now I had someone to be responsible
for.''

They were together now, but still trapped, and the windows of Laczko's
truck were getting hotter. Until then, the fires blooming erratically
around Paradise were spot fires, birthed from embers that the wildfire
sprayed ahead of itself as it grew. Now an impregnable riot of heat and
flame was cresting the hillside under Pearson Road. This was the fire
itself.

\hypertarget{---4}{%
\subparagraph{❈ ❈ ❈}\label{---4}}

\textbf{T}here's a dismaying randomness to how a megafire can start: The
tire on a trailer goes flat and scrapes against the pavement, producing
sparks; the D.I.Y. wiring job on someone's hot tub melts. (These were
the causes of
\href{https://www.nytimes.com/2018/07/31/us/carr-fires-california-explained.html}{the
2018 Carr} and
\href{https://www.latimes.com/local/lanow/la-me-ln-napa-wildfire-hot-tub-20160810-snap-story.html}{2015
Valley fires}, respectively. More than 300,000 acres burned, combined.)
But by now, there is also a feeling of predictability: In 2017, for
example, 17 of 21 major fires in California were started accidentally by
equipment owned by Pacific Gas and Electric (PG\&E), which, as
California's largest electrical utility, is in the precarious business
of shooting electricity through 175,000 miles of live wire, stitched
across an increasingly flammable state. Under state law, the company may
be liable for damage from those fires, whether or not the initial spark
resulted from its negligence. And so, PG\&E found itself looking for
ways to adapt.

Two days before the Camp Fire, as horrendously blustery and dry
conditions began settling on the Ridge and the risk of fire turned
severe, PG\&E began warning 70,000 of its electricity customers in the
area, including the entire town of Paradise, that it might shut off
their power as a precaution. This was one of the new tactics that the
company had adopted --- a ``last resort,'' PG\&E called it: In periods
of extreme fire danger, if weather conditions aligned to make any
accidental spark potentially calamitous, PG\&E was prepared to flip the
switch, preventively cutting the electricity from its lines. Life would
go dark, maybe for days --- whatever it took. It was clear that the
unforgiving environment in which PG\&E had been operating for the last
few years was, as the company put it, California's ``new normal.''

Wildfires have always remade California's landscape. Historically, they
were sparked by lightning, switching on haphazardly to sweep forests of
their dead and declining vegetation and prime them for new, healthier
growth. Noticing this cycle --- the natural ``fire regimes'' at work ---
Native Americans mimicked it, lighting targeted fires to engineer areas
for better foraging and hunting. But white settlers were oblivious to
nature's fire regimes; when blazes sprung up around their towns, they
stamped them out.

Those towns grew into cities; the land around them, suburbs. More than a
century of fire suppression left the ecosystems abutting them misshapen
and dysfunctional. To set things right, the maintenance once performed
naturally by fire would have to be conducted by state and federal
bureaucracies, timber companies, private citizens and all the other
entities through whose jurisdictions that land splinters. The approach
has been feeble and piecemeal, says William Stewart, a co-director of
Berkeley Forests at the University of California, Berkeley: ``Little
pinpricks of fuel reduction on the landscape.'' We effectively turned
nature into another colossal infrastructure project and endlessly
deferred its maintenance.

Then came climate change. Summers in Northern California are now 2.5
degrees hotter than they were in the early 1970s, speeding up
evaporation and baking the forests dry. Nine of the 10 largest fires in
state history, since record-keeping started in 1932, have happened in
the last 16 years. Ten of the 20 most destructive fires occurred in the
last four; eight in the last two. California's Department of Forestry
and Fire Protection, known as Cal Fire, expects that these trends will
only get worse. It's possible that we've entered an era of ``megafires''
and ``megadisturbances,'' the agency noted in its 2018 Strategic Fire
Plan. And these fires are no longer restricted to the summer and early
fall: ``Climate change has rendered the term `fire season' obsolete.''

Even deep into last fall, much of the landscape still seemed restless,
eager to burn. A bout of heavy rains that spring produced a record
growth of grasses around the Ridge --- the fastest-burning fuels in a
landscape. But then the rain stopped. By the time of the Camp Fire, in
November, there hadn't been any significant precipitation since late
May, and July had been California's hottest month on record: All that
vegetation dried out. ``Everything is here,'' explained a veteran
wild-land firefighter named Jon Paul. ``All you need is ignition.''

The Camp Fire glinted into existence around 6:15 that Thursday morning.
Cal Fire hasn't yet released its full investigation, but the available
evidence indicates that a hook on a PG\&E electrical tower near the
community of Pulga snapped, allowing a wire to spring free. The wire
flapped against the skeletal metal tower, throwing sparks into the wind,
most likely for a fraction of a second before the system's safety
controls could have flipped. Still, it was enough: The sparks started a
fire; the fire spread.

In the end, PG\&E chose not to de-energize its lines. Even with warm,
dry air gushing through the canyon early that morning, blowing 30 miles
per hour and gusting up to 51, the company claimed that conditions never
reached the thresholds it had determined would necessitate a shut-off.
``That revealed a failure of imagination on PG\&E's part,'' says Michael
Wara, who directs the Climate and Energy Policy Program at the Stanford
Woods Institute for the Environment. PG\&E was largely forced into the
position of having to shut off people's power in the first place, Wara
argues, because it failed for decades to invest in the kind of
maintenance and innovation that would allow its infrastructure to stand
up to more hostile conditions, as climate change gradually exacerbated
the overall risk. But now, Wara said, the decision to keep operating
that morning suggested that the company still hadn't fully accepted the
kind of resoluteness this new reality demanded. Three weeks earlier,
PG\&E instituted its first, and ultimately only, shutdown of the 2018
fire season, cutting electricity during a windstorm to nearly 60,000
customers in seven counties. It took two days to restore everyone's
power; citizens and local governments fumed. ``One has to wonder,'' Wara
says, ``if the negative publicity and pushback PG\&E received influenced
decision making on the day of the Camp Fire.'' (``We will not speculate
on past events,'' a PG\&E spokesman said in an email. ``The devastating
wildfires of the past two years have made it clear that more must be
done, and with greater urgency, to adapt and address the issue.'')

An even starker truth: It probably wouldn't have mattered. The lines at
that particular tower in Pulga wouldn't have been included in a shut-off
that morning anyway; PG\&E's protocols at the time appeared not to
consider such high-voltage transmission lines a severe risk. A shutdown,
however, would have de-energized other lower-voltage lines a few miles
west of the tower, where, shortly after the first fire started, some
vegetation, most likely a tree branch, blew into the equipment and
triggered a second blaze.

In a preliminary report, Cal Fire's investigators seemed to regard this
subsequent event as negligible, however. Within 30 minutes of igniting,
the second fire had been consumed by the first, which was ripping
through a fast-burning landscape, powered forward by its own metabolism
and pushed by the wind. It had advanced four miles and was already
swallowing the small town of Concow. The Camp Fire was moving too fast
to be fought.

\hypertarget{---5}{%
\subparagraph{❈ ❈ ❈}\label{---5}}

``It was pretty much complete chaos,'' Joe Kennedy said. Kennedy is a
Cal Fire heavy-equipment operator based in Nevada City, southeast of
Paradise. He was called to the Camp Fire at 7:16 that morning and
hurtled toward the Ridge with his siren on, in an 18-wheeler flatbed
with his bulldozer lashed to the back.

Kennedy is 36, a fantastically giant man with a shaved head and a
friendly face but the affect of a granite wall; he spoke quietly and, it
seemed, never a syllable more than necessary. He had operated heavy
equipment his entire adult life, working as a contractor in the same
small mountain towns around the Sierra where he grew up, then joined Cal
Fire in 2014, just before he and his wife had a child. He claimed his
supremely taciturn nature was a byproduct of fatherhood; until then, he
explained, he was a more reckless adrenaline junkie. But Kennedy loved
bulldozers, and he loved the rush of barreling toward a fire in one.
``Dozer driver'' seemed to be less his job description than his
identity, his tribe. ``In 10 years,'' he joked, ``they'll probably
consider it a mental disorder.''

Kennedy was dispatched to the Adventist Health Feather River hospital on
Pentz Road. By the time he arrived, spot fires were igniting everywhere.
The chatter on the radio was hard to penetrate. Now that he was in
position, Kennedy couldn't get in touch with anyone to give him a
specific assignment, so he fell back on his training and a precept known
as ``leader's intent'': If someone were to give him an order, he asked
himself, what would it be?

By then, the hospital staff had completed a swift evacuation of the
facility. Nurses later described doing precisely what they practiced in
their annual drills but at three or four times the speed: wheeling
patients through the halls at a sprint, staging everyone in the E.R.
lobby, then sorting all 67 inpatients into a haphazard fleet of
ambulances and civilians' cars arriving outside to carry them away. Many
didn't make it far. One ambulance, carrying a woman who had just had a
C-section and was still immobilized from the waist down, quickly caught
fire in the traffic on Pentz. Paramedics hustled the woman into a nearby
empty house. Others took shelter there as well. A Cal Fire officer,
David Hawks, mobilized them into an ad hoc fire brigade to rake out the
gutters and hose down the roof as structures on either side began to
burn.

Kennedy caught snippets on his radio about this group and others
hunkered in nearby houses. He'd found his assignment. He climbed into
his bulldozer, a colossal Caterpillar D5H that traveled on towering
treads like a tank and was outfitted with a huge steel shovel, or
``blade.'' It also had a pretty killer sound system, and as Kennedy
turned the ignition, the stereo automatically connected to his phone
through Bluetooth and started playing Pantera. Kennedy's technical skill
and experience as a heavy-machinery operator was formidable; so was his
knowledge of wild-land firefighting tactics. But, given the scale of
disaster unfolding around him, all that expertise now concentrated into
one urgent, almost blockheadedly simple directive: ``Take the fire away
from the houses.''

Of course, Kennedy had no idea which houses any of these people were in.
All around the hospital lay a sprawl of mostly ranch homes, packed
together on small, wooded lots. A great many were already burning, so
Kennedy homed in on the others and started clearing anything flammable,
or anything already in flames, away from them. Ornamental landscaping,
woodpiles, trees --- he ripped it all out of the ground, pushed it aside
or plowed straight through it, clearing a buffer around each home. He
worked quickly, brutally, unhindered by any remorse over the collateral
damage he was causing; it's impossible, he explained, to maneuver an
18-ton bulldozer between two adjacent houses and not scrape up a few
corners.

Before long, Kennedy lost track of exactly where he was; he hadn't even
bothered to switch on the GPS in his dozer yet. ``It seemed like
forever, but it was probably a half-hour,'' he said. ``I think I got
eight or nine houses. I made a pretty big mess.''

\hypertarget{---6}{%
\subparagraph{❈ ❈ ❈}\label{---6}}

Wildfires are typically attacked by strategically positioned columns of
firefighters who advance on the fire's head, heel or flanks like knights
confronting a dragon. If a fire is spreading too rapidly for such an
offensive, they instead work to contain it, drawing boundaries around
the blaze --- a ``big box,'' it's called. Work crews or bulldozers clear
vegetation and cut fire breaks to harden that perimeter. Aircraft drop
retardants. Everything in the big box can be ceded to the fire; if you
have to, you let it burn. But ideally, you hold those lines, and the
flames don't spread any farther.

As wildfires get fiercer and more unruly, firefighters aren't just
unable to mount direct attacks but are also forced to draw larger and
larger boxes to keep from being overrun themselves. ``The big box is a
lot bigger now,'' one Cal Fire officer explained. (He asked not to be
named, hesitant to publicly concede that ``our tactics need to
change.'') But this strategy breaks down when the fire is racing toward
a populated area. The extra space you would surrender to the fire might
contain a neighborhood of several hundred homes.

Wildfires aren't solid objects, moving in a particular direction at a
particular speed. They are frequently erratic and fluid, ejecting embers
in all directions, producing arrays of spot fires that then pull
together and ingest any empty space between them. On Nov. 8, the wind
was so strong that gusts easily lofted embers from one rim of the
Feather River Canyon to the other like a trebuchet, launching fire out
of the wilderness into Fisher's neighborhood.

As this swirl of live embers descended, like the flecks in a snow globe,
each had the potential to land in a receptive fuel bed: the dry leaves
in someone's yard, the pine needles in a gutter. Those kinds of fuels
were easy to find. It was November, after all, past the time of year
that wildfires traditionally start, and Paradise's trees had carpeted
the town with tinder. And every speck of flame that rose up in it had
the potential to leap into an air vent and engulf a home. Now it is a
spot fire --- a beachhead in the built environment, spattering its own
embers everywhere, onto other houses, rebooting the entire process.

Within two hours of the first spot fires being reported near Fisher's
house, others leapfrogged from one end of Paradise to the other. The
progression was unintelligible from any one point on the ground. As one
man who was at the hospital later told me, ``I thought that the only
part of Paradise that was on fire was the part of Paradise we were
looking at.'' And, as happened with Fisher, this generated a horrifying
kind of dissonance: scurrying away from the fire only to discover that
the fire was suddenly ahead of you and alongside you, too.

\hypertarget{---7}{%
\subparagraph{❈ ❈ ❈}\label{---7}}

``Take deep breaths,'' Laczko said.

Fisher had just told him about trying to kill herself. They were barely
moving. Embers darted by like schools of bioluminescent fish. Evergreen
trees alongside them burned top to bottom. These were the town's famous
pines, stressed from years of drought; the pitch inside was heating to
its boiling point and, the moment it vaporized, the length of the trunk
would flash into flame all at once. This became one of the more
nightmarish and stupefying sights that morning on the Ridge: giant trees
suddenly combusting.

The topography of that particular stretch of Pearson Road made it a
distinctly horrible place to be stranded. Beyond the guardrail to Fisher
and Laczko's left, a densely wooded ravine yawned open, with a stream
known as Dry Creek Drainage far below. Already, the spot fires and
burning trees on either side of the road were casting heat inward. But
as the mass of the wildfire moved in, the ravine appeared to create a
chimney effect, funneling flames up and over the street --- only to be
overridden periodically by the prevailing winds, which pushed the flames
back. Everyone on Pearson was caught in the middle.

A Cal Fire branch director, Tony Brownell, told me that he was
astonished to watch fire doubling back across Pearson, washing over the
same land it had just scorched, only the second such immediate
``reburn'' he witnessed in his 31-year career. This was about 15 minutes
before Fisher's car ignited. Brownell, it turns out, was the fireman in
the white pickup truck whom she initially followed into that gridlock
from back on Pentz Road. Brownell managed to escape quickly, but as he
turned his vehicle around and drove away, he told me, he looked at the
flames in his rearview mirror and thought, I just killed that girl.

``You'd think that people would just hurry up and go,'' Fisher said.

``There's no place to go,'' Laczko told her. ``They're trying. Cal
Fire's here to help.''

He could see a fire engine a few car-lengths ahead. After fighting to
weave forward, it, too, had been more or less swallowed by the same
intractable traffic. Laczko silently made the calculation that if his
own truck caught fire, he and Fisher would make a run for it and climb
inside to safety.

This would have been a mistake. The Cal Fire captain driving the fire
engine, John Jessen, later estimated that the outside temperature was
more than 200 degrees, the air swirling with lethally hot gases. Cars
were catching fire everywhere, and four drivers fled toward that fire
truck and, one after another, crammed themselves into the cab alongside
its three-man crew. When two more people came knocking, Jessen turned
them away --- no more room, he said. ``That was probably the worst thing
I've ever had to do,'' Jessen said later. ``I don't know if those people
made it to another car. I don't know what happened to them.''

This was Jessen's 24th fire season in California. He'd fought five of
the 10 most destructive fires in state history and was beginning to feel
beaten down. ``When I started this career 25 years ago, a 10,000-acre
fire was a big deal,'' he said. ``And it was a big deal if we weren't
able to do structure defense and the fire consumed five homes. We took
that to heart. We felt like we lost a major battle.'' Just moments
earlier, around the corner on Pentz, Jessen watched fire consume dozens
of homes within minutes. He was knocking on the door of another, to
evacuate any stragglers, when he saw the actual fire front for the first
time. It was already climbing the near side of the canyon, pounding
toward town. The wall of flame was 200 feet tall, he estimated, and
stretched for more than two and a half miles. That was the moment Jessen
scrambled back to his truck and told his crew it was time to move.

Now, marooned on Pearson, Jessen radioed for air support. Later, he
would seem embarrassed by this request, chalking it up to ``muscle
memory'': The smoke was too thick for aircraft to fly in. The paint on
his hood started burbling from the heat. Inside, the plastic on his
steering console was smoking; the stench of its off-gasing filled the
cab. The barrel-shaped fuel tanks beneath the doors were splashing
diesel around the truck; the brass plugs in their openings got so hot
they liquefied.

Jessen, meanwhile, was making a desperate calculation of his own: If
their truck caught fire, he decided, they would extinguish it quickly
and take off, saving the civilians aboard by pushing other cars out of
the way with the front of his fire engine.

Maybe this was the lowest point. The mega-fire overwhelmed every system
people put in place to fight or escape it; now it was scrambling their
consciences too. ``That's something I never imagined I would be thinking
about,'' Jessen confessed, ``pushing people closer to the fire so that I
could get out.''

Jessen sat there, watching for signs that his truck was about to catch
fire. Laczko sat watching his own truck, ready to run for Jessen's. Then
someone shouted, ``Let's go, let's go, let's go!'' Laczko saw a
bulldozer churn into view behind him, clobbering one burning car after
another.

\hypertarget{---8}{%
\subparagraph{❈ ❈ ❈}\label{---8}}

Joe Kennedy had been mashing through people's landscaping on Pentz when
he heard Jessen's distress call. There was no time for the standard,
numeric identifiers. ``John,'' Kennedy radioed. ``Where you at?''

He switched on the iPad in his dozer and found Jessen's position near
the corner of Pearson and Stearns Road. It was more than a mile away.
The dozer's maximum speed was 6.3 miles per hour. But Kennedy clipped
that distance by disregarding the right angles on his street map and
barging his bulldozer through backyards, then eventually barreling down
the steep, wooded incline overlooking Pearson and spilling, sloppily,
into the middle of the street.

He produced a spectacular ruckus as he pushed the machine down the hill
on its treads. From the road, it sounded like trees crashing --- and
some of it probably was. As Kennedy leveled off, he came upon a group of
people, including four nurses from the hospital in scrubs. They were
stranded in the middle of Pearson, battered by gusts of embers roaring
out of the ravine, buckling over, struggling to breathe and keep
walking. One nurse, Jeff Roach, was walking straight at Kennedy's
bulldozer, with his arms in the air. Later, Roach explained that he had
decided the bulldozer driver would either see him and rescue him and his
three friends, or would not see him, keep advancing and crush him under
the vehicle's treads. The burning in Roach's lungs was so bad, he said,
that he had made his peace with either outcome.

Kennedy stopped. Two of the nurses climbed aboard then scampered to
rejoin the others who had piled into a fire engine that appeared behind
him. Kennedy began fighting his way up Pearson, toward Jessen, but found
cars crammed into both lanes and the shoulder. Some people were idling
right beside other vehicles that were expelling fountains of flame.
Kennedy turned up the Pantera. He knew what he had to do: take the fire
away from the people.

He approached the first burning car and pushed it off the embankment and
into the ravine with his dozer blade, then backed up to discover a
flaming rectangle of asphalt underneath it. He drove through that,
pushed more cars. ``I was basically on fire,'' Kennedy said. A photo
later surfaced of an old Land Cruiser shoved so far up the adjacent
hillside that it became snared in some sagging power lines. ``That was
me,'' Kennedy explained with a noticeable quantum of pride.

At least one of the vehicles Kennedy was shoving around had a body in
it: Evva Holt, an 85-year-old retired dietitian who lived at Feather
Canyon Gracious Retirement Living, close to Fisher's house. Holt had
phoned her daughter that morning to come get her --- her daughter and
son-in-law lived nearby and frequently came to perform for Holt and the
other residents with their choral group --- but there was no time. An
independent caretaker named Lori LeBoa was readying to leave with a
103-year-old woman, and a police officer put Holt in her Chevy Silverado
as well. The three women wound up stuck on Pearson. As the fire curled
over LeBoa's pickup, she jumped out and handed off the older woman to
another driver. Turning back for Holt, she saw only fire and two arms
reaching out.

Months later, over coffee, I asked Kennedy if he remembered moving that
Silverado. He did. The memory seemed painful; he preferred not to talk
about it on the record, except to stress that it was clear that he
arrived too late to help whoever was inside.

I asked if he knew any details about the woman, if he wanted me to tell
him. ``I like the story in my head,'' he said.

\hypertarget{---9}{%
\subparagraph{❈ ❈ ❈}\label{---9}}

Kennedy opened enough space for the stranded drivers on Pearson to
maneuver and slowly advance. Moments earlier, one nurse who'd leapt into
his dozer accidentally knocked into his iPad, switching his GPS into
satellite view. Eventually, when Kennedy looked down at the map again,
his eyes locked onto a conspicuous, bare rectangle, free of any
vegetation or structures --- any fuels to burn. It was a large gravel
lot right near Jessen's fire engine; the firefighters just couldn't see
it through the smoke. Once Kennedy arrived, the firefighters began
herding the entire traffic jam --- more than a hundred cars, Jessen says
--- into that clearing.

``Pull over there,'' a firefighter hollered at Laczko and Fisher as they
crept uphill.

``And then what?'' Laczko asked.

``Hunker down and keep your windows rolled up.''

``Are you serious?'' Fisher erupted. She was hoping for a more
sophisticated plan.

``They wouldn't have put us here unless it was safer than where we
were,'' Laczko said.

He eased his truck into place, parallel with the others. Directly in
front of them, through the windshield, the frame of a large house burned
and burned. For a moment, it was quiet. Then Fisher broke down again,
very softly this time. ``I don't have anything,'' she said. ``I don't
have anything.''

\hypertarget{---10}{%
\subparagraph{❈ ❈ ❈}\label{---10}}

Fires are unique among natural disasters: Unlike earthquakes or
hurricanes, they can be fought, slowed down or thwarted. And virtually
every summer in Paradise, until that Thursday morning, they had been.
There was always trepidation as fire season approached but also
skepticism that evacuation would ever truly be necessary and worth the
hassle. ``I confess my sense of denial,'' said Jacky Hoiland, who had
lived in Paradise most of her life and worked for the school district
for 20 years. Initially, after hearing about the Camp Fire, she took a
look at the sky and then made herself a smoothie.

Still, even before the Camp Fire, many people in Paradise and around
California had started to look at the recent succession of devastating
fires ---
\href{https://www.sfchronicle.com/bayarea/article/Wine-Country-fires-first-fatal-hours-12278092.php}{the
Tubbs Fire}, the
\href{https://www.latimes.com/local/lanow/la-me-ln-thomas-fire-edison-cause-20190313-story.html}{Thomas
Fire}, blazes that ate through suburban-seeming neighborhoods and took
lives --- and intuit that our dominion over fire might be slipping.
Something was different now: Fire was winning, finding ways to outstrip
our fight response, to rear up recklessly and break us down. That
morning, in Paradise, there hadn't even been time for that fight
response to kick in. And the flight response was failing, too. Those who
study wildfire have long argued that we need to reshuffle our
relationship to it --- move from reflexively trying to conquer fire to
designing ways for communities to outfox and withstand it. And in a
sense, that's what was happening with Laczko and Fisher, though only in
a hasty and desperate way: Hunkered in that gravel lot, everyone was
playing dead.

After the fire, stories surfaced of people retreating into similar
so-called temporary refuge areas all over the Ridge: clearings that
offered some minimal protection or structures that could be easily
defended. One large group sheltered in the Paradise Alliance Church,
which had been scouted and fortified in advance as part of the town's
emergency planning. Another group sheltered outside a bar on Skyway and,
when it caught fire, scampered, en masse, to an adjacent building and
sheltered there. The Kmart parking lot became an impromptu refuge. So
did an antique shop called Needful Things. In Concow, one firefighter
instructed at least a dozen people to jump into a reservoir as the fire
approached.

The group on Pearson wasn't in the gravel clearing long, less than 10
minutes, it seems, from videos on Fisher's phone. Eventually, there was
a knock on Laczko's window. ``We're going to get out of here,'' a
firefighter said, though he didn't specify where they would go. Moments
later, another firefighter on a bullhorn shouted, ``We're going to go
toward the hospital.''

``Oh, shit,'' Laczko blurted.

``We're going back?'' Fisher said. She sounded both terrified and
incensed. The hospital was on Pentz Road, near where she started. The
trauma of the last two hours appeared to be flooding back.

Joe Kennedy led the way in his bulldozer, crawling through the thick
smoke on Pearson to batter any obstacles out of their way. The core of
the fire had passed, though it had left a kind of living residue
everywhere: All the wooden posts of a roadside metal barricade were
still burning, and shoals of flames dotted the road where Kennedy had
removed burning cars.

The cars were still burning, too, wherever he had deposited them,
belching solid black smoke as the caravan of survivors slowly passed.

``There's my car,'' Fisher said and turned to film it. Fire spouted from
its roof like the plume of a Roman helmet. ``It has my Raggedy Ann in
it!'' she said. The doll was one of the few things she grabbed before
evacuating. She had had it since she was 6 and had expected to be buried
with it one day. ``Oh, my God,'' she said. ``I'm crying over something
so stupid!''

\hypertarget{---11}{%
\subparagraph{❈ ❈ ❈}\label{---11}}

At the hospital, a fire alarm quacked robotically as a small
outbuilding, not far from where Laczko and Fisher were parked, expelled
smoke from behind a fence. A group of nurses had scavenged supplies from
the evacuated emergency room and erected a makeshift triage center under
the awning to treat any wounded trickling in. Laczko got out of his
truck to see how he could help.

The hospital campus was ringed and speckled with fire. Some of the men
were peeing on the little spot fires that danced in the parking lot's
landscaped medians. Still, the influx of firefighters that morning had
largely succeeded in defending the main building when the fire front
moved through. Eventually, a call went out on the radio that the
hospital campus was ``actually the safest place to be.''

Fisher and Laczko's group waited in the parking lot for close to three
hours. Then, those lingering fires nearby began to swell and expand,
threatening the hospital again. The firefighters were losing pressure in
their hoses. The nurses were told to pack everything up. The road out
was clear; they had a window in which it was safe to move. Everyone
would finally be driving off the Ridge.

As they pulled out of the parking lot, back onto Pentz Road, Laczko
noticed his eye-doctor's office burning top to bottom, directly in front
of them.

``It's gone,'' Fisher said.

``It's gone,'' Laczko said.

``It's gone,'' Fisher repeated. ``That house is gone! And that house is
gone!''

They went on gesturing at everything as they drove --- or rather, at its
absence: all the homes still burning and others that had already settled
into static masses of scrap and ash. As happens in any small town, every
part of Paradise was overlaid with memories and meanings; each resident
had his or her own idiosyncratic map of associations. As Fisher and
Laczko coasted down Pentz, they tried to reconcile their maps with the
disfigured reality in front of them, speaking the names of each
flattened side street, noting who lived there or the last time they had
been down there themselves. The iconic home at the corner of Pearson,
with the ornate metal fence and sculptures of lions, had been devoured:
``It used to be on the garden tour,'' Laczko said.

``Right here, that was my dog groomer's house.''

``My sister is just right up here.''

``Are these the people that used to have the Halloween stuff up?''

It was 1:45 p.m. Thirty-nine minutes later, and 460 miles away, a small
brush fire would be reported near a Southern California Edison
substation north of Malibu. Firefighters wouldn't contain the Woolsey
Fire until it had swallowed nearly 100,000 acres and 1,600 structures
and charged all the way to the Pacific, where it ran out of earth to
consume. This time, as photos surfaced, all of America could find
reference points on the map the fire had clawed apart: Lady Gaga
evacuated. Miley Cyrus's home was a ruin. The mansion from ``The
Bachelor'' was encircled and singed.

``Oh, God, it's all gone,'' Fisher said again. She gaped at the east
side of Pentz Road, facing the canyon, where there didn't appear to be a
single home left: just chimneys, wreckage, the slumping carcasses of
cars, everything dun-colored and dead.

\hypertarget{---12}{%
\subparagraph{❈ ❈ ❈}\label{---12}}

Five months after the Camp Fire, at the end of March, the wreckage in
Paradise was still overpowering: parcel after parcel of incinerated
storefronts, cars, outbuildings, fast-food restaurants and homes.
Patches of rutted pavement, like erratic rumble strips, still scarred
Paradise's roadways wherever vehicles had burned. On Pearson Road, I
knelt beside one and found a circular shred of yellow plastic, fused
into a ring of tar: a piece of Fisher's car. It was startling how
similar Paradise looked to when I first came, 10 days after the fire.
Except that it was spring now: Clusters of daffodils were blooming,
carefully arranged, bordering what had been fences or front steps.

That week, the city issued its first rebuilding permit, though roughly
1,000 residents were already back, somehow making a go of it, either in
trailers or inside the scant number of houses that survived, even as
public-health officials discovered that the municipal water system was
contaminated with high levels of benzene, a carcinogen released by the
burning homes and household appliances, then sucked through the pipes as
firefighters drew water into their hoses. Driving around at dusk one
evening, letting acre after acre of obliterated houses wash over me, I
spotted a lone little boy in what appeared to be the head of a
cul-de-sac --- it was hard to tell --- with heaps of houses all around
him. He was standing with his arms raised, like a victor or a king, then
he hopped back on his scooter and zipped away.

Jim Broshears, Paradise's emergency-operations coordinator, pointed out
that many of the homes still standing tended to be in clusters: ``A
shadow effect,'' he called it, where one property broke the chain of
ignitions --- maybe because its owners employed certain fire-wise
landscaping or design features, or maybe just by chance. It showed that,
while the destructiveness of any fire is largely random, there are ways
a community can collectively lower the odds. ``It's really a cultural
shift that requires people to look at their home in a different way,''
Broshears said: to see the unkempt azalea bush or split rail fence
touching your home as a hazard that will carry the next fire forward
like a fuse, not just to your house but also to the others around it ---
to recognize that everyone is joined in one massive pool of incalculable
and unconquerable risk.

The free market, meanwhile, has continued adjusting to that risk
according to its own unsparing logic. Insurance companies have steadily
raised premiums or even ceased to renew policies in many fire-prone
areas of California, as payouts for wildfire claims will now exceed \$10
billion for the second year in a row. Two months after the Camp Fire,
\href{https://www.nytimes.com/2019/07/11/business/energy-environment/wildfire-california-utilities.html}{PG\&E
filed for bankruptcy protection}. Then it announced, along with two of
California's other major utilities, that it would be expanding its
Public Safety Power Shutoff program this year. The company is now
prepared to preventively cut electricity to a larger share of its
infrastructure --- high-voltage wires, as well as lower-voltage ones ---
and across its entire range. Nearly five and a half million customers
could be subject to shutdowns at one time or another this summer,
``which is all of our customer base,'' a PG\&E vice president, Aaron
Johnson, told me: every single one. ``With the increasing fire risk that
we're seeing in the state,'' he added, ``and the increasing extreme
weather, this program is going to be with us for some time to come.''

In California, the prospect of life without electricity from time to
time --- a signature convenience of the 20th century --- has apparently
become an unavoidable, even sensible, feature of the 21st.

\hypertarget{---13}{%
\subparagraph{❈ ❈ ❈}\label{---13}}

How did it end? With smoke --- with colossal shapes of smoke gurgling
out of Paradise behind Laczko and Fisher as they glided downhill, and
with a stoic figure somewhere inside the smoke, single-mindedly grinding
through neighborhoods in his bulldozer, music blaring, chasing after
flames as they stampeded uphill, but mostly failing to get ahead of them
as he and every other firefighter labored to keep fire away from
structures that seemed, in the end, determined to burn.

The houses had revealed themselves: They were just another crop of
tightly clustered and immaculately dried-out dead trees, a forest that
had grown, been felled and milled, then rearranged sideways and hammered
together by clever human beings who, over time, came to forget the
volatile ecosystem that spawned that material and still surrounded it
now. Some of that wood most likely lived 100 years or more and had been
lumber for almost as long: a storehouse of energy that was now bursting
open, joining with the burning forests around the Ridge into a single,
furious outpouring of smoke --- ominous because it was dark and high
enough to challenge the sun, but also because it was largely composed of
carbon: an estimated 3.6 million metric tons of greenhouse gases that,
as seems to happen at least once every fire season lately, was more than
enough to obliterate the progress made by all of California's
climate-change policies in a typical year.

How did it end? With smoke --- with smoke that signaled the world that
Fisher knew at the beginning of the day was gone and that surely
signaled something just as grave for the rest of us. Within hours, and
for nearly two weeks after that, smoke would swamp the lucid blue sky
over the valley where Fisher was now heading; where, for weeks, she
would be afraid to be left alone and, for months, refuse to drive,
terrified by the sensation of slowing down in traffic, even momentarily;
where she found herself repeatedly checking the sky to make sure it
wasn't black; where she kept showering but swore she still smelled the
smoke on her skin. And before long, the smoke had floated all the way to
the coast, where it forced the city of San Francisco to close its
schools.

How did it end? It hasn't. It won't.

\begin{center}\rule{0.5\linewidth}{\linethickness}\end{center}

Jon Mooallem is a writer at large for the magazine. His book about the
great Alaska earthquake of 1964, ``This Is Chance!'' will be out early
next year. Katy Grannan is a photographer and filmmaker based in
Berkeley, Calif.
\href{https://www.nytimes.com/2018/12/12/magazine/oakland-warehouse-fire-ghost-ship.html}{She
last took photographs for the magazine for an article about the Ghost
Ship fire.}

Videos from Tamra Fisher

\subsection{}

\begin{itemize}
\item
  \href{https://www.nytimes.com/interactive/2018/11/18/us/california-camp-fire-paradise.html}{}

  \includegraphics{https://static01.nyt.com/images/2018/11/17/us/california-camp-fire-paradise-map-promo-1542504191694/california-camp-fire-paradise-map-promo-1542504191694-mediumThreeByTwo225-v2.jpg}

  \hypertarget{hell-on-earth-the-first-12-hours-of-californias-deadliest-wildfire}{%
  \subsection{`Hell on Earth': The First 12 Hours of California's
  Deadliest
  Wildfire}\label{hell-on-earth-the-first-12-hours-of-californias-deadliest-wildfire}}

  Nov. 20, 2018
\item
  \href{https://www.nytimes.com/interactive/2019/03/18/business/pge-california-wildfires.html}{}

  \hypertarget{how-californias-biggest-utility-overlooked-wildfire-risks}{%
  \subsection{How California's Biggest Utility Overlooked Wildfire
  Risks}\label{how-californias-biggest-utility-overlooked-wildfire-risks}}

  March 20, 2019

  Pacific Gas \& Electric, California's largest utility, has been
  responsible for wildfires in recent years that destroyed hundreds of
  thousands of acres. Several proved deadly.
\item
  \href{https://www.nytimes.com/2019/05/15/business/pge-fire.html}{}

  \includegraphics{https://static01.nyt.com/images/2019/05/15/business/15utility1/merlin_146590140_be0c2b07-3e4b-464b-866c-acac14e69ec7-mediumThreeByTwo225.jpg}

  \hypertarget{california-says-pge-power-lines-caused-fire-that-killed-85-people}{%
  \subsection{California Says PG\&E Power Lines Caused Fire That Killed
  85
  People}\label{california-says-pge-power-lines-caused-fire-that-killed-85-people}}

  May 16, 2019
\item
  \href{https://www.nytimes.com/2018/11/10/us/california-wildfires-paradise-malibu.html}{}

  \includegraphics{https://static01.nyt.com/images/2018/11/14/business/14warfire-01-print/00warfire-01-mediumThreeByTwo225.jpg}

  \hypertarget{paradise-is-gone-25-are-dead-in-devastating-california-fires}{%
  \subsection{`Paradise Is Gone': 25 Are Dead in Devastating California
  Fires}\label{paradise-is-gone-25-are-dead-in-devastating-california-fires}}

  Nov. 12, 2018
\item
  \href{https://www.nytimes.com/2018/11/22/us/survivors-california-wildfires.html}{}

  \includegraphics{https://static01.nyt.com/images/2018/11/23/us/23holdout-01-print/23holdout-01-print-mediumThreeByTwo225-v2.jpg}

  \hypertarget{everything-around-him-burned-he-stayed-put-and-lived-to-tell-the-tale}{%
  \subsection{Everything Around Him Burned. He Stayed Put and Lived to
  Tell the
  Tale.}\label{everything-around-him-burned-he-stayed-put-and-lived-to-tell-the-tale}}

  Nov. 23, 2018
\end{itemize}

Advertisement

\hypertarget{site-information-navigation}{%
\subsection{Site Information
Navigation}\label{site-information-navigation}}

\begin{itemize}
\tightlist
\item
  \href{https://help.nytimes.com/hc/en-us/articles/115014792127-Copyright-notice}{©
  2020 The New York Times Company}
\item
  \href{https://www.nytimes.com}{Home}
\item
  \href{https://www.nytimes.com/search/}{Search}
\item
  Accessibility concerns? Email us at
  \href{mailto:accessibility@nytimes.com}{\nolinkurl{accessibility@nytimes.com}}.
  We would love to hear from you.
\item
  \href{https://help.nytimes.com/hc/en-us/articles/115015385887-Contact-Us}{Contact
  Us}
\item
  \href{https://www.nytco.com/careers/}{Work with us}
\item
  \href{https://nytmediakit.com/}{Advertise}
\item
  \href{https://help.nytimes.com/hc/en-us/articles/115014892108-Privacy-policy\#pp}{Your
  Ad Choices}
\item
  \href{https://help.nytimes.com/hc/en-us/articles/115014892108-Privacy-policy}{Privacy}
\item
  \href{https://help.nytimes.com/hc/en-us/articles/115014893428-Terms-of-service}{Terms
  of Service}
\item
  \href{https://help.nytimes.com/hc/en-us/articles/115014893968-Terms-of-sale}{Terms
  of Sale}
\end{itemize}

\hypertarget{site-information-navigation-1}{%
\subsection{Site Information
Navigation}\label{site-information-navigation-1}}

\begin{itemize}
\tightlist
\item
  \href{https://spiderbites.nytimes.com}{Site Map}
\item
  \href{https://help.nytimes.com/hc/en-us}{Help}
\item
  \href{https://help.nytimes.com/hc/en-us/articles/115015385887-Contact-Us?redir=myacc}{Site
  Feedback}
\item
  \href{https://www.nytimes.com/subscription?campaignId=37WXW}{Subscriptions}
\end{itemize}
