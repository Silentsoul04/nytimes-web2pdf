Sections

SEARCH

\protect\hyperlink{site-content}{Skip to
content}\protect\hyperlink{site-index}{Skip to site index}

\hypertarget{comments}{%
\subsection{\texorpdfstring{\protect\hyperlink{commentsContainer}{Comments}}{Comments}}\label{comments}}

\href{}{How Trump Reshaped the Presidency in Over 11,000
Tweets}\href{}{Skip to Comments}

The comments section is closed. To submit a letter to the editor for
publication, write to
\href{mailto:letters@nytimes.com}{\nolinkurl{letters@nytimes.com}}.

\hypertarget{how-trump-reshaped-the-presidency-in-over-11000-tweets}{%
\section{How Trump Reshaped the Presidency in Over 11,000
Tweets}\label{how-trump-reshaped-the-presidency-in-over-11000-tweets}}

By \href{https://www.nytimes.com/by/michael-d-shear}{Michael D. Shear},
\href{https://www.nytimes.com/by/maggie-haberman}{Maggie Haberman},
\href{https://www.nytimes.com/by/nicholas-confessore}{Nicholas
Confessore}, \href{https://www.nytimes.com/by/karen-yourish}{Karen
Yourish}, \href{https://www.nytimes.com/by/larry-buchanan}{Larry
Buchanan} and \href{https://www.nytimes.com/by/keith-collins}{Keith
Collins}Nov. 2, 2019

\begin{itemize}
\item
\item
\item
\item
\item
  \emph{+}
\end{itemize}

The president's tweeting was once a sideshow. But it transformed how he
exerted power, leaving the White House and Twitter to grapple over
whether, and how, to rein it in.

\includegraphics{https://static01.nyt.com/newsgraphics/2019/09/23/trump-tweets/a7695ee15e891945cc2ab768dd99d417e94e832b/whtweet.jpg}

On the morning of Inauguration Day 2017, Donald J. Trump tweeted an
opening message to the United States.

What followed was a barrage of personal attacks, outrage and boasting,
in a near-constant stream of more than 11,000 tweets over 33 months.

At the beginning of his presidency, Mr. Trump tweeted about nine times
per day.

In the past three months, President Trump's tweets have spilled out at
triple the rate he set in 2017.

\hypertarget{how-trump-reshaped-the-presidency-in-over-11000-tweets-1}{%
\section{How Trump Reshaped the Presidency in Over 11,000
Tweets}\label{how-trump-reshaped-the-presidency-in-over-11000-tweets-1}}

The president's tweeting was once a sideshow. But it transformed how he
exerted power, leaving the White House and Twitter to grapple over
whether, and how, to rein it in.

By \href{https://www.nytimes.com/by/michael-d-shear}{MICHAEL D. SHEAR},
\href{https://www.nytimes.com/by/maggie-haberman}{MAGGIE HABERMAN},
\href{https://www.nytimes.com/by/nicholas-confessore}{NICHOLAS
CONFESSORE}, \href{https://www.nytimes.com/by/karen-yourish}{KAREN
YOURISH}, \href{https://www.nytimes.com/by/larry-buchanan}{LARRY
BUCHANAN} and \href{https://www.nytimes.com/by/keith-collins}{KEITH
COLLINS} Nov. 2, 2019

SHARE

In the Oval Office, an annoyed President Trump ended an argument he was
having with his aides. He reached into a drawer, took out his iPhone and
threw it on top of the historic Resolute Desk:

``Do you want me to settle this right now?''

There was no missing Mr. Trump's threat that day in early 2017, the
aides recalled. With a tweet, he could fling a directive to the world,
and there was nothing they could do about it.

When Mr. Trump entered office, Twitter was a political tool that had
helped get him elected and a digital howitzer that he relished firing.
In the years since, he has fully integrated Twitter into the very fabric
of his administration, reshaping the nature of the presidency and
presidential power.

After Turkey invaded northern Syria this past month, he crafted his
response not only in White House meetings but also in a series of
contradictory tweets. This summer, he announced increased tariffs on
\$300 billion worth of Chinese goods, using a tweet to deepen tensions
between the two countries. And in March, Mr. Trump cast aside more than
50 years of American policy, tweeting his recognition of Israel's
sovereignty in the Golan Heights. He openly delighted in the reaction he
provoked.

``Boom. I press it,'' Mr. Trump recalled months later at a White House
conference attended by conservative social media personalities, ``and,
within two seconds, `We have breaking news.'''

Early on, top aides wanted to restrain the president's Twitter habit,
even considering asking the company to impose a 15-minute delay on Mr.
Trump's messages. But 11,390 presidential tweets later, many
administration officials and lawmakers embrace his Twitter obsession,
flocking to his social media chief with suggestions. Policy meetings are
hijacked when Mr. Trump gets an idea for a tweet, drawing in cabinet
members and others for wordsmithing. And as a president often at war
with his own bureaucracy, he deploys Twitter to break through logjams,
overrule or humiliate recalcitrant advisers and pre-empt his staff.

``He needs to tweet like we need to eat,'' Kellyanne Conway, his White
House counselor, said in an interview.

In a presidency unlike any other, where Mr. Trump wakes to Twitter, goes
to bed with it and is comforted by how much it revolves around him, the
person he most often singled out for praise was himself --- more than
2,000 times, according to an analysis by The New York Times.

President Trump is tweeting more than ever. The second week of October
was his busiest, with 271 tweets.

He regularly takes to Twitter to lash out at his perceived enemies. In
fact, he attacks someone or something in more than half of his tweets.

250

tweets

per

week

200

150

100

50

2017

April

July

Oct.

2018

April

July

Oct.

2019

April

July

Oct.

250

tweets per week

200

150

100

50

2017

2018

2019

250

tweets

per

week

200

150

100

50

2017

April

July

Oct.

2018

April

July

Oct.

2019

April

July

Oct.

250

tweets

per

week

200

150

100

50

2017

April

July

Oct.

2018

April

July

Oct.

2019

April

July

Oct.

250

tweets per week

200

150

100

50

2017

2018

2019

250

tweets

per

week

200

150

100

50

2017

April

July

Oct.

2018

April

July

Oct.

2019

April

July

Oct.

\begin{itemize}
\item
\item
\end{itemize}

Most of these attacks occur in the early morning or later in the
evening, when Mr. Trump is more likely to be without his advisers.

1,200

tweets

1,000

Total tweets,

by hour tweeted

800

600

400

All tweets

200

Attacks

2 a.m.

6 a.m.

10 a.m.

2 p.m.

6 p.m.

10 p.m.

1,200

tweets

1,000

Total tweets,

by hour tweeted

800

600

400

All tweets

200

Attacks

2 a.m.

6 a.m.

10 a.m.

2 p.m.

6 p.m.

10 p.m.

The Times examined Mr. Trump's use of Twitter since taking office,
reviewing all his tweets, retweets and followers, and interviewing
nearly 50 current and former administration officials, lawmakers and
Twitter executives and employees. What has emerged is a rich account,
with new analysis, previously unreported episodes and fresh details of
how the president exploits the platform to exert power.

\emph{{[}Read how
\href{https://www.nytimes.com/2019/11/02/insider/trump-twitter-data.html}{Times
journalists} sorted more than 11,000 of Donald Trump's Twitter
posts.{]}}

\begin{longtable}[]{@{}ll@{}}
\toprule
\textbf{No. of tweets ...} & \textbf{that}\tabularnewline
\midrule
\endhead
5,889 & attacked someone or something\tabularnewline
4,876 & praised someone or something\tabularnewline
2,405 & attacked Democrats\tabularnewline
2,065 & attacked investigations\tabularnewline
2,026 & praised President Trump\tabularnewline
1,710 & promoted conspiracy theories\tabularnewline
1,308 & attacked news organizations\tabularnewline
851 & attacked minority groups\tabularnewline
758 & praised or promoted Fox News and other conservative
media\tabularnewline
570 & attacked immigrants\tabularnewline
453 & attacked previous presidential administrations\tabularnewline
417 & conducted presidential business on Twitter\tabularnewline
256 & attacked Hillary Clinton\tabularnewline
233 & attacked ally nations\tabularnewline
183 & bragged about crowd size and applause\tabularnewline
132 & praised dictators\tabularnewline
95 & referred to a Trump business\tabularnewline
40 & promoted voter fraud conspiracy theories\tabularnewline
36 & called the news media the ``enemy of the people''\tabularnewline
16 & referred to himself as everyone's ``favorite''
president\tabularnewline
\bottomrule
\end{longtable}

Show more

Tweets from Jan. 20, 2017 to Oct. 15, 2019.

It is often by brute repetition. He has taken to Twitter to demand
action 1,159 times on immigration and his border wall, a top priority,
and 521 times on tariffs, another key agenda item. Twitter is an
instrument of his foreign policy: He has praised dictators more than a
hundred times, while complaining nearly twice as much about America's
traditional allies. Twitter is the Trump administration's de facto
personnel office: The chief executive has announced the departures of
more than two dozen top officials, some fired by tweet.

More than half of the president's posts --- 5,889 --- have been attacks;
no other category even comes close. His targets include the Russia
investigation, a Federal Reserve that won't bow to his whims, previous
administrations, entire cities that are led by Democrats, and
adversaries from outspoken athletes to chief executives who displease
him. Like no other modern president, Mr. Trump has publicly harangued
businesses to advance his political goals and silence criticism, often
with talk of government intervention. Using Twitter, he threatened
``Saturday Night Live'' with an investigation by the Federal
Communications Commission and accused Amazon, led by Jeff Bezos, owner
of The Washington Post, of cheating the United States Postal Service.

As much as anything, Twitter is the broadcast network for Mr. Trump's
parallel political reality --- the ``alternative facts'' he has used to
spread conspiracy theories, fake information and extremist content,
including material that energizes some of his base.

Mr. Trump's use of Twitter has accelerated sharply since the end of the
special counsel's Russia investigation and reached a new high as
Democrats opened an impeachment inquiry, the analysis shows. He tweeted
more than 500 times during the first two weeks of October, a pace that
put him on track to triple his monthly average. (The Times analyzed Mr.
Trump's tweets through Oct. 15. The total by the end of the month
reached 11,887.)

His more than 66 million Twitter followers have become his private
polling service, offering what he sees as validation for his performance
in office. But fewer than one-fifth of his followers are voting-age
Americans, according to a Times analysis of Pew Research national
surveys of adults who use Twitter.

The White House press office declined to comment for this article and
turned down an interview request with the president. Now, as Mr. Trump
anticipates a bitter re-election battle and faces an impeachment inquiry
by Democrats, the stakes are higher than ever before, and Twitter even
more central to his presidency.

His top campaign aides are embracing the outrage that Mr. Trump stirs
with his tweets to reinforce his anti-establishment brand and strengthen
his bond with the fiercely loyal supporters who propelled him into
office. And as public backing for impeachment grows, the president is
using the platform to build a defensive echo chamber.

\includegraphics{https://static01.nyt.com/newsgraphics/2019/10/25/trump-attack-charts/2a5cd29e05c9e308a68eb5118ec0aa29d1816a28/trump_tweet_top_sm.png}

\includegraphics{https://static01.nyt.com/newsgraphics/2019/10/25/trump-attack-charts/2a5cd29e05c9e308a68eb5118ec0aa29d1816a28/trump_tweet_top_sm.png}

Mr. Trump tweeted his first attack as president on his third day in
office.

He fired off more than 1,100 attacks over the next year.

The most frequent targets of Mr. Trump's ire are Democrats, news
organizations and investigations --- specifically the Russia and
impeachment inquiries.

Mr. Trump's attack on four Democratic congresswomen of color in July
received a lot of attention, and blowback.

It was not the first time he had used Twitter to attack minority groups
--- and it would not be the last.

The president has tweeted more attacks so far this year than in the
previous two years combined. In total, he has attacked at least 630
people and things in 5,889 tweets since taking office.

While people around Mr. Trump acknowledge that his tweets can cause
political damage, the president is confident in his mastery of Twitter.

This past week, as he announced that American Special Forces had killed
\href{https://www.nytimes.com/2019/10/27/world/middleeast/al-baghdadi-dead.html?module=inline}{Abu
Bakr al-Baghdadi}, the leader of the Islamic State, Mr. Trump noted the
terror group's digital prowess. ``They use the internet better than
almost anybody in the world,'' he said. ``Perhaps other than Donald
Trump.''

\hypertarget{policy-via-twitter}{%
\paragraph{Policy via Twitter}\label{policy-via-twitter}}

With a single tweet last fall, Mr. Trump sent his administration into a
tailspin. ``I must, in the strongest of terms, ask Mexico to stop this
onslaught,'' he wrote in October 2018, angry about a caravan of migrants
from Central America. ``If unable to do so I will call up the U.S.
Military and CLOSE OUR SOUTHERN BORDER!''

Mr. Trump's aides had tried for weeks to talk him out of shutting down
the border --- the logistics would be impossible and the economic pain
extreme. The tweet prompted an emergency meeting down the hall from the
Oval Office as aides scrambled to head off Mr. Trump's impulse,
according to people familiar with the frantic scene. Like others in this
article, they spoke on the condition of anonymity for fear of angering
the president.

The aides succeeded in temporarily holding him off, but the tweet
crystallized for cautious bureaucrats exactly what he wanted: to stop
people from coming into the country. In the months that followed, Mr.
Trump's threat helped to set off an effort inside the government to find
ever more restrictive ways to block immigrants. Nearly six months later,
Kirstjen Nielsen, the homeland security secretary, was still trying to
prevent a border shutdown when the president brought her resistance to
an end.

``Kirstjen Nielsen,'' he
\href{https://twitter.com/realDonaldTrump/status/1115011884154064896?ref_src=twsrc\%5Etfw}{tweeted},
``will be leaving her position.''

This is governing in the Trump era. For President Barack Obama, a tweet
about a presidential proposal might mark the conclusion of a long,
deliberative process. For Mr. Trump, Twitter is often the beginning of
how policy is made.

``Suddenly there's a tweet, and everything gets upended and you spent
the week trying to defend something else,'' said Representative Peter
King, Republican of New York. ``This person thrives on chaos. What we
may find disconcerting or upsetting or whatever, it is actually what
keeps him going.''

In October 2017, Rex W. Tillerson, the president's first secretary of
state, was in China with a team of diplomats negotiating sanctions on
Kim Jong-un, the North Korean leader, when Mr. Trump weighed in on
Twitter. Mr. Tillerson was ``wasting his time trying to negotiate with
Little Rocket Man,'' he wrote. ``Save your energy Rex, we'll do what has
to be done!''

Two months later,
\href{https://www.reuters.com/article/us-usa-trump-wells-fargo-exclusive/exclusive-wells-fargo-sanctions-are-on-ice-under-trump-official-sources-idUSKBN1E12Y5}{a
Reuters headline} blared that Mick Mulvaney, who then was Mr. Trump's
new pick to lead the Consumer Financial Protection Bureau, had decided
to put ``on ice'' sanctions against Wells Fargo for consumer abuses. It
was little surprise: Mr. Mulvaney was an ally of the financial industry.
But Mr. Trump had other ideas.

``Fines and penalties against Wells Fargo Bank for their bad acts
against their customers and others will not be dropped, as has
incorrectly been reported, but will be pursued and, if anything,
substantially increased,'' he tweeted.

Political appointees at the bureau wanted to affirm Mr. Trump's desire
publicly, despite longstanding policies against commenting on active
investigations, according to former officials there. A spokesman for Mr.
Mulvaney issued a statement saying only that he ``shares the president's
firm commitment to punishing bad actors and protecting American
consumers.''

According to two people with direct knowledge of the Wells Fargo
inquiry, career bureau officials took Mr. Trump's outburst as a green
light to pursue aggressive negotiations with the bank, even as Mr.
Mulvaney's team prepared to dial back penalties in other cases or shelve
them. Wells Fargo ultimately agreed to a billion-dollar federal
settlement, the bureau's largest-ever civil penalty.

Over time, Mr. Trump has turned Twitter into a means of presidential
communication as vital as a statement from the White House press
secretary or an Oval Office address. The press secretary has not held a
daily on-camera press briefing --- a decades-long ritual of presidential
messaging --- since March. Instead, Mr. Trump's Twitter activity drives
the day.

And Mr. Trump has removed any doubt that his tweets carry the weight
once reserved for more formal pronouncements.

In summer 2018, his aides repeatedly tried to reassure Republican
lawmakers that the president backed their hard-line immigration bill,
despite his remarks suggesting otherwise. But privately, Mr. Trump told
several senators that there was only one certain sign of his support.

``If I don't tweet it,'' he said, according to two former senior
advisers, ``don't listen to my staff.''

\hypertarget{adapting-a-platform}{%
\paragraph{Adapting a Platform}\label{adapting-a-platform}}

When Mr. Trump entered office, aides were determined to rein in his
itchy Twitter fingers.

In a series of informal conversations in early 2017, top White House
officials discussed the possibility of a 15-minute delay on the
president's account, a technical change not unlike the five-second
naughty-word system used by television networks. But, one former senior
official said, they quickly abandoned the idea after recognizing the
political peril if it leaked to the press --- or to their boss.

Several weeks later, a trio of close advisers presented Mr. Trump with
another idea. Gary Cohn, the top economic adviser; Hope Hicks, the
president's director of strategic communications; and Rob Porter, his
staff secretary, argued that they should see the tweets before he sent
them out.

Mr. Trump was skeptical, worrying that delayed tweets would be
irrelevant, according to a former White House official. But he agreed to
a weeklong trial. Within 72 hours, the president had resumed tweeting
from his golf club in Bedminster, N.J.

Three thousand miles away, in Silicon Valley, similar conversations were
unfolding at Twitter's offices, where executives faced the same dilemma
as Mr. Trump's inner circle: whether, and how, to restrain him.

At the time, Twitter lagged far behind larger competitors like Facebook.
While popular among politicians and journalists, it was struggling
financially. But the president's incessant tweeting gave the company
more currency.

His Twitter account often drove more ``impressions'' --- a key company
metric --- than any other in the world. But some of his messages seemed
to violate the company's policies against abuse and incitement.

\includegraphics{https://static01.nyt.com/newsgraphics/2019/10/25/trump-attack-charts/2a5cd29e05c9e308a68eb5118ec0aa29d1816a28/trump_tweet_top_sm.png}

\includegraphics{https://static01.nyt.com/newsgraphics/2019/10/25/trump-attack-charts/2a5cd29e05c9e308a68eb5118ec0aa29d1816a28/trump_tweet_top_sm.png}

Many of the president's tweets promoted conspiracy theories or tried to
erode faith in democratic institutions. On his sixth day in office, he
advanced the false claim that millions of people voted illegally in the
2016 election, depriving him of a popular-vote majority.

He has tweeted 40 times about voter fraud and a ``rigged'' electoral
system.

Less than a month into his presidency, Mr. Trump tweeted that Democrats
made up Russian interference in the 2016 election to justify Hillary
Clinton's loss. He then accused Mr. Obama of illegally wiretapping Trump
Tower during the campaign.

He has since sown doubt about Russian interference and the resulting
investigations in more than 1,400 tweets.

Mr. Trump has also used Twitter to attack the credibility of
journalists, intelligence agencies and the judicial system. He has
spoken of a nefarious ``deep state'' undermining his presidency, a
judiciary that puts the country in ``peril'' and a news media that is
``the enemy of the people,'' a phrase historically used by autocrats.

The president also pushed unfounded claims that Big Tech is biased
against conservatives (102 tweets), stoked fears that caravans of
migrants were going to ``invade'' the United States (43 tweets), and
questioned the number of people killed in Puerto Rico as a result of
Hurricane Maria (5 tweets). All told, Mr. Trump tweeted conspiratorial
language more than 1,700 times.

On a now-defunct internal company message board known as Twitter Buzz,
some left-leaning employees favored barring the president. Mr. Trump's
behavior came up at almost every all-hands gathering and at many smaller
meetings of executives. Some of them had set their phones to alert them
whenever the president tweeted, according to a former employee who spoke
on the condition of confidentiality.

``What I saw was a company coming to grips with an entirely new
situation, a new level of scrutiny, a new level of vitriol,'' said
Dianna Colasurdo, a former account executive on Twitter's political
advertising sales team, ``and working to adapt their policies in the
moment to align with that.''

A turning point came in fall 2017, at the height of tensions with North
Korea, when Mr. Trump tweeted that the rogue nation might not ``be
around much longer!'' The country's foreign minister called that a
declaration of war. On Twitter, users wondered if the company would
allow Mr. Trump to tweet his way into a nuclear conflict.

The response came the next day. Referring back to Mr. Trump's online
declaration, Twitter announced in a tweet that it took
``newsworthiness'' into account when
\href{https://twitter.com/Policy/status/912438046515220480}{evaluating
whether to remove} a post that violated its policies.

In an interview, Twitter executives said that newsworthiness had long
figured into the company's internal enforcement guidelines and that
officials there had been formulating the announcement, which applied
worldwide, months before Mr. Trump's North Korea tweet. But former
employees said they understood the announcement to be Trump-driven.
Twitter did not want to be in the business of censoring the president.

Late in summer 2018, White House insiders tried again to curb Mr.
Trump's use of social media, according to two former aides. After a
series of over-the-top weeks of tweeting --- including calling Omarosa
Manigault Newman, his onetime aide, ``wacky'' and ``a lowlife'' ---
several advisers suggested he go just two days without Twitter and see
what happened. Mr. Trump nodded, and then promptly discarded the advice.

Mr. King, who said most of his Republican colleagues wished the
president would tweet less, added that whenever he had raised the issue
with White House staff members, they shrugged helplessly.

``It's not going to stop,'' he recalled their saying. ``Forget it; we've
all tried.''

Soon enough, Mr. Trump was as prolific as ever.

On Sept. 13, he mocked Jamie Dimon, the chief executive of JPMorgan
Chase, for claiming he could beat Mr. Trump in an election. ``He doesn't
have the aptitude or `smarts' \& is a poor public speaker \& nervous
mess,'' the president tweeted. Over the next 12 hours, Mr. Trump
attacked two former F.B.I. officials, accused The Wall Street Journal of
getting a tariff story wrong and blasted former Secretary of State John
Kerry for holding ``illegal meetings'' with Iran.

``BAD!'' he wrote.

\hypertarget{first-things-first}{%
\paragraph{First Things First}\label{first-things-first}}

Mr. Trump's Twitter habit is most intense in the morning, when he is in
the White House residence, watching Fox News, scrolling through his
Twitter mentions and turning the social media platform into what one
aide called the ``ultimate weapon of mass dissemination.''

Of the attack tweets identified in the Times analysis, nearly half were
sent between 6 a.m. and 10 a.m., hours that Mr. Trump spends mostly
without advisers present.

After waking early, Mr. Trump typically watches news shows recorded the
previous night on his ``Super TiVo,'' several DVRs connected to a single
remote. (The devices are set to record ``Lou Dobbs Tonight'' on Fox
Business Network; ``Hannity,'' ``Tucker Carlson Tonight'' and ``The
Story With Martha MacCallum'' on Fox News; and ``Anderson Cooper 360''
on CNN.)

He takes in those shows, and the ``Fox \& Friends'' morning program,
then flings out comments on his iPhone. Then he watches as his tweets
reverberate on cable channels and news sites.

\includegraphics{https://static01.nyt.com/newsgraphics/2019/10/25/trump-attack-charts/2a5cd29e05c9e308a68eb5118ec0aa29d1816a28/trump_tweet_top_sm.png}

The symbiotic relationship between Mr. Trump and Fox News is apparent
through the president's tweets. In fact, he praised the network in his
first tweet on the first morning he woke up in the White House.

He has since praised and promoted the network, individual shows and
conservative news media personalities more than 750 times.

Over all, at least 15 percent of the content in Mr. Trump's tweets
seemed to come directly from Fox News and other conservative media
outlets.

Early on Sept. 2 --- the start of a week in which he tweeted 198 times
--- the president sent a few benign tweets, then lashed out at Paul
Krugman as a ``Failing New York Times columnist'' who ``never got it!''
Over the next 44 minutes, he fired off 10 more tweets. He disparaged
Richard Trumka, the president of the A.F.L.-C.I.O. (``Likes what we are
doing until the cameras go on.'') He called James B. Comey, the former
F.B.I. director, and his ``dwindling group of friends'' liars and
traitors. He railed against The Washington Post and four women of color
in Congress who called themselves ``the Squad.''

Almost every morning that week, Mr. Trump kicked off the day with an
attack on one critic or another: the ``incompetent Mayor of London,'' or
``Bad `actress' Debra The Mess Messing'' --- whom he accused of being
racist --- or the ``Fake News Media.'' He referred to conservative media
outlets 45 times, berated the mainstream media 32 times and tweeted
about conspiracy theories 12 times.

Sometimes the president's apparent fury on Twitter is meant to troll his
critics and get a rise out of them, many of his closest aides said. But
they still brace themselves, knowing that they are likely to be
blindsided by one of his tweets. Aides who gather for the early-morning
staff meetings in the West Wing said their agenda was regularly blown up
when their phones simultaneously went off with a tweet from the boss.

Once Mr. Trump arrives in the West Wing --- usually after 10 a.m. ---
Dan Scavino, the White House social media director, takes control of the
Twitter account, tweeting as @realDonaldTrump from his own phone or
computer. Mr. Trump rarely tweets in front of others, those close to him
say, because he does not like to wear the reading glasses he needs to
see the screen.

Instead, the president dictates tweets to Mr. Scavino, who sits in a
closet-size room just off the Oval Office until Mr. Trump
\href{https://www.politico.com/story/2019/05/16/trump-scavino-1327921}{calls
out}
\href{https://www.politico.com/story/2019/05/16/trump-scavino-1327921}{``Scavino!''}
Often, he prints out suggested tweets in extra-large fonts for the
president to sign off on. (A single-page article that Mr. Scavino
recently printed out for him ran to six pages after the fonts were
enlarged, according to one person who saw it.)

Mr. Scavino's role in Mr. Trump's Twitter machine has made him an
unlikely White House power broker and the go-to person for aides,
business executives, friends and lawmakers who want the president to
tweet something. Ms. Conway noted what she called the hypocrisy of many
Republicans who begged her to get Mr. Trump to stop tweeting during the
2016 campaign and now come to Mr. Scavino with suggestions. Mr. Scavino
declined to be interviewed for this article.

He sometimes acts as a brake --- or tries to --- on the president's
tweeting impulses. When Mr. Trump started angrily posting about the
``Squad,'' Mr. Scavino told him it was a bad idea, according to an aide
who witnessed the conversation. Along with Michael Dubke, who served as
White House communications director for several months in 2017 and is
from Buffalo, home of the famous chicken wings, Mr. Scavino presented
some tweets to Mr. Trump in degrees of outrageousness: ``hot,''
``medium'' or ``mild.'' Mr. Trump, said one former official who saw the
proposed messages, always picked the most incendiary ones and often
wanted to make them even more provocative.

And while many of Mr. Trump's tweets are shoot-from-the-hip attacks, he
chews over others for days or even weeks, waiting for just the right
moment to maximize the reaction, aides say.

He plotted for days to tweet about Mika Brzezinski, the liberal co-host
of the popular MSNBC morning program, according to former White House
officials, before finally posting one morning in June 2017. He called
her ``low I.Q. Crazy Mika'' and wrote that she had been ``bleeding badly
from a face-lift'' during a New Year's Eve party.

In October of last year, the president started telling his aides that he
planned to denounce Stormy Daniels, a pornographic-film actress who
claimed to have had an affair with him more than a decade earlier. He
said he wanted to call her a ``horse face.''

Several current and former aides recalled telling Mr. Trump that it was
a terrible idea and would renew accusations of misogyny against him. But
he persisted.

Finally, after watching a Fox News report days later about how a federal
judge had thrown out a lawsuit by Ms. Daniels, the president tapped out
the tweet.

``Great, now I can go after Horseface and her 3rd rate lawyer in the
Great State of Texas,'' he wrote.

\hypertarget{a-love-of-likes}{%
\paragraph{A Love of `Likes'}\label{a-love-of-likes}}

For Mr. Trump, Twitter reinforces his instincts about his performance as
president.

After a rally in Dallas in mid-October, Mr. Trump's aides prepared a
large-type printout of tweets gushing over his speech that day,
including one from Tomi Lahren, a Fox News commentator and the host of a
show on the Fox Nation site. Mr. Trump scrawled a thank-you note on one
copy to Ms. Lahren --- who then
\href{https://twitter.com/TomiLahren/status/1187105545997602816?s=20}{tweeted
a picture of the letter} back at the president.

Aides said they often compiled positive feedback for Mr. Trump. He
revels in the stream of praise from his most loyal followers, on paper
or as he scrolls through his phone early in the morning and late at
night. He considers his following to be like the ratings on a TV show,
better than any approval poll. After one weekend Twitter spree, the
president told Sarah Huckabee Sanders, his press secretary at the time,
he had expected a tweet he was particularly proud of to get more
response than it did, according to a former administration official. Ms.
Sanders said that if he tweeted 60 times, people wouldn't pay as much
attention, the official said.

\hypertarget{twitter-vs-reality}{%
\subsection{Twitter vs. Reality}\label{twitter-vs-reality}}

The polling firm YouGov asks Americans to rate Mr. Trump's tweets every
day. Tweets that get the most likes on Twitter tend to be more poorly
received by the American public, while those with lower engagement tend
to be viewed more positively.

Americans disapproved

Americans approved

Average Twitter ``likes''

(in thousands)

Attacks

N.F.L. players

Makes

appeals to

black people

120

Promotes voter

fraud conspiracy

Attacks

people

of color

Tweets that attacked individuals or spread conspiracy theories were
popular on Twitter, but were poorly rated over all by the American
public.

110

Attacks

immigrants

Attacks

Hillary Clinton

or her campaign

100

Attacks Democrats

Attacks former presidents

Praises himself

Attacks

Russia

investigation

Attacks

Barack

Obama

Praises

someone

or something

2020

election

Mentions wars

or the military

90

Brags about

crowd sizes

and applause

Mentions law

enforcement

Attacks

impeachment

inquiry

80

Mentions guns

Mentions

veterans

2018

election

Official pronouncements

Stems from

conservative media

70

Praises Fox and other

conservative media

Mentions

drug addiction

Official pronouncements and tweets about law enforcement and veterans
scored higher with many Americans, but they were less popular on
Twitter.

60

Twitter

popularity

50

Promotes a news

media appearance

-40

-35

-30

-25

-20

-15

-10

-5

0

5

10

15

20

25

Average YouGov TweetIndex score

Among Republicans only

Among Democrats only

Attacks N.F.L. players

Attacks N.F.L. players

Voter

fraud

Makes

appeals

to black

people

120

120

Voter fraud

Makes appeals to black people

Attacks

people

of color

Attacks people of color

110

110

Attacks immigrants

Attacks

immigrants

Attacks Hillary Clinton

or her campaign

100

100

Attacks Hillary Clinton or her campaign

90

90

Mentions wars or the military

Law enforcement

Mentions law enforcement

80

80

Mentions guns

Mentions veterans

Official

pronouncements

Official pronouncements

Mentions

veterans

70

70

Mentions

drug

addiction

Praises Fox and

other conservative media

Mentions

drug addiction

60

60

Promotes a news

media appearance

Promotes a news

media appearance

50

50

-120

-100

-80

-60

-40

-20

0

20

-20

0

+20

40

60

80

100

120

Americans disapproved

Americans approved

Attacks

N.F.L. players

Average Twitter ``likes''

(in thousands)

120

Makes

appeals to

black people

Promotes voter

fraud conspiracy

Attacks

people

of color

Tweets that attacked individuals or spread conspiracy theories were
popular on Twitter, but were poorly rated over all by the American
public.

110

Attacks

immigrants

Attacks

Hillary Clinton

or her campaign

100

Attacks Democrats

Attacks former presidents

2020

election

Praises himself

Attacks

Russia

investigation

Mentions wars

or the military

90

Brags about

crowd sizes

and applause

Mentions law

enforcement

Attacks

impeachment

inquiry

80

Mentions guns

Mentions

veterans

2018

election

Official pronouncements

Stems from

conservative media

70

Mentions

drug addiction

Praises Fox and other

conservative media

Official pronouncements and tweets about law enforcement and veterans
scored higher with many Americans, but they were less popular on
Twitter.

60

Twitter

popularity

50

Promotes a news

media appearance

-40

-35

-30

-25

-20

-15

-10

-5

0

5

10

15

20

25

Average YouGov TweetIndex score

Among Democrats only

Among Republicans only

Attacks N.F.L. players

Attacks N.F.L. players

120

120

Makes appeals

to black people

Voter fraud

Voter fraud

Attacks

people of color

Attacks

people

of color

110

110

Attacks

immigrants

Attacks

immigrants

Attacks Hillary Clinton

or her campaign

100

100

Attacks Hillary Clinton or her campaign

90

90

Mentions wars or the military

Law enforcement

Mentions law

enforcement

80

80

Mentions veterans

Official

pronouncements

Mentions

veterans

2018

election

70

70

Mentions

drug

addiction

Mentions

drug addiction

60

60

Promotes a news

media appearance

Promotes a news

media appearance

50

50

-120

-100

-80

-60

-40

-20

0

20

-20

0

20

40

60

80

100

120

Americans

disapproved

Americans

approved

Average

Twitter ``likes''

(in thousands)

Attacks

N.F.L. players

Makes

appeals

to black

people

Attacks

people

of color

Voter fraud

110

Attacks

immigrants

100

Mentions wars

or the military

Praises

himself

90

Mentions law

enforcement

80

Mentions

veterans

2018

election

Pronouncements

70

Mentions

drug

addiction

Praises Fox and other

conservative media

60

Twitter

popularity

Promotes a news

media appearance

50

-40

-30

-20

-10

0

10

20

Average YouGov TweetIndex score

Among Democrats only

Attacks N.F.L. players

120

Makes appeals

to black people

Voter fraud

Attacks

people of color

110

Attacks

immigrants

100

Attacks Hillary Clinton or her campaign

90

Mentions wars or the military

Mentions law

enforcement

80

Official

pronouncements

Mentions

veterans

2018

election

70

Mentions

drug addiction

60

Promotes a news

media appearance

50

-120

-100

-80

-60

-40

-20

0

20

Among Republicans only

Attacks N.F.L. players

120

Voter fraud

Attacks

people

of color

110

Attacks

immigrants

Attacks Hillary Clinton

or her campaign

100

90

Law enforcement

80

Mentions veterans

70

Mentions

drug

addiction

60

Promotes a news

media appearance

50

-20

0

20

40

60

80

100

120

Americans

disapproved

Americans

approved

Average

Twitter likes

(in thousands)

Attacks

N.F.L. players

Makes

appeals

to black

people

Attacks

people

of color

Voter fraud

110

Attacks

immigrants

100

Mentions wars

or the military

Praises

himself

90

Mentions law

enforcement

80

Mentions

veterans

2018

election

Official pronouncements

70

Mentions

drug addiction

Praises Fox and other

conservative media

60

Twitter

popularity

Promotes a news

media appearance

50

-40

-30

-20

-10

0

10

20

Average YouGov TweetIndex score

Among Democrats only

Attacks N.F.L. players

120

Makes appeals

to black people

Voter fraud

Attacks

people of color

110

Attacks

immigrants

100

Attacks Hillary Clinton or her campaign

90

Mentions wars or the military

Mentions law

enforcement

80

Official

pronouncements

Mentions

veterans

2018

election

70

Mentions

drug addiction

60

Promotes a news

media appearance

50

-120

-100

-80

-60

-40

-20

0

20

Among Republicans only

Attacks N.F.L. players

120

Voter fraud

Attacks

people

of color

110

Attacks

immigrants

Attacks Hillary Clinton

or her campaign

100

90

Law enforcement

80

Mentions veterans

70

Mentions

drug

addiction

60

Promotes a news

media appearance

50

-20

0

20

40

60

80

100

120

Note: YouGov TweetIndex scores range from --200 to 200. Retweets and
deleted tweets are not included in the analysis.

·Source: YouGov·By The New York Times

The president is keenly aware of his number of followers and reluctant
to acknowledge that any of them are not real. Mr. Trump has accused
Twitter of political bias for its periodic purges of bot accounts across
the platform, which have cost him --- and other prominent users ---
hundreds of thousands of followers. When he met with the company's chief
executive, Jack Dorsey, in April, Mr. Trump
\href{https://www.washingtonpost.com/technology/2019/04/23/trump-meets-with-twitter-ceo-jack-dorsey-white-house/}{reportedly
pressed} him at length about the lost followers.

There is plenty of evidence that Mr. Trump's Twitter following may not
be a reliable proxy for what the American people think of the job he is
doing.

It is difficult, if not impossible, to determine with certainty how many
of Mr. Trump's more than 66 million followers are fake. Some studies of
his followers have estimated that a high proportion are likely to be
automated bots, fake accounts or inactive. But even a conservative
analysis by The Times found that nearly a third of them, about 22
million, included no biographical information and used the service's
default profile image --- two signs the accounts may be rarely used or
inactive. Fourteen percent have automatically generated user names,
another indication that an account may not belong to a real person.

Even if Mr. Trump is not shouting into the void on Twitter, he is often
preaching to the converted. Data from Stirista, an analytics firm, shows
that his followers tend to be the kind of users who are most likely to
be his supporters --- disproportionately older, white and male compared
with Twitter users over all.

And they constitute just a fraction of the electorate. According to the
Times analysis of Pew data, only about four percent of American adults,
or about 11 million people, follow him on Twitter. Those followers
represent less that one-fifth of his total, the analysis shows.

According to data from YouGov, which polls about most of the president's
tweets, some of the topics on which Mr. Trump got the most likes and
retweets --- jabs at the N.F.L., posts about the special counsel's
investigation, unfounded allegations of widespread voter fraud --- poll
poorly with the general public.

But people close to Mr. Trump said there was no dissuading him that the
``likes'' a tweet got were evidence that a decision or policy proposal
was well received.

Last December, after Mr. Trump announced plans to withdraw some troops
from Syria, lawmakers came to the White House to argue against it.
According to Politico, Mr. Trump responded by calling in Mr. Scavino.

``Tell them how popular my policy is,'' Mr. Trump asked Mr. Scavino, who
described for the lawmakers social media postings that had praised Mr.
Trump's decision. Aides said that for Mr. Trump, his Twitter ``likes''
were proof that he had made the right call.

The reaction in the outside world was far less favorable. Within weeks,
Mr. Trump's defense secretary and the special anti-ISIS envoy quit over
the decision. American allies were enraged. More than two-thirds of the
Senate voted to rebuke Mr. Trump, who agreed under pressure to keep the
troops in Syria.

Almost a year later, American troops in Syria became an issue again
after Mr. Trump appeared to give President Recep Tayyip Erdogan of
Turkey a green light to invade Kurdish-controlled areas in northern
Syria.

That resulted in another congressional rebuke for Mr. Trump and
complaints even from loyal Republican allies. In subsequent days, Mr.
Trump sought to defend himself on Twitter, alternately denying he had
abandoned the Kurds and suggesting the United States had no stake in
their safety, threatening Mr. Erdogan if the incursion continued and
praising Turkey as an important trading partner.

Many people took note of the back-and-forth, including Mr. Erdogan.
``When we take a look at Mr. Trump's Twitter posts, we can no longer
follow them,'' the Turkish president told reporters mockingly in
mid-October, according to Hurriyet, a Turkish newspaper. ``We cannot
keep track.''

\includegraphics{https://static01.nyt.com/newsgraphics/2019/10/25/trump-attack-charts/2a5cd29e05c9e308a68eb5118ec0aa29d1816a28/trump_tweet_top_sm.png}

\includegraphics{https://static01.nyt.com/newsgraphics/2019/10/25/trump-attack-charts/2a5cd29e05c9e308a68eb5118ec0aa29d1816a28/trump_tweet_top_sm.png}

Mr. Trump was not all negative. He also used Twitter to express
admiration for people, from Aretha Franklin to Ryan Zinke.

He often coupled his praise with a jab at someone else, or vice versa.

He praised himself more than he did anyone else.

All told, Mr. Trump sent 4,876 tweets that expressed approval or praise.

\hypertarget{a-tool-for-re-election}{%
\paragraph{A Tool for Re-Election}\label{a-tool-for-re-election}}

In the months ahead, the man tasked with winning Mr. Trump a second term
is hoping to focus the president's Twitter habit on its original
purpose: connecting with voters.

Brad Parscale, who served as Mr. Trump's digital director in 2016 and is
now campaign manager, has worked closely with Mr. Scavino to shape
perceptions of the president through social media. The two men speak a
half-dozen times a day, according to people familiar with their
interactions.

Mr. Parscale criticized Twitter after it announced on Wednesday that it
would no longer allow paid political advertising on the platform,
calling it ``yet another attempt to silence conservatives.'' But the
change may benefit Mr. Trump: He has a far larger organic Twitter
following than any of his likely Democratic opponents, and is therefore
less reliant on paid ads to spread his message through the platform.

While some campaign aides say Mr. Trump's tweets can be a distraction,
they also view Twitter as an essential tool to present him as someone
strong, willing to stand up to so-called political elites and what the
president recently called the ``unholy alliance of corrupt Democrat
politicians, deep-state bureaucrats and the fake-news media.''

The aides seek to cultivate the image of a man who understands ``regular
people.'' Mr. Trump's team believes that his unvarnished writing, poor
punctuation and increasing profanity on Twitter signal authenticity ---
a contrast to the polished, vetted, often anodyne social media style of
most candidates.

Twitter, Ms. Conway said, is the president's most potent weapon when it
comes to bypassing the powerful people he believes have controlled the
flow of information too long.

``It's the democratization of information,'' she said. Everyone receives
Mr. Trump's tweets at once --- the stay-at-home mom, the plumber working
on the sink, the billionaire executive, the White House correspondent.

``They all hear `ping,''' she said, ``at the same time.''

The New York Times reviewed every tweet and retweet sent by President
Trump from Jan. 20, 2017, through Oct. 15, 2019. Each one was evaluated
and tagged for several factors: whether it included an attack or praise;
who or what was attacked or praised; and for topics including trade,
immigration, the military, the economy, the 2018 midterm elections, the
Russia investigation and the House impeachment inquiry. In the Times
analysis, retweets in each of those categories were counted as tweets.

The Times reviewed each Twitter account that followed Mr. Trump by
analyzing profile information, tweet frequency and the date the account
was created. The Times also used data from Pew Research to estimate how
many American adults follow Mr. Trump on Twitter. Pew Research conducted
a nationally representative sample of American adults with personal,
public Twitter accounts to analyze how many follow American politicians.

Sources: Trump Twitter Archive, Internet Archive, Politwoops, Census
Bureau, Pew Research

Reporting was contributed by Rich Harris, Blacki Migliozzi, Matthew
Rosenberg and Rachel Shorey. Produced by Gray Beltran, Rumsey Taylor and
Jon Huang. Additional graphics by Guilbert Gates.

The Twitter Presidency

\begin{itemize}
\tightlist
\item
  \href{https://www.nytimes.com/interactive/2019/11/02/us/politics/trump-twitter-disinformation.html\%0A}{Extremists
  and Spies}
\item
  Reshaping the White House
\item
  \href{https://www.nytimes.com/interactive/2019/11/02/us/politics/trump-twitter-retweets.html\%0A}{Plucked
  From Obscurity}
\end{itemize}

\hypertarget{the-twitter-presidency}{%
\subsection{The Twitter Presidency}\label{the-twitter-presidency}}

\begin{itemize}
\tightlist
\item
  \href{https://www.nytimes.com/interactive/2019/11/02/us/politics/trump-twitter-disinformation.html\%0A}{}
\item
  \href{https://www.nytimes.com/interactive/2019/11/02/us/politics/trump-twitter-retweets.html\%0A}{}
\item
  \href{https://www.nytimes.com/2019/11/02/us/trump-twitter-takeaways.html}{}
\end{itemize}

Reporting was contributed by Rich Harris, Blacki Migliozzi, Matthew
Rosenberg and Rachel Shorey. Produced by Gray Beltran, Rumsey Taylor and
Jon Huang. Additional graphics by Guilbert Gates.

Write a comment

\begin{itemize}
\item
\item
\item
\item
\end{itemize}

Advertisement

\protect\hyperlink{after-bottom}{Continue reading the main story}

\hypertarget{site-index}{%
\subsection{Site Index}\label{site-index}}

\hypertarget{site-information-navigation}{%
\subsection{Site Information
Navigation}\label{site-information-navigation}}

\begin{itemize}
\tightlist
\item
  \href{https://help.nytimes.com/hc/en-us/articles/115014792127-Copyright-notice}{©~2020~The
  New York Times Company}
\end{itemize}

\begin{itemize}
\tightlist
\item
  \href{https://www.nytco.com/}{NYTCo}
\item
  \href{https://help.nytimes.com/hc/en-us/articles/115015385887-Contact-Us}{Contact
  Us}
\item
  \href{https://www.nytco.com/careers/}{Work with us}
\item
  \href{https://nytmediakit.com/}{Advertise}
\item
  \href{http://www.tbrandstudio.com/}{T Brand Studio}
\item
  \href{https://www.nytimes.com/privacy/cookie-policy\#how-do-i-manage-trackers}{Your
  Ad Choices}
\item
  \href{https://www.nytimes.com/privacy}{Privacy}
\item
  \href{https://help.nytimes.com/hc/en-us/articles/115014893428-Terms-of-service}{Terms
  of Service}
\item
  \href{https://help.nytimes.com/hc/en-us/articles/115014893968-Terms-of-sale}{Terms
  of Sale}
\item
  \href{https://spiderbites.nytimes.com}{Site Map}
\item
  \href{https://help.nytimes.com/hc/en-us}{Help}
\item
  \href{https://www.nytimes.com/subscription?campaignId=37WXW}{Subscriptions}
\end{itemize}
