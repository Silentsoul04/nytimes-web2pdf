 **NYTimes.com no longer supports Internet Explorer 9 or earlier. Please
upgrade your browser.
\href{http://www.nytimes.com/content/help/site/ie9-support.html}{LEARN
MORE »}

**Sections

**Home

**Search

\hypertarget{the-new-york-times}{%
\subsection{\texorpdfstring{\href{http://www.nytimes.com/}{The New York
Times}}{The New York Times}}\label{the-new-york-times}}

 \href{/section/world/asia}{Asia Pacific} \textbar{}What Is Article 370,
and Why Does It Matter in Kashmir?

Log In

**0

**Settings

**Close search

\hypertarget{site-search-navigation}{%
\subsection{Site Search Navigation}\label{site-search-navigation}}

Search NYTimes.com

**Clear this text input

Go

\url{https://nyti.ms/2IJbac6}

\begin{enumerate}
\def\labelenumi{\arabic{enumi}.}
\item
  Loading...
\end{enumerate}

See next articles

See previous articles

\hypertarget{site-navigation}{%
\subsection{Site Navigation}\label{site-navigation}}

\hypertarget{site-mobile-navigation}{%
\subsection{Site Mobile Navigation}\label{site-mobile-navigation}}

Advertisement

\hypertarget{-asia-pacific-}{%
\subsection{\texorpdfstring{ \href{/section/world/asia}{Asia Pacific}
}{ Asia Pacific }}\label{-asia-pacific-}}

\hypertarget{what-is-article-370-and-why-does-it-matter-in-kashmir}{%
\section{What Is Article 370, and Why Does It Matter in
Kashmir?}\label{what-is-article-370-and-why-does-it-matter-in-kashmir}}

Disputed area

PAKISTAN

INDIA

Kashmir, a mountainous valley that borders Pakistan and India, has been
a center of conflict between the two nuclear-armed countries since the
1947 partition of British India.

At the time of the partition, the British agreed to divide their former
colony into two countries: Pakistan, with a Muslim majority, and India,
with a Hindu majority. Both nations covet Kashmir, which is Muslim
majority, and occupy portions of it with military forces.

For decades, an uneasy stalemate has prevailed, broken by occasional
military incursions, terrorist attacks and police crackdowns. But on
Monday,
\href{https://www.nytimes.com/2019/08/05/world/asia/india-pakistan-kashmir-jammu.html}{the
Indian government decided to permanently incorporate the territory it
controls into the rest of India}.

The administration of Prime Minister Narendra Modi revoked Article 370
of the Indian constitution, a 70-year-old provision that had given
autonomy to the state of Jammu and Kashmir, which includes the
Hindu-majority area of Jammu and the Muslim-majority Kashmir valley.

The government also introduced a bill to strip the region of statehood
and divide it into two parts, both under direct control of the central
government.

But Mr. Modi, a Hindu nationalist, had campaigned for re-election in
part by stoking patriotic fervor against Muslim-led Pakistan. He
promised the full integration of Kashmir, a cause which his party has
championed for decades, and now he is delivering on that pledge.

Pakistan condemned India's moves. Pakistan's prime minister, Imran Khan,
called on President Trump to follow through on an
\href{https://www.nytimes.com/2019/07/22/world/asia/trump-pakistan-afghanistan.html}{offer
he made two weeks ago} to mediate the Kashmir dispute.

\hypertarget{what-are-the-roots-of-the-conflict}{%
\subsection{What are the roots of the
conflict?}\label{what-are-the-roots-of-the-conflict}}

\includegraphics{https://static01.nyt.com/packages/flash/multimedia/ICONS/transparent.png}

Muhammad Ali Jinnah, the founder of Pakistan, on the country's first day
as a nation. Lord Louis Mountbatten, left, had overseen the partition of
the subcontinent. Associated Press

In 1947, the sudden separation of the area into Pakistan and India
prompted millions of people to migrate between the two countries and led
to religious violence that killed hundreds of thousands.

afghanistan

china

Gilgit-Baltistan

Controlled by Pakistan

boundary

undefined

line of control

Feb. 26

Airstrikes

Jammu and

Kashmir

Controlled by India

Feb. 14

Suicide bombing

pakistan

pak.

india

India

afghan.

china

Gilgit-Baltistan

Controlled by Pakistan

boundary

undefined

line of control

Feb. 26

Airstrikes

Jammu and Kashmir

Controlled by India

pakistan

Feb. 14

Suicide bombing

pak.

india

India

afghan.

china

Gilgit-Baltistan

Controlled by Pakistan

boundary

undefined

line of control

Feb. 26

Balakot

Jammu and

Kashmir

Controlled by India

Feb. 14

Suicide bombing

pakistan

pak.

India

india

By Scott Reinhard

Left undecided was the status of Jammu and Kashmir, a Muslim-majority
state in the Himalayas that had been ruled by a local prince. Fighting
quickly broke out, and both countries eventually sent in troops, with
Pakistan occupying about one-third of the state and India two-thirds.

The prince signed an agreement for the territory to become part of
India. Regional autonomy, which was formalized through Article 370, was
a key inducement.

\includegraphics{https://static01.nyt.com/packages/flash/multimedia/ICONS/transparent.png}

The village of Gujran, in the upper section of Tulail Valley, Kashmir.
Michael Benanav for The New York Times

Despite efforts by the United Nations to mediate the Kashmir dispute,
India and Pakistan continue to administer their portions of the former
princely territory while hoping to get full control of it. Troops on
both sides of the so-called ``line of control'' regularly fire volleys
at each other.

Muslim militants have frequently resorted to violence to expel the
Indian troops from the territory. Pakistan has backed many of those
militants, as well as terrorists who have struck deep inside India ---
most brutally in a four-day killing spree in Mumbai in 2008, which left
more than 160 people dead.

\hypertarget{what-is-article-370}{%
\subsection{What is Article 370?}\label{what-is-article-370}}

Article 370 was added to the Indian constitution shortly after the
partition of British India to give autonomy to the former princely state
of Jammu and Kashmir until a decision was made about its rule. It
limited the power of India's central government over the territory. A
related provision gave state lawmakers the power to decide who could buy
land and be a permanent resident -\/- a provision that irked many
non-Kashmiris.

Although it was intended to be temporary, Article 370 says that it can
only be abrogated with the consent of the legislative body that drafted
the state constitution. That body dissolved itself in 1957, and India's
Supreme Court ruled last year that Article 370 is therefore a permanent
part of the constitution.

The Modi government disagrees and says the president of India, who is
beholden to the ruling party, has the power to revoke the article.

\hypertarget{why-did-the-conflict-heat-up-this-year}{%
\subsection{Why did the conflict heat up this
year?}\label{why-did-the-conflict-heat-up-this-year}}

\includegraphics{https://static01.nyt.com/packages/flash/multimedia/ICONS/transparent.png}

Indian soldiers examine debris after an explosion in Pulwama, in
southern Kashmir. Younis Khaliq/Reuters

The immediate cause was the Feb. 14 suicide bombing by a young Islamic
militant, who blew up a convoy of trucks carrying paramilitary forces in
Pulwama in southern Kashmir.

Gilgit-Baltistan

Controlled by Pakistan

PAKISTAN

line of control

Feb. 26

Balakot

Airstrikes

Jammu and Kashmir

Controlled by India

Feb. 14

Suicide bombing

Pulwama

pak.

india

Gilgit-Baltistan

Controlled by Pakistan

Feb. 26

Balakot

Airstrikes

line of control

Jammu and Kashmir

Controlled by India

Feb. 14

Suicide bombing

Pulwama

pak.

india

PAKISTAN

Gilgit-Baltistan

Controlled by Pakistan

line of control

Feb. 26

Balakot

Airstrikes

Jammu and Kashmir

Controlled by India

Feb. 14

Suicide bombing

Pulwama

pak.

india

By Scott Reinhard

Indian aircraft responded to that attack by flying into Pakistan and
\href{https://www.nytimes.com/2019/02/26/world/asia/india-pakistan-kashmir-airstrikes.html}{firing
airstrikes near the town of Balakot}. The Indian government claimed it
was attacking a training camp for Jaish-e-Mohammed, the terrorist group
that claimed responsibility for the bombing.

The next day,
\href{https://www.nytimes.com/2019/02/27/world/asia/kashmir-india-pakistan-aircraft.html}{Pakistani
and Indian fighter jets engaged in a skirmish over Indian-controlled
territory}, and Pakistani forces downed an Indian aircraft --- an
\href{https://www.nytimes.com/2019/03/03/world/asia/india-military-united-states-china.html}{aging
Soviet-era MiG-21} --- and captured its pilot. It was the first aerial
clash between the rivals
\href{https://www.nytimes.com/2019/03/03/world/asia/india-military-united-states-china.html?module=inline}{in
five decades}.

Pakistan quickly
\href{https://www.nytimes.com/2019/03/01/world/asia/india-pakistan-plane-abhinandan-varthaman-india.html?module=inline}{returned
the pilot}, easing the diplomatic tensions. But Mr. Modi exploited a
wave of a nationalist fervor over the Pulwama attack as part of his
re-election campaign that helped his Bharatiya Janata Party win a
\href{https://www.nytimes.com/2019/05/23/world/asia/narendra-modi-election-win.html}{sweeping
victory}.

Pakistan's prime minister, Imran Khan, was elected last year with the
backing of his country's powerful military, and he wants to show that he
can stand up to India, even as his country's economy is so weak that he
\href{https://www.nytimes.com/2019/02/18/world/asia/saudi-arabia-prince-mohammed-pakistan.html}{sought
bailouts from Saudi Arabia} and China.

\hypertarget{will-the-united-states-and-other-global-powers-get-involved}{%
\subsection{Will the United States and other global powers get
involved?}\label{will-the-united-states-and-other-global-powers-get-involved}}

On July 22, Mr. Trump hosted Mr. Khan at the White House. Although the
meeting was focused on how to end the war in Afghanistan,
\href{https://www.nytimes.com/2019/07/22/world/asia/trump-pakistan-afghanistan.html}{Mr.
Trump told reporters that Mr. Modi had asked him to help mediate the
Kashmir dispute}. Mr. Khan welcomed his involvement. The Indian
government denied making any mediation request and has long insisted on
direct negotiations with Pakistan to resolve the dispute.

Under Mr. Trump, American foreign policy has shifted away from Pakistan,
a longtime recipient of American aid, toward India, which the
administration views as a bulwark against China's rising influence in
Asia.

China, meanwhile, has
\href{https://www.nytimes.com/2018/12/19/world/asia/pakistan-china-belt-road-military.html}{become
a close ally and financial patron of Pakistan}. The Chinese government
recently urged India and Pakistan to settle their conflicts through
bilateral discussions. China shares a border with Jammu and Kashmir
state, and India and China still do not agree on the demarcation line.

\hypertarget{what-is-likely-to-happen-next}{%
\subsection{What is likely to happen
next?}\label{what-is-likely-to-happen-next}}

\includegraphics{https://static01.nyt.com/packages/flash/multimedia/ICONS/transparent.png}

Indian soldiers near the remains of an Indian aircraft after it crashed
on Wednesday. Danish Ismail/Reuters

The constitutional changes, issued through a presidential order, could
face legal challenges. Last year,
\href{https://timesofindia.indiatimes.com/india/article-370-has-acquired-permanent-status-supreme-court/articleshow/63603527.cms}{India's
Supreme Court ruled that Article 370 could not be abrogated} because the
state-level body that would have to approve the change went out of
existence in 1957.

``My view is that this presidential notification is illegal,'' said
Shubhankar Dam, a law professor at the University of Portsmouth in
Britain and the author of a book on executive power in India. ``The
question is one of jurisdiction: Does the government of India have the
power to do this?''

Pakistan, for its part, said it will ``exercise all possible options to
counter the illegal steps'' taken by India.

Mr. Modi's moves to integrate Kashmir into India are likely to be
popular in much of the country. But there is widespread panic in
Kashmir, where there have been decades of protests against Indian rule.

Suhasini Raj contributed reporting from New Delhi and Ayesha
Venkataraman contributed research from Mumbai.

\hypertarget{more-on-nytimescom}{%
\subsection{More on NYTimes.com}\label{more-on-nytimescom}}

Advertisement

\hypertarget{site-information-navigation}{%
\subsection{Site Information
Navigation}\label{site-information-navigation}}

\begin{itemize}
\tightlist
\item
  \href{https://help.nytimes.com/hc/en-us/articles/115014792127-Copyright-notice}{©
  2020 The New York Times Company}
\item
  \href{https://www.nytimes.com}{Home}
\item
  \href{https://www.nytimes.com/search/}{Search}
\item
  Accessibility concerns? Email us at
  \href{mailto:accessibility@nytimes.com}{\nolinkurl{accessibility@nytimes.com}}.
  We would love to hear from you.
\item
  \href{https://help.nytimes.com/hc/en-us/articles/115015385887-Contact-Us}{Contact
  Us}
\item
  \href{https://www.nytco.com/careers/}{Work with us}
\item
  \href{https://nytmediakit.com/}{Advertise}
\item
  \href{https://help.nytimes.com/hc/en-us/articles/115014892108-Privacy-policy\#pp}{Your
  Ad Choices}
\item
  \href{https://help.nytimes.com/hc/en-us/articles/115014892108-Privacy-policy}{Privacy}
\item
  \href{https://help.nytimes.com/hc/en-us/articles/115014893428-Terms-of-service}{Terms
  of Service}
\item
  \href{https://help.nytimes.com/hc/en-us/articles/115014893968-Terms-of-sale}{Terms
  of Sale}
\end{itemize}

\hypertarget{site-information-navigation-1}{%
\subsection{Site Information
Navigation}\label{site-information-navigation-1}}

\begin{itemize}
\tightlist
\item
  \href{https://spiderbites.nytimes.com}{Site Map}
\item
  \href{https://help.nytimes.com/hc/en-us}{Help}
\item
  \href{https://help.nytimes.com/hc/en-us/articles/115015385887-Contact-Us?redir=myacc}{Site
  Feedback}
\item
  \href{https://www.nytimes.com/subscription?campaignId=37WXW}{Subscriptions}
\end{itemize}
