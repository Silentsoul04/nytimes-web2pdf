Sections

SEARCH

\protect\hyperlink{site-content}{Skip to
content}\protect\hyperlink{site-index}{Skip to site index}

\hypertarget{comments}{%
\subsection{\texorpdfstring{\protect\hyperlink{commentsContainer}{Comments}}{Comments}}\label{comments}}

\href{}{`We Take the Dead From Morning Till Night'}\href{}{Skip to
Comments}

The comments section is closed. To submit a letter to the editor for
publication, write to
\href{mailto:letters@nytimes.com}{\nolinkurl{letters@nytimes.com}}.

\hypertarget{we-take-the-dead-from-morning-till-night}{%
\section{`We Take the Dead From Morning Till
Night'}\label{we-take-the-dead-from-morning-till-night}}

By Fabio Bucciarelli and
\href{https://www.nytimes.com/by/jason-horowitz}{Jason Horowitz}March
27, 2020

\begin{itemize}
\item
\item
\item
\item
\item
  \emph{+}
\end{itemize}

No country has been hit harder by the coronavirus than Italy, and no
province has suffered as many losses as Bergamo. Photos and voices from
there evoke a portrait of despair.

\hypertarget{-bergamo-italy}{%
\subsection{**** Bergamo, Italy}\label{-bergamo-italy}}

This is the bleak heart of the world's deadliest coronavirus outbreak.

There have been 11,591 deaths in Italy, more than China and Spain
combined, many times more than the United States.

And in Italy the most deaths are in the Bergamo area. Officially 1,969
people have died there. The actual toll may be four times higher, so
many that the local paper is given over to death notices.

\hypertarget{we-take-the-dead-from-morning-till-night-1}{%
\section{`We Take the Dead From Morning Till
Night'}\label{we-take-the-dead-from-morning-till-night-1}}

Photographs by Fabio Bucciarelli Written by Jason Horowitz March 27,
2020
\href{https://www.nytimes.com/it/interactive/2020/03/29/world/europe/coronavirus-italy-bergamo.html}{Leggere
in italiano}

The streets of Bergamo are empty. As in all of Italy, people can leave
their homes only for food and medicines and work. The factories and
shops and schools are closed. There is no more chatting on the corners
or in the coffee bars.

But what won't stop are the sirens.

While the world's attention now shifts to its own centers of contagion,
the sirens keep sounding. Like the air raid sirens of the Second World
War, they are the ambulance sirens that many survivors of this war will
remember. They blare louder as they get closer, coming to collect the
parents and grandparents, the keepers of Italy's memory.

The grandchildren wave from terraces, and spouses sit back on the
corners of now empty beds. And then the sirens start again, becoming
fainter as the ambulances drive away toward hospitals crammed with
coronavirus patients.

``At this point, all you hear in Bergamo is sirens,'' said Michela
Travelli.

On March 7, her father, Claudio Travelli, 60, was driving a food
delivery truck all around northern Italy. The next day, he developed a
fever and flu-like symptoms. His wife had run a fever in recent days,
and so he called his family doctor, who told him to take a common
Italian fever reducer and rest up.

For much of the prior month, Italian officials had sent mixed messages
about the virus.

On Feb. 19, some 40,000 people from Bergamo, a province of about a
million people in the region of Lombardy, traveled 30 miles to Milan to
watch a Champions League soccer game between Atalanta and the Spanish
team Valencia. (The mayor of Bergamo, Giorgio Gori, this week called the
match ``a strong accelerator of contagion.'') Mr. Travelli and his wife
didn't take the threat of the virus seriously back then, their daughter
said, ``because it wasn't sold as a grave thing.''

But Mr. Travelli could not shake his fever, and he got sicker.

On Friday, March 13, he felt unbearable pressure on his chest and
suffered dry heaves. His temperature spiked and his family called an
ambulance. An ambulance crew found her father with low levels of oxygen
in his blood but, following the advice of Bergamo's hospitals,
recommended he stay home. ``They said, `We have seen worse, and the
hospitals are like the trenches of a war,''' Ms. Travelli said.

Another day at home led to a night of coughing fits and fever. On
Sunday, Mr. Travelli woke up and wept, saying, ``I'm sick. I can't do it
anymore,'' his daughter said. He took more fever suppressant but his
temperature climbed to nearly 103 degrees and his skin became yellow.

This time, as the ambulance arrived, his daughters, both wearing gloves
and masks, packed a bag with two pairs of pajamas, a bottle of water, a
cellphone and a charger. His oxygen levels had dipped.

Red Cross workers hovered over him on a bed, where he lay below a
painting of the Virgin Mary. They brought him into the ambulance. His
granddaughters, 3 and 6, waved goodbye from the terrace. He looked up at
them, at the balconies draped with Italian flags. Then the ambulance
left and there was nothing to hear. ``Only the police and the sirens,''
his daughter said.

The ambulance crew that took Mr. Travelli away had started early that
morning.

At 7:30 a.m., a crew of three Red Cross volunteers met to make sure the
ambulance was certified as cleaned and stocked with oxygen. Like masks
and gloves, the tanks had become an increasingly rare resource. They
blasted one another in sprays of alcohol disinfectants. They sanitized
their cellphones.

``We can't be the \emph{untori},'' said Nadia Vallati, 41, a Red Cross
volunteer, whose day job is working in the city's tax office. She was
referring to the infamous ``anointers,'' suspected in Italian lore of
spreading contagion during the 17th century plague. After sanitizing,
Ms. Vallati and her colleagues wait for an alarm to sound in their
headquarters. It never takes long.

Indistinguishable from one another in the white medical scrubs pulled
over their red uniforms, crew members entered Mr. Travelli's home on
March 15 with tanks of oxygen. ``Always with oxygen,'' Ms. Vallati said.

One of the biggest dangers for coronavirus patients is hypoxemia, or low
blood oxygen. Normal readings are between 95 and 100, and doctors worry
when the number dips below 90.

Ms. Vallati said she had found coronavirus patients with readings of 50.
Their lips are blue. Their fingertips turn violet. They take rapid,
shallow breaths and use their stomach muscles to pull in air. Their
lungs are too weak.

In many of the apartments they visit, patients clutch small oxygen
tanks, the size of SodaStreams, that are procured for them with a
doctor's prescription by family members. They lie in bed next to them.
They eat with them at the kitchen table. They watch the nightly reports
of Italy's dead and infected with them on their couches.

On March 15, Ms. Vallati put her hand, wrapped in two layers of blue
gloves, on the chest of Teresina Coria, 88, as they measured her oxygen
level. The next day, Antonio Amato, an outlier at the age of 40, sat in
his armchair, holding his oxygen tank as his children, whom he could not
hold for fear of contagion, waved to him from across the room.

On a Saturday, Ms. Vallati found herself in the bedroom of a 90-year-old
man. She asked his two granddaughters if he had had any contact with
anyone who had the coronavirus. Yes, they said, the man's son, their
father, who had died on Wednesday. Their grandmother, they told her, had
been taken away on Friday and was in critical condition.

They weren't crying, she said, because ``they didn't have any tears
left.''

On another recent tour in the highly infected Valle Seriana under the
Alps, Ms. Vallati said, they picked up a woman of about 80. Her husband
of many decades asked to kiss her goodbye. But Ms. Vallati told him he
could not, because the risk of contagion was too high. As the man
watched the crew take his wife away, Ms. Vallati saw him go into another
room and close the door behind him, she said.

While those suspected of infection are taken to hospitals, the hospitals
themselves are not safe. Bergamo officials first detected the
coronavirus at the Pesenti Fenaroli di Alzano Lombardo hospital.

By then, officials say, it had already been present for some time,
masked as ordinary pneumonia, infecting other patients, doctors, and
nurses. People carried it out of the hospital and into the city, out of
the city and into the province. Young people passed it to their parents
and grandparents. It spread around bingo halls and over coffee cups.

The mayor, Mr. Gori, has talked about how infections have ravaged his
town and nearly broken one of Europe's wealthiest and most sophisticated
health care systems. Doctors estimate that 70,000 people in the province
have the virus. Bergamo has had to send 400 bodies to other provinces
and regions and countries because there is no room for them there.

``If we have to identify a spark,'' he said, ``it was the hospital.''

When an ambulance arrives, its crew proceeds with extreme caution. Only
one of the three, the team leader, accompanies the patient inside. If
the patient is heavy, another helps.

This weekend, a group of doctors from one Bergamo hospital
\href{https://catalyst.nejm.org/doi/full/10.1056/CAT.20.0080?fbclid=IwAR0wa6jzq-t_YYlZlYQtWiVmphT8pjyGBCndLhJGSN34dBaeZJoGP0sfneo}{wrote
in a medical journal} associated with The New England Journal of
Medicine that ``we are learning that hospitals might be the main
Covid-19 carriers'' and ``as they are rapidly populated by infected
patients, facilitating transmission to uninfected patients.''

Ambulances and their personnel get infected, they said, but perhaps show
no symptoms, and spread the virus further. As a result, the doctors
urged home care and mobile clinics to avoid bringing people to the
hospital unless absolutely necessary.

But Ms. Vallati said they had no choice with the gravest cases. The
authors of the paper work at Bergamo's Papa Giovanni XXIII, where Ms.
Vallati's crew have taken many of the sick.

Dr. Ivano Riva, an anesthesiologist there, said the hospital was still
admitting up to 60 new coronavirus patients a day. They are tested for
the virus he said, but at this point the clinical evidence --- the
coughs, the low oxygen levels, the fevers --- is a better indicator,
especially since 30 percent of the tests produced false negatives.

The hospital had 500 coronavirus patients, who occupied all 90 I.C.U.
beds. About a month ago, the hospital had seven such beds.

Oxygen flows everywhere through Lombardy's hospitals now, and workers
are constantly pushing carts of tanks around the corridors. A tanker
truck filled with oxygen is parked outside. Patients are jammed next to
supply closets and in hallways.

Dr. Riva said 26 of his hospital's 101 medical staff members were out of
work with the virus. ``It's a situation that no one has ever seen, I
don't think in any other part of the world,'' he said.

If people don't stay at home, he said, ``the system will fail.''

His colleagues wrote in the paper that intensive care unit beds were
reserved for coronavirus patients with ``a reasonable chance to
survive.'' Older patients, they said, ``are not being resuscitated and
die alone.''

Mr. Travelli ended up at the nearby Humanitas Gavazzeni hospital, where,
after a false negative, he tested positive for the virus. He is still
alive.

``Papi, you were lucky because you found a bed --- now you have to
fight, fight, fight,'' his daughter Michela told him in a telephone
call, their last before he was fitted with a helmet to ease his
breathing. ``He was scared,'' she said. ``He thought he was dying.''

In the meantime, Ms. Travelli said she had been quarantined and had lost
her sense of taste for food, a frequent complaint among people without
symptoms, but who have had close contact with the virus.

So many people are dying so quickly, the hospital mortuaries and funeral
workers cannot keep up. ``We take the dead from the morning till night,
one after the other, constantly,'' said Vanda Piccioli, who runs one of
the last funeral homes to remain open. Others have closed as a result of
sick funeral directors, some in intensive care. ``Usually we honor the
dead. Now it's like a war and we collect the victims.''

Ms. Piccioli said one member of her staff had died of the virus on
Sunday. She considered closing but decided they had a responsibility to
keep going, despite what she said was constant terror of infection and
emotional trauma. ``You are a sponge and you take the pain of
everybody,'' she said.

She said her staff moved 60 infected bodies daily, from Papa Giovanni
and Alzano hospitals, from clinics, from nursing homes and apartments.
``It's hard for us to get masks and gloves,'' she said. ``We are a
category in the shadows.''

Ms. Piccioli said that in the beginning, they sought to get the personal
effects of the dead, kept in red plastic bags, back to their loved ones.
A tin of cookies. A mug. Pajamas. Slippers. But now they simply don't
have time.

Still, the calls to the Red Cross crew do not stop.

On March 19, Ms. Vallati and her crew entered the apartment of Maddalena
Peracchi, 74, in Gazzaniga. She had run out of oxygen. Her daughter
Cinzia Cagnoni, 43, who lives in the apartment downstairs, had placed an
order for a new tank that would arrive on Monday, but the Red Cross
volunteers told her she wouldn't hold out that long.

``We were a little agitated because we knew that this could be the last
time we saw each other,'' Ms. Cagnoni said. ``It's like sending someone
to die alone.''

She and her sister and her father put on a brave face under their masks,
she said. ``You can do it,'' they told her mother, she said. ``We will
wait for you, there are still so many things we need to do with you.
Fight back.''

The volunteers brought Ms. Peracchi down to the ambulance. One of her
daughters urged her stunned grandchildren to bid farewell with louder
voices. ``I thought a thousand things,'' Ms. Cagnoni said. ``Don't
abandon me. God help us. God save my mother.'' The ambulance doors
closed. The sirens sounded, as they do ``all the hours of the day,'' Ms.
Cagnoni said.

The crew drove to Pesenti Fenaroli di Alzano Lombardo, where Ms.
Peracchi was found to have the coronavirus and pneumonia on both sides
of her lungs. On Thursday night, her daughter said she was ``holding on
by a thread.''

Ms. Peracchi is a woman of deep Catholic faith, said her daughter, who
spiked a temperature herself the night the ambulance took her mother
away and has remained quarantined since.

It pained her mother, she said, that if it came to it, ``we cannot have
a funeral.''

To contain the virus, all religious and civil celebrations are banned in
Italy. That includes funerals. Bergamo's cemetery is locked shut. A
chilling backlog of coffins waits in a traffic jam for the crematorium
inside the cemetery's church.

Officials have banned changing the clothes of the dead and require that
people be buried or cremated in the pajamas or medical gowns they perish
in. Corpses need to be wrapped in an extra bag or cloaked in a
disinfecting cloth. The lids of coffins, which usually cannot be closed
without a formal death certificate, now can be, though they still have
to wait to be sealed. Bodies often linger in homes for days, as stairs
and stuffy rooms become especially dangerous.

``We are trying to avoid it,'' the funeral director, Ms. Piccioli, said
of home visits. Nursing homes were much easier because you could arrive
with five or six coffins to be filled and loaded directly into the vans.
``I know it's terrible to say,'' she said.

Through a network of local priests, she helps arrange quick prayers,
rather than proper funerals, for the dead and the families who are not
quarantined.

That was the case for Teresina Gregis, who was interred at the Alzano
Lombardo cemetery on March 21 after she died at home. Ambulance workers
had told her family that there was no room in the hospitals.

``All the beds are full,'' they told the family, according to her
daughter-in-law, Romina Mologni, 34. Since she was 75, she said, ``they
gave priority to others who were younger.''

In her last weeks at home, her family struggled to find tanks of oxygen,
driving all over the province as she sat facing her garden and the
pinwheels she adored.

When she died, all the flower shops were closed because of the lockdown.
Ms. Mologni instead brought to the cemetery one of the pinwheels her own
daughter had given her grandmother. ``She liked that one.''

Photo editing by David Furst and Gaia Tripoli. Design and development by
Rebecca Lieberman and Matt Ruby.

Obituary from L'Eco di Bergamo, March 13, 2020.

Related Stories

\href{https://www.nytimes.com/2020/03/21/world/europe/italy-coronavirus-center-lessons.html}{\includegraphics{https://static01.nyt.com/images/2020/03/21/world/21italy-virus1sub-copy/21italy-virus1sub-copy-slide-v3.jpg}}

\hypertarget{italy-pandemics-new-epicenter-has-lessons-for-the-world}{%
\subsubsection{Italy, Pandemic's New Epicenter, Has Lessons for the
World}\label{italy-pandemics-new-epicenter-has-lessons-for-the-world}}

The country's experience shows that steps to isolate the coronavirus and
limit people's movement need to be put in place early, with absolute
clarity, then strictly enforced.

\href{https://www.nytimes.com/2020/03/16/world/europe/italy-coronavirus-funerals.html}{\includegraphics{https://static01.nyt.com/images/2020/03/16/world/16Italy-Bodies01/16Italy-Bodies01-slide.jpg}}

\hypertarget{italys-coronavirus-victims-face-death-alone-with-funerals-postponed}{%
\subsubsection{Italy's Coronavirus Victims Face Death Alone, With
Funerals
Postponed}\label{italys-coronavirus-victims-face-death-alone-with-funerals-postponed}}

As morgues are inundated, coffins pile up and mourners grieve in
isolation: ``This is the bitterest part.''

\href{https://www.nytimes.com/2020/03/12/world/europe/12italy-coronavirus-health-care.html}{\includegraphics{https://static01.nyt.com/images/2020/03/12/world/12italy-virus1/merlin_170410329_41442a1b-44e6-425a-91f5-11ba9e0faf04-slide.jpg}}

\hypertarget{italys-health-care-system-groans-under-coronavirus--a-warning-to-the-world}{%
\subsubsection{Italy's Health Care System Groans Under Coronavirus --- a
Warning to the
World}\label{italys-health-care-system-groans-under-coronavirus--a-warning-to-the-world}}

In less than three weeks, the virus has overloaded hospitals in northern
Italy, offering a glimpse of what countries face if they cannot slow the
contagion.

Write a comment

\begin{itemize}
\item
\item
\item
\item
\end{itemize}

Advertisement

\protect\hyperlink{after-bottom}{Continue reading the main story}

\hypertarget{site-index}{%
\subsection{Site Index}\label{site-index}}

\hypertarget{site-information-navigation}{%
\subsection{Site Information
Navigation}\label{site-information-navigation}}

\begin{itemize}
\tightlist
\item
  \href{https://help.nytimes.com/hc/en-us/articles/115014792127-Copyright-notice}{©~2020~The
  New York Times Company}
\end{itemize}

\begin{itemize}
\tightlist
\item
  \href{https://www.nytco.com/}{NYTCo}
\item
  \href{https://help.nytimes.com/hc/en-us/articles/115015385887-Contact-Us}{Contact
  Us}
\item
  \href{https://www.nytco.com/careers/}{Work with us}
\item
  \href{https://nytmediakit.com/}{Advertise}
\item
  \href{http://www.tbrandstudio.com/}{T Brand Studio}
\item
  \href{https://www.nytimes.com/privacy/cookie-policy\#how-do-i-manage-trackers}{Your
  Ad Choices}
\item
  \href{https://www.nytimes.com/privacy}{Privacy}
\item
  \href{https://help.nytimes.com/hc/en-us/articles/115014893428-Terms-of-service}{Terms
  of Service}
\item
  \href{https://help.nytimes.com/hc/en-us/articles/115014893968-Terms-of-sale}{Terms
  of Sale}
\item
  \href{https://spiderbites.nytimes.com}{Site Map}
\item
  \href{https://help.nytimes.com/hc/en-us}{Help}
\item
  \href{https://www.nytimes.com/subscription?campaignId=37WXW}{Subscriptions}
\end{itemize}
