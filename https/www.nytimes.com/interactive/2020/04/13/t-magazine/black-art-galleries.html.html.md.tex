\hypertarget{how-a-trio-of-black-owned-galleries-changed-the-art-world}{%
\section{How a Trio of Black-Owned Galleries Changed the Art
World}\label{how-a-trio-of-black-owned-galleries-changed-the-art-world}}

April 13, 2020

\begin{itemize}
\item
\item
\item
\item
\end{itemize}

In the 1960s and '70s, Brockman Gallery, Gallery 32 and JAM led the way
in showing the work of artists now among the most influential of our
time.

\href{https://www.nytimes.com/interactive/2020/04/13/t-magazine/culture-issue-2020.html}{We
Are Family}

\hypertarget{chapter-1-heirs-and-alumni}{%
\subparagraph{Chapter 1: Heirs and
Alumni}\label{chapter-1-heirs-and-alumni}}

\hypertarget{previous}{%
\subparagraph{Previous}\label{previous}}

\hypertarget{next}{%
\subparagraph{Next}\label{next}}

\hypertarget{how-a-trio-of-black-owned-galleries-changed-the-art-world-1}{%
\section{How a Trio of Black-Owned Galleries Changed the Art
World}\label{how-a-trio-of-black-owned-galleries-changed-the-art-world-1}}

\hypertarget{the-artists}{%
\subsection{The Artists}\label{the-artists}}

For decades, the art world ignored artists of color --- an institutional
neglect it's now trying to correct. But in the 1960s and '70s, in Los
Angeles and New York, three galleries led the way in showing the work of
black artists, many of whom are now among the most influential of our
time.

By \href{https://www.nytimes.com/by/m-h-miller}{M.H. Miller}

April 13, 2020

SHARE

IN THE PAST few years, cultural institutions have been trying to create
a more inclusive narrative of contemporary art history, one that
contains more women and people of color --- people who were denied
successful careers a half-century ago simply because they weren't white
men. Today, it's not uncommon to see black artists with solo shows at
museums and galleries that just five years ago might have ignored them
entirely.

Despite this correction, black-owned commercial galleries remain
rarities in America. For a brief period in the 1960s and '70s, however,
there was an alternative art world --- first in Los Angeles, then in New
York --- that offered a view of contemporary art that was vibrant and
welcoming. Five decades later, it's even more influential than it was
then.

The first major gallery run by and for black artists was Brockman
Gallery, founded in 1967 by two artist brothers, Alonzo Davis and Dale
Brockman Davis, in Los Angeles's Leimert Park neighborhood. As the
historian Kellie Jones notes in her 2017 study,
``\href{https://read.dukeupress.edu/books/book/2349/South-of-PicoAfrican-American-Artists-in-Los}{South
of Pico: African American Artists in Los Angeles in the 1960s and
1970s},'' storefront space was easy to come by in the wake of the
\href{https://www.nytimes.com/2015/08/11/us/50-years-after-watts-riots-a-recovery-is-in-progress.html}{Watts
rebellion}, a series of riots that took place in August 1965 in
predominantly black Los Angeles communities. The Davis brothers overcame
difficult odds to run their own business, having grown up in the Jim
Crow South, where being an artist, not to mention a black artist, was
unheard-of. Over the next 23 years, Brockman --- which was named for the
brothers' maternal grandmother --- helped cultivate a roster of young,
mostly unknown artists who are now familiar names, among them Dan
Concholar,
\href{https://www.nytimes.com/2019/07/15/t-magazine/most-important-contemporary-art.html}{David
Hammons}, Maren Hassinger, Ulysses Jenkins, Senga Nengudi, John
Outterbridge and Noah Purifoy.

The T List \textbar{}

Sign up here

The same year the Davises opened their gallery, a young painter and
ballet dancer named Suzanne Jackson arrived in Los Angeles from San
Francisco. In 1968, she began taking figure-drawing classes at the Otis
Art Institute with Charles White, the best-known black artist of the
'40s and '50s, who became a mentor to many of the city's younger
talents. The art community was so small that Jackson encountered Alonzo
Davis at an artist's house in Echo Park not long after moving to the
city, and later met Hammons in one of White's classes. At the time, she
was searching for a new studio and found one not far from the school,
just west of downtown. To sign the lease, she told her landlord she was
going to open a gallery, thinking the landlord wouldn't understand the
concept of an artist's studio. ``I was just going to use the space to
paint,'' she said, but Hammons --- ``being kind of nosy, he was always
around,'' Jackson said of him --- encouraged her to actually use the
building as she promised her landlord she would. Named after its
address, Gallery 32 opened in March 1969 and created an energized
community of artists who had previously been relegated to showing their
work at community centers or in people's backyards.

Hammons was just one unifying thread of this community, though there
were others. Brockman and Gallery 32 usually coordinated their openings
to fall on the same weekends, and they shared collectors and artists as
well, including
\href{https://www.nytimes.com/2019/09/04/arts/design/betye-saar.html}{Betye
Saar}, Timothy Washington and Nengudi. After a year and a half of losing
money on postage and printing invitations, Jackson closed her space in
1970 but later worked for Brockman Gallery.

THEN, IN THE 1970S, Hammons moved to New York and became a major reason
a young single mother named
\href{https://www.nytimes.com/2019/10/11/nyregion/linda-goode-bryant-project-eats.html}{Linda
Goode Bryant} opened her own gallery, Just Above Midtown (known as JAM):
In 1973, Bryant was working as the director of education at the Studio
Museum in Harlem, another important touchstone for black artists of the
era. She was familiar with Hammons's work and asked if he'd ever show in
a New York gallery. He told Bryant, ``I don't show in white galleries.''
Her response: ``Well, I guess I have to start a gallery.'' She was the
only black gallery owner in a building on 57th Street full of exhibition
spaces. (Bryant recalls that whenever she ran into Allan Frumkin, a
dealer of mostly realist paintings, he'd tell her, ``You don't belong
here.'')

An exterior view of Brockman Gallery on the opening night of the first
gallery exhibition in 1967. The gallery was located at 4334 Degnan
Boulevard in the Leimert Park neighborhood of Los Angeles. Los Angeles
Public Library Special Collections

A poster for ``Contemporary Black Imagery'' at Brockman Gallery, a group
exhibition featuring work by Herman Bailey, David Bradford, Elizabeth
Catlett, David Hammons, Suzanne Jackson, Timothy Washington and Roland
Welton. Los Angeles Public Library Special Collections

In an undated photograph, Dale Brockman Davis is shown at top right with
the Bell Tower group installation in Watts, Los Angeles. Los Angeles
Public Library Special Collections

An undated poster for an exhibition at Brockman Gallery featuring the
artists Noah Purifoy and David Hammons. Los Angeles Public Library
Special Collections

An undated poster for a show at Gallery 32 featuring a photo of Annie
Bianucci, George Evans Jr. and Suzanne Jackson by Greg Edwards. Photo ©
Gregory Wiley Edwards. Courtesy of Suzanne Jackson

On the reverse side of an exhibition poster for Gallery 32, a hand-drawn
map by Suzanne Jackson depicting the gallery's location. Courtesy of
Suzanne Jackson

A poster for a show at Gallery 32 featuring David Hammons. Courtesy of
Suzanne Jackson

A poster for an exhibition at Gallery 32 featuring a photograph by
Phillip Jackson of a sculpture by Timothy Washington. Courtesy of
Suzanne Jackson

Emory Douglas's ``See Revolutionary Art Exhibit'' (1969), held at
Gallery 32. Collection of the Center for the Study of Political
Graphics. © 2018 Emory Douglas/Artist Rights Society (ARS), New York

An installation view from David Hammons's 1975 JAM show, ``Greasy Bags
and BBQ Bones.'' Courtesy Linda Goode Bryant

Lorraine O'Grady's ``Untitled (Mlle Bourgeoise Noire Shouts Out Her
Poem)'' (1980-83/2009), performances made with unannounced appearances
at openings at both Just Above Midtown (JAM) and the New Museum of
Contemporary Art. Courtesy Alexander Gray Associates, New York. ©
Lorraine O'Grady/Artists Rights Society (ARS), New York

A postcard for the artist Senga Nengudi's 1977 exhibition ``R.S.V.P'' at
JAM. Senga Nengudi papers, 1947, circa 1962-2017, Archives of American
Art, Smithsonian Institution

An undated color photocopy with installation instructions for Nengudi's
1977 ``R.S.V.P'' exhibit at JAM. Senga Nengudi papers, 1947, circa
1962-2017, Archives of American Art, Smithsonian Institution

Nengundi's artist's questionnaire for JAM, circa 1977. Senga Nengudi
papers, 1947, circa 1962-2017, Archives of American Art, Smithsonian
Institution

Many of the artists who first worked with Brockman and Gallery 32 went
on to show with JAM, which was, for a time, the only place in New York
that would give them an arena. Each of these places was as important a
gallery as the most storied exhibition spaces of the '60s and '70s: Los
Angeles's Ferus Gallery, where many West Coast artists (Ed Ruscha,
Robert Irwin, Larry Bell) debuted, and Leo Castelli's townhouse on East
77th Street, which introduced New York to
\href{https://www.nytimes.com/2019/02/18/t-magazine/jasper-johns.html}{Jasper
Johns},
\href{https://www.nytimes.com/topic/person/robert-rauschenberg}{Robert
Rauschenberg} and
\href{https://www.nytimes.com/topic/person/andy-warhol}{Andy Warhol}.
There remains a consensus that these galleries were legendary, though
both largely overlooked artists of color. (According to the critic
Miranda Sawyer, Castelli, who ran his space from 1957 until his death in
1999, even rejected Jean-Michel Basquiat, who in the 1980s became the
most famous black artist to show in white galleries, as ``too
troublesome.'') Castelli became shorthand for the rapidly growing
commercial market of contemporary art, but not 20 blocks away was an
entirely different world, one that was largely ignored by the
traditional power brokers; The Times, for instance, did not review any
JAM gallery shows. The artists who got their starts here --- Bryant
would also give
\href{https://www.nytimes.com/2019/05/01/arts/design/dawoud-bey-david-hammons-jam-frieze.html}{Dawoud
Bey}, Lorraine O'Grady,
\href{https://www.nytimes.com/2019/03/23/arts/design/black-artists-older-success.html}{Howardena
Pindell} and Fred Wilson their first shows before closing her space in
1986 --- are now canonical.

As a result, anecdotes from this era take on an almost mythological
quality. In 1975, Bryant was preparing a show of Hammons's body prints,
works on paper that the artist made by covering his body in margarine or
grease. He'd press himself against the paper, and then dust the
impression with ground pigment, producing a ghostlike image that was
somewhere between a self-portrait and a Rorschach test. Hammons had been
showing these works in Los Angeles, and they had become popular with
collectors, so to cover postage for her announcement cards and to pay
her printer, Bryant got people to commit to buying new body-print works
in advance, before Hammons arrived to install the show. When she called
Hammons to discuss logistics, she asked him how many body prints he was
bringing.

``I ain't doing that anymore,'' Hammons told her.

``Oh,'' Bryant replied, trying to mask her anxiety. ``What \emph{are}
you doing?''

``Brown paper bags, barbecue bones, grease and hair,'' he said. He was
making sculptures out of these materials, gathering hair from the floors
of Harlem barbershops.

``Oh really?'' Bryant said into the phone, and then covered the
mouthpiece to shout an expletive.

There had been a longstanding division between black artists who made
figurative work and black artists who worked in abstraction, and people
from both camps came to the opening of the show, which was called,
appropriately, ``Greasy Bags and Barbecue Bones.'' ``The place was
packed,'' Bryant said. All of them felt a little betrayed. The
figurative artists were upset that Hammons had turned his back on the
form, and the abstract artists felt he was debasing their work by using
such blatantly discarded materials. Tensions were high, so Bryant hushed
the group and suggested everyone ``sit down and just talk.'' They did,
while Hammons stood on the sidelines, not speaking, only listening, and
at the end of the night, ``the artists that worked figuratively and the
artists that worked abstractly shook hands and walked out the door,''
Bryant said. The rift between the two sides was resolved. ``That was the
end of that debate.'' In the end, she wasn't able to pay her printer.

M.H. Miller is a features director at T Magazine. Wayne Lawrence is a
documentary photographer. L.A. grooming: Olivia Fischa at
celestineagency.com. L.A. production: Peter McClafferty. N.Y. grooming:
Laura de Leon at Joe Management. Grooming assistants: Emily Klein,
Robert Reyes, Jo Franco and Destiny Venice.

\href{https://www.nytimes.com/2019/11/19/t-magazine/suzanne-jackson-artist.html}{}

\hypertarget{an-artist-who-makes-paintings-without-a-canvasnov-19-2019}{%
\paragraph{An Artist Who Makes Paintings Without a CanvasNov. 19,
2019}\label{an-artist-who-makes-paintings-without-a-canvasnov-19-2019}}

\includegraphics{https://static01.nyt.com/images/2019/11/19/t-magazine/19tmag-jackson-slide-USMT/19tmag-jackson-slide-USMT-mediumThreeByTwo210-v2.jpg}
\href{https://www.nytimes.com/2018/09/28/t-magazine/art/charles-white-moma-retrospective.html}{}

\hypertarget{the-man-who-taught-a-generation-of-black-artists-gets-his-own-retrospectivesept-28-2018}{%
\paragraph{The Man Who Taught a Generation of Black Artists Gets His Own
RetrospectiveSept. 28,
2018}\label{the-man-who-taught-a-generation-of-black-artists-gets-his-own-retrospectivesept-28-2018}}

\includegraphics{https://static01.nyt.com/images/2018/09/28/t-magazine/art/charles-white-slide-7RSF/charles-white-slide-7RSF-mediumThreeByTwo210.jpg}
\href{https://www.nytimes.com/2018/06/20/t-magazine/black-art-dealers.html}{}

\hypertarget{why-have-there-been-no-great-black-art-dealers-june-20-2018}{%
\paragraph{Why Have There Been No Great Black Art Dealers? June 20,
2018}\label{why-have-there-been-no-great-black-art-dealers-june-20-2018}}

\includegraphics{https://static01.nyt.com/images/2018/06/21/t-magazine/art/art-dealers-slide-21LU/art-dealers-slide-21LU-mediumThreeByTwo210.jpg}

\hypertarget{we-are-family-1}{%
\subsubsection{We Are Family}\label{we-are-family-1}}

\hypertarget{chapter-1-heirs-and-alumni-1}{%
\paragraph{Chapter 1: Heirs and
Alumni}\label{chapter-1-heirs-and-alumni-1}}

\href{/interactive/2020/04/13/t-magazine/black-art-galleries.html}{}

\hypertarget{the-artists-1}{%
\subparagraph{The Artists}\label{the-artists-1}}

\href{/interactive/2020/04/13/t-magazine/italian-fashion-design-houses.html}{}

\hypertarget{the-dynasties}{%
\subparagraph{The Dynasties}\label{the-dynasties}}

\href{/interactive/2020/04/13/t-magazine/gordon-parks.html}{}

\hypertarget{the-directors}{%
\subparagraph{The Directors}\label{the-directors}}

\href{/interactive/2020/04/13/t-magazine/enrique-olvera-chef.html}{}

\hypertarget{the-disciples}{%
\subparagraph{The Disciples}\label{the-disciples}}

\href{/interactive/2020/04/13/t-magazine/royal-academy-antwerp.html}{}

\hypertarget{the-graduates}{%
\subparagraph{The Graduates}\label{the-graduates}}

\hypertarget{chapter-2-reunions-and-reconsiderations}{%
\paragraph{Chapter 2: Reunions and
Reconsiderations}\label{chapter-2-reunions-and-reconsiderations}}

\href{/interactive/2020/04/13/t-magazine/ninth-street-greenwich-village-neighbors.html}{}

\hypertarget{the-neighbors}{%
\subparagraph{The Neighbors}\label{the-neighbors}}

\href{/interactive/2020/04/13/t-magazine/omen-restaurant-nyc.html}{}

\hypertarget{the-regulars}{%
\subparagraph{The Regulars}\label{the-regulars}}

\href{/interactive/2020/04/13/t-magazine/hair-musical-broadway.html}{}

\hypertarget{hair-1967}{%
\subparagraph{Hair (1967)}\label{hair-1967}}

\href{/interactive/2020/04/13/t-magazine/sweeney-todd-revival.html}{}

\hypertarget{sweeney-todd-2005-revival}{%
\subparagraph{Sweeney Todd (2005
Revival)}\label{sweeney-todd-2005-revival}}

\href{/interactive/2020/04/13/t-magazine/daughters-of-the-dust.html}{}

\hypertarget{daughters-of-the-dust-1991}{%
\subparagraph{Daughters of the Dust
(1991)}\label{daughters-of-the-dust-1991}}

\hypertarget{chapter-3-legends-pioneers-and-survivors}{%
\paragraph{Chapter 3: Legends Pioneers and
Survivors}\label{chapter-3-legends-pioneers-and-survivors}}

\href{/interactive/2020/04/13/t-magazine/butch-stud-lesbian.html}{}

\hypertarget{the-renegades}{%
\subparagraph{The Renegades}\label{the-renegades}}

\href{/interactive/2020/04/13/t-magazine/act-up-aids.html}{}

\hypertarget{the-activists}{%
\subparagraph{The Activists}\label{the-activists}}

\href{/interactive/2020/04/13/t-magazine/artist-recluse.html}{}

\hypertarget{the-shadows}{%
\subparagraph{The Shadows}\label{the-shadows}}

\href{/interactive/2020/04/13/t-magazine/black-actresses-bassett-berry-blige-henson-whitfield-elise.html}{}

\hypertarget{the-veterans}{%
\subparagraph{The Veterans}\label{the-veterans}}

\hypertarget{chapter-4-the-new-guard}{%
\paragraph{Chapter 4: The New Guard}\label{chapter-4-the-new-guard}}

\href{/interactive/2020/04/13/t-magazine/asian-american-fashion-designers.html}{}

\hypertarget{the-designers}{%
\subparagraph{The Designers}\label{the-designers}}

\href{13tmag-beauties.html}{}

\hypertarget{the-beauties}{%
\subparagraph{The Beauties}\label{the-beauties}}

\href{/interactive/2020/04/13/t-magazine/nyc-downtown-nightlife-party-scene.html}{}

\hypertarget{the-scenemakers}{%
\subparagraph{The Scenemakers}\label{the-scenemakers}}

\href{/interactive/2020/04/13/t-magazine/maria-cornejo-olivier-rousteing-telfar-clemens-alessandro-michele.html\#olivier-rousteing-and-co}{}

\hypertarget{olivier-rousteing-and-co}{%
\subparagraph{Olivier Rousteing and
Co.}\label{olivier-rousteing-and-co}}

\href{/interactive/2020/04/13/t-magazine/maria-cornejo-olivier-rousteing-telfar-clemens-alessandro-michele.html\#maria-cornejo-and-co}{}

\hypertarget{maria-cornejo-and-co}{%
\subparagraph{Maria Cornejo and Co.}\label{maria-cornejo-and-co}}

\href{/interactive/2020/04/13/t-magazine/maria-cornejo-olivier-rousteing-telfar-clemens-alessandro-michele.html\#telfar-clemens-and-co}{}

\hypertarget{telfar-clemens-and-co}{%
\subparagraph{Telfar Clemens and Co.}\label{telfar-clemens-and-co}}

\href{/interactive/2020/04/13/t-magazine/maria-cornejo-olivier-rousteing-telfar-clemens-alessandro-michele.html\#alessandro-michele-and-co}{}

\hypertarget{alessandro-michele-and-co}{%
\subparagraph{Alessandro Michele and
Co.}\label{alessandro-michele-and-co}}

\href{/interactive/2020/04/13/t-magazine/foreign-correspondents.html}{}

\hypertarget{the-journalists}{%
\subparagraph{The Journalists}\label{the-journalists}}

\begin{itemize}
\item
\item
\item
\item
\end{itemize}

Advertisement

\protect\hyperlink{after-bottom}{Continue reading the main story}

\hypertarget{site-index}{%
\subsection{Site Index}\label{site-index}}

\hypertarget{site-information-navigation}{%
\subsection{Site Information
Navigation}\label{site-information-navigation}}

\begin{itemize}
\tightlist
\item
  \href{https://help.nytimes.com/hc/en-us/articles/115014792127-Copyright-notice}{©~2020~The
  New York Times Company}
\end{itemize}

\begin{itemize}
\tightlist
\item
  \href{https://www.nytco.com/}{NYTCo}
\item
  \href{https://help.nytimes.com/hc/en-us/articles/115015385887-Contact-Us}{Contact
  Us}
\item
  \href{https://www.nytco.com/careers/}{Work with us}
\item
  \href{https://nytmediakit.com/}{Advertise}
\item
  \href{http://www.tbrandstudio.com/}{T Brand Studio}
\item
  \href{https://www.nytimes.com/privacy/cookie-policy\#how-do-i-manage-trackers}{Your
  Ad Choices}
\item
  \href{https://www.nytimes.com/privacy}{Privacy}
\item
  \href{https://help.nytimes.com/hc/en-us/articles/115014893428-Terms-of-service}{Terms
  of Service}
\item
  \href{https://help.nytimes.com/hc/en-us/articles/115014893968-Terms-of-sale}{Terms
  of Sale}
\item
  \href{https://spiderbites.nytimes.com}{Site Map}
\item
  \href{https://help.nytimes.com/hc/en-us}{Help}
\item
  \href{https://www.nytimes.com/subscription?campaignId=37WXW}{Subscriptions}
\end{itemize}
