Sections

SEARCH

\protect\hyperlink{site-content}{Skip to
content}\protect\hyperlink{site-index}{Skip to site index}

\hypertarget{comments}{%
\subsection{\texorpdfstring{\protect\hyperlink{commentsContainer}{Comments}}{Comments}}\label{comments}}

\href{}{The Creative Circles Defining the Culture}\href{}{Skip to
Comments}

The comments section is closed. To submit a letter to the editor for
publication, write to
\href{mailto:letters@nytimes.com}{\nolinkurl{letters@nytimes.com}}.

\hypertarget{the-creative-circles-defining-the-culture}{%
\section{The Creative Circles Defining the
Culture}\label{the-creative-circles-defining-the-culture}}

April 13, 2020

\begin{itemize}
\item
\item
\item
\item
\item
  \emph{+}
\end{itemize}

The Veterans

The Graduates

The Activists

The Beauties

The Neighbors

The Scenemakers

The Renegades

The Disciples

Presents

\hypertarget{we-are-family}{%
\subsection{We Are Family}\label{we-are-family}}

In our 2020 Culture issue, out April 19, T celebrates various groups of
creative people who, whether united by outlook or identity, happenstance
or choice, built communities that have shaped the larger cultural
landscape. \textbf{Read more about the making of the issue.}

EVERY MAGAZINE IS by its nature retrospective, a time capsule from the
near past. A magazine such as this takes months to photograph, write,
edit and research, and a few weeks to print; this means that the things
that were true at its conception are sometimes no longer so when it's
published. As I write this, two days before this issue ships to the
printers, it is March 19, 2020, nine months since we started working on
it, six days since The Times urged its employees to work from home, four
days since the mayor ordered the closure of New York's restaurants and
bars and the first of what will likely be many days of our living in a
never-ending Samuel Beckett play. Yet while the world around us has
changed in ways that were --- just a few weeks ago --- once reserved for
the realm of fiction, the spirit and thesis of this issue has not.

One of the things that has defined our age has been the rise and
dominance of what we can colloquially call tribes, groups of people
bound not by blood or genetics or law, but by something more profound
and just as durable --- call it an affinity, if you will. Sometimes that
affinity has its roots in race, or gender, or sexuality, but it's just
as often based in something not innate, but developed: taste, say, or
sensibility, or experience, or history. These are assemblages of people
not born unto one another, but who \emph{find} one another, and as a
result, their bond is more charged, more powerful, more intimate.

Of course, tribalism --- the word is used most often by pundits, usually
despairingly --- has been a scourge of our age as well, splintering the
romance of the national body. But we are not speaking here of those who
try to rally people with the idea of a shared history in order to
protect their own way of life \emph{against} another's ambitions and
strivings; we are instead celebrating groups that comprise creative
people, each of whom has changed or defined their respective artistic
mediums and, by extension, changed or defined the culture at large.

Admittedly, ```tribe' is a troublesome word, bearing the weight of
decades of anthropological study that privileged Western civilization
over all other traditions,''
\href{https://www.nytimes.com/interactive/2020/04/13/t-magazine/tribe-meaning.html}{writes
Ligaya Mishan} in her lovely, wise essay about the etymology of the
term. But, she adds, ``let us rescue it here, pare it down to its
simplest meaning, as a name for the first human communities that formed
beyond the primal bonds of kinship --- the beginnings of the great
experiment we call society, which taught us to be human.''

\href{https://www.nytimes.com/interactive/2020/04/13/t-magazine/butch-stud-lesbian.html}{}

\includegraphics{https://static01.nyt.com/packages/flash/multimedia/ICONS/transparent.png}

\includegraphics{https://static01.nyt.com/newsgraphics/2020/03/07/tmag-culture-landing/assets/images/cover1-2000.jpg}

Photos by Collier Schorr. Styled by Brian Molloy. Produced by Casey
Legler.

\href{https://www.nytimes.com/interactive/2020/04/13/t-magazine/black-actresses-bassett-berry-blige-henson-whitfield-elise.html}{}

\includegraphics{https://static01.nyt.com/packages/flash/multimedia/ICONS/transparent.png}

\includegraphics{https://static01.nyt.com/newsgraphics/2020/03/07/tmag-culture-landing/assets/images/cover3-2000.jpg}

Photos by Mickalene Thomas and Racquel Chevremont. Styled by Shiona
Turini.

\href{https://www.nytimes.com/interactive/2020/04/13/t-magazine/dominican-republic-models.html}{}

\includegraphics{https://static01.nyt.com/packages/flash/multimedia/ICONS/transparent.png}

\includegraphics{https://static01.nyt.com/newsgraphics/2020/03/07/tmag-culture-landing/assets/images/cover2-2000.jpg}

Photo by Willy Vanderperre. Styled by Olivier Rizzo.

\href{https://www.nytimes.com/interactive/2020/04/13/t-magazine/maria-cornejo-olivier-rousteing-telfar-clemens-alessandro-michele.html\#alessandro-michele-and-co}{}

\includegraphics{https://static01.nyt.com/packages/flash/multimedia/ICONS/transparent.png}

\includegraphics{https://static01.nyt.com/newsgraphics/2020/03/07/tmag-culture-landing/assets/images/cover4-2000.jpg}

Photo by Nick Waplington.

SO WHAT MAKES a 21st-century tribe? A shared adversity, for one. In many
cases, its members have spent their lives fighting for recognition and
respect, sometimes even from within their own families or communities,
among people who were supposed to protect them. This is why certain
places, certain industries, are more conducive to tribes than others.
New York is
\href{https://www.nytimes.com/interactive/2020/04/13/t-magazine/ninth-street-greenwich-village-neighbors.html}{one
such place}; fashion
\href{https://www.nytimes.com/interactive/2020/04/13/t-magazine/maria-cornejo-olivier-rousteing-telfar-clemens-alessandro-michele.html}{is
another}. Both are significantly populated by people who at one point
(or many) in their childhoods realized that they didn't quite belong to
the people to whom they were assigned, who decided that they were
someday going to discover --- or create --- their own commune. Then
there's the kind of adversity faced by people like the
\href{https://www.nytimes.com/interactive/2020/04/13/t-magazine/black-actresses-bassett-berry-blige-henson-whitfield-elise.html}{renowned
black actresses} we gathered; now middle-aged, most of them have been
acting or performing for decades, but due to entrenched systemic racism
and sexism, it has only been in the past few years that they've found
themselves considered commercial assets in addition to artistic ones. Or
there's
\href{https://www.nytimes.com/interactive/2020/04/13/t-magazine/act-up-aids.html}{the
members of ACT UP}, the unapologetically impolite activist group
co-founded by the writer Larry Kramer in 1987, at the height of the AIDS
crisis in America. Like all significant activist groups in this country,
it owes a debt to its predecessors in the civil rights and feminism
movements. It also inspired the next generation of activists: As David
France writes, its influence --- its members' understanding of the media
and the theater of demonstration; their emphasis on self-education and
skepticism about authority --- lives on in Black Lives Matter, the
Women's March, the Parkland student-led gun-control groups and many
more. Its influence is even and especially felt in this moment, as we
stagger through the worst plague since H.I.V. and AIDS (with an
essential difference: this disease, unlike its 20th-century counterpart,
did not at its start disproportionately affect a community denied their
fundamental human rights). The transparency that we now expect public
health officials to offer is a direct legacy of ACT UP's necessary
provocations.

BUT THE OTHER thing that binds many of these tribes is love. Notice I
didn't say ``like'': Not all of the people in these taxonomies
necessarily like one another. They hadn't necessarily even met one
another.

But they all \emph{knew} one another, or of one another. One of the
great joys of working on this issue --- one I experienced firsthand, as
did the various editors, writers, stylists and art directors who
attended the 22 shoots that our indefatigable photo team oversaw in a
five-month-long process --- was witnessing the astonishment, wonder and
affection expressed by so many of the subjects upon encountering one
another. ``\emph{You're} here?'' someone would ask. ``I can't believe
\emph{you're} here!'' It was the kind of pure delight that's reserved
for the most special of meetings, the one in which you're finally
reunited with the long-lost relative you had waited your whole life to
see and be seen by. \emph{Here you are, one of my people. You belong
with me: With you, I belong}. Sometimes, in our recognizing them as a
group, they saw themselves as one, too. The result was a series of
almost giddy reunions, sometimes made giddier for being so unexpected. A
\href{https://www.nytimes.com/interactive/2020/04/13/t-magazine/black-art-galleries.html}{group
of black artists}, now elderly and among the most celebrated in the
world, all of whom had once showed at one or all of the three
black-owned galleries that would change contemporary art in the '70s and
'80s, broke into an impromptu version of ``Signed, Sealed, Delivered I'm
Yours''; many of the
\href{https://www.nytimes.com/interactive/2020/04/13/t-magazine/butch-stud-lesbian.html}{butch
and stud lesbians}, who had admired one another from afar, left together
for a bar after an eight-hour-long shoot; the reconvened casts of
genre-altering productions of
``\href{https://www.nytimes.com/interactive/2020/04/13/t-magazine/hair-musical-broadway.html}{Hair}''
and
``\href{https://www.nytimes.com/interactive/2020/04/13/t-magazine/sweeney-todd-revival.html}{Sweeney
Todd}'' burst into spontaneous song.

As I said, the members of these groups didn't necessarily all like one
another (though watching them, it was hard to believe otherwise). But
they all recognized one another. And isn't that what love is? Having
someone see you as not only the person you are, but as someone who is
part of something larger, some community you yourself hadn't seen?
Certainly that's true for me. And though I don't like to speak for other
people's definition of love, I'm going to say --- just because I saw it
--- that it was true for them as well. --- HANYA YANAGIHARA

\href{https://www.nytimes.com/interactive/2020/04/13/t-magazine/tribe-meaning.html}{\includegraphics{https://static01.nyt.com/newsgraphics/2020/03/07/tmag-culture-landing/assets/images/what_is_a_tribe.png}}

\includegraphics{https://static01.nyt.com/packages/flash/multimedia/ICONS/transparent.png}

\includegraphics{https://static01.nyt.com/newsgraphics/2020/03/07/tmag-culture-landing/assets/images/ch1-title.png}

\hypertarget{heirs-and-alumni}{%
\subsection{Heirs and Alumni}\label{heirs-and-alumni}}

\href{https://www.nytimes.com/interactive/2020/04/13/t-magazine/black-art-galleries.html}{}

\includegraphics{https://static01.nyt.com/packages/flash/multimedia/ICONS/transparent.png}

\includegraphics{https://static01.nyt.com/images/2020/04/13/t-magazine/13tmag-cultureimages-slide-8PFV/13tmag-cultureimages-slide-8PFV-master1050.jpg}

Wayne Lawrence

\hypertarget{the-artists}{%
\subsubsection{The Artists}\label{the-artists}}

\hypertarget{by-m-h-miller}{%
\subparagraph{By M. H. Miller}\label{by-m-h-miller}}

For decades, the art world ignored artists of color --- an institutional
neglect it's now trying to correct. But in the 1960s and '70s, in Los
Angeles and New York, three galleries led the way in showing the work of
black artists, many of whom are now among the most influential of our
time.

\href{https://www.nytimes.com/interactive/2020/04/13/t-magazine/italian-fashion-design-houses.html}{}

\includegraphics{https://static01.nyt.com/packages/flash/multimedia/ICONS/transparent.png}

\includegraphics{https://static01.nyt.com/images/2020/04/13/t-magazine/13tmag-cultureimages-slide-4H6Q/13tmag-cultureimages-slide-4H6Q-master1050.jpg}

Simon Watson

\hypertarget{the-dynasties}{%
\subsubsection{The Dynasties}\label{the-dynasties}}

\hypertarget{by-nancy-hass}{%
\subparagraph{By Nancy Hass}\label{by-nancy-hass}}

For centuries, Italy has prized the art of fashion and furniture design
--- and like no other country in the world, its makers have served as
cultural custodians, with generations dedicated to the craftsmanship,
continuity and traditions of the family-run company.

\href{https://www.nytimes.com/interactive/2020/04/13/t-magazine/gordon-parks.html}{}

\includegraphics{https://static01.nyt.com/packages/flash/multimedia/ICONS/transparent.png}

\includegraphics{https://static01.nyt.com/images/2020/04/13/t-magazine/13tmag-cultureimages-slide-0QS8/13tmag-cultureimages-slide-0QS8-master1050.jpg}

Bon Duke

\hypertarget{the-directors}{%
\subsubsection{The Directors}\label{the-directors}}

\hypertarget{by-a-o-scott}{%
\subparagraph{By A. O. Scott}\label{by-a-o-scott}}

In 1969, Gordon Parks became the first black director to make a major
Hollywood studio film. His career made it possible for the next
generation to fight their way into the mainstream --- only to face the
same opposition Parks had.

\href{https://www.nytimes.com/interactive/2020/04/13/t-magazine/enrique-olvera-chef.html}{}

Video by Tony Floyd

\hypertarget{the-disciples}{%
\subsubsection{The Disciples}\label{the-disciples}}

\hypertarget{by-ligaya-mishan}{%
\subparagraph{By Ligaya Mishan}\label{by-ligaya-mishan}}

In the age of the visionary cook-as-auteur, those who have trained under
the Mexican chef Enrique Olvera have not only reconceived their
country's cuisine --- but have gone on to change how and what we want to
eat.

\href{https://www.nytimes.com/interactive/2020/04/13/t-magazine/royal-academy-antwerp.html}{}

Video by Pascal Gambarte

\hypertarget{the-graduates}{%
\subsubsection{The Graduates}\label{the-graduates}}

\hypertarget{by-alice-newell-hanson}{%
\subparagraph{By Alice Newell-Hanson}\label{by-alice-newell-hanson}}

How the Royal Academy of Fine Arts in Antwerp --- a city poised on the
edge of Europe and the rest of the world --- became the incubator for
the contemporary avant-garde.

\includegraphics{https://static01.nyt.com/packages/flash/multimedia/ICONS/transparent.png}

\includegraphics{https://static01.nyt.com/newsgraphics/2020/03/07/tmag-culture-landing/assets/images/ch2-title.png}

\hypertarget{reunions-and-reconsiderations}{%
\subsection{Reunions and
Reconsiderations}\label{reunions-and-reconsiderations}}

\href{https://www.nytimes.com/interactive/2020/04/13/t-magazine/ninth-street-greenwich-village-neighbors.html}{}

Video by Sean Donnola

\hypertarget{the-neighbors}{%
\subsubsection{The Neighbors}\label{the-neighbors}}

\hypertarget{introduction-by-kate-guadagnino}{%
\subparagraph{Introduction by Kate
Guadagnino}\label{introduction-by-kate-guadagnino}}

As told to Merrell Hambleton and Samuel Rutter

The two blocks in Greenwich Village that have been home to a
disproportionate number of New York City's writers, artists, actors and
designers for decades.

\href{https://www.nytimes.com/interactive/2020/04/13/t-magazine/omen-restaurant-nyc.html}{}

\includegraphics{https://static01.nyt.com/packages/flash/multimedia/ICONS/transparent.png}

\includegraphics{https://static01.nyt.com/images/2020/04/13/t-magazine/13tmag-cultureimages-slide-RWL4/13tmag-cultureimages-slide-RWL4-master1050.jpg}

Jason Schmidt

\hypertarget{the-regulars}{%
\subsubsection{The Regulars}\label{the-regulars}}

\hypertarget{by-patti-smith}{%
\subparagraph{By Patti Smith}\label{by-patti-smith}}

From its opening, the Japanese restaurant Omen has been a gathering
place for artists, writers, actors and fashion designers. Forty years
later, in a vastly different downtown New York, it still is.

\hypertarget{the-forerunners}{%
\subsubsection{The Forerunners}\label{the-forerunners}}

Every now and then, a piece of American performance is so memorable that
it both redefines its medium and reframes the culture at large. From a
controversial, youth-inflected 1967 musical that captured and encouraged
the popular (and political) consciousness to a 2005 Broadway revival
that brought a much-needed austerity and economy to the stage to a
seminal 1991 film that centered the black female experience and gaze,
here are appraisals of three such enduring --- and heavily referenced
--- works, alongside gatherings of the stars who not only made them but
were made by them, too.

\href{https://www.nytimes.com/interactive/2020/04/13/t-magazine/hair-musical-broadway.html}{}

\includegraphics{https://static01.nyt.com/packages/flash/multimedia/ICONS/transparent.png}

\includegraphics{https://static01.nyt.com/images/2020/04/13/t-magazine/13tmag-cultureimages-slide-1KZG/13tmag-cultureimages-slide-1KZG-master1050.jpg}

Nicholas Calcott

\hypertarget{hair-1967}{%
\subsubsection{Hair (1967)}\label{hair-1967}}

\hypertarget{by-ben-brantley}{%
\subparagraph{By Ben Brantley}\label{by-ben-brantley}}

\href{https://www.nytimes.com/interactive/2020/04/13/t-magazine/sweeney-todd-revival.html}{}

\includegraphics{https://static01.nyt.com/packages/flash/multimedia/ICONS/transparent.png}

\includegraphics{https://static01.nyt.com/images/2020/04/13/t-magazine/13tmag-cultureimages-slide-2D85/13tmag-cultureimages-slide-2D85-master1050.jpg}

Jennifer Livingston

\hypertarget{sweeney-todd-2005-revival}{%
\subsubsection{Sweeney Todd (2005
Revival)}\label{sweeney-todd-2005-revival}}

\hypertarget{by-patricia-cohen}{%
\subparagraph{By Patricia Cohen}\label{by-patricia-cohen}}

\href{https://www.nytimes.com/interactive/2020/04/13/t-magazine/daughters-of-the-dust.html}{}

\includegraphics{https://static01.nyt.com/packages/flash/multimedia/ICONS/transparent.png}

\includegraphics{https://static01.nyt.com/newsgraphics/2020/03/07/tmag-culture-landing/assets/images/daughters-2000.png}

Andres Gonzalez and David Chow

\hypertarget{daughters-of-the-dust-1991}{%
\subsubsection{Daughters of the Dust
(1991)}\label{daughters-of-the-dust-1991}}

\hypertarget{by-a-o-scott-1}{%
\subparagraph{By A. O. Scott}\label{by-a-o-scott-1}}

\includegraphics{https://static01.nyt.com/packages/flash/multimedia/ICONS/transparent.png}

\includegraphics{https://static01.nyt.com/newsgraphics/2020/03/07/tmag-culture-landing/assets/images/ch3-title.png}

\hypertarget{legends-pioneers-and-survivors}{%
\subsection{Legends Pioneers and
Survivors}\label{legends-pioneers-and-survivors}}

\href{https://www.nytimes.com/interactive/2020/04/13/t-magazine/butch-stud-lesbian.html}{}

Video by Caroline Berler

\hypertarget{the-renegades}{%
\subsubsection{The Renegades}\label{the-renegades}}

\hypertarget{by-kerry-manders}{%
\subparagraph{By Kerry Manders}\label{by-kerry-manders}}

Queer culture and the arts would be much poorer without the presence and
contribution of butch and stud lesbians, whose identity is both its own
aesthetic and a defiant repudiation of the male gaze.

\href{https://www.nytimes.com/interactive/2020/04/13/t-magazine/act-up-aids.html}{}

Video by Rosie Marks

\hypertarget{the-activists}{%
\subsubsection{The Activists}\label{the-activists}}

\hypertarget{by-david-france}{%
\subparagraph{By David France}\label{by-david-france}}

How ACT UP --- the coalition that fought against AIDS stigma and won
medications that slowed the plague --- forever changed patients' rights,
protests and American political organizing as it's practiced today.

\href{https://www.nytimes.com/interactive/2020/04/13/t-magazine/artist-recluse.html}{}

\includegraphics{https://static01.nyt.com/packages/flash/multimedia/ICONS/transparent.png}

\includegraphics{https://static01.nyt.com/newsgraphics/2020/03/07/tmag-culture-landing/assets/images/shadows-2000.png}

Artwork by Jessica Wohl

\hypertarget{the-shadows}{%
\subsubsection{The Shadows}\label{the-shadows}}

\hypertarget{by-megan-ogrady}{%
\subparagraph{By Megan O'Grady}\label{by-megan-ogrady}}

These days, artists of all kinds are expected to be available for public
consumption. But a small and highly influential group of them has chosen
to disappear from society in favor of letting their work speak for
itself. What does it mean to be inaccessible in an age of oversharing?

\href{https://www.nytimes.com/interactive/2020/04/13/t-magazine/black-actresses-bassett-berry-blige-henson-whitfield-elise.html}{}

\includegraphics{https://static01.nyt.com/packages/flash/multimedia/ICONS/transparent.png}

\includegraphics{https://static01.nyt.com/newsgraphics/2020/03/07/tmag-culture-landing/assets/images/actresses-2000.jpg}

Mickalene Thomas and Racquel Chevremont

\hypertarget{the-veterans}{%
\subsubsection{The Veterans}\label{the-veterans}}

\hypertarget{by-brian-keith-jackson}{%
\subparagraph{By Brian Keith Jackson}\label{by-brian-keith-jackson}}

The journey of black actresses in Hollywood is hard and steep, though a
number of women have fought the odds to the top over long and
distinguished careers. They're not done yet.

\includegraphics{https://static01.nyt.com/packages/flash/multimedia/ICONS/transparent.png}

\includegraphics{https://static01.nyt.com/newsgraphics/2020/03/07/tmag-culture-landing/assets/images/ch4-title.png}

\hypertarget{the-new-guard}{%
\subsection{The New Guard}\label{the-new-guard}}

\href{https://www.nytimes.com/interactive/2020/04/13/t-magazine/asian-american-fashion-designers.html}{}

\includegraphics{https://static01.nyt.com/packages/flash/multimedia/ICONS/transparent.png}

\includegraphics{https://static01.nyt.com/images/2020/04/13/t-magazine/13tmag-cultureimages-slide-HDR2/13tmag-cultureimages-slide-HDR2-master1050.jpg}

Renee Cox

\hypertarget{the-designers}{%
\subsubsection{The Designers}\label{the-designers}}

\hypertarget{by-thessaly-la-force}{%
\subparagraph{By Thessaly La Force}\label{by-thessaly-la-force}}

Why there are so many Asian-Americans in fashion, and how they changed
the industry.

\href{https://www.nytimes.com/interactive/2020/04/13/t-magazine/dominican-republic-models.html}{}

\includegraphics{https://static01.nyt.com/packages/flash/multimedia/ICONS/transparent.png}

\includegraphics{https://static01.nyt.com/newsgraphics/2020/03/07/tmag-culture-landing/assets/images/models-2000.jpg}

Willy Vanderperre

\hypertarget{the-beauties}{%
\subsubsection{The Beauties}\label{the-beauties}}

\hypertarget{by-concepcioux301n-de-leoux301n}{%
\subparagraph{By Concepción de
León}\label{by-concepcioux301n-de-leoux301n}}

How a new generation of Dominican models has come to define the runways
--- and continues to shape our definition of what beauty looks like.

\href{https://www.nytimes.com/interactive/2020/04/13/t-magazine/nyc-downtown-nightlife-party-scene.html}{}

Video by Sacred Pact

\hypertarget{the-scenemakers}{%
\subsubsection{The Scenemakers}\label{the-scenemakers}}

\hypertarget{by-mitchell-kuga}{%
\subparagraph{By Mitchell Kuga}\label{by-mitchell-kuga}}

Long a capital of nightlife, from '70s discos to '90s raves, New York
City is now home to a robust scene of roving parties where anyone --- of
any race, gender or sexuality --- is welcome. These are the people who
make the party happen.

\href{https://www.nytimes.com/interactive/2020/04/13/t-magazine/maria-cornejo-olivier-rousteing-telfar-clemens-alessandro-michele.html}{}

\hypertarget{the-fashion-gangs}{%
\subsubsection{The Fashion Gangs}\label{the-fashion-gangs}}

\hypertarget{introduction-by-alice-newell-hanson}{%
\subparagraph{Introduction by Alice
Newell-Hanson}\label{introduction-by-alice-newell-hanson}}

Clothing design is often perceived as the work of a solitary and
singular talent, of an outsider looking in. But as Balmain's Olivier
Rousteing, Maria Cornejo, Gucci's Alessandro Michele and Telfar Clemens
prove, making clothes is in fact the ultimate act of collaboration.

\href{https://www.nytimes.com/interactive/2020/04/13/t-magazine/maria-cornejo-olivier-rousteing-telfar-clemens-alessandro-michele.html\#olivier-rousteing-and-co}{}

\includegraphics{https://static01.nyt.com/packages/flash/multimedia/ICONS/transparent.png}

\includegraphics{https://static01.nyt.com/images/2020/04/13/t-magazine/13tmag-cultureimages-slide-RJAK/13tmag-cultureimages-slide-RJAK-master1050.jpg}

Ilya Lipkin

\hypertarget{olivier-rousteing-and-co}{%
\subsubsection{Olivier Rousteing and
Co.}\label{olivier-rousteing-and-co}}

\hypertarget{by-m-h-miller-1}{%
\subparagraph{By M. H. Miller}\label{by-m-h-miller-1}}

\href{https://www.nytimes.com/interactive/2020/04/13/t-magazine/maria-cornejo-olivier-rousteing-telfar-clemens-alessandro-michele.html\#maria-cornejo-and-co}{}

\includegraphics{https://static01.nyt.com/packages/flash/multimedia/ICONS/transparent.png}

\includegraphics{https://static01.nyt.com/images/2020/04/13/t-magazine/13tmag-cultureimages-slide-NZ9C/13tmag-cultureimages-slide-NZ9C-master1050.jpg}

Olivia Arthur

\hypertarget{maria-cornejo-and-co}{%
\subsubsection{Maria Cornejo and Co.}\label{maria-cornejo-and-co}}

\hypertarget{by-kate-guadagnino}{%
\subparagraph{By Kate Guadagnino}\label{by-kate-guadagnino}}

\href{https://www.nytimes.com/interactive/2020/04/13/t-magazine/maria-cornejo-olivier-rousteing-telfar-clemens-alessandro-michele.html\#telfar-clemens-and-co}{}

\includegraphics{https://static01.nyt.com/packages/flash/multimedia/ICONS/transparent.png}

\includegraphics{https://static01.nyt.com/images/2020/04/13/t-magazine/13tmag-cultureimages-slide-XHEF/13tmag-cultureimages-slide-XHEF-master1050.jpg}

John Edmonds

\hypertarget{telfar-clemens-and-co}{%
\subsubsection{Telfar Clemens and Co.}\label{telfar-clemens-and-co}}

\hypertarget{by-brian-keith-jackson-1}{%
\subparagraph{By Brian Keith Jackson}\label{by-brian-keith-jackson-1}}

\href{https://www.nytimes.com/interactive/2020/04/13/t-magazine/maria-cornejo-olivier-rousteing-telfar-clemens-alessandro-michele.html\#alessandro-michele-and-co}{}

\includegraphics{https://static01.nyt.com/packages/flash/multimedia/ICONS/transparent.png}

\includegraphics{https://static01.nyt.com/newsgraphics/2020/03/07/tmag-culture-landing/assets/images/alessandro-michele-2000.jpg}

Nick Waplington

\hypertarget{alessandro-michele-and-co}{%
\subsubsection{Alessandro Michele and
Co.}\label{alessandro-michele-and-co}}

\hypertarget{by-alexa-brazilian}{%
\subparagraph{By Alexa Brazilian}\label{by-alexa-brazilian}}

\href{https://www.nytimes.com/interactive/2020/04/13/t-magazine/foreign-correspondents.html}{}

\includegraphics{https://static01.nyt.com/packages/flash/multimedia/ICONS/transparent.png}

\includegraphics{https://static01.nyt.com/images/2020/04/13/t-magazine/13tmag-cultureimages-slide-41WW/13tmag-cultureimages-slide-41WW-master1050.jpg}

Brian Finke

\hypertarget{the-journalists}{%
\subsubsection{The Journalists}\label{the-journalists}}

\hypertarget{introduction-by-jane-perlez}{%
\subparagraph{Introduction by Jane
Perlez}\label{introduction-by-jane-perlez}}

As told to Heather Corcoran, Tal McThenia and Mimi Vu

In this tumultuous period of American politics, there are perhaps more
foreign correspondents in Washington, D.C., than ever before, reporting
back to everywhere from Sweden to Singapore. What unites them is their
fight against the threat of misinformation and their struggle to
accurately inform their fellow citizens about what's happening here ---
and how it might affect them.

Digital production and design by Katherine Cusumano, Hilary Moss, Jacky
Myint and Caroline Newton.

Write a comment

\begin{itemize}
\item
\item
\item
\item
\end{itemize}

Advertisement

\protect\hyperlink{after-bottom}{Continue reading the main story}

\hypertarget{site-index}{%
\subsection{Site Index}\label{site-index}}

\hypertarget{site-information-navigation}{%
\subsection{Site Information
Navigation}\label{site-information-navigation}}

\begin{itemize}
\tightlist
\item
  \href{https://help.nytimes.com/hc/en-us/articles/115014792127-Copyright-notice}{©~2020~The
  New York Times Company}
\end{itemize}

\begin{itemize}
\tightlist
\item
  \href{https://www.nytco.com/}{NYTCo}
\item
  \href{https://help.nytimes.com/hc/en-us/articles/115015385887-Contact-Us}{Contact
  Us}
\item
  \href{https://www.nytco.com/careers/}{Work with us}
\item
  \href{https://nytmediakit.com/}{Advertise}
\item
  \href{http://www.tbrandstudio.com/}{T Brand Studio}
\item
  \href{https://www.nytimes.com/privacy/cookie-policy\#how-do-i-manage-trackers}{Your
  Ad Choices}
\item
  \href{https://www.nytimes.com/privacy}{Privacy}
\item
  \href{https://help.nytimes.com/hc/en-us/articles/115014893428-Terms-of-service}{Terms
  of Service}
\item
  \href{https://help.nytimes.com/hc/en-us/articles/115014893968-Terms-of-sale}{Terms
  of Sale}
\item
  \href{https://spiderbites.nytimes.com}{Site Map}
\item
  \href{https://help.nytimes.com/hc/en-us}{Help}
\item
  \href{https://www.nytimes.com/subscription?campaignId=37WXW}{Subscriptions}
\end{itemize}
