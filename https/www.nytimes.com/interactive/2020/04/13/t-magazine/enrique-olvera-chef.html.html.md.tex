\hypertarget{enrique-olvera-and-his-culinary-heirs-have-changed-how-and-what-we-eat}{%
\section{Enrique Olvera and His Culinary Heirs Have Changed How and What
We
Eat}\label{enrique-olvera-and-his-culinary-heirs-have-changed-how-and-what-we-eat}}

April 13, 2020

\begin{itemize}
\item
\item
\item
\item
\end{itemize}

The influential chef has reconceived Mexican cuisine, both in his own
country and beyond.

\href{https://www.nytimes.com/interactive/2020/04/13/t-magazine/culture-issue-2020.html}{We
Are Family}

\hypertarget{chapter-1-heirs-and-alumni}{%
\subparagraph{Chapter 1: Heirs and
Alumni}\label{chapter-1-heirs-and-alumni}}

\hypertarget{previous}{%
\subparagraph{Previous}\label{previous}}

\hypertarget{next}{%
\subparagraph{Next}\label{next}}

\hypertarget{enrique-olvera-and-his-culinary-heirs-have-changed-how-and-what-we-eat-1}{%
\section{Enrique Olvera and His Culinary Heirs Have Changed How and What
We
Eat}\label{enrique-olvera-and-his-culinary-heirs-have-changed-how-and-what-we-eat-1}}

\hypertarget{the-disciples}{%
\subsection{The Disciples}\label{the-disciples}}

In the age of the visionary cook-as-auteur, those who have trained under
the Mexican chef Enrique Olvera have not only reconceived their
country's cuisine --- but have gone on to change how and what we want to
eat.

By Ligaya Mishan

April 13, 2020

SHARE

THESE ARE TRIBAL markings, the way one veteran of the kitchen knows
another before a word is said. The penny's width of a fingertip forever
lost. The scars from the oyster knife through the hand, the skin crisped
by the oven rack, the counter where someone let a hot pan sit too long.

Those who choose cooking as a profession make up a great, sprawling
community that transcends borders. But within it are smaller sects, more
tightly bonded, often centered around a visionary chef who draws
acolytes from around the world with the promise of transforming the way
people eat. This has become an international phenomenon in recent
decades, as the restaurants of certain chefs --- from
\href{https://www.nytimes.com/topic/person/ferran-adria}{Ferran Adrià}
and his molecular sorcery at the now shuttered El Bulli on Spain's Costa
Brava to
\href{https://www.nytimes.com/interactive/2014/09/10/style/tmagazine/redzepi-searches-for-the-perfect-taco.html}{René
Redzepi} and his excavation of forgotten Nordic foodways at Noma in
Copenhagen to
\href{https://www.nytimes.com/2019/09/03/dining/ella-restaurant-manoella-buffara.html}{Manoella
Buffara} and her marriage of haute cuisine and social and environmental
activism at Manu in Curitiba, Brazil --- have become not just desirable
places to work but rites of passage, imprimaturs for all who pass
through them. But of all these groups of alumni, perhaps none has
achieved as much influence as the chefs who have risen through the
kitchens of the Mexican chef
\href{https://www.instagram.com/enriqueolveraf/?hl=en}{Enrique Olvera}.
Their allegiance --- to Olvera and his mission of showing the greatness
of Mexican cuisine --- goes deeper: a tattoo on the heart.

All restaurants are built on trust; diners must have faith in the
kitchen and in the person who leads it. In an industry that prizes
hands-on experience over schooling, to have trained under a great chef
can be the highest of credentials --- a testament to one's endurance,
technical skill and ideological commitment. Still, not every chef is
willing to be a teacher. Those who command the staunchest loyalty are
the ones who never forget the labor that goes into every dish --- who
believe, like Olvera, that their job is to create chefs, not cooks, and
to establish a different kind of lineage.

The T List \textbar{}

Sign up here

Tony Floyd

IF EVERY TRIBE has an origin story, this one begins two decades ago, on
Calle Francisco Petrarca in Mexico City, when Olvera was 24, barely out
of culinary school, a chef by self-declaration only. To open a
restaurant with such a featherweight résumé was an act of wild ambition,
even hubris. Typically, ``you go to cooking school, then go stage'' ---
apprentice --- ``for a few years, then open your own place,'' Olvera
said. ``That's the opposite of what I did.''

It would be easy to cast Olvera as a kind of prophet, defying convention
and insisting from the start on the power of Mexican cooking. But like a
hero in myth, he had to stumble. The mood in the kitchen was dour at
first, the dining room solemn and underlit. Olvera still relied heavily
on the European techniques he'd been taught in the late '90s at the
Culinary Institute of America in Hyde Park, N.Y., which were considered
the standard for haute cuisine. Rather than break new ground, he was
``more concerned about simply surviving.'' It took years before his
restaurant, \href{https://pujol.com.mx/en/}{Pujol}, was enshrined in the
pantheon of the city's finest --- and for Olvera to realize that this
wasn't enough. After an epiphany in 2004, he turned away from those
European teachings, back to his Mexican heritage, and found a new way to
speak through food.

``Enrique Olvera is not only a mentor to many Mexican chefs --- he is
the mentor of Mexico,'' said Eduardo García, known as Lalo, who at age
42 runs Máximo Bistrot in Mexico City with his wife, Gabriela López, who
also worked under Olvera. (His kitchens have brokered many marriages.)
The child of migrant workers who became one himself, picking oranges in
Florida, blueberries in Michigan and mushrooms in Pennsylvania, García
went to work for Olvera after he was deported from the United States in
2007. ``It blew my mind,'' he said, to see Mexican ingredients
transformed into dishes like mole madre, two stark concentric circles of
mole --- one newly made and one aged for up to a thousand days --- at
once avant-garde and comforting, forward-thinking yet respectful of the
past.

Everyone in the kitchen was young. ``We wanted to do things right ---
because of Enrique but also because of us,'' said Jorge Vallejo, who
came to Pujol in 2006, when he was 25, and is today the chef of
\href{https://quintonil.com/en/restaurant-quintonil/}{Quintonil} in
Mexico City, which has been ranked alongside Pujol on the World's 50
Best Restaurants list since 2015. (His partner at the restaurant is his
wife and fellow Pujol alum, Alejandra Flores.) To
\href{https://www.nytimes.com/2019/01/02/t-magazine/daniela-soto-innes-recipes-wellness.html}{Daniela
Soto-Innes}, who started at Pujol in 2013, at age 19, and is now a
partner with Olvera in his North American restaurants, including
\href{https://www.cosmenyc.com/}{Cosme} in New York and
\href{https://eliolv.com/}{Elio} in Las Vegas, the sense of commitment
goes beyond Olvera to Mexico itself. ``Especially because of everything
that's going on with the government in the U.S.,'' she said, ``for us to
be representing Mexico outside of Mexico --- we're all together,
united.''

As Olvera changed course in the kitchen, he started to take a chance on
cooks with less experience, trusting in their work ethic. Sofía Cortina,
the pastry chef at the restaurant at
\href{https://hotelcarlota.mx/}{Hotel Carlota} in Mexico City (where she
works alongside her fellow Pujol alum Joaquín Cardoso), said that when
she started working with Olvera in 2011, at age 18, ``I didn't even know
how to put my jacket on properly.'' The typical kitchen, Olvera said, is
``almost like a monarchy, where there's this king that everybody needs
to listen to,'' but at Pujol, he wanted the learning to be horizontal
rather than vertical, with cooks learning from one another as much as
from the chefs above them. Olvera doesn't see himself as a mentor,
although the chefs who've worked for him continue to seek his advice.
``I tell them to find their own path,'' he said, because his career arc
was so atypical. In that sense, they are his teachers, too.

For them in turn, the bond remains, with Olvera and with one another, as
a vanguard changing the way Mexican cuisine is seen not just by the
world but by Mexicans themselves, and as a scrappy family, flaunting
their burns and scars, sharing memories of forcing mole through a
chinois --- a task that invariably took several people and ruined
whatever you were wearing --- or simply sitting in Olvera's office
talking for hours about how to make tortillas. ``We were suffering
together, when Enrique was mad sometimes; we were scared together,''
Cortina said with a laugh. But mostly the chef remains patient: He
listens and tries to give people ``the security of knowing that there's
not just one way of doing things,'' Olvera said, noting that he's most
proud of the fact that the chefs who have left his kitchens ``still have
a strong personal voice,'' separate from his own. They have spoken for
him. Now they speak for themselves.

Ligaya Mishan is a writer at large for T Magazine. Tony Floyd is a
commercial photographer and director. Production: Maritza Carbajal.

\href{https://www.nytimes.com/2020/04/13/t-magazine/enrique-olvera-vegetable-soup-recipe.html}{}

\hypertarget{enrique-olveras-satisfying-adaptable-vegetable-soupapril-13-2020}{%
\paragraph{Enrique Olvera's Satisfying, Adaptable Vegetable SoupApril
13,
2020}\label{enrique-olveras-satisfying-adaptable-vegetable-soupapril-13-2020}}

\includegraphics{https://static01.nyt.com/images/2020/04/13/t-magazine/13tmag-olvera-recipe/13tmag-olvera-recipe-mediumThreeByTwo210.jpg}
\href{https://www.nytimes.com/2019/11/27/t-magazine/spices.html}{}

\hypertarget{how-spices-have-made-and-unmade-empiresnov-27-2019}{%
\paragraph{How Spices Have Made, and Unmade, EmpiresNov. 27,
2019}\label{how-spices-have-made-and-unmade-empiresnov-27-2019}}

\includegraphics{https://static01.nyt.com/images/2019/11/27/t-magazine/27tmag-spice/27tmag-spice-mediumThreeByTwo210.jpg}
\href{https://www.nytimes.com/2020/01/13/t-magazine/best-chef-cook-books.html}{}

\hypertarget{the-cookbooks-you-need-for-2020-as-selected-by-chefsjan-13-2020}{%
\paragraph{The Cookbooks You Need for 2020, as Selected by ChefsJan. 13,
2020}\label{the-cookbooks-you-need-for-2020-as-selected-by-chefsjan-13-2020}}

\includegraphics{https://static01.nyt.com/images/2019/12/19/t-magazine/19tmag-cookbooks-slide-JTJX-copy/19tmag-cookbooks-slide-JTJX-mediumThreeByTwo210.jpg}

\hypertarget{we-are-family-1}{%
\subsubsection{We Are Family}\label{we-are-family-1}}

\hypertarget{chapter-1-heirs-and-alumni-1}{%
\paragraph{Chapter 1: Heirs and
Alumni}\label{chapter-1-heirs-and-alumni-1}}

\href{/interactive/2020/04/13/t-magazine/black-art-galleries.html}{}

\hypertarget{the-artists}{%
\subparagraph{The Artists}\label{the-artists}}

\href{/interactive/2020/04/13/t-magazine/italian-fashion-design-houses.html}{}

\hypertarget{the-dynasties}{%
\subparagraph{The Dynasties}\label{the-dynasties}}

\href{/interactive/2020/04/13/t-magazine/gordon-parks.html}{}

\hypertarget{the-directors}{%
\subparagraph{The Directors}\label{the-directors}}

\href{/interactive/2020/04/13/t-magazine/enrique-olvera-chef.html}{}

\hypertarget{the-disciples-1}{%
\subparagraph{The Disciples}\label{the-disciples-1}}

\href{/interactive/2020/04/13/t-magazine/royal-academy-antwerp.html}{}

\hypertarget{the-graduates}{%
\subparagraph{The Graduates}\label{the-graduates}}

\hypertarget{chapter-2-reunions-and-reconsiderations}{%
\paragraph{Chapter 2: Reunions and
Reconsiderations}\label{chapter-2-reunions-and-reconsiderations}}

\href{/interactive/2020/04/13/t-magazine/ninth-street-greenwich-village-neighbors.html}{}

\hypertarget{the-neighbors}{%
\subparagraph{The Neighbors}\label{the-neighbors}}

\href{/interactive/2020/04/13/t-magazine/omen-restaurant-nyc.html}{}

\hypertarget{the-regulars}{%
\subparagraph{The Regulars}\label{the-regulars}}

\href{/interactive/2020/04/13/t-magazine/hair-musical-broadway.html}{}

\hypertarget{hair-1967}{%
\subparagraph{Hair (1967)}\label{hair-1967}}

\href{/interactive/2020/04/13/t-magazine/sweeney-todd-revival.html}{}

\hypertarget{sweeney-todd-2005-revival}{%
\subparagraph{Sweeney Todd (2005
Revival)}\label{sweeney-todd-2005-revival}}

\href{/interactive/2020/04/13/t-magazine/daughters-of-the-dust.html}{}

\hypertarget{daughters-of-the-dust-1991}{%
\subparagraph{Daughters of the Dust
(1991)}\label{daughters-of-the-dust-1991}}

\hypertarget{chapter-3-legends-pioneers-and-survivors}{%
\paragraph{Chapter 3: Legends Pioneers and
Survivors}\label{chapter-3-legends-pioneers-and-survivors}}

\href{/interactive/2020/04/13/t-magazine/butch-stud-lesbian.html}{}

\hypertarget{the-renegades}{%
\subparagraph{The Renegades}\label{the-renegades}}

\href{/interactive/2020/04/13/t-magazine/act-up-aids.html}{}

\hypertarget{the-activists}{%
\subparagraph{The Activists}\label{the-activists}}

\href{/interactive/2020/04/13/t-magazine/artist-recluse.html}{}

\hypertarget{the-shadows}{%
\subparagraph{The Shadows}\label{the-shadows}}

\href{/interactive/2020/04/13/t-magazine/black-actresses-bassett-berry-blige-henson-whitfield-elise.html}{}

\hypertarget{the-veterans}{%
\subparagraph{The Veterans}\label{the-veterans}}

\hypertarget{chapter-4-the-new-guard}{%
\paragraph{Chapter 4: The New Guard}\label{chapter-4-the-new-guard}}

\href{/interactive/2020/04/13/t-magazine/asian-american-fashion-designers.html}{}

\hypertarget{the-designers}{%
\subparagraph{The Designers}\label{the-designers}}

\href{13tmag-beauties.html}{}

\hypertarget{the-beauties}{%
\subparagraph{The Beauties}\label{the-beauties}}

\href{/interactive/2020/04/13/t-magazine/nyc-downtown-nightlife-party-scene.html}{}

\hypertarget{the-scenemakers}{%
\subparagraph{The Scenemakers}\label{the-scenemakers}}

\href{/interactive/2020/04/13/t-magazine/maria-cornejo-olivier-rousteing-telfar-clemens-alessandro-michele.html\#olivier-rousteing-and-co}{}

\hypertarget{olivier-rousteing-and-co}{%
\subparagraph{Olivier Rousteing and
Co.}\label{olivier-rousteing-and-co}}

\href{/interactive/2020/04/13/t-magazine/maria-cornejo-olivier-rousteing-telfar-clemens-alessandro-michele.html\#maria-cornejo-and-co}{}

\hypertarget{maria-cornejo-and-co}{%
\subparagraph{Maria Cornejo and Co.}\label{maria-cornejo-and-co}}

\href{/interactive/2020/04/13/t-magazine/maria-cornejo-olivier-rousteing-telfar-clemens-alessandro-michele.html\#telfar-clemens-and-co}{}

\hypertarget{telfar-clemens-and-co}{%
\subparagraph{Telfar Clemens and Co.}\label{telfar-clemens-and-co}}

\href{/interactive/2020/04/13/t-magazine/maria-cornejo-olivier-rousteing-telfar-clemens-alessandro-michele.html\#alessandro-michele-and-co}{}

\hypertarget{alessandro-michele-and-co}{%
\subparagraph{Alessandro Michele and
Co.}\label{alessandro-michele-and-co}}

\href{/interactive/2020/04/13/t-magazine/foreign-correspondents.html}{}

\hypertarget{the-journalists}{%
\subparagraph{The Journalists}\label{the-journalists}}

\begin{itemize}
\item
\item
\item
\item
\end{itemize}

Advertisement

\protect\hyperlink{after-bottom}{Continue reading the main story}

\hypertarget{site-index}{%
\subsection{Site Index}\label{site-index}}

\hypertarget{site-information-navigation}{%
\subsection{Site Information
Navigation}\label{site-information-navigation}}

\begin{itemize}
\tightlist
\item
  \href{https://help.nytimes.com/hc/en-us/articles/115014792127-Copyright-notice}{©~2020~The
  New York Times Company}
\end{itemize}

\begin{itemize}
\tightlist
\item
  \href{https://www.nytco.com/}{NYTCo}
\item
  \href{https://help.nytimes.com/hc/en-us/articles/115015385887-Contact-Us}{Contact
  Us}
\item
  \href{https://www.nytco.com/careers/}{Work with us}
\item
  \href{https://nytmediakit.com/}{Advertise}
\item
  \href{http://www.tbrandstudio.com/}{T Brand Studio}
\item
  \href{https://www.nytimes.com/privacy/cookie-policy\#how-do-i-manage-trackers}{Your
  Ad Choices}
\item
  \href{https://www.nytimes.com/privacy}{Privacy}
\item
  \href{https://help.nytimes.com/hc/en-us/articles/115014893428-Terms-of-service}{Terms
  of Service}
\item
  \href{https://help.nytimes.com/hc/en-us/articles/115014893968-Terms-of-sale}{Terms
  of Sale}
\item
  \href{https://spiderbites.nytimes.com}{Site Map}
\item
  \href{https://help.nytimes.com/hc/en-us}{Help}
\item
  \href{https://www.nytimes.com/subscription?campaignId=37WXW}{Subscriptions}
\end{itemize}
