\hypertarget{the-foreign-correspondents-explaining-america-to-the-world}{%
\section{The Foreign Correspondents Explaining America to the
World}\label{the-foreign-correspondents-explaining-america-to-the-world}}

April 13, 2020

\begin{itemize}
\item
\item
\item
\item
\end{itemize}

In this tumultuous period of U.S. politics, there are perhaps more
international journalists in Washington, D.C., than ever before.

\href{https://www.nytimes.com/interactive/2020/04/13/t-magazine/culture-issue-2020.html}{We
Are Family}

\hypertarget{chapter-4-the-new-guard}{%
\subparagraph{Chapter 4: The New Guard}\label{chapter-4-the-new-guard}}

\hypertarget{previous}{%
\subparagraph{Previous}\label{previous}}

\hypertarget{the-foreign-correspondents-explaining-america-to-the-world-1}{%
\section{The Foreign Correspondents Explaining America to the
World}\label{the-foreign-correspondents-explaining-america-to-the-world-1}}

\hypertarget{the-journalists}{%
\subsection{The Journalists}\label{the-journalists}}

In this tumultuous period of American politics, there are perhaps more
foreign correspondents in Washington, D.C., than ever before, reporting
back to everywhere from Sweden to Singapore. What unites them is their
fight against the threat of misinformation and their struggle to
accurately inform their fellow citizens about what's happening here ---
and how it might affect them.

By \href{https://www.nytimes.com/by/jane-perlez}{Jane Perlez}

April 13, 2020

SHARE

LALIT K. JHAPress Trust of India, India

CHIDANAND RAJGHATTAThe Times of India, India

PAUL HUNTERCBC News, Canada

ARJEN VAN DER HORSTNOS, the Netherlands

YOSHITA SINGHPress Trust of India, India

GIUSEPPE SARCINACorriere della Sera, Italy

MARI KARPPINENMTV News, Finland

WATARU SAWAMURAThe Asahi Shimbun, Japan

SUZANNE LYNCHThe Irish Times, Ireland

NIRMAL GHOSHThe Straits Times, Singapore

PAUL DANAHARBBC, United Kingdom

MARCIN WRONATVN Discovery, Poland

BEN RILEY-SMITHThe Telegraph, United Kingdom

MARIEKE DE VRIESNOS, the Netherlands

JON SOPELBBC, United Kingdom

KERSTIN KOHLENBERGDie Zeit, Germany

DAVID SMITHThe Guardian, United Kingdom

SANDRA BROVALLPolitiken, Denmark

RICHARD MADANCTV News, Canada

JOHANNA BRUCKNERSüddeutsche Zeitung, Germany

ALAN CASSIDYTages-Anzeiger, Switzerland

ZHANG QICaixin, China

RENÉ PFISTERDer Spiegel, Germany

WANG YOUYOUPhoenix TV, China

NADIA BILBASSY-CHARTERSAl Arabiya News, Saudi Arabia

OMRI NAHMIASThe Jerusalem Post, Israel

AMIR TIBONHaaretz, Israel

BJÖRN AF KLEENDagens Nyheter, Sweden

ABDERRAHIM FOUKARAAl Jazeera, Qatar

JOSÉ DÍAZ-BRISEÑOReforma, Mexico

ADRIAN MORROWThe Globe and Mail, Canada

PHILIP CROWTHERAP Global Media Services

AMANDA MARSEl País, Spain

JOY MALBONCTV News, Canada

SONIA DRIDIFrance 24, France

BEATRIZ NAVARROLa Vanguardia, Spain

ANNE CORPETRFI, France

MATTHEW KNOTTSydney Morning Herald and The Age, Australia

RAQUEL KRÄHENBÜHLGloboNews, Brazil

ANNA-SOFIA BERNERHelsingin Sanomat, Finland

JAMES GLENDAYABC News, Australia

Standing, from left: \textbf{LALIT K. JHA}, Press Trust of India, India;
\textbf{CHIDANAND RAJGHATTA}, The Times of India, India; \textbf{PAUL
HUNTER}, CBC News, Canada; \textbf{ARJEN VAN DER HORST}, NOS, the
Netherlands; \textbf{YOSHITA SINGH}, Press Trust of India, India;
\textbf{GIUSEPPE SARCINA}, Corriere della Sera, Italy; \textbf{MARI
KARPPINEN}, MTV News, Finland; \textbf{WATARU SAWAMURA}, The Asahi
Shimbun, Japan; \textbf{SUZANNE LYNCH}, The Irish Times, Ireland;
\textbf{NIRMAL GHOSH}, The Straits Times, Singapore; \textbf{PAUL
DANAHAR}, BBC, United Kingdom; \textbf{MARCIN WRONA}, TVN Discovery,
Poland; \textbf{BEN RILEY-SMITH}, The Telegraph, United Kingdom;
\textbf{MARIEKE DE VRIES}, NOS, the Netherlands; \textbf{JON SOPEL},
BBC, United Kingdom; \textbf{KERSTIN KOHLENBERG}, Die Zeit, Germany;
\textbf{DAVID SMITH}, The Guardian, United Kingdom; \textbf{SANDRA
BROVALL}, Politiken, Denmark; \textbf{RICHARD MADAN}, CTV News, Canada;
\textbf{JOHANNA BRUCKNER}, Süddeutsche Zeitung, Germany; \textbf{ALAN
CASSIDY}, Tages-Anzeiger, Switzerland; \textbf{ZHANG QI}, Caixin, China;
\textbf{RENÉ PFISTER}, Der Spiegel, Germany; \textbf{WANG YOUYOU},
Phoenix TV, China; \textbf{NADIA BILBASSY-CHARTERS}, Al Arabiya News,
Saudi Arabia; \textbf{OMRI NAHMIAS}, The Jerusalem Post, Israel;
\textbf{AMIR TIBON}, Haaretz, Israel; \textbf{BJÖRN AF KLEEN}, Dagens
Nyheter, Sweden; and \textbf{ABDERRAHIM FOUKARA}, Al Jazeera, Qatar.
Seated, from left: \textbf{JOSÉ DÍAZ-BRISEÑO}, Reforma, Mexico;
\textbf{ADRIAN MORROW}, The Globe and Mail, Canada; \textbf{PHILIP
CROWTHER}, AP Global Media Services; \textbf{AMANDA MARS}, El País,
Spain; \textbf{JOY MALBON}, CTV News, Canada; \textbf{SONIA DRIDI},
France 24, France; \textbf{BEATRIZ NAVARRO}, La Vanguardia, Spain;
\textbf{ANNE CORPET}, RFI, France; \textbf{MATTHEW KNOTT}, Sydney
Morning Herald and The Age, Australia; \textbf{RAQUEL KRÄHENBÜHL},
GloboNews, Brazil; \textbf{ANNA-SOFIA BERNER}, Helsingin Sanomat,
Finland; and \textbf{JAMES GLENDAY}, ABC News, Australia. Photographed
at the Line DC in Washington, D.C., on Jan. 16, 2020. Brian Finke

The call came late at night: ``Get to the airport and a small plane will
take you and your photographer to Somalia.'' It was 1992, and I was a
foreign correspondent for The New York Times in Kenya, anxious that I
would be the first Western reporter to confirm rumors of Africa's hidden
tragedy: a
\href{https://www.nytimes.com/1992/07/19/world/deaths-in-somalia-outpace-delivery-of-food.html}{hellish
famine} caused by a civil war that had swept the country after the
collapse of the regime of
\href{https://www.nytimes.com/1995/01/03/obituaries/somalia-s-overthrown-dictator-mohammed-siad-barre-is-dead.html}{Mohammed
Siad Barre} in 1991. We saw skeletal bodies littered across the desert,
parents weeping for their perished children, overworked gravediggers.
Our words and images, prominent in The Times that July, caught the eye
of then
\href{https://www.nytimes.com/2018/11/30/us/politics/george-hw-bush-dies.html}{President
George H.W. Bush}, who a month later ordered American military forces to
deliver aid to the region.

Ever since The Times of London dispatched the Irish reporter William
Howard Russell to the Crimean War in 1854, foreign correspondents have
traveled to hard places to send eyewitness reports back home. Russell,
who shocked the British public with his exposés of incompetent British
commanders, called himself the ``miserable parent of a luckless tribe.''
By that he meant that foreign correspondents --- no matter what
nationality or how stiff the competition for a story --- work as a clan,
united in their mission to describe what is happening on the ground,
often in harsh conditions.

The T List \textbar{}

Sign up here for T's newsletter, a weekly roundup of what our editors
are noticing and coveting now.

Washington in the era of
\href{https://www.nytimes.com/topic/person/donald-trump}{Donald Trump}
is not quite a war zone. But there is a zealous fight over information,
over what's true and what's not. Social media confirms prejudices,
distorts the lens. In this age of disinformation, governments,
militaries --- and yes, the White House --- try to muzzle the truth. In
that tussle, foreign correspondents fight back by accumulating sources,
assessing what they say and making sense of the chaos of decisions. They
give as full a picture as possible of complicated events. They aspire to
make a difference. If they are lucky --- as I was in writing about
Somalia --- they can do so.

The presence in Washington of correspondents from independent media
outlets around the globe shows how Russell's ideal from more than a
century ago has matured into an established fixture of global
journalism. As the presidential primary season
\href{https://www.nytimes.com/interactive/2019/us/elections/2020-presidential-election-calendar.html}{intensifies},
these reporters typically travel to key states, sending stories from
diners, fairgrounds and town-hall meetings. They explain to their
readers and viewers at home how the tumult in Washington around
impeachment, immigration and democratic politics can have precipitous
consequences in their own countries.

President Trump has roiled America's longstanding alliances:
Correspondents from Western Europe and Asia must interpret what his
antagonistic rhetoric really portends. Is he seriously
\href{https://www.nytimes.com/2018/05/03/world/asia/trump-troops-south-korea.html}{considering}
pulling American troops from South Korea? Does he
\href{https://www.nytimes.com/2019/01/14/us/politics/nato-president-trump.html}{intend}
to undo NATO? Thoughtful analysis from Washington outclasses tropes on
social media. And though President Trump likes to call the mainstream
media ``fake news,'' he covets close-in and frequent coverage, inviting
correspondents to his pugnacious, freewheeling news conferences in the
Oval Office.

At a rough moment in U.S.-China relations last fall, a couple of Chinese
reporters asked Trump about the tensions between the two countries. The
president, sitting behind the Resolute desk just a few yards away, gave
rosy answers that often stretched reality. But the reporters' proximity
to Trump illustrated a more important point. It would be virtually
impossible for an American correspondent based in Beijing to ask the
authoritarian Chinese leader,
\href{https://www.nytimes.com/topic/person/xi-jinping}{Xi Jinping}, such
questions. By their very presence in Washington, these foreign
correspondents illuminate a foundation of American democracy: a free and
open press, which feels under pressure these days but lives on,
fortified by their efforts. --- JANE PERLEZ

Jane Perlez was most recently the Beijing bureau chief for The New York
Times. She has also served as bureau chief in Kenya, Poland, Austria,
Indonesia and Pakistan. Brian Finke is a photojournalist specializing in
visual cultural commentary. Producer: Mike Schoen. Photo assistant:
Carlos Jaramillo.

\hypertarget{here-41-correspondents-interviewed-between-jan-19-and-feb-3-2020-share-their-thoughts-on-reporting-from-the-united-states-during-the-trump-administration-read-a-behind-the-scenes-account-of-the-making-of-the-group-portrait-which-was-snapped-on-january-16-the-first-day-of-the-impeachment-trial-in-a-rare-moment-when-the-journalists-werent-following-the-news-on-their-phones}{%
\subsubsection{Here, 41 correspondents, interviewed between Jan. 19 and
Feb. 3, 2020, share their thoughts on reporting from the United States
during the Trump administration. Read a behind-the-scenes account of the
making of the group portrait, which was snapped on January 16, the first
day of the impeachment trial, in a rare moment when the journalists
weren't following the news on their
phones.}\label{here-41-correspondents-interviewed-between-jan-19-and-feb-3-2020-share-their-thoughts-on-reporting-from-the-united-states-during-the-trump-administration-read-a-behind-the-scenes-account-of-the-making-of-the-group-portrait-which-was-snapped-on-january-16-the-first-day-of-the-impeachment-trial-in-a-rare-moment-when-the-journalists-werent-following-the-news-on-their-phones}}

\hypertarget{bjuxf6rn-af-kleen-39}{%
\paragraph{BJÖRN AF KLEEN, 39}\label{bjuxf6rn-af-kleen-39}}

\hypertarget{dagens-nyheter-sweden}{%
\subparagraph{\texorpdfstring{\textbf{\href{https://www.dn.se/}{Dagens
Nyheter},
Sweden}}{Dagens Nyheter, Sweden}}\label{dagens-nyheter-sweden}}

\hypertarget{reporting-from-washington-dc-since-2017-first-foreign-posting}{%
\subparagraph{\texorpdfstring{\textbf{Reporting from Washington, D.C.,
since 2017; first foreign
posting.}}{Reporting from Washington, D.C., since 2017; first foreign posting.}}\label{reporting-from-washington-dc-since-2017-first-foreign-posting}}

A lot of Swedes are interested in how a person who is close to 80 years
old could become the next American president. In Sweden, the welfare
state means that people typically retire when they're 65 or so, and it's
very uncommon to find an 80-year-old person in an important position in
politics or business. I was in the Senate Chamber recently, and it still
fascinates me that some of the members are close to or over 80. Swedes
are also, of course, very interested in Trump, and how America, this
rich, sophisticated powerful nation, was so upset about things that 46
percent of its citizens chose him as its president. As a reporter, I'm
often surprised by how friendly his supporters can be, though. At Trump
rallies, the president will regularly direct his anger toward the press
--- and I don't want to diminish the violence in what he's doing --- but
later, when I speak with people in the audience, they're often very
curious about the foreign media: They want to know what we think about
their lives and the choices they make.

\hypertarget{anna-sofia-berner-34}{%
\paragraph{ANNA-SOFIA BERNER, 34}\label{anna-sofia-berner-34}}

\hypertarget{helsingin-sanomat-finland}{%
\subparagraph{\texorpdfstring{\textbf{\href{https://www.hs.fi/}{Helsingin
Sanomat},
Finland}}{Helsingin Sanomat, Finland}}\label{helsingin-sanomat-finland}}

\hypertarget{reporting-from-washington-dc-since-2019-first-foreign-posting}{%
\subparagraph{\texorpdfstring{\textbf{Reporting from Washington, D.C.,
since 2019; first foreign
posting.}}{Reporting from Washington, D.C., since 2019; first foreign posting.}}\label{reporting-from-washington-dc-since-2019-first-foreign-posting}}

When I wake up at 6:30 every morning, there are already emails from
Helsinki, so I start my day from bed. The afternoons and evenings are
quieter, because everybody is sleeping at home, but of course I'm still
working. This job easily turns into a 24/7 role. I'm the only Helsingin
Sanomat reporter based here, covering the whole country --- and
officially, Mexico and Canada as well --- so it's impossible to chase
every story. That's why I think it's important to have a basic idea of
what you want to do here and why you want to do it. It would be very
easy to play out stereotypes, to do an abortion or gun story, or seek
out the kind of crazy characters that have been part of American culture
for years --- it's not like they don't exist --- but I try to find
stories that offer a new angle.

\hypertarget{nadia-bilbassy-charters-57}{%
\paragraph{NADIA BILBASSY-CHARTERS,
57}\label{nadia-bilbassy-charters-57}}

\hypertarget{al-arabiya-news-channel-saudi-arabia}{%
\subparagraph{\texorpdfstring{\textbf{\href{https://english.alarabiya.net/}{Al
Arabiya News Channel}, Saudi
Arabia}}{Al Arabiya News Channel, Saudi Arabia}}\label{al-arabiya-news-channel-saudi-arabia}}

\hypertarget{reporting-from-washington-dc-since-2003-previous-foreign-postings-include-jerusalem-1987-89-colombo-sri-lanka-1990-94-dakar-senegal-1994-96-addis-ababa-ethiopia-1996-97-and-nairobi-kenya-1997-2003}{%
\subparagraph{\texorpdfstring{\textbf{Reporting from Washington, D.C.,
since 2003; previous foreign postings include Jerusalem (1987-89);
Colombo, Sri Lanka (1990-94); Dakar, Senegal (1994-96); Addis Ababa,
Ethiopia (1996-97) and Nairobi, Kenya
(1997-2003).}}{Reporting from Washington, D.C., since 2003; previous foreign postings include Jerusalem (1987-89); Colombo, Sri Lanka (1990-94); Dakar, Senegal (1994-96); Addis Ababa, Ethiopia (1996-97) and Nairobi, Kenya (1997-2003).}}\label{reporting-from-washington-dc-since-2003-previous-foreign-postings-include-jerusalem-1987-89-colombo-sri-lanka-1990-94-dakar-senegal-1994-96-addis-ababa-ethiopia-1996-97-and-nairobi-kenya-1997-2003}}

Before I came to Washington in 2003, I led a nomadic life as a field
reporter, covering wars and famines and often traveling great distances
to get a story. When I arrived here, I was shocked by how organized it
is. There's all this talk about ``briefings,'' but really the press
secretary just throws you a bone in an effort to distract you, to give
you the illusion of access to information. Ten years ago, I helped found
the White House Foreign Reporters Group, with the idea that if we come
together, we will have better access. Especially in this administration,
when we hardly have any information, correspondents are much better when
we're united in trying to hold officials accountable. With President
George W. Bush, I did six interviews; every time there was a major event
in the Middle East, I was able to convince his staff that it was
important to speak directly to the people in that region. The challenge
with this administration is that there aren't people around that you can
get through to.

\hypertarget{sandra-brovall-35}{%
\paragraph{SANDRA BROVALL, 35}\label{sandra-brovall-35}}

\hypertarget{politiken-denmark}{%
\subparagraph{\texorpdfstring{\textbf{\href{https://politiken.dk/}{Politiken},
Denmark}}{Politiken, Denmark}}\label{politiken-denmark}}

\hypertarget{reporting-from-new-york-since-2018-first-foreign-posting}{%
\subparagraph{\texorpdfstring{\textbf{Reporting from New York since
2018; first foreign
posting.}}{Reporting from New York since 2018; first foreign posting.}}\label{reporting-from-new-york-since-2018-first-foreign-posting}}

One thing that I've noticed while reporting from America --- especially
when writing about the impeachment trial --- is that, in a way that I've
never experienced before, I'm getting emails from people calling it
``fake news.'' Even readers in Denmark, where just a few years ago
people didn't talk about ``fake news,'' are saying it. It's alarming. I
want to understand where those people are coming from, what they're
reading; their point of view makes me curious. But as a journalist, it's
scary to see that the trust in the press has been eroded, especially
taking into account some of the things that Trump has been saying since
the beginning of his campaign.

\hypertarget{johanna-bruckner-35}{%
\paragraph{JOHANNA BRUCKNER, 35}\label{johanna-bruckner-35}}

\hypertarget{suxfcddeutsche-zeitung-germany}{%
\subparagraph{\texorpdfstring{\textbf{\href{https://www.sueddeutsche.de/}{Süddeutsche
Zeitung},
Germany}}{Süddeutsche Zeitung, Germany}}\label{suxfcddeutsche-zeitung-germany}}

\hypertarget{reporting-from-new-york-since-2017-first-foreign-posting}{%
\subparagraph{\texorpdfstring{\textbf{Reporting from New York since
2017; first foreign
posting.}}{Reporting from New York since 2017; first foreign posting.}}\label{reporting-from-new-york-since-2017-first-foreign-posting}}

As a journalist, if you haven't lived in a country you're writing about,
you tend to fall back on general assumptions --- or return to what seems
most logical. A lot of people in Germany are still trying to make sense
of how Trump got elected; there's a profound sense of disbelief. If I
was in Germany, maybe I would also think America is a country where
there's a history of showman-type politicians who came from the
entertainment business. A lot of Germans might just say, ``OK, America
is far away and has strange political traditions.'' Of course, the
reality is much more complicated. There are various reasons he was
elected, and why he might be elected again. You have to live here, watch
the news, talk to people when you go out, listen to the conversations
around you on the subway --- otherwise you will only cover a fraction of
the truth.

\hypertarget{alan-cassidy-36}{%
\paragraph{ALAN CASSIDY, 36}\label{alan-cassidy-36}}

\hypertarget{tages-anzeiger-switzerland}{%
\subparagraph{\texorpdfstring{\textbf{\href{https://www.tagesanzeiger.ch/}{Tages-Anzeiger},
Switzerland}}{Tages-Anzeiger, Switzerland}}\label{tages-anzeiger-switzerland}}

\hypertarget{reporting-from-washington-dc-since-2018-first-foreign-posting}{%
\subparagraph{\texorpdfstring{\textbf{Reporting from Washington, D.C.,
since 2018; first foreign
posting.}}{Reporting from Washington, D.C., since 2018; first foreign posting.}}\label{reporting-from-washington-dc-since-2018-first-foreign-posting}}

I've lost count of the times I've written a story, filed it, then
checked Twitter only to see that Trump has reversed a decision. From a
personal point of view, it can be very annoying. But European
journalists covering America are also beneficiaries of the Trump
presidency in that the interest in U.S. politics has never been so high,
at least during my lifetime. People in Switzerland are fascinated by
what's happening here. When I go back home, everybody wants to talk
about it. There's a real demand for stories from the U.S., and we know
that our stories get read, which is good for foreign correspondents. So
I'm not complaining --- covering this presidency has got upsides and
downsides.

\hypertarget{anne-corpet-52}{%
\paragraph{ANNE CORPET, 52}\label{anne-corpet-52}}

\hypertarget{rfi-france}{%
\subparagraph{\texorpdfstring{\textbf{\href{http://www.rfi.fr/fr/tag/anne-corpet/}{RFI},
France}}{RFI, France}}\label{rfi-france}}

\hypertarget{reporting-from-washington-dc-since-2017-first-foreign-posting-1}{%
\subparagraph{\texorpdfstring{\textbf{Reporting from Washington, D.C.,
since 2017; first foreign
posting.}}{Reporting from Washington, D.C., since 2017; first foreign posting.}}\label{reporting-from-washington-dc-since-2017-first-foreign-posting-1}}

Washington is a never-ending story. I'm constantly exhausted, but I also
feel extremely fortunate to be here at this particular moment in U.S.
history. I had wanted to be posted here 12 years ago, and then again six
years ago, but I now realize I was lucky, because this is the best time
there's ever been to be a journalist in America. People here have always
been more used to speaking to the media than they are in Europe, and now
especially, they're not afraid to share their opinions. You can ask a
question of anyone in the street and they will give you a good
punchline. People express themselves and they don't care if it's
politically incorrect, in the same way that Trump says things that no
one else would say. The other interesting thing is that whatever happens
here is coming to Europe. Every tendency --- the rise of the
anti-abortion movement, protectionism, nationalism --- begins here
before it comes to France. When you're covering the United States, it's
as if you're on the front line of Western culture.

\hypertarget{philip-crowther-38}{%
\paragraph{PHILIP CROWTHER, 38}\label{philip-crowther-38}}

\hypertarget{associated-press-global-media-services-international-affiliate-reporter}{%
\subparagraph{\texorpdfstring{\textbf{\href{https://www.ap.org/en-us/}{Associated
Press}} \textbf{Global Media Services, international affiliate
reporter}}{Associated Press Global Media Services, international affiliate reporter}}\label{associated-press-global-media-services-international-affiliate-reporter}}

\hypertarget{reporting-from-washington-dc-since-2011-previously-reported-for-france-24-2011-18}{%
\subparagraph{\texorpdfstring{\textbf{Reporting from Washington, D.C.,
since 2011; previously reported for France 24
(2011-18).}}{Reporting from Washington, D.C., since 2011; previously reported for France 24 (2011-18).}}\label{reporting-from-washington-dc-since-2011-previously-reported-for-france-24-2011-18}}

I'm a reporter who goes on air for the AP's broadcast clients ---
international, national and sometimes even local news channels --- and I
do this in six different languages. In a given week, I could be on air
for RTL Luxembourg, Euronews or Voice of America, which is based here,
in Portuguese, French, English, German, Luxemburgish or Spanish. With
each channel, I have to adapt a little to the audience: How much do the
viewers know about American politics? Do they want me to speak to a
particular subject matter? Foreign correspondents are able to step back
from the day-to-day business that most American reporters here have to
concentrate on. Our role is to explain what's going on without a lot of
the minute details, which is a wonderful luxury to have. One of the
challenges, though, is making sure you know the right vocabulary. I've
had to learn how to say ``impeachment'' in a lot of languages.

\hypertarget{paul-danahar-52}{%
\paragraph{PAUL DANAHAR, 52}\label{paul-danahar-52}}

\hypertarget{bbc-news-united-kingdom}{%
\subparagraph{\texorpdfstring{\href{https://www.bbc.com/news}{BBC News},
United
Kingdom}{BBC News, United Kingdom}}\label{bbc-news-united-kingdom}}

\hypertarget{reporting-from-washington-dc-since-2014-previous-foreign-postings-include-johannesburg-2000-01-new-delhi-2002-06-beijing-2007-09-and-jerusalem-2010-13}{%
\subparagraph{Reporting from Washington, D.C., since 2014; previous
foreign postings include Johannesburg (2000-01), New Delhi (2002-06),
Beijing (2007-09), and Jerusalem
(2010-13).}\label{reporting-from-washington-dc-since-2014-previous-foreign-postings-include-johannesburg-2000-01-new-delhi-2002-06-beijing-2007-09-and-jerusalem-2010-13}}

I spent many of the past 20 years at the sharp end of American foreign
policy, watching it play out on the ground. I was deployed to
Afghanistan during the American invasion and to Baghdad when Saddam
Hussein's statue came down. I've been in Africa and the Middle East and
Asia. So coming here, where it all starts, has been really interesting,
and I think that's true for a lot of my colleagues. Part of the job of a
foreign journalist in Washington is to explain to our audience how a
button being pressed here can impact another country several thousand
miles away. It's also our job to join the dots for people who haven't
got the big picture before them. When we come to Washington, we try to
use the experience we've acquired in other places to provide context
rather than just straightforward reports about what happened in the
White House on a particular day.

\hypertarget{marieke-de-vries-47}{%
\paragraph{MARIEKE DE VRIES, 47}\label{marieke-de-vries-47}}

\hypertarget{nos-the-netherlands}{%
\subparagraph{\texorpdfstring{\textbf{\href{https://over.nos.nl/onze-mensen/147/marieke-de-vries}{NOS},
the Netherlands}}{NOS, the Netherlands}}\label{nos-the-netherlands}}

\hypertarget{reporting-from-washington-dc-since-2019-previously-posted-in-beijing-2012-19}{%
\subparagraph{\texorpdfstring{\textbf{Reporting from Washington, D.C.,
since 2019; previously posted in Beijing
(2012-19).}}{Reporting from Washington, D.C., since 2019; previously posted in Beijing (2012-19).}}\label{reporting-from-washington-dc-since-2019-previously-posted-in-beijing-2012-19}}

A colleague once told me that you have to be at eye level with a person
that you're trying to understand; you have to physically sit down or
kneel beside them. For that reason, it's very important to have people
on the ground. I believe, too, that journalists are there not only to
listen but to provide checks and balances as well. My experience as a
correspondent in China was that nobody there dared to say anything that
might be construed as critical of the government. I met regularly with
the authorities, with the Ministry of Foreign Affairs, for example, and
I would say, ``Why don't you listen to people who have critiques?'' But
they just didn't want to talk about it. And so our journalistic
colleagues in China are under a lot of pressure, but they still do their
jobs; they're real heroes. I hope that the local journalists in all
countries will always be able to do their jobs, that they won't be let
go because of cutbacks. The threat to regional journalism is worrisome
and doesn't get enough attention. People say, ``It's just a local
newspaper. Who cares if it's going bankrupt?'' But it would matter
greatly if those institutions disappeared.

\hypertarget{josuxe9-duxedaz-briseuxf1o-45}{%
\paragraph{JOSÉ DÍAZ-BRISEÑO, 45}\label{josuxe9-duxedaz-briseuxf1o-45}}

\hypertarget{reforma-mexico}{%
\subparagraph{\texorpdfstring{\textbf{\href{https://www.reforma.com/}{Reforma},
Mexico}}{Reforma, Mexico}}\label{reforma-mexico}}

\hypertarget{reporting-from-washington-dc-since-2005-including-a-stint-with-mundofox-television-2012-15-previous-foreign-postings-include-tucson-ariz-for-the-mexican-newspaper-el-imparcial-2003-04}{%
\subparagraph{\texorpdfstring{\textbf{Reporting from Washington, D.C.,
since 2005, including a stint with MundoFox television (2012-15);
previous foreign postings include Tucson, Ariz., for the Mexican
newspaper El Imparcial
(2003-04).}}{Reporting from Washington, D.C., since 2005, including a stint with MundoFox television (2012-15); previous foreign postings include Tucson, Ariz., for the Mexican newspaper El Imparcial (2003-04).}}\label{reporting-from-washington-dc-since-2005-including-a-stint-with-mundofox-television-2012-15-previous-foreign-postings-include-tucson-ariz-for-the-mexican-newspaper-el-imparcial-2003-04}}

In many ways, foreign correspondents are survivors. We're the last men
and women standing who are covering international news. It's extremely
expensive to have correspondents in the United States, so there are very
few organizations that are able to have a bureau here or send
journalists to the primaries. Some of us are working from home rather
than offices. But how do we not become isolated? I know only a handful
of full-time correspondents from Latin America, and that has encouraged
me to become friends with them. We don't have 10 people at the White
House like The New York Times, so we help each other --- if there's
something of interest to another country, we exchange information. Some
years ago, I appeared on a Colombian TV show, ``Club de Prensa,'' a
round-table discussion featuring Spanish, Mexican, Colombian, Argentine
and Brazilian correspondents, and we became very close after that. Now,
we have WhatsApp groups in which we occasionally make fun of the
situation we're in. We hang out with our families --- in some cases, we
even travel together.

\hypertarget{sonia-dridi-34}{%
\paragraph{SONIA DRIDI, 34}\label{sonia-dridi-34}}

\hypertarget{france-24-europe-1-and-bfm-tv-france}{%
\subparagraph{\texorpdfstring{\href{https://www.france24.com/en/}{France
24}, Europe 1 and BFM TV,
France}{France 24, Europe 1 and BFM TV, France}}\label{france-24-europe-1-and-bfm-tv-france}}

\hypertarget{reporting-from-washington-dc-since-2015-previously-posted-in-cairo-2011-15}{%
\subparagraph{Reporting from Washington, D.C., since 2015; previously
posted in Cairo
(2011-15).}\label{reporting-from-washington-dc-since-2015-previously-posted-in-cairo-2011-15}}

I knew that the U.S. was a very diverse country before I moved here, but
now I'm constantly reminded that it's almost as if it's many nations in
one. I did a year abroad at U.C.L.A. from 2005 to 2006, and what struck
me most when I returned in 2015 was how polarized the country has
become. I was arriving from Egypt, which was really divided between the
supporters of the Muslim Brotherhood and those who were pro-army. Donald
Trump announced his candidacy two weeks after I landed in Washington,
and I would sometimes be able to report the same style of stories here
that I did in Egypt, about families that were split down the middle by
politics or how the press itself is a target. And the divide has grown
since 2015. It's worrying that people talk to each other much less, that
there's less dialogue.

\hypertarget{abderrahim-foukara-52}{%
\paragraph{ABDERRAHIM FOUKARA, 52}\label{abderrahim-foukara-52}}

\hypertarget{al-jazeera-qatar}{%
\subparagraph{\texorpdfstring{\textbf{\href{https://www.aljazeera.com/}{Al
Jazeera}, Qatar}}{Al Jazeera, Qatar}}\label{al-jazeera-qatar}}

\hypertarget{reporting-from-new-york-and-washington-dc-since-2002-previously-reported-from-boston-for-the-bbc-1999-2002}{%
\subparagraph{\texorpdfstring{\textbf{Reporting from New York and
Washington, D.C., since 2002; previously reported from Boston for the
BBC
(1999-2002).}}{Reporting from New York and Washington, D.C., since 2002; previously reported from Boston for the BBC (1999-2002).}}\label{reporting-from-new-york-and-washington-dc-since-2002-previously-reported-from-boston-for-the-bbc-1999-2002}}

As an Arab correspondent working in D.C., I'm not just a journalist but
also an interpreter. I have to explain the political culture of the
United States to people in a part of the world that has a very different
political culture, and it has always been fascinating. Al Jazeera had a
very difficult relationship with the Bush administration during its
first term because we covered civilian casualties in Afghanistan and
Iraq, and the administration didn't like that, though it eased off with
his second term. Obama, by comparison, saw reaching out to public
opinion in the Middle East as something very important, and Al Jazeera
was a key component of that. With the Trump administration, we haven't
had any major problems so far. During the blockade of Qatar, at the top
of the list of conditions was closing down Al Jazeera. We know that
conversation was percolating in different parts of D.C., and we also
know that
\href{https://www.nytimes.com/2020/03/05/us/politics/jared-kushner-cadre.html}{Jared
Kushner} has good relationships with the Emiratis and the Saudis, who
had set the condition, and so when we interviewed him in June 2019, we
wondered what the dynamic would be. He had his own ideas and agenda, we
had our issues with that agenda, and it was our duty to challenge him.
He accepted that, and at the end of the day, our relationship with him
became very professional. That's the way it goes between a journalist
and a politician --- or should go.

\hypertarget{nirmal-ghosh-60}{%
\paragraph{NIRMAL GHOSH, 60}\label{nirmal-ghosh-60}}

\hypertarget{the-straits-times-singapore}{%
\subparagraph{\texorpdfstring{\textbf{\href{https://www.straitstimes.com/authors/nirmal-ghosh}{The
Straits Times},
Singapore}}{The Straits Times, Singapore}}\label{the-straits-times-singapore}}

\hypertarget{reporting-from-washington-dc-since-2016-previous-foreign-postings-include-manila-1994-99-new-delhi-1999-2003-and-bangkok-2003-16}{%
\subparagraph{\texorpdfstring{\textbf{Reporting from Washington, D.C.,
since 2016; previous foreign postings include Manila (1994-99), New
Delhi (1999-2003) and Bangkok
(2003-16).}}{Reporting from Washington, D.C., since 2016; previous foreign postings include Manila (1994-99), New Delhi (1999-2003) and Bangkok (2003-16).}}\label{reporting-from-washington-dc-since-2016-previous-foreign-postings-include-manila-1994-99-new-delhi-1999-2003-and-bangkok-2003-16}}

As the competition between China and America grows, there is an acute
awareness in Singapore of the need for these two powers to balance each
other in a somewhat benign way. So what happens in America matters ---
perhaps more so now than ever, because you have a president who is bent
on disrupting the world order. Recently, I've noticed that the frenetic
pace of the news cycle has seeped into my dreams. I dream about being
late for an appointment or a flight, of missing a deadline, of trying to
find somebody unsuccessfully. The dreams are full of stress and anxiety.
I miss the organic nature of Asia, its colors, its intensity of life.
The way I've stayed connected is by remaining involved in wildlife
conservation there, specifically in India. We work to preserve four
tiger habitats, of which two are also elephant habitats, as well as the
grassland habitat of the great Indian bustard. It provides a useful
perspective for a journalist. You realize that humans are only passing
through, really.

\hypertarget{james-glenday-34}{%
\paragraph{JAMES GLENDAY, 34}\label{james-glenday-34}}

\hypertarget{abc-news-australia}{%
\subparagraph{\texorpdfstring{\textbf{\href{https://www.abc.net.au/news/}{ABC
News}, Australia}}{ABC News, Australia}}\label{abc-news-australia}}

\hypertarget{reporting-from-washington-dc-since-2018-previously-posted-in-london-2016-18}{%
\subparagraph{\texorpdfstring{\textbf{Reporting from Washington, D.C.,
since 2018; previously posted in London
(2016-18).}}{Reporting from Washington, D.C., since 2018; previously posted in London (2016-18).}}\label{reporting-from-washington-dc-since-2018-previously-posted-in-london-2016-18}}

After the Port Arthur massacre in 1996, Australia introduced sweeping,
strict gun-control laws, so every time there's a mass shooting in
America, there's a lot of interest back home. When I report on attacks
in the U.S., my Twitter feed blows up with Australians asking, ``Are
they going to do anything? Is anything going to change?'' I was in El
Paso, Texas, after the shooting there last August, and so many families
were grieving, but people were still saying guns weren't to blame.
America, in many ways, does feel like Australia, but that was one thing
that was very hard to wrap my head around. There's also the pace of the
work itself, which has accelerated over the past few years. On most
days, I will file something for radio, something for TV and something
for our website. I think we've kind of hit peak in terms of what we can
realistically do.

\hypertarget{paul-hunter-59}{%
\paragraph{PAUL HUNTER, 59}\label{paul-hunter-59}}

\hypertarget{cbc-news-canada}{%
\subparagraph{\texorpdfstring{\textbf{\href{https://www.cbc.ca/news/world/paul-hunter-1.2449165}{CBC
News}, Canada}}{CBC News, Canada}}\label{cbc-news-canada}}

\hypertarget{reporting-from-washington-dc-since-2008-first-foreign-posting}{%
\subparagraph{\texorpdfstring{\textbf{Reporting from Washington, D.C.,
since 2008; first foreign
posting.}}{Reporting from Washington, D.C., since 2008; first foreign posting.}}\label{reporting-from-washington-dc-since-2008-first-foreign-posting}}

Something I found interesting when I first came down here was the degree
to which people told me that they valued journalism. After the
\href{https://www.nytimes.com/2013/05/21/us/tornado-oklahoma.html}{Moore,
Okla., tornado} in 2013, I saw a couple sifting through the rubble of
their house. They gave me an interview about everything that they'd lost
simply to let people know that they were still standing and that life
would be OK. It was as if they wanted to give a pep talk to the rest of
the United States and Canada. Those days are why I feel so grateful to
be a reporter in this country. But the obsession with Trump has eaten up
all the real estate. What gets lost is the question: ``Where is the
country going?'' And the democratic process suffers. It's very dark, and
all the partisan fighting wears on you after a while. But my job is to
give the helicopter view to Canadians, to provide broad strokes and not
to get into the minutiae. To me, that's the only comfort.

\hypertarget{lalit-k-jha-47}{%
\paragraph{LALIT K. JHA, 47}\label{lalit-k-jha-47}}

\hypertarget{press-trust-of-india-india}{%
\subparagraph{\texorpdfstring{\textbf{\href{http://www.ptinews.com/}{Press
Trust of India},
India}}{Press Trust of India, India}}\label{press-trust-of-india-india}}

\hypertarget{reporting-from-washington-dc-since-2005-first-foreign-posting}{%
\subparagraph{\texorpdfstring{\textbf{Reporting from Washington, D.C.,
since 2005; first foreign
posting.}}{Reporting from Washington, D.C., since 2005; first foreign posting.}}\label{reporting-from-washington-dc-since-2005-first-foreign-posting}}

This is the best country in the world for a foreign correspondent to be
posted in because of its values and ethos. In Washington, I interact
with journalists from around the globe and I see the difference between
the access a correspondent has here compared to other places. Not many
governments allow the kind of close-up coverage that the U.S. does. For
instance, in the last three years, as a foreign pool reporter for the
White House Correspondents' Association, I've attended more than half a
dozen of the president's cabinet meetings. But there are challenges as
well: Access is always relative, and for the foreign correspondents who,
like me, are one-person bureaus, **** perhaps the biggest dilemma is how
to cover everything in person. Because there's no alternative to human
intelligence; the best information for a journalist comes from
face-to-face interactions with other people --- not phone calls or
emails.

\hypertarget{mari-karppinen-35}{%
\paragraph{MARI KARPPINEN, 35}\label{mari-karppinen-35}}

\hypertarget{mtv-news-finland}{%
\subparagraph{\texorpdfstring{\textbf{\href{https://www.mtvuutiset.fi/}{MTV
News}, Finland}}{MTV News, Finland}}\label{mtv-news-finland}}

\hypertarget{reporting-from-new-york-since-2014-first-foreign-posting}{%
\subparagraph{\texorpdfstring{\textbf{Reporting from New York since
2014; first foreign
posting.}}{Reporting from New York since 2014; first foreign posting.}}\label{reporting-from-new-york-since-2014-first-foreign-posting}}

When I say I'm the U.S. correspondent for MTV News --- which is the
largest commercial television news network in Finland (and not the music
channel) --- it means that I'm the only one here. I'm a reporter, but
I'm also my own camerawoman, editor and producer; some of my colleagues
call me a one-woman show. It's a little chaotic at times, but I feel
very privileged to be here and witness this era. I feel rewarded if I'm
able to explain what's happening here to the Finnish people, so that
they understand what it means to them --- because big changes in the
U.S. influence everybody, especially small countries like Finland. Of
course, sometimes I also just feel very tired and sad, especially about
the way the president is undermining the press constantly; that's scary
to see and live.

\hypertarget{matthew-knott-32}{%
\paragraph{MATTHEW KNOTT, 32}\label{matthew-knott-32}}

\hypertarget{sydney-morning-herald-and-the-age-australia}{%
\subparagraph{\texorpdfstring{\textbf{\href{https://www.smh.com.au/}{Sydney
Morning Herald}} \textbf{and}
\textbf{\href{https://www.theage.com.au/}{The Age},
Australia}}{Sydney Morning Herald and The Age, Australia}}\label{sydney-morning-herald-and-the-age-australia}}

\hypertarget{reporting-from-new-york-and-washington-dc-since-2017-first-foreign-posting}{%
\subparagraph{\texorpdfstring{\textbf{Reporting from New York and
Washington, D.C., since 2017; first foreign
posting.}}{Reporting from New York and Washington, D.C., since 2017; first foreign posting.}}\label{reporting-from-new-york-and-washington-dc-since-2017-first-foreign-posting}}

After the election in 2016, there was a general feeling among
journalists that we had ``lost touch with real people,'' as some of us
would say. I was very struck by that. I felt I needed to do whatever I
could to come to America because, as a journalist, you run toward the
turmoil. Before I left, I was told that all foreign correspondents
become paranoid within the first year of a posting because they worry
that people back home don't understand, or even read, their work. I
tried to fight that, but I can see how that happens: relevance
deprivation syndrome. You go from being inside the beast, or affecting
things from within your own country, to interpreting events from the
outside. As a foreigner in power-hungry Washington, people don't see a
benefit in speaking to you from the inside. But when you travel to
different parts of the country, Americans really want to talk to you
because they're fascinated by how an Australian has shown up in rural
Pennsylvania or New Mexico. They don't approach you with all the
preconceptions that they might have of a reporter from a publication
here.

\hypertarget{kerstin-kohlenberg-49}{%
\paragraph{KERSTIN KOHLENBERG, 49}\label{kerstin-kohlenberg-49}}

\hypertarget{die-zeit-germany}{%
\subparagraph{\texorpdfstring{\textbf{\href{https://www.zeit.de/index}{Die
Zeit}, Germany}}{Die Zeit, Germany}}\label{die-zeit-germany}}

\hypertarget{reporting-from-new-york-since-2014-first-foreign-posting-1}{%
\subparagraph{\texorpdfstring{\textbf{Reporting from New York since
2014; first foreign
posting.}}{Reporting from New York since 2014; first foreign posting.}}\label{reporting-from-new-york-since-2014-first-foreign-posting-1}}

I was the deputy head of the investigative team at my paper in Berlin,
and our former correspondent was due to come back in 2014. After six
years of Barack Obama, everyone imagined that Hillary Clinton would
continue what he started, so the paper decided to send a reporter
instead of a hardcore political journalist; they asked me. I told the
paper I wanted to live in New York, and they said, ``Sure, America might
get a little bit boring anyway.'' So I took my family to New York, and
then nothing turned out as expected. But one big advantage of not being
based in D.C. during the Trump primary campaign was that I was outside
of the Washington bubble. I spent more time in the neighboring states,
and I was able to see the whole story. Some days, when I'm walking
around New York or D.C. or spending time with my family, I feel like
everything is fine. But then I remember that, no, these are not normal
times. Something is changing here in the fabric of the country.

\hypertarget{raquel-kruxe4henbuxfchl-36}{%
\paragraph{RAQUEL KRÄHENBÜHL, 36}\label{raquel-kruxe4henbuxfchl-36}}

\hypertarget{globonews-brazil}{%
\subparagraph{\texorpdfstring{\textbf{\href{https://g1.globo.com/globonews/}{GloboNews},
Brazil}}{GloboNews, Brazil}}\label{globonews-brazil}}

\hypertarget{reporting-from-washington-dc-since-2006-first-foreign-posting}{%
\subparagraph{\texorpdfstring{\textbf{Reporting from Washington, D.C.,
since 2006; first foreign
posting.}}{Reporting from Washington, D.C., since 2006; first foreign posting.}}\label{reporting-from-washington-dc-since-2006-first-foreign-posting}}

During the Obama administration, I remember that people would sometimes
sleep in the briefing room. Everybody knew what was coming. Now,
everyone's on edge. Suddenly a staffer will announce that an event in
the Oval Office that was previously going to be closed to the press is
now open to the pool, and so everyone will start running. I'm always
thinking, ``Do I have time to go to the bathroom?'' On the other hand,
we have much more access than we had before. Now we can get close to the
president. I think I've already asked Trump dozens of questions about
Brazil-U.S. relations. He knows who we are and he comes to us --- I
think maybe sometimes to change the subject from the American press.
There's so much going on this year --- with the
\href{https://www.nytimes.com/2020/02/05/us/politics/impeachment-vote.html}{impeachment
trial}, the election and
\href{https://www.nytimes.com/2020/01/05/world/middleeast/Iran-us-trump.html}{Iran}
--- that we need to let off steam by the end of the week. So some
journalists have started something we call survival parties. We go
dancing or meet at a bar near the White House; we just have fun --- and
relieve the stress together.

\hypertarget{suzanne-lynch-41}{%
\paragraph{SUZANNE LYNCH, 41}\label{suzanne-lynch-41}}

\hypertarget{the-irish-times-ireland}{%
\subparagraph{\texorpdfstring{\textbf{\href{https://www.irishtimes.com/}{The
Irish Times},
Ireland}}{The Irish Times, Ireland}}\label{the-irish-times-ireland}}

\hypertarget{reporting-from-washington-dc-since-2017-previously-posted-in-brussels-2013-17}{%
\subparagraph{\texorpdfstring{\textbf{Reporting from Washington, D.C.,
since 2017; previously posted in Brussels
(2013-17).}}{Reporting from Washington, D.C., since 2017; previously posted in Brussels (2013-17).}}\label{reporting-from-washington-dc-since-2017-previously-posted-in-brussels-2013-17}}

I was taken aback by how open the system here is. I regularly go to
Congress and I've been in the chamber for decisive moments like the
House impeachment vote on December 18. I was there, behind Nancy Pelosi,
thinking, ``This is history being made.'' I was there when John McCain,
in 2017, gave his famous thumbs down to the Senate on the repeal of
parts of the Affordable Care Act. He had recently learned he had brain
cancer --- he would die the following year --- and it was a hugely
symbolic and emotional moment. It was the signature Republican
legislative priority back then, and he voted against it. On Capitol Hill
in particular, you can really walk around the halls of power, go into
the offices of members of Congress and talk to them directly.

\hypertarget{richard-madan-47}{%
\paragraph{RICHARD MADAN, 47}\label{richard-madan-47}}

\hypertarget{ctv-news-canada}{%
\subparagraph{\texorpdfstring{\textbf{CTV News,
Canada}}{CTV News, Canada}}\label{ctv-news-canada}}

\hypertarget{reporting-from-washington-dc-since-2016-first-foreign-posting}{%
\subparagraph{\texorpdfstring{\textbf{Reporting from ​Washington, D.C.,
since 2016; first foreign
posting.}}{Reporting from ​Washington, D.C., since 2016; first foreign posting.}}\label{reporting-from-washington-dc-since-2016-first-foreign-posting}}

I was born in Athens, Ga., and my family moved to Canada when I was very
young. I was back in Georgia for the 2018 midterms and I have a lot of
memories from that time that will stay with me, including covering the
state's income inequality and the racial attacks against gubernatorial
candidate Stacey Abrams. We saw firsthand the impact of gerrymandering,
and it was very sobering. In Canada, electoral boundaries are overseen
by a nonpartisan group, and gerrymandering doesn't really exist. But the
painful legacy of race relations in the U.S. has not gone away; it's
part of American culture that has not healed yet. I find it interesting,
too, how the press is increasingly antagonized here. When our team
covers a Trump rally, we put a Canadian-flag sticker on top of the
camera's viewfinder. People often start yelling at us, calling us ``fake
news,'' but when they see the sticker, they become friendly and more
open to conversation.

\hypertarget{joy-malbon-62}{%
\paragraph{JOY MALBON, 62}\label{joy-malbon-62}}

\hypertarget{ctv-news-canada-1}{%
\subparagraph{\texorpdfstring{\textbf{\href{https://www.ctvnews.ca/}{CTV
News}, Canada}}{CTV News, Canada}}\label{ctv-news-canada-1}}

\hypertarget{reporting-from-washington-dc-since-2005-previous-foreign-postings-include-london-1997-2000-and-jerusalem-2003}{%
\subparagraph{\texorpdfstring{\textbf{Reporting from Washington, D.C.,
since 2005; previous foreign postings include London (1997-2000) and
Jerusalem
(2003).}}{Reporting from Washington, D.C., since 2005; previous foreign postings include London (1997-2000) and Jerusalem (2003).}}\label{reporting-from-washington-dc-since-2005-previous-foreign-postings-include-london-1997-2000-and-jerusalem-2003}}

When I go back to Canada, people often come up to me, at the airport or
even when I'm at the grocery store with my mom, and ask me about Trump.
They say, ``How can this be happening?'' I'm very fortunate because my
husband, Paul Hunter, is also a correspondent in Washington. We can vent
to each other about how exhausting this is, how difficult it is to get
at the truth, and then cook a nice dinner and do it all over again. This
is such a historic, crazy story that we're covering. One day we'll look
back on it all and think, ``Wow, I was there,'' but right now, we're
just surviving. Washington's museums are a wonderful escape, too. I like
dropping into the National Museum of American History, if only for an
hour, to look at Muhammad Ali's gloves, say, or the Museum of Natural
History for the dinosaurs. It's a reminder that there's a bigger world
out there.

\hypertarget{amanda-mars-41}{%
\paragraph{AMANDA MARS, 41}\label{amanda-mars-41}}

\hypertarget{el-pauxeds-spain}{%
\subparagraph{\texorpdfstring{\textbf{\href{https://elpais.com/america/}{El
País}, Spain}}{El País, Spain}}\label{el-pauxeds-spain}}

\hypertarget{reporting-from-washington-dc-since-2017-previous-foreign-postings-include-new-york-2015-17}{%
\subparagraph{\texorpdfstring{\textbf{Reporting from Washington, D.C.,
since 2017; previous foreign postings include New York
(2015-17).}}{Reporting from Washington, D.C., since 2017; previous foreign postings include New York (2015-17).}}\label{reporting-from-washington-dc-since-2017-previous-foreign-postings-include-new-york-2015-17}}

It's been very surprising to me, since I first moved here, how readily
strangers in America talk to each other --- and talk to me. I might be
working in a cafe and someone will sit next to me, tell me their life
story, and then leave, often without giving their name. They'll say,
``Nice to talk with you,'' and I'm left thinking, ``OK, what happened?
You just told me your daughter died, and then you say, `Bye.''' People
have shared very intimate things with me on the subway. My first thought
was that perhaps America is a country of lonely people. But it's not
just that. In New York, there are just so many people and they mix every
day for a short while on the subway; they share their stories and then
they disappear. It's pretty amazing.

\hypertarget{adrian-morrow-33}{%
\paragraph{ADRIAN MORROW, 33}\label{adrian-morrow-33}}

\hypertarget{the-globe-and-mail-canada}{%
\subparagraph{\texorpdfstring{\textbf{\href{https://www.theglobeandmail.com/}{The
Globe and Mail},
Canada}}{The Globe and Mail, Canada}}\label{the-globe-and-mail-canada}}

\hypertarget{reporting-from-washington-dc-since-2017-first-foreign-posting-2}{%
\subparagraph{\texorpdfstring{\textbf{Reporting from Washington, D.C.,
since 2017; first foreign
posting.}}{Reporting from Washington, D.C., since 2017; first foreign posting.}}\label{reporting-from-washington-dc-since-2017-first-foreign-posting-2}}

Ninety-five percent of the time, living in the U.S. doesn't feel
dissimilar from living in Canada. But that five-percent difference can
be really striking. Last year, I was reporting a story on the Medicare
for All debate and I went to Virginia's Shenandoah Valley to cover a
pop-up clinic there run by a nonprofit that provides free medical care
in rural areas. The organization took over the county fairground for a
weekend and offered services on a first-come first-served basis. The
doors opened at 6 a.m., and people would line up overnight, pacing in
the cold. I met Virginians who had serious medical conditions that they
had put off treating for years because they couldn't afford to go to a
doctor. One woman had severe sleep apnea and regularly fell asleep at
the wheel of her car; there was a man who had injured his knee so badly
that he could feel it popping out as he walked. He'd spent two years
living like this without seeing a doctor. It felt like being in a
developing country, but we were here, in the world's wealthiest, most
powerful country, just two hours away from the capital.

\hypertarget{omri-nahmias-34}{%
\paragraph{OMRI NAHMIAS, 34}\label{omri-nahmias-34}}

\hypertarget{the-jerusalem-post-israel}{%
\subparagraph{\texorpdfstring{\textbf{\href{https://www.jpost.com/}{The
Jerusalem Post},
Israel}}{The Jerusalem Post, Israel}}\label{the-jerusalem-post-israel}}

\hypertarget{reporting-from-washington-dc-since-2016-first-foreign-posting-1}{%
\subparagraph{\texorpdfstring{\textbf{Reporting from Washington, D.C.,
since 2016; first foreign
posting.}}{Reporting from Washington, D.C., since 2016; first foreign posting.}}\label{reporting-from-washington-dc-since-2016-first-foreign-posting-1}}

You have to be ready to live this kind of life. When something dramatic
happens, I often need to be on a plane within two or three hours. I've
had to leave brunches with friends to cover breaking news. I'll find
myself working for three hours in the middle of Sunday. I don't have
plans for this weekend because Trump is expected to make an announcement
at any moment and I need to be at home near my computer: When the news
breaks, it's going to happen fast. I enjoy the adrenaline of this kind
of reporting. On weekends when I'm not on high alert, and when I miss
home, I go to Little Sesame, a hummus place in D.C. run by an Israeli
guy. And I'm the only person I know here who has a subscription for the
Israeli Basketball League. The N.B.A. is very exciting, but I miss my
team, Hapoel Jerusalem.

\hypertarget{beatriz-navarro-42}{%
\paragraph{BEATRIZ NAVARRO, 42}\label{beatriz-navarro-42}}

\hypertarget{la-vanguardia-spain}{%
\subparagraph{\texorpdfstring{\textbf{\href{https://www.lavanguardia.com/}{La
Vanguardia}, Spain}}{La Vanguardia, Spain}}\label{la-vanguardia-spain}}

\hypertarget{reporting-from-washington-dc-since-2018-previously-posted-in-brussels-2007-18}{%
\subparagraph{\texorpdfstring{\textbf{Reporting from Washington, D.C.,
since 2018; previously posted in Brussels
(2007-18).}}{Reporting from Washington, D.C., since 2018; previously posted in Brussels (2007-18).}}\label{reporting-from-washington-dc-since-2018-previously-posted-in-brussels-2007-18}}

In Spain, everyone asks me, ``Have you been to the White House? Have you
been into the briefing rooms they show in `The West Wing'?'' And I have.
But the thing is, shortly after I arrived, Sarah Huckabee Sanders ended
the traditional midday briefing at the White House, so I've never been
able to attend one. The only time that I've been into the White House
briefing room during this time and found somebody --- or something ---
on the stage, it was a turkey, and not the press secretary. It was the
ceremony of the pardoning of the White House turkey.

\hypertarget{brian-odonovan-40}{%
\paragraph{BRIAN O'DONOVAN, 40}\label{brian-odonovan-40}}

\hypertarget{rtuxe9-news-ireland}{%
\subparagraph{\texorpdfstring{\textbf{\href{https://www.rte.ie/news/}{RTÉ
News}, Ireland}}{RTÉ News, Ireland}}\label{rtuxe9-news-ireland}}

\hypertarget{reporting-from-washington-dc-since-2018-first-foreign-posting-1}{%
\subparagraph{\texorpdfstring{\textbf{Reporting from Washington, D.C.,
since 2018; first foreign
posting.}}{Reporting from Washington, D.C., since 2018; first foreign posting.}}\label{reporting-from-washington-dc-since-2018-first-foreign-posting-1}}

RTÉ has had Washington correspondents since the Bill Clinton
administration. I spoke with almost all of them before I started, and
they each had different advice to share. When I arrived here, I felt an
amazing buzz because, as a correspondent, you have to instantly hit the
ground running. I remember filing stories on my very first day while I
was also trying to figure out the move, relocate my family and find a
house. It was exciting. At RTÉ, each posting is four years, and I've
done two now, so I have two more to go. I look at this as one of the
best jobs within the station: You're a witness to history. Whether you
love him or hate him, Trump is a newsmaker. This will be remembered as a
unique time, and I'm privileged to be covering it and watching it
firsthand.

\hypertarget{renuxe9-pfister-46}{%
\paragraph{RENÉ PFISTER, 46}\label{renuxe9-pfister-46}}

\hypertarget{der-spiegel-germany}{%
\subparagraph{\texorpdfstring{\textbf{\href{https://www.spiegel.de/}{Der
Spiegel}, Germany}}{Der Spiegel, Germany}}\label{der-spiegel-germany}}

\hypertarget{reporting-from-washington-dc-since-2019-first-foreign-posting-1}{%
\subparagraph{\texorpdfstring{\textbf{Reporting from Washington, D.C.,
since 2019; first foreign
posting.}}{Reporting from Washington, D.C., since 2019; first foreign posting.}}\label{reporting-from-washington-dc-since-2019-first-foreign-posting-1}}

Trump tries to frame what I would call the value-based media --- the
organizations that are bound to certain values like truth --- as the
enemy of the people. He tells his supporters he's on their side in a
battle with the so-called liberal media mob. And I think American media
outlets are prone to take the bait and be dragged into the fight, which
is exactly what he wants. He can then say certain publications are
biased against him. He wants to destroy the credibility of the press
because he knows that if it's covering him correctly, it damages him.
And so the big challenge for the liberal and value-based media is to
cover and criticize Trump without giving the impression that they're in
a fight with him. It's a hard balance to get right.

\hypertarget{zhang-qi-29}{%
\paragraph{ZHANG QI, 29}\label{zhang-qi-29}}

\hypertarget{caixin-media-china}{%
\subparagraph{\texorpdfstring{\textbf{\href{https://www.caixinglobal.com/}{Caixin
Media}, China}}{Caixin Media, China}}\label{caixin-media-china}}

\hypertarget{reporting-from-washington-dc-since-2018-first-foreign-posting-2}{%
\subparagraph{\texorpdfstring{\textbf{Reporting from Washington, D.C.,
since 2018; first foreign
posting.}}{Reporting from Washington, D.C., since 2018; first foreign posting.}}\label{reporting-from-washington-dc-since-2018-first-foreign-posting-2}}

It can be hard for any foreign correspondent to get access to the
administration in Washington and I'm a Chinese journalist, which likely
adds another layer of difficulty. I once spoke to a staffer off the
record and asked him why I never heard back from certain politicians. He
replied, ``Because we don't really trust the Chinese press.'' And so
I've learned that I have to be very persistent, and not take things
personally. A reporter here always has to push to see what they can get
--- I think that's the most important thing. There is a phrase I learned
from an American White House reporter that really captures the attitude
a foreign correspondent needs to have in D.C.: ``It's better to ask for
forgiveness than ask for permission.'' I still think Washington is
relatively open, because I have been given access to press briefings and
can ask the president questions.

\hypertarget{chidanand-rajghatta-59}{%
\paragraph{CHIDANAND RAJGHATTA, 59}\label{chidanand-rajghatta-59}}

\hypertarget{the-times-of-india-india}{%
\subparagraph{\texorpdfstring{\textbf{\href{https://timesofindia.indiatimes.com/us}{The
Times of India},
India}}{The Times of India, India}}\label{the-times-of-india-india}}

\hypertarget{reporting-from-washington-dc-since-1995-first-foreign-posting}{%
\subparagraph{\texorpdfstring{\textbf{Reporting from Washington, D.C.,
since 1995; first foreign
posting.}}{Reporting from Washington, D.C., since 1995; first foreign posting.}}\label{reporting-from-washington-dc-since-1995-first-foreign-posting}}

When I first arrived in Washington 25 years ago, India was just a little
blip on America's radar. So much has changed since then: Now, the
India-U.S. partnership is India's principal international relationship.
I've also witnessed the rise of the Indian-American community, which is
now the best-educated and highest-earning immigrant community in the
U.S. There's a great deal of interest there. It's also been fascinating,
while based in the U.S., to watch the ascent of right-wing politicians
around the world, including in India. The chief guest at India's
Republic Day this year, for example, was the Brazilian president, Jair
Bolsonaro. A pattern has emerged: Bolsonaro, Boris Johnson, Donald Trump
and Narendra Modi are all right-of-center, populist and polarizing
figures. And Trump has a large following in India. Part of it has to do
with residual Islamophobia --- many right-wing Indians have problems
with Pakistan because they conflate, wrongly, Islam with terrorism ---
and they think Trump is tough on Islamic countries. So any
administration that's tough on Pakistan or Muslims is appreciated by the
Indian right wing.

\hypertarget{ben-riley-smith-32}{%
\paragraph{BEN RILEY-SMITH, 32}\label{ben-riley-smith-32}}

\hypertarget{the-daily-telegraph-united-kingdom}{%
\subparagraph{\texorpdfstring{\textbf{\href{https://www.telegraph.co.uk/}{The
Daily Telegraph}, United
Kingdom}}{The Daily Telegraph, United Kingdom}}\label{the-daily-telegraph-united-kingdom}}

\hypertarget{reporting-from-washington-dc-since-2017-first-foreign-posting-3}{%
\subparagraph{\texorpdfstring{\textbf{Reporting from Washington, D.C.
since 2017; first foreign
posting.}}{Reporting from Washington, D.C. since 2017; first foreign posting.}}\label{reporting-from-washington-dc-since-2017-first-foreign-posting-3}}

When I go to bed, my phone is sending me alerts that Trump is tweeting,
and when I get up in the morning there are more. He appears absolutely
attuned to the coverage of his presidency and is very apt at changing
the news cycle when he feels it's focusing on something he doesn't like
--- which can be hard for foreign correspondents whose print deadlines
are in the middle of the afternoon here, as mine is. In November 2018,
when I was writing about the results of the midterms, he suddenly held a
press conference in which he said about 20 newsworthy things --- so I
changed course and wrote about the press conference, but then he
announced that he had sacked Jeff Sessions. The paper was about to go to
print, and I quickly rewrote the top of my story. It's an endless wave
of news. It's fascinating to be part of, but also pretty exhausting.

\hypertarget{giuseppe-sarcina-58}{%
\paragraph{GIUSEPPE SARCINA, 58}\label{giuseppe-sarcina-58}}

\hypertarget{corriere-della-sera-italy}{%
\subparagraph{\texorpdfstring{\textbf{\href{https://www.corriere.it/}{Corriere
della Sera},
Italy}}{Corriere della Sera, Italy}}\label{corriere-della-sera-italy}}

\hypertarget{reporting-from-new-york-and-washington-dc-since-2016-previous-foreign-postings-include-brussels-2003-07}{%
\subparagraph{\texorpdfstring{\textbf{Reporting from New York and
Washington, D.C., since 2016; previous foreign postings include Brussels
(2003-07).}}{Reporting from New York and Washington, D.C., since 2016; previous foreign postings include Brussels (2003-07).}}\label{reporting-from-new-york-and-washington-dc-since-2016-previous-foreign-postings-include-brussels-2003-07}}

I belong to the old school of journalism. I think it's always better to
have someone who is directly accountable for their reporting. The news
on social media is sometimes accurate, sometimes totally fabricated and
often simply uninformed --- and nobody is accountable to the networks'
readers. But if you buy a newspaper, especially if it's an old
newspaper, it tends to be more solid. There is a bond between the
readers and the journalists, an agreement based on trust. My goal is
always, first of all, to understand what is going on, and second, to
explain it to the reader in plain language, presenting all the
information as objectively as possible. Correspondents are here to make
people aware of what is going on --- and to enable them to form their
own opinions.

\hypertarget{wataru-sawamura-56}{%
\paragraph{WATARU SAWAMURA, 56}\label{wataru-sawamura-56}}

\hypertarget{the-asahi-shimbun-japan}{%
\subparagraph{\texorpdfstring{\textbf{\href{http://www.asahi.com/ajw/}{The
Asahi Shimbun},
Japan}}{The Asahi Shimbun, Japan}}\label{the-asahi-shimbun-japan}}

\hypertarget{reporting-from-washington-dc-since-2017-previous-foreign-postings-include-new-york-1993-96-london-1998-2001-paris-2003-07-london-2011-13-and-beijing-2013-14}{%
\subparagraph{\texorpdfstring{\textbf{Reporting from Washington, D.C.,
since 2017; previous foreign postings include New York (1993-96), London
(1998-2001), Paris (2003-07), London (2011-13) and Beijing
(2013-14).}}{Reporting from Washington, D.C., since 2017; previous foreign postings include New York (1993-96), London (1998-2001), Paris (2003-07), London (2011-13) and Beijing (2013-14).}}\label{reporting-from-washington-dc-since-2017-previous-foreign-postings-include-new-york-1993-96-london-1998-2001-paris-2003-07-london-2011-13-and-beijing-2013-14}}

Last year, the White House press office stopped giving briefings, and so
the way we've been working recently is quite different from how our
correspondents operated here even two or three years ago. There are also
so many unknown people inside the government now. It's not as easy as it
might have been once for foreign journalists to find connections within
the administration. And anyway, it's the president who makes the
decisions, which means everything is unpredictable. You never know
what's going to happen next. So our bureau is not only watching what's
taking place inside the beltway, we're also encouraging our
correspondents to travel beyond D.C. --- to the Midwest, to the South
--- to meet the American people and local officials and politicians.
Even our Washington correspondents are now reporting from elsewhere in
the United States.

\hypertarget{yoshita-singh-37}{%
\paragraph{YOSHITA SINGH, 37}\label{yoshita-singh-37}}

\hypertarget{press-trust-of-india-india-1}{%
\subparagraph{\texorpdfstring{\textbf{Press Trust of India,
India}}{Press Trust of India, India}}\label{press-trust-of-india-india-1}}

\hypertarget{reporting-from-new-york-since-2011-previously-posted-in-chicago-2008-10-and-boston-2010-11}{%
\subparagraph{\texorpdfstring{\textbf{Reporting from New York since
2011; previously posted in Chicago (2008-10) and Boston
(2010-11).}}{Reporting from New York since 2011; previously posted in Chicago (2008-10) and Boston (2010-11).}}\label{reporting-from-new-york-since-2011-previously-posted-in-chicago-2008-10-and-boston-2010-11}}

I was based in New Delhi between 2006 and 2008, covering finance and the
commerce ministry, which is a very specific beat. Here, I have to cover
everything: the United Nations, India-U.S. relations, New York,
business, economics, culture. There's so much to learn. I'm based at the
U.N., so I'm surrounded by journalists from all over the world, working
together not just for their audiences, but trying to understand how this
administration's policies are unfolding for everyone. We look to each
other for help and to get a sense of how other countries see certain
issues. And so the more we work with each other, the more our worldviews
widen. But it's also a demanding job. At the end of the day, I just need
to see my kids. My son is 7 and my daughter is 2, and I don't have to
talk to them about world politics or the impeachment trial.

\hypertarget{david-smith-44}{%
\paragraph{DAVID SMITH, 44}\label{david-smith-44}}

\hypertarget{the-guardian-united-kingdom}{%
\subparagraph{\texorpdfstring{\textbf{\href{https://www.theguardian.com/us}{The
Guardian}, United
Kingdom}}{The Guardian, United Kingdom}}\label{the-guardian-united-kingdom}}

\hypertarget{reporting-from-washington-dc-since-2015-previously-posted-in-johannesburg-2009-15}{%
\subparagraph{\texorpdfstring{\textbf{Reporting from Washington, D.C.,
since 2015; previously posted in Johannesburg
(2009-15).}}{Reporting from Washington, D.C., since 2015; previously posted in Johannesburg (2009-15).}}\label{reporting-from-washington-dc-since-2015-previously-posted-in-johannesburg-2009-15}}

I was at Trump's election watch party in 2016, and a group of his
supporters heard my accent and thanked me, as a British person, for
Brexit. They believed that decision had started a political wave that
had now reached America --- and they were jubilant. Little did they
realize that, as a Guardian journalist, I'm not really a Brexit kind of
guy. The Guardian is an important liberal voice, and I think now more
than ever, during the Trump presidency, it's vital that we stand for
liberal values and investigate this administration when it does things
like put children in cages at the U.S.-Mexico border, or when the
president makes racist comments. Our readers are looking for that kind
of coverage at this time. When I write for an international audience,
too, it's interesting that stories that take a step back and look at the
big picture are often particularly popular with American readers.

\hypertarget{jon-sopel-60}{%
\paragraph{JON SOPEL, 60}\label{jon-sopel-60}}

\hypertarget{bbc-news-united-kingdom-1}{%
\subparagraph{\texorpdfstring{\textbf{\href{https://www.bbc.com/news/correspondents/jonsopel}{BBC
News}, United
Kingdom}}{BBC News, United Kingdom}}\label{bbc-news-united-kingdom-1}}

\hypertarget{reporting-from-washington-dc-since-2014-previous-foreign-postings-include-paris-1999-2003}{%
\subparagraph{\texorpdfstring{\textbf{Reporting from Washington, D.C.,
since 2014; previous foreign postings include Paris
(1999-2003).}}{Reporting from Washington, D.C., since 2014; previous foreign postings include Paris (1999-2003).}}\label{reporting-from-washington-dc-since-2014-previous-foreign-postings-include-paris-1999-2003}}

I've always thought that being a journalist is a good and quite
important job. During my time here, I've learned that, because of the
First Amendment, American journalists tend to take themselves more
seriously than British journalists do. We call ourselves ``hacks,'' and
there's a certain Hemingway stereotype of journalism that involves large
tumblers of whiskey, smoking late into the night and doing it all again
the next day. I am not nearly as important as an aid worker or a
teacher, or a doctor or a nurse --- people who do vital, socially
important work --- but at this moment, journalism is facing a real
battle for relevance. It can be very tempting at times, for example,
when you receive a torrent of abuse over a report you've filed, to
think, ``Oh God, I've had enough. I've done this for 30-plus years. I'm
out of here.'' But I feel this job has never been more important. I
think that truth exists, and it is our job to disseminate it, and do it
fairly. I'm sure people will scream at me, or the BBC, and say, ``You're
not fair, you're this, or you're that,'' and I think we still have a
mighty amount of work to do. But I'm more passionate about journalism
now than I ever have been, because it feels like there is something to
fight for.

\hypertarget{amir-tibon-31}{%
\paragraph{AMIR TIBON, 31}\label{amir-tibon-31}}

\hypertarget{haaretz-israel}{%
\subparagraph{\texorpdfstring{\textbf{\href{https://www.haaretz.com/misc/writers/WRITER-1.4699333}{Haaretz},
Israel}}{Haaretz, Israel}}\label{haaretz-israel}}

\hypertarget{reporting-from-washington-dc-since-2017-first-foreign-posting-4}{%
\subparagraph{\texorpdfstring{\textbf{Reporting from Washington, D.C.,
since 2017; first foreign
posting.}}{Reporting from Washington, D.C., since 2017; first foreign posting.}}\label{reporting-from-washington-dc-since-2017-first-foreign-posting-4}}

One of the challenges of my role is translating Trump's speeches into
Hebrew. For example, he might go to a rally in West Virginia and say
something about Israel and Palestine. So of course, there's a
one-sentence headline: ``He said something about Israel.'' But you also
want to give your readers the full context, and do you know what the
entire quote actually sounds like? He might begin to talk about Israel
and Palestine, but then he'll open parentheses to talk about Jewish
voters, then say something about the states that all the pundits in the
media said he would never win, and then maybe he'll start to talk about
the jobs that went to China, but he didn't finish the sentence about
Israel and the Palestinians yet --- we're still in the sentence! --- and
now he's talking about how he's doing a trade deal, and then he comes
back to Israel and Palestine. How do you translate that so your readers
will actually understand what he said and why it's meaningful and
important to them? Do you give them the full quote with all the
digressions in the middle? Do you just paraphrase it? This is a constant
dilemma.

\hypertarget{arjen-van-der-horst-46}{%
\paragraph{ARJEN VAN DER HORST, 46}\label{arjen-van-der-horst-46}}

\hypertarget{nos-the-netherlands-1}{%
\subparagraph{\texorpdfstring{\textbf{NOS, the
Netherlands}}{NOS, the Netherlands}}\label{nos-the-netherlands-1}}

\hypertarget{reporting-from-washington-dc-since-2014-previously-posted-in-london-2006-14}{%
\subparagraph{\texorpdfstring{\textbf{Reporting from Washington, D.C.,
since 2014; previously posted in London
(2006-14).}}{Reporting from Washington, D.C., since 2014; previously posted in London (2006-14).}}\label{reporting-from-washington-dc-since-2014-previously-posted-in-london-2006-14}}

Living here has changed my view of American exceptionalism in two ways:
On the one hand, the country is exceptionally bad at certain things. Its
carbon footprint is enormous, and there's so much inequality. My
colleague visited a homeless shelter in Montana, and 40 percent of the
people in that shelter had a job, which means something is wrong. I've
met people who have three jobs and can't make ends meet. The cliché used
to be that if you worked hard in America, you could move up. I don't
think that's the case anymore. On the other hand, I think that America
is truly exceptional in the diversity and energy of its people, and in
its landscapes. In a sense, it's a giant social experiment: ``How can we
live and work together in a peaceful way?'' It's the world on a micro
level.

\hypertarget{marcin-wrona-50}{%
\paragraph{MARCIN WRONA, 50}\label{marcin-wrona-50}}

\hypertarget{tvn-discovery-poland}{%
\subparagraph{\texorpdfstring{\textbf{\href{https://www.tvn.pl/}{TVN
Discovery},
Poland}}{TVN Discovery, Poland}}\label{tvn-discovery-poland}}

\hypertarget{reporting-from-washington-dc-since-2006-first-foreign-posting-1}{%
\subparagraph{\texorpdfstring{\textbf{Reporting from Washington, D.C.,
since 2006; first foreign
posting.}}{Reporting from Washington, D.C., since 2006; first foreign posting.}}\label{reporting-from-washington-dc-since-2006-first-foreign-posting-1}}

I remember Obama going without a press conference for 200 or so days
once, but Trump talks to the press almost daily --- which is amazing,
especially for a foreign correspondent. He was like this even before the
election. In October 2016, I was at the opening of the Trump
International Hotel in Washington, standing with my cameraman, when I
saw Trump passing by. I yelled, ``Sir, Sir!'' He stopped and said,
``Poland? OK, come over here.'' And we talked. The unpredictability of
this job gets exhausting, and because of the time difference, many
foreign correspondents have to just sleep whenever they can, like
soldiers. Sometimes I nap in my car. There was one reporter who found a
way of putting his suit jacket over his head so that he could get some
rest on the desk in the press room.

\hypertarget{wang-youyou}{%
\paragraph{WANG YOUYOU}\label{wang-youyou}}

\emph{(Wang Youyou of China's Phoenix TV, who has been reporting from
Washington, D.C., since 2015, declined to comment.)}

\textbf{As told to Heather Corcoran, Tal McThenia and Mimi Vu. Quotes
have been edited and condensed.}

\begin{center}\rule{0.5\linewidth}{\linethickness}\end{center}

\emph{\textbf{A behind-the-scenes account of the making of the group
portrait:}}

It was hard to predict which day in January would be the least
inconvenient for tens of foreign correspondents based in Washington,
D.C., and New York to suspend their work for a two-hour photo shoot.
Jan. 16, midway between the holidays and the Iowa caucuses, seemed a
safer date than most. Still, in the preceding days, a number of
newsworthy events loomed on the horizon, threatening to reduce turnout.
The House was preparing to deliver the impeachment articles to the
Senate. A trade deal with China appeared imminent. And for several days,
there was the very real possibility of a war with Iran. As an editor who
works mainly in the arts, I had never before had to worry that the
sudden unveiling of a geopolitical proposal (Trump was then teasing the
release of his long-awaited Middle East peace plan) might force the
subjects of a story to cancel. But trying to plan this shoot, alongside
T's photo team, gave me just the smallest glimpse into the uncertainty
--- what Joy Malbon, the Washington bureau chief for Canada's CTV News,
later described to me as the ``constant chaos'' --- that these
journalists live with and adjust for every day.

As it turned out, Jan. 16 was the first day of the impeachment trial.
Just two journalists had to pull out, but many of the 42 correspondents
who did attend the shoot were forced to talk to sources and file stories
from the set; only 41 people appear in the final image because, midway
through, Brian O'Donovan of the Irish broadcaster RTÉ had to find a
quiet corner from which to record a radio segment. Still, many of the
reporters, who come from 22 different countries, had not met before and
managed to find time to compare notes. A group that included Malbon and
José Díaz-Briseño of the Mexican newspaper Reforma discussed the
exhaustion inflicted by the ever-accelerating news cycle. Richard Madan,
also of CTV News, and James Glenday, of Australia's ABC News, spoke
about their organizations' respective coverage of the wildfires that
were then raging in Victoria and New South Wales. Wataru Sawamura, the
Washington bureau chief of Japan's Asahi Shimbun, arrived straight from
the airport --- he had been on a reporting trip to Chile --- and
conferred with his international colleagues about the challenges of
covering not only the United States but the entirety of the Americas, as
many of these correspondents do. By 11:30 a.m., though, when they were
meant to take their places for the portrait, Nancy Pelosi was holding
her weekly press conference three miles across town at the Capitol, and
the impeachment articles were about to be read. The resulting image for
this story, taken by Brian Finke, was captured in one of just a few
seconds-long windows during which his sitters' faces were not lit up by
the glowing screens of their phones. But even then, despite their
composed expressions, they were making sense of things, preparing to
explain yet another nearly unprecedented event to their audiences back
home. --- ALICE NEWELL-HANSON

\hypertarget{we-are-family-1}{%
\subsubsection{We Are Family}\label{we-are-family-1}}

\hypertarget{chapter-1-heirs-and-alumni}{%
\paragraph{Chapter 1: Heirs and
Alumni}\label{chapter-1-heirs-and-alumni}}

\href{/interactive/2020/04/13/t-magazine/black-art-galleries.html}{}

\hypertarget{the-artists}{%
\subparagraph{The Artists}\label{the-artists}}

\href{/interactive/2020/04/13/t-magazine/italian-fashion-design-houses.html}{}

\hypertarget{the-dynasties}{%
\subparagraph{The Dynasties}\label{the-dynasties}}

\href{/interactive/2020/04/13/t-magazine/gordon-parks.html}{}

\hypertarget{the-directors}{%
\subparagraph{The Directors}\label{the-directors}}

\href{/interactive/2020/04/13/t-magazine/enrique-olvera-chef.html}{}

\hypertarget{the-disciples}{%
\subparagraph{The Disciples}\label{the-disciples}}

\href{/interactive/2020/04/13/t-magazine/royal-academy-antwerp.html}{}

\hypertarget{the-graduates}{%
\subparagraph{The Graduates}\label{the-graduates}}

\hypertarget{chapter-2-reunions-and-reconsiderations}{%
\paragraph{Chapter 2: Reunions and
Reconsiderations}\label{chapter-2-reunions-and-reconsiderations}}

\href{/interactive/2020/04/13/t-magazine/ninth-street-greenwich-village-neighbors.html}{}

\hypertarget{the-neighbors}{%
\subparagraph{The Neighbors}\label{the-neighbors}}

\href{/interactive/2020/04/13/t-magazine/omen-restaurant-nyc.html}{}

\hypertarget{the-regulars}{%
\subparagraph{The Regulars}\label{the-regulars}}

\href{/interactive/2020/04/13/t-magazine/hair-musical-broadway.html}{}

\hypertarget{hair-1967}{%
\subparagraph{Hair (1967)}\label{hair-1967}}

\href{/interactive/2020/04/13/t-magazine/sweeney-todd-revival.html}{}

\hypertarget{sweeney-todd-2005-revival}{%
\subparagraph{Sweeney Todd (2005
Revival)}\label{sweeney-todd-2005-revival}}

\href{/interactive/2020/04/13/t-magazine/daughters-of-the-dust.html}{}

\hypertarget{daughters-of-the-dust-1991}{%
\subparagraph{Daughters of the Dust
(1991)}\label{daughters-of-the-dust-1991}}

\hypertarget{chapter-3-legends-pioneers-and-survivors}{%
\paragraph{Chapter 3: Legends Pioneers and
Survivors}\label{chapter-3-legends-pioneers-and-survivors}}

\href{/interactive/2020/04/13/t-magazine/butch-stud-lesbian.html}{}

\hypertarget{the-renegades}{%
\subparagraph{The Renegades}\label{the-renegades}}

\href{/interactive/2020/04/13/t-magazine/act-up-aids.html}{}

\hypertarget{the-activists}{%
\subparagraph{The Activists}\label{the-activists}}

\href{/interactive/2020/04/13/t-magazine/artist-recluse.html}{}

\hypertarget{the-shadows}{%
\subparagraph{The Shadows}\label{the-shadows}}

\href{/interactive/2020/04/13/t-magazine/black-actresses-bassett-berry-blige-henson-whitfield-elise.html}{}

\hypertarget{the-veterans}{%
\subparagraph{The Veterans}\label{the-veterans}}

\hypertarget{chapter-4-the-new-guard-1}{%
\paragraph{Chapter 4: The New Guard}\label{chapter-4-the-new-guard-1}}

\href{/interactive/2020/04/13/t-magazine/asian-american-fashion-designers.html}{}

\hypertarget{the-designers}{%
\subparagraph{The Designers}\label{the-designers}}

\href{13tmag-beauties.html}{}

\hypertarget{the-beauties}{%
\subparagraph{The Beauties}\label{the-beauties}}

\href{/interactive/2020/04/13/t-magazine/nyc-downtown-nightlife-party-scene.html}{}

\hypertarget{the-scenemakers}{%
\subparagraph{The Scenemakers}\label{the-scenemakers}}

\href{/interactive/2020/04/13/t-magazine/maria-cornejo-olivier-rousteing-telfar-clemens-alessandro-michele.html\#olivier-rousteing-and-co}{}

\hypertarget{olivier-rousteing-and-co}{%
\subparagraph{Olivier Rousteing and
Co.}\label{olivier-rousteing-and-co}}

\href{/interactive/2020/04/13/t-magazine/maria-cornejo-olivier-rousteing-telfar-clemens-alessandro-michele.html\#maria-cornejo-and-co}{}

\hypertarget{maria-cornejo-and-co}{%
\subparagraph{Maria Cornejo and Co.}\label{maria-cornejo-and-co}}

\href{/interactive/2020/04/13/t-magazine/maria-cornejo-olivier-rousteing-telfar-clemens-alessandro-michele.html\#telfar-clemens-and-co}{}

\hypertarget{telfar-clemens-and-co}{%
\subparagraph{Telfar Clemens and Co.}\label{telfar-clemens-and-co}}

\href{/interactive/2020/04/13/t-magazine/maria-cornejo-olivier-rousteing-telfar-clemens-alessandro-michele.html\#alessandro-michele-and-co}{}

\hypertarget{alessandro-michele-and-co}{%
\subparagraph{Alessandro Michele and
Co.}\label{alessandro-michele-and-co}}

\href{/interactive/2020/04/13/t-magazine/foreign-correspondents.html}{}

\hypertarget{the-journalists-1}{%
\subparagraph{The Journalists}\label{the-journalists-1}}

\begin{itemize}
\item
\item
\item
\item
\end{itemize}

Advertisement

\protect\hyperlink{after-bottom}{Continue reading the main story}

\hypertarget{site-index}{%
\subsection{Site Index}\label{site-index}}

\hypertarget{site-information-navigation}{%
\subsection{Site Information
Navigation}\label{site-information-navigation}}

\begin{itemize}
\tightlist
\item
  \href{https://help.nytimes.com/hc/en-us/articles/115014792127-Copyright-notice}{©~2020~The
  New York Times Company}
\end{itemize}

\begin{itemize}
\tightlist
\item
  \href{https://www.nytco.com/}{NYTCo}
\item
  \href{https://help.nytimes.com/hc/en-us/articles/115015385887-Contact-Us}{Contact
  Us}
\item
  \href{https://www.nytco.com/careers/}{Work with us}
\item
  \href{https://nytmediakit.com/}{Advertise}
\item
  \href{http://www.tbrandstudio.com/}{T Brand Studio}
\item
  \href{https://www.nytimes.com/privacy/cookie-policy\#how-do-i-manage-trackers}{Your
  Ad Choices}
\item
  \href{https://www.nytimes.com/privacy}{Privacy}
\item
  \href{https://help.nytimes.com/hc/en-us/articles/115014893428-Terms-of-service}{Terms
  of Service}
\item
  \href{https://help.nytimes.com/hc/en-us/articles/115014893968-Terms-of-sale}{Terms
  of Sale}
\item
  \href{https://spiderbites.nytimes.com}{Site Map}
\item
  \href{https://help.nytimes.com/hc/en-us}{Help}
\item
  \href{https://www.nytimes.com/subscription?campaignId=37WXW}{Subscriptions}
\end{itemize}
