\hypertarget{the-downtown-new-york-restaurant-with-a-whos-who-list-of-devotees}{%
\section{The Downtown New York Restaurant With a Who's Who List of
Devotees}\label{the-downtown-new-york-restaurant-with-a-whos-who-list-of-devotees}}

April 13, 2020

\begin{itemize}
\item
\item
\item
\item
\end{itemize}

From the start, Omen was a gathering place for artists, writers, actors
and designers --- 40 years later, in a vastly different city, it still
is.

\href{https://www.nytimes.com/interactive/2020/04/13/t-magazine/culture-issue-2020.html}{We
Are Family}

\hypertarget{chapter-2-reunions-and-reconsiderations}{%
\subparagraph{Chapter 2: Reunions and
Reconsiderations}\label{chapter-2-reunions-and-reconsiderations}}

\hypertarget{previous}{%
\subparagraph{Previous}\label{previous}}

\hypertarget{next}{%
\subparagraph{Next}\label{next}}

\hypertarget{the-downtown-new-york-restaurant-with-a-whos-who-list-of-devotees-1}{%
\section{The Downtown New York Restaurant With a Who's Who List of
Devotees}\label{the-downtown-new-york-restaurant-with-a-whos-who-list-of-devotees-1}}

From its opening, the Japanese restaurant Omen has been a gathering
place for artists, writers, actors and fashion designers. Forty years
later, in a vastly different downtown New York, it still is.

By Patti Smith

April 13, 2020

SHARE

THE STORY OF \href{http://www.omen-azen.com/}{OMEN} is the story of how
the world comes about without design. Like a simple musical phrase drawn
by a lone saxophone, steadily augmenting to form a memorable motif,
there was no particular strategy in the creation of Omen's song. It came
about organically, gently guided by the humble hand of its founder and
proprietor, Mikio Shinagawa, evolving as a treasured presence in SoHo
for nearly four decades.

In the '60s and '70s, SoHo was all but deserted. The public developer
\href{https://www.nytimes.com/topic/person/robert-moses}{Robert Moses}
had projected the idea of building an expressway across Broome Street,
so no one was prepared to invest in the area. Artists moved into the
abundance of empty lofts, grateful for ample space and low rents. There
were few local restaurants back then.
\href{https://www.nytimes.com/2019/10/14/t-magazine/fanelli-cafe.html}{Fanelli
Cafe}, on Prince Street, had been around (under different names and
owners) since the mid-19th century, offering a continuous communal
atmosphere. Gordon Matta-Clark and his friends
\href{https://tmagazine.blogs.nytimes.com/2013/05/10/food-matters-when-eating-and-art-became-one/}{started
the restaurant} called Food in 1971, on the corner of Prince and Wooster
Streets, where artists would take turns cooking for friends who gathered
there to eat. But most convened in the coffee houses on MacDougal
Street.

Mikio came to New York in 1976 from Kyoto --- which was and still is a
center of higher learning, with its Golden Pavilion, scurrying monks and
the sound of rakes combing the stones of temple gardens. He had been a
Buddhist monk himself, but gave up traditional practice when he realized
that anything he did could be considered a practice.

In 1981, assessing the needs of his downtown neighborhood, Mikio decided
to open a place where people could partake of healthy Japanese country
food. He also wanted to provide a haven for discourse. He took to heart
the wishes of his parents --- his father was the poet and calligrapher
Tetsuzan Shinagawa --- who desired that their children promote peace and
share the most enlightened aspects of their culture in their ventures.

The T List \textbar{}

Sign up here

At the same time, the painter
\href{https://www.nytimes.com/1981/06/26/obituaries/david-grossblatt-61-a-painter-and-operator-of-the-cafe-rienzi.html}{David
Grossblatt}, who ran the popular Cafe Rienzi, and co-owned the O.G.
Dining Rooms, passed away. The O.G. closed and stood vacant. Mikio took
over the space, brought craftsmen from Japan and built a warm, spare and
rustic interior. Mikio never thought of it as a business but instead as
a kind of Japanese salon in the city. The brick walls were hung with the
calligraphy of his father, whose elegant brush strokes represented the
Japanese character \emph{mu}, meaning ``nothingness.''

Note by note, Omen came into being, attracting artists, musicians,
actors and architects. Kyoto's country-style cuisine was served --- fine
tea and handmade udon noodles in a rich broth --- creating the serene
yet spirited atmosphere of Mikio's ancestral community.
\href{https://www.nytimes.com/2015/10/23/t-magazine/yoko-ono-illustrated-interview.html}{Yoko
Ono} frequented it. She has often shared meals there with her son, Sean.

Through the years, one could find many leading figures in the arts.
There seems to be an unspoken rule of maintaining a congenial but
private atmosphere, with no sense of celebrity. There are regulars who
have patronized Omen throughout its entire existence; at times, one gets
a sense of downtown as it once was, a place of kinship, common mind and
residence. Everyone shares a part of the atmosphere, and that is how New
York City once was, with its air of casual discretion. The last time I
saw \href{https://www.nytimes.com/topic/person/lou-reed}{Lou Reed}, not
long before he died, he was in Omen with
\href{https://www.nytimes.com/2018/04/26/t-magazine/larry-clark-laurie-anderson-bill-t-jones-80s-work.html}{Laurie
Anderson}, and what he spoke of, at that moment, was love.

In Omen, people talk to one another. It is not a place for cellphones.
The conversational din is complemented by the spiritual horns of
\href{https://www.nytimes.com/topic/person/miles-davis}{Miles Davis} and
\href{https://www.nytimes.com/topic/person/john-coltrane}{John Coltrane}
at a civilized volume. Dreamy, soft-spoken and impeccable, like a spirit
visitor, Mikio glides though the brick-walled dining room, stopping to
give a warm hello or to sit and share a sake, noting the loss of a
friend and the blessings of a new one. Green tea is brought to the
table, and for a while, all is right with the world.

MUCH HAS CHANGED in our downtown area since Mikio opened Omen in 1981.
The relentless demolition, construction and astronomical rise in rents
to accommodate a burgeoning corporate culture has had a deep effect on a
traditionally creative community. One by one, we lose beloved landmarks,
diners, community parks and a sense of our own history. Our Omen
stoically remains.

An omen is a sign, at best a portent to good. Following a ribbon of
notes up the stairs, I pass through the braided curtain, sit at a table
beneath an ideogram symbolizing nothingness before a bowl of perfect
broth. Here I have found the warmth of friendship. Here I have come in
tears and left comforted. Here I have come hungry and left satisfied.

Where shall we go? Let's go to Omen. And all agree.

Patti Smith is an award-winning poet, author, singer-songwriter and
visual artist. Jason Schmidt specializes in artist portraiture,
architecture and interiors. Photo assistants: Kyle Knodell, Wil Pierce
and Julia Wilson.

\begin{center}\rule{0.5\linewidth}{\linethickness}\end{center}

\emph{\textbf{A behind-the-scenes account of the making of the group
portrait:}}

``Why don't we talk about it over a cup of tea?'' So began the answer
from Mikio Shinagawa, the owner of Omen Azen, after I initially
approached him several months ago (before the arrival of the pandemic)
about doing a story for
\href{https://www.nytimes.com/interactive/2020/04/13/t-magazine/culture-issue-2020.html}{T's
Culture issue} on the SoHo restaurant he has owned and operated ever
since he opened its doors in 1981. Omen, with its dark wood and tasteful
touches --- from the polka-dot navy napkins to the soft light cast by
the paper chandeliers --- has a warm welcoming quality to it, like the
whiskey bars or jazz clubs one can find only in Tokyo. As Patti Smith
writes in her essay for the issue: ``In Omen, people talk to one
another. It is not a place for cellphones. The conversational din is
complemented by the spiritual horns of Miles Davis and John Coltrane at
a civilized volume. \ldots{} Green tea is brought to the table, and for
a while, all is right with the world.''

​And so, for the first time in my life, I was producing a magazine story
in person --- not over email or on the phone, but over green tea that
had been poured into a delicate ceramic cup. Sitting across from
Shinagawa, I explained that T wanted to feature not just Omen but also
the people who frequent it, and have been doing so for the last several
decades. Omen has always been the sort of place where, on any given
night, the tables are populated by fashion designers, magazine editors,
artists, architects, poets and whoever else has managed to drift through
downtown New York and stay around. ``The thing about Omen,'' the actress
and producer Stella Schnabel told me, ``is that you will always get a
great meal, but be ready to bump into the likes of anyone you might
know.''

Yet while there are those restaurants that insist on an aura of
exclusivity, flaunting the status gleaned from and imparted to their
guests, Omen is different. It never invites anything but a sense of
gathering. Even as Shinagawa opened up to the idea of inviting longtime
patrons together for a group portrait, he never gave the slightest
impression of regarding the story as an opportunity. Such discretion may
be the secret to why Omen is such a perfect restaurant. ``For Mikio,''
Smith said to me when I asked her to write her essay, ``anything.''

​A few weeks later, on a wintry afternoon, the restaurant's regulars,
who could also be described as friends of Shinagawa's, gathered over
platters of sushi and other small bites. They greeted each other warmly,
making jokes and light banter. There was the great choreographer and
dancer Bill T. Jones (his husband, Bjorn Amelan, couldn't make it)
chatting with the interior architect designer Bill Katz. There was the
artist Rachel Feinstein, with her dog, Mr. Greenjeans; she knows
Shinagawa so well that when she and her husband, the painter John
Currin, were last in Kyoto (where Shinagawa splits his time), they
called him up to show them around. There was the artist Francesco
Clemente and the three children of fellow artist Julian **** Schnabel
--- Lola, Stella and Vito (their father, another Omen devotee, also
couldn't make it). Shinagawa told me he remembers them as little kids at
the restaurant. And there was the movie star Richard Gere, who bellowed
affectionately Shinagawa --- ``Mikio!'' **** --- when he walked in with
his pregnant wife, Alejandra.

​Afterward, a few bites of uni and pickled cucumber still sat on the
platters, and I picked one up to nibble. Smith was chatting with Lola
Schnabel. Others were exchanging pleasantries and goodbyes. For a
moment, before everyone scattered to return to their lives, I let myself
imagine being part of Shinagawa's circle, and part of the sort of world
--- one where art offers a sense of camaraderie, and where a
neighborhood is a community, not a place to make a commercial investment
--- that increasingly seems to be disappearing from a city such as New
York. --- THESSALY LA FORCE

\href{https://www.nytimes.com/2019/10/14/t-magazine/fanelli-cafe.html}{}

\hypertarget{the-bar-that-has-fed-soho-for-almost-a-centuryoct-14-2019}{%
\paragraph{The Bar That Has Fed SoHo for Almost a CenturyOct. 14,
2019}\label{the-bar-that-has-fed-soho-for-almost-a-centuryoct-14-2019}}

\includegraphics{https://static01.nyt.com/images/2019/10/14/t-magazine/14tmag-fanelli-slide-6GYD/14tmag-fanelli-slide-6GYD-mediumThreeByTwo210.jpg}
\href{https://www.nytimes.com/2018/08/14/t-magazine/raouls-new-york-restaurant-history.html}{}

\hypertarget{the-new-york-restaurant-where-almost-nothing-has-changed-since-1975aug-15-2018}{%
\paragraph{The New York Restaurant Where (Almost) Nothing Has Changed
Since 1975Aug. 15,
2018}\label{the-new-york-restaurant-where-almost-nothing-has-changed-since-1975aug-15-2018}}

\includegraphics{https://static01.nyt.com/images/2018/08/14/t-magazine/food/Raouls-slide-7M1Z/Raouls-slide-7M1Z-mediumThreeByTwo210.jpg}
\href{https://www.nytimes.com/2019/12/20/t-magazine/bemelmans-bar.html}{}

\hypertarget{a-bar-revered-for-its-old-fashioned-charm-and-witty-muralsdec-20-2019}{%
\paragraph{A Bar Revered for Its Old-Fashioned Charm and Witty
MuralsDec. 20,
2019}\label{a-bar-revered-for-its-old-fashioned-charm-and-witty-muralsdec-20-2019}}

\includegraphics{https://static01.nyt.com/images/2019/12/20/t-magazine/entertainment/20-tmag-bemelmans-slide-DTPP/20-tmag-bemelmans-slide-DTPP-mediumThreeByTwo210.jpg}
\href{https://www.nytimes.com/2019/01/24/t-magazine/indochine-restaurant-history.html}{}

\hypertarget{the-restaurant-that-has-helped-new-yorkers-feel-famous-since-1984jan-24-2019}{%
\paragraph{The Restaurant That Has Helped New Yorkers Feel Famous Since
1984Jan. 24,
2019}\label{the-restaurant-that-has-helped-new-yorkers-feel-famous-since-1984jan-24-2019}}

\includegraphics{https://static01.nyt.com/images/2019/01/15/t-magazine/food/indochine-slide-14IB/indochine-slide-14IB-mediumThreeByTwo210.jpg}

\hypertarget{correction-april-14-2020}{%
\subparagraph{\texorpdfstring{\textbf{Correction} April 14,
2020}{Correction April 14, 2020}}\label{correction-april-14-2020}}

An earlier version of a picture caption with this article misstated the
surname of a furniture designer. He is Dakota Jackson, not Johnson.

\hypertarget{we-are-family-1}{%
\subsubsection{We Are Family}\label{we-are-family-1}}

\hypertarget{chapter-1-heirs-and-alumni}{%
\paragraph{Chapter 1: Heirs and
Alumni}\label{chapter-1-heirs-and-alumni}}

\href{/interactive/2020/04/13/t-magazine/black-art-galleries.html}{}

\hypertarget{the-artists}{%
\subparagraph{The Artists}\label{the-artists}}

\href{/interactive/2020/04/13/t-magazine/italian-fashion-design-houses.html}{}

\hypertarget{the-dynasties}{%
\subparagraph{The Dynasties}\label{the-dynasties}}

\href{/interactive/2020/04/13/t-magazine/gordon-parks.html}{}

\hypertarget{the-directors}{%
\subparagraph{The Directors}\label{the-directors}}

\href{/interactive/2020/04/13/t-magazine/enrique-olvera-chef.html}{}

\hypertarget{the-disciples}{%
\subparagraph{The Disciples}\label{the-disciples}}

\href{/interactive/2020/04/13/t-magazine/royal-academy-antwerp.html}{}

\hypertarget{the-graduates}{%
\subparagraph{The Graduates}\label{the-graduates}}

\hypertarget{chapter-2-reunions-and-reconsiderations-1}{%
\paragraph{Chapter 2: Reunions and
Reconsiderations}\label{chapter-2-reunions-and-reconsiderations-1}}

\href{/interactive/2020/04/13/t-magazine/ninth-street-greenwich-village-neighbors.html}{}

\hypertarget{the-neighbors}{%
\subparagraph{The Neighbors}\label{the-neighbors}}

\href{/interactive/2020/04/13/t-magazine/omen-restaurant-nyc.html}{}

\hypertarget{the-regulars}{%
\subparagraph{The Regulars}\label{the-regulars}}

\href{/interactive/2020/04/13/t-magazine/hair-musical-broadway.html}{}

\hypertarget{hair-1967}{%
\subparagraph{Hair (1967)}\label{hair-1967}}

\href{/interactive/2020/04/13/t-magazine/sweeney-todd-revival.html}{}

\hypertarget{sweeney-todd-2005-revival}{%
\subparagraph{Sweeney Todd (2005
Revival)}\label{sweeney-todd-2005-revival}}

\href{/interactive/2020/04/13/t-magazine/daughters-of-the-dust.html}{}

\hypertarget{daughters-of-the-dust-1991}{%
\subparagraph{Daughters of the Dust
(1991)}\label{daughters-of-the-dust-1991}}

\hypertarget{chapter-3-legends-pioneers-and-survivors}{%
\paragraph{Chapter 3: Legends Pioneers and
Survivors}\label{chapter-3-legends-pioneers-and-survivors}}

\href{/interactive/2020/04/13/t-magazine/butch-stud-lesbian.html}{}

\hypertarget{the-renegades}{%
\subparagraph{The Renegades}\label{the-renegades}}

\href{/interactive/2020/04/13/t-magazine/act-up-aids.html}{}

\hypertarget{the-activists}{%
\subparagraph{The Activists}\label{the-activists}}

\href{/interactive/2020/04/13/t-magazine/artist-recluse.html}{}

\hypertarget{the-shadows}{%
\subparagraph{The Shadows}\label{the-shadows}}

\href{/interactive/2020/04/13/t-magazine/black-actresses-bassett-berry-blige-henson-whitfield-elise.html}{}

\hypertarget{the-veterans}{%
\subparagraph{The Veterans}\label{the-veterans}}

\hypertarget{chapter-4-the-new-guard}{%
\paragraph{Chapter 4: The New Guard}\label{chapter-4-the-new-guard}}

\href{/interactive/2020/04/13/t-magazine/asian-american-fashion-designers.html}{}

\hypertarget{the-designers}{%
\subparagraph{The Designers}\label{the-designers}}

\href{13tmag-beauties.html}{}

\hypertarget{the-beauties}{%
\subparagraph{The Beauties}\label{the-beauties}}

\href{/interactive/2020/04/13/t-magazine/nyc-downtown-nightlife-party-scene.html}{}

\hypertarget{the-scenemakers}{%
\subparagraph{The Scenemakers}\label{the-scenemakers}}

\href{/interactive/2020/04/13/t-magazine/maria-cornejo-olivier-rousteing-telfar-clemens-alessandro-michele.html\#olivier-rousteing-and-co}{}

\hypertarget{olivier-rousteing-and-co}{%
\subparagraph{Olivier Rousteing and
Co.}\label{olivier-rousteing-and-co}}

\href{/interactive/2020/04/13/t-magazine/maria-cornejo-olivier-rousteing-telfar-clemens-alessandro-michele.html\#maria-cornejo-and-co}{}

\hypertarget{maria-cornejo-and-co}{%
\subparagraph{Maria Cornejo and Co.}\label{maria-cornejo-and-co}}

\href{/interactive/2020/04/13/t-magazine/maria-cornejo-olivier-rousteing-telfar-clemens-alessandro-michele.html\#telfar-clemens-and-co}{}

\hypertarget{telfar-clemens-and-co}{%
\subparagraph{Telfar Clemens and Co.}\label{telfar-clemens-and-co}}

\href{/interactive/2020/04/13/t-magazine/maria-cornejo-olivier-rousteing-telfar-clemens-alessandro-michele.html\#alessandro-michele-and-co}{}

\hypertarget{alessandro-michele-and-co}{%
\subparagraph{Alessandro Michele and
Co.}\label{alessandro-michele-and-co}}

\href{/interactive/2020/04/13/t-magazine/foreign-correspondents.html}{}

\hypertarget{the-journalists}{%
\subparagraph{The Journalists}\label{the-journalists}}

\begin{itemize}
\item
\item
\item
\item
\end{itemize}

Advertisement

\protect\hyperlink{after-bottom}{Continue reading the main story}

\hypertarget{site-index}{%
\subsection{Site Index}\label{site-index}}

\hypertarget{site-information-navigation}{%
\subsection{Site Information
Navigation}\label{site-information-navigation}}

\begin{itemize}
\tightlist
\item
  \href{https://help.nytimes.com/hc/en-us/articles/115014792127-Copyright-notice}{©~2020~The
  New York Times Company}
\end{itemize}

\begin{itemize}
\tightlist
\item
  \href{https://www.nytco.com/}{NYTCo}
\item
  \href{https://help.nytimes.com/hc/en-us/articles/115015385887-Contact-Us}{Contact
  Us}
\item
  \href{https://www.nytco.com/careers/}{Work with us}
\item
  \href{https://nytmediakit.com/}{Advertise}
\item
  \href{http://www.tbrandstudio.com/}{T Brand Studio}
\item
  \href{https://www.nytimes.com/privacy/cookie-policy\#how-do-i-manage-trackers}{Your
  Ad Choices}
\item
  \href{https://www.nytimes.com/privacy}{Privacy}
\item
  \href{https://help.nytimes.com/hc/en-us/articles/115014893428-Terms-of-service}{Terms
  of Service}
\item
  \href{https://help.nytimes.com/hc/en-us/articles/115014893968-Terms-of-sale}{Terms
  of Sale}
\item
  \href{https://spiderbites.nytimes.com}{Site Map}
\item
  \href{https://help.nytimes.com/hc/en-us}{Help}
\item
  \href{https://www.nytimes.com/subscription?campaignId=37WXW}{Subscriptions}
\end{itemize}
