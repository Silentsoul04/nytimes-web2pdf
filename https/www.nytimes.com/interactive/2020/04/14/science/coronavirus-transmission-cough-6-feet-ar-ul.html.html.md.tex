Sections

SEARCH

\protect\hyperlink{site-content}{Skip to
content}\protect\hyperlink{site-index}{Skip to site index}

\hypertarget{this-3-d-simulation-shows-why-social-distancing-is-so-important}{%
\section{This 3-D Simulation Shows Why Social Distancing Is So
Important}\label{this-3-d-simulation-shows-why-social-distancing-is-so-important}}

By \href{https://www.nytimes.com/by/yuliya-parshina-kottas}{Yuliya
Parshina-Kottas}, \href{https://www.nytimes.com/by/bedel-saget}{Bedel
Saget}, \href{https://www.nytimes.com/by/karthik-patanjali}{Karthik
Patanjali}, Or Fleisher and Gabriel GianordoliApril 14, 2020

\begin{itemize}
\item
\item
\item
\item
\end{itemize}

\includegraphics{https://static01.nyt.com/newsgraphics/2020/03/20/covid-19-safe-zone/assets/images/cover-8-2000.png}

Public health experts and elected officials have emphasized again and
again that social distancing is the best tool we have to slow
\href{https://www.nytimes.com/news-event/coronavirus}{the coronavirus
outbreak}.

The Centers for Disease Control and Prevention encourages people to stay
home. If you must venture out, you should stay at least six feet away
from others. The World Health Organization
\href{https://www.who.int/emergencies/diseases/novel-coronavirus-2019/advice-for-public}{recommends}
a minimum of three feet of separation.

Scientists are learning about the novel coronavirus in real time, and
those who study similar respiratory illnesses say that until it is
better understood,
\href{https://www.nytimes.com/2020/04/14/health/coronavirus-six-feet.html}{no
guideline is likely
to}\href{https://www.nytimes.com/2020/04/14/health/coronavirus-six-feet.html}{offer
perfect safety}. Instead, understanding the possible transmission routes
for the virus can help us see why keeping our distance is so important.

Scientists who study the transmission of respiratory illnesses like
influenza say that infections typically happen when a healthy person
comes into contact with respiratory droplets from an infected person's
cough, sneeze or breath.

This simulation, created using research data from the Kyoto Institute of
Technology, offers one view of what can happen when someone coughs
indoors. A cough produces respiratory droplets of varying sizes. Larger
droplets fall to the floor, or break up into smaller droplets.

The heaviest coughs release about a quarter-teaspoon of fluid, with
droplets dispersing quickly throughout the room. The simulation shows
their spread over a minute, inside a room of about 600 square feet.
Under other conditions, the particles could behave differently.

The C.D.C. says keeping at least six feet away from others can help you
avoid contact with these respiratory droplets and lower the risk of
infection. That guidance is based on the assumption that transmission
mainly occurs through large droplets that fall in close proximity.

But as this simulation suggests, and scientists have argued, droplets
can travel farther than six feet. And small droplets known as aerosols
can remain suspended or travel through the air before they eventually
settle on surfaces. This is how they could disperse over the next 20
minutes.

``It's not like, `Oh, it's six feet, they've all fallen and there's
nothing,''' said Donald K. Milton, an
\href{https://www.pnas.org/content/115/5/1081}{infectious
aerosol}\href{https://www.pnas.org/content/115/5/1081}{s scientist} at
the University of Maryland's School of Public Health. ``It's more like
it's a continuum.''

In fact,
\href{https://jamanetwork.com/journals/jama/fullarticle/2763852}{researchers
at M.I.T.} studying coughs and sneezes observed particles from a cough
traveling as far as 16 feet and those from a sneeze traveling as far as
26 feet.

All of this suggests that keeping a distance of six feet or more can
greatly reduce the possibility of transmission, compared with being
closer. ``The farther you get away, the more diluted it is,'' Dr. Milton
said of the aerosol.

Coughing and sneezing may not be the only causes for concern. Studies of
influenza have shown that infected people with mild or no symptoms may
also generate infectious droplets through speaking and breathing.

An infected person talking for five minutes in a poorly ventilated space
can produce as many viral droplets as one infectious cough. ``If there
are 10 people in there, it's going to build up,'' said Pratim Biswas, an
aerosols expert at Washington University in St. Louis.

In recent days, public health officials have suggested that, in addition
to social distancing, more people should wear face masks to help slow
the spread of the virus.

A mask disrupts the trajectory of a cough, sneeze or breath and captures
some respiratory droplets before they can spew out. A mask can also
prevent large infectious droplets from landing on the nose and mouth,
even though it provides minimal protection against inhaling the smaller
droplets.

Wearing a mask can help protect yourself and others. So if you do need
to leave home,
\href{https://www.nytimes.com/2020/04/10/well/live/coronavirus-face-masks-guides-protection-personal-protective-equipment.html}{wear
a mask} and be sure to keep your distance.

**

See How to Social Distance in Your Space

\hypertarget{see-how-to-social-distance-in-your-space}{%
\subsection{See How to Social Distance in Your
Space}\label{see-how-to-social-distance-in-your-space}}

We used augmented reality to show you how social distancing guidelines
can apply in real life: at the grocery store, on the sidewalk, or
anywhere else. See the video of this here. To experience this in your
space, you will need to use the NYTimes iOS App on a newer iPhone or
iPad.

Shot while using the NYTimes app for iOS

The augmented reality experience is available only on newer iPhones and
iPads using the NYTimes app. To view on the app, open the camera on your
device and point to the QR tag
below.\includegraphics{https://static01.nyt.com/newsgraphics/2020/03/20/covid-19-safe-zone/97227895b097a7cc171f8bcda85bc85223ed10f9/qr-tag-apr14.png}

Note: This visualization is based on a scientific simulation that has
not yet been peer reviewed.

Sources: Masashi Yamakawa, Kyoto Institute of Technology; Donald K.
Milton, Maryland Institute for Applied Environmental Health, University
of Maryland School of Public Health; Pratim Biswas, Aerosol and Air
Quality Research Laboratory at Washington University in St. Louis; Barry
Scharfman, Head of Data Science at PA Aerospace \& Defense.

Makiko Inoue contributed reporting from Tokyo. Additional work by Jon
Huang and Mark McKeague.

\begin{itemize}
\item
\item
\item
\item
\end{itemize}

Advertisement

\protect\hyperlink{after-bottom}{Continue reading the main story}

\hypertarget{site-index}{%
\subsection{Site Index}\label{site-index}}

\hypertarget{site-information-navigation}{%
\subsection{Site Information
Navigation}\label{site-information-navigation}}

\begin{itemize}
\tightlist
\item
  \href{https://help.nytimes.com/hc/en-us/articles/115014792127-Copyright-notice}{©~2020~The
  New York Times Company}
\end{itemize}

\begin{itemize}
\tightlist
\item
  \href{https://www.nytco.com/}{NYTCo}
\item
  \href{https://help.nytimes.com/hc/en-us/articles/115015385887-Contact-Us}{Contact
  Us}
\item
  \href{https://www.nytco.com/careers/}{Work with us}
\item
  \href{https://nytmediakit.com/}{Advertise}
\item
  \href{http://www.tbrandstudio.com/}{T Brand Studio}
\item
  \href{https://www.nytimes.com/privacy/cookie-policy\#how-do-i-manage-trackers}{Your
  Ad Choices}
\item
  \href{https://www.nytimes.com/privacy}{Privacy}
\item
  \href{https://help.nytimes.com/hc/en-us/articles/115014893428-Terms-of-service}{Terms
  of Service}
\item
  \href{https://help.nytimes.com/hc/en-us/articles/115014893968-Terms-of-sale}{Terms
  of Sale}
\item
  \href{https://spiderbites.nytimes.com}{Site Map}
\item
  \href{https://help.nytimes.com/hc/en-us}{Help}
\item
  \href{https://www.nytimes.com/subscription?campaignId=37WXW}{Subscriptions}
\end{itemize}
