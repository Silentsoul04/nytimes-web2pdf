Sections

SEARCH

\protect\hyperlink{site-content}{Skip to
content}\protect\hyperlink{site-index}{Skip to site index}

\href{https://www.nytimes.com/section/business}{Business}

\href{https://myaccount.nytimes.com/auth/login?response_type=cookie\&client_id=vi}{}

\href{https://www.nytimes.com/section/todayspaper}{Today's Paper}

\href{/section/business}{Business}\textbar{}Where the Small-Business
Relief Loans Have Gone

\url{https://nyti.ms/2WOT5xh}

\begin{itemize}
\item
\item
\item
\item
\item
\end{itemize}

\href{https://www.nytimes.com/news-event/coronavirus?action=click\&pgtype=Article\&state=default\&region=TOP_BANNER\&context=storylines_menu}{The
Coronavirus Outbreak}

\begin{itemize}
\tightlist
\item
  live\href{https://www.nytimes.com/2020/08/04/world/coronavirus-cases.html?action=click\&pgtype=Article\&state=default\&region=TOP_BANNER\&context=storylines_menu}{Latest
  Updates}
\item
  \href{https://www.nytimes.com/interactive/2020/us/coronavirus-us-cases.html?action=click\&pgtype=Article\&state=default\&region=TOP_BANNER\&context=storylines_menu}{Maps
  and Cases}
\item
  \href{https://www.nytimes.com/interactive/2020/science/coronavirus-vaccine-tracker.html?action=click\&pgtype=Article\&state=default\&region=TOP_BANNER\&context=storylines_menu}{Vaccine
  Tracker}
\item
  \href{https://www.nytimes.com/2020/08/02/us/covid-college-reopening.html?action=click\&pgtype=Article\&state=default\&region=TOP_BANNER\&context=storylines_menu}{College
  Reopening}
\item
  \href{https://www.nytimes.com/live/2020/08/04/business/stock-market-today-coronavirus?action=click\&pgtype=Article\&state=default\&region=TOP_BANNER\&context=storylines_menu}{Economy}
\end{itemize}

Advertisement

\protect\hyperlink{after-top}{Continue reading the main story}

\hypertarget{where-the-small-business-relief-loans-have-gone}{%
\section{Where the Small-Business Relief Loans Have
Gone}\label{where-the-small-business-relief-loans-have-gone}}

By \href{https://www.nytimes.com/by/karl-russell}{Karl Russell} and
\href{https://www.nytimes.com/by/stacy-cowley}{Stacy Cowley}May 7, 2020

\begin{itemize}
\item
\item
\item
\item
\end{itemize}

Total value of loans per small-

business employee in each state

\$2.7

\$6.0

\$7.0

\$7.5

\$8.1

Thousands

of dollars

N.D.

\$8.1

Minn.

\$7.8

Mass.

\$7.7

Neb.

\$7.8

Ill.

\$7.7

W.Va.

\$2.7

Colo.

\$8.0

Kan.

\$7.5

Calif.

\$7.6

N.C. \$6.0

Ark.

\$5.7

N.M.

\$5.6

S.C.

\$5.8

Miss.

\$5.9

Total value of loans per small-business employee in each state

\$2.7

\$6.0

\$7.0

\$7.5

\$8.1

\$7.2

Thousands of dollars

\$7.0

N.D.

\$8.1

\$6.4

Minn.

\$7.8

\$6.5

\$6.4

\$6.7

\$7.1

Mass.

\$7.7

\$7.1

\$7.1

\$6.8

6.9

\$7.3

\$6.9

\$6.2

\$6.1

\$7.0

Neb.

\$7.8

\$7.4

\$6.5

\$6.9

\$6.1

Ill.

\$7.7

\$5.1

\$7.5

\$6.6

W.Va.

\$2.7

Colo.

\$8.0

Calif.

\$7.6

Kan.

\$7.5

\$6.9

\$7.0

\$6.3

N.C. \$6.0

\$6.5

\$6.9

Ark.

\$5.7

\$7.0

S.C.

\$5.8

N.M.

\$5.6

\$6.6

\$7.0

Miss.

\$5.9

\$7.3

\$6.6

\$7.3

\$6.7

\$7.4

Total value of loans per small-business employee in each state

\$7,230

\$2,700

\$6,000

\$7,000

\$7,500

\$8,100

\$7,010

N.D.

\$8,120

\$6,400

Minn.

\$7,830

\$6,490

\$6,390

\$6,710

\$7,070

Mass.

\$7,120

\$7,150

\$7,670

\$6,820

\$7,300

\$6,870

\$6,900

\$6,200

\$6,120

\$6,950

Neb.

\$7,770

\$7,390

\$6,550

\$6,880

\$6,110

Ill.

\$7,680

\$5,100

\$7,500

\$6,580

Colo.

\$7,960

W.Va.

\$2,710

Calif.

\$7,590

Kan.

\$7,520

\$6,900

\$6,960

\$6,310

N.C.

\$5,980

\$6,520

\$6,860

Ark.

\$5,680

\$6,950

S.C.

\$5,810

N.M.

\$5,560

\$6,640

\$6,950

Miss.

\$5,860

\$7,340

\$6,550

\$7,320

\$6,670

\$7,410

Source: Small Business Administration

A centerpiece of the federal government's economic relief plan is to
provide billions in forgivable loans to small businesses struggling
during the coronavirus pandemic.

But analyses of government data show that the lending program, which is
overseen by the Small Business Administration, allowed many of the
earliest funds to go to parts of the country that were not as hard hit
by the coronavirus, as well as to a small number of companies seeking
millions in assistance.

\hypertarget{midwestern-businesses-got-an-outsize-share-during-the-first-lending-round}{%
\subsection{Midwestern businesses got an outsize share during the first
lending
round.}\label{midwestern-businesses-got-an-outsize-share-during-the-first-lending-round}}

Total value of loans per small-

business employee in each state

1st round

2nd round

\$1.8

3.5

4.7

5.5

7.0

\$0.8

1.4

2.4

3.0

4.7

In thousands of dollars

Total value of loans per small-business employee in each state

1st round

2nd round

\$1,800

\$3,500

\$4,700

\$5,500

\$7,000

\$800

\$1,400

\$2,350

\$3,000

\$4,700

Wash.

N.D.

Ore.

N.Y.

Wisc.

Minn.

Conn.

Iowa

Neb.

N.J.

Nev.

Kan.

Calif.

Okla.

Ariz.

Fla.

Hawaii

Total value of loans per small-business employee in each state

1st round

2nd round

\$1,800

\$3,500

\$4,700

\$5,500

\$7,000

\$800

\$1,400

\$2,350

\$3,000

\$4,700

Wash.

Wash.

Me.

Minn.

Vt.

Mont.

N.D.

N.D.

Ore.

Ore.

N.Y.

Wisc.

Wis.

N.Y.

SD.

Conn.

N.J.

Iowa

Iowa

Neb.

Neb.

N.J.

Nev.

Del.

Nev.

Md.

W.Va.

W.Va.

Kan.

Kan.

Calif.

Calif.

N.C.

Okla.

Ariz.

Okla.

Ariz.

N.M.

S.C.

Ark.

Fla.

Hawaii

2nd round data through May 1·Source: Small Business Administration

The country's largest banks are often heavy lenders to small businesses,
but during the first of the program's two rounds, community banks and
regional institutions did most of the lending, according to an
\href{https://www.nber.org/papers/w27095.pdf}{analysis by a group of
University of Chicago and M.I.T. economists}.

That contributed to a disproportionately large share of loans going to
areas that were not as hard-hit by the virus.

Businesses in Iowa, Nebraska and North Dakota were among the biggest
beneficiaries of the early aid when accounting for the number of people
working for small businesses in each state, a Times analysis shows. All
three states are below the
\href{https://www.nytimes.com/interactive/2020/us/coronavirus-us-cases.html}{national
median}\href{https://www.nytimes.com/interactive/2020/us/coronavirus-us-cases.html}{for
cases of the virus per capita}, and
\href{https://www.nytimes.com/interactive/2020/us/states-reopen-map-coronavirus.html}{none
imposed statewide lockdowns} as the outbreak began to spread nationwide.

One reason for the uneven distribution is because big banks were
\href{https://www.nytimes.com/2020/04/02/business/small-business-coronavirus-stimulus.html}{slow
to lend} when the program first began, in part because of bureaucratic
delays, and they imposed rules that blocked many people seeking help.
The 20 largest banks accounted for 41 percent of small-business lending
throughout the country before the pandemic, but issued only 20 percent
of the first-round loans, the Chicago and M.I.T. economists found.

Since then, big banks made vastly more loans. Businesses in harder-hit
states like California and New York have claimed a larger share of the
money so far in the second round, which started in late April after
Congress approved a fresh round of funds when money quickly ran out
during the initial wave.

\hypertarget{many-small-loans-were-recently-issued-but-loans-worth-more-than-1-million-made-up-a-large-share-of-the-early-money-handed-out}{%
\subsection{Many small loans were recently issued, but loans worth more
than \$1 million made up a large share of the early money handed
out.}\label{many-small-loans-were-recently-issued-but-loans-worth-more-than-1-million-made-up-a-large-share-of-the-early-money-handed-out}}

1st round

Size of loans

70\%

Share of loans in the round

UP TO

\$150,000

15\%

Number of loans

15\%

\$150,000

TO \$350,000

Value of total loans

14\%

10\%

\$350,000

TO \$1 MILLION

24\%

5\%

OVER

\$1 MILLION

47\%

2nd round

90\%

UP TO

\$150,000

37\%

6\%

\$150,000

TO \$350,000

17\%

3\%

\$350,000

TO \$1 MILLION

19\%

1\%

OVER

\$1 MILLION

27\%

1st round

Size of loans

70\%

Share of loans in the round

UP TO

\$150,000

15\%

15\%

\$150,000

TO \$350,000

Number of loans

14\%

Value of total loans

10\%

\$350,000

TO \$1 MILLION

24\%

5\%

OVER

\$1 MILLION

47\%

2nd round

90\%

UP TO

\$150,000

37\%

6\%

\$150,000

TO \$350,000

17\%

3\%

\$350,000

TO \$1 MILLION

19\%

1\%

OVER

\$1 MILLION

27\%

2nd round data through May 1·Source: Small Business Administration

Loans of more than \$1 million made up just 5 percent of those approved
in the first round, but they accounted for roughly half of the overall
money. That favored large companies seeking greater sums.

Many of those companies already had deep relationships with their banks,
which helped them get to the front of the line, and some banks
\href{https://www.nytimes.com/2020/04/22/business/sba-loans-ppp-coronavirus.html}{prioritized
their most lucrative customers}. More than
\href{https://www.nytimes.com/2020/05/04/business/live-stock-market-coronavirus.html\#link-48f11fd9}{300
public companies disclosed receiving the loans}, though many have since
returned the money after a growing backlash.

Scrutiny over the program prompted officials to
\href{https://www.nytimes.com/2020/04/28/us/politics/coronavirus-treasury-payment-protection-program.html}{impose
new eligibility rules} for companies. Lending has since shifted toward
smaller amounts, suggesting smaller businesses were benefiting from the
assistance. In the first round, the average loan size was \$206,000,
according to the S.B.A. After one week of the second round, it was
\$79,000.

\hypertarget{small-businesses-employ-roughly-half-of-the-countrys-nongovernment-workers-and-most-have-under-100-employees}{%
\subsection{Small businesses employ roughly half of the country's
nongovernment workers, and most have under 100
employees.}\label{small-businesses-employ-roughly-half-of-the-countrys-nongovernment-workers-and-most-have-under-100-employees}}

Share of U.S. employment by company size

NUMBER

OF EMPLOYEES:

1-4

5\%

5-9

5\%

10 - 19

7\%

Small

businesses

52\%

20 - 49

10\%

50 - 99

8\%

100 - 249

10\%

250 - 499

7\%

500 - 999

7\%

Large

businesses

48\%

1,000

or more

41\%

Share of U.S. employment by company size

NUMBER OF EMPLOYEES:

100-

249

250-

499

500-

999

1-4

5-9

10-19

20-49

50-99

1,000 or more

7\%

41\%

5\%

5\%

7\%

10\%

8\%

10\%

7\%

Large businesses

Small businesses

48\%

52\%

Share of U.S. employment by company size

NUMBER OF EMPLOYEES:

1-4

5-9

10-19

20-49

50-99

100-249

250-499

500-999

1,000 or more

5\%

5\%

7\%

10\%

8\%

10\%

7\%

7\%

41\%

Large businesses

Small businesses

48\%

52\%

Figures are as of the first quarter of 2019 and are not seasonally
adjusted.·Source: Bureau of Labor Statistics

The government generally defines a small business as one that has up to
500 employees. In 2016, the last year for which the government has
released
\href{https://cdn.advocacy.sba.gov/wp-content/uploads/2019/04/23142719/2019-Small-Business-Profiles-US.pdf}{annual
estimates}, there were nearly 31 million small businesses operating in
the United States, employing almost 60 million workers.

Employees of small businesses,

in thousands

161

199

4,106

131

2,498

188

7,129

4,707

3,398

141

Employees of small businesses, in thousands

1,380

289

199

245

161

1,255

296

853

4,106

1,490

1,259

316

211

1,869

230

131

750

2,498

1,810

652

412

2,193

2,486

1,143

188

487

1,222

1,117

573

1,535

559

1,169

7,129

604

702

1,672

1,096

1,039

713

490

335

795

1,638

789

440

4,707

903

3,398

141

275

Employees of small businesses

1,379,600

289,200

245,400

199,000

161,100

1,255,000

295,900

853,000

4,106,100

1,258,900

315,800

1,490,400

210,500

1,868,900

230,000

131,500

750,500

2,497,500

1,810,100

651,600

412,300

487,400

2,192,900

2,485,900

187,600

1,143,200

1,222,400

1,117,200

572,900

7,129,200

1,535,100

558,900

1,168,700

604,200

702,100

1,672,100

1,096,200

1,039,200

712,600

334,900

490,500

794,700

440,400

1,637,900

789,400

4,706,900

902,800

3,397,900

141,100

275,100

Figures rounded to the nearest 100·Source: Small Business Administration

The stimulus program was created to help businesses pay their workers
for an eight-week period in April, May or June. Applicants did not have
to prove a sharp drop in sales or other specific harm. They simply had
to certify that ``current economic uncertainty makes this loan request
necessary'' to support their operations.

But many of the most devastated businesses --- like restaurants and
service providers --- have already laid off workers and are uncertain
when their sales will return. Because the program's rules require
companies to maintain their head count at pre-pandemic levels (those
that cut positions have a brief window to rehire) if they want their
loans forgiven, those businesses have a much harder time taking
advantage of the program than companies that still have their full work
force intact.

In effect, the Chicago and M.I.T. economists argued, the early stages of
the program ``functioned less as social insurance to support the hardest
hit areas'' and more as a cash infusion ``for less affected firms.''

\begin{itemize}
\item
\item
\item
\item
\end{itemize}

Advertisement

\protect\hyperlink{after-bottom}{Continue reading the main story}

\hypertarget{site-index}{%
\subsection{Site Index}\label{site-index}}

\hypertarget{site-information-navigation}{%
\subsection{Site Information
Navigation}\label{site-information-navigation}}

\begin{itemize}
\tightlist
\item
  \href{https://help.nytimes.com/hc/en-us/articles/115014792127-Copyright-notice}{©~2020~The
  New York Times Company}
\end{itemize}

\begin{itemize}
\tightlist
\item
  \href{https://www.nytco.com/}{NYTCo}
\item
  \href{https://help.nytimes.com/hc/en-us/articles/115015385887-Contact-Us}{Contact
  Us}
\item
  \href{https://www.nytco.com/careers/}{Work with us}
\item
  \href{https://nytmediakit.com/}{Advertise}
\item
  \href{http://www.tbrandstudio.com/}{T Brand Studio}
\item
  \href{https://www.nytimes.com/privacy/cookie-policy\#how-do-i-manage-trackers}{Your
  Ad Choices}
\item
  \href{https://www.nytimes.com/privacy}{Privacy}
\item
  \href{https://help.nytimes.com/hc/en-us/articles/115014893428-Terms-of-service}{Terms
  of Service}
\item
  \href{https://help.nytimes.com/hc/en-us/articles/115014893968-Terms-of-sale}{Terms
  of Sale}
\item
  \href{https://spiderbites.nytimes.com}{Site Map}
\item
  \href{https://help.nytimes.com/hc/en-us}{Help}
\item
  \href{https://www.nytimes.com/subscription?campaignId=37WXW}{Subscriptions}
\end{itemize}
