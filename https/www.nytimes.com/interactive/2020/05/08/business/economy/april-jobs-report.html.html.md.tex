Sections

SEARCH

\protect\hyperlink{site-content}{Skip to
content}\protect\hyperlink{site-index}{Skip to site index}

\href{https://www.nytimes.com/section/business/economy}{Economy}

\href{https://myaccount.nytimes.com/auth/login?response_type=cookie\&client_id=vi}{}

\href{https://www.nytimes.com/section/todayspaper}{Today's Paper}

\href{/section/business/economy}{Economy}\textbar{}How Bad Is
Unemployment? `Literally Off the Charts'

\url{https://nyti.ms/2WJhtQS}

\begin{itemize}
\item
\item
\item
\item
\item
\item
\end{itemize}

\href{https://www.nytimes.com/news-event/coronavirus?action=click\&pgtype=Article\&state=default\&region=TOP_BANNER\&context=storylines_menu}{The
Coronavirus Outbreak}

\begin{itemize}
\tightlist
\item
  live\href{https://www.nytimes.com/2020/08/04/world/coronavirus-cases.html?action=click\&pgtype=Article\&state=default\&region=TOP_BANNER\&context=storylines_menu}{Latest
  Updates}
\item
  \href{https://www.nytimes.com/interactive/2020/us/coronavirus-us-cases.html?action=click\&pgtype=Article\&state=default\&region=TOP_BANNER\&context=storylines_menu}{Maps
  and Cases}
\item
  \href{https://www.nytimes.com/interactive/2020/science/coronavirus-vaccine-tracker.html?action=click\&pgtype=Article\&state=default\&region=TOP_BANNER\&context=storylines_menu}{Vaccine
  Tracker}
\item
  \href{https://www.nytimes.com/2020/08/02/us/covid-college-reopening.html?action=click\&pgtype=Article\&state=default\&region=TOP_BANNER\&context=storylines_menu}{College
  Reopening}
\item
  \href{https://www.nytimes.com/live/2020/08/04/business/stock-market-today-coronavirus?action=click\&pgtype=Article\&state=default\&region=TOP_BANNER\&context=storylines_menu}{Economy}
\end{itemize}

Advertisement

\protect\hyperlink{after-top}{Continue reading the main story}

\hypertarget{comments}{%
\subsection{\texorpdfstring{\protect\hyperlink{commentsContainer}{Comments}}{Comments}}\label{comments}}

\href{}{How Bad Is Unemployment? `Literally Off the Charts'}\href{}{Skip
to Comments}

The comments section is closed. To submit a letter to the editor for
publication, write to
\href{mailto:letters@nytimes.com}{\nolinkurl{letters@nytimes.com}}.

\hypertarget{how-bad-is-unemployment-literally-off-the-charts}{%
\section{How Bad Is Unemployment? `Literally Off the
Charts'}\label{how-bad-is-unemployment-literally-off-the-charts}}

By \href{https://www.nytimes.com/by/nelson-d-schwartz}{Nelson D.
Schwartz}, \href{https://www.nytimes.com/by/ben-casselman}{Ben
Casselman} and Ella KoezeMay 8, 2020

\begin{itemize}
\item
\item
\item
\item
\item
  \emph{411}
\end{itemize}

The American economy plunged deeper into crisis last month, losing 20.5
million jobs as the unemployment rate jumped to 14.7 percent, the worst
devastation since the Great Depression.

The Labor Department's monthly report on Friday provided the clearest
picture yet of the breadth and depth of the economic damage --- and how
swiftly it spread --- as the coronavirus pandemic swept the country.

Job losses have encompassed the entire economy, affecting every major
industry. Areas like leisure and hospitality had the biggest losses in
April, but even health care shed more than a million jobs. Low-wage
workers, including many women and members of racial and ethnic
minorities, have been hit especially hard.

``It's literally off the charts,'' said Michelle Meyer, head of U.S.
economics at Bank of America. ``What would typically take months or
quarters to play out in a recession happened in a matter of weeks this
time.''

\hypertarget{unemployment-rate}{%
\subsubsection{Unemployment rate}\label{unemployment-rate}}

Source: Department of Labor

From almost any vantage point, it was a bleak report. The share of the
adult population with a job, at 51.3 percent, was the lowest on record.
Nearly 11 million people reported working part time because they
couldn't find full-time work, up from about four million before the
pandemic.

If anything, the numbers probably understate the economic distress.

Millions more Americans have filed unemployment claims since the data
was collected in mid-April. What's more, because of issues with the way
workers are classified, the Labor Department said the actual
unemployment rate last month might have been closer to 20 percent.

It remains possible that the recovery, too, will be swift, and that as
the pandemic retreats, businesses that were fundamentally healthy before
the virus will reopen, rehire and return more or less to normal. The one
bright spot in Friday's report was that nearly 80 percent of the
unemployed said they had been temporarily laid off and expected to
return to their jobs in the coming months.

\hypertarget{share-of-unemployed-on-temporary-layoffs}{%
\subsubsection{Share of unemployed on temporary
layoffs}\label{share-of-unemployed-on-temporary-layoffs}}

Source: Department of Labor

President Trump endorsed this view in an interview Friday morning on Fox
News. ``Those jobs will all be back, and they'll be back very soon,''
Mr. Trump said, ``and next year we're going to have a phenomenal year.''

But Diane Swonk, chief economist at Grant Thornton, said that such
optimism was misplaced, and that many of the jobs could not be
recovered.

``This is going to be a hard reality,'' Ms. Swonk said. ``These
furloughs are permanent, not temporary.''

Many businesses have indicated that employees can work from home
throughout the summer, hurting sales at downtown restaurants. Meetings
and conferences have been put off as well, reducing demand at hotels and
other gathering places. And the longer the pandemic lasts, the more
businesses will fail, deepening the downturn.

The broad nature of the job cuts, too, means it will take longer for the
labor market to recover than if the losses were confined to one or two
areas.

``There is no safe place in the labor market right now,'' said Martha
Gimbel, an economist and labor market expert at Schmidt Futures, a
philanthropic initiative. ``Once people are unemployed, once they've
lost their jobs, once their spending has been sucked out of the economy,
it takes so long to come back from that.''

\hypertarget{share-of-the-population-that-is-employed}{%
\subsubsection{Share of the population that is
employed}\label{share-of-the-population-that-is-employed}}

Source: Department of Labor

Carrie Hines, a managing director at an advertising firm in Austin,
Texas, had the kind of professional job --- adaptable to working from
home --- that seemed insulated from the pandemic's effects. But her firm
worked closely with companies in the airline, hotel and amusement park
industries. When their business evaporated as a result of the outbreak,
it was only a matter of time before Ms. Hines's firm felt the impact.
She was laid off April 20.

``I was shocked,'' she said. ``I've never had a gap in work since
college.''

Ms. Hines and her husband are cutting back where they can, and they have
canceled plans to send their three children to summer camp. ``I never
imagined this kind of job market where the entire advertising industry
has been crushed,'' she said.

The scale of the job losses last month alone far exceed the 8.7 million
lost in the last recession, when unemployment peaked at 10 percent in
October 2009.

``I thought the Great Recession was once in a lifetime, but this is much
worse,'' said Beth Ann Bovino, chief U.S. economist at S\&P Global.

The only comparable period is when unemployment reached about 25 percent
in 1933, before the government began publishing official statistics.
Then as now, workers from a variety of backgrounds found themselves with
few prospects for quickly landing a new job.

The government's official definition of unemployment typically requires
people to be actively looking for work, making the measure ill suited to
a crisis in which the government is encouraging people to stay home.
Some 6.4 million people left the labor force entirely in April, meaning
they were neither working nor looking for work.

Joblessness --- by any measure --- could be even higher in the report
for May, which will reflect conditions next week. Some economists say
the unemployment rate should fall over the summer as people begin to
return to work. Several states have begun to reopen their economies, and
others are expected to do so in coming weeks.

\hypertarget{monthly-change-in-jobs-by-industry}{%
\subsubsection{Monthly change in jobs by
industry}\label{monthly-change-in-jobs-by-industry}}

But with the virus untamed, it's not clear how quickly customers will
return to businesses. And epidemiologists and economists warn that if
states move too quickly, they could risk a second wave of infections,
imperiling public health and the economy.

``That would stop people from shopping and cause austerity,'' Ms. Bovino
said.

For businesses, the uncertainty about the path of the pandemic and about
consumers' response to it is making planning difficult.

When Austin Ramirez heard about the new coronavirus earlier this year,
his initial concern was for his supply chain. Mr. Ramirez runs Husco
International, a manufacturer of hydraulic and electromechanical
components for cars and other equipment. The company has a factory in
China and receives parts from suppliers there and around the world.

By April, virtually the entire U.S. auto industry was shut down, Husco
included. (The company's nonautomotive production continued at a reduced
rate.) Mr. Ramirez said he didn't know when business would bounce back.
His goal is to weather the storm.

``There's no visibility or certainty on what the future demand is going
to look like,'' he said. ``We can't build a business model that relies
on there being a big recovery six months from now.''

While most of Husco's roughly 750 North American workers have been
furloughed during the crisis, the company has mostly avoided
large-scale, permanent job cuts. Mr. Ramirez said he expected that most
of his workers would come back when he needs them.

But particularly in industries like retail and hospitality, layoffs that
were initially temporary might not remain so as bankruptcies mount and
business owners confront shifts in consumer behavior.

Most forecasters expect the unemployment rate to remain elevated at
least through 2021, and probably longer. That means that it will be
years before workers enjoy the bargaining power that was beginning to
bring them faster wage gains and better benefits before the crisis.

``Job seekers are going to have less leverage,'' said Julia Pollak, a
labor economist at the employment marketplace ZipRecruiter. ``We're no
longer going to see mostly employed job seekers browsing and looking for
better matches and higher pay. You're going to see job seekers desperate
to pay the bills.''

\href{https://bfi.uchicago.edu/wp-content/uploads/BFI_WP_202058-1.pdf}{Research
this week} from economists at the ADP Research Institute, the University
of Chicago and the Federal Reserve found that low-wage workers have
suffered a disproportionate share of job losses in the crisis so far.

``Recessions always tend to affect employment for low-wage, low-skilled
workers, but the magnitude of the difference between low-wage and
high-wage workers, that's remarkable,'' said Ahu Yildirmaz, an economist
at the ADP Research Institute and an author of the study.

Ibelis Gonzalez worked as a server for Ruth's Chris Steak House in
Jersey City, N.J., until she was let go in March. She is hoping her job
will return when the chain reopens, but she knows there are no
guarantees, as patrons may be hesitant to dine out at first.

``We don't know if they will have a skeleton staff,'' said Ms. Gonzalez,
who earned \$600 to \$800 a week, nearly all of it from tips. ``People
may not have the money to go out and have a \$100 steak.''

She has been trying to file for unemployment insurance but hasn't been
able to reach the state's Department of Labor and Workforce Development.
``I'm not looking for a handout; I'm just looking for these benefits,''
she said. ``I don't have a dollar to my name.''

Michael Crowley contributed reporting.

Read 411 Comments

\begin{itemize}
\item
\item
\item
\item
\end{itemize}

Advertisement

\protect\hyperlink{after-bottom}{Continue reading the main story}

\hypertarget{site-index}{%
\subsection{Site Index}\label{site-index}}

\hypertarget{site-information-navigation}{%
\subsection{Site Information
Navigation}\label{site-information-navigation}}

\begin{itemize}
\tightlist
\item
  \href{https://help.nytimes.com/hc/en-us/articles/115014792127-Copyright-notice}{©~2020~The
  New York Times Company}
\end{itemize}

\begin{itemize}
\tightlist
\item
  \href{https://www.nytco.com/}{NYTCo}
\item
  \href{https://help.nytimes.com/hc/en-us/articles/115015385887-Contact-Us}{Contact
  Us}
\item
  \href{https://www.nytco.com/careers/}{Work with us}
\item
  \href{https://nytmediakit.com/}{Advertise}
\item
  \href{http://www.tbrandstudio.com/}{T Brand Studio}
\item
  \href{https://www.nytimes.com/privacy/cookie-policy\#how-do-i-manage-trackers}{Your
  Ad Choices}
\item
  \href{https://www.nytimes.com/privacy}{Privacy}
\item
  \href{https://help.nytimes.com/hc/en-us/articles/115014893428-Terms-of-service}{Terms
  of Service}
\item
  \href{https://help.nytimes.com/hc/en-us/articles/115014893968-Terms-of-sale}{Terms
  of Sale}
\item
  \href{https://spiderbites.nytimes.com}{Site Map}
\item
  \href{https://help.nytimes.com/hc/en-us}{Help}
\item
  \href{https://www.nytimes.com/subscription?campaignId=37WXW}{Subscriptions}
\end{itemize}
