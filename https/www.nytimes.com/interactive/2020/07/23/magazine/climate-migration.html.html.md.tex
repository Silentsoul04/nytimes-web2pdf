Sections

SEARCH

\protect\hyperlink{site-content}{Skip to
content}\protect\hyperlink{site-index}{Skip to site index}

\hypertarget{comments}{%
\subsection{\texorpdfstring{\protect\hyperlink{commentsContainer}{Comments}}{Comments}}\label{comments}}

\href{}{The Great Climate Migration Has Begun}\href{}{Skip to Comments}

The comments section is closed. To submit a letter to the editor for
publication, write to
\href{mailto:letters@nytimes.com}{\nolinkurl{letters@nytimes.com}}.

\hypertarget{the-great-climate-migration-has-begun}{%
\section{The Great Climate Migration Has
Begun}\label{the-great-climate-migration-has-begun}}

By Abrahm LustgartenJuly 23, 2020

\begin{itemize}
\item
\item
\item
\item
\item
  \emph{571}
\end{itemize}

\includegraphics{https://static01.nyt.com/newsgraphics/2020/07/14/mag-climate-migration/assets/images/globe1-720_x2.png}

\includegraphics{https://static01.nyt.com/newsgraphics/2020/07/14/mag-climate-migration/assets/images/globe2-720_x2.png}

\includegraphics{https://static01.nyt.com/newsgraphics/2020/07/14/mag-climate-migration/assets/images/globe3-720_x2.png}

\begin{itemize}
\item
\item
\item
\item
\item
\end{itemize}

Today, 1\% of the world is a barely livable hot zone.

By 2070, that portion could go up to 19\%.

Billions of people call this land home.

Where will they go?

The Great Climate Migration By Abrahm Lustgarten \textbar{} Photographs
by Meridith Kohut

\includegraphics{https://static01.nyt.com/packages/flash/multimedia/ICONS/transparent.png}

\includegraphics{https://static01.nyt.com/images/2020/07/26/magazine/26mag-Migration-Images/26mag-Migratiion-Images-master1050.jpg}

ALTA VERAPAZ, GUATEMALA. Carlos Tiul, an Indigenous farmer whose maize
crop has failed, with his children.

\textbf{Early in 2019,} a year before the world shut its borders
completely, Jorge A. knew he had to get out of Guatemala. The land was
turning against him. For five years, it almost never rained. Then it did
rain, and Jorge rushed his last seeds into the ground. The corn sprouted
into healthy green stalks, and there was hope --- until, without
warning, the river flooded. Jorge waded chest-deep into his fields
searching in vain for cobs he could still eat. Soon he made a last
desperate bet, signing away the tin-roof hut where he lived with his
wife and three children against a \$1,500 advance in okra seed. But
after the flood, the rain stopped again, and everything died. Jorge knew
then that if he didn't get out of Guatemala, his family might die, too.

This article, the first in a series on global climate migration, is a
partnership between ProPublica and The New York Times Magazine, with
support from the Pulitzer Center.
\href{https://www.propublica.org/article/2020-climate-migration-part-1-methodology}{Read
more about the data project} that underlies the reporting.

Even as hundreds of thousands of Guatemalans fled north toward the
United States in recent years, in Jorge's region --- a state called Alta
Verapaz, where precipitous mountains covered in coffee plantations and
dense, dry forest give way to broader gentle valleys --- the residents
have largely stayed. Now, though, under a relentless confluence of
drought, flood, bankruptcy and starvation, they, too, have begun to
leave. Almost everyone here experiences some degree of uncertainty about
where their next meal will come from. Half the children are chronically
hungry, and many are short for their age, with weak bones and bloated
bellies. Their families are all facing the same excruciating decision
that confronted Jorge.

\includegraphics{https://static01.nyt.com/packages/flash/multimedia/ICONS/transparent.png}

\includegraphics{https://static01.nyt.com/images/2020/07/26/magazine/26mag-Migration-Images-02/26mag-Migratiion-Images-02-master1050.jpg}

ALTA VERAPAZ. An ear of maize from a failed crop.

The odd weather phenomenon that many blame for the suffering here ---
the drought and sudden storm pattern known as El Niño --- is expected to
become more frequent as the planet warms. Many semiarid parts of
Guatemala will soon be more like a desert. Rainfall is expected to
decrease by 60 percent in some parts of the country, and the amount of
water replenishing streams and keeping soil moist will drop by as much
as 83 percent. Researchers project that by 2070, yields of some staple
crops in the state where Jorge lives will decline by nearly a third.

Scientists have learned to project such changes around the world with
surprising precision, but --- until recently --- little has been known
about the human consequences of those changes. As their land fails them,
hundreds of millions of people from Central America to Sudan to the
Mekong Delta will be forced to choose between flight or death. The
result will almost certainly be the greatest wave of global migration
the world has seen.

In March, Jorge and his 7-year-old son each packed a pair of pants,
three T-shirts, underwear and a toothbrush into a single thin black
nylon sack with a drawstring. Jorge's father had pawned his last four
goats for \$2,000 to help pay for their transit, another loan the family
would have to repay at 100 percent interest. The coyote called at 10
p.m. --- they would go that night. They had no idea then where they
would wind up, or what they would do when they got there.

From decision to departure, it was three days. And then they were gone.

\includegraphics{https://static01.nyt.com/packages/flash/multimedia/ICONS/transparent.png}

\includegraphics{https://static01.nyt.com/images/2020/07/26/magazine/26mag-Migration-Images-04/26mag-Migratiion-Images-04-master1050.jpg}

ALTA VERAPAZ. Jorge A.'s wife, Eva María H., at home with two of their
children.

\textbf{For most of human} history, people have lived within a
surprisingly narrow range of temperatures, in the places where the
climate supported abundant food production. But as the planet warms,
that band is suddenly shifting north. According to
\href{https://www.pnas.org/content/117/21/11350}{a pathbreaking recent
study in the journal Proceedings of the National Academy of Sciences,}
the planet could see a greater temperature increase in the next 50 years
than it did in the last 6,000 years combined. By 2070, the kind of
extremely hot zones, like in the Sahara, that now cover less than 1
percent of the earth's land surface could cover nearly a fifth of the
land, potentially placing one of every three people alive outside the
climate niche where humans have thrived for thousands of years. Many
will dig in, suffering through heat, hunger and political chaos, but
others will be forced to move on.
\href{https://advances.sciencemag.org/content/3/8/e1603322}{A 2017 study
in Science Advances} found that by 2100, temperatures could rise to the
point that just going outside for a few hours in some places, including
parts of India and Eastern China, ``will result in death even for the
fittest of humans.''

People are already beginning to flee. In Southeast Asia, where
increasingly unpredictable monsoon rainfall and drought have made
farming more difficult,
\href{https://www.worldbank.org/en/news/infographic/2018/03/19/groundswell---preparing-for-internal-climate-migration}{the
World Bank points to} more than eight million people who have moved
toward the Middle East, Europe and North America. In the African Sahel,
millions of rural people have been streaming toward the coasts and the
cities amid drought and widespread crop failures. Should the flight away
from hot climates reach the scale that current research suggests is
likely, it will amount to a vast remapping of the world's populations.

\hypertarget{listen-to-this-article}{%
\subsubsection{Listen to This Article}\label{listen-to-this-article}}

To hear more audio stories from publishers like The New York Times,
\href{https://www.audm.com/}{download Audm for iPhone or Android}.

Migration can bring great opportunity not just to migrants but also to
the places they go. As the United States and other parts of the global
North face a demographic decline, for instance, an injection of new
people into an aging work force could be to everyone's benefit. But
securing these benefits starts with a choice: Northern nations can
relieve pressures on the fastest-warming countries by allowing more
migrants to move north across their borders, or they can seal themselves
off, trapping hundreds of millions of people in places that are
increasingly unlivable. The best outcome requires not only good will and
the careful management of turbulent political forces; without
preparation and planning, the sweeping scale of change could prove
wildly destabilizing. The United Nations and others warn that in the
worst case, the governments of the nations most affected by climate
change could topple as whole regions devolve into war.

The stark policy choices are already becoming apparent. As refugees
stream out of the Middle East and North Africa into Europe and from
Central America into the United States, an anti-immigrant backlash has
propelled nationalist governments into power around the world. The
alternative, driven by a better understanding of how and when people
will move, is governments that are actively preparing, both materially
and politically, for the greater changes to come.

Projected percentage decrease by 2070 in the yield of the rice crop in
Alta Verapaz, Guatemala:

\hypertarget{32}{%
\subparagraph{32}\label{32}}

Last summer, I went to Central America to learn how people like Jorge
will respond to changes in their climates. I followed the decisions of
people in rural Guatemala and their routes to the region's biggest
cities, then north through Mexico to Texas. I found an astonishing need
for food and witnessed the ways competition and poverty among the
displaced broke down cultural and moral boundaries. But the picture on
the ground is scattered. To better understand the forces and scale of
climate migration over a broader area, The New York Times Magazine and
ProPublica joined with the Pulitzer Center in an effort to model, for
the first time, how people will move across borders.

We focused on changes in Central America and used climate and
economic-development data to examine a range of scenarios. Our model
projects that migration will rise every year regardless of climate, but
that the amount of migration increases substantially as the climate
changes. In the most extreme climate scenarios, more than 30 million
migrants would head toward the U.S. border over the course of the next
30 years.

Migrants move for many reasons, of course. The model helps us see which
migrants are driven primarily by climate, finding that they would make
up as much as 5 percent of the total. If governments take modest action
to reduce climate emissions, about 680,000 climate migrants might move
from Central America and Mexico to the United States between now and
2050. If emissions continue unabated, leading to more extreme warming,
that number jumps to more than a million people. (None of these figures
include undocumented immigrants, whose numbers could be twice as high.)

The model shows that the political responses to both climate change and
migration can lead to drastically different futures.

\includegraphics{https://static01.nyt.com/packages/flash/multimedia/ICONS/transparent.png}

\includegraphics{https://static01.nyt.com/packages/flash/multimedia/ICONS/transparent.png}

\includegraphics{https://static01.nyt.com/packages/flash/multimedia/ICONS/transparent.png}

\includegraphics{https://static01.nyt.com/packages/flash/multimedia/ICONS/transparent.png}

\includegraphics{https://static01.nyt.com/packages/flash/multimedia/ICONS/transparent.png}

\includegraphics{https://static01.nyt.com/packages/flash/multimedia/ICONS/transparent.png}

\includegraphics{https://static01.nyt.com/packages/flash/multimedia/ICONS/transparent.png}

\includegraphics{https://static01.nyt.com/packages/flash/multimedia/ICONS/transparent.png}

United States

Mexico City

Guatemala City

San Salvador

Colombia

Fewer People

More People

More open borders, high development

More closed borders, low development

\begin{itemize}
\item
\item
\item
\item
\item
\item
\item
\item
\end{itemize}

2020

In one scenario, globalization --- with its relatively open borders ---
continues.

As the climate changes, drought and food insecurity drive rural
residents in Mexico and Central America out of the countryside.

Millions seek relief, first in big cities, spurring a rapid and
increasingly overwhelming urbanization.

Then they move farther north, pushing the largest number of migrants
toward the United States. The projected number of migrants arriving from
Central America and Mexico rises to 1.5 million a year by 2050, from
about 700,000 a year in 2025.

We modeled another scenario in which the United States hardens its
borders. People are turned back, and economic growth in Central America
slows, as does urbanization.

In this case, Central America's population surges, and the rural
hollowing reverses as the birthrate rises, poverty deepens and hunger
grows --- all with hotter weather and less water.

That version of the world leaves tens of millions of people more
desperate and with fewer options. Misery reigns, and large populations
become trapped.

As with much modeling work, the point here is not to provide concrete
numerical predictions so much as it is to provide glimpses into possible
futures. Human movement is notoriously hard to model, and as many
climate researchers have noted, it is important not to add a false
precision to the political battles that inevitably surround any
discussion of migration. But our model offers something far more
potentially valuable to policymakers: a detailed look at the staggering
human suffering that will be inflicted if countries shut their doors.

In recent months, the coronavirus pandemic has offered a test run on
whether humanity has the capacity to avert a predictable --- and
predicted --- catastrophe. Some countries have fared better. But the
United States has failed. The climate crisis will test the developed
world again, on a larger scale, with higher stakes. The only way to
mitigate the most destabilizing aspects of mass migration is to prepare
for it, and preparation demands a sharper imagining of where people are
likely to go, and when.

\hypertarget{i-a-different-kind-of-climate-model}{%
\subsection{I. A Different Kind of Climate
Model}\label{i-a-different-kind-of-climate-model}}

In November 2007, Alan B. Krueger, a labor economist known for his
statistical work on inequality, walked into the Princeton University
offices of Michael Oppenheimer, a leading climate geoscientist, and
asked him whether anyone had ever tried to quantify how and where
climate change would cause people to move.

Earlier that year, Oppenheimer helped write the U.N. Intergovernmental
Panel on Climate Change report that, for the first time, explored in
depth how climate disruption might uproot large segments of the global
population. But as groundbreaking as the report was --- the U.N. was
recognized for its work with a Nobel Peace Prize --- the academic
disciplines whose work it synthesized were largely siloed from one
another. Demographers, agronomists and economists were all doing their
work on climate change in isolation, but understanding the question of
migration would have to include all of them.

Together, Oppenheimer and Krueger, who died in 2019, began to chip away
at the question, asking whether tools typically used by economists might
yield insight into the environment's effects on people's decision to
migrate. They began to examine the statistical relationships --- say,
between census data and crop yields and historical weather patterns ---
in Mexico to try to understand how farmers there respond to drought. The
data helped them create a mathematical measure of farmers' sensitivity
to environmental change --- a factor that Krueger could use the same way
he might evaluate fiscal policies, but to model future migration.

Their study, \href{https://www.pnas.org/content/107/32/14257}{published
in 2010 in Proceedings of the National Academy of Sciences,} found that
Mexican migration to the United States pulsed upward during periods of
drought and projected that by 2080, climate change there could drive 6.7
million more people toward the Southern U.S. border. ``It was,''
Oppenheimer said, ``one of the first applications of econometric
modeling to the climate-migration problem.''

\includegraphics{https://static01.nyt.com/packages/flash/multimedia/ICONS/transparent.png}

\includegraphics{https://static01.nyt.com/images/2020/07/26/magazine/26mag-Migration-Images-03/26mag-Migratiion-Images-03-master1050.jpg}

TABASCO, MEXICO. Migrants from Central America riding north on the
Bestia freight rail line.

The modeling was a start. But it was hyperlocal instead of global, and
it left open huge questions: how cultural differences might change
outcomes, for example, or how population shifts might occur across
larger regions. It was also controversial, igniting
\href{http://rogerpielkejr.blogspot.com/2010/07/silly-science.html}{a
backlash among climate-change skeptics,} who attacked the modeling
effort as ``guesswork'' built on ``tenuous assumptions'' and argued that
a model couldn't untangle the effect of climate change from all the
other complex influences that determine human decision-making and
migration. That argument eventually found some traction with migration
researchers, many of whom remain reluctant to model precise migration
figures.

But to Oppenheimer and Krueger, the risks of putting a specific shape to
this well established but amorphous threat seemed worth taking. In the
early 1970s, after all, many researchers had made a similar argument
against using computer models to forecast climate change, arguing that
scientists shouldn't traffic in predictions. Others ignored that advice,
producing some of the earliest projections about the dire impact of
climate change, and with them some of the earliest opportunities to try
to steer away from that fate. Trying to project the consequences of
climate-driven migration, to Oppenheimer, called for similarly
provocative efforts. ``If others have better ideas for estimating how
climate change affects migration,'' he wrote in 2010, ``they should
publish them.''

Since then, Oppenheimer's approach has become common. Dozens more
studies have applied econometric modeling to climate-related problems,
seizing on troves of data to better understand how environmental change
and conflict each lead to migration and clarify how the cycle works.
Climate is rarely the main cause of migration, the studies have
generally found, but it is almost always an exacerbating one.

As they have looked more closely, migration researchers have found
climate's subtle fingerprints almost everywhere.
\href{https://www.pnas.org/content/early/2015/02/23/1421533112.abstract}{Drought
helped push} many Syrians into cities before the war, worsening tensions
and leading to rising discontent;
\href{https://www.americanprogress.org/issues/security/reports/2013/02/28/54579/the-arab-spring-and-climate-change/}{crop
losses led to unemployment that stoked Arab Spring uprisings in Egypt
and Libya}; Brexit, even, was arguably a ripple effect of the influx of
migrants brought to Europe by the wars that followed. And all those
effects were bound up with the movement of just two million people. As
the mechanisms of climate migration have come into sharper focus ---
food scarcity, water scarcity and heat --- the latent potential for
large-scale movement comes to seem astronomically larger.

\includegraphics{https://static01.nyt.com/packages/flash/multimedia/ICONS/transparent.png}

\includegraphics{https://static01.nyt.com/images/2020/07/26/magazine/26mag-Migration-Images-06/26mag-Migratiion-Images-06-master1050.jpg}

TABASCO. Bayron Coto (front) left his home in Honduras to support his
family after a hurricane destroyed local maize, bean and coffee crops.

North Africa's Sahel provides an example. In the nine countries
stretching across the continent from Mauritania to Sudan, extraordinary
population growth and steep environmental decline are on a collision
course. Past droughts, most likely caused by climate change, have
already killed more than 100,000 people there. And the region --- with
more than 150 million people and growing --- is threatened by rapid
desertification, even more severe water shortages and deforestation.
Today researchers at the United Nations estimate that some 65 percent of
farmable lands have already been degraded. ``My deep fear,'' said
Solomon Hsiang, a climate researcher and economist at the University of
California, Berkeley, is that Africa's transition into a
post-climate-change civilization ``leads to a constant outpouring of
people.''

The story is similar in South Asia, where nearly one-fourth of the
global population lives. The World Bank projects that the region will
soon have the highest prevalence of food insecurity in the world. While
some 8.5 million people have fled already --- resettling mostly in the
Persian Gulf --- 17 million to 36 million more people may soon be
uprooted, the World Bank found. If past patterns are a measure, many
will settle in India's Ganges Valley; by the end of the century, heat
waves and humidity will become so extreme there that people without
air-conditioning will simply die.

If it is not drought and crop failures that force large numbers of
people to flee, it will be the rising seas. We are now learning that
\href{https://www.nature.com/articles/s41467-019-12808-z}{climate
scientists have been underestimating} the future displacement from
rising tides by a factor of three, with the likely toll being some 150
million globally. New projections show high tides subsuming much of
Vietnam by 2050 --- including most of the Mekong Delta, now home to 18
million people --- as well as parts of China and Thailand, most of
southern Iraq and nearly all of the Nile Delta, Egypt's breadbasket.
Many coastal regions of the United States are also at risk.

Through all the research, rough predictions have emerged about the scale
of total global climate migration --- they range from 50 million to 300
million people displaced --- but the global data is limited, and
uncertainty remained about how to apply patterns of behavior to specific
people in specific places. Now, though, new research on both fronts has
created an opportunity to improve the models tremendously. A few years
ago, climate geographers from Columbia University and the City
University of New York began working with the World Bank to build a
next-generation tool to establish plausible migration scenarios for the
future. The idea was to build on the Oppenheimer-style measure of
response to the environment with other methods of analysis, including a
``gravity'' model, which assesses the relative attractiveness of
destinations with the hope of mathematically anticipating where migrants
might end up.
\href{https://www.worldbank.org/en/news/infographic/2018/03/19/groundswell---preparing-for-internal-climate-migration}{The
resulting report,} published in early 2018, involved six European and
American institutions and took nearly two years to complete.

The bank's work targeted climate hot spots in sub-Saharan Africa, South
Asia and Latin America, focusing not on the emergency displacement of
people from natural disasters but on their premeditated responses to
what researchers call ``slow-onset'' shifts in the environment. They
determined that as climate change progressed in just these three regions
alone, as many as 143 million people would be displaced within their own
borders, moving mostly from rural areas to nearby towns and cities. The
study, though, wasn't fine-tuned to specific climatic changes like
declining groundwater. And it didn't even try to address the elephant in
the room: How would the climate push people to migrate across
international borders?

\includegraphics{https://static01.nyt.com/packages/flash/multimedia/ICONS/transparent.png}

\includegraphics{https://static01.nyt.com/images/2020/07/26/magazine/26mag-Migration-Images-07/26mag-Migratiion-Images-07-master1050.jpg}

CHIAPAS, MEXICO. Coto (right) hopping a train with other migrants.

\textbf{In early 2019,} The Times Magazine and ProPublica, with support
from the Pulitzer Center, hired an author of the World Bank report ---
Bryan Jones, a geographer at Baruch College --- to add layers of
environmental data to its model, making it even more sensitive to
climatic change and expanding its reach. Our goal was to pick up where
the World Bank researchers left off, in order to model, for the first
time, how people would move between countries, especially from Central
America and Mexico toward the United States.

First we gathered existing data sets --- on political stability,
agricultural productivity, food stress, water availability, social
connections, weather and much more --- in order to approximate the
kaleidoscopic complexity of human decision-making.

Then we started asking questions: If crop yields continue to decline
because of drought, for instance, and people are forced to respond by
moving, as they have in the past, can we see where they will go and see
what new conditions that might introduce? It's very difficult to model
how individual people think or to answer these questions using
individual data points --- often the data simply doesn't exist. Instead
of guessing what Jorge A. will do and then multiplying that decision by
the number of people in similar circumstances, the model looks across
entire populations, averaging out trends in community decision-making
based on established patterns, then seeing how those trends play out in
different scenarios.

Projected percentage of city dwellers who will live in slums by 2030:

\hypertarget{40}{%
\subparagraph{40}\label{40}}

In all, we fed more than 10 billion data points into our model. Then we
tested the relationships in the model retroactively, checking where
historical cause and effect could be empirically supported, to see if
the model's projections about the past matches what really happened.
Once the model was built and layered with both approaches ---
econometric and gravity --- we looked at how people moved as global
carbon concentrations increased in five different scenarios, which
imagine various combinations of growth, trade and border control, among
other factors. (These scenarios have become standard among climate
scientists and economists in modeling different pathways of global
socioeconomic development.)

Only a supercomputer could efficiently process the work in its entirety;
estimating migration from Central America and Mexico in one case
required uploading our query to a federal mainframe housed in a building
the size of a small college campus outside Cheyenne, Wyo., run by the
National Center for Atmospheric Research, where even there it took four
days for the machine to calculate its answers. (A more detailed
description of the data project can be found at
\href{http://propublica.org/migration-methodology}{propublica.org/migration-methodology}.)

The results are built around a number of assumptions about the
relationships between real-world developments that haven't all been
scientifically validated. The model also assumes that complex
relationships --- say, how drought and political stability relate to
each other --- remain consistent and linear over time (when in reality
we know the relationships will change, but not how). Many people will
also be trapped by their circumstances, too poor or vulnerable to move,
and the models have a difficult time accounting for them.

All this means that our model is far from definitive. But every one of
the scenarios it produces points to a future in which climate change,
currently a subtle disrupting influence, becomes a source of major
disruption, increasingly driving the displacement of vast populations.

\includegraphics{https://static01.nyt.com/packages/flash/multimedia/ICONS/transparent.png}

\includegraphics{https://static01.nyt.com/images/2020/07/26/magazine/26mag-Migration2-images/26mag-Migration2-images-master1050.jpg}

GUATEMALA CITY. Crop failures are causing more rural residents to
migrate to urban areas.

\hypertarget{--ii-how-climate-moves-people}{%
\subsection{ **** II. How Climate Moves
People}\label{--ii-how-climate-moves-people}}

Delmira de Jesús Cortez Barrera moved to the outskirts of San Salvador
six years ago, after her life in the rural western edge of El Salvador
--- just 90 miles from Jorge A.'s village in Guatemala --- collapsed.
Now she sells \emph{pupusas} on a block not far from where teenagers
stand guard for the Mara Salvatrucha gang. When we met last summer, she
was working six days a week, earning \$7 a day, or less than \$200 a
month. She relied on the kindness of her boss, who gave her some free
meals at work. But everything else for her and her infant son she had to
provide herself. Cortez commuted before dawn from San Marcos, where she
lived with her sister in a cheap room off a pedestrian alleyway. But her
apartment still cost \$65 each month. And she sent \$75 home to her
parents each month --- enough for beans and cheese to feed the two
daughters she left with them. ``We're going backward,'' she said.

Her story --- that of an uneducated, unskilled woman from farm roots who
can't find high-paying work in the city and falls deeper into poverty
--- is a familiar one, the classic pattern of in-country migration all
around the world. San Salvador, meanwhile, has become notorious as one
of the most dangerous cities in the world, a capital in which gangs have
long controlled everything from the majestic colonial streets of its
downtown squares to the offices of the politicians who reside in them.
It is against this backdrop of war, violence, hurricanes and poverty
that one in six of El Salvador's citizens have fled for the United
States over the course of the last few decades, with some 90,000
Salvadorans apprehended at the U.S. border in 2019 alone.

Cortez was born about a mile from the Guatemalan border, in El Paste, a
small town nestled on the side of a volcano. Her family were
\emph{jornaleros} --- day laborers who farmed on the big maize and bean
plantations in the area --- and they rented a two-room mud-walled hut
with a dirt floor, raising nine children there. Around 2012, a coffee
blight worsened by climate change virtually wiped out El Salvador's
crop, slashing harvests by 70 percent. Then drought and unpredictable
storms led to what a U.N.-affiliated food-security organization
describes as ``a progressive deterioration'' of Salvadorans'
livelihoods.

That's when Cortez decided to leave. She married and found work as a
brick maker at a factory in the nearby city of Ahuachapán. But the gangs
found easy prey in vulnerable farmers and spread into the Salvadoran
countryside and the outlying cities, where they made a living by
extorting local shopkeepers. Here we can see how climate change can act
as what Defense Department officials sometimes refer to as a ``threat
multiplier.'' For Cortez, the threat could not have been more dire.
After two years in Ahuachapán, a gang-connected hit man knocked on
Cortez's door and took her husband, whose ex-girlfriend was a gang
member, executing him in broad daylight a block away.

In other times, Cortez might have gone back home. But there was no work
in El Paste, and no water. So she sent her children there and went to
San Salvador instead.

\includegraphics{https://static01.nyt.com/packages/flash/multimedia/ICONS/transparent.png}

\includegraphics{https://static01.nyt.com/images/2020/07/26/magazine/26mag-Migration-Images-05/26mag-Migratiion-Images-05-master1050.jpg}

SAN SALVADOR. Delmira de Jesús Cortez Barrera (left) and her sister
(center) moved to the area after their family's agriculture work dried
up.

\includegraphics{https://static01.nyt.com/packages/flash/multimedia/ICONS/transparent.png}

\includegraphics{https://static01.nyt.com/images/2020/07/26/magazine/26mag-Migration2-images-06/26mag-Migration2-images-06-master1050.jpg}

SAN SALVADOR. Delmira de Jesús Cortez Barrera moved to the area after
her family's agriculture work dried up.

For all the ways in which human migration is hard to predict, one trend
is clear: Around the world, as people run short of food and abandon
farms, they gravitate toward cities, which quickly grow overcrowded.
It's in these cities, where waves of new people stretch infrastructure,
resources and services to their limits, that migration researchers warn
that the most severe strains on society will unfold. Food has to be
imported --- stretching reliance on already-struggling farms and
increasing its cost. People will congregate in slums, with little water
or electricity, where they are more vulnerable to flooding or other
disasters. The slums fuel extremism and chaos.

It is a shift that is already well underway, which is why the World Bank
has raised concerns about the mind-boggling influx of people into East
African cities like Addis Ababa, in Ethiopia, where the population has
doubled since 2000 and is expected to nearly double again by 2035. In
Mexico, the World Bank estimates, as many as 1.7 million people may
migrate away from the hottest and driest regions, many of them winding
up in Mexico City.

But like so much of the rest of the climate story, the urbanization
trend is also just the beginning. Right now a little more than half of
the planet's population lives in urban areas, but by the middle of the
century, the World Bank estimates, 67 percent will. In just a decade,
four out of every 10 urban residents --- two billion people around the
world --- will live in slums. The International Committee of the Red
Cross warns that 96 percent of future urban growth will happen in some
of the world's most fragile cities, which already face a heightened risk
of conflict and have governments that are least capable of dealing with
it. Some cities will be unable to sustain the influx. In the case of
Addis Ababa, the World Bank suggests that in the second half of the
century, many of the people who fled there will be forced to move again,
leaving that city as local agriculture around it dries up.

Percentage of El Salvador's 6.4 million residents who currently lack a
reliable source of food:

\hypertarget{42}{%
\subparagraph{42}\label{42}}

Our modeling effort is premised on the notion that in these cities as
they exist now, we can see the seeds of their future growth.
Relationships between quality-of-life factors like household income in
specific neighborhoods, education levels, employment rates and so forth
--- and how each of those changed in response to climate --- would
reveal patterns that could be projected into the future. As moisture
raises the grain in a slab of wood, the information just needed to be
elicited.

Under every scientific forecast for global climate change, El Salvador
gets hotter and drier, and our model was in accord with what other
researchers said was likely: San Salvador will continue to grow as a
result, putting still more people in its dense outer rings. What happens
in its farm country, though, is more dependent on which climate and
development policies governments to the north choose to deploy in
dealing with the warming planet. High emissions, with few global policy
changes and relatively open borders, will drive rural El Salvador ---
just like rural Guatemala --- to empty out, even as its cities grow.

Should the United States and other wealthy countries change the
trajectory of global policy, though --- by, say, investing in climate
mitigation efforts at home but also hardening their borders --- they
would trigger a complex cascade of repercussions farther south,
according to the model. Central American and Mexican cities continue to
grow, albeit less quickly, but their overall wealth and development
slows drastically, most likely concentrating poverty further. Far more
people also remain in the countryside for lack of opportunity, becoming
trapped and more desperate than ever.

\includegraphics{https://static01.nyt.com/packages/flash/multimedia/ICONS/transparent.png}

\includegraphics{https://static01.nyt.com/images/2020/07/26/magazine/26mag-Migration-Images-12/26mag-Migratiion-Images-12-master1050.jpg}

ALTA VERAPAZ. Residents near the trickle that remains of the river that
once flowed through the Nuevo Paraíso Indigenous community.

People move to cities because they can seem like a refuge, offering the
facade of order --- tall buildings and government presence --- and the
mirage of wealth. I met several men who left their farm fields seeking
extremely dangerous work as security guards in San Salvador and
Guatemala City. I met a 10-year-old boy washing car windows at a
stoplight, convinced that the coins in his jar would help buy back his
parents' farmland. Cities offer choices, and a sense that you can
control your destiny.

These same cities, though, can just as easily become traps, as the
challenges that go along with rapid urbanization quickly pile up. Since
2000, San Salvador's population has ballooned by more than a third as it
has absorbed migrants from the rural areas, even as tens of thousands of
people continue to leave the country and migrate north. By midcentury,
the U.N. estimates that El Salvador --- which has 6.4 million people and
is the most densely populated country in Central America --- will be 86
percent urban.

Our models show that much of the growth will be concentrated in the
city's slumlike suburbs, places like San Marcos, where people live in
thousands of ramshackle structures, many without electricity or fresh
water. In these places, even before the pandemic and its fallout, good
jobs were difficult to find, poverty was deepening and crime was
increasing. Domestic abuse has also been rising, and declining sanitary
conditions threaten more disease. As society weakens, the gangs ---
whose members outnumber the police in parts of El Salvador by an
estimated three to one --- extort and recruit. They have made San
Salvador's murder rate one of the highest in the world.

Cortez hoped to escape the violence, but she couldn't. The gangs run
through her apartment block, stealing televisions and collecting
protection payments. She had recently witnessed a murder inside a
medical clinic where she was delivering food. The lack of security, the
lack of affordable housing, the lack of child care, the lack of
sustenance --- all influence the evolution of complex urban systems
under migratory pressure, and our model considers such stresses by
incorporating data on crime, governance and health care. They are
signposts for what is to come.

A week before our meeting last year, Cortez had resolved to make the
trip to the United States at almost any cost. For months she had ``felt
like going far away,'' but moving home was out of the question. ``The
climate has changed, and it has provoked us,'' she said, adding that it
had scarcely rained in three years. ``My dad, last year, he just gave
up.''

Cortez recounted what she did next. As her boss dropped potato
\emph{pupusas} into the smoking fryer, Cortez turned to her and made an
unimaginable request: Would she take Cortez's baby? It was the only way
to save the child, Cortez said. She promised to send money from the
United States, but the older woman said no --- she couldn't imagine
being able to care for the infant.

Today San Salvador is shut down by the coronavirus pandemic, and Cortez
is cooped up inside her apartment in San Marcos. She hasn't worked in
three months and is unable to see her daughters in El Paste. She was
allowed a forbearance on rent during the country's official lockdown,
but that has come to an end. She remains convinced that the United
States is her only salvation --- border walls be damned. She'll leave,
she said, ``the first chance I get.''

\includegraphics{https://static01.nyt.com/packages/flash/multimedia/ICONS/transparent.png}

\includegraphics{https://static01.nyt.com/images/2020/07/26/magazine/26mag-Migration-Images-09/26mag-Migratiion-Images-09-master1050.jpg}

ALTA VERAPAZ. Isabel Max Mez with her daughter Katerin Michel Xol Max.
The girl has a skin infection that doctors say was caused by
contaminated water.

\textbf{Most would-be migrants} don't want to move away from home.
Instead, they'll make incremental adjustments to minimize change, first
moving to a larger town or a city. It's only when those places fail them
that they tend to cross borders, taking on ever riskier journeys, in
what researchers call ``stepwise migration.'' Leaving a village for the
city is hard enough, but crossing into a foreign land --- vulnerable to
both its politics and its own social turmoil --- is an entirely
different trial.

Seven miles from the Suchiate River, which marks Guatemala's border with
Mexico, sits Siglo XXI, one of Mexico's largest immigration detention
centers, a squat concrete compound with 30-foot walls, barred windows
and a punishment cell. In early 2019, the 960-bed facility was largely
empty, as Mexico welcomed passing migrants instead of detaining them.
But by March, as the United States increased pressure to stop Central
Americans from reaching its borders, Mexico had begun to detain migrants
who crossed into its territory, packing almost 2,000 people inside this
center near the city of Tapachula. Detainees slept on mattresses thrown
down in the white-tiled hallways, waited in lines to use toilets
overflowing with feces and crammed shoulder to shoulder for hours to get
a meal of canned meat spooned onto a metal tray.

Projected decrease in percentage of annual rainfall by 2070 in many
parts of Guatemala:

\hypertarget{60}{%
\subparagraph{60}\label{60}}

On April 25, imprisoned migrants stormed the stairway leading to a
fortified security platform in the center's main hall, overpowering the
guards and then unlocking the main gates. More than 1,000 Guatemalans,
Cubans, Salvadorans, Haitians and others streamed into the Tapachula
night.

I arrived in Tapachula five weeks after the breakout to find a city
cracking in the crucible of migration. Just months earlier, passing
migrants on Mexico's southern border were offered rides and
\emph{tortas} and medicine from a sympathetic Mexican public. Now
migrant families were being hunted down in the countryside by armed
national-guard units, as if they were enemy soldiers.

Mexico has not always welcomed migrants, but President Andrés Manuel
López Obrador was trying to make his country a model for increasingly
open borders. This idealistic effort was also pragmatic: It was meant to
show the world an alternative to the belligerent wall-building
xenophobia he saw gathering momentum in the United States. More open
borders, combined with strategic foreign aid and help with human rights
to keep Central American migrants from leaving their homes in the first
place, would lead to a better outcome for all nations. ``I want to tell
them they can count on us,'' López Obrador had declared, promising the
migrants work permits and temporary jobs.

The architects of Mexico's policies assumed that its citizens had the
patience and the capacity to absorb --- economically, environmentally
and socially --- such an influx of people. But they failed to anticipate
how President Trump would hold their economy hostage to press his own
anti-immigrant crackdown, and they were caught off-guard by how the
burdens brought by the immigration traffic weighed on Mexico's own
people.

\includegraphics{https://static01.nyt.com/packages/flash/multimedia/ICONS/transparent.png}

\includegraphics{https://static01.nyt.com/images/2020/07/26/magazine/26mag-Migration-Images-08/26mag-Migratiion-Images-08-master1050.jpg}

CHIAPAS. Juan Francisco Murcia (left), a climate migrant from Honduras,
studying a map of shelters near northbound train routes.

In the six months after López Obrador took office in December 2018, some
420,000 people entered Mexico without documentation, according to
Mexico's National Migration Institute. Many floated across the Suchiate
on boards tied atop large inner tubes, paying guides a couple of dollars
for passage. In Ciudad Hidalgo, a border town outside Tapachula,
migrants camped in the square and fought in the streets. In a late-night
interview in his cinder-block office, under the glare of fluorescent
lights, the town's director of public security, Luis Martínez López,
rattled off statistics about their impact: Armed robberies jumped 45
percent; murders increased 15 percent.

Whether the crimes were truly attributable to the migrants was a matter
of significant debate, but the perception that they were fueled a rising
impatience. That March, Martínez told me, a confrontation between a
crowd of about 400 migrants and the local police turned rowdy, and the
migrants tied up five officers in the center of town. No one was hurt,
but the incident stoked locals' concern that things were getting out of
control. ``We used to open doors for them like brothers and feed them,''
said Martínez, who has since left his government job. ``I was
disappointed and angry.''

In Tapachula, a much larger city, tourism and commerce began to suffer.
Whole families of migrants huddled in downtown doorways overnight,
crowding sidewalks and sleeping on thin, oil-stained sheets of
cardboard. Hotels --- normally almost sold out in December --- were less
than 65 percent full as visitors stayed away, fearful of crime. Clinics
ran short of medication. The impact came at a vulnerable moment: While
many northern Mexican states enjoyed economic growth of 3 to 11 percent
in 2018, Chiapas --- its southernmost state --- had a 3 percent drop in
its gross domestic product. ``They are overwhelmed,'' said the Rev.
César Cañaveral Pérez, who earned a Ph.D. in the theology of human
mobility in Rome and now runs Tapachula's largest Catholic migrant
shelter.

\includegraphics{https://static01.nyt.com/packages/flash/multimedia/ICONS/transparent.png}

\includegraphics{https://static01.nyt.com/images/2020/07/26/magazine/26mag-Migration2-images-02/26mag-Migration2-images-02-master1050.jpg}

TAPACHULA, MEXICO. Young migrants eating breakfast at a shelter.

Models can't say much about the cultural strain that might result from a
climate influx; there is no data on anger or prejudice. What they do say
is that over the next two decades, if climate emissions continue as they
are, the population in southern Mexico will grow sharply.

At the same time, Mexico has its own serious climate concerns and will
most likely see its own climate exodus. One in six Mexicans now rely on
farming for their livelihood, and close to half the population lives in
poverty. Studies estimate that with climate change, water availability
per capita could decrease by as much as 88 percent in places, and crop
yields in coastal regions may drop by a third. If that change does
indeed push out a wave of Mexican migrants, many of them will most
likely come from Chiapas.

Yet a net increase in population at the same time --- which is what our
models assume --- suggests that even as one million or so climate
migrants make it to the U.S. border, many more Central Americans will
become trapped in protracted transit, unable to move forward or backward
in their journey, remaining in southern Mexico and making its current
stresses far worse.

Percentage of future urban growth that, according to the International
Committee of the Red Cross, is likely to take place in some of the
world's most fragile cities, where risk of social unrest is high:

\hypertarget{96}{%
\subparagraph{96}\label{96}}

Already, by late last year, the Mexican government's ill-planned
policies had begun to unravel into something more insidious: rising
resentment and hate. Now that the coronavirus pandemic has effectively
sealed borders, those sentiments risk bubbling over. Migrants, with
nowhere to go and no shelters able to take them in, roam the streets,
unable to socially distance and lacking even basic sanitation.

It has angered many Mexican citizens, who have begun to describe the
migrants as economic parasites and question foreign aid aimed at helping
people cope with the drought in places where Jorge A. and Cortez come
from.

``How dare AMLO give \$30 million to El Salvador when we have no
services here?'' asked Javier Ovilla Estrada, a community-group leader
in the southern border town Ciudad Hidalgo, referring to López Obrador's
participation in a multibillion-dollar development plan with Guatemala,
Honduras and El Salvador. Ovilla has become a strident defender of a new
Mexico-first movement, organizing thousands to march against immigrants.
Months before the coronavirus spread, we met in the sterile dining room
of a Chinese restaurant that he frequents in Ciudad Hidalgo, and he
echoed the same anti-immigrant sentiments rising in the U.S. and Europe.

The migrants ``don't love this country,'' he said. He points to
anti-immigrant Facebook groups spreading rumors that migrants stole
ballots and rigged the Mexican presidential election, that they murder
with impunity and run brothels. He's not the first to tell me that the
migrants traffic in disease --- that Suchiate will soon be overwhelmed
by Ebola. ``They should close the borders once and for all,'' he said.
If they don't, he warns, the country will sink further into lawlessness
and conflict. ``We're going to go out into the streets to defend our
homes and our families.''

\includegraphics{https://static01.nyt.com/packages/flash/multimedia/ICONS/transparent.png}

\includegraphics{https://static01.nyt.com/images/2020/07/26/magazine/26mag-Migration-Images-10/26mag-Migratiion-Images-10-master1050.jpg}

SAN MATEO, MEXICO. A joint forces operation including Mexican National
Guard soldiers, federal police officers and immigration agents detaining
migrants during a raid on a train.

\textbf{One afternoon last} summer, I sat on a black pleather couch in a
borrowed airport-security office at the Tapachula airfield to talk with
Francisco Garduño Yáñez, Mexico's new commissioner for immigration.
Garduño had abruptly succeeded a man named Tonatiuh Guillén López, a
strong proponent of more open borders, whom I'd been trying to reach for
weeks to ask how Mexico had strayed so far from the mission he laid out
for it.

But in between, Trump had, as another senior government official told
me, ``held a gun to Mexico's head,'' demanding a crackdown at the
Guatemalan border under threat of a 25 percent tariff on trade. Such a
tax could break the back of Mexico's economy overnight, and so López
Obrador's government immediately agreed to dispatch a new militarized
force to the border. Guillén resigned as a result, four days before I
hoped to meet him.

Number of people projected to be displaced from their homes by rising
sea levels alone by 2050:

\hypertarget{150m}{%
\subparagraph{150M}\label{150m}}

Garduño, a cheerful man with short graying hair, a broad smile and a
ceaseless handshake, had been on the job for less than 36 hours. He had
flown to Tapachula because another riot had broken out in one of the
city's smaller fortified detention centers, and a starving Haitian
refugee was filmed by news crews there, begging for help for her and her
young son. I wanted to know how it had come to this --- from signing an
international humanitarian migrant bill of rights to a mother lying with
her face pressed to the ground in a detention center begging for food,
in the space of a few months. He demurred, laying blame at the feet of
neoliberal economics, which he said had produced a ``poverty factory''
with no regional development policies to address it. It was the system
--- capitalism itself --- that had abandoned human beings, not Mexico's
leaders. ``We didn't anticipate that the globalization of the economy,
the globalization of the law \ldots{} would have such a devastating
effect,'' Garduño told me.

It seemed telling that Garduño's previous role had been as Mexico's
commissioner of federal prisons. Was this the start of a new, punitive
Mexico? I asked him. Absolutely not, he replied. But Mexico was now
pursuing a policy of ``containment,'' he said, rejecting the notion that
his country was obligated to ``receive a global migration.''

No policy, though, would be able to stop the forces --- climate,
increasingly, among them --- that are pushing migrants from the south to
breach Mexico's borders, legally or illegally. So what happens when
still more people --- many millions more --- float across the Suchiate
River and land in Chiapas? Our model suggests that this is what is
coming --- that between now and 2050, nearly nine million migrants will
head for Mexico's southern border, more than 300,000 of them because of
climate change alone.

Before leaving Mexico last summer, I went to Huixtla, a small town 25
miles west of Tapachula that, because it sat on the Bestia freight rail
line used by migrants, had long been a waypoint on Mexico's superhighway
for Central Americans on their way north. Joining several local police
officers as they headed out on patrol, I watched as our pickup truck's
red and blue lights reflected in the barred windows of squat
cinder-block homes. Two officers stood in back, holding tight to the
truck's roll bars, black combat boots firmly planted in the cargo bed,
as the driver, dodging mangy dogs, navigated the town's slender
alleyways.

The operations commander, a soft-spoken bureaucratic type named José
Gozalo Rodríguez Méndez, sat in the front seat. I asked him if he
thought Mexico could sustain the number of migrants who might soon come.
He said Mexico would buckle. There is no money from the federal
government, no staffing to address services, no housing, let alone
shelter, no more good will. ``We couldn't do it.''

Rodríguez had already been tested. When the first caravan of thousands
of migrants reached Huixtla in late 2018, throngs of tired, destitute
people --- many of them carrying children in their emaciated arms ---
packed the central square and spilled down the city's side streets.
Rodríguez and his wife went through their cupboards, gathering corn,
fried beans and tortillas, and collected clothing outgrown by their
children and hauled all of it to the town center, where church and civic
groups had set up tents and bathrooms.

But as the caravans continued, he said, his good will began to
disintegrate. ``It's like inviting somebody to your place for dinner,''
he said. ``You'll invite them once, even twice. But will you invite them
six times?'' When the fourth caravan of migrants approached the city
last March, Rodríguez told me, he stayed home.

In the center of town, the truck lurched to a stop amid a busy market,
where stalls sell vegetables and toys under blue light filtered through
plastic tarps overhead. A short way away, five men sheltered from the
searing heat under the shade of a metal awning on the platform of a
crumbling railway station, never repaired after Hurricane Stan 14 years
earlier. Rodríguez peppered the group --- two from Honduras, three from
Guatemala --- with questions. Together they said they had suffered the
totality of misfortune that Central America offers: muggings, gang
extortion and environmental disaster. Either they couldn't grow food or
the drought made it too expensive to buy.

``We can't stand the hunger,'' said one Honduran farmer, Jorge Reyes,
his gaunt face dripping with sweat. At his feet was a gift from a
shopkeeper: a plastic bag filled with a cut of raw meat, pooled in its
own blood, flies circling around it in the heat. Reyes had nowhere to
cook it. ``If we are going to die anyway,'' he said, ``we might as well
die trying to get to the United States.''

\includegraphics{https://static01.nyt.com/packages/flash/multimedia/ICONS/transparent.png}

\includegraphics{https://static01.nyt.com/images/2020/07/26/magazine/26mag-Migration2-images-03/26mag-Migration2-images-03-master1050-v2.jpg}

EL PASO. People waiting to enter the United States at a Customs and
Border Protection point of entry.

Reyes had made his decision. Like Jorge A., Cortez and millions of
others, he was going to the U.S. The next choice --- how to respond and
prepare for the migrants --- ultimately falls to America's elected
leaders.

Over the course of 2019, El Paso, Texas, had endured a crush of people
at its border crossings, peaking at more than 4,000 migrants in a single
day, as the same caravans of Central Americans that had worn out their
welcome in Tapachula made their way here. It put El Paso in a delicate
spot, caught between the forces of politically charged anti-immigrant
federal policy and its own deep roots as a diverse, largely Hispanic
city whose identity was virtually inextricable from its close ties to
Mexico. This surge, though, stretched the city's capacity. When the
migrants arrived, city officials argued over who should pay the tab for
the emergency services, aid and housing, and in the end crossed their
fingers and hoped the city's active private charities would figure it
out. Church groups rented thousands of hotel rooms across the city,
delivered food, offered counseling and so on.

Conjoined to the Mexican city of Juárez, the El Paso area is the
second-largest binational metroplex in the Western Hemisphere. It sits
smack in the middle of the Chihuahuan Desert, a built-up oasis amid a
barren and bleached-bright rocky landscape. Much of its daily work force
commutes across the border, and Spanish is as common as English.

Downtown, new buildings are rising in a weary business district where
boot shops and pawnshops compete amid boarded-up and barred storefronts.
The only barriers between the American streets --- home to more than
800,000 people --- and their Juárez counterparts are the concrete
viaduct of a mostly dry Rio Grande and a rusted steel border fence.

\includegraphics{https://static01.nyt.com/packages/flash/multimedia/ICONS/transparent.png}

\includegraphics{https://static01.nyt.com/images/2020/07/26/magazine/26mag-Migration2-images-05/26mag-Migration2-images-05-master1050.jpg}

EL PASO. Last year, the city endured a crush of people at its border
crossings --- peaking at 4,000 people in a single day.

To some migrants, this place is Eden. But El Paso is also a place with
oppressive heat and very little water, another front line in the climate
crisis. Temperatures already top 90 degrees here for three months of the
year, and by the end of the century it will be that hot one of every two
days. The heat, according to researchers at the University of
California, Berkeley, will drive deaths that soon outpace those from car
crashes or opioid overdoses. Cooling costs --- already a third of some
residents' budgets --- will get pricier, and warming will drive down
economic output by 8 percent, perhaps making El Paso just as unlivable
as the places farther south.

In 2014, El Paso created a new city government position --- chief
resilience officer --- aimed, in part, at folding climate concerns into
its urban planning. Soon enough, the climate crisis in Guatemala --- not
just the one in El Paso --- became one of the city's top concerns. ``I
apologize if I'm off topic,'' the resilience chief, Nicole Ferrini, told
municipal leaders and other attendees at a water conference in Phoenix
in 2019 as she raised the question of ``massive amounts of climate
refugees, and are we prepared as a community, as a society, to deal with
that?''

Ferrini, an El Paso native, did her academic training as an architect.
She worries that El Paso will struggle to adapt if its leadership, and
the nation's, continue to react to daily or yearly spikes rather than
view the problem as a systematic one, destined to become steadily worse
as the planet warms. She sees her own city as an object lesson in what
U.N. officials and climate-migration scientists have been warning of:
Without a decent plan for housing, feeding and employing a growing
number of climate refugees, cities on the receiving end of migration can
never confidently pilot their own economic future.

For the moment, the coronavirus pandemic has largely choked off legal
crossings into El Paso, but that crisis will eventually fade. And when
it does, El Paso will face the same enduring choice that all wealthier
societies everywhere will eventually face: determining whether it is a
society of walls or --- in the vernacular of aid organizations working
to fortify infrastructure and resilience to stem migration --- one that
builds wells.

\includegraphics{https://static01.nyt.com/packages/flash/multimedia/ICONS/transparent.png}

\includegraphics{https://static01.nyt.com/images/2020/07/26/magazine/26mag-Migration-Images-11/26mag-Migratiion-Images-11-master1050.jpg}

EL PASO. A mother and daughter from Central America, hoping for asylum,
turning themselves in to Border Patrol agents.

Around the world, nations are choosing walls. Even before the pandemic,
Hungary fenced off its boundary with Serbia, part of more than 1,000
kilometers of border walls erected around the European Union states
since 1990. India has built a fence along most of its 2,500-mile border
with Bangladesh, whose people are among the most vulnerable in the world
to sea-level rise.

The United States, of course, has its own wall-building agenda ---
literal ones, and the figurative ones that can have a greater effect. On
a walk last August from one of El Paso's migrant shelters, an
inconspicuous brick home called Casa Vides, the Rev. Peter Hinde told me
that El Paso's security-oriented economy had created a cultural barrier
that didn't exist when he moved here 25 years earlier. Hinde, who is 97,
helps run the Carmelite order in Juárez but was traveling to volunteer
at Casa Vides on a near-daily basis. A former Army Air Forces captain
and fighter pilot who grew up in Chicago, Hinde said the United States
is turning its own fears into reality when it comes to immigration,
something he witnesses in a growing distrust of everyone who crosses the
border.

That fear creates other walls. The United States refused to join 164
other countries in signing a global migration treaty in 2018, the first
such agreement to recognize climate as a cause of future displacement.
At the same time, the U.S. is cutting off foreign aid --- money for
everything from water infrastructure to greenhouse agriculture --- that
has been proved to help starving families like Jorge A.'s in Guatemala
produce food, and ultimately stay in their homes. Even those migrants
who legally make their way into El Paso have been turned back, relegated
to cramped and dangerous shelters in Juárez to wait for the hearings
they are owed under law.

\textbf{There is no more natural} and fundamental adaptation to a
changing climate than to migrate. It is the obvious progression the
earliest Homo sapiens pursued out of Africa, and the same one the Mayans
tried 1,200 years ago. As Lorenzo Guadagno at the U.N.'s International
Organization for Migration told me recently, ``Mobility is resilience.''
Every policy choice that allows people the flexibility to decide for
themselves where they live helps make them safer.

Are you a teacher looking for a way to use this project in your
classroom? You can find resources from the Pulitzer Center
\href{https://pulitzercenter.org/event/webinar-educators-exploring-climate-migration-classroom}{here}.

But it isn't always so simple, and relocating across borders doesn't
have to be inevitable. I thought about Jorge A. from Guatemala. He made
it to the United States last spring, climbing the steel border barrier
and dropping his 7-year-old son 20 feet down the other side into the
California desert. (We are abbreviating his last name in this article
because of his undocumented status.) Now they live in Houston, where
until the pandemic, Jorge found steady work in construction, earning
enough to pay his debts and send some money home. But the separation
from his wife and family has proved intolerable; home or away, he can't
win, and as of early July, he was wondering if he should go back to
Guatemala.

And therein lies the basis for what may be the worst-case scenario: one
in which America and the rest of the developed world refuse to welcome
migrants but also fail to help them at home. As our model demonstrated,
closing borders while stinting on development creates a somewhat
counterintuitive population surge even as temperatures rise, trapping
more and more people in places that are increasingly unsuited to human
life.

In that scenario, the global trend toward building walls could have a
profound and lethal effect. Researchers suggest that the annual death
toll, globally, from heat alone will eventually rise by 1.5 million. But
in this scenario, untold more will also die from starvation, or in the
conflicts that arise over tensions that food and water insecurity will
bring.

\includegraphics{https://static01.nyt.com/packages/flash/multimedia/ICONS/transparent.png}

\includegraphics{https://static01.nyt.com/images/2020/07/26/magazine/26mag-Migration2-images-04/26mag-Migration2-images-04-master1050.jpg}

JUÁREZ, MEXICO. José Cruz and his daughter Yakelin (center), climate
migrants from Honduras, have waited months in a shelter for their asylum
request to be processed.

If this happens, the United States and Europe risk walling themselves
in, as much as walling others out. And so the question then is: What are
policymakers and planners prepared to do about that? America's
demographic decline suggests that more immigrants would play a
productive role here, but the nation would have to be willing to invest
in preparing for that influx of people so that the population growth
alone doesn't overwhelm the places they move to, deepening divisions and
exacerbating inequalities. At the same time, the United States and other
wealthy countries can help vulnerable people where they live, by funding
development that modernizes agriculture and water infrastructure. A U.N.
World Food Program effort to help farmers build irrigated greenhouses in
El Salvador, for instance, has drastically reduced crop losses and
improved farmers' incomes. It can't reverse climate change, but it can
buy time.

Thus far, the United States has done very little at all. Even as the
scientific consensus around climate change and climate migration builds,
in some circles the topic has become taboo. This spring, after
Proceedings of the National Academy of Sciences published the explosive
study estimating that, barring migration, one-third of the planet's
population may eventually live outside the traditional ecological niche
for civilization, Marten Scheffer, one of the study's authors, told me
that he was asked to tone down some of his conclusions through the
peer-review process and that he felt pushed to ``understate'' the
implications in order to get the research published. The result:
Migration is only superficially explored in the paper. (A spokeswoman
for the journal declined to comment because the review process is
confidential.)

``There's flat-out resistance,'' Scheffer told me, acknowledging what he
now sees as inevitable, that migration is going to be a part of the
global climate crisis. ``We have to face it.''

Our modeling and the consensus of academics point to the same bottom
line: If societies respond aggressively to climate change and migration
and increase their resilience to it, food production will be shored up,
poverty reduced and international migration slowed --- factors that
could help the world remain more stable and more peaceful. If leaders
take fewer actions against climate change, or more punitive ones against
migrants, food insecurity will deepen, as will poverty. Populations will
surge, and cross-border movement will be restricted, leading to greater
suffering. Whatever actions governments take next --- and when they do
it --- makes a difference.

The window for action is closing. The world can now expect that with
every degree of temperature increase, roughly a billion people will be
pushed outside the zone in which humans have lived for thousands of
years. For a long time, the climate alarm has been sounded in terms of
its economic toll, but now it can increasingly be counted in people
harmed. The worst danger, Hinde warned on our walk, is believing that
something so frail and ephemeral as a wall can ever be an effective
shield against the tide of history. ``If we don't develop a different
attitude,'' he said, ``we're going to be like people in the lifeboat,
beating on those that are trying to climb in.''

\includegraphics{https://static01.nyt.com/packages/flash/multimedia/ICONS/transparent.png}

\includegraphics{https://static01.nyt.com/images/2020/07/26/magazine/26mag-Migration-Images-13/26mag-Migratiion-Images-13-master1050.jpg}

ALTA VERAPAZ. An Indigenous agricultural worker, Martin Yat Chen, on
farmland that is too dry to plant in anymore.

\textbf{Abrahm Lustgarten} is a senior environmental reporter at
ProPublica. His 2015 series examining the causes of water scarcity in
the American West, ``Killing the Colorado,'' was a finalist for the 2016
Pulitzer Prize for national reporting. \textbf{Meridith Kohut} is an
award-winning photojournalist based in Caracas, Venezuela, who has
documented global health and humanitarian crises in Latin America for
The New York Times for more than a decade. Her recent assignments
include photographing migration and childbirth in Venezuela,
antigovernment protests in Haiti and the killing of women in Guatemala.

Reporting and translation were contributed by Pedro Pablo Solares in
Guatemala and El Salvador, and Louisa Reynolds and Juan de Dios García
Davish in Mexico.

Data for opening globe graphic from ``Future of the Human Climate
Niche,'' by Chi Xu, Timothy A. Kohler, Timothy M. Lenton, Jens-Christian
Svenning and Marten Scheffer, from Proceedings of the National Academy
of Sciences. Graphic by Bryan Christie Design/Joe Lertola.

Maps in Central America graphics sequence show total population shift
under the SSP5 / RCP 8.5 and SSP3 / RCP 8.5 scenarios used by the U.N.'s
Intergovernmental Panel on Climate Change, and it is calculated on a
15-kilometer grid. A cube-root scale was used to compress the largest
peaks.

Projections based on research by The New York Times Magazine and
ProPublica, with support from the Pulitzer Center. Model graphics and
additional data analysis by Matthew Conlen.

Additional design and development by Jacky Myint and Shannon Lin.

\textbf{Abrahm Lustgarten} is a senior environmental reporter at
ProPublica. His 2015 series examining the causes of water scarcity in
the American West, ``Killing the Colorado,'' was a finalist for the 2016
Pulitzer Prize for national reporting. \textbf{Meridith Kohut} is an
award-winning photojournalist based in Caracas, Venezuela, who has
documented global health and humanitarian crises in Latin America for
The New York Times for more than a decade. Her recent assignments
include photographing migration and childbirth in Venezuela,
antigovernment protests in Haiti and the killing of women in Guatemala.

Read 571 Comments

\begin{itemize}
\item
\item
\item
\item
\end{itemize}

Advertisement

\protect\hyperlink{after-bottom}{Continue reading the main story}

\hypertarget{site-index}{%
\subsection{Site Index}\label{site-index}}

\hypertarget{site-information-navigation}{%
\subsection{Site Information
Navigation}\label{site-information-navigation}}

\begin{itemize}
\tightlist
\item
  \href{https://help.nytimes.com/hc/en-us/articles/115014792127-Copyright-notice}{©~2020~The
  New York Times Company}
\end{itemize}

\begin{itemize}
\tightlist
\item
  \href{https://www.nytco.com/}{NYTCo}
\item
  \href{https://help.nytimes.com/hc/en-us/articles/115015385887-Contact-Us}{Contact
  Us}
\item
  \href{https://www.nytco.com/careers/}{Work with us}
\item
  \href{https://nytmediakit.com/}{Advertise}
\item
  \href{http://www.tbrandstudio.com/}{T Brand Studio}
\item
  \href{https://www.nytimes.com/privacy/cookie-policy\#how-do-i-manage-trackers}{Your
  Ad Choices}
\item
  \href{https://www.nytimes.com/privacy}{Privacy}
\item
  \href{https://help.nytimes.com/hc/en-us/articles/115014893428-Terms-of-service}{Terms
  of Service}
\item
  \href{https://help.nytimes.com/hc/en-us/articles/115014893968-Terms-of-sale}{Terms
  of Sale}
\item
  \href{https://spiderbites.nytimes.com}{Site Map}
\item
  \href{https://help.nytimes.com/hc/en-us}{Help}
\item
  \href{https://www.nytimes.com/subscription?campaignId=37WXW}{Subscriptions}
\end{itemize}
