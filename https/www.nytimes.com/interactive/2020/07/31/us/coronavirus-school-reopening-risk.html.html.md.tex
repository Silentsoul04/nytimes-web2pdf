Sections

SEARCH

\protect\hyperlink{site-content}{Skip to
content}\protect\hyperlink{site-index}{Skip to site index}

\href{https://www.nytimes.com/section/us}{U.S.}

\href{https://myaccount.nytimes.com/auth/login?response_type=cookie\&client_id=vi}{}

\href{https://www.nytimes.com/section/todayspaper}{Today's Paper}

\href{/section/us}{U.S.}\textbar{}The Risk That Students Could Arrive at
School With the Coronavirus

\url{https://nyti.ms/315fGaL}

\begin{itemize}
\item
\item
\item
\item
\item
\item
\end{itemize}

\href{https://www.nytimes.com/news-event/coronavirus?action=click\&pgtype=Article\&state=default\&region=TOP_BANNER\&context=storylines_menu}{The
Coronavirus Outbreak}

\begin{itemize}
\tightlist
\item
  live\href{https://www.nytimes.com/2020/08/02/world/coronavirus-updates.html?action=click\&pgtype=Article\&state=default\&region=TOP_BANNER\&context=storylines_menu}{Latest
  Updates}
\item
  \href{https://www.nytimes.com/interactive/2020/us/coronavirus-us-cases.html?action=click\&pgtype=Article\&state=default\&region=TOP_BANNER\&context=storylines_menu}{Maps
  and Cases}
\item
  \href{https://www.nytimes.com/interactive/2020/science/coronavirus-vaccine-tracker.html?action=click\&pgtype=Article\&state=default\&region=TOP_BANNER\&context=storylines_menu}{Vaccine
  Tracker}
\item
  \href{https://www.nytimes.com/interactive/2020/07/29/us/schools-reopening-coronavirus.html?action=click\&pgtype=Article\&state=default\&region=TOP_BANNER\&context=storylines_menu}{What
  School May Look Like}
\item
  \href{https://www.nytimes.com/live/2020/07/31/business/stock-market-today-coronavirus?action=click\&pgtype=Article\&state=default\&region=TOP_BANNER\&context=storylines_menu}{Economy}
\end{itemize}

Advertisement

\protect\hyperlink{after-top}{Continue reading the main story}

\hypertarget{comments}{%
\subsection{\texorpdfstring{\protect\hyperlink{commentsContainer}{Comments}}{Comments}}\label{comments}}

\href{}{The Risk That Students Could Arrive at School With the
Coronavirus}\href{}{Skip to Comments}

The comments section is closed. To submit a letter to the editor for
publication, write to
\href{mailto:letters@nytimes.com}{\nolinkurl{letters@nytimes.com}}.

\hypertarget{the-risk-that-students-could-arrive-at-school-with-the-coronavirus}{%
\section{The Risk That Students Could Arrive at School With the
Coronavirus}\label{the-risk-that-students-could-arrive-at-school-with-the-coronavirus}}

By \href{https://www.nytimes.com/by/james-glanz}{James Glanz},
\href{https://www.nytimes.com/by/benedict-carey}{Benedict Carey} and
Matthew ConlenJuly 31, 2020

\begin{itemize}
\item
\item
\item
\item
\item
  \emph{289}
\end{itemize}

As schools grapple with how to reopen, new estimates show that large
parts of the country would likely see infected students if classrooms
opened now.

Estimated infected people arriving in the first week

1

2

3

5

10+

+

-

Pod of 10

School of 100

School of 500

School of 1,000

Source: Lauren Ancel Meyers and Spencer Fox, the University of Texas at
Austin; Michael Lachmann, Santa Fe Institute

Millions of families face an excruciating choice this fall: Should their
children attend if local schools reopen their classrooms, and risk being
exposed to the coronavirus? Or should they stay home and lose out on
in-person instruction?

No single factor can settle such a fraught decision. But new estimates
provide a rough gauge of the risk that students and educators could
encounter at school in each county in the United States.

The estimates, from researchers at the University of Texas at Austin,
range from sobering to surprisingly reassuring, depending on the area
and the size of the school.

Based on current infection rates, more than 80 percent of Americans live
in a county where at least one infected person would be expected to show
up to a school of 500 students and staff in the first week, if school
started today.

In the highest-risk areas --- including Miami, Fort Lauderdale,
Nashville and Las Vegas --- at least five students or staff would be
expected to show up infected with the virus at a school of 500 people.

The high numbers reflect the rapid spread of the virus in those areas,
where more than 1 in 70 people are estimated to be currently infected.

At the same time, smaller, isolated groups of students face a much lower
risk. Some schools are considering
\href{https://www.nytimes.com/interactive/2020/07/29/us/schools-reopening-coronavirus.html}{narrowing
classes down} to small ``pods,'' with students who mainly come in
contact with their teacher and each other. While the chance of having an
infected person at the school would stay the same, the risk of exposure
within those pods would be much lower.

If they remain isolated from the rest of the school --- a tall order ---
10-person pods in every part of the country would be unlikely to include
an infected person in that first week.

\hypertarget{how-many-infected-people-might-arrive-if-classes-started-today}{%
\subsubsection{How many infected people might arrive if classes started
today?}\label{how-many-infected-people-might-arrive-if-classes-started-today}}

Pod of 10

School of 100

School of 500

School of 1,000

Note: Estimates show potential infected people arriving during the first
week of instruction. A zero indicates a low probability that an infected
person will show up in the school or pod during that week.

Education experts and disease researchers said information that reflects
local conditions could be critical in shaping decisions by parents,
teachers, administrators and political leaders.

``It's meant to guide schools so they can anticipate when it might be
safe, or easier, to open and bring kids in,'' said Lauren Ancel Meyers,
an epidemiologist at the University of Texas at Austin who led the
research team.

The projections are rough guidelines based on the estimated prevalence
of the virus in each county, which is drawn from a New York Times
database of cases, and estimates that five people may be infected for
each known case. Those estimates reflect current levels of infection
around the country and are likely to change, improving or worsening in
individual communities over the next weeks and months.

The estimates assume that children are as likely to carry and transmit
the virus as adults --- ``a large assumption, given the unknowns about
children,'' said Spencer Fox, a member of the research team.

``This is meant to be a rough guide, a first step,'' Dr. Fox said.

Some preliminary studies have suggested that children are infected less
often, or that young ones do not transmit the disease as readily, which
could reduce the risk, said Carl T. Bergstrom, a professor of biology at
the University of Washington. But those questions
\href{https://www.nytimes.com/2020/07/30/health/coronavirus-children.html}{remain
unresolved}, he said.

Still, the information ``really helps put things into context for
parents,'' Dr. Bergstrom said. ``Anything that could help you do that
both helps you make better decisions and offers a level of comfort and
assurance.''

Many districts will start the school year remotely. Those that do open
buildings will hedge the risks by
\href{https://www.nytimes.com/interactive/2020/07/29/us/schools-reopening-coronavirus.html}{taking
various measures}, such as requiring masks and social distancing,
holding classes outside when possible or bringing students to school on
alternating schedules.

Plans announced by some of the nation's largest school systems already
show the range of choices in play. Districts
in\href{https://www.nytimes.com/2020/07/13/us/lausd-san-diego-school-reopening.html}{}\href{https://www.nytimes.com/2020/07/13/us/lausd-san-diego-school-reopening.html}{San
Diego and Los Angeles}, citing the risk of crowded classrooms, said they
would operate online in the fall, as will the
\href{https://www.nytimes.com/2020/07/17/us/california-schools-reopening-newsom.html}{vast
majority of schools in California} under guidelines issued by the
state.\href{https://www.nytimes.com/2020/07/08/nyregion/nyc-schools-reopening-plan.html}{}\href{https://www.nytimes.com/2020/07/08/nyregion/nyc-schools-reopening-plan.html}{New
York City}, though, is planning a partial reopening, allowing classroom
attendance one to three times a week.

But decisions on remote learning come with their own concerns, said Greg
J. Duncan, an education professor at the University of California,
Irvine.
\href{https://www.nytimes.com/2020/06/05/us/coronavirus-education-lost-learning.html}{Studies
have shown} that younger children and those in lower-income districts do
not learn as well online as they do in person. For lower-income
children, that gap in learning can persist, he said.

Wealthy families, which have more resources and workarounds, will be far
more risk-averse than others, Dr. Duncan said.

``One infection is too many'' will likely be the refrain of wealthier
families, he said. ``Any slight chance that their child is going to be
infected is probably going to get them to jump to a decision more
quickly than lower-income families.''

Although the risk varies by school size, in the hardest-hit areas of the
country, even small schools face significant risks.

In eight states, most people live in counties where even a school of
only 100 people would probably see an infected person in the first week
if school started today, the estimates say: Louisiana, Alabama,
Mississippi, Florida, Nevada, Tennessee, Arizona and Georgia.

The list is even longer for schools of 500 people: The vast majority of
people in 19 states, including California, Texas and Illinois, live in
counties where at least one infected person would likely show up to
school in the first week if in-person classes were held. Many of those
areas have elected to hold classes online for now.

Many parents are consumed with the question of returning to school, and
there is hunger for solid guidance, said Annette Campbell Anderson,
deputy director of the Johns Hopkins Center for Safe and Healthy
Schools.

``They want to see the data to make them feel that they have a model
that they can trust,'' Dr. Anderson said. ``And we need it. We need this
kind of data.''

Notes: Estimates approximate the proportion of the population that is
infectious based on the number who were infected during the preceding
seven days, from data ending July 28. The calculations assume that
students and teachers come in to school at least once a week and won't
come in if they are symptomatic. The estimates are likely to change as
the epidemic gets better or worse in certain areas. The estimates assume
that five people are infected for each known case. The estimated range
of infected people shown on the map is based on scenarios with between
three and 10 people infected for each known case.

Read 289 Comments

\begin{itemize}
\item
\item
\item
\item
\end{itemize}

Advertisement

\protect\hyperlink{after-bottom}{Continue reading the main story}

\hypertarget{site-index}{%
\subsection{Site Index}\label{site-index}}

\hypertarget{site-information-navigation}{%
\subsection{Site Information
Navigation}\label{site-information-navigation}}

\begin{itemize}
\tightlist
\item
  \href{https://help.nytimes.com/hc/en-us/articles/115014792127-Copyright-notice}{©~2020~The
  New York Times Company}
\end{itemize}

\begin{itemize}
\tightlist
\item
  \href{https://www.nytco.com/}{NYTCo}
\item
  \href{https://help.nytimes.com/hc/en-us/articles/115015385887-Contact-Us}{Contact
  Us}
\item
  \href{https://www.nytco.com/careers/}{Work with us}
\item
  \href{https://nytmediakit.com/}{Advertise}
\item
  \href{http://www.tbrandstudio.com/}{T Brand Studio}
\item
  \href{https://www.nytimes.com/privacy/cookie-policy\#how-do-i-manage-trackers}{Your
  Ad Choices}
\item
  \href{https://www.nytimes.com/privacy}{Privacy}
\item
  \href{https://help.nytimes.com/hc/en-us/articles/115014893428-Terms-of-service}{Terms
  of Service}
\item
  \href{https://help.nytimes.com/hc/en-us/articles/115014893968-Terms-of-sale}{Terms
  of Sale}
\item
  \href{https://spiderbites.nytimes.com}{Site Map}
\item
  \href{https://help.nytimes.com/hc/en-us}{Help}
\item
  \href{https://www.nytimes.com/subscription?campaignId=37WXW}{Subscriptions}
\end{itemize}
