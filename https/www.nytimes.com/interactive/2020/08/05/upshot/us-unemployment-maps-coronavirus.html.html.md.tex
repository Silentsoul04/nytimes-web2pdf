Sections

SEARCH

\protect\hyperlink{site-content}{Skip to
content}\protect\hyperlink{site-index}{Skip to site index}

\href{https://www.nytimes.com/section/upshot}{The Upshot}

\href{https://myaccount.nytimes.com/auth/login?response_type=cookie\&client_id=vi}{}

\href{https://www.nytimes.com/section/todayspaper}{Today's Paper}

\href{/section/upshot}{The Upshot}\textbar{}In These Neighborhoods, the
Jobless Rate May Top 30 Percent

\href{https://nyti.ms/3fwuW5R}{https://nyti.ms/3fwuW5R}

\begin{itemize}
\item
\item
\item
\item
\item
\item
\end{itemize}

\href{https://www.nytimes.com/news-event/coronavirus?action=click\&pgtype=Article\&state=default\&region=TOP_BANNER\&context=storylines_menu}{The
Coronavirus Outbreak}

\begin{itemize}
\tightlist
\item
  live\href{https://www.nytimes.com/2020/08/08/world/coronavirus-updates.html?action=click\&pgtype=Article\&state=default\&region=TOP_BANNER\&context=storylines_menu}{Latest
  Updates}
\item
  \href{https://www.nytimes.com/interactive/2020/us/coronavirus-us-cases.html?action=click\&pgtype=Article\&state=default\&region=TOP_BANNER\&context=storylines_menu}{Maps
  and Cases}
\item
  \href{https://www.nytimes.com/interactive/2020/science/coronavirus-vaccine-tracker.html?action=click\&pgtype=Article\&state=default\&region=TOP_BANNER\&context=storylines_menu}{Vaccine
  Tracker}
\item
  \href{https://www.nytimes.com/interactive/2020/world/coronavirus-tips-advice.html?action=click\&pgtype=Article\&state=default\&region=TOP_BANNER\&context=storylines_menu}{F.A.Q.}
\item
  \href{https://www.nytimes.com/live/2020/08/07/business/stock-market-today-coronavirus?action=click\&pgtype=Article\&state=default\&region=TOP_BANNER\&context=storylines_menu}{Markets
  \& Economy}
\end{itemize}

Advertisement

\protect\hyperlink{after-top}{Continue reading the main story}

\hypertarget{comments}{%
\subsection{\texorpdfstring{\protect\hyperlink{commentsContainer}{Comments}}{Comments}}\label{comments}}

\href{}{In These Neighborhoods, the Jobless Rate May Top 30
Percent}\href{}{Skip to Comments}

The comments section is closed. To submit a letter to the editor for
publication, write to
\href{mailto:letters@nytimes.com}{\nolinkurl{letters@nytimes.com}}.

Upshot

\hypertarget{in-these-neighborhoods-the-jobless-rate-may-top-30-percent}{%
\section{In These Neighborhoods, the Jobless Rate May Top 30
Percent}\label{in-these-neighborhoods-the-jobless-rate-may-top-30-percent}}

By \href{https://www.nytimes.com/by/quoctrung-bui}{Quoctrung Bui} and
\href{https://www.nytimes.com/by/emily-badger}{Emily Badger}Aug. 5, 2020

\begin{itemize}
\item
\item
\item
\item
\item
  \emph{115}
\end{itemize}

\hypertarget{unemployment-has-soared-in-some-census-tracts-spreading-pain-unevenly}{%
\subsubsection{Unemployment has soared in some census tracts, spreading
pain
unevenly}\label{unemployment-has-soared-in-some-census-tracts-spreading-pain-unevenly}}

The economic damage from the coronavirus is most visible in areas like
Midtown Manhattan, where lunch spots have closed, businesses have gone
dark and once-crowded sidewalks have emptied.

But some of the worst economic pain lies in other neighborhoods, in the
places where workers who've endured the broadest job losses live. In
corners of the Bronx, South Los Angeles or the South Side of Chicago,
unemployment is concentrated to a breathtaking degree. And that means
that other problems still to come --- a wave of evictions, deepening
poverty, more \href{https://t.co/EbkPprWGbz?amp=1}{childhood hunger} ---
will be geographically concentrated, too.

Data estimating neighborhood-level unemployment rates suggests that as
many as one in three workers in these areas are jobless, deeply widening
economic disparities within cities.

In New York City, it's as if parts of the Bronx were experiencing the
Great Depression while the Upper East Side faced only modest drops in
employment, according to Yair Ghitza and Mark Steitz, analysts who have
\href{https://github.com/Catalist-LLC/unemployment/blob/master/deep_maps_20200804.pdf}{estimated
unemployment at the census tract level} based on national economic
statistics over the last six months.

\hypertarget{unemployment-in-new-york-city}{%
\subsubsection{Unemployment in New York
City}\label{unemployment-in-new-york-city}}

2\%

4\%

6\%

8\%

10\%

12\%

14\%

16\%

18\%

20\%

22\%

24\%

26\%

28\%

30\%

\hypertarget{february-2020}{%
\subsubsection{February 2020}\label{february-2020}}

\hypertarget{june-2020}{%
\subsubsection{June 2020}\label{june-2020}}

The federal government doesn't report unemployment data down to the
neighborhood level, so the two researchers modeled these fine-grained
statistics in a way that makes them ** consistent with state and
national surveys*.* Through June, they found most neighborhoods in the
Bronx had unemployment rates in excess of 20 percent, while most
neighborhoods south of 95th Street in Manhattan had rates less than half
that.

``What's salient and visible right now is the businesses that are
shuttered, and the office buildings that are empty,'' said Ingrid Gould
Ellen, a professor of urban policy and planning at N.Y.U. ``What we're
not quite seeing at least the physical manifestations of yet is the
really just stark decline in incomes in so many neighborhoods around the
city, and in a lot of working-class neighborhoods.''

``We will see them,'' she predicted, warning that concentrated distress
in these neighborhoods could also have long-term consequences for the
children growing up there.

Mr. Ghitza, the chief scientist at
\href{https://catalist.us/}{Catalist}, a Democratic data firm, and Mr.
Steitz, a principal at TSD Communications, have tried to solve a large
multiplication problem in modeling neighborhood-level unemployment.
Official government statistics estimate, for example, the share of
residents in a given census tract who are women, the share who are
African-American, and the share who work in food service. Using such
data, Mr. Ghitza and Mr. Steitz created an educated guess of the number
of Black female food-service workers in each tract, then matched those
demographics to national monthly unemployment statistics on the
occupations and demographic groups most severely affected in this
downturn.

The approach makes it possible to gauge employment differences at a
finer level of geography than what the government reports. But these
estimates also come with much wider room for error than official
statistics, and the researchers warn that the results should be viewed
alongside other data as policymakers try to understand an economy in
free fall.

The resulting maps capture the flip side of
\href{https://www.tracktherecovery.org/}{recent analyses} of
private-sector data showing where restaurants have cut hours
\href{https://www.nytimes.com/2020/08/03/nyregion/nyc-small-businesses-closing-coronavirus.html}{or
where stores have closed their doors}. Those business closings have been
clustered, too, often in downtown districts where office workers no
longer come in, or in
\href{https://www.nytimes.com/2020/06/17/upshot/coronavirus-spending-rich-poor.html}{wealthy
neighborhoods where residents have sharply reduced their spending} (or
where they have
\href{https://www.nytimes.com/interactive/2020/05/15/upshot/who-left-new-york-coronavirus.html}{left
town altogether}).

These maps reflect, instead, where the workers who once staffed those
restaurants, bars, hotels and offices commuted home at night:

\hypertarget{unemployment-in-chicago}{%
\subsubsection{Unemployment in Chicago}\label{unemployment-in-chicago}}

2\%

4\%

6\%

8\%

10\%

12\%

14\%

16\%

18\%

20\%

22\%

24\%

26\%

28\%

30\%

\hypertarget{february-2020-1}{%
\subsubsection{February 2020}\label{february-2020-1}}

\hypertarget{june-2020-1}{%
\subsubsection{June 2020}\label{june-2020-1}}

The maps also highlight how the distinct nature of the coronavirus
economic shock has divided cities into neighborhoods where most people
can work from home and neighborhoods where most can't. And because the
latter group is
\href{https://www.bls.gov/cps/effects-of-the-coronavirus-covid-19-pandemic.htm}{disproportionately
made up of Black and Hispanic workers}, those lines also largely follow
patterns of racial segregation, as in Chicago.

As of June, the Chicago metro area had an unemployment rate of
\href{https://www.bls.gov/web/metro/laulrgma.htm}{15.6 percent},
according to the Bureau of Labor Statistics. But Mr. Ghitza and Mr.
Steitz estimate that in some neighborhoods on the predominantly
African-American South Side, the unemployment rate was more than double
that. Wealthier neighborhoods on the North Side had unemployment rates
of less than 10 percent.

In a recent analysis, Peter Ganong, an economics professor at the
University of Chicago, found that workers in the lowest quintile of
income have experienced three times as many job losses as workers in the
highest quintile. But that's just looking through the lens of income
alone. He says layering race, age and gender could push the differences
even further at the census tract level.

Jesse Rothstein, an economist who is part of a team that has been
\href{https://irle.berkeley.edu/labor-market-impacts-of-covid-19-on-hourly-workers-in-small-and-medium-sized-businesses-four-facts-from-homebase-data-2/}{tracking
the effects of the pandemic on the labor market}, agrees that it's
possible for unemployment rates in some neighborhoods to barely budge
while others soar across town.

``There aren't that many food-service workers that live in Beverly
Hills,'' he said.

In Los Angeles, job losses appear to be most severe in South Los
Angeles, in predominantly Hispanic parts of the city.

\hypertarget{unemployment-in-los-angeles}{%
\subsubsection{Unemployment in Los
Angeles}\label{unemployment-in-los-angeles}}

2\%

4\%

6\%

8\%

10\%

12\%

14\%

16\%

18\%

20\%

22\%

24\%

26\%

28\%

30\%

\hypertarget{february-2020-2}{%
\subsubsection{February 2020}\label{february-2020-2}}

\hypertarget{june-2020-2}{%
\subsubsection{June 2020}\label{june-2020-2}}

Until now, some of the worst pain of the recession has been eased in
these neighborhoods by a major federal expansion of unemployment
benefits, including weekly \$600 supplemental payments to millions of
workers. Research shows that this aid significantly lifted the
\href{https://www.nytimes.com/live/2020/07/16/business/stock-market-today-coronavirus\#adding-600-to-weekly-jobless-pay-is-found-to-be-an-economic-tonic}{spending
of unemployed workers}; the money from the government might well have
circulated through businesses in their own neighborhoods, too.

But those jobless benefits expired a few days ago. Now, as
\href{https://www.nytimes.com/2020/08/03/us/politics/congress-jobless-aid-talks-trump.html}{Congress
and the White House wrangle} over whether and how to extend the aid,
these maps offer one more insight: These are the neighborhoods where
workers --- and the businesses that depend on their spending --- would
most acutely suffer without more federal help.

\hypertarget{unemployment-in-the-place-where-you-live}{%
\subsubsection{Unemployment in the place where you
live}\label{unemployment-in-the-place-where-you-live}}

2\%

4\%

6\%

8\%

10\%

12\%

14\%

16\%

18\%

20\%

22\%

24\%

26\%

28\%

30\%

Note: Stark differences in unemployment along state lines may reflect
differences in state-level B.L.S. unemployment estimates incorporated
into the model.

Read 115 Comments

\begin{itemize}
\item
\item
\item
\item
\end{itemize}

Advertisement

\protect\hyperlink{after-bottom}{Continue reading the main story}

\hypertarget{site-index}{%
\subsection{Site Index}\label{site-index}}

\hypertarget{site-information-navigation}{%
\subsection{Site Information
Navigation}\label{site-information-navigation}}

\begin{itemize}
\tightlist
\item
  \href{https://help.nytimes.com/hc/en-us/articles/115014792127-Copyright-notice}{©~2020~The
  New York Times Company}
\end{itemize}

\begin{itemize}
\tightlist
\item
  \href{https://www.nytco.com/}{NYTCo}
\item
  \href{https://help.nytimes.com/hc/en-us/articles/115015385887-Contact-Us}{Contact
  Us}
\item
  \href{https://www.nytco.com/careers/}{Work with us}
\item
  \href{https://nytmediakit.com/}{Advertise}
\item
  \href{http://www.tbrandstudio.com/}{T Brand Studio}
\item
  \href{https://www.nytimes.com/privacy/cookie-policy\#how-do-i-manage-trackers}{Your
  Ad Choices}
\item
  \href{https://www.nytimes.com/privacy}{Privacy}
\item
  \href{https://help.nytimes.com/hc/en-us/articles/115014893428-Terms-of-service}{Terms
  of Service}
\item
  \href{https://help.nytimes.com/hc/en-us/articles/115014893968-Terms-of-sale}{Terms
  of Sale}
\item
  \href{https://spiderbites.nytimes.com}{Site Map}
\item
  \href{https://help.nytimes.com/hc/en-us}{Help}
\item
  \href{https://www.nytimes.com/subscription?campaignId=37WXW}{Subscriptions}
\end{itemize}
