Sections

SEARCH

\protect\hyperlink{site-content}{Skip to
content}\protect\hyperlink{site-index}{Skip to site index}

\hypertarget{comments}{%
\subsection{\texorpdfstring{\protect\hyperlink{commentsContainer}{Comments}}{Comments}}\label{comments}}

\href{}{Here's What Extreme Heat Looks Like: Profoundly
Unequal}\href{}{Skip to Comments}

The comments section is closed. To submit a letter to the editor for
publication, write to
\href{mailto:letters@nytimes.com}{\nolinkurl{letters@nytimes.com}}.

\hypertarget{heres-what-extreme-heat-looks-like-profoundly-unequal}{%
\section{Here's What Extreme Heat Looks Like: Profoundly
Unequal}\label{heres-what-extreme-heat-looks-like-profoundly-unequal}}

By \href{https://www.nytimes.com/by/somini-sengupta}{Somini
Sengupta}Aug. 6, 2020

\begin{itemize}
\item
\item
\item
\item
\item
  \emph{163}
\end{itemize}

\emph{KC Nwakalor}

\includegraphics{https://static01.nyt.com/packages/flash/multimedia/ICONS/transparent.png}

\includegraphics{https://static01.nyt.com/newsgraphics/2020/07/28/climate-heat-v2/assets/images/nigeria_top-2000.jpg}

\emph{Saumya Khandelwal}

\includegraphics{https://static01.nyt.com/packages/flash/multimedia/ICONS/transparent.png}

\includegraphics{https://static01.nyt.com/newsgraphics/2020/07/28/climate-heat-v2/assets/images/lucknow_top-2000.jpg}

\emph{Myrto Papadopoulos}

\includegraphics{https://static01.nyt.com/packages/flash/multimedia/ICONS/transparent.png}

\includegraphics{https://static01.nyt.com/newsgraphics/2020/07/28/climate-heat-v2/assets/images/greece_top-2000.jpg}

\begin{itemize}
\item
\item
\item
\end{itemize}

Earth is overheating. Millions are already feeling the pain.

For the past 60 years, every decade has been hotter than the last, and
2020 is poised to be among the hottest years ever.

The agony of extreme heat, though, is profoundly unequal.

\hypertarget{this-is-inequity-at-the-boiling-point}{%
\subsection{This Is Inequity at the Boiling
Point}\label{this-is-inequity-at-the-boiling-point}}

\hypertarget{-by-somini-sengupta-aug-7-2020}{%
\subsection{\texorpdfstring{ \textbf{By
\href{https://www.nytimes.com/by/somini-sengupta}{Somini Sengupta}} Aug.
7,
2020}{ By Somini Sengupta Aug. 7, 2020}}\label{-by-somini-sengupta-aug-7-2020}}

It was a record 125 degrees Fahrenheit in Baghdad in July, and 100
degrees above the Arctic Circle this June. Australia shattered its
summer heat records as wildfires, fueled by prolonged drought, turned
the sky fever red.

For 150 years of industrialization, the combustion of coal, oil and gas
has steadily released heat-trapping gases into the atmosphere, driving
up average global temperatures and setting heat records. Nearly
everywhere around the world,
\href{https://climateextremes.org.au/heatwave-trends-accelerate-worldwide/}{heat
waves are more frequent and longer lasting} than they were 70 years ago.

But a hotter planet does not hurt equally. If you're poor and
marginalized, you're likely to be much more vulnerable to extreme heat.
You might be unable to afford an air-conditioner, and you might not even
have electricity when you need it. You may have no choice but to work
outdoors under a sun so blistering that first your knees feel weak and
then delirium sets in. Or the heat might bring a drought so punishing
that, no matter how hard you work under the sun, your corn withers and
your children turn to you in hunger.

It's not like you can just pack up and leave. So you plant your corn
higher up the mountain. You bathe several times a day if you can afford
the water. You powder your baby to prevent heat rash. You sleep outdoors
when the power goes out, slapping mosquitoes. You sit in front of a fan
by yourself, cursed by the twin dangers of isolation and heat.

Extreme heat is not a future risk. It's now. It endangers human health,
food production and the \href{https://www.nber.org/papers/w27599}{fate
of entire economies}. And it's worst for those at the bottom of the
economic ladder in their societies. See what it's like to live with one
of the most dangerous and stealthiest hazards of the modern era.

\emph{Photographs by Myrto Papadopoulos in}
\protect\hyperlink{greece}{Athens}\emph{, Ilana Panich-Linsman in}
\protect\hyperlink{houston}{Houston}\emph{, KC Nwakalor in the}
\protect\hyperlink{nigeria}{Niger Delta, Nigeria}\emph{, Daniele Volpe
in} \protect\hyperlink{guatemala}{Jocotán, Guatemala}\emph{, Saumya
Khandelwal in} \protect\hyperlink{india}{Lucknow, India}\emph{, and Juan
Arredondo in} \protect\hyperlink{newyork}{New York City}\emph{.}

Days per year above 95°F

≤1

10

50

100

200

The average number of extremely hot days expected per year between 2020
to 2039 under a moderate warming scenario.

By The New York Times \textbar{} Source: Climate Impact Lab

Greece

United States

China

India

Guatemala

Nigeria

Brazil

Australia

Days per year above 95°F

≤1

10

50

100

200

The average number of extremely hot days expected per year between 2020
to 2039 under a moderate warming scenario.

By The New York Times \textbar{} Source: Climate Impact Lab

Greece

United States

China

Atlantic Ocean

Pacific Ocean

India

Guatemala

Nigeria

Brazil

Pacific Ocean

Indian Ocean

Australia

Days per year above 95°F

100

200

≤1

10

50

The average number of extremely hot days expected per year between 2020
to 2039 under a moderate warming scenario.

By The New York Times

Source: Climate Impact Lab

Greece

United States

Atlantic Ocean

China

Pacific Ocean

India

Guatemala

Nigeria

Brazil

Pacific Ocean

Indian Ocean

Australia

Days per year above 95°F

200

≤1

10

50

100

The average number of extremely hot days expected per year between 2020
to 2039 under a moderate warming scenario.

By The New York Times \textbar{} Source: Climate Impact Lab

\hypertarget{heat-waves-are-becoming-more-frequent-in-athens-its-toughest-in-the-citys-treeless-concrete-neighborhoods}{%
\subsection{Heat waves are becoming more frequent in Athens. It's
toughest in the city's treeless, concrete
neighborhoods.}\label{heat-waves-are-becoming-more-frequent-in-athens-its-toughest-in-the-citys-treeless-concrete-neighborhoods}}

\includegraphics{https://static01.nyt.com/packages/flash/multimedia/ICONS/transparent.png}

\includegraphics{https://static01.nyt.com/images/2020/08/06/climate/06heat-intro-2/merlin_175139394_f75a6930-1d83-44fd-8c56-7334963755c5-master1050.jpg}

\hypertarget{photographs-by}{%
\subsection{Photographs by}\label{photographs-by}}

Myrto Papadopoulos

\includegraphics{https://static01.nyt.com/packages/flash/multimedia/ICONS/transparent.png}

\includegraphics{https://static01.nyt.com/newsgraphics/2020/07/28/climate-heat-v2/assets/images/globe-athens4-2000.png}

Hasib Hotak, 21, has been sleeping on a rooftop in Athens. To be
precise, he has been sleeping on a carpet, under the stars, on a rooftop
in Athens. There's a small room on the roof, with a sheet of corrugated
tin on top and a curtain for a door. The heat of the day turns it into
an oven. It is suffocatingly hot to sleep inside. It belongs to a friend
who, like Mr. Hotak, is a homeless Afghan refugee, and who sleeps on a
bed on the roof, draped with a mosquito net.

In late July, peak summer in Athens, the sun burned the rooftop by
midday. Mr. Hotak walked through the city to one of Athens's largest
public parks, Pedion Areos. Some days, he volunteered with an aid group
that gives out sandwiches to homeless refugees like him. Other days, he
sat under a wide-armed tree and scrolled through his phone. There aren't
a lot of places where a young Afghan man feels welcome in Athens, he
said. Once, he and a friend went to a cafe, hoping to chat over a cup of
coffee, only to be thrown out. The owner said Greeks wouldn't patronize
his establishment if they saw refugees at a table.

\includegraphics{https://static01.nyt.com/packages/flash/multimedia/ICONS/transparent.png}

\includegraphics{https://static01.nyt.com/newsgraphics/2020/07/28/climate-heat-v2/assets/images/heat_greece_07-2000.JPG}

\includegraphics{https://static01.nyt.com/packages/flash/multimedia/ICONS/transparent.png}

\includegraphics{https://static01.nyt.com/newsgraphics/2020/07/28/climate-heat-v2/assets/images/heat_greece_10-2000.jpg}

\includegraphics{https://static01.nyt.com/packages/flash/multimedia/ICONS/transparent.png}

\includegraphics{https://static01.nyt.com/newsgraphics/2020/07/28/climate-heat-v2/assets/images/heat_greece_28-2000.JPG}

\begin{itemize}
\item
\item
\item
\end{itemize}

Hasib Hotak set up a makeshift home on a roof in the Kolonos area of
Athens.

He spent much of his time in Athens seeking shade with other Afghan
refugees.

With no running water on the roof, showers on the beach offered a chance
to wash.

Mr. Hotak was 16 when he left his home in the Sholgara district of
Afghanistan, the only one among his 11 brothers and sisters to do so.
After one failed attempt to enter Europe and two years in a refugee
camp, he was granted asylum in Greece. That's when he arrived on the
rooftop refuge with a friend, in the crowded warrens of Kolonos, a
working class Athens neighborhood where many migrants have settled.

The city has grown hotter by the decade. According to temperature
records kept by the National Observatory of Athens, there were fewer
than 20 hot days (with temperatures over 99 degrees Fahrenheit, or 37
Celsius) per year in the first decade of the 1900s. By the mid-1980s,
there were still fewer than 50 hot days. Between 2006 and 2017, though,
the number had risen to 120 hot days.

\includegraphics{https://static01.nyt.com/packages/flash/multimedia/ICONS/transparent.png}

\includegraphics{https://static01.nyt.com/images/2020/08/06/climate/06heat-greece-07_v3/06-heat-greece-07_v3-master1050.jpg}

Mr. Hotak cooled down at an Athens beach. Heat waves have increased
fivefold in the city over the last century.

\includegraphics{https://static01.nyt.com/packages/flash/multimedia/ICONS/transparent.png}

\includegraphics{https://static01.nyt.com/images/2020/08/06/climate/06heat-greece-09/merlin_175139385_0235ea67-eb99-4a2f-8eab-91c7ee007c5c-master1050.jpg}

``I don't feel welcome in the country. Whenever I go out people look at
me like I'm a refugee. I don't want that. I'm human.''

\includegraphics{https://static01.nyt.com/packages/flash/multimedia/ICONS/transparent.png}

\includegraphics{https://static01.nyt.com/images/2020/08/06/climate/06heat-greece-08/merlin_175139358_fa3ff53a-8273-4080-96d4-ce3200643a5a-master1050.jpg}

\hypertarget{houston-is-getting-hotter-fast-staying-cool-is-an-unaffordable-luxury-for-the-rodriguez-family}{%
\subsection{Houston is getting hotter, fast. Staying cool is an
unaffordable luxury for the Rodriguez
family.}\label{houston-is-getting-hotter-fast-staying-cool-is-an-unaffordable-luxury-for-the-rodriguez-family}}

\includegraphics{https://static01.nyt.com/packages/flash/multimedia/ICONS/transparent.png}

\includegraphics{https://static01.nyt.com/images/2020/08/06/climate/06heat-houston-01/06heat-houston-01-master1050-v2.jpg}

\hypertarget{photographs-by-1}{%
\subsection{Photographs by}\label{photographs-by-1}}

Ilana Panich-Linsman

\includegraphics{https://static01.nyt.com/packages/flash/multimedia/ICONS/transparent.png}

\includegraphics{https://static01.nyt.com/newsgraphics/2020/07/28/climate-heat-v2/assets/images/globe-houston2-2000.png}

The air conditioner in her room gives Norma Rodriguez some breathing
space at the end of a long day.

At 18, just out of high school, Ms. Rodriguez is working two jobs to
help her family. One at a shoe store, the other at a restaurant. Her
father, Candelario Rodriguez, a roofer by profession, is unemployed. The
family's truck has broken down, so she has to hustle for rides. Her
mother, Dominga, is a part-time housekeeper in a nearby hotel where
business is slow. Her brother, Noe, 9, is on summer vacation from
school. Money is tight. Bills are juggled. Windows are covered during
the day to keep out the sun. Air-conditioners are turned on only at
night. Showers are limited to every other day.

The summer air is steamy in Houston. Even when you move slowly, you drip
with sweat. When you're working outdoors, in construction, as Norma's
father used to before the pandemic, sweat pools in your work boots.
Three of his co-workers have collapsed from heat exhaustion over the
years.

\includegraphics{https://static01.nyt.com/packages/flash/multimedia/ICONS/transparent.png}

\includegraphics{https://static01.nyt.com/newsgraphics/2020/07/28/climate-heat-v2/assets/images/heat_houston_03-2000.jpg}

\includegraphics{https://static01.nyt.com/packages/flash/multimedia/ICONS/transparent.png}

\includegraphics{https://static01.nyt.com/newsgraphics/2020/07/28/climate-heat-v2/assets/images/heat_houston_02-2000.jpg}

\includegraphics{https://static01.nyt.com/packages/flash/multimedia/ICONS/transparent.png}

\includegraphics{https://static01.nyt.com/newsgraphics/2020/07/28/climate-heat-v2/assets/images/heat_houston_6347_v2-2000.jpg}

\begin{itemize}
\item
\item
\item
\end{itemize}

Norma Rodriguez's father, Candelario, a roofer, is unemployed but keeps
busy, still outside in the heat, helping neighbors.

Her mother, Dominga, works three days a week as a housekeeper in a hotel
near the family home.

Nightfall offers a small respite from the heat.

The perils of the past haunt them. Their East Houston neighborhood, home
to mainly Latinos like the Rodriguez family, was hit particularly hard
by Hurricane Harvey. The heat packed into the atmosphere brought
exceptionally heavy rains, flooding the Rodriguez's two-bedroom trailer
and a car. They waded through floodwaters to be rescued by an 18-wheeler
truck, Norma carrying a pet chicken and a cat in her backpack, and
Dominga, who can't swim, wearing a life jacket. ``This year,'' Dominga
said, ``we just hope there isn't another hurricane.'' Hurricane Hanna
came close in July, but spared the city.

Houston is one of the
\href{https://www.climatecentral.org/news/report-american-warming-us-heats-up-earth-da}{country's
fastest warming cities}. Average temperatures have risen by more than
3.5 degrees Fahrenheit since 1970. In mid-July, the city's heat index
peaked above 110 degrees. It offered a glimpse of the future. If
emissions of greenhouse gases continue to rise at their current pace,
Houston could see 109 days each year, on average, where the heat index
tops 100 degrees. ****

\includegraphics{https://static01.nyt.com/packages/flash/multimedia/ICONS/transparent.png}

\includegraphics{https://static01.nyt.com/images/2020/08/06/climate/06heat-houston-05/06heat-houston-05-master1050-v2.jpg}

\includegraphics{https://static01.nyt.com/packages/flash/multimedia/ICONS/transparent.png}

\includegraphics{https://static01.nyt.com/images/2020/08/06/climate/06heat-houston-08/06heat-houston-08-master1050-v2.jpg}

Norma worked on her college orientation one evening in July when the
air-conditioner was turned on. She plans to begin as a freshman in the
fall.

\includegraphics{https://static01.nyt.com/packages/flash/multimedia/ICONS/transparent.png}

\includegraphics{https://static01.nyt.com/images/2020/08/06/climate/06heat-houston-09/merlin_175131228_ca8d070c-8efd-4309-9c85-f753afc10d8f-master1050.jpg}

Her younger brother, Noe, had his own idea to keep the family cool: a
squirt gun mounted on his bike.

\includegraphics{https://static01.nyt.com/packages/flash/multimedia/ICONS/transparent.png}

\includegraphics{https://static01.nyt.com/images/2020/08/06/climate/06heat-houston-10/06heat-houston-10-master1050-v2.jpg}

Ms. Rodriguez, who plans to attend community college this fall, has more
immediate concerns. A co-worker at the restaurant where she works
complained of symptoms of the coronavirus in late July. He said he would
get tested, and she hasn't heard from him since. She needs the job, but
she worries about getting sick --- and even more about infecting her
family.

``If I bring it home to my family, it's something else for me to worry
about,'' Ms. Rodriguez said over the weekend. ``If we're all going to be
sick, who's going to take care of us?''

\includegraphics{https://static01.nyt.com/packages/flash/multimedia/ICONS/transparent.png}

\includegraphics{https://static01.nyt.com/images/2020/08/06/climate/06heat-intro-3/merlin_175131063_133a4049-3a63-49c4-afc0-3e53b85ff30a-master1050.jpg}

\hypertarget{in-nigeria-rising-temperatures-are-supercharged-by-nonstop-gas-flares-you-can-feel-them-singe-the-skin}{%
\subsection{In Nigeria, rising temperatures are supercharged by nonstop
gas flares. You can feel them singe the
skin.}\label{in-nigeria-rising-temperatures-are-supercharged-by-nonstop-gas-flares-you-can-feel-them-singe-the-skin}}

\includegraphics{https://static01.nyt.com/packages/flash/multimedia/ICONS/transparent.png}

\includegraphics{https://static01.nyt.com/images/2020/08/06/climate/06heat-nigeria-01/06heat-nigeria-01-master1050-v2.jpg}

\hypertarget{photographs-by-2}{%
\subsection{Photographs by}\label{photographs-by-2}}

KC Nwakalor

\includegraphics{https://static01.nyt.com/packages/flash/multimedia/ICONS/transparent.png}

\includegraphics{https://static01.nyt.com/newsgraphics/2020/07/28/climate-heat-v2/assets/images/globe-nigeria2-2000.png}

Darkness never falls on Faith Osi's village.

Five tall methane gas flares loom over Obrikom, in the heart of the
oil-rich delta in southeastern Nigeria. They are part of the huge
petrochemicals operations of the Italian multinational Agip, and they
burn 24 hours a day, like blowtorches through the steamy tropical air.

It's normally hot here. Temperatures reach 91 degrees Fahrenheit on
average in the hot season and drop only slightly in the rainy months.
The flares make it even hotter, even at night, and particularly if
you're too poor to live anywhere other than within a few hundred meters
of the flares, where land is cheaper. One study found that
\href{https://www.aaas.org/resources/eyes-nigeria-technical-report/gas-flaring}{temperatures
were 22 degrees Fahrenheit higher around homes closest to the gas
flares}.

For decades, oil extraction has poisoned the air, land and water of the
Niger Delta region, while its people have reaped little by way of jobs
or development in the area.

\includegraphics{https://static01.nyt.com/packages/flash/multimedia/ICONS/transparent.png}

\includegraphics{https://static01.nyt.com/newsgraphics/2020/07/28/climate-heat-v2/assets/images/heat_nigeria_4-2000.jpg}

\includegraphics{https://static01.nyt.com/packages/flash/multimedia/ICONS/transparent.png}

\includegraphics{https://static01.nyt.com/newsgraphics/2020/07/28/climate-heat-v2/assets/images/heat_nigeria_19-2000.jpg}

\includegraphics{https://static01.nyt.com/packages/flash/multimedia/ICONS/transparent.png}

\includegraphics{https://static01.nyt.com/newsgraphics/2020/07/28/climate-heat-v2/assets/images/heat_nigeria_09-2000.jpg}

\begin{itemize}
\item
\item
\item
\end{itemize}

Faith Osi says she worries constantly about heat rash on her youngest
child, a girl named Miracle.

The family home, within clear view of gas flares that burn constantly.

Ms. Osi's older children, Ada, 16, and Promise, 14, helped with
dishwashing.

Heat is arguably the least understood of these threats. It's everyday.
It's invisible. And, for Ms. Osi, who is in her mid-30s, it's
exhausting.

It saps her. She can barely work for three hours a day on her cassava
farm, and even then, she feels like she can hardly breathe. Headaches
torment her often. Relief comes only from a bucket of cold water over
her head.

It was worse when she was pregnant. She would pat her belly with a wet
cloth. At night she would lay on the bare floor. The bed was too hot.
She would barely sleep.

The other day, with the air clammy from the rains, she bathed the
youngest of her eight children, Miracle, who is 1 year old. In minutes,
Miracle was glistening with sweat and screaming from discomfort. Ms. Osi
worried about heat rash. She emptied nearly a can of talcum powder on
the baby.

\includegraphics{https://static01.nyt.com/packages/flash/multimedia/ICONS/transparent.png}

\includegraphics{https://static01.nyt.com/images/2020/08/06/climate/06heat-nigeria-05/merlin_174916359_1000d229-069d-47be-b73e-6609082ccb94-master1050.jpg}

Her husband, Azubuike Osi, 42, turned to cigarettes for relief. The kids
flapped their clothes to air their bodies. They all try to sleep under
the fan, at least until the electricity goes out, which it sometimes
does on the hottest nights, and then some of the children sleep outside
on the balcony, battling mosquitoes. Malaria is rampant.

The dangerous extremes of climate change are already affecting Nigeria's
poorest people.
\href{https://rmets.onlinelibrary.wiley.com/doi/10.1002/met.1791}{Hotter
days and hotter nights} are more frequent, while the number of cool days
and nights has decreased,
\href{https://www.ipcc.ch/site/assets/uploads/2018/02/WGIIAR5-Chap22_FINAL.pdf}{a
trend that studies have observed throughout West Africa}.

\includegraphics{https://static01.nyt.com/packages/flash/multimedia/ICONS/transparent.png}

\includegraphics{https://static01.nyt.com/images/2020/08/06/climate/06heat-nigeria-08/06heat-nigeria-08-master1050-v2.jpg}

A bath gave Miracle a moment of cool comfort.

\includegraphics{https://static01.nyt.com/packages/flash/multimedia/ICONS/transparent.png}

\includegraphics{https://static01.nyt.com/images/2020/08/06/climate/06heat-nigeria-06/06heat-nigeria-06-master1050-v2.jpg}

When the power goes out, some of Ms. Osi's children sleep on the
balcony.

\includegraphics{https://static01.nyt.com/packages/flash/multimedia/ICONS/transparent.png}

\includegraphics{https://static01.nyt.com/images/2020/08/06/climate/06heat-nigeria-07/merlin_174917028_0ec83666-14f0-43c8-8fd3-75ba3fc3236a-master1050.jpg}

\hypertarget{the-dry-season-is-getting-longer-and-drier-in-guatemala-indigenous-farmers-could-see-crop-yields-fall-sharply}{%
\subsection{The dry season is getting longer and drier in Guatemala.
Indigenous farmers could see crop yields fall
sharply.}\label{the-dry-season-is-getting-longer-and-drier-in-guatemala-indigenous-farmers-could-see-crop-yields-fall-sharply}}

\includegraphics{https://static01.nyt.com/packages/flash/multimedia/ICONS/transparent.png}

\includegraphics{https://static01.nyt.com/images/2020/08/06/climate/06heat-guatemala-01/merlin_175092447_3159861b-9ccd-42ce-8f7b-4d98d82228d3-master1050.jpg}

\hypertarget{photographs-by-3}{%
\subsection{Photographs by}\label{photographs-by-3}}

\emph{Daniele Volpe}

\includegraphics{https://static01.nyt.com/packages/flash/multimedia/ICONS/transparent.png}

\includegraphics{https://static01.nyt.com/newsgraphics/2020/07/28/climate-heat-v2/assets/images/globe-guatemala2-2000.png}

Eduardo Roque, 38, is among Guatemala's original people, part of the
Ch'orti Mayan community living in one of the poorest and driest corners
of the Americas, known as the Dry Corridor.

Rising temperatures are ravaging the land.

The early summer rains that nourish his small fields have diminished
measurably in recent years, according to scientists, and five long and
harsh late summer droughts have cursed this region in the last decade.
The country as a whole is warmer by about 1.8 degrees Fahrenheit since
1960, with far more frequent hot days and nights. The rains don't come
when he needs them for his crop, Mr. Roque says. ``When we need the sun,
suddenly, we are receiving water.''

\includegraphics{https://static01.nyt.com/packages/flash/multimedia/ICONS/transparent.png}

\includegraphics{https://static01.nyt.com/newsgraphics/2020/07/28/climate-heat-v2/assets/images/heat_guatemala_2856-2000.jpg}

\includegraphics{https://static01.nyt.com/packages/flash/multimedia/ICONS/transparent.png}

\includegraphics{https://static01.nyt.com/newsgraphics/2020/07/28/climate-heat-v2/assets/images/heat_guatemala_9582-2000.jpg}

\includegraphics{https://static01.nyt.com/packages/flash/multimedia/ICONS/transparent.png}

\includegraphics{https://static01.nyt.com/newsgraphics/2020/07/28/climate-heat-v2/assets/images/heat_guatemala_1522-2000.jpg}

\begin{itemize}
\item
\item
\item
\end{itemize}

Eduardo Roque and his neighbors planted corn this season even though
drought hit them hard the previous three years.

He earns about \$5 a day working on a coffee field.

Everyone in the family pitches in to make ends meet. His sons hauled
firewood home.

Mr. Roque's harvests of corn and beans, staple foods, failed three years
in a row. Desperate, he hustled for work in the capital, Guatemala City,
bought a patch of land near a small creek, planted rows of corn there.
On his old corn fields, he has planted trees, and in their shade, he is
trying coffee.

Malnutrition runs higher in the largely indigenous region, called
Chiquimula, where Mr. Roque lives with his wife and nine children. Water
has to be rationed.

The amount of greenhouse gases emitted by the average Guatemalan each
year is tiny ---
\href{https://data.worldbank.org/indicator/EN.ATM.CO2E.PC?locations=GT}{}
\href{https://data.worldbank.org/indicator/EN.ATM.CO2E.PC?locations=GT}{1.1
metric tons, compared with 16.5 tons per person in the United States}
--- and Mr. Roque's carbon footprint is, very likely, smaller still.
Electricity came to his village only recently. The family doesn't have a
car, motorcycle or tractor. He built his house by hand, from mud, with
only a few pillars of concrete.

\includegraphics{https://static01.nyt.com/packages/flash/multimedia/ICONS/transparent.png}

\includegraphics{https://static01.nyt.com/images/2020/08/06/climate/06heat-guatemala-05/merlin_175092423_33b18c47-9656-48dc-9a4c-d2204db241ba-master1050.jpg}

But Guatemala is poised to feel the effect of a hotter planet acutely.
Yields of maize and beans could fall by around 14 percent by 2050,
according to
\href{https://www.ifpri.org/publication/climate-change-agriculture-and-adaptation-options-guatemala}{}
\href{https://www.ifpri.org/publication/climate-change-agriculture-and-adaptation-options-guatemala}{a
recent study}; coffee grown in lower elevations is unlikely to be
``economically viable.''

Climate models project longer dry periods in the future.

``The models show that this should happen in the next decades,'' said
Edwin Castellanos, director of the center for environmental studies at
the University of the Valley of Guatemala and a co-author of the study,
``but it's already happening.''

\includegraphics{https://static01.nyt.com/packages/flash/multimedia/ICONS/transparent.png}

\includegraphics{https://static01.nyt.com/images/2020/08/06/climate/06heat-guatemala-06/merlin_175092207_8f79acf6-d1ca-4b4d-8329-d5f0c8d675d0-master1050.jpg}

Eduardo's grandparents, Magdaleno Roque, 78, and Eulofia Aldana, 64,
live near his home.

\includegraphics{https://static01.nyt.com/packages/flash/multimedia/ICONS/transparent.png}

\includegraphics{https://static01.nyt.com/images/2020/08/06/climate/06heat-guatemala-07/merlin_175092651_496c8ddb-fec2-49a2-a7c7-198fa4cf82fa-master1050.jpg}

A small but solid defender: Eduardo Roque, left, filmed his son Eddy, 2,
on a parched soccer field.

\includegraphics{https://static01.nyt.com/packages/flash/multimedia/ICONS/transparent.png}

\includegraphics{https://static01.nyt.com/images/2020/08/06/climate/06heat-guatemala-08/06heat-guatemala-08-master1050-v2.jpg}

\hypertarget{india-is-already-hot-an-increase-of-just-a-few-degrees-can-be-dangerous-for-people-who-work-outdoors}{%
\subsection{India is already hot. An increase of just a few degrees can
be dangerous for people who work
outdoors.}\label{india-is-already-hot-an-increase-of-just-a-few-degrees-can-be-dangerous-for-people-who-work-outdoors}}

\includegraphics{https://static01.nyt.com/packages/flash/multimedia/ICONS/transparent.png}

\includegraphics{https://static01.nyt.com/images/2020/08/06/climate/06heat-india-10/merlin_175328292_a249f4bd-a138-4595-a7d5-6ff70beedb76-master1050.jpg}

\hypertarget{photographs-by-4}{%
\subsection{Photographs by}\label{photographs-by-4}}

Saumya Khandelwal

\includegraphics{https://static01.nyt.com/packages/flash/multimedia/ICONS/transparent.png}

\includegraphics{https://static01.nyt.com/newsgraphics/2020/07/28/climate-heat-v2/assets/images/globe-lucknow2-2000.png}

Rabita bends down, fills a bowl with sand, lifts it atop her head,
climbs up and down the stairs. Up and down, countless times each day,
even as the heat rises through the morning and the air gets sticky. Her
legs ache from the climbing. Her head spins sometimes. Breaks can't be
longer than five minutes, or she'll get a hectoring from the foreman on
the construction site. Occasionally, she comes down with a fever and has
to take a day off. When she's on her period, it's the worst.

The other day, she tried to shake the sand off herself, to no avail. The
sweat had glued the sand to her skin.

\includegraphics{https://static01.nyt.com/packages/flash/multimedia/ICONS/transparent.png}

\includegraphics{https://static01.nyt.com/newsgraphics/2020/07/28/climate-heat-v2/assets/images/heat_lucknow_working-2000.jpg}

\includegraphics{https://static01.nyt.com/packages/flash/multimedia/ICONS/transparent.png}

\includegraphics{https://static01.nyt.com/newsgraphics/2020/07/28/climate-heat-v2/assets/images/heat_lucknow_2657-2000.jpg}

\includegraphics{https://static01.nyt.com/packages/flash/multimedia/ICONS/transparent.png}

\includegraphics{https://static01.nyt.com/newsgraphics/2020/07/28/climate-heat-v2/assets/images/heat_lucknow_child2-2000.jpg}

\begin{itemize}
\item
\item
\item
\end{itemize}

Rabita and her husband, Ashok Kumar, live and work at a construction
site in Lucknow.

Breaks are short and the heat often makes her feel ill.

Rabita's youngest son, Sumari, also spends his days at the site.

Rabita, who does not use a surname, is helping to build a government
housing project. She and her husband, Ashok Kumar, are Dalits, at the
bottom of the Hindu caste ladder. They own no land in their village in
Bihar, which has long been one of the most terrifying places to be a
Dalit. They work on other people's lands, when there is work, and Rabita
gets paid less than half what a man makes.

And then there's the extreme vagaries of the rain. It rains when it's
not supposed to, she says, and washes away the crops. People like her
have to leave home to put food in their stomachs.

For years, Mr. Kumar had been working hauling sacks of vegetables at a
city market, sending home money. The pandemic changed all that. Mr.
Kumar came back home, borrowed money to make ends meet. Now he and
Rabita work to pay off those debts. Their oldest son, Guddu, 15, works
alongside them. Their 3-year-old, Sumari, hangs around.

\includegraphics{https://static01.nyt.com/packages/flash/multimedia/ICONS/transparent.png}

\includegraphics{https://static01.nyt.com/images/2020/08/06/climate/06heat-india-04/merlin_175251276_ab7f98a2-408e-45c6-9b99-d7b31f27657c-master1050.jpg}

Episodes of extreme humid heat at levels the human body cannot tolerate
for many hours at a time have more than doubled in frequency since 1979,
\href{https://blogs.ei.columbia.edu/2020/05/08/fatal-heat-humidity-emerging/?shareadraft=baba77757_5e8c736da48a5}{according
to a recent scientific paper}. South Asia and the Gulf Coast of the
United States are among the places hardest hit. Sweat can't evaporate as
fast. The body can't cool down.

The International Labor Organization calls
\href{https://www.ilo.org/wcmsp5/groups/public/---dgreports/---dcomm/---publ/documents/publication/wcms_711919.pdf}{}
\href{https://www.ilo.org/wcmsp5/groups/public/---dgreports/---dcomm/---publ/documents/publication/wcms_711919.pdf}{heat
an occupational health hazard}, with construction workers like Rabita
especially vulnerable. Most people can work only at half their capacity
when temperatures exceed 91 degrees Fahrenheit, and exposure to many
hours of heat can be fatal, the group warns.

Economic losses from heat stress are projected to increase to \$2.4
trillion in 2030. But this cost, too, is expected to be unequally
spread.

South Asia and West Africa are expected to be the worst affected, not
just because of high heat and humidity, but because of how vulnerable
laborers like Rabita are to begin with.

\includegraphics{https://static01.nyt.com/packages/flash/multimedia/ICONS/transparent.png}

\includegraphics{https://static01.nyt.com/images/2020/08/06/climate/06heat-india-09/06heat-india-09-master1050.jpg}

Watching sports on a phone on a day when the humidity made it feel like
107 Fahrenheit.

\includegraphics{https://static01.nyt.com/packages/flash/multimedia/ICONS/transparent.png}

\includegraphics{https://static01.nyt.com/images/2020/08/06/climate/06heat-india-06/merlin_175251303_05caea7e-7c84-4908-8ba9-8311c3718bb7-master1050.jpg}

Rabita slept on a day off. The high for the day was 95 Fahrenheit.

\includegraphics{https://static01.nyt.com/packages/flash/multimedia/ICONS/transparent.png}

\includegraphics{https://static01.nyt.com/images/2020/08/06/climate/06heat-india-07/merlin_175268157_9402e5e2-2f33-4414-89c0-f8637a8777b7-master1050.jpg}

\hypertarget{heat-is-the-deadliest-form-of-extreme-weather-for-older-americans-in-new-york-city-isolation-is-its-sly-accomplice}{%
\subsection{Heat is the deadliest form of extreme weather for older
Americans. In New York City, isolation is its sly
accomplice.}\label{heat-is-the-deadliest-form-of-extreme-weather-for-older-americans-in-new-york-city-isolation-is-its-sly-accomplice}}

\includegraphics{https://static01.nyt.com/packages/flash/multimedia/ICONS/transparent.png}

\includegraphics{https://static01.nyt.com/images/2020/08/06/climate/06heat-newyork-01/06heat-newyork-01-master1050-v2.jpg}

\hypertarget{photographs-by-5}{%
\subsection{Photographs by}\label{photographs-by-5}}

Juan Arrendondo

\includegraphics{https://static01.nyt.com/packages/flash/multimedia/ICONS/transparent.png}

\includegraphics{https://static01.nyt.com/newsgraphics/2020/07/28/climate-heat-v2/assets/images/globe-nyc2-2000.png}

On a sweltering Sunday in July, with temperatures soaring to 93
Fahrenheit, Rafael Velasquez, 66, sat in the courtyard of his apartment
complex with a cold bottle of water pressed to his face. He liked
sitting outside on a bench, feeding the pigeons. He kept a hand towel to
wipe the sweat.

There wasn't much to do inside. He's lived alone since his wife died a
couple of years ago. He can't afford to buy an air-conditioner, and he
said he had no idea how to get a free one from a city program designed
to help seniors stay cool during the pandemic, when cooling centers are
mostly closed.

He had a window fan in the living room, and one standing fan that he
dragged from the bedroom to the living room every morning. Mostly, he
watched stuff on his phone. He can't afford cable.

\includegraphics{https://static01.nyt.com/packages/flash/multimedia/ICONS/transparent.png}

\includegraphics{https://static01.nyt.com/images/2020/08/06/climate/06heat-newyork-02/merlin_175089870_66f37f12-f60a-424a-b0e8-864881b12013-master1050.jpg}

In the United States, heat kills older people more than any other
extreme weather event, including hurricanes, and the problem is part of
an ignominious national pattern: Black people and Latinos like Mr.
Velasquez are \href{https://ehp.niehs.nih.gov/doi/10.1289/ehp.1205919}{}
\href{https://ehp.niehs.nih.gov/doi/10.1289/ehp.1205919}{far more likely
to live in the hottest parts} of American cities.

His neighborhood is exceptionally
\href{http://a816-dohbesp.nyc.gov/IndicatorPublic/HeatHub/hvi.html}{}
\href{http://a816-dohbesp.nyc.gov/IndicatorPublic/HeatHub/hvi.html}{vulnerable}
to heat extremes. According to the most recent available data, from
2018,
\href{http://a816-dohbesp.nyc.gov/IndicatorPublic/VisualizationData.aspx?id=2141,719b87,107,Summarize}{}
\href{http://a816-dohbesp.nyc.gov/IndicatorPublic/VisualizationData.aspx?id=2141,719b87,107,Summarize}{Brownsville
was among New York City's hottest}, with average daytime highs around
two degrees Fahrenheit higher than the city as a whole.

Those neighborhoods are often
\href{http://a816-dohbesp.nyc.gov/IndicatorPublic/HeatHub/hvi.html}{the
same areas that have faced some of the highest rates of coronavirus
deaths}. This spring, around 10 residents of Mr. Velasquez's senior
housing complex died from the virus.

``Inequality exacerbates climate and environmental risks,'' said Kizzy
Charles-Guzman, a deputy director for resilience efforts in the New York
City Mayor's office.

Isolation makes it worse.

With no one to check in on you, even a mild case of dehydration can take
a quick turn for the worse if you're frail or suffer from other
ailments, like heart disease. According to the Centers for Disease
Control and Prevention, 600 Americans die each year from extreme heat. A
recent academic study, though, estimated that as many
\href{https://www.climatecentral.org/news/2020-Heat-and-Seniors}{as
12,000 people may be dying} of heat-related ailments; 80 percent of
them, the researchers said, are over the age of 60.

\includegraphics{https://static01.nyt.com/packages/flash/multimedia/ICONS/transparent.png}

\includegraphics{https://static01.nyt.com/images/2020/08/06/climate/06heat-newyork-03/06heat-newyork-03-master1050-v2.jpg}

Mr. Velasquez, a retired hospital cook, spent most of his life in the
Bronx.

\includegraphics{https://static01.nyt.com/packages/flash/multimedia/ICONS/transparent.png}

\includegraphics{https://static01.nyt.com/images/2020/08/06/climate/06heat-newyork-04/merlin_175089780_8810576f-9aa4-4f64-b979-840e5d5a886f-master1050.jpg}

When he applied for senior housing, he was offered an apartment in the
Ocean Hill-Brownsville area of Brooklyn.

\includegraphics{https://static01.nyt.com/packages/flash/multimedia/ICONS/transparent.png}

\includegraphics{https://static01.nyt.com/images/2020/08/06/climate/06heat-newyork-05/merlin_175089840_01c48c19-fd07-4d30-8eaa-ca0c58472175-master1050.jpg}

Mr. Velasquez's five daughters live in the Bronx. He said he hadn't seen
them in months because of the pandemic.

The other day, when the city issued a heat alert and the senior center
on the ground floor opened for the first time in months as a cooling
center, he went to pick up two plastic bags of groceries: black beans,
breakfast cereal, peanut butter and other provisions that would last.

He won a round of bingo, and a roll of paper towels as a prize.

Additional reporting by Shola Lawal and Orji Sunday. Map by Nadja
Popovich. Designed and produced by Becky Lebowitz Hanger, Claire O'Neill
and Eden Weingart.

\textbf{Correction:}~Aug. 6, 2020

An earlier version of this article misspelled the given name of Norma
Rodriguez's father. He is Candelario Rodriguez, not Candelaria.

Read 163 Comments

\begin{itemize}
\item
\item
\item
\item
\end{itemize}

Advertisement

\protect\hyperlink{after-bottom}{Continue reading the main story}

\hypertarget{site-index}{%
\subsection{Site Index}\label{site-index}}

\hypertarget{site-information-navigation}{%
\subsection{Site Information
Navigation}\label{site-information-navigation}}

\begin{itemize}
\tightlist
\item
  \href{https://help.nytimes.com/hc/en-us/articles/115014792127-Copyright-notice}{©~2020~The
  New York Times Company}
\end{itemize}

\begin{itemize}
\tightlist
\item
  \href{https://www.nytco.com/}{NYTCo}
\item
  \href{https://help.nytimes.com/hc/en-us/articles/115015385887-Contact-Us}{Contact
  Us}
\item
  \href{https://www.nytco.com/careers/}{Work with us}
\item
  \href{https://nytmediakit.com/}{Advertise}
\item
  \href{http://www.tbrandstudio.com/}{T Brand Studio}
\item
  \href{https://www.nytimes.com/privacy/cookie-policy\#how-do-i-manage-trackers}{Your
  Ad Choices}
\item
  \href{https://www.nytimes.com/privacy}{Privacy}
\item
  \href{https://help.nytimes.com/hc/en-us/articles/115014893428-Terms-of-service}{Terms
  of Service}
\item
  \href{https://help.nytimes.com/hc/en-us/articles/115014893968-Terms-of-sale}{Terms
  of Sale}
\item
  \href{https://spiderbites.nytimes.com}{Site Map}
\item
  \href{https://help.nytimes.com/hc/en-us}{Help}
\item
  \href{https://www.nytimes.com/subscription?campaignId=37WXW}{Subscriptions}
\end{itemize}
