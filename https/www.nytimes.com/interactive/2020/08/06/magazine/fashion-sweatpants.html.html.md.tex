Sections

SEARCH

\protect\hyperlink{site-content}{Skip to
content}\protect\hyperlink{site-index}{Skip to site index}

\hypertarget{comments}{%
\subsection{\texorpdfstring{\protect\hyperlink{commentsContainer}{Comments}}{Comments}}\label{comments}}

\href{}{Sweatpants Forever: How the Fashion Industry
Collapsed}\href{}{Skip to Comments}

The comments section is closed. To submit a letter to the editor for
publication, write to
\href{mailto:letters@nytimes.com}{\nolinkurl{letters@nytimes.com}}.

\hypertarget{sweatpants-forever-how-the-fashion-industry-collapsed}{%
\section{Sweatpants Forever: How the Fashion Industry
Collapsed}\label{sweatpants-forever-how-the-fashion-industry-collapsed}}

By Irina AleksanderAug. 6, 2020

\begin{itemize}
\item
\item
\item
\item
\item
  \emph{461}
\end{itemize}

\includegraphics{https://static01.nyt.com/packages/flash/multimedia/ICONS/transparent.png}

\includegraphics{https://static01.nyt.com/newsgraphics/2020/08/02/entireworld/assets/images/topper-2000.jpg}

\hypertarget{sweatpants-forever}{%
\subsection{Sweatpants Forever}\label{sweatpants-forever}}

Even before the pandemic, the whole fashion industry\\
had started to unravel. What happens now that no one\\
has a reason to dress up? By Irina Aleksander\\
Photographs by Stephanie Gonot • August 6, 2020

It's difficult, in retrospect, to pinpoint when exactly panic about
coronavirus took hold in the United States, but March 12 stands out.
Stores ran out of canned goods. Streets emptied of cars. Tom Hanks had
just tested positive for the virus. That evening, Scott Sternberg, a
fashion designer, was lying awake at home in the Silver Lake
neighborhood of Los Angeles, thinking about Entireworld, a line of
basics he founded two years earlier. Would people still buy clothes? How
much cash did he have to keep going? When would he have to lay people
off? ``My Band of Outsiders battle scars just opened wide,'' he said.

\includegraphics{https://static01.nyt.com/newsgraphics/2020/08/02/entireworld/assets/images/marginalia/bandofoutsiders.png}
\includegraphics{https://static01.nyt.com/newsgraphics/2020/08/02/entireworld/assets/images/marginalia/michelleobama.png}

Band of Outsiders was Sternberg's previous company. He founded it in
2004 as a line of slim shirts and ties. (Remember the skinny-tie boom?
That was Sternberg.) Eventually it grew into full men's and women's
collections that won over the fashion world with self-consciously preppy
clothes. Sternberg took home two Council of Fashion Designers of America
(C.F.D.A.) awards, the industry's equivalent of the Oscars. He posed for
photos with Kanye West. Michelle Obama wore one of his dresses. He
opened stores in Tokyo and New York. Then, in 2015, to everyone's
surprise, Sternberg announced that Band was going out of business. An
investment with some Belgians had gone bad, but that didn't feel like
the whole story. Sternberg knew the whole story. Every choice he made at
Entireworld was to prevent it from happening again. Now a global
pandemic had hit. He couldn't foresee that. No one did.

Unlike other designers, Sternberg studied not design but economics, a
major he chose in part because the year he entered Washington University
in St. Louis, the economist Douglass North, a professor there, won a
Nobel Prize. Sternberg graduated summa cum laude. His senior thesis was
about the economics of actors in Hollywood, which is how he wound up in
Los Angeles in the first place. This is all to say that Sternberg knew
what uncertainty does to consumer behavior.

\hypertarget{listen-to-this-article}{%
\subsubsection{Listen to This Article}\label{listen-to-this-article}}

To hear more audio stories from publishers like The New York Times,
\href{https://www.audm.com/}{download Audm for iPhone or Android}.

``What was going through my head was: Man, I don't know how big
businesses are going to deal with this,'' he said. ``But for a small
business this is enough to take all of us out'' --- he snapped his
fingers --- ``in one shot.''

\includegraphics{https://static01.nyt.com/newsgraphics/2020/08/02/entireworld/assets/images/marginalia/dvf.png}

As it happened, it was the giants who would fall first. Over the next
few months, J. Crew, Neiman Marcus, Brooks Brothers and J.C. Penney
filed for bankruptcy. Gap Inc. couldn't pay rent on its 2,785 North
American stores. By July, Diane von Furstenberg announced she would lay
off 300 employees and close 18 of her 19 stores. The impending damage to
small businesses was inconceivable.

The next morning, a Friday, Sternberg drove to Entireworld's offices in
Koreatown. He sat down at his desk and began drafting an email: ``Wow. I
mean, WTF.''

He didn't run the email by his staff. There was no meeting about it. He
just sat down and wrote it.

``Am I sick already? Can I leave my house? What do I tell my employees?
Will my mom be OK on her flight home today? Can Zod'' --- Sternberg's
dog --- ``get coronavirus? Did I buy enough T.P.? How long will this
last? Who's in charge? What's next?''

\includegraphics{https://static01.nyt.com/newsgraphics/2020/08/02/entireworld/assets/images/marginalia/teleben.gif}

The email went out to the brand's 30,000 subscribers on Sunday, March
15. It was, in a sea of daily promotional emails, a distinctly human
one. But this was still a promotion: for a sweatsuit, the brand's top
seller, a ``hero item'' in industry speak. Inspired by a French
children's film, Entireworld's sweatsuits come in a prism of cheery
colors and, in Sternberg's vision, ``sort of make you look like a cross
between a Teletubbie, Ben Stiller in `The Royal Tenenbaums,' and a J.C.
Penney ad from 1979.''

It wasn't long before Sternberg's employees began texting him happy-face
emoji. On an average day, the brand --- still in its nascent stage ---
sells 46 sweats. That day they sold more than 1,000. When they ran out
of sweatsuits, shoppers moved through the T-shirts, socks and underwear.
By month's end, the brand's sales were up 662 percent over March the
previous year.

The day we met, April 24, was the highest-grossing day in the company's
history. A new shipment came in that morning and promptly sold out
again. Entireworld had now grossed more in two months than in its entire
first year in business.

\includegraphics{https://static01.nyt.com/newsgraphics/2020/08/02/entireworld/assets/images/marginalia/dov.png}
\includegraphics{https://static01.nyt.com/newsgraphics/2020/08/02/entireworld/assets/images/marginalia/lavendersock.gif}

By ``met,'' I mean that we were in Sternberg's backyard in chairs
positioned 20 feet apart, with a setup of disinfectant wipes between us.
At this point, Sternberg hadn't been leaving the house much, instead
subsisting on deliveries from BlueApron, the meal-kit service, and
rationing the ingredients into multiple meals. Entireworld's managing
director, Jordan Schiff --- formerly of Dov Charney's American Apparel,
whose heyday Sternberg's line openly pays homage to --- had just come
down with Covid-19. But he was still tracking the numbers. Just a few
days before, Schiff reported that the company had sold out of 600 pairs
of lavender women's socks.

Sternberg was in a good mood. This was obviously not just because of an
email. Nor was it simply because America had settled into sweatpants for
the foreseeable future. He'd been laying down this groundwork since Band
of Outsiders imploded. Entireworld wasn't a departure in name only,
suggesting as it does the opposite of the in crowd. It was also
Sternberg's rejection of the traditional fashion system, the one that
once vaulted him to success. No more fashion shows, no more seasonal
collections, no more wholesale accounts that had become unreliable
(R.I.P. Barneys) or the markups required to pay for it all. (Band's
shirts started at \$220; Entireworld's are \$95.)

\includegraphics{https://static01.nyt.com/packages/flash/multimedia/ICONS/transparent.png}

\includegraphics{https://static01.nyt.com/newsgraphics/2020/08/02/entireworld/assets/images/scott-2000.jpg}

Scott SternbergStephanie Gonot for The New York Times

For years, Sternberg had been saying that the fashion industry was a
giant bubble heading toward collapse. Now the pandemic was just speeding
up the inevitable. In fact, it had already begun. An incredible surplus
of clothing was presently sitting in warehouses and in stores, some of
which might never reopen. ``That whole channel is dead,'' Sternberg
said. ``And there's no sign of when it's turning on again.''

In April, clothing sales fell 79 percent in the United States, the
largest dive on record. Purchases of sweatpants, though, were up 80
percent. Entireworld was like the rare life form that survives the
apocalypse. By betting that the luxury market would fail, Sternberg had
evaded the very forces that were bringing down the rest of the industry.
``Because you could see the writing on the wall,'' he said. ``The
Neimans writing on the wall, the Barneys. ... Listen, Barneys? That was
not a shock to anyone.''

\includegraphics{https://static01.nyt.com/newsgraphics/2020/08/02/entireworld/assets/images/marginalia/marcjacobsinterview.gif}

If there's one image that I will remember from the last days of the
fashion industry as it has existed for the last two decades, it's Marc
Jacobs streaming live from the Mercer Hotel in New York in pearls and
perfect makeup. The broadcast ran to 75 minutes in length over two
different virtual events. It began on April 15, with Vogue's Global
Conversations, a series the magazine introduced to figure out how to fix
the fashion industry, and continued a month later, on May 15, with
Business of Fashion, the industry's go-to news website.

``I'm in the process of grief right now,'' Jacobs told Vogue.

\emph{Why are you grieving, Marc?} the moderator asked.

``Why? Because this is all very sad.''

Then, later: \emph{How are you going to present your spring/summer '21
collection?}

``I'm not sure there will be a spring/summer '21 collection.''

`This has been a very difficult business to be in for a long time, I
think.'

Jacobs had come to see his fall 2020 show as a kind of farewell. ``I've
said this to my psychiatrist, my lovely Dr. Richardson,'' he told
Business of Fashion, after taking a long drag from his vape pen, ``that
I would be very happy if that were my last show.'' That collection would
never be produced. Buyers couldn't place orders, and even if they had,
factories were shut down. Jacobs said he had to lay off ``a bunch of
people'' and ask others to take pay cuts. Not that this began with the
pandemic. Since 2013, Jacobs's business had shrunk from 250 stores to
just four. Speaking to Vogue, he said, ``This has been a very difficult
business to be in for a long time, I think.''

\includegraphics{https://static01.nyt.com/newsgraphics/2020/08/02/entireworld/assets/images/marginalia/wwd.jpg}
\includegraphics{https://static01.nyt.com/newsgraphics/2020/08/02/entireworld/assets/images/marginalia/wanglamlim.png}

Things looked different in 2005. I'm choosing that year somewhat
subjectively, because that's when I started as an intern at Women's Wear
Daily. It was a thrilling time in American fashion. A new guard of young
designers had just entered the scene, displacing the stars of the 1980s
and '90s (Donna Karan, Calvin Klein, Michael Kors, et al.) and
re-energizing the runways. Interns don't see much, but occasionally
fashion week invites trickle down. My first show was Zac Posen, in
something like Row 8. My second was Proenza Schouler. Those designers,
along with Alexander Wang, Derek Lam, Phillip Lim, Rag \& Bone, Rodarte,
Jason Wu and later Joseph Altuzarra, seemed to grow into global brands
overnight, with the help of store buyers and fashion editors eager to
usher in a post 9/11 generation of American talent.

\includegraphics{https://static01.nyt.com/newsgraphics/2020/08/02/entireworld/assets/images/marginalia/rodartesisters.png}
\includegraphics{https://static01.nyt.com/newsgraphics/2020/08/02/entireworld/assets/images/marginalia/Prada.png}

Band of Outsiders was part of that. Sternberg was 29 when he started the
brand in 2004. Like the Rodarte sisters, who had no formal training and
lived with their parents in Pasadena, Calif., Sternberg, a former agent
at Creative Artists Agency designing a line in what was then a fashion
desert, was an outsider instantly embraced. Within months he had a
full-page photo of his ties in GQ and was picked up by Barneys. ``We
were next to Dries, Balenciaga, Prada,'' he said. ``And `we' were ...
me, making shirts and ties in L.A.''

\includegraphics{https://static01.nyt.com/newsgraphics/2020/08/02/entireworld/assets/images/marginalia/tombrowne.png}

Along with brands like Thom Browne, Band joined the wave of the
nerdy-preppy resurgence --- shrunken blazers, polos, boat shoes --- or
what Sternberg called ``preppy clothes about preppy clothes.'' Once he
expanded into women's wear, the brand grew into a \$15 million wholesale
business, sold in 250 stores worldwide. ``It wasn't by the end all that
good for us, obviously, because we weren't building a sound business,''
Sternberg said. ``But it's pretty incredible the power of what that
global fashion system could do.''

When Sternberg says ``global fashion system,'' he's referring to the
ecosystem of designers, fashion media and stores that puts us all in
clothes. Fashion week is where those entities meet. The reason spring
collections are shown in the fall (and vice versa) is so they can be
ordered, reviewed and produced in time for the actual season. As with
most things, this system was upended by the internet. Once normal people
could view collections online --- which, confusingly, they couldn't buy
until six months later --- everything began to accelerate. Now stores
needed deliveries earlier to fill demand, and two deliveries simply
weren't enough. Suddenly midseason collections --- mainly, pre-fall and
resort (also known as cruise) --- became the norm, even for smaller
designers whose customers were not necessarily among the small subset of
people who jet off to Capri or St.-Tropez for the winter months.

\includegraphics{https://static01.nyt.com/newsgraphics/2020/08/02/entireworld/assets/images/marginalia/hathawayrihanna.png}

So designers went from making two collections a year to four. If you had
a men's line, maybe it was actually six, and if you were Dior or
Givenchy, you were also doing couture. As fashion shows had grown into
huge marketing events because Rihanna or Anne Hathaway or whoever was
sitting in the front row, each of those collections was also a show.
Somehow this was all still going pretty well. Consumers were consuming,
store buyers were buying more and designers produced more and faster.
Business boomed. And everyone just kept growing.

If there was a turning point, it might have been fall 2008. That year,
New York Fashion Week drew an estimated 232,000 attendees and generated
\$466 million in visitor spending. Three days after it ended in
September, the economy collapsed. The luxury market was already
oversaturated, and now there was no one to buy the stuff. Stores
panicked and marked everything down early. But then they did it again
the next year, and the year after that, relying on markdowns to generate
revenue and training consumers to shop on sale. So now you had summer
dresses arriving in January and being discounted before the weather
would even allow you to wear them.

\includegraphics{https://static01.nyt.com/newsgraphics/2020/08/02/entireworld/assets/images/marginalia/gallianorant.gif}
\includegraphics{https://static01.nyt.com/newsgraphics/2020/08/02/entireworld/assets/images/marginalia/toelessboot.png}

The fashion cycle stopped making sense. Despite dwindling budgets,
thousands of people were still flying all over the world every two
months for the shows. Designers started to crack under the pace, most
notably John Galliano, who attributed his 2011 anti-Semitic rant (and
subsequent firing from Dior) to work-related stress. And the clothes
themselves got kind of weird. The sped-up calendar gave birth to
``seasonless dressing,'' a trend of Frankenstein clothing items: toeless
boots, sleeveless coats --- you get it. When you're delivering fall in
July, it's really not about the weather anymore.

This might have been the time to rethink things. Instead, everyone
doubled down and made more stuff.

As online retailers like Net-a-Porter and Matches Fashion gained
traction, and everything was suddenly sold everywhere, department stores
looked for new ways to draw customers. Enter ``novelty,'' a term for the
sometimes-literal bells and whistles that buyers increasingly asked
designers to add to collections in order to entice straying customers
like cats. If in the last decade you've gone looking for a simple
cashmere sweater and instead encountered ones with zippers, giant animal
faces, glitter shoulders or ``distressed'' anything --- that's novelty.
If you found yourself annoyed, you were not alone. ``That was so we
could sell to Saks, Neiman, Barneys, Nordstrom, Colette, and everybody
could have their own special thing,'' Sternberg recalled. ``I was
basically making stuff I didn't like because I thought a buyer wanted
it, not even the customer.''

`I was basically making stuff I didn't like because I thought a buyer
wanted it, not even the customer.'

\includegraphics{https://static01.nyt.com/newsgraphics/2020/08/02/entireworld/assets/images/marginalia/batsheva_portrait.png}
\includegraphics{https://static01.nyt.com/newsgraphics/2020/08/02/entireworld/assets/images/marginalia/batsheva_dress.png}

It used to be that stores attracted shoppers with the promise of an
exclusively carried designer. Once designers could no longer afford to
remain exclusive to a certain store, the compromise was exclusive
styles. In addition to a presented collection, buyers requested slightly
altered looks --- lengthen a hem here, add a sleeve there, take the
print from that dress and make it into pants --- that could then be
exclusive to their customers. This is still going on. ``The amount of
work you do for exclusives is out of control,'' Batsheva Hay, a former
litigator who started her namesake line of off-kilter prairie dresses in
2016, told me. `` `I want \emph{this,} can you make \emph{this} with a
little \emph{this. ... '} Some of it is because they think it might
sell, but some is just so they can say it's exclusive.''

Molly Nutter, a former V.P. for merchandising at Barneys, worked for the
department store for 19 years. ``The system has been broken for a long
time,'' said Nutter, who is now the president of ByGeorge, a specialty
store in Austin, Texas. ``There was a lot of pressure on designers to
produce more collections, and therefore more product. I would say it
wasn't a real demand by the customer; I think it was just retailers
trying to grab market share. They thought, If I can get more in, and
earlier, then I can get more clients through my door. But with everyone
doing this, it just compounds the problem. Then of course all of these
stores end up with too much inventory, and this is where all of the
promotional activity starts to take place. You're basically putting
luxury product out there and devaluing it almost right away. It was just
this vicious cycle.''

This is what Jacobs would later be mourning in his hotel room. While
everyone seemed eager to define fashion's future, he was holding space
for its present. He was lucid, candid, somehow smarter than everyone. (I
was relieved when he declined to be interviewed for this article.)

``We've done everything to such excess that there is no consumer for all
of it,'' Jacobs told Vogue. ``Everyone is exhausted by it. The designers
are exhausted by it. The journalists are exhausted from following it.''
He added, ``When you're just told to produce, to produce, to produce,
it's like having a gun to your head and saying, you know, \emph{Dance,
monkey!''}

\includegraphics{https://static01.nyt.com/packages/flash/multimedia/ICONS/transparent.png}

\includegraphics{https://static01.nyt.com/newsgraphics/2020/08/02/entireworld/assets/images/flag-2000.jpg}

Stephanie Gonot for The New York Times

In 2013, Sternberg sat down with the chief executive at Barneys at the
time, Mark Lee, who Sternberg says overpromised how much inventory the
department store would be able to sell. ``Barneys promised us the world
and never delivered on any of it,'' Sternberg said. (Lee did not respond
to requests for comment.) ``And it was stupid of us to listen to them.
But we trusted them. That was a complete killer. And you feel insecure,
like, I need Barneys to be cool. And then there are these things called
R.T.V.s.''

R.T.V. stands for ``return to vendor,'' which is what it sounds like: If
a collection --- the one that the store has asked you to pad out with
novelty and exclusives --- doesn't sell, the retailer can return it and
ask for its money back. According to Nutter, as stores struggled, the
terms of this deal got worse. In some cases, stores asked designers to
sell on consignment or to share costs if a certain percentage of the
collection didn't sell at full price. So let's say a store decided to
mark the collection down early: You now owed it for those losses. ``Even
as I'm telling you this,'' Nutter said, ``I'm like, Isn't that crazy?''

It is. It is crazy. And here's where it got even crazier: In order to
protect exclusivity, stores had to commit to even larger buys, ordering
more clothes than they could possibly sell. Then, when they couldn't
move the stuff, they'd return it. Thanks to the rise of fast fashion and
the luxury market's simultaneous attempt to keep up with its impossible
pace, it all started to feel disposable. So detrimental was the cycle of
overproduction and discounting to luxury goods that in 2018, Burberry,
the British label, revealed that it had been burning --- not
metaphorically but literally: \emph{burning} --- \$37 million of worth
of merchandise per year to maintain ``brand value.''

`I was just a kid in a candy store, waiting for an adult to step into
the room and rein it all in.'

In short, fashion seemed to slowly annihilate itself. Remember fashion
week? While incurring all those losses, designers were still putting on
shows roughly every three months, productions that ran hundreds of
thousands of dollars. (Or millions, if you were Chanel.) The problem is
that everyone who attended the shows and streamed them out via endless
blurry Instagram videos was actively making the case for the demise of
their jobs. Because if you're there watching via the tiny screen on your
phone while the real live show is happening feet away, why even go?
``God bless fashion media,'' Sternberg said. ``They still have not
caught up to the idea that everyone is seeing it at the same time.''

``It's such a little scam, fashion week,'' he continued. ``I love doing
shows, but you get caught up in it. And then you can't stop. Because if
you stop, they're going to write about you stopping, and you're going to
look like a failure. Or the stores will stop buying your stuff, and you
don't really know why they're buying your stuff, but they're buying it.
And you're not relevant anymore if you're not doing a show.''

Sternberg acknowledged that there were other factors that killed Band of
Outsiders, chief among them his own inexperience in scaling a niche
brand, but ultimately he was underfunded and overleveraged. The day he
opened the store in SoHo --- with a Momofuku Milk Bar attached --- he
knew it was over. Sternberg took a \$2 million convertible loan from
CLCC, a fashion fund backed by a Belgian shipping magnate, and defaulted
six months later. The brand was collateral. (Band has since been reborn
as a zombie version of itself, run by the Belgians.) In May 2015, he
handed off passwords, keys and a storage locker in Pomona, Calif., with
the brand's archive and walked away. ``But it wasn't some big
disaster,'' he said. ``Well ... by the end it was a little bit of a
disaster.''

\includegraphics{https://static01.nyt.com/newsgraphics/2020/08/02/entireworld/assets/images/marginalia/rosen.png}
\includegraphics{https://static01.nyt.com/newsgraphics/2020/08/02/entireworld/assets/images/marginalia/ragnbone.png}

Sternberg's story was not unique among his peers. In Europe, luxury
fashion conglomerates like LVMH and Kering paired young designers with
experienced businesspeople. ``In America, it was much more
entrepreneurial,'' Andrew Rosen, a founder of Theory and an early
investor in Proenza and Rag \& Bone, told me. ``You had a lot of these
incredibly talented young designers that frankly didn't have the
business partnership to go along with it.''

I asked Sternberg if he felt as if he'd lost the narrative. ``To some
extent, I didn't lose the narrative, because I never had one,'' he said.
``I started making shirts and ties for men, and everybody loved them.
Then I made men's clothes for women, and everybody loved them. All these
amazing stores and magazines were eating them up. I was just a kid in a
candy store, waiting for an adult to step into the room and rein it all
in.''

The adult never came. Proenza Schouler has gone through myriad
investors, ending up with one that specializes in distressed assets.
Last summer, Derek Lam shut down his high-end line. In November, Zac
Posen went out of business the same week as Barneys, the store that once
discovered him, followed closely by Opening Ceremony in January.

Then Covid-19 hit.

Consumers stopped having any need for fashionable clothing. Retailers
scrambled to cancel and return orders. (Remember R.T.V.s?) Designers
were unable to cover basic expenses like rent and payroll, let alone
upcoming collections. Suddenly an industry that was already on the brink
ground to a complete halt.

\includegraphics{https://static01.nyt.com/newsgraphics/2020/08/02/entireworld/assets/images/marginalia/wintour-zoom.png}

``It crystallized a lot of conversations that the fashion industry had
been having for some time,'' Anna Wintour, editor of Vogue and artistic
director of Condé Nast, told me when we spoke via Zoom in May. ``For an
industry that is meant to be about change, sometimes we take a long time
to do just that, because it's so big and there are so many moving parts.
But now we were really forced into a moment when we had to reset and
rethink.'' (Full disclosure: I've written for Vogue.)

Later, I asked Wintour why so many designers of that generation were now
struggling. ``I think in general, we've created a system that is
unrealistic and a strain for even the largest of brands,'' she wrote in
an email. ``It could be that some younger designers were playing the
same game and trying to keep up with the big brands rather than
determining what's best for them.''

In March, Vogue partnered with the C.F.D.A. to set up A Common Thread, a
pandemic-relief initiative that has raised \$4.9 million to date. By
May, more than 1,000 companies had applied for aid. ``I was truly
saddened by the number,'' Wintour said, adding: ``I think it really is a
time where we need to learn from what's happened, almost about how
fragile and on the edge we were all living. And that it wasn't that
solid.'' Steven Kolb, the president of the C.F.D.A., was even more
blunt. ``I think there will be brands that don't come out of this still
a business,'' he said.

How did we get here? This is a question I asked almost everyone.

``I think everybody would say it's the other and not themselves,'' Kolb
told me.

``I don't think you can blame one person, or one part of the industry,''
Wintour said. ``Certainly the media had something to do with it as
everything went so instant through digital and the emphasis on what's
new.''

`Certainly the media had something to do with it as everything went so
instant through digital and the emphasis on what's new.'

In May, I called Jeffrey Kalinsky, the retail pioneer who opened Jeffrey
in New York's meatpacking district in 1999, transforming the
neighborhood into the retail zone it is today. Kalinsky was first in New
York to sell Band of Outsiders. In 2005, his stores were acquired by
Nordstrom, one of the department stores said to be well positioned to
survive the pandemic. ``I think all of us played a part,'' Kalinsky
said. ``It was the stores and the customers and the brands and ... all
of us. I hate what's happening in the world. But I think if there's
anything good that can come out of this, it's the chance to look at
ourselves.'' Four days after we spoke, Nordstrom announced that it was
closing Jeffrey.

\includegraphics{https://static01.nyt.com/packages/flash/multimedia/ICONS/transparent.png}

\includegraphics{https://static01.nyt.com/newsgraphics/2020/08/02/entireworld/assets/images/shoe-2000.jpg}

Stephanie Gonot for The New York Times

Sternberg never intended to design a uniform for sheltering in place.
After Band of Outsiders folded in 2015, he padded around his house for a
few weeks and avoided the press. Then, he got an email from Gwyneth
Paltrow. ``I was so sad when Band closed,'' she wrote. ``It was a dark
day for fashion. I'm not sure what you're doing, where your head is at
or if you have a noncompete, but I have an idea I'd love to run by
you.''

Soon Sternberg had a job designing Paltrow's clothing line for Goop, her
wellness-and-lifestyle business. Meanwhile, he thought about what he
might like to do next.

Sternberg surveyed the fashion scene and saw a lot of noise: the luxury
minimalism of countless Celine copycats; the maximalism of brands like
Gucci; the full gamut of streetwear, from Supreme to Vetements. He
wanted to do something that felt like a palate cleanser. Sternberg took
meetings with Target and Amazon fashion and pitched Superproduct, a line
of well-designed basics that he hoped could be what the Gap once was.
When neither went anywhere, he decided to do it on his own.

Entireworld was born in 2018 as a D.T.C. (direct-to-consumer) line, with
no seasons, no shows, no novelty. ``I wanted complete freedom from
that,'' he said. You probably know what D.T.C. is even without knowing
it. Reformation, Everlane, Outdoor Voices, Warby Parker, Allbirds ---
all those sans-serif, venture capital-funded brands that have
proliferated so much in the last decade that you're probably wearing one
of them right now. Have you ever bought clothes from an Instagram ad?
That's D.T.C. Entireworld is sort of \emph{post}-D.T.C., which is to say
that there is no Silicon Valley boardroom trying to solve a problem for
you. It's just Sternberg, a fashion-industry refugee, feeling his way
through it.

``I'm incredibly business-minded,'' Sternberg said. ``But we're
design-driven. I come out of fashion. I'm not coming out of a PowerPoint
deck.''

Most styles in his line are perennial. There are pleated trousers that
are sort of the cooler version of what your '80s dad might wear, and a
``Giant Shirt'' inspired by Ralph Lauren's ``Big Shirt'' of the '90s.
The sweatsuit, made of fabric that Sternberg developed from scratch,
feels like the sartorial version of a hug. Something about its
combination of color, fabric and fit makes it feel OK to wear not only
to bed but also out. (In January, I saw a woman in New York wearing it
under a Burberry coat.) Unlike Band's slim fit, most things by
Entireworld are roomy and wide. Its slogan is ``The stuff you live in.''

\includegraphics{https://static01.nyt.com/newsgraphics/2020/08/02/entireworld/assets/images/marginalia/thakoon.png}
\includegraphics{https://static01.nyt.com/newsgraphics/2020/08/02/entireworld/assets/images/marginalia/mellon.png}

In recent years, the collapse of the fashion industry has pushed other
runway designers, like Thakoon Panichgul and the shoe designer Tamara
Mellon, to redefine themselves as D.T.C. companies. Those who haven't
are now being nudged in that direction. Take Batsheva Hay, for instance,
who in April had more than half of her wholesale orders slashed and
\$100,000 owed to her by retailers. When I reached her, she was
packaging web orders from a lake house in upstate New York and selling
face masks via Instagram. She estimated that before the pandemic D.T.C.
was about 10 percent of her business. ``But now, it's kind of all my
business,'' she said.

Emily Adams Bode, a men's-wear designer who won a C.F.D.A. award last
year, was until recently sold in 120 stores worldwide, with e-commerce
accounting for less than 10 percent of her sales. In May, Bode was at
her fiancé's parents' home in Canada, rushing to put her spring/summer
collection online. ``Stores that we've had in our Excel sheets on the
probability of getting paid at 90 percent now call us and say they're
closing,'' she told me. ``We have to completely rely on our own selling,
because at the end of the day, I don't know how many stores are going to
be able to carry the weight in another six months.'' Last November, just
as everyone declared that retail was dead, Bode opened her own
brick-and-mortar store on the Lower East Side. The store, which is sort
of the old-school version of D.T.C., ended up saving her. What she
projected to sell in a month she started selling in a day. ``I don't
think we'd be here without the store,'' she said. Hay was also looking
at store space just as the crisis began, and planned to again. ``There's
going to be a ton of empty retail space,'' she said, ``I'm sure I can
find an amazing deal.''

`There will definitely be something, but nothing resembling fashion week
as we knew it,' Wintour told me.

The pandemic has also forced a correction of the calendar. With
factories shut down and deliveries delayed, many of this year's fall
collections will, for the first time in a long while, actually arrive in
season. Some in the industry have even talked about pushing the unseen
and unsold 2020 collections to 2021 to avoid losses. ``Which, by the
way, is not a bad idea,'' Sternberg said. ``It's what the clothing
industry has over the food industry: In the food industry, the aged
inventory rots.'' The fascinating part is that in order to do that ---
to give that aged inventory value again --- requires literally killing
fashion, that nebulous deity that says something is ``in'' this year and
not the next.

\includegraphics{https://static01.nyt.com/newsgraphics/2020/08/02/entireworld/assets/images/marginalia/abloh.png}

In May, two separate groups of designers banded together to put forth
proposals on how to change the industry. Each essentially pushed for the
same thing: later deliveries, delayed markdowns, fewer collections. ``I
think a lot of us are aligned on this idea that seasons have to go back
to what they were,'' Joseph Altuzarra, who signed both proposals, told
me. The only person who didn't think fashion had been moving too fast
was the designer Virgil Abloh, even though he had to skip his own
fashion show in Paris last September, reportedly because of exhaustion.
(Abloh juggles his streetwear label, Off-White, with Louis Vuitton men's
wear, as well as collaborations with Nike, Ikea, Evian, Jimmy Choo and
others.) ``I work at the pace of my ideas, and those come often,'' he
told me. ``The consumer today is a hyper being. I'm not one to say,
Let's go back to the old days when we had rotary phones or something.''
He called revising the delivery schedule an ``obvious fix, more so than
a profound idea or anything.''

What does all of this mean for the shows?

``There will definitely be something, but nothing resembling fashion
week as we knew it,'' Wintour told me.

\includegraphics{https://static01.nyt.com/newsgraphics/2020/08/02/entireworld/assets/images/marginalia/alessandro.png}

Abloh announced that he will no longer show on a seasonal schedule, or
base his shows in one place. The Belgian designer Dries Van Noten will
not show until 2021. Chanel premiered a virtual resort show the week
that the George Floyd protests began and came off as mostly tone-deaf.
Alessandro Michele, the Gucci designer, has reduced the number of shows
from five to two, doing away with seasons and gender altogether. There
has also been talk of virtual reality and films accompanied by fabric
samples. In New York, the C.F.D.A. will still be the official scheduler
of New York Fashion Week in September, though it's unclear why mostly
digital shows would have to be scheduled.

``I think fashion week is over,'' Hay said. ``I'm pretty sure it's over
forever.'' If not the shows, then certainly the collective circus that
travels from New York to London to Milan to Paris twice a year.

\includegraphics{https://static01.nyt.com/newsgraphics/2020/08/02/entireworld/assets/images/marginalia/altuzarra_portrait.png}

The more important question is whether people will buy clothes that
aren't sweatpants in the near future. Some are already designing with
that uncertainty in mind. Altuzarra, who makes the opposite of homebody
clothes, told me he was adding softer fabrics and more relaxed
silhouettes to his spring '21 collection. ``Not necessarily like
loungewear or athleisure,'' he said. ``But I think after spending months
in sweatpants, people are going to want to feel comfortable.'' Hay,
meanwhile, was pivoting from party dresses to housedresses. ``I'm just
like, OK, we're home more, but why does that have to be sweatpants?''
she said. ``Can it be a dress? A housedress is completely easy. You can
throw it on, zip it off, whatever. Maybe I'm going too far imagining a
future where we're constantly in and out of quarantine, but
businesswise, I'm sort of preparing for that.''

And if that's the case, what happens to designers like Jacobs? When
asked about online shopping, Jacobs told Business of Fashion: ``I love
to go to a shop. I like to see everything. I like to touch it. I like to
try it on. I like to have a coffee. I like to have a bottle of water. I
like to get dressed up.'' He raised his eyebrows for emphasis. ``But
ordering online, in a pair of grubby sweats, is not my idea of living
life.''

Incidentally, Jacobs's fall 2020 show, in February, was among his very
best. The clothes referenced a pre-internet New York while modern
dancers charged at unsuspecting audience members seated at cafe tables
in a way that now feels prescient. In 2008, Sternberg used to sneak into
Jacobs's shows at the Lexington Avenue Armory, as everyone did then.
(``I'm a \emph{huge} Marc Jacobs fan,'' he told me.) That was the year
that Santigold and M.I.A. played on every runway, and there was a magic
to the way that the music, the stomping models and the fabric in motion
gave fashion its heartbeat. The incredible talent of someone like Jacobs
is that his clothes didn't even have to be produced or worn to have
influence. He's all about starting a conversation that then threads its
way through the system, eventually landing in a consumer's hands via a
perfume or an accessory, if at all. ``So what happens to Marc?''
Sternberg asked. ``Where does he end up?''

He answered his own question. ``I guess in the Mercer Hotel wearing
pearls.''

\includegraphics{https://static01.nyt.com/packages/flash/multimedia/ICONS/transparent.png}

\includegraphics{https://static01.nyt.com/newsgraphics/2020/08/02/entireworld/assets/images/sweats-mann-2000.jpg}

Stephanie Gonot for The New York Times

In June, I stopped by Sternberg's garage, where he keeps a personal
archive of Band of Outsiders designs. There are crates labeled
``turbs,'' for the turbans he sent down the runway for fall 2013 --- a
collection inspired by Billie Holiday and Atari video games --- and
``SS12'' for spring/summer 2012, which referenced Peter Weir's ``Picnic
at Hanging Rock.'' There are also polos from his ``This is not a polo
shirt'' line; fur jackets (before he got off fur) from the show that
opened with mountain climbers rappelling from the ceiling; and bandage
skirts stitched out of suspenders. ``I made that, yay me,'' Sternberg
said flatly. ``This is some ugly print that Rashida Jones wore on `Good
Morning America,''' he said. (Sternberg loves Jones; it's his own work
he's ambivalent about.) ``What do you do with all this {[}expletive{]}?
You don't want to throw it out. Give it away? Should someone be wearing
it? It's not art, for God's sake.''

Going through this stuff, Sternberg was a bit like a musician revisiting
the hits he made before he got sober. He loves them, he really does, but
the excess of it weighs on him --- all those ideas that never became
anything, all those materials, all that waste. Like the shoes: lace-up
Manolo Blahniks and golf-cleat Oxfords and platforms with watch bands as
straps, all developed just for the shows, at 30 pairs per show, and
never even produced. ``And it's season after season,'' he said. ``It's
not like you're making an iPhone, where you're going to mass-produce it
and then iterate on it.''

\includegraphics{https://static01.nyt.com/newsgraphics/2020/08/02/entireworld/assets/images/marginalia/kolb.png}

Last year, Sternberg let his C.F.D.A. membership lapse. He saw it as a
largely New York Fashion Week-centric institution. ``They don't offer
anything for what I'm doing,'' he said. ``They should be trying to
figure out what all this is and how they could support it.'' The
C.F.D.A. subsequently reached out to Sternberg. ``They were sort of
like, `What are you doing?' And I just said: `\emph{This} is what I'm
doing. What are \emph{you} doing? When you're in my zone, let's talk.'''
When I asked Kolb if the C.F.D.A. could do more to support D.T.C.
companies, he said: ``I think that's a big question. That's not an
answer I have.'' It was ultimately up to the board, he added. ``But I
know we have those conversations all the time.''

Whatever tensions there may be, everyone I spoke to praised Sternberg's
reinvention, in the way that fashion people praise things, which is to
say with a tiny bit of shade. ``Love Scott,'' Anna Wintour said. ``It
seems very honest to me and very realistic. I understand not everyone
can afford Marc Jacobs or Chanel.''

Kolb told me, ``I think Scott is a brilliant marketer,'' adding, ``It
works really well with a basics brand.'' But he also credited him with
anticipating this moment. ``Whatever happened between him and the
investors and however he got out of that maybe at the time was painful,
but it enabled him to start over. I think brands that are in it now,
it's much harder to make that change.''

Even Virgil Abloh, the designer of Vuitton men's wear, was excited when
I brought up Sternberg's name. ``Oh, I loved Band of Outsiders!'' he
said. ``My question is, where did he go?''

By June, U.S. clothing sales rebounded, but they were still down overall
from the year before. Market analysts predicted that with infections
soaring again and stimulus money running out, that uptick might be
temporary. The anomalies have been mostly athleisure companies, like
Lululemon, the purveyor of bougie leggings, whose shares have surged in
recent months.

Entireworld is still tiny. But in its second year, Sternberg says its
revenue is already eight times that of Band of Outsiders by the same
point, and that's while selling much more product (\$15 underwear and
socks, \$32 tees, \$88 sweatshirts). Despite the recent good sales,
Sternberg has still had to scale back. In February, he expected to get a
round of financing from investors in Korea, but then the virus hit there
first, and that evaporated. The same week that the sweatsuits were
selling out, he laid off three of his nine employees and cut styles he
planned to add in the fall. Even before the pandemic, persuading
investors to bet on clothing brands had become a drag. ``This is the
shmatte business,'' he told me. ``It's no longer sexy. Investors want
something disruptive. When they're with their investor friends they want
to say they invested in, like, flavored water or an operating system
that changes the way we walk.''

Investors that do pump money into D.T.C. brands are after swift returns,
pushing companies to grow big and fast in a way that's unsustainable.
One such casualty was Outdoor Voices, the athletic-apparel company that
reportedly took in \$60 million of venture-capital money and faltered in
February, with its C.E.O. ousted and its valuation plummeting. After
what happened to Band, the last thing Sternberg wants is to grow too
fast for his own good. ``Investors are only interested in, like:
`Billion-dollar company! Unicorns!''' Sternberg said. Sternberg doesn't
want to be a unicorn. He just wants to be profitable by next year. ``The
second Band tried to grow, that's when we stopped being profitable,'' he
said.

Sternberg wouldn't remember this, but we met briefly a long time ago,
when I covered his spring show in September 2008, mere weeks before the
financial crash. He seemed different now --- sort of softer around the
edges, which also happens to be how he describes his new line. ``I'm
much lighter as a person,'' he said. ``I know that whatever I'm doing
for work is not the end-all, be-all of my life. That doesn't mean I
don't emotionally invest in all this and want it to thrive. But my
identity and sense of self-worth isn't tied to its success or failure.
Would I like this to work? Sure. But is it going to ruin me? No.''

`Is there a place for a \$30 million brand that can self-sustain and be
around year after year?'

The last week, Sternberg admitted, had been rough. Though Schiff, his
managing director, had recovered from Covid-19, a billionaire seed
investor informed Sternberg that he would not be investing any more
money. ``And it's not like we haven't hit our numbers,'' Sternberg said.
In a way, if it weren't for the pandemic, this might have been the end
of Entireworld. When the pandemic hit, he had maybe six weeks of runway
left. The sales boom has extended that to at least the end of the
summer. Still, he had to get more product up on the website, and for
that, he had to pay his factories.

He found the whole thing depressing. Here he was, perhaps the only one
in fashion who couldn't sell merchandise fast enough in a pandemic, and
no one was interested in investing. ``It's a slog,'' Sternberg said.
``It's a constant series of disappointing conversations.''

\includegraphics{https://static01.nyt.com/newsgraphics/2020/08/02/entireworld/assets/images/marginalia/marc2.png}

He thought it was indicative of where the industry was now. Someone like
Marc Jacobs would probably be OK, because he was backed by LVMH. But
what would happen to the upstarts? If the wholesale model could no
longer be relied on to fund young designers, and private equity and
venture capitalists pushed them to expand so quickly that they
inevitably imploded, was there any hope for brands to grow slowly and
thoughtfully over time? If not, fashion might go the way of other
industries, like film, in which there are the blockbusters and the tiny
indies and nothing in between. ``Band didn't need to be a \$100 million
brand,'' Sternberg said. ``But is there a place for a \$30 million brand
that can self-sustain and be around year after year? Certainly not with
big backers, because that's not interesting to them. Wholesale used to
be able to support that, but it also ultimately killed it.''

Fashion is, by definition, unpredictable. People buy clothes for
illogical, emotional reasons. The challenge, as Sternberg saw it, was to
build a brand that could be immune to trends and novelty and whatever
dystopian disaster was coming next. ``The trick with fashion is that
we're not selling toilet paper,'' he said, ``which of course during
Covid, toilet-paper sales go up. But ultimately it will level out,
because there's only so many butts in the world. That hasn't changed ---
people are just hoarding. Fashion is really different. You have to
assume the cycle will change even if you're doing commodity. And how
will you keep up with that? How do you build a business that can sustain
those fluctuations over time?''

That was his pitch, anyway. But so far, no one seemed to be listening.
One investor suggested that maybe Sternberg should turn Entireworld into
a TV show that would advertise the clothes. (Sternberg: ``Sounds
easy!'') Another told him, ``Wow, it's great that you're doing well, but
I'm actually looking into distressed assets now.'' Instead of investing
into a young business that was actually making money, the investor was
looking to swoop in and pick off bigger brands that were now on the
brink of bankruptcy. Reviving a corpse was easier than tending to a
newborn. As this investor saw it, that, in the end, held the promise of
a bigger payoff.

\textbf{Irina Aleksander} is a contributing writer for the magazine. Her
last cover story was about Oliver Stone's quest to make a biopic about
Edward Snowden. \textbf{Stephanie Gonot} is a Los Angeles-based
photographer and director known for her use of vivid colors and playful
style.

Prop Styling: Machen Machen Studio

Margin photographs in order of appearance: Polaroids: Band of Outsiders;
Obama: Manuel Balce Ceneta/Associated Press, via Shutterstock;
Furstenberg: Anna Moneymaker/The New York Times; Teletubbies: Handout,
Stiller: RGR Collection/Alamy; Charney: Ann Johansson for The New York
Times; Socks: Entireworld; Marc Jacobs broadcast: Screen grab from
YouTube; Women's Wear Daily; Runway, from left to right: Karl
Prouse/Catwalking, via Getty Images, Marilynn K. Yee/The New York Times,
Erin Baiano for The New York Times; Rodartes: Brinson+Banks for The New
York Times; Runway, from left to right: Firstview (2), Stefano
Rellandini/Reuters; Browne: Donna Ward/Getty Images; Rihanna and
Hathaway: Jamie McCarthy/Getty Images; Galliano: Screen grab from
YouTube; Beckham boots: Raymond Hall/GC Images, via Getty Images; Hay:
Roy Rochlin/Getty Images; Model in pink dress: Alexei Hay; Rosen: Andy
Lyons/Getty Images, for The New York Times; Runway: JP Yim/Getty Images,
Slaven Vlasic/Getty Images; Wintour: Screen grab from Zoom; Panichgul:
JP Yim/Getty Images; Mellon: Andrew Toth/FilmMagic, via Getty Images;
Abloh: Daniel Zuchnik/Getty Images; Michele: Karwai Tang/Getty Images;
Altuzarra: Lars Niki/Getty Images; Kolb: Ben Gabbe/Getty Images; Jacobs
waving: Raymond Hall/GC Images, via Getty Images.

Design and development by Shannon Lin.

Read 461 Comments

\begin{itemize}
\item
\item
\item
\item
\end{itemize}

Advertisement

\protect\hyperlink{after-bottom}{Continue reading the main story}

\hypertarget{site-index}{%
\subsection{Site Index}\label{site-index}}

\hypertarget{site-information-navigation}{%
\subsection{Site Information
Navigation}\label{site-information-navigation}}

\begin{itemize}
\tightlist
\item
  \href{https://help.nytimes.com/hc/en-us/articles/115014792127-Copyright-notice}{©~2020~The
  New York Times Company}
\end{itemize}

\begin{itemize}
\tightlist
\item
  \href{https://www.nytco.com/}{NYTCo}
\item
  \href{https://help.nytimes.com/hc/en-us/articles/115015385887-Contact-Us}{Contact
  Us}
\item
  \href{https://www.nytco.com/careers/}{Work with us}
\item
  \href{https://nytmediakit.com/}{Advertise}
\item
  \href{http://www.tbrandstudio.com/}{T Brand Studio}
\item
  \href{https://www.nytimes.com/privacy/cookie-policy\#how-do-i-manage-trackers}{Your
  Ad Choices}
\item
  \href{https://www.nytimes.com/privacy}{Privacy}
\item
  \href{https://help.nytimes.com/hc/en-us/articles/115014893428-Terms-of-service}{Terms
  of Service}
\item
  \href{https://help.nytimes.com/hc/en-us/articles/115014893968-Terms-of-sale}{Terms
  of Sale}
\item
  \href{https://spiderbites.nytimes.com}{Site Map}
\item
  \href{https://help.nytimes.com/hc/en-us}{Help}
\item
  \href{https://www.nytimes.com/subscription?campaignId=37WXW}{Subscriptions}
\end{itemize}
