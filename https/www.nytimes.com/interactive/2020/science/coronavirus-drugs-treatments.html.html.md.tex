Sections

SEARCH

\protect\hyperlink{site-content}{Skip to
content}\protect\hyperlink{site-index}{Skip to site index}

\href{https://www.nytimes.com/section/science}{Science}

\href{https://myaccount.nytimes.com/auth/login?response_type=cookie\&client_id=vi}{}

\href{https://www.nytimes.com/section/todayspaper}{Today's Paper}

\href{/section/science}{Science}\textbar{}Coronavirus Drug and Treatment
Tracker

\url{https://nyti.ms/3h684eG}

\begin{itemize}
\item
\item
\item
\item
\item
\end{itemize}

\href{https://www.nytimes.com/news-event/coronavirus?action=click\&pgtype=Article\&state=default\&region=TOP_BANNER\&context=storylines_menu}{The
Coronavirus Outbreak}

\begin{itemize}
\tightlist
\item
  live\href{https://www.nytimes.com/2020/08/04/world/coronavirus-covid-19.html?action=click\&pgtype=Article\&state=default\&region=TOP_BANNER\&context=storylines_menu}{Latest
  Updates}
\item
  \href{https://www.nytimes.com/interactive/2020/us/coronavirus-us-cases.html?action=click\&pgtype=Article\&state=default\&region=TOP_BANNER\&context=storylines_menu}{Maps
  and Cases}
\item
  \href{https://www.nytimes.com/interactive/2020/science/coronavirus-vaccine-tracker.html?action=click\&pgtype=Article\&state=default\&region=TOP_BANNER\&context=storylines_menu}{Vaccine
  Tracker}
\item
  \href{https://www.nytimes.com/2020/08/02/us/covid-college-reopening.html?action=click\&pgtype=Article\&state=default\&region=TOP_BANNER\&context=storylines_menu}{College
  Reopening}
\item
  \href{https://www.nytimes.com/live/2020/08/03/business/stock-market-today-coronavirus?action=click\&pgtype=Article\&state=default\&region=TOP_BANNER\&context=storylines_menu}{Economy}
\end{itemize}

\hypertarget{coronavirus-drug-and-treatment-tracker}{%
\section{Coronavirus Drug and Treatment
Tracker}\label{coronavirus-drug-and-treatment-tracker}}

By \href{https://www.nytimes.com/by/jonathan-corum}{Jonathan Corum},
\href{https://www.nytimes.com/by/katherine-j--wu}{Katherine J. Wu} and
\href{https://www.nytimes.com/by/carl-zimmer}{Carl Zimmer}Updated July
22, 2020

\href{https://www.nytimes.com/es/interactive/2020/science/coronavirus-tratamientos-curas.html}{Leer
en español}

\begin{itemize}
\item
\item
\item
\item
\end{itemize}

\href{https://www.nytimes.com/interactive/2020/world/coronavirus-maps.html}{World}~

COUNTRIES

\textbar{}
\href{https://www.nytimes.com/interactive/2020/us/coronavirus-us-cases.html}{U.S.A.}~

STATES

~
\href{https://www.nytimes.com/interactive/2020/us/coronavirus-testing.html}{Testing}

\href{https://www.nytimes.com/interactive/2020/world/americas/brazil-coronavirus-cases.html}{Brazil}\href{https://www.nytimes.com/interactive/2020/world/canada/canada-coronavirus-cases.html}{Canada}\href{https://www.nytimes.com/interactive/2020/world/europe/france-coronavirus-cases.html}{France}\href{https://www.nytimes.com/interactive/2020/world/europe/germany-coronavirus-cases.html}{Germany}\href{https://www.nytimes.com/interactive/2020/world/asia/india-coronavirus-cases.html}{India}\href{https://www.nytimes.com/interactive/2020/world/europe/italy-coronavirus-cases.html}{Italy}\href{https://www.nytimes.com/interactive/2020/world/americas/mexico-coronavirus-cases.html}{Mexico}\href{https://www.nytimes.com/interactive/2020/world/europe/spain-coronavirus-cases.html}{Spain}\href{https://www.nytimes.com/interactive/2020/world/europe/united-kingdom-coronavirus-cases.html}{U.K.}

\href{https://www.nytimes.com/interactive/2020/us/alabama-coronavirus-cases.html}{Alabama}\href{https://www.nytimes.com/interactive/2020/us/alaska-coronavirus-cases.html}{Alaska}\href{https://www.nytimes.com/interactive/2020/us/arizona-coronavirus-cases.html}{Arizona}\href{https://www.nytimes.com/interactive/2020/us/arkansas-coronavirus-cases.html}{Arkansas}\href{https://www.nytimes.com/interactive/2020/us/california-coronavirus-cases.html}{California}\href{https://www.nytimes.com/interactive/2020/us/colorado-coronavirus-cases.html}{Colorado}\href{https://www.nytimes.com/interactive/2020/us/connecticut-coronavirus-cases.html}{Connecticut}\href{https://www.nytimes.com/interactive/2020/us/delaware-coronavirus-cases.html}{Delaware}\href{https://www.nytimes.com/interactive/2020/us/florida-coronavirus-cases.html}{Florida}\href{https://www.nytimes.com/interactive/2020/us/georgia-coronavirus-cases.html}{Georgia}\href{https://www.nytimes.com/interactive/2020/us/hawaii-coronavirus-cases.html}{Hawaii}\href{https://www.nytimes.com/interactive/2020/us/idaho-coronavirus-cases.html}{Idaho}\href{https://www.nytimes.com/interactive/2020/us/illinois-coronavirus-cases.html}{Illinois}\href{https://www.nytimes.com/interactive/2020/us/indiana-coronavirus-cases.html}{Indiana}\href{https://www.nytimes.com/interactive/2020/us/iowa-coronavirus-cases.html}{Iowa}\href{https://www.nytimes.com/interactive/2020/us/kansas-coronavirus-cases.html}{Kansas}\href{https://www.nytimes.com/interactive/2020/us/kentucky-coronavirus-cases.html}{Kentucky}\href{https://www.nytimes.com/interactive/2020/us/louisiana-coronavirus-cases.html}{Louisiana}\href{https://www.nytimes.com/interactive/2020/us/maine-coronavirus-cases.html}{Maine}\href{https://www.nytimes.com/interactive/2020/us/maryland-coronavirus-cases.html}{Maryland}\href{https://www.nytimes.com/interactive/2020/us/massachusetts-coronavirus-cases.html}{Massachusetts}\href{https://www.nytimes.com/interactive/2020/us/michigan-coronavirus-cases.html}{Michigan}\href{https://www.nytimes.com/interactive/2020/us/minnesota-coronavirus-cases.html}{Minnesota}\href{https://www.nytimes.com/interactive/2020/us/mississippi-coronavirus-cases.html}{Mississippi}\href{https://www.nytimes.com/interactive/2020/us/missouri-coronavirus-cases.html}{Missouri}\href{https://www.nytimes.com/interactive/2020/us/montana-coronavirus-cases.html}{Montana}\href{https://www.nytimes.com/interactive/2020/us/nebraska-coronavirus-cases.html}{Nebraska}\href{https://www.nytimes.com/interactive/2020/us/nevada-coronavirus-cases.html}{Nevada}\href{https://www.nytimes.com/interactive/2020/us/new-hampshire-coronavirus-cases.html}{New
Hampshire}\href{https://www.nytimes.com/interactive/2020/us/new-jersey-coronavirus-cases.html}{New
Jersey}\href{https://www.nytimes.com/interactive/2020/us/new-mexico-coronavirus-cases.html}{New
Mexico}\href{https://www.nytimes.com/interactive/2020/us/new-york-coronavirus-cases.html}{New
York}\href{https://www.nytimes.com/interactive/2020/us/north-carolina-coronavirus-cases.html}{North
Carolina}\href{https://www.nytimes.com/interactive/2020/us/north-dakota-coronavirus-cases.html}{North
Dakota}\href{https://www.nytimes.com/interactive/2020/us/ohio-coronavirus-cases.html}{Ohio}\href{https://www.nytimes.com/interactive/2020/us/oklahoma-coronavirus-cases.html}{Oklahoma}\href{https://www.nytimes.com/interactive/2020/us/oregon-coronavirus-cases.html}{Oregon}\href{https://www.nytimes.com/interactive/2020/us/pennsylvania-coronavirus-cases.html}{Pennsylvania}\href{https://www.nytimes.com/interactive/2020/us/puerto-rico-coronavirus-cases.html}{Puerto
Rico}\href{https://www.nytimes.com/interactive/2020/us/rhode-island-coronavirus-cases.html}{Rhode
Island}\href{https://www.nytimes.com/interactive/2020/us/south-carolina-coronavirus-cases.html}{South
Carolina}\href{https://www.nytimes.com/interactive/2020/us/south-dakota-coronavirus-cases.html}{South
Dakota}\href{https://www.nytimes.com/interactive/2020/us/tennessee-coronavirus-cases.html}{Tennessee}\href{https://www.nytimes.com/interactive/2020/us/texas-coronavirus-cases.html}{Texas}\href{https://www.nytimes.com/interactive/2020/us/utah-coronavirus-cases.html}{Utah}\href{https://www.nytimes.com/interactive/2020/us/vermont-coronavirus-cases.html}{Vermont}\href{https://www.nytimes.com/interactive/2020/us/virginia-coronavirus-cases.html}{Virginia}\href{https://www.nytimes.com/interactive/2020/us/washington-coronavirus-cases.html}{Washington}\href{https://www.nytimes.com/interactive/2020/us/washington-dc-coronavirus-cases.html}{Washington,
D.C.}\href{https://www.nytimes.com/interactive/2020/us/west-virginia-coronavirus-cases.html}{West
Virginia}\href{https://www.nytimes.com/interactive/2020/us/wisconsin-coronavirus-cases.html}{Wisconsin}\href{https://www.nytimes.com/interactive/2020/us/wyoming-coronavirus-cases.html}{Wyoming}

~

The Covid-19 pandemic is one of the greatest challenges modern medicine
has ever faced. Doctors and scientists are scrambling to find treatments
and drugs that can save the lives of infected people and perhaps even
prevent infection.

Below is an updated list of \textbf{19 of the most-talked-about
treatments for the coronavirus}. While some are accumulating evidence
that they're effective, most are still at early stages of research. We
also included a warning about a few that are just bunk.

We are following 19 coronavirus treatments for effectiveness and safety:

2

2

10

2

3

Widely

used

Promising

evidence

Tentative or

mixed evidence

Not

promising

Pseudoscience

or fraud

We are following 19 coronavirus treatments

for effectiveness and safety:

2

10

2

Promising

evidence

Tentative or

mixed evidence

2

3

Widely

used

Not

promising

Pseudoscience

or fraud

We are following 19 coronavirus treatments

for effectiveness and safety:

2

10

2

Promising

evidence

Mixed

evidence

2

3

Widely

used

Not

promising

Pseudoscience

or fraud

There is no cure yet for Covid-19. And even the most promising
treatments to date only help certain groups of patients, and await
validation from further trials. The F.D.A. has not fully licensed any
treatment specifically for the coronavirus, but it has granted
\href{https://www.fda.gov/emergency-preparedness-and-response/mcm-legal-regulatory-and-policy-framework/emergency-use-authorization}{emergency
use authorization} to a few.

This list provides a snapshot of the latest research on the coronavirus,
but does not constitute medical endorsements. Always consult your doctor
about treatments for Covid-19.

New additions and recent updates:

•~ We \protect\hyperlink{note}{adjusted} some labels used in the tracker
after additional discussions with experts. July 17

We will update and expand the list as new evidence emerges. For details
on evaluating treatments, see the
\href{https://www.covid19treatmentguidelines.nih.gov/}{N.I.H. Covid-19
Treatment Guidelines}. For the current status of vaccine development,
see our
\href{https://www.nytimes.com/interactive/2020/science/coronavirus-vaccine-tracker.html}{Coronavirus
Vaccine Tracker}.

\hypertarget{what-the-labels-mean}{%
\subsection{What the Labels~Mean}\label{what-the-labels-mean}}

\textbf{WIDELY USED}: These treatments have been used widely by doctors
and nurses to treat patients hospitalized for diseases that affect the
respiratory system, including Covid-19.

\textbf{PROMISING EVIDENCE}: Early evidence from studies on patients
suggests effectiveness, but more research is needed. This category
includes treatments that have shown improvements in morbidity, mortality
and recovery in at least one randomized controlled trial, in which some
people get a treatment and others get a placebo.

\textbf{TENTATIVE OR MIXED EVIDENCE}: Some treatments show promising
results in cells or animals, which need to be confirmed in people.
Others have yielded encouraging results in retrospective studies in
humans, which look at existing datasets rather than starting a new
trial. Some treatments have produced different results in different
experiments, raising the need for larger, more rigorously designed
studies to clear up the confusion.

\textbf{NOT PROMISING}: **** Early evidence suggests that these
treatments do not work.

\textbf{PSEUDOSCIENCE OR FRAUD}: These are not treatments that
researchers have ever considered using for Covid-19. Experts have warned
against trying them, because they do not help against the disease and
can instead be dangerous. Some people have even been arrested for their
false promises of a Covid-19 cure.

Filter the list of treatments:

All treatments

Widely used

Promising

Tentative or mixed

Not promising

Pseudoscience

\hypertarget{blocking-the-virus}{%
\subsection{Blocking the~Virus}\label{blocking-the-virus}}

\emph{Antivirals can stop viruses such as H.I.V. and hepatitis C from
hijacking our cells. Scientists are searching for antivirals that work
against the new coronavirus.}

PROMISING EVIDENCE EMERGENCY USE AUTHORIZATION\\
Remdesivir\\
\href{https://www.nytimes.com/2020/05/23/health/coronavirus-remdesivir.html}{Remdesivir},
made by Gilead Sciences, was the first drug to get emergency
authorization from the F.D.A. for use on Covid-19. It stops viruses from
replicating by inserting itself into new viral genes. Remdesivir was
originally tested as an antiviral against Ebola and Hepatitis C, only to
deliver lackluster results. But preliminary data from trials that began
this spring suggested the drug can reduce the recovery time of people
hospitalized with Covid-19 from
\href{https://www.nejm.org/doi/full/10.1056/NEJMoa2007764}{15 to 11
days}. (The study defined recovery as ``either discharge from the
hospital or hospitalization for infection-control purposes only.'')
These early results did not show any effect on mortality, though
retrospective data released in July hints that the drug might
\href{https://www.gilead.com/news-and-press/press-room/press-releases/2020/7/gilead-presents-additional-data-on-investigational-antiviral-remdesivir-for-the-treatment-of-covid-19}{reduce
death rates} among those who are very ill.

TENTATIVE OR MIXED EVIDENCE\\
Favipiravir\\
Originally designed to beat back influenza, favipiravir blocks a virus's
ability to copy its genetic material. A
\href{https://www.sciencedirect.com/science/article/pii/S2095809920300631?via\%3Dihub}{small
study} in March indicated the drug might help purge the coronavirus from
the airway, but results from larger, well-designed clinical trials are
still pending.

TENTATIVE OR MIXED EVIDENCE\\
EIDD-2801\\
Another antiviral originally designed to fight the flu, EIDD-2801 has
had promising
\href{https://stm.sciencemag.org/content/12/541/eabb5883}{results}
against the new coronavirus in studies in cells and on animals. It is
still being tested in humans.

TENTATIVE OR MIXED EVIDENCE\\
Recombinant ACE-2\\
To enter cells, the coronavirus must first
\href{https://www.nytimes.com/interactive/2020/03/11/science/how-coronavirus-hijacks-your-cells.html}{unlock
them} --- a feat it accomplishes by latching onto a human protein called
ACE-2. Scientists have created artificial ACE-2 proteins which might be
able to act as decoys, luring the coronavirus away from vulnerable
cells. Recombinant ACE-2 proteins have shown promising
\href{https://doi.org/10.1016/j.cell.2020.04.004}{results} in
experiments on cells, but not yet in animals or people.

NOT PROMISING\\
Lopinavir and ritonavir\\
Twenty years ago, the F.D.A. approved this combination of drugs to treat
H.I.V. Recently, researchers tried them out on the new coronavirus and
found that they stopped the virus from replicating. But clinical trials
in patients proved disappointing. In early July, the World Health
Organization
\href{https://www.who.int/news-room/detail/04-07-2020-who-discontinues-hydroxychloroquine-and-lopinavir-ritonavir-treatment-arms-for-covid-19}{suspended}
trials on patients hospitalized for Covid-19. But they didn't rule out
studies to see if the drugs could help patients not sick enough to be
hospitalized, or to prevent people exposed to the new coronavirus from
falling ill. The drug could also still have a role to play in certain
\href{https://www.thelancet.com/journals/lancet/article/PIIS0140-6736(20)31042-4/fulltext}{combination
treatments}.

NOT PROMISING\\
Hydroxychloroquine and chloroquine\\
German chemists synthesized chloroquine in the 1930s as a drug against
malaria. A less toxic version, called hydroxychloroquine, was
\href{https://www.nature.com/articles/s41421-020-0156-0\#:~:text=Hydroxychloroquine\%20(HCQ)\%20sulfate\%2C\%20a,than\%20CQ\%20in\%20animals4.}{invented
in 1946}, and later was approved for other diseases such as lupus and
rheumatoid arthritis. At the start of the Covid-19 pandemic, researchers
discovered that both drugs could stop the coronavirus from replicating
in cells. Since then, they've had a tumultuous ride. A few small studies
on patients offered some hope that hydroxychloroquine could treat
Covid-19. The World Health Organization launched a randomized clinical
trial in March to see if it was indeed safe and effective for Covid-19,
as did Novartis and a number of universities.

Meanwhile, President Trump repeatedly
\href{https://www.nytimes.com/2020/04/06/us/politics/coronavirus-trump-malaria-drug.html}{promoted
hydroxychloroquine} at press conferences, touting it as a ``game
changer,'' and even
\href{https://www.nytimes.com/2020/05/18/us/politics/trump-hydroxychloroquine-covid-coronavirus.html}{took
it himself}. The F.D.A. temporarily granted hydroxychloroquine emergency
authorization for use in Covid-19 patients --- which a whistleblower
later
\href{https://www.buzzfeednews.com/article/zahrahirji/fda-eua-hydroxychloroquine-chloroquine}{claimed}
was the result of political pressure. In the wake of the drug's newfound
publicity,
\href{https://www.nytimes.com/2020/04/25/us/coronavirus-trump-chloroquine-hydroxychloroquine.html}{demand
spiked}, resulting in
\href{https://ard.bmj.com/content/early/2020/07/01/annrheumdis-2020-218164}{shortages}
for people who rely on hydroxychloroquine as a treatment for other
diseases.

But more detailed studies proved disappointing. A
\href{https://www.nature.com/articles/s41586-020-2558-4}{study} on
monkeys found that hydroxychloroquine didn't prevent the animals from
getting infected and didn't clear the virus once they got sick.
Randomized clinical trials found that hydroxychloroquine
\href{https://www.recoverytrial.net/news/statement-from-the-chief-investigators-of-the-randomised-evaluation-of-covid-19-therapy-recovery-trial-on-hydroxychloroquine-5-june-2020-no-clinical-benefit-from-use-of-hydroxychloroquine-in-hospitalised-patients-with-covid-19}{didn't
help people with Covid-19 get better} or
\href{https://www.nytimes.com/2020/06/03/health/hydroxychloroquine-coronavirus-trump.html}{prevent
healthy people from contracting the coronavirus}. Another
\href{https://www.acpjournals.org/doi/10.7326/M20-4207}{randomized
clinical trial} found that giving hydroxychloroquine to people right
after being diagnosed with Covid-19 didn't reduce the severity of their
disease. (One large-scale study that concluded the drug was harmful as
well was later
\href{https://www.nytimes.com/2020/06/04/health/coronavirus-hydroxychloroquine.html?searchResultPosition=1}{retracted}.)
The
\href{https://www.who.int/news-room/detail/04-07-2020-who-discontinues-hydroxychloroquine-and-lopinavir-ritonavir-treatment-arms-for-covid-19}{World
Health Organization}, the National Institutes of Health and Novartis
have since halted trials investigating hydroxychloroquine as a treatment
for Covid-19, and the F.D.A.
\href{https://www.nytimes.com/2020/06/15/health/fda-hydroxychloroquine-malaria.html}{revoked
its emergency approval}. The F.D.A. now
\href{https://www.fda.gov/drugs/drug-safety-and-availability/fda-cautions-against-use-hydroxychloroquine-or-chloroquine-covid-19-outside-hospital-setting-or}{warns}
that the drug can cause a host of serious side effects to the heart and
other organs when used to treat Covid-19.

In July, researchers at Henry Ford hospital in Detroit published a
\href{https://www.statnews.com/2020/07/08/a-flawed-covid-19-study-gets-the-white-houses-attention-and-the-fda-may-pay-the-price/}{study}
finding that hydroxychloroquine reduced mortality in Covid-19 patients.
President Trump
\href{https://twitter.com/realDonaldTrump/status/1280328830218051584}{praised
the study on Twitter}, but experts raised doubts about it because it was
not a randomized controlled trial. Still, the White House has initiated
a push for the F.D.A. to reauthorize hydroxychloroquine as an emergency
Covid-19 treatment.

Despite negative results, a number of hydroxychloroquine trials have
continued. A recent analysis by STAT and Applied XL found more than 180
\href{https://www.statnews.com/2020/07/06/data-show-panic-and-disorganization-dominate-the-study-of-covid-19-drugs/}{ongoing
clinical trials} testing hydroxychloroquine or chloroquine, for treating
or preventing Covid-19. Although it's clear the drugs are no panacea,
it's possible they could work in combination with other treatments, or
when given in early stages of the disease.\\
Updated July 22

\hypertarget{mimicking-the-immune-system}{%
\subsection{Mimicking the
Immune~System}\label{mimicking-the-immune-system}}

\emph{Most people who get Covid-19 successfully fight off the virus with
a strong immune response. Drugs might help people who can't mount an
adequate defense.}

TENTATIVE OR MIXED EVIDENCE\\
Convalescent plasma\\
A century ago, doctors filtered plasma from the blood of recovered flu
patients. So-called convalescent plasma, rich with antibodies, helped
people sick with flu fight their illness. Now researchers are trying out
this
\href{https://www.nytimes.com/2020/04/24/smarter-living/coronavirus-convalescent-plasma-antibodies.html?searchResultPosition=1}{strategy}
on Covid-19. Early trials with convalescent plasma have yielded
\href{https://www.nytimes.com/2020/05/22/health/coronarvirus-convalescent-serum.html?searchResultPosition=2}{promising,
if preliminary, results}, and the F.D.A. has authorized its use on very
sick patients infected by the coronavirus.

TENTATIVE OR MIXED EVIDENCE\\
Monoclonal antibodies\\
Convalescent plasma contains a mix of different antibodies, some of
which can attack the coronavirus, and some of which can't. Researchers
have been sifting through the slurry for the most potent antibodies
against Covid-19. Synthetic copies of these molecules, known as
\href{https://www.nytimes.com/2020/07/09/health/regeneron-monoclonal-antibodies.html}{monoclonal
antibodies}, can be manufactured in bulk and then injected into
patients. Safety trials for this treatment have
\href{https://investor.lilly.com/news-releases/news-release-details/lilly-begins-worlds-first-study-potential-covid-19-antibody}{only
just begun}, with several more on the way.

TENTATIVE OR MIXED EVIDENCE\\
Interferons\\
Interferons are molecules our cells naturally produce in response to
viruses, rousing the immune system to attack. Injecting synthetic
interferons is now a standard treatment for a number of immune
disorders.
\href{https://www.nationalmssociety.org/Treating-MS/Medications/Rebif}{Rebif},
for example, is prescribed for multiple sclerosis. Early studies,
including experiments in
\href{https://pubmed.ncbi.nlm.nih.gov/32511406/}{mice} and
\href{https://academic.oup.com/jid/article/doi/10.1093/infdis/jiaa350/5860074}{cells},
hint that injecting interferons may help against Covid-19. An open-label
study in China suggested that the molecules could
\href{https://www.medrxiv.org/content/10.1101/2020.04.11.20061473v2}{help
prevent healthy people from getting infected}. On July 20, the British
pharmaceutical company Synairgen
\href{https://www.nytimes.com/2020/07/20/world/covid-19-treatment-synairgen-interferon-beta.html}{announced}
that an inhaled form of interferon called SNG001 lowered the risk of
severe Covid-19 in infected patients in a small clinical trial. The full
data have not yet been released to the public, or published in a
scientific journal.\\
Updated July 20

\hypertarget{putting-out-friendly-fire}{%
\subsection{Putting Out Friendly~Fire}\label{putting-out-friendly-fire}}

\emph{The most severe symptoms of Covid-19 are the result of the immune
system's overreaction to the virus. Scientists are testing drugs that
can rein in its attack.}

PROMISING EVIDENCE\\
Dexamethasone\\
This cheap and widely available steroid blunts many types of immune
responses. Doctors have long used it to treat allergies, asthma and
inflammation. In June, it became the first drug shown to
\href{https://www.nytimes.com/2020/06/16/world/europe/dexamethasone-coronavirus-covid.html?searchResultPosition=5}{reduce
Covid-19 deaths}. That
\href{https://www.nejm.org/doi/full/10.1056/NEJMoa2021436?query=featured_home}{study}
of more than 6,000 people, which in July was published in the New
England Journal of Medicine, found that dexamethasone reduced deaths by
one-third in patients on ventilators, and by one-fifth in patients on
oxygen. It may be
\href{https://www.nytimes.com/2020/06/24/health/coronavirus-dexamethasone.html?searchResultPosition=2}{less
likely to help} --- and may even harm --- patients who are at an earlier
stage of Covid-19 infections, however. In its Covid-19 treatment
guidelines, the National Institutes of Health
\href{https://www.covid19treatmentguidelines.nih.gov/dexamethasone/}{recommends}
only using dexamethasone in patients with COVID-19 who are on a
ventilator or are receiving supplemental oxygen.

TENTATIVE OR MIXED EVIDENCE\\
Cytokine Inhibitors\\
The body produces signaling molecules called cytokines to fight off
diseases. But manufactured in excess, cytokines can trigger the immune
system to wildly overreact to infections, in a process sometimes called
a cytokine storm. Researchers have created a number of drugs to halt
cytokine storms, and they have proven effective against arthritis and
other inflammatory disorders. Some turn off the supply of molecules that
launch the production of the cytokines themselves. Others block the
receptors on immune cells to which cytokines would normally bind. A few
block the cellular messages they send.

Against the coronavirus, several of these drugs, including tocilizumab,
sarilumab and
\href{https://onlinelibrary.wiley.com/doi/10.1002/art.41422}{anakinra},
have
\href{https://www.medrxiv.org/content/10.1101/2020.06.01.20119149v2}{offered
modest help} in some trials, but faltered in others. The drug company
Regeneron recently announced that a branded version of sarilumab,
Kevzara,
\href{https://www.nytimes.com/reuters/2020/07/02/us/02reuters-health-coronavirus-regeneron-pharms.html}{failed}
Phase 3 clinical trials.

TENTATIVE OR MIXED EVIDENCE EMERGENCY USE AUTHORIZATION\\
Cytosorb\\
Cytosorb is a cartridge that
\href{https://www.nytimes.com/2020/06/11/health/coronavirus-cytokine-storm.html}{filters
cytokines}from the blood in an attempt to cool cytokine storms. The
machine can purify a patient's entire blood supply about 70 times in a
24-hour period. It was
\href{https://www.nhlbi.nih.gov/news/2020/novel-blood-filter-approved-fda-emergency-treatment-covid-19}{granted
emergency use authorization} by the F.D.A. for Covid-19 based on a small
study in March suggesting that it had helped dozens of severely ill
Covid-19 patients in Europe and China.
\href{https://clinicaltrials.gov/ct2/results?cond=covid-19\&term=cytosorb\&cntry=\&state=\&city=\&dist=}{Further
trials} on patients with Covid-19 are now underway.

TENTATIVE OR MIXED EVIDENCE\\
Stem cells\\
Certain kinds of stem cells can secrete anti-inflammatory molecules.
Over the years, researchers have tried to use them as a
\href{https://celltrials.org/news/role-msc-treat-coronavirus-pneumonia-and-ards-part-1-is-emperor-wearing-clothes}{treatment
for cytokine storms}, and now dozens of clinical
\href{https://clinicaltrials.gov/ct2/results?term=stem+cell\&cond=COVID-19\&age_v=\&gndr=\&type=\&rslt=\&phase=0\&phase=1\&phase=2\&phase=3\&Search=Apply}{trials}
are under way to see if they can help patients with Covid-19. But these
stem cell treatments haven't worked well in the past, and it's not clear
yet if they'll work against the coronavirus.

\hypertarget{other-treatments}{%
\subsection{Other Treatments}\label{other-treatments}}

\emph{Doctors and nurses often administer other supportive treatments to
help patients with Covid-19.}

WIDELY USED\\
Prone positioning\\
The simple act of flipping Covid-19 patients onto their bellies
\href{https://www.nytimes.com/2020/05/13/health/coronavirus-proning-lungs.html}{opens
up the lungs}. The maneuver has become commonplace in hospitals around
the world since the start of the pandemic. It might help some
individuals avoid the need for ventilators entirely. The treatment's
benefits continue to be tested in a range of clinical trials.

WIDELY USED EMERGENCY USE AUTHORIZATION\\
Ventilators and other respiratory support devices\\
Devices that help people breathe are an essential tool in the fight
against deadly respiratory illnesses. Some patients do well if they get
an extra supply of oxygen through the nose or via a mask connected to an
oxygen machine. Patients in severe respiratory distress may need to have
a
\href{https://www.nytimes.com/interactive/2020/05/08/health/coronavirus-covid-lungs-ventilators.html}{ventilator
breathe for them} until their lungs heal. Doctors are divided about how
long to treat patients with noninvasive oxygen before deciding whether
or not they need a ventilator. Not all Covid-19 patients who go on
ventilators survive, but the devices are thought to be
\href{https://www.nytimes.com/2020/04/26/health/coronavirus-patient-ventilator.html}{lifesaving
in many cases}.

TENTATIVE OR MIXED EVIDENCE\\
Anticoagulants\\
The coronavirus can invade cells in the lining of blood vessels, leading
to tiny clots that can cause strokes and other serious harm.
Anticoagulants are commonly used for other conditions, such as heart
disease, to slow the formation of clots, and doctors sometimes use them
on patients with Covid-19 who have clots. Many clinical trials teasing
out this relationship are now underway. Some of these trials are looking
at whether giving anticoagulants before any sign of clotting is
beneficial.

\hypertarget{pseudoscience-and-fraud}{%
\subsection{Pseudoscience and~Fraud}\label{pseudoscience-and-fraud}}

\emph{False claims about Covid-19 cures abound. The F.D.A. maintains a}
\href{https://www.fda.gov/consumers/health-fraud-scams/fraudulent-coronavirus-disease-2019-covid-19-products}{list}
\emph{of more than 80 fraudulent Covid-19 products, and the W.H.O.}
\href{https://www.who.int/emergencies/diseases/novel-coronavirus-2019/advice-for-public/myth-busters}{debunks}
\emph{many myths about the disease.}

WARNING: DO NOT DO THIS\\
Drinking or injecting bleach and disinfectants\\
In April, President Trump
\href{https://www.nytimes.com/2020/04/24/health/sunlight-coronavirus-trump.html}{suggested}
that disinfectants such as alcohol or bleach might be effective against
the coronavirus if directly injected into the body. His comments were
immediately
\href{https://www.nytimes.com/2020/04/24/us/politics/trump-inject-disinfectant-bleach-coronavirus.html}{refuted}
by health professionals and researchers around the world --- as well as
the
\href{https://www.prweek.com/article/1681380/lysol-clorox-respond-trump-comment-injecting-disinfectant}{makers
of Lysol and Clorox}. Ingesting disinfectant would not only be
ineffective against the virus, but also hazardous --- possibly even
deadly. In July, Federal prosecutors
\href{https://www.wtsp.com/article/news/regional/florida/miracle-mineral-solution-genesis-ii-church-of-health-and-healing/67-b33b7f2e-2b0c-4853-8434-90732359d730}{charged}
four Florida men with marketing bleach as a cure for COVID-19.

WARNING: NO EVIDENCE\\
UV light\\
President Trump also
\href{https://www.nytimes.com/2020/04/24/health/sunlight-coronavirus-trump.html}{speculated}
about hitting the body with ``ultraviolet or just very powerful light.''
Researchers have used UV light to sterilize surfaces, including killing
viruses, in carefully managed laboratories. But UV light would not be
able to purge the virus from within a sick persons' body. This kind of
radiation can also damage the skin. Most skin cancers are a result of
exposure to the UV rays naturally present in sunlight.

WARNING: NO EVIDENCE\\
Silver\\
The F.D.A. has threatened legal action against a host of people claiming
silver-based products are safe and effective against Covid-19 ---
including televangelist
\href{https://www.fda.gov/inspections-compliance-enforcement-and-criminal-investigations/warning-letters/jim-bakker-show-604820-03062020}{Jim
Bakker} and InfoWars host
\href{https://www.fda.gov/inspections-compliance-enforcement-and-criminal-investigations/warning-letters/free-speech-systems-llc-dba-infowarscom-605802-04092020}{Alex
Jones}. Several metals do have
\href{https://www.nytimes.com/article/copper-coronavirus-masks.html}{natural
antimicrobial properties}. But products made from them have not been
shown to prevent or treat the coronavirus.

\hypertarget{tracking-the-coronavirus}{%
\subsection{Tracking the Coronavirus}\label{tracking-the-coronavirus}}

\begin{itemize}
\tightlist
\item
  \href{https://www.nytimes.com/interactive/2020/world/coronavirus-maps.html}{World}
\item
  \href{https://www.nytimes.com/interactive/2020/04/21/world/coronavirus-missing-deaths.html}{World
  Deaths}
\item
  \href{https://www.nytimes.com/interactive/2020/04/23/upshot/five-ways-to-monitor-coronavirus-outbreak-us.html}{U.S.
  Cities}
\item
  \href{https://www.nytimes.com/interactive/2020/05/05/us/coronavirus-death-toll-us.html}{U.S.
  Deaths}
\item
  \href{https://www.nytimes.com/interactive/2020/us/coronavirus-testing.html}{Testing}
\item
  \href{https://www.nytimes.com/interactive/2020/us/coronavirus-nursing-homes.html}{Nursing
  homes}
\item
  \href{https://www.nytimes.com/interactive/2020/nyregion/new-york-city-coronavirus-cases.html}{New
  York City}
\item
  \href{https://www.nytimes.com/interactive/2020/us/states-reopen-map-coronavirus.html}{Reopening}
\item
  \href{https://www.nytimes.com/interactive/2020/science/coronavirus-vaccine-tracker.html}{Vaccines}
\end{itemize}

Countries

\begin{itemize}
\tightlist
\item
  \href{https://www.nytimes.com/interactive/2020/world/americas/brazil-coronavirus-cases.html}{Brazil}
\item
  \href{https://www.nytimes.com/interactive/2020/world/canada/canada-coronavirus-cases.html}{Canada}
\item
  \href{https://www.nytimes.com/interactive/2020/world/europe/france-coronavirus-cases.html}{France}
\item
  \href{https://www.nytimes.com/interactive/2020/world/europe/germany-coronavirus-cases.html}{Germany}
\item
  \href{https://www.nytimes.com/interactive/2020/world/asia/india-coronavirus-cases.html}{India}
\item
  \href{https://www.nytimes.com/interactive/2020/world/europe/italy-coronavirus-cases.html}{Italy}
\item
  \href{https://www.nytimes.com/interactive/2020/world/americas/mexico-coronavirus-cases.html}{Mexico}
\item
  \href{https://www.nytimes.com/interactive/2020/world/europe/spain-coronavirus-cases.html}{Spain}
\item
  \href{https://www.nytimes.com/interactive/2020/world/europe/united-kingdom-coronavirus-cases.html}{U.K.}
\item
  \href{https://www.nytimes.com/interactive/2020/us/coronavirus-us-cases.html}{United
  States}
\end{itemize}

State by state

\begin{itemize}
\tightlist
\item
  \href{https://www.nytimes.com/interactive/2020/us/alabama-coronavirus-cases.html}{Alabama}
\item
  \href{https://www.nytimes.com/interactive/2020/us/alaska-coronavirus-cases.html}{Alaska}
\item
  \href{https://www.nytimes.com/interactive/2020/us/arizona-coronavirus-cases.html}{Arizona}
\item
  \href{https://www.nytimes.com/interactive/2020/us/arkansas-coronavirus-cases.html}{Arkansas}
\item
  \href{https://www.nytimes.com/interactive/2020/us/california-coronavirus-cases.html}{California}
\item
  \href{https://www.nytimes.com/interactive/2020/us/colorado-coronavirus-cases.html}{Colorado}
\item
  \href{https://www.nytimes.com/interactive/2020/us/connecticut-coronavirus-cases.html}{Connecticut}
\item
  \href{https://www.nytimes.com/interactive/2020/us/delaware-coronavirus-cases.html}{Delaware}
\item
  \href{https://www.nytimes.com/interactive/2020/us/florida-coronavirus-cases.html}{Florida}
\item
  \href{https://www.nytimes.com/interactive/2020/us/georgia-coronavirus-cases.html}{Georgia}
\item
  \href{https://www.nytimes.com/interactive/2020/us/hawaii-coronavirus-cases.html}{Hawaii}
\item
  \href{https://www.nytimes.com/interactive/2020/us/idaho-coronavirus-cases.html}{Idaho}
\item
  \href{https://www.nytimes.com/interactive/2020/us/illinois-coronavirus-cases.html}{Illinois}
\item
  \href{https://www.nytimes.com/interactive/2020/us/indiana-coronavirus-cases.html}{Indiana}
\item
  \href{https://www.nytimes.com/interactive/2020/us/iowa-coronavirus-cases.html}{Iowa}
\item
  \href{https://www.nytimes.com/interactive/2020/us/kansas-coronavirus-cases.html}{Kansas}
\item
  \href{https://www.nytimes.com/interactive/2020/us/kentucky-coronavirus-cases.html}{Kentucky}
\item
  \href{https://www.nytimes.com/interactive/2020/us/louisiana-coronavirus-cases.html}{Louisiana}
\item
  \href{https://www.nytimes.com/interactive/2020/us/maine-coronavirus-cases.html}{Maine}
\item
  \href{https://www.nytimes.com/interactive/2020/us/maryland-coronavirus-cases.html}{Maryland}
\item
  \href{https://www.nytimes.com/interactive/2020/us/massachusetts-coronavirus-cases.html}{Massachusetts}
\item
  \href{https://www.nytimes.com/interactive/2020/us/michigan-coronavirus-cases.html}{Michigan}
\item
  \href{https://www.nytimes.com/interactive/2020/us/minnesota-coronavirus-cases.html}{Minnesota}
\item
  \href{https://www.nytimes.com/interactive/2020/us/mississippi-coronavirus-cases.html}{Mississippi}
\item
  \href{https://www.nytimes.com/interactive/2020/us/missouri-coronavirus-cases.html}{Missouri}
\item
  \href{https://www.nytimes.com/interactive/2020/us/montana-coronavirus-cases.html}{Montana}
\item
  \href{https://www.nytimes.com/interactive/2020/us/nebraska-coronavirus-cases.html}{Nebraska}
\item
  \href{https://www.nytimes.com/interactive/2020/us/nevada-coronavirus-cases.html}{Nevada}
\item
  \href{https://www.nytimes.com/interactive/2020/us/new-hampshire-coronavirus-cases.html}{New
  Hampshire}
\item
  \href{https://www.nytimes.com/interactive/2020/us/new-jersey-coronavirus-cases.html}{New
  Jersey}
\item
  \href{https://www.nytimes.com/interactive/2020/us/new-mexico-coronavirus-cases.html}{New
  Mexico}
\item
  \href{https://www.nytimes.com/interactive/2020/us/new-york-coronavirus-cases.html}{New
  York}
\item
  \href{https://www.nytimes.com/interactive/2020/us/north-carolina-coronavirus-cases.html}{North
  Carolina}
\item
  \href{https://www.nytimes.com/interactive/2020/us/north-dakota-coronavirus-cases.html}{North
  Dakota}
\item
  \href{https://www.nytimes.com/interactive/2020/us/ohio-coronavirus-cases.html}{Ohio}
\item
  \href{https://www.nytimes.com/interactive/2020/us/oklahoma-coronavirus-cases.html}{Oklahoma}
\item
  \href{https://www.nytimes.com/interactive/2020/us/oregon-coronavirus-cases.html}{Oregon}
\item
  \href{https://www.nytimes.com/interactive/2020/us/pennsylvania-coronavirus-cases.html}{Pennsylvania}
\item
  \href{https://www.nytimes.com/interactive/2020/us/puerto-rico-coronavirus-cases.html}{Puerto
  Rico}
\item
  \href{https://www.nytimes.com/interactive/2020/us/rhode-island-coronavirus-cases.html}{Rhode
  Island}
\item
  \href{https://www.nytimes.com/interactive/2020/us/south-carolina-coronavirus-cases.html}{South
  Carolina}
\item
  \href{https://www.nytimes.com/interactive/2020/us/south-dakota-coronavirus-cases.html}{South
  Dakota}
\item
  \href{https://www.nytimes.com/interactive/2020/us/tennessee-coronavirus-cases.html}{Tennessee}
\item
  \href{https://www.nytimes.com/interactive/2020/us/texas-coronavirus-cases.html}{Texas}
\item
  \href{https://www.nytimes.com/interactive/2020/us/utah-coronavirus-cases.html}{Utah}
\item
  \href{https://www.nytimes.com/interactive/2020/us/vermont-coronavirus-cases.html}{Vermont}
\item
  \href{https://www.nytimes.com/interactive/2020/us/virginia-coronavirus-cases.html}{Virginia}
\item
  \href{https://www.nytimes.com/interactive/2020/us/washington-coronavirus-cases.html}{Washington}
\item
  \href{https://www.nytimes.com/interactive/2020/us/washington-dc-coronavirus-cases.html}{Washington,
  D.C.}
\item
  \href{https://www.nytimes.com/interactive/2020/us/west-virginia-coronavirus-cases.html}{West
  Virginia}
\item
  \href{https://www.nytimes.com/interactive/2020/us/wisconsin-coronavirus-cases.html}{Wisconsin}
\item
  \href{https://www.nytimes.com/interactive/2020/us/wyoming-coronavirus-cases.html}{Wyoming}
\end{itemize}

Note: After additional discussions with experts we have adjusted several
labels on the tracker. The ``Strong evidence'' label has been removed
until further research identifies treatments that consistently benefit
groups of patients infected by the coronavirus. In its place,
``Promising evidence'' will be used for drugs such as remdesivir and
dexamethasone that have shown promise in at least one randomized
controlled trial, and ``Widely used'' for treatments such as proning and
ventilators that are often used with severely ill patients, including
those with Covid-19. And we may reintroduce the ``Ineffective'' label
when ongoing clinical trials repeatedly end with disappointing results.

Sources: National Library of Medicine; National Institutes of Health;
Paul Knoepfler, University of California, Davis; Phyllis Tien,
University of California, San Francisco; John Moore and Douglas Nixon,
Weill Cornell Medical College; Noah Haber, Stanford University.

\begin{itemize}
\item
\item
\item
\item
\end{itemize}

Advertisement

\protect\hyperlink{after-bottom}{Continue reading the main story}

\hypertarget{site-index}{%
\subsection{Site Index}\label{site-index}}

\hypertarget{site-information-navigation}{%
\subsection{Site Information
Navigation}\label{site-information-navigation}}

\begin{itemize}
\tightlist
\item
  \href{https://help.nytimes.com/hc/en-us/articles/115014792127-Copyright-notice}{©~2020~The
  New York Times Company}
\end{itemize}

\begin{itemize}
\tightlist
\item
  \href{https://www.nytco.com/}{NYTCo}
\item
  \href{https://help.nytimes.com/hc/en-us/articles/115015385887-Contact-Us}{Contact
  Us}
\item
  \href{https://www.nytco.com/careers/}{Work with us}
\item
  \href{https://nytmediakit.com/}{Advertise}
\item
  \href{http://www.tbrandstudio.com/}{T Brand Studio}
\item
  \href{https://www.nytimes.com/privacy/cookie-policy\#how-do-i-manage-trackers}{Your
  Ad Choices}
\item
  \href{https://www.nytimes.com/privacy}{Privacy}
\item
  \href{https://help.nytimes.com/hc/en-us/articles/115014893428-Terms-of-service}{Terms
  of Service}
\item
  \href{https://help.nytimes.com/hc/en-us/articles/115014893968-Terms-of-sale}{Terms
  of Sale}
\item
  \href{https://spiderbites.nytimes.com}{Site Map}
\item
  \href{https://help.nytimes.com/hc/en-us}{Help}
\item
  \href{https://www.nytimes.com/subscription?campaignId=37WXW}{Subscriptions}
\end{itemize}
