Sections

SEARCH

\protect\hyperlink{site-content}{Skip to
content}\protect\hyperlink{site-index}{Skip to site index}

\href{https://www.nytimes.com/news-event/2020-election}{Elections}

\href{https://myaccount.nytimes.com/auth/login?response_type=cookie\&client_id=vi}{}

\href{https://www.nytimes.com/section/todayspaper}{Today's Paper}

\hypertarget{comments}{%
\subsection{\texorpdfstring{\protect\hyperlink{commentsContainer}{Comments}}{Comments}}\label{comments}}

\href{}{Which Democrats Are Leading the 2020 Presidential
Race?}\href{}{Skip to Comments}

The comments section is closed. To submit a letter to the editor for
publication, write to
\href{mailto:letters@nytimes.com}{\nolinkurl{letters@nytimes.com}}.

\hypertarget{which-democrats-are-leading-the-2020-presidential-race}{%
\section{Which Democrats Are Leading the 2020 Presidential
Race?}\label{which-democrats-are-leading-the-2020-presidential-race}}

By \href{https://www.nytimes.com/by/jasmine-c-lee}{Jasmine C. Lee},
\href{https://www.nytimes.com/by/annie-daniel}{Annie Daniel},
\href{https://www.nytimes.com/by/rebecca-lieberman}{Rebecca Lieberman},
\href{https://www.nytimes.com/by/blacki-migliozzi}{Blacki Migliozzi},
\href{https://www.nytimes.com/by/alexander-burns}{Alexander Burns} and
\href{https://www.nytimes.com/by/sarah-almukhtar}{Sarah
Almukhtar}Updated March 20, 2020

\begin{itemize}
\item
\item
\item
\item
\item
  \emph{25}
\end{itemize}

This page is no longer being updated and
\href{https://www.nytimes.com/interactive/2020/us/elections/delegate-count-primary-results.html}{primaries
are delayed} because of the coronavirus outbreak.

Each week, The Times is bringing you the latest political data and
analysis to track the race for the 2020 Democratic presidential
nomination.

Jump to: Overview Polls Campaign Money News Coverage

\hypertarget{current-state-of-the-race}{%
\subsection{Current State of the Race}\label{current-state-of-the-race}}

National polling averageNat. Polling

Pledged delegatesDelegates

Individual contributions†Individual contrib.†

Weekly news coverageWeekly media

\href{https://www.nytimes.com/interactive/2020/us/elections/joe-biden.html}{Joseph
R. Biden Jr.Biden}

52\%

1,185

\$69.7m

\#1

\href{https://www.nytimes.com/interactive/2020/us/elections/bernie-sanders.html}{Bernie
SandersSanders}

34\%

884

\$121.0m

\#2

\href{https://www.nytimes.com/interactive/2020/us/elections/tulsi-gabbard.html}{Tulsi
GabbardGabbard}

Dropped out March 19, 2020

\href{https://www.nytimes.com/interactive/2020/us/elections/elizabeth-warren.html}{Elizabeth
WarrenWarren}

Dropped out March 5

\href{https://www.nytimes.com/interactive/2020/us/elections/michael-bloomberg.html}{Michael
R. BloombergBloomberg}

Dropped out March 4

\href{https://www.nytimes.com/interactive/2020/us/elections/amy-klobuchar.html}{Amy
KlobucharKlobuchar}

Dropped out March 2

\href{https://www.nytimes.com/interactive/2020/us/elections/pete-buttigieg.html}{Pete
ButtigiegButtigieg}

Dropped out March 1

\href{https://www.nytimes.com/interactive/2020/us/elections/tom-steyer.html}{Tom
SteyerSteyer}

Dropped out Feb. 29

\href{https://www.nytimes.com/interactive/2020/us/elections/deval-patrick.html}{Deval
PatrickPatrick}

Dropped out Feb. 12

\href{https://www.nytimes.com/interactive/2020/us/elections/andrew-yang.html}{Andrew
YangYang}

Dropped out Feb. 11

\href{https://www.nytimes.com/interactive/2020/us/elections/michael-bennet.html}{Michael
BennetBennet}

Dropped out Feb. 11

\href{https://www.nytimes.com/interactive/2020/us/elections/john-delaney.html}{John
DelaneyDelaney}

Dropped out Jan. 31

\href{https://www.nytimes.com/interactive/2020/us/elections/cory-booker.html}{Cory
BookerBooker}

Dropped out Jan. 13

\href{https://www.nytimes.com/interactive/2020/us/elections/marianne-williamson.html}{Marianne
WilliamsonWilliamson}

Dropped out Jan. 10

\href{https://www.nytimes.com/interactive/2020/us/elections/julian-castro.html}{Julián
CastroCastro}

Dropped out Jan. 2

\href{https://www.nytimes.com/interactive/2020/us/elections/kamala-harris.html}{Kamala
HarrisHarris}

Dropped out Dec. 3, 2019

Steve BullockBullock

Dropped out Dec. 2

Joe SestakSestak

Dropped out Dec. 1

Wayne MessamMessam

Dropped out Nov. 20

\href{https://www.nytimes.com/interactive/2020/us/elections/beto-orourke.html}{Beto
O'RourkeO'Rourke}

Dropped out Nov. 1

Tim RyanRyan

Dropped out Oct. 24

Bill de Blasiode Blasio

Dropped out Sept. 20

Kirsten GillibrandGillibrand

Dropped out Aug. 28

Seth MoultonMoulton

Dropped out Aug. 23

Jay InsleeInslee

Dropped out Aug. 21

John HickenlooperHickenlooper

Dropped out Aug. 15

Eric SwalwellSwalwell

Dropped out July 8

+ View all candidates

† Campaign finance data through Jan. 31.\\
Arrows show recent changes in value or rank.\\
\href{https://www.nytimes.com/interactive/2020/us/elections/delegate-count-primary-results.html}{Delegate
data} from The Associated Press as of March 20.

\hypertarget{heres-the-latest}{%
\subsection{Here's the latest.}\label{heres-the-latest}}

March 20, 2020

The Democratic race is not officially over, but in every other way the
contest has been resolved. \textbf{Joseph R. Biden Jr.} has taken an
enormous lead over \textbf{Bernie Sanders} in national polls. The former
vice president has amassed
\href{https://www.nytimes.com/interactive/2020/us/elections/delegate-count-primary-results.html}{hundreds
more delegates} than the senator from Vermont, building an advantage
that is all but insurmountable. The only other candidate who was still
running, \textbf{Tulsi Gabbard},
\href{https://www.nytimes.com/2020/03/19/us/politics/tulsi-gabbard-drops-out.html}{dropped
out and endorsed Mr. Biden}.

Mr. Sanders may take days or weeks --- even months --- to concede
defeat. But even his close allies now acknowledge that he has
\href{https://www.nytimes.com/2020/03/18/us/politics/bernie-sanders-campaign.html}{no
realistic shot} of overtaking Mr. Biden. After losing the Florida,
Illinois and Arizona primaries, Mr. Sanders was assessing the future of
his campaign, his campaign manager said, a step that often begins the
process of winding down a candidacy.

There are a few reasons for continued uncertainty: The coronavirus
outbreak has
\href{https://www.nytimes.com/article/2020-campaign-primary-calendar-coronavirus.html}{upended
the remainder of the primary calendar}, causing states to delay
elections well into the spring for reasons of public health. Mr. Biden
and Mr. Sanders have stopped holding campaign events, and Mr. Sanders
has been spending time in Washington as the Senate crafts emergency
legislation to address the outbreak. In such a chaotic environment, it
is impossible to completely rule out some major shift brought on by
entirely unforeseeable events.

But based on all the information we have now, and everything we can
anticipate about the remainder of the primary process, Mr. Biden is an
almost prohibitive favorite to win the Democratic nomination.

His dominance over Mr. Sanders was built mainly on the strengths Mr.
Biden had all along: an image of seasoned experience, a reputation for
empathy and decency in the eyes of many Democrats, and a powerful base
of support among older voters, moderates and, most of all,
African-Americans. As the race advanced and other candidates withdrew,
Mr. Biden created an even broader coalition by winning over larger
numbers of college-educated white voters and liberal women.

A significant bloc of Democrats remains aligned with Mr. Sanders, even
in his beleaguered present state --- perhaps about a third of the party,
according to our national polling average. His most loyal supporters are
young people and ideological progressives, as well as Latino voters in
many parts of the country.

In the coming months, one of the great questions of the 2020 race may be
whether Mr. Biden can manage to win over those people and mobilize them
for the general election --- and how quickly Mr. Sanders might be
persuaded to join forces and help.

--- Alexander Burns

Data through March 19

\hypertarget{who-is-leading-the-polls}{%
\subsection{Who Is Leading the Polls?}\label{who-is-leading-the-polls}}

National polls are a flawed tool for predicting elections. That's even
truer in a primary that unfolds in stages, with one or several states
voting at a time. But the broad national picture is still important,
offering a sense of which candidates are gaining support overall.

\hypertarget{national-polling-average}{%
\subsubsection{National Polling
Average}\label{national-polling-average}}

Candidate polling average

Individual polls shown on hover

Individual polls shown on tap

\hypertarget{latest-national-polls}{%
\subsubsection{Latest National Polls}\label{latest-national-polls}}

Remember, political fortunes can shift rapidly in a national campaign.

\hypertarget{on-march-20-in-previous-election-cycles-}{%
\subsubsection{On March 20 in previous election cycles
...}\label{on-march-20-in-previous-election-cycles-}}

Source: RealClearPolitics

Data through Jan. 31

\hypertarget{who-is-leading-the-money-race}{%
\subsection{Who Is Leading the Money
Race?}\label{who-is-leading-the-money-race}}

Presidential campaigns are expensive, and candidates' ability to compete
often depends on their prowess at collecting large sums of money.
Candidates used to focus on courting a few thousand wealthy individuals;
many now spend more time raising money in small increments from millions
of people online.

These statistics show which candidates are inspiring financial
enthusiasm, either from a cluster of deep-pocketed donors or from a
larger army of supporters.
\href{https://www.nytimes.com/interactive/2020/02/21/us/politics/democratic-fundraising-numbers-february.html}{See
full fund-raising numbers from January 2020 here »}

Contributions, Jan.Contributions, Jan.

Bernie SandersSanders

\$25.1m

Joseph R. Biden Jr.Biden

\$8.9m

Current numbers are as of the Jan. 31 filing deadline. The next filing
deadline is March 20.·Source: Federal Election Commission

Data through March 18

\hypertarget{who-is-getting-news-coverage}{%
\subsection{Who Is Getting News
Coverage?}\label{who-is-getting-news-coverage}}

A candidate's ability to make news and draw the attention of voters ---
and cameras --- is a major asset in any campaign. This statistic tracks
which candidates are breaking through on cable television, which helps
drive perceptions of the race among highly engaged voters and the wider
media.

Being talked about isn't always a good thing: It can also mean a
candidate made a major mistake or confronted damaging information from
his or her past.

\hypertarget{total-mentions-since-2019}{%
\subsubsection{Total Mentions Since
2019}\label{total-mentions-since-2019}}

Mentions are the number of 15-second clips in which a candidate's full
name is mentioned on any of the three cable news networks. A more
detailed methodology can be found
\href{https://blog.gdeltproject.org/the-new-television-explorer-launches/}{here}.·Source:
Internet Archive's Television News Archive via the GDELT Project.

\hypertarget{follow-our-coverage}{%
\subsection{Follow Our Coverage}\label{follow-our-coverage}}

\href{https://www.nytimes.com/interactive/2019/us/politics/2020-presidential-candidates.html}{}

\hypertarget{the-candidates}{%
\subsection{The Candidates}\label{the-candidates}}

Here's who's running for president in 2020.

\href{https://www.nytimes.com/interactive/2019/us/politics/2020-candidate-interviews.html}{}

18 questions. 21 Democrats. Here's what they said.

\href{https://www.nytimes.com/interactive/2020/us/politics/democratic-candidates-20-questions.html}{}

20 (More) Questions With Democrats

\href{https://www.nytimes.com/interactive/2020/us/elections/delegate-count-primary-results.html}{}

\hypertarget{delegate-count}{%
\subsection{Delegate Count}\label{delegate-count}}

See which state each candidate has won and how many delegates are still
available.

\href{https://www.nytimes.com/interactive/2020/02/01/us/elections/democratic-q4-fundraising.html}{}

\hypertarget{campaign-finance}{%
\subsection{Campaign finance}\label{campaign-finance}}

See which Democrats spent the most at the end of 2019.

\href{https://www.nytimes.com/interactive/2020/02/01/us/politics/democratic-presidential-campaign-donors.html}{}

Detailed donor maps show who's powering the Democratic campaigns.

\href{https://www.nytimes.com/newsletters/politics}{}

\hypertarget{sign-up-for-our-politics-newsletter}{%
\subsection{Sign Up for Our Politics
Newsletter}\label{sign-up-for-our-politics-newsletter}}

Join our conversation about the 2020 race.

\href{https://www.nytimes.com/news-event/2020-election}{}

\hypertarget{a-collection-of-our-latest-stories}{%
\subsection{A Collection of Our Latest
Stories}\label{a-collection-of-our-latest-stories}}

Get the latest news on the 2020 race.

\hypertarget{key-dates}{%
\subsection{Key Dates}\label{key-dates}}

2020

July 13-16

Democratic National Convention

Nov. 3

Election Day

\href{https://www.nytimes.com/interactive/2019/us/elections/2020-presidential-election-calendar.html}{View
our full election calendar »}

Note: Reuters was removed from the D.N.C.'s approved pollster list for
the September debate, but its earlier polls are still included for
consistency.

Sources: Polling data from ABC News/The Washington Post, Reuters,
Monmouth University, Quinnipiac University, Fox News, USA Today/Suffolk,
University of New Hampshire, CBS News/YouGov, CNN, The Des Moines
Register, NBC News/The Wall Street Journal, Winthrop University, NPR,
NBC News/Marist. Historical primary polling data from RealClearPolitics.
Campaign finance data from Federal Election Commission. News media
mentions data from Internet Archive's Television News Archive via the
GDELT Project. Delegate data from The Associated Press.

Additional reporting contributed by Rachel Shorey.

\textbf{Correction:}~Feb. 13, 2020

An earlier version of this article misstated the date of the Nevada
Democratic caucuses this month. It is Feb. 22, not Feb. 20.

\textbf{Correction:}~Feb. 13, 2020

An earlier version of this article misstated the time frame of news
mentions per candidate. The mentions are since 2019, not in 2019.

Read 25 Comments

\begin{itemize}
\item
\item
\item
\item
\end{itemize}

Advertisement

\protect\hyperlink{after-bottom}{Continue reading the main story}

\hypertarget{site-index}{%
\subsection{Site Index}\label{site-index}}

\hypertarget{site-information-navigation}{%
\subsection{Site Information
Navigation}\label{site-information-navigation}}

\begin{itemize}
\tightlist
\item
  \href{https://help.nytimes.com/hc/en-us/articles/115014792127-Copyright-notice}{©~2020~The
  New York Times Company}
\end{itemize}

\begin{itemize}
\tightlist
\item
  \href{https://www.nytco.com/}{NYTCo}
\item
  \href{https://help.nytimes.com/hc/en-us/articles/115015385887-Contact-Us}{Contact
  Us}
\item
  \href{https://www.nytco.com/careers/}{Work with us}
\item
  \href{https://nytmediakit.com/}{Advertise}
\item
  \href{http://www.tbrandstudio.com/}{T Brand Studio}
\item
  \href{https://www.nytimes.com/privacy/cookie-policy\#how-do-i-manage-trackers}{Your
  Ad Choices}
\item
  \href{https://www.nytimes.com/privacy}{Privacy}
\item
  \href{https://help.nytimes.com/hc/en-us/articles/115014893428-Terms-of-service}{Terms
  of Service}
\item
  \href{https://help.nytimes.com/hc/en-us/articles/115014893968-Terms-of-sale}{Terms
  of Sale}
\item
  \href{https://spiderbites.nytimes.com}{Site Map}
\item
  \href{https://help.nytimes.com/hc/en-us}{Help}
\item
  \href{https://www.nytimes.com/subscription?campaignId=37WXW}{Subscriptions}
\end{itemize}
