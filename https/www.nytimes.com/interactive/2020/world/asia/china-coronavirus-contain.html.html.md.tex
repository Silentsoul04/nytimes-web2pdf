Sections

SEARCH

\protect\hyperlink{site-content}{Skip to
content}\protect\hyperlink{site-index}{Skip to site index}

\href{https://www.nytimes.com/section/world/asia}{Asia Pacific}

\href{https://myaccount.nytimes.com/auth/login?response_type=cookie\&client_id=vi}{}

\href{https://www.nytimes.com/section/todayspaper}{Today's Paper}

\href{/section/world/asia}{Asia Pacific}\textbar{}How Bad Will the
Coronavirus Outbreak Get? Here Are 6 Key Factors

\url{https://nyti.ms/38Xc1ho}

\begin{itemize}
\item
\item
\item
\item
\item
\item
\end{itemize}

\href{https://www.nytimes.com/news-event/coronavirus?action=click\&pgtype=Article\&state=default\&region=TOP_BANNER\&context=storylines_menu}{The
Coronavirus Outbreak}

\begin{itemize}
\tightlist
\item
  live\href{https://www.nytimes.com/2020/08/02/world/coronavirus-updates.html?action=click\&pgtype=Article\&state=default\&region=TOP_BANNER\&context=storylines_menu}{Latest
  Updates}
\item
  \href{https://www.nytimes.com/interactive/2020/us/coronavirus-us-cases.html?action=click\&pgtype=Article\&state=default\&region=TOP_BANNER\&context=storylines_menu}{Maps
  and Cases}
\item
  \href{https://www.nytimes.com/interactive/2020/science/coronavirus-vaccine-tracker.html?action=click\&pgtype=Article\&state=default\&region=TOP_BANNER\&context=storylines_menu}{Vaccine
  Tracker}
\item
  \href{https://www.nytimes.com/interactive/2020/07/29/us/schools-reopening-coronavirus.html?action=click\&pgtype=Article\&state=default\&region=TOP_BANNER\&context=storylines_menu}{What
  School May Look Like}
\item
  \href{https://www.nytimes.com/live/2020/07/31/business/stock-market-today-coronavirus?action=click\&pgtype=Article\&state=default\&region=TOP_BANNER\&context=storylines_menu}{Economy}
\end{itemize}

Advertisement

\protect\hyperlink{after-top}{Continue reading the main story}

\hypertarget{comments}{%
\subsection{\texorpdfstring{\protect\hyperlink{commentsContainer}{Comments}}{Comments}}\label{comments}}

\href{}{How Bad Will the Coronavirus Outbreak Get? Here Are 6 Key
Factors}\href{}{Skip to Comments}

The comments section is closed. To submit a letter to the editor for
publication, write to
\href{mailto:letters@nytimes.com}{\nolinkurl{letters@nytimes.com}}.

\hypertarget{how-bad-will-the-coronavirus-outbreak-get-here-are-6-key-factors}{%
\section{How Bad Will the Coronavirus Outbreak Get? Here Are 6 Key
Factors}\label{how-bad-will-the-coronavirus-outbreak-get-here-are-6-key-factors}}

By \href{https://www.nytimes.com/by/knvul-sheikh}{Knvul Sheikh},
\href{https://www.nytimes.com/by/derek-watkins}{Derek Watkins},
\href{https://www.nytimes.com/by/jin-wu}{Jin Wu} and
\href{https://www.nytimes.com/by/mika-grondahl}{Mika Gröndahl}Updated
Feb. 28, 2020

\href{https://cn.nytimes.com/china/20200201/china-coronavirus-contain/}{阅读简体中文版}\href{https://cn.nytimes.com/china/20200201/china-coronavirus-contain/zh-hant/}{閱讀繁體中文版}

\begin{itemize}
\item
\item
\item
\item
\item
  \emph{219}
\end{itemize}

As the coronavirus continues to spread around the world, a flurry of
early research is drawing a clearer picture of how the pathogen behaves
and the factors that will determine how much it can be contained.

\protect\hyperlink{transmission}{\textbf{How contagious is the virus?}}

It seems to spread very easily, making containment efforts difficult.

\protect\hyperlink{virulence}{\textbf{How deadly is the virus?}}

It's hard to know yet. But the fatality rate may be more than 1 percent,
much higher than the seasonal flu.

\protect\hyperlink{contagiousperiod}{\textbf{How long does it take to
show symptoms?}}

Typically between five and seven days, allowing the illness to go
undetected.

\protect\hyperlink{travel}{\textbf{How much have infected people
traveled?}}

Enough to spread the outbreak all over the world.

\protect\hyperlink{response}{\textbf{How effective will the response
be?}}

China has slowed new cases for now, but the spread around the world is
accelerating.

\protect\hyperlink{vaccine}{\textbf{How long will it take to develop a
treatment or vaccine?}}

A few drugs are being tested in clinical trials, but a vaccine is still
at least a year away.

While the virus is a serious public health concern, you are very
unlikely to get infected outside of
\href{https://www.nytimes.com/interactive/2020/world/coronavirus-maps.html}{areas
where it is spreading widely}, including China, Italy, Iran and South
Korea. To avoid any viral illness, experts advise
\href{https://www.nytimes.com/2020/01/28/opinion/coronavirus-prevention-tips.html}{washing
your hands} frequently and avoiding your office or school when you're
sick. Most healthy people don't need masks, and hoarding them may
contribute to
\href{https://www.nytimes.com/2020/01/29/health/coronavirus-masks-hoarding.html}{shortages}
for health workers who do need them, experts say.

\hypertarget{how-contagious-is-the-virus}{%
\subsection{How contagious is the
virus?}\label{how-contagious-is-the-virus}}

It seems to spread very easily, making containment efforts difficult.

The scale of an outbreak depends on how quickly and easily a virus is
transmitted from person to person.

The new coronavirus seems to spread very easily, especially in homes,
hospitals, churches, cruise ships and other confined spaces. It is much
more contagious than SARS, another coronavirus that circulated in China
in 2003 and sickened about 8,000 people.

The pathogen can travel through the air, enveloped in tiny respiratory
droplets that are produced when a sick person breathes, talks, coughs or
sneezes.

These droplets fall to the ground within a few feet. That makes the
virus harder to get than pathogens like measles, chickenpox and
tuberculosis, which can travel 100 feet through the air. But it is
easier to catch than H.I.V. or hepatitis, which spread only through
direct contact with the bodily fluids of an infected person.

\hypertarget{how-far-viruses-travel}{%
\subsubsection{How far viruses travel}\label{how-far-viruses-travel}}

Coronaviruses can travel only about six feet from the infected person.
It's unknown how long they live on surfaces.

Some other viruses, like measles, can travel up to 100 feet and stay
alive on surfaces for hours.

Coronaviruses can travel only about six feet from the infected person.
It's unknown how long they live on surfaces.

Some other viruses, like measles, can travel up to 100 feet and stay
alive on surfaces for hours.

Coronaviruses can travel only about six feet from the infected person.
It's unknown how long they live on surfaces.

Some other viruses, like measles, can travel up to 100 feet and stay
alive on surfaces for hours.

Coronaviruses can travel only about six feet from the infected person.
It's unknown how long they live on surfaces.

Some other viruses, like measles, can travel up

to 100 feet and stay alive on surfaces for hours.

Research is still in its early stages, but some estimates suggest that
each person with the new coronavirus could infect between
\href{https://academic.oup.com/jtm/advance-article/doi/10.1093/jtm/taaa021/5735319}{two
and four} people without effective containment measures. That is enough
to sustain and accelate an outbreak, if nothing is done to reduce it.

Here's how that works. In the animation below, a group of five infected
people could spread the virus to about 368 people over just five cycles
of infection.

If 5 people with \textbf{new coronavirus} each infected 2.6 others ...

... there could be 5 sick after 1 generation.

Compare that with a less contagious virus, like the seasonal flu, which
can be slowed by vaccines and immunity from past epidemics. People with
the flu tend to infect 1.3 other individuals, on average. The difference
may seem small, but the result is a striking contrast: Only about 45
people might be infected in the same scenario.

If 5 people with \textbf{seasonal flu} each infected 1.3 others ...

... there could be 5 sick after 1 generation.

The transmission numbers of any disease aren't set in stone. They can
change depending on how much people interact at school, work or
\href{https://www.nytimes.com/2020/02/21/world/asia/south-korea-coronavirus-shincheonji.html}{religious
gatherings}. When global health authorities methodically tracked and
isolated people infected with SARS in 2003, they were able to bring the
average number each sick person infected down to 0.4, enough to stop the
outbreak.

Health authorities around the world are
\href{https://www.nytimes.com/2020/01/29/health/china-coronavirus-outbreak.html}{expending
enormous effort} trying to repeat that. But the number of people
infected globally is rising quickly, with large clusters of cases in
Italy, Iran, Japan and South Korea.

The virus's high rate of transmission means containment measures ---
such as wearing masks, keeping a distance from infected people and
implementing quarantines if people are exposed --- must block more than
60 percent of transmissions in order to effectively control the
outbreak, which is difficult.

Coronavirus cases have far surpassed the rate of new SARS cases in 2003:

75,000 reported cases

Numbers rose drastically after

China changed its diagnostic criteria.

50,000

Covid-19

25,000

SARS

0

Day 20

40

60

80

100

120

140

The first day that W.H.O. received reports of the outbreaks

Covid-19

75,000

reported

cases

Numbers rose drastically

after China changed its

diagnostic criteria.

50,000

25,000

SARS

0

Day 20

40

60

80

100

120

140

The first day that W.H.O. received reports of the outbreaks

Notes: The official World Health Organization case count for SARS was
delayed at the beginning of the outbreak. Some cases were suspected but
not confirmed; SARS is a diagnosis of exclusion, so previously reported
cases may have been discarded after further investigation. New
coronavirus data as of Feb. 27.

\hypertarget{how-deadly-is-the-virus}{%
\subsection{How deadly is the virus?}\label{how-deadly-is-the-virus}}

It's hard to know yet. But the fatality rate may be more than 1 percent,
much higher than the seasonal flu.

This is one of the most important factors in how damaging the outbreak
will be, and one of the least understood.

It's tough to assess the lethality of a new virus. The worst cases are
usually detected first, which can skew our understanding of how likely
patients are to die. People with mild illness may never visit a doctor,
and there may be more cases than China is counting, leading to a lower
death rate than initially thought.

``It's easier to miss mild cases that resolve by themselves than it is
to miss dead people,'' said Dr. Angela Rasmussen, a virologist at
Columbia University's Mailman School of Public Health.

But early research indicates the virus may be significantly more deadly
than the seasonal flu, which kills roughly one in 1,000 people. An
analysis of outcomes for more than
\href{https://jamanetwork.com/journals/jama/fullarticle/2762130}{44,000
confirmed patients in China} found that roughly one in 50 died.
Eighty-one percent of patients infected with the new coronavirus had
mild illness, 14 percent had severe illness and 5 percent had critical
illness, according to the study.

The pathogen is considerably less dangerous than other coronaviruses,
such as MERS, which kills about a third of people who become infected,
and SARS, which kills about 1 in 10. All of the diseases appear to latch
on to proteins on the surface of lung cells, but MERS and SARS seem to
be more destructive to lung tissue.

Here's how the new coronavirus compares with other infectious diseases:

Fatality rate

(log scale)

100\%

Bird flu

Ebola

50

Smallpox

More

deadly

MERS

20

SARS

10

5

Spanish flu

New coronavirus

Most estimates put the

fatality rate below 3\%,

and the number of

transmissions between

2 and 4.

2

Spreads faster

1

Measles

Seasonal

flu

0.1

Polio

Common

cold

2009 flu

Chickenpox

0

1

5

10

15

Average number of people infected by each sick person

Fatality rate

(log scale)

100\%

Bird flu

Ebola

50

Smallpox

MERS

More

deadly

20

SARS

10

5

Spanish flu

2

New coronavirus

Most estimates put the

fatality rate below 3\%,

and the number of

transmissions between

2 and 4.

Spreads faster

1

Seasonal

flu

Measles

0.1

Polio

2009

flu

Common

cold

Chickenpox

0

1

5

10

15

Average number of people infected by each sick person

Fatality rate

(log scale)

100\%

Bird flu

50

Ebola

More

deadly

Smallpox

MERS

20

SARS

Spreads faster

10

5

Spanish flu

2

New coronavirus

Most estimates put the

fatality rate below 3\%,

and the number of

transmissions between

2 and 4.

1

Measles

0.1

Seasonal

flu

Polio

Common

cold

Chickenpox

0

1

5

10

15

Avg. number of people infected by each sick person

Note: Average case-fatality rates and transmission numbers are shown.
Estimates of case-fatality rates can vary, and numbers for the new
coronavirus are preliminary estimates.

The chart above uses a
\href{https://blog.datawrapper.de/weeklychart-logscale/}{logarithmic}
vertical scale: data near the top is compressed into a smaller space to
make the variation between less-deadly diseases easier to see. Diseases
near the top of the chart are much deadlier than those in the middle.

Older people are much more likely to face serious illness than younger
people, the analysis of Chinese patients found. In that study, nearly 15
percent of infected people over 80 died, along with 8 percent of people
in their 70s.
\href{https://www.nytimes.com/2020/02/05/health/coronavirus-children.html}{Very
few young children} seem to be falling ill, a pattern seen with some
other respiratory viruses.

Those numbers could be reduced as more cases are discovered. And it is
possible that death rates at the center of the outbreak in China, where
hospitals were overwhelmed, will end up higher than elsewhere in the
world.

Pathogens can still be very dangerous even if their fatality rate is
low. Even though influenza has a case fatality rate below one per 1,000,
roughly 200,000 people end up hospitalized with the virus each year in
the United States, and about 35,000 people die.

\hypertarget{how-long-does-it-take-to-show-symptoms}{%
\subsection{How long does it take to show
symptoms?}\label{how-long-does-it-take-to-show-symptoms}}

Typically between five and seven days, allowing the illness to go
undetected.

The time it takes for symptoms to appear after a person is infected can
be vital for prevention and control. Known as the incubation period,
this time can allow health officials to quarantine or observe people who
may have been exposed to the virus. But if the incubation period is too
long or too short, these measures may be difficult to implement.

Some illnesses, like influenza, have a short incubation period of two or
three days. People may be shedding infectious virus particles before
they exhibit flu symptoms, making it almost impossible to identify and
isolate people who have the virus. SARS had an incubation period of
about five days, and it took four or five days after symptoms started
before sick people could transmit the virus. That gave officials time to
stop the virus and effectively contain the outbreak.

Officials at the Centers for Disease Control and Prevention estimate
that the new coronavirus has an incubation period of
\href{https://www.cdc.gov/coronavirus/2019-ncov/about/symptoms.html}{two
to 14 days}. When symptoms do start to appear, they can include fever,
cough and difficulty breathing or shortness of breath.

But mild cases may simply resemble the flu or a bad cold, and people may
be able to pass on the new coronavirus
\href{https://www.nytimes.com/2020/02/26/health/coronavirus-asymptomatic.html}{even
before they develop obvious symptoms}.

``That concerns me because it means the infection could elude
detection,'' said Dr. Mark Denison, an infectious disease expert at
Vanderbilt University in Nashville, Tenn.

\hypertarget{how-much-have-infected-people-traveled}{%
\subsection{How much have infected people
traveled?}\label{how-much-have-infected-people-traveled}}

Enough to spread the outbreak all over the world.

Wuhan was a difficult place to contain an outbreak. It has 11 million
people, more than New York City. On an average day, 3,500 passengers
take direct flights from Wuhan to cities in other countries. These
cities were among the first to report cases of the virus outside China.

\hypertarget{passengers-flying-from-wuhan-to-other-countries}{%
\subsubsection{Passengers flying from Wuhan to other
countries}\label{passengers-flying-from-wuhan-to-other-countries}}

\hypertarget{october-to-november-2019}{%
\paragraph{October to November 2019}\label{october-to-november-2019}}

Russia

United Kingdom

Italy

France

United

States

8,000

Japan

23,000

passengers

CHINA

Turkey

Wuhan

South Korea

Taiwan

U.A.E.

Philippines

Malaysia

Thailand

55,000

Indonesia

5,000

25,000

Australia

50,000

passengers

Russia

United Kingdom

Italy

France

United

States

8,000

CHINA

Turkey

Wuhan

Japan

23,000

passengers

U.A.E.

Philippines

Malaysia

Thailand

55,000

Indonesia

5,000

25,000

Australia

50,000

passengers

Japan

23,000

passengers

United

States

8,000

Wuhan

Thailand

55,000

5,000

50,000

passengers

Note: Map shows passenger volume from October to November 2019, the most
recent data available.

Wuhan is also a major transportation hub within China, linked to
Beijing, Shanghai and other major cities by high-speed railways and
domestic airlines. In October and November of last year, close to two
million people flew from Wuhan to other places within China.

Beijing

136,000

CHINA

Wuhan

Shanghai

119,000

Passengers flying from Wuhan

to other cities in China

October to November 2019

5,000

25,000

Hong Kong

18,000

Kunming

95,000 passengers

50,000

passengers

Passengers flying from Wuhan to other cities in China

October to November 2019

Beijing

136,000

Wuhan

Shanghai

119,000

5,000

25,000

50,000

passengers

Kunming

95,000 passengers

Hong Kong

18,000

Passengers flying from Wuhan to

other cities in China

Oct. to Nov. 2019

Beijing

136,000

Wuhan

Shanghai

119,000

5,000

50,000

passengers

Kunming

95,000 passengers

Hong Kong

18,000

Note: Map shows passenger volume from October to November 2019, the most
recent data available. Destinations with fewer than 1,000 passengers are
not shown.

China was not nearly as well connected in 2003 during the SARS outbreak.
Large numbers of migrant workers now travel domestically and
internationally --- to Africa, other parts of Asia and Latin America,
where China is making an enormous infrastructure push with its Belt and
Road Initiative. This travel creates a high risk for outbreaks in
countries with health systems that are not equipped to handle them, like
Zimbabwe, which is facing a worsening hunger and economic crisis.

Over all, China has about four times as many train and air passengers as
it did during the SARS outbreak:

4 billion travelers

Passenger traffic has quadrupled, opening more routes for infection.

3

When SARS broke out, there were about 1 billion travelers.

2

Train

passengers

1

Air passengers

1990

'00

'10

2019

Passenger traffic has quadrupled, opening more routes for infection.

4 billion travelers

3

When SARS broke out, there were about 1 billion travelers.

2

Train

passengers

1

Air passengers

1990

'00

'10

2019

Note: Air travel data includes passengers only on Chinese airlines.

In January, China took the unprecedented step of imposing travel
restrictions on tens of millions of people living in Wuhan and nearby
cities. Some
\href{https://www.nytimes.com/2020/01/26/world/asia/coronavirus-wuhan-china-hubei.html}{experts
questioned the effectiveness of the lockdown}, and Wuhan's mayor
acknowledged that five million people had left the city before the
restrictions began, in the run-up to the Lunar New Year.

``You can't board up a germ. A novel infection will spread,'' said
Lawrence O. Gostin, a law professor at Georgetown University and
director of the World Health Organization Collaborating Center on
National and Global Health Law. ``It will get out; it always does.''

Several countries, including Italy, Iran and South Korea, are already
discovering clusters of cases with no clear ties to the outbreak's
epicenter in China. On Feb. 26, the C.D.C. also reported what it called
possibly
\href{https://www.nytimes.com/2020/02/26/health/coronavirus-cdc-usa.html}{the
first case of community spread in the United
States}\href{https://www.nytimes.com/2020/02/26/health/coronavirus-cdc-usa.html}{.}

\hypertarget{how-effective-will-the-response-be}{%
\subsection{How effective will the response
be?}\label{how-effective-will-the-response-be}}

China has slowed new cases for now, but the spread around the world is
accelerating.

World Health Organization officials have
\href{https://www.who.int/docs/default-source/coronaviruse/who-china-joint-mission-on-covid-19-final-report.pdf}{praised}
China's aggressive response to the virus --- walling off cities, forcing
people to stay home and tracking large numbers of contacts of infected
people --- saying that it helped curb the spread of more cases. The
daily tally of new cases there peaked and then plateaued between Jan. 23
and Feb. 2, and has steadily declined since.

Many countries have also enacted travel restrictions and bans, closing
their doors to people from countries with sustained transmission of the
virus. Governments around the world have been screening incoming
passengers for signs of illness. Airlines and cruise lines have canceled
service to many Asian destinations.

Critics fear those measures won't be enough.

The rate at which transmissions are spreading in several countries makes
it seem ``unlikely that containment will be a strategy that will
completely stop this virus,'' said Clarence Tam, an assistant professor
of infectious diseases at the School of Public Health at the National
University of Singapore.

The ability of nations to
\href{https://www.nytimes.com/2020/02/07/health/hospitals-coronavirus.html}{prepare
for the arrival of coronavirus cases} will depend on the strength of
their health systems; their capacity to test, provide hospital beds,
drugs and respirators for severely ill patients; and their effectiveness
in communicating to the public.

\hypertarget{how-long-will-it-take-to-develop-a-treatment-or-vaccine}{%
\subsection{How long will it take to develop a treatment or
vaccine?}\label{how-long-will-it-take-to-develop-a-treatment-or-vaccine}}

A few drugs are being tested in clinical trials, but a vaccine is still
at least a year away.

There are no approved treatments for any coronavirus diseases, including
the new coronavirus.

Several drugs are
\href{https://www.nytimes.com/2020/02/06/health/coronavirus-treatments.html}{being
tested}, and some initial findings are expected soon. An antiviral
medication called remdesivir appears to be effective in animals, and it
was used to treat the first American patient in Washington State.
Researchers are now testing the drug
\href{https://www.nytimes.com/2020/02/26/health/coronavirus-gilead-drug-trials.html}{in
clinical trials} in the United States, China and other countries.

Several groups are also working to develop a vaccine for the virus in
order to stop the spread of the disease. But vaccines take time.

After the SARS outbreak in 2003, it took researchers about 20 months to
get a vaccine ready for human trials. (The vaccine was never needed,
because the disease was eventually contained.) By the time of the Zika
outbreak in 2015, researchers had brought the development timeline down
to six months.

Now, they hope that work from past outbreaks will help cut the timeline
even further. Researchers have already
studied\href{https://ncbiinsights.ncbi.nlm.nih.gov/2020/01/13/novel-coronavirus/}{}\href{https://ncbiinsights.ncbi.nlm.nih.gov/2020/01/13/novel-coronavirus/}{the
genome of the new coronavirus} and found the proteins that are crucial
for infection. Scientists from the National Institutes of Health, in
Australia and at least three companies are working on vaccine
candidates.

``If we don't run into any unforeseen obstacles, we'll be able to get a
Phase 1 trial going within the next three months,'' said Dr. Anthony
Fauci, director of the National Institute of Allergy and Infectious
Diseases.

Dr. Fauci cautioned that it could still take months, and even years,
after initial trials to conduct extensive testing that can prove a
vaccine is safe and effective. In the best case, a vaccine may become
available to the public
\href{https://www.nytimes.com/2020/01/28/health/coronavirus-vaccine.html}{a
year from now}.

\textbf{Sources:}\\
Data on daily reported cases from the Health Commission of Hubei
Province, National Health Commission of the People's Republic of China
and World Health Organization.

Data on fatality rates and number of transmissions per sick person from
the World Health Organization, U.S. Centers for Disease Control and
Prevention, Johns Hopkins University, Global Health Data Exchange, Food
and Agriculture Organization of the United States, Global Mapping of
Infectious Disease Risk, National Institutes of Health, European Centre
for Disease Prevention and Control, University of Oxford, Korea
Institute of Oriental Medicine, Inserm, Imperial College, Harvard
University, Hong Kong University, Lancaster University and University of
Bern.

Travel volume data from International Air Transport Association DDS and,
via CEIC, China Railway Corporation, National Railway Administration and
Civil Aviation Administration of China.

\textbf{Sources:} Data on daily reported cases from the Health
Commission of Hubei Province, National Health Commission of the People's
Republic of China and World Health Organization.\\
Data on fatality rates and number of transmissions per sick person from
the World Health Organization, U.S. Centers for Disease Control and
Prevention, Johns Hopkins University, Global Health Data Exchange, Food
and Agriculture Organization of the United States, Global Mapping of
Infectious Disease Risk, National Institutes of Health, European Centre
for Disease Prevention and Control, University of Oxford, Korea
Institute of Oriental Medicine, Inserm, Imperial College, Harvard
University, Hong Kong University, Lancaster University and University of
Bern.\\
Travel volume data from International Air Transport Association DDS and,
via CEIC, China Railway Corporation, National Railway Administration and
Civil Aviation Administration of China.

Read 219 Comments

\begin{itemize}
\item
\item
\item
\item
\end{itemize}

Advertisement

\protect\hyperlink{after-bottom}{Continue reading the main story}

\hypertarget{site-index}{%
\subsection{Site Index}\label{site-index}}

\hypertarget{site-information-navigation}{%
\subsection{Site Information
Navigation}\label{site-information-navigation}}

\begin{itemize}
\tightlist
\item
  \href{https://help.nytimes.com/hc/en-us/articles/115014792127-Copyright-notice}{©~2020~The
  New York Times Company}
\end{itemize}

\begin{itemize}
\tightlist
\item
  \href{https://www.nytco.com/}{NYTCo}
\item
  \href{https://help.nytimes.com/hc/en-us/articles/115015385887-Contact-Us}{Contact
  Us}
\item
  \href{https://www.nytco.com/careers/}{Work with us}
\item
  \href{https://nytmediakit.com/}{Advertise}
\item
  \href{http://www.tbrandstudio.com/}{T Brand Studio}
\item
  \href{https://www.nytimes.com/privacy/cookie-policy\#how-do-i-manage-trackers}{Your
  Ad Choices}
\item
  \href{https://www.nytimes.com/privacy}{Privacy}
\item
  \href{https://help.nytimes.com/hc/en-us/articles/115014893428-Terms-of-service}{Terms
  of Service}
\item
  \href{https://help.nytimes.com/hc/en-us/articles/115014893968-Terms-of-sale}{Terms
  of Sale}
\item
  \href{https://spiderbites.nytimes.com}{Site Map}
\item
  \href{https://help.nytimes.com/hc/en-us}{Help}
\item
  \href{https://www.nytimes.com/subscription?campaignId=37WXW}{Subscriptions}
\end{itemize}
