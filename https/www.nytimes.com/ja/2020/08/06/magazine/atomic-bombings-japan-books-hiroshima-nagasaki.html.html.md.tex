Sections

SEARCH

\protect\hyperlink{site-content}{Skip to
content}\protect\hyperlink{site-index}{Skip to site index}

\href{https://myaccount.nytimes.com/auth/login?response_type=cookie\&client_id=vi}{}

\href{https://www.nytimes.com/section/todayspaper}{Today's Paper}

死者の声を運ぶ小舟

\href{https://nyti.ms/30uPW8E}{https://nyti.ms/30uPW8E}

\begin{itemize}
\item
\item
\item
\item
\item
\end{itemize}

Advertisement

\protect\hyperlink{after-top}{Continue reading the main story}

Supported by

\protect\hyperlink{after-sponsor}{Continue reading the main story}

\hypertarget{ux6b7bux8005ux306eux58f0ux3092ux904bux3076ux5c0fux821f}{%
\section{死者の声を運ぶ小舟}\label{ux6b7bux8005ux306eux58f0ux3092ux904bux3076ux5c0fux821f}}

理屈から自由になり、矛盾を受け止める必要に迫られた時、人は自然と文学に心を寄せるようになる

\includegraphics{https://static01.nyt.com/images/2020/08/06/multimedia/ja-06ww2-bombing-ogawa-01/merlin_175276872_dfcc7576-6b4a-46d4-9658-02bc79839fd2-articleLarge.jpg?quality=75\&auto=webp\&disable=upscale}

By 小川洋子

\begin{itemize}
\item
  Aug. 6, 2020
\item
  \begin{itemize}
  \item
  \item
  \item
  \item
  \item
  \end{itemize}
\end{itemize}

\href{https://www.nytimes.com/2020/08/06/magazine/hiroshima-nagasaki-japan-literature.html}{Read
in English}

広島の原爆の日は8月6日。長崎は8月9日。そして終戦の日が8月15日。日本にとって8月は、死者を思う季節である。

本当なら今年、75年めの原爆の日を、私たちは東京オリンピックの期間中に迎えるはずだった。しかし、新型コロナウイルスの蔓延によりオリンピックは延期され、思いがけない静けさの中で人々は、死者のために黙祷を捧げることになった。

1964年の東京オリンピック大会で聖火の最終ランナーを務めたのは、19歳の、無名の陸上選手だった。その青年は、原爆投下の当日、広島で生を受けていた。真っ白いランニングシャツと短パンを身に着け、聖火台に続く長い階段を駆け上がる彼の姿は、実に清潔で、均整がとれ、全身に若々しさが満ちあふれていた。この映像を目にするたび、敗戦からわずか19年で、世界中の人々が集まるスポーツの祭典が日本で催された、という現実に驚かされる。人類が経験したことのない徹底的な破壊の中から誕生した、一人の生命が、炎をなびかせながら、一段一段、火を運んでゆく。最終聖火ランナー選出の裏に、政治的な意図が入り乱れていたとしても、広島で生まれた19歳の青年が放つ生命力には、何のごまかしもなかった。

核兵器廃絶の理想は実現しないままに、やがて、若い肉体が復興の証となる時は過ぎていった。世界で唯一の被爆国として、核兵器の非人道性を訴えてゆくことの難しさに、日本は直面し続けてきた。その難しさは年々、複雑さを増しているように思える。2015年、NHK放送文化研究所が行った原爆意識調査によれば、広島原爆投下の年月日を正確に答えられた割合は、広島市で69%、長崎市で50%、全国では30%にとどまったという。忘却の壁はどんどん高くなってゆく。そう遠くない未来に、被爆した人から直接話を聴ける時代は終わりを迎える。

\includegraphics{https://static01.nyt.com/images/2020/08/06/multimedia/ja-06ww2-bombing-ogawa-03/merlin_175232727_aa8d085c-fad1-4f77-827f-80c9b629b844-articleLarge.jpg?quality=75\&auto=webp\&disable=upscale}

体験者の世代で記憶を途切れさせず、体験していない者がどうやってそれを受け継いでゆくか。二度と繰り返してはならない重大な過ちを犯した時、人類は繰り返し、記憶の継承という問題に立ち向かってきた。数々の戦争、ホロコースト、チェルノブイリ、フクシマ\ldots\ldots。もちろんヒロシマ・ナガサキも例外ではない。

自分が生まれる前、遠いどこかで起こった無関係なはずの事実を、単に知識として得るだけでなく、直接の体験と同様に自らに刻み込み、記憶の小舟に載せて次の世代につなげてゆく。この困難を乗り越えるためには、政治や学問の助けだけでは足りない。なぜなら、他人の記憶を共有するなど、全く非論理的な足掻きだからだ。

ここで文学の力が求められる。理屈から自由になり、矛盾を受け止める必要に迫られた時、人は自然と文学に心を寄せるようになる。文学の言葉を借りてようやく、名前も知らない誰かの痛みに共感できる。あるいは、取り返しのつかない過ちを犯してしまう人間の、愚かさの影が、自らの内にも潜んでいないか、じっと目を凝らすことができるのだ。

私自身、『アンネの日記』や『夜と霧』(V.E.フランクル)や『これが人間か』(プリーモ・レーヴィ)を幾度となく読み返すことで、ナチスドイツ時代を生きた人々の声に、耳を澄ましている。アンネ・フランクにより、たとえ隠れ家に閉じ込められていようとも、人は成長できるという尊い事実を教えられ、フランクルの〝すなわち最もよき人々は帰ってこなかった〞の一文に、強制収容所を生き延びた後に続く苦悩の果てしなさを感じ取った。そんなふうに、今の自分と、自分が存在しない時間がつながった時、人生に新たな地平が拓けるのを実感した。

そしてまた日本文学も、原爆を描き続けている。小説、詩歌、戯曲、ルポルタージュ等々、あらゆるジャンルで原爆というテーマは、特別な場所を占めている。1962年生まれの私にとってなじみ深いのは、例えば、不思議な喋る椅子に導かれ、日めくりカレンダーが6のまま止まった家で、二人の少女が時を超えて交差する童話『ふたりのイーダ』(松谷みよ子)。原爆の後遺症の無残さを描き、日本の文学史には決して欠くことのできない小説となっている『黒い雨』(井伏鱒二)。デビューして間もない20代の大江健三郎が広島を訪れ、過酷な日々を忍耐し続ける被爆者たちに、最も人間的な威厳を見出してゆくさまを記したレポート『ヒロシマ・ノート』\ldots\ldots。挙げてゆけばきりがない。

ここでどうしても取り上げたい小説がある。教科書にも載るほど、日本では大事に読み継がれている、原民喜の『夏の花』。体験者によってまさにその時が記された小説だ。

1905年、広島市生まれの原は文芸誌に詩や小説を発表する生活の中、1944年に妻と死別し、関東での生活を切り上げ、45年の2月、広島に一人帰郷。〝まるで広島の惨劇に遭ふために移つたやうなものだつた〞と自ら書いているとおり、当日、実家のトイレに入っている時、その瞬間を迎える。幸い大きな怪我を免れた原は、手帳にメモを取りながら、燃え盛る広島の町を逃げ惑う。この記録がのちに『夏の花』となる。

Image

原民喜著「夏の花」。自らの被爆体験を元に原爆投下当時の様子と体験を克明に描いた小説 Credit...新潮文庫

小説は原爆投下の前々日、妻の墓参りの場面からはじまる。水を掛け、夏の花を飾った墓は主人公の目に清々しく映る。人生の支えであった最愛の妻の死と、二日後に彼が目にするおびただしい数の死が、同じ死という言葉におさまりきらず、引き裂かれてしまう予感が、書き出しの一ページに悲しく漂っている。

川を目指して避難する主人公の様子を、作者は冷静に細やかに描写してゆく。簡潔な文章には、感情を表す言葉が見当たらない。かつて誰一人目にしたことのない、生々しい現実だけが次から次へと主人公の前に立ち現れ、感情などというあいまいなものを飲み込んでしまうのだ。

男か女か区別もつかないほどに腫れ上がった顔。全体が黒豆の粒々で出来上がっているような黒焦の頭。「水をくれ、水をくれ」と狂いまわる声。「お母さん、お父さん」とかすかに静かな声とともになされる合掌。死体から剥ぎ取られる形見の爪とバンド\ldots\ldots。死臭に満ちる町は、こんなふうに描写されている。〝\ldots\ldots 銀色の虚無のひろがりの中に、路があり、川があり、橋があった、そして、赤むけの膨れ上った屍体がところどころに配置されていた。これは精密巧緻な方法で実現された新地獄\ldots\ldots〞

人間的なものを根こそぎ奪うのが原子爆弾なのだとすれば、この時、言葉も燃え尽きてしまったのかもしれない。しかし何ものかの導きか、原が避難するのに持ち出した非常用の鞄の中には、食料や医薬品と一緒に手帳と鉛筆が入っていた。彼が記したのは、単なる言葉ではない。死に行く者や、死者たちから発せられる、到底言葉にはできない何かを聴き取った印だ。無言のまま去っていかなければならない人々が、確かにここに存在したという証となる痕跡だ。

妻を病で亡くし、孤独の中で原爆に遭った原の創作の礎には、いつも死者たちの無言があった。広場の真ん中に立ち、社会に向かって大きな声で物申すのではなく、言葉を奪われた者の声なき声を言葉にする、という矛盾に黙々と耐えた作家、詩人だった。

原の書いた、『コレガ人間ナノデス』という短い詩がある。怒りや恨みを超え、人間とは思えない姿になってしまった人のか細い声を、ただそっと抱き留める詩だ。

\begin{quote}
コレガ人間ナノデス\\
原子爆弾ニ依ル変化ヲゴラン下サイ\\
肉体ガ恐ロシク膨脹シ\\
男モ女モスベテ一ツノ型ニカヘル\\
オオ ソノ真黒焦ゲノ滅茶苦茶ノ\\
爛レタ顔ノムクンダ唇カラ洩レテ来ル声ハ\\
「助ケテ下サイ」\\
ト カ細イ 静カナ言葉\\
コレガ コレガ人間ナノデス\\
人間ノ顔ナノデス
\end{quote}

これを読む時、ナチスの強制収容所から生還したイタリア人化学者、プリーモ・レーヴィの『これが人間か』を思い出さずにはいられない。冒頭、こんな問いが掲げられている。

\begin{quote}
これが人間か、考えてほしい\\
泥にまみれて働き\\
平安を知らず\\
パンのかけらを争い\\
他人がうなずくだけで死に追いやられるものが。
\end{quote}

私には、互いを知るはずもなかったはずのレーヴィと原、二人の言葉が呼応し合っているように感じられる。ある者は、これが人間か、と問い、ある者は、これが人間なのです、と答える。人間らしくあろうともがく者と、人間らしさを見失うまいとする者が、文学の言葉を通して一つに重なり合い、未来にまで届く思いを響かせている。文学の世界では、単なる無意味な偶然、で済ませられてしまうものの中に、最も大切な真理が映し出される。文学の助けにより、死者の言葉が小舟にすくい上げられ、真実の川を連なって流れてゆく。

Image

原爆投下から約一月後、1945年9月8日の広島の様子Credit...Wayne
Miller/Magnum Photos

偶然、ということで言えば、原民喜は1951年、プリーモ・レーヴィは1987年、生き残った者の使命を果たし終えた、と自ら悟ったかのように、ともに自死している。

今、私の手元に、広島平和記念資料館の収蔵品を撮影した写真集『Hiroshima
Collection』(撮影土田ヒロミ)がある。中学1年生、折免滋(おりめんしげる)君の弁当箱と水筒の写真を見つめている。滋君は動員学徒として作業中に、爆心地から500メートルで被爆。川の土手に積み重ねられた遺体のカバンから、お母さんがこれを発見した。「今日は大豆ご飯だから、昼飯が楽しみだ」と言って出かけたという。弁当箱は歪み、蓋には穴が開き、中身は真っ黒に炭化している。

この小さな箱には、息子を思う母親の愛情と、質素な大豆ご飯を楽しみにしていた少年の無邪気さが詰まっている。たとえ原爆の体験者が一人もいなくなっても、弁当箱が朽ちて化石になっても、小さな箱に潜む声を聴き取ろうとする者がいる限り、記憶は途絶えない。死者の声は永遠であり、人間はそれを運ぶための小舟、つまり文学の言葉を持っているのだから。

Advertisement

\protect\hyperlink{after-bottom}{Continue reading the main story}

\hypertarget{site-index}{%
\subsection{Site Index}\label{site-index}}

\hypertarget{site-information-navigation}{%
\subsection{Site Information
Navigation}\label{site-information-navigation}}

\begin{itemize}
\tightlist
\item
  \href{https://help.nytimes.com/hc/en-us/articles/115014792127-Copyright-notice}{©~2020~The
  New York Times Company}
\end{itemize}

\begin{itemize}
\tightlist
\item
  \href{https://www.nytco.com/}{NYTCo}
\item
  \href{https://help.nytimes.com/hc/en-us/articles/115015385887-Contact-Us}{Contact
  Us}
\item
  \href{https://www.nytco.com/careers/}{Work with us}
\item
  \href{https://nytmediakit.com/}{Advertise}
\item
  \href{http://www.tbrandstudio.com/}{T Brand Studio}
\item
  \href{https://www.nytimes.com/privacy/cookie-policy\#how-do-i-manage-trackers}{Your
  Ad Choices}
\item
  \href{https://www.nytimes.com/privacy}{Privacy}
\item
  \href{https://help.nytimes.com/hc/en-us/articles/115014893428-Terms-of-service}{Terms
  of Service}
\item
  \href{https://help.nytimes.com/hc/en-us/articles/115014893968-Terms-of-sale}{Terms
  of Sale}
\item
  \href{https://spiderbites.nytimes.com}{Site Map}
\item
  \href{https://help.nytimes.com/hc/en-us}{Help}
\item
  \href{https://www.nytimes.com/subscription?campaignId=37WXW}{Subscriptions}
\end{itemize}
