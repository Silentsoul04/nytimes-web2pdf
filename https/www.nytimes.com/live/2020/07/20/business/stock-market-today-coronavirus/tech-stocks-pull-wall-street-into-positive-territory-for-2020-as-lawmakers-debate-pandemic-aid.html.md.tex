Sections

SEARCH

\protect\hyperlink{site-content}{Skip to
content}\protect\hyperlink{site-index}{Skip to site index}

\href{https://myaccount.nytimes.com/auth/login?response_type=cookie\&client_id=vi}{}

\href{https://www.nytimes.com/section/todayspaper}{Today's Paper}

\href{https://www.nytimes.com/news-event/coronavirus}{The Coronavirus
Outbreak}

\begin{itemize}
\tightlist
\item
  live\href{https://www.nytimes.com/2020/08/01/world/coronavirus-covid-19.html}{Latest
  Updates}
\item
  \href{https://www.nytimes.com/interactive/2020/us/coronavirus-us-cases.html}{Maps
  and Cases}
\item
  \href{https://www.nytimes.com/interactive/2020/science/coronavirus-vaccine-tracker.html}{Vaccine
  Tracker}
\item
  \href{https://www.nytimes.com/interactive/2020/07/29/us/schools-reopening-coronavirus.html}{What
  School May Look Like}
\item
  \href{https://www.nytimes.com/live/2020/07/31/business/stock-market-today-coronavirus}{Economy}
\end{itemize}

Last Updated

July 21, 2020, 9:45 a.m. ET

July 21, 2020, 9:45 a.m. ET

\hypertarget{stocks-erase-2020-losses-as-tech-shares-rally}{%
\section{Stocks Erase 2020 Losses as Tech Shares
Rally}\label{stocks-erase-2020-losses-as-tech-shares-rally}}

This briefing is no longer being updated. Follow live updates
\href{https://www.nytimes.com/live/2020/07/21/business/stock-market-today-coronavirus}{here}.

\hypertarget{heres-what-you-need-to-know}{%
\subsubsection{Here's what you need to
know:}\label{heres-what-you-need-to-know}}

\begin{itemize}
\item
  \protect\hyperlink{tech-stocks-pull-wall-street-into-positive-territory-for-2020-as-lawmakers-debate-pandemic-aid}{}

  Tech stocks pull Wall Street into positive territory for 2020 as
  lawmakers debate pandemic aid.
\item
  \protect\hyperlink{congressional-republicans-and-the-white-house-are-still-debating-the-next-aid-bill}{}

  Congressional Republicans and the White House are still debating the
  next aid bill.
\item
  \protect\hyperlink{chinese-companies-are-using-a-contentious-labor-program-for-uighurs-to-satisfy-demand-for-ppe}{}

  Chinese companies are using a contentious labor program for Uighurs to
  satisfy demand for P.P.E.
\item
  \protect\hyperlink{googles-virus-tracing-apps-can-allow-it-to-track-some-users-locations}{}

  Google's virus-tracing apps can allow it to track some users'
  locations.
\item
  \protect\hyperlink{the-feds-bond-buying-program-draws-congressional-scrutiny}{}

  The Fed's bond-buying program draws congressional scrutiny.
\item
  \protect\hyperlink{delta-and-united-are-stepping-up-safety-measures-to-reassure-travelers}{}

  Delta and United are stepping up safety measures to reassure
  travelers.
\item
  \protect\hyperlink{chevron-will-acquire-noble-energy-for-5-billion}{}

  Chevron will acquire Noble Energy for \$5 billion.
\item
  \protect\hyperlink{warner-bros-backs-off-its-aug-12-release-date-for-tenet}{}

  Warner Bros. backs off its Aug. 12 release date for `Tenet.'
\item
  \protect\hyperlink{volkswagen-begins-sales-of-its-crucial-electric-model}{}

  Volkswagen begins sales of its crucial electric model.
\item
  \protect\hyperlink{chief-executives-gird-for-a-long-period-of-economic-pain}{}

  Chief executives gird for a long period of economic pain.
\item
  \protect\hyperlink{catch-up-heres-what-else-is-happening}{}

  Catch up: Here's what else is happening.
\end{itemize}

\hypertarget{tech-stocks-pull-wall-street-into-positive-territory-for-2020-as-lawmakers-debate-pandemic-aid}{%
\subsection{\texorpdfstring{\protect\hyperlink{tech-stocks-pull-wall-street-into-positive-territory-for-2020-as-lawmakers-debate-pandemic-aid}{Tech
stocks pull Wall Street into positive territory for 2020 as lawmakers
debate pandemic
aid.}}{Tech stocks pull Wall Street into positive territory for 2020 as lawmakers debate pandemic aid.}}\label{tech-stocks-pull-wall-street-into-positive-territory-for-2020-as-lawmakers-debate-pandemic-aid}}

Copied to clipboard.

\includegraphics{https://static01.nyt.com/images/2020/07/20/business/20markets-brf-markets1/merlin_174645228_9dc11e84-44d1-4216-8c11-915708f5708b-articleLarge.jpg?quality=75\&auto=webp\&disable=upscale}

Stocks on Wall Street climbed on Monday, pulled higher by a rally in
technology companies, as lawmakers in the United States began to discuss
another coronavirus aid package.

The S\&P 500 rose almost 1 percent to return to positive territory for
2020. \textbf{Amazon}, which jumped nearly 8 percent, and other
technology companies like \textbf{Citrix Systems, Microsoft} and
\textbf{Adobe} were among the best performing stocks in the S\&P 500.

The tech-heavy Nasdaq composite rose 2.5 percent.

Energy stocks also climbed after \textbf{Chevron} announced that it had
agreed to acquire \textbf{Noble Energy}, a Houston-based oil and gas
explorer with an international dimension, for \$5 billion. Noble rose
more than 5 percent, and shares of smaller energy companies like
\textbf{Apache} and \textbf{Hess} saw gains.

This isn't the first time this year that Wall Street has clawed its way
back into positive territory. It did so in June, before the coronavirus
pandemic began to spike again in parts of the United States and other
countries, sending shares lower again.

News this week on plans for more federal spending and on corporate
earnings could set the direction for markets in the days ahead.

Lawmakers have returned to Washington to
\href{https://www.nytimes.com/2020/07/02/business/economy/congress-economy-coronavirus.html}{begin
an intense week of negotiations} over what would be the fourth
significant bailout package since the virus shuttered large swaths of
the U.S. economy earlier this year. The talks come as millions of
Americans are about to see their
\href{https://www.nytimes.com/interactive/2020/04/23/business/economy/unemployment-benefits-stimulus-coronavirus.html}{expanded}
unemployment insurance benefits expire.

The House, controlled by Democrats, has signaled that it wants
\href{https://www.nytimes.com/2020/05/15/us/politics/house-simulus-vote.html}{\$3
trillion} in aid, while the Republican-controlled Senate appears to want
something around \$1 trillion. President Trump has said he's interested
in including a payroll tax cut in the next round of aid.

Companies will also report on the state of their businesses as they
release their results for the three months through June. Earnings
reports are expected from \textbf{Microsoft}, \textbf{Chipotle},
\textbf{Hershey} and \textbf{United Airlines}, among dozens of others.

--- \href{https://www.nytimes.com/by/kevin-granville}{Kevin Granville}
and Mohammed Hadi

\hypertarget{congressional-republicans-and-the-white-house-are-still-debating-the-next-aid-bill}{%
\subsection{\texorpdfstring{\protect\hyperlink{congressional-republicans-and-the-white-house-are-still-debating-the-next-aid-bill}{Congressional
Republicans and the White House are still debating the next aid
bill.}}{Congressional Republicans and the White House are still debating the next aid bill.}}\label{congressional-republicans-and-the-white-house-are-still-debating-the-next-aid-bill}}

Copied to clipboard.

\includegraphics{https://static01.nyt.com/images/2020/07/20/business/20markets-brf-mnuchin/merlin_174766554_bf36d63e-590a-43b6-8ade-a9a687ad1764-articleLarge.jpg?quality=75\&auto=webp\&disable=upscale}

Republicans leaders worked Monday to resolve their differences with
President Trump over the next coronavirus relief package ahead of what
promises to be an intense battle with Democrats.

Senate Republicans and Trump administration officials worked Monday to
resolve their differences over the next coronavirus relief package ahead
of what promises to be an intense battle with Democrats.

They were coalescing around a roughly \$1 trillion package that would
most likely include tax breaks, direct payments and jobless aid for
Americans struggling amid the pandemic; money for schools and health
care; and liability protections for employers and health providers.

But President Trump and congressional leaders remained at odds over
several ideas Mr. Trump is pushing that many Republicans oppose,
including conditioning all new education funding on the resumption of
in-person school, removing funding for testing and tracing efforts
nationwide, and including a payroll tax cut.

The division among Republicans further complicated what was already
expected to be an intense round of negotiations on the package with
Democrats, who have vowed to accept no less than the \$3 trillion
stimulus proposal the House approved in May. Time is of the essence,
with expanded jobless aid for the tens of millions of Americans laid off
during the pandemic set to expire at the end of the month.

The focus of the aid package ``is really about kids and jobs and
vaccines,'' Steven Mnuchin, the Treasury secretary, told reporters
during the Oval Office meeting with Mark Meadows, the White House chief
of staff, and Vice President Mike Pence.

Mr. Mnuchin said that there was urgency to pass something before the
expanded unemployment benefits expired but did not deviate from the
administration's position that any additional funding should provide
only as much as workers were earning on the job. Congress's plodding
speed is more likely to produce a bill for Mr. Trump to sign by early
August.

The bill will represent an opening bid from Republicans, and its overall
cost is almost certain to grow in negotiations with Democrats because it
is likely to exclude any additional money to help state and local
governments avert massive layoffs of public employees amid plunging tax
revenues. The legislation will most likely include additional aid to a
popular federal loan program for small businesses to help maintain
payroll, albeit with more stringent restrictions, as well as aid for
schools and hospitals.

Mr. Mnuchin said he would begin talks with Democrats after Republicans
are briefed on the proposal.

--- \href{https://www.nytimes.com/by/emily-cochrane}{Emily Cochrane},
\href{https://www.nytimes.com/by/jim-tankersley}{Jim Tankersley},
\href{https://www.nytimes.com/by/nicholas-fandos}{Nicholas Fandos} and
\href{https://www.nytimes.com/by/alan-rappeport}{Alan Rappeport}

\hypertarget{advertisement}{%
\subsubsection{Advertisement}\label{advertisement}}

\protect\hyperlink{after-dfp-ad-mid1}{Continue reading the main story}

\hypertarget{chinese-companies-are-using-a-contentious-labor-program-for-uighurs-to-satisfy-demand-for-ppe}{%
\subsection{\texorpdfstring{\protect\hyperlink{chinese-companies-are-using-a-contentious-labor-program-for-uighurs-to-satisfy-demand-for-ppe}{Chinese
companies are using a contentious labor program for Uighurs to satisfy
demand for
P.P.E.}}{Chinese companies are using a contentious labor program for Uighurs to satisfy demand for P.P.E.}}\label{chinese-companies-are-using-a-contentious-labor-program-for-uighurs-to-satisfy-demand-for-ppe}}

Copied to clipboard.

\includegraphics{https://static01.nyt.com/images/2020/07/21/world/asia/cover-uighur-ppe/cover-uighur-ppe-videoSixteenByNineJumbo1600.jpg}

The coronavirus pandemic has companies across China rushing to produce
personal protective equipment. A
\href{https://www.nytimes.com/2020/07/19/world/asia/china-mask-forced-labor.html}{New
York Times visual investigation has found} that some of those companies
are using Uighur labor through a contentious government-sponsored
program that experts say often puts people to work against their will,
and that some of that P.P.E. has ended up in the United States and other
countries.

Uighurs, a largely Muslim ethnic minority, have long been persecuted by
the Chinese government. The government promotes a
\href{https://www.nytimes.com/2019/12/30/world/asia/china-xinjiang-muslims-labor.html}{program
that sends Uighurs and other ethnic minorities} into factory and service
jobs as a way to reduce poverty, but quotas on the number of workers to
be transferred and penalties for those who refuse mean that
participation is often coerced.

In Xinjiang, the region where the majority of Uighurs live, only four
companies produced medical-grade protective equipment before the
pandemic. Now there are 51, and at least 17 of them participate in the
program.

The Times spent months sifting through Chinese state media, government
documents, satellite images and shipping data to establish how Uighur
labor is being used to produce P.P.E. and where that equipment is ending
up.

These companies produce P.P.E. primarily for domestic use, but The Times
identified several other companies using Uighur labor that are exporting
around the world. One shipment of masks was traced from a factory in
China's Hubei province, where Uighur workers are subject to mandatory
flag-raising ceremonies and Mandarin language classes, all the way to a
medical supply company in the U.S. state of Georgia.

--- \href{https://www.nytimes.com/by/muyi-xiao}{Muyi Xiao},
\href{https://www.nytimes.com/by/haley-willis}{Haley Willis},
\href{https://www.nytimes.com/by/christoph-koettl}{Christoph Koettl},
\href{https://www.nytimes.com/by/natalie-reneau}{Natalie Reneau} and
\href{https://www.nytimes.com/by/drew-jordan}{Drew Jordan}

\hypertarget{googles-virus-tracing-apps-can-allow-it-to-track-some-users-locations}{%
\subsection{\texorpdfstring{\protect\hyperlink{googles-virus-tracing-apps-can-allow-it-to-track-some-users-locations}{Google's
virus-tracing apps can allow it to track some users'
locations.}}{Google's virus-tracing apps can allow it to track some users' locations.}}\label{googles-virus-tracing-apps-can-allow-it-to-track-some-users-locations}}

Copied to clipboard.

\includegraphics{https://static01.nyt.com/images/2020/07/17/business/00google-virus-apps1/merlin_173941842_65ad5822-5d1d-4017-bf02-88404a880ad1-articleLarge.jpg?quality=75\&auto=webp\&disable=upscale}

When Google and Apple announced plans in April for
\href{https://www.nytimes.com/2020/04/10/technology/apple-google-coronavirus-contact-tracing.html}{free
software} to help alert people of their possible exposure to the
coronavirus, the companies promoted it as
``\href{https://www.apple.com/covid19/contacttracing}{privacy
preserving}'' and said it
\href{https://covid19-static.cdn-apple.com/applications/covid19/current/static/contact-tracing/pdf/ExposureNotification-FAQv1.1.pdf}{would
not track} users' locations. Encouraged by those guarantees,
\href{https://play.google.com/store/apps/details?id=de.rki.coronawarnapp}{Germany},
\href{https://www.thelocal.ch/20200609/how-will-switzerlands-coronavirus-tracing-app-work}{Switzerland}
and other countries used the code to develop national virus alert apps
that have been downloaded more than
\href{https://www.zdnet.com/article/germanys-contact-tracing-app-gets-downloaded-6-5-million-times-in-a-day/}{20
million times}.

But for the apps to work on smartphones with Google's Android operating
system --- the most popular in the world --- users must first turn on
the device location setting, which enables GPS and may allow Google to
determine their locations.

Some government officials seemed surprised that the company could detect
Android users' locations. After learning about it, Cecilie Lumbye
Thorup, a spokeswoman for Denmark's Health Ministry, said her agency
intended to ``start a dialogue with Google about how they in general use
location data.''

Switzerland said it had pushed Google for weeks to alter the location
setting requirement.

Google's location requirement adds to the slew of
\href{https://www.nytimes.com/2020/07/08/technology/virus-tracing-apps-privacy.html}{privacy
and security concerns} with virus-tracing apps, many of which were
developed by governments before the new Apple-Google software became
available. Government officials and epidemiologists say the apps can be
a helpful complement to public health efforts to stem the pandemic. But
human rights groups and technologists have warned that aggressive data
collection and security flaws in many apps put hundreds of millions of
people at risk for stalking, scams, identity theft or oppressive
government tracking.

--- \href{https://www.nytimes.com/by/natasha-singer}{Natasha Singer}

\hypertarget{the-feds-bond-buying-program-draws-congressional-scrutiny}{%
\subsection{\texorpdfstring{\protect\hyperlink{the-feds-bond-buying-program-draws-congressional-scrutiny}{The
Fed's bond-buying program draws congressional
scrutiny.}}{The Fed's bond-buying program draws congressional scrutiny.}}\label{the-feds-bond-buying-program-draws-congressional-scrutiny}}

Copied to clipboard.

\includegraphics{https://static01.nyt.com/images/2020/07/20/business/20markets-brf-bonds/merlin_82263362_a6d4973d-5b23-4703-a0e5-40284762648c-articleLarge.jpg?quality=75\&auto=webp\&disable=upscale}

The Federal Reserve's effort to keep the corporate debt market running
smoothly continued to raise questions on Monday, with a Congressional
Oversight Commission
\href{https://www.toomey.senate.gov/files/documents/Oversight\%20Commission\%20-\%203rd\%20Report\%20(FINAL)_7.20.20.pdf}{report}
asking whether it was a necessary use of government funds.

The Fed announced plans in late March and early April to buy corporate
bonds, an effort backed by funding that Congress supplied to the
Treasury Department. It
\href{https://www.nytimes.com/2020/05/12/business/economy/fed-corporate-debt-coronavirus.html}{began
those purchases}in May with exchange-traded funds, bundles of debt that
trade like stocks, and more recently began to buy already-outstanding
individual
\href{https://www.nytimes.com/2020/06/30/business/economy/federal-reserve-jerome-powell-corporate-bonds.html}{corporate
bonds}.

Senator Patrick J. Toomey, Republican of Pennsylvania and a member of
the commission that oversees the Fed and Treasury's various
taxpayer-backed coronavirus response programs, has questioned whether
those purchases are necessary, given that corporate bond markets are
functioning well and big companies are
\href{https://www.nytimes.com/2020/07/20/business/corporate-debt-federal-reserve.html}{issuing
bonds at a rapid pace}. The skepticism reared its head again in the
latest oversight report.

``We recognize the importance of the Federal Reserve following through
on its commitments,'' the report said. ``At the same time, the secondary
market for corporate bonds is functioning well already and continued
Federal Reserve intervention can have distortionary effects in both the
short term and the long term.''

Jerome H. Powell, the Federal Reserve chair, has said that it was
important for the central bank to follow through on its announcements.
But he has noted that the Fed did not want ``to run through the bond
market like an elephant'' snuffing out price signals.

The Fed has also noted that it has decreased the pace of purchases and
stands ready to slow them even more.

The commission also raised concerns about another stimulus measure ---
Treasury's decision to give \textbf{YRC Worldwide}, a trucking company,
a \$700 million loan.

The loan money was given on the basis that YRC was important for
national security reasons. The government got a 30 percent stake in the
company in exchange for the loan.

But the commission said that it did not appear that YRC met the national
security criteria. The commission also expressed concern that YRC had
faced financial struggles for years, suggesting that the government
could incur losses.

--- \href{https://www.nytimes.com/by/jeanna-smialek}{Jeanna Smialek} and
\href{https://www.nytimes.com/by/alan-rappeport}{Alan Rappeport}

\hypertarget{advertisement-1}{%
\subsubsection{Advertisement}\label{advertisement-1}}

\protect\hyperlink{after-dfp-ad-mid2}{Continue reading the main story}

\hypertarget{delta-and-united-are-stepping-up-safety-measures-to-reassure-travelers}{%
\subsection{\texorpdfstring{\protect\hyperlink{delta-and-united-are-stepping-up-safety-measures-to-reassure-travelers}{Delta
and United are stepping up safety measures to reassure
travelers.}}{Delta and United are stepping up safety measures to reassure travelers.}}\label{delta-and-united-are-stepping-up-safety-measures-to-reassure-travelers}}

Copied to clipboard.

\includegraphics{https://static01.nyt.com/images/2020/07/20/business/20markets-brf-united/merlin_174361755_11cb93d5-e41b-47fe-bed0-d8e3997e6260-articleLarge.jpg?quality=75\&auto=webp\&disable=upscale}

In an effort to improve confidence in air travel, \textbf{United
Airlines} said on Monday that it would step up air filtration as
passengers board and disembark, and \textbf{Delta Air Lines} started
screening travelers unable to wear masks because of health conditions.

Airlines continue to lose tens of millions of dollars every day, and
several, including United, will offer an outlook this week on what is
expected to be a long and uneven recovery.

United, \textbf{American Airlines} and \textbf{Southwest Airlines} will
report financial results from the industry's disastrous second quarter
in the next few days. Last week, Delta said its revenue during those
three months fell about 88 percent compared with the same period last
year.

United said that it would leave its high-efficiency particulate air, or
HEPA, filtration systems running as passengers get on and off most
planes. The move, which it will put into place next week, is intended to
maximize air flow.

``The quality of the air, combined with a strict mask policy and
regularly disinfected surfaces, are the building blocks toward
preventing the spread of Covid-19 on an airplane,'' the airline's chief
executive, Scott Kirby, said in a statement. ``We expect that air travel
is not likely to get back to normal until we're closer to a widely
administered vaccine --- so we're in this for the long haul.''

Starting Monday, Delta **** said it would require passengers unable to
wear face masks because of health conditions to undergo a private
medical consultation by phone before boarding. Passengers who falsify
health claims could be barred from future flights.

Delta and Southwest are limiting the number of passengers on their
flights through at least September in an effort to leave middle seats
empty. United, which is not capping passengers, has described such
policies as marketing strategies. American is also not limiting the
number of passengers on its flights. Of course, no policy can guarantee
protection from germs spread by nearby passengers at the gate, during
boarding or in flight.

Airline executives say the recovery will take years to unfold, with tens
of thousands of employees expected to lose their jobs in the coming
months. United has warned that it could furlough up to 36,000 people
when federal stimulus funding expires this fall. American could furlough
as many as 20,000. Delta and Southwest have said that they may follow
suit.

Delta's pilots union said Monday that 2,235 of the airline's 14,000
pilots volunteered for early retirement during a sign-up period that
ended Sunday. It was not immediately clear how that would affect the
nearly 2,600 pilots who were warned of a possible furlough weeks ago.
The airline told its pilots on Friday that it would~avoid a furlough for
a year if they agreed to a 15 percent cut to guaranteed pay, a move that
the union criticized as sidestepping negotiations.

--- \href{https://www.nytimes.com/by/niraj-chokshi}{Niraj Chokshi}

\hypertarget{chevron-will-acquire-noble-energy-for-5-billion}{%
\subsection{\texorpdfstring{\protect\hyperlink{chevron-will-acquire-noble-energy-for-5-billion}{Chevron
will acquire Noble Energy for \$5
billion.}}{Chevron will acquire Noble Energy for \$5 billion.}}\label{chevron-will-acquire-noble-energy-for-5-billion}}

Copied to clipboard.

\includegraphics{https://static01.nyt.com/images/2020/07/17/us/20markets-brf-chevronnoble/20markets-brf-chevronnoble-articleLarge-v3.jpg?quality=75\&auto=webp\&disable=upscale}

\textbf{Chevron}, the American oil giant, said on Monday that it had
agreed to acquire \textbf{Noble Energy}, a Houston-based oil and gas
explorer with an international dimension, for \$5 billion.

Smaller oil and gas firms have been hit hard by the effects of the
coronavirus pandemic, making them look like potential bargains. Noble's
share price is down about 60 percent from the beginning of the year,
allowing Chevron to buy its oil and gas resources at a low cost.
Chevron, based in San Ramon, Calif., said the deal would add around 18
percent to its oil and gas reserves at a cost of less than \$5 a barrel.

``This is a cost-effective opportunity for Chevron,'' said Michael
Wirth, the company's chairman and chief executive.

Noble would bring Chevron properties in shale drilling regions in the
United States.

The deal would also give Chevron a leading position in potentially
lucrative if politically daunting natural gas fields that have been
discovered in the Eastern Mediterranean region.
\href{https://www.nytimes.com/2014/12/15/business/energy-environment/israels-natural-gas-supply-offers-lifeline-for-peace.html?searchResultPosition=1}{Noble
has led the way in developing resources in Israeli waters}, shrugging
off geopolitical risks that once kept other companies out.

Noble has also made a discovery off Cyprus, where western companies are
looking for gas, but where a simmering dispute between Cyprus and Turkey
presents an obstacle to developing the island's resources.

Last year, Chevron agreed to acquire \textbf{Anadarko Petroleum},
another explorer, but opted for taking a \$1 billion termination fee
\href{https://www.nytimes.com/2019/05/09/business/chevron-anadarko-occidental-permian.html?searchResultPosition=5}{when
Occidental Petroleum topped its bid}.

--- \href{https://www.nytimes.com/by/stanley-reed}{Stanley Reed}

\hypertarget{warner-bros-backs-off-its-aug-12-release-date-for-tenet}{%
\subsection{\texorpdfstring{\protect\hyperlink{warner-bros-backs-off-its-aug-12-release-date-for-tenet}{Warner
Bros. backs off its Aug. 12 release date for
`Tenet.'}}{Warner Bros. backs off its Aug. 12 release date for `Tenet.'}}\label{warner-bros-backs-off-its-aug-12-release-date-for-tenet}}

Copied to clipboard.

\includegraphics{https://static01.nyt.com/images/2020/07/20/business/20markets-brf-tenet/merlin_173956605_dcb41441-97dd-4e97-842d-50df44f48a84-articleLarge.jpg?quality=75\&auto=webp\&disable=upscale}

With coronavirus cases still on the rise in the United States,
\textbf{Warner Bros.} announced on Monday that it was abandoning its
Aug. 12 release date for Christopher Nolan's film ``Tenet,'' the
one-time marker for when Hollywood hoped moviegoing would return in
earnest.

The studio, which has been delicately trying to balance its desire to
return its movies to theaters with the realities of the global pandemic,
did not announce a new release date for the film, which stars John David
Washington and Robert Pattinson.

The studio will move its upcoming installment of the horror film ``The
Conjuring 3'' to June 4, 2021, from Sept. 10. Release dates for ``Wonder
Woman 1984'' (Oct. 2) and ``Dune'' (Dec. 18) remain as scheduled.

``Our goals throughout this process have been to ensure the highest odds
of success for our films while also being ready to support our theater
partners with new content as soon as they could safely reopen,'' said
Toby Emmerich, the chairman of Warner Bros. Pictures Group. ``We are not
treating `Tenet' like a traditional global day-and-date release, and our
upcoming marketing and distribution plans will reflect that.''

``Tenet'' had originally been scheduled to come out on July 17 before
the pandemic hit. The next big-budget film that is set to be released is
Disney's ``Mulan,'' scheduled for Aug. 21. Disney has not yet said
whether it will push the movie's release back again.

Warner Bros. did not offer concrete details for the release of
``Tenet,'' but it is likely that the studio will open the movie in the
locations around the world where it is safe to do so before unveiling it
in the United States.

--- \href{https://www.nytimes.com/by/nicole-sperling}{Nicole Sperling}

\hypertarget{advertisement-2}{%
\subsubsection{Advertisement}\label{advertisement-2}}

\protect\hyperlink{after-dfp-ad-mid3}{Continue reading the main story}

\hypertarget{volkswagen-begins-sales-of-its-crucial-electric-model}{%
\subsection{\texorpdfstring{\protect\hyperlink{volkswagen-begins-sales-of-its-crucial-electric-model}{Volkswagen
begins sales of its crucial electric
model.}}{Volkswagen begins sales of its crucial electric model.}}\label{volkswagen-begins-sales-of-its-crucial-electric-model}}

Copied to clipboard.

\includegraphics{https://static01.nyt.com/images/2020/07/20/business/20markets-brf-volkswagen1/merlin_160607259_48bb1464-bcdf-4bc7-96ac-520cb72322bf-articleLarge.jpg?quality=75\&auto=webp\&disable=upscale}

Volkswagen began taking orders Monday for its first mass-produced
electric vehicle, an attempt by the world's largest carmaker to fend off
an increasingly serious challenge from Tesla.

The
\href{https://www.nytimes.com/2019/09/08/business/volkswagen-trademark-electric-vehicles.html}{ID.3},
a four-seat hatchback, is Volkswagen's bid to make electric cars as
accessible as a Golf, but the introduction was delayed for several
months by software problems. The entry level model has a list price in
Germany of 35,575 euros including sales tax, or \$40,700, about 10,000
euros less than a Tesla Model 3. With government subsidies, a temporary
reduction of the German value-added tax and manufacturer discounts, the
price of the ID.3 falls to 26,000 euros.

The ID.3 is not likely to be very profitable for Volkswagen. But the
company needs to establish credibility as a maker of electric vehicles
to prevent Tesla from dominating the nascent market. Tesla's stock
market value has soared recently, and is more than three times that of
Volkswagen, an indication of who investors think owns the future.

Sales of electric cars in Europe have held up better in the pandemic
than those of conventional vehicles, in part because carmakers are
offering discounts to meet stricter quotas on carbon dioxide emissions.

Volkswagen has not announced plans to export the ID.3 to the United
States, saying its first electric vehicle in the country will be an
S.U.V. Germans who order an ID.3 now will receive their cars in October,
Volkswagen said.

--- \href{https://www.nytimes.com/by/jack-ewing}{Jack Ewing}

\hypertarget{chief-executives-gird-for-a-long-period-of-economic-pain}{%
\subsection{\texorpdfstring{\protect\hyperlink{chief-executives-gird-for-a-long-period-of-economic-pain}{Chief
executives gird for a long period of economic
pain.}}{Chief executives gird for a long period of economic pain.}}\label{chief-executives-gird-for-a-long-period-of-economic-pain}}

Copied to clipboard.

\includegraphics{https://static01.nyt.com/images/2020/07/21/business/21virus-ceos-print01/00virus-ceos-articleLarge.jpg?quality=75\&auto=webp\&disable=upscale}

Chief executives of major American companies are bracing for prolonged
pain as the pandemic keeps squeezing the economy.

Arne Sorenson of Marriott International said he was ``less optimistic
than I was 30 days ago.'' Ed Bastian of Delta Air Lines said he was now
taking a ``more cautious view.'' Julia Hartz of Eventbrite said she was
expecting ``one step forward, two steps back.''

As coronavirus cases spike in the South and West, many top executives
believe that reopening plans will be disrupted and a return to normalcy
will be difficult, especially amid high unemployment and the lack of a
vaccine.

Executives from CVS and Chipotle have weighed in on how best to proceed,
pushing for more mask-wearing and a boost in government aid in the form
of expanded jobless benefits and small business support. ``Getting this
wrong --- overreacting or acting irresponsibly --- could be far more
devastating to the global economy and the health of Americans,'' said
Jamie Dimon, the chief executive of JPMorgan Chase.

--- \href{https://www.nytimes.com/by/david-gelles}{David Gelles}

\hypertarget{catch-up-heres-what-else-is-happening}{%
\subsection{\texorpdfstring{\protect\hyperlink{catch-up-heres-what-else-is-happening}{Catch
up: Here's what else is
happening.}}{Catch up: Here's what else is happening.}}\label{catch-up-heres-what-else-is-happening}}

Copied to clipboard.

\begin{itemize}
\item
  \textbf{Gap}, the owner of its namesake chain, Old Navy, Banana
  Republic and Intermix, said it would require customers to wear masks
  in its North American stores. \textbf{Ulta Beauty} also said it would
  require masks starting on Monday, following similar announcements last
  week from major retailers like \textbf{Walmart}, \textbf{Best Buy},
  \textbf{Home Depot} and \textbf{Lowe's}. Gap said the policy would
  apply to everyone but small children or those who were exempt because
  of an underlying medical condition.
\item
  The outdoor equipment co-op \textbf{REI}
  \href{https://www.nytimes.com/2020/07/19/business/coronavirus-rei-staff.html}{is
  facing a backlash} over its handling of coronavirus cases among its
  employees. An employee at a store in Grand Rapids, Mich., told
  colleagues about testing positive on July 2 for Covid-19 --- a result
  that managers did not acknowledge to workers until July 9, several
  days after the store had reopened. The employee described being asked
  to stay quiet about the positive test. REI changed its notification
  policy on Tuesday to allow managers to inform employees of known
  Covid-19 cases among their colleagues, after employees created an
  online petition accusing the co-op of prioritizing sales above
  workers.
\item
  \textbf{Ant Group}, the financial affiliate of the e-commerce giant
  \textbf{Alibaba}, said Monday that it was planning an initial public
  offering in Hong Kong and Shanghai. Ant competes primarily with
  China's other internet giant, Tencent, to provide payments and other
  financial services. The I.P.O. is expected to be big --- the company
  was valued at \$150 billion two years ago --- and it offers a ray of
  light for Hong Kong, which has struggled under the dual uncertainties
  of the pandemic and a new national security law.
\item
  \textbf{Marks \& Spencer}, the British high-street retailer, said it
  would cut 950 jobs as the company speeds up a restructuring plan. The
  cuts will be in its head office and in stores. The company said on
  Monday that the accelerated plan, intended to squeeze three years of
  change into a single year, would make store managers more accountable
  for profits and losses and customer service.
\end{itemize}

\hypertarget{site-index}{%
\subsection{Site Index}\label{site-index}}

\hypertarget{site-information-navigation}{%
\subsection{Site Information
Navigation}\label{site-information-navigation}}

\begin{itemize}
\tightlist
\item
  \href{https://help.nytimes.com/hc/en-us/articles/115014792127-Copyright-notice}{©~2020~The
  New York Times Company}
\end{itemize}

\begin{itemize}
\tightlist
\item
  \href{https://www.nytco.com/}{NYTCo}
\item
  \href{https://help.nytimes.com/hc/en-us/articles/115015385887-Contact-Us}{Contact
  Us}
\item
  \href{https://www.nytco.com/careers/}{Work with us}
\item
  \href{https://nytmediakit.com/}{Advertise}
\item
  \href{http://www.tbrandstudio.com/}{T Brand Studio}
\item
  \href{https://www.nytimes.com/privacy/cookie-policy\#how-do-i-manage-trackers}{Your
  Ad Choices}
\item
  \href{https://www.nytimes.com/privacy}{Privacy}
\item
  \href{https://help.nytimes.com/hc/en-us/articles/115014893428-Terms-of-service}{Terms
  of Service}
\item
  \href{https://help.nytimes.com/hc/en-us/articles/115014893968-Terms-of-sale}{Terms
  of Sale}
\item
  \href{https://spiderbites.nytimes.com}{Site Map}
\item
  \href{https://help.nytimes.com/hc/en-us}{Help}
\item
  \href{https://www.nytimes.com/subscription?campaignId=37WXW}{Subscriptions}
\end{itemize}
