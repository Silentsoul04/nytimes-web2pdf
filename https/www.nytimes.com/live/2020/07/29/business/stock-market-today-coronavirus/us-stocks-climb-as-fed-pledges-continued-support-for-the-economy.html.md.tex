Sections

SEARCH

\protect\hyperlink{site-content}{Skip to
content}\protect\hyperlink{site-index}{Skip to site index}

\href{https://myaccount.nytimes.com/auth/login?response_type=cookie\&client_id=vi}{}

\href{https://www.nytimes.com/section/todayspaper}{Today's Paper}

\href{https://www.nytimes.com/news-event/coronavirus}{The Coronavirus
Outbreak}

\begin{itemize}
\tightlist
\item
  live\href{https://www.nytimes.com/2020/08/01/world/coronavirus-covid-19.html}{Latest
  Updates}
\item
  \href{https://www.nytimes.com/interactive/2020/us/coronavirus-us-cases.html}{Maps
  and Cases}
\item
  \href{https://www.nytimes.com/interactive/2020/science/coronavirus-vaccine-tracker.html}{Vaccine
  Tracker}
\item
  \href{https://www.nytimes.com/interactive/2020/07/29/us/schools-reopening-coronavirus.html}{What
  School May Look Like}
\item
  \href{https://www.nytimes.com/live/2020/07/31/business/stock-market-today-coronavirus}{Economy}
\end{itemize}

Last Updated

July 30, 2020, 6:18 a.m. ET

July 30, 2020, 6:18 a.m. ET

\hypertarget{stocks-rise-as-fed-reiterates-support-for-the-economy}{%
\section{Stocks Rise as Fed Reiterates Support for the
Economy}\label{stocks-rise-as-fed-reiterates-support-for-the-economy}}

This briefing is no longer being updated. Follow the latest developments
\href{https://www.nytimes.com/live/2020/07/30/business/stock-market-today-coronavirus}{here}.

\hypertarget{heres-what-you-need-to-know}{%
\subsubsection{Here's what you need to
know:}\label{heres-what-you-need-to-know}}

\begin{itemize}
\item
  \protect\hyperlink{5-takeaways-from-wednesdays-fed-meeting-and-news-conference}{}

  5 takeaways from Wednesday's Fed meeting and news conference.
\item
  \protect\hyperlink{democrats-accused-big-tech-of-stifling-competition-republicans-accused-big-tech-of-stifling-speech}{}

  Democrats accused Big Tech of stifling competition. Republicans
  accused Big Tech of stifling speech.
\item
  \protect\hyperlink{us-employment-picked-up-last-week-new-data-shows}{}

  U.S. employment picked up last week, new data shows.
\item
  \protect\hyperlink{kodak-stock-soars-after-765-million-loan-to-make-drug-ingredients}{}

  Kodak stock soars after \$765 million loan to make drug ingredients.
\item
  \protect\hyperlink{us-stocks-climb-as-fed-pledges-continued-support-for-the-economy}{}

  U.S. stocks climb as Fed pledges continued support for the economy.
\end{itemize}

\hypertarget{5-takeaways-from-wednesdays-fed-meeting-and-news-conference}{%
\subsection{\texorpdfstring{\protect\hyperlink{5-takeaways-from-wednesdays-fed-meeting-and-news-conference}{5
takeaways from Wednesday's Fed meeting and news
conference.}}{5 takeaways from Wednesday's Fed meeting and news conference.}}\label{5-takeaways-from-wednesdays-fed-meeting-and-news-conference}}

Copied to clipboard.

\includegraphics{https://static01.nyt.com/images/2020/07/29/world/29virus-briefing-vid-cover/merlin_168830211_32d0e4c5-cdab-4c3d-ab66-f861783004ee-videoSixteenByNine3000.jpg}

The chair of the Federal Reserve, Jerome H. Powell, reiterated at a news
conference on Wednesday that the health of the economy will continue to
be tied to the whether the coronavirus can be kept in check. He spoke
after
\href{https://www.nytimes.com/2020/07/29/business/economy/federal-reserve-meeting-interest-rates.html}{the
central bank pledged to keep supporting the United States economy} as
the pandemic continues to depress economic growth and sideline millions
of workers.

\begin{itemize}
\item
  ``We're not even thinking about thinking about thinking about raising
  rates.'' The Federal Reserve left interest rates near zero on
  Wednesday, and Mr. Powell made clear officials plan to keep rates at
  rock-bottom for the foreseeable future.
\item
  Mr. Powell said it was important that the Fed keep its programs in
  place ``until we're very confident that the turmoil from the pandemic
  and the economic fallout are behind us.''

  The Fed said on Wednesday that it would also extend its programs meant
  to keep dollar funding readily available to foreign central banks
  through March 2021, and on Tuesday, officials announced that they
  would extend the Fed's emergency lending programs through the end of
  2020.
\item
  ``The pace of recovery looks like it has slowed'' since the virus
  began to spike again in June.

  ``Recent labor market indicators point to a slowing in job growth,
  particularly among smaller businesses,'' Mr. Powell said, and consumer
  surveys ``look like they may be softening again now.''
\item
  ``There's probably going to be a long tail where a large number of
  people are struggling to get back to work.'' Mr. Powell predicted a
  long slog ahead for workers in certain sectors of the economy,
  particularly restaurants, bars and hotels.

  ``There won't be enough jobs for them,'' he said. ``There will be a
  need both for more support from us and for more fiscal policy.''
\item
  ``It's kept people in their homes, it's kept businesses in business.''
  Mr. Powell said the support Congress had provided so far had been
  critical to helping workers and businesses, though he did not weigh in
  on whether lawmakers should extend the enhanced employment benefits.
\end{itemize}

--- \href{https://www.nytimes.com/by/jeanna-smialek}{Jeanna Smialek}

\hypertarget{democrats-accused-big-tech-of-stifling-competition-republicans-accused-big-tech-of-stifling-speech}{%
\subsection{\texorpdfstring{\protect\hyperlink{democrats-accused-big-tech-of-stifling-competition-republicans-accused-big-tech-of-stifling-speech}{Democrats
accused Big Tech of stifling competition. Republicans accused Big Tech
of stifling
speech.}}{Democrats accused Big Tech of stifling competition. Republicans accused Big Tech of stifling speech.}}\label{democrats-accused-big-tech-of-stifling-competition-republicans-accused-big-tech-of-stifling-speech}}

Copied to clipboard.

\includegraphics{https://static01.nyt.com/images/2020/07/29/business/29markets-brf-tech/merlin_175078407_2e6503e3-2eb5-417b-aac0-5cb9177a507f-articleLarge.jpg?quality=75\&auto=webp\&disable=upscale}

\href{https://www.nytimes.com/2020/07/27/business/jeff-bezos-amazon-congress.html}{Jeff
Bezos} of \textbf{Amazon},
\href{https://www.nytimes.com/2014/06/15/technology/tim-cook-making-apple-his-own.html}{Tim
Cook} of \textbf{Apple},
\href{https://www.nytimes.com/2018/03/21/technology/mark-zuckerberg-q-and-a.html}{Mark
Zuckerberg} of \textbf{Facebook} and
\href{https://www.nytimes.com/2019/12/03/technology/google-alphabet-ceo-larry-page-sundar-pichai.html}{Sundar
Pichai} of \textbf{Google}
\href{https://www.nytimes.com/live/2020/07/29/technology/tech-ceos-hearing-testimony}{testified
before Congress} on Wednesday, making their case about why their
companies actually are not that powerful.

Democratic lawmakers criticized the tech companies for buying start-ups
to stifle them and for unfairly using their data hoards to clone and
kill off competitors, while Republicans questioned whether the platforms
had muzzled conservative viewpoints and were unpatriotic.

``As gatekeepers to the digital economy, these platforms enjoy the power
to pick winners and losers, shake down small businesses and enrich
themselves while choking off competitors,'' said Representative David
Cicilline, Democrat of Rhode Island and chairman of the House Judiciary
Committee's antitrust subcommittee. ``Our founders would not bow before
a king. Nor should we bow before the emperors of the online economy.''

In response, Mr. Pichai, Mr. Zuckerberg, Mr. Cook and Mr. Bezos, who
testified via videoconference because of the coronavirus pandemic, were
forced to show humility. They presented themselves as participants in
enormously competitive and fast-changing digital marketplaces, and they
evaded questions about the decisions that turned their companies into
giants.

Follow our full coverage of the hearings:

\hypertarget{live-updates-big-tech-hearing}{%
\section{\texorpdfstring{\href{https://www.nytimes.com/live/2020/07/29/technology/tech-ceos-hearing-testimony}{Live
Updates: Big Tech
Hearing}}{Live Updates: Big Tech Hearing}}\label{live-updates-big-tech-hearing}}

\href{https://www.nytimes.com/live/2020/07/29/technology/tech-ceos-hearing-testimony\#tim-cook-pressed-on-apples-app-store}{86h
ago}

\href{https://www.nytimes.com/live/2020/07/29/technology/tech-ceos-hearing-testimony\#tim-cook-pressed-on-apples-app-store}{Tim
Cook pressed on Apple's App Store.}

\href{https://www.nytimes.com/live/2020/07/29/technology/tech-ceos-hearing-testimony\#tech-executives-looked-like-they-work-in-well-tech-offices}{86h
ago}

\href{https://www.nytimes.com/live/2020/07/29/technology/tech-ceos-hearing-testimony\#tech-executives-looked-like-they-work-in-well-tech-offices}{Tech
executives looked like they work in, well, tech offices.}

\href{https://www.nytimes.com/live/2020/07/29/technology/tech-ceos-hearing-testimony\#heres-which-tech-ceo-was-asked-the-most-questions-by-lawmakers}{86h
ago}

\href{https://www.nytimes.com/live/2020/07/29/technology/tech-ceos-hearing-testimony\#heres-which-tech-ceo-was-asked-the-most-questions-by-lawmakers}{Here's
which tech C.E.O. was asked the most questions by lawmakers.}

\href{https://www.nytimes.com/live/2020/07/29/technology/tech-ceos-hearing-testimony}{See
more updates}

\hypertarget{advertisement}{%
\subsubsection{Advertisement}\label{advertisement}}

\protect\hyperlink{after-dfp-ad-mid1}{Continue reading the main story}

\hypertarget{regal-cinemas-rejects-idea-of-shortening-exclusive-theatrical-runs}{%
\subsection{\texorpdfstring{\protect\hyperlink{regal-cinemas-rejects-idea-of-shortening-exclusive-theatrical-runs}{Regal
Cinemas rejects idea of shortening exclusive theatrical
runs.}}{Regal Cinemas rejects idea of shortening exclusive theatrical runs.}}\label{regal-cinemas-rejects-idea-of-shortening-exclusive-theatrical-runs}}

Copied to clipboard.

\includegraphics{https://static01.nyt.com/images/2020/07/29/business/29markets-brf-regal/merlin_172351464_38342ddd-232c-4193-877a-ffd8a48bcddc-articleLarge.jpg?quality=75\&auto=webp\&disable=upscale}

\textbf{Regal Cinemas}, the second-largest multiplex chain in the United
States, on Wednesday said it did not plan to follow suit of its rival
\textbf{AMC Entertainment} in allowing \textbf{Universal Pictures}
movies to arrive in homes
\href{https://www.nytimes.com/2020/07/28/business/media/universal-amc-movies-at-home.html}{just
17 days} after rolling out in theaters. The norm in the industry has
long been to restrict distribution through streaming for 90 days after a
movie's release.

``We do not see any business sense in this model,'' Mooky Greidinger,
the chief executive of Cineworld, the British owner of Regal, said in a
statement. ``This is a wrong move at the wrong time.''

``We are not changing our policy with regard to showing only movies that
respect the theatrical window,'' he said.

Mr. Greidinger said that Cineworld did not know the full details of
AMC's agreement with Universal and that, per standard practice, it would
analyze it more thoroughly.

``The first big movie from Universal is not releasing for six months, so
there is no pressure here,'' he said. (It was unclear what film he was
referring to; nothing on Universal's calendar quite matches, with the
closest being the animated ``The Croods 2'' in late December.)

Studios --- and \textbf{Netflix} --- have long wanted to shorten the
exclusive window given to theaters, but the big chains have recoiled.
The pandemic, however, shifted the playing field. With most theaters in
the United States closed since March, Universal has made new films
available online for premium rental, to apparent success.

AMC, which has been staring down bankruptcy, finally broke ranks with
other cinema chains and agreed to a shorter exclusivity window in return
for big concession from Universal: For the first time, the studio agreed
to share a portion of all premium on-demand revenue with AMC, reducing
the risk of early home availability.

AMC operates about 8,000 screens in the United States. Regal has 7,128.
\textbf{Cinemark}, the third-largest chain, has still not commented on
the deal, which was announced on Tuesday. The National Association of
Theater Owners, a trade group, said it would not comment on ``individual
member company decisions.''

--- \href{https://www.nytimes.com/by/brooks-barnes}{Brooks Barnes}

\hypertarget{us-employment-picked-up-last-week-new-data-shows}{%
\subsection{\texorpdfstring{\protect\hyperlink{us-employment-picked-up-last-week-new-data-shows}{U.S.
employment picked up last week, new data
shows.}}{U.S. employment picked up last week, new data shows.}}\label{us-employment-picked-up-last-week-new-data-shows}}

Copied to clipboard.

U.S. employment data has been relentlessly grim in recent weeks, showing
\href{https://www.nytimes.com/2020/07/23/business/economy/unemployment-economy-coronavirus.html}{rising
layoffs} and falling job postings. But Wednesday brought a glimmer of
optimism: Data from the Census Bureau showed an increase in jobs last
week, after four straight weekly declines.

The data, from the bureau's experimental
\href{https://www.census.gov/programs-surveys/household-pulse-survey/data.html}{Household
Pulse Survey}, showed that the number of employed people rose by about
1.9 million last week, partly reversing a
\href{https://www.nytimes.com/live/2020/07/22/business/stock-market-today-coronavirus/us-employment-has-declined-sharply-a-new-report-shows}{decline
of more than four million} the week before.

Overall employment is still down by nearly five million since mid-June.
Just over 52 percent of American adults were employed last week,
according the survey, down from 54 percent last month.

The Census Bureau started the survey as an effort to track the
pandemic's economic impact in a more timely manner than is possible
using standard monthly or quarterly indicators. The survey has drawn
more attention since correctly signaling the big increase in employment
in the
\href{https://www.nytimes.com/2020/07/02/business/economy/jobs-unemployment-coronavirus.html}{jobs
report} for June.

Still, economists caution against reading too much into week-to-week
changes in such a new data source. It is possible, for example, that
last week's report overstated the extent of job losses, and that the
latest report is best interpreted as a return to the previous trend of
gradual declines.

--- \href{https://www.nytimes.com/by/ben-casselman}{Ben Casselman}

\hypertarget{kodak-stock-soars-after-765-million-loan-to-make-drug-ingredients}{%
\subsection{\texorpdfstring{\protect\hyperlink{kodak-stock-soars-after-765-million-loan-to-make-drug-ingredients}{Kodak
stock soars after \$765 million loan to make drug
ingredients.}}{Kodak stock soars after \$765 million loan to make drug ingredients.}}\label{kodak-stock-soars-after-765-million-loan-to-make-drug-ingredients}}

Copied to clipboard.

\includegraphics{https://static01.nyt.com/images/2020/07/29/business/29-markets-brf-kodak/merlin_92635288_d49c92ca-527a-4fb9-88c7-f25af0c16cd3-articleLarge.jpg?quality=75\&auto=webp\&disable=upscale}

The stock of \textbf{Eastman Kodak Company} is nearly fifteen times
higher on Wednesday than it was at the start of the week --- and the
surge has nothing to do with photography.

The Trump administration said on Tuesday that it would
\href{https://www.nytimes.com/live/2020/07/28/business/stock-market-today-coronavirus/the-united-states-will-lend-kodak-765-million-to-make-drug-components}{lend
\$765 million} to the company to produce critical pharmaceutical
components to decrease America's reliance on foreign nations for
medicines.

``If we have learned anything from the global pandemic, it is that
Americans are dangerously dependent on foreign supply chains for their
essential medicines,'' said Peter Navarro, a White House adviser on
trade and the economy, said
\href{https://www.dfc.gov/media/press-releases/dfc-sign-letter-interest-investment-kodaks-expansion-pharmaceuticals}{in
a statement}.

Kodak's stock price tripled on Tuesday after the announcement and then
climbed more than six times higher at one point on Wednesday. The
company said the project, to be established in the Eastman Business Park
in Rochester, N.Y., will support 360 direct jobs.

The firm, founded in 1888, employed more than 60,000 people in Rochester
at its peak in the early 1980s. It struggled to keep up with the rise of
digital photography, and filed for bankruptcy in 2012. In an attempt to
remain relevant as its fortunes waned, in 2018 Kodak launched a
``\href{https://www.nytimes.com/2018/01/30/technology/kodak-blockchain-bitcoin.html}{photo-centric
cryptocurrency}.'' Its beaten-down stock got a short-lived bump from
that project, but nothing like what it's seen over the past 24 hours.

--- Gillian Friedman

\hypertarget{advertisement-1}{%
\subsubsection{Advertisement}\label{advertisement-1}}

\protect\hyperlink{after-dfp-ad-mid2}{Continue reading the main story}

\hypertarget{us-stocks-climb-as-fed-pledges-continued-support-for-the-economy}{%
\subsection{\texorpdfstring{\protect\hyperlink{us-stocks-climb-as-fed-pledges-continued-support-for-the-economy}{U.S.
stocks climb as Fed pledges continued support for the
economy.}}{U.S. stocks climb as Fed pledges continued support for the economy.}}\label{us-stocks-climb-as-fed-pledges-continued-support-for-the-economy}}

Copied to clipboard.

\includegraphics{https://static01.nyt.com/images/2020/07/29/world/29markets-brf-markets/merlin_175050378_bcd066d9-3a91-407a-9bca-aee22dc35fad-articleLarge.jpg?quality=75\&auto=webp\&disable=upscale}

U.S. stocks rose on Wednesday as the U.S. Federal Reserve announced it
would
\href{https://www.nytimes.com/live/2020/07/29/business/stock-market-today-coronavirus/federal-reserve-leaves-rates-near-zero-and-pledges-ongoing-economic-support}{leave
interest rates near-zero} and continue to support the U.S. economy amid
the continuing coronavirus pandemic.

The S\&P 500 rose more than 1 percent. European markets were mixed,
while Asian markets had a mostly positive day.

Investors had expected the Fed to leave rates unchanged and were looking
for further indication that policymakers would continue to use the
central bank's power to cushion the economy as millions of Americans
remain unemployed and economic growth is depressed.

On Wednesday, policymakers said they would extend programs meant to keep
dollar funding available to foreign central banks. That announcement
comes after
\href{https://www.nytimes.com/2020/07/28/business/economy/coronavirus-federal-reserve-policy.html}{officials
announced} on Tuesday that they would extend their nine emergency
lending programs through the end of 2020.

Shares of the ``Big Four'' technology companies --- \textbf{Amazon},
\textbf{Apple}, \textbf{Facebook} and \textbf{Alphabet}, the parent
company of \textbf{Google} --- all rose at least 1 percent as the chiefs
of the companies
\href{https://www.nytimes.com/live/2020/07/29/technology/tech-ceos-hearing-testimony}{testified
before Congress}.

Among the European companies reporting quarterly earnings on Wednesday
were \textbf{Deutsche Bank} (down 1.9 percent despite a surge in trading
revenue), \textbf{Barclays} (down 4.5 percent after a net loss that it
attributed to the pandemic) and the luxury group \textbf{Kering}, the
parent company of \textbf{Gucci} (which gained 4 percent after a
smaller-than-expected drop in sales).

In Hong Kong, after the markets closed, the government reported that the
territory's economy
\href{https://www.nytimes.com/live/2020/07/29/business/stock-market-today-coronavirus/hong-kongs-economy-shrank-by-9-percent-in-the-second-quarter}{contracted
9 percent} in the second quarter from a year ago, the fourth consecutive
quarter of declines.

--- \href{https://www.nytimes.com/by/kevin-granville}{Kevin Granville}

\hypertarget{as-the-600-weekly-booster-to-jobless-pay-ends-the-bills-dont-stop}{%
\subsection{\texorpdfstring{\protect\hyperlink{as-the-600-weekly-booster-to-jobless-pay-ends-the-bills-dont-stop}{As
the \$600 weekly booster to jobless pay ends, the bills don't
stop.}}{As the \$600 weekly booster to jobless pay ends, the bills don't stop.}}\label{as-the-600-weekly-booster-to-jobless-pay-ends-the-bills-dont-stop}}

Copied to clipboard.

\includegraphics{https://static01.nyt.com/images/2020/07/29/business/29virus-cliff1/29virus-cliff1-articleLarge.jpg?quality=75\&auto=webp\&disable=upscale}

Sara Gard is one of roughly 30 million Americans who are getting
unemployment payments --- a staggering figure that reflects one of the
country's most calamitous economic events.

Jobless benefit payments began arriving a few days after she was
furloughed in April from an entertainment company in Atlanta. They
included a \$600 weekly supplement, part of an emergency federal aid
program.

She is still without a job, but the
\href{https://www.nytimes.com/2020/07/21/business/economy/coronavirus-unemployment-benefits.html}{benefit
booster has run out}, leaving her with \$300 in weekly payments from the
state. ``We're going to totally have to rethink our lives,'' said Ms.
Gard, the primary breadwinner for a family of four.

Ms. Gard and her family are among those left in the lurch by the
congressional impasse over restoring or replacing the \$600 supplement.
Democrats favor extending it in full. Republicans would substitute a
\$200 payment, saying the larger sum discourages looking for work.

The Gards recognize that they are luckier than many. Ms. Gard's husband,
Matt, has kept his hospital maintenance job, and her employer continues
to pay its portion of the cost of her medical insurance. But she has to
come up with her part --- \$350 a month --- while dealing with other
bills.

``I have the month of August to figure out where September's mortgage
payment and everything else will come from,'' she said.

--- \href{https://www.nytimes.com/by/patricia-cohen}{Patricia Cohen}

\hypertarget{gap-is-suing-its-landlords}{%
\subsection{\texorpdfstring{\protect\hyperlink{gap-is-suing-its-landlords}{Gap
is suing its
landlords.}}{Gap is suing its landlords.}}\label{gap-is-suing-its-landlords}}

Copied to clipboard.

\includegraphics{https://static01.nyt.com/images/2020/07/29/business/29-markets-brf-gap1/merlin_172620972_326412c6-05fa-4005-81b9-e927cb1c87a0-articleLarge.jpg?quality=75\&auto=webp\&disable=upscale}

\textbf{Gap Inc.}, the retailer that oversees its namesake chain, Old
Navy and Banana Republic, has been facing a wave
of\href{https://www.nytimes.com/2020/06/05/business/economy/coronavirus-commercial-real-estate.html}{lawsuits
from major mall owners and smaller landlords} after it stopped paying
tens of millions of dollars in rent for its roughly 2,800 stores in
North America. Now, the company is filing its own lawsuits and
counterclaims, saying that it is actually owed money for rent it paid in
March and that leases must be renegotiated or terminated based on
unexpected closures and major shifts in the shopping landscape.

The disputes highlight the growing tension between big retail chains and
landlords because of the pandemic. Gap is a huge tenant, saying in a
\href{http://d18rn0p25nwr6d.cloudfront.net/CIK-0000039911/e24ba13d-fd33-4f86-9dd5-c88ec4ded708.pdf}{filing}
last month that it suspended rent payments in April for stores that had
temporarily closed, a roughly \$115 million cost per month in North
America. The company said it resumed rent payments when the stores
opened again.

\textbf{Simon Property Group}, the biggest mall operator in the U.S.,
\href{https://www.nytimes.com/2020/06/04/business/unemployment-jobless-claims-coronavirus.html}{sued
Gap} last month, saying the company owed it about \$66 million in unpaid
rent for April, May and June, while the upscale \textbf{Brookfield
Properties} sued Gap in Texas, saying that the company owed more than
\$2 million in unpaid rent in that state. Numerous other smaller
properties have also filed lawsuits against Gap and its chains for
unpaid rent.

Gap said in a complaint to Simon Property last week that ``shopping for
apparel in physical stores will look nothing like what was contemplated
by the leases when they were executed.'' The company added that if the
parties knew when entering the leases that Gap ``would not be permitted
to operate a retail store for the entire duration of the leases, or to
do so only with restricted limits on the occupancy of the premises, the
parties would not have agreed on the same amount of rent and other
terms.''

Gap, based in San Francisco, declined to comment but confirmed it has
filed multiple counterclaims and five of its own lawsuits against
landlords, including Brookfield. Simon Property declined to comment.

Lindsay Kahn, a spokeswoman for Brookfield, said in an emailed statement
that while the company could not speak to the specifics of the ongoing
litigation, ``it should be clear from the filings that The Gap Inc. has
taken inappropriate positions at a time when we should be working
together.''

``Brookfield has worked hard to reopen its shopping centers safely and
consistent with guidance from local authorities, which is important for
the many businesses and jobs that depend on commercial activity at our
properties,'' she wrote. ``The filing is a matter of contract law and
Brookfield intends to hold Gap accountable.''

--- \href{https://www.nytimes.com/by/sapna-maheshwari}{Sapna Maheshwari}

\hypertarget{advertisement-2}{%
\subsubsection{Advertisement}\label{advertisement-2}}

\protect\hyperlink{after-dfp-ad-mid3}{Continue reading the main story}

\hypertarget{the-us-postal-service-and-treasury-agree-to-undisclosed-terms-for-a-10-billion-loan}{%
\subsection{\texorpdfstring{\protect\hyperlink{the-us-postal-service-and-treasury-agree-to-undisclosed-terms-for-a-10-billion-loan}{The
U.S. Postal Service and Treasury agree to (undisclosed) terms for a \$10
billion
loan.}}{The U.S. Postal Service and Treasury agree to (undisclosed) terms for a \$10 billion loan.}}\label{the-us-postal-service-and-treasury-agree-to-undisclosed-terms-for-a-10-billion-loan}}

Copied to clipboard.

\includegraphics{https://static01.nyt.com/images/2020/07/29/business/29-markets-brf-USPS/merlin_172297068_5836d54b-7fdb-46b9-9e87-f43101d84e3f-articleLarge.jpg?quality=75\&auto=webp\&disable=upscale}

The \textbf{Treasury Department} and the \textbf{United States Postal
Service} said on Wednesday that they had reached an agreement on terms
that would allow the Postal Service to access \$10 billion in loan money
that was approved by Congress in the March economic relief package.

The \href{https://home.treasury.gov/news/press-releases/sm1071}{terms of
the agreement were not disclosed}. Treasury said in a statement that the
Postal Service was authorized to borrow the money if it determined that
it could no longer fund its operating expenses.

The Postal Service had warned earlier this year that it could run out of
money by September without financial assistance. Democrats wanted to
provide it with a cash infusion, but Treasury Secretary Steven Mnuchin
insisted that the money be allocated as a loan that would only be
distributed if the Postal Service implemented reforms such as price
increases for shipping.

``While the USPS is able to fund its operating expenses without
additional borrowing at this time, we are pleased to have reached an
agreement on the material terms and conditions of a loan, should the
need arise,'' Mr. Mnuchin said in a statement.

The new Postmaster General, Louis DeJoy, started imposing cost cutting
measures earlier this month that critics warned would slow mail
delivery. The fate of the Postal Service is an especially fraught issue
this year, as millions of American could be relying on its services to
vote by mail in the November presidential election.

Mr. DeJoy, a long
time\href{https://www.nytimes.com/2020/05/07/us/politics/postmaster-general-louis-dejoy.html}{Republican
donor and financial backer} of President Trump who was given the job in
May, said in a statement that access to the loan would delay ``an
approaching liquidity crisis'' but that more cost cutting was in store.

``The Postal Service, however, remains on an unsustainable path and we
will continue to focus on improving operational efficiency and pursuing
other reforms in order to put the Postal Service on a trajectory for
long-term financial stability,'' Mr. DeJoy said.

--- \href{https://www.nytimes.com/by/alan-rappeport}{Alan Rappeport}

\hypertarget{deutsche-bank-ekes-out-profit-as-trading-fees-surge}{%
\subsection{\texorpdfstring{\protect\hyperlink{deutsche-bank-ekes-out-profit-as-trading-fees-surge}{Deutsche
Bank ekes out profit as trading fees
surge.}}{Deutsche Bank ekes out profit as trading fees surge.}}\label{deutsche-bank-ekes-out-profit-as-trading-fees-surge}}

Copied to clipboard.

\includegraphics{https://static01.nyt.com/images/2020/07/29/business/29markets-brf-deutschebank/29markets-brf-deutschebank-articleLarge.jpg?quality=75\&auto=webp\&disable=upscale}

\textbf{Deutsche Bank}, Germany's largest lender, reported a small but
better-than-expected
\href{https://www.db.com/newsroom_news/2020/deutsche-bank-reports-pre-tax-profit-of-158-million-euros-in-second-quarter-of-2020-with-transformation-fully-on-en-11651.htm}{profit}
Wednesday as it cashed in on market turbulence caused by the pandemic.

The bank said its net profit during the second quarter of the year was
61 million euros, or \$72 million, compared to a loss of more than 3
billion euros a year earlier. Like other
\href{https://www.nytimes.com/2020/07/14/business/big-banks-quarterly-results.html?searchResultPosition=4}{big
banks}, Deutsche Bank recorded a surge in trading fees as clients
frantically adjusted their portfolios in response to the pandemic.

Deutsche Bank, based in Frankfurt, has been paring back its investment
banking but remains a force in currency trading and bond markets. Sales
in the unit that handles those activities surged 40 percent to more than
2 billion euros, helping to offset a nearly fivefold increase in the
amount that Deutsche Bank had to set aside to cover problem loans.

Among other banks issuing earnings reports, \textbf{Barclays} reported a
smaller-than-expected profit and said it would increase its reserve for
bad loans by 1.6 billion pounds (\$2.1 billion). And \textbf{Santander}
fell to the first loss in its 163-year history after a big write-down on
its assets.

--- \href{https://www.nytimes.com/by/jack-ewing}{Jack Ewing}

\hypertarget{boeing-suffers-24-billion-quarterly-loss}{%
\subsection{\texorpdfstring{\protect\hyperlink{boeing-suffers-2-4-billion-quarterly-loss}{Boeing
suffers \$2.4 billion quarterly
loss.}}{Boeing suffers \$2.4 billion quarterly loss.}}\label{boeing-suffers-24-billion-quarterly-loss}}

Copied to clipboard.

\includegraphics{https://static01.nyt.com/images/2020/07/29/business/29markets-brf-boeing/29markets-brf-boeing-articleLarge.jpg?quality=75\&auto=webp\&disable=upscale}

\textbf{Boeing} said on Wednesday that it was slowing plane production
and might cut jobs as it reels from the grounding of the 737 Max and a
devastating aviation slowdown brought on by the coronavirus pandemic.
The company also reported a \$2.4 billion loss in the second quarter.

``These past few months have been unlike anything we've seen,'' Boeing's
president and chief executive, Dave Calhoun, said in a message to staff.
``The pandemic's effect on our communities and industry is ongoing. And
the challenges we face as a company are still unfolding.''

Boeing's revenues plunged 25 percent, to \$11.8 billion, in the quarter
compared with last year, a loss driven by its struggling commercial
business and partially offset by its government, defense and space
programs. The company has previously announced plans to
\href{https://www.nytimes.com/2020/04/29/business/boeing-layoffs-coronavirus.html}{slash
about 16,000 jobs worldwide}, or about a tenth of its work force, and
warned on Wednesday that more cuts could come. Boeing's share price was
down nearly 4 percent around noon.

The quarterly update came a day after a global airline industry group,
the \textbf{International Air Transport Association}, downgraded its
forecast for when air travel will return to normal, blaming poor virus
containment in the United States and other developing economies, a slow
rebound for business travel, and low consumer confidence. Now, passenger
traffic is not expected to return to last year's levels until 2024, the
group said. Boeing said it believes the industry could reach that
milestone somewhat sooner.

The industry's middling recovery is being led by short, domestic trips
typically on single-aisle, planes like the 737 Max. Longer international
flights aboard wide-body planes, like Boeing's 787 or 777 lines, are
expected to lag as much as a year behind.

The plane maker said it would make fewer of the 777 and 787 planes for
now and pushed back production plans for the 737 family of planes,
including the troubled 737 Max jet, which has been grounded since March
2019 after two fatal crashes. Boeing now expects to reach a production
rate of 31 737s per month by the beginning of 2022. That is about half
the rate Boeing had targeted before the Max was grounded and represents
a delay from the company's previous plan to reach that rate sometime
next year.

The Max moved closer to flying passengers again this month, after the
Federal Aviation Administration concluded a series of test flights and
said
\href{https://www.nytimes.com/live/2020/07/21/business/stock-market-today-coronavirus\#the-faa-moves-the-boeing-737-max-one-step-closer-to-flying-again}{it
was close to formally proposing changes} that would address its safety
concerns. In a call with analysts, Boeing's chief financial officer,
Greg Smith, said that the company expects to start delivering the jet to
customers by the end of the year and expects to distribute most of the
approximately 450 Max jets in storage by 2022.

In the note to staff, Mr. Calhoun also confirmed that Boeing would end
production of the famed 747 in 2022. Once nicknamed ``Queen of the
Skies,'' the plane was the world's first jumbo jet and celebrated the
50th anniversary of its first flight last year.

--- \href{https://www.nytimes.com/by/niraj-chokshi}{Niraj Chokshi}

\hypertarget{advertisement-3}{%
\subsubsection{Advertisement}\label{advertisement-3}}

\protect\hyperlink{after-dfp-ad-mid4}{Continue reading the main story}

\hypertarget{gm-lost-758-million-in-the-second-quarter}{%
\subsection{\texorpdfstring{\protect\hyperlink{gm-lost-758-million-in-the-second-quarter}{G.M.
lost \$758 million in the second
quarter.}}{G.M. lost \$758 million in the second quarter.}}\label{gm-lost-758-million-in-the-second-quarter}}

Copied to clipboard.

\includegraphics{https://static01.nyt.com/images/2020/07/29/business/29-markets-brf-GM/29-markets-brf-GM-articleLarge.jpg?quality=75\&auto=webp\&disable=upscale}

\textbf{General Motors} suffered a loss in the second quarter as the
coronavirus pandemic took a heavy toll on its operations in most regions
of the world.

The automaker lost \$758 million as its second-quarter revenue was more
than halved, to \$16.8 billion compared with \$36.1 billion in the same
period a year ago. The company used \$9 billion in cash during the
quarter but still has \$28.3 billon on hand.

``Covid-19 has affected every aspect of our business,'' the company's
chief executive, Mary T. Barra, said in a conference call.

Ms. Barra added that G.M. had reduced costs and had ample cash to spur
sales as the economy improves. ``We have put the company in position for
continued recovery in the third and fourth quarters and beyond,'' she
said.

In North America, which generates the bulk of G.M.'s profit, new vehicle
deliveries fell 36 percent to 565,000 cars and light trucks. The
pandemic forced the company to halt production in North America for two
months, and depressed purchases of new vehicles by both consumers and
fleet customers like rental-car companies.

Its smaller South American unit suffered an even harsher blow, as
deliveries fell 65 percent in the quarter, to 57,000 vehicles. In China,
where the virus outbreak has receded faster than in the United States,
deliveries fell 5 percent.

G.M. stock was down about 2 percent in early afternoon trading.

Ms. Barra also said companies like G.M. have an ``increasing
responsibility to take a stand against racial injustice.'' The automaker
recently formed an inclusion board, chaired by Ms. Barra, to increase
diversity in the company. ``General Motors has a strong track record of
diversity by many objective standards, but we must do more and we
will,'' she said.

--- \href{https://www.nytimes.com/by/neal-e-boudette}{Neal E. Boudette}

\hypertarget{thursdays-gdp-report-is-likely-to-show-a-big-decline-in-dueling-numbers}{%
\subsection{\texorpdfstring{\protect\hyperlink{thursdays-gdp-report-is-likely-to-show-a-big-decline-in-dueling-numbers}{Thursday's
G.D.P. report is likely to show a big decline, in dueling
numbers.}}{Thursday's G.D.P. report is likely to show a big decline, in dueling numbers.}}\label{thursdays-gdp-report-is-likely-to-show-a-big-decline-in-dueling-numbers}}

Copied to clipboard.

\includegraphics{https://static01.nyt.com/images/2020/07/28/business/28gdp-explain2/merlin_172459122_1f997b8d-abb8-4a5b-91f1-199d1eaad2d1-articleLarge.jpg?quality=75\&auto=webp\&disable=upscale}

On Thursday morning, when the Commerce Department announces the nation's
second-quarter economic output, the data is likely to reflect the
biggest decline in the 70-plus years that such statistics have been
compiled.
\href{https://www.nytimes.com/2020/07/29/business/economy/us-gdp-report.html}{But
you may see two widely different numbers}.

Forecasters expect the report to show that gross domestic product ---
the broadest measure of goods and services produced --- fell at an
annual rate of about 35 percent. That doesn't mean, however, that the
economy shrank by more than a third in a mere three months.

The United States has traditionally reported the figure as an annual
rate --- that is, how much the economy would grow or shrink if the
change were sustained for a full year. But when annual rates are applied
to short-term changes, the results can be misleading.

For that reason, The New York Times plans to emphasize the nonannualized
percentage change in its coverage. And on that basis, if the forecasters
are on target, the G.D.P. should be about 10 percent lower in the second
quarter than in the first.

--- \href{https://www.nytimes.com/by/ben-casselman}{Ben Casselman}

\hypertarget{site-index}{%
\subsection{Site Index}\label{site-index}}

\hypertarget{site-information-navigation}{%
\subsection{Site Information
Navigation}\label{site-information-navigation}}

\begin{itemize}
\tightlist
\item
  \href{https://help.nytimes.com/hc/en-us/articles/115014792127-Copyright-notice}{©~2020~The
  New York Times Company}
\end{itemize}

\begin{itemize}
\tightlist
\item
  \href{https://www.nytco.com/}{NYTCo}
\item
  \href{https://help.nytimes.com/hc/en-us/articles/115015385887-Contact-Us}{Contact
  Us}
\item
  \href{https://www.nytco.com/careers/}{Work with us}
\item
  \href{https://nytmediakit.com/}{Advertise}
\item
  \href{http://www.tbrandstudio.com/}{T Brand Studio}
\item
  \href{https://www.nytimes.com/privacy/cookie-policy\#how-do-i-manage-trackers}{Your
  Ad Choices}
\item
  \href{https://www.nytimes.com/privacy}{Privacy}
\item
  \href{https://help.nytimes.com/hc/en-us/articles/115014893428-Terms-of-service}{Terms
  of Service}
\item
  \href{https://help.nytimes.com/hc/en-us/articles/115014893968-Terms-of-sale}{Terms
  of Sale}
\item
  \href{https://spiderbites.nytimes.com}{Site Map}
\item
  \href{https://help.nytimes.com/hc/en-us}{Help}
\item
  \href{https://www.nytimes.com/subscription?campaignId=37WXW}{Subscriptions}
\end{itemize}
