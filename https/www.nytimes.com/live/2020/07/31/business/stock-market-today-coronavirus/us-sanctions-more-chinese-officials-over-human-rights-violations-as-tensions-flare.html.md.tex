Sections

SEARCH

\protect\hyperlink{site-content}{Skip to
content}\protect\hyperlink{site-index}{Skip to site index}

\href{https://myaccount.nytimes.com/auth/login?response_type=cookie\&client_id=vi}{}

\href{https://www.nytimes.com/section/todayspaper}{Today's Paper}

\href{https://www.nytimes.com/news-event/coronavirus}{The Coronavirus
Outbreak}

\begin{itemize}
\tightlist
\item
  live\href{https://www.nytimes.com/2020/08/04/world/coronavirus-covid-19.html}{Latest
  Updates}
\item
  \href{https://www.nytimes.com/interactive/2020/us/coronavirus-us-cases.html}{Maps
  and Cases}
\item
  \href{https://www.nytimes.com/interactive/2020/science/coronavirus-vaccine-tracker.html}{Vaccine
  Tracker}
\item
  \href{https://www.nytimes.com/2020/08/02/us/covid-college-reopening.html}{College
  Reopening}
\item
  \href{https://www.nytimes.com/live/2020/08/03/business/stock-market-today-coronavirus}{Economy}
\end{itemize}

Last Updated

Aug. 3, 2020, 7:12 a.m. ET

Aug. 3, 2020, 7:12 a.m. ET

\hypertarget{us-debt-outlook-is-downgraded}{%
\section{U.S. Debt Outlook is
Downgraded}\label{us-debt-outlook-is-downgraded}}

\emph{This briefing is no longer being updated. Follow the latest
developments}
\href{https://www.nytimes.com/live/2020/08/03/business/stock-market-today-coronavirus}{\emph{here.}}

\hypertarget{heres-what-you-need-to-know}{%
\subsubsection{Here's what you need to
know:}\label{heres-what-you-need-to-know}}

\begin{itemize}
\item
  \protect\hyperlink{fitch-ratings-downgrades-its-outlook-on-us-debt}{}

  Fitch Ratings downgrades its outlook on U.S. debt.
\item
  \protect\hyperlink{stocks-climb-as-big-tech-rallies-after-strong-earnings}{}

  Stocks climb as Big Tech rallies after strong earnings.
\item
  \protect\hyperlink{united-will-add-international-flights-despite-travel-restrictions-limiting-us-visitors}{}

  United will add international flights despite travel restrictions
  limiting U.S. visitors.
\item
  \protect\hyperlink{europes-contraction-is-its-worst-on-record}{}

  Europe's contraction is its worst on record.
\item
  \protect\hyperlink{economic-snapshots-france-germany-italy-and-spain}{}

  Economic snapshots: France, Germany, Italy and Spain.
\item
  \protect\hyperlink{exxon-reports-a-record-loss-and-chevron-writes-off-venezuela-investments}{}

  Exxon reports a record loss and Chevron writes off Venezuela
  investments.
\end{itemize}

\hypertarget{fitch-ratings-downgrades-its-outlook-on-us-debt}{%
\subsection{\texorpdfstring{\protect\hyperlink{fitch-ratings-downgrades-its-outlook-on-us-debt}{Fitch
Ratings downgrades its outlook on U.S.
debt.}}{Fitch Ratings downgrades its outlook on U.S. debt.}}\label{fitch-ratings-downgrades-its-outlook-on-us-debt}}

Copied to clipboard.

\includegraphics{https://static01.nyt.com/images/2020/07/31/business/31Markets-brief-fitch/merlin_150696402_fd549eef-c722-4ef7-b480-00c10f1b6c28-articleLarge.jpg?quality=75\&auto=webp\&disable=upscale}

The credit rating firm Fitch left the United States' AAA rating
untouched, but downgraded its outlook on what is effectively the
national credit score, suggesting the country's status as one of the
world's most trustworthy borrowers could be put at risk by the enormous
deficits the federal government is running to combat the fallout from
the pandemic.

``The outlook has been revised to negative to reflect the ongoing
deterioration in the U.S. public finances and the absence of a credible
fiscal consolidation plan,''
\href{https://www.fitchratings.com/research/sovereigns/fitch-revises-united-states-outlook-to-negative-affirms-at-aaa-31-07-2020}{Fitch
analysts wrote} on Friday in a report announcing the decision.

Cratering tax revenues and surging expenditures have driven record
levels of red ink for the federal government in recent months. The
\href{https://www.nytimes.com/live/2020/07/13/business/stock-market-today-coronavirus\#the-us-budget-deficit-hits-another-monthly-record}{United
States' budget deficit hit a record} \$864 billion in June as the
government continued pumping money into the economy to support workers
and businesses slammed by the pandemic. Some analysts expect monthly
deficits to soon top \$1 trillion.

Ballooning deficits have led to an explosion of new borrowing. Fitch
noted that the Treasury Department borrowed just under \$3 trillion
dollars from the end of February to the end of June.

Much of the supply of new government bonds was,
\href{https://www.nytimes.com/2020/04/15/business/coronavirus-stimulus-money.html}{essentially,
purchased by the Federal Reserve}, which has bought \$2.6 trillion in
financial assets since the middle of March, Fitch noted.

The presence of the Federal Reserve, which can essentially create
whatever money it wants and use it to buy assets, such as U.S.
government debt, has depressed yields on government bonds even as debts
and deficits rise sharply.

On Friday, the yield on the 10-year note fell to 0.53 percent, one of
\href{https://www.marketwatch.com/story/10-year-treasury-yield-plunged-to-its-lowest-in-234-years-says-deutsche-bank-11596214464\#:~:text=The\%2010\%2Dyear\%20Treasury\%20note,scurrying\%20into\%20safe\%20haven\%20assets.}{the
lowest levels in recorded history}, suggesting there is virtually no
concern among investors about the country's ability to service its
growing debts.

--- \href{https://www.nytimes.com/by/matt-phillips}{Matt Phillips}

\hypertarget{kodaks-chief-executive-was-given-stock-options-then-the-share-price-spiked-1000-percent}{%
\subsection{\texorpdfstring{\protect\hyperlink{kodaks-chief-executive-was-given-stock-options-then-the-share-price-spiked-1000-percent}{Kodak's
chief executive was given stock options. Then the share price spiked
1,000
percent.}}{Kodak's chief executive was given stock options. Then the share price spiked 1,000 percent.}}\label{kodaks-chief-executive-was-given-stock-options-then-the-share-price-spiked-1000-percent}}

Copied to clipboard.

\includegraphics{https://static01.nyt.com/images/2020/08/01/business/31virus-kodak-print/merlin_77472113_4c707a84-2d62-47ed-bea4-1ad9ba7aae9f-articleLarge.jpg?quality=75\&auto=webp\&disable=upscale}

Earlier this week, The Times reported on
\href{https://www.nytimes.com/2020/07/25/business/coronavirus-vaccine-profits-vaxart.html}{the
well-timed stock bets that have generated big profits} for senior
executives and board members at companies developing vaccines and
treatments. Jesse Drucker and Ellen Gabler have
\href{https://www.nytimes.com/2020/07/31/business/kodak-ceo-stock-options.html}{the
latest example}:

\begin{quote}
At the beginning of this week, the Eastman Kodak Company handed Jim
Continenza, its chief executive, 1.75 million stock options.

It was the type of compensation decision that generally wouldn't attract
much notice, except for one thing: The day after the stock options were
granted, the White House announced that the company would receive
\href{https://www.nytimes.com/live/2020/07/28/business/stock-market-today-coronavirus\#the-united-states-will-lend-kodak-765-million-to-make-drug-components}{a
\$765 million federal loan} to produce ingredients to make
pharmaceuticals in the United States.

The news of the deal caused Kodak's shares to soar more than 1,000
percent. Within 48 hours of the options grants, their value had
ballooned, at least on paper, to about \$50 million.

A Kodak spokeswoman declined to comment on the timing of the
stock-options grants and emphasized that the value of the options could
change before Mr. Continenza uses them to buy Kodak shares.

Starting in May, Kodak began talks with the Trump administration about
manufacturing the ingredients for pharmaceuticals, Mr. Continenza said
in a television interview this week.

The deal was announced on Tuesday. President Trump said the federal loan
from the U.S. International Development Finance Corporation would help
reduce the United States' reliance on other countries, in particular
China and India, for the vast majority of ingredients used to make
generic drugs. Mr. Trump called the Kodak deal ``a breakthrough in
bringing pharmaceutical manufacturing back to the United States.''
\end{quote}

\hypertarget{advertisement}{%
\subsubsection{Advertisement}\label{advertisement}}

\protect\hyperlink{after-dfp-ad-mid1}{Continue reading the main story}

\hypertarget{stocks-climb-as-big-tech-rallies-after-strong-earnings}{%
\subsection{\texorpdfstring{\protect\hyperlink{stocks-climb-as-big-tech-rallies-after-strong-earnings}{Stocks
climb as Big Tech rallies after strong
earnings.}}{Stocks climb as Big Tech rallies after strong earnings.}}\label{stocks-climb-as-big-tech-rallies-after-strong-earnings}}

Copied to clipboard.

Stocks rallied to end Friday as investors looked past gnawing concerns
about the economic toll of the pandemic and instead were cheered by a
surge in profits reported by America's largest tech companies.

The S\&P 500 rose more than three-quarters of a percent, and ended July
with a gain of more than 5 percent. The index has climbed for four
consecutive months ---~rising more than 26 percent since the end of
February.

A big factor behind that rally has been the success of big technology
companies, which were well positioned to benefit from a shift to remote
work and limits on public activity.

On Thursday, investors heard just how much they benefited.
\textbf{\href{https://www.nytimes.com/live/2020/07/30/business/stock-market-today-coronavirus/alphabets-revenue-drops-but-beats-wall-street-expectations}{Amazon}}\href{https://www.nytimes.com/live/2020/07/30/business/stock-market-today-coronavirus/alphabets-revenue-drops-but-beats-wall-street-expectations}{,}\textbf{\href{https://www.nytimes.com/live/2020/07/30/business/stock-market-today-coronavirus/alphabets-revenue-drops-but-beats-wall-street-expectations}{Apple}}\href{https://www.nytimes.com/live/2020/07/30/business/stock-market-today-coronavirus/alphabets-revenue-drops-but-beats-wall-street-expectations}{and}\textbf{\href{https://www.nytimes.com/live/2020/07/30/business/stock-market-today-coronavirus/alphabets-revenue-drops-but-beats-wall-street-expectations}{Facebook}}\href{https://www.nytimes.com/live/2020/07/30/business/stock-market-today-coronavirus/alphabets-revenue-drops-but-beats-wall-street-expectations}{}reported
surging profits. The blockbuster earnings seemed to briefly put aside
the uncertainty and pessimism surrounding the economic impact of the
pandemic, but also perhaps underscored the
\href{https://www.nytimes.com/2020/07/29/technology/big-tech-hearing-apple-amazon-facebook-google.html}{concerns
of lawmakers}, expressed on Wednesday, that American's tech giants have
gotten too big.

Apple gained nearly 10.5 percent on Friday, reaching a record, as the
company announced a four-for-one stock split, and shares of Amazon and
Facebook also rose. \textbf{Alphabet}, the parent company of
\textbf{Google,} which reported its first-ever decline in quarterly
revenue on Thursday, ended Friday down more than 3 percent.

\textbf{Microsoft} also climbed late in the day, erasing its earlier
losses after
\href{https://www.nytimes.com/2020/07/31/technology/tiktok-microsoft.html}{reports
that it is in talks to buy TikTok}, the popular video sharing app. The
gains helped lift the Nasdaq composite by about 1.5 percent.

But the virus continues spreading, and its damage is mounting. On
Thursday, the United States reported that its
\href{https://www.nytimes.com/live/2020/07/30/business/stock-market-today-coronavirus/the-us-economys-contraction-in-the-second-quarter-was-the-worst-on-record}{economy
fell 9.5 percent} in the second quarter, compared with the previous
quarter, the most on record. On Friday, the authorities reported that
the
\href{https://www.nytimes.com/live/2020/07/31/business/stock-market-today-coronavirus/europes-contraction-is-its-worst-on-record}{eurozone
contracted 12.1 percent} in the second quarter. Both the United States
and Europe are in deep recessions caused by shutdowns in economic
activity to curb the spread of the disease.

--- \href{https://www.nytimes.com/by/kevin-granville}{Kevin Granville}

\hypertarget{united-will-add-international-flights-despite-travel-restrictions-limiting-us-visitors}{%
\subsection{\texorpdfstring{\protect\hyperlink{united-will-add-international-flights-despite-travel-restrictions-limiting-us-visitors}{United
will add international flights despite travel restrictions limiting U.S.
visitors.}}{United will add international flights despite travel restrictions limiting U.S. visitors.}}\label{united-will-add-international-flights-despite-travel-restrictions-limiting-us-visitors}}

Copied to clipboard.

\includegraphics{https://static01.nyt.com/images/2020/07/31/business/31markets-brf-united1/merlin_174811215_f9df7454-d680-4a0c-a377-252d876e83a0-articleLarge.jpg?quality=75\&auto=webp\&disable=upscale}

\textbf{United Airlines} plans to add more than 25 international routes
to its September schedule, a sign of limited optimism in a battered
industry at a time when coronavirus cases continue to rise across the
country.

Many of the new routes include destinations in Europe and Asia, where
governments restrict or limit American visitors. United said it would
adjust its schedule as necessary to deal with travel and quarantine
restrictions.

``We continue to be realistic in our approach to building back our
international and domestic schedules by closely monitoring customer
demand and flying where people want to go,'' Patrick Quayle, United's
vice president of international network and alliances, said in a
statement.

Many people are still flying for essential business, to visit friends
and family or to return home. Some of the shorter international flights
United is adding will serve limited demand for leisure travel.

The airline said it would launch a new route connecting Chicago and Tel
Aviv if it could obtain government approval. The airline will also
resume service between some of its American hubs and Amsterdam,
Frankfurt, Munich, Sydney, Costa Rica, St. Thomas, Ecuador and several
destinations in Mexico. United also plans to continue to fly to New
Delhi and Mumbai and between Chicago and Hong Kong, pending government
approval.

Overall, the airline plans to operate about 37 percent of the flights it
flew last September, a relative increase from August. The Transportation
Security Administration has only screened about 26 percent as many
people at its checkpoint in recent days as it did on the same days a
year ago.

The news comes a day after United dealt what appeared to be a fatal blow
to ExpressJet, a regional carrier that operates under the United Express
brand. United has a 49.9 percent stake in ExpressJet. In a note to staff
on Thursday, ExpressJet's chief executive, Subodh Karnik, said that the
two airlines would work together to wind down ExpressJet's operations
after United decided to make another regional carrier, CommutAir, the
sole operator of United Express flights aboard the small Embraer ERJ145
jet.

--- \href{https://www.nytimes.com/by/niraj-chokshi}{Niraj Chokshi}

\hypertarget{europes-contraction-is-its-worst-on-record}{%
\subsection{\texorpdfstring{\protect\hyperlink{europes-contraction-is-its-worst-on-record}{Europe's
contraction is its worst on
record.}}{Europe's contraction is its worst on record.}}\label{europes-contraction-is-its-worst-on-record}}

Copied to clipboard.

Eurozone G.D.P.

+2\%

0

-2

-4

-6

-8

-10

--12.1\%

Percentage change from previous quarter

-12

2008

2010

2012

2014

2016

2018

2020

Eurozone G.D.P.

+2\%

0

-2

-4

-6

-8

-10

-12

--12.1\%

Percentage change from previous quarter

-14

2008

2010

2012

2014

2016

2018

2020

Note: Adjusted for inflation and seasonality.

Source: Eurostat

By The New York Times

The European economy tumbled into its worst recession on record, as
quarantines in countries across the continent brought business, trade
and consumer spending to a grinding halt in the second quarter.

From April to
June,\href{https://ec.europa.eu/eurostat/documents/2995521/11156775/2-31072020-BP-EN.pdf/cbe7522c-ebfa-ef08-be60-b1c9d1bd385b}{economic
activity fell 12.1} percent from the previous quarter among the
countries that use the euro currency. It was sharpest contraction since
1995, when the data was first collected, according to Eurostat, the
European Union's statistics agency.

Compared to the same period a year ago, the decline was sharper:
Economic activity shrank 15 percent from the 2nd quarter of 2019.

The collapse marks the severe economic disruption caused by the
pandemic. Governments ordered lockdowns that silenced many cities, and
residents were told to stay home to prevent the virus's spread. On
Thursday, the
\href{https://www.nytimes.com/2020/07/30/business/economy/q2-gdp-coronavirus-economy.html}{United
States announced} its economy contracted 9.5 percent in the 2nd quarter
compared to the previous three-month period.

But there are signs the worst may have passed since then, and that a
tentative recovery is gaining some traction as European governments
unleashed enormous stimulus spending. The lengthy lockdowns have helped
curb a widespread resurgence of the pandemic in most countries.

The data was especially grim for nations on Europe's southern rim, which
were among the worst affected by the virus and which faced longer
quarantine periods than northern European countries.

In Spain, which has had one of Europe's highest death tolls, the economy
shrank by a staggering 18.5 percent from the previous quarter. France,
the eurozone's second-largest economy, shrank by 13.8 percent; and
Italy, the third-largest economy in the zone, contracted by 12.4
percent. France is officially in recession, with three straight quarters
of contraction.

On Thursday, the authorities reported that the German economy, Europe's
largest, shrank by 10.1 percent from the previous quarter.

European Union leaders last week agreed to
\href{https://www.nytimes.com/2020/07/20/world/europe/eu-stimulus-coronavirus.html}{a
landmark stimulus of 750 billion euros}, or about \$884 billion, to
rescue their economies and to anchor a mild turnaround that had started
to take hold after lockdowns began to be lifted.

But risks abound as surges in new cases are reported, increasing the
possibility of more quarantines.

``The hard part of this recovery is set to start about now,'' Bert
Colijn, senior economist for the eurozone at ING Bank, said in a note to
clients.

--- \href{https://www.nytimes.com/by/liz-alderman}{Liz Alderman}

\hypertarget{advertisement-1}{%
\subsubsection{Advertisement}\label{advertisement-1}}

\protect\hyperlink{after-dfp-ad-mid2}{Continue reading the main story}

\hypertarget{economic-snapshots-france-germany-italy-and-spain}{%
\subsection{\texorpdfstring{\protect\hyperlink{economic-snapshots-france-germany-italy-and-spain}{Economic
snapshots: France, Germany, Italy and
Spain.}}{Economic snapshots: France, Germany, Italy and Spain.}}\label{economic-snapshots-france-germany-italy-and-spain}}

Copied to clipboard.

\includegraphics{https://static01.nyt.com/images/2020/07/31/business/31briefing-markets-europe/merlin_175097730_1d272a52-4596-40da-b084-434675f2e035-articleLarge.jpg?quality=75\&auto=webp\&disable=upscale}

European countries have, for the most part, contained the spread of
coronavirus. But the outbreak, which was early and widespread, has left
a
\href{https://www.nytimes.com/live/2020/07/31/business/stock-market-today-coronavirus\#europes-economic-contraction-is-its-worst-on-record}{deep
scar} on the region's
economy:\href{https://ec.europa.eu/eurostat/documents/2995521/11156775/2-31072020-BP-EN.pdf/cbe7522c-ebfa-ef08-be60-b1c9d1bd385b}{a
12 percent contraction} in the second quarter of the year compared with
the first quarter. Different government interventions and infection
rates means the impact has been uneven. Here are snapshots from the
region's largest economies in the three months that ended in June.

\hypertarget{france}{%
\subsubsection{France}\label{france}}

Though France's 13.8 percent decline is stark, a mild
\href{https://www.nytimes.com/2020/07/14/business/as-europes-economies-reopen-consumers-go-on-a-spending-spree.html}{rebound
in consumer spending} and business activity after quarantines were
lifted has helped the country avoid a far sharper decline. In fact, the
nation's central bank recently revised its economic forecasts, expecting
slightly less damage in the next few years.

The government's largess has been key: It spent over 100 billion euros
(\$118 billion) to pay businesses not to lay off workers; it delayed
deadlines for business taxes and loan payments; and it deployed over 300
billion euros in state-guaranteed loans to struggling companies.

\hypertarget{germany}{%
\subsubsection{Germany}\label{germany}}

The 10.1 percent drop in Germany's G.D.P., the largest since the country
began keeping quarterly records, might already be painting a darker
picture of the economy than is warranted. Separate data released
Thursday showed the labor market stabilized in July and surveys of
\href{https://www.nytimes.com/2020/07/30/business/a-fan-maker-explains-why-german-managers-are-upbeat.html}{business
activity indicate a quick rebound}.

But the continuation of this recovery is at risk. Germany was in a
better position than other European Union countries because the
government was effective in containing the spread of the coronavirus.
However, there is now an increase in new infections as Germans return
from holidays abroad, stoking fear of a second wave.

\hypertarget{italy}{%
\subsubsection{Italy}\label{italy}}

The devastating economic impact of Italy's outbreak and lockdown, the
first in Europe, was a 12.4 percent drop in G.D.P. While the central
bank estimates that two government relief packages mitigated the
contraction, a slow return in tourism, consumer spending, and business
investment is dragging the recovery down.

``At least for Italy, the possibility of a V-shaped recovery is not what
we have in front of us,'' Bank of Italy's governor, Daniele Franco,
said. One slice of the economy is experiencing a stronger rebound:
industrial production. During the first phase of the lockdown, which
ended in early May, half of the Italian companies that were forced to
shut managed to reopen, the central bank said.

\hypertarget{spain}{%
\subsubsection{Spain}\label{spain}}

Spain's recession is the deepest of all the European countries that have
reported second-quarter G.D.P. so far. The economy contracted 18.5
percent compared to the first three months of the year, and the outlook
for the rest of the year is grim. Spain officially ended its Covid-19
state of emergency on June 21, but it has since been struggling with an
increase in the number of new cases and over 300 local outbreaks,
particularly severe in the northeast.

Tourism is a substantial component of the Spanish economy but hopes of a
tourism-led economic recovery this summer have been undermined by
quarantine restrictions placed on the nation and its islands by Britain
and other countries.

--- \href{https://www.nytimes.com/by/liz-alderman}{Liz Alderman},
\href{https://www.nytimes.com/by/jack-ewing}{Jack Ewing},
\href{https://www.nytimes.com/by/emma-bubola}{Emma Bubola},
\href{https://www.nytimes.com/by/raphael-minder}{Raphael Minder} and
\href{https://www.nytimes.com/by/eshe-nelson}{Eshe Nelson}

\hypertarget{exxon-reports-a-record-loss-and-chevron-writes-off-venezuela-investments}{%
\subsection{\texorpdfstring{\protect\hyperlink{exxon-reports-a-record-loss-and-chevron-writes-off-venezuela-investments}{Exxon
reports a record loss and Chevron writes off Venezuela
investments.}}{Exxon reports a record loss and Chevron writes off Venezuela investments.}}\label{exxon-reports-a-record-loss-and-chevron-writes-off-venezuela-investments}}

Copied to clipboard.

\includegraphics{https://static01.nyt.com/images/2020/07/31/business/31markets-brf-oil/merlin_173437968_821af7a5-648c-4195-85a7-e02a7d784bd4-articleLarge.jpg?quality=75\&auto=webp\&disable=upscale}

\textbf{Exxon Mobil} announced a record-breaking quarterly loss of \$1.1
billion, blaming the
\href{https://www.nytimes.com/2020/03/31/business/energy-environment/crude-oil-companies-coronavirus.html}{coronavirus
pandemic} for lowering oil and gas prices and sales volumes.

The results from the largest American oil producer were further evidence
of the deepest downturn for the industry in the modern era. Oil prices
have recovered in recent weeks to around \$40 a barrel, but that is
still roughly a third below the oil price of the beginning of the year.

\textbf{Chevron}, the second largest U.S. oil company, **** also posted
disappointing results for the quarter on Friday and said it was writing
off its \$2.6 billion investment in Venezuela because of the country's
political instability and American sanctions against its government.

Exxon's
\href{https://corporate.exxonmobil.com/Investors/Investor-relations}{oil
production was down 3 percent} and natural gas output was down 12
percent, compared to the quarter a year ago, a reflection of the
crippling of global demand for energy due to a worldwide recession.

Darren W. Woods, Exxon's chairman and chief executive, attempted to put
the best face on the results.

``The global pandemic and oversupply conditions significantly impacted
our second quarter financial results,'' he said. ``We responded
decisively by reducing near-term spending and continuing work to improve
efficiency. The progress we've made to date gives us confidence that we
will meet or exceed our cost-reduction targets.''

The \$1.1 billion loss compares to a profit of \$3.1 billion a year ago.
At the same time the company's capital and exploration expenditures were
down to \$5.3 billion from \$8.1 billion in the quarter last year.

Chevron said it lost \$8.3 billion in the quarter; a year earlier it
reported a \$4.3 billion profit.

The company reported an adjusted quarterly loss of \$3 billion,
excluding one-time items, compared to adjusted earnings of \$3.4 billion
in the same quarter of 2019. In addition to the \$2.6 billion Venezuelan
write down, Chevron also took a \$1.8 billion write down based on the
company's oil and gas price outlook.

Chevron reported sales and other revenue of \$16 billion, compared to
\$36 billion in the same period a year earlier.

``We're focused on what we can control,'' Michael K. Wirth, Chevron's
chairman and chief executive, said in a statement. ``We're transforming
our company to be more efficient, agile and innovative.''

Exxon and Chevron said they would maintain their dividends.

--- \href{https://www.nytimes.com/by/clifford-krauss}{Clifford Krauss}

\hypertarget{fiat-chrysler-lost-12-billion-in-the-second-quarter}{%
\subsection{\texorpdfstring{\protect\hyperlink{fiat-chrysler-lost-1-2-billion-in-the-second-quarter}{Fiat
Chrysler lost \$1.2 billion in the second
quarter}}{Fiat Chrysler lost \$1.2 billion in the second quarter}}\label{fiat-chrysler-lost-12-billion-in-the-second-quarter}}

Copied to clipboard.

\includegraphics{https://static01.nyt.com/images/2020/07/31/business/31markets-brf-fiat/31markets-brf-fiat-articleLarge.jpg?quality=75\&auto=webp\&disable=upscale}

\textbf{Fiat Chrysler} reported a net loss of 1 billion euros (\$1.2
billion) in the second quarter, but said it expects improving economic
conditions to lift its fortunes in the second half of the year.

Forced to shut down operations in Europe and North American for much of
the quarter because of the pandemic, Fiat Chrysler said revenue dropped
56 percent, to 11.7 billion euros. It also used some 5 billion euros in
cash.

In a conference call, the automaker's chief executive, Mike Manley, said
auto sales are recovering faster than had been expected, and the company
has been able to ramp production back to normal levels in North America.
Its European plants should return to typical production levels in the
third quarter, the company said.

``We expect significant improvement in profitability and cash flows,''
he said. ``We expect a much, much better second half.''

The automaker also plans to introduce five new electric vehicles in the
coming months, including plug-in hybrid versions of three different Jeep
models.

Fiat Chrysler is in the process of merging with French automaker PSA
Group, maker of the Peugeot and Citroën brands. The combined company
will be called Stellantis.

--- \href{https://www.nytimes.com/by/neal-e-boudette}{Neal E. Boudette}

\hypertarget{advertisement-2}{%
\subsubsection{Advertisement}\label{advertisement-2}}

\protect\hyperlink{after-dfp-ad-mid3}{Continue reading the main story}

\hypertarget{government-payments-helped-lift-income-and-spending-but-that-could-soon-end}{%
\subsection{\texorpdfstring{\protect\hyperlink{government-payments-helped-lift-income-and-spending-but-that-could-soon-end}{Government
payments helped lift income and spending, but that could soon
end.}}{Government payments helped lift income and spending, but that could soon end.}}\label{government-payments-helped-lift-income-and-spending-but-that-could-soon-end}}

Copied to clipboard.

\includegraphics{https://static01.nyt.com/images/2020/07/31/business/31markets-brf-eco/merlin_174645234_553cabe3-1c7c-4589-bf27-ef8740169de8-articleLarge.jpg?quality=75\&auto=webp\&disable=upscale}

Government payments played a critical role in propping up the American
economy, data released Friday shows.

\href{https://www.bea.gov/news/2020/personal-income-and-outlays-june-2020-and-annual-update}{Consumer
spending} rose 5.6 percent in June, the Commerce Department said, the
second straight monthly increase after a record-setting plunge in April.

But the end of some benefits, namely the \$1,200 payment made to many
individuals, also meant that personal income fell 1.1 percent last
month. Incomes could fall further now that the federal government's
additional unemployment benefits have ended, at least temporarily.

To understand what's happening, it helps to go back to the beginning of
the pandemic. When businesses began shutting their doors and furloughing
workers in March, both incomes and spending fell. Congress then stepped
in with a multi-trillion-dollar rescue package, which included sending
\$1,200 checks to most American families and expanding the unemployment
insurance system.

As a result, personal incomes rose a record 12.1 percent in April,
despite a big drop in wage and salary earnings. But spending still fell,
at least in part because people had fewer opportunities to go shopping
and dine out.
(\href{https://www.nytimes.com/2020/06/17/upshot/coronavirus-spending-rich-poor.html}{Other
data} suggests spending fell sharply among the wealthy, while rebounding
more quickly for other income groups once government checks began
arriving.)

In May and June, those patterns began to reverse. Spending picked back
up as the economy reopened. Wage and salary incomes rose too, as
companies began rehiring furloughed workers. Government payments fell
with the end of the \$1,200 checks, but remained high.

The net result: Overall personal income was higher in June than in
February. But without government intervention --- especially the
expanded unemployment benefits, which are injecting money into the
economy at a rate of \$1.4 trillion a year --- incomes would be lower
now than when the crisis began.

Spending has rebounded but remains almost 7 percent below its precrisis
level, even with the government help. And now, that help is in danger of
running out: The \$600 a week in extra unemployment benefits expires
today, and senators have
\href{https://www.nytimes.com/2020/07/30/us/politics/senate-virus-aid.html}{left
for the weekend}.

--- \href{https://www.nytimes.com/by/ben-casselman}{Ben Casselman}

\hypertarget{us-sanctions-more-chinese-officials-over-human-rights-violations-as-tensions-flare}{%
\subsection{\texorpdfstring{\protect\hyperlink{us-sanctions-more-chinese-officials-over-human-rights-violations-as-tensions-flare}{U.S.
sanctions more Chinese officials over human rights violations as
tensions
flare}}{U.S. sanctions more Chinese officials over human rights violations as tensions flare}}\label{us-sanctions-more-chinese-officials-over-human-rights-violations-as-tensions-flare}}

Copied to clipboard.

The Trump administration announced new sanctions Friday on two Chinese
officials and one government entity, citing human rights abuses against
predominantly Muslim ethnic minorities in the Xinjiang region in China's
far west.

The sanctions, administered by the Treasury Department's Office of
Foreign Assets Control, effectively cut the Xinjiang Production and
Construction Corps and two of its former officials, Sun Jinlong and Peng
Jiarui, off from American property and the financial system. The
Xinjiang Production and Construction Corps is an economic and
paramilitary organization in charge of economic development in the
region.

``The United States is committed to using the full breadth of its
financial powers to hold human rights abusers accountable in Xinjiang
and across the world,'' Steven T. Mnuchin, the Treasury Secretary, said
in a statement.

Ties between the United States and China have been fraying as the Trump
administration takes an increasingly critical posture on China's
handling of coronavirus, its growing influence over Hong Kong, its
territorial disputes in the South China Sea and its treatment of a
largely Muslim minority in Xinjiang.

The Chinese government has carried out a
\href{https://www.nytimes.com/interactive/2019/11/16/world/asia/china-xinjiang-documents.html}{campaign
of mass detentions} in Xinjiang, placing one million or more members of
Muslim and other minority groups into large internment camps intended to
increase their loyalty to the Communist Party.

On July 20, the Trump administration
\href{https://www.nytimes.com/2020/07/20/business/economy/china-sanctions-uighurs-labor.html}{added
11 new Chinese entities}, including firms supplying major American
brands like Apple, Ralph Lauren and Tommy Hilfiger, to a list that cuts
them off from purchasing American products without a special license,
saying the firms were complicit in human rights violations in Xinjiang.
On July 1, the administration issued a warning to businesses with supply
chains that run through Xinjiang to consider the reputational, economic
and legal risks of doing so.

--- \href{https://www.nytimes.com/by/ana-swanson}{Ana Swanson}

\hypertarget{the-economy-is-in-record-decline-but-not-for-the-tech-giants}{%
\subsection{\texorpdfstring{\protect\hyperlink{the-economy-is-in-record-decline-but-not-for-the-tech-giants}{The
economy is in record decline, but not for the tech
giants.}}{The economy is in record decline, but not for the tech giants.}}\label{the-economy-is-in-record-decline-but-not-for-the-tech-giants}}

Copied to clipboard.

\includegraphics{https://static01.nyt.com/images/2020/07/30/technology/30tech-earnings/oakImage-1596148409520-articleLarge.jpg?quality=75\&auto=webp\&disable=upscale}

A day after lawmakers grilled the chief executives of the biggest tech
companies about their size and power, \textbf{Alphabet},
\textbf{Amazon}, \textbf{Apple} and \textbf{Facebook} reported
surprisingly healthy quarterly financial results, defying one of the
\href{https://www.nytimes.com/live/2020/07/30/business/stock-market-today-coronavirus/the-us-economys-contraction-in-the-second-quarter-was-the-worst-on-record}{worst
economic downturns on record}.

Even though the companies felt some sting from the spending slowdown,
they demonstrated, as critics have argued, that they are operating on a
different playing field from the rest of the economy.

\begin{itemize}
\item
  \href{https://www.nytimes.com/live/2020/07/30/business/stock-market-today-coronavirus/amazons-earnings-double-as-sales-surge}{Amazon's
  sales were up} 40 percent from a year ago and its profit doubled.
\item
  \href{https://www.nytimes.com/live/2020/07/30/business/stock-market-today-coronavirus/facebook-nearly-doubles-its-profit-but-warns-of-fallout-from-ad-boycotts}{Facebook's
  profit jumped} 98 percent.
\item
  Even though the pandemic shuttered many of its stores,
  \href{https://www.nytimes.com/live/2020/07/30/business/stock-market-today-coronavirus/apple-blows-past-expectations-with-surging-sales-and-profits}{Apple
  increased sales} of all its products in every part of the world and
  posted \$11.25 billion in profit.
\item
  Advertising revenue
  \href{https://www.nytimes.com/live/2020/07/30/business/stock-market-today-coronavirus/alphabets-revenue-drops-but-beats-wall-street-expectations}{dropped
  for Alphabet}, the laggard of the bunch, but it still did better than
  Wall Street had expected.
\end{itemize}

Combined, the companies reported \$28.6 billion in quarterly net profit,
underscoring how
\href{https://www.nytimes.com/2020/07/29/technology/big-tech-hearing-apple-amazon-facebook-google.html}{regulatory
scrutiny} remains more background noise and a distraction for them
rather than an imminent threat to their businesses.

``The strong continue to get stronger,'' said Dan Ives, managing
director of equity research at Wedbush Securities. ``As many companies
are falling by the wayside, the tech stalwarts continue to gain muscle
and power in this environment.''

--- \href{https://www.nytimes.com/by/daisuke-wakabayashi}{Daisuke
Wakabayashi}, \href{https://www.nytimes.com/by/karen-weise}{Karen
Weise}, \href{https://www.nytimes.com/by/jack-nicas}{Jack Nicas} and
\href{https://www.nytimes.com/by/mike-isaac}{Mike Isaac}

\hypertarget{advertisement-3}{%
\subsubsection{Advertisement}\label{advertisement-3}}

\protect\hyperlink{after-dfp-ad-mid4}{Continue reading the main story}

\hypertarget{what-we-overheard-on-earnings-calls}{%
\subsection{\texorpdfstring{\protect\hyperlink{what-we-overheard-on-earnings-calls}{What
we overheard on earnings
calls.}}{What we overheard on earnings calls.}}\label{what-we-overheard-on-earnings-calls}}

Copied to clipboard.

The editors and reporters for the DealBook newsletter sift through a lot
of company reports and dial into many earnings conference calls. A huge
number of companies reported their latest financial results on Thursday,
and aside from the
\href{https://www.nytimes.com/2020/07/30/technology/tech-company-earnings-amazon-apple-facebook-google.htmlhttps://www.nytimes.com/2020/07/31/business/dealbook/tech-earnings-economy.html}{tech
giants' bumper profits} these are some of the things that caught our
notice, from lapsed cereal eaters to ``coronabeards.''

🍺 ``To put a finer point in the level of demand we're seeing, we
eclipsed July 4 week shipment days in the United States four times
already this year. That's unheard of.'' \emph{--- Gavin Hattersley, the
C.E.O. of Molson Coors}

🇯🇵 ``We would be in Tokyo right now under normal circumstances. So it's
a
\href{https://www.nytimes.com/live/2020/07/30/business/stock-market-today-coronavirus/comcast-saw-10-million-sign-ups-for-its-streaming-service-peacock}{total
bummer for our company} that we don't have the Olympics.'' \emph{---
Jeff Shell, the C.E.O. of NBCUniversal}

🥣 ``Special K gained share in quarter two as did Mini-Wheats and Raisin
Bran. We are also excited about the consumer trial and rediscovery we
are seeing from new and lapsed users in cereal.'' --- \emph{Steven
Cahillane, the C.E.O. of Kellogg's}

🧔 ``As people go back to work in offices and outside the home, we'll see
a pickup in the wet shave rate.'' --- \emph{David Taylor, the C.E.O. of
Procter \& Gamble, in response to an analyst question about the rise of}
\href{https://www.nytimes.com/2020/07/30/business/coronabeards-aside-procter-gamble-saw-strong-sales.html}{\emph{mullets
and ``coronabeards''}} \emph{during lockdowns}

🍩 ``I love when we really get on our doughnut mojo, but look, we are
leaning into beverages in a big way.'' \emph{--- David Hoffmann, the
C.E.O. of Dunkin' Brands}

--- \href{https://www.nytimes.com/by/jason-karaian}{Jason Karaian}

\hypertarget{site-index}{%
\subsection{Site Index}\label{site-index}}

\hypertarget{site-information-navigation}{%
\subsection{Site Information
Navigation}\label{site-information-navigation}}

\begin{itemize}
\tightlist
\item
  \href{https://help.nytimes.com/hc/en-us/articles/115014792127-Copyright-notice}{©~2020~The
  New York Times Company}
\end{itemize}

\begin{itemize}
\tightlist
\item
  \href{https://www.nytco.com/}{NYTCo}
\item
  \href{https://help.nytimes.com/hc/en-us/articles/115015385887-Contact-Us}{Contact
  Us}
\item
  \href{https://www.nytco.com/careers/}{Work with us}
\item
  \href{https://nytmediakit.com/}{Advertise}
\item
  \href{http://www.tbrandstudio.com/}{T Brand Studio}
\item
  \href{https://www.nytimes.com/privacy/cookie-policy\#how-do-i-manage-trackers}{Your
  Ad Choices}
\item
  \href{https://www.nytimes.com/privacy}{Privacy}
\item
  \href{https://help.nytimes.com/hc/en-us/articles/115014893428-Terms-of-service}{Terms
  of Service}
\item
  \href{https://help.nytimes.com/hc/en-us/articles/115014893968-Terms-of-sale}{Terms
  of Sale}
\item
  \href{https://spiderbites.nytimes.com}{Site Map}
\item
  \href{https://help.nytimes.com/hc/en-us}{Help}
\item
  \href{https://www.nytimes.com/subscription?campaignId=37WXW}{Subscriptions}
\end{itemize}
