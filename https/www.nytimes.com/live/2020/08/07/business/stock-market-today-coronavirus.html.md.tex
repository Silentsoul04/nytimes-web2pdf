Sections

SEARCH

\protect\hyperlink{site-content}{Skip to
content}\protect\hyperlink{site-index}{Skip to site index}

\href{https://myaccount.nytimes.com/auth/login?response_type=cookie\&client_id=vi}{}

\href{https://www.nytimes.com/section/todayspaper}{Today's Paper}

\href{https://www.nytimes.com/news-event/coronavirus}{The Coronavirus
Outbreak}

\begin{itemize}
\tightlist
\item
  live\href{https://www.nytimes.com/2020/08/08/world/coronavirus-updates.html}{Latest
  Updates}
\item
  \href{https://www.nytimes.com/interactive/2020/us/coronavirus-us-cases.html}{Maps
  and Cases}
\item
  \href{https://www.nytimes.com/interactive/2020/science/coronavirus-vaccine-tracker.html}{Vaccine
  Tracker}
\item
  \href{https://www.nytimes.com/interactive/2020/world/coronavirus-tips-advice.html}{F.A.Q.}
\item
  \href{https://www.nytimes.com/live/2020/08/07/business/stock-market-today-coronavirus}{Markets
  \& Economy}
\end{itemize}

Last Updated

Aug. 7, 2020, 10:34 p.m. ET

Aug. 7, 2020, 10:34 p.m. ET

\hypertarget{us-added-18-million-jobs-in-july}{%
\section{U.S. Added 1.8 Million Jobs in
July}\label{us-added-18-million-jobs-in-july}}

This briefing is no longer being updated. Follow live updates
\href{https://www.nytimes.com/2020/08/07/world/covid-19-news.html}{here}.

\hypertarget{jobs-remain-far-below-pre-pandemic-levels}{%
\subsubsection{Jobs remain far below pre-pandemic
levels}\label{jobs-remain-far-below-pre-pandemic-levels}}

\hypertarget{cumulative-change-in-jobs-since-july-2016}{%
\paragraph{Cumulative change in jobs since July
2016}\label{cumulative-change-in-jobs-since-july-2016}}

By Allison McCann·Data is seasonally adjusted.·Source: Bureau of Labor
Statistics

\hypertarget{heres-what-you-need-to-know}{%
\subsubsection{Here's what you need to
know:}\label{heres-what-you-need-to-know}}

\begin{itemize}
\item
  \protect\hyperlink{us-employers-added-1-8-million-jobs-in-july-despite-a-coronavirus-surge}{}

  U.S. employers added 1.8 million jobs in July despite a coronavirus
  surge.
\item
  \protect\hyperlink{wealthy-families-are-throwing-a-lifeline-to-distressed-businesses}{}

  Wealthy families are throwing a lifeline to distressed businesses.
\item
  \protect\hyperlink{canada-outlines-its-response-to-the-new-us-aluminum-tariff}{}

  Canada outlines its response to the new U.S. aluminum tariff.
\item
  \protect\hyperlink{an-expert-on-economic-calamities-sees-very-very-dangerous-territory}{}

  An expert on economic calamities sees `very, very dangerous
  territory.'
\item
  \protect\hyperlink{wall-street-is-held-back-by-china-tensions-and-gridlock-in-washington}{}

  Wall Street is held back by China tensions and gridlock in Washington.
\end{itemize}

\hypertarget{us-employers-added-18-million-jobs-in-july-despite-a-coronavirus-surge}{%
\subsection{\texorpdfstring{\protect\hyperlink{us-employers-added-1-8-million-jobs-in-july-despite-a-coronavirus-surge}{U.S.
employers added 1.8 million jobs in July despite a coronavirus
surge.}}{U.S. employers added 1.8 million jobs in July despite a coronavirus surge.}}\label{us-employers-added-18-million-jobs-in-july-despite-a-coronavirus-surge}}

Copied to clipboard.

The American economy gained 1.8 million jobs last month, even as the
coronavirus surged in many parts of the country and newly reintroduced
restrictions caused some businesses to close for a second time.

Still, the increase
\href{https://www.bls.gov/news.release/empsit.nr0.htm}{reported Friday
by the Labor Department} was well below the 4.8 million jump in jobs in
June and a sign that momentum is slowing after a burst of economic
activity in late spring. The unemployment rate fell to 10.2 percent.

\hypertarget{unemployment-rate}{%
\subsubsection{Unemployment rate}\label{unemployment-rate}}

By Allison McCann·Unemployment rates are seasonally adjusted. The
government began collecting standardized unemployment statistics in
1948.·Source: Bureau of Labor Statistics

``The labor market continues to heal, which is encouraging, but there is
a long road ahead,'' said Michelle Meyer, head of U.S. economics at Bank
of America.

She noted that 42 percent of the jobs lost since the pandemic hit had
now been recovered, but warned the remainder would be harder to make up.

``In the very early stages of the recovery it's easier to bring back
workers quickly just to have a functioning operation,'' she said. ``It's
not a snap back to pre-Covid levels by any means. It's a healing
process.''

But dig a little deeper and the picture isn't so bright.

The \href{https://www.bls.gov/news.release/empsit.t15.htm}{Labor
Department's U-6 measure}, which includes discouraged workers who have
given up the search for work as well as those who are in part-time jobs
because they can't find full-time positions, stands at 16.5 percent,
seasonally adjusted.

The U-6 figure has come down from 22.8 percent in April. Still, the U-6
is among a series of data points that underscore just how difficult the
labor market remains for those out of work.

``The rate of churn in the labor market remains incredibly high,''
concluded Morgan Stanley's economics team. In plain English, that means
millions of workers finding a job only to be fired soon afterward, or
being let go permanently after assuming a layoff was temporary.

``I think the U-6 is a better indicator of the job market than the 10
percent unemployment rate,'' said Beth Ann Bovino, chief U.S. economist
at S\&P Global. ``The traditional unemployment rate doesn't capture
what's happening on the ground.''

--- \href{https://www.nytimes.com/by/nelson-d-schwartz}{Nelson D.
Schwartz}

\hypertarget{the-publisher-of-the-onion-jezebel-and-other-websites-lays-off-15-employees}{%
\subsection{\texorpdfstring{\protect\hyperlink{the-publisher-of-the-onion-jezebel-and-other-websites-lays-off-15-employees}{The
publisher of The Onion, Jezebel and other websites lays off 15
employees.}}{The publisher of The Onion, Jezebel and other websites lays off 15 employees.}}\label{the-publisher-of-the-onion-jezebel-and-other-websites-lays-off-15-employees}}

Copied to clipboard.

\includegraphics{https://static01.nyt.com/images/2020/08/07/business/07markets-brf-media/07markets-brf-media-articleLarge.jpg?quality=75\&auto=webp\&disable=upscale}

\textbf{G/O Media}, the publisher of \textbf{The Onion},
\textbf{Gizmodo}, \textbf{Jezebel} and several other websites, laid off
more than a dozen staff members in its video department Friday,
prompting accusations from the employees' union that websites' editors
in chief had not been consulted.

A G/O Media spokeswoman said that 15 employees had been let go. A
statement said the decision was necessary to allow the company to invest
in other areas.

``In our efforts to strengthen editorial teams at G/O Media, we
completed a thorough evaluation of our traffic and sites,'' it said.
``In doing so, we're making the unfortunate but necessary decision to
change our current process of video production.''

But in a statement posted on Twitter, the GMG Union accused the company
of taking the video department's legs out from under it.

``Instead of working with editorial leadership to figure out how to
execute a successful video strategy,'' it said, ``they laid off many
talented and vital staff members, leaving a serious gap in sites'
abilities to produce any videos at all.''

\begin{quote}
Our statement regarding today's layoff of more than half our video
staff: \href{https://t.co/bWXngHm91r}{pic.twitter.com/bWXngHm91r}

--- GMG Union (@gmgunion)
\href{https://twitter.com/gmgunion/status/1291858333868097542?ref_src=twsrc\%5Etfw}{August
7, 2020}
\end{quote}

In April, G/O laid off 14 workers in response to the coronavirus
pandemic, which has caused a sharp drop in advertising.

But the media company had been struggling before the pandemic. Last
fall, nearly two dozen employees resigned from the sports news site
\textbf{Deadspin} after its acting editor in chief was fired, causing
the sports blog to fall silent for several months. Also that fall, G/O
shuttered \textbf{Splinter}, a politics-focused blog.

G/O is owned by the Boston private equity firm \textbf{Great Hill
Partners}, which purchased what was then known as \textbf{Gizmodo Media
Group} from \textbf{Univision} in April 2019. Univision in turn had
bought several of the sites from \textbf{Gawker Media}, which had filed
for bankruptcy in 2016 following a \$140 million judgment in an
invasion-of-privacy lawsuit secretly bankrolled by the Silicon Valley
executive Peter Thiel.

--- \href{https://www.nytimes.com/by/marc-tracy}{Marc Tracy}

\hypertarget{advertisement}{%
\subsubsection{Advertisement}\label{advertisement}}

\protect\hyperlink{after-dfp-ad-mid1}{Continue reading the main story}

\hypertarget{wealthy-families-are-throwing-a-lifeline-to-distressed-businesses}{%
\subsection{\texorpdfstring{\protect\hyperlink{wealthy-families-are-throwing-a-lifeline-to-distressed-businesses}{Wealthy
families are throwing a lifeline to distressed
businesses.}}{Wealthy families are throwing a lifeline to distressed businesses.}}\label{wealthy-families-are-throwing-a-lifeline-to-distressed-businesses}}

Copied to clipboard.

\includegraphics{https://static01.nyt.com/images/2020/08/08/business/07Wealth-01/07Wealth-01-articleLarge.jpg?quality=75\&auto=webp\&disable=upscale}

During the pandemic, wealthy families have continued to use their
investment pools, known as family offices, to gain access to the type of
high-return opportunities once reserved for institutional investors. But
they are taking a more hands-on role in those financial decisions.

These family offices have chosen to bypass private equity and venture
capital funds --- which have high minimum investments and sizable fees
--- to invest directly in companies, either by themselves or with other
significantly wealthy families,
\href{https://www.fintrx.com/fintrx-charles-schwab-2020-family-office-report}{a
report} released on Friday found.

Half of all family offices in the world make direct investments in
companies, according to the report, which was released by Fintrx, a data
and research company, and sponsored by Charles Schwab's family office
arm.

``Family offices add value in times of crisis,'' said Russ D'Argento,
founder and chief executive of Fintrx. ``That's a big component of how
they stand out and can be different from other fund structures.''

With so many distressed companies looking for investors in the pandemic,
family offices have an opportunity to leverage the ``family alpha,'' or
the operating knowledge that a family has in the area from which its
wealth came, said Kristi Kuechler, managing director of client relations
at Vernal Point Advisors, a multifamily office.

``There are families who have as much knowledge of a sector as a private
equity firm,'' Ms. Kuechler said.

--- \href{https://www.nytimes.com/by/paul-sullivan}{Paul Sullivan}

\hypertarget{canada-outlines-its-response-to-the-new-us-aluminum-tariff}{%
\subsection{\texorpdfstring{\protect\hyperlink{canada-outlines-its-response-to-the-new-us-aluminum-tariff}{Canada
outlines its response to the new U.S. aluminum
tariff.}}{Canada outlines its response to the new U.S. aluminum tariff.}}\label{canada-outlines-its-response-to-the-new-us-aluminum-tariff}}

Copied to clipboard.

The Canadian government said on Friday that it planned to impose tariffs
on aluminum products from the United States, in a tit-for-tat response
to a similar measure announced by President Trump on Thursday.

Canada's tariffs, on ``aluminum and aluminum-containing products from
the U.S.,'' will take effect by Sept. 16 and will remain in place until
the United States eliminates its tariffs, the Canadian government said
in a statement on Friday.

The statement included a list of products that might be subject to the
tariffs, from aluminum wire to refrigerators and washing machines.

President Trump said Thursday that he would impose a
\href{https://www.nytimes.com/2020/08/06/business/economy/trump-canadian-aluminum-tariffs.html}{10
percent tariff on Canadian aluminum} in an effort to help struggling
American producers. In a speech at a Whirpool factory in Clyde, Ohio,
Mr. Trump accused Canada of ``taking advantage of us as usual.''

In early 2018, Mr. Trump imposed tariffs on steel and aluminum from
Canada, Mexico and the European Union, which caused those countries to
impose their own tariffs on good from the United States. The tariffs
from Canada and Mexico were not lifted until the next year.

--- Gillian Friedman

\hypertarget{an-expert-on-economic-calamities-sees-very-very-dangerous-territory}{%
\subsection{\texorpdfstring{\protect\hyperlink{an-expert-on-economic-calamities-sees-very-very-dangerous-territory}{An
expert on economic calamities sees `very, very dangerous
territory.'}}{An expert on economic calamities sees `very, very dangerous territory.'}}\label{an-expert-on-economic-calamities-sees-very-very-dangerous-territory}}

Copied to clipboard.

\includegraphics{https://static01.nyt.com/images/2020/08/07/business/07markets-brf-rogoff/merlin_57344744_18f4a987-93cd-4335-84f1-67c3ce2653ff-articleLarge.jpg?quality=75\&auto=webp\&disable=upscale}

With 13 million fewer people working since the pandemic hit, according
to the \href{https://www.bls.gov/news.release/empsit.nr0.htm}{monthly
jobs report released on Friday}, the economist Kenneth S. Rogoff --- an
expert on financial crises --- says the American economy is at a
precarious point.

``We are going to clock the worst recession since the Great Depression,
regardless of how fast we bounce back at this point,'' he said. ``The
virus is coming back, hard and fast. It really does look like this is
going to have profound long-term impacts.''

A Harvard University professor, Mr. Rogoff is a noted historian of
economic calamities. His books include ``This Time Is Different: Eight
Centuries of Financial Folly,'' written with Carmen M. Reinhart in 2009.

Mr. Rogoff said the current state of virus was reminiscent of 1918
Spanish Flu, in which the second wave of virus proved even more
devastating from an economic and public health perspective than the
first. At this point, the economic damage from the coronavirus has far
surpassed the 2008 recession, he said.

Small businesses will be hit hardest, Mr. Rogoff said.

``We're going to start to see a lot of small businesses fall by the
wayside, a lot of people who are unemployed become chronically
unemployed,'' he said. ``We're in very, very dangerous territory.''

Large corporations will be more shielded from the impact of the virus,
accelerating their ability to crush smaller competitors, a trend that
the United States has been experiencing over the last 40 years, he said.

``They have cash reserves to survive this,'' he said. ``And so their
monopoly power is going to grow.''

--- Gillian Friedman

\hypertarget{advertisement-1}{%
\subsubsection{Advertisement}\label{advertisement-1}}

\protect\hyperlink{after-dfp-ad-mid2}{Continue reading the main story}

\hypertarget{leisure-and-hospitality-jobs-led-the-july-gains-but-the-pandemics-dent-remains-deep}{%
\subsection{\texorpdfstring{\protect\hyperlink{leisure-and-hospitality-jobs-led-the-july-gains-but-the-pandemics-dent-remains-deep}{Leisure
and hospitality jobs led the July gains, but the pandemic's dent remains
deep.}}{Leisure and hospitality jobs led the July gains, but the pandemic's dent remains deep.}}\label{leisure-and-hospitality-jobs-led-the-july-gains-but-the-pandemics-dent-remains-deep}}

Copied to clipboard.

\hypertarget{industries-are-rebounding-but-none-have-fully-recovered}{%
\subsubsection{Industries are rebounding, but none have fully
recovered}\label{industries-are-rebounding-but-none-have-fully-recovered}}

\hypertarget{cumulative-change-in-jobs-since-july-2016-by-industry}{%
\paragraph{Cumulative change in jobs since July 2016, by
industry}\label{cumulative-change-in-jobs-since-july-2016-by-industry}}

By Allison McCann·Data is seasonally adjusted.·Source: Bureau of Labor
Statistics

Despite renewed restrictions on business activity in some parts of the
country, the leisure and hospitality industry managed to show some signs
of life in July, gaining 592,000 jobs, or one-third of the total gain in
payrolls for the month.

The sector was among the hardest hit when restaurants and bars closed
abruptly in March as the pandemic hit. July's increase follows a jump of
3.4 million in May and June, seasonally adjusted, but still leaves
employment in the leisure and hospitality field 4.3 million below where
it was in February.

Retail, another hard-hit sector which has seen numerous bankruptcies in
recent months, added 258,000 jobs.

``Retail and leisure and hospitality are two of the sectors most
sensitive to coronavirus, and I was pleasantly surprised by the pace of
job creation there,'' said Michelle Meyer, head of U.S. economics at
Bank of America.

The plunge in employment in these sectors hit lower-paid workers
especially hard, including millions who depend on tips. For big
increases in hiring at restaurants and bars, employees may need to wait
until indoor dining is again permitted in states like New York ---
something unlikely to occur until a vaccine is found.

--- \href{https://www.nytimes.com/by/nelson-d-schwartz}{Nelson D.
Schwartz}

\hypertarget{state-and-local-government-added-jobs-back-but-its-not-what-it-seems}{%
\subsection{\texorpdfstring{\protect\hyperlink{state-and-local-government-added-jobs-back-but-its-not-what-it-seems}{State
and local government added jobs back, but it's not what it
seems.}}{State and local government added jobs back, but it's not what it seems.}}\label{state-and-local-government-added-jobs-back-but-its-not-what-it-seems}}

Copied to clipboard.

Employment in state and local government arrested its decline in July
--- but the change was largely the result of a quirk in how the numbers
are adjusted, and it left the combined work forces much smaller than
February.

Local governments have cut about 970,000 jobs since the month before the
pandemic took hold, while state governments now employ 200,000 fewer
people, on a seasonally adjusted basis. Combined, they have shed nearly
6 percent of their pre-pandemic work force.

Economists and policymakers are concerned that the job losses will
continue as local government budgets come under extreme strain. While
July offered what seemed to be a reprieve, with state and local hiring
ticking up, the improvement was heavily driven by education hiring as
seasonal adjustments made the numbers look rosier. On an unadjusted
basis, the figures showed
\href{https://fred.stlouisfed.org/series/CEU9093000001}{continued
declines}.

``Typically, public-sector education employment declines in July,'' the
Bureau of Labor Statistics
\href{https://www.bls.gov/news.release/empsit.nr0.htm}{said} in its
release, but ``declines occurred earlier than usual this year due to the
pandemic, resulting in unusually large July increases'' after the
seasonal adjustment.

Aid to state and local governments remains a flash point in negotiations
over a new federal relief package. Democrats
\href{https://www.nytimes.com/2020/08/05/us/politics/congress-coronavirus-stimulus.html}{are
pushing for} more assistance, something that congressional Republicans
and the White House have resisted or opposed.

The Federal Reserve and Treasury Department
\href{https://www.nytimes.com/2020/08/06/business/house-democrats-want-fed-to-give-cities-and-states-more-help-by-improving-program-terms.html}{have
established a program} to buy short-term municipal debt from certain
state and local governments, but it has not been used much. The terms
are not generous, experts have said, and many local governments are
hoping for grants rather than loans that they would have to pay back.

Without help, further job losses could be in store.

``Unlike small businesses or restaurants --- which respond immediately
to economic shocks --- deep budget and job cuts in state and local
government will likely grow in the next few months and fester for years
to come,''
\href{https://www.brookings.edu/blog/the-avenue/2020/08/03/state-and-local-governments-employ-the-highest-share-of-essential-workers-congress-is-failing-to-protect-them/}{researchers}
at the Brookings Institution wrote in a recent post.

--- \href{https://www.nytimes.com/by/jeanna-smialek}{Jeanna Smialek}

\hypertarget{wall-street-is-held-back-by-china-tensions-and-gridlock-in-washington}{%
\subsection{\texorpdfstring{\protect\hyperlink{wall-street-is-held-back-by-china-tensions-and-gridlock-in-washington}{Wall
Street is held back by China tensions and gridlock in
Washington.}}{Wall Street is held back by China tensions and gridlock in Washington.}}\label{wall-street-is-held-back-by-china-tensions-and-gridlock-in-washington}}

Copied to clipboard.

Stocks on Wall Street lost their footing on Friday, as investors moved
cautiously amid escalating tensions between the United States and China
and little indication that lawmakers in Washington were close to
resolving their differences over the next economic aid package.

It helped, somewhat, that the monthly employment report showed that
American employers
\href{https://www.nytimes.com/live/2020/08/07/business/stock-market-today-coronavirus/us-employers-added-1-8-million-jobs-in-july-despite-a-coronavirus-surge}{added
1.8 million jobs} in July, continuing a rebound that began earlier this
year. But even in that report there were reasons for caution: The rate
of hiring slowed substantially from June, and the unemployment rate
remained above 10 percent.

After recouping early losses, the S\&P 500 was essentially unchanged by
the end of trading Friday. Technology stocks that have led shares higher
in recent days tumbled on Friday, with the Nasdaq composite down nearly
1 percent as \textbf{Apple}, \textbf{Amazon}, \textbf{Alphabet} and
\textbf{Microsoft} all declined.

Hanging over Wall Street was President Trump's decision late Thursday to
order sweeping restrictions on two popular
\href{https://www.nytimes.com/2020/08/06/technology/trump-wechat-tiktok-china.html}{Chinese
social media networks}, \textbf{TikTok} and \textbf{WeChat.} Two
executive orders cited national security concerns in barring
transactions with WeChat or TikTok by any person or property subject to
the jurisdiction of the United States. The orders take effect in 45
days.

The moves are expected to prompt
\href{https://www.nytimes.com/2020/08/07/business/trump-china-wechat-tiktok.html}{retaliation
from China}. A Chinese Ministry of Foreign Affairs spokesman called the
executive orders a ``nakedly hegemonic act.'' Shares in
\textbf{Tencent}, the parent company of WeChat, fell almost 6 percent,
and markets in Asia dropped.

Investors were also watching talks in Washington over what shape another
economic aid package would take. Federal unemployment benefits, a
moratorium on evictions and aid for small businesses shuttered during
the pandemic hang in the balance, and economists have repeatedly warned
that failure to extend the assistance could imperil the American
economy.

On Friday, negotiations between the White House and Democrats stalled,
as both sides said they remained deeply divided on the package.
President Trump's advisers said they would recommend that he bypass
Congress and act on his own to provide relief.

--- \href{https://www.nytimes.com/by/kevin-granville}{Kevin Granville}
and Mohammed Hadi

\hypertarget{advertisement-2}{%
\subsubsection{Advertisement}\label{advertisement-2}}

\protect\hyperlink{after-dfp-ad-mid3}{Continue reading the main story}

\hypertarget{more-news-uber-and-amc-post-big-losses-dow-jones-financials-revealed}{%
\subsection{\texorpdfstring{\protect\hyperlink{more-news-uber-and-amc-post-big-losses-dow-jones-financials-revealed}{More
news: Uber and AMC post big losses, Dow Jones financials
revealed.}}{More news: Uber and AMC post big losses, Dow Jones financials revealed.}}\label{more-news-uber-and-amc-post-big-losses-dow-jones-financials-revealed}}

Copied to clipboard.

Here's some of the news you might have missed.

\begin{itemize}
\item
  \textbf{The Evening Standard}, a free daily newspaper in London, is
  planning to lay off 139 employees, or about a third of the staff, as
  well as other noncontract workers. The paper's main readership ---
  commuters in central London --- has all but vanished because of the
  pandemic, but the company was already facing financial difficulty,
  having reported a pre-tax loss of £13.6 million (\$17.8 million) last
  year. A spokesperson said the company would focus on growing the
  paper's digital and live events business. .
\item
  \textbf{Uber} said on Thursday that its revenue in the second quarter
  dropped 29 percent to \$2.2 billion from a year ago and that its net
  loss narrowed to \$1.8 billion, as the ride-hailing giant deals with
  the fallout from the coronavirus pandemic. The revenue decline was the
  steepest since
  \href{https://www.nytimes.com/live/2020/08/06/business/stock-market-today-coronavirus/uber-reports-steep-revenue-decline-as-delivery-outpaces-ride-hailing}{Uber}
  went public in May 2019.
\item
  Rupert Murdoch's
  \textbf{\href{https://www.nytimes.com/live/2020/08/06/business/stock-market-today-coronavirus/dow-jones-was-news-corps-only-growing-division-this-past-fiscal-year}{News
  Corp}} reported a \$401 million loss for the three months ending in
  June, with much of the decline related to impairment charges for some
  of its assets in Britain and Australia and restructuring costs related
  to the coronavirus pandemic. The company revealed for the first time
  financial details of its \textbf{Dow Jones} division, the group that
  publishes The Wall Street Journal. The unit was News Corp's only
  growing business on an annual basis.
\item
  With operations ceased for the entirety of the quarter and most of its
  employees laid off or furloughed,
  \textbf{\href{https://www.nytimes.com/live/2020/08/06/business/stock-market-today-coronavirus/amcs-quarterly-revenues-dropped-98-7-percent-from-last-year}{AMC
  Entertainment}}, the largest theater chain in the United States,
  posted a quarterly loss for the period ended June of \$561.2 million.
  Revenues totaled \$18.9 million, a 98.7 percent plunge from the same
  period last year for the Kansas-based company. The coronavirus has
  laid waste to AMC's 1,000 theaters scattered across the globe, calling
  into question whether it would be able to stay financially viable.
\item
  The Trump administration is considering forcing Chinese companies to
  delist their shares from stock exchanges in the United States unless
  they share their audits with American regulators, a move that would
  further ratchet up tension between the world's two largest economies.
  The
  \href{https://www.nytimes.com/live/2020/08/06/business/stock-market-today-coronavirus/us-may-insist-chinese-companies-share-audits-or-delist-their-shares-from-american-exchanges}{President's
  Working Group on Financial Markets} recommended the move in a report
  released on Thursday as a way to protect American investors from what
  it described as the
  \href{https://www.nytimes.com/2012/07/13/business/in-china-inspecting-the-inspectors.html}{risks
  posed by Chinese companies}.
\end{itemize}

\hypertarget{site-index}{%
\subsection{Site Index}\label{site-index}}

\hypertarget{site-information-navigation}{%
\subsection{Site Information
Navigation}\label{site-information-navigation}}

\begin{itemize}
\tightlist
\item
  \href{https://help.nytimes.com/hc/en-us/articles/115014792127-Copyright-notice}{©~2020~The
  New York Times Company}
\end{itemize}

\begin{itemize}
\tightlist
\item
  \href{https://www.nytco.com/}{NYTCo}
\item
  \href{https://help.nytimes.com/hc/en-us/articles/115015385887-Contact-Us}{Contact
  Us}
\item
  \href{https://www.nytco.com/careers/}{Work with us}
\item
  \href{https://nytmediakit.com/}{Advertise}
\item
  \href{http://www.tbrandstudio.com/}{T Brand Studio}
\item
  \href{https://www.nytimes.com/privacy/cookie-policy\#how-do-i-manage-trackers}{Your
  Ad Choices}
\item
  \href{https://www.nytimes.com/privacy}{Privacy}
\item
  \href{https://help.nytimes.com/hc/en-us/articles/115014893428-Terms-of-service}{Terms
  of Service}
\item
  \href{https://help.nytimes.com/hc/en-us/articles/115014893968-Terms-of-sale}{Terms
  of Sale}
\item
  \href{https://spiderbites.nytimes.com}{Site Map}
\item
  \href{https://help.nytimes.com/hc/en-us}{Help}
\item
  \href{https://www.nytimes.com/subscription?campaignId=37WXW}{Subscriptions}
\end{itemize}
