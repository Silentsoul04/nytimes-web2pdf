Sections

SEARCH

\protect\hyperlink{site-content}{Skip to
content}\protect\hyperlink{site-index}{Skip to site index}

\href{https://myaccount.nytimes.com/auth/login?response_type=cookie\&client_id=vi}{}

\href{https://www.nytimes.com/section/todayspaper}{Today's Paper}

\href{https://www.nytimes.com/news-event/coronavirus}{The Coronavirus
Outbreak}

\begin{itemize}
\item
   •
  \href{https://www.nytimes.com/2020/06/22/world/coronavirus-updates.html}{Latest
  Updates}
\item
  \href{https://www.nytimes.com/interactive/2020/us/coronavirus-us-cases.html}{Maps
  and Cases}
\item
  \href{https://www.nytimes.com/interactive/2020/us/states-reopen-map-coronavirus.html}{Reopenings
  by State}
\item
  \href{https://www.nytimes.com/2020/06/22/nyregion/nyc-phase-2-reopening-coronavirus.html}{N.Y.C.
  Enters Phase 2}
\end{itemize}

Last Updated

March 24, 2020, 8:01 a.m. ET

March 24, 2020, 8:01 a.m. ET

\hypertarget{covid-19s-impact-in-the-us}{%
\section{Covid-19's Impact in the
U.S.}\label{covid-19s-impact-in-the-us}}

\includegraphics{https://static01.nyt.com/images/2020/03/23/world/23virus-lives-lede-sub/merlin_170874204_b4a22fd1-c58e-4155-bacd-81749116d91d-articleLarge.jpg?quality=75\&auto=webp\&disable=upscale}

The coronavirus is changing how we live our daily lives. Taking a look
at how the global pandemic has affected various aspects of life in the
United States reveals the unique nature of this crisis.

\begin{itemize}
\item
  For the latest on how the coronavirus is changing lives in the U.S.,
  \href{https://www.nytimes.com/live/2020/coronavirus-usa-live-03-24}{read
  Tuesday's live updates.}
\item
  A
  \href{https://www.nytimes.com/2020/03/23/us/coronavirus-westport-connecticut-party-zero.html?action=click\&module=Spotlight\&pgtype=Homepage}{soirée
  in Connecticut} is a story of how, in the Gilded Age of money, social
  connectedness and air travel, a pandemic has spread at lightning
  speed.
\item
  Coronavirus is one of the biggest stories most publications will ever
  cover. But it has left
  \href{https://www.nytimes.com/2020/03/23/business/media/coronavirus-local-news.html?action=click\&module=RelatedLinks\&pgtype=Article}{many
  of them struggling to stay solvent.}
\item
  In the sporting world,
  \href{https://www.nytimes.com/2020/03/23/sports/olympics/track-and-field-poll-Olympics-coronavirus.html}{athletes
  are overwhelmingly in favor} of delaying the Summer Olympics. Two of
  basketball's biggest stars, LeBron James and Giannis Antetokounmpo,
  are
  \href{https://www.nytimes.com/live/2020/coronavirus-updates-news-03-23\#you-cant-watch-giannis-antetokounmpo-hoop-but-you-can-watch-him-strum}{trying
  to find ways to pass the time}.
\item
  The uncertainty about how long this will last and what will happen
  next leaves many of us mourning current losses as well as ones we
  haven't experienced yet.
  \href{https://www.nytimes.com/2020/03/23/well/family/coronavirus-grief-loss.html?action=click\&module=RelatedLinks\&pgtype=Article}{Here's
  how to cope.}
\end{itemize}

\href{https://www.nytimes.com/by/elizabeth-williamson}{\includegraphics{https://static01.nyt.com/images/2019/10/22/reader-center/author-elizabeth-williamson/author-elizabeth-williamson-thumbLarge-v2.png}}\href{https://www.nytimes.com/by/kristin-hussey}{\includegraphics{https://static01.nyt.com/images/2018/12/10/multimedia/author-kristin-hussey/author-kristin-hussey-thumbLarge.png}}

March 23, 2020, 9:00 p.m. ET

March 23, 2020, 9:00 p.m. ET

By \href{https://www.nytimes.com/by/elizabeth-williamson}{Elizabeth
Williamson} and \href{https://www.nytimes.com/by/kristin-hussey}{Kristin
Hussey}

\hypertarget{an-upscale-party-in-connecticut-became-a-super-spreader-heres-how}{%
\subsection{\texorpdfstring{\protect\hyperlink{an-upscale-party-in-connecticut-became-a-super-spreader-heres-how}{An
upscale party in Connecticut became a `super spreader.' Here's
how.}}{An upscale party in Connecticut became a `super spreader.' Here's how.}}\label{an-upscale-party-in-connecticut-became-a-super-spreader-heres-how}}

Image

Businesses have in Westport, Conn., where a surge in coronavirus cases
has been reported.Credit...Dave Sanders for The New York Times

About 50 guests gathered on March 5 at a home in the stately suburb of
Westport, Conn., to toast the hostess on her 40th birthday and greet old
friends, including one visiting from South Africa. They shared
reminiscences, a lavish buffet and, unknown to anyone, the coronavirus.

Then they scattered.

The Westport soirée --- Party Zero in southwestern Connecticut and
beyond --- is a story of how, in the Gilded Age of money, social
connectedness and air travel, a
\href{https://www.nytimes.com/interactive/2020/world/coronavirus-maps.html}{pandemic
has spread} at lightning speed. The partygoers --- more than half of
whom are now infected --- left that evening for Johannesburg, New York
City and other parts of Connecticut and the United States, all seeding
infections on the way.

The party ``may be an example of the kind of thing we call a
super-spreading event,'' said William Hanage, an associate professor of
epidemiology at Harvard, especially since some of the partygoers later
attended large social events in the New York metropolitan area, where
cases of the virus are high.

Worry, rumors and recriminations engulfed the town. Political leaders
fielded hundreds of emails and phone calls from residents terrified that
their children or vulnerable family members had been exposed. Who threw
the party, and who attended? They wanted to know. Rumors flew that some
residents were telling health officials they had attended the party so
they could obtain a scarce test.

As the disease spread, many residents kept mum, worried about being
ostracized by their neighbors and that their children would be kicked
off coveted sports teams or miss school events.

One local woman compared going public with a Covid-19 diagnosis to
``having an S.T.D.''

``I don't think that's a crazy comparison,'' said Will Haskell, the
state senator who represents Westport. He has been fielding frantic
phone calls from constituents.

``This is life or death,'' he said in an interview. ``Westport really is
a cautionary tale of what we're soon to see.''

Sheila Kaplan contributed reporting from Washington. Kitty Bennett
contributed research.

Read more

\includegraphics{https://static01.nyt.com/images/icons/t_logo_291_black.png}

March 23, 2020, 8:30 p.m. ET

March 23, 2020, 8:30 p.m. ET

By Lori Gottlieb

\hypertarget{when-handling-loss-let-each-person-do-it-their-own-way}{%
\subsection{\texorpdfstring{\protect\hyperlink{when-handling-loss-let-each-person-do-it-their-own-way}{When
handling loss, let each person do it their own
way.}}{When handling loss, let each person do it their own way.}}\label{when-handling-loss-let-each-person-do-it-their-own-way}}

There is collective anxiety surrounding Covid-19, but there's also
collective loss. Here are some ways to help navigate through our losses.

\textbf{1. Acknowledge the grief}

Although anxiety is unpleasant, it can be easier to acknowledge anxiety
than to acknowledge grief. That's because there are two kinds of
anxiety: productive anxiety and unproductive anxiety.

Grieving, on the other hand, is a much quieter process. It requires us
to sit with our pain, to feel a kind of sadness that makes many of us so
uncomfortable that we try to get rid of it. In the age of coronavirus, a
child might say: ``I'm so sad that I'm missing seeing my friends every
day'' and the parent, trying to lessen the child's pain, might say:
``But honey, we're so lucky that we're not sick and you'll get to see
your friends soon!'' A more helpful response might be: ``I know how sad
you are about this. You miss being with your friends so much. It's a big
loss not to have that.''

Just as our kids need to have their grief acknowledged, we need to
acknowledge our own. The more we can say to ourselves and the people
around us, ``Yes, these are meaningful losses,'' the more seen and
soothed we will feel.

\textbf{2. Stay in the present}

There's a term to describe the kind of loss many of us are experiencing:
\href{https://www.researchgate.net/publication/258193450_Ambiguous_Loss_and_the_Family_Grieving_Process}{ambiguous
grief}. In ambiguous grief, there's a murkiness to the loss.

With Covid-19, on top of the tangible losses, there's the uncertainty
about how long this will last and what will happen next that leaves us
mourning our current losses as well as ones we haven't experienced yet.
(No Easter, no prom, and what if this means we can't go on summer
vacation?)

Ambiguous grief can leave us in a state of ongoing mourning, so it's
important for us to stay grounded in the present. Instead of futurizing
or catastrophizing --- ruminating about losses that haven't actually
happened yet (and may never happen) --- we can focus on the present by
adopting a concept I call
\href{https://www.theatlantic.com/family/archive/2020/03/a-therapists-guide-to-emotional-health-in-a-pandemic/608161/}{``both/and.''}
Both/and means that we can feel loss in the present and also feel safe
exactly where we are --- snuggled up with a good book, eating lunch with
our kids who are home from school, taking a walk with a family member,
and even celebrating a birthday via FaceTime.

We may have lost our sense of normalcy, but we can still stay present
for the ordinary right in front of us.

\textbf{3. Let people experience loss in their own way}

Although loss is universal, the ways in which we grieve are deeply
personal. For some, the loss of stability leads to a reckoning with
mortality, while for others, it leads to a rehaul of one's closet or
stress-baking.

In other words, there's no one-size-fits-all for grief. Even Elisabeth
Kübler-Ross's familiar stages of grieving --- denial, anger, bargaining,
depression, acceptance --- aren't meant to be linear. Everyone moves
through loss in a unique way, so it's important to let people do their
grieving in whatever way works for them without diminishing their losses
or pressuring them to grieve the way you are. A good rule of thumb: you
do you (and let others do them).

\begin{center}\rule{0.5\linewidth}{\linethickness}\end{center}

\href{https://lorigottlieb.com/}{\emph{Lori Gottlieb}} \emph{is a
therapist and the author of ``Maybe You Should Talk to Someone.''}

Read more

\hypertarget{advertisement}{%
\subsubsection{Advertisement}\label{advertisement}}

\protect\hyperlink{after-dfp-ad-mid1}{Continue reading the main story}

\includegraphics{https://static01.nyt.com/images/icons/t_logo_291_black.png}

March 23, 2020, 8:15 p.m. ET

March 23, 2020, 8:15 p.m. ET

By \href{https://www.nytimes.com/by/audra-d-s-burch}{Audra D. S. Burch}

\hypertarget{a-city-meeting-in-florida-grows-tense}{%
\subsection{\texorpdfstring{\protect\hyperlink{a-city-meeting-in-florida-grows-tense}{A
city meeting in Florida grows
tense.}}{A city meeting in Florida grows tense.}}\label{a-city-meeting-in-florida-grows-tense}}

\includegraphics{https://static01.nyt.com/images/2020/03/23/us/23virus-threats/23virus-threats-articleLarge.png?quality=75\&auto=webp\&disable=upscale}

HOLLYWOOD, Fla. --- The mayor of Lake Worth Beach, Fla., and a city
commissioner had an exchange last week, captured in a
two-minute-and-17-second
\href{https://www.youtube.com/watch?v=2bgCCXSrwHA\&t=59s}{video clip},
that provided a glimpse into the high-stakes tension facing local
government officials across the country amid the coronavirus outbreak.
Debates over shutdowns and containment measures and their economic
impact are boiling over as stakes rise.

The city commissioner, Omari Hardy, had watched the news as the
coronavirus transformed from a distant outbreak to a threat to local
lives. Mr. Hardy thought the city of about 38,000, which stretches seven
square miles not far from President Trump's Mar-a-Lago resort, needed to
act quickly to blunt the spread of the coronavirus and to protect the
city's most vulnerable.

He wanted officials to immediately ban large public gatherings, stop
shutting off delinquent electric and water accounts, establish more
protections for city workers and find out who had lawful emergency
powers.

Mr. Hardy said that he had tried to arrange a special commission meeting
for a week as cases in Florida multiplied, but that at every turn, his
requests were dismissed by the city manager, Michael Bornstein, who at
one point told him to ``calm down.''

So by the time the five-member commission met on Thursday, Mr. Hardy was
seething. And it would not be long before everybody knew it.

At the end of a fraught two-hour meeting, Mr. Hardy unleashed his
exasperation on Mayor Pam Triolo and Mr. Bornstein in a fiery speech
that would be alternatively characterized on social media as heroic and
disrespectful after it was
\href{https://www.youtube.com/watch?v=2bgCCXSrwHA\&t=59s}{posted online
last week} by The Palm Beach Post.

``This is a banana republic is what you're turning this place into with
your so-called leadership,'' Mr. Hardy shouted at Ms. Triolo in the
meeting, his voice booming. ``We should have been talking about this
last week. We cut off people's utilities this week and made them pay
what could have been their last check --- to us --- to turn their lights
on in a global health pandemic. But you don't care about that. You
didn't want to meet.''

Ms. Triolo did not sit idly by. She repeatedly slammed the gavel to
recess the meeting and quiet Mr. Hardy. And she yelled right back at
him.

Ms. Triolo, in her fourth term as mayor, suggested that Mr. Hardy's rant
was intentional grandstanding for attention --- he is running for a
Florida State House seat.

Mr. Bornstein acknowledged that service to dozens of residents had been
suspended earlier, but he said no utility disconnections had taken place
since a moratorium was announced on Wednesday, the day before the
explosive meeting. Those customers had already had their service
restored, he said, and fines had been reversed.

Read more

\href{https://www.nytimes.com/by/tiffany-hsu}{\includegraphics{https://static01.nyt.com/images/2018/12/06/multimedia/author-tiffany-hsu/author-tiffany-hsu-thumbLarge.png}}\href{https://www.nytimes.com/by/marc-tracy}{\includegraphics{https://static01.nyt.com/images/2018/02/20/multimedia/author-marc-tracy/author-marc-tracy-thumbLarge.jpg}}

March 23, 2020, 8:00 p.m. ET

March 23, 2020, 8:00 p.m. ET

By \href{https://www.nytimes.com/by/tiffany-hsu}{Tiffany Hsu} and
\href{https://www.nytimes.com/by/marc-tracy}{Marc Tracy}

\hypertarget{it-is-the-story-of-a-lifetime-and-its-crippling-some-publications}{%
\subsection{\texorpdfstring{\protect\hyperlink{it-is-the-story-of-a-lifetime-and-its-crippling-some-publications}{It
is the story of a lifetime, and it's crippling some
publications.}}{It is the story of a lifetime, and it's crippling some publications.}}\label{it-is-the-story-of-a-lifetime-and-its-crippling-some-publications}}

\includegraphics{https://static01.nyt.com/images/2020/03/23/business/00VIRUS-MEDIA-01/merlin_170848431_0024213b-e8b2-414a-be35-e67f5af535c1-articleLarge.jpg?quality=75\&auto=webp\&disable=upscale}

The coronavirus pandemic is one of the
\href{https://www.nytimes.com/2020/03/17/business/media/china-expels-american-journalists.html}{biggest
stories most publications will ever cover}. But it has left many of them
struggling to stay solvent.

Alternative weeklies and daily papers in small and midsize cities across
the United States were already suffering because of the recession last
decade, the migration of readers from print to online and the decline of
the advertising business. Since 2004, roughly one-fourth of American
newspapers --- more than 2,000 --- have been lost to mergers or
shutdowns, according to researchers at the University of North Carolina.
Most were weeklies.

The arrival of the coronavirus shook the industry's already weakened
economic foundation. As ad revenue and the money generated by events
sponsored by small publications started to evaporate, many papers have
canceled print editions, laid off workers or asked readers for
donations.

Among those affected:
\href{https://www.instagram.com/p/B95aZIbBRP3/}{Metro Weekly}, a
magazine about gay issues in Washington, D.C.;
\href{https://www.firsttouchonline.com/first-touch-to-suspend-print-publication-until-april/}{First
Touch}, a soccer publication in New York; and
\href{https://twitter.com/stevecarp56/status/1240307958015078405?s=20}{Gaming
Today}, a gambling newspaper in Las Vegas.

``One of the big problems with all of this is you don't know when this
is going to end,'' said Doyle Murphy, the editor in chief of Riverfront
Times. ``Even when people can go out of their houses again, it's going
to take a long time for business to come back to what it was.''

Read more

\href{https://www.nytimes.com/by/ellen-barry}{\includegraphics{https://static01.nyt.com/images/2018/10/08/multimedia/author-ellen-barry/author-ellen-barry-thumbLarge.png}}

March 23, 2020, 7:30 p.m. ET

March 23, 2020, 7:30 p.m. ET

By \href{https://www.nytimes.com/by/ellen-barry}{Ellen Barry}

\hypertarget{on-march-11-loretta-dionisio-became-a-data-point-this-is-her-story}{%
\subsection{\texorpdfstring{\protect\hyperlink{on-march-11-loretta-dionisio-became-a-data-point-this-is-her-story}{On
March 11, Loretta Dionisio became a data point. This is her
story.}}{On March 11, Loretta Dionisio became a data point. This is her story.}}\label{on-march-11-loretta-dionisio-became-a-data-point-this-is-her-story}}

\includegraphics{https://static01.nyt.com/images/2020/03/20/us/00virus-death04alt/00virus-death04alt-articleLarge.jpg?quality=75\&auto=webp\&disable=upscale}

Her name was Loretta, but they called her Lettie. She stood 4 feet 10
inches tall. She was outrageously friendly, the kind of person liable to
invite the sales clerk at T-Mobile to join the family for dinner. This
made her children cringe but was also something they loved. Pure Lettie.

She was tough. At work, she could stare down colleagues who were hairy,
blustery and taller than her by a foot or two. And it was true of her
husband, Roddy. He could not say no to her.

Roddy had not wanted to go on their February trip to the Philippines. He
was watching the early news about the coronavirus, and worried it would
put his wife, a cancer survivor, in danger. But she was adamant. There
was something she needed to finish.

On March 11, Loretta Dionisio became a data point.

In the ongoing tally of fatalities associated with the coronavirus, hers
was the 37th death in the United States, the first in Los Angeles
County.

After she tested positive for the virus, the family was occupied with
crisis management, five or six hours a day of phone calls to public
health officials, the crematory, hospital staff. Not only their father,
but also their aunt and uncle, and another aunt and cousin, have been
ordered to self-quarantine.

A memorial gathering, for now, is out of the question.

``We don't want to put any other family members in harm's way,'' Ms.
Dionisio's son, Rembert, said. ``That's what makes everything really
rough right now. It's almost taken away from what is happening with my
mother.''

Sarah Mervosh contributed reporting from New York, Amy Qin from Beijing
and Jason Horowitz from Rome. Kitty Bennett contributed research from
New York.

Read more

\hypertarget{advertisement-1}{%
\subsubsection{Advertisement}\label{advertisement-1}}

\protect\hyperlink{after-dfp-ad-mid2}{Continue reading the main story}

\href{https://www.nytimes.com/by/stacy-cowley}{\includegraphics{https://static01.nyt.com/images/2018/10/03/multimedia/author-stacy-cowley/author-stacy-cowley-thumbLarge.png}}\href{https://www.nytimes.com/by/tiffany-hsu}{\includegraphics{https://static01.nyt.com/images/2018/12/06/multimedia/author-tiffany-hsu/author-tiffany-hsu-thumbLarge.png}}

March 23, 2020, 7:00 p.m. ET

March 23, 2020, 7:00 p.m. ET

By \href{https://www.nytimes.com/by/stacy-cowley}{Stacy Cowley} and
\href{https://www.nytimes.com/by/tiffany-hsu}{Tiffany Hsu}

\hypertarget{small-businesses-seek-a-lifeline-beyond-loans}{%
\subsection{\texorpdfstring{\protect\hyperlink{small-businesses-seek-a-lifeline-beyond-loans}{Small
businesses seek a lifeline beyond
loans.}}{Small businesses seek a lifeline beyond loans.}}\label{small-businesses-seek-a-lifeline-beyond-loans}}

\includegraphics{https://static01.nyt.com/images/2020/03/27/world/27virus-lives-biz-sub/merlin_170788977_1388468e-3abf-408e-a5f5-df96d76fe483-articleLarge.jpg?quality=75\&auto=webp\&disable=upscale}

The Federal Reserve on Monday resurrected an asset-purchase facility
from the 2008 financial crisis intended to encourage banks and
financiers to make loans to small businesses and households. It also
plans to announce a new Main Street Business Lending Program designed to
support lending to ``eligible small and medium-sized businesses'' but
offered few details.

A disaster loan program is already up and running. Congress
\href{https://www.nytimes.com/2020/03/04/us/politics/coronavirus-emergency-aid-congress.html}{authorized
up to \$7 billion early this month} for
\href{https://www.sba.gov/page/coronavirus-covid-19-small-business-guidance-loan-resources\#section-header-0}{small
business disaster loans} through the Small Business Administration.
Unlike the agency's flagship loans, which are made by banks, disaster
loans are issued directly by the government.

But the main federal lifeline offered so far --- low-interest disaster
loans --- is unappealing to many small business owners running on thin
margins and leery of taking on debt they can't afford to repay as
they've been forced to close and lay off employees.

``All we do is make enough money to make it through the off-season,''
said Donna Benefiel, who owns the Sunset Produce Market in Banks, Ore.,
and a grocery store on the Oregon coast. ``We're not that profitable. We
don't have any reserves. How do we borrow a year's worth of money and
then have to pay it back?''

Borrowers who own their homes often risk losing the property if they
can't repay what they borrowed. Terms like that spook business owners,
especially now, when there is little clarity around when and how the
coronavirus pandemic will subside, and whether mom-and-pop shops will
ever recover.

The Trump administration and lawmakers have discussed plans for
\href{https://www.nytimes.com/2020/03/18/business/bailout-economy-coronavirus.html}{a
bailout that could top \$2 trillion}, including direct payments to
individuals and aid for battered industries like the airlines. A memo
circulated on Wednesday by the Treasury Department proposed \$300
billion for small business ``interruption'' loans.

That would be a vastly larger program than anything the government has
previously run. Last year, the Small Business Administration backed \$28
billion in loans issued by banks; its disaster program lent out just
over \$2 billion.

The agency is used to ramping up quickly to disburse loans after natural
disasters like floods and earthquakes --- after Hurricane Harvey in
2017, it processed most loan applications in less than three weeks ---
but its track record with large economic disasters is troubled.

In the aftermath of the 2008 financial crisis, Congress ordered the
S.B.A. to partner with banks on zero-interest loans of up to \$35,000 to
``viable'' small companies hurt by the recession. The program was laden
with complex rules, and fewer than 9,000 companies took the loans.
Nearly half of the applications approved
\href{https://boss.blogs.nytimes.com/2011/05/02/a-r-c-loans-gone-but-hardly-forgotten/}{did
not meet all of the agency's rules}, auditors estimated.

And many vulnerable businesses cannot afford to wait weeks for a cash
infusion. The median small company takes in \$381 a day and spends
\$374, a 2016
\href{http://jpmorganchase.com/corporate/institute/document/jpmc-institute-small-business-report.pdf}{analysis}
by the JPMorgan Chase Institute found. The typical business has enough
savings to survive just 27 days.

Read more

\href{https://www.nytimes.com/by/jodi-kantor}{\includegraphics{https://static01.nyt.com/images/2018/02/16/multimedia/author-jodi-kantor/author-jodi-kantor-thumbLarge-v2.png}}

March 23, 2020, 6:30 p.m. ET

March 23, 2020, 6:30 p.m. ET

By \href{https://www.nytimes.com/by/jodi-kantor}{Jodi Kantor}

\hypertarget{remote-locations-worry-about-a-rash-of-crisis-tourists}{%
\subsection{\texorpdfstring{\protect\hyperlink{remote-locations-worry-about-a-rash-of-crisis-tourists}{Remote
locations worry about a rash of `crisis
tourists.'}}{Remote locations worry about a rash of `crisis tourists.'}}\label{remote-locations-worry-about-a-rash-of-crisis-tourists}}

Khara Tapay Jabola-Carolus was recently at a Target in Honolulu and
noticed an influx of ``crisis tourists,'' people who've traveled from
the contiguous United States to Hawaii seeking more isolation amid the
coronavirus pandemic.

``We have a weak social safety net and an economy overly dependent on
tourism,'' she wrote to The New York Times's Dilemmas column. ``These
outsiders could push us over the edge, especially because tourists are
often prioritized at the expense of local residents.''

It's an instant national ethical dilemma, exacerbating already-tense
relationships between rich and poor, urban and rural, and, in the case
of Hawaii, largely white outsiders and more diverse locals. Who gets to
shelter where? Or take the last sack of flour at a small supermarket?

Destinations known for welcoming visitors are now closing themselves
off. After Ms. Jabola-Carolus wrote, Hawaii announced a mandatory 14-day
quarantine for all incoming travelers. Southeast Utah has
\href{https://www.nytimes.com/2020/03/21/travel/coronavirus-tourists-conflict.html}{prohibited}
lodging for nonessential visitors, and Colorado has announced it doesn't
want tourists either. The Outer Banks of North Carolina are shut to
nonresidents. The Maine island of North Haven went even further,
\href{https://www.pressherald.com/2020/03/16/north-haven-votes-to-keep-nonresidents-off-island-out-of-coronavirus-fears/?fbclid=IwAR2NnA5B8jgufxiPK5OHQSofOX0VrNB_GZo7zs9QBO0svs9I4g3ojUt3svw}{barring
all visitors}, including seasonal residents.

History shows that may be the correct call. During the 1918 influenza
pandemic, Gunnison, Colo., erected barricades over its highways
(``against the world,'' in the words of a
\href{http://chm.med.umich.edu/wp-content/uploads/sites/20/2015/02/gnc05.pdf}{county
physician}) and quarantined anyone who entered. Neighboring towns were
decimated, but Gunnison's losses were low.
\href{http://chm.med.umich.edu/research/1918-influenza-escape-communities/yerba-buena/}{Yerba
Buena}, an island in San Francisco Bay with a military base and a
population of 5,000, locked down for two months with similar results.

\href{https://www.nytimes.com/interactive/2020/03/20/us/coronavirus-model-us-outbreak.html?action=click\&module=Spotlight\&pgtype=Homepage}{Projections
of the virus's spread} show the brutal truth: Fellow city dwellers, we
pose a threat to everyone else.

But these remote locations are already flooded with visitors who may not
be going home for months. Some sort of compact will be necessary,
according to interviews with officials in rural and resort towns,
second-home owners and coronavirus expatriates.

So, before relocating, consider whether farther truly equals safer,
especially if you'll be far from the kind of vast medical corps found in
major cities, as well as friends and neighbors to count on in an
emergency.

\begin{center}\rule{0.5\linewidth}{\linethickness}\end{center}

Grace Ashford contributed research.

\emph{The Times's new Dilemmas column offers guidance on how to navigate
daily life during the coronavirus. Whether or not the virus has reached
your neighborhood, what's swept into everyone's lives is a set of
confounding dilemmas. So send yours to}
\href{mailto:dilemmas@nytimes.com}{\emph{dilemmas@nytimes.com}}\emph{.
There's a saying in journalism: The solution is always more reporting.
And that's what we'll do here.}

Read more

\href{https://www.nytimes.com/by/elisabeth-vincentelli}{\includegraphics{https://static01.nyt.com/images/2018/12/10/multimedia/author-elisabeth-vincentelli/author-elisabeth-vincentelli-thumbLarge.png}}

March 23, 2020, 6:00 p.m. ET

March 23, 2020, 6:00 p.m. ET

By \href{https://www.nytimes.com/by/elisabeth-vincentelli}{Elisabeth
Vincentelli}

\hypertarget{rosie-odonnell-hosts-a-fund-raiser-with-broadways-stars}{%
\subsection{\texorpdfstring{\protect\hyperlink{rosie-odonnell-hosts-a-fund-raiser-with-broadways-stars}{Rosie
O'Donnell hosts a fund-raiser with Broadway's
stars.}}{Rosie O'Donnell hosts a fund-raiser with Broadway's stars.}}\label{rosie-odonnell-hosts-a-fund-raiser-with-broadways-stars}}

\includegraphics{https://static01.nyt.com/images/2020/03/23/arts/23ROSIE/23ROSIE-articleLarge-v2.jpg?quality=75\&auto=webp\&disable=upscale}

A lot happened during the just-this-once edition of ``The Rosie
O'Donnell Show'' that streamed live Sunday night --- that would be
expected with an event lasting three and a half hours. But it will be
hard to top
\href{https://www.nytimes.com/2019/10/30/theater/adrienne-warren-tina-turner-broadway-musical.html?searchResultPosition=2}{Adrienne
Warren,} the star of
\href{https://www.nytimes.com/2019/11/07/theater/tina-turner-musical-review.html?searchResultPosition=1}{``Tina:
The Tina Turner Musical,''} singing ``Simply the Best'' in a bathtub,
clad in a bathing suit worthy of a 1970s Bond girl. At the end of the
song, she whipped out a tiny toy saxophone.

What made this surreal scene even more memorable was that by then a
beaming Warren had already attempted the song twice, each time foiled by
her audio feed. As a running gag, it was sheer perfection, and the
obvious accidental nature just made it more endearing.

This was not the only time technical difficulties hampered the online
``The Rosie O'Donnell Show,'' in which the participants --- the vast
majority of them from the Broadway or Broadway-adjacent community ---
called in from their homes. But these glitches only added to the D.I.Y.
charm of the evening, a benefit for the
\href{https://actorsfund.org/}{Actors Fund} (an organization that
supports a wide range of professionals in film, theater, television,
music, opera, radio and dance). It felt like a hybrid talk show, Jerry
Lewis telethon and the entertainment the United Service Organizations
provided during World War II.

Hosting ``from the comfort of my garage-slash-art studio'' and wearing a
``Hamilton'' hoodie, O'Donnell eased back into the role she held on
daytime from 1996 to 2002, years during which she established herself as
one of Broadway's most devoted superfans. The community repaid the
affection by turning up in droves for the virtual shindig, most of them
live and some on video message. The star-studded lineup --- first names
are not necessary with the likes of Chenoweth, Benanti, LuPone, Menzel,
McDonald, Fierstein, Salonga --- included performers who, in normal
times, could rarely be free on the same evening. This sudden
availability was a bittersweet reminder that these are not normal times.

Read more

\hypertarget{advertisement-2}{%
\subsubsection{Advertisement}\label{advertisement-2}}

\protect\hyperlink{after-dfp-ad-mid3}{Continue reading the main story}

\includegraphics{https://static01.nyt.com/images/icons/t_logo_291_black.png}\href{https://www.nytimes.com/by/abby-goodnough}{\includegraphics{https://static01.nyt.com/images/2018/06/14/multimedia/author-abby-goodnough/author-abby-goodnough-thumbLarge-v2.png}}

March 23, 2020, 5:30 p.m. ET

March 23, 2020, 5:30 p.m. ET

By Penina Krieger and
\href{https://www.nytimes.com/by/abby-goodnough}{Abby Goodnough}

\hypertarget{medical-students-largely-sidelined-are-finding-creative-ways-to-help-out}{%
\subsection{\texorpdfstring{\protect\hyperlink{medical-students-largely-sidelined-are-finding-creative-ways-to-help-out}{Medical
students, largely sidelined, are finding creative ways to help
out.}}{Medical students, largely sidelined, are finding creative ways to help out.}}\label{medical-students-largely-sidelined-are-finding-creative-ways-to-help-out}}

Image

Georgetown University School of Medicine students volunteering at the
Capitol Area Food Bank in Washington on Thursday after their rounds were
canceled.Credit...Carolyn Kaster/Associated Press

As hospitals around the United States brace for an ongoing surge in
coronavirus cases, one question they are grappling with is whether
medical students should be deployed to help care for patients infected
with the virus.

For now, the nation's 90,000 medical students have been largely
sidelined from patient care during the crisis. The reasoning is that
sending medical students home helps conserve scarce personal protective
equipment --- including masks, gloves and gowns. It also gives schools
time to educate students on Covid-19 should the students eventually be
needed for patient care.

Disappointed by the abrupt halt to their training, medical students
around the country have responded with grass-roots efforts to secure
masks, staff patient call centers and even provide child care for
beleaguered doctors.

``I thought I'd be learning to listen to the heart and lungs or conduct
an outpatient interview, but that's not what is needed right now,'' said
Elyse Berlinberg, a second-year medical student at N.Y.U. ``Part of the
role of being a physician is being part of the community and knowing
their needs and responding to them. I think the service we are doing now
is part of forming our professional identity.''

Read more

\includegraphics{https://static01.nyt.com/images/icons/t_logo_291_black.png}

March 23, 2020, 5:00 p.m. ET

March 23, 2020, 5:00 p.m. ET

By The New York Times

\hypertarget{missing-the-fine-arts-weve-got-you-covered}{%
\subsection{\texorpdfstring{\protect\hyperlink{missing-the-fine-arts-weve-got-you-covered}{Missing
the fine arts? We've got you
covered.}}{Missing the fine arts? We've got you covered.}}\label{missing-the-fine-arts-weve-got-you-covered}}

Image

The Solomon R. Guggenheim Museum is offering a virtual 360-degree tour
of its spiraling rotunda.

Patrick Stewart reads Shakespeare on Twitter, Ballet Hispánico is on
Instagram and art galleries are expanding online offerings.

If you're stuck at home and hankering for the fine arts, there's plenty
online. Since the coronavirus pandemic began
\href{https://www.nytimes.com/2020/03/12/theater/coronavirus-broadway-shutdown.html}{temporarily}
\href{https://www.nytimes.com/2020/03/12/arts/ny-events-cancellations-coronavirus.html}{shutting
down performing arts venues} and museums around the world, cultural
organizations have been finding ways to share their work digitally.
Performances are being live-streamed, archival material is being
resurfaced and social media platforms like Instagram, YouTube and
Facebook are serving as makeshift stages, concert halls and gallery
spaces.

Here's a list of some of what's streaming and otherwise available on the
internet. The offerings are increasing by the day, so be sure to check
in with your favorite arts institutions to see what they're providing as
things develop. And check back here for updates.

Read more

\href{https://www.nytimes.com/by/victor-mather}{\includegraphics{https://static01.nyt.com/images/2018/02/20/multimedia/author-victor-mather/author-victor-mather-thumbLarge.jpg}}

March 23, 2020, 4:30 p.m. ET

March 23, 2020, 4:30 p.m. ET

By \href{https://www.nytimes.com/by/victor-mather}{Victor Mather}

\hypertarget{you-cant-watch-giannis-antetokounmpo-hoop-but-you-can-watch-him-strum}{%
\subsection{\texorpdfstring{\protect\hyperlink{you-cant-watch-giannis-antetokounmpo-hoop-but-you-can-watch-him-strum}{You
can't watch Giannis Antetokounmpo hoop, but you can watch him
strum.}}{You can't watch Giannis Antetokounmpo hoop, but you can watch him strum.}}\label{you-cant-watch-giannis-antetokounmpo-hoop-but-you-can-watch-him-strum}}

During the 1918 flu pandemic, we had no way of knowing what Babe Ruth or
Jack Dempsey were up to in their spare time. Thanks to social media, we
know that today's athletes are just like us: Really bored.

\begin{quote}
This ladies and gentlemen is why we all need basketball back...
\href{https://twitter.com/Giannis_An34?ref_src=twsrc\%5Etfw}{@Giannis\_An34}
🎸😂 \href{https://t.co/9F4aJIOIiW}{pic.twitter.com/9F4aJIOIiW}

--- Mariah Danae (@mariahdanae15)
\href{https://twitter.com/mariahdanae15/status/1239003226139250693?ref_src=twsrc\%5Etfw}{March
15, 2020}
\end{quote}

Giannis Antetokounmpo is clearly not a professional guitarist, but the
Milwaukee Bucks superstar managed to
\href{https://twitter.com/mariahdanae15/status/1239003226139250693}{pluck
out ``Smoke on the Water}'' to pass the time. Also, LeBron James
\href{https://www.instagram.com/p/B-C6OQEg2j0/}{did a TikTok dance}. He
certainly can't play basketball in Hudson County, N.J.,
\href{https://www.nj.com/hudson/2020/03/hudson-county-to-remove-rims-on-basketball-hoops-to-slow-coronavirus-spread.html}{where
they are removing all the rims}. Those are just two examples of how
sports starts are killing time.

\hypertarget{advertisement-3}{%
\subsubsection{Advertisement}\label{advertisement-3}}

\protect\hyperlink{after-dfp-ad-mid4}{Continue reading the main story}

\href{https://www.nytimes.com/by/jodi-kantor}{\includegraphics{https://static01.nyt.com/images/2018/02/16/multimedia/author-jodi-kantor/author-jodi-kantor-thumbLarge-v2.png}}

March 23, 2020, 4:00 p.m. ET

March 23, 2020, 4:00 p.m. ET

By \href{https://www.nytimes.com/by/jodi-kantor}{Jodi Kantor}

\hypertarget{a-close-family-copes-with-social-distancing-after-their-mothers-cancer-diagnosis}{%
\subsection{\texorpdfstring{\protect\hyperlink{a-close-family-copes-with-social-distancing-after-their-mothers-cancer-diagnosis}{A
close family copes with social distancing after their mother's cancer
diagnosis.}}{A close family copes with social distancing after their mother's cancer diagnosis.}}\label{a-close-family-copes-with-social-distancing-after-their-mothers-cancer-diagnosis}}

The Wilkinson family --- two parents and four teenage-to-20-something
children --- all live in the same house in California. They're so close
that, ordinarily, their idea of fun is to all pile onto Mom and Dad's
bed to watch ``The Bachelor*.''*

But in late January, Diana, a 56-year-old math teacher, learned she had
a rare, aggressive form of endometrial cancer and a 50 percent chance of
survival. Her husband and children have been told to keep their
distance, because chemotherapy has left her vulnerable to infection. The
family is especially jittery because one daughter recently returned from
Italy after her study abroad program skidded to a halt. Another is a
U.S. Marine who spends weekdays on a base where others have tested
positive for the novel coronavirus. Meagan Wilkerson has become the
enforcer, telling her mother to separate herself. Mrs. Wilkinson feels
hurt. Meagan feels guilty.

So the Wilkinsons are living a more acute version of our collective
dilemma: longing for connections they once took for granted, terrified
of making mistakes and unsure how to get through the coming months.

**``**I break the rules sometimes,'' Mrs. Wilkinson confessed, choking
back tears. ```Just put on some clean clothes and lie down next to
me,''' she said she tells her children when she can no longer bear being
several feet away.

Image

The Wilkinsons are living a more acute version of America's collective
dilemma, trying to abide by seemingly impossible new rules.Credit...Anna
Wilkinson

To seek help for the Wilkinsons, and the rest of us, I turned to
authorities on how to cope with
\href{https://www.nytimes.com/2020/03/19/well/live/coronavirus-quarantine-social-distancing.html}{social
distancing}: therapists who advise cancer patients and their families,
including during treatments like stem-cell transplants, which can
involve prolonged isolation and other measures to guard against
infection.

Ian Sadler, a psychologist at Columbia University Medical Center, gently
dismissed Meagan Wilkinson's fear that she would damage her mother by
not hugging her. Distancing can be an act of care, he said. A lack of
embrace is now an embrace.

Allison Applebaum, a psychologist at Memorial Sloan Kettering Cancer
Center in Manhattan, worried about the burden on the Wilkinson children.
Maintaining complete distance and cleanliness, monitoring contact and
food, is ``an enormous source of anxiety,'' she said, for them and now
everyone else. ``You can't do it perfectly a hundred percent of the
time.'' Note to everyone who is wiping, washing and Clorox-ing these
days: in some cancer caregivers, the responsibility of constantly trying
to eliminate germs contributes to post-traumatic stress symptoms, she
said.

To find a substitute for cuddling, Dr. Applebaum suggested an exercise
she does with her patients. If a father longs to play ball outside with
his son, she asks him: ``What was it about playing ball in the backyard
that was meaningful? Was it about the ball, or was it connecting with
him?'' The goal is to find an alternate activity that delivers similar
satisfaction.

\begin{center}\rule{0.5\linewidth}{\linethickness}\end{center}

Grace Ashford contributed research.

\emph{The Times's new Dilemmas column offers guidance on how to navigate
daily life during the coronavirus. Whether or not the virus has reached
your neighborhood, what's swept into everyone's lives is a set of
confounding dilemmas. So send yours to}
\href{mailto:dilemmas@nytimes.com}{\emph{dilemmas@nytimes.com}}\emph{.
There's a saying in journalism: The solution is always more reporting.
And that's what we'll do here.}

Read more

\includegraphics{https://static01.nyt.com/images/icons/t_logo_291_black.png}

March 23, 2020, 3:30 p.m. ET

March 23, 2020, 3:30 p.m. ET

By \href{https://www.nytimes.com/by/mike-baker}{Mike Baker}

\hypertarget{a-group-challenges-washingtons-plans-for-disabled-people-with-the-coronavirus}{%
\subsection{\texorpdfstring{\protect\hyperlink{a-group-challenges-washingtons-plans-for-disabled-people-with-the-coronavirus}{A
group challenges Washington's plans for disabled people with the
coronavirus.}}{A group challenges Washington's plans for disabled people with the coronavirus.}}\label{a-group-challenges-washingtons-plans-for-disabled-people-with-the-coronavirus}}

\includegraphics{https://static01.nyt.com/images/2020/03/27/world/27virus-lives-washington-sub/merlin_170822835_851b2000-406a-4538-a400-5db0441c8c7c-articleLarge.jpg?quality=75\&auto=webp\&disable=upscale}

In what could prove to be a hot topic nationwide, groups representing
people with disabilities on Monday challenged
\href{https://www.nytimes.com/2020/03/20/us/coronavirus-in-seattle-washington-state.html}{a
plan that would guide hospitals in Washington State} dealing with the
coronavirus in the event that they do not have enough lifesaving
resources for all the patients who need them.

The triage care plan could result in end-of-life decisions that
disadvantage those with disabilities, said David Carlson, the director
of advocacy at Disability Rights Washington. The group's complaint calls
for the federal government to quickly intervene to investigate, issue
findings and make sure that doctors and hospitals do not discriminate
against people with disabilities when making treatment decisions.

``Washington's rationing scheme places the lives of disabled people at
serious risk,'' the advocacy groups wrote in the complaint to the U.S.
Department of Health and Human Service's Office of Civil Rights.

The complaint in Washington State was the first to be filed, but
advocates for the disabled said they expect to scrutinize similar triage
plans around the country to see if they provide equal access to
lifesaving care to people with disabilities.

The federal government ``has a very brief moment to intercede,'' the
complaint said. If it does not, it said, ``there will be no way to undo
the lethal outcome of the discriminatory plans that have been formulated
without O.C.R.'s guidance.''

Read more

\includegraphics{https://static01.nyt.com/images/icons/t_logo_291_black.png}

March 23, 2020, 3:00 p.m. ET

March 23, 2020, 3:00 p.m. ET

By \href{https://www.nytimes.com/by/guy-trebay}{Guy Trebay}

\hypertarget{doormen-are-on-the-front-lines-of-new-york-citys-crisis}{%
\subsection{\texorpdfstring{\protect\hyperlink{doormen-are-on-the-front-lines-of-new-york-citys-crisis}{Doormen
are on the front lines of New York City's
crisis.}}{Doormen are on the front lines of New York City's crisis.}}\label{doormen-are-on-the-front-lines-of-new-york-citys-crisis}}

\includegraphics{https://static01.nyt.com/images/2020/03/27/world/27virus-lives-doormen-sub/merlin_170705256_2fad5c2e-8260-4e88-8a08-72f9c9e105fd-articleLarge.jpg?quality=75\&auto=webp\&disable=upscale}

They are the white-gloved sentries, standing guard at
\href{https://www.nytimes.com/2020/03/23/nyregion/coronavirus-new-york-update.html\#link-6b4a2a81}{New
York City}'s better addresses, so signature a feature of life here that
we tend to forget that in other metropolitan centers uniformed
\href{https://www.nytimes.com/2020/03/23/style/doormen-coronavirus.html}{doormen}
(and, less often, women) barely exist.

They form a small army, some 35,000 residential doormen, concierges,
porters, handymen and supers represented by a powerful union, 32BJ. They
open doors, of course, load cars, receive packages, pass dogs off to
professional walkers and in general make themselves indispensable to an
ease of life many in this demanding town take for granted.

Yet as protectors of the border between public and private,
\href{https://www.nytimes.com/2020/03/23/style/doormen-coronavirus.html}{doormen
play a role crucial to the currents of the metropolis}, one never more
evident than now when
\href{https://www.nytimes.com/2020/03/22/nyregion/Coronavirus-new-York-epicenter.html}{the
front line of a global pandemic is that threshold}.

``Yes, it's a job, but we also try to keep the building as a home,''
said Alberto Ventura, 65, who has worked the door at the same Park
Avenue building for 42 years. ``With the virus, we're trying to take it
a day at a time and be as calm as we can.''

It has become the job of those in his line of work not just to swab
elevator buttons with Lysol five times daily, but also to restore the
psychic equilibrium of those to whom bad things are not supposed to
happen.

At increased risk to themselves, the staffs at most high-end buildings
throughout the five boroughs find themselves scrambling to institute
hygienic measures. They are also enforcing daily changing guidelines,
establishing ad hoc networks of notification, caring for the old and
vulnerable left behind by an exodus that has rendered the Upper East
Side a ghost town.

``We're not a hazmat crew but we're doing what we can,'' Jimmy Brennan,
40, the resident manager of a cooperative building on Fifth Avenue, said
of his nine-member team. ``The trick here is anticipating the needs and
problems before they come up.''

At Mr. Brennan's building in the East 70s, that meant appealing to his
board of directors --- well before the Centers for Disease Control and
Prevention issued stringent updated safety guidelines --- for a go-ahead
to shut down the gym and common areas, to suspend nonemergency
contractors, to seal off floors left vacant by tenants who fled the city
and to establish a phone tree and daily wellness check-ins with those
that remained.

``Our population is mostly aged 60 to 100, 85 is a pretty common number
here,'' said Mr. Brennan, a third-generation building manager whose
extended family oversees 30 separate buildings around New York.
``History will judge whether it was better for us to be proactive than
reactive. But, for now, I'd rather be effective than popular.''

Read more

\hypertarget{advertisement-4}{%
\subsubsection{Advertisement}\label{advertisement-4}}

\protect\hyperlink{after-dfp-ad-mid5}{Continue reading the main story}

\includegraphics{https://static01.nyt.com/images/icons/t_logo_291_black.png}

March 23, 2020, 2:30 p.m. ET

March 23, 2020, 2:30 p.m. ET

By \href{https://www.nytimes.com/by/emma-goldberg}{Emma Goldberg}

\hypertarget{coronavirus-is-a-unique-challenge-for-doctors-who-specialize-in-other-things}{%
\subsection{\texorpdfstring{\protect\hyperlink{coronavirus-is-a-unique-challenge-for-doctors-who-specialize-in-other-things}{Coronavirus
is a unique challenge for doctors who specialize in other
things.}}{Coronavirus is a unique challenge for doctors who specialize in other things.}}\label{coronavirus-is-a-unique-challenge-for-doctors-who-specialize-in-other-things}}

\includegraphics{https://static01.nyt.com/images/2020/03/27/world/27virus-lives-docs-sub/merlin_170846328_79110f41-96e4-4ea0-ba75-d5167e539176-articleLarge.jpg?quality=75\&auto=webp\&disable=upscale}

The coronavirus has created an unusual situation for many medical
specialists who serve as the primary physicians of patients with
particular medical needs. Physicians across every field who are trained
to care for very specific medical problems are confronting a surge of
patient questions and they are scrambling to keep up with rapid changes
in case numbers and advisories from governments and health agencies.

``We're hearing a lot of anxieties from specialists who don't know what
the right thing to do is for their patients,'' said Dr. Megan Ranney, an
emergency physician in Rhode Island. ``Dermatologists, ophthalmologists,
we're even hearing from dentists.''

Discussions with some of those doctors reveals a changing world for
doctors of all specialties.

\href{https://www.nytimes.com/by/anahad-oconnor}{\includegraphics{https://static01.nyt.com/images/2018/10/17/multimedia/author-anahad-oconnor/author-anahad-oconnor-thumbLarge.png}}

March 23, 2020, 2:00 p.m. ET

March 23, 2020, 2:00 p.m. ET

By \href{https://www.nytimes.com/by/anahad-oconnor}{Anahad O'Connor}

\hypertarget{sales-of-vitamins-rise-as-shoppers-look-to-fortify-their-immune-systems}{%
\subsection{\texorpdfstring{\protect\hyperlink{sales-of-vitamins-rise-as-shoppers-look-to-fortify-their-immune-systems}{Sales
of vitamins rise as shoppers look to fortify their immune
systems.}}{Sales of vitamins rise as shoppers look to fortify their immune systems.}}\label{sales-of-vitamins-rise-as-shoppers-look-to-fortify-their-immune-systems}}

\includegraphics{https://static01.nyt.com/images/2020/03/27/world/27virus-lives-vitamins-sub/merlin_170471772_a835ad4d-7405-473b-bafc-63ee0293f5a0-articleLarge.jpg?quality=75\&auto=webp\&disable=upscale}

Dietary supplement sales have surged nationwide as panicked consumers
stock up on vitamins, herbs, extracts, and cold and flu remedies. None
of these products have been shown to lower the likelihood of contracting
the coronavirus or shortening its course, and taking large doses of them
can potentially do harm. But experts say that the jump in sales suggests
many people are desperate to strengthen their body's immune defenses and
ease their heightened anxiety levels.

There are times when taking a supplement can be very useful, like during
pregnancy or to address a clear nutrient deficiency. But for healthy
adults who are worried about the coronavirus, eating a nutritious diet
and getting proper sleep and exercise are the best ways to strengthen
your immune system, said Linda Van Horn, chief of nutrition in the
department of preventive medicine at the Northwestern University
Feinberg School of Medicine.

Whole foods like fruits, vegetables, fish, poultry, nuts, legumes and
milk contain a wide range of vitamins, minerals and phytochemicals ---
including zinc and vitamin D --- that work in synergy to protect your
health.

\href{https://www.nytimes.com/by/jessica-grose}{\includegraphics{https://static01.nyt.com/images/2019/05/06/multimedia/00-headshot-test-Jess/00-headshot-test-Jess-thumbLarge-v5.png}}

March 23, 2020, 1:30 p.m. ET

March 23, 2020, 1:30 p.m. ET

By \href{https://www.nytimes.com/by/jessica-grose}{Jessica Grose}

\hypertarget{throw-away-one-worry-this-week}{%
\subsection{\texorpdfstring{\protect\hyperlink{throw-away-one-worry-this-week}{Throw
away one worry this
week.}}{Throw away one worry this week.}}\label{throw-away-one-worry-this-week}}

Yesterday I opened my cupboard and stared at a pair of garish, orange
plastic cups. I use these cups to serve my children milk, and as
recently as last week, I would look at them and think: I really should
throw these out. I don't even remember where they came from, and they're
probably leeching all sorts of unpronounceable toxins into my kids'
bloodstreams every time they take a sip --- which
\href{https://parenting.nytimes.com/childrens-health/plastics-to-avoid}{I
know from our own coverage of plastics}!

But when I look at them now, nested cheerfully on the shelf, when we're
barely leaving the house as the coronavirus whips around the country, I
think: I don't give a rat's patoot about these cups anymore. And, in
that spirit, I invite you all to take one minor thing you used to worry
about and throw it out the window for the next several months. If you
can take anything off your considerable mental loads right now, please
do so.

This week, Parenting has stories about
\href{https://www.nytimes.com/2020/03/17/parenting/seattle-child-care.html?type=roundup\&link=intro}{how
folks in Seattle are managing work with their children} at home (the
answer is: it's a nightmare);
\href{https://www.nytimes.com/2020/03/18/parenting/coronavirus-kids-events-cancelled.html?type=roundup\&link=intro}{how
to handle your kids' disappointment} about the changes in their lives;
\href{https://www.nytimes.com/2020/03/20/parenting/coronavirus-work-from-home-spouse.html?type=roundup\&link=cta}{how
to work from home with your partner without losing it}; and a bright
spot:
\href{https://www.nytimes.com/2020/03/20/parenting/coronavirus-divorce-coparenting.html?type=roundup\&link=intro}{an
essay from Hanna Ingber,} an editor at The Times, about how her
co-parenting relationship with her ex has improved during this pandemic.

If you're a divorced or separated parent,
\href{mailto:parenting_submissions@nytimes.com?subject=Newsletter\%20Suggestions}{Parenting
wants to hear from you} about your experiences co-parenting during the
coronavirus.

Read more

\hypertarget{advertisement-5}{%
\subsubsection{Advertisement}\label{advertisement-5}}

\protect\hyperlink{after-dfp-ad-mid6}{Continue reading the main story}

\href{https://www.nytimes.com/by/c-j-hughes}{\includegraphics{https://static01.nyt.com/images/2018/12/05/multimedia/author-c-j-hughes/author-c-j-hughes-thumbLarge.png}}

March 23, 2020, 1:00 p.m. ET

March 23, 2020, 1:00 p.m. ET

By \href{https://www.nytimes.com/by/c-j-hughes}{C. J. Hughes}

\hypertarget{home-buying-and-mortgage-refinancing-plans-face-complications}{%
\subsection{\texorpdfstring{\protect\hyperlink{home-buying-and-mortgage-refinancing-plans-face-complications}{Home
buying and mortgage refinancing plans face
complications.}}{Home buying and mortgage refinancing plans face complications.}}\label{home-buying-and-mortgage-refinancing-plans-face-complications}}

\includegraphics{https://static01.nyt.com/images/2020/03/29/us/21-live-county-records/21-live-county-records-articleLarge-v2.jpg?quality=75\&auto=webp\&disable=upscale}

The last few weeks saw a surge in mortgage applications, especially from
borrowers seeking to refinance in the face of low interest rates, and
many of those loan closings are scheduled now.

But the coronavirus outbreak has snarled many spring home-buying plans,
as apartment buildings ban open houses, real estate agents shutter
brokerages and quarantines make it tough to even step outside.

Now, another obstacle: The closure of government recording offices, as
all nonessential employees in New York and other states have been told
to stay home.

Squirreled away in county buildings, and probably not high on the list
of things buyers care about, these offices are nevertheless vital to the
buying and refinancing processes, as title searches and deed filings
happen inside. Most lenders require a title search for refinancing.

``The machine is being overwhelmed at this point,'' said Bob Jennings,
the chief executive of ClosingCorp, a tech platform involved in a third
of the country's home-loan applications.

All told, as of Friday, about 1,000 of the country's 3,600 recording
offices had shut down or curtailed their hours, according to the
American Land Title Association, a trade group crowdsourcing
\href{https://www.alta.org/business-tools/coronavirus.cfm}{a closures
list}.

A lack of staff doesn't totally derail business. About 2,100 of the
3,600 offices allow electronic filings. But a human being ultimately has
to process those filings, known as e-recordings, ``so if no one's there,
the pipeline is still blocked at the end,'' said Steve Gottheim, a
senior counsel with the title association, which is urging officials to
leave at least skeleton crews in the offices.

If a deed fails to be recorded in a timely manner, lenders can get
spooked by the potential for fraud. Without a public record, a devious
seller could technically sell a house twice.

Read more

\href{https://www.nytimes.com/by/melissa-clark}{\includegraphics{https://static01.nyt.com/images/2018/06/21/multimedia/author-melissa-clark/author-melissa-clark-thumbLarge.png}}

March 23, 2020, 12:30 p.m. ET

March 23, 2020, 12:30 p.m. ET

By \href{https://www.nytimes.com/by/melissa-clark}{Melissa Clark}

\hypertarget{if-cooking-from-your-pantry-make-a-vegetarian-skillet-chili}{%
\subsection{\texorpdfstring{\protect\hyperlink{if-cooking-from-your-pantry-make-a-vegetarian-skillet-chili}{If
cooking from your pantry, make a vegetarian skillet
chili.}}{If cooking from your pantry, make a vegetarian skillet chili.}}\label{if-cooking-from-your-pantry-make-a-vegetarian-skillet-chili}}

\includegraphics{https://static01.nyt.com/images/2020/03/23/dining/23live-bhh-pantryrecipe/23live-bhh-pantryrecipe-articleLarge.jpg?quality=75\&auto=webp\&disable=upscale}

Are you ready for another bean dish? I sure am, which is a good thing
considering how many kinds of beans I have on hand. (I'm not hoarding;
I'm just extremely well stocked.)

Over the weekend, I put a couple of cans of beans to work in a quick
vegetarian skillet chili. Like every recipe I'm writing about lately,
it's very adaptable. You can use any kind (or kinds) of beans, swap out
the spices, skip the tomatoes --- or double them, if your can is bigger
than mine. Skillet chili also happens to be fast and easy, good recipe
traits whether you're overwhelmingly busy or anxious, or a little of
each.

This recipe starts, like many great dishes, by sautéing an onion or
shallots or leeks in some oil with a pinch of salt (any kind of oil, any
kind of allium). When tender and golden at the edges, add minced garlic
and a jalapeño or other chile if you have one. Let it all cook until it
starts to smell delicious, then add spices --- chili powder, cumin,
coriander --- and let those toast for a minute to bring out their
flavors. Crumble in some dried oregano or marjoram, if you like, and add
two 15-ounce cans of drained beans (any kind) and any size can of
tomatoes with their liquid. (I used a 15-ounce can of diced tomatoes.)
Simmer it all for 15 to 20 minutes, so the flavors can meld.

Taste, and add more salt and spices if the chili needs it. I like to
serve this with sliced red onions that I've soaked in lime juice, and a
pinch each of salt and sugar. But jarred pickled peppers are great, too,
as are sliced scallions and a nice, fat, optional dollop of sour cream
or yogurt. This makes enough for three or four people, or freezes
perfectly if you're by yourself.

\begin{center}\rule{0.5\linewidth}{\linethickness}\end{center}

\emph{In this series, Melissa Clark will teach you how to cook with
pantry staples. Check back Tuesday for another installment.}

\emph{\textbf{Last week's recipes:}}\\
\emph{\textbf{Monday}}*:*
\href{https://cooking.nytimes.com/recipes/1020947-big-pot-of-beans}{\emph{Dried
beans}}\emph{.}\\
\emph{\textbf{Tuesday}}*:*
\href{https://cooking.nytimes.com/recipes/1020948-baked-steel-cut-oats-with-nut-butter}{\emph{Baked
oats}}\emph{.}\\
\emph{\textbf{Wednesday:}} **
\href{https://cooking.nytimes.com/recipes/1020949-pasta-with-tuna-capers-and-scallions}{\emph{Canned
tuna pasta}}\emph{.}\\
\emph{\textbf{Thursday}}*:*
\href{https://cooking.nytimes.com/recipes/1020950-any-vegetable-soup}{\emph{Any-vegetable
soup}}\emph{.}\\
\emph{\textbf{Friday:}} **
\href{https://cooking.nytimes.com/recipes/1020946-pantry-crumb-cake}{\emph{Pantry
crumb cake}}\emph{.}

Read more

\includegraphics{https://static01.nyt.com/images/icons/t_logo_291_black.png}

March 23, 2020, 12:00 p.m. ET

March 23, 2020, 12:00 p.m. ET

By The New York Times

\hypertarget{with-many-labs-closed-science-is-trying-to-find-a-way}{%
\subsection{\texorpdfstring{\protect\hyperlink{with-many-labs-closed-science-is-trying-to-find-a-way}{With
many labs closed, science is trying to find a
way.}}{With many labs closed, science is trying to find a way.}}\label{with-many-labs-closed-science-is-trying-to-find-a-way}}

\includegraphics{https://static01.nyt.com/images/2020/03/23/science/23SCI-VIRUS-CLOSEDLABS-promo/23SCI-VIRUS-CLOSEDLABS-promo-articleLarge.jpg?quality=75\&auto=webp\&disable=upscale}

In recent weeks, the coronavirus led to the shutdown of many university
campuses and other institutions for research and learning in the United
States and around the world.

There's always work that you can do from home. But parts of the
scientific process can only be completed in the lab, or at another
location where fieldwork or other hands-on research occurs. What's a
scientist to do when it's time to put some of their experiments on the
shelf?

Science put together a collection of stories from around the world on
how professors, graduate students and others in the sciences are coping
with the effects of coronavirus on their lives and work.

\hypertarget{site-index}{%
\subsection{Site Index}\label{site-index}}

\hypertarget{site-information-navigation}{%
\subsection{Site Information
Navigation}\label{site-information-navigation}}

\begin{itemize}
\tightlist
\item
  \href{https://help.nytimes.com/hc/en-us/articles/115014792127-Copyright-notice}{©~2020~The
  New York Times Company}
\end{itemize}

\begin{itemize}
\tightlist
\item
  \href{https://www.nytco.com/}{NYTCo}
\item
  \href{https://help.nytimes.com/hc/en-us/articles/115015385887-Contact-Us}{Contact
  Us}
\item
  \href{https://www.nytco.com/careers/}{Work with us}
\item
  \href{https://nytmediakit.com/}{Advertise}
\item
  \href{http://www.tbrandstudio.com/}{T Brand Studio}
\item
  \href{https://www.nytimes.com/privacy/cookie-policy\#how-do-i-manage-trackers}{Your
  Ad Choices}
\item
  \href{https://www.nytimes.com/privacy}{Privacy}
\item
  \href{https://help.nytimes.com/hc/en-us/articles/115014893428-Terms-of-service}{Terms
  of Service}
\item
  \href{https://help.nytimes.com/hc/en-us/articles/115014893968-Terms-of-sale}{Terms
  of Sale}
\item
  \href{https://spiderbites.nytimes.com}{Site Map}
\item
  \href{https://help.nytimes.com/hc/en-us}{Help}
\item
  \href{https://www.nytimes.com/subscription?campaignId=37WXW}{Subscriptions}
\end{itemize}
