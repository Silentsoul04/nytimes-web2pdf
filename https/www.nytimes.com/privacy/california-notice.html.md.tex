The Times and Your Data

\hypertarget{california-notice}{%
\section{California Notice}\label{california-notice}}

The Times and Your Data

\hypertarget{the-trust-of-our-readers-is-essential}{%
\section{The trust of our readers is
essential.}\label{the-trust-of-our-readers-is-essential}}

\hypertarget{new-york-times-privacy-company-notice-for-california-residents}{%
\section{New York Times Privacy Company Notice for California
Residents}\label{new-york-times-privacy-company-notice-for-california-residents}}

This notice is part of The New York Times Company's
\href{/privacy/privacy-policy}{Privacy Policy}, but it applies only to
California residents. It explains some additional rights and details
that were introduced by the California Consumer Privacy Act of 2018 (the
CCPA).

The words here use the meanings given to them in the CCPA, which are
often very broad. For example, the definition of ``personal
information'' under the CCPA includes your name as well as more general
information like how you navigate parts of our website.

\hypertarget{categories-of-information-we-collect}{%
\subsection{Categories of Information We
Collect}\label{categories-of-information-we-collect}}

This notice lists the categories of personal information we have
collected through any of the
\href{http://www.nytimes.com/privacy/privacy-policy}{Times Services} in
the past 12 months. Because some Times Services work differently than
others, your use of the Times Services may differ from other users. This
notice is likely to be overinclusive for any one person. For more
information on how we obtain this information, please refer to our
\href{http://www.nytimes.com/privacy/privacy-policy\#what-information-do-we-gather-about-you}{Privacy
Policy}.

\begin{enumerate}
\def\labelenumi{\arabic{enumi}.}
\item
  Identifiers --- including name, address, email address, account name,
  IP address, cookie ID, mobile advertising ID and an ID number assigned
  to your account.
\item
  Other personal information --- including phone number, billing
  address, credit or debit card information, employment or education
  information.\\
  This category includes personal information protected under
  pre-existing California law (Cal. Civ. Code 1798.80(e)) and overlaps
  with other categories.
\item
  Demographic information --- including your age or gender.\\
  This type of personal information includes what is also considered a
  protected classification characteristic under pre-existing California
  or federal laws.
\item
  Commercial information on your interactions with the Times Services
  --- including purchases and other commercial engagements with The New
  York Times Company.
\item
  Internet or other electronic network activity information ---
  including browsing activity on our sites and apps, browser type and
  browser language. This also encompasses other information that gets
  collected automatically when you use our sites and apps or interact
  with us through social media.
\item
  Geolocation data, inferred from your IP address, to help us deliver
  relevant content and enhance your experience.
\item
  Professional or employment-related data --- as part of our group
  subscriptions and when provided by you.
\item
  Inferences drawn from any of the above personal information --- about
  your preferences, predispositions and behavior as they relate to our
  sites and apps.
\end{enumerate}

\hypertarget{the-purposes-for-our-collection}{%
\subsection{The Purposes for Our
Collection}\label{the-purposes-for-our-collection}}

This notice offers broad information about the business and commercial
purposes for which we collect the information in the categories listed
above. Details about how we use the information we collect can be found
in our
\href{http://www.nytimes.com/privacy/privacy-policy\#what--do-we-do-with-the-information-we-gather}{Privacy
Policy}.

We collect the above information for the business purpose of giving you
access to our sites and apps. This information is a key part of how we
carry out subscription services, account access, customer service,
orders and transactions, customer research and feedback programs,
analytics, advertising and marketing services.

For commercial purposes, we collect information from the following
categories: identifiers, demographic information, commercial
information, internet activity, geolocation data (based on your IP
address but not your precise GPS location), professional data and
inferences.

\hypertarget{notice-of-your-right-to-opt-out-of-sale-of-personal-information}{%
\subsection{Notice of Your Right to Opt-Out of Sale of Personal
Information}\label{notice-of-your-right-to-opt-out-of-sale-of-personal-information}}

The New York Times Company does not sell personal information of its
readers as the term ``sell'' is traditionally understood. But ``sell''
under the CCPA is broadly defined. It includes the sharing of personal
information with third parties in exchange for something of value, even
if no money changes hands. For example, sharing an advertising or device
identifier to a third party may be considered a ``sale'' under the CCPA.

To the extent The New York Times Company ``sells'' your personal
information (as the term ``sell'' is defined under the CCPA), you have
the right to opt-out of that ``sale'' on a going-forward basis at any
time. Loading...

Once you have opted out, you will see a change to ``We No Longer Sell
Your Personal Information.'' If you have an account with certain Times
Services (specifically nytimes.com, cooking.nytimes.com,
nytimes.com/crosswords, the New York Times app, the New York Times
Cooking app and the New York Times Crossword app) and are logged in, we
will save your preference and honor your opt-out request across browsers
and devices so long as you remain logged in. If you are not logged in,
or do not have an account with any Times Services listed above, your
opt-out of the ``sale'' of personal information will be specific to the
browser or device from which you have clicked ``Do Not Sell My Personal
Information'' and until you clear your cookies (or local storage in
apps) on this browser or device.

If your browser or device is using a ``do not track'' setting, we will
detect it and honor it on that specific browser or device only. If you
wish to have a ``do not track'' experience across all of your browsers
and devices, please make sure that all of your browsers and devices are
set on ``do not track.''

After you opt out of the ``sale'' of your personal information, we will
no longer ``sell'' your personal information to third parties (except in
an aggregated or de-identified manner so it is no longer personal
information), but we will continue to share your personal information
with our service providers, which process it on our behalf. Exercising
your right to opt out of the ``sale'' of your personal information does
not mean that you will stop seeing ads on our sites and apps.

To opt-out of interest-based advertising as much as technically
possible, go to
``\href{http://www.nytimes.com/privacy/cookie-policy\#how-do-i-manage-trackers}{How
Do I Manage Trackers}'' in our Cookie Policy. To opt out of the ``sale''
of your personal information from participating companies, please visit
the Digital Advertising Alliance
\href{https://optout.privacyrights.info/?c=1}{website} or
\href{https://www.privacyrights.info/appchoices}{apps}. We do not
control these opt-out mechanisms and are not responsible for their
operation.

You can designate someone else to make a request on your behalf. To
protect your information, we will ask for a signed permission from you
authorizing the other person to submit a request on your behalf. We will
contact you to verify your identity before we respond to your authorized
agent's request.

After 12 months, we may ask you if you want to opt into the ``sale'' of
your personal information.

For further information, please refer to our
\href{http://www.nytimes.com/privacy/privacy-policy}{Privacy Policy}.

\hypertarget{right-to-request-accessible-alternative-format}{%
\subsection{Right to Request Accessible Alternative
Format}\label{right-to-request-accessible-alternative-format}}

This notice has been designed to be accessible to people with
disabilities. If you experience difficulties accessing this notice,
please contact us at
\href{mailto:privacy@nytimes.com}{\nolinkurl{privacy@nytimes.com}}.

\hypertarget{right-to-nondiscrimination}{%
\subsection{Right to
Nondiscrimination}\label{right-to-nondiscrimination}}

You have the right not to receive discriminatory treatment by us for the
exercise of any of your rights.

©2020 The New York Times Company

\href{/privacy}{Privacy F.A.Q.}\href{/privacy/privacy-policy}{Privacy
Policy}\href{/privacy/cookie-policy}{Cookie
Policy}\href{/privacy/california-notice}{California
Notice}\href{https://help.nytimes.com/hc/en-us/articles/115014893428-Terms-of-service}{Terms
of Service}

The Times and your Data

\hypertarget{main-menu}{%
\subsection{Main Menu}\label{main-menu}}

\href{/privacy}{Privacy F.A.Q.}\href{/privacy/privacy-policy}{Privacy
Policy}\href{/privacy/cookie-policy}{Cookie Policy}
