**NYTimes.com no longer supports Internet Explorer 9 or earlier. Please
upgrade your browser.
\href{https://www.nytimes.com/content/help/site/ie9-support.html}{LEARN
MORE »}

**Sections

** Home

**Search

\href{https://www.nytimes.com/projects/2020-report/index.html\#main}{Skip
to content}

\hypertarget{the-new-york-times}{%
\subsection{\texorpdfstring{\href{https://www.nytimes.com/}{The New York
Times}}{The New York Times}}\label{the-new-york-times}}

\href{}{insider}\textbar{}Journalism That Stands Apart

**Search

Subscribe Now

Log In

**0

**Settings

**Close search

\hypertarget{site-search-navigation}{%
\subsection{Site Search Navigation}\label{site-search-navigation}}

Search NYTimes.com

**Clear this text input

Go

\hypertarget{site-navigation}{%
\subsection{Site Navigation}\label{site-navigation}}

\hypertarget{site-mobile-navigation}{%
\subsection{Site Mobile Navigation}\label{site-mobile-navigation}}

\hypertarget{journalism-that-stands-apart}{%
\section{Journalism That Stands
Apart}\label{journalism-that-stands-apart}}

\hypertarget{the-report-of-the-2020-group}{%
\subsection{The Report of
the~2020~Group}\label{the-report-of-the-2020-group}}

\hypertarget{january-2017}{%
\subsection{January 2017}\label{january-2017}}

\texttt{\ \ } \texttt{\ \ }

This report, by a team of seven Times journalists, outlines the
newsroom's strategy and aspirations. For additional details,
\href{http://www.nytco.com/from-dean-and-joe-the-year-ahead}{see this
memo from Dean Baquet, The Times's executive editor, and Joe Kahn, the
managing editor}.

\begin{center}\rule{0.5\linewidth}{\linethickness}\end{center}

This is a vital moment in the life of The New York Times. Journalists
across the organization are hungry to make change a reality, and we have
new leaders ready to push us forward. Most important, The Times is
uniquely well positioned to take advantage of today's changing media
landscape --- but also vulnerable to decline if we do not transform
ourselves quickly.

While the past two years have been a time of significant innovation, the
pace must accelerate. Too often, digital progress has been accomplished
through workarounds; now we must tear apart the barriers. We must
differentiate between mission and tradition: what we do because it's
essential to our values and what we do because we've always done it.

The New York Times has staked its future on being a destination for
readers --- an authoritative, clarifying and vital destination. These
qualities have long prompted people to subscribe to our expertly curated
print newspaper. Today, they also lead people to devote valuable space
on their smartphone's homescreen to our app, to seek us out on social
media amid the cacophony and to subscribe to our newsletters and
briefings.

We are, in the simplest terms, a subscription-first business. Our focus
on subscribers sets us apart in crucial ways from many other media
organizations. We are not trying to maximize clicks and sell low-margin
advertising against them. We are not trying to win a pageviews arms
race. We believe that the more sound business strategy for The Times is
to provide journalism so strong that several million people around the
world are willing to pay for it. Of course, this strategy is also deeply
in tune with our longtime values. Our incentives point us toward
journalistic excellence.

And our strategy is working. The Times is unrivaled in its investment in
original, quality journalism. In 2016, our journalists
\href{https://www.nytimes.com/2016/12/17/world/we-were-there.html}{filed
from more than}
\href{https://www.nytimes.com/2016/12/17/world/we-were-there.html}{150}
\href{https://www.nytimes.com/2016/12/17/world/we-were-there.html}{countries}
--- nearly 80 percent of all countries on the planet. No newsroom in the
world has
\href{https://www.nytimes.com/interactive/2016/12/28/us/year-in-interactive-graphics.html}{more
journalists who can code}. We remain the employer of choice for top
journalists, receiving job queries from our peers at other leading
publications every week and hiring many of the field's most creative,
distinguished people.

Most important, our readers pay us the highest compliments: They are
willing to give us both their time and their money. The Times is by far
the most cited news publisher by other media organizations, the most
discussed on Twitter and the most searched on Google. Thanks to our
journalism, our digital revenue towers above that of any news
competitor. Recent media accounts have made clear the gap: Last year,
The Times brought in almost \$500 million in purely digital revenue,
which is far more than the digital revenues reported by many other
leading publications (including BuzzFeed, The Guardian and The
Washington Post) --- combined.

New York Times revenue, in billions

1.2

1.0

0.8

0.6

0.4

0.2

0

Advertising

Consumer

2003

2004

2005

2006

2007

2008

2009

2010

2011

2012

2013

2014

2015

\includegraphics{https://int.nyt.com/chartmaker/2016/12/01/20160919-yearconsumer-advertising/10/artboard-600px.png}

New York Times revenue, in billions

\$1.2

1.0

0.8

0.6

0.4

0.2

0

Advertising

Consumer

2004

2006

2008

2010

2012

2014

2015

\includegraphics{https://int.nyt.com/chartmaker/2016/12/01/20160919-yearconsumer-advertising/10/artboard-540px.png}

New York Times revenue, in billions

\$1.2

1.0

0.8

0.6

0.4

0.2

0

Advertising

Consumer

2005

2010

2015

\includegraphics{https://int.nyt.com/chartmaker/2016/12/01/20160919-yearconsumer-advertising/10/artboard-300px.png}

Our digital-subscription revenue also continues to grow at a strong
pace, while revenue from digital advertising is growing in spite of the
long-term shift of ad dollars to platforms like Google and Facebook. In
the third quarter of 2016, our digital subscriptions grew at the fastest
pace since the launch of the pay model in 2011 --- and growth then
exceeded that pace during the fourth quarter, in a postelection surge.
We now have more than 1.5 million digital-only subscriptions, up from
one million a year ago and from zero only six years ago. We also have
more than one million print subscriptions, and our readers are receiving
a product better than it has ever been, with rich new standalone
sections.

Yet to continue succeeding --- to continue providing journalism that
stands apart and to create an ever-more-appealing destination --- we
need to change. Indeed, we need to change even more rapidly than we have
been changing.

\begin{center}\rule{0.5\linewidth}{\linethickness}\end{center}

Why must we change? Because our ambitions are grand: to prove that there
is a digital model for
\href{https://www.nytimes.com/video/multimedia/100000003957700/the-times-celebrates-a-milestone.html}{original,
time-consuming, boots-on-the-ground, expert reporting that the world
needs}. For all the progress we have made, we still have not built a
digital business large enough on its own to support a newsroom that can
fulfill our ambitions. To secure our future, we need to expand
substantially our number of subscribers by 2020.

As Dean wrote to the newsroom, when explaining Project 2020, ``Make no
mistake, this is the only way to protect our journalistic ambitions. To
do nothing, or to be timid in imagining the future, would mean being
left behind.'' There are many
\href{https://www.nytimes.com/interactive/business/kodak-timeline.html}{once-mighty}
companies that believed their history of success would inevitably
protect them from technological change, only to be done in by their
complacency.

Our focus on subscribers stems from a challenge confronting us: the
weakness in the markets for print advertising and traditional forms of
digital-display advertising. But by focusing on subscribers, The Times
will also maintain a stronger advertising business than many other
publications. Advertisers crave engagement: readers who linger on
content and who return repeatedly. Thanks to the strength and innovation
of our journalism --- not just major investigative work and dispatches
from around the world but also interactive graphics, virtual reality and
Emmy-winning videos that redefine storytelling --- The Times attracts an
audience that advertisers want to reach.

A year ago, in the
\href{http://www.nytco.com/wp-content/uploads/Our-Path-Forward.pdf}{``Our
Path Forward''} document, the company announced its intention to double
its digital revenue by 2020, to \$800 million. The center of this
strategy is increasing our digital subscriptions. Doing so requires our
news report and newsroom to move past many habits that are holding us
back.

These realities led to the creation of the 2020 group. Our seven members
have spent the past year working closely with newsroom leaders;
conducting hundreds of conversations with Times journalists and with
outsiders; studying reader behavior and focus groups; and conducting a
written survey of the newsroom. (An appendix contains excerpts from the
survey responses.)

Our group is the heir to the Innovation Committee,
\href{http://www.niemanlab.org/2014/05/the-leaked-new-york-times-innovation-report-is-one-of-the-key-documents-of-this-media-age/}{whose
2014 report} and related work changed the culture of the newsroom. But
2020 has been different from the Innovation Committee in two important
respects.

First, we have had the benefit of working closely with Times leadership
over the past year to begin implementing changes. As a result, this
report is not intended to be the detailed guide for change that the
Innovation Report was. Many of the changes we advocate
\href{http://www.nytco.com/press/blog/}{are already well underway}. The
details will continue to come from Dean, Joe and the rest of the
leadership. This report is instead a statement of principles, priorities
and goals --- a guide to help members of the newsroom understand more
fully the direction that The Times is moving and to play an even bigger
role in making that change happen.

Second, the 2020 group was charged with questioning the assumption
behind the very first sentence of the Innovation Report: ``The New York
Times is winning at journalism.'' We are indeed winning, but not at a
scale sufficient to achieve the company's goals or sustain our cherished
newsroom operations.

We have not yet created a news report that takes full advantage of all
the storytelling tools at our disposal and, in the process, does the
best possible job of speaking to our potential audience. More of our
journalism needs to match what a large and growing number of curious and
sophisticated readers have told us they value most --- distinctive
journalism, in a comfortable form, that expands their understanding of
the world and helps them navigate it. Our work too often instead
reflects conventions built up over many decades, when we spoke to our
readers once a day, when we cultivated an aura of detachment from them
and when by far our most powerful tool was the written word. To keep our
current readers and attract new ones we must more often apply Times
values to the new forms of journalism now available to us.

For The Times to become an even more attractive destination to readers
--- and to maintain and strengthen its position in the years ahead ---
three broad areas of change are necessary. Our report must change. Our
staff must change. And the way we work must change.

\hypertarget{our-report}{%
\subsection{Our report}\label{our-report}}

The Times publishes about 200 pieces of journalism every day. This
number typically includes some of the best work published anywhere. It
also includes too many stories that lack significant impact or audience
--- that do not help make The Times a valuable destination.

What kinds of stories? Incremental news stories that are little
different from what can be found in the freely available competition.
Features and columns with little urgency. Stories written in a dense,
institutional language that fails to clarify important subjects and
feels alien to younger readers. A long string of text, when a
photograph, video or chart would be more eloquent.

We devote a large amount of resources to stories that relatively few
people read. Except in some mission-driven areas or in areas where
evidence suggests that the articles have disproportionate value to
subscribers, there is little justification for this. It wastes time ---
of reporters, backfielders, copy editors, photo editors and others ---
and dilutes our report.

The most poorly read stories, it turns out, are often the most
``dutiful'' --- incremental pieces, typically with minimal added
context, without visuals and largely undifferentiated from the
competition. They frequently do not clear the bar of journalism worth
paying for, because similar versions are available free elsewhere.

Our journalism must change to match, and anticipate, the habits, needs
and desires of our readers, present and future. We need a report that
even more people consider an indispensable destination, worthy of their
time every day and of their subscription dollars. Specifically:

\hypertarget{1-the-report-needs-to-become-more-visual}{%
\subsubsection{1. The report needs to become more
visual.}\label{1-the-report-needs-to-become-more-visual}}

The Times has an unparalleled reputation for excellence in visual
journalism. We have defined multimedia storytelling for the news
industry and established ourselves as the clear leader. Yet despite our
excellence, not enough of our report uses digital storytelling tools
that allow for richer and more engaging journalism. Too much of our
daily report remains dominated by long strings of text.

Share of stories with deliberately placed visual elements

\%

30

20

10

0

12.1\%

2014

April

July

Oct.

2015

April

July

Oct.

2016

April

July

Sep.

\includegraphics{https://int.nyt.com/chartmaker/2016/12/03/20160921-share-of-all-stories-wit/6/artboard-600px.png}

Share of stories with deliberately placed visual elements

\%

30

20

10

0

12.1\%

2014

April

July

Oct.

2015

April

July

Oct.

2016

April

Sep.

\includegraphics{https://int.nyt.com/chartmaker/2016/12/03/20160921-share-of-all-stories-wit/6/artboard-540px.png}

Share of stories with deliberately placed visual elements

\%

30

20

10

0

12.1\%

2014

2015

2016

Sep.

\includegraphics{https://int.nyt.com/chartmaker/2016/12/03/20160921-share-of-all-stories-wit/6/artboard-300px.png}

An example of the problem: When we ran a
\href{https://www.nytimes.com/2016/05/23/nyregion/f-train-express-plan-brings-anger-and-joy-depending-on-the-neighborhood.html}{story}
in 2016 about the roiling debate over subway routes in New York, a
reader mocked us in the comments for not including a simple map of the
train line at the heart of the debate. Similarly, when we write about
dance or art, our reporters and critics are able to include video or
photography but only in a limited way; they lack the proper training to
embed visuals contextually, and our content management system, Scoop,
makes the placement of visuals an afterthought. (The advent of Oak, our
new story creation tool in Scoop, is encouraging because it is designed
to address these problems.) The same issues apply to our critics writing
reviews on other topics, our sports reporters writing about
well-executed plays and our foreign correspondents trying to convey a
sense of place.

Reporters, editors and critics are eager to make progress here, and we
need to train and empower them. ``It's sort of demoralizing to know that
your story could be stronger with the help of a graphic,'' one reporter
told the 2020 group, ``but to also know that you will probably receive
no help with it.'' To solve the problem, we need to expand the number of
visual experts who work at The Times and also expand the number who are
in leadership roles.

We also need to become more comfortable with our photographers,
videographers and graphics editors playing the primary role covering
some stories, rather than a secondary role. The excellent journalism
already being produced by these desks serves as a model.

Given our established excellence in this area, creating a more visual
daily report is an enormous opportunity.

\includegraphics{https://static01.nyt.com/packages/flash/multimedia/ICONS/transparent.png}

Recent articles about the
\href{https://www.nytimes.com/interactive/2016/09/15/arts/design/national-museum-of-african-american-history-and-culture.html}{The
National Museum of African American History and Culture},
\href{https://www.nytimes.com/interactive/2016/12/07/world/asia/rodrigo-duterte-philippines-drugs-killings.html}{President
Rodrigo Duterte's brutal antidrug campaign} and
\href{https://www.nytimes.com/interactive/2016/08/05/sports/olympics-gymnast-simone-biles.html}{Simone
Biles} highlight the power of visual journalism.

\hypertarget{2-our-written-work-should-also-use-a-more-digitally-native-mix-of-journalistic-forms}{%
\subsubsection{2. Our written work should also use a more digitally
native mix of journalistic
forms.}\label{2-our-written-work-should-also-use-a-more-digitally-native-mix-of-journalistic-forms}}

\includegraphics{https://static01.nyt.com/packages/flash/multimedia/ICONS/transparent.png}

We should embrace and expand new storytelling forms, like the morning,
evening and
\href{https://www.nytimes.com/2017/01/12/us/politics/trump-cabinet-hearing.html}{live
briefings}.

The daily briefings are among the most successful products that The
Times has launched in recent years. They have a big, loyal audience,
among both Times subscribers and nonsubscribers. They also largely build
on journalistic investments The Times has already made. The briefings
are in many ways a digital manifestation of a daily newspaper: They take
advantage of the available technology and our curatorial judgment to
explain the world to readers on a frequent, predictable rhythm that
matches the patterns of readers' lives.

We need more innovations like the briefings.

We have dozens of regularly appearing features built for the print
edition but not enough for a digital ecosystem. We need more
journalistic forms that make The Times a habit by frequently
enlightening readers on major running stories, through email
newsletters, alerts, FAQs, scoreboards, audio, video and forms yet to be
invented.

These forms are not only consistent with our readers' habits, but they
also naturally encourage our journalists to use a less institutional and
more conversational writing style. Our journalists comfortably use this
style on social media, television and radio, and it is consistent with
the lingua franca of the Internet. One of its biggest advantages is that
it can convey the distinctiveness of The Times, making clear that we're
covering stories on the ground and doing so with expert journalists. In
our own report, however, we still do not use this more approachable
writing style often enough, and, when we do, we too often equate it with
the first-person voice. The Times has rightly become more comfortable
with the first person, but clear, conversational writing does not depend
on it.

One major problem is the bottlenecks that limit our ability to launch
new features, even when the tools already exist. A developer in
interactive news put it well: ``We should be approaching the shape of
our coverage with the same intent that we bring to our formal
newsgathering and reporting.''

To be clear, The Times is making progress in employing a richer, more
digital mix of journalistic forms. The progress in audio, video and
virtual reality are obvious examples. But the overall pace should
accelerate, and more of our journalists should participate in the
creative and production process. The value of The New York Times does
not depend on conveying information in the forms that made the most
sense for a print newspaper or for desktop computers.

\hypertarget{3-we-need-a-new-approach-to-features-and-service-journalism}{%
\subsubsection{3. We need a new approach to features and service
journalism.}\label{3-we-need-a-new-approach-to-features-and-service-journalism}}

Our largely print-centric strategy, while highly successful, has kept us
from building a sufficiently successful digital presence and attracting
new audiences for our features content. At the same time, we should make
a small number of big digital bets on areas where The Times has a
competitive opportunity, the way we did with
\href{http://cooking.nytimes.com/}{Cooking} and
\href{https://www.nytimes.com/watching}{Watching}.

The Times's current features strategy dates to the creation of new
sections in the 1970s. The driving force behind these sections, such as
Living and Home, was a desire to attract advertising. The main
attractions for readers were our ability to delight and to offer useful
advice about what to cook, what to wear and what to do. The strategy
succeeded brilliantly.

Today, we need a new strategy, both for traditional features (meant to
delight and inform) and for guidance (meant to be useful in tangible
ways). Our approach has kept us from building as large a digital
presence as the Times brand and journalistic quality make possible, and
kept us from making our print sections as imaginative, modern and
relevant for readers as they could possibly be. To be blunt, we have not
yet been as ambitious or innovative as our predecessors were in the
1970s.

Our readers are hungry for advice from The Times. Too often, we don't
offer it, or offer it only in print-centric forms. Our ability to
collaborate with \href{http://thewirecutter.com/}{The Wirecutter}, the
company's newest
\href{http://investors.nytco.com/press/press-releases/press-release-details/2016/The-New-York-Times-Company-Acquires-The-Wirecutter-and-The-Sweethome/default.aspx}{acquisition},
and the advent of
\href{https://www.nytimes.com/spotlight/times-tips}{Smarter Living} are
promising first steps in rethinking The Times's role as a guide, but we
remain far from reaching our potential here.

\includegraphics{https://static01.nyt.com/packages/flash/multimedia/ICONS/transparent.png}

Well's \href{https://www.nytimes.com/spotlight/well-guides}{series of
guides} expertly teach readers how to do something new or improve their
technique.

The audience and revenue goals laid out in
\href{http://www.nytco.com/wp-content/uploads/Our-Path-Forward.pdf}{``Our
Path Forward''} are highly ambitious. It is possible --- probable, in
the view of 2020 --- that The Times will not be able to meet them simply
by getting better at what we already do. In all likelihood, we will need
a modern version of the 1970s features expansion: devoting newsroom
resources to new areas, primarily to attract subscribers and engage new
readers (which in turn will attract advertisers). There would be nothing
wrong or new about doing so. The success of the 1970s features strategy
helped The Times afford great investigative journalism and foreign
correspondents stationed around the world. The 1970s features sections
also produced troves of wonderful journalism on their own.

We expect that the bigger opportunities are in providing guidance rather
than traditional features. We can help people curate the culture at a
moment when the culture, from television and movies to fashion and
style, is changing.

As we expand service, however, we should not forget traditional
features. We should continue producing trend pieces, profiles, essays
and other journalism that provides us a foundation of authority and are
essential to our most loyal readers.

\hypertarget{4-our-readers-must-become-a-bigger-part-of-our-report}{%
\subsubsection{4. Our readers must become a bigger part of our
report.}\label{4-our-readers-must-become-a-bigger-part-of-our-report}}

\includegraphics{https://static01.nyt.com/packages/flash/multimedia/ICONS/transparent.png}

The Times received
\href{https://mobile.nytimes.com/2016/10/29/world/middleeast/saudi-arabia-women.html?referer=}{nearly
6,000 responses} from a call-out asking women from Saudi Arabia about
their lives, frustrations and ambitions.

Perhaps nothing builds reader loyalty as much as engagement --- the
feeling of being part of a community. And the readers of The New York
Times are very much a community. They want to talk with each other and
learn from each other, not only about food, books, travel, technology
and crossword puzzles but about politics and foreign affairs, too.

We have developed one of
\href{https://www.nytimes.com/times-insider/2014/04/17/a-comments-path-to-publication/}{the
most civil and successful comment sections} in the news business, but we
still don't do nearly enough to allow our readers to have these
interactions.

Our richest community engagement right now is mainly in nooks and
crannies of the site: the robust discussion of philosophers on Opinion's
\href{https://www.nytimes.com/2017/01/09/opinion/is-humanism-really-humane.html}{``The
Stone''} series; the crossword fanatics on the
\href{https://www.nytimes.com/column/wordplay}{Wordplay} column; the
stories of
\href{http://well.blogs.nytimes.com/projects/breast-cancer-stories}{cancer
survivors} on Well; or the helpful notes on Cooking's best recipe for
\href{https://cooking.nytimes.com/recipes/1015819-chocolate-chip-cookies}{chocolate
chip cookies}.

We know from research and anecdotes that readers value the limited
opportunities we provide to engage in discussion. ``I have a friend who
emails me every time The Times approves one of her comments. It's an
accomplishment for her, akin to getting a letter to the editor
published,'' wrote the author of a recent Columbia Journalism Review
\href{http://www.cjr.org/first_person/comments_articles_publishers.php}{story
on commenting}.

Asking readers to invest their time on our platform creates a natural
cycle of loyalty. Network effects are the growth engine of every
successful startup, Facebook being the prime example. But the Times
experience doesn't get more interesting or valuable as more of a
reader's friends, relatives and colleagues use it. That must change.

\hypertarget{our-staff}{%
\subsection{Our staff}\label{our-staff}}

The Times employs \href{http://www.nytco.com/pulitzer-prizes/}{the
finest staff of journalists} in the world and remains the employer of
choice for many top journalists. Much about our newsroom staff must
remain unchanged. We should continue to employ a healthy mix of
newshounds, wordsmiths and analysts. We should continue to place
rigorous editing at the heart of our journalism. We should continue to
give journalists the time and resources to pursue work that has real
impact.

But we also must change our staff, and not primarily for budget reasons.
We must align the skills of our journalists with the demands of our
journalistic ambitions. We need a staff that makes The Times even more
of a reader destination than it is today, able to attract a larger
paying audience and able to become an even more influential source of
news and information. Specifically:

\hypertarget{1-the-times-needs-a-major-expansion-of-its-training-operation-starting-as-soon-as-feasible}{%
\subsubsection{1. The Times needs a major expansion of its training
operation, starting as soon as
feasible.}\label{1-the-times-needs-a-major-expansion-of-its-training-operation-starting-as-soon-as-feasible}}

The 2020 group's survey of the newsroom uncovered a deep desire among
many reporters and editors to acquire new skills. They understand that
Times journalism has already changed and will need to change even more.
They want to play a bigger role in making that change happen. To do so,
they need new kinds of knowledge, so that they are able to create
digitally native journalism that meets Times standards of excellence.

Our newsroom training efforts have improved markedly over the past year,
but they need to expand further. One recently hired reporter told us,
``The ability to maneuver and be trained on different platforms would be
ideal,'' adding that, ``training is always haphazard.''

Our staff is made up of the world's best journalists. Training will
allow them to combine their expertise and knowledge with the powerful
new storytelling tools at our disposal.

\hypertarget{2-we-need-to-accelerate-the-pace-of-hiring-top-outside-journalists}{%
\subsubsection{2. We need to accelerate the pace of hiring top outside
journalists.}\label{2-we-need-to-accelerate-the-pace-of-hiring-top-outside-journalists}}

We do not now have the right mix of skills in the newsroom to carry out
the ambitious plan for change. A few areas are especially important:
visual journalists; reporters who have both unmatched beat authority and
strong writing skills; and backfield editors with expertise in
sharpening ideas and shaping more analytical, conversational stories.

Above all, this new batch of talent must help us move away from
traditional, print-focused roles and toward new, multimedia-focused
roles, like senior visual journalists shaping both the form and content
of coverage. The most high-priority hires should be those of creators,
such as reporters, graphics editors, photographers and others who make
journalism. The hiring of star backfielders, well suited to the digital
age, is also crucial.

\includegraphics{https://static01.nyt.com/packages/flash/multimedia/ICONS/transparent.png}

Recent job posts, for The Times's first executive producer for audio;
\href{https://www.nytimes.com/interactive/2017/01/11/jobs/nyt-job-national-desk.html}{new
roles on the National desk}; and an editor to cover gender issues.

Some of our hiring needs have nothing to do with new journalistic tools.
They instead revolve around traditional beat authority. In the past, it
was acceptable for Times coverage to be merely solid in some areas, so
long as the total package was better than any other publication's. It no
longer is acceptable. The Internet is brutal to mediocrity. When
journalists make mistakes, miss nuances or lack sharpness, they're
called out quickly on Twitter, Facebook and elsewhere. Free alternatives
abound, often reporting the same commoditized information. As a result,
the returns to expertise have risen.

This new reality forces The Times to take a clear-eyed look at the
coverage of every subject that is central to our report and to evaluate
whether it is good enough. Put simply, is it so much better than the
competition's coverage --- which is largely free --- that we can
plausibly ask readers to pay for our own?

In many areas, the answer is yes; we employ journalists who are
recognized leaders in their field. No other media organization has a
report that is nearly as strong as ours overall. Yet we are not seeking
merely to be better. We are seeking to be so much better than the
competition that The Times is a destination that attracts several
million paying subscribers.

In recent years, the newsroom has hired about 70 new people a year, as
part of normal turnover to keep the newsroom population flat. In very
rough terms, about half of these hires have fallen into the categories
with the most direct impact on journalism: coverage leaders, reporters,
videographers, graphics editors and others. This pace needs to
accelerate, even though doing so will increase the need for newsroom
turnover given budget realities. The 2020 group does not make this
recommendation lightly; we also believe it is among the most important
recommendations we are making.

\hypertarget{3-diversity-needs-to-be-a-top-priority-for-our-newsroom}{%
\subsubsection{3. Diversity needs to be a top priority for our
newsroom.}\label{3-diversity-needs-to-be-a-top-priority-for-our-newsroom}}

Increasing the diversity of our newsroom -- more people of color, more
women, more people from outside major metropolitan areas, more younger
journalists and more non-Americans -- is critical to our ability to
produce a richer and more engaging report. It is also vital to our
strategic ambitions. Expanding our international audience and attracting
more young readers, which will go a long way toward determining whether
The Times meets its audience goals, depend on having a more diversified
report and a more diverse staff.

Every open position is an opportunity to improve diversity. We should
make an extra effort to broaden our lens. We should also think beyond
recruiting --- to career development --- to ensure that we create paths
for people in a variety of personal situations, including parents. When
big news breaks or investigations are launched, the people running
toward the action and the people sitting around the table plotting
coverage should reflect the audience we seek.

The
\href{http://investors.nytco.com/press/press-releases/press-release-details/2016/The-New-York-Times-Names-Ellen-Shultz-Executive-Vice-President-Talent-and-Inclusion/default.aspx}{recent
hiring} of an executive vice president for talent and inclusion creates
an important opportunity to make progress, because it can create
processes to ensure greater diversity. In addition, the Design, Product
and Technology groups recently
\href{http://www.nytco.com/a-note-from-kinsey-wilson-nytco-named-to-abis-2016-top-companies-for-women-technologists-leadership-index/}{took
concrete steps} to make diversity a priority and have seen results.
These efforts provide a model for other parts of the organization.

\hypertarget{4-we-should-rethink-our-approach-to-freelance-work-expanding-it-in-some-areas-and-shrinking-it-in-others}{%
\subsubsection{4. We should rethink our approach to freelance work,
expanding it in some areas and shrinking it in
others.}\label{4-we-should-rethink-our-approach-to-freelance-work-expanding-it-in-some-areas-and-shrinking-it-in-others}}

Inside the newsroom, we sometimes conflate Times quality with Times
staff, but our readers have a different view. If something appears in
The New York Times, they see it as Times quality (either positively or
negatively). The best of that work elevates The Times, and it's often
the quickest and most economical way to reach new audiences or improve
an aspect of our report.

Indeed, freelance work is often among our best-read journalism, in both
the newsroom and in Opinion. The successes are easy to name: Op-Eds,
Op-Docs, book reviews, photography, pieces for the Magazine, Science,
Styles, Travel, Upshot, Well and elsewhere, as well as news dispatches
that fill crucial coverage needs. These are not merely isolated cases,
either. On a per-dollar basis, our freelance-written journalism attracts
a larger audience on average than our staff-written journalism.

Yet the landscape is bifurcated. We also use contributors to provide
obligatory coverage that doesn't resonate with readers and help to make
The Times a destination. Much of this work exists because of print
legacies or an aversion to relying on wire reporting even for dutiful,
incremental stories. We rely on stringers in every state and around the
world for routine coverage of stories that too often does not surpass
the quality or speed of the wires and that requires considerable effort
editing and coordinating.

We need to be more creative, and ambitious, with the money spent each
year on outside contributors. But we should not conflate changing our
freelance spending with cutting it. When a newsroom budget is under
pressure, freelance is often the most obvious candidate for cuts. Taking
an across-the-board approach now would be a mistake. It is likely, in
fact, that overall freelance spending should increase. But parts of it
should be eliminated, as part of a rigorous review.

\hypertarget{the-way-we-work}{%
\subsection{The way we work}\label{the-way-we-work}}

We should reorganize the newsroom to reflect our digital present and
future rather than our print legacy. The Times needs a newsroom more
nimble and better at taking risks than in the past. It needs to take the
notion of management more seriously and run itself less by gut instinct.

We have spent the last 20 years tinkering with organizational structures
and processes born of print demands. Even today, our operation is still
largely a reflection of the physical newspaper. It is time to become
more aggressive. Specifically:

\hypertarget{1-every-department-should-have-a-clear-vision-that-is-well-understood-by-its-staff}{%
\subsubsection{1. Every department should have a clear vision that is
well understood by its
staff.}\label{1-every-department-should-have-a-clear-vision-that-is-well-understood-by-its-staff}}

Our most successful forays into digital journalism, from both existing
departments and new ones, have depended on distinct visions established
by their leaders --- visions supported and shaped by the masthead, and
enthusiastically shared by the members of the department. The list
includes Graphics, the Briefings, Cooking, Well and others.

This isn't an accident. The rise of digital journalism has given us many
more ways to tell stories and to reach readers. But we need to make
choices about what we're going to do and not do. We need to be more
proactive than we were during the decades of a stable, thriving print
business.

These departments with clear, widely understood missions remain unusual.
Most Times journalists cannot describe the vision or mission of their
desks, and the identities of those desks remain closely tied to
eponymous print sections. Most departments have not made clear decisions
about who their primary audience is and and which journalistic forms are
a priority (and which are not). Many people in the newsroom are hungry
for such clarity and believe it will make them more effective
journalists.

The 2020 group believes that an effective vision spans three main areas:

\begin{itemize}
\tightlist
\item
  \textbf{Journalism} What will the team cover (and not cover), and in
  what forms? How will it distinguish its coverage from competitors'
  coverage?
\item
  \textbf{Audience} Who is the target audience for each aspect of the
  team's report? How will these audiences find and experience the
  coverage, and what role will it play in making The Times a habit? What
  does success look like, and how will departments know when they have
  achieved it?
\item
  \textbf{Operations} What skills does the group need? What, for
  instance, is the appropriate balance between reporters/content
  creators and managers/editors? How will the group interact with the
  print hub and other cross-department teams?
\end{itemize}

\hypertarget{2-we-should-set-goals-and-track-our-progress-toward-them}{%
\subsubsection{2. We should set goals and track our progress toward
them.}\label{2-we-should-set-goals-and-track-our-progress-toward-them}}

In a print era, when the newspaper business was stable, the newsroom
could do without tracking the success of individual elements of the
report --- or the report as a whole. The excellence of the overall
bundle overshadowed specific deficiencies. And it was cumbersome to
quantify success. The Times continued to make money and to have a strong
reputation. That was enough.

But today our business is changing rapidly. We have much better data
than we once did. And as strong as our reputation remains, our position
in the market is under attack.

Our management practices, however, remain mostly unchanged. Much of the
newsroom does not set tangible goals, much less feel accountable for
reaching goals. Even those with some access to data are exposed to just
a narrow slice of it (like pageviews about individual articles via
Stela), and they don't know what success looks like.

Multiple people told the 2020 group that they were frustrated by a lack
of understanding and transparency about newsroom goals. One said: ``I
think people would appreciate our willingness to try different things if
they were allowed a better understanding of why we're trying something
an alternate way and what we hope to achieve.''

As we saw with Cooking, the mere exercise of setting targets, even rough
ones, can be a powerful focusing mechanism. It allows for clear-eyed
assessment of what is and is not working.

Ultimately, goals will work only if they are coupled with
accountability. The Times should be more willing to expand teams that
are thriving, to change course for teams that don't appear to have the
right approach, to shift resources away from teams that appear to be
failing and to change leadership when appropriate. We're no longer in a
period when most coverage leaders have the luxury of ``figuring it out''
over multiple years.

\hypertarget{3-we-need-to-redefine-success}{%
\subsubsection{3. We need to redefine
success.}\label{3-we-need-to-redefine-success}}

The newsroom has embraced data and analytics over the past year, with
positive effects. We now have a better sense for which of our work
resonates with readers and which does not. We're producing more resonant
work, and we have largely resisted the lures of clickbait.

Now we need to take the next steps. The newsroom needs a clearer
understanding that pageviews, while a meaningful yardstick, do not equal
success. To repeat, The Times is a subscription-first business; it is
not trying to maximize pageviews. The most successful and valuable
stories are often not those that receive the largest number of
pageviews, despite widespread newsroom assumptions. A story that
receives 100,000 or 200,000 pageviews and makes readers feel as if
they're getting reporting and insight that they can't find anywhere else
is more valuable to The Times than a fun piece that goes viral and yet
woos few if any new subscribers.

The data and audience insights group, under Laura Evans, is in the
latter stages of creating a more sophisticated metric than pageviews,
one that tries to measure an article's value to attracting and retaining
subscribers. This metric seems a promising alternative to pageviews.

Yet the newsroom should also understand that no metric is perfect. To a
significant extent, we will need to rely on a mix of quantitative
measures and qualitative judgments when deciding which stories to do and
to promote. Achieving the right balance is tricky. We neither want to
equate audience size with journalistic value nor do we want to return to
the days when we persuaded ourselves that a piece of journalism was
valuable for the mere reason that it appeared in The New York Times.

\hypertarget{4-we-need-a-greater-focus-on-conceptual-front-end-editing}{%
\subsubsection{4. We need a greater focus on conceptual, front-end
editing.}\label{4-we-need-a-greater-focus-on-conceptual-front-end-editing}}

The 2020 group's survey of the newsroom found that many reporters wished
their editors had more time to help them sharpen stories in the early
stages of reporting and writing. At the same time, many reporters, and
editors, believe The Times wastes time and resources on repeated
line-editing of individual stories, making changes of limited value.

The 2020 group believes strongly in the value of copy-editing. There is
a high price for easily identifiable errors, such as spelling and
grammar mistakes. An increase in such errors would send the wrong
message to readers --- that our product is sloppy and lacks high value.
When we publish sloppy stories, readers complain to us in significant
numbers. At the same time, The Times spends too much time on low-value
line-editing, such as the moving, unmoving and removing of paragraphs,
and too little on conceptual editing and story sharpening, including on
questions like what form a story should take. A shift toward front-end
editing will need to involve changes in multiple parts of the newsroom,
including the copy desk, the backfield and the masthead.

The Times currently devotes too many resources to low-value editing ---
and, by extension, too many to editing overall. Our journalism and our
readers would be better served if we instead placed an even higher
priority on newsgathering in all of its forms.

\hypertarget{5-the-newsroom-and-our-product-teams-should-work-together-more-closely}{%
\subsubsection{5. The newsroom and our product teams should work
together more
closely.}\label{5-the-newsroom-and-our-product-teams-should-work-together-more-closely}}

\includegraphics{https://static01.nyt.com/packages/flash/multimedia/ICONS/transparent.png}

New products, like \href{https://www.nytimes.com/watching}{Watching},
benefit from close collaboration across teams and lead to new ways of
presenting Times journalism.

For The Times to remain a destination --- a high bar in an age of
social-media platforms --- the experience of reading, watching and
listening to our work needs to be as compelling as the journalism
itself. Achieving this goal will be far easier if our journalists and
our product teams (comprising product managers, designers and
developers) work more closely together. We need both journalists and
product specialists to understand reader behavior, to develop a sharp
view of the competition and to understand how different areas of
coverage fit into the broader Times experience.

Each group needs a better understanding of what the other does. Despite
great strides over the past two years, many product teams don't have a
deep understanding of the newsroom, including how we think about our
coverage and how we do our jobs. Much of the newsroom, similarly,
doesn't understand what the product teams do.

The central flaw in the current setup is that the newsroom ends up
focusing on short-term problem solving (How do we make today's report
excellent?), while the product teams focus on longer-term questions
(What's the best future news experience?). Our editors still aren't
involved closely enough in thinking about how the Times experience
across different platforms should evolve, and our product managers often
aren't aware of coverage priorities. The results can be problematic. For
example, the design and functionality of our homepage have remained
effectively static for the past decade.

A closer working relationship would cause both the newsroom and the
product teams to function more effectively.

\hypertarget{6-we-need-to-reduce-the-dominant-role-that-the-print-newspaper-still-plays-in-our-organization-and-rhythms-while-making-the-print-paper-even-better}{%
\subsubsection{6. We need to reduce the dominant role that the print
newspaper still plays in our organization and rhythms, while making the
print paper even
better.}\label{6-we-need-to-reduce-the-dominant-role-that-the-print-newspaper-still-plays-in-our-organization-and-rhythms-while-making-the-print-paper-even-better}}

The print version of The New York Times remains a daily marvel, beloved
by a large number of loyal readers. It is a curated version of our best
stories, photography, graphics and art.

But the newsroom's current organization creates dangers for the print
newspaper --- and is also holding back our ability to create the best
digital report. Today, department heads and other coverage leaders must
organize much of their day around print rhythms even as they find
themselves gravitating toward digital journalism. The current setup is
holding back our ability to make further digital changes, and it is also
starting to rob the print newspaper of the attention it needs to become
even better.

The print hub made impressive strides in 2016, beginning to take over
some functions from departments while also creating a series of
successful new print-only sections and features. Progress in these
directions needs to accelerate in the early months of 2017, to ensure
that the print hub becomes more autonomous. A Times working group is
examining how to continue improving the print newspaper, building off
the recent progress.

A more muscular print hub will also allow for the creation of more
subject-focused newsroom teams, which can make our coverage more
authoritative and sophisticated and allow it to rise above the
competition more often. Our big news desks were built to fill sections
in the print edition. As a result, high-priority coverage areas are
spread across multiple desks, diluting them and limiting collaboration
among journalists covering the same subjects. There is not enough
coordination among some Times journalists who cover similar beats, and
there is even less consideration about which audiences we're targeting
and how they're expected to consume our journalism. The pending creation
of
\href{https://www.nytimes.com/interactive/2016/jobs/nyt-climate-change-editor.html}{climate}
and
\href{https://www.nytimes.com/interactive/2016/jobs/nyt-gender-editor.html}{gender}
teams is a step in the right direction.

\includegraphics{https://static01.nyt.com/packages/flash/multimedia/ICONS/transparent.png}

Special print sections produced last year, including one on the opening
of the
\href{https://www.nytimes.com/interactive/2016/09/15/arts/design/national-museum-of-african-american-history-and-culture.html}{National
Museum of African American History and Culture}; the story of a woman
\href{https://www.nytimes.com/interactive/2016/05/01/nyregion/living-with-alzheimers.html}{finding
her way through the early stages of Alzheimer's disease}; an entire
issue of the Sunday Magazine
\href{https://www.nytimes.com/interactive/2016/08/11/magazine/isis-middle-east-arab-spring-fractured-lands.html}{dedicated
to the Middle East}; and
\href{https://www.nytimes.com/2016/03/28/sports/boxing-youngstown-anthony-taylor-hamzah-aljahmi.html}{``Fight,''}
the tale of boxers Hamzah Aljahmi and Anthony Taylor.

\begin{center}\rule{0.5\linewidth}{\linethickness}\end{center}

The idea that The Times must change can seem daunting and
counterintuitive. We continue to be the most influential news
organization in the country, with a large and growing group of loyal
readers. But the notion of a changing New York Times is not new. The
institution's great success over the past century has depended on its
ability to change.

The Times
\href{https://timesmachine.nytimes.com/timesmachine/1917/01/17/issue.html}{was
once filled} with short, dry articles documenting incremental news in
business and public life. As recently as the early 1980s, our front page
included 10 stories a day and a smattering of small black-and-white
photos. There was even a time when Times editors considered a crossword
puzzle to be beneath the institution's dignity.

But as readers' habits and needs changed, The Times changed with them.
Our values did not change; our expression of them did. Previous
generations of editors introduced a magazine, a book review, readers'
letters, daily features sections and color photography. The most recent
manifestation of these changes is the creation of a digital report,
first on desktop computers and then on phones, that is widely regarded
as the world's finest.

The digital revolution, however, has not stopped. If anything, the
changes in our readers' habits --- the ways that they receive news and
information and engage with the world --- have accelerated in the last
several years. We must keep up with these changes.

The members of the 2020 group have emerged from this process both
optimistic and anxious. We are optimistic, deeply so, because The Times
is better positioned than any other media organization to deliver the
coverage that millions of people are seeking. The institution's values
are exactly right for the moment. The strongest daily journalism, the
meatiest enterprise, the hardest-hitting investigations and the most
useful and delightful features will continue to make us stand out from
the crowd. Thanks to our values and our great strengths, The Times has
the potential in coming years to become an even stronger, larger, more
influential news source.

But we must not fall prey to wishful thinking and believe that such an
outcome is inevitable. It is not. We also face real challenges ---
journalism challenges and business challenges. If we do not address
them, we will give our competitors an opportunity to overtake us. We
will leave ourselves vulnerable to the same kind of technology-related
decline that has afflicted other long-successful businesses, both inside
and outside media.

The task facing the leadership of The Times is more daunting than what
those earlier generations faced, because of the scope of the digital
revolution. Yet the essential challenge remains the same. We must be
steadfast with our values and creative in realizing them. We must act
with urgency.

By David Leonhardt, Jodi Rudoren, Jon Galinsky, Karron Skog, Marc Lacey,
Tom Giratikanon and Tyson Evans.

Research and analysis contributed by Samarth Bhaskar and Dan Gendler.

\hypertarget{newsroom-survey-responses}{%
\subsection{Newsroom survey responses}\label{newsroom-survey-responses}}

\hypertarget{the-2020-group-conducted-a-newsroom-survey-last-summer-asking-what-the-newsroom-of-the-future-should-look-like-nearly-200-people-responded-in-writing-while-others-met-with-members-of-our-group-in-person-the-following-is-a-compilation-of-responses-that-represents-some-of-the-strongest-themes}{%
\paragraph{The 2020 group conducted a newsroom survey last summer asking
what the newsroom of the future should look like. Nearly 200 people
responded in writing, while others met with members of our group in
person. The following is a compilation of responses that represents some
of the strongest
themes.}\label{the-2020-group-conducted-a-newsroom-survey-last-summer-asking-what-the-newsroom-of-the-future-should-look-like-nearly-200-people-responded-in-writing-while-others-met-with-members-of-our-group-in-person-the-following-is-a-compilation-of-responses-that-represents-some-of-the-strongest-themes}}

\hypertarget{reporting-and-writing}{%
\subsubsection{Reporting and writing}\label{reporting-and-writing}}

\hypertarget{there-was-a-broad-consensus-throughout-the-responses-that-we-are-doing-too-many-dutiful-800-word-stories-and-that-we-should-do-less-coverage-of-incremental-news-many-people-said-we-should-do-more-profiles-investigations-and-long-form-narratives-as-well-as-quick-explainers-lists-and-live-blogs-quite-a-few-editors-said-they-would-like-to-write-occasionally-and-several-reporters-said-they-would-appreciate-a-player-coach-model}{%
\paragraph{There was a broad consensus throughout the responses that we
are doing too many dutiful 800-word stories and that we should do less
coverage of incremental news. Many people said we should do more
profiles, investigations and long-form narratives, as well as quick
explainers, lists and live blogs. Quite a few editors said they would
like to write occasionally, and several reporters said they would
appreciate a player-coach
model.}\label{there-was-a-broad-consensus-throughout-the-responses-that-we-are-doing-too-many-dutiful-800-word-stories-and-that-we-should-do-less-coverage-of-incremental-news-many-people-said-we-should-do-more-profiles-investigations-and-long-form-narratives-as-well-as-quick-explainers-lists-and-live-blogs-quite-a-few-editors-said-they-would-like-to-write-occasionally-and-several-reporters-said-they-would-appreciate-a-player-coach-model}}

``The 800-word news story is the bread and butter of the print product,
but time and again we have seen studies (and can see in our own traffic
statistics) that those stories struggle mightily to perform well online.
Everyone in the room seems to know this, but we continue to produce them
out of some rote allegiance to a product that fewer and fewer people
read.''

``I would like the burden of managing a coverage area to be more on
creating original work, less on covering all the bases.''

``I would love us to be more agile in how we chose to report stories. A
reporter in the field with a good sense of all the tools available
should be able to make the call over how best to tell the story, not be
in the middle of a live protest and have to call 5 different people and
be on three different email threads just to launch a Facebook Live or
Snapchat takeover from their phone.''

``There needs to be more versatility and movement. Becoming an editor
shouldn't mean the end of writing for strong reporters and writers; and
strong reporters and writers, especially with subject expertise, should
be encouraged to think of stories for others to write that they could
possibly edit or advise. Our structure and our approach (and hiring)
should encourage shared responsibility, not rigid roles and
hierarchies.''

\hypertarget{editing}{%
\subsubsection{Editing}\label{editing}}

\hypertarget{reporters-said-they-wanted-more-helpful-interaction-with-their-editors-at-the-outset-less-editing-in-the-middle-and-more-attention-to-presentation-and-promotion-there-was-much-frustration-about-stories-being-held-because-of-print-considerations-and-several-editors-and-reporters-said-they-would-like-to-see-a-copy-editing-process-that-was-more-responsive-to-the-complexity-of-the-story-and-the-urgency-of-the-news}{%
\paragraph{Reporters said they wanted more helpful interaction with
their editors at the outset; less editing in the middle; and more
attention to presentation and promotion. There was much frustration
about stories being held because of print considerations. And several
editors and reporters said they would like to see a copy editing process
that was more responsive to the complexity of the story and the urgency
of the
news.}\label{reporters-said-they-wanted-more-helpful-interaction-with-their-editors-at-the-outset-less-editing-in-the-middle-and-more-attention-to-presentation-and-promotion-there-was-much-frustration-about-stories-being-held-because-of-print-considerations-and-several-editors-and-reporters-said-they-would-like-to-see-a-copy-editing-process-that-was-more-responsive-to-the-complexity-of-the-story-and-the-urgency-of-the-news}}

``Every story feels like a fire hydrant --- it gets passed from dog to
dog, and no one can let it go by without changing a few words.''

``We spend too little time thinking about how stories will be told,
which means we get too many stories that are middling in every way. I'd
like to see more time spent brainstorming and workshopping ideas at the
front end, and being more willing to kill ideas that don't rise to the
level of memorable.''

``Hire editors and reporters who don't need to have their hands held.
Honestly, how can we still afford to have five editors arguing for hours
over a routine day story? The print mentality still rules the newsroom,
from the top down. But it is important to maintain the commitment to
copy editing, as it is essential to the quality of the journalism and
the reputation of the news site.''

``There is too much editing on the copy desks, where editors are
adhering to a style that is increasingly becoming far too rigid for the
Times.''

``Too often, on breaking, competitive stories, the time from the
reporter filing, to the slot publishing, is far too long. I get the
impression that the backfield and copy desk are overloaded and have
trouble prioritizing.''

``Most of the time, you time and edit stories to print requirements, no
matter what the official doctrine says. I've had things hold for weeks
while waiting for a print slot.''

\hypertarget{visual-journalism}{%
\subsubsection{Visual journalism}\label{visual-journalism}}

\hypertarget{many-people-said-they-were-enthusiastic-about-the-mandate-to-think-more-visually-but-many-also-said-the-obstacles-to-getting-there-were-far-too-high-citing-little-or-no-access-to-graphics-editors-or-the-video-unit-several-people-said-they-wished-they-had-the-tools-and-ability-to-make-simple-graphics-themselves}{%
\paragraph{Many people said they were enthusiastic about the mandate to
think more visually. But many also said the obstacles to getting there
were far too high, citing little or no access to graphics editors or the
video unit. Several people said they wished they had the tools and
ability to make simple graphics
themselves.}\label{many-people-said-they-were-enthusiastic-about-the-mandate-to-think-more-visually-but-many-also-said-the-obstacles-to-getting-there-were-far-too-high-citing-little-or-no-access-to-graphics-editors-or-the-video-unit-several-people-said-they-wished-they-had-the-tools-and-ability-to-make-simple-graphics-themselves}}

``It is too hard for a reporter or editor to get help on a special
project. Each pod should have a graphics and/or interactives point
person. They should be involved with reporting from the beginning,
identifying which stories are ripe for media and using their knowledge
to make the most of a story.''

``Some of our visual work is too polished. Intimacy and serendipity is a
huge part of the internet. We currently don't have the editorial courage
to pull that lever.''

``I'm a reporter and I have almost never spoken to a video person.''

``A friend at BuzzFeed has told me that he effectively has to argue FOR
a traditional story format, rather than for non-traditional formats. In
his context, all formats are effectively equal, and all need to be
justified as useful. I do not propose we emulate BuzzFeed (Times readers
come to us for specific reasons, obviously), but forcing us to justify
traditional stories could make us re-think how we use non-traditional
formats.''

``If every desk had someone who could produce a nimble graphic, and
people didn't need special `keys' to make a simple chart or a map, we
could get a lot more done. It's sort of demoralizing to know that your
story could be stronger with the help of a graphic, but to also know
that you will probably receive no help with it.''

\hypertarget{conversational-tone}{%
\subsubsection{Conversational tone}\label{conversational-tone}}

\hypertarget{there-was-quite-a-bit-of-ambivalence-about-changing-the-tone-or-sensibility-of-writing-some-were-eager-to-try-new-voice-and-forms-but-werent-quite-sure-how-others-said-they-were-stymied-by-the-backfield-or-copy-desk-when-they-tried-others-still-felt-we-should-be-very-cautious-about-making-any-such-changes}{%
\paragraph{There was quite a bit of ambivalence about changing the tone
or sensibility of writing. Some were eager to try new voice and forms
but weren't quite sure how. Others said they were stymied by the
backfield or copy desk when they tried. Others still felt we should be
very cautious about making any such
changes.}\label{there-was-quite-a-bit-of-ambivalence-about-changing-the-tone-or-sensibility-of-writing-some-were-eager-to-try-new-voice-and-forms-but-werent-quite-sure-how-others-said-they-were-stymied-by-the-backfield-or-copy-desk-when-they-tried-others-still-felt-we-should-be-very-cautious-about-making-any-such-changes}}

``In simple terms, we need less head and more heart in our storytelling.
Emotion is not something we tend to embrace, and we should. It's a major
driver of loyalty. Of connection.... We write too often in a male
executive voice, which tends to push away many readers we should be
bringing close.''

``We frequently hear from the top editors at the paper that they want
more voice and less institutional-ese in our stories. But typically when
you try to make the prose more playful or engaging in a news story, or
just generally inject a bit more personality, the copy desk is quick to
ferret it out, and it can be exhausting to push back on every single
word or phrase. If we're going to loosen our style up a bit, the copy
desk is going to be the key swing demographic.''

``We really get nailed in a way that other publications do not when
we're wrong or even just a little tonally false. People hold us to
higher expectations than other newspapers or Web sites, with an almost
visceral sense of betrayal when we're wrong. I think we've chosen to go
the route of a high-quality publication and standards are a big part of
that. We need editors to keep that quality up.''

\hypertarget{newsroom-organization}{%
\subsubsection{Newsroom organization}\label{newsroom-organization}}

\hypertarget{there-was-wide-agreement-that-separating-print-production-is-crucial-to-fostering-change-in-the-newsroom-there-was-little-consensus-however-on-structure-lack-of-collaboration-among-the-desks-was-a-top-complaint-several-people-advocated-for-less-rigid-lines-between-being-an-editor-and-reporter}{%
\paragraph{There was wide agreement that separating print production is
crucial to fostering change in the newsroom. There was little consensus,
however, on structure. Lack of collaboration among the desks was a top
complaint. Several people advocated for less rigid lines between being
an editor and
reporter.}\label{there-was-wide-agreement-that-separating-print-production-is-crucial-to-fostering-change-in-the-newsroom-there-was-little-consensus-however-on-structure-lack-of-collaboration-among-the-desks-was-a-top-complaint-several-people-advocated-for-less-rigid-lines-between-being-an-editor-and-reporter}}

``We should experiment with more hybrid jobs, in which reporters edit
and editors write, as is done at many other news organizations ---
flatten the org chart, encourage collegiality, diversify skill sets,
vary how people spend their days.''

``The Times still suffers from a drastic lack of teamwork, camaraderie
and coordination between desks and reporters for different desks.''

``How do we find a way for reporters to work with more and different
kinds of editors to learn more skills? Also, how do we make it easier
for reporters to write across the paper? We are encouraged to do this,
but in practice it's really hard and weird.''

``I believe every editor must be directly ordered to think beyond his or
her desk, must be evaluated on how collaborative they are (based on
interviews or assessments by their peers on other desks) and must be
penalized when they play keep-away with stories.''

``I would like to report to an enterprise editor who has the authority
to offer my work to the department where the story best fits. I can see
this working for a variety of topics, such as immigration, drugs, etc.
``

\hypertarget{hiring-training-and-development}{%
\subsubsection{Hiring, training and
development}\label{hiring-training-and-development}}

\hypertarget{several-respondents-said-they-were-frustrated-by-a-lack-of-transparency-in-the-hiring-process-as-well-as-a-lack-of-diversity-in-race-gender-and-experience-in-top-positions-some-said-they-had-received-little-or-no-training-in-their-years-at-the-times-and-saw-too-few-opportunities-for-career-development}{%
\paragraph{Several respondents said they were frustrated by a lack of
transparency in the hiring process, as well as a lack of diversity (in
race, gender and experience) in top positions. Some said they had
received little or no training in their years at The Times and saw too
few opportunities for career
development.}\label{several-respondents-said-they-were-frustrated-by-a-lack-of-transparency-in-the-hiring-process-as-well-as-a-lack-of-diversity-in-race-gender-and-experience-in-top-positions-some-said-they-had-received-little-or-no-training-in-their-years-at-the-times-and-saw-too-few-opportunities-for-career-development}}

``The Times should invest more in career planning, and should do more to
not only hire people of color or people who aren't from the usual talent
pipelines but also help them with mentorship and career advancement.''

``We need more diversity at the top, in the traditional sense and in the
sense of diversity of skills. There are too many people at the top who
are reporters or former reporters --- and that's just one set of skills.
Production and administration skills are essential, and should be more
fully represented at the top.''

``I think it's really important to do a better job of communicating
general strategy with regards to our audience with the entire newsroom
(alerts, scheduling things for morning publication, homepage play,
liveblogging, Listys, mobile presentation, etc.). For the past year or
two, I've sensed a lot of frustration with our ever-changing direction
--- but, I don't think that's purely a product of the experimentation, I
think people would appreciate our willingness to try different things if
they were allowed a better understanding of why we're trying something
an alternate way and what we hope to achieve.''

``The ability to maneuver and be trained on different platforms would be
ideal. The Times, unlike other places, does a great job of mixing things
up and changing the jobs/positions of people so they do not get bored
and always have a fresh take. But from what I can tell (I'm still pretty
new) the people don't get a huge say in where they may end up and
training is always haphazard.''

``Leaders should be held accountable for stated priorities, whatever
they are. We focus on external metrics, but no one is saying `your desk
needs to have striking visuals in 40 percent of its stories next month'
and demanding accountability.''

\hypertarget{more-on-nytimescom}{%
\subsection{More on NYTimes.com}\label{more-on-nytimescom}}

\hypertarget{site-information-navigation}{%
\subsection{Site Information
Navigation}\label{site-information-navigation}}

\begin{itemize}
\tightlist
\item
  \href{https://www.nytimes.com/content/help/rights/copyright/copyright-notice.html}{©
  2017 The New York Times Company}
\item
  \href{https://www.nytimes.com}{Home}
\item
  \href{https://query.nytimes.com/search/sitesearch/\#/}{Search}
\item
  Accessibility concerns? Email us at
  \href{mailto:accessibility@nytimes.com}{\nolinkurl{accessibility@nytimes.com}}.
  We would love to hear from you.
\item
  \href{https://www.nytimes.com/ref/membercenter/help/infoservdirectory.html}{Contact
  Us}
\item
  \href{http://www.nytco.com/careers}{Work With Us}
\item
  \href{http://nytmediakit.com/}{Advertise}
\item
  \href{https://www.nytimes.com/content/help/rights/privacy/policy/privacy-policy.html\#pp}{Your
  Ad Choices}
\item
  \href{https://www.nytimes.com/privacy}{Privacy}
\item
  \href{https://www.nytimes.com/ref/membercenter/help/agree.html}{Terms
  of Service}
\item
  \href{https://www.nytimes.com/content/help/rights/sale/terms-of-sale.html}{Terms
  of Sale}
\end{itemize}

\hypertarget{site-information-navigation-1}{%
\subsection{Site Information
Navigation}\label{site-information-navigation-1}}

\begin{itemize}
\tightlist
\item
  \href{http://spiderbites.nytimes.com/}{Site Map}
\item
  \href{https://www.nytimes.com/membercenter/sitehelp.html}{Help}
\item
  \href{https://myaccount.nytimes.com/membercenter/feedback.html}{Site
  Feedback}
\item
  \href{https://www.nytimes.com/subscriptions/Multiproduct/lp5558.html?campaignId=37WXW}{Subscriptions}
\end{itemize}
