Sections

SEARCH

\protect\hyperlink{site-content}{Skip to
content}\protect\hyperlink{site-index}{Skip to site index}

\href{https://myaccount.nytimes.com/auth/login?response_type=cookie\&client_id=vi}{}

\href{https://www.nytimes.com/section/todayspaper}{Today's Paper}

\href{/section/opinion}{Opinion}\textbar{}Por que os brasileiros
deveriam ter medo do gabinete do ódio

\url{https://nyti.ms/2XmiUWk}

\begin{itemize}
\item
\item
\item
\item
\item
\end{itemize}

Advertisement

\protect\hyperlink{after-top}{Continue reading the main story}

\href{/section/opinion}{Opinion}

Supported by

\protect\hyperlink{after-sponsor}{Continue reading the main story}

\hypertarget{por-que-os-brasileiros-deveriam-ter-medo-do-gabinete-do-uxf3dio}{%
\section{Por que os brasileiros deveriam ter medo do gabinete do
ódio}\label{por-que-os-brasileiros-deveriam-ter-medo-do-gabinete-do-uxf3dio}}

O presidente Jair Bolsonaro e seus aliados têm espalhado o ódio online
contra as instituições que protegem a democracia. Agora, essa cólera
está transbordando para as ruas.

Por Patrícia Campos Mello

Campos Mello é uma jornalista brasileira.

\begin{itemize}
\item
  Aug. 4, 2020, 5:00 a.m. ET
\item
  \begin{itemize}
  \item
  \item
  \item
  \item
  \item
  \end{itemize}
\end{itemize}

\includegraphics{https://static01.nyt.com/images/2020/08/05/opinion/05campos/04campos-articleLarge.jpg?quality=75\&auto=webp\&disable=upscale}

\href{https://www.nytimes.com/2020/08/04/opinion/bolsonaro-office-of-hate-brazil.html}{Read
in
English}\href{https://www.nytimes.com/es/2020/08/04/espanol/opinion/bolsonaro-oficina-odio-brasil.html}{Leer
en español}

SÃO PAULO --- Em 13 de junho, integrantes do ``300 do Brasil'', uma
milícia de extrema direita formada por apoiadores do presidente Jair
Bolsonaro,
\href{https://www1.folha.uol.com.br/poder/2020/05/sara-winter-xinga-moraes-diz-querer-trocar-socos-com-ele-e-promete-inferniza-lo.shtml}{atiraram
fogos de artifício em direção ao prédio do Supremo Tribunal Federal em
Brasília}, simulando um bombardeio. ``Se preparem, Supremo dos
bandidos\ldots{}bando de bandidos, vocês estão levando o país para o
caos, para o comunismo'', disse um dos líderes, que
\href{https://www.metropoles.com/brasil/video-bolsonaristas-lancam-fogos-de-artificio-em-predio-do-stf}{transmitiu
o protesto ao vivo em um vídeo}. ``Acabou, porra'', disse um dos
manifestantes,
\href{https://www.youtube.com/watch?v=I2bZoC8FHJI}{ecoando as palavras
usadas por Bolsonaro} ao condenar a investigação do Supremo contra
alguns dos apoiadores do presidente, envolvidos em campanhas de
desinformação e ameaças contra os ministros do STF.

De onde veio esse ódio contra o mais alto tribunal do Brasil?

Durante os meses que antecederam o incidente com os fogos, milhares de
contas nas redes sociais, muitas delas falsas, ligadas a apoiadores de
Bolsonaro ou a blogueiros de extrema direita, vinham postando
obscenidades e
\href{https://www1.folha.uol.com.br/poder/2020/05/sara-winter-xinga-moraes-diz-querer-trocar-socos-com-ele-e-promete-inferniza-lo.shtml}{ameaças
contra os ministros do Supremo}. Alguns apoiadores do presidente até
falaram em matar e
\href{https://g1.globo.com/politica/noticia/2020/06/17/moraes-vota-pela-legalidade-e-continuidade-do-inquerito-das-fake-news.ghtml}{esquartejar
ministros e seus familiares}. Era só uma questão de tempo até essa
animosidade transbordar para as ruas.

Esse ambiente tóxico tem sido estimulado pelo que os brasileiros chamam
de ``gabinete do ódio''. Trata-se de um gabinete de assessores de
Bolsonaro, que apoiam uma rede de blogs bolsonaristas e perfis em redes
sociais que espalham desinformação e ataques contra jornalistas,
políticos, artistas e veículos de imprensa críticos ao presidente. O
gabinete do ódio não é oficial, nem tem um orçamento específico, mas é
bancado com dinheiro público. Não está claro quantas pessoas trabalham
na operação, e nem se sabe quem são todos os envolvidos. Na realidade,
Bolsonaro e seus aliados negam que exista um gabinete do ódio. Mas o
fato é que as sementes do ódio e do sectarismo que vêm sendo espalhadas
são uma ameaça à nossa democracia.

O governo Bolsonaro enfrenta atualmente três investigações diretamente
ligadas a essa máquina do ódio. O Supremo Tribunal Federal está
investigando ataques contra seus ministros, financiados por empresários
e disseminado pela rede bolsonarista. Outro inquérito do Supremo examina
o financiamento dos atos antidemocráticos, protestos pedindo o
fechamento do Congresso e intervenção no Judiciário. Quatro ações correm
no Tribunal Superior Eleitoral investigando o uso de disparos em massa
de WhatsApp na tentativa de influenciar a campanha eleitoral de 2018,
que teriam sido financiados por empresários.

\includegraphics{https://static01.nyt.com/images/2020/08/05/opinion/04campos-PORT-2/04Campos2-articleLarge.jpg?quality=75\&auto=webp\&disable=upscale}

Em 8 de julho, o Facebook
\href{https://www1.folha.uol.com.br/poder/2020/07/facebook-remove-contas-falsas-ligadas-aos-bolsonaros-e-ao-gabinete-da-presidencia.shtml}{removeu
inúmeras contas}, algumas usadas por um funcionário de Bolsonaro e por
seus filhos. Algumas estavam registradas no nome de
\href{https://elpais.com/internacional/2020-07-10/facebook-rompe-la-oficina-del-odio-una-red-de-88-cuentas-de-apoyo-a-jair-bolsonaro.html}{Tércio
Arnaud Tomaz}, assessor de Bolsonaro que é apontado como um dos líderes
do \href{https://apnews.com/0c58cccec2004bf250c8dab38172cbc9}{chamado
gabinete do ódio}.

Infelizmente, eu conheço bem essa máquina do ódio. Nos últimos dois
anos, venho cobrindo o uso da desinformação na política. E acabei me
tornando um dos alvos dessas campanhas desde 2018, quando revelei no
jornal Folha de S.Paulo que empresários estavam pagando pelo envio de
milhões de mensagens por WhatsApp para influenciar a eleição
presidencial daquele ano.

Como resultado, tenho enfrentado uma campanha violenta de ameaças e
ataques pessoais. Trolls e até políticos têm compartilhado memes onde
meu rosto aparece em montagens pornográficas, e se referem a mim como
prostituta. Recebi mensagens de pessoas dizendo que eu deveria ser
estuprada. Estou processando o presidente Bolsonaro; seu filho, o
deputado federal Eduardo Bolsonaro, e um blogueiro bolsonarista
\href{http://www.fundamedios.us/incidentes/patriciacampos-demanda-jairbolsonaro-ofensas-periodista/}{por
danos morais}, por dizerem ou insinuarem que eu ofereço sexo em troca de
reportagens exclusivas.

Não sou a única. Muitas jornalistas têm sido vítimas de ataques
misóginos no Brasil. A imprensa, ao lado do judiciário e do Congresso, é
uma das últimas barreiras contendo medidas autoritárias do presidente.
Mas não sei quanto tempo conseguiremos resistir às pressões de Bolsonaro
e seus apoiadores. A retórica e as ações cada vez mais agressivas do
presidente, seus filhos e aliados funcionam como um sinal verde para que
milícias bolsonaristas partam das palavras para as vias de fato.

Em 25 de maio, jornalistas que cobrem as coletivas improvisadas de
Bolsonaro no palácio do Alvorada mais uma vez foram submetidos a uma
chuva de xingamentos dos apoiadores do presidente.
\href{https://twitter.com/folha/status/1264913877399212034}{Vídeo
daquele dia mostra} os repórteres sendo chamados de ``cambada de
safados'' e ``mídia comprada''. ``Escória! Lixos! Ratos! Ratazanas!
Bolsonaro até 2050! Imprensa podre! Comunistas!''.

Image

Apoiadores do presidente do Brasil, Jair Bolsonaro, em
Brasília.Credit...Eraldo Peres/Associated Press

Jornalistas estão longe de serem os únicos alvos. Ao longo do último
ano, a máquina do ódio vem colocando os brasileiros uns contra os
outros, e contra aqueles que têm funcionado como freios e contrapesos
para os arroubos autoritários de Bolsonaro. Essa máquina vem corroendo a
confiança das pessoas nas instituições concebidas justamente para
protegê-las.

Esse grupo disseminador de ódio está por trás de uma campanha de
difamação que acusava o ex-juiz Sergio Moro, que
\href{https://www.nytimes.com/2017/09/18/opinion/brazil-corruption-car-wash.html?searchResultPosition=1}{liderou
a LavaJato} e foi ministro da justiça de Bolsonaro, de ser ``traidor'' e
``comunista''. Moro pediu demissão em abril,
\href{https://www.nytimes.com/2017/09/18/opinion/brazil-corruption-car-wash.html?searchResultPosition=1}{acusando
Bolsonaro de tentar interferir na Polícia Federal}, que investigava um
dos filhos de Bolsonaro e alguns aliados. Logo depois de ele deixar o
cargo, contas falsas inundaram as redes sociais com memes atacando Moro.

Com a disseminação do coronavírus, fake news e
\href{https://www.bbc.com/news/53361876}{curas supostamente milagrosas}
começaram a proliferar nas redes sociais, muitas vezes compartilhadas
por legisladores com centenas de milhares de seguidores. Bolsonaro tem
\href{https://www.hrw.org/news/2020/04/10/brazil-bolsonaro-sabotages-anti-covid-19-efforts}{sabotado
as medidas de distanciamento social} implementadas por governadores, e
contas ligadas a assessores como Tércio Arnaud
\href{https://www.bbc.com/portuguese/brasil-53353594}{divulgaram que a
reação à Covid-19 era exagerada} e que a cloroquina,
\href{https://www.nytimes.com/2020/06/13/world/americas/virus-brazil-bolsonaro-chloroquine.html}{alardeada
por Bolsonaro como uma cura para o coronavírus}, consegue matar o vírus.

Em abril, o governo começou a divulgar no
\href{https://twitter.com/secomvc/status/1257836970518200323}{Twitter} e
\href{https://www.facebook.com/minsaude/posts/3549590468392877}{Facebook}
o que chama de ``Placar da Vida'', que enumera apenas o número de
pacientes que se recuperaram da Covid-19. Depois, em junho, o Ministério
da Saúde deixou de divulgar o número acumulado de casos e mortes, e
mostrava apenas os registrados nas últimas 24 horas, além de mudar a
maneira de contabilizar. O
\href{https://www.nytimes.com/2020/06/19/world/coronavirus-live-updates.html}{STF
determinou que o governo tinha de voltar a publicar os dados} de forma
integral.

Agendas políticas, no entanto, não conseguem demover o coronavírus. A
\href{https://www.cnn.com/2020/05/23/americas/brazil-coronavirus-hospitals-intl/index.html}{``gripezinha''}
-- termo que Bolsonaro usou para se referir à doença que foi contraída
por ele e sua mulher em julho --
\href{https://www.nytimes.com/interactive/2020/world/americas/brazil-coronavirus-cases.html}{já
matou mais de 92 mil brasileiros}. Trata-se do país com o segundo maior
número de mortes pela doença.

Com as campanhas de difamação e a manipulação de informações, o objetivo
da máquina do ódio é muito mais nefasto -- é enfraquecer as instituições
democráticas do Brasil. Investigações da Procuradoria Geral da República
apontam que alguns legisladores bolsonaristas usaram a cota parlamentar,
dinheiro público, para promover pelas redes sociais protestos contra o
Supremo Tribunal Federal e a favor de intervenção militar.

Esse tipo de incitamento tem como objetivo convencer apoiadores de que
os ministros do Supremo são ditadores, e que o Congresso e a mídia são
golpistas e impedem o presidente de governar. Dessa maneira, o governo
pode estar preparando o terreno para justificar uma intervenção militar.
E em uma democracia jovem como a brasileira, as instituições podem ser
mais frágeis do que aparentam.

Apesar de ter sido eleito democraticamente, Bolsonaro já declarou
repetidamente sua admiração pela ditadura militar que vigorou no Brasil
entre 1964 e 1985. Muito antes de se candidatar à presidência,
\href{https://www.youtube.com/watch?v=qIDyw9QKIvw\&t=577s}{Bolsonaro
disse que só uma guerra civil faria} o trabalho que o regime militar não
havia feito e afirmou que fecharia o Congresso se fosse presidente.
Durante a campanha eleitoral de 2018,
\href{https://congressoemfoco.uol.com.br/especial/noticias/fas-usam-imagem-de-torturador-para-promover-bolsonaro/}{seus
filhos e apoiadores usaram camisetas estampadas com o rosto do coronel
Carlos Alberto Brilhante Ustra}, condenado pela Justiça como um dos
maiores torturadores da ditadura --- uma figura celebrada pelo
presidente.

Bolsonaro tem tentado implementar essa visão. Para contornar o
Congresso, ele assinou um número recorde de medidas provisórias que
reduziam a independência das universidades, que ele considera antros de
comunistas, restringir acesso à informação, enfraquecer sindicatos e
jornais. Ele também ameaçou desobedecer ordens judiciais.

Image

Um manifestante em Rio de Janeiro.Credit...Bruna Prado/Getty Images

O presidente afirmou que seu objetivo era armar toda a população, para
que as pessoas pudessem se defender da ``ditadura'' dos governadores e
do Supremo. ``Eu quero todo mundo armado! Que o povo armado jamais será
escravizado'', disse o presidente em uma reunião ministerial em maio.
Depois da reunião, ele assinou um decreto que facilitou a importação de
armas e aumentou a quantidade de munição que as pessoas podem comprar
por ano. Em qualquer democracia, isso seria considerado um chamado à
insurreição.

O presidente e seus aliados adorariam silenciar aqueles que tentam
investigar as facetas obscuras de seu governo.

No último ano, o objetivo do gabinete do ódio tem ficado cada vez mais
claro: fazer os brasileiros se voltarem contra aqueles que têm servido
de freios e contrapesos às medidas autoritárias de Bolsonaro.

Ataques como o ocorrido contra o Supremo e como a recente agressão
contra um fotojornalista em um protesto contra o Congresso são uma
amostra de que a máquina do ódio tem sido, de certa maneira, bem
sucedida em seu chamado à insurreição.

Na última quarta-feira, dois homens acompanhados de um carro de som
foram até a entrada
\href{https://esportes.yahoo.com/noticias/aliados-jair-bolsonaro-atacam-casa-felipe-neto-010129218.html}{do
condomínio onde mora Felipe Neto} e o acusaram de destruir ``a
instituição mais importante de todas, que é a família'', em uma
tentativa de intimidação ao famoso youtuber. Um dos homens que ameaçou
Felipe Neto havia participado do ataque com fogos de artifício contra o
prédio do Supremo em Brasília. Alguns dias antes, esta seção havia
veiculado um vídeo em que Felipe Neto chamava a Bolsonaro
\href{https://www.nytimes.com/2020/07/15/opinion/coronavirus-covid-brazil-bolsonaro.html}{``o
pior presidente da pandemia''.} O ataque é mais um exemplo de que, cada
vez mais, o fel propagado pela máquina do ódio está se espalhando para
além da internet.

\includegraphics{https://static01.nyt.com/images/2020/07/16/autossell/15op-brazil-thumb-print/15op-brazil-thumb-videoSixteenByNineJumbo1600.jpg}

Bolsonaro e seus aliados adorariam nos intimidar e silenciar. Mas para
que haja esperança de que nossa democracia vai resistir, precisamos nos
manter vigilantes e garantir que o governo seja responsabilizado por
suas ações. Não apenas as vidas de brasileiros estão em jogo, mas as
próprias instituições -- Congresso, Judiciário, mundo acadêmico e a
imprensa -- que, por enquanto, têm conseguido brecar a ascensão do
autoritarismo.

Patrícia Campos Mello é jornalista do jornal brasileiro Folha de S.Paulo
e autora da ``Máquina do Ódio'', sobre campanhas de desinformação e
Bolsonaro.

Advertisement

\protect\hyperlink{after-bottom}{Continue reading the main story}

\hypertarget{site-index}{%
\subsection{Site Index}\label{site-index}}

\hypertarget{site-information-navigation}{%
\subsection{Site Information
Navigation}\label{site-information-navigation}}

\begin{itemize}
\tightlist
\item
  \href{https://help.nytimes.com/hc/en-us/articles/115014792127-Copyright-notice}{©~2020~The
  New York Times Company}
\end{itemize}

\begin{itemize}
\tightlist
\item
  \href{https://www.nytco.com/}{NYTCo}
\item
  \href{https://help.nytimes.com/hc/en-us/articles/115015385887-Contact-Us}{Contact
  Us}
\item
  \href{https://www.nytco.com/careers/}{Work with us}
\item
  \href{https://nytmediakit.com/}{Advertise}
\item
  \href{http://www.tbrandstudio.com/}{T Brand Studio}
\item
  \href{https://www.nytimes.com/privacy/cookie-policy\#how-do-i-manage-trackers}{Your
  Ad Choices}
\item
  \href{https://www.nytimes.com/privacy}{Privacy}
\item
  \href{https://help.nytimes.com/hc/en-us/articles/115014893428-Terms-of-service}{Terms
  of Service}
\item
  \href{https://help.nytimes.com/hc/en-us/articles/115014893968-Terms-of-sale}{Terms
  of Sale}
\item
  \href{https://spiderbites.nytimes.com}{Site Map}
\item
  \href{https://help.nytimes.com/hc/en-us}{Help}
\item
  \href{https://www.nytimes.com/subscription?campaignId=37WXW}{Subscriptions}
\end{itemize}
