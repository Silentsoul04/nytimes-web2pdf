Sections

SEARCH

\protect\hyperlink{site-content}{Skip to
content}\protect\hyperlink{site-index}{Skip to site index}

\href{https://myaccount.nytimes3xbfgragh.onion/auth/login?response_type=cookie\&client_id=vi}{}

\href{https://www.nytimes3xbfgragh.onion/section/todayspaper}{Today's
Paper}

Archives\textbar{}Battling Harvard Law Over Diversity

\begin{itemize}
\item
\item
\item
\item
\item
\end{itemize}

Advertisement

\protect\hyperlink{after-top}{Continue reading the main story}

Supported by

\protect\hyperlink{after-sponsor}{Continue reading the main story}

\hypertarget{battling-harvard-law-over-diversity}{%
\section{Battling Harvard Law Over
Diversity}\label{battling-harvard-law-over-diversity}}

By Sharon Cotliar,

\begin{itemize}
\item
  March 6, 1992
\item
  \begin{itemize}
  \item
  \item
  \item
  \item
  \item
  \end{itemize}
\end{itemize}

\includegraphics{https://s1.graylady3jvrrxbe.onion/timesmachine/pages/1/1992/03/06/front_page_360W.png?quality=75\&auto=webp\&disable=upscale}

See the article in its original context from\\
March 6, 1992, Section B, Page
8\href{https://store.nytimes3xbfgragh.onion/collections/new-york-times-page-reprints?utm_source=nytimes\&utm_medium=article-page\&utm_campaign=reprints}{Buy
Reprints}

\href{http://timesmachine.nytimes3xbfgragh.onion/timesmachine/1992/03/06/004692.html}{View
on timesmachine}

TimesMachine is an exclusive benefit for home delivery and digital
subscribers.

About the Archive

This is a digitized version of an article from The Times's print
archive, before the start of online publication in 1996. To preserve
these articles as they originally appeared, The Times does not alter,
edit or update them.

Occasionally the digitization process introduces transcription errors or
other problems; we are continuing to work to improve these archived
versions.

Two third-year Harvard law students, Caroline Wittcoff and Laura
Hankins, have asked the state's highest court to allow them to bring
charges of employment discrimination against the law school, though they
are not employees of Harvard.

The setting was unusual for students, who customarily present their
cases in classrooms. In this instance, their presentation was before the
Massachusetts Supreme Judicial Court.

The students are members of an organization that filed a lawsuit in
1990. The group, the Harvard Law School Coalition for Civil Rights,
seeks to bring more racial and cultural diversity to the school's
faculty.

The suit was rejected by a lower court, which ruled that the students
had no standing to sue Harvard over an issue of employment
discrimination. But the state's high court decided to review that
decision.

On Tuesday, portraying their case as another fight in the continuing
struggle for civil rights, the two law students urged the panel of five
justices to allow the suit to go to trial. They contend that the law
school's hiring practices are discriminatory and that students'
education has been harmed by a resulting lack of diversity.

Under state and Federal laws, only employees or people whose job
applications are rejected can bring such a suit. But the students say
that Massachusetts anti-discrimination laws and the state's Equal Rights
Act give them the right to challenge discriminatory hiring practices at
the university because of its promise of equal education opportunities.

The university has denied the students' accusations and argues that the
issues raised in the suit are essentially about diversity, a debate,
Harvard says, that does not belong in the courts.

In their brief appearance before the high court, the students argued
that Brown v. the Board of Education, the landmark school desegregation
case, had established the "insidiousness of discrimination in education
and the legally cognizable harms that students suffer." Ms. Wittcoff,
who is white, argued that students are similarly injured when they are
denied the benefits of learning from an integrated faculty.

To try to establish those injuries, Ms. Wittcoff argued that women and
minority students are "stamped with a badge of inferiority" when those
like them are passed over for professorships and when such students are
denied the benefits of role models that their white male counterparts
have.

Ms. Hankins, a black student, compared the role of women and minorities
at Harvard Law School with that of Rosa Parks in asserting the rights of
blacks. "When Rosa Parks said she wanted to ride in the front of the
bus, everybody wanted to know what she was complaining about," Ms.
Hankins said, speaking of the woman whose actions inspired the
Montgomery, Ala., bus boycott in 1955. "She got where she wanted to go.
She got to ride the bus.

"Today we say that we are allowed to sit in the back of the classroom,
but we receive the message that we will never stand up front."

The lawyer representing Harvard, Allan Ryan Jr., argued that the
students' suit could not be compared with civil rights disputes. "This
case is not Brown v. Board of Education," he said. "And Rosa Parks is
not in this courtroom."

He dismissed the students' assertions that they had a right to bring
charges of employment discrimination, saying that state and Federal law
gives only employees and prospective employees the right to bring such
suits. "There is no actual case of discrimination that has been filed by
a faculty member or candidate," Mr. Ryan said. He suggested that the
students' argument is simply a diversity debate, saying, "Students'
argument is that Harvard Law School is too male, too white and too
heterosexual."

Of the law school's 64 professors with tenure or positions that lead to
tenure, five are women and six are black men. There is no tenured
professor who is a black woman, no Hispanic professor, no openly
homosexual or lesbian professors and none who are physically
handicapped.

Women represent 7 percent of the law school's tenured faculty, and
minority members represent 9 percent. Of the school's 1,620 students, 45
percent are women and 22 percent are minority members.

In a 1991 survey of 175 law schools, the American Bar Association found
that minorities represented 9.5 percent of all full-time professors,
which includes visiting professors and those without tenure, and that
women represented 25 percent of the total.

Mr. Ryan rejected statistical comparisons because of the school's
relatively small faculty. If Harvard Law School hired even one member of
the groups mentioned in the students' suit, he said, the representation
of that group on the law school faculty would equal or surpass the
national average.

The court is not expected to make a ruling for at least a month.

Advertisement

\protect\hyperlink{after-bottom}{Continue reading the main story}

\hypertarget{site-index}{%
\subsection{Site Index}\label{site-index}}

\hypertarget{site-information-navigation}{%
\subsection{Site Information
Navigation}\label{site-information-navigation}}

\begin{itemize}
\tightlist
\item
  \href{https://help.nytimes3xbfgragh.onion/hc/en-us/articles/115014792127-Copyright-notice}{©~2020~The
  New York Times Company}
\end{itemize}

\begin{itemize}
\tightlist
\item
  \href{https://www.nytco.com/}{NYTCo}
\item
  \href{https://help.nytimes3xbfgragh.onion/hc/en-us/articles/115015385887-Contact-Us}{Contact
  Us}
\item
  \href{https://www.nytco.com/careers/}{Work with us}
\item
  \href{https://nytmediakit.com/}{Advertise}
\item
  \href{http://www.tbrandstudio.com/}{T Brand Studio}
\item
  \href{https://www.nytimes3xbfgragh.onion/privacy/cookie-policy\#how-do-i-manage-trackers}{Your
  Ad Choices}
\item
  \href{https://www.nytimes3xbfgragh.onion/privacy}{Privacy}
\item
  \href{https://help.nytimes3xbfgragh.onion/hc/en-us/articles/115014893428-Terms-of-service}{Terms
  of Service}
\item
  \href{https://help.nytimes3xbfgragh.onion/hc/en-us/articles/115014893968-Terms-of-sale}{Terms
  of Sale}
\item
  \href{https://spiderbites.nytimes3xbfgragh.onion}{Site Map}
\item
  \href{https://help.nytimes3xbfgragh.onion/hc/en-us}{Help}
\item
  \href{https://www.nytimes3xbfgragh.onion/subscription?campaignId=37WXW}{Subscriptions}
\end{itemize}
