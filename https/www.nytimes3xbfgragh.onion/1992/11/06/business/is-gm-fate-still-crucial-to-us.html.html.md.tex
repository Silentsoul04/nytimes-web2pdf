Sections

SEARCH

\protect\hyperlink{site-content}{Skip to
content}\protect\hyperlink{site-index}{Skip to site index}

\href{https://www.nytimes3xbfgragh.onion/section/business}{Business}

\href{https://myaccount.nytimes3xbfgragh.onion/auth/login?response_type=cookie\&client_id=vi}{}

\href{https://www.nytimes3xbfgragh.onion/section/todayspaper}{Today's
Paper}

\href{/section/business}{Business}\textbar{}Is G.M. Fate Still Crucial
to U.S.?

\url{https://nyti.ms/29bnOw5}

\begin{itemize}
\item
\item
\item
\item
\item
\end{itemize}

Advertisement

\protect\hyperlink{after-top}{Continue reading the main story}

Supported by

\protect\hyperlink{after-sponsor}{Continue reading the main story}

\hypertarget{is-gm-fate-still-crucial-to-us}{%
\section{Is G.M. Fate Still Crucial to
U.S.?}\label{is-gm-fate-still-crucial-to-us}}

By Peter Passell

\begin{itemize}
\item
  Nov. 6, 1992
\item
  \begin{itemize}
  \item
  \item
  \item
  \item
  \item
  \end{itemize}
\end{itemize}

\includegraphics{https://s1.graylady3jvrrxbe.onion/timesmachine/pages/1/1992/11/06/243892_360W.png?quality=75\&auto=webp\&disable=upscale}

See the article in its original context from\\
November 6, 1992, Section D, Page
1\href{https://store.nytimes3xbfgragh.onion/collections/new-york-times-page-reprints?utm_source=nytimes\&utm_medium=article-page\&utm_campaign=reprints}{Buy
Reprints}

\href{http://timesmachine.nytimes3xbfgragh.onion/timesmachine/1992/11/06/243892.html}{View
on timesmachine}

TimesMachine is an exclusive benefit for home delivery and digital
subscribers.

About the Archive

This is a digitized version of an article from The Times's print
archive, before the start of online publication in 1996. To preserve
these articles as they originally appeared, The Times does not alter,
edit or update them.

Occasionally the digitization process introduces transcription errors or
other problems; we are continuing to work to improve these archived
versions.

"What is good for the country is good for General Motors, and what's
good for General Motors is good for the country," Charles E. Wilson, the
G.M. president who was to become President Eisenhower's Secretary of
Defense, is said to have told the Senate Armed Forces Committee. The
pronouncement by Engine Charlie Wilson instantly became a symbol of
corporate arrogance for a nation that simultaneously loathed and
worshiped big business, for there was rough truth in his words.

America was the world's largest consumer of cars, and G.M. was the
world's largest producer, a mighty engine of prosperity for G.M.
stockholders and the families and communities of workers.

Four decades later, with G.M. hemorrhaging jobs and cash, its love-hate
affair with America may be ending. Although almost everyone wishes its
new managers well, few economists think there is cause for great alarm
if they fail to restore the company's profitability or market share.
Moving On

"A unique strength of this economy is its capacity to acknowledge
failure and move on," said Alfred E. Kahn, a specialist in industrial
organization at Cornell University.

Irwin Stelzer, an economist at the American Enterprise Institute, added,
"Even in bankruptcy, G.M. would not disappear." One way or another, its
productive plants and accumulated expertise in engineering and marketing
would be recycled.

But isn't the decline of the corporation that symbolized America's
postwar industrial dominance a prospect that should be viewed with
exceptional anxiety? After all, at the end of 1991 some 370,000 people
were on G.M.'s \$12 billion North American payroll, while the
livelihoods of hundreds of thousands more were indirectly linked to the
company. More pointedly, if G.M. reaches the brink, should Uncle Sugar
step in to save it, as it saved Chrysler in 1980?

F. M. Scherer, an economist at Harvard's John F. Kennedy School of
Government, cautiously offers a rationale for treating G.M. as a special
case. While the company's jobs are widely distributed across the
continent, he notes that big cutbacks would still be "a very severe blow
to the already badly damaged city of Detroit." Moreover, the timing of
the job implosion, in the midst of a long recession, could hardly be
worse. Mr. Scherer also expects some jobs lost at G.M. to go overseas.
And he worries that a failure by the company in the car business may
mean a permanent loss of the sorts of highly skilled, highly paid jobs
that Americans need to remain prosperous. But many other economists are
unpersuaded. Mr. Stelzer says the focus on the fortunes of the corporate
entity misses the mark. G.M. is certain to shed payrolls and plants, he
notes, whether or not the new managers succeed. And presumably, some of
those losses would be gains for workers at other American companies
making more popular models.

Paradoxically, an unsuccessful salvage effort -\/- one in which G.M.
fails to cut its payroll and capacity fast enough, and slowly bleeds its
way into bankruptcy -\/- might be least traumatic for most of the
workers and communities. They would, at least, have longer to make the
transition.

Since there is excess production capacity in the American car industry
-\/- and, for that matter, in the world's car industry -\/- the number
of employees in the business is almost sure to continue to slip. But
many analysts think the location of the highly-paid white collar jobs,
the jobs that Mr. Scherer worries about most, is no longer tightly
linked to the location of manufacturing operations.

What apparently counts most in the good jobs equation is the quality of
education and other amenities -\/- safe streets, clean air, good
transportation -\/- that attract what Robert B. Reich of Harvard calls
"knowledge workers." Seen in this light, it should be no surprise that
Toyota and Mazda are designing cars in the United States for production
worldwide.

Of course, many of the good jobs at risk at G.M. and in the rest of the
auto industry are semiskilled, where high pay is a matter of inertia and
strong unions -\/- much as it is in the steel, rubber and heavy
machinery industries. Last year, G.M. spent \$34.60 an hour for labor,
including benefits, roughly twice the average total compensation for the
economy as a whole. But Robert Crandall of the Brookings Institution
argues that high pay in autos and steel has always been a mixed
blessing, since it translates into higher prices for consumers. Indeed,
Mr. Crandall contends that the high pay-high price trade-off has been a
rotten deal for the economy. He estimates that the effort in the 1980's
to protect blue-collar jobs with import quotas cost car buyers tens of
billions of dollars, with much of the money lining the pockets of
foreign manufacturers who imported higher-priced, more profitable
models.

A common theme running through much of the nascent debate about the
public's stake in G.M.'s future is the tension between allowing
competitive markets to discipline business, and sparing the innocent the
consequences of corporate incompetence. Many economists, including many
conservatives, would gladly strengthen the safety net for the
underskilled and unemployed.

By the same token, most economists are deeply skeptical about the
impulse to reverse the sometimes-harsh judgment of free markets, even
when the potential victims are icons of America's industrial might.

The economist Milton Friedman, a Nobel Prize winner, concluded, "If you
get into the game of too-big-to-fail, you get into real trouble."

Advertisement

\protect\hyperlink{after-bottom}{Continue reading the main story}

\hypertarget{site-index}{%
\subsection{Site Index}\label{site-index}}

\hypertarget{site-information-navigation}{%
\subsection{Site Information
Navigation}\label{site-information-navigation}}

\begin{itemize}
\tightlist
\item
  \href{https://help.nytimes3xbfgragh.onion/hc/en-us/articles/115014792127-Copyright-notice}{©~2020~The
  New York Times Company}
\end{itemize}

\begin{itemize}
\tightlist
\item
  \href{https://www.nytco.com/}{NYTCo}
\item
  \href{https://help.nytimes3xbfgragh.onion/hc/en-us/articles/115015385887-Contact-Us}{Contact
  Us}
\item
  \href{https://www.nytco.com/careers/}{Work with us}
\item
  \href{https://nytmediakit.com/}{Advertise}
\item
  \href{http://www.tbrandstudio.com/}{T Brand Studio}
\item
  \href{https://www.nytimes3xbfgragh.onion/privacy/cookie-policy\#how-do-i-manage-trackers}{Your
  Ad Choices}
\item
  \href{https://www.nytimes3xbfgragh.onion/privacy}{Privacy}
\item
  \href{https://help.nytimes3xbfgragh.onion/hc/en-us/articles/115014893428-Terms-of-service}{Terms
  of Service}
\item
  \href{https://help.nytimes3xbfgragh.onion/hc/en-us/articles/115014893968-Terms-of-sale}{Terms
  of Sale}
\item
  \href{https://spiderbites.nytimes3xbfgragh.onion}{Site Map}
\item
  \href{https://help.nytimes3xbfgragh.onion/hc/en-us}{Help}
\item
  \href{https://www.nytimes3xbfgragh.onion/subscription?campaignId=37WXW}{Subscriptions}
\end{itemize}
