Sections

SEARCH

\protect\hyperlink{site-content}{Skip to
content}\protect\hyperlink{site-index}{Skip to site index}

\href{https://www.nytimes3xbfgragh.onion/section/us}{U.S.}

\href{https://myaccount.nytimes3xbfgragh.onion/auth/login?response_type=cookie\&client_id=vi}{}

\href{https://www.nytimes3xbfgragh.onion/section/todayspaper}{Today's
Paper}

\href{/section/us}{U.S.}\textbar{}Tobacco Chiefs Say Cigarettes Aren't
Addictive

\href{https://nyti.ms/2987lbK}{https://nyti.ms/2987lbK}

\begin{itemize}
\item
\item
\item
\item
\item
\end{itemize}

Advertisement

\protect\hyperlink{after-top}{Continue reading the main story}

Supported by

\protect\hyperlink{after-sponsor}{Continue reading the main story}

\hypertarget{tobacco-chiefs-say-cigarettes-arent-addictive}{%
\section{Tobacco Chiefs Say Cigarettes Aren't
Addictive}\label{tobacco-chiefs-say-cigarettes-arent-addictive}}

By Philip J. Hilts

\begin{itemize}
\item
  April 15, 1994
\item
  \begin{itemize}
  \item
  \item
  \item
  \item
  \item
  \end{itemize}
\end{itemize}

\includegraphics{https://s1.graylady3jvrrxbe.onion/timesmachine/pages/1/1994/04/15/784079_360W.png?quality=75\&auto=webp\&disable=upscale}

See the article in its original context from\\
April 15, 1994, Section A, Page
1\href{https://store.nytimes3xbfgragh.onion/collections/new-york-times-page-reprints?utm_source=nytimes\&utm_medium=article-page\&utm_campaign=reprints}{Buy
Reprints}

\href{http://timesmachine.nytimes3xbfgragh.onion/timesmachine/1994/04/15/784079.html}{View
on timesmachine}

TimesMachine is an exclusive benefit for home delivery and digital
subscribers.

About the Archive

This is a digitized version of an article from The Times's print
archive, before the start of online publication in 1996. To preserve
these articles as they originally appeared, The Times does not alter,
edit or update them.

Occasionally the digitization process introduces transcription errors or
other problems; we are continuing to work to improve these archived
versions.

The top executives of the seven largest American tobacco companies
testified in Congress today that they did not believe that cigarettes
were addictive, but that they would rather their own children did not
smoke.

The executives, sitting side by side at a conference table in what
seemed to many a counterpoint to the growing antismoking sentiment in
Congress, faced more than six hours of sharp questioning by members of
the House Energy and Commerce Subcommittee on Health and the
Environment.

Under persistent questioning, each of the executives agreed to give
Congress extensive, previously unpublicized research on humans and
animals that their companies had done concerning nicotine and addiction.

Democratic Congressmen on the panel, inspired by recent news reports,
pressed the executives on whether their companies manipulated the
content of nicotine to keep smokers addicted to cigarettes. The
executives acknowledged that nicotine levels could be and were
controlled by altering the blends of tobacco, but they said this was
done to enhance flavor, not to insure addiction.

The executives also made a number of other notable admissions, including
these:

*Cigarettes may cause lung cancer, heart disease and other health
problems, but the evidence is not conclusive.

*Despite earlier denials, a Philip Morris study that suggested that
animals could become addicted to nicotine was suppressed in 1983 and
1985.

The hearing was televised live by the Cable News Network and C-Span
cable channels, as an overflow crowd stood or sat in the hallways of the
Rayburn House Office Building for what several members of Congress said
marked a high tide of anti smoking sentiment.

The executives seemed to agree, saying that they felt besieged and that
the sweep of antitobacco fervor in recent months had led them to fear
that the Government would try to ban cigarettes.

What the "antitobacco industry wants is prohibition," said James W.
Johnston, chairman and chief executive of R. J. Reynolds. "We hear about
the addiction and the threat. If cigarettes are too dangerous to be
sold, then ban them. Some smokers will obey the law, but many will not.
People will be selling cigarettes out of the trunks of cars, cigarettes
made by who knows who, made of who knows what."

Among the most significant statements by the executives were those that
confirmed that tobacco companies could control the amount of nicotine in
cigarettes by varying the types of tobacco and the parts of the tobacco
plant that were used in a particular blend. They said a number of their
cigarettes, primarily low-tar brands, did use high-nicotine blends,
which gave more nicotine to the smoker than the cigarettes might have
otherwise given. They use these blends for flavor, they explained.

On the Reynolds company's widely criticized use of the cartoon figure
Joe Camel to promote its Camel brand, Mr. Johnston of Reynolds
apologized for an ad that recommended that young men seeking dates at
the beach drag women from the water, pretending to save them from
drowning.

"That ad ran once," he said. "It never should have run. I apologize. It
was offensive. It was stupid. We do make mistakes." Concerned About
Fires

While most of the exchanges focused on the health risks of cigarettes,
the executives were also asked about other risks posed by their
products, like fire. The president and chief executive of Philip Morris,
William I. Campbell, was asked about the feasibility of making
cigarettes whose paper tubes would pose less danger of starting fires.
Cigarette companies have said this type of cigarette would be difficult
to draw smoke through and would taste bad. Representative Albert R.
Wyden, Democrat of Oregon, noted that the Virginia Slims brand was
considered less of a fire hazard than others, and he asked Mr. Campbell,
whose company makes the brand, if a Virginia Slim was difficult to
smoke.

"As a matter of fact, it is too hard to smoke, and doesn't taste very
good," snapped Mr. Campbell. He said the company had been unable to make
a commercially acceptable and fire-safe cigarette.

Pressed by the subcommittee's chairman, Henry A. Waxman, Democrat of
California, and by Representatives Wyden and Mike Synar, Democrat of
Oklahoma, the companies agreed to supply many private company papers,
including all the research on humans and animals concerning nicotine and
addiction, all the market research and internal memoranda on Reynolds'
Joe Camel advertising campaign and all the research done by the Philip
Morris researcher whose scientific paper on addiction was blocked from
publication by company executives.

At one point during the hearing, Mr. Wyden presented a stack of data
from medical groups and a 1989 Surgeon General's report on the perils of
smoking, asking each executive in turn if he believed that cigarettes
were addictive. Each answered no.

When Mr. Johnston said that all products, from cola to Twinkies, had
risks associated with them, Mr. Waxman replied, "Yes, but the difference
between cigarettes and Twinkies is death."

"How many smokers die each year from cancer?" Mr. Waxman asked Mr.
Johnston

"I do not know how many," he replied, adding that estimates of death are
"generated by computers and are only statistical."

Mr. Waxman asked, "Does smoking cause heart disease?"

"It may," Mr. Johnston said.

"Does it cause lung cancer?"

"It may."

"Emphysema?

"It may."

The list continued through several other ailments. Mr. Waxman asked
Andrew H. Tisch, the chairman and chief executive of the Lorillard
Tobacco Company whether he knew that cigarettes caused cancer. "I do not
believe that," Mr. Tisch answered.

"Do you understand how isolated you are from the scientific community in
your belief?" Mr. Waxman asked.

"I do, sir," Mr. Tisch said. Advice to Children

Although each of the six executives who have children said he would
prefer that his own children not smoke, several added that they would
give no advice to their children but let them decide on their own.

Mr. Johnston of Reynolds, who is a smoker himself, said his daughter did
not smoke and "my preference is, she wouldn't smoke."

Earlier this year the Food and Drug Administration said it believed that
it had the authority to regulate cigarettes as drugs if it could
determine that cigarettes were addictive. The agency said there was
evidence that the companies intentionally controlled the amount of
nicotine in cigarettes to maintain their addictive potential. Dr. David
A. Kessler, the agency commissioner, said there was already much
evidence on the first point and some evidence showing that the companies
purposely maintained nicotine levels in cigarettes. Today's hearings
were partly a response to his request that Congress debate the matter
and give the agency guidance.

A bill has also been introduced in the House to regulate tobacco sales
and cigarette composition, while another bill would ban cigarette
smoking in all buildings nationwide that were visited by more than 10
people per week. That bill gained surprising support from the trade
association for chain restaurants, whose members would have to ban
smoking in their businesses.

In his testimony, Mr. Campbell of Philip Morris admitted twice stopping
publication of a study, in 1983 and 1985, that showed that laboratory
animals could be conditioned to press levers repeatedly to get nicotine,
the sort of study that is key to proving that a drug is addictive. He
also agreed to waive the secrecy agreement that has kept a former
company researcher, Dr. Victor DeNoble, from publicly discussing his
work.

A Lorillard executive, Dr. Alexander Spears, admitted, when pressed in
the hearing, that data he gave to Congress three weeks ago showing a
drop in the amount of nicotine in cigarettes since 1982 was wrong. The
chart he presented then before the same subcommittee showed a 10 percent
drop in nicotine, when in fact the Surgeon General's report from which
the data were taken showed an increase of the total nicotine in
cigarettes by more than 10 percent.

Asked after the hearing how the error was made, Dr. Spears said, "I
don't know."

Senior F.D.A. officials and strong opponents of tobacco in Congress have
said that they do not want to ban cigarettes outright but that some way
ought to be found to regulate them to lessen the health and safety
dangers that they pose. Some tobacco company executives, who asked not
to be named said privately that they could accept some regulation. This
hearing, they said, may be the opening of a discussion about the future
for cigarettes in the United States. Possible Regulations

Among the possibilities suggested by F.D.A. officials, members of
Congress and tobacco company executives were limits on the amounts of
nicotine and tar permitted in cigarettes, stricter control over the
distribution of cigarettes and efforts to reduce other hazards, like the
risk of fires and of secondhand tobacco smoke.

It is unclear how much sentiment there will be for such moves in
Congress, but a vote later this year on Mr. Waxman's bill to limit
smoking in public places is to be considered by the health and
environment subcommittee, which includes many members from
tobacco-producing states and several members who are swing votes. Its
fate may indicate the strength of the sentiment for reform,
Congressional aides said. The Witnesses, Identified

Pictured on page 1, from left to right, are these top executives of the
seven leading American tobacco companies: * Donald S. Johnston,
president and chief executive of American Tobacco Company * Thomas
Sandelur Jr., chairman and chief executive of Brown and William Tobacco
Corporation * Edward A. Horrigan, chairman and chief executive of
Liggett Group Inc. * Andrew H. Tisch, chief executive Lorillard Tobacco
Company * Joseph Taddeo, president of United States Tobacco Company *
James W. Johnston, chief executive of R. J. Reynolds * William I.
Campbell, chief executive of Philip Morris

Advertisement

\protect\hyperlink{after-bottom}{Continue reading the main story}

\hypertarget{site-index}{%
\subsection{Site Index}\label{site-index}}

\hypertarget{site-information-navigation}{%
\subsection{Site Information
Navigation}\label{site-information-navigation}}

\begin{itemize}
\tightlist
\item
  \href{https://help.nytimes3xbfgragh.onion/hc/en-us/articles/115014792127-Copyright-notice}{©~2020~The
  New York Times Company}
\end{itemize}

\begin{itemize}
\tightlist
\item
  \href{https://www.nytco.com/}{NYTCo}
\item
  \href{https://help.nytimes3xbfgragh.onion/hc/en-us/articles/115015385887-Contact-Us}{Contact
  Us}
\item
  \href{https://www.nytco.com/careers/}{Work with us}
\item
  \href{https://nytmediakit.com/}{Advertise}
\item
  \href{http://www.tbrandstudio.com/}{T Brand Studio}
\item
  \href{https://www.nytimes3xbfgragh.onion/privacy/cookie-policy\#how-do-i-manage-trackers}{Your
  Ad Choices}
\item
  \href{https://www.nytimes3xbfgragh.onion/privacy}{Privacy}
\item
  \href{https://help.nytimes3xbfgragh.onion/hc/en-us/articles/115014893428-Terms-of-service}{Terms
  of Service}
\item
  \href{https://help.nytimes3xbfgragh.onion/hc/en-us/articles/115014893968-Terms-of-sale}{Terms
  of Sale}
\item
  \href{https://spiderbites.nytimes3xbfgragh.onion}{Site Map}
\item
  \href{https://help.nytimes3xbfgragh.onion/hc/en-us}{Help}
\item
  \href{https://www.nytimes3xbfgragh.onion/subscription?campaignId=37WXW}{Subscriptions}
\end{itemize}
