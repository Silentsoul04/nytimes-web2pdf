Sections

SEARCH

\protect\hyperlink{site-content}{Skip to
content}\protect\hyperlink{site-index}{Skip to site index}

\href{https://www.nytimes3xbfgragh.onion/section/arts}{Arts}

\href{https://myaccount.nytimes3xbfgragh.onion/auth/login?response_type=cookie\&client_id=vi}{}

\href{https://www.nytimes3xbfgragh.onion/section/todayspaper}{Today's
Paper}

\href{/section/arts}{Arts}\textbar{}Restaurants

\begin{itemize}
\item
\item
\item
\item
\item
\end{itemize}

Advertisement

\protect\hyperlink{after-top}{Continue reading the main story}

Supported by

\protect\hyperlink{after-sponsor}{Continue reading the main story}

\hypertarget{restaurants}{%
\section{Restaurants}\label{restaurants}}

By Ruth Reichl

\begin{itemize}
\item
  Feb. 23, 1996
\item
  \begin{itemize}
  \item
  \item
  \item
  \item
  \item
  \end{itemize}
\end{itemize}

See the article in its original context from\\
February 23, 1996, Section C, Page
26\href{https://store.nytimes3xbfgragh.onion/collections/new-york-times-page-reprints?utm_source=nytimes\&utm_medium=article-page\&utm_campaign=reprints}{Buy
Reprints}

\href{http://timesmachine.nytimes3xbfgragh.onion/timesmachine/1996/02/23/063177.html}{View
on timesmachine}

TimesMachine is an exclusive benefit for home delivery and digital
subscribers.

About the Archive

This is a digitized version of an article from The Times's print
archive, before the start of online publication in 1996. To preserve
these articles as they originally appeared, The Times does not alter,
edit or update them.

Occasionally the digitization process introduces transcription errors or
other problems; we are continuing to work to improve these archived
versions.

What do New Yorkers want in their restaurants? The proprietors of Gotham
Bar and Grill have spent 11 years studying the question and they seem to
have the answer down pat.

To begin, there's the look of the restaurant, a vast comfortable space
with a lively bar along one side. The high-ceilinged room stretches out,
as open as a large theater. And yet, through some trick of design, each
table offers intimacy: seated, you have the sense of watching without
being watched.

Then there is the unthreatening ambiance that invites people to come as
they are. You could walk in wearing blue jeans or a tuxedo and feel
entirely comfortable. Starched men in suits mingle with rumpled guys in
blue jeans and nobody looks out of place. It is a very appealing
quality.

The service is a big part of this equation. The people who work at
Gotham Bar and Grill have figured out how to make Americans feel at ease
in the presence of fancy food. This is harder than it sounds; there are
no rules and different people have different expectations. At Gotham,
the waiters take their cues from the customer, tuning in to almost
imperceptible vibrations. Give the waiter the slightest hint that you
are interested and he will describe the food in loving detail. Indicate
that you are not and your order will be taken without a word. The
waiters are similarly sensitive to your wishes about wine; they know the
list but don't press their information. I've had fast meals when I was
in a hurry, slow ones when I was not. The result is that each meal has
left me feeling remarkably well cared for.

Finally, there's the food. The chef, Alfred Portale, became famous for
his vertical dishes. The presentation is stunning, but to focus on that
is to miss the point. Mr. Portale is working with an architecture of
flavor, composing his dishes so that each element contributes something
vital. The vegetables are not an afterthought but an integral part of
each offering. His food may seem modern but it is almost classic in its
balance.

Mr. Portale's signature dish is his seafood salad, a spiral of scallops,
squid, octopus, lobster and avocado that swirls onto the plate like a
mini-tornado. Dressed only in lemon and olive oil, it depends on the
flavors of the seafood itself; the textures are underlined by the smooth
avocado, crisp leaves of lettuce and crunchy bits of flying-fish roe.
His octopus salad has a similar look, but the swirl is composed of
chickpeas, new potatoes, charred tomatoes and leeks, so the intensity of
the textures and flavors is heightened.

Mr. Portale makes a fine foie gras and guinea hen terrine, the flavors
piqued by a green lentil salad. His tuna tartar is lively with lime,
scallions and ginger. He clearly has fun with the pastas, which change
daily. One day it was a perfect bowl of linguine with clams in a great
white wine sauce dotted with diced tomatoes. Another night the pasta was
wide, buttery ribbons painted with bright green parsley sauce and coiled
into a tower in the center of the plate. On top was a dollop of caviar,
on the side two large sea scallops. Knocking it down and mixing it up
made me feel like a child playing with my food.

Mr. Portale also does interesting things with less popular fish, pairing
skate with eggplant and white beans in a mint and olive oil dressing.
Mackerel comes with potatoes no larger than marbles, more eggplant and
just enough rosemary to cut through the richness of the fish.

The prettiest appetizer is veal carpaccio, translucent circles of meat
with thin red lines running through them like paint. These turn out to
be slivers of bresaola (air-dried beef). Bits of basil are scattered
across the top like confetti, joining leaves of baby arugula and fat
curls of Parmesan cheese. It tastes as good as it looks.

The main courses are more straightforward than the appetizers, each dish
composed around a central element. Rosy slices of duck breast are set
off by a single caramelized endive, peeled back like a flower in the
center of the plate. The bitterness of the vegetable plays against a
sweet potato puree, each vegetable bringing forth a different quality of
the duck.

Squab is topped with foie gras, which brings out the peculiarly livery
taste of the bird. Meanwhile the vegetables on the plate -\/- sweet
corn, polenta and cranberry beans -\/- all emphasize the tender richness
of the squab. Pheasant is normally a dull bird, but Mr. Portale plunks
it onto a fabulous pile of sauerkraut, alive with the taste of juniper
berries, a puree of potatoes and a single poached crab apple. In this
company, an old shoe would taste terrific.

Wonderful rack of lamb comes with chard and garlic mashed potatoes. The
steak, however, is served without potatoes; there are deep fried
shallots on the side and a silky little custard whose delicacy brings
out the hearty quality of the meat.

Each fish is also paired with something to bring out its most important
qualities. Tuna, so red it is almost blue, comes with a swirl of
pappardelle and a mush of eggplant. Sturdy salmon is served with
mushrooms, peas and potatoes. Halibut, whether steamed or sauteed, is
served with lots of vegetables, emphasizing the clean delicacy of the
fish.

Desserts are intense and very American, from a vertical banana split
with caramel, macadamia nuts and hot fudge to a warm apple tart with
vanilla ice cream and a cranberry compote. There are a fine creme brulee
served with a poached pear compote and a warm chocoalte cake of great
complexity. There is also a fabulous chocolate bread pudding served with
vanilla ice cream and chocolate sauce that tastes like what My-T-Fine
wishes it could be when it grows up.

"Everything was wonderful," a man murmured one night as he was leaving.
"But it wasn't very romantic." He has a point. With its bright lights
and loud music, Gotham has everything but romance. I suspect the owners
are onto something there, too; after all, romantic meals are rare
special events but Gotham is good for every ordinary occasion.

Gotham Bar and Grill ***

12 East 12th Street, Greenwich Village, (212) 620-4020.

Ambiance: The big, loud, open room is cheerful and welcoming.

Service: The waiters are extraordinary, taking their cues from their
customers and anticipating every wish.

Recommended dishes: Seafood salad, veal carpaccio, tuna tartar, seared
mackerel, pastas, foie gras and guinea hen terrine, sauteed skate wings,
herbed ricotta ravioli, duck breast with endive and sweet potato puree,
grilled salmon, seared tuna, roast chicken with shoestring potatoes,
pheasant with sauerkraut, grilled steak, rack of lamb, squab with foie
gras, vertical banana split, chocolate cake, warm apple tart, pear
napoleon, chocolate bread pudding.

Wine list: The waiters are very helpful in navigating the unusual and
well-chosen list. All the wines are good, but it would be nice if there
were more bottles in the lower price ranges.

Price range: Lunch: three-course prix fixe at \$19.96, appetizers \$7 to
\$14.50, main courses \$14 to \$19; dinner: appetizers \$9.25 to
\$16.50, main courses \$26.50 to \$33, desserts \$7.50 to \$8.50.

Hours: Lunch: noon to 2:30, Mondays through Fridays; dinner: 5:30 to
10:30, Sundays through Thursdays, until 11:30, Fridays and Saturdays.

Credit cards: All major cards.

Wheelchair accessibility: There is a ramp for the steps to the dining
room, but the restrooms are downstairs.

Advertisement

\protect\hyperlink{after-bottom}{Continue reading the main story}

\hypertarget{site-index}{%
\subsection{Site Index}\label{site-index}}

\hypertarget{site-information-navigation}{%
\subsection{Site Information
Navigation}\label{site-information-navigation}}

\begin{itemize}
\tightlist
\item
  \href{https://help.nytimes3xbfgragh.onion/hc/en-us/articles/115014792127-Copyright-notice}{©~2020~The
  New York Times Company}
\end{itemize}

\begin{itemize}
\tightlist
\item
  \href{https://www.nytco.com/}{NYTCo}
\item
  \href{https://help.nytimes3xbfgragh.onion/hc/en-us/articles/115015385887-Contact-Us}{Contact
  Us}
\item
  \href{https://www.nytco.com/careers/}{Work with us}
\item
  \href{https://nytmediakit.com/}{Advertise}
\item
  \href{http://www.tbrandstudio.com/}{T Brand Studio}
\item
  \href{https://www.nytimes3xbfgragh.onion/privacy/cookie-policy\#how-do-i-manage-trackers}{Your
  Ad Choices}
\item
  \href{https://www.nytimes3xbfgragh.onion/privacy}{Privacy}
\item
  \href{https://help.nytimes3xbfgragh.onion/hc/en-us/articles/115014893428-Terms-of-service}{Terms
  of Service}
\item
  \href{https://help.nytimes3xbfgragh.onion/hc/en-us/articles/115014893968-Terms-of-sale}{Terms
  of Sale}
\item
  \href{https://spiderbites.nytimes3xbfgragh.onion}{Site Map}
\item
  \href{https://help.nytimes3xbfgragh.onion/hc/en-us}{Help}
\item
  \href{https://www.nytimes3xbfgragh.onion/subscription?campaignId=37WXW}{Subscriptions}
\end{itemize}
