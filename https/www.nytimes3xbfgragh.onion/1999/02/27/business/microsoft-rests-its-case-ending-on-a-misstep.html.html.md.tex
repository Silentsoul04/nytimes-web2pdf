Sections

SEARCH

\protect\hyperlink{site-content}{Skip to
content}\protect\hyperlink{site-index}{Skip to site index}

\href{https://www.nytimes3xbfgragh.onion/section/business}{Business}

\href{https://myaccount.nytimes3xbfgragh.onion/auth/login?response_type=cookie\&client_id=vi}{}

\href{https://www.nytimes3xbfgragh.onion/section/todayspaper}{Today's
Paper}

\href{/section/business}{Business}\textbar{}Microsoft Rests Its Case,
Ending On a Misstep

\begin{itemize}
\item
\item
\item
\item
\item
\end{itemize}

Advertisement

\protect\hyperlink{after-top}{Continue reading the main story}

Supported by

\protect\hyperlink{after-sponsor}{Continue reading the main story}

\hypertarget{microsoft-rests-its-case-ending-on-a-misstep}{%
\section{Microsoft Rests Its Case, Ending On a
Misstep}\label{microsoft-rests-its-case-ending-on-a-misstep}}

By
\href{https://topics.nytimes3xbfgragh.onion/top/reference/timestopics/people/b/joel_brinkley/index.html}{Joel
Brinkley}

\begin{itemize}
\item
  Feb. 27, 1999
\item
  \begin{itemize}
  \item
  \item
  \item
  \item
  \item
  \end{itemize}
\end{itemize}

See the article in its original context from\\
February 27, 1999, Section C, Page
1\href{https://store.nytimes3xbfgragh.onion/collections/new-york-times-page-reprints?utm_source=nytimes\&utm_medium=article-page\&utm_campaign=reprints}{Buy
Reprints}

\href{http://timesmachine.nytimes3xbfgragh.onion/timesmachine/1999/02/27/528269.html}{View
on timesmachine}

TimesMachine is an exclusive benefit for home delivery and digital
subscribers.

After more than five months of testimony, the Microsoft Corporation
rested its case today in the Government's landmark antitrust suit, but
not before the presiding judge had shouted angrily at the company's
final witness and ordered him to stop talking.

Today's incident was yet another blow in several months of missteps and
embarrassments as Microsoft has tried to defend itself against
Government charges that the company bullied the computer industry and
used its monopoly in computer operating systems to the disadvantage of
competitors.

The trial is not over; court will reconvene in mid-April for several
weeks of testimony from rebuttal witnesses, three to a side. But as
Microsoft completed its central case today, six weeks after the
Government completed its case, both sides said they were confident of
victory.

John Warden, Microsoft's lead trial lawyer, acknowledged that others
believed that the Government had ''succeeded in undermining our
witnesses.'' But he called this a desperation tactic. ''When you don't
have the laws or the facts, you try credibility, and that's what I think
has driven them to this strategy.''

David Boies, the Government's lead trial lawyer, who has tripped up and
embarrassed most of Microsoft's witnesses, said he believed that casting
doubt on witnesses' credibility was not all that had been achieved.

''They've admitted monopoly power,'' he said. ''They've admitted the
absence of competitive constraints. They've admitted raising prices to
hurt consumers. They've admitted depriving consumers of choice.''

In the witness box today, Robert Muglia, a Microsoft senior vice
president, tried to put the best face on his company's relationship with
Sun Microsystems, the creator and owner of the Java programming
language. The Government charges that Microsoft tried to sabotage Sun
because it saw Java as a competitive threat.

Mr. Muglia, who said Microsoft's relationship with Sun was his
responsibility, repeatedly asserted that Microsoft was interested in
cooperating with Sun. But Mr. Boies presented numerous E-mail messages
and memos from senior Microsoft executives, saying in one manner or
another that they wanted to defeat Sun.

The combined effect of the memos was to leave the impression that if Mr.
Muglia was to be believed, he was either out of touch or naive. And his
continued defense of his position, even in the face of a contradictory
E-mail from William H. Gates, the company's chairman, set off the judge.

In May 1997, Mr. Gates wrote: ''I am hard-core about NOT supporting''
the latest version of Java. Messages in the same string of E-mail from
other senior executives made the same statement, but with exclamation
points and expletives.

Yet Mr. Muglia tried to make the case that Mr. Gates had not really
meant what he wrote, adding, ''I don't exactly know what Bill meant by
support.''

At that, Judge Thomas Penfield Jackson, who is hearing the case without
a jury, shook his head and interrupted with an irritated tone, saying:
''There's no question he says he does not like the idea of supporting
it. Let's not argue about it.''

Mr. Muglia persisted, pleading with the judge, ''Can I say one more
thing, please?''

But a few seconds after he began what promised to be a long discourse
defending his position, Judge Jackson exploded. One hand covering his
face, the other held up at the witness, he bellowed: ''No! Stop! There
is no question pending!''

He then called a recess.

Microsoft spokesmen said they believed that Mr. Muglia needed to keep
his answers shorter.

Earlier, Mr. Boies had showed him a Microsoft memo setting out the
company's strategy on Java. The first line was: ''Kill cross-platform
Java by growing the polluted Java market.'' Sun and the Government
accuse Microsoft of creating its own ''polluted'' version of Java to
undermine Sun's version. Microsoft argues that its version is better.

Mr. Muglia said the document was written by a junior employee and was
later revised by her supervisor.

Judge Jackson's outburst followed questions he had asked earlier in the
day suggesting that he was skeptical of Microsoft's case.

While lawyers warn that it is dangerous to read too much into a judge's
remarks, it is also true that judges often pose questions at the end of
a trial that are intended to test conclusions they are considering for
use in their final ruling.

This morning Microsoft's lawyer was questioning the preceding witness,
Joachim Kempin, a Microsoft vice president, prompting him to list the
modifications Microsoft was now allowing computer manufacturers to make
to its Windows operating system. A year ago, the company forbade most or
all such changes, which contributed to Federal antitrust charges.

Judge Jackson interrupted the questions to ask in an even tone: ''Are
all these rights manufacturers now possess a matter of sufferance and
grace on the part of Microsoft, or are they expressly written into the
contracts?''

Mr. Kempin said some were granted in personal letters to the companies,
others in phone conversations -\/- not in contracts.

''So you have chosen to waive or give up certain rights you have in your
contract?'' the judge said.

That's right, Mr. Kempin said. The judge's questions appeared to mirror
the Government's assertions that Microsoft's new generosity to
manufacturers could be temporary -\/- lasting only as long as
Microsoft's previous behavior is the subject of antitrust charges.

As Mr. Kempin completed his testimony, he turned to the next topic of
testimony: Java. He offered an eloquent argument for the tenuous nature
of Microsoft's dominant position in the industry. Among other threats,
he said, Java was such a powerful idea that it could cause ''a paradigm
shift'' that would unseat Microsoft.

But that exposed a contradiction in Microsoft's defense. Mr. Muglia, in
his written, direct testimony, said Sun's version of Java was so
dreadful that Microsoft had every right to create its own version.

Sun's Java, he wrote, ''is truly the great equalizing software; it has
reduced all computers to mediocrity and buggyness.''

At the end of the day, Mr. Muglia described a conversation in which Mr.
Gates had told him that Microsoft, which had licensed Java from Sun,
could meet its contractual obligations by simply posting Sun's version
of Java on Microsoft's Web site, rather than including it in Windows.

Then Mr. Boies introduced a memo in which the company executives charged
with doing that described what they had done.

The Java file was posted, one executive wrote, ''but there will be no
entry index'' for it, and as a result a consumer would ''have to stumble
across it.''

''I put it in a directory with 37 other old files,'' the executive
wrote. ''In this directory I'd say it's pretty buried.''

''Awesome,'' the other executive wrote back. ''Thanks.''

Mr. Muglia insisted that he had ''said specifically that I didn't want
it to be hidden.'' And he suggested that the hidden copy was just a
draft.

But Mr. Boies said simply, ''I have no more questions.'' The trial
adjourned.

Advertisement

\protect\hyperlink{after-bottom}{Continue reading the main story}

\hypertarget{site-index}{%
\subsection{Site Index}\label{site-index}}

\hypertarget{site-information-navigation}{%
\subsection{Site Information
Navigation}\label{site-information-navigation}}

\begin{itemize}
\tightlist
\item
  \href{https://help.nytimes3xbfgragh.onion/hc/en-us/articles/115014792127-Copyright-notice}{©~2020~The
  New York Times Company}
\end{itemize}

\begin{itemize}
\tightlist
\item
  \href{https://www.nytco.com/}{NYTCo}
\item
  \href{https://help.nytimes3xbfgragh.onion/hc/en-us/articles/115015385887-Contact-Us}{Contact
  Us}
\item
  \href{https://www.nytco.com/careers/}{Work with us}
\item
  \href{https://nytmediakit.com/}{Advertise}
\item
  \href{http://www.tbrandstudio.com/}{T Brand Studio}
\item
  \href{https://www.nytimes3xbfgragh.onion/privacy/cookie-policy\#how-do-i-manage-trackers}{Your
  Ad Choices}
\item
  \href{https://www.nytimes3xbfgragh.onion/privacy}{Privacy}
\item
  \href{https://help.nytimes3xbfgragh.onion/hc/en-us/articles/115014893428-Terms-of-service}{Terms
  of Service}
\item
  \href{https://help.nytimes3xbfgragh.onion/hc/en-us/articles/115014893968-Terms-of-sale}{Terms
  of Sale}
\item
  \href{https://spiderbites.nytimes3xbfgragh.onion}{Site Map}
\item
  \href{https://help.nytimes3xbfgragh.onion/hc/en-us}{Help}
\item
  \href{https://www.nytimes3xbfgragh.onion/subscription?campaignId=37WXW}{Subscriptions}
\end{itemize}
