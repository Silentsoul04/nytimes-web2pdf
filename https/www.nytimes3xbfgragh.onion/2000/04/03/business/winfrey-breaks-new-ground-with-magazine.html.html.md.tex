Sections

SEARCH

\protect\hyperlink{site-content}{Skip to
content}\protect\hyperlink{site-index}{Skip to site index}

\href{https://www.nytimes3xbfgragh.onion/section/business}{Business}

\href{https://myaccount.nytimes3xbfgragh.onion/auth/login?response_type=cookie\&client_id=vi}{}

\href{https://www.nytimes3xbfgragh.onion/section/todayspaper}{Today's
Paper}

\href{/section/business}{Business}\textbar{}Winfrey Breaks New Ground
With Magazine

\begin{itemize}
\item
\item
\item
\item
\item
\end{itemize}

Advertisement

\protect\hyperlink{after-top}{Continue reading the main story}

Supported by

\protect\hyperlink{after-sponsor}{Continue reading the main story}

\hypertarget{winfrey-breaks-new-ground-with-magazine}{%
\section{Winfrey Breaks New Ground With
Magazine}\label{winfrey-breaks-new-ground-with-magazine}}

By \href{https://www.nytimes3xbfgragh.onion/by/alex-kuczynski}{Alex
Kuczynski}

\begin{itemize}
\item
  April 3, 2000
\item
  \begin{itemize}
  \item
  \item
  \item
  \item
  \item
  \end{itemize}
\end{itemize}

See the article in its original context from\\
April 3, 2000, Section C, Page
1\href{https://store.nytimes3xbfgragh.onion/collections/new-york-times-page-reprints?utm_source=nytimes\&utm_medium=article-page\&utm_campaign=reprints}{Buy
Reprints}

\href{http://timesmachine.nytimes3xbfgragh.onion/timesmachine/2000/04/03/587800.html}{View
on timesmachine}

TimesMachine is an exclusive benefit for home delivery and digital
subscribers.

You would have to live under a rock in a particularly desolate stretch
of the Sahara to not know who Oprah Winfrey is.

The talk show she started in 1985, ''The Oprah Winfrey Show,'' is seen
by 22 million viewers a week in the United States and is broadcast in
119 countries. She is an Oscar-nominated actress. And in what is perhaps
the truest measure of celebrity in America, the fluctuations of her
weight are chronicled almost weekly in the nation's tabloids.

In the magazine industry, though, Ms. Winfrey is known for selling more
magazines than bikini-clad models, Internet billionaires or even
pregnant actresses.

Her record is auspicious. When Ms. Winfrey appeared on the cover of In
Style in November 1998, it was that magazine's best-selling issue ever,
with almost 900,000 copies in newsstand sales. She was on the cover of
Vogue the prior month, selling 810,000 copies on newsstands, and of Good
Housekeeping in December, selling 1.4 million copies on the newsstand
-\/- best sellers for both magazines for that year.

So when Ms. Winfrey announced last year that her company, Harpo Inc.,
and Hearst Magazines, a division of the Hearst Corporation, would start
a magazine using her name and image, it made sense. Ms. Winfrey gets a
new audience -\/- one that is both younger and more affluent than the
one that watches her talk show, said Alyce Alston, the magazine's
publisher. And Hearst gets to use the Oprah Winfrey brand to sell
magazines.

The magazine, which will arrive on newsstands on April 17, is, after
months of Hearst teeth-gnashing, called O: The Oprah Magazine. It is, as
Ms. Winfrey describes it, ''a personal growth guide'' for women 25 to
49. And Ms. Winfrey said that her first few months working on the
magazine were a period of personal growth for her, because she learned
that even fame cannot inure one from the slings and arrows -\/- or plain
squabbling -\/- of a magazine start-up.

''This magazine is the book that I never wrote,'' Ms. Winfrey said on
Friday afternoon, after a television taping. ''It's an opportunity every
month to use my voice, but also to share what I have learned from other
people's wisdom. It's about challenging readers, inspiring them and
getting them to figure out what's important.''

What is important to Hearst is that the magazine be a successful
newsstand sale -\/- Hearst has bought 50,000 racks in supermarkets
nationwide -\/- but that it also generate millions of dollars in
advertising revenue.

Roberta Garfinkle, director of print advertising at McCann-Erickson,
said that would not be an issue for now.

''She is what every woman wants to be when they grow up,'' Ms. Garfinkle
said. ''She transcends race and class. And probably most categories of
advertisers.''

The first issue carries a hefty 166 pages of advertising; the entire
magazine, with editorial content, is 324 pages, Ms. Alston, the
publisher, said. Advertisers include technology companies like Microsoft
and Hewlett-Packard, fashion designers like Carolina Herrera and Calvin
Klein, mass-market retailers like J. C. Penney, Schering-Plough, maker
of the allergy drug Claritin, and automakers like Toyota and General
Motors.

Ms. Winfrey's level of personal involvement is such that she will not
allow certain types of advertisers -\/- tobacco companies, for instance
-\/- in the magazine. And her involvement in the editorial process, she
said, has not been easy.

''Doing the TV show is like breathing to me,'' she said. ''I do not find
that stressful at all. But this is the most stressed I have been since I
first started working in TV and they assigned me to cover a city council
meeting and I didn't know what they were talking about.''

A famous perfectionist, Ms. Winfrey asked that articles be rewritten as
late as three weeks ago, staff members said.

''I do know that if I read a piece and if I think the writing is
condescending, I do know how to say, 'Find another writer,' '' Ms.
Winfrey said. ''And that has happened a couple of times.''

She also fought with the staff of the magazine late in the process of
selling ad pages to move the table of contents from the position it
occupies in most women's magazines -\/- 20 or so pages in, so readers
will have to sift through many pages of advertising -\/- to the front.

''I said, 'Of course we're not going to have the table of contents so
far in you can't find it and people think the whole thing is about ads,'
'' Ms. Winfrey said. ''Everyone said, 'Oh, the pages are already sold,'
and certain advertisers refuse to be after the T.O.C. I said, 'No, I am
here to represent the reader. And we don't want to go to Page 18.' So
we're sitting at the meeting, and everyone said, 'So where do you want
it to be?' I said, 'Page 2.' '' The table of contents will now appear on
Page 2.

And the process of editing the magazine has been fraught with
territorial issues. Because Ms. Winfrey required a liaison between her
and the magazine's editors, she asked that her friend Gayle King act in
an editorial capacity, making choices for Ms. Winfrey when she was
unavailable.

Ms. King, in turn, had to make editorial choices with Ellen Levine, the
editor in chief of Good Housekeeping, who is shepherding the magazine at
Hearst, and Ellen Kunes, the editor in chief of O: The Oprah Magazine.

There were too many cooks in the kitchen at times; two weeks ago, Ms.
Winfrey flew the editorial staff to Miami on her private jet and invited
everyone to her house on Fisher's Island, one of Miami's most posh
resorts, to dance and eat brunch and generally clear the air. At one
point, several staff members complained loudly about the diffusion of
editorial vision.

''The problem has been that we are on a learning curve,'' Ms. Winfrey
said. ''And the problem is that Ellen Kunes does not really know me. She
is trying to get to know me. That is part of the reason I had the
retreat, to say to them, 'Look, I know that to you guys the Oprah name
is a brand. But for me, it is my life, it's the way I live my life, and
the way I behave and everything I stand for.' The retreat was to sync
ourselves up.''

Ms. Kunes said: ''She and I are partners. My job is basically to make
sure that what she sees and wants to work and have in the magazine
happens. It is surprisingly straightforward.''

One area of possible contention is how Hearst, which owns Women.com and
a piece of the Lifetime Television cable network, and Oxygen Media,
which is building Internet sites and cable television properties for
women and is backed in part by Ms. Winfrey, will all promote the
magazine.

Earlier this year, Ms. Winfrey called a meeting that included Geraldine
Laybourne, the chairwoman and chief executive of Oxygen and several
Hearst Magazine executives, including Ms. Levine.

''It was the big white elephant in the room,'' Ms. Winfrey said. ''But
we have worked it out for now. One will promote the other.'' Oxygen will
promote the magazine on the Oxygen Web site, and Oprah.com, which is an
Oxygen Web site, will have a link to the magazine's Web site.

Most important, Ms. Winfrey said, the magazine should not be construed
as her preparation for leaving television.

''Gee whizzer,'' Ms. Winfrey said. ''No way. The Oprah show is the
mother lode.''

She added, ''And it's so much less stressful.''

Advertisement

\protect\hyperlink{after-bottom}{Continue reading the main story}

\hypertarget{site-index}{%
\subsection{Site Index}\label{site-index}}

\hypertarget{site-information-navigation}{%
\subsection{Site Information
Navigation}\label{site-information-navigation}}

\begin{itemize}
\tightlist
\item
  \href{https://help.nytimes3xbfgragh.onion/hc/en-us/articles/115014792127-Copyright-notice}{©~2020~The
  New York Times Company}
\end{itemize}

\begin{itemize}
\tightlist
\item
  \href{https://www.nytco.com/}{NYTCo}
\item
  \href{https://help.nytimes3xbfgragh.onion/hc/en-us/articles/115015385887-Contact-Us}{Contact
  Us}
\item
  \href{https://www.nytco.com/careers/}{Work with us}
\item
  \href{https://nytmediakit.com/}{Advertise}
\item
  \href{http://www.tbrandstudio.com/}{T Brand Studio}
\item
  \href{https://www.nytimes3xbfgragh.onion/privacy/cookie-policy\#how-do-i-manage-trackers}{Your
  Ad Choices}
\item
  \href{https://www.nytimes3xbfgragh.onion/privacy}{Privacy}
\item
  \href{https://help.nytimes3xbfgragh.onion/hc/en-us/articles/115014893428-Terms-of-service}{Terms
  of Service}
\item
  \href{https://help.nytimes3xbfgragh.onion/hc/en-us/articles/115014893968-Terms-of-sale}{Terms
  of Sale}
\item
  \href{https://spiderbites.nytimes3xbfgragh.onion}{Site Map}
\item
  \href{https://help.nytimes3xbfgragh.onion/hc/en-us}{Help}
\item
  \href{https://www.nytimes3xbfgragh.onion/subscription?campaignId=37WXW}{Subscriptions}
\end{itemize}
