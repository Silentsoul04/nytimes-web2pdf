Sections

SEARCH

\protect\hyperlink{site-content}{Skip to
content}\protect\hyperlink{site-index}{Skip to site index}

\href{https://www.nytimes3xbfgragh.onion/section/food}{Food}

\href{https://myaccount.nytimes3xbfgragh.onion/auth/login?response_type=cookie\&client_id=vi}{}

\href{https://www.nytimes3xbfgragh.onion/section/todayspaper}{Today's
Paper}

\href{/section/food}{Food}\textbar{}\$25 AND UNDER; A Spicy Indian Spot
That Doesn't Care to Fit In

\begin{itemize}
\item
\item
\item
\item
\item
\end{itemize}

Advertisement

\protect\hyperlink{after-top}{Continue reading the main story}

Supported by

\protect\hyperlink{after-sponsor}{Continue reading the main story}

\$25 AND UNDER

\hypertarget{25-and-under-a-spicy-indian-spot-that-doesnt-care-to-fit-in}{%
\section{\$25 AND UNDER; A Spicy Indian Spot That Doesn't Care to Fit
In}\label{25-and-under-a-spicy-indian-spot-that-doesnt-care-to-fit-in}}

By \href{https://www.nytimes3xbfgragh.onion/by/sam-sifton}{Sam Sifton}

\begin{itemize}
\item
  July 31, 2002
\item
  \begin{itemize}
  \item
  \item
  \item
  \item
  \item
  \end{itemize}
\end{itemize}

See the article in its original context from\\
July 31, 2002, Section F, Page
8\href{https://store.nytimes3xbfgragh.onion/collections/new-york-times-page-reprints?utm_source=nytimes\&utm_medium=article-page\&utm_campaign=reprints}{Buy
Reprints}

\href{http://timesmachine.nytimes3xbfgragh.onion/timesmachine/2002/07/31/175382.html}{View
on timesmachine}

TimesMachine is an exclusive benefit for home delivery and digital
subscribers.

THE old saw has it that there is a single communal kitchen under the
cluster of Indian restaurants on East Sixth Street from which they all
take their food. The arrival in late 2000 of Banjara, a nonfacsimile on
the corner of First Avenue, did some damage to this moldy chestnut.
Brick Lane Curry House, a spare and casual place that opened next door
to Banjara about three months ago, breaks it wide.

The food is powerfully spiced, distinctively flavored, extremely fresh.
From the sweetly moist mint-heightened minced lamb shammi kebabs (\$8)
that might start a meal to the frothy mango lassi (\$3) that might come
at its end, almost everything about Brick Lane calls to mind a skilled
and careful hand. Sitting in the tiny front dining room, listening to
the world-beat soundtrack beneath walls stained to a soft-focus sunset
red, it is possible to feel transported -\/- if not to Bombay, than at
least to England.

Brick Lane is not, in a strict sense, a regional Indian restaurant. It
takes its name from the English analogue to Sixth Street -\/- a stretch
of Indian restaurants in East London -\/- and it serves, along with a
diverse menu of Indian kebabs and rolls, the standard fare of modern
England: chicken tikka masala, curries, lamb vindaloo.

The difference is that this Brick Lane prepares its food with unusual
care and ability. The silky chicken tikka masala (\$15) is wonderfully
smooth, redolent of the roasted tomatoes in the cream sauce and the
fenugreek used to amp its flavor, with plump hunks of tender chicken
lounging in their bath. The balti (\$14), a curry that has taken deep
root in the West Midlands of England, offers sweet roasted onions and
bell peppers along with moist chicken or lamb, and a sauce bright with
mustard seed and fenugreek, with an amiable hint of fire. (All the
curries are available in vegetarian versions, as well as with lamb,
shrimp or chicken.) And the lamb vindaloo (\$15) offers the smoky heat
that Goan dish is known best for, though without the throbbing mouth
pain (or, alternatively, anemic mouth boredom) so many versions of the
dish deliver.

For fans of the heat challenge, though, Brick Lane can oblige. The chef,
Eric McCarthy, who was born in the former Portuguese colony of Goa,
offers a gimmick curry, phal (\$9 to \$17), that the menu calls ''an
excruciatingly hot curry, more pain and sweat than flavor.'' The
statement is factual. The lamb version I sampled left me gasping,
sweating, in a slight panic. Much better are the flavorful lamb jalfrezi
(\$15), heated with green chilies and balanced by ginger, and the
mustardy, ever-so-slightly-sweet shrimp Madras (\$17).

On the opposite end of the spectrum are a splendid vegan dish called
peeli dal (\$10) -\/- yellow lentils with a pleasant, almost musky bite
-\/- and Mr. McCarthy's dal makhani (\$11), a creamy, gently spiced stew
of smoky black lentils that despite its simplicity may be the
restaurant's finest creation. It is perfect to pair with hunks of lovely
soft garlic nan (\$2.50), pillowy Basmati rice shot through with
frizzled onions or engaging versions of richly flavored biriyani (\$9 to
\$17).

Boti rolls (\$14), nan or roti bread wrapped around chicken tikkas and
fresh onions in the manner of an Indian burrito are likewise sublime,
especially dipped into smoky tomato chutney, as is the stuffed calamari
(\$14), or squid filled with chopped seafood run through with ginger and
onions, then baked and sliced. The effect is strangely delicious. The
menu says the dish is inspired by the flavors of Kerala, along India's
southern coast.

Brick Lane is not flawless, of course. Desserts, when available, run to
plain-Jane milk-curd balls. (That delicious mango lassi is the wiser
choice.) Beer may or may not be served cold, and the wine list,
apparently limited one night to basic red or white, did not intrigue.
Service can be cheerfully lackadaisical. Sometimes courses arrive
willy-nilly. Other times orders are forgotten. This is particularly true
on Tuesday nights, when the restaurant plays host to world-beat
concerts, with live marimba music, Middle Eastern drumming and the like.
Then the dining room fills with hipsters and swaying middle-aged
dancers. But the restaurant's mood is invariably pleasant and happy -\/-
a hectic establishment serving wonderful food.

Brick Lane Curry House

342 East Sixth Street (First Avenue), East Village; (212) 979-2900.

BEST DISHES Shammi kebabs, tofu kebabs, stuffed calamari, boti rolls,
chicken tikka masala, chicken balti, lamb jalfrezi, dal makhani, lamb
biriyani, mango lassi.

PRICE RANGE Appetizers, \$5; kebabs and rolls, \$8 to \$14; entrees, \$9
to \$17.

CREDIT CARDS All major cards.

HOURS Sunday through Thursday, 5:30 to 11 p.m.; Friday and Saturday,
5:30 p.m. to 1 a.m.; Saturday and Sunday brunch, noon to 3 p.m.

WHEELCHAIR ACCESS Entrance is a step up; bathrooms are small.

Advertisement

\protect\hyperlink{after-bottom}{Continue reading the main story}

\hypertarget{site-index}{%
\subsection{Site Index}\label{site-index}}

\hypertarget{site-information-navigation}{%
\subsection{Site Information
Navigation}\label{site-information-navigation}}

\begin{itemize}
\tightlist
\item
  \href{https://help.nytimes3xbfgragh.onion/hc/en-us/articles/115014792127-Copyright-notice}{©~2020~The
  New York Times Company}
\end{itemize}

\begin{itemize}
\tightlist
\item
  \href{https://www.nytco.com/}{NYTCo}
\item
  \href{https://help.nytimes3xbfgragh.onion/hc/en-us/articles/115015385887-Contact-Us}{Contact
  Us}
\item
  \href{https://www.nytco.com/careers/}{Work with us}
\item
  \href{https://nytmediakit.com/}{Advertise}
\item
  \href{http://www.tbrandstudio.com/}{T Brand Studio}
\item
  \href{https://www.nytimes3xbfgragh.onion/privacy/cookie-policy\#how-do-i-manage-trackers}{Your
  Ad Choices}
\item
  \href{https://www.nytimes3xbfgragh.onion/privacy}{Privacy}
\item
  \href{https://help.nytimes3xbfgragh.onion/hc/en-us/articles/115014893428-Terms-of-service}{Terms
  of Service}
\item
  \href{https://help.nytimes3xbfgragh.onion/hc/en-us/articles/115014893968-Terms-of-sale}{Terms
  of Sale}
\item
  \href{https://spiderbites.nytimes3xbfgragh.onion}{Site Map}
\item
  \href{https://help.nytimes3xbfgragh.onion/hc/en-us}{Help}
\item
  \href{https://www.nytimes3xbfgragh.onion/subscription?campaignId=37WXW}{Subscriptions}
\end{itemize}
