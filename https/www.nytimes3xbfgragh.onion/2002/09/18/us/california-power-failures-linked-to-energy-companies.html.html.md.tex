Sections

SEARCH

\protect\hyperlink{site-content}{Skip to
content}\protect\hyperlink{site-index}{Skip to site index}

\href{https://www.nytimes3xbfgragh.onion/section/us}{U.S.}

\href{https://myaccount.nytimes3xbfgragh.onion/auth/login?response_type=cookie\&client_id=vi}{}

\href{https://www.nytimes3xbfgragh.onion/section/todayspaper}{Today's
Paper}

\href{/section/us}{U.S.}\textbar{}California Power Failures Linked to
Energy Companies

\begin{itemize}
\item
\item
\item
\item
\item
\end{itemize}

Advertisement

\protect\hyperlink{after-top}{Continue reading the main story}

Supported by

\protect\hyperlink{after-sponsor}{Continue reading the main story}

\hypertarget{california-power-failures-linked-to-energy-companies}{%
\section{California Power Failures Linked to Energy
Companies}\label{california-power-failures-linked-to-energy-companies}}

By \href{https://www.nytimes3xbfgragh.onion/by/john-m-broder}{John M.
Broder}

\begin{itemize}
\item
  Sept. 18, 2002
\item
  \begin{itemize}
  \item
  \item
  \item
  \item
  \item
  \end{itemize}
\end{itemize}

See the article in its original context from\\
September 18, 2002, Section A, Page
22\href{https://store.nytimes3xbfgragh.onion/collections/new-york-times-page-reprints?utm_source=nytimes\&utm_medium=article-page\&utm_campaign=reprints}{Buy
Reprints}

\href{http://timesmachine.nytimes3xbfgragh.onion/timesmachine/2002/09/18/934941.html}{View
on timesmachine}

TimesMachine is an exclusive benefit for home delivery and digital
subscribers.

Widespread power failures during California's energy crisis of 2000 and
2001 could have been avoided if five independent energy companies had
not withheld electricity they were capable of producing, a study by
state regulators said today.

The investigation by the California Public Utilities Commission said the
five companies -\/- Duke, Dynegy, Mirant, Reliant and AES/Williams -\/-
had withheld power from their California plants.

This contributed to the ''unconscionable, unjust and unreasonable
electricity price spike that California experienced during the energy
crisis,'' the report said.

The commission did not directly accuse the companies of deliberately
trying to drive prices up. Officials said investigations were continuing
into possible price manipulation and collusion among the companies.

The commission said that the companies took plants off line for
necessary maintenance and were subjected to sharp increases in natural
gas prices but that they still had sufficient generating capacity to
avoid all four days of blackouts in Southern California and 65 percent
of the blackout hours in Northern California.

''On all but 2 of the 32 statewide blackout and service interruption
days shown, the five biggest independent electricity generators did not
supply well over 500 megawatts of power they could have generated,'' the
study found.

A megawatt of electricity is roughly the amount needed to power 750
homes.

The generators responded that there were legitimate reasons for the
service disruptions and that their plants were running as hard or harder
during the crisis as before or since.

''California's merchant generators ran at historically high levels to
power our state throughout the crisis,'' said Jan Smutny-Jones,
executive director of the Independent Energy Producers, a trade group.
''The average age of these plants is over 36 years old. Despite their
age, California's power plants ran 88 percent harder in 2001 than they
did in 1999, and some increased output as much as 206 percent to meet
our state's energy needs, making up for decreased energy imports from
the Northwest caused by a drought.''

Company officials also said that some plants were taken off line to add
equipment required by state environmental regulations. They also blamed
the public utilities commission for ordering continued sales of
electricity to Mexico at a time of power failures in San Diego.

The report looked at the 38 days from November 2000 through May 2001
when millions of Californians experienced large-scale blackouts and
selective power failures. The blackouts were the first in California
since World War II and caused thousands of businesses to close and
interrupted power to homes, hospitals, schools, theaters and shopping
malls.

Wholesale prices for power leaped to as high as \$1,500 a megawatt-hour
from \$40 per megawatt-hour, causing electricity bills to jump, forcing
two major utilities into bankruptcy and plunging the state into debt as
it spent billions of dollars buying electricity on the deregulated
market. The state entered into long-term energy contracts at the
inflated prices of 2000-2001, one factor that has led to the state's
\$24 billion budget deficit.

The utility commission found that on each of the days when power was
interrupted all of the five main electricity generators had capacity
that went unused. For example, on May 8, 2001, there were two-hour
blackouts caused by a shortage of 400 megawatts of power in Northern and
Southern California. But Duke Energy had about 1,000 megawatts of
available capacity that was not used that day, the commission report
said. ''Thus, Duke alone had more available and unused power than the
total amount of power that was needed to avoid the blackout that day,''
it said.

Terry Francisco, a Duke spokesman, responded: ''Duke's plants have run
harder, broken down less and been available more than any other time
during utility ownership and never more than during the crisis. Our
capacity utilization during the crisis was among the highest in
California among all the generators.''

The report mentions the role of Enron Corporation only in passing,
noting that federal officials had found that its energy trading
strategies contributed to California's energy woes and might have been
an effort to manipulate prices.

Enron, employing a variety of trading schemes known by such names as
Death Star, Fat Boy and Get Shorty, tried to ''game'' the state's power
grid to make quick trading profits, according to internal company
documents.

''Thus, the potential for simultaneous withholding and gaming (including
strategies not yet revealed) is very serious,'' the report stated, ''and
probably accounts for a large part of the rapid increase in costs in
California during the crisis.''

An Enron spokesman, Eric Thode, said that company officials had not seen
the report, but that Enron ''continues to cooperate in the various
investigations.''

Frank Wolak, a professor of economics at Stanford University and
chairman of the market surveillance committee for the state's power
grid, said the utilities commission did not establish that the
generators deliberately withheld power they were capable of producing.
But he compared that task to an employer's proving that an employee was
playing hooky when he called in sick.

''It sure looks suspicious,'' Mr. Wolak said, because the shortages led
to huge increases in prices and profits. ''But it comes down to a 'he
said-she said' sort of story.''

Advertisement

\protect\hyperlink{after-bottom}{Continue reading the main story}

\hypertarget{site-index}{%
\subsection{Site Index}\label{site-index}}

\hypertarget{site-information-navigation}{%
\subsection{Site Information
Navigation}\label{site-information-navigation}}

\begin{itemize}
\tightlist
\item
  \href{https://help.nytimes3xbfgragh.onion/hc/en-us/articles/115014792127-Copyright-notice}{©~2020~The
  New York Times Company}
\end{itemize}

\begin{itemize}
\tightlist
\item
  \href{https://www.nytco.com/}{NYTCo}
\item
  \href{https://help.nytimes3xbfgragh.onion/hc/en-us/articles/115015385887-Contact-Us}{Contact
  Us}
\item
  \href{https://www.nytco.com/careers/}{Work with us}
\item
  \href{https://nytmediakit.com/}{Advertise}
\item
  \href{http://www.tbrandstudio.com/}{T Brand Studio}
\item
  \href{https://www.nytimes3xbfgragh.onion/privacy/cookie-policy\#how-do-i-manage-trackers}{Your
  Ad Choices}
\item
  \href{https://www.nytimes3xbfgragh.onion/privacy}{Privacy}
\item
  \href{https://help.nytimes3xbfgragh.onion/hc/en-us/articles/115014893428-Terms-of-service}{Terms
  of Service}
\item
  \href{https://help.nytimes3xbfgragh.onion/hc/en-us/articles/115014893968-Terms-of-sale}{Terms
  of Sale}
\item
  \href{https://spiderbites.nytimes3xbfgragh.onion}{Site Map}
\item
  \href{https://help.nytimes3xbfgragh.onion/hc/en-us}{Help}
\item
  \href{https://www.nytimes3xbfgragh.onion/subscription?campaignId=37WXW}{Subscriptions}
\end{itemize}
