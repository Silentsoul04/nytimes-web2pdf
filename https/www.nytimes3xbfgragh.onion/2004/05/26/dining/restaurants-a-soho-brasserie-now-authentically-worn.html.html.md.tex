Sections

SEARCH

\protect\hyperlink{site-content}{Skip to
content}\protect\hyperlink{site-index}{Skip to site index}

\href{https://www.nytimes3xbfgragh.onion/section/food}{Food}

\href{https://myaccount.nytimes3xbfgragh.onion/auth/login?response_type=cookie\&client_id=vi}{}

\href{https://www.nytimes3xbfgragh.onion/section/todayspaper}{Today's
Paper}

\href{/section/food}{Food}\textbar{}RESTAURANTS; A SoHo Brasserie, Now
Authentically Worn

\begin{itemize}
\item
\item
\item
\item
\item
\end{itemize}

Advertisement

\protect\hyperlink{after-top}{Continue reading the main story}

Supported by

\protect\hyperlink{after-sponsor}{Continue reading the main story}

RESTAURANTS

\hypertarget{restaurants-a-soho-brasserie-now-authentically-worn}{%
\section{RESTAURANTS; A SoHo Brasserie, Now Authentically
Worn}\label{restaurants-a-soho-brasserie-now-authentically-worn}}

By \href{https://www.nytimes3xbfgragh.onion/by/amanda-hesser}{Amanda
Hesser}

\begin{itemize}
\item
  May 26, 2004
\item
  \begin{itemize}
  \item
  \item
  \item
  \item
  \item
  \end{itemize}
\end{itemize}

EVERY diner I know who has been to Balthazar has a story about being
snubbed by the maître d'hôtel, hung up on by the reservationist, or
ignored by a waiter tending to a more important guest.

Yet everyone who has a grievance (myself included) returns again and
again. Something draws us back. After seven years in business, Balthazar
has reached that stage in the life of a restaurant when diners' eyes
soften as they call to mind memories of meals past, memories of Lillet
and oysters and crisp, peaked croissants and the evenings of carefree
youth. A married friend recently gave me a sentimental tour of the many
tables at which she had had dates.

Balthazar is no longer hip, but it's still bustling. At 8:30 on a recent
Wednesday night, groups of waiting diners chatted outside, customers
streamed out of the bakery attached to the restaurant, and the
predictable cluster of beggars were at the reservation desk, being
snubbed by the maître d'hôtel.

It is never easy getting in, but once you are at your table, your
cocoon, the charms of Balthazar take over. The low-backed banquettes
allow diners to see almost every other table through the room's hazy
yellow glow. The space is loud, yet within your cocoon, you can hear
your companions. And the wait staff has a machine-tooled finesse -\/-
particularly the busboys, who zip by your table, pouring water and
gliding on to the next table in continuous motion. After the main
course, they change the paper on your table in a flutter of origami
folds.

The menu is much the same as always. Most of the best dishes are still
there, and so are the less successful items. The beet salad still does
not coalesce. Large beet cubes, walnut halves and a wedge of blue cheese
rest under a tuft of mâche. It has the potential to be a great
combination, just as a pile of lumber has the potential to be a great
house. And steer clear of the garganelli pasta with tomatoes and the
sautéed skate.

But the Balthazar salad, romaine, frisée, asparagus, ricotta salata and
truffle oil all mashed together, is as good as ever in its slick and
wilted way. The escargots drew my companions' attention, large brown
coils filled with juicy snails with plenty of the most important part
-\/- the butter, garlic and parsley at the bottom of the baking dish.

I adore the brandade, which is coarse and rustic, a mound of potatoes
and salt cod marked with rivulets of olive oil, and topped with thin
shards of toast. You spread a patch of brandade on the toast, and the
delicate toast shatters in your mouth. And the crisp, salty French
fries, which are served with the steak and a few other dishes, are still
the best in the city.

The menu maintains a backbone of classics throughout the year, as well
as an ample infusion of seasonal dishes. Right now, for instance, there
is navarin d'agneau with baby turnips and carrots, and asparagus spears,
which are warm and come blanketed with a tangy hollandaise and small
fragrant morels.

One reason the food has held up so well is that Riad Nasr and Lee
Hanson, the co-chefs who started when the restaurant opened, are still
here. Lately, they have put more energy into the desserts. None of them
will win awards for creativity, but almost all are excellent versions of
the familiar. The apples on the tarte Tatin are cooked down so that they
are chewy and candied on the edges. The tart of the day, when it is the
cherry and almond, should not be missed. It is chewy, sharp and
permeated with the essence of almonds. The profiteroles are as cold and
sloppy as you could wish, and the Pavlova, a crisp meringue filled with
ricotta whipped cream, is soaked with juicy blueberries and
strawberries.

In 1997, when Keith McNally opened Balthazar, it seemed to be merely a
simulacrum of a turn-of-the-century French brasserie, notable for its
museum-quality distressed tiles, faded mirrors and dented and worn zinc
bar. But because Balthazar is animated from early in the morning to late
at night by diners who treat it like a brasserie, rather than a
sanctified restaurant (it is closed about six hours a day), the place
now feels authentically worn, and it is difficult to distinguish between
the real faded and the fake faded. The stairs leading down to the
bathroom dip in the center and are worn down to raw wood. The restroom
feels as if it has been there for 100 years, with clean but creaky
stalls. A friendly woman is there to hand you a towel, and when you
don't have money, she waves you off warmly. ''Next time,'' she says.
''Next time.''

Lunch is a good hour at Balthazar. You can tuck into a platter of
oysters and a glass of wine; the lamb sandwich spread with harissa
mayonnaise is also excellent. But my favorite hour at Balthazar is
breakfast. Then the light is pure, the sound of forks and glass on
tabletops is crisp. You can have Nutella spread on a baguette with a
latte served in a large bowl, and a soft-boiled egg with toast
''soldiers.'' People are tapping away at their computers, sitting up
straight for business meetings or nodding sleepily in their chairs. The
atmosphere -\/- fake French, authentic New York -\/- buoys the room.
Keith McNally, Balthazar's owner, was right all along -\/- a dining
institution needs more than just great food.

Balthazar

** {[}rating: two stars{]}

80 Spring Street (Crosby Street), SoHo; (212) 965-1414.

ATMOSPHERE -\/- Faux Parisian brasserie with faded mirrors, bright
lighting and red banquettes.

SOUND LEVEL -\/- Loud.

RECOMMENDED DISHES -\/- Brandade; goat cheese tart; escargots; seafood
platter; Balthazar salad; steak frites; chicken paprikash; Pavlova;
tarte Tatin; profiteroles.

SERVICE -\/- Brisk and efficient, not always friendly.

WINE LIST -\/- Respectable French wines, with plenty around \$40. Good
selection of aperitifs, wines by the glass, ciders and digestifs.

HOURS -\/- Monday to Thursday, 7:30 a.m. to 1 a.m.; Friday, 7:30 a.m. to
2 a.m.; Saturday, 8 a.m. to 2 a.m.; Sunday, 8 a.m. to midnight.

PRICE RANGE -\/- Dinner, appetizers, \$9 to \$23; entrees, \$16 to \$36;
desserts, \$8.

CREDIT CARDS -\/- All major cards.

WHEELCHAIR ACCESS -\/- Restroom on street level.

Advertisement

\protect\hyperlink{after-bottom}{Continue reading the main story}

\hypertarget{site-index}{%
\subsection{Site Index}\label{site-index}}

\hypertarget{site-information-navigation}{%
\subsection{Site Information
Navigation}\label{site-information-navigation}}

\begin{itemize}
\tightlist
\item
  \href{https://help.nytimes3xbfgragh.onion/hc/en-us/articles/115014792127-Copyright-notice}{©~2020~The
  New York Times Company}
\end{itemize}

\begin{itemize}
\tightlist
\item
  \href{https://www.nytco.com/}{NYTCo}
\item
  \href{https://help.nytimes3xbfgragh.onion/hc/en-us/articles/115015385887-Contact-Us}{Contact
  Us}
\item
  \href{https://www.nytco.com/careers/}{Work with us}
\item
  \href{https://nytmediakit.com/}{Advertise}
\item
  \href{http://www.tbrandstudio.com/}{T Brand Studio}
\item
  \href{https://www.nytimes3xbfgragh.onion/privacy/cookie-policy\#how-do-i-manage-trackers}{Your
  Ad Choices}
\item
  \href{https://www.nytimes3xbfgragh.onion/privacy}{Privacy}
\item
  \href{https://help.nytimes3xbfgragh.onion/hc/en-us/articles/115014893428-Terms-of-service}{Terms
  of Service}
\item
  \href{https://help.nytimes3xbfgragh.onion/hc/en-us/articles/115014893968-Terms-of-sale}{Terms
  of Sale}
\item
  \href{https://spiderbites.nytimes3xbfgragh.onion}{Site Map}
\item
  \href{https://help.nytimes3xbfgragh.onion/hc/en-us}{Help}
\item
  \href{https://www.nytimes3xbfgragh.onion/subscription?campaignId=37WXW}{Subscriptions}
\end{itemize}
