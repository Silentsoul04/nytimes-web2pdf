Sections

SEARCH

\protect\hyperlink{site-content}{Skip to
content}\protect\hyperlink{site-index}{Skip to site index}

\href{https://www.nytimes3xbfgragh.onion/section/politics}{Politics}

\href{https://myaccount.nytimes3xbfgragh.onion/auth/login?response_type=cookie\&client_id=vi}{}

\href{https://www.nytimes3xbfgragh.onion/section/todayspaper}{Today's
Paper}

\href{/section/politics}{Politics}\textbar{}Senator John Edwards's
Remarks to the Democratic National Convention

\begin{itemize}
\item
\item
\item
\item
\item
\end{itemize}

Advertisement

\protect\hyperlink{after-top}{Continue reading the main story}

Supported by

\protect\hyperlink{after-sponsor}{Continue reading the main story}

\hypertarget{senator-john-edwardss-remarks-to-the-democratic-national-convention}{%
\section{Senator John Edwards's Remarks to the Democratic National
Convention}\label{senator-john-edwardss-remarks-to-the-democratic-national-convention}}

July 27, 2004

\begin{itemize}
\item
\item
\item
\item
\item
\end{itemize}

The following is the text of Senator John Edwards's remarks to the
Democratic National Convention, as recorded by The New York Times.

JOHN EDWARDS. Thank you. Thank you. Thank you. Now you know why
Elizabeth is so amazing, right? I am such a lucky man to have the love
of my life at my side. Both of us have been blessed with four
extraordinary children: Wade, Cate, who you heard from, Emma Claire and
Jack. We are having such an extraordinary time, myself and my entire
family at this convention. And by the way, how great was Teresa Heinz
Kerry last night?

My father and mother, Wallace and Bobbie Edwards, are also here tonight.
You taught me the values that I carry in my heart: faith, family,
responsibility, opportunity for everybody. You taught me that there's
dignity and honor in a hard day's work. You taught me to always look out
for our neighbors, to never look down on anybody, and treat everybody
with respect.

Those are the values that John Kerry and I believe in. And nothing makes
me prouder than standing with him in this campaign. I am so humbled to
be your candidate for vice president of the United States.

I want to talk about our next president. For those who want to know what
kind of leader he'll be, I want to take you back about thirty years.
When John Kerry graduated from college, he volunteered for military
service, volunteered to go to Vietnam, volunteered to captain a swift
boat, one of the most dangerous duties in Vietnam that you could have.
As a result he was wounded, honored for his valor.

If you have any question about what he's made of, just spend three
minutes with the men who served with him then and who stand with him
now. They saw up close what he's made of. They saw him reach into the
river and pull one of his men to safety and save his life. They saw him
in the heat of battle make a decisions in a split second to turn his
boat around, drive it through an enemy position, and chase down the
enemy to save his crew. Decisive. Strong. Is this not what we need in a
commander in chief?

You know, we hear a lot of talk about values. Where I come from, you
don't judge somebody's values based upon how they use that word in a
political ad. You judge their values based upon what they've spent their
life doing.

So when a man volunteers to serve his country, a man volunteers and puts
his life on the line for others that's a man who represents real
American values. This is a man who is prepared to keep the American
people safe, to make America stronger at home and more respected in the
world. John is a man who knows the difference between right and wrong.
He wants to serve you, your cause is his cause. And that is why we must
and we will elect him the next president of the United States.

You know, for the last few months, John has been traveling around the
country talking about his positive, optimistic vision for America,
talking about his plan to move this country in the right direction.

But what have we seen? Relentless negative attacks against John. So in
the weeks ahead, we know what's coming don't we? More negative attacks.
Aren't you sick of it? They are doing all they can to take the campaign
for the highest office in the land down the lowest possible road. But
this is where you come in. Between now and November, you, the American
people, you can reject this tired, old, hateful, negative politics of
the past. And instead you can embrace the politics of hope, the politics
of what's possible because this is America, where everything is
possible.

I am here tonight for a very simple reason, because I love my country.
And I have every reason to love my country. I have grown up in the
bright light of America. I grew up in a small town in rural North
Carolina, a place called Robbins. My father, he worked in a mill all his
life. And I still remember vividly the men and women who worked in that
mill with him. I can see them. Some of them had lint in their hair. Some
of them had grease on their faces. They worked hard. And they tried to
put a little money away so that their kids and their grandkids could
have a better life. The truth is they are just like the auto workers,
the office workers, the teachers, and the shop keepers on Main Streets
all across this cuontry.

My mother had a number of jobs. She worked at the post office so she and
my father could have health care. She owned her own small business, she
refinished furniture to help pay for my education.

I have had such incredible opportunities in my life.I was blessed to be
the first person in my family to be able to go to college. I worked my
way through, and I had opportunities beyond my wildest dreams. . And the
heart of this campaign, your campaign, our campaign, is to make sure all
Americans have exactly the same kind of opportunities that I had, no
matter where you live, no matter who your family is, no matter what the
color of your skin. This is the America we believe in.

I have spent my life fighting for the kind of people that I grew up
with. For two decades, I stood with kids and families against big HMOs
and big insurance companies. When I got to the Senate I fought those
same fights against the Washington lobbyists and for causes like the
Patients' Bill of Rights.

I stand here tonight ready to work with you and John to make America
stronger. And we have much work to do. Because the truth is, we still
live in a country where there are two different Americas: one for all
those people who have lived the American Dream and don't have to worry,
and another for most Americans, everybody else, who struggle to make
ends meet every single day. It doesn't have to be that way.

We can build one America. We can build one America where we no longer
have two healthcare systems. One for families who get the best
healthcare money can buy and then one for everybody else, rationed out
by insurance companies, drug companies, and HMOs. Millions of Americans
who don't have any health insurance at all. It doesn't have to be that
way.

We have a plan that will offer all Americans the same health care that
your Senator has. We can give you tax breaks to help you pay for your
health care. And when we're in office, we will sign a real Patients'
Bill of Rights into law so that you can make your own health care
decisions.

We shouldn't have two public school systems in this country: one for the
most affluent communities, and one for everybody else. None of us
believe that the quality of a child's education should be controlled by
where they live or the affluence of the community they live in. It
doesn't have to be that way.

We can build one school system that works for all our kids, gives them a
chance to do what they're capable of doing. Our plan will reform our
schools and raise standards. We can give our schools the resources that
they need. We can provide incentives to put our best teachers in the
subjects and the places where we need them the most. And we can ensure
that three million children have a safe place when they leave school in
the afternoon. We can do together, you and I.

John Kerry and I believe that we shouldn't have two different economies
in this country: one for people who are set for life, they know their
kids and their grandkids are going to be just fine, and then one for
most Americans, people who live paycheck to paycheck. You don't need me
to explain this to you, do you?

You know exactly what I'm talking about Can't save any money, can you?
Takes every dime you make just to pay your bills. And you know what
happens if something goes wrong, if you have a child that gets sick, a
financial problem, a layoff in the family, you go right off the cliff.
And when that happens, what's the first thing that goes? Your dreams. It
doesn't have to be that way. We can strengthen and lift up your
families.

Your agenda is our agenda. So let me give you some specifics. First, we
can create good paying jobs in this country again. We're going to get
rid of tax cuts for companies who are outsourcing your jobs. And
instead, we're going to give tax breaks to American companies that are
keeping jobs right here in America. And we will invest in the jobs of
the future, in the technologies and innovation to ensure that America
stays ahead of the competition.

And we're going to do this because John and I understand understand that
a job is about more than a paycheck. It's about dignity and
self-respect. Hard work should be valued in this country. So we're going
to reward work, not just wealth. We don't want people to just get by; we
want people to get ahead. So let me give you some specifics about what
we're going to do.. First, we're going to help you pay for your health
care by having a tax break and health care reform that can save you up
to a thousand dollars on your premiums.. We're going to help you cover
the rising costs of child care with a tax credit up to \$1,000 so that
your kids have a place to go when you're at work that they're safe and
well taken care of. If your child wants to be the first in your family
to go to college, we're going to give you a tax break on up to \$4,000
in tuition.

And everybody listening here and at home is thinking one thing right
now. O.K., how are we going to pay for it, right? Well, let me tell you
how we're going to pay for it. And I want to be very clear about this.
We are going to keep and protect the tax cuts for 98 percent of
Americans, 98 percent. We're going to roll back the tax cuts for the
wealthiest Americans. We're going to close corporate loopholes. We're
going to cut government contractors and wasteful spending. We can move
this country forward without passing the burden to our children and our
grandchildren.

We can also do something about 35 million Americans who live in poverty
every day. And here's why we shouldn't just talk about but do something
about the millions of Americans who live in poverty. Because it is
wrong. And we have a moral responsibility to lift those families up.

I mean the very idea that in a country of our wealth and our prosperity,
we have children going to bed hungry. We have children who don't have
the clothes to keep them warm. We have millions of Americans who work
full-time every day to support their families, working for minimum wage
and still live in poverty. It's wrong. These are men and women who are
living up to their bargain. They're working hard, they're supporting
their families. Their families are doing their part; it's time we did
our part.

And that's what we're going to do, that's what we're going to do when
John is in the White House. Because we're going to raise the minimum
wage. We're going to finish the job on welfare reform. And we're going
to bring good paying jobs to the places where we need them the most. .
And by doing all those things we're going to say no forever to any
American working full-time and living in poverty. Not in our America,
not in our America. Not in our America. Not in our America.

And let me talk - let me talk about - let me talk about - let me talk
about about why we need to build one America. Because I, like many of
you, I saw up close what having two Americas - what having two Americas
can do to our country. From the time I was very young, I saw the ugly
face of segregation and discrimination. I saw young African-American
kids being sent upstairs in movie theaters. I saw white only signs on
restaurant doors and luncheon counters. I feel such an enormous personal
responsibility when it comes to issues of race and equality and civil
rights.

And I've heard some discussions- I've heard some discussions and debates
around America about where, and in front of what audiences we ought to
talk about race and equality and civil rights. I have an answer to that
question. Everywhere. Everywhere. Everywhere.

This is not an African-American issue, this is not a Latino issue, this
is not an Asian-American issue, this is an American issue. It is about
who we are, what our values are and what kind of country we live in.

The truth is - the truth is is that what John and I want - what we all
want - is for our children and our grandchildren to be the first
generations that grow up in an America that's no longer divided by race.
We must build one America. We must be one America, strong and united for
another very important reason - because we are at war.

None of us will ever forget where we were on September the 11th. We all
share the same terrible images: the Towers falling in New York, the
Pentagon in flames, smoldering field in Pennsylvania. We share the
profound sadness for the nearly 3,000 lives that were lost.

As a member of the Senate Intelligence Committee, I know that we have to
do more to fight the war on terrorism and keep the American people safe.
And we can do that. We're approaching the third anniversary of Sept. 11,
and one thing I can tell you: when we're in office, it won't take three
years to get the reforms in our intelligence that are necessary to keep
the American people safe. We will do whatever it takes, as long as it
takes, to make sure this never happens again in our America.

And when John is president, we will listen to the wisdom of the Sept. 11
Commission. We will lead strong alliances. We will safeguard and secure
our weapons of mass destruction. We will strengthen our homeland
security, protect our ports, protect our chemical plants, and support
our firefighters, police officers, EMT's. We will always - we will
always use our military might to keep the American people safe.

And we - John and I -we will have one clear unmistakable message for Al
Qaida and these terrorists: You cannot run. You cannot hide. We will
destroy you.

John understands personally about fighting in a war. And he knows what
our brave men and women are going through right now in another war - the
war in Iraq. The human cost and the extraordinary heroism of this war,
it surrounds us. It surrounds us in our cities and our towns. And we'll
win this war because of the strength and courage of our own people.

Some of our friends and neighbors, they saw their last images in
Baghdad. Some took their last steps outside of Fallujah. Some buttoned
their uniform for the final time before they went out and saved their
unit. Men and women who used to take care of themselves, they now count
on others to see them through the day. They need their mother to tie
their shoe. Their husband to brush their hair. Their wife's arm to help
them across the room.

The stars and stripes wave for them. The word hero was made for them.
They are the best and the bravest. And they will never be left behind.
You - you - you understand that. And they deserve a president who
understands - understands it on the most personal level what they've
gone through, what they've given and what they've given up for their
country.

To us, the real test of patriotism is how we treat the men and women who
have put their lives on the lines to protect our values. And let me tell
you, the 26 million veterans in this country will not have to wonder,
when they're in office - when we're in office, whether they'll have
health care next week or next year. We will take care of them because
they have taken care of us.

But today, our great United States military is stretched thin. We've got
more than 140,000 are in Iraq, almost 20,000 in Afghanistan. I visited
the men and women there and we're praying as they try to give that
country hope.

Like all of those brave men and women, John put his life on the line for
our country. He knows that when authority is given to a president, much
is expected in return. That's why we will strengthen and modernize our
military. We will double our Special Forces. We will invest in the new
equipment and technologies so that our military remains the best
equipped and the best prepared in the world. This will make our military
stronger. It'll make it sure that can defeat any enemy in this new
world.

But we can't do this alone. We have got to restore our respect in the
world to bring our allies to us and with us. It is how we won the cold
war. It is how we won two world wars. And it is how we will build a
stable Iraq.

With a new president who strengthens and leads our alliances, we can get
NATO to help secure Iraq. We can ensure that Iraq's neighbors like Syria
and Iran, don't stand in the way of a democratic Iraq. We can help
Iraq's economy by getting other countries to forgive their enormous debt
and participate in the reconstruction. We can do this for the Iraqi
people and we can do it for our own soldiers. And we will get this done
right.

A new president will bring the world to our side, and with it a stable
Iraq, a real chance for freedom and peace in the Middle East, including
a safe and secure Israel. And John and I will bring the world together -
John and I will bring the world together to face the most dangerous
threat we have: the possibility of terrorists getting their hands on a
chemical, biological or nuclear weapon.

With our credibility restored, we can work with other nations to secure
stockpiles of the world's most dangerous weapons and safeguard this
extraordinarily dangerous material. We can finish the job and secure the
loose nukes in Russia. We can close the loophole in the Nuclear
Nonproliferation Treaty that allows rogue nations access to the tools
they need to develop these weapons.

That's how we can address the new threats we face. That's how we can
keep you safe. And that's how we can restore America's respect around
the world. And together, we will ensure that the image of America, the
image all of us love, America this great shining light, this beacon of
freedom, democracy and human rights that the world looks up to, is
always lit.

And the truth is - the truth is that every child, every family in
America will be safer and more secure if they grow up in a world where
America is once again looked up to and respected. That is the world we
can create together.

Tonight - tonight, as we celebrate in this hall, somewhere in America, a
mother sits at her kitchen table. She can't sleep becauseshe's worried.
She can't pay her bills. She's working hard trying to pay her rent,
trying to feed her kids but she just can't catch up. Didn't used to be
that way in her house. Her husband was called up in the Guard. Now he's
been in Iraq for over a year. They thought hewas going to come home last
month, but now he's got to stay longer. She thinks she's alone. But
tonight in this hall and in your homes, you know what? She's got a lot
of friends. We want her to know that we hear her. It is time to bring
opportunity and an equal chance to her door.

We're here to make America stronger at home so that she can get ahead.
And we're here to make America respected in the world again so that we
can bring him home and American soldiers don't have to fight this war in
Iraq or this war on terrorism alone.

So when you return home some night, you might pass a mother on her way
to work the late-shift. You tell her: Hope is on the way.

When your brother calls - when your brother calls and says that he's
spending his entire life at the office and he still can't get ahead, you
tell him: Hope is on the way.

When your parents call and tell you their medicine's going through the
roof, they can't keep up, you tell them: Hope is on the way.

And when your neighbor calls you and says her daughter's worked hard and
she wants to go to college, you tell her: Hope is on the way.

And when your son or daughter who's serving this country heroically in
Iraq calls, you tell them: Hope is on the way.

When you wake up and you're sitting at the kitchen table with your kids
and you're talking about the great possibilities in America, your kids
should know that John and I believe to our core that tomorrow can be
better than today.

Like all of us, I've learned a lot of lessons in my life. Two of the
most important are that first, there will always be heartache and
struggle. We can't make it go away. But the second is that people of
good and strong will can make a difference. One's a sad lesson; the
other's inspiring. We are Americans and we choose to be inspired.

We choose hope over despair; possibilities over problems, optimism over
cynicism. We choose to do what's right even when those around us say,
"You can't do that." We choose to be inspired because we know that we
can do better because this is America where everything is still
possible.

What we believe - what John Kerry and I believe - is that we should
never look down on anybody, we ought to lift people up. We don't believe
in tearing people apart. We believe in bringing them together. What we
believe - what I believe - is that the family you're born into and the
color of your skin in our America should never control your destiny.

Join us in this cause. Let's make America stronger at home and more
respected in the world. Let's ensure that once again, in our one America

\begin{itemize}
\tightlist
\item
  our one America - tomorrow will always be better than today.
\end{itemize}

Thank you. God bless you. And God bless the United States of America.

Advertisement

\protect\hyperlink{after-bottom}{Continue reading the main story}

\hypertarget{site-index}{%
\subsection{Site Index}\label{site-index}}

\hypertarget{site-information-navigation}{%
\subsection{Site Information
Navigation}\label{site-information-navigation}}

\begin{itemize}
\tightlist
\item
  \href{https://help.nytimes3xbfgragh.onion/hc/en-us/articles/115014792127-Copyright-notice}{©~2020~The
  New York Times Company}
\end{itemize}

\begin{itemize}
\tightlist
\item
  \href{https://www.nytco.com/}{NYTCo}
\item
  \href{https://help.nytimes3xbfgragh.onion/hc/en-us/articles/115015385887-Contact-Us}{Contact
  Us}
\item
  \href{https://www.nytco.com/careers/}{Work with us}
\item
  \href{https://nytmediakit.com/}{Advertise}
\item
  \href{http://www.tbrandstudio.com/}{T Brand Studio}
\item
  \href{https://www.nytimes3xbfgragh.onion/privacy/cookie-policy\#how-do-i-manage-trackers}{Your
  Ad Choices}
\item
  \href{https://www.nytimes3xbfgragh.onion/privacy}{Privacy}
\item
  \href{https://help.nytimes3xbfgragh.onion/hc/en-us/articles/115014893428-Terms-of-service}{Terms
  of Service}
\item
  \href{https://help.nytimes3xbfgragh.onion/hc/en-us/articles/115014893968-Terms-of-sale}{Terms
  of Sale}
\item
  \href{https://spiderbites.nytimes3xbfgragh.onion}{Site Map}
\item
  \href{https://help.nytimes3xbfgragh.onion/hc/en-us}{Help}
\item
  \href{https://www.nytimes3xbfgragh.onion/subscription?campaignId=37WXW}{Subscriptions}
\end{itemize}
