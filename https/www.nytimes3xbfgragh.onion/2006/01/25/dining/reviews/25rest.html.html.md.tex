Sections

SEARCH

\protect\hyperlink{site-content}{Skip to
content}\protect\hyperlink{site-index}{Skip to site index}

\href{https://www.nytimes3xbfgragh.onion/pages/dining/index.html}{Dining
\& Wine}

\href{https://myaccount.nytimes3xbfgragh.onion/auth/login?response_type=cookie\&client_id=vi}{}

\href{https://www.nytimes3xbfgragh.onion/section/todayspaper}{Today's
Paper}

\href{/pages/dining/index.html}{Dining \& Wine}\textbar{}Stuffed Pork

\begin{itemize}
\item
\item
\item
\item
\item
\end{itemize}

Advertisement

\protect\hyperlink{after-top}{Continue reading the main story}

Supported by

\protect\hyperlink{after-sponsor}{Continue reading the main story}

Restaurants

\hypertarget{stuffed-pork}{%
\section{Stuffed Pork}\label{stuffed-pork}}

By FRANK BRUNI

\begin{itemize}
\item
  Jan. 25, 2006
\item
  \begin{itemize}
  \item
  \item
  \item
  \item
  \item
  \end{itemize}
\end{itemize}

IN just one example of my lidless optimism and bottomless foolishness, I
recently visited the Spotted Pig on a Friday night at about 7:30, not
exactly an off hour.

What I encountered looked less like a restaurant than a mosh pit. I
spent 10 minutes trying to press through the mass of bodies around the
front door and flag down the host. He told me that the wait for a table

\begin{itemize}
\tightlist
\item
  the Spotted Pig accepts reservations only in special cases - would be
  nearly two hours. I took a pass, because I had this thing called
  hunger gnawing at me, and vowed to be cleverer about my next Pigward
  journey.
\end{itemize}

That happened on a Sunday night at 6:15. And didn't go much better. The
mass at the door was thinner, but where oh where was the host?

"In hiding," cracked a server who passed by, which was funny but then
again not. When the host finally appeared, he projected a wait of at
least 20 minutes. Not so bad, but no stools were available at the
downstairs bar or the upstairs bar and it was freezing out on the
sidewalk, where many a Pig aspirant loiters.

My friends and I climbed the stairs, pushed to get within shouting
distance of a bartender and clamored, without much success, for his
attention. It would be about a half-hour before we were shown to an
insanely cramped table downstairs and 15 minutes more before our harried
server could deal with us.

The Spotted Pig may well be Manhattan's most unforgiving, uncomfortable
trough, the gastropub as gastromelee. Almost immediately after it opened
in March 2004 and began serving its sometimes heroically satisfying
combination of English and Italian cooking, the throngs started to
descend, and they have never stopped.

So the Pig, inevitably, has porked up. Late last month, with the opening
of that upstairs room, it more than doubled in size, to about 110 seats
from about 50. But so far, it seems, the waits at dinnertime are as
long, and the crowds as dense, as ever. The Pig should give you more
than a menu. It should hand out a special Kama Sutra on the contortions
necessary to get to and from your seat.

Like so much in life, the Pig proves contradictory truths: that New
Yorkers are fools, willing to endure any manner of nonsense to run with
the pack, and that New Yorkers are sages, able to divine and embrace
genuine merit in the middle of bedlam.

On the tables there is merit aplenty. Two hours might in fact be an
acceptable wait for the Pig's fantastic gnudi, which are soft, rich
dumplings made of sheep's milk ricotta and topped with fried sage
leaves, browned butter and Parmesan.

Refined or unruly, few restaurants devise such a prudent assignment for
the overextended beet, paired here with luscious smoked trout, crème
fraîche and a generous smattering of chives. Few produce a chowder with
as much depth of flavor as the Pig's, which combines smoked haddock,
pancetta, cream and a barely visible sheen of olive oil.

April Bloomfield, the chef and a principal owner, favors smoked things,
cured things, rich things and salty things. Her menu encourages grazing,
presenting many snacks (roasted almonds, marinated olives, duck egg with
tuna bottarga) and few conventionally portioned entrees.

It can be broken down into pub-ready fare that logically suits the
ambience (chicken liver toast, prunes wrapped in bacon, grilled burger
with Roquefort cheese) and more elevated compositions (sautéed quail
with pomegranates and a balsamic drizzle, a squash salad with pine nuts
and pecorino).

Want evidence of how expertly a kitchen can operate in the eye of a
storm? Order the roasted black bass. When I had it, the fish was
astonishingly moist, and the herbs atop it (cilantro, thyme, flat-leaf
parsley) were so fresh that my companion wondered aloud if the Pig had
its own roof garden.

In fact, it has plans for one. The restaurant's obvious commitment to
superior ingredients helps explain the Michelin star it received. Or
maybe the Michelin inspectors were won over by the interesting selection
of ales and the individual press pots in which coffee and tea are
served.

A random Pig fact: Ms. Bloomfield's main partner in this endeavor, Ken
Friedman, who got it off the ground, said he wanted to call it the
Prodigal Pig, but his friends kept asking what prodigal meant. "Maybe
they're not so smart," he said in a recent telephone conversation.

He said he liked the sound of the Spotted Pig and considered it an
allusion to one of his advisers and investors, Mario Batali.

How so?

"He's got freckles," Mr. Friedman said. "That's on the record."

Pieces of Pig advice: Go for lunch on a weekday. It's not as crowded, a
hefty sandwich of pork and pickled jalapeño is available, and light
streams in the big downstairs windows.

Skip weekend brunch. Although the Bloody Mary is made with freshly
grated horseradish, too much roasted garlic and excessive saltiness
overwhelmed a frittata. French toast was more like barely cooked bread
that had barely been touched by its "warm bourbon maple syrup."

If you're intent on going at a normal dinner hour, do a searing personal
inventory of the sturdiness of various body parts. Knees O.K.? You may
be on your feet a good long while. Lower vertebrae in good shape? You
may be seated on one of many tiny stools without backs.

Bladder strong? The line for the small unisex bathroom downstairs can be
long. But it can also be interesting. Two young, pretty women entered
the bathroom together. And stayed for a bit.

While they were in there, a server and I exchanged amused glances, and
my weariness with so much standing and waiting went away. With its
festive spirit and with the best of its food, the Pig can make that
happen.

\textbf{The Spotted Pig}

*

314 West 11th Street (Greenwich Street), Greenwich Village; (212)
620-0393.

ATMOSPHERE A charmingly scruffy, dark-wooded corner space is the theater
for this severely chaotic and extremely cramped gastropub.

SOUND LEVEL Take a guess.

RECOMMENDED DISHES Smoked haddock chowder; beets with smoked trout;
gnudi; braised veal breast; black bass; pork sandwich with pickled
jalapeño; burger with Roquefort; prune and Armagnac tart.

WINE LIST A serious but not sprawling international selection with many
bottles between \$26 and \$50, supplemented by an interesting selection
of beers.

PRICE RANGE Brunch dishes, \$11 to \$18. Lunch and dinner snacks, \$2.50
to \$7; small and large plates, \$12 to \$30; desserts, \$7.

HOURS Noon to 2 a.m. Monday through Friday, with limited menu from 3 to
5 p.m. From 11 a.m. to 5 p.m. and from 5:30 p.m. to 2 a.m. on Saturday
and Sunday.

RESERVATIONS Not accepted, except from neighborhood regulars and special
guests in special cases.

CREDIT CARDS All major cards.

WHEELCHAIR ACCESS Ramp available for small step up to entrance;
accessible restroom, but conditions are very crowded.

WHAT THE STARS MEAN:\\
(None) Poor to satisfactory\\
* Good\\
** Very good\\
*** Excellent\\
**** Extraordinary\\
Ratings reflect the reviewer's reaction to food ambience and service,
with price taken into consideration. Menu listings and prices are
subject to change.

Advertisement

\protect\hyperlink{after-bottom}{Continue reading the main story}

\hypertarget{site-index}{%
\subsection{Site Index}\label{site-index}}

\hypertarget{site-information-navigation}{%
\subsection{Site Information
Navigation}\label{site-information-navigation}}

\begin{itemize}
\tightlist
\item
  \href{https://help.nytimes3xbfgragh.onion/hc/en-us/articles/115014792127-Copyright-notice}{©~2020~The
  New York Times Company}
\end{itemize}

\begin{itemize}
\tightlist
\item
  \href{https://www.nytco.com/}{NYTCo}
\item
  \href{https://help.nytimes3xbfgragh.onion/hc/en-us/articles/115015385887-Contact-Us}{Contact
  Us}
\item
  \href{https://www.nytco.com/careers/}{Work with us}
\item
  \href{https://nytmediakit.com/}{Advertise}
\item
  \href{http://www.tbrandstudio.com/}{T Brand Studio}
\item
  \href{https://www.nytimes3xbfgragh.onion/privacy/cookie-policy\#how-do-i-manage-trackers}{Your
  Ad Choices}
\item
  \href{https://www.nytimes3xbfgragh.onion/privacy}{Privacy}
\item
  \href{https://help.nytimes3xbfgragh.onion/hc/en-us/articles/115014893428-Terms-of-service}{Terms
  of Service}
\item
  \href{https://help.nytimes3xbfgragh.onion/hc/en-us/articles/115014893968-Terms-of-sale}{Terms
  of Sale}
\item
  \href{https://spiderbites.nytimes3xbfgragh.onion}{Site Map}
\item
  \href{https://help.nytimes3xbfgragh.onion/hc/en-us}{Help}
\item
  \href{https://www.nytimes3xbfgragh.onion/subscription?campaignId=37WXW}{Subscriptions}
\end{itemize}
