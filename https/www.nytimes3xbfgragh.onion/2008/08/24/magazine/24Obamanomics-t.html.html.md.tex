Sections

SEARCH

\protect\hyperlink{site-content}{Skip to
content}\protect\hyperlink{site-index}{Skip to site index}

\href{https://myaccount.nytimes3xbfgragh.onion/auth/login?response_type=cookie\&client_id=vi}{}

\href{https://www.nytimes3xbfgragh.onion/section/todayspaper}{Today's
Paper}

Obamanomics

\begin{itemize}
\item
\item
\item
\item
\item
\end{itemize}

Advertisement

\protect\hyperlink{after-top}{Continue reading the main story}

Supported by

\protect\hyperlink{after-sponsor}{Continue reading the main story}

\hypertarget{obamanomics}{%
\section{Obamanomics}\label{obamanomics}}

By \href{https://www.nytimes3xbfgragh.onion/by/david-leonhardt}{David
Leonhardt}

\begin{itemize}
\item
  Aug. 20, 2008
\item
  \begin{itemize}
  \item
  \item
  \item
  \item
  \item
  \end{itemize}
\end{itemize}

\textbf{I. A Broken Economy}

As Barack Obama prepares to accept the Democratic nomination this week,
it is clear that the economic policies of the next president are going
to be hugely important. Ever since Wall Street bankers were called back
from their vacations last summer to deal with the convulsions in the
mortgage market, the economy has been lurching from one crisis to the
next. The International Monetary Fund has described the situation as
``the largest financial shock since the Great Depression.'' The details
are too technical for most of us to understand. (They're too technical
for many bankers to understand, which is part of the problem.) But the
root cause is simple enough. In some fundamental ways, the American
economy has stopped working.

The fact that the economy grows --- that it produces more goods and
services one year than it did in the previous one --- no longer ensures
that most families will benefit from its growth. For the first time on
record, an economic expansion seems to have ended without family income
having risen substantially. Most families are still
\href{http://www.census.gov/hhes/www/income/histinc/f07ar.html}{making
less}, after accounting for inflation, than they were in 2000. For these
workers, roughly the bottom 60 percent of the income ladder, economic
growth has become a theoretical concept rather than the wellspring of
better medical care, a new car, a nicer house --- a better life than
their parents had.

Americans have still been buying such things, but they have been doing
so with debt. A big chunk of that debt will never be repaid, which is
the most basic explanation for the financial crisis. Even after the
crisis has passed, the larger problem of income stagnation will remain.
It's hardly the economy's only serious problem either. There is also the
slow unraveling of the employer-based health-insurance system and the
fact that, come 2011, the baby boomers will start to turn 65, setting
off an enormous rise in the government's Medicare and Social Security
obligations.

Most of these problems aren't immediate, which helps explain why they
have gone unaddressed for so long. And the United States remains a
fabulously prosperous country, relative to almost any other country, at
any point in history. Yet Americans seem to realize that something has
gone wrong. In recent polls, about 80 percent of
\href{http://nytimes3xbfgragh.onion/polls}{respondents say} the economy
is in bad shape, and almost 70 percent say it's going to get worse.
Together, these answers make for the most downbeat assessment since at
least the early 1980s, and underscore that the next president will be
inheriting a set of domestic problems as serious as any the country has
faced in a long time.

John McCain's
\href{http://www.johnmccain.com/Informing/News/Speeches/4c980d5b-dfd3-40a3-9663-7d14df1f1468.htm}{economic
vision}, as he has laid it out during the campaign, amounts to a
slightly altered version of Republican orthodoxy, with tax cuts at the
core. Obama, on the other hand, has more-detailed proposals but a less
obvious ideology.

Well before this point on the presidential calendar, it's usually clear
where a candidate fits within the political spectrum of his party. With
Obama, there is vast disagreement about just how liberal he is,
especially on the economy. My favorite example came in mid-June, shortly
after Obama named Jason Furman, a protégé of Robert Rubin, the centrist
former Treasury secretary, as his lead economic adviser. Labor leaders
\href{http://articles.latimes.com/2008/jun/11/nation/na-furman11}{recoiled},
and John Sweeney, the head of the A.F.L.-C.I.O., worried aloud about
``corporate influence on the Democratic Party.'' Then, the following
week, Kimberley Strassel, a member of The Wall Street Journal editorial
board, wrote a column titled,
``\href{http://www.wsj.com/public/article_print/SB121391937825890363.html}{Farewell,
New Democrats},'' concluding that Obama's economic policies amounted to
the end of Clintonian centrism and a reversion to old liberal ways.

Some of the confusion stems from Obama's own strategy of presenting
himself as a postpartisan figure. A few weeks ago, I joined him on a
flight from Orlando to Chicago and began our conversation by asking
about his economic approach. He started to answer, but then interrupted
himself. ``My core economic theory is pragmatism,'' he said, ``figuring
out what works.''

This, of course, is not the whole story. Invoking pragmatism doesn't
help the average voter much; ideology, though it often gets a bad name,
matters, because it offers insight into how a candidate might actually
behave as president. I have spent much of this year trying to get a
handle on what is sometimes called Obamanomics and have come away
thinking that Obama does have an economic ideology. It's just not a
completely familiar one. Depending on how you look at it, he is both
more left-wing and more right-wing than many people realize.

\textbf{II. A New Democratic Consensus, of Sorts}

To understand where Obama stands, you first have to know that, for 15
years, Democratic Party economics have been defined by a struggle that
took place during the start of the Clinton administration. It was the
battle of the Bobs. On one side was Clinton's labor secretary and
longtime friend, Bob Reich, who argued that the government should invest
in roads, bridges, worker training and the like to stimulate the economy
and help the middle class. On the other side was Bob Rubin, a former
Goldman Sachs executive turned White House aide, who favored reducing
the deficit to soothe the bond market, bring down interest rates and get
the economy moving again. Clinton cast his lot with Rubin, and to this
day the first question about any Democrat's economic outlook is often
where his heart lies, with Reich or Rubin, the left or the center, the
government or the market.

Obama has obviously studied this debate, and early on during the flight
to Chicago, he told me a story about Reich and Rubin. The previous week,
Obama convened a discussion with a high-powered group of economists and
chief executives. He was sitting at a conference table, with Rubin two
seats to his left and Reich across from him. ``One of the points I
raised,'' Obama told me, ``is if you just use you, Bob, and you, Bob, as
caricatures, the truth is, both of you acknowledge the world is more
complicated.'' By this, Obama didn't simply mean that their views were
more nuanced than many outsiders understood. He meant that both have
come to acknowledge that the other man is, in part, correct. The two now
occupy more similar ideological places than they did in 1993. The battle
of the Bobs may not be completely over, but it has certainly been
\href{http://www.nytimes3xbfgragh.onion/2007/06/10/magazine/10wwln-summers-t.html}{suspended}.

Among the policy experts and economists who make up the Democratic
government-in-waiting, there is now something of a consensus. They agree
that deficit reduction did an enormous amount of good. It helped usher
in the 1990s boom and the only period of strong, broad-based income
growth in a generation. But that boom also depended on a technology
bubble and historically low oil prices. In the current decade, the
economy has continued to grow at a decent pace, yet most families have
seen little benefit. Instead, the benefits have flowed mostly to a small
slice of workers at the very top of the income distribution. As Rubin
told me, comparing the current moment with 1993, ``The distributional
issues are obviously more serious now.'' From today's vantage point,
inequality looks likes a bigger problem than economic growth; fiscal
discipline seems necessary but not sufficient.

In practical terms, the new consensus means that the policies of an
Obama administration would differ from those of the Clinton
administration, but not primarily because of differences between the two
men. ``The economy has changed in the last 15 years, and our
understanding of economic policy has changed as well,'' Furman says.
``And that means that what was appropriate in 1993 is no longer
appropriate.'' Obama's agenda starts not with raising taxes to reduce
the deficit, as Clinton's ended up doing, but with changing the tax code
so that families making more than \$250,000 a year pay more taxes and
nearly everyone else pays less. That would begin to address inequality.
Then there would be Reich-like investments in alternative energy,
physical infrastructure and such, meant both to create middle-class jobs
and to address long-term problems like global warming.

All of this raises the question of what will happen to the deficit.
Obama's aides optimistically insist he will reduce it, thanks to his tax
increases on the affluent and his plan to wind down the Iraq war.
Relative to McCain, whose promised spending cuts are
\href{http://www.nytimes3xbfgragh.onion/2008/04/23/business/23leonhardt.html}{extremely
vague}, Obama does indeed look like a fiscal conservative. But the
larger point is that the immediate deficit isn't as big as it was in
1992. Then, it was equal to
\href{http://www.cbo.gov/budget/data/historical.pdf}{4.7 percent} of
gross domestic product. Right now it's about 2.5 percent.

During our conversation, Obama made it clear that he considered the
deficit to be only one of the long-term problems requiring immediate
attention, and he sounded more worried about the others, like global
warming, health care and the economic hangover that could follow the
housing bust. Tellingly, he said that while he admired what Clinton did,
he might have been more open to Reich's argument --- even in 1993. ``I
still would have probably made a slightly different choice than Clinton
did,'' Obama said. ``I probably wouldn't have been as obsessed with
deficit reduction.''

The new Democratic consensus isn't complete, obviously. Labor unions, in
particular, would prefer more trade barriers than many other Democrats.
During the primaries Obama nodded, and at times pandered, in this
direction. Since then, he has
\href{http://money.cnn.com/2008/06/18/magazines/fortune/easton_obama.fortune/index.htm?postversion=2008061815}{disavowed}
that rhetoric, to almost no one's surprise. Yet his zig-zagging on the
issue did highlight the biggest weak spot in his, and his party's,
economic agenda. He still hasn't quite figured out how to sell it. For
all his skills as a storyteller and a speaker, he has not settled on a
compelling message about how to put the economy on the right path.

Image

Credit...Tim Davis for The New York Times

The lack of such a message has contributed to several of his worst
moments over the last year. Most recently, the campaign has come out
with a series of small-bore, populist energy plans --- a
windfall-profits tax on oil companies, a crackdown on speculators, a
partial opening of the strategic oil reserve --- that seem more
political than economic. The most glaring misstep on this score was his
comment this spring about bitter rural voters clinging to guns and
religion. It was, in effect, an admission that his own message about the
economy hadn't yet broken through.

\textbf{III. A `University of Chicago' Democrat}

Starting in the early 1990s, Obama spent 12 years at the University of
Chicago, mostly as a senior lecturer on constitutional law. It was a
part-time job that helped him make money while he began to build his
political career. But it also happened to place him inside what is
arguably the intellectual center of modern American economic
conservatism, the home of Milton Friedman and the laissez-faire
philosophy known as the Chicago School of economics. By all accounts,
Obama didn't spend much time with Friedman's disciples at the law
school. Instead, he became friendly with another crowd: liberals who had
come to think that Friedman was right about a lot, just not everything.

The
\href{http://www.pbs.org/wgbh/commandingheights/shared/video/qt/mini_p01_11_a_56.html}{Chicago
School} believes that markets --- that is, millions of individuals
making separate decisions --- almost always function better than
economies that are managed by governments. In a market system, prices
adjust whenever there is a shortage or a glut, and the problem soon
resolves itself. Just as important, companies constantly compete with
each other, which helps bring down prices, improves the quality of goods
and ultimately lifts living standards.

In its more extreme forms, the Chicago School's ideas have some obvious
flaws. History has shown that free markets aren't so good at, say,
preventing pollution or the issuance of fantastically unrealistic
mortgages. But over the last few decades, as Europe's regulated
economies have struggled and Asia's move toward capitalism has spurred
its fabulous boom, many liberals have also come to appreciate the
virtues of markets.

One of these liberals is
\href{http://www.law.harvard.edu/faculty/directory/facdir.php?id=552}{Cass
Sunstein}, a prolific law professor who sometimes ate with Obama in the
open, sunlit cafeteria off the lobby of the main building at Chicago's
law school. Over sandwiches in that cafeteria this spring, Sunstein told
me that he didn't think that Obama arrived at the law school as an
old-style liberal or departed as anything like a Friedmanite. Yet
Sunstein and other former Chicago colleagues I spoke with said they
believed that Chicago had helped give Obama an intellectual framework
for his instincts, at the least, and probably made him come to
appreciate markets more.

Obama, when I asked him, agreed that his years surrounded by Chicago
School thinking affected him. He tends to assign his motives to more
intimate narratives, though, and he said that his grandmother, a
high-school graduate who rose to become the vice president of a bank and
was the family's main breadwinner, had the biggest impact. ``She had to
think very practically about, How do you make money?'' he told me. ``How
does the system work? That led me to have an orientation to ask
hardheaded questions. During my formative years, there was still
ideological competition between a social-democratic or even socialist
agenda and a free-market, Milton Friedman agenda. I think it was natural
for me to ask questions of both sides and maybe try to synthesize
approaches.''

There is plenty of evidence that this synthesis isn't merely a part of a
candidate's inevitable tack to the center for a general election. In
Obama's memoir, ``Dreams From My Father,'' he sympathetically recounts a
conversation he had with a Kenyan farmer, in which the man complains
both about rich people who won't pay their fair share of taxes and about
burdensome government regulations on coffee growing. In Obama's second
book, ``The Audacity of Hope,'' he goes further: ``Reagan's central
insight --- that the liberal welfare state had grown complacent and
overly bureaucratic, with Democratic policy makers more obsessed with
slicing the economic pie than with growing that pie --- contained a good
deal of truth.''

The partial embrace of Reaganomics is a typical bit of Obama's
postpartisan veneer. In a single artful sentence, he dismissed the old
liberals, aligned himself with the Bill Clinton centrists and did so by
reaching back to a conservative icon who remains widely popular. But the
words have significance at face value too. Compared with many other
Democrats, Obama simply is more comfortable with the apparent successes
of laissez-faire economics.

Sunstein, now on the faculty at Harvard, has a name for this approach:
``I like to think of him as a `University of Chicago' Democrat.''

It's a useful label. Today's Democratic consensus has moved the party to
the left, and on issues like inequality and climate change, Obama
appears willing to be even more aggressive than many fellow Democrats.
From this standpoint, he's a true liberal. Yet he also says he believes
that there are significant parts of Reaganism worth preserving. So his
policies often involve setting up a government program to address a
market failure but then trying to harness the power of the market within
that program. This, at times, makes him look like a conservative
Democrat.

From the beginning, Obama has sought out academic economists, rather
than lawyers or former White House aides. His first economic adviser,
\href{http://faculty.chicagogsb.edu/austan.goolsbee/website/}{Austan
Goolsbee}, is a young University of Chicago professor who shares Obama's
market-oriented Democratic views. This summer, Obama added Furman, who
has a more traditional background, having worked for both the Clinton
administration and the Kerry campaign. But he, too, has a Ph.D. in
economics, from Harvard.

As anyone who has spent time with Obama knows, he likes experts, and his
choice of advisers stems in part from his interest in empirical
research. (James Heckman, a Nobel laureate who critiqued the campaign's
education plan at Goolsbee's request, said, ``I've never worked with a
campaign that was more interested in what the research shows.'') By
surrounding himself with economists, however, Obama was also making a
decision with ideological consequences. Far more than many other policy
advisers, economists believe in the power of markets. What tends to
distinguish Democratic economists is that they set out to uncover
imperfections of the market and then come up with incremental,
market-based solutions to these imperfections. This helps explain the
Obama campaign's
\href{http://www.nytimes3xbfgragh.onion/2008/01/02/business/02leonhardt.html}{interest
in} behavioral economics, a relatively new field that has pointed out
many ways in which people make irrational, short-term decisions. To deal
with one example of such myopia, Obama would require companies to
automatically set aside a portion of their workers' salary in a 401(k)
plan. Any worker could override the decision --- and save nothing at all
or save even more --- but the default would be to save.

A more controversial version of Obama's market friendliness came from
his health-care proposal, which, unlike Hillary Clinton's, would not
mandate that people have health insurance. Like other Democrats, he was
pushing for a big government program to deal with what he saw as market
failures in health care and to bring down the price of insurance. Once
the program was in place, though, he trusted a market of individuals to
make its own decisions; once the government had subsidized health
insurance, he thought the vast majority of the uninsured would sign up.

There are similar strains in Obama's proposals on housing and education,
and it's worth remembering that these all came out before he was the
presumptive nominee. The best example of his approach, however, may be
his climate policy. By last year, Democrats in Congress essentially
agreed that to reduce greenhouse-gas emissions, the government should
place a nationwide cap on these emissions and then issue tradable
permits giving companies the right to produce them (thus the term ``cap
and trade''). Most Congressional bills envisioned giving away many of
the permits to power companies. Economists, by and large, considered
this giveaway to be the worst part of the plan. It would require
Congress to decide how many free permits each company should get and
would set off a frenzy of corporate lobbying.

The alternative was to auction off the permits --- to let the market set
their value. ``If you don't auction 100 percent of the permits,''
Goolsbee told me, ``this could be one of the biggest pieces of corporate
welfare ever.'' With Congress making the decisions, the power companies
with the best political connections might get the permits. With a
\href{http://gregmankiw.blogspot.com/2008/05/mccain-vs-obama-carbon-auctions.html}{full
auction}, the permits would end up with companies willing to make the
highest bids. Presumably, these would be the most efficient companies,
the ones able to produce the most energy (and profits) for a given
amount of greenhouse-gas pollution.

The auctions would have another big advantage too. They would raise
billions of dollars for the government, money that could then be
returned to taxpayers to offset the higher energy prices created by the
emissions cap.

Image

Credit...Photo Illustration by Victor Schrager for The New York Times;
Prop Stylist: Megan Caponetto

It seems likely that a President Obama would sign a cap-and-trade bill
even if it did give away some permits. But candidate Obama has at least
moved the debate toward a more pro-market solution.

\textbf{IV. The End of the Age of Reagan?}

``The market is the best mechanism ever invented for efficiently
allocating resources to maximize production,'' Obama told me. ``And I
also think that there is a connection between the freedom of the
marketplace and freedom more generally.'' But, he continued, ``there are
certain things the market doesn't automatically do.'' In other words,
free-market policy isn't likely to dominate his agenda; his project
would be fixing the market.

And it does seem to need fixing. For three decades now, the American
economy has been in what the historian Sean Wilentz calls the
\href{http://www.harpercollins.com/books/9780060744809/The_Age_of_Reagan/index.aspx}{Age
of Reagan}. The government has deregulated industries, opened the
economy more to market forces and, above all, cut income taxes. Much
good has come of this --- the end of 1970s stagflation, infrequent and
relatively mild recessions, faster growth than that of the more
regulated economies of Europe. Yet laissez-faire capitalism hasn't
delivered nearly what its proponents promised. It has created big budget
deficits, the most pronounced income inequality since the 1920s and the
current financial crisis. As Lawrence Summers, the former Treasury
secretary and Rubin ally from the Clinton administration, says: ``We've
probably done a better job of the last 20 years on the problems the
market can solve than the problems the market can't solve. We're doing
pretty well on the size of people's houses and televisions and the like.
We're not looking so good on infrastructure and education.''

The closest thing to an Obama doctrine on market regulation was a
\href{http://www.nytimes3xbfgragh.onion/2008/03/27/us/politics/27text-obama.html?pagewanted=print}{speech}
he gave in March at Cooper Union in New York, called ``Renewing the
American Economy.'' It included his usual praise of market forces, and
his prescriptions for regulating the financial system were mostly
mainstream Democratic fare, like tougher penalties for loan fraud,
tighter rules and closer oversight for Wall Street. These steps might or
might not prevent the next crisis, but they would certainly place a
bigger emphasis on trying to do so. And the speech, if anything,
probably placed Obama on the more aggressively liberal side of the
Democratic platform. Afterward, Robert Kuttner, an unabashedly
left-leaning Democrat,
\href{http://www.prospect.org/cs/articles?article=obama_v_krugman}{praised}
Obama for going ``well beyond the current Democratic Party consensus.''

Shortly before Obama's speech, the Federal Reserve made emergency loans
to investment banks that hadn't officially been under its supervision.
Obama argued that, going forward, the Fed had to be given permanent
oversight of any such institutions, because their executives would
henceforth assume that the government would come to their rescue. If
taxpayers were going to be on the hook for those banks when they failed,
he suggested, the government should have the chance to minimize the risk
of failure. (Since March, Fed officials themselves have inched toward a
similar position.)

There is, plainly, a big potential conflict between the University of
Chicago side of Obama and the regulator side. A regulation that sounds
sensible today can end up having nasty unintended consequences. But in
Obama's view, the risks to market-based capitalism now have more to do
with too little regulation than too much. He can sound almost righteous
on this point. He talked to me about the need for a moral element to
capitalism and said that the crony capitalism of recent years should be
the nightmare of any market-loving economist. At times, this part of his
message can seem to overwhelm his respect for the market. Obama's aides
have justified his proposed windfall-profits tax on oil companies, for
example, by saying that it makes up for the unjustifiable tax breaks the
energy industry has received in the past. But that doesn't change the
fact that it's a tax targeted at a specific industry, which, as some
economists have pointed out, is just the sort of tinkering that the
Chicago School detests.

\textbf{V. Spreading the Wealth}

The most tangible way that today's economy feels unfair is the lack of
real income growth for most families. Earlier this year, when
\href{http://www.nytimes3xbfgragh.onion/2008/02/02/us/politics/02obama.html?_r=1\&oref=slogin}{I
interviewed} Obama during the primaries, he was careful to say that he
didn't think President Bush deserved all that much blame for the
stagnant incomes of the current decade. Income growth for most families
began to slow in the 1970s, and the causes of the great pay slowdown
were complex. Obama didn't name them all, but a decent list would look
something like this: new technologies that have made some blue-collar
work obsolete; a slowing in the nation's educational attainment; the
shriveling of labor unions; the increase in one-parent families, which
are far less economically secure; and the rise of other countries that
have huge low-wage work forces.

What Obama blamed the current administration for, he said, was
aggravating these trends with the tax code. To a large extent, Obama's
own economic agenda revolves around reversing Bush's tax policies and
then going a bit further in the other direction. Here, more than in his
regulatory approach, Obama stands on the left side of the Democratic
Party, but not exactly in the traditional tax-and-spend ways.

It's helpful to start with a little
\href{http://elsa.berkeley.edu/~saez/piketty-saezJEP07taxprog.pdf}{history}.
When Reagan was elected, in 1980, tax rates on top incomes were so high
that even liberal economists now say the economy was suffering. There
simply wasn't enough of an incentive for rich people to start new
companies or expand existing ones, because so much of their profits
would have gone to the federal government. Someone making the equivalent
of \$5 million in 1980 --- in inflation-adjusted terms --- would have
paid a combined federal tax rate of almost 60 percent, according to
research by Emmanuel Saez and Thomas Piketty, two academic economists.
(These calculations cover not only income taxes but also payroll taxes,
capital-gains taxes and others.) Reagan, by the end of his second term,
had cut this rate to about 35 percent. Clinton raised it above 40
percent, but the current President Bush has reduced it to 34 percent. So
over the same period that the rich have been getting much richer before
taxes, their tax rates have also been falling far faster than the rates
of any other income group.

Dating back to Reagan, Republicans have packaged tax cuts on high
earners with more modest middle-class tax cuts and then maneuvered the
Democrats into an unwinnable choice: are you for tax cuts or against
them? Obama, however, argues that this is the moment when the politics
of taxes can be changed.

To do this, he is proposing tax cuts for most families that are
significantly larger than those McCain is offering, along with major tax
increases for families making more than \$250,000 a year. ``That's
essentially a major part of our economic plan,'' Obama said. ``But it's
also a political message.'' Economically, he is trying to use the tax
code to spread the bounty from the market-based American economy to a
far wider group of families. Politically, he is trying to drive a wedge
through the great Reagan tax gambit.

The Tax Policy Center, a research group run by the Brookings Institution
and the Urban Institute, has done the most detailed analysis of the
Obama and McCain tax plans, and it has published a series of
\href{http://www.taxpolicycenter.org/UploadedPDF/411749_update_candidates.pdf}{fascinating
tables}. For the bottom 80 percent of the population --- those
households making \$118,000 or less --- McCain's various tax cuts would
mean a net savings of about \$200 a year on average. Obama's proposals
would bring \$900 a year in savings. So for most people, Obama is the
tax cutter in this campaign.

If there is a theme to the Obama tax philosophy, it's that the tax code
is not quite as progressive as you think it is. Most of the public
discussion about taxes tends to focus on the income tax, which taxes the
affluent at a considerably higher rate than anyone else. But the income
tax doesn't take the biggest bite out of most families' annual tax bill.
The payroll tax does. And even as the federal government has been
reducing income taxes over the last few decades, it has allowed the
payroll tax, which finances Social Security and Medicare, to creep up.
That's a big reason that overall tax rates for the bottom 80 percent of
earners have not fallen as much as rates for the affluent.

Obama's second-most-expensive proposal, after his health-care plan, is
the equivalent of a \$500 cut in the payroll tax for most workers. (It
is actually a credit that is applied toward income taxes based on
payroll taxes paid.) In a speech this month in Florida, he proposed that
the cut take effect immediately, in the form of a rebate, to stimulate
the economy. For most workers, it would be the first significant cut in
the payroll tax in decades, if not ever.

The other way that he would cut taxes involves a series of
technicalities. But since the campaign began, Goolsbee has been arguing
that those technicalities offer one of the best glimpses of how Obama
thinks about the tax code. Right now, several big tax breaks that sound
broad-based --- like those for child care and mortgage interest ---
don't always benefit middle-income and lower-income families. Another
example is the Hope Credit for college tuition, a creation of the
Clinton administration. Obama wants to more than double the credit, to
\$4,000. More to the point, he would make it ``fully refundable.'' As a
result, a family with an income-tax bill of \$3,000 wouldn't merely have
that bill eliminated; it would also receive a \$1,000 check.
Increasingly, the income-tax system becomes a way to transfer money to
poor families.

All told, Obama would not only cut taxes for most people more than
McCain would. He would cut them more than Bill Clinton did and more than
Hillary Clinton proposed doing. These tax cuts are really the essence of
his market-oriented redistributionist philosophy (though he made it
clear that he doesn't like the word ``redistributionist''). They are an
attempt to address the middle-class squeeze by giving people a chunk of
money to spend as they see fit.

Image

Credit...Photo Illustration by Victor Schrager for The New York Times;
Prop Stylist: Megan Caponetto

He would then pay for the cuts, at least in part, by raising taxes on
the affluent to a point where they would eventually be slightly higher
than they were under Clinton. For these upper-income families, the Tax
Policy Center's comparisons with McCain are even starker. McCain, by
continuing the basic thrust of Bush's tax policies and adding a few new
wrinkles, would cut taxes for the top 0.1 percent of earners --- those
making an average of \$9.1 million --- by another \$190,000 a year, on
top of the Bush reductions. Obama would raise taxes on this top 0.1
percent by an average of \$800,000 a year.

It's hard not to look at that figure and be a little stunned. It would
represent a huge tax increase on the wealthy families. But it's also
worth putting the number in some context. The bulk of Obama's tax
increases on the wealthy --- about \$500,000 of that \$800,000 --- would
simply take away Bush's tax cuts. The remaining \$300,000 wouldn't
nearly reverse their pretax income gains in recent years. Since the
mid-1990s, their inflation-adjusted pretax income has
\href{http://elsa.berkeley.edu/~saez/}{roughly doubled}.

To put it another way, the wealthy have done so well over the past few
decades, with their incomes soaring and tax rates plummeting, that
Obama's plan would not come close to erasing their gains. The same would
be true of households making a few hundred thousand dollars a year (who
have gotten smaller raises than the very rich but would also face
smaller tax increases). As ambitious as Obama's proposals might be, they
would still leave the gap between the rich and everyone else far wider
than it was 15 or 30 years ago. It just wouldn't be quite as wide as it
is now.

\textbf{VI. Is He a European-Model Neoliberal?}

Even some Republicans have started to wonder whether the Reagan strategy
on taxes has run its course. Earlier this year, two young conservative
writers, Ross Douthat and Reihan Salam, came out with a book called
``Grand New Party.'' Their basic thesis is that the Republican Party,
for all its successes over the past generation, has failed to cement its
majority because of economics. If the party's agenda continues to
revolve around tax cuts that mostly benefit the well off, the
\href{http://www.nytimes3xbfgragh.onion/2008/06/29/books/review/Ornstein-t.html}{book
argues}, Republicans risk allowing a generation-long Democratic
majority, like the kind that ruled the country from F.D.R. to L.B.J. To
avoid this outcome, the authors offer an agenda of what they call Sam's
Club Republicanism, focused on the working class.

For now, the people running the party, be they in the Bush
administration or the McCain campaign, evidently do not share this
concern. They have responded to Obama's tax proposals with the same kind
of attacks that the party has been using since the 1980s. First, they
have argued that Obama's tax increases would end up hitting every income
group. Strictly speaking, this is true. Obama's increase on the
corporate income tax would ultimately fall on all stockholders, even
poor ones. In practical terms, though, most families own little enough
stock that the other features of the tax plan would matter far, far
more. That's why the Tax Policy Center numbers, which include the
corporate tax increase, come out as they do.

The second criticism is that Obama's tax increases would send an
already-weak economy into a tailspin. The
\href{http://wsj.com/article/SB121728762442091427.html}{problem with
this argument} is that it's been made before, fairly recently, and it
proved to be spectacularly wrong. When Bill Clinton raised taxes on
upper-income families in 1993, his supply-side critics insisted that he
would ruin the economy. As we now know, Clinton presided over the
longest economic expansion on record, the fastest income growth most
workers had experienced in a generation and the disappearance of the
federal-budget deficit. His successor, Bush, then did exactly what the
supply-siders wanted, cutting upper-income tax rates, and the results
were much worse. Economic growth wasn't quite as strong or nearly as
widespread, and the deficit returned. At the very least, Clinton's
increases did no discernible economic damage. Rubin, citing academic
work on tax rates, made the case to me that rates under an Obama
administration would not be nearly high enough to stifle innovation.

There is, however, a more philosophical critique of Obama's tax
policies. It's one that Douthat and Salam make in ``Grand New Party.''
The book doesn't mention Obama by name, but it contains one of the best
summaries of his economic policy that I have read. The authors describe
a new-model liberal consensus that weds ``the free-market centrism of
the Clinton years to a revived push for European-style social
democracy.'' This neoliberalism, as they call it, wouldn't involve the
big-government programs of the postwar years, but the government would
come to play a larger role in the economy and would redistribute much
more income from the rich to everyone else. ``This is, in many respects,
a deeply un-American solution to the problems facing our country,'' the
authors write, ``one that would emphasize dependence over
self-sufficiency and bureaucratic condescension over self-help.''

Douglas Holtz-Eakin, a former head of the Congressional Budget Office
who has been advising McCain since the primaries, made a more specific
version of this same point to me. Since Social Security was founded, its
benefits have been based on the amount of payroll taxes that an
individual worker paid over his or her lifetime. The system is
progressive, in that the rich contribute more than the poor and do not
get out everything they put in. But Obama would make it vastly more
progressive. Currently, only income up to \$102,000 is subject to the
tax. After a decade, he would leave income between \$102,000 and
\$250,000 untaxed, but would begin taxing income above that. The people
paying this new tax probably would not get any additional retirement
benefits in return. ``As a political matter,''
\href{http://www.leighbureau.com/speaker.asp?id=344}{Holtz-Eakin
argued}, ``it reveals a lack of judgment.'' A program with almost
unrivaled political support, he added, could turn into yet another
government transfer program.

During my recent conversation with Obama, he mentioned Sam's Club
Republicanism in a different context, and I asked him if he had read
``Grand New Party.'' He hadn't, he said, so I read him the line about
dependence and condescension and asked for his reaction.

He said it made him think of Warren Buffett, an Obama supporter, who, if
anything, might argue that he wasn't going far enough to change the tax
code. ``If you talk to Warren, he'll tell you his preference is not to
meddle in the economy at all --- let the market work, however way it's
going to work, and then just tax the heck out of people at the end and
just redistribute it,'' Obama said. ``That way you're not impeding
efficiency, and you're achieving equity on the back end.'' He continued
by saying that he thought there was some merit in Buffett's argument.
But, he said: ``I do think that what the argument may miss is the sense
of control that we want individuals to have in determining their own
career paths, making their own life choices and so forth. And I also
think you want to instill that sense of self-reliance and that what you
do will help determine outcomes.''

\textbf{VII. The New New Deal}

Last summer, just before a highway bridge in Minneapolis collapsed,
Obama was meeting with a small group of economists. At one point,
according to several people who were at the meeting, Obama said he
agreed that blue-collar workers were struggling primarily because their
skills weren't as much in demand as they used to be. Technology has
remade the economy, and education and retraining were the best ways for
workers to keep up. But any public-policy response couldn't be about
just education; it also had to take account of the psychology of the
workplace, Obama continued. Some laid-off steelworkers might indeed be
able to go back to school to become health-care workers. But many of
them don't want to work in health care or any service job. Factory
workers, he said, want to make something. It's part of their identity.

From there, Obama moved the conversation toward a discussion of how the
government could improve the nation's infrastructure --- its backbone of
bridges, roads, tunnels, airports and the like, much of which has seen
better days. Since the dawn of the Age of Reagan, the idea that
government spending can be a good thing for the economy has been out of
favor, even among Democrats. But it's now making something of a
comeback, particularly within Obama's camp. His agenda calls for about
\$50 billion in new annual spending on various investments, including
infrastructure, alternative energy and scientific research. (To put that
in perspective, the cut in the payroll tax would cost about \$70 billion
a year.)

These investments might pay off in all sorts of ways. They are a classic
form of stimulus that could help the economy emerge from the housing
hangover. They would provide jobs for former factory workers and others
without college degrees, many of whom have struggled over the past
generation, and for whom the current home-building slump has been yet
another blow. Above all, the investments would have the potential to pay
big long-term dividends, in the form of a national economy that operated
more smoothly.

I came to think of this part of Obama's agenda as the Virginia model,
thanks to Tim Kaine, Virginia's governor, who was one of the first
Democrats to endorse Obama. Last year, Kaine began making the case to
Goolsbee that the campaign should view Virginia as a model for the rest
of the country. In just a few decades, the state has managed to
transform itself in precisely the way that economists think the United
States now must --- to a higher-wage economy with a more-educated
population, a place that has prospered even while losing many of its
old-line manufacturing jobs. And it did so with a crucial shove from the
government.

For much of the 20th century, Virginia was a poor state, but after World
War II, with the cold war under way and the military growing,
well-paying defense contractors began to sprout up around the Pentagon,
in northern Virginia. By the 1970s,
\href{http://www.darpa.mil/body/overtheyears.html}{Darpa}, the
Pentagon's research arm, began working on a computer network, which soon
spawned a new form of communication: electronic mail. That computer
system eventually became the Internet, and Northern Virginia suddenly
had the beginnings of a brand-new industry. In recent decades, Virginia
has also invested money in the port near Norfolk and has vastly expanded
its colleges and universities. Today the state's per-capita income is 7
percent higher than the national average.

The trick for someone trying to replicate Virginia's success is figuring
out which investments to make. As any Chicago School economist would
remind you, the federal government has made its share of mistakes in
this area, a recent example being subsidies for ethanol, which Obama, a
farm-state senator, has championed and McCain has opposed. But Obama at
least seems to have learned one lesson from the experience: His proposed
new infrastructure spending would be overseen by a bipartisan board of
unelected officials, rather than members of Congress.

Image

Credit...Photo Illustration by Victor Schrager for The New York Times;
Prop Stylist: Megan Caponetto

More important, perhaps, is the fact that a single success, like the
Internet or the Interstate highway system, can make up for a lot of
failures. Jason Grumet, a Washington lawyer who is the Obama campaign's
lead environmental adviser, made this point to me after I asked him why
anyone should have confidence in the government's ability to pick
winners. ``We all talk about
\href{http://www.nasm.si.edu/collections/imagery/Apollo/AS11/a11.htm}{Apollo
11}, but there were some pretty public, pretty awful failures along the
way,'' Grumet said. ``The United States didn't say: `Well, we had some
failures. We're going to give up getting to the moon.' ''

\textbf{VIII. Lots of Beef, Shortage of Message}

When Obama gives a speech about his economic plan, there is often a
moment when you can sense him shift from poetry to prose. He can be
inspiring when talking about how the country ended up being the envy of
the world. But when he comes to the part about what he wants to do next,
how he wants to keep America the envy of the world, it can sound a
little like a State of the Union laundry list.

His advisers are divided about how much of a problem this is. Some of
them told me that he did have a unifying theme --- the middle-class
squeeze --- and that it would become clearer to voters as they began
paying closer attention to the race. Others said they didn't think Obama
had yet come up with a simple way to explain how he would alleviate that
squeeze. Obama himself seems well aware of the stakes. In 2005, on a
call-in public-radio show, he told a listener that Democrats hadn't been
as effective in telling a story about the country as Republicans. In the
end, he said, people voted not for a hodgepodge of position papers but
for someone who could explain to them where the country should be going.

So I asked Obama whether he thought he had been able to tell an
effective story about the economy during this campaign. Specifically, I
wondered, did he think he had a message that compared with Reagan's
simple call for less government and lower taxes.

He paused for a few seconds and then said this:

``I think I can tell a pretty simple story. Ronald Reagan ushered in an
era that reasserted the marketplace and freedom. He made people aware of
the cost involved of government regulation or at least a
command-and-control-style regulation regime. Bill Clinton to some extent
continued that pattern, although he may have smoothed out the edges of
it. And George Bush took Ronald Reagan's insight and ran it over a
cliff. And so I think the simple way of telling the story is that when
Bill Clinton said the era of big government is over, he wasn't arguing
for an era of no government. So what we need to bring about is the end
of the era of unresponsive and inefficient government and short-term
thinking in government, so that the government is laying the groundwork,
the framework, the foundation for the market to operate effectively and
for every single individual to be able to be connected with that market
and to succeed in that market. And it's now a global marketplace.

``Now, that's the story. Now, telling it elegantly --- `low taxes,
smaller government' --- the way the Republicans have, I think is more of
a challenge.''

Even if Obama does figure out how to meet the challenge well enough to
get elected, there are any number of ways in which his plans could fail.
He has never run any government entity --- no state, no city, not even a
municipal agency --- and he may not prove to be good at doing so. The
economy could deteriorate further, leaving him with a Clinton-like
choice between manageable deficits and direct help for the middle class.
Or maybe the many economists who like his agenda are simply wrong. Maybe
his health-care program won't bring down costs. Maybe the Virginia model
won't work for the rest of the country.

But it's not entirely clear what the alternative is, at least in the
broad sense and at least for the time being. A much more left-wing
agenda than Obama's would consist of erecting new trade barriers,
reregulating various industries and otherwise getting the government
even more involved in the economy than Obama would. This program has the
dubious distinction of being disliked by both voters and experts alike.
Populism hasn't won a national election, or even the Democratic
nomination, in decades, and economists can point to any number of ways
why it wouldn't work anyway.

Republicans, on the other hand, have an economic strategy that may still
sell politically. But is there much reason to think that it would lead
to a very different result from Bush's? There have now been two
presidents in the last 30 years --- Bush and Reagan --- who cut taxes
and promised that deficits would not follow. But the deficits did come,
and they went away only after two other presidents --- George H. W. Bush
and Bill Clinton --- raised taxes. It also seems fairly clear by now
that tax cuts for the affluent do not necessarily trickle down to
everyone else.

For Democrats who want to think the worst about their opponents,
McCain's reliance on these ideas may be affirming. But it's really a
shame. For the time being, only one party is applying the lessons of
history to the country's biggest economic problems. There is no great
battle of new ideas, and that can't make it more likely that those
problems will be solved.

\textbf{Shortly after I boarded} Obama's campaign plane this month, one
of his press aides warned me that the conversation might not last long.
She explained that he was exhausted from two days of campaigning in
Florida and might decide to nap as soon as he got on the plane. But a
few minutes later he summoned me to the plane's first-class section,
evidently choosing an economics discussion over a DVD of ``Mad Men,''
which was sitting on his side table. His eyes were tired, and he looked
a good deal older than he had only four years ago, on the night that he
became famous at the 2004 Democratic convention. But we ended up talking
for an hour. After I returned to my seat, the press aide walked back to
tell me that Obama had more to say.

``Two things,'' he said, as we were standing outside the first-class
bathroom. ``One, just because I think it really captures where I was
going with the whole issue of balancing market sensibilities with moral
sentiment. One of my favorite quotes is --- you know that famous Robert
F. Kennedy quote about the measure of our G.D.P.?''

I didn't, I said.

``Well, I'll send it to you, because it's one of the most beautiful of
his speeches,'' Obama said.

In it, Kennedy argues that a country's health can't be measured simply
by its economic output. That output, he said, ``counts special locks for
our doors and the jails for those who break them'' but not ``the health
of our children, the quality of their education or the joy of their
play.''

The second point Obama wanted to make was about sustainability. The
current concerns about the state of the planet, he said, required
something of a paradigm shift for economics. If we don't make serious
changes soon, probably in the next 10 or 15 years, we may find that it's
too late.

Both of these points, I realized later, were close cousins of two of the
weaker arguments that liberals have made in recent decades. Liberals
have at times dismissed the
\href{http://www.nytimes3xbfgragh.onion/2008/04/16/business/16leonhardt.html}{enormous
benefits} that come with prosperity. And for decades some liberals have
been
\href{http://www.nytimes3xbfgragh.onion/1990/12/02/magazine/120290-tierney-magazine.html}{wrongly}
predicting that economic growth was sure to leave the world without
enough food or enough oil or enough something. Obama acknowledged as
much, saying that technology had thus far always overcome any concerns
about sustainability and that Kennedy's notion had to be tempered with
an appreciation of prosperity.

What's new about the current moment, however, is that both of these
arguments are actually starting to look relevant. Based on the
collective wisdom of scientists, global warming really does seem to be
different from any previous environmental crisis. For the first time on
record, meanwhile, economic growth has not translated into better living
standards for most Americans. These are two enormous challenges that are
part of the legacy of the Reagan Age. They will be waiting for the next
president, whether he is Obama or McCain, and they'll probably be around
for another couple of presidents too.

Advertisement

\protect\hyperlink{after-bottom}{Continue reading the main story}

\hypertarget{site-index}{%
\subsection{Site Index}\label{site-index}}

\hypertarget{site-information-navigation}{%
\subsection{Site Information
Navigation}\label{site-information-navigation}}

\begin{itemize}
\tightlist
\item
  \href{https://help.nytimes3xbfgragh.onion/hc/en-us/articles/115014792127-Copyright-notice}{©~2020~The
  New York Times Company}
\end{itemize}

\begin{itemize}
\tightlist
\item
  \href{https://www.nytco.com/}{NYTCo}
\item
  \href{https://help.nytimes3xbfgragh.onion/hc/en-us/articles/115015385887-Contact-Us}{Contact
  Us}
\item
  \href{https://www.nytco.com/careers/}{Work with us}
\item
  \href{https://nytmediakit.com/}{Advertise}
\item
  \href{http://www.tbrandstudio.com/}{T Brand Studio}
\item
  \href{https://www.nytimes3xbfgragh.onion/privacy/cookie-policy\#how-do-i-manage-trackers}{Your
  Ad Choices}
\item
  \href{https://www.nytimes3xbfgragh.onion/privacy}{Privacy}
\item
  \href{https://help.nytimes3xbfgragh.onion/hc/en-us/articles/115014893428-Terms-of-service}{Terms
  of Service}
\item
  \href{https://help.nytimes3xbfgragh.onion/hc/en-us/articles/115014893968-Terms-of-sale}{Terms
  of Sale}
\item
  \href{https://spiderbites.nytimes3xbfgragh.onion}{Site Map}
\item
  \href{https://help.nytimes3xbfgragh.onion/hc/en-us}{Help}
\item
  \href{https://www.nytimes3xbfgragh.onion/subscription?campaignId=37WXW}{Subscriptions}
\end{itemize}
