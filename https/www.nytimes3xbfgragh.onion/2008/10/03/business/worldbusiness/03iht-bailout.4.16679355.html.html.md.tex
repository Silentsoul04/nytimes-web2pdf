Sections

SEARCH

\protect\hyperlink{site-content}{Skip to
content}\protect\hyperlink{site-index}{Skip to site index}

\href{https://www.nytimes3xbfgragh.onion/section/business}{International
Business}

\href{https://myaccount.nytimes3xbfgragh.onion/auth/login?response_type=cookie\&client_id=vi}{}

\href{https://www.nytimes3xbfgragh.onion/section/todayspaper}{Today's
Paper}

\href{/section/business}{International Business}\textbar{}Congress
approves \$700 billion Wall Street bailout

\begin{itemize}
\item
\item
\item
\item
\item
\end{itemize}

Advertisement

\protect\hyperlink{after-top}{Continue reading the main story}

Supported by

\protect\hyperlink{after-sponsor}{Continue reading the main story}

\hypertarget{congress-approves-700-billion-wall-street-bailout}{%
\section{Congress approves \$700 billion Wall Street
bailout}\label{congress-approves-700-billion-wall-street-bailout}}

By
\href{https://www.nytimes3xbfgragh.onion/by/david-m-herszenhorn}{David
M. Herszenhorn}

\begin{itemize}
\item
  Oct. 3, 2008
\item
  \begin{itemize}
  \item
  \item
  \item
  \item
  \item
  \end{itemize}
\end{itemize}

\textbf{WASHINGTON ---} The U.S. House of Representatives gave final
approval Friday to the \$700 billion bailout for the financial system,
reversing course to authorize what may be the most expensive U.S.
government intervention in history.

The vote was 263 to 171, with a number of Democrats and Republicans
switching sides to give the rescue package a majority. More Republicans
continued to oppose the measure than support it, however.

The Senate approved the plan Wednesday night by 74 to 25, after adding a
portfolio of popular tax provisions. The bill now heads to President
George W. Bush, who has promised to sign it.

Financial markets had a positive but hardly exuberant response to the
House action and the prospect that the Federal Reserve would also move
to cut interest rates to help the ailing economy. Just after the bill
was approved, the Dow Jones industrial average was up about 115 points.

The move to rescue the financial industry came as the United States was
weighed down by another round of bleak economic data, including a report
showing 159,000 jobs were lost in September. It is the last jobs report
to be issued before the Nov. 4 election. (Page 18)

After the surprise defeat of the bailout package Monday, congressional
leaders took no chances Friday. Democrats had said they would not bring
the bill back to the floor unless they were certain of victory.

Even before the vote, however, there was evidence that the prospect of
the rescue package was encouraging further consolidation in the banking
industry. On Friday, Wells Fargo, a big bank concentrated in the western
United States, announced that it would buy another bank, Wachovia,
without a government guarantee, grabbing it away from Citigroup, which
said it might try to block the deal. (Page 15)

Policy makers and investors worldwide were closely watching the outcome
of the vote because they believed the measure was critical to prevent a
further weakening of the U.S. and global economy. Leaders of some of the
largest countries in Europe planned to meet Saturday to discuss dealing
with the credit crisis there.

The revolt this week by the House rank and file was quelled both by
fears of a global economic meltdown and by old-fashioned political
inducements added by the Senate to sweeten the deal.

Many lawmakers who changed sides said they had agonized over the
decision amid a torrent of telephone calls and e-mail messages from
voters, and several referred to a provision added by the Senate
increasing the amount of bank savings insured by the federal government
to \$250,000 per account from \$100,000.

Several Democrats in the Congressional Black Caucus said they were
persuaded to support the bill by Senator Barack Obama, the party's
presidential nominee. But many lawmakers said they were motivated most
by fears of economic calamity.

"Nobody in East Tennessee hates the fact more than me that I am going to
vote 'yes' today after voting 'no' on Monday," Representative Zach Wamp,
a Republican, said in a speech on the House floor. "Monday I cast a blue
collar vote for the American people. Today I am going to cast a
red-white-and-blue collar vote with my hand over my heart for this
country, because things are really bad and we don't have any choice.
We're out of choices and our backs are up against the wall."

The action Friday capped an extraordinary two-week dénouement to the
110th Congress. Lawmakers, eager to get home for the autumn campaign
season, had intended to wrap up by adopting a budget bill to finance
government operations through early March.

Instead, they found themselves still in Washington just five weeks
before Election Day, facing the most important vote of the year - the
most important vote of their lives, many lawmakers said. They were under
extreme pressure from the White House, the presidential nominees and
congressional leaders of both parties to make a quick decision.

Last week they balked, defeating the bailout package by 228 to 205 and
sending the Dow down 777 points.

Supporters said the bailout was needed to prevent economic collapse;
opponents said it was hasty, ill-conceived and risked too much taxpayer
money to help Wall Street executives, while providing no guarantees of
success.

In the Senate, lawmakers who opposed the plan Wednesday warned that it
still did not address the root problems in the U.S. financial system,
including lax regulation.

The rescue plan allows the Treasury to buy troubled debt from financial
institutions in an effort to ease a deepening credit crisis that is
choking off business and consumer loans, the lifeblood of the global
economy, and contributing to a string of bank failures in the United
States and Europe. The hope is that clearing the balance sheets of bad
debt will keep credit flowing and prevent normal economic activity from
stalling.

Whether the plan succeeds or fails, elected officials and business
leaders alike said it stands to fundamentally alter the relationship
between government and the private markets - perhaps in ways that are
not immediately clear.

At the White House, Bush hailed the vote. But it was a hollow victory
for the administration. After long favoring a hands-off approach and
relentlessly pursuing deregulation of the financial industry, the Bush
administration, along with the Federal Reserve, found itself interceding
repeatedly in the private market this year to avert one calamity after
another.

And after proposing perhaps the biggest government intervention in
decades, Bush found himself abandoned by fellow Republicans in the
House.

When the House rejected the plan Monday, the Senate stepped in and
attached a \$150.5 billion package of popular provisions, including tax
breaks for the production and use of renewable energy, and protection
for millions of U.S. families from paying the alternative minimum tax,
which was initially aimed at the wealthy but now effects growing numbers
of upper-middle-income taxpayers.

The approval of the bailout plan came just 13 days after the
administration put forward a three-page proposal that would have given
the Treasury secretary unfettered authority to run the \$700 billion
effort, in what the House speaker, Nancy Pelosi, called "czar-like
powers."

Tense negotiations over eight days, including an extraordinary and
contentious meeting at the White House between Bush, top lawmakers and
the two presidential candidates, Senator John McCain and Obama, produced
a compromise measure that all sides said they could support, albeit
unenthusiastically.

The final agreement called for the \$700 billion to be disbursed in
parts: \$250 billion at first, to get the program started, followed by
\$100 billion at the discretion of Bush and the remaining \$350 billion
upon the request of the Treasury, with Congress empowered to block the
last installment by acting within 15 days.

It is impossible to predict the final cost of the bailout to taxpayers,
but officials insist it will be far less than \$700 billion. The
Treasury will purchase and then resell assets, potentially at a higher
price than it paid, making it likely the program will recover much of
the initial outlay.

The deal provides for tight oversight of the rescue program by two
boards, including an independent congressional panel. And it requires
the government to use its new status as an large-scale owner of
distressed, mortgage-backed securities to take more aggressive steps to
prevent home foreclosures.

The bill also seeks to limit the pay of executives of some companies
that sell bad debt to the government, including restrictions on "golden
parachute" retirement plans.

It also provides several taxpayer protections, including a mechanism for
the government to take an equity stake, in the form of stock warrants,
in some of the companies that seek government help, which will give
taxpayers a chance to make money should the companies profit in the
months and years ahead.

And, if the rescue plan has lost money after five years, the bill
requires the president to submit a plan to Congress for recouping those
losses from the financial industry, perhaps through new fees or a tax on
securities transactions.

Carl Hulse and Robert Pear contributed reporting.

Advertisement

\protect\hyperlink{after-bottom}{Continue reading the main story}

\hypertarget{site-index}{%
\subsection{Site Index}\label{site-index}}

\hypertarget{site-information-navigation}{%
\subsection{Site Information
Navigation}\label{site-information-navigation}}

\begin{itemize}
\tightlist
\item
  \href{https://help.nytimes3xbfgragh.onion/hc/en-us/articles/115014792127-Copyright-notice}{©~2020~The
  New York Times Company}
\end{itemize}

\begin{itemize}
\tightlist
\item
  \href{https://www.nytco.com/}{NYTCo}
\item
  \href{https://help.nytimes3xbfgragh.onion/hc/en-us/articles/115015385887-Contact-Us}{Contact
  Us}
\item
  \href{https://www.nytco.com/careers/}{Work with us}
\item
  \href{https://nytmediakit.com/}{Advertise}
\item
  \href{http://www.tbrandstudio.com/}{T Brand Studio}
\item
  \href{https://www.nytimes3xbfgragh.onion/privacy/cookie-policy\#how-do-i-manage-trackers}{Your
  Ad Choices}
\item
  \href{https://www.nytimes3xbfgragh.onion/privacy}{Privacy}
\item
  \href{https://help.nytimes3xbfgragh.onion/hc/en-us/articles/115014893428-Terms-of-service}{Terms
  of Service}
\item
  \href{https://help.nytimes3xbfgragh.onion/hc/en-us/articles/115014893968-Terms-of-sale}{Terms
  of Sale}
\item
  \href{https://spiderbites.nytimes3xbfgragh.onion}{Site Map}
\item
  \href{https://help.nytimes3xbfgragh.onion/hc/en-us}{Help}
\item
  \href{https://www.nytimes3xbfgragh.onion/subscription?campaignId=37WXW}{Subscriptions}
\end{itemize}
