Sections

SEARCH

\protect\hyperlink{site-content}{Skip to
content}\protect\hyperlink{site-index}{Skip to site index}

\href{https://www.nytimes3xbfgragh.onion/pages/dining/index.html}{Dining
\& Wine}

\href{https://myaccount.nytimes3xbfgragh.onion/auth/login?response_type=cookie\&client_id=vi}{}

\href{https://www.nytimes3xbfgragh.onion/section/todayspaper}{Today's
Paper}

\href{/pages/dining/index.html}{Dining \& Wine}\textbar{}Such Adventures
Often Start With Tapas

\begin{itemize}
\item
\item
\item
\item
\item
\end{itemize}

Advertisement

\protect\hyperlink{after-top}{Continue reading the main story}

Supported by

\protect\hyperlink{after-sponsor}{Continue reading the main story}

Restaurants

\hypertarget{such-adventures-often-start-with-tapas}{%
\section{Such Adventures Often Start With
Tapas}\label{such-adventures-often-start-with-tapas}}

\includegraphics{https://static01.graylady3jvrrxbe.onion/images/2009/04/22/dining/22rest-600.jpg?quality=75\&auto=webp\&disable=upscale}

By \href{https://www.nytimes3xbfgragh.onion/by/frank-bruni}{Frank Bruni}

\begin{itemize}
\item
  April 21, 2009
\item
  \begin{itemize}
  \item
  \item
  \item
  \item
  \item
  \end{itemize}
\end{itemize}

JOSH DeCHELLIS is the opposite of a man without a country. He's a man
with possibly too many countries --- on his résumé, at least.

There's France, where he apprenticed toward the beginning of his career
as a chef, and Japan, a reference point for both Sumile, the New York
restaurant that established him, and BarFry, a later failed bet on
tempura.

Then there's Italy, the inspiration for Jovia, which came just before
BarFry and also didn't hang in there for long, partly because he never
mastered any real fluency in Italian.

Is Spanish the language that Mr. DeChellis --- born, as it happens, in
Colombia --- was meant to speak? Is Spain his culinary home?

During the peak moments of my best meals at La Fonda del Sol, which
opened in January with Mr. DeChellis manning the stoves, I was inclined
to say yes.

Although the menu has weak spots, with a few too many dishes not from
the heart but from a marketing plan, his cooking here feels less forced
and more exuberant than it did at any of the other restaurants where I
tried it.

More important, it reflects a steady, precise hand. A tried-and-true
combination of octopus with potato seemed fresh again, because the
kitchen got precisely the tenderness it wanted from the octopus and the
firmness it sought in the potato, so that each was a textural mirror and
mimic of the other.

At lunchtime, when so many restaurants put on their B if not C games, La
Fonda served me a fillet of wild striped bass so vividly white in color
and melting in consistency it could have been a snowdrift. The fish got
a thrillingly salty, nutty charge --- and some nice crunch --- from the
pumpkinseeds scattered over it.

Even better was a lunchtime braised chicken with saffron-seasoned rice,
which sopped up juices and braising liquid from the bird. Packed into a
deep bowl instead of arrayed on a plate, the dish negotiated some
midpoint between coq au vin and paella, and it had some of paella's
gently smoky, porky accents.

La Fonda del Sol takes its name and general bearings from a long-gone
restaurant that opened in the Time \& Life Building in 1960, well before
Spanish food had as much cachet with American diners as it has now.

That original La Fonda was the creation of Joe Baum, he of the Four
Seasons and other flashy enterprises that heralded an era of dining as
conceptual extravaganza, during which the motifs would matter as much as
the moulard.

This restaurant is billed as a revival, along the lines of ``Gypsy'' or
yet another Fleetwood Mac tour. But that's more marketing gimmick than
reality, even though one of the cooks from the first Fonda, Victor
Broceaux, has been on hand to consult with Mr. DeChellis.

La Fonda 2009 is in a different location, on Vanderbilt Avenue, smack
against Grand Central Terminal. Its menu is more expressly Spanish and
less pan-Latin. Its look isn't remotely the same.

That the new owners, the Patina Restaurant Group, mean business is
reflected in part by their hiring of Adam Tihany's firm to design the
space, split into two distinct environments.

Up front, looking onto the street, is a long room with a long bar and a
festive décor that makes you want to drink sangria. This area doesn't
accept reservations and serves a menu entirely of tapas, except at
lunchtime, when it also offers a few main courses, including those bass
and chicken dishes and equally fine scallops with salsa verde.

Up a half flight of stairs is a carpeted, plush, windowless dining room
that makes you want to drink martinis. It serves a menu mostly of
conventionally portioned appetizers and entrees, with some tapas thrown
in.

It's up here in this somewhat glum hideaway that you can eat the superb
scallops tiradito, a sort of carpaccio-cum-ceviche with flares of
cilantro and chili, or a gorgeously brooding, thick soup of oxtail,
white beans, bread and cheese.

I preferred the casual front, which has the right feel, food and natural
light (until sunset) for an easygoing lunch or a dinner that proceeds in
improvisatory chapters of tapas, the final tally of food determined in
part by the final tally of drink.

With another glass of a refreshing blend of garnacha blanca and macabeo,
I would ask for yet another order of the classic pa amb tomàquet,
toasted bread with a jammy spread of tomatoes, olive oil and salt. Each
time I had it, it was just about perfect.

For less comfort but more adventure you can check out another Spanish
newcomer, and the adventure unfolds on several fronts.

Image

Txikito takes Basque fare to Chelsea.Credit...Evan Sung for The New York
Times

There's the location, a once-groggy block of Ninth Avenue suddenly
swarmed by the food-obsessed, thanks largely to the new-wave pizzeria
Co. There's the food, which has included, on different nights, beef
tongue and hake jowls and even a mix of beef tendon, tripe and pig
trotters.

And there's the phonetics, a daunting barrage of t's, x's, k's and p's
twisted into such pronunciation-defying dishes as txangurro (crab meat
gratin), txiki txanpi (grilled cheese with mushroom and shrimp),
txipiron encebollado (squid ribbons a la plancha) and txarripatak (the
trotter-tripe-tendon jubilee).

In honor of the Spanish region that guides the kitchen, the menu is
written in the Basque language of Euskara, which also yields the
restaurant's name, Txikito (cheek-ee-toe), meaning ``little.'' And
little this restaurant is --- a boxy room whose dark blue paint works
with reclaimed barn wood and grayish tiles on other walls to produce a
drearier effect than the owners, Alexandra Raij and her husband and
co-chef, Eder Montero, who is from the Basque area, could have intended.

Ms. Raij used to be in charge of the food at the Chelsea favorites Tia
Pol and El Quinto Pino, and justly amassed a loyal following through her
wickedly enjoyable work there. El Quinto Pino's uni panino will go down
as an era-defining treat, pressing an exotic delicacy into the service
of a paper-sleeved finger food to be eaten on the run.

Her menu at Txikito maintains her focus on tapas-style dishes, including
pintxoak (Basque canapés), but is by far the longest and most
complicated she's done --- maybe too long and complicated, given the
constraints of a visibly tiny kitchen.

Across many meals here I had wonderfully memorable food (suckling pig as
fine as any in New York beyond Eleven Madison Park's); ridiculous food
(a rib-eye so excessively fatty and undercooked it was almost inedible);
food that fell somewhere in between (the crosscut spareribs, with too
much bone and too little pork); and food that never tasted the same
twice. The meatballs in a shellfish broth could be hard and dull or
tender and nuanced. It depended on the night.

Although the prices on individual items are low, the bill can climb
surprisingly high, especially considering the plainness and tightness of
the quarters.

But the staff is cheerful and welcoming. The short wine list has many
Spanish pleasures. And a terrific meal is definitely a possibility,
speaking to Ms. Raij's indisputable talents. Get two orders of her
``crispy creamy croquettes'' of béchamel and bacalao, which live up to
both adjectives and have a way of disappearing fast.

\textbf{La Fonda del Sol}

\textbf{**}

200 Park Avenue (entrance on Vanderbilt near 44th Street); (212)
867-6767. patinagroup.com/east/lafondadelsol

\textbf{ATMOSPHERE} A festively decorated lounge area facing the street;
a half staircase up to a more private, formal, carpeted dining room.

\textbf{SOUND LEVEL} Moderately loud up front; quiet in back.

\textbf{RECOMMENDED DISHES} Tuna tacos; empanadas; meatball skewers; pa
amb tomàquet; octopus with potatoes; braised chicken with saffron rice;
striped bass with pumpkinseeds; scallops tiradito; oxtail soup; cod with
clams; lamb with pumpkinseed crust; Mexican chocolate cake; cinnamon
fritters.

\textbf{WINE LIST} International and varied, with a slight focus on
Spain.

\textbf{PRICE RANGE} Dinner tapas, \$3 to \$12; appetizers, \$12 to
\$18; entrees, \$24 to \$39; desserts, \$9.

\textbf{HOURS} For tapas lounge, 11:30 a.m. to 11 p.m. Monday to
Saturday. For dining room, lunch 11:30 a.m. to 3 p.m. Monday to Friday,
and dinner 5 to 10:30 p.m. Monday to Saturday.

\textbf{RESERVATIONS} Not accepted for lounge; for dining room, call at
least three days ahead.

\textbf{CREDIT CARDS} All major cards.

\textbf{WHEELCHAIR ACCESS} Entrance and lounge at street level; elevator
to dining room; accessible restrooms.

\textbf{Txikito}

\textbf{*}

240 Ninth Avenue (25th Street); (212) 242-4730. txikitonyc.com

\textbf{ATMOSPHERE} A boxy, relatively no-frills room with a bar up
front and barn wood on the walls.

\textbf{SOUND LEVEL} Tolerably loud when crowded.

\textbf{RECOMMENDED DISHES} Sofrito, chorizo and quail egg; Basque tuna
sandwich; chorizo hash sandwich; croquettes; crab meat gratin; squid in
its ink; lamb chops; suckling pig; copa de chocolate.

\textbf{WINE LIST} Spanish, interesting and accessibly priced.

\textbf{PRICE RANGE} Dinner dishes (other than bar snacks and specials),
\$5 to \$20. Desserts, \$6 to \$8.

\textbf{HOURS} Lunch from noon to 3 p.m. Tuesday to Friday. Dinner from
5 to 11 p.m. Tuesday to Thursday and Sunday, and to midnight Friday and
Saturday.

\textbf{RESERVATIONS} Accepted only for groups of at least six who order
their meal in advance.

\textbf{CREDIT CARDS} All major cards.

\textbf{WHEELCHAIR ACCESS} Entire restaurant and accessible restroom on
street level.

\textbf{WHAT THE STARS MEAN} Ratings range from zero to four stars and
reflect the reviewer's reaction to food, ambience and service, with
price taken into consideration. Menu listings and prices are subject to
change.

Advertisement

\protect\hyperlink{after-bottom}{Continue reading the main story}

\hypertarget{site-index}{%
\subsection{Site Index}\label{site-index}}

\hypertarget{site-information-navigation}{%
\subsection{Site Information
Navigation}\label{site-information-navigation}}

\begin{itemize}
\tightlist
\item
  \href{https://help.nytimes3xbfgragh.onion/hc/en-us/articles/115014792127-Copyright-notice}{©~2020~The
  New York Times Company}
\end{itemize}

\begin{itemize}
\tightlist
\item
  \href{https://www.nytco.com/}{NYTCo}
\item
  \href{https://help.nytimes3xbfgragh.onion/hc/en-us/articles/115015385887-Contact-Us}{Contact
  Us}
\item
  \href{https://www.nytco.com/careers/}{Work with us}
\item
  \href{https://nytmediakit.com/}{Advertise}
\item
  \href{http://www.tbrandstudio.com/}{T Brand Studio}
\item
  \href{https://www.nytimes3xbfgragh.onion/privacy/cookie-policy\#how-do-i-manage-trackers}{Your
  Ad Choices}
\item
  \href{https://www.nytimes3xbfgragh.onion/privacy}{Privacy}
\item
  \href{https://help.nytimes3xbfgragh.onion/hc/en-us/articles/115014893428-Terms-of-service}{Terms
  of Service}
\item
  \href{https://help.nytimes3xbfgragh.onion/hc/en-us/articles/115014893968-Terms-of-sale}{Terms
  of Sale}
\item
  \href{https://spiderbites.nytimes3xbfgragh.onion}{Site Map}
\item
  \href{https://help.nytimes3xbfgragh.onion/hc/en-us}{Help}
\item
  \href{https://www.nytimes3xbfgragh.onion/subscription?campaignId=37WXW}{Subscriptions}
\end{itemize}
