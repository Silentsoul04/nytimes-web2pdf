Sections

SEARCH

\protect\hyperlink{site-content}{Skip to
content}\protect\hyperlink{site-index}{Skip to site index}

\href{https://www.nytimes3xbfgragh.onion/pages/dining/index.html}{Dining
\& Wine}

\href{https://myaccount.nytimes3xbfgragh.onion/auth/login?response_type=cookie\&client_id=vi}{}

\href{https://www.nytimes3xbfgragh.onion/section/todayspaper}{Today's
Paper}

\href{/pages/dining/index.html}{Dining \& Wine}\textbar{}Beef and Décor,
Aged to Perfection

\begin{itemize}
\item
\item
\item
\item
\item
\end{itemize}

Advertisement

\protect\hyperlink{after-top}{Continue reading the main story}

Supported by

\protect\hyperlink{after-sponsor}{Continue reading the main story}

Restaurants

\hypertarget{beef-and-duxe9cor-aged-to-perfection}{%
\section{Beef and Décor, Aged to
Perfection}\label{beef-and-duxe9cor-aged-to-perfection}}

\includegraphics{https://static01.graylady3jvrrxbe.onion/images/2009/05/19/dining/20rest600.1.jpg?quality=75\&auto=webp\&disable=upscale}

By \href{https://www.nytimes3xbfgragh.onion/by/frank-bruni}{Frank Bruni}

\begin{itemize}
\item
  May 19, 2009
\item
  \begin{itemize}
  \item
  \item
  \item
  \item
  \item
  \end{itemize}
\end{itemize}

THE minute you heard that Keith McNally was dusting off Minetta Tavern
--- that musty, sputtering Greenwich Village relic from the late 1930s
--- you probably figured he'd get the look and atmosphere right. This is
a project perfectly suited to a restaurateur with a gift for breathing
fresh life into familiar genres, for grafting the present onto the past.

You also knew he'd conjure buzz. He always conjures buzz. Where Mr.
McNally goes, models, movie honchos and magazine scribes follow, because
they're sure to find themselves among other members of their slavishly
fashionable tribe, coddled in an environment that's as much stage set as
mess hall.

And maybe, just maybe, you counted on decent food. Even when his
restaurants miss their mark, they usually reflect at least some culinary
thought. He cares about cooking, more than he must. Balthazar, for
example, has typically been about twice as good as it needs to be.

But were you prepared for a côte de boeuf like Minetta's, a sublime hunk
of glorious meat that you dream about hours later, pine for the next day
and extol in a manner so rapturous and nonstop that friends begin to
worry less about your cholesterol than about your sanity?

And did you expect that Mr. McNally, with the chefs Riad Nasr and Lee
Hanson, would come up with the best steakhouse in the city? That's what
Minetta Tavern turns out to be.

For starters it's serving some of the most expertly aged, flavorful and
exquisitely prepared prime beef in New York. This beef is showcased in
one trailblazing burger and two titanic steaks, the côte de boeuf for
two and a bone-in New York strip, that have for two months now been the
incessant talk of insatiable carnivores, who can't get enough of them.

Are you better off with the côte de boeuf, which is more generously
marbled with fat, or the strip, with its steelier, brawnier taste? You
could ponder, discuss and dither over this question as long as you could
any stimulus package, only it would be a lot more fun. You could
dedicate half a dozen visits to Minetta in the service of a resolution
and still it might elude you.

Or you could abandon the debate for a while and explore other areas of
the menu --- and you wouldn't be disappointed. Although little of the
rest of Minetta's food rises all the way to the extraordinarily high
level of the beef, much of it is terrific. Minetta's claim to being New
York's best steakhouse rests in large part on its versatility.

While Sparks has a justly renowned strip of its own, it doesn't have
appetizers as quietly sophisticated as Minetta's creamy mussel soup.
While the best porterhouse on the best night at Peter Luger can be an
amazement, there's no seafood there as fine as the tender, sweet lobster
in a big, crisp salad at Minetta or as this restaurant's trout meunière,
buttery and bedecked with crab meat. And at Minetta the servers don't
bark at you.

Minetta's potatoes Anna, which are like a love affair between scalloped
potatoes and hash browns, rival the best sides at Strip House, where the
beef itself isn't in Minetta's league. Minetta has a succinct selection
of American cheeses so accomplished you could mistake them for European
--- you don't get that at Keens. It doesn't charge tariffs as steep as
Craftsteak's or the BLT empire's.

And with just 70 seats in two rooms, it feels more intimate than, say,
Primehouse New York or Porter House New York. Yes, it's cramped and
loud, but that's in keeping with the genre and, as such, a facet of its
charm. Minetta captures the clubby, chaotic spirit of a handsomely
timeworn saloon to a T: the long, heavy wood bar up front; the glowing
tiers of liquor behind it; the tiled floors; the tin ceiling. Plus there
are all those framed etchings of celebrities on the walls and the gauzy
painted murals in the back.

Mr. McNally has buffed what needed buffing, added what needed adding ---
the dark red booths, for example, are new --- and left the rest of the
place intact. It's high-gloss nostalgia: McNally assoluta.

It finds him in a more ambitious mood than at other restaurants he's
opened over the last decade, in terms of what's on the plate. Neither
Pastis nor Schiller's Liquor Bar seek to do any food as superior as
Minetta's meat, including the crisp lamb saddle and succulent veal chop.

And Morandi doesn't compete against the city's Italian standouts the way
Minetta takes on the most prized temples of sirloin.

As Mr. McNally put this restaurant together with Mr. Nasr and Mr.
Hanson, who are veterans of Balthazar and Pastis, the three arranged
through the distributor Pat La Frieda to get their hands on the
fantastic grain-fed Black Angus beef from Creekstone Farms, which has a
richness that's indulgent without crossing into unctuous overkill as
wagyu sometimes does.

They had Mr. La Frieda set up a special aging room where Minetta's
strips, its côte de boeuf and the rib-eye used in a blend (with short
rib, brisket and skirt) for its much-ballyhooed Black Label burger are
stored for six to seven weeks.

And in the kitchen they installed a high-temperature broiler that gives
the meat precisely the char it wants. The steaks don't develop a surface
that's too crunchy, which can happen at Luger, and they're not finished
with an excess of butter, another Luger liability. They're spot-on ---
at least the ones on the ``grillades'' section of the menu, where the
best cuts are clustered. A lower-priced bar steak elsewhere isn't prime
or dry-aged.

Wrongly, the Black Label burger has received more public attention than
the steaks, on account of its \$26 price tag. It's without question a
riveting experience, because burgers seldom pack the discernible tang
and funk of aged beef. But for that same reason, it's unsettling and
arguably \emph{too} intense.

Besides, the Minetta burger, a blend of short rib and brisket, manages a
comparable juiciness at a price of \$16, including a heap of crunchy,
salty, addictive fries. And with a burger, you look to be comforted, not
awed.

Comfort, all in all, isn't this restaurant's strong point. Good luck
penetrating the bodies around the host station after 7:30 p.m. And good
luck nabbing a reservation any time between 7 and 10 unless you have and
use an inside phone line, which I didn't.

Image

Credit...Robert Presutti for The New York Times

Minetta has additional drawbacks. The desserts need slight improvement,
especially the coarse, flat-footed sorbets, though you'll have no
complaint whatsoever with the sumptuous chocolate dacquoise.

The wine list, better than those at older and squarer steakhouses, can
nonetheless be frustrating, with too few accessibly priced reds that
beckon you.

Should you want to tread more lightly than the \$36 strip and the \$90
côte de boeuf --- both generously portioned and neither out of line with
the cost of prime beef these days --- the trout is \$24. An equally fine
grilled dorade is \$21. A large carbonara-style pasta dish that's better
than three-quarters of what I had at Morandi is \$16. The excellent
fried pig's trotter, served over lentils, is \$19.

Given his name and his stardust, Mr. McNally didn't have to provide this
reasonable a path through his newest restaurant, not even now. But at
Minetta he's made a series of decisions that go admirably beyond the
bottom line. And he's made the kitchen the focal point of a
resuscitation that's ultimately about eating more than anything else.

By all means take delight in the vintage décor. Stay for the steak.

\textbf{Minetta Tavern}

***

113 Macdougal Street (Minetta Lane), Greenwich Village; (212) 475-3850.

\textbf{ATMOSPHERE} A cramped late 1930s saloon gets the McNally
makeover, which is a respectful combination of period fidelity and
contemporary gloss, along with the jostling crowd that usually entails.

\textbf{SOUND LEVEL} Tiled floor, tin ceiling, hungry throngs: what do
you think?

\textbf{RECOMMENDED DISHES} Mussel and cream soup; lobster salad;
roasted marrow bones; omelet with morels; trout; pasta za za; pig's
trotter; roasted chicken; New York strip; côte de boeuf; lamb saddle;
veal chop; potatoes Anna; chocolate dacquoise; tarte Tropézienne.

\textbf{WINE LIST} International, with a French emphasis and too few
appealing selections under \$60 a bottle.

\textbf{PRICE RANGE} Appetizers, \$10 to \$24; entrees, \$16 to \$45;
desserts, \$9.

\textbf{HOURS} From 5:30 p.m. to midnight daily, limited menu until 1
a.m. from Sunday through Thursday and until 2 a.m. Friday and Saturday.
Weekend brunch to come soon.

\textbf{RESERVATIONS} For prime times, taken up to three weeks in
advance; that will expand to a month soon.

\textbf{CREDIT CARDS} All major cards.

\textbf{WHEELCHAIR ACCESS} Inaccessible.

\textbf{WHAT THE STARS MEAN} Ratings range from zero to four stars and
reflect the reviewer's reaction to food, ambience and service, with
price taken into consideration. Menu listings and prices are subject to
change.

Advertisement

\protect\hyperlink{after-bottom}{Continue reading the main story}

\hypertarget{site-index}{%
\subsection{Site Index}\label{site-index}}

\hypertarget{site-information-navigation}{%
\subsection{Site Information
Navigation}\label{site-information-navigation}}

\begin{itemize}
\tightlist
\item
  \href{https://help.nytimes3xbfgragh.onion/hc/en-us/articles/115014792127-Copyright-notice}{©~2020~The
  New York Times Company}
\end{itemize}

\begin{itemize}
\tightlist
\item
  \href{https://www.nytco.com/}{NYTCo}
\item
  \href{https://help.nytimes3xbfgragh.onion/hc/en-us/articles/115015385887-Contact-Us}{Contact
  Us}
\item
  \href{https://www.nytco.com/careers/}{Work with us}
\item
  \href{https://nytmediakit.com/}{Advertise}
\item
  \href{http://www.tbrandstudio.com/}{T Brand Studio}
\item
  \href{https://www.nytimes3xbfgragh.onion/privacy/cookie-policy\#how-do-i-manage-trackers}{Your
  Ad Choices}
\item
  \href{https://www.nytimes3xbfgragh.onion/privacy}{Privacy}
\item
  \href{https://help.nytimes3xbfgragh.onion/hc/en-us/articles/115014893428-Terms-of-service}{Terms
  of Service}
\item
  \href{https://help.nytimes3xbfgragh.onion/hc/en-us/articles/115014893968-Terms-of-sale}{Terms
  of Sale}
\item
  \href{https://spiderbites.nytimes3xbfgragh.onion}{Site Map}
\item
  \href{https://help.nytimes3xbfgragh.onion/hc/en-us}{Help}
\item
  \href{https://www.nytimes3xbfgragh.onion/subscription?campaignId=37WXW}{Subscriptions}
\end{itemize}
