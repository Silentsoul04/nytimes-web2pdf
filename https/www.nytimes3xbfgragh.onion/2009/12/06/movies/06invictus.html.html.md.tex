Sections

SEARCH

\protect\hyperlink{site-content}{Skip to
content}\protect\hyperlink{site-index}{Skip to site index}

\href{https://www.nytimes3xbfgragh.onion/section/movies}{Movies}

\href{https://myaccount.nytimes3xbfgragh.onion/auth/login?response_type=cookie\&client_id=vi}{}

\href{https://www.nytimes3xbfgragh.onion/section/todayspaper}{Today's
Paper}

\href{/section/movies}{Movies}\textbar{}An Actor Nails the Cadence and
the Charm

\begin{itemize}
\item
\item
\item
\item
\item
\end{itemize}

Advertisement

\protect\hyperlink{after-top}{Continue reading the main story}

Supported by

\protect\hyperlink{after-sponsor}{Continue reading the main story}

Film

\hypertarget{an-actor-nails-the-cadence-and-the-charm}{%
\section{An Actor Nails the Cadence and the
Charm}\label{an-actor-nails-the-cadence-and-the-charm}}

\includegraphics{https://static01.graylady3jvrrxbe.onion/images/2009/12/06/arts/06invictus_span_CA0/articleLarge.jpg?quality=75\&auto=webp\&disable=upscale}

By
\href{https://topics.nytimes3xbfgragh.onion/top/reference/timestopics/people/k/bill_keller/index.html}{Bill
Keller}

\begin{itemize}
\item
  Dec. 2, 2009
\item
  \begin{itemize}
  \item
  \item
  \item
  \item
  \item
  \end{itemize}
\end{itemize}

\href{https://archive.nytimes3xbfgragh.onion/www.nytimes3xbfgragh.onion/2009/12/06/movies/06invictus.html}{See
how this article appeared when it was originally published on
NYTimes.com.}

MORGAN FREEMAN has been cast as God --- twice --- so he evidently has no
trouble projecting moral authority. The challenge of portraying Nelson
Mandela, then, was not the size of the halo, but knowing the performance
would be measured against the real, familiar Mandela, and his myth. ``If
we can say any part of acting is hard, then playing someone who is
living and everybody knows would be the hardest,'' Mr. Freeman said in a
phone interview.

The role has defeated actors as varied as Danny Glover (the 1987 TV film
``Mandela''), Sidney Poitier (``Mandela and de Klerk,'' 1997, also for
TV) and Dennis Haysbert (``Goodbye Bafana,'' 2007), in vehicles that
were reverential and mostly forgettable.

But as someone who studied Mr. Mandela over the course of three years
while he replaced an apartheid regime with a genuine democracy, I found
Mr. Freeman's performance in the film ``Invictus,'' directed by Clint
Eastwood, uncanny --- less an impersonation than an incarnation.

He gets the rumble and halting rhythm of Mr. Mandela's speech, the erect
posture and stiff gait. There is a striking physical resemblance,
enhanced by the fact that Mr. Freeman, 72, is just a few years younger
than Mr. Mandela was in the period the film covers. More important, Mr.
Freeman conveys the manipulative charm, the serene confidence, the force
of purpose, the hint of mischief and the lonely regret that made Mr.
Mandela one of the most fascinating political figures of his time. This
is not, as the film's screenwriter, Anthony Peckham, put it, ``Rich
Little doing Mandela in Vegas.''

It's hard to say whether Americans at this moment in their history crave
a 130-minute parable of racial reconciliation built around a 1995 World
Cup rugby match in South Africa. Audiences and movie critics will render
their verdict on ``Invictus,'' which reaches theaters Friday.

But we could probably do worse, as an antidote to the cynicism on the
noisy margins of our political life, than spending a couple of hours
watching Mr. Mandela calculating how to knit together a grotesquely
divided society.

The story of ``Invictus,'' drawn from John Carlin's book ``Playing the
Enemy: Nelson Mandela and the Game That Made a Nation,'' begins with the
newly inaugurated president of post-apartheid South Africa looking for
ways to enlist his fearful white minority --- with its talent, wealth,
resentment and capacity for insurrection --- in the business of
governing a democracy. His inspired stratagem is to embrace the
Springboks national rugby team, the darlings of the formerly ruling
Afrikaners and, for most nonwhite South Africans, a symbol of brutal and
humiliating repression.

The new president sets the team's captain (François Pienaar, played by
Matt Damon) the improbable goal of winning the World Cup; the tournament
is to be held in South Africa in a year, and the Springboks are given
little chance. Mr. Mandela sets himself the considerably more improbable
goal of uniting country behind the team.

So loathed were the Springboks that those few blacks who showed up for
matches rooted loudly for the other side. So the rugby campaign was one
of Mr. Mandela's boldest strokes of statecraft, no less impressive for
the fact that the euphoria he achieved could barely begin to extinguish
three centuries of racial antagonism.

Mr. Freeman's occupational association with South Africa began with a
role in the 1992 film ``The Power of One,'' the pious tale of a white
boy coming to enlightenment in apartheid South Africa. Soon thereafter
Mr. Freeman made his directing debut with a more tough-minded film,
``Bopha!,'' the story of a conflicted black South African cop, played by
Mr. Glover. (Lori McCreary, who was a producer on that film and is a
producer of ``Invictus,'' said she tried to lure Mr. Freeman for the
lead part in ``Bopha!,'' but was told he ``doesn't do accents.'')

Image

Morgan Freeman as Nelson Mandela.Credit...Keith Bernstein

According to Mr. Freeman, his mission to portray Mr. Mandela on the
screen began with a public invitation from the subject himself. At a
press conference to promote the publication of his 1994 memoir, ``Long
Walk to Freedom,'' someone asked Mr. Mandela who should play him in the
movie.

``And he said he wanted me,'' Mr. Freeman recalled. ``So it became. That
was the whole sanction, right there.''

The South African film producer Anant Singh, who bought the movie rights
to ``Long Walk,'' arranged for Mr. Mandela and Mr. Freeman to meet.

``I told him that if I was going to play him, I was going to have to
have access to him,'' the actor said. ``That I would have to hold his
hand and watch him up close and personal.'' As president Mr. Mandela
could be surprisingly approachable --- he once allowed me, the New York
Times correspondent in South Africa at the time, to shadow him during a
day of his presidency, something I can scarcely imagine an American
president allowing. But since stepping down in 1999, and especially
since his memory began to fail him, he has become more reclusive,
protected by a staff that worries he might embarrass himself. But he
obliged Mr. Freeman.

``Whenever we've been in proximity in one city or another, I have had
access to him,'' the actor said. Their encounters ranged from tea at Mr.
Mandela's home in Johannesburg to a charity fund-raiser in Monaco. But
through multiple screenplays Mr. Mandela's sprawling memoir proved too
unwieldy for a film, and Mr. Freeman abandoned the project.

``There's just too much to whittle down to movie size,'' Mr. Freeman
said.

Then, in 2006, Mr. Carlin, a British journalist who had covered Mr.
Mandela in the 1990s, was in Mississippi to write an article on poverty
in the American South for El Pais, the Spanish daily that now employs
him. He ended up in the Clarksdale living room of Mr. Freeman's business
partner. When the host went to the kitchen for a bottle of wine, Mr.
Carlin recalls, he turned to Mr. Freeman.

``This is your lucky day,'' he said. ``I have a movie for you.''

``Oh, really,'' Mr. Freeman replied. ``What's it about?''

``It's based on a book I am writing about an event that distills the
essence of Mandela's genius, and the essence of the South African
miracle.''

``Oh,'' Mr. Freeman replied, ``you mean the rugby game?''

Mr. Carlin's proposal for his book had already been circulating in
Hollywood, and it had caught Mr. Freeman's eye.

Mr. Freeman sought Mr. Mandela's blessing, bought the rights and
persuaded Mr. Eastwood to direct. (Their two previous collaborations,
``Unforgiven'' and ``Million Dollar Baby,'' both won best picture
Oscars.) They hired Mr. Peckham, a South African émigré, to write the
script.

Mr. Freeman insists that if the portrayal transcends impersonation, that
is largely Mr. Peckham's doing.

As an actor, ``you're looking for the physical: how he stands, how he
walks, how he talks,'' he said. ``Nuances he has in terms of tics or
movements. Things that sort of define him. The inner life has to come
off the page. Whatever he's thinking, I don't know. You have a script,
and you stick to that script, and the script is going to inform you of
everything.''

Image

Matt Damon, center, plays the South African rugby player François
Pienaar in ``Invictus,'' directed by Clint Eastwood.Credit...Keith
Bernstein/Warner Brothers Pictures

While Mr. Freeman brought to the project a decade of firsthand
observation, Mr. Peckham, who left South Africa in 1981, had never ---
and still has not --- met Mr. Mandela.

``He was a nonperson for my entire growing up,'' Mr. Peckham said in a
phone interview from his home in California. ``You weren't even supposed
to have pictures of him. Everything I learned about him I learned from a
distance, after I came here.''

For the feel of Mr. Mandela's everyday speech, the screenwriter mined
written documents, especially transcripts of a 1998 court case in which
the South African president was subjected to a hostile grilling by
lawyers for the national rugby hierarchy. (It tells you something about
the incompleteness of the redemptive turn depicted in ``Invictus'' that,
three years after the famous rugby match, Mr. Mandela appointed a
commission to study whether the powerful rugby union was thwarting the
advancement of black players.)

Mr. Peckham's main difficulty in writing a script, he found, was to do
justice to such a familiar and beloved figure without tipping into
idolatry.

``It was extremely difficult, because in the period I write about he was
in many respects at his most saintly --- leading the country the way he
did,'' Mr. Peckham said. The danger of hagiography ``was something we
all knew was an issue and that I struggled with every day while I was
writing it. With the additional complication that we didn't want to be
offensive and disrespectful either. It's easy enough to kind of show
someone's feet of clay if you're prepared to be brutal about it, but
it's not so easy when you want to be respectful without
hero-worshiping.''

The notion they settled on to humanize the hero was that while Mr.
Mandela was making a nation he was neglecting his own family. It is
certainly true that Mr. Mandela's marriage to the cause contributed to
his two divorces and his estrangement from some of his children. In the
movie there is a scene of Mr. Mandela, who could always summon the words
to move a crowd, failing to connect with his resentful grown daughter
Zinzi.

``Knowing what I know of Madiba personally,'' Mr. Freeman said, using
Mr. Mandela's clan name, ``his real concern is not for what he did, but
more for what he didn't do. He had family obligations that he couldn't
live up to, one, because he was in prison, and they just wouldn't allow
it, and he had so many other obligations. The father of the nation is
usually less than the father of his family.''

South Africans listening to Mr. Freeman's rendering may agree that he
``doesn't do accents.'' (He says ``Spring-BAHK'' where Mandela would say
``Spring-BOHK.'') But Mr. Mandela's distinctive voice is less about
accent than cadence, and Mr. Freeman gets that precisely right.

Mr. Carlin, who covered Mr. Mandela in his political prime and spent
many hours with him for the rugby book, said Mr. Freeman ``channels
Mandela beautifully.''

Most important, Mr. Carlin said, Mr. Freeman, abetted by the
screenwriter, ``impressively conveys the giant solitude of Mandela.''

Though an admirer of Mr. Freeman, Mr. Carlin has seen Mr. Mandela gotten
wrong often enough that he braced himself for disappointment. After
attending a screening in Paris last month, he sent an ecstatic e-mail
message: ``They didn't screw it up!'' he wrote. ``WHAT a relief!''

For me the realization that Mr. Freeman had nailed it came as the film
ended. Alongside the closing credits came still photos of the actual
rugby match, and the actual Mandela. And for a second I wondered, ``Who
is that impostor?''

Advertisement

\protect\hyperlink{after-bottom}{Continue reading the main story}

\hypertarget{site-index}{%
\subsection{Site Index}\label{site-index}}

\hypertarget{site-information-navigation}{%
\subsection{Site Information
Navigation}\label{site-information-navigation}}

\begin{itemize}
\tightlist
\item
  \href{https://help.nytimes3xbfgragh.onion/hc/en-us/articles/115014792127-Copyright-notice}{©~2020~The
  New York Times Company}
\end{itemize}

\begin{itemize}
\tightlist
\item
  \href{https://www.nytco.com/}{NYTCo}
\item
  \href{https://help.nytimes3xbfgragh.onion/hc/en-us/articles/115015385887-Contact-Us}{Contact
  Us}
\item
  \href{https://www.nytco.com/careers/}{Work with us}
\item
  \href{https://nytmediakit.com/}{Advertise}
\item
  \href{http://www.tbrandstudio.com/}{T Brand Studio}
\item
  \href{https://www.nytimes3xbfgragh.onion/privacy/cookie-policy\#how-do-i-manage-trackers}{Your
  Ad Choices}
\item
  \href{https://www.nytimes3xbfgragh.onion/privacy}{Privacy}
\item
  \href{https://help.nytimes3xbfgragh.onion/hc/en-us/articles/115014893428-Terms-of-service}{Terms
  of Service}
\item
  \href{https://help.nytimes3xbfgragh.onion/hc/en-us/articles/115014893968-Terms-of-sale}{Terms
  of Sale}
\item
  \href{https://spiderbites.nytimes3xbfgragh.onion}{Site Map}
\item
  \href{https://help.nytimes3xbfgragh.onion/hc/en-us}{Help}
\item
  \href{https://www.nytimes3xbfgragh.onion/subscription?campaignId=37WXW}{Subscriptions}
\end{itemize}
