Sections

SEARCH

\protect\hyperlink{site-content}{Skip to
content}\protect\hyperlink{site-index}{Skip to site index}

\href{https://www.nytimes3xbfgragh.onion/pages/theater/index.html}{Theater
Reviews}

\href{https://myaccount.nytimes3xbfgragh.onion/auth/login?response_type=cookie\&client_id=vi}{}

\href{https://www.nytimes3xbfgragh.onion/section/todayspaper}{Today's
Paper}

\href{/pages/theater/index.html}{Theater Reviews}\textbar{}Stomping Onto
Broadway With a Punk Temper Tantrum

\begin{itemize}
\item
\item
\item
\item
\item
\end{itemize}

Advertisement

\protect\hyperlink{after-top}{Continue reading the main story}

Supported by

\protect\hyperlink{after-sponsor}{Continue reading the main story}

Theater Review \textbar{} 'American Idiot'

\hypertarget{stomping-onto-broadway-with-a-punk-temper-tantrum}{%
\section{Stomping Onto Broadway With a Punk Temper
Tantrum}\label{stomping-onto-broadway-with-a-punk-temper-tantrum}}

\includegraphics{https://static01.graylady3jvrrxbe.onion/images/2010/04/21/theater/21idiot-span/21idiot-span-articleLarge.jpg?quality=75\&auto=webp\&disable=upscale}

\begin{itemize}
\tightlist
\item
  American Idiot\\
  **NYT Critic's Pick Broadway, Musical Closing Date: April 24, 2011 St.
  James Theater, 246 W. 44th St. 866-870-2717
\end{itemize}

By
\href{https://www.nytimes3xbfgragh.onion/by/charles-isherwood}{Charles
Isherwood}

\begin{itemize}
\item
  April 20, 2010
\item
  \begin{itemize}
  \item
  \item
  \item
  \item
  \item
  \end{itemize}
\end{itemize}

Rage and love, those consuming emotions felt with a particularly acute
pang in youth, all but burn up the stage in ``American Idiot,'' the
thrillingly raucous and gorgeously wrought Broadway musical adapted from
the blockbuster pop-punk album by Green Day.

Pop on Broadway, sure. But punk? Yes, indeed, and served straight up,
with each sneering lyric and snarling riff in place. A stately old pile
steps from the tourist-clogged Times Square might seem a strange place
for the music of Green Day, and for theater this blunt, bold and
aggressive in its attitude. Not to mention loud. But from the moment the
curtain rises on a panorama of baleful youngsters at the venerable St.
James Theater, where the show opened on Tuesday night, it's clear that
these kids are going to make themselves at home, even if it means
tearing up the place in the process.

Which they do, figuratively speaking. ``American Idiot,'' directed by
Michael Mayer and performed with galvanizing intensity by a terrific
cast, detonates a fierce aesthetic charge in this ho-hum Broadway
season. A pulsating portrait of wasted youth that invokes all the
standard genre conventions --- bring on the sex, drugs and rock 'n'
roll, please! --- only to transcend them through the power of its music
and the artistry of its execution, the show is as invigorating and
ultimately as moving as anything I've seen on Broadway this season. Or
maybe for a few seasons past.

Burning with rage and love, and knowing how and when to express them,
are two different things, of course. The young men we meet in the first
minutes of ``American Idiot'' are too callow and sullen and restless ---
too young, basically --- to channel their emotions constructively. The
show opens with a glorious 20-minute temper tantrum kicked off by the
title song.

``Don't want to be an American idiot!'' shouts one of the gang. The
song's signature electric guitar riff slashes through the air, echoing
the testy challenge of the cry. A sharp eight-piece band, led by the
conductor Carmel Dean, is arrayed around the stage, providing a sonic
frame for the action. The simple but spectacular set, designed by
Christine Jones, suggests an epically scaled dive club, its looming
walls papered in punk posters and pimpled by television screens, on
which frenzied video collages flicker throughout the show. (They're the
witty work of Darrel Maloney.)

Image

Christina Sajous, left, as the Extraordinary Girl, and Stark Sands as
Tunny in ``American Idiot,'' with music by Green Day, lyrics by Billie
Joe Armstrong and directed by Michael Mayer.Credit...Sara Krulwich/The
New York Times

Who's the American idiot being referred to? Well, as that curtain slowly
rose, we heard the familiar voice of George W. Bush break through a haze
of television chatter: ``Either you are with us, or with the
terrorists.'' That kind of talk could bring out the heedless rebel in
any kid, particularly one who is already feeling itchy at the lack of
prospects in his dreary suburban burg.

But while ``American Idiot'' is nominally a portrait of youthful malaise
of a particular era --- the album dates from 2004, the midpoint of the
Bush years, and the show is set in ``the recent past'' --- its depiction
of the crisis of post-adolescence is essentially timeless. Teenagers
eager for their lives to begin, desperate to slough off their old selves
and escape boredom through pure sensation, will probably always be
making the same kinds of mistakes, taking the same wrong turns on the
road to self-discovery.

``American Idiot'' is a true rock opera, almost exclusively using the
music of Green Day and the lyrics of its kohl-eyed frontman, Billie Joe
Armstrong, to tell its story. (The score comprises the whole of the
title album as well as several songs from the band's most recent
release, ``21st Century Breakdown.'') The book, by Mr. Armstrong and Mr.
Mayer, consists only of a series of brief, snarky dispatches sent home
by the central character, Johnny, played with squirmy intensity by the
immensely gifted John Gallagher Jr. (``Spring Awakening,'' ``Rabbit
Hole'').

``I held up my local convenience store to get a bus ticket,'' Johnny
says with a smirk as he and a pal head out of town.

``Actually I stole the money from my mom's dresser.''

Beat.

``Actually she lent me the cash.''

Such is the sheepish fate of a would-be rebel today. But at least Johnny
and his buddy Tunny (Stark Sands) do manage to escape deadly suburbia
for the lively city, bringing along just their guitars and the anomie
and apathy that are the bread and butter of teenage attitudinizing the
world over. (``I don't care if you don't care,'' a telling lyric, could
be their motto.)

The friend they meant to bring along, Will (Michael Esper), was forced
to stay home when he discovered that his girlfriend (Mary Faber) was
pregnant. Lost and lonely, and far from ready for the responsibilities
of fatherhood, he sinks into the couch, beer in one hand and bong in the
other, as his friends set off for adventure.

Beneath the swagger of indifference, of course, are anxiety, fear and
insecurity, which Mr. Gallagher, Mr. Esper and Mr. Sands transmit with
aching clarity in the show's more reflective songs, like the hit
``Boulevard of Broken Dreams'' or the lilting anthem ``Are We the
Waiting.'' The city turns out to be just a bigger version of the place
Johnny and Tunny left behind, a ``land of make believe that don't
believe in me.'' The boys discover that while a fractious 21st-century
America may not offer any easy paths to fulfillment, the deeper problem
is that they don't know how to believe in themselves.

Image

John Gallagher Jr., center, and Tony Vincent, upper right, in ``American
Idiot.''Credit...Sara Krulwich/The New York Times

Johnny strolls the lonely streets with his guitar, vaguely yearning for
love and achievement. He eventually hooks up with a girl (a vivid
Rebecca Naomi Jones) but falls more powerfully under the spell of an
androgynous goth drug pusher, St. Jimmy, played with mesmerizing
vitality and piercing vocalism by Tony Vincent. Tunny mostly stays in
bed, clicker affixed to his right hand, dangerously susceptible to a
pageant of propaganda about military heroism on the tube, set to the
song ``Favorite Son.'' By the time the song's over, he's enlisted and
off to Iraq.

In both plotting and its emotional palette, ``American Idiot'' is drawn
in brash, primary-colored strokes, maybe too crudely for those looking
for specifics of character rather than cultural archetypes. But operas
--- rock or classical --- often trade in archetypes, and the actors
flesh out their characters' journeys through their heartfelt
interpretations of the songs, with the help of Mr. Mayer's poetic
direction and the restless, convulsive choreography of Steven Hoggett
(``Black Watch''), which exults in both the grace and the awkwardness of
energy-generating young metabolisms.

Line by line, a skeptic could fault Mr. Armstrong's lyrics for their
occasional glibness or grandiosity. That's to be expected, too: rock
music exploits heightened emotion and truisms that can fit neatly into a
memorable chorus. The songs are precisely as articulate --- and
inarticulate --- as the characters are, reflecting the moment in youth
when many of us feel that pop music has more to say about us than we
have to say for ourselves. (And, really, have you ever worked your way
through a canonical Italian opera libretto, line by line?)

In any case the music is thrilling: charged with urgency, rich in
memorable melody and propulsive rhythms that sometimes evolve midsong.
The orchestrations by Tom Kitt (the composer of ``Next to Normal'') move
from lean and mean to lush, befitting the tone of each number. Even if
you are unfamiliar with Green Day's music, you are more likely to emerge
from this show humming one of the guitar riffs than you are to find a
tune from ``The Addams Family'' tickling your memory.

But the emotion charge that the show generates is as memorable as the
music. ``American Idiot'' jolts you right back to the dizzying roller
coaster of young adulthood, that turbulent time when ecstasy and misery
almost seem interchangeable states, flip sides of the coin of
exaltation. It captures with a piercing intensity that moment in life
when everything seems possible, and nothing seems worth doing, or maybe
it's the other way around.

\textbf{AMERICAN IDIOT}

Music by Green Day; lyrics by Billie Joe Armstrong; book by Mr.
Armstrong and Michael Mayer; directed by Mr. Mayer; choreography by
Steven Hoggett; musical supervision, arrangements and orchestrations by
Tom Kitt; sets by Christine Jones; costumes by Andrea Lauer; lighting by
Kevin Adams; sound by Brian Ronan; video and projections by Darrel
Maloney; technical supervision by Hudson Theatrical Associates; music
coordinator, Michael Keller; music director, Carmel Dean; associate
choreographer, Lorin Latarro; associate director, Johanna McKean.
Presented by Tom Hulce and Ira Pittelman, Ruth and Stephen Hendel, Vivek
J. Tiwary and Gary Kaplan, Aged in Wood and Burnt Umber, Scott M.
Delman, Latitude Link, HOP Theatricals and Jeffrey Finn, Larry Welk,
Bensinger Filerman and Maellenberg Taylor, Allan S. Gordon and Élan V.
McAllister and Berkeley Repertory Theater, in association with Awaken
Entertainment and John Pinckard and John Domo. At the St. James Theater,
246 West 44th Street, Manhattan; (212) 239-6200. Running time: 1 hour 30
minutes.

WITH: John Gallagher Jr. (Johnny), Stark Sands (Tunny), Michael Esper
(Will), Rebecca Naomi Jones (Whatshername), Christina Sajous (the
Extraordinary Girl), Mary Faber (Heather) and Tony Vincent (St. Jimmy).

Advertisement

\protect\hyperlink{after-bottom}{Continue reading the main story}

\hypertarget{site-index}{%
\subsection{Site Index}\label{site-index}}

\hypertarget{site-information-navigation}{%
\subsection{Site Information
Navigation}\label{site-information-navigation}}

\begin{itemize}
\tightlist
\item
  \href{https://help.nytimes3xbfgragh.onion/hc/en-us/articles/115014792127-Copyright-notice}{©~2020~The
  New York Times Company}
\end{itemize}

\begin{itemize}
\tightlist
\item
  \href{https://www.nytco.com/}{NYTCo}
\item
  \href{https://help.nytimes3xbfgragh.onion/hc/en-us/articles/115015385887-Contact-Us}{Contact
  Us}
\item
  \href{https://www.nytco.com/careers/}{Work with us}
\item
  \href{https://nytmediakit.com/}{Advertise}
\item
  \href{http://www.tbrandstudio.com/}{T Brand Studio}
\item
  \href{https://www.nytimes3xbfgragh.onion/privacy/cookie-policy\#how-do-i-manage-trackers}{Your
  Ad Choices}
\item
  \href{https://www.nytimes3xbfgragh.onion/privacy}{Privacy}
\item
  \href{https://help.nytimes3xbfgragh.onion/hc/en-us/articles/115014893428-Terms-of-service}{Terms
  of Service}
\item
  \href{https://help.nytimes3xbfgragh.onion/hc/en-us/articles/115014893968-Terms-of-sale}{Terms
  of Sale}
\item
  \href{https://spiderbites.nytimes3xbfgragh.onion}{Site Map}
\item
  \href{https://help.nytimes3xbfgragh.onion/hc/en-us}{Help}
\item
  \href{https://www.nytimes3xbfgragh.onion/subscription?campaignId=37WXW}{Subscriptions}
\end{itemize}
