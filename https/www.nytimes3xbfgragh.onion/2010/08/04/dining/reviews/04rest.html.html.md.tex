Sections

SEARCH

\protect\hyperlink{site-content}{Skip to
content}\protect\hyperlink{site-index}{Skip to site index}

\href{https://www.nytimes3xbfgragh.onion/pages/dining/index.html}{Dining
\& Wine}

\href{https://myaccount.nytimes3xbfgragh.onion/auth/login?response_type=cookie\&client_id=vi}{}

\href{https://www.nytimes3xbfgragh.onion/section/todayspaper}{Today's
Paper}

\href{/pages/dining/index.html}{Dining \& Wine}\textbar{}Tamarind
Tribeca

\begin{itemize}
\item
\item
\item
\item
\item
\end{itemize}

Advertisement

\protect\hyperlink{after-top}{Continue reading the main story}

Supported by

\protect\hyperlink{after-sponsor}{Continue reading the main story}

Restaurant Review

\hypertarget{tamarind-tribeca}{%
\section{Tamarind Tribeca}\label{tamarind-tribeca}}

\includegraphics{https://static01.graylady3jvrrxbe.onion/images/2010/08/04/dining/04restspan-1/04restspan-1-articleLarge.jpg?quality=75\&auto=webp\&disable=upscale}

By \href{https://www.nytimes3xbfgragh.onion/by/sam-sifton}{Sam Sifton}

\begin{itemize}
\item
  Aug. 3, 2010
\item
  \begin{itemize}
  \item
  \item
  \item
  \item
  \item
  \end{itemize}
\end{itemize}

IT is hard to know what to make of Tamarind Tribeca at first, seeing it
rise up from the corner of Hudson and Franklin Streets, sleek and
glass-fronted, like a bank.

The restaurant is gigantic, occupying more than 10,000 square feet in
the space where a sticky little coffee shop called Socrates and a grim
sports bar used to be. (Nobu is across the street.) There are towering
windows and acres of warm Brazilian teak, a run of marble at the bar,
stairs up to a mezzanine. Roughly 500 staff members seem to be milling
about in the mango-hued light.

It hardly seems intimate. The foyer has enough room for a few dozen
people. You can feel a little lost. Families are scarce in the dining
room --- dates, friendships, too. Pressed shirts abound, and wide
English ties. Suit jackets are thrown over the backs of chairs and bar
stools. The restaurant seems almost ostentatious, with its excess of
design and wide passageways: as if it might be a pan-Indian food mill
for the professional class, of little use or interest to those who chase
the delicious.

The city has another Tamarind up in the Flatiron district, a decade-old
exercise in sophisticated Indian cuisine, and when this TriBeCa version
opened in April, it appeared simply to be a newer model of the first
one, brought downtown to capitalize on the neighborhood's wealth and
proximity to the financial markets.

But then! Here is a booth in which to sit, quiet and slightly removed
from the bustle of the rest of the room, separated from others by a
sheer curtain; or a little table in the restaurant's back, like a small
square off a town's main one; or one of the tables upstairs on the
mezzanine that looks over the dining room as if at a city.

It comes as a surprise: Tamarind is an extremely pleasant place to dine,
and despite the size of the room, it is possible for a group to have a
conversation there as if in a private home.

Here, too, is Gary Walia, the restaurant's manager (nephew to Avtar
Walia, the owner), directing his well-trained and helpful staff as if he
were conducting an orchestra, greeting guests in the manner of a
subcontinental Julian Niccolini, of the Four Seasons restaurant in
Midtown. He treats strangers as regulars, and regulars with glee.

The bar serves an excellent gin and tonic, cold and tall. And the
restaurant has an extensive wine list, well suited to the cuisine, if a
little vast and expensive at the high end. So, have a drink and consider
some curry-laced crab cakes and crisp pomegranate samosas, and the
promise beyond them of a menu that can take diners across India in the
name of flavor, and represent that nation's varied cuisine with pride
and great skill.

In London, where marvelous Indian food is as much a part of the culinary
landscape as French restaurants or steakhouses are here, Tamarind
Tribeca might rate a pleasant shrug. But in Manhattan, it is shaping up
to be the best thing to happen to Indian food since Hemant Mathur and
Suvir Saran opened Devi in 2004.

The menu is more extensive than at Tamarind's Flatiron branch. Under the
direction of Peter Beck, the restaurant's executive chef, it sprawls:
cooking from Punjab, Goa, Hyderabad, Madras, Calcutta and Lucknow are
all represented.

And its desserts are ambitious, Westernized, recalling in sweets what
the chef Floyd Cardoz has done at Tabla with savory Indian food, with a
list that offers both a goat cheese crème brûlée and a brilliant coconut
mousse ``bombe'' with a chocolate-Darjeeling ganache and pineapple
butter cake.

By all means save room for those. But begin your meal with an appetizer
called bataki kosha, which offers a kind of rice-crepe egg roll of
shredded duck scented with garam masala, mustard and onion, ginger and
garlic, then deep-fried and served with tangerine chutney. Another,
galouti kebab, traditionally made with mutton, brings small patties of
lamb scented with dozens of spices --- coriander, cumin, cardamom and
cloves are definitely four of them --- that call out for Kingfisher beer
and perhaps a second order.

Nizami keema is a marvel: an appetizing sandwich of grilled and minced
lamb, pungent with cinnamon, cloves, nutmeg and a high pitch of lemon
zest, in a small bun. As are turmeric-hued sautéed scallops, with poppy
seeds and a bright lemon sauce.

It is hardly soup weather. But Tamarind offers two worth tasting all the
same, buttery and rich: a regal lobster with carrots and garlic, Cognac
and cayenne pepper; and a silky butternut squash with fenugreek seeds,
lemon grass and a low roar of green chilies.

The entrees are as ambitious, though a simple, luscious, heat-infused
chicken tikka masala proves a fine explanation of why this totally
inauthentic dish might displace roast beef in the hearts of the British.
Offset by a pile of goat-meat biriyani, funky and rich, some excellent
stewed garbanzo beans with pomegranate powder, ginger and tomatoes, and
some of the cheese-infused spinach known as saag paneer, it made for a
memorable meal.

So did lamb shank braised in Indian rum, with a soft, addictive sauce of
onions, cashews, saffron and nutmeg, a perfect topping for the
restaurant's pillowy basmati rice, and a good partner to a glossy bowl
of dal makhani as well, lentils simmered with garlic, ginger and a great
deal of clarified butter.

Dry, chalky pan-seared halibut was disappointing, and a flavorful
roasted lobster managed to appear on the plate as if it had been
prepared in a wood-chipper. But sea bass cooked in the tandoor oven was
breathtakingly well prepared, the intense heat of the oven somehow
sealing in the flavor of the fish, and a dressing of thick yogurt, dill,
lime zest and peppercorns enhancing it.

Venison chops also benefited from the tandoor treatment, marinated in
pickling spices, then coated in chickpea flour and yogurt before
cooking. You might put these up against roasted lamb chops for a taste
test, marinated in yogurt, cardamom, garlic, black cumin seeds and
nutmeg. It's a fair fight.

Winner takes on the prawns: just charred by the oven, big and meaty,
spiced to blaze mouths and bring laughter. Everyone wins.

\textbf{Tamarind Tribeca}

★★

99 Hudson Street (Franklin Street), TriBeCa; (212) 775-9000.

\textbf{ATMOSPHERE} Sleek and elegant, airy and open: a glass of Bombay
Sapphire, clear and cool.

\textbf{SOUND LEVEL} Given the size of the space and the hardness of the
surfaces, surprisingly conversational.

\textbf{RECOMMENDED DISHES} Crab cakes, samosas, bataki kosha, braised
lamb shank, chicken tikka masala, goat biriyani, saag paneer,
coconut-mousse bombe.

\textbf{WINE LIST} A much longer and more varied list than you might be
used to seeing in an Indian restaurant. It is well paired to the food,
though there are a few four-figure howlers at the top end.

\textbf{PRICE RANGE} Appetizers, \$7.50 to \$14; entrees, \$13.50 to
\$32.

\textbf{HOURS} Daily, 11:30 a.m. to 3 p.m.; Sunday to Thursday, 5:30 to
11:30 p.m.; Friday and Saturday to midnight

\textbf{RESERVATIONS} Recommended, at least a week in advance.

\textbf{CREDIT CARDS} All major cards.

\textbf{WHEELCHAIR ACCESS} Restaurant is mainly on one level; restrooms
are large.

\textbf{WHAT THE STARS MEAN} Ratings range from zero to four stars and
reflect the reviewer's reaction to food, ambience and service, with
price taken into consideration. Menu listings and prices are subject to
change.

Advertisement

\protect\hyperlink{after-bottom}{Continue reading the main story}

\hypertarget{site-index}{%
\subsection{Site Index}\label{site-index}}

\hypertarget{site-information-navigation}{%
\subsection{Site Information
Navigation}\label{site-information-navigation}}

\begin{itemize}
\tightlist
\item
  \href{https://help.nytimes3xbfgragh.onion/hc/en-us/articles/115014792127-Copyright-notice}{©~2020~The
  New York Times Company}
\end{itemize}

\begin{itemize}
\tightlist
\item
  \href{https://www.nytco.com/}{NYTCo}
\item
  \href{https://help.nytimes3xbfgragh.onion/hc/en-us/articles/115015385887-Contact-Us}{Contact
  Us}
\item
  \href{https://www.nytco.com/careers/}{Work with us}
\item
  \href{https://nytmediakit.com/}{Advertise}
\item
  \href{http://www.tbrandstudio.com/}{T Brand Studio}
\item
  \href{https://www.nytimes3xbfgragh.onion/privacy/cookie-policy\#how-do-i-manage-trackers}{Your
  Ad Choices}
\item
  \href{https://www.nytimes3xbfgragh.onion/privacy}{Privacy}
\item
  \href{https://help.nytimes3xbfgragh.onion/hc/en-us/articles/115014893428-Terms-of-service}{Terms
  of Service}
\item
  \href{https://help.nytimes3xbfgragh.onion/hc/en-us/articles/115014893968-Terms-of-sale}{Terms
  of Sale}
\item
  \href{https://spiderbites.nytimes3xbfgragh.onion}{Site Map}
\item
  \href{https://help.nytimes3xbfgragh.onion/hc/en-us}{Help}
\item
  \href{https://www.nytimes3xbfgragh.onion/subscription?campaignId=37WXW}{Subscriptions}
\end{itemize}
