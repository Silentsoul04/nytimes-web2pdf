Sections

SEARCH

\protect\hyperlink{site-content}{Skip to
content}\protect\hyperlink{site-index}{Skip to site index}

\href{https://www.nytimes3xbfgragh.onion/pages/dining/index.html}{Dining
\& Wine}

\href{https://myaccount.nytimes3xbfgragh.onion/auth/login?response_type=cookie\&client_id=vi}{}

\href{https://www.nytimes3xbfgragh.onion/section/todayspaper}{Today's
Paper}

\href{/pages/dining/index.html}{Dining \& Wine}\textbar{}El Bulli and a
Meal for the Ages

\begin{itemize}
\item
\item
\item
\item
\item
\end{itemize}

Advertisement

\protect\hyperlink{after-top}{Continue reading the main story}

Supported by

\protect\hyperlink{after-sponsor}{Continue reading the main story}

\href{/column/the-pour}{The Pour}

\hypertarget{el-bulli-and-a-meal-for-the-ages}{%
\section{El Bulli and a Meal for the
Ages}\label{el-bulli-and-a-meal-for-the-ages}}

By \href{https://www.nytimes3xbfgragh.onion/by/eric-asimov}{Eric Asimov}

\begin{itemize}
\item
  Sept. 21, 2010
\item
  \begin{itemize}
  \item
  \item
  \item
  \item
  \item
  \end{itemize}
\end{itemize}

MORE than a year ago, before anybody had the slightest inkling that a
potential closing was in the offing, I achieved the unexpected coup of a
reservation at El Bulli, Ferran Adrià's temple of avant-garde dining in
Spain.

I had never been there before, and it was a bit of a departure for me.
In covering food and wine, I've had an awful lot of opportunities to eat
in Michelin three-star restaurants or the equivalent. Most of the time,
I have tried to avoid them. Sure, the food is great. The meals are also
complicated, incredibly expensive, time consuming and fatiguing.

Maybe I've become that horrid caricature, the jaded guy who dreads the
punishment of having to go to a fancy restaurant. And in this economy! I
know, we all should have such problems.

I didn't always feel this way. Up until a few years ago, nothing would
have thrilled me more than dining at one of the most famous restaurants
on earth. I remember mind-blowing, life-changing meals at Boyer Les
Crayères outside Reims, France; and Taillevent in Paris; and fabulous
gatherings here in New York, at Le Bernardin, Per Se and Jean Georges.
Last year, I took my family to Daniel on my older son's 18th birthday as
a rite of passage, and he and I had the joy of discovering Alinea
together on a college trip to Chicago.

But over time, I have come to prefer simplicity over an exhausting
complexity and numbing elaboration. One of the greatest meals in my
recent memory took place in a rustic Italian wine cellar in Campania,
where, leaning against barrels in the dim light, we bit into balls of
warm, creamy buffalo mozzarella, bread torn from a crusty loaf and
chunks of spicy dried pork sausage. We drank rough red wine, the perfect
complement. The flavors were so elemental, direct and pure that I cannot
imagine a better meal, even if we'd had cutlery.

I've experienced a parallel progression with wine. Who wouldn't love a
great Burgundy, a first-growth Bordeaux or a beautifully aged Barolo
every night of the week?

Should I feel ashamed to raise my hand here? Perhaps, again, because I
have had the pleasure of drinking many of these wines, I can afford to
feel choosy. Deserving wine lovers would trade an awful lot of Sancerre
for a bottle of Montrachet, one of the pinnacles of white-wine making. I
know this feeling deep in my gut. I certainly have not drunk enough
Montrachet in my life to be jaded about it, but at least I've had the
opportunity to savor it. Until you've had such experiences, the quest
for great wines --- for benchmarks --- is of paramount intellectual
importance.

\includegraphics{https://static01.graylady3jvrrxbe.onion/images/2010/09/22/dining/26pour/26pour-jumbo.jpg?quality=75\&auto=webp\&disable=upscale}

But if you have had these enviable opportunities, what becomes clear is
that majestic old bottles of magnitude and complexity, as well as more
modest bottles, have their proper place. Young, fresh wines offer
carefree pleasures, and build the foundation for a lifetime of enjoying
wine. These unpretentious wines form a basis for distinguishing wines of
greatness and consequence, just as humble, everyday meals provide the
earthbound frame of reference for heavenly feasts.

Context is everything. Leaving aside moral consequences, the regular
consumption of monumental wines and exquisitely prepared foods
diminishes their capacity to inspire. When everything is awesome, people
lose the capacity to be awed. But reserving them for the rare moments
when the right company and the right time coalesce can't help but
intensify their meaning.

So it was with El Bulli. It had been a gorgeous, sunny day in Spain, in
the city of Roses on the Costa Brava, the takeoff point for our journey
to the restaurant. And it is indeed a journey. A painfully narrow road
winds precariously over a promontory to the quiet cove where the
restaurant is. Dizzying drops, unimpeded by guardrails, plant gruesome
images of smoking wrecks in the canyon below --- please, God, let it be
after the meal, not before!

Finally we arrived at the restaurant, a tranquil oasis where the only
sound seemed to be the waves lapping at the shore. For me, the drive was
liberating, opening the mind and soul to whatever followed. It was not
merely a cab ride across town, but a crossing to another world, like
arriving at Brigadoon.

It almost doesn't matter what we ate, and after 38 courses, none bigger
than a few bites, who can remember specifics? Oh, if you insist, I can't
forget extraordinary beetroot cookies and tomato crackers, each
capturing sweet essences in suspended animation. Or succulent oysters
blended with luscious bone marrow, or tiny ravioli, the skins like clear
plastic teabags, which melt in your mouth until they dissolve in an
explosion of pine nut flavor.

What remains lodged in my memory is the totality --- the laughter, the
surprise, the moments of discovered deliciousness as each novel
presentation unmoored the food from the burden of expectations. I
thought back many years, to the look of absolute astonishment and
delight on the face of my son Peter the first time he tasted chocolate
cake. I think that look was plastered on my face over the course of our
five-hour meal.

One of my friends brought along some older Riojas that were astounding
in their own right. But how do you pair wines with so many different
dishes? Left to the extensive but quite reasonably priced wine list, I
would have stuck happily with Champagne throughout the meal. Not
ordinary Champagne either, but superb, hard-to-find bottles like Selosse
Brut Initiale, which retails for about \$125 but was on the list for
\$165, or Jérôme Prévost for \$140, or maybe both.

My meal was a transcendent experience, as it was meant to be. Once in a
lifetime is perfect, though I wouldn't put it past El Bulli if twice
were even more so.

Advertisement

\protect\hyperlink{after-bottom}{Continue reading the main story}

\hypertarget{site-index}{%
\subsection{Site Index}\label{site-index}}

\hypertarget{site-information-navigation}{%
\subsection{Site Information
Navigation}\label{site-information-navigation}}

\begin{itemize}
\tightlist
\item
  \href{https://help.nytimes3xbfgragh.onion/hc/en-us/articles/115014792127-Copyright-notice}{©~2020~The
  New York Times Company}
\end{itemize}

\begin{itemize}
\tightlist
\item
  \href{https://www.nytco.com/}{NYTCo}
\item
  \href{https://help.nytimes3xbfgragh.onion/hc/en-us/articles/115015385887-Contact-Us}{Contact
  Us}
\item
  \href{https://www.nytco.com/careers/}{Work with us}
\item
  \href{https://nytmediakit.com/}{Advertise}
\item
  \href{http://www.tbrandstudio.com/}{T Brand Studio}
\item
  \href{https://www.nytimes3xbfgragh.onion/privacy/cookie-policy\#how-do-i-manage-trackers}{Your
  Ad Choices}
\item
  \href{https://www.nytimes3xbfgragh.onion/privacy}{Privacy}
\item
  \href{https://help.nytimes3xbfgragh.onion/hc/en-us/articles/115014893428-Terms-of-service}{Terms
  of Service}
\item
  \href{https://help.nytimes3xbfgragh.onion/hc/en-us/articles/115014893968-Terms-of-sale}{Terms
  of Sale}
\item
  \href{https://spiderbites.nytimes3xbfgragh.onion}{Site Map}
\item
  \href{https://help.nytimes3xbfgragh.onion/hc/en-us}{Help}
\item
  \href{https://www.nytimes3xbfgragh.onion/subscription?campaignId=37WXW}{Subscriptions}
\end{itemize}
