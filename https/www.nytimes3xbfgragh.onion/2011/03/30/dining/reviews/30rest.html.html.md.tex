Sections

SEARCH

\protect\hyperlink{site-content}{Skip to
content}\protect\hyperlink{site-index}{Skip to site index}

\href{https://www.nytimes3xbfgragh.onion/pages/dining/index.html}{Dining
\& Wine}

\href{https://myaccount.nytimes3xbfgragh.onion/auth/login?response_type=cookie\&client_id=vi}{}

\href{https://www.nytimes3xbfgragh.onion/section/todayspaper}{Today's
Paper}

\href{/pages/dining/index.html}{Dining \& Wine}\textbar{}Crosstown Tour
of India

\begin{itemize}
\item
\item
\item
\item
\item
\end{itemize}

Advertisement

\protect\hyperlink{after-top}{Continue reading the main story}

Supported by

\protect\hyperlink{after-sponsor}{Continue reading the main story}

Restaurant Review

\hypertarget{crosstown-tour-of-india}{%
\section{Crosstown Tour of India}\label{crosstown-tour-of-india}}

\includegraphics{https://static01.graylady3jvrrxbe.onion/images/2011/03/30/dining/30rest-span/30rest-span-articleLarge.jpg?quality=75\&auto=webp\&disable=upscale}

By \href{https://www.nytimes3xbfgragh.onion/by/sam-sifton}{Sam Sifton}

\begin{itemize}
\item
  March 29, 2011
\item
  \begin{itemize}
  \item
  \item
  \item
  \item
  \item
  \end{itemize}
\end{itemize}

TWO new Indian restaurants have joined the Manhattan fray, competitors
for a place in the upper castes of the city's Indian fine-dining scene.
Junoon rises lush and enormous off a dark stretch of West 24th Street,
an Indian restaurant remade as a luxe Western palace, with a marvelous
wine list and perfect rye manhattans. Tulsi glitters bright and almost
romantic on East 46th Street across the street from Sparks, a Midtown
bet on a great chef's talent.

Hemant Mathur is the force behind Tulsi. He was an owner and a chef at
the pretty, bejeweled Devi, in the Flatiron district, where he often
could be seen fussing nervously at the edges of the dining room before
retreating to the kitchen to cook with rare brilliance. At Tulsi he does
the same dance and, if the setting is less attractive, louder, less
special than at his former restaurant, it places his cooking in sharper
relief.

No one in New York makes lamb chops like Mr. Mathur --- heavy, ugly
things caked in yogurt but tasting of gamy perfection --- and it is a
relief to know that he has brought that entree north with him. His
Manchurian cauliflower has made the transition as well: a magical dish,
sweet and fiery, worth ordering in multiples even at a small table, so
everyone present can thrill to its sweet and heat, and guess at the
secret ingredient (spoiler: it's ketchup, well caramelized over the
fried cornstarch batter).

He serves a wickedly fine duck moilee --- a delicate coastal curry
softened by coconut milk --- and a deeply flavorful curried monkfish
with pomegranate sauce, the spices within each dish rendered distinct
and powerful in the cooking. And there are very good tandoori prawns,
courtesy of one of Mr. Mathur's partners in this enterprise, the chef
Dhandu Ram, whom older students may remember from his time at Bukhara
Grill. Roasted to crispness, the prawns emerge from the tandoor sweetly
juicy within, a lovely match for the okra that comes with them, and
tingling eggplant chutney.

Tulsi's vegetable dishes do not disappoint either, most notably the
house dal, simmered until its primary ingredients, lentils and clarified
butter, appear to become one: a creamy yogurt made of legumes.

The breads, in particular a rich and salty rosemary-garlic naan, perform
marvels alongside the food. There is even a soft ginger panna cotta for
dessert, from Mr. Mathur's wife, Surbhi Sahni, to undercut the notion
that all Indian sweets must be wickedly so, as if to make up for the
heat of the entrees.

Too bad, then, that Tulsi's lighting is harsh, and that the service
style runs more to chain-restaurant gab than anything approaching the
hushed, pajama-clad grace of Devi. Too bad, too, that behind its warm
and comfortable bar, the restaurant opens into a bazaar of tables more
reminiscent of Home Depot's patio furniture department than anything
filmed by Merchant-Ivory. It is not a particularly enjoyable place to
eat dinner. The floating curtains between tables have the stiffness not
of silk but of polyester. They look itchy. And under that light, only a
few people can look good.

AT Junoon, just west of Madison Square Park, everyone looks good. Little
expense has been spared to make sure that this is the case. The
restaurant is lighted as if by softly glowing amber. The teak lounge
furnishings and long bar up front might have been pulled straight off
the floor at ABC Carpet \& Home. And the enormous dining room, at once
airy and filled with deeply upholstered chairs and banquettes, features
both large sandstone sculptures and a steel reflecting pool in which
float lotus blossoms under candlelight.

Rajesh Bhardwaj, the Cafe Spice tycoon, is the founder and chief
executive of this restaurant, now his glittery flagship. Sommeliers
wheel linen-covered wine carts about the dining room with quiet
efficiency, and the conversation bubbles along at a hushed level, as at
an old-line Italian or French establishment. Imagine a kind of
dream-state Del Posto, the sort of restaurant where of course you would
order a 2006 Valpolicella Grassi with your monkfish tikka.

Never mind knowing why this is so. (But it is so!) Ask one of those
sommeliers for advice, at whatever price you consider comfortable.
Junoon's cellar has within it much to injure the belief that the best
accompaniment to Indian food is beer.

In the open kitchen is Vikas Khanna, late of Salaam Bombay and numerous
television appearances. His menu offers a wide-ranging examination of
Indian cuisine, divided not by region but cooking style.

It is, over all, very good. The tandoor dishes are delicate beneath
their crusts: sweet lobster under a cloak of cumin, cayenne and lemon,
with ground fennel; venison amped up on ginger and nutmeg. A lentil soup
--- three-lentil shorba, it's called here --- rides silky in its elegant
bowl, enlivened by cilantro, over a seat of fresh turmeric.

Lamb is given a wonderful send-off by the patthar, a style in which the
meat is cooked on hot stones. Even better are the curries, in particular
the grilled eggplant, and a rich and fragrant duck version with an
enormous amount of Tellicherry pepper to counterbalance the richness of
the meat, along with a faint hum of tamarind, a zing of curry leaves.

The restaurant's desserts are as good as they are few in number, with
but three options above the traditional kulfi and (less traditional)
sorbet. The best of these is a date pudding cake held together with
caramel, cranberries and streusel, and an orange-buttermilk ice cream.
It is rich and smoky, with a creamy tang, amazing: Brahmin food that
recalls Boston more than Bengal.

Junoon is comfortable and elegant, almost more than the two Tamarind
restaurants, in the Flatiron and in TriBeCa, to which it will inevitably
be compared. Its design and service style hint at the kind of
European-style sumptuousness that used to be common to upscale
restaurants in Manhattan, at least before the recession removed
tablecloths and quiet from so many of our dining rooms.

It is a very nice place to spend a few hours, dressed and dining and
drinking well. Tulsi may indeed offer a slightly more attractive bill of
fare. But restaurants are always about more than simply the food.

~

\textbf{Tulsi}

★

211 East 46th Street, Midtown; (212) 888-0820, tulsinyc.com.

\textbf{ATMOSPHERE} A well-intentioned suburban feel, in the heart of
Midtown.

\textbf{SOUND LEVEL} Raised voices are unnecessary.

\textbf{RECOMMENDED DISHES} Manchurian cauliflower, butter chicken,
tandoor lamb chops, duck moilee, ginger panna cotta.

\textbf{WINE LIST} Here is your chardonnay, your Napa cab. Better is a
bottle of 1947, an Indian lager.

\textbf{PRICE RANGE} Appetizers, \$7 to \$14; entrees, \$16 to \$34.

\textbf{HOURS} Lunch: Monday to Saturday, noon to 2:30 p.m. Dinner: 5:30
to 10:30 p.m.; Sunday, 5 to 10 p.m.

\textbf{RESERVATIONS} Recommended at least one week ahead.

\textbf{CREDIT CARDS} All major cards.

\textbf{WHEELCHAIR ACCESS} The bar, dining room and restrooms are all on
one level, and aisles are reasonably wide. Restrooms are large.

\textbf{WHAT THE STARS MEAN} Ratings range from zero to four stars and
reflect the reviewer's reaction to food, ambience and service, with
price taken into consideration. Menu listings and prices are subject to
change.

~

\textbf{Junoon}

★★

27 West 24th Street, Flatiron; (212) 490-2100, junoonnyc.com.

\textbf{ATMOSPHERE} Opulent, warm and comfortable.

\textbf{SOUND LEVEL} Hushed and self-satisfied.

\textbf{RECOMMENDED DISHES} Three-lentil shorba, piri-piri shrimp,
lobster tandoori, monkfish tikka, duck with Tellicherry pepper, grilled
eggplant curry, date pudding cake.

\textbf{WINE LIST} Extensive, interesting and well worth discussing with
a sommelier. Roussillon with the eggplant? Why, um, sure!

\textbf{PRICE RANGE} Appetizers, \$10 to \$15; entrees, \$16 to \$33.

\textbf{HOURS} Lunch: Monday to Friday, noon to 3 p.m. Dinner: Sunday to
Thursday, 5:30 to 10:30 p.m.; Friday and Saturday, to 11 p.m.

\textbf{RESERVATIONS} Recommended at least one week ahead.

\textbf{CREDIT CARDS} All major cards.

\textbf{WHEELCHAIR ACCESS} The bar and dining room are at street level,
and there is plenty of space between tables.

Advertisement

\protect\hyperlink{after-bottom}{Continue reading the main story}

\hypertarget{site-index}{%
\subsection{Site Index}\label{site-index}}

\hypertarget{site-information-navigation}{%
\subsection{Site Information
Navigation}\label{site-information-navigation}}

\begin{itemize}
\tightlist
\item
  \href{https://help.nytimes3xbfgragh.onion/hc/en-us/articles/115014792127-Copyright-notice}{©~2020~The
  New York Times Company}
\end{itemize}

\begin{itemize}
\tightlist
\item
  \href{https://www.nytco.com/}{NYTCo}
\item
  \href{https://help.nytimes3xbfgragh.onion/hc/en-us/articles/115015385887-Contact-Us}{Contact
  Us}
\item
  \href{https://www.nytco.com/careers/}{Work with us}
\item
  \href{https://nytmediakit.com/}{Advertise}
\item
  \href{http://www.tbrandstudio.com/}{T Brand Studio}
\item
  \href{https://www.nytimes3xbfgragh.onion/privacy/cookie-policy\#how-do-i-manage-trackers}{Your
  Ad Choices}
\item
  \href{https://www.nytimes3xbfgragh.onion/privacy}{Privacy}
\item
  \href{https://help.nytimes3xbfgragh.onion/hc/en-us/articles/115014893428-Terms-of-service}{Terms
  of Service}
\item
  \href{https://help.nytimes3xbfgragh.onion/hc/en-us/articles/115014893968-Terms-of-sale}{Terms
  of Sale}
\item
  \href{https://spiderbites.nytimes3xbfgragh.onion}{Site Map}
\item
  \href{https://help.nytimes3xbfgragh.onion/hc/en-us}{Help}
\item
  \href{https://www.nytimes3xbfgragh.onion/subscription?campaignId=37WXW}{Subscriptions}
\end{itemize}
