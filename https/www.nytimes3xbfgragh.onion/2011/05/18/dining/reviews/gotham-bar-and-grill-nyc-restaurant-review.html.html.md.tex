Sections

SEARCH

\protect\hyperlink{site-content}{Skip to
content}\protect\hyperlink{site-index}{Skip to site index}

\href{https://www.nytimes3xbfgragh.onion/pages/dining/index.html}{Dining
\& Wine}

\href{https://myaccount.nytimes3xbfgragh.onion/auth/login?response_type=cookie\&client_id=vi}{}

\href{https://www.nytimes3xbfgragh.onion/section/todayspaper}{Today's
Paper}

\href{/pages/dining/index.html}{Dining \& Wine}\textbar{}Gotham Bar and
Grill

\begin{itemize}
\item
\item
\item
\item
\item
\end{itemize}

Advertisement

\protect\hyperlink{after-top}{Continue reading the main story}

Supported by

\protect\hyperlink{after-sponsor}{Continue reading the main story}

Restaurant Review

\hypertarget{gotham-bar-and-grill}{%
\section{Gotham Bar and Grill}\label{gotham-bar-and-grill}}

\includegraphics{https://static01.graylady3jvrrxbe.onion/images/2011/05/18/dining/18rest-span/18rest-span-articleLarge.jpg?quality=75\&auto=webp\&disable=upscale}

By \href{https://www.nytimes3xbfgragh.onion/by/sam-sifton}{Sam Sifton}

\begin{itemize}
\item
  May 17, 2011
\item
  \begin{itemize}
  \item
  \item
  \item
  \item
  \item
  \end{itemize}
\end{itemize}

IN a city obsessed with the shiny and novel, Gotham Bar and Grill is an
outlier. Open since 1984, it celebrates stability and excellence,
perhaps even opulence. Fabric chandeliers billow beneath soaring
ceilings. Elaborate flowers are everywhere. People dress for dinner and
are escorted past marvelous photographs on the way to their tables. It
is all very civilized there.

Gotham has an informality of service, though, that is purely American,
despite the starched white tablecloths.

And Alfred Portale, who came to this Greenwich Village restaurant the
year after it opened, runs a kitchen that is committed to innovation
even as it celebrates the past. His life must occasionally resemble
those of perpetually touring rock stars who need every night to sing the
songs they wrote as teenagers. But even after more than 20 years, new
songs keep coming.

A sweet, tomato-flecked risotto with red Maine shrimp, nuggets of bacon
and a few artful tangles of wild arugula arrived as an appetizer. The
bacon had heft and smoke and fatty crunch, and would not have been out
of place at a pop-up restaurant in some corner of Bushwick, Brooklyn,
served by a bearish kitchen poet with a rutabaga tattoo on his forearm.
It provided a fine complement to the tender flesh of the shrimp, the
bite of the greens, the tender, chewy, broth-thickened rice. The
combination tasted absolutely of now.

Beside it on the table: Mr. Old School, a thick disk of perfectly diced
yellowfin tuna, with Japanese cucumber and shiso leaf, the fish barely
dressed in a sweet miso vinaigrette that tingled with ginger. Crouton
obelisks rose out of this base, with greens between them: an
architectural flourish that has been a highlight of Mr. Portale's
cooking since the start of his career. It was a deeply familiar dish,
beautifully rendered: the original article, not a knockoff. It hummed
with flavor.

Both dishes were delivered without pretense, with minor, quiet ceremony.
A waiter approached and placed them on the table, made sure all was well
and departed. He had things to do. At Gotham, there is no recitation of
the provenance of the bacon, or explanation of the wild arugula's
journey, before you eat. No one offers the name of the boat whose
captain whispered the tuna aboard.

This is refreshing, as it happens. A meal at Gotham is about you and
your interests, not of those who made it.

That fact may weigh sometimes on Mr. Portale. Being a success for so
long comes with mighty, and surely sometimes boring, responsibilities.
(``Five tuna tartares,'' a waitress might tell the kitchen at 6, making
her first order of the night. Many, many more will follow.) But if this
is the case, he does not show it.

Take as an example the restaurant's seafood salad, a dish that
\href{http://www.nytimes3xbfgragh.onion/1985/10/04/arts/resaturants.html}{Bryan
Miller raved about} in The New York Times in 1985, when he awarded the
restaurant three stars; that Molly O'Neill loved in a 1993 review that
gave the same rating; that Ruth Reichl called Mr. Portale's signature
dish in the newspaper's most recent previous review of the restaurant,
in 1996, which also awarded three stars.

That salad is still on the menu, and still seemingly on most of the
date-night tables running south along the bar: a molded tangle of
scallops, squid, octopus and sweet, briny lobster, swathed by a slice of
ripe avocado, dressed simply in lemon and olive oil. The wrap of the
avocado is dashing, if no longer really in style. It recalls suits with
shoulder pads and hair teased up with
\href{http://www.flickr.com/photos/ajbear/3293720226/}{Tenax}, Steve
Winwood singing
``\href{http://vodpod.com/watch/1476909-steve-winwood-higher-love}{Higher
Love.}''

It still tastes terrific, though, with every flavor in balance. It still
offers excitement. For those many for whom Gotham is a place to
celebrate birthdays, graduations and anniversaries, it needs always to
be on the menu, tasting exactly as it does.

None of which is to say Gotham is perfect. The restaurant is very
expensive, almost aggressively so, with starters that hover in the
mid-\$20s and entrees that can go to twice that number, with a wine list
that hides its bargains well.

Above the lights is a ceiling that is deeply unattractive, a stucco
vision of those '70s lofts with coffee-can track lighting. And the
casual service seems sometimes to be perfunctory, even if it never
really lags. (You will always get your water. Sometimes it comes with a
splash.) The restaurant does not always show its age well. Some surfaces
need fresh paint.

But Mr. Portale does have plenty of new material to showcase. On the
current menu, there is an appetizer of cold-smoked Tasmanian sea trout,
served with tender baby fennel and celery hearts, with Meyer lemon for
acidity and tiny pumpernickel croutons for texture against the firm
slickness of the fish. It is unreasonably good. There is another of
spring-pea ravioli, luxurious and sweet, with fava beans, delicate pea
tendrils and some Parmesan, in a bacon broth of real weight and
seriousness --- a Greenmarket treat enlivened by a trip to the
smokehouse.

He makes shiitake soup with hazelnuts, crème fraîche and aged sherry
vinegar. It might be beef consommé and French onion soup and veal yogurt
combined into some liquid ambrosia, and a taste of it may serve as
complete explanation for all those who do not understand food's power to
make people laugh from pleasure. It is an appetizer that excites and
relaxes at once: a pair of velvet slippers for those who order it.

For entrees, Mr. Portale cooks out of a number of larders without ever
compromising the integrity of his overall vision. This is a remarkable
trick. A single table might have on it a Thai-spiced Maine lobster next
to a free-range chicken dusted with the French curry powder known as
Vadouvan, with a seared fist of Atlantic halibut with morels and
asparagus across the way, next to a dry-aged New York strip steak served
with batter-fried Vidalia onion rings.

On the night when that actually happened, each dish tasted absolutely
and elaborately true to itself: a celebration of American diversity that
avoided melting pots entirely. The lobster stood at attention among
sheets of water spinach and shards of crisp snow peas, with ginger root
and a pillow of soft rice noodles, in a heady bath of lemongrass broth
amplified by kaffir lime. The chicken meanwhile, sweet and perfumed,
with perfect skin, tacked toward the north and south coasts of the
Mediterranean with layered vegetables, onion confit and preserved lemon,
with a potato purée. That halibut, meanwhile, was forager-chic, utterly
unfettered in its flavors, with a lick of white wine across its flesh.

And the steak? It came with a marrow-mustard custard of remarkable
intensity and a bordelaise you could use to adhere stamps. Only an
enormous and enormously delicious dry-aged porterhouse-for-two was
better, with a potato gratin, spring onions, sweet English peas, grilled
asparagus and more of that bordelaise: British food made by Frenchmen,
for Americans to cheer with massive zinfandels, with heady California
cabernets.

Speaking of which, care should be taken with the wine list, which
contains a lot of these California jam pots at prices to put a sheik
back on his heels. Ask for one of the sommeliers, put forward a price
and discuss what you are eating and what you like to drink. They know
the list, its traps and honey holes. Let them work it to your advantage:
they might suggest a nebbiolo instead, at a price you can afford.

``Desserts are intense and very American,'' Ms. Reichl wrote of Gotham's
offerings in 1996. This is still the case. Deborah Racicot, the pastry
chef, makes a rich and dashing plate of chocolate peanut-butter mousse
with dark-chocolate caramel sauce and a raspberry lambic sorbet, and
another that is a take on the classic s'more, with a smoked bittersweet
chocolate tart, toffee and an excellent quenelle of root-beer ice cream.
There is also a marvelous maple-glazed pineapple cake with hibiscus mint
granité and a bourbon-buttermilk ice cream.

It is a crazy-quilt finish to a meal that deserves three cheers.

\textbf{Gotham Bar and Grill}

★★★

12 East 12th Street, Greenwich Village; (212) 620-4020,
gothambarandgrill.com.

\textbf{ATMOSPHERE} A once-modern, now classic take on American haute
cuisine in a soaring dining room filled with celebrants, tourists and
regulars.

\textbf{SOUND LEVEL} Moderate but for the occasional chorus of ``Happy
Birthday'' from a table nearby.

\textbf{RECOMMENDED DISHES} Seafood salad, tuna tartare, Maine ruby red
shrimp risotto, cold-smoked sea trout, spring-pea ravioli, shiitake
mushroom soup, Thai-spiced lobster, Vadouvan spiced chicken, halibut,
lamb, steaks, s'mores, chocolate peanut-butter mousse, maple-glazed
pineapple cake.

\textbf{WINE LIST} Deceptively expensive and stacked toward big American
flavors. Ask for a sommelier's advice.

\textbf{PRICE RANGE} Appetizers, \$20 to \$28; entrees, \$34 to shared
dishes of \$130 or more. Five-course tasting menu, \$95.

\textbf{HOURS} Monday to Friday, noon to 2:30 p.m.; Monday to Thursday,
5:30 to 10 p.m., Friday, 5:30 to 11 p.m., Saturday, 5 to 11 p.m.,
Sunday, 5 to 10 p.m.

\textbf{RESERVATIONS} Recommended at least two weeks ahead.

\textbf{CREDIT CARDS} All major cards.

\textbf{WHEELCHAIR ACCESS} There are steps up to the bar, down to the
dining room. Restrooms are down a flight of stairs.

\textbf{WHAT THE STARS MEAN} Ratings range from zero to four stars and
reflect the reviewer's reaction to food, ambience and service, with
price taken into consideration. Menu listings and prices are subject to
change.

Advertisement

\protect\hyperlink{after-bottom}{Continue reading the main story}

\hypertarget{site-index}{%
\subsection{Site Index}\label{site-index}}

\hypertarget{site-information-navigation}{%
\subsection{Site Information
Navigation}\label{site-information-navigation}}

\begin{itemize}
\tightlist
\item
  \href{https://help.nytimes3xbfgragh.onion/hc/en-us/articles/115014792127-Copyright-notice}{©~2020~The
  New York Times Company}
\end{itemize}

\begin{itemize}
\tightlist
\item
  \href{https://www.nytco.com/}{NYTCo}
\item
  \href{https://help.nytimes3xbfgragh.onion/hc/en-us/articles/115015385887-Contact-Us}{Contact
  Us}
\item
  \href{https://www.nytco.com/careers/}{Work with us}
\item
  \href{https://nytmediakit.com/}{Advertise}
\item
  \href{http://www.tbrandstudio.com/}{T Brand Studio}
\item
  \href{https://www.nytimes3xbfgragh.onion/privacy/cookie-policy\#how-do-i-manage-trackers}{Your
  Ad Choices}
\item
  \href{https://www.nytimes3xbfgragh.onion/privacy}{Privacy}
\item
  \href{https://help.nytimes3xbfgragh.onion/hc/en-us/articles/115014893428-Terms-of-service}{Terms
  of Service}
\item
  \href{https://help.nytimes3xbfgragh.onion/hc/en-us/articles/115014893968-Terms-of-sale}{Terms
  of Sale}
\item
  \href{https://spiderbites.nytimes3xbfgragh.onion}{Site Map}
\item
  \href{https://help.nytimes3xbfgragh.onion/hc/en-us}{Help}
\item
  \href{https://www.nytimes3xbfgragh.onion/subscription?campaignId=37WXW}{Subscriptions}
\end{itemize}
