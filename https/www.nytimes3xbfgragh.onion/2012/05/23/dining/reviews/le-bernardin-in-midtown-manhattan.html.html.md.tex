Sections

SEARCH

\protect\hyperlink{site-content}{Skip to
content}\protect\hyperlink{site-index}{Skip to site index}

\href{https://www.nytimes3xbfgragh.onion/pages/dining/index.html}{Dining
\& Wine}

\href{https://myaccount.nytimes3xbfgragh.onion/auth/login?response_type=cookie\&client_id=vi}{}

\href{https://www.nytimes3xbfgragh.onion/section/todayspaper}{Today's
Paper}

\href{/pages/dining/index.html}{Dining \& Wine}\textbar{}Moving Ever
Forward, Like a Fish

\url{https://nyti.ms/Lj8drF}

\begin{itemize}
\item
\item
\item
\item
\item
\end{itemize}

Advertisement

\protect\hyperlink{after-top}{Continue reading the main story}

Supported by

\protect\hyperlink{after-sponsor}{Continue reading the main story}

Restaurant Review

\hypertarget{moving-ever-forward-like-a-fish}{%
\section{Moving Ever Forward, Like a
Fish}\label{moving-ever-forward-like-a-fish}}

\includegraphics{https://static01.graylady3jvrrxbe.onion/images/2012/05/23/dining/23REST_SPAN/23REST-articleLarge.jpg?quality=75\&auto=webp\&disable=upscale}

By \href{https://www.nytimes3xbfgragh.onion/by/pete-wells}{Pete Wells}

\begin{itemize}
\item
  May 22, 2012
\item
  \begin{itemize}
  \item
  \item
  \item
  \item
  \item
  \end{itemize}
\end{itemize}

ERIC RIPERT has been the executive chef of Le Bernardin for 18 years,
and people sometimes ask him if he gets tired of cooking fish. Clearly
these people have never eaten at Le Bernardin. No other restaurant in
the city makes the simple cooking of fish (and the fish at Le Bernardin
is cooked simply, when it is cooked at all) seem so ripe with
opportunities for excitement.

Some of the thrills are the hushed kind, like the way black garlic,
pomegranate and lime support the crisp skin and white flesh of sautéed
black bass. Others are scene-stealers, as when a white slab of steamed
halibut is slowly surrounded by a crimson pool of beet sauce that, with
crème fraîche stirred in, will turn the delirious pink of summer
borscht.

A few are flat-out luxurious, like a small boulder of caviar nested
inside a heap of sea urchin on a carpet of little gnocchi. I blinked my
eyes a few times at the \$70 supplement on top of the \$125 set price
for four courses at dinner. Then I decided not to worry, because a
chance like this might not come along again. A year from now the sea
urchin and caviar, along with almost everything else on a menu of nearly
40 items, may well have made way for a new crop of thrills.

For a restaurant so determined to stay on top, keeping such a deep
repertory and refreshing it so often would seem to be a risk. It is
also, of course, one source of its enduring success. Le Bernardin's
four-star rating in The New York Times has been confirmed every time the
restaurant has been assessed, from 1986, when it opened, through 2005,
when
\href{http://www.nytimes3xbfgragh.onion/2005/03/16/dining/reviews/16rest.html?pagewanted=all}{Frank
Bruni wrote} its most recent review. Why wait to say it: today I fall in
line, happily, with my predecessors.

Not that I am reviewing the same restaurant, exactly. Under the
relentless guidance of Mr. Ripert and Maguy LeCoze, his partner in the
business, Le Bernardin moves forward without a pause. To rest for a
minute might mean growing old. Change, typically gradual, came in a rush
last summer, when the interior was given a shake-up by the architecture
firm Bentel \& Bentel.

\includegraphics{https://static01.graylady3jvrrxbe.onion/images/2012/05/23/dining/23rest2/23rest2-jumbo.jpg?quality=75\&auto=webp\&disable=upscale}

The old dining room was always compared to a corporate boardroom, but
for some reason its monumental scale and profusion of framed canvases in
an antiquated style made me think of the atrium of a minor art museum.
That's all different now, starting with the art. Now, just one enormous
painting of a brooding sea, ``Deep Water No. 1'' by Ran Ortner, looms
over the space, imparting a sense of motion and immediacy.

Wavy blades of twisted aluminum ripple like reeds along another wall.
Opposite are shimmering, swaying curtains woven from vines and aluminum
fibers. This room may never be sexy, exactly, but now it has a
suggestive invitation in its eye. The downside of the redesign, however,
is the removal of a good number of tables for two. Just as the
restaurant has worked up a little romance, couples have a harder time
getting reservations.

The sleepy little bar was remodeled, too, and is now a sleek
leather-and-steel lounge. Cocktails, when appropriate, come with a
stainless-steel swizzle stick, not (please pound your fist on the bar
along with me) a cheap plastic straw. The lounge is also something like
Branson, Mo., for Le Bernardin's greatest hits, bringing back from
retirement classics like the irresistible smoked-salmon croque monsieur
overflowing with caviar. The full menu is available, though I can't
imagine perching on a tuffet not much higher than a footstool to eat a
\$125 dinner. One French Connection cocktail and a brioche filled with
warm truffled lobster, and I'd be ready to move along.

Comfort, though, may not be the point. Simply having a lounge at all
gives the entire restaurant a pulse that was missing before. The
achievement of Bentel \& Bentel's design is that the interior now walks
in step with Le Bernardin's cuisine. Both are up-to-date, lively,
intimate and playful.

A corporate boardroom is no place for Laurie Jon Moran's elegantly
disassembled desserts, for instance. Mr. Moran's plates are a bit busier
than those of the last pastry chef, Michael Laiskonis, who left late
last year, but the flavors quickly and agreeably reassemble themselves
as you eat.

And a museum is no place for the muscular sancocho sauce served with
lacquered grouper, or more recently with striped bass. Inspired by the
Puerto Rican stew, the sancocho is enveloping and warming, made from
oxtails and chicken, with a low current of heat. The fish is treated
like meat and likes it.

Another attempt at the same treatment perplexed me: Dover sole in a red
wine and cassis sauce the color of grape chewing gum. Out of more than
two dozen I tasted, this was the only dish that didn't come together for
me. And at each meal just one minor detail struck me as not quite
keeping up with the rest: the breads, which are outclassed at a number
of places around town.

\href{https://www.nytimes3xbfgragh.onion/slideshow/2012/05/23/dining/20120523-REST.html}{}

\hypertarget{le-bernardin}{%
\subsection{Le Bernardin}\label{le-bernardin}}

13 Photos

View Slide Show ›

Daniel Krieger for The New York Times

Mr. Ripert has been stocking up on ingredients from Asia for years now,
but somehow he makes the whole enterprise feel new. Sliced geoduck set
down on a fluffy mousseline of smoked edamame and given a bath of wasabi
and lime has obvious roots in Japan, yet doesn't taste Japanese at all.
The original impulse has been transformed.

Like nearly all the savory dishes, this one depends upon the kitchen's
expert sauciers, especially Vincent Robinson, who has been on the job
since 1985. Standing over his stockpots, Mr. Robinson has the control of
Mariano Rivera on the mound. (Get well soon, Mr. Rivera.) When he makes
a sauce of sweet pimentón for red snapper, the level of heat will be
just perceptible; in a red wine and squid ink sauce for sepia, it will
rise a bit higher and stop, right there. When he blends bergamot with
grapefruit and other citrus for lobster, or jalapeño with lime for fluke
sashimi, the nip of acidity will touch down precisely on \emph{this}
spot of your tongue, and nowhere else.

Every time I went to Le Bernardin, somebody ordered a vintage Bordeaux,
and the decanting apparatus would be trundled out. What on earth did
those people order, I wondered, and did they ask Aldo Sohm for advice?
Mr. Sohm, who holds the title chef sommelier, has studded his list with
the required blockbusters, but he also has hidden oddities and
discoveries to drink with raw salmon bathed in cardamom and ginger. He
was joined on the floor at all times by at least two other sommeliers,
silver tastevins swinging from their necks.

They and other servers patrol the room like the Secret Service at a
parade, on the lookout for the slight muscular shift indicating a guest
is about to stand. One, two, three brisk steps, and someone is there to
pull back the chair.

There are slightly more women on the dining room staff these days,
although men are still overrepresented as they are at other restaurants
in this style. Ms. LeCoze and her maître d'hôtel, Ben Chekroun, are
zealous about enforcing correct procedure, and woe to the novice who
clatters dishes or forgets to look customers in the eye the minute they
walk in the door.

There is one other aspect of the service at Le Bernardin that sets the
place apart from some of its peers. In spite of Mr. Ripert's television
appearances, in spite of the restaurant's global acclaim, no one ever
tried to let me know I was lucky to be there.

But I was.

Advertisement

\protect\hyperlink{after-bottom}{Continue reading the main story}

\hypertarget{site-index}{%
\subsection{Site Index}\label{site-index}}

\hypertarget{site-information-navigation}{%
\subsection{Site Information
Navigation}\label{site-information-navigation}}

\begin{itemize}
\tightlist
\item
  \href{https://help.nytimes3xbfgragh.onion/hc/en-us/articles/115014792127-Copyright-notice}{©~2020~The
  New York Times Company}
\end{itemize}

\begin{itemize}
\tightlist
\item
  \href{https://www.nytco.com/}{NYTCo}
\item
  \href{https://help.nytimes3xbfgragh.onion/hc/en-us/articles/115015385887-Contact-Us}{Contact
  Us}
\item
  \href{https://www.nytco.com/careers/}{Work with us}
\item
  \href{https://nytmediakit.com/}{Advertise}
\item
  \href{http://www.tbrandstudio.com/}{T Brand Studio}
\item
  \href{https://www.nytimes3xbfgragh.onion/privacy/cookie-policy\#how-do-i-manage-trackers}{Your
  Ad Choices}
\item
  \href{https://www.nytimes3xbfgragh.onion/privacy}{Privacy}
\item
  \href{https://help.nytimes3xbfgragh.onion/hc/en-us/articles/115014893428-Terms-of-service}{Terms
  of Service}
\item
  \href{https://help.nytimes3xbfgragh.onion/hc/en-us/articles/115014893968-Terms-of-sale}{Terms
  of Sale}
\item
  \href{https://spiderbites.nytimes3xbfgragh.onion}{Site Map}
\item
  \href{https://help.nytimes3xbfgragh.onion/hc/en-us}{Help}
\item
  \href{https://www.nytimes3xbfgragh.onion/subscription?campaignId=37WXW}{Subscriptions}
\end{itemize}
