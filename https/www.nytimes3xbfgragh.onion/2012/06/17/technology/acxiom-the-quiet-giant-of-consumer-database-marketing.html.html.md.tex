Sections

SEARCH

\protect\hyperlink{site-content}{Skip to
content}\protect\hyperlink{site-index}{Skip to site index}

\href{https://www.nytimes3xbfgragh.onion/section/technology}{Technology}

\href{https://myaccount.nytimes3xbfgragh.onion/auth/login?response_type=cookie\&client_id=vi}{}

\href{https://www.nytimes3xbfgragh.onion/section/todayspaper}{Today's
Paper}

\href{/section/technology}{Technology}\textbar{}Mapping, and Sharing,
the Consumer Genome

\url{https://nyti.ms/LcBw0f}

\begin{itemize}
\item
\item
\item
\item
\item
\item
\end{itemize}

Advertisement

\protect\hyperlink{after-top}{Continue reading the main story}

Supported by

\protect\hyperlink{after-sponsor}{Continue reading the main story}

You for Sale

\hypertarget{mapping-and-sharing-the-consumer-genome}{%
\section{Mapping, and Sharing, the Consumer
Genome}\label{mapping-and-sharing-the-consumer-genome}}

\includegraphics{https://static01.graylady3jvrrxbe.onion/images/2012/06/17/business/17-DATA-JP1/17-DATA-JP1-articleLarge.jpg?quality=75\&auto=webp\&disable=upscale}

By \href{https://www.nytimes3xbfgragh.onion/by/natasha-singer}{Natasha
Singer}

\begin{itemize}
\item
  June 16, 2012
\item
  \begin{itemize}
  \item
  \item
  \item
  \item
  \item
  \item
  \end{itemize}
\end{itemize}

IT knows who you are. It knows where you live. It knows what you do.

It peers deeper into American life than the F.B.I. or the I.R.S., or
those prying digital eyes at Facebook and Google. If you are an American
adult, the odds are that it knows things like your age, race, sex,
weight, height, marital status, education level, politics, buying
habits, household health worries, vacation dreams --- and on and on.

Right now in Conway, Ark., north of Little Rock, more than 23,000
computer servers are collecting, collating and analyzing consumer data
for a company that, unlike Silicon Valley's marquee names, rarely makes
headlines. It's called the
\href{http://www.acxiom.com/about-acxiom/}{Acxiom Corporation}, and it's
the quiet giant of a multibillion-dollar industry known as database
marketing.

Few consumers have ever heard of Acxiom. But analysts say it has amassed
the world's largest commercial database on consumers --- and that it
wants to know much, much more. Its servers process more than 50 trillion
data ``transactions'' a year. Company executives have said its database
contains information about 500 million active consumers worldwide, with
about 1,500 data points per person. That includes a majority of adults
in the United States.

Such large-scale data mining and analytics --- based on information
available in public records, consumer surveys and the like --- are
perfectly legal. Acxiom's customers have included big banks like Wells
Fargo and HSBC, investment services like E*Trade, automakers like Toyota
and Ford, department stores like Macy's --- just about any major company
looking for insight into its customers.

For Acxiom, based in Little Rock, the setup is lucrative. It posted
profit of \$77.26 million in its latest fiscal year, on sales of \$1.13
billion.

But such profits carry a cost for consumers. Federal authorities say
current laws may not be equipped to handle the rapid expansion of an
industry whose players often collect and sell sensitive financial and
health information yet are nearly invisible to the public. In essence,
it's as if the ore of our data-driven lives were being mined, refined
and sold to the highest bidder, usually without our knowledge --- by
companies that most people rarely even know exist.

\href{http://www.ftc.gov/commissioners/brill/index.shtml}{Julie Brill, a
member} of the \href{http://www.ftc.gov/}{Federal Trade Commission},
says she would like data brokers in general to tell the public about the
data they collect, how they collect it, whom they share it with and how
it is used. ``If someone is listed as diabetic or pregnant, what is
happening with this information? Where is the information going?'' she
asks. ``We need to figure out what the rules should be as a society.''

Although Acxiom employs a chief privacy officer, Jennifer Barrett
Glasgow, she and other executives declined requests to be interviewed
for this article, said Ines Rodriguez Gutzmer, director of corporate
communications.

In March, ~however, Ms. Barrett Glasgow~ endorsed increased industry
openness. ``It's not an unreasonable request to have more transparency
among data brokers,'' she said in an interview with The New~York Times.~
In marketing materials, Acxiom promotes itself as
\href{https://isapps.acxiom.com/AppFiles/Download18/AcxiomPersonicX_VisionScape-723200794458.pdf}{``a
global thought leader in addressing consumer privacy issues and earning
the public trust.''}

But, in interviews, security experts and consumer advocates paint a
portrait of a company with practices that privilege corporate clients'
interests over those of consumers and contradict the company's stance on
transparency. Acxiom's marketing materials, for example, promote a
special security system for clients and associates to encrypt the data
they send. Yet cybersecurity experts who examined Acxiom's Web site for
The Times found basic security lapses on an online form for consumers
seeking access to their own profiles. (Acxiom says it has fixed the
broken link that caused the problem.)

In a fast-changing digital economy, Acxiom is developing even more
advanced techniques to mine and refine data. It has recruited talent
from Microsoft, Google, Amazon.com and Myspace and is using a powerful,
multiplatform approach to predicting consumer behavior that could raise
its standing among investors and clients.

Of course, digital marketers already customize pitches to users, based
on their past activities. Just think of ``cookies,'' bits of computer
code placed on browsers to keep track of online activity. But Acxiom,
analysts say, is pursuing far more comprehensive techniques in an effort
to influence consumer decisions. It is integrating what it knows about
our offline, online and even mobile selves, creating in-depth behavior
portraits in pixilated detail. Its executives have called this approach
a ``360-degree view'' on consumers.

``There's a lot of players in the digital space trying the same thing,''
says
\href{http://www.piperjaffray.com/1col.aspx?id=7\&analystid=455\&title=Analyst\%20Information\%20for\%20Mark\%20Zgutowicz}{Mark
Zgutowicz, a Piper Jaffray analyst}. ``But Acxiom's advantage is they
have a database of offline information that they have been collecting
for 40 years and can leverage that expertise in the digital world.''

Yet some prominent privacy advocates worry that such techniques could
lead to a new era of consumer profiling.

Jeffrey Chester, executive director of
\href{http://www.democraticmedia.org/}{the Center for Digital
Democracy}, a nonprofit group in Washington, says: ``It is Big Brother
in Arkansas.''

SCOTT HUGHES, an up-and-coming small-business owner and Facebook
denizen, is Acxiom's ideal consumer. Indeed, it created him.

Mr. Hughes is a fictional character who appeared in
\href{http://www.docstoc.com/docs/65163921/Acxiom-Corporation-Data-Demand-Respect}{an
Acxiom investor presentation} in 2010. A frequent shopper, he was
designed to show the power of Acxiom's multichannel approach.

\includegraphics{https://static01.graylady3jvrrxbe.onion/images/2012/06/17/business/17-DATA-JP2/17-DATA-JP2-jumbo.jpg?quality=75\&auto=webp\&disable=upscale}

In the presentation, he logs on to Facebook and sees that his friend
Ella has just become a fan of Bryce Computers, an imaginary electronics
retailer and Acxiom client. Ella's update prompts Mr. Hughes to check
out Bryce's fan page and do some digital window-shopping for a fast
inkjet printer.

Such browsing seems innocuous --- hardly data mining. But it cues an
Acxiom system designed to recognize consumers, remember their actions,
classify their behaviors and influence them with tailored marketing.

When Mr. Hughes follows a link to Bryce's retail site, for example, the
system recognizes him from his Facebook activity and shows him a printer
to match his interest. He registers on the site, but doesn't buy the
printer right away, so the system tracks him online. Lo and behold, the
next morning, while he scans baseball news on ESPN.com, an ad for the
printer pops up again.

That evening, he returns to the Bryce site where, the presentation says,
``he is instantly recognized'' as having registered. It then offers a
sweeter deal: a \$10 rebate and free shipping.

It's not a random offer. Acxiom has its own classification system,
PersonicX, which assigns consumers to one of 70 detailed socioeconomic
clusters and markets to them accordingly. In this situation, it pegs Mr.
Hughes as a ``savvy single'' --- meaning he's in a cluster of mobile,
upper-middle-class people who do their banking online, attend pro sports
events, are sensitive to prices --- and respond to free-shipping offers.

Correctly typecast, Mr. Hughes buys the printer.

But the multichannel system of Acxiom and its online partners is just
revving up. Later, it sends him coupons for ink and paper, to be
redeemed via his cellphone, and a personalized snail-mail postcard
suggesting that he donate his old printer to a nearby school.

Analysts say companies design these sophisticated ecosystems to prompt
consumers to volunteer enough personal data --- like their names, e-mail
addresses and mobile numbers --- so that marketers can offer them
customized appeals any time, anywhere.

Still, there is a fine line between customization and stalking. While
many people welcome the convenience of personalized offers, others may
see the surveillance engines behind them as intrusive or even
manipulative.

``If you look at it in cold terms, it seems like they are really out to
trick the customer,'' says
\href{http://www.forrester.com/Dave-Frankland}{Dave Frankland, the
research director} for customer intelligence at Forrester Research.
``But they are actually in the business of helping marketers make sure
that the right people are getting offers they are interested in and
therefore establish a relationship with the company.''

DECADES before the Internet as we know it, a businessman named Charles
Ward planted the seeds of Acxiom. It was 1969, and Mr. Ward started a
data processing company in Conway called Demographics Inc., in part to
help the Democratic Party reach voters. In a time when Madison Avenue
was deploying one-size-fits-all national ad campaigns, Demographics and
its lone computer used public phone books to compile lists for direct
mailing of campaign material.

Today, Acxiom maintains its own database on about 190 million
individuals and 126 million households in the United States. Separately,
it manages customer databases for or works with 47 of the Fortune 100
companies. It also worked with the government after the September 2001
terrorist attacks, providing information about 11 of the 19 hijackers.

To beef up its digital services, Acxiom recently mounted an aggressive
hiring campaign. Last July, it named
\href{http://www.acxiom.com/about-acxiom/corporate-governance/board-of-directors/scott-e--howe/}{Scott
E. Howe}, a former corporate vice president for Microsoft's advertising
business group, as C.E.O. Last month, it hired
\href{http://www.acxiom.com/about-acxiom/corporate-governance/company-leadership/phil-mui,-ph-d-/}{Phil
Mui}, formerly~group product manager for Google Analytics, as its chief
product and engineering officer.

In interviews, Mr. Howe has laid out a vision of Acxiom as a
new-millennium ``data refinery'' rather than a data miner. That
description posits Acxiom as a nimble provider of customer analytics
services, able to compete with Facebook and Google, rather than as a
stealth engine of consumer espionage.

Still, the more that information brokers mine powerful consumer data,
the more they become attractive targets for hackers --- and draw
scrutiny from consumer advocates.

This year, Advertising Age ranked
\href{http://www.epsilon.com/}{Epsilon, another database marketing
firm}, as the biggest advertising agency in the United States, with
Acxiom second. Most people know Epsilon, if they know it at all, because
it experienced a major security breach last year,
\href{http://www.nytimes3xbfgragh.onion/2011/04/05/business/05hack.html?_r=1}{exposing
the e-mail addresses of millions of customers} of Citibank, JPMorgan
Chase, Target, Walgreens and others. In 2003, Acxiom had its own
security breaches.

But privacy advocates say they are more troubled by data brokers'
ranking systems, which classify some people as high-value prospects, to
be offered marketing deals and discounts regularly, while dismissing
others as low-value --- known in industry slang as ``waste.''

Exclusion from a vacation offer may not matter much, says Pam Dixon, the
executive director of \href{http://www.worldprivacyforum.org/}{the World
Privacy Forum}, a nonprofit group in San Diego, but if marketing
algorithms judge certain people as not worthy of receiving promotions
for higher education or health services, they could have a serious
impact.

``Over time, that can really turn into a mountain of pathways not
offered, not seen and not known about,'' Ms. Dixon says.

Until now, database marketers operated largely out of the public eye.
Unlike consumer reporting agencies that sell sensitive financial
information about people for credit or employment purposes, database
marketers aren't required by law to show consumers their own reports and
allow them to correct errors. That may be about to change. This year,
the F.T.C.
\href{http://www.nytimes3xbfgragh.onion/2012/03/27/business/ftc-seeks-privacy-legislation.html?pagewanted=all}{published
a report} calling for greater transparency among data brokers and asking
Congress to give consumers the right to access information these firms
hold about them.

Image

Jennifer Barrett Glasgow is the company's chief privacy
officer.Credit...Ken Cedeno/Bloomberg News

ACXIOM'S Consumer Data Products Catalog offers hundreds of details ---
called ``elements'' --- that corporate clients can buy about individuals
or households, to augment their own marketing databases. Companies can
buy data to pinpoint households that are concerned, say, about
allergies, diabetes or ``senior needs.'' Also for sale is information on
sizes of home loans and household incomes.

Clients generally buy this data because they want to hold on to their
best customers or find new ones --- or both.

A bank that wants to sell its best customers additional services, for
example, might buy details about those customers' social media, Web and
mobile habits to identify more efficient ways to market to them. Or,
says Mr. Frankland at Forrester, a sporting goods chain whose best
customers are 25- to 34-year-old men living near mountains or beaches
could buy a list of a million other people with the same
characteristics. The retailer could hire Acxiom, he says, to manage a
campaign aimed at that new group, testing how factors like consumers'
locations or sports preferences affect responses.

But the catalog also offers delicate information that has set off alarm
bells among some privacy advocates, who worry about the potential for
misuse by third parties that could take aim at vulnerable groups. Such
information includes consumers' interests --- derived, the catalog says,
``from actual purchases and self-reported surveys'' --- like ``Christian
families,'' ``Dieting/Weight Loss,'' ``Gaming-Casino,'' ``Money
Seekers'' and ``Smoking/Tobacco.'' Acxiom also sells data about an
individual's race, ethnicity and country of origin. ``Our Race model,''
the catalog says, ``provides information on the major racial category:
Caucasians, Hispanics, African-Americans, or Asians.'' Competing
companies sell similar data.

Acxiom's data about race or ethnicity is ``used for engaging those
communities for marketing purposes,'' said Ms. Barrett Glasgow, the
privacy officer, in an e-mail response to questions.

There may be a legitimate commercial need for some businesses, like
ethnic restaurants, to know the race or ethnicity of consumers, says
\href{http://law.fordham.edu/faculty/1134.htm}{Joel R. Reidenberg}, a
privacy expert and a professor at the Fordham Law School.

``At the same time, this is ethnic profiling,'' he says. ``The people on
this list, they are being sold based on their ethnic stereotypes. There
is a very strong citizen's right to have a veto over the commodification
of their profile.''

He says the sale of such data is troubling because race coding may be
incorrect. And even if a data broker has correct information, a person
may not want to be marketed to based on race.

``DO you really know your customers?'' Acxiom asks in marketing
materials for its shopper recognition system, a program that uses ZIP
codes to help retailers confirm consumers' identities --- without asking
their permission.

``Simply asking for name and address information poses many challenges:
transcription errors, increased checkout time and, worse yet, losing
customers who feel that you're invading their privacy,'' Acxiom's fact
sheet explains. In its system, a store clerk need only ``capture the
shopper's name from a check or third-party credit card at the point of
sale and then ask for the shopper's ZIP code or telephone number.'' With
that data Acxiom can identify shoppers within a 10 percent margin of
error, it says, enabling stores to reward their best customers with
special offers. Other companies offer similar services.

``This is a direct way of circumventing people's concerns about
privacy,'' says Mr. Chester of the Center for Digital Democracy.

Ms. Barrett Glasgow of Acxiom says that its program is a ``standard
practice'' among retailers, but that the company encourages its clients
to report consumers who wish to opt out.

Acxiom has positioned itself as an industry leader in data privacy, but
some of its practices seem to undermine that image. It created the
position of chief privacy officer in 1991, well ahead of its rivals. It
even offers an online
\href{http://www.acxiom.com/about-acxiom/privacy/us-consumer-choices/}{request
form}, promoted as an easy way for consumers to access information
Acxiom collects about them.

But the process turned out to be not so user-friendly for a reporter for
The Times.

In early May, the reporter decided to request her record from Acxiom, as
any consumer might. Before submitting a Social Security number and other
personal information, however, she asked for advice from a cybersecurity
expert at The Times. The expert examined Acxiom's Web site and
immediately noticed that the online form did not employ a standard
encryption protocol --- called https --- used by sites like Amazon and
American Express. When the expert tested the form, using software that
captures data sent over the Web, he could clearly see that the sample
Social Security number he had submitted had not been encrypted. At that
point, the reporter was advised not to request her file, given the risk
that the process might expose her personal information.

Later in May, \href{http://www.ashkansoltani.org/}{Ashkan Soltani}, an
independent security researcher and former technologist in identity
protection at the F.T.C., also examined Acxiom's site and came to the
same conclusion. ``Parts of the site for corporate clients are
encrypted,'' he says. ``But for consumers, who this information is about
and who stand the most to lose from data collection, they don't provide
security.''

Ms. Barrett Glasgow says that the form has always been encrypted with
https but that on May 11, its security monitoring system detected a
``broken redirect link'' that allowed unencrypted access. Since then,
she says, Acxiom has fixed the link and determined that no unauthorized
person had gained access to information sent using the form.

On May 25, the reporter submitted an online request to Acxiom for her
file, along with a personal check, sent by Express Mail, for the \$5
processing fee. Three weeks later, no response had arrived.

Regulators at the F.T.C. declined to comment on the practices of
individual companies. But
\href{http://www.ftc.gov/commissioners/leibowitz/index.shtml}{Jon
Leibowitz, the commission chairman,} said consumers should have the
right to see and correct personal details about them collected and sold
by data aggregators.

After all, he said, ``they are the unseen cyberazzi who collect
information on all of us.''

Advertisement

\protect\hyperlink{after-bottom}{Continue reading the main story}

\hypertarget{site-index}{%
\subsection{Site Index}\label{site-index}}

\hypertarget{site-information-navigation}{%
\subsection{Site Information
Navigation}\label{site-information-navigation}}

\begin{itemize}
\tightlist
\item
  \href{https://help.nytimes3xbfgragh.onion/hc/en-us/articles/115014792127-Copyright-notice}{©~2020~The
  New York Times Company}
\end{itemize}

\begin{itemize}
\tightlist
\item
  \href{https://www.nytco.com/}{NYTCo}
\item
  \href{https://help.nytimes3xbfgragh.onion/hc/en-us/articles/115015385887-Contact-Us}{Contact
  Us}
\item
  \href{https://www.nytco.com/careers/}{Work with us}
\item
  \href{https://nytmediakit.com/}{Advertise}
\item
  \href{http://www.tbrandstudio.com/}{T Brand Studio}
\item
  \href{https://www.nytimes3xbfgragh.onion/privacy/cookie-policy\#how-do-i-manage-trackers}{Your
  Ad Choices}
\item
  \href{https://www.nytimes3xbfgragh.onion/privacy}{Privacy}
\item
  \href{https://help.nytimes3xbfgragh.onion/hc/en-us/articles/115014893428-Terms-of-service}{Terms
  of Service}
\item
  \href{https://help.nytimes3xbfgragh.onion/hc/en-us/articles/115014893968-Terms-of-sale}{Terms
  of Sale}
\item
  \href{https://spiderbites.nytimes3xbfgragh.onion}{Site Map}
\item
  \href{https://help.nytimes3xbfgragh.onion/hc/en-us}{Help}
\item
  \href{https://www.nytimes3xbfgragh.onion/subscription?campaignId=37WXW}{Subscriptions}
\end{itemize}
