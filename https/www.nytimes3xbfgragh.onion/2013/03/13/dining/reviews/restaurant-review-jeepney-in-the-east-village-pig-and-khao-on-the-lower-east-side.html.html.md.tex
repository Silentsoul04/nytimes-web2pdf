Sections

SEARCH

\protect\hyperlink{site-content}{Skip to
content}\protect\hyperlink{site-index}{Skip to site index}

\href{https://www.nytimes3xbfgragh.onion/section/food}{Food}

\href{https://myaccount.nytimes3xbfgragh.onion/auth/login?response_type=cookie\&client_id=vi}{}

\href{https://www.nytimes3xbfgragh.onion/section/todayspaper}{Today's
Paper}

\href{/section/food}{Food}\textbar{}Two Roads to the Philippines

\url{https://nyti.ms/Zjzlwf}

\begin{itemize}
\item
\item
\item
\item
\item
\end{itemize}

Advertisement

\protect\hyperlink{after-top}{Continue reading the main story}

Supported by

\protect\hyperlink{after-sponsor}{Continue reading the main story}

\hypertarget{two-roads-to-the-philippines}{%
\section{Two Roads to the
Philippines}\label{two-roads-to-the-philippines}}

\includegraphics{https://static01.graylady3jvrrxbe.onion/images/2013/03/13/dining/13REST_SPAN/13REST_SPAN-articleLarge.jpg?quality=75\&auto=webp\&disable=upscale}

\begin{itemize}
\tightlist
\item
  Jeepney\\
  **NYT Critic's Pick ★★ Philippine \$\$\$ 201 First Avenue 212-533-4121
\end{itemize}

\begin{itemize}
\tightlist
\item
  Pig and Khao\\
  **NYT Critic's Pick ★★ Asian \$\$ 68 Clinton Street 212-920-4485
\end{itemize}

\href{http://www.opentable.com/single.aspx?ref=4201\&rid=103108}{Reserve
a Table}

When you make a reservation at an independently reviewed restaurant
through our site, we earn an affiliate commission.

By \href{https://www.nytimes3xbfgragh.onion/by/pete-wells}{Pete Wells}

\begin{itemize}
\item
  March 12, 2013
\item
  \begin{itemize}
  \item
  \item
  \item
  \item
  \item
  \end{itemize}
\end{itemize}

\href{https://archive.nytimes3xbfgragh.onion/www.nytimes3xbfgragh.onion/2013/03/13/dining/reviews/restaurant-review-jeepney-in-the-east-village-pig-and-khao-on-the-lower-east-side.html}{See
how this article appeared when it was originally published on
NYTimes.com.}

The cry goes up around the restaurant.

``Ba-luuuut!''

The word volleys from Jeepney's kitchen to its dining room as a server
carries the hard-boiled duck egg to a man at the bar.

\href{http://www.nytimes3xbfgragh.onion/video/2013/03/11/multimedia/100000002110714/filipino-egg.html\#media/vid1}{``Ba-luuuuuuuuuuut!''}

Grabbing the customer's hand, she whacks the flesh between his thumb and
forefinger with a tablespoon to demonstrate the force needed to crack
the balut's shell. Then she prepares him for what he'll find inside.

Right on top, she says, is ``a really smoky chicken broth.'' Under that
is the white, hard-boiled, and the yolk. ``And down at the bottom,
there's E.T.''

The extraterrestrial is a two- or three-week-old duckling that will
never hatch, a ball of spindly legs and tucked wings and fine threads of
feathers. The unabashed embrace of a delicacy with major freakout
potential is typical of the deep-end approach of
\href{http://www.jeepneynyc.com/}{Jeepney}, a self-described ``Filipino
gastro pub'' in the East Village.

Another approach is on display about a mile south at
\href{http://pigandkhao.com/}{Pig and Khao}, which has a strong Filipino
imprint, too. The two places have many things in common. Open since last
fall, they are small, casual, fun and often loud --- Jeepney with
American and Filipino party rock, Pig and Khao with slow-rolling
Southern hip-hop. Neither stocks hard liquor, but each still manages to
shake up very entertaining cocktails. Recently, I'm glad to report, both
dropped their no-reservations policies.

In their styles of presenting Asian cuisine, though, they go their own
ways. Before opening Pig and Khao, Leah Cohen, the chef, spent a year
eating and cooking in countries like Vietnam, Thailand and the
Philippines, where her mother was raised. Eating at her restaurant, I
felt as if I were poring over an album of carefully edited postcards
from her travels. Dinner at Jeepney, on the other hand, felt more like
parachuting into Manila myself. I didn't know all the vocabulary and
didn't always know what I was putting in my mouth, but I knew I had left
home.

I've grown fond of both places, but I would take different sets of
friends to each.

For Pig and Khao, I'd round up the ones who love Asian flavors, don't
have significant hearing loss yet, think it's fun to get endless refills
of beer from a keg in the back garden and won't be heartbroken to learn
that fertilized duck embryos are not an option.

They'd be pork lovers who would go for the sugary chunks of Chinese
sausage in a bowl of mussels first before dipping the so-so shellfish in
a redeemingly aromatic, basil-scented broth of dashi, butter and yuzu.

They would be delighted by a salad of watermelon cubes with strips of
grilled pork jowl and golden puffs of fried pork rind. Like nearly every
dish on the menu, it has enough crunch, salt and acidity to satisfy the
most curmudgeonly judge on ``Top Chef,'' where Ms. Cohen came to
national attention while she was cooking at
\href{http://events.nytimes3xbfgragh.onion/2007/10/17/dining/reviews/17rest.html?pagewanted=all}{Centro
Vinoteca} before taking over the kitchen there.

\href{https://www.nytimes3xbfgragh.onion/slideshow/2013/03/13/dining/20130313-REST.html}{}

\hypertarget{jeepney-and-pig-and-khao}{%
\subsection{Jeepney, and Pig and Khao}\label{jeepney-and-pig-and-khao}}

14 Photos

View Slide Show ›

\includegraphics{https://static01.graylady3jvrrxbe.onion/images/2013/03/13/dining/20130313-REST-slide-ZQW5/20130313-REST-slide-ZQW5-articleLarge.jpg?quality=75\&auto=webp\&disable=upscale}

Brian Harkin for The New York Times

Then they'd see nothing wrong with moving on to another part of the
pig's face with sizzling sisig, a modern Filipino classic. They'd devour
the chopped bits of pork --- some crisp, some gelatinous, some meaty ---
seasoned with soy and slicked with the yolk of an egg cracked over the
platter at the last minute.

They wouldn't be fanatics about the authenticity of a menu that leans
repeatedly on ``crispy'' over less familiar words. When ``crispy red
curry rice salad'' turned out to be a less-spicy Thai pork larb tricked
out with fragments of curried rice cooked into a crust, they'd nod their
heads over Ms. Cohen's talent for bright, clear flavors instead of
quibbling. And when the adobo in ``crispy quail adobo'' was not a
stewing liquid, as it might be in the Philippines, but a rich soy and
garlic sauce tossed with deep-fried quail, they wouldn't care.

I didn't care, either.

The friends I'd take to Jeepney would be the explorers, the ones who see
every meal as a chance to learn something.

They wouldn't fume if it took three tries to bolt the restroom door.
They wouldn't clutch their pearls at the sight of nipples on the back
room's photomurals, one set belonging to the first Filipina to appear in
Playboy, the other to the first in German Playboy. (When children show
up, the staff applies discreet bands of tape.)

The menu nonchalantly tosses out ingredients like sawsawan and bagoong
as if they were peas and carrots, and these friends would listen as the
high-spirited servers explained it all.

They'd pay \$7 more for an extra marrow bone to supplement the
impressive one that rides on top of a short-rib and vegetable soup
called bulalo. They would mash the marrow with a fork into a lump of
jasmine rice, then drink the deeply restorative broth from their bowls
and ladle some into mine.

If they came on a Thursday, when Jeepney puts away the utensils and
plates and serves dinner on banana leaves, they'd wash their hands and
get right down to business.

They'd mold jasmine rice around bits of longanisa sausage; strips of
tocino, pork that the chef, Miguel Trinidad, cures in 7Up; and cubes of
pork stewed in a fascinating chocolate-colored sauce of beef blood, bay
leaves and vinegar. Facing fried crabs sautéed with garlic, they would
crack open the shells and excavate the meat with their fingers.

Would they eat balut? Maybe.

Would I? I did. The liquid on top tasted, yes, like chicken broth; the
yolk was chalky; parts of the white were confusingly hard. As for the
little embryo, it gave way to the spoon as easily as custard and tasted
something like duck liver mixed with duck breast. If you didn't grow up
eating balut, it probably helps to stop thinking about the feathers.

Jeepney's batting average could be higher. A dish called the Defeated
Chicken, with nondescript roasted chicken in adobo sauce, defeated me,
too, and the slow-roasted pork shoulder in one called Bicol Express had
been cut across the grain into short, dry strands.

But the people I'd take there know that discoveries require both trial
and error. If they prefer the reliable rewards of skillful cooking
edited for New Yorkers, I've got a restaurant for that, too.

Advertisement

\protect\hyperlink{after-bottom}{Continue reading the main story}

\hypertarget{site-index}{%
\subsection{Site Index}\label{site-index}}

\hypertarget{site-information-navigation}{%
\subsection{Site Information
Navigation}\label{site-information-navigation}}

\begin{itemize}
\tightlist
\item
  \href{https://help.nytimes3xbfgragh.onion/hc/en-us/articles/115014792127-Copyright-notice}{©~2020~The
  New York Times Company}
\end{itemize}

\begin{itemize}
\tightlist
\item
  \href{https://www.nytco.com/}{NYTCo}
\item
  \href{https://help.nytimes3xbfgragh.onion/hc/en-us/articles/115015385887-Contact-Us}{Contact
  Us}
\item
  \href{https://www.nytco.com/careers/}{Work with us}
\item
  \href{https://nytmediakit.com/}{Advertise}
\item
  \href{http://www.tbrandstudio.com/}{T Brand Studio}
\item
  \href{https://www.nytimes3xbfgragh.onion/privacy/cookie-policy\#how-do-i-manage-trackers}{Your
  Ad Choices}
\item
  \href{https://www.nytimes3xbfgragh.onion/privacy}{Privacy}
\item
  \href{https://help.nytimes3xbfgragh.onion/hc/en-us/articles/115014893428-Terms-of-service}{Terms
  of Service}
\item
  \href{https://help.nytimes3xbfgragh.onion/hc/en-us/articles/115014893968-Terms-of-sale}{Terms
  of Sale}
\item
  \href{https://spiderbites.nytimes3xbfgragh.onion}{Site Map}
\item
  \href{https://help.nytimes3xbfgragh.onion/hc/en-us}{Help}
\item
  \href{https://www.nytimes3xbfgragh.onion/subscription?campaignId=37WXW}{Subscriptions}
\end{itemize}
