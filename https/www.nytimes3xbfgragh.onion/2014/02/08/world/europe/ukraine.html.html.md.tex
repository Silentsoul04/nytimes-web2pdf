Sections

SEARCH

\protect\hyperlink{site-content}{Skip to
content}\protect\hyperlink{site-index}{Skip to site index}

\href{https://www.nytimes3xbfgragh.onion/section/world/europe}{Europe}

\href{https://myaccount.nytimes3xbfgragh.onion/auth/login?response_type=cookie\&client_id=vi}{}

\href{https://www.nytimes3xbfgragh.onion/section/todayspaper}{Today's
Paper}

\href{/section/world/europe}{Europe}\textbar{}Leaked Recordings Lay Bare
E.U. and U.S. Divisions in Goals for Ukraine

\url{https://nyti.ms/1iBDfdS}

\begin{itemize}
\item
\item
\item
\item
\item
\end{itemize}

Advertisement

\protect\hyperlink{after-top}{Continue reading the main story}

Supported by

\protect\hyperlink{after-sponsor}{Continue reading the main story}

\hypertarget{leaked-recordings-lay-bare-eu-and-us-divisions-in-goals-for-ukraine}{%
\section{Leaked Recordings Lay Bare E.U. and U.S. Divisions in Goals for
Ukraine}\label{leaked-recordings-lay-bare-eu-and-us-divisions-in-goals-for-ukraine}}

\includegraphics{https://static01.graylady3jvrrxbe.onion/images/2014/02/08/world/08ukraine4/08ukraine4-articleLarge.jpg?quality=75\&auto=webp\&disable=upscale}

By \href{http://www.nytimes3xbfgragh.onion/by/alison-smale}{Alison
Smale}

\begin{itemize}
\item
  Feb. 7, 2014
\item
  \begin{itemize}
  \item
  \item
  \item
  \item
  \item
  \end{itemize}
\end{itemize}

BERLIN --- ``Really Pretty Stupid'' was the headline chosen by the
august Frankfurter Allgemeine Zeitung on Friday to describe an editorial
on the latest eruption between the United States and Europe, this time
over who should take the lead in trying to calm the crisis in Ukraine,
and how to do it.

The headline spoke to the tensions that flared this week over the
release of a recording in which a top American diplomat disparaged the
European Union's efforts in Ukraine. On Friday, a second recording
surfaced in which European diplomats complained about the Americans.

But it was also a reflection of the disarray that has marked much of the
West's dealings with Ukraine since late November, when President Viktor
F. Yanukovych spurned a pact with the European Union. He then turned to
Russia for a \$15 billion aid package that the Kremlin has since
suspended because of continuing antigovernment protests in Kiev, the
capital.

Ever since Ukraine became independent as the Soviet Union crumbled in
1991, the United States and Europe have had different aims for the
country, a large, troubled nation of 45 million whose very name means
``on the edge.''

With strategic considerations uppermost in American diplomacy, the
United States helped, for instance, to rid Ukraine of old Soviet nuclear
weapons. Europe, meanwhile, saw opportunities for trade.

As the European Union expanded eastward with the inclusion of Poland and
Romania, the perception grew that neighboring Ukraine needed formal ties
to regulate commerce and legal systems to facilitate the growing
cross-border transactions. In 2012, Poland and Ukraine were even joint
hosts of the continent's premier sports event, the European soccer
championship.

Russia, which has centuries of shared history with Ukraine and under
Vladimir V. Putin has grown ever more painfully conscious of its loss of
Soviet empire, looked on with mounting suspicion, and now seems to be
intent on exploiting Western disarray.

The release of the recordings has further roiled the waters. In the
\href{https://www.youtube.com/watch?v=MSxaa-67yGM\#t=89}{first one},
posted anonymously on YouTube, Victoria Nuland, the American assistant
secretary of state for European affairs, profanely dismissed European
efforts in Ukraine as weak and inadequate to the challenge posed by the
Kremlin.

On Friday, a second recording was posted that featured a senior German
diplomat, Helga Schmid, complaining in her native tongue to the European
Union envoy in Kiev about ``unfair'' American criticism of Europe's
diplomacy.

\includegraphics{https://static01.graylady3jvrrxbe.onion/images/2014/02/08/world/08ukraine1/08ukraine1-articleLarge.jpg?quality=75\&auto=webp\&disable=upscale}

``We are not in a race to be the strongest,'' retorted the envoy, Jan
Tombinski, a Pole. ``We have good instruments'' for dealing with the
crisis.

Yes, replied Ms. Schmid, but journalists were telling European officials
that the Americans were running around saying the Europeans were weak.
So she advised Mr. Tombinski to have a word with the United States
ambassador to Ukraine, Geoffrey Pyatt, the man whom Ms. Nuland was
talking to in her recorded conversation.

While the Obama administration accused the Russians of making mischief
by recording and then posting the Nuland conversation, neither the
European Union nor Germany blamed the Kremlin for the second recording.

Illustrating how testy relations with Washington have become, Chancellor
Angela Merkel of Germany, earlier the target of American monitoring of
her cellphone, issued an unusually sharp statement saying that Ms.
Nuland's remarks were ``completely unacceptable.''

Germany, as befits its status as Europe's largest economy and a country
with centuries of dealings with lands to its East, has been heavily
involved in the crisis over Ukraine. In a speech to the German
Parliament on Nov. 18, Ms. Merkel, herself raised in Communist East
Germany, emphasized that the Cold War should be over for everyone,
including countries once allied with Russia but now independent. She
made a forceful case for Ukraine to sign the European pact.

Julianne Smith, a former national security aide to Vice President Joseph
R. Biden Jr. who is now at the Center for a New American Security, said
there was a structural tension between the European Union and the United
States because the Americans can speak with one voice and grow impatient
waiting for decisions from a union with many voices.

``They all have different sovereign issues, different threat
perceptions, different priorities,'' she said. ``As a result, there has
always been this longstanding deep frustration on the part of the United
States with the inability to get quick answers, quick responses and
broker some sort of U.S.-E.U. agreement on whatever the issue of the day
might be.''

The back-and-forth this week illustrates how many interests are a part
of the mix in Ukraine --- a mix that Western diplomats seem unable to
keep free of their own differences.

In the editorial with the headline ``Really Pretty Stupid,''
Klaus-Dieter Frankenberger, the newspaper's foreign editor, noted how
the latest issue had been stoked by months of ``bad blood'' with
Washington. ``You can certainly criticize some parts of European policy
toward Ukraine, but it is not as if American diplomacy has found the
font of all wisdom. In fact, they can't think of anything more than a
few mini-sanctions against the regime in Kiev.''

Meanwhile, Mr. Frankenberger said, Mr. Putin ``should certainly be
laughing himself stupid.''

``If a top American diplomat could not care less about the Europeans,''
he added, ``then he will certainly bear more easily their absence from
the opening of the Olympic Games in Sochi. And he will see in Ms.
Nuland's remark, which Moscow presumably disseminated, a confirmation of
the bad opinion he already has of Europeans.''

The moral of the tale? ``No disparaging remarks about partners on the
phone.''

Advertisement

\protect\hyperlink{after-bottom}{Continue reading the main story}

\hypertarget{site-index}{%
\subsection{Site Index}\label{site-index}}

\hypertarget{site-information-navigation}{%
\subsection{Site Information
Navigation}\label{site-information-navigation}}

\begin{itemize}
\tightlist
\item
  \href{https://help.nytimes3xbfgragh.onion/hc/en-us/articles/115014792127-Copyright-notice}{©~2020~The
  New York Times Company}
\end{itemize}

\begin{itemize}
\tightlist
\item
  \href{https://www.nytco.com/}{NYTCo}
\item
  \href{https://help.nytimes3xbfgragh.onion/hc/en-us/articles/115015385887-Contact-Us}{Contact
  Us}
\item
  \href{https://www.nytco.com/careers/}{Work with us}
\item
  \href{https://nytmediakit.com/}{Advertise}
\item
  \href{http://www.tbrandstudio.com/}{T Brand Studio}
\item
  \href{https://www.nytimes3xbfgragh.onion/privacy/cookie-policy\#how-do-i-manage-trackers}{Your
  Ad Choices}
\item
  \href{https://www.nytimes3xbfgragh.onion/privacy}{Privacy}
\item
  \href{https://help.nytimes3xbfgragh.onion/hc/en-us/articles/115014893428-Terms-of-service}{Terms
  of Service}
\item
  \href{https://help.nytimes3xbfgragh.onion/hc/en-us/articles/115014893968-Terms-of-sale}{Terms
  of Sale}
\item
  \href{https://spiderbites.nytimes3xbfgragh.onion}{Site Map}
\item
  \href{https://help.nytimes3xbfgragh.onion/hc/en-us}{Help}
\item
  \href{https://www.nytimes3xbfgragh.onion/subscription?campaignId=37WXW}{Subscriptions}
\end{itemize}
