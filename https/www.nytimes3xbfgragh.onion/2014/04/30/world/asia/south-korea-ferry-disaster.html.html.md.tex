Sections

SEARCH

\protect\hyperlink{site-content}{Skip to
content}\protect\hyperlink{site-index}{Skip to site index}

\href{https://www.nytimes3xbfgragh.onion/section/world/asia}{Asia
Pacific}

\href{https://myaccount.nytimes3xbfgragh.onion/auth/login?response_type=cookie\&client_id=vi}{}

\href{https://www.nytimes3xbfgragh.onion/section/todayspaper}{Today's
Paper}

\href{/section/world/asia}{Asia Pacific}\textbar{}Vowing Changes, South
Korean Leader Apologizes for Ferry Disaster

\url{https://nyti.ms/1m5KGzG}

\begin{itemize}
\item
\item
\item
\item
\item
\end{itemize}

Advertisement

\protect\hyperlink{after-top}{Continue reading the main story}

Supported by

\protect\hyperlink{after-sponsor}{Continue reading the main story}

\hypertarget{vowing-changes-south-korean-leader-apologizes-for-ferry-disaster}{%
\section{Vowing Changes, South Korean Leader Apologizes for Ferry
Disaster}\label{vowing-changes-south-korean-leader-apologizes-for-ferry-disaster}}

\includegraphics{https://static01.graylady3jvrrxbe.onion/images/2014/04/29/multimedia/skorea-president-apology/skorea-president-apology-videoSixteenByNine1050.jpg}

By \href{http://www.nytimes3xbfgragh.onion/by/choe-sang-hun}{Choe
Sang-Hun}

\begin{itemize}
\item
  April 29, 2014
\item
  \begin{itemize}
  \item
  \item
  \item
  \item
  \item
  \end{itemize}
\end{itemize}

SEOUL, South Korea --- Under mounting public pressure, President Park
Geun-hye apologized on Tuesday for failing to prevent a
\href{http://www.nytimes3xbfgragh.onion/2014/04/17/world/asia/south-korean-ferry-accident.html}{ferry
disaster that left 302 people}, the vast majority of them high school
students, dead or missing, and promised broad reforms to make her
country a safer place.

``My heart aches thinking how I can best apologize and ease the grief
and pain,'' she said during a cabinet meeting, admitting to her
government's fumbling in the early stages of rescue operations. ``I am
sorry that so many precious lives were lost.''

It was a humbling moment for Ms. Park, the daughter of the military
strongman Park Chung-hee. Ever since she took office in February 2013,
Ms. Park has built a reputation for steely leadership in the face of
military threats from North Korea. But the political opposition has
often accused her of being an imperious leader blind to criticism.

Although the prime minister, Chung Hong-won, resigned on Sunday,
apologizing for the disaster, few analysts have yet suggested that it
has threatened Ms. Park's ability to govern. On Tuesday, Ms. Park said
she would create a central government agency to ensure better
coordination in rescue efforts in major disasters.

She also vowed to eliminate what she and the local news media called ``a
government mafia'' --- in which retirees from ministries and regulatory
agencies find jobs in industry lobbies. The collusive links between the
regulatory agencies and the industries, forged through these retirees,
have long been blamed for widespread corruption and lax safety
enforcement of the kind that caused many nuclear power plants to be shut
down last year.

Ms. Park got a stinging taste of anger from the grieving families when
she visited Ansan, south of Seoul, where the students' school is
situated. The families turned away the memorial wreaths donated by Ms.
Park and other senior government officials. According to local news
reports, some family members shouted at her, demanding that she take
responsibility for the disaster.

For the most part, Ms. Park is popular, thanks partly to older and
conservative South Koreans who support her tough stance in the standoff
with North Korea over its nuclear weapons threat and with Japan over
historical issues. But her approval rating dropped by nearly seven
percentage points, to about 58 percent, in the week after the sinking of
the ferry, as ``the people's discontent deepened over the government's
ability to manage a crisis,'' according to a Seoul-based polling
company, Realmeter, which surveyed 2,520 voters by landline and
cellphone from April 21 to 25. (The survey's margin of sampling error
was plus or minus two percentage points.)

According to video released by the coast guard, the first government
rescue boats to arrive at the 6,825-ton ferry, which was sinking off
southwestern South Korea on April 16, helped the captain and other crew
members off the ship, while hundreds of passengers remained trapped
inside. Fifteen crew members, including the captain, Lee Jun-seok, have
been arrested on criminal charges of deserting their passengers during
an emergency.

\includegraphics{https://static01.graylady3jvrrxbe.onion/images/2014/04/30/world/asia/30korea/30korea-articleLarge.jpg?quality=75\&auto=webp\&disable=upscale}

Early investigations have left little doubt that the disaster came about
from a combination of the poor work ethics of the crew, loopholes in
safety standards, lax regulatory enforcement and compromised industry
watchdogs, whose top ranks are filled with retirees from government
ministries.

On Tuesday, it was revealed that a former employee of the Chonghaejin
Marine Company, the operator of the ferry, had alerted the government to
corruption and lax safety measures. Writing on a government-run
whistle-blowing website in January, the former employee reported
violations by Chonghaejin Marine, including overloading ships, covering
up accidents and illegal treatment of contract workers, including
failing to pay them, the Hankyoreh newspaper reported.

The government said it had helped resolve the petitioner's grievances
about wages. But it remained unclear whether it investigated his other
allegations.

Investigators have expanded their inquiry into the government's
emergency response system. They raided the offices of vessel traffic
controllers amid allegations that their fumbling contributed to the high
death toll.

A coast guard emergency dispatcher has been accused of delaying early
rescue efforts by asking a student who called for help on his cellphone
to provide coordinates. Vessel traffic controllers are accused of
failing to monitor the ferry's whereabouts even after it was tilting and
drifting in a notoriously dangerous waterway.

When the ship set sail from Incheon, west of Seoul, on April 15, it was
top-heavy with cabins recently added to its upper decks, investigators
said. The ship was reportedly overloaded with poorly lashed cargo, and
crew members did not keep a correct record of passengers. Still, the
Korea Shipping Association, a lobbying group for shipping companies that
also serves as a safety monitor, ruled the ship fit to sail.

On Tuesday, prosecutors sought to arrest two officials at the shipping
association on charges of destroying documents and deleting computer
files before their offices were raided last week by investigators.
Prosecutors were looking for evidence of corrupt ties between the
association and shipping companies.

``It's deeply regrettable that this incident happened because we have
failed to remove these layers of long-running evils,'' said Ms. Park,
referring to what she called entrenched corrupt ties between industries
and regulators.

By Tuesday, the death toll had risen to 205, as divers struggled with
strong currents in their search for the 97 people who remained missing.

Advertisement

\protect\hyperlink{after-bottom}{Continue reading the main story}

\hypertarget{site-index}{%
\subsection{Site Index}\label{site-index}}

\hypertarget{site-information-navigation}{%
\subsection{Site Information
Navigation}\label{site-information-navigation}}

\begin{itemize}
\tightlist
\item
  \href{https://help.nytimes3xbfgragh.onion/hc/en-us/articles/115014792127-Copyright-notice}{©~2020~The
  New York Times Company}
\end{itemize}

\begin{itemize}
\tightlist
\item
  \href{https://www.nytco.com/}{NYTCo}
\item
  \href{https://help.nytimes3xbfgragh.onion/hc/en-us/articles/115015385887-Contact-Us}{Contact
  Us}
\item
  \href{https://www.nytco.com/careers/}{Work with us}
\item
  \href{https://nytmediakit.com/}{Advertise}
\item
  \href{http://www.tbrandstudio.com/}{T Brand Studio}
\item
  \href{https://www.nytimes3xbfgragh.onion/privacy/cookie-policy\#how-do-i-manage-trackers}{Your
  Ad Choices}
\item
  \href{https://www.nytimes3xbfgragh.onion/privacy}{Privacy}
\item
  \href{https://help.nytimes3xbfgragh.onion/hc/en-us/articles/115014893428-Terms-of-service}{Terms
  of Service}
\item
  \href{https://help.nytimes3xbfgragh.onion/hc/en-us/articles/115014893968-Terms-of-sale}{Terms
  of Sale}
\item
  \href{https://spiderbites.nytimes3xbfgragh.onion}{Site Map}
\item
  \href{https://help.nytimes3xbfgragh.onion/hc/en-us}{Help}
\item
  \href{https://www.nytimes3xbfgragh.onion/subscription?campaignId=37WXW}{Subscriptions}
\end{itemize}
