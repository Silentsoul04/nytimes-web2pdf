Sections

SEARCH

\protect\hyperlink{site-content}{Skip to
content}\protect\hyperlink{site-index}{Skip to site index}

\href{https://www.nytimes3xbfgragh.onion/section/business}{International
Business}

\href{https://myaccount.nytimes3xbfgragh.onion/auth/login?response_type=cookie\&client_id=vi}{}

\href{https://www.nytimes3xbfgragh.onion/section/todayspaper}{Today's
Paper}

\href{/section/business}{International Business}\textbar{}Samsung Leader
Stable After Heart Attack

\url{https://nyti.ms/1kZalVx}

\begin{itemize}
\item
\item
\item
\item
\item
\end{itemize}

Advertisement

\protect\hyperlink{after-top}{Continue reading the main story}

Supported by

\protect\hyperlink{after-sponsor}{Continue reading the main story}

\hypertarget{samsung-leader-stable-after-heart-attack}{%
\section{Samsung Leader Stable After Heart
Attack}\label{samsung-leader-stable-after-heart-attack}}

\includegraphics{https://static01.graylady3jvrrxbe.onion/images/2014/05/12/business/12samsungpic/12samsungpic-articleLarge-v2.jpg?quality=75\&auto=webp\&disable=upscale}

By Mark Scott and
\href{http://www.nytimes3xbfgragh.onion/by/brian-x-chen}{Brian X. Chen}

\begin{itemize}
\item
  May 11, 2014
\item
  \begin{itemize}
  \item
  \item
  \item
  \item
  \item
  \end{itemize}
\end{itemize}

The chairman of Samsung Electronics, Lee Kun-hee, who helped transform
the business into a technology giant, was in stable condition Sunday
after suffering a heart attack, the company said.

Mr. Lee, 72, was recovering at Samsung Medical Center in Seoul, said the
company, which declined to comment further on his condition. According
to Reuters, he was admitted to a hospital Saturday night before being
transferred to the medical center.

While he has not directly overseen many of Samsung's products, including
its popular smartphones, Mr. Lee is credited with shaping Samsung into
one of the most profitable consumer electronics companies in the world.
The company is now a leader in smartphones and flat-screen televisions,
as well as semiconductors and washing machines.

Mr. Lee has previously been treated for lung cancer and pneumonia, and
his latest health problem will almost certainly renew calls for a
concrete succession plan. His son, Lee Jae-yong, who served as the
company's chief operating officer until 2012 and is now the Samsung's
vice chairman, is widely expected to eventually take over from his
father.

Shares of Samsung climbed almost 4 percent on Monday, the company's
largest increase in more than six months. Samsung's share price had
fallen almost 9 percent in the last 12 months, as shareholders fretted
over its growth prospects, with the company saying it would reinvest its
large war chest of cash in research and development projects and in new
sectors like health care.

Investors and industry analysts speculated on Monday that the company
would restructure in the wake of Mr. Lee's health.

Mr. Lee's hospitalization highlights the issue of succession in Samsung,
said Park Joong-sun, an analyst at Kiwoon Securities in Seoul. He noted
that only a little over 17 percent of Samsung Electronics is owned by
other Samsung subsidiaries or Lee family members. ``The stock price rose
because there was an expectation among investors that Samsung
Electronics may buy up its own shares to help defend the management
right.''

Mr. Lee --- who is also chairman of the Samsung Group, the conglomerate
that includes Samsung Electronics --- owns 11.9 trillion won, or \$11.6
billion, in stocks in Samsung Electronics, the insurer Samsung Life and
the construction and trading company Samsung C\&T, analysts said. Lee
Jae-yong would have to pay billions of dollars in taxes if he wanted to
inherit his father's shares --- a sum the son would find difficult to
raise, they said.

Though the elder Mr. Lee's contributions to Samsung have been vital, the
implications to Samsung of his declining health do not seem to carry as
much weight as they might at other companies. At Apple, for instance,
its former chief executive Steven P. Jobs had a hands-on role in the
company's creations and inventions, and his death raised concerns among
investors that the company might not continue to produce successful,
innovative products.

By contrast, Samsung, a South Korean company, does not lean so heavily
on just one person's vision. Instead, the group is a huge and complex
organization with many executives overseeing each different part of the
company. That includes J. K. Shin, who as one of three chief executives
at Samsung Electronics, runs the mobile device division, and Kim
Hyun-suk, another executive, who heads the company's television
business.

\href{https://www.nytimes3xbfgragh.onion/interactive/2013/12/15/technology/samsung-timeline.html}{}

\includegraphics{https://static01.graylady3jvrrxbe.onion/images/2013/12/15/technology/samsung-timeline-founding/20131215_samsung-slide-UNOA-videoLarge.jpg}

\hypertarget{from-fish-trader-to-smartphone-maker}{%
\subsection{From Fish Trader to Smartphone
Maker}\label{from-fish-trader-to-smartphone-maker}}

Samsung has its origins as a trading company established in Korea in
1938. It is now one of the largest and most recognizable technology
brands in the world.

Chetan Sharma, an independent telecommunications analyst who also acts
as a consultant to phone companies, underscored the contrast between Mr.
Lee's role at Samsung and that of Mr. Jobs at Apple.

``Steve was the driving force behind all the products down to the last
detail,'' Mr. Sharma said. At Samsung, ``there are many senior
executives who can step in and the world won't notice.''

Mr. Sharma added, ``While Mr. Lee built an empire in Samsung, he isn't
identified with the brand or the products as Mr. Jobs was with Apple.''

Still, Samsung's strong position in the smartphone market, which has
helped cement its reputation as a global powerhouse, is now being
challenged by new low-cost rivals from China and on the high end by
Apple.

Samsung remains one of the few handset makers to profit from selling
smartphones, with its phones like the Galaxy S5 winning over critics and
consumers. Yet last month, the company reported
\href{http://www.nytimes3xbfgragh.onion/2014/04/09/business/international/samsung-electronics-hit-by-earnings-decline-as-smartphone-sales-slow.html?_r=0}{the
lowest quarterly sales} in over a year at its handset unit, which
generated more than three-quarters of Samsung's operating income.

The company also has been fighting a bitter legal battle with Apple over
patent infringement claims, and is facing tough competition from Chinese
rivals like Huawei and Xiaomi in fast-growing emerging markets.

In response, Samsung has been branching out into new areas like wearable
devices and tablets in an effort to protect itself from the competition.

Although Mr. Lee, a billionaire, is not directly overseeing the creation
of products, he remains a powerful figure at the company and plays a key
role in the company's strategic plans. He is also chairman of the
Samsung Group, the conglomerate that includes Samsung Electronics and is
also involved in financial services, among other businesses. Mr. Lee and
his three children control more than 70 companies connected to the
Samsung Group.

Mr. Lee and Samsung have been known for their aggressive tactics since
1987, when Mr. Lee succeeded his father as chairman.

Over the last 15 years, Samsung's share price has risen almost 400
percent on consumers' seemingly insatiable appetite for its products.

Investors, however, are growing wary of the company's future plans, and
some want the tech giant to return part of its large cash stockpile,
which topped more than \$50 billion at the end of last year.

Samsung's share price has fallen almost 9 percent in the last 12 months,
as shareholders fret over its future growth prospects. The company says
it will reinvest its large cash war chest in research and development
projects and in new sectors like health care.

Advertisement

\protect\hyperlink{after-bottom}{Continue reading the main story}

\hypertarget{site-index}{%
\subsection{Site Index}\label{site-index}}

\hypertarget{site-information-navigation}{%
\subsection{Site Information
Navigation}\label{site-information-navigation}}

\begin{itemize}
\tightlist
\item
  \href{https://help.nytimes3xbfgragh.onion/hc/en-us/articles/115014792127-Copyright-notice}{©~2020~The
  New York Times Company}
\end{itemize}

\begin{itemize}
\tightlist
\item
  \href{https://www.nytco.com/}{NYTCo}
\item
  \href{https://help.nytimes3xbfgragh.onion/hc/en-us/articles/115015385887-Contact-Us}{Contact
  Us}
\item
  \href{https://www.nytco.com/careers/}{Work with us}
\item
  \href{https://nytmediakit.com/}{Advertise}
\item
  \href{http://www.tbrandstudio.com/}{T Brand Studio}
\item
  \href{https://www.nytimes3xbfgragh.onion/privacy/cookie-policy\#how-do-i-manage-trackers}{Your
  Ad Choices}
\item
  \href{https://www.nytimes3xbfgragh.onion/privacy}{Privacy}
\item
  \href{https://help.nytimes3xbfgragh.onion/hc/en-us/articles/115014893428-Terms-of-service}{Terms
  of Service}
\item
  \href{https://help.nytimes3xbfgragh.onion/hc/en-us/articles/115014893968-Terms-of-sale}{Terms
  of Sale}
\item
  \href{https://spiderbites.nytimes3xbfgragh.onion}{Site Map}
\item
  \href{https://help.nytimes3xbfgragh.onion/hc/en-us}{Help}
\item
  \href{https://www.nytimes3xbfgragh.onion/subscription?campaignId=37WXW}{Subscriptions}
\end{itemize}
