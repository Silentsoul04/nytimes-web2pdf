Sections

SEARCH

\protect\hyperlink{site-content}{Skip to
content}\protect\hyperlink{site-index}{Skip to site index}

\href{https://www.nytimes3xbfgragh.onion/section/science}{Science}

\href{https://myaccount.nytimes3xbfgragh.onion/auth/login?response_type=cookie\&client_id=vi}{}

\href{https://www.nytimes3xbfgragh.onion/section/todayspaper}{Today's
Paper}

\href{/section/science}{Science}\textbar{}Looks Like Rain Again. And
Again.

\url{https://nyti.ms/1le5sH9}

\begin{itemize}
\item
\item
\item
\item
\item
\end{itemize}

Advertisement

\protect\hyperlink{after-top}{Continue reading the main story}

Supported by

\protect\hyperlink{after-sponsor}{Continue reading the main story}

\href{/column/by-degrees}{By Degrees}

\hypertarget{looks-like-rain-again-and-again}{%
\section{Looks Like Rain Again. And
Again.}\label{looks-like-rain-again-and-again}}

\includegraphics{https://static01.graylady3jvrrxbe.onion/images/2014/05/13/science/13GILL/13GILL-articleLarge.jpg?quality=75\&auto=webp\&disable=upscale}

By \href{http://www.nytimes3xbfgragh.onion/by/justin-gillis}{Justin
Gillis}

\begin{itemize}
\item
  May 12, 2014
\item
  \begin{itemize}
  \item
  \item
  \item
  \item
  \item
  \end{itemize}
\end{itemize}

The acid test of a scientific theory is whether it makes predictions
that eventually come true. So consider this old prediction, from a pair
of researchers in Australia and New Zealand. They were summarizing the
results of then-primitive computerized forecasts about global warming:

``The available evidence suggests that a warmer world is likely to
experience an increase in the frequency of heavy precipitation events,
associated with a more intense hydrological cycle and the increased
water-holding capacity of a warmer atmosphere.''

That was published in 1995, and it was based on research going back to
the 1980s. Fast forward to 2014.

In the \href{http://nca2014.globalchange.gov/}{National Climate
Assessment}, published last week, researchers in the United States
reported that ``large increases in heavy precipitation have occurred in
the Northeast, Midwest and Great Plains, where heavy downpours have
frequently led to runoff that exceeded the capacity of storm drains and
levees, and caused flooding events and accelerated erosion.''

The future, it would seem, has arrived.

Climate is a difficult branch of science, full of ambiguities and
uncertainties. But scientists can justly claim to have demonstrated some
predictive skill about many of the potential implications of the human
release of greenhouse gases.

Their track record actually goes back to 1896, when a Swede named
\href{http://earthobservatory.nasa.gov/Features/Arrhenius/}{Svante
Arrhenius} first
\href{http://www.globalwarmingart.com/images/1/18/Arrhenius.pdf}{predicted}
that emissions of carbon dioxide would cause the planet to warm. It took
more than 80 years to be sure he was right. At roughly the same time
that realization was taking hold, climate scientists running computer
models of the atmosphere began to focus on the likelihood of heavier
rains in a future climate.

Many people are still catching up with the science, but it is hard to
miss the ubiquity of these heavy rainstorms in recent years.

People in the Florida Panhandle recently
\href{http://www.nytimes3xbfgragh.onion/2014/05/01/us/severe-flooding-in-south-and-midwest-as-storm-system-tapers-off.html?_r=0}{had
to dodge flash floods} after two feet of rain fell in 26 hours.
Torrential rains caused a Washington State hillside to
\href{http://www.nytimes3xbfgragh.onion/2014/03/25/us/search-continues-after-washington-state-landslide.html}{collapse}
and bury a community earlier this year. Tumultuous
\href{http://www.nytimes3xbfgragh.onion/2013/09/14/us/colorado-flooding.html}{rainstorms
and floods} overwhelmed Colorado last year, and sudden floods swept
through
\href{http://www.nytimes3xbfgragh.onion/2010/05/04/us/04flood.html?_r=1\&adxnnl=1\&adxnnlx=1399809918-8on824VB1bK8a1dRDBI2Bg}{Nashville}
in 2010, and
\href{http://www.cnn.com/2009/TECH/science/09/22/atlanta.weather.science/}{Atlanta}
in 2009.

We're seeing a pattern here.

In the National Climate Assessment, the experts reported huge increases
since the mid-20th century in the amount of precipitation falling in
very heavy rainstorms: up 71 percent in the Northeast, 37 percent in the
Midwest and 27 percent in the Southeast. The effect was seen on a
smaller scale west of the Mississippi River, too, even in parts of the
country where the climate is drying out over all.

What led the researchers to expect this long before it actually
happened?

Initially, the forecast was based on
\href{http://tinyurl.com/latnp6g}{simple physics} from the 19th century.
As we pour carbon dioxide into the air, the lower atmosphere has to
warm. As it does, it is able to hold more moisture, and as the surface
of the ocean also warms, more moisture tends to evaporate from it.

In the United States, the increase in water vapor has been on the order
of 3 percent or 4 percent since the 1970s (most of the human-caused
global warming has occurred since then). That may not sound like a big
jump, but the effect is enormous.

Two leading scientists,
\href{http://www.cgd.ucar.edu/staff/trenbert/}{Kevin E. Trenberth} at
the \href{http://ncar.ucar.edu/}{National Center for Atmospheric
Research} and
\href{http://www.ametsoc.org/boardpges/cwce/docs/profiles/EasterlingDavidR/profile.html}{David
R. Easterling} at the \href{http://www.noaa.gov/}{National Oceanic and
Atmospheric Administration}, ran some calculations and agreed that the
warming has, on average, put more than a trillion gallons of extra water
into the air over the contiguous 48 states, probably closer to two
trillion.

That extra water has to fall as rain or snow. But from the elementary
physics, it was long unclear whether this would mean more rainy days
over all, or more intense rains, or both.

It was the computer models of the climate that suggested, starting in
the late 1980s, that the answer would be the latter, and so it has
turned out. One way to think of it is that even with a lot of moisture
in the air, conditions are not always right for rain, but when they are
right, the skies have a lot more water to dump.

``It rains harder than it used to,'' said Dr. Trenberth, who could not
resist adding: ``When it rains, it pours.''

Researchers sponsored by the Australian government were the first to
really drill into the implications of the finding. In their 1995
overview \href{http://tinyurl.com/mmcmr7z}{paper}, A. M. Fowler of the
University of Auckland in New Zealand and K. J. Hennessy of Australia's
national research program warned that society needed to start thinking
about the risks. They suggested toughening standards for the designs of
levees and dams, and hardening roads and culverts against the
possibility of more flash floods.

Society responded by ignoring them. For someone sitting in Pensacola,
Fla., wondering why the roads were washed out the other day, that
longstanding refusal to confront reality might be a good part of the
answer.

The warming of the planet has
\href{http://www.nytimes3xbfgragh.onion/2013/06/11/science/earth/what-to-make-of-a-climate-change-plateau.html}{slowed}
in recent years, but scientists think that is likely temporary. They
expect it to get much, much warmer as this century progresses, and that
can only mean that the rains will fall harder still.

So if you are still a little amazed at what these heavy downpours have
been doing to communities around the country, the message from science
is pretty blunt: Get used to it.

Advertisement

\protect\hyperlink{after-bottom}{Continue reading the main story}

\hypertarget{site-index}{%
\subsection{Site Index}\label{site-index}}

\hypertarget{site-information-navigation}{%
\subsection{Site Information
Navigation}\label{site-information-navigation}}

\begin{itemize}
\tightlist
\item
  \href{https://help.nytimes3xbfgragh.onion/hc/en-us/articles/115014792127-Copyright-notice}{©~2020~The
  New York Times Company}
\end{itemize}

\begin{itemize}
\tightlist
\item
  \href{https://www.nytco.com/}{NYTCo}
\item
  \href{https://help.nytimes3xbfgragh.onion/hc/en-us/articles/115015385887-Contact-Us}{Contact
  Us}
\item
  \href{https://www.nytco.com/careers/}{Work with us}
\item
  \href{https://nytmediakit.com/}{Advertise}
\item
  \href{http://www.tbrandstudio.com/}{T Brand Studio}
\item
  \href{https://www.nytimes3xbfgragh.onion/privacy/cookie-policy\#how-do-i-manage-trackers}{Your
  Ad Choices}
\item
  \href{https://www.nytimes3xbfgragh.onion/privacy}{Privacy}
\item
  \href{https://help.nytimes3xbfgragh.onion/hc/en-us/articles/115014893428-Terms-of-service}{Terms
  of Service}
\item
  \href{https://help.nytimes3xbfgragh.onion/hc/en-us/articles/115014893968-Terms-of-sale}{Terms
  of Sale}
\item
  \href{https://spiderbites.nytimes3xbfgragh.onion}{Site Map}
\item
  \href{https://help.nytimes3xbfgragh.onion/hc/en-us}{Help}
\item
  \href{https://www.nytimes3xbfgragh.onion/subscription?campaignId=37WXW}{Subscriptions}
\end{itemize}
