Sections

SEARCH

\protect\hyperlink{site-content}{Skip to
content}\protect\hyperlink{site-index}{Skip to site index}

\href{https://www.nytimes3xbfgragh.onion/section/food}{Food}

\href{https://myaccount.nytimes3xbfgragh.onion/auth/login?response_type=cookie\&client_id=vi}{}

\href{https://www.nytimes3xbfgragh.onion/section/todayspaper}{Today's
Paper}

\href{/section/food}{Food}\textbar{}You Start in Spain, but There's Room
to Roam

\url{https://nyti.ms/RCxrsd}

\begin{itemize}
\item
\item
\item
\item
\item
\item
\end{itemize}

Advertisement

\protect\hyperlink{after-top}{Continue reading the main story}

Supported by

\protect\hyperlink{after-sponsor}{Continue reading the main story}

\hypertarget{you-start-in-spain-but-theres-room-to-roam}{%
\section{You Start in Spain, but There's Room to
Roam}\label{you-start-in-spain-but-theres-room-to-roam}}

Slide 1 of 11

1/11

The curved bar at El Quinto Pino, packed with guests. Last fall the
restaurant's owners and chefs, Alex Raij and her husband, Eder Montero,
took over the lease on the apartment next door and filled it with
tables, making their sardine can a real restaurant.

Credit...Michael Falco for The New York Times

\begin{itemize}
\item
  \includegraphics{https://static01.graylady3jvrrxbe.onion/images/2014/05/14/dining/20140514-REST-slide-34EL/20140514-REST-slide-34EL-superJumbo.jpg}
\item
  \includegraphics{https://static01.graylady3jvrrxbe.onion/images/2014/05/14/dining/20140514-REST-slide-R3O0/20140514-REST-slide-R3O0-superJumbo.jpg}
\item
  \includegraphics{https://static01.graylady3jvrrxbe.onion/images/2014/05/14/dining/20140514-REST-slide-I6OT/20140514-REST-slide-I6OT-superJumbo.jpg}
\item
  \includegraphics{https://static01.graylady3jvrrxbe.onion/images/2014/05/14/dining/20140514-REST-slide-CS0M/20140514-REST-slide-CS0M-superJumbo.jpg}
\item
  \includegraphics{https://static01.graylady3jvrrxbe.onion/images/2014/05/14/dining/20140514-REST-slide-E8QU/20140514-REST-slide-E8QU-superJumbo.jpg}
\item
  \includegraphics{https://static01.graylady3jvrrxbe.onion/images/2014/05/14/dining/20140514-REST-slide-YV69/20140514-REST-slide-YV69-superJumbo.jpg}
\item
  \includegraphics{https://static01.graylady3jvrrxbe.onion/images/2014/05/14/dining/20140514-REST-slide-SY6P/20140514-REST-slide-SY6P-superJumbo.jpg}
\item
  \includegraphics{https://static01.graylady3jvrrxbe.onion/images/2014/05/14/dining/20140514-REST-slide-JG9A/20140514-REST-slide-JG9A-superJumbo.jpg}
\item
  \includegraphics{https://static01.graylady3jvrrxbe.onion/images/2014/05/14/dining/20140514-REST-slide-KTQD/20140514-REST-slide-KTQD-superJumbo.jpg}
\item
  \includegraphics{https://static01.graylady3jvrrxbe.onion/images/2014/05/14/dining/20140514-REST-slide-LR9E/20140514-REST-slide-LR9E-superJumbo.jpg}
\item
  \includegraphics{https://static01.graylady3jvrrxbe.onion/images/2014/05/14/dining/20140514-REST-slide-YJJM/20140514-REST-slide-YJJM-superJumbo.jpg}
\end{itemize}

\begin{itemize}
\tightlist
\item
  El Quinto Pino\\
  ★★ Spanish \$\$ 401 West 24th Street 212-206-6900
\end{itemize}

\href{http://www.opentable.com/single.aspx?ref=4201\&rid=148285}{Reserve
a Table}

When you make a reservation at an independently reviewed restaurant
through our site, we earn an affiliate commission.

By \href{http://www.nytimes3xbfgragh.onion/by/pete-wells}{Pete Wells}

\begin{itemize}
\item
  May 13, 2014
\item
  \begin{itemize}
  \item
  \item
  \item
  \item
  \item
  \item
  \end{itemize}
\end{itemize}

The name means ``the fifth pine,'' a Spanish idiom for the boondocks,
which was probably a stretch even seven years ago when El Quinto Pino
opened on West 24th Street, across from the canyon wall of London
Terrace. The thumbnail curve of the sardine-can tapas bar is just the
way it's been since the beginning, packed, every seat there and on the
opposite wall a few inches away taken almost any time you show up.

Turn right at the door, though, and you'll find something new. Last
fall, El Quinto Pino's owners and chefs, Alex Raij and her husband, Eder
Montero, took over the lease on the apartment next door and filled it
with tables, making their sardine can a real restaurant. ``It won't make
a bid for your entire night: You will snack at El Quinto Pino and almost
inevitably eat more somewhere else,'' Peter Meehan wrote in a
``\href{http://events.nytimes3xbfgragh.onion/2007/10/31/dining/reviews/31unde.html?_r=0}{\$25
and Under'' column} in 2007, its last review in The New York Times. With
a reservation for that side room you won't need to go anywhere else for
dinner. You won't want to, either, once the food starts zooming out of
the kitchen.

Ms. Raij and Mr. Montero also own Txikito, across Ninth Avenue, which
digs into Basque cuisine, and La Vara, in Cobble Hill, Brooklyn, where
the menu is a master's thesis on Moorish and Jewish imprints upon
Spanish cooking. A short ``menú turístico'' at El Quinto Pino
investigates a different region of the country every few months, which
gives regulars a reason to keep coming and gives the kitchen a launching
pad for dishes that may hit the permanent roster. Other than that, the
two chefs haven't tied themselves down. Having laid down their Iberian
credentials in their other restaurants, they've given El Quinto Pino a
passport to roam around.

Among the bar sandwiches tightly wrapped in paper is a visitor from New
Orleans, a po'boy with closely packed, terrifically crunchy fried squid
legs at its core. Another is a fortified Cubano, in which the ham and
crisp pickle have more oomph than usual, the cheese doesn't taste
processed, for once, and the braised pork is supplemented with a red
wallop of blood sausage. The chefs do a takeoff on Catalonia's version
of the croque monsieur, the bikini, but they aim a double-barreled blast
of Mexico at it, topping the melted cheese with roasted poblanos and
huitlacoche, the flamboyantly weird, mushroom-tasting black fungus that
grows on corn ears.

Geographic freedom between pieces of bread was written into El Quinto
Pino's charter from the start, when Ms. Raij brought out (all rise,
please) the uni panino. A pressed ficelle filled with creamy sea urchin,
melted butter and a lashing of mustard oil, it became one of the city's
essential sandwiches. It still is, even though one night mustard-oil
supplies must have run low and the panino wasn't the same without its
throat-catching rumble.

\includegraphics{https://static01.graylady3jvrrxbe.onion/images/2014/05/14/dining/20140514-REST-slide-YV69/20140514-REST-slide-YV69-articleLarge.jpg?quality=75\&auto=webp\&disable=upscale}

The rest of the menu is made up of tapas-style plates, and the servers
are careful not to overload the small tables, bringing just one or two
at a time. The kitchen mostly presents these dishes as tangles that are
easily shared, and resists the impulse toward multicomponent platings.
With Raij-Montero portions, four people seems to be the breaking point
at which you're not so much eating as nibbling. I enjoyed El Quinto Pino
more in a party of three, and best of all with just one accomplice.

Some of the seafood is so appealing and out of the ordinary that
dividing it can test your ability to play well with others. I hated
surrendering the last fried lump of sea anemone folded into soft
scrambled eggs. Called ortiguillas, they tasted almost like fried
oysters, but not quite, and I wanted to get to the bottom of that ``not
quite.'' My curiosity and my appetite also wanted a few more runs at the
Catalan salad xató, which mixed chicory and canned bonito with raw salt
cod, soaked to pull out the salt and to reveal a flavor of quietly
intensified fish. Dressed with both romesco and an uncooked tomato
sauce, this salad seemed to get more lively with each bite.

Tender, apple-blossom-pink gambas al ajillo get a little fresh ginger
along with the garlic, an addition that made me want to eat them twice
as fast. (Mashed and sliced avocado don't have the same effect on an
oddly inert salpicon of shrimp.)

Smaller shrimp from Cadiz, the size of a paper clip and intensely
flavorful in their shells, are pressed into a tortillita, a wonderful
fritter that looks like a latke cooked under a brick. Fried on their
own, these shrimp turn up again to bring a marine undertow to a jiggly
poached egg with slivered snow peas.

The kitchen's hand is so steady that it's easy to pass over its
occasional bobbles, unless you're unlucky enough to be served two or
three in a row. My disappointments were spread out: dully spiced lamb
skewers; underseasoned bits of fried pork whose name, ``bag of bacon,''
raised undue expectations; seafood fideua, like paella made from
noodles, that lacked the oceanic depth I loved when I had the same dish
at La Vara.

And it's unclear how the xocolata dessert is meant to be served. One
night the salt-sprinkled ingot of chocolate ganache was filled with
fruity green olive oil that spilled out at the touch of a fork; I don't
think I've ever enjoyed the pairing of the two ingredients more. On
another, the oil was cold and congealed like Vaseline. Fortunately I had
the excellent crema catalana, roused from its usual custardy slumber by
cinnamon and lemon zest, to fall back on.

Everything I know about dating could be carved onto the head of a pin
with a butter knife, but a younger man who lives down the block from El
Quinto Pino assures me it's a great date place. He didn't mean the bar,
with its was-that-your-foot? dimensions, but the new dining room.
Separated from the drinkers by a galley kitchen, its 30 seats feel
secluded and romantic, but not in an obvious way. The architect Silvia
Zofio gave it a quietly domestic, midcentury look, with an earth-toned
tapestry covering one wall and a chandelier that suggests the swoosh of
a flamenco dancer's skirt in mid-twirl. It's a room for talking, and if
it's not exactly the boondocks, it's easy enough to get lost there for
an hour or two.

Advertisement

\protect\hyperlink{after-bottom}{Continue reading the main story}

\hypertarget{site-index}{%
\subsection{Site Index}\label{site-index}}

\hypertarget{site-information-navigation}{%
\subsection{Site Information
Navigation}\label{site-information-navigation}}

\begin{itemize}
\tightlist
\item
  \href{https://help.nytimes3xbfgragh.onion/hc/en-us/articles/115014792127-Copyright-notice}{©~2020~The
  New York Times Company}
\end{itemize}

\begin{itemize}
\tightlist
\item
  \href{https://www.nytco.com/}{NYTCo}
\item
  \href{https://help.nytimes3xbfgragh.onion/hc/en-us/articles/115015385887-Contact-Us}{Contact
  Us}
\item
  \href{https://www.nytco.com/careers/}{Work with us}
\item
  \href{https://nytmediakit.com/}{Advertise}
\item
  \href{http://www.tbrandstudio.com/}{T Brand Studio}
\item
  \href{https://www.nytimes3xbfgragh.onion/privacy/cookie-policy\#how-do-i-manage-trackers}{Your
  Ad Choices}
\item
  \href{https://www.nytimes3xbfgragh.onion/privacy}{Privacy}
\item
  \href{https://help.nytimes3xbfgragh.onion/hc/en-us/articles/115014893428-Terms-of-service}{Terms
  of Service}
\item
  \href{https://help.nytimes3xbfgragh.onion/hc/en-us/articles/115014893968-Terms-of-sale}{Terms
  of Sale}
\item
  \href{https://spiderbites.nytimes3xbfgragh.onion}{Site Map}
\item
  \href{https://help.nytimes3xbfgragh.onion/hc/en-us}{Help}
\item
  \href{https://www.nytimes3xbfgragh.onion/subscription?campaignId=37WXW}{Subscriptions}
\end{itemize}
