Sections

SEARCH

\protect\hyperlink{site-content}{Skip to
content}\protect\hyperlink{site-index}{Skip to site index}

\href{https://www.nytimes3xbfgragh.onion/section/technology}{Technology}

\href{https://myaccount.nytimes3xbfgragh.onion/auth/login?response_type=cookie\&client_id=vi}{}

\href{https://www.nytimes3xbfgragh.onion/section/todayspaper}{Today's
Paper}

\href{/section/technology}{Technology}\textbar{}Never Forgetting a Face

\href{https://nyti.ms/1lK7Rw7}{https://nyti.ms/1lK7Rw7}

\begin{itemize}
\item
\item
\item
\item
\item
\end{itemize}

Advertisement

\protect\hyperlink{after-top}{Continue reading the main story}

Supported by

\protect\hyperlink{after-sponsor}{Continue reading the main story}

You for Sale

\hypertarget{never-forgetting-a-face}{%
\section{Never Forgetting a Face}\label{never-forgetting-a-face}}

\includegraphics{https://static01.graylady3jvrrxbe.onion/images/2014/05/18/business/18face-illo/18face-illo-articleLarge.jpg?quality=75\&auto=webp\&disable=upscale}

By \href{http://www.nytimes3xbfgragh.onion/by/natasha-singer}{Natasha
Singer}

\begin{itemize}
\item
  May 17, 2014
\item
  \begin{itemize}
  \item
  \item
  \item
  \item
  \item
  \end{itemize}
\end{itemize}

Joseph J. Atick cased the floor of the Ronald Reagan Building and
International Trade Center in Washington as if he owned the place. In a
way, he did. He was one of the organizers of the event, a conference and
trade show for the biometrics security industry. Perhaps more to the
point, a number of the wares on display, like an airport face-scanning
checkpoint, could trace their lineage to his work.

A physicist, Dr. Atick is one of the pioneer entrepreneurs of modern
face recognition. Having helped advance the fundamental face-matching
technology in the 1990s, he went into business and promoted the systems
to government agencies looking to identify criminals or prevent identity
fraud. ``We saved lives,'' he said during the conference in mid-March.
``We have solved crimes.''

Thanks in part to his boosterism, the global business of biometrics ---
using people's unique physiological characteristics, like their
fingerprint ridges and facial features, to learn or confirm their
identity --- is booming. It generated an estimated \$7.2 billion in
2012, according to reports by Frost \& Sullivan.

Making his rounds at \href{http://www.connectidexpo.com/}{the trade
show}, Dr. Atick, a short, trim man with an indeterminate Mediterranean
accent, warmly greeted industry representatives at their exhibition
booths. Once he was safely out of earshot, however, he worried aloud
about what he was seeing. What were those companies' policies for
retaining and reusing consumers' facial data? Could they identify
individuals without their explicit consent? Were they running
face-matching queries for government agencies on the side?

Now an industry consultant, Dr. Atick finds himself in a delicate
position. While promoting and profiting from an industry that he helped
foster, he also feels compelled to caution against its unfettered
proliferation. He isn't so much concerned about government agencies that
use face recognition openly for specific purposes --- for example, the
many state
\href{http://www.oregon.gov/ODOT/DMV/pages/faqs/facial_recognition.aspx}{motor
vehicle departments that scan drivers' faces} as a way to prevent
license duplications and fraud. Rather, what troubles him is the
potential exploitation of face recognition to identify ordinary and
unwitting citizens as they go about their lives in public. Online, we
are all tracked. But to Dr. Atick, the street remains a haven, and he
frets that he may have abetted a technology that could upend the social
order.

Face-matching today could enable mass surveillance, ``basically robbing
everyone of their anonymity,'' he says, and inhibit people's normal
behavior outside their homes. Pointing to the intelligence documents
made public by Edward J. Snowden, he adds that once companies amass
consumers' facial data, government agencies might obtain access to it,
too.

To many in the biometrics industry, Dr. Atick's warning seems
Cassandra-like. Face recognition to them is no different from a car, a
neutral technology whose advantages far outweigh the risks. The
conveniences of biometrics seem self-evident: Your unique code
automatically accompanies you everywhere. They envision a world where,
instead of having to rely on losable ID cards or on a jumble of easily
forgettable --- not to mention hackable --- passwords, you could unlock
your smartphone or gain entry to banks, apartment complexes, parking
garages and health clubs just by showing your face.

Dr. Atick sees convenience in these kinds of uses as well. But he
provides a cautionary counterexample to make his case. Just a few months
back, he heard about \href{http://www.nametag.ws/\#mainscreen}{NameTag,
an app} that, according to its news release, was available in an early
form to people trying out Google Glass. Users had only to glance at a
stranger and NameTag would instantly return a match complete with that
stranger's name, occupation and public Facebook profile information.
``We are basically allowing our fellow citizens to surveil us,'' Dr.
Atick told me on the trade-show floor.

(His sentiments were shared by Senator Al Franken, Democrat of Minnesota
and chairman of the Senate subcommittee on
\href{http://www.judiciary.senate.gov/about/subcommittees\#privacy}{privacy,
technology and the law}. Concerned that NameTag might facilitate
stalking,
\href{http://www.franken.senate.gov/?p=press_release\&id=2699}{Mr.
Franken} requested that its public introduction be delayed; in late
April, the app's developer said he would comply with the request. Google
has said that it will not approve facial recognition apps on Google
Glass.)

Dr. Atick is just as bothered by what could be brewing quietly in larger
companies. Over the past few years, several tech giants have acquired
face-recognition start-up businesses. In 2011,
\href{http://blogs.wsj.com/digits/2011/07/22/google-acquires-facial-recognition-technology-company/}{Google
bought Pittsburgh Pattern Recognition}, a computer vision business
developed by researchers at Carnegie Mellon University. In 2012,
\href{http://www.bloomberg.com/news/2012-06-18/facebook-buys-face-com-adds-facial-recognition-software.html}{Facebook
bought Face.com, an Israeli start-up}.

Google and Facebook both declined to comment for this article about
their plans for the technology.

Dr. Atick says the technology he helped cultivate requires some special
safeguards. Unlike fingerprinting or other biometric techniques, face
recognition can be used at a distance, without people's awareness; it
could then link their faces and identities to the many pictures they
have put online. But in the United States, no specific federal law
governs face recognition. A division of the Commerce Department is
organizing a meeting of
\href{http://www.nytimes3xbfgragh.onion/2014/02/02/technology/when-no-one-is-just-a-face-in-the-crowd.html}{industry
representatives and consumer advocates} on Tuesday to start hammering
out a voluntary code of conduct for the technology's commercial use.

Dr. Atick has been working behind the scenes to influence the outcome.
He is part of a tradition of scientists who have come to feel
responsible for what their work has wrought. ``I think that the industry
has to own up,'' he asserts. ``If we do not step up to the plate and
accept responsibility, there could be unexpected apps and
consequences.''

\textbf{`Not an Innocent Machine'}

A few uses of face recognition are already commonplace. It's what allows
\href{https://www.facebookcorewwwi.onion/help/122175507864081}{Facebook}
and \href{https://support.google.com/plus/answer/2370300?hl=en}{Google
Plus} to automatically suggest name tags for members or their friends in
photographs.

And more applications could be in the works. Google has applied for a
patent on a \href{http://1.usa.gov/1dxTzys}{method to identify faces in
videos} and on one to \href{http://1.usa.gov/1lq4JCw}{allow people to
log on to devices by winking} or making other facial expressions.
Facebook researchers recently reported how the company had developed a
powerful pattern-recognition system, called
\href{https://www.facebookcorewwwi.onion/publications/546316888800776/}{DeepFace,
which had achieved} near-human accuracy in identifying people's faces.

Image

Joseph Atick, a pioneer in the industry, now fears that if face-matching
is taken too far, it could allow mass surveillance, ``basically robbing
everyone of their anonymity.''Credit...Tony Cenicola/The New York Times

But real-time, automated face recognition is a relatively recent
phenomenon and, at least for now, a niche technology. In the early
1990s, several academic researchers, including Dr. Atick, hit upon the
idea of programming computers to identify a face's most distinguishing
features; the software then used those local points to recognize that
face when it reappeared in other images.

To work, the technology needs a large data set, called an image gallery,
containing the photographs or video stills of faces already identified
by name. Software automatically converts the topography of each face in
the gallery into a unique mathematical code, called a faceprint. Once
people are faceprinted, they may be identified in existing or subsequent
photographs or as they walk in front of a video camera.

The technology is already in use in law enforcement and casinos. In
\href{http://nypost.com/2012/03/16/nypd-uses-high-tech-facial-recognition-software-to-nab-barbershop-shooting-suspect/}{New
York},
\href{http://www.govtech.com/public-safety/Pennsylvania-Facial-Recognition-Systems-Integrate-to-Widen-Search-for-Criminals.html}{Pennsylvania}
and
\href{http://www.nbcsandiego.com/news/local/Chula-Vista-Police-Dept-New-Facial-Recognition-Technology-231835401.html}{California,}
police departments with face-recognition systems can input the image of
a robbery suspect taken from a surveillance video in a bank, for
instance, and compare the suspect's faceprint against their image
gallery of convicted criminals, looking for a match. And some casinos
faceprint visitors, seeking to identify repeat big-spending customers
for special treatment. In Japan, a few grocery stores use face-matching
\href{http://www.scmp.com/news/asia/article/1466536/115-japanese-stores-sharing-customers-facial-data}{to
classify some shoppers as shoplifters} or even ``complainers'' and
blacklist them.

Whether society embraces face recognition on a larger scale will
ultimately depend on how legislators, companies and consumers resolve
the argument about its singularity. Is faceprinting as innocuous as
photography, an activity that people may freely perform? Or is a
faceprint a unique indicator, like a fingerprint or a DNA sequence, that
should require a person's active consent before it can be collected,
matched, shared or sold?

Dr. Atick is firmly in the second camp.

His upbringing influenced both his interest in identity authentication
and his awareness of the power conferred on those who control it. He was
born in Jerusalem in 1964 to Christian parents of Greek and French
descent. Conflict based on ethnic and religious identity was the
backdrop of his childhood. He was an outsider, neither Jewish nor
Muslim, and remembers often having to show an identity booklet listing
his name, address and religion.

``As a 5- or 6-year old boy, seeing identity as a foundation for trust,
I think it marked me,'' Dr. Atick says. To this day, he doesn't feel
comfortable leaving his New York apartment without his driver's license
or passport.

After a childhood accident damaged his eyesight, he became interested in
the mechanics of human vision. Eventually, he dropped out of high school
to write a physics textbook. His family moved to Miami, and he decided
to skip college. It did not prove a setback; at 17, he was accepted to a
doctoral program in physics at Stanford.

Still interested in how the brain processes visual information, he
started a computational neuroscience lab at Rockefeller University in
Manhattan, where he and two colleagues began programming computers to
recognize faces. To test the accuracy of their algorithms, they acquired
the most powerful computer they could find, a Silicon Graphics desktop,
for their lab and mounted a video camera on it. They added a speech
synthesizer so the device could read certain phrases aloud.

As Dr. Atick tells it, he concluded that the system worked after he
walked into the lab one day and the computer called out his name, along
with those of colleagues in the room. ``We were just milling about and
you heard this metallic voice saying: `I see Joseph. I see Norman. I see
Paul,'~'' Dr. Atick recounts. Until then, most face recognition had
involved analyzing static images, he says, not identifying a face amid a
group of live people. ``We had made a breakthrough.''

The researchers left academia to start their own face-recognition
company, called Visionics, in 1994. Dr. Atick says he hadn't initially
considered the ramifications of their product, named
\href{http://www.prnewswire.com/news-releases/face-recognition-software-from-visionics-wins-pc-week-best-of-show-award-at-comdex-77726772.html}{FaceIt.}
But when intelligence agencies began making inquiries, he says, it
``started dawning on me that this was not an innocent machine.''

He helped start an \href{http://www.ibia.org/}{international biometrics
trade group}, and it came up with guidelines like requiring notices in
places where face recognition was in use. But even in a nascent industry
composed of a few companies, he had little control.

In 2001, his worst-case scenario materialized. A competitor supplied the
Tampa police with a face-recognition system; officers covertly deployed
it on fans attending Super Bowl XXXV. The police scanned tens of
thousands of fans without their awareness, identifying a handful of
petty criminals, but no one was detained.

Journalists coined it
\href{http://news.cnet.com/Firm-defends-snooper-bowl-technology/2100-1023_3-253884.html}{the
``Snooper Bowl.}'' Public outrage and congressional criticism ensued,
raising issues about the potential intrusiveness and fallibility of face
recognition that have yet to be resolved.

Dr. Atick says he thought this fiasco had doomed the industry: ``I had
to explain to the media this was not responsible use.''

\includegraphics{https://static01.graylady3jvrrxbe.onion/images/2014/04/11/multimedia/tech-face-recognition/tech-face-recognition-videoSixteenByNine1050.jpg}

Then, a few months later, came the Sept. 11 terrorist attacks. Dr. Atick
immediately went to Washington to promote biometrics as a new method of
counterterrorism. He testified before congressional committees and made
the rounds on nightly news programs where he argued that terrorism might
be prevented if airports, motor vehicle departments, law enforcement and
immigration agencies used face recognition to authenticate people's
identities.

``Terror is not faceless,'' he said in one segment on ABC's ``World News
Tonight.'' ``Terror has measurable identity, has a face that can be
detected through technology that's available today.''

It was an optimistic spin, given that the technology at that early stage
did not work well in uncontrolled environments.

Still, Dr. Atick prospered. He merged his original business with other
biometrics enterprises, eventually forming a company called L-1 Identity
Solutions. In 2011, Safran, a military contractor in France,
\href{http://www.safran-group.com/site-safran-en/finance-397/financial-publications/financial-press-releases/article/safran-completes-the-acquisition-11359}{bought
the bulk of that company} for about \$1.5 billion, including debt.

Dr. Atick had waited 17 years for a cash payout from his endeavors; his
take amounted to tens of millions of dollars.

In fact, some experts view his contribution to the advancement of face
recognition as not so much in research but in recognizing its business
potential and capitalizing on it.

``He actually was one of the early commercializers of face-recognition
algorithms,'' says P. Jonathon Phillips, an electronics engineer at the
National Institute of Standards and Technology, which
\href{http://www.nist.gov/itl/iad/ig/face.cfm}{evaluates the accuracy}
of commercial face-recognition engines.

\textbf{Ovals, Squares and Matches}

At Knickerbocker Village, a 1,600-unit
\href{http://www.knickvill.com/en/}{red-brick apartment complex} in
Lower Manhattan where Julius and Ethel Rosenberg once lived, the
entryways click open as residents walk toward the doors. It is one of
the first properties in New York City to install a biometrics system
that uses both face and motion recognition, and it is a showcase for
\href{http://www.fst21.com/}{FST Biometrics}, the Israeli security firm
that designed the program.

``This development will make obsolete keys, cards and codes --- because
your identity is the key,'' says Aharon Zeevi Farkash, the chief
executive of FST. ``Your face, your behavior, your biometrics are the
key.''

On a recent visit to New York, Mr. Farkash offered to demonstrate how it
worked. We met at the Knickerbocker security office on the ground floor.
There, he posed before a webcam, enabling the system to faceprint and
enroll him. To test it, he walked outside into the courtyard and
approached one of the apartment complex entrances. He pulled open an
outer glass door, heading directly toward a camera embedded in the wall
near an inner door.

Back in the security office, a monitor broadcast video of the process.

First, a yellow oval encircled Mr. Farkash's face in the video,
indicating that the system had detected a human head. Then a green
square materialized around his head. The system had found a match. A
message popped up on the screen: ``Recognized, Farkash Aharon.
Confidence: 99.7 percent.''

On his third approach, the system pegged him even sooner --- while he
was opening the outer door.

Mr. Farkash says he believes that systems like these, which are designed
to identify people in motion, will soon make obsolete the cumbersome,
time-consuming security process at most airports.

``The market needs convenient security,'' he told me; the company's
system is now being tested at one airport.

Mr. Farkash served in the Israeli army for nearly 40 years, eventually
as chief of military intelligence. Now a major general in the army
reserves, he says he became interested in biometrics because of two
global trends: the growth of densely populated megacities and the
attraction that dense populations hold for terrorists.

In essence, he started FST Biometrics because he wanted to improve urban
security. Although the company has residential, corporate and government
clients, Mr. Farkash's larger motive is to convince average citizens
that face identification is in their best interest. He hopes that people
will agree to have their faces recognized while banking, attending
school, having medical treatments and so on.

If all the ``the good guys'' were to volunteer to be faceprinted, he
theorizes, ``the bad guys'' would stand out as obvious outliers. Mass
public surveillance, Mr. Farkash argues, should make us all safer.

Safer or not, it could have chilling consequences for human behavior.

A private high school in Los Angeles also has an FST system. The school
uses the technology to recognize students when they arrive --- a
security measure intended to keep out unwanted interlopers. But it also
serves to keep the students in line.

``If a girl will come to school at 8:05, the door will not open and she
will be registered as late,'' Mr. Farkash explained. ``So you can use
the system not only for security but for education, for better
discipline.''

\textbf{Faceprints and Civil Liberties}

In February, Dr. Atick was invited to speak at a public meeting on face
recognition convened by the \href{http://www.ntia.doc.gov/}{National
Telecommunications and Information Administration}. It was part of an
agency effort to corral industry executives and consumer advocates into
devising
\href{http://www.nytimes3xbfgragh.onion/2014/02/02/technology/when-no-one-is-just-a-face-in-the-crowd.html}{a
code for the technology's commercial use}.

But some tech industry representatives in attendance were reluctant to
describe their plans or make public commitments to limit face
recognition. Dr. Atick, who was serving on a panel, seemed to take their
silence as an affront to his sense of industry accountability.

``Where is Google? Where is Facebook?'' he loudly asked the audience at
one point.

``Here,'' one voice in the auditorium volunteered. That was about the
only public contribution from the two companies that day.

The agency meetings on face recognition are continuing. In a statement,
Matt Kallman, a Google spokesman, said the company was ``participating
in discussions to advance our view that the industry should make sure
technology is in line with people's expectations.''

A Facebook spokeswoman, Jodi Seth, said in a statement that the company
was participating in the process. ``Multi-stakeholder dialogues like
this are critical to promoting people's privacy,'' she said, ``but until
a code of conduct exists, we can't say whether we will sign it.''

The fundamental concern about faceprinting is the possibility that it
would be used to covertly identify a live person by name --- and then
serve as the link that would connect them, without their awareness or
permission, to intimate details available online, like their home
addresses, dating preferences, employment histories and religious
beliefs. It's not a hypothetical risk. In 2011, researchers at Carnegie
Mellon \href{http://bit.ly/NxlORs}{reported in a study} that they had
used a face-recognition app to identify some students on campus by name,
linking them to their public Facebook profiles and, in some cases, to
their Social Security numbers.

As with many emerging technologies, the arguments tend to coalesce
around two predictable poles: those who think the technology needs rules
and regulation to prevent violations of civil liberties and those who
fear that regulation would stifle innovation. But face recognition
stands out among such technologies: While people can disable smartphone
geolocation and other tracking techniques, they can't turn off their
faces.

``Facial recognition involves the intersection of multiple research
disciplines that have serious consequences for privacy, consumer
protection and human rights,'' wrote Jeffrey Chester, executive director
of the nonprofit Center for Digital Democracy,
\href{http://www.democraticmedia.org/dont-ask-dont-tell-ntia-multistakeholder-group-doesnt-want-know-how-facial-recognition-worksgoogfb-r}{in
a recent blog post}.

``Guidelines at this stage could stymie progress in a very promising
market, and could kill investment,'' Paul Schuepp, the chief executive
of Animetrics, a company that supplies mobile face-recognition systems
to the military, recently
\href{http://animetrics.com/animetrics-present-ntia-meeting/}{wrote on
the company's blog}.

Dr. Atick takes a middle view.

To maintain the status quo around public anonymity, he says, companies
should take a number of steps: They should post public notices where
they use face recognition; seek permission from a consumer before
collecting a faceprint with a unique, repeatable identifier like a name
or code number; and use faceprints only for the specific purpose for
which they have received permission. Those steps, he says, would inhibit
sites, stores, apps and appliances from covertly linking a person in the
real world with their multiple online personas.

``Some people believe that I am maybe inhibiting the industry from
growing. I disagree,'' Dr. Atick told me. `` I am helping industry make
difficult choices, but the right choices.''

Advertisement

\protect\hyperlink{after-bottom}{Continue reading the main story}

\hypertarget{site-index}{%
\subsection{Site Index}\label{site-index}}

\hypertarget{site-information-navigation}{%
\subsection{Site Information
Navigation}\label{site-information-navigation}}

\begin{itemize}
\tightlist
\item
  \href{https://help.nytimes3xbfgragh.onion/hc/en-us/articles/115014792127-Copyright-notice}{©~2020~The
  New York Times Company}
\end{itemize}

\begin{itemize}
\tightlist
\item
  \href{https://www.nytco.com/}{NYTCo}
\item
  \href{https://help.nytimes3xbfgragh.onion/hc/en-us/articles/115015385887-Contact-Us}{Contact
  Us}
\item
  \href{https://www.nytco.com/careers/}{Work with us}
\item
  \href{https://nytmediakit.com/}{Advertise}
\item
  \href{http://www.tbrandstudio.com/}{T Brand Studio}
\item
  \href{https://www.nytimes3xbfgragh.onion/privacy/cookie-policy\#how-do-i-manage-trackers}{Your
  Ad Choices}
\item
  \href{https://www.nytimes3xbfgragh.onion/privacy}{Privacy}
\item
  \href{https://help.nytimes3xbfgragh.onion/hc/en-us/articles/115014893428-Terms-of-service}{Terms
  of Service}
\item
  \href{https://help.nytimes3xbfgragh.onion/hc/en-us/articles/115014893968-Terms-of-sale}{Terms
  of Sale}
\item
  \href{https://spiderbites.nytimes3xbfgragh.onion}{Site Map}
\item
  \href{https://help.nytimes3xbfgragh.onion/hc/en-us}{Help}
\item
  \href{https://www.nytimes3xbfgragh.onion/subscription?campaignId=37WXW}{Subscriptions}
\end{itemize}
