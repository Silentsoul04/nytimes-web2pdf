Sections

SEARCH

\protect\hyperlink{site-content}{Skip to
content}\protect\hyperlink{site-index}{Skip to site index}

\href{https://www.nytimes3xbfgragh.onion/section/food}{Food}

\href{https://myaccount.nytimes3xbfgragh.onion/auth/login?response_type=cookie\&client_id=vi}{}

\href{https://www.nytimes3xbfgragh.onion/section/todayspaper}{Today's
Paper}

\href{/section/food}{Food}\textbar{}Wylie Dufresne Says He Is Forced to
Close WD-50

\url{https://nyti.ms/1kk921w}

\begin{itemize}
\item
\item
\item
\item
\item
\item
\end{itemize}

Advertisement

\protect\hyperlink{after-top}{Continue reading the main story}

Supported by

\protect\hyperlink{after-sponsor}{Continue reading the main story}

\hypertarget{wylie-dufresne-says-he-is-forced-to-close-wd-50}{%
\section{Wylie Dufresne Says He Is Forced to Close
WD-50}\label{wylie-dufresne-says-he-is-forced-to-close-wd-50}}

\includegraphics{https://static01.graylady3jvrrxbe.onion/images/2014/06/11/dining/11wylie/11wylie-articleLarge.jpg?quality=75\&auto=webp\&disable=upscale}

By \href{http://www.nytimes3xbfgragh.onion/by/jeff-gordinier}{Jeff
Gordinier}

\begin{itemize}
\item
  June 10, 2014
\item
  \begin{itemize}
  \item
  \item
  \item
  \item
  \item
  \item
  \end{itemize}
\end{itemize}

\textbf{Updated \textbar{} June 11, 2014}

In the end, a prophet of change was pushed out by change itself.

WD-50, a Lower East Side landmark for modernist cooking and one of the
most influential restaurants in the world, will close at the end of
November.
\href{http://www.nytimes3xbfgragh.onion/2012/05/02/dining/at-wd-50-wylie-dufresne-is-shaking-up-the-entire-menu.html}{Wylie
Dufresne}, the chef whose imaginative vision has propelled the
restaurant for 11 years, announced Tuesday evening on Twitter that Nov.
30 ``will be our final night of service'' at the
\href{http://wd-50.com}{50 Clinton Street} spot. ``Come celebrate with
us for the next 173 days.''

In a phone interview Wednesday morning,
\href{http://dinersjournal.blogs.nytimes3xbfgragh.onion/2012/05/01/wylie-wonkas-food-factory/?ref=dining}{Mr.
Dufresne}, who turned 44 last week, explained that the pioneering
restaurant had succumbed to a classic New York City story: A developer,
\href{http://www.iconrealtymgmt.com/}{Icon Realty Management}, is
planning to put up a new building on the site. ``It's a real estate
thing,'' the chef said. The developer declined to comment for this
article.

So far, Mr. Dufresne hasn't peeked at plans for that building. ``I
haven't been particularly eager to look at what my headstone's going to
look like,'' he said with a rueful laugh.

For a while, the chef and his team thought there might be a way to stay
put while the structure came together around them, but that began to
feel untenable. ``The more I ruminated on that, the more it just didn't
sit well,'' he said. ``We weren't going to be able to give the diners
the same experience we'd been giving them.'' He added that having the
restaurant exist in a construction site, with all the attendant dust and
noise and inconvenience, seemed wrong for an enterprise where every dish
comes across as an edible thought experiment.

Mr. Dufresne said he planned to eventually move his vision to a
different location in the city, although he had no specific spots in
mind. ``I have a great love of fine dining and I hope to continue
that,'' he said. ``We don't quite know yet where we're going. It's like
they say, `You don't have to go home, but you can't stay here.'''

In food-obsessed circles, the shuttering of the WD-50 space --- like
\href{http://www.nytimes3xbfgragh.onion/2011/06/15/dining/el-bulli-is-closing-but-spain-looks-forward.html?pagewanted=all}{the
closing of El Bulli} in Spain in 2011 --- represents the loss of a
history-making culinary laboratory.

Mr. Dufresne is also the gastronomic conceptualist behind
\href{http://www.nytimes3xbfgragh.onion/2013/07/10/dining/reviews/restaurant-review-alder-in-the-east-village.html}{Alder},
an East Village tavern that offers
\href{http://www.nytimes3xbfgragh.onion/2013/02/06/dining/wylie-dufresne-prepares-you-for-a-cubist-spin-on-pub-grub.html}{a
playfully cubist version} of pub grub, but the chef made his mark on the
global food scene with WD-50, an atelier of appetite where diners might
find foie gras that had been turned into delectable aerated puffs and
eggs Benedict broken down into a Lego-like tableau of flavor. At WD-50,
Mr. Dufresne and his team weren't content to spread mayonnaise; instead,
\href{http://www.esquire.com/features/food-drink/ESQ0305EATSCHEF_142}{they
fried it}.

On Tuesday night, testaments to the restaurant's enduring influence were
popping up all over Twitter. Pete Wells, the restaurant critic for The
New York Times, wrote: ``In the future we're going to realize WD-50 was
the CBGB of this era, with way nicer bathrooms.''

``I'm sad about it,''
\href{http://www.nytimes3xbfgragh.onion/2010/07/07/dining/07chef.html?pagewanted=all}{René
Redzepi}, the chef behind Noma, a restaurant in Copenhagen that has been
called the best in the world, said in an email. ``I'm a frequent visitor
in New York, and WD-50 has always been the place to go and see the edgy
side of cooking. It was the place to have your preconceptions
challenged.''

Mr. Redzepi mused that Mr. Dufresne's approach to cooking --- ``wild,
totally unafraid, setting new standards and constantly exploring new
territories'' --- had for a while turned WD-50 into ``perhaps the most
influential restaurant in the world.''

``There was definitely a time, if you were a young cook and you wanted
to be in the know, that you'd be checking out WD-50's website all the
time,'' he said. (In April, Mr. Redzepi and a legion of top-ranked chefs
from around the world
\href{http://www.newyorker.com/online/blogs/culture/2014/04/operation-surprise-wylie.html}{gathered
in New York} for a surprise party in honor of Mr. Dufresne.) Mr. Redzepi
added, ``I've never understood why New York hasn't given him the `key to
the city,' like they have, rightly, to some of his peers who do a nice
pasta or a new burger.''

David Chang, the chef and entrepreneur behind the expanding Momofuku
empire, expressed his disappointment at hearing the WD-50 news in blunt
language, adding, ``I don't know how else to describe it.'' For years,
Mr. Chang went on, Mr. Dufresne ``was doing stuff that no one else was
doing. He was so ahead of the curve that people took it for granted.''

Although clearly wistful and emotional about the closing, Mr. Dufresne
said he would not snuff out WD-50's experimental spirit in the last few
months of its existence. ``We are going to continue to push and continue
to do new stuff,'' he said. ``We're still actively working on new
ideas.'' The restaurant may also opt for a dash of crowdsourcing, using
social media to ask diners which famous dishes from the early years
they'd like to see resurrected for the final stretch.

And although there won't be another New Year's Eve party at WD-50,
``maybe we'll do New Year's Eve in October,'' Mr. Dufresne said. ``We
want this to be fun. We want this to be a celebration.''

Beyond his scientific bent in the kitchen, Mr. Dufresne is known for
having an eye for young talent. Many veterans of the restaurant's
kitchen and bar, such as Christina Tosi, Mario Carbone, Alex Stupak and
Paul Carmichael, have become forces in cooking around New York City.

Mr. Dufresne is viewed as a pioneer, too, when it comes to bringing
\href{http://www.nytimes3xbfgragh.onion/2008/03/05/dining/reviews/05rest.html?pagewanted=all}{three-star
dining} to a somewhat dingy patch of downtown Manhattan. ``Wylie helped
change the Lower East Side completely,'' Mr. Chang said. ``If any other
restaurant is there now, it's because Wylie was there first.''

In fact, it could be argued that WD-50's presence in that part of town
helped usher in the gentrification that is now forcing it out. ``This is
unfortunately what's probably going to happen to almost everybody,
unless you can buy the building,'' Mr. Chang said.

Mr. Dufresne appeared to be approaching the shift with equanimity.
``That's the story of New York,'' he said. ``Neighborhoods change. In
some ways it's part of the beauty of New York City. It's in a constant
state of flux.''

Advertisement

\protect\hyperlink{after-bottom}{Continue reading the main story}

\hypertarget{site-index}{%
\subsection{Site Index}\label{site-index}}

\hypertarget{site-information-navigation}{%
\subsection{Site Information
Navigation}\label{site-information-navigation}}

\begin{itemize}
\tightlist
\item
  \href{https://help.nytimes3xbfgragh.onion/hc/en-us/articles/115014792127-Copyright-notice}{©~2020~The
  New York Times Company}
\end{itemize}

\begin{itemize}
\tightlist
\item
  \href{https://www.nytco.com/}{NYTCo}
\item
  \href{https://help.nytimes3xbfgragh.onion/hc/en-us/articles/115015385887-Contact-Us}{Contact
  Us}
\item
  \href{https://www.nytco.com/careers/}{Work with us}
\item
  \href{https://nytmediakit.com/}{Advertise}
\item
  \href{http://www.tbrandstudio.com/}{T Brand Studio}
\item
  \href{https://www.nytimes3xbfgragh.onion/privacy/cookie-policy\#how-do-i-manage-trackers}{Your
  Ad Choices}
\item
  \href{https://www.nytimes3xbfgragh.onion/privacy}{Privacy}
\item
  \href{https://help.nytimes3xbfgragh.onion/hc/en-us/articles/115014893428-Terms-of-service}{Terms
  of Service}
\item
  \href{https://help.nytimes3xbfgragh.onion/hc/en-us/articles/115014893968-Terms-of-sale}{Terms
  of Sale}
\item
  \href{https://spiderbites.nytimes3xbfgragh.onion}{Site Map}
\item
  \href{https://help.nytimes3xbfgragh.onion/hc/en-us}{Help}
\item
  \href{https://www.nytimes3xbfgragh.onion/subscription?campaignId=37WXW}{Subscriptions}
\end{itemize}
