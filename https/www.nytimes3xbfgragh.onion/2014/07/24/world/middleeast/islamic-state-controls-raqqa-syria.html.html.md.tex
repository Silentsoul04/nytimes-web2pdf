Sections

SEARCH

\protect\hyperlink{site-content}{Skip to
content}\protect\hyperlink{site-index}{Skip to site index}

\href{https://www.nytimes3xbfgragh.onion/section/world/middleeast}{Middle
East}

\href{https://myaccount.nytimes3xbfgragh.onion/auth/login?response_type=cookie\&client_id=vi}{}

\href{https://www.nytimes3xbfgragh.onion/section/todayspaper}{Today's
Paper}

\href{/section/world/middleeast}{Middle East}\textbar{}Life in a
Jihadist Capital: Order With a Darker Side

\url{https://nyti.ms/1z2sB8q}

\begin{itemize}
\item
\item
\item
\item
\item
\item
\end{itemize}

Advertisement

\protect\hyperlink{after-top}{Continue reading the main story}

Supported by

\protect\hyperlink{after-sponsor}{Continue reading the main story}

\hypertarget{life-in-a-jihadist-capital-order-with-a-darker-side}{%
\section{Life in a Jihadist Capital: Order With a Darker
Side}\label{life-in-a-jihadist-capital-order-with-a-darker-side}}

\includegraphics{https://static01.graylady3jvrrxbe.onion/images/2014/07/23/world/23RAQQA/23RAQQA-videoSixteenByNine1050-v3.jpg}

By an Employee of The New York Times and
\href{http://www.nytimes3xbfgragh.onion/by/ben-hubbard}{Ben Hubbard}

\begin{itemize}
\item
  July 23, 2014
\item
  \begin{itemize}
  \item
  \item
  \item
  \item
  \item
  \item
  \end{itemize}
\end{itemize}

RAQQA, Syria --- When his factory was bombed in the northern Syrian city
of Aleppo, the businessman considered two bleak options: remain at home
and risk dying in the next airstrike, or flee like hundreds of thousands
of others to a refugee camp in Turkey.

Instead, he took his remaining cash east and moved to a neighboring
city, Raqqa, the de facto capital of the world's fastest growing
jihadist force. There he found a degree of order and security absent in
other parts of Syria.

``The fighting in Syria will continue, so we have to live our lives,''
said the businessman, who gave only a first name, Qadri, as he oversaw a
dozen workers in his new children's clothing factory in Raqqa.

Long before extremists rolled through Iraq and seized a large piece of
territory, the group known as the Islamic State in Iraq and Syria, or
ISIS, took over most of Raqqa Province, home to about a million people,
and established a headquarters in its capital. Through strategic
management and brute force, the group, which now calls itself simply the
Islamic State, has begun imposing its vision of a state that blends its
fundamentalist interpretation of Islam with the practicalities of
governance.

In time, it has won the surprising respect of some war-weary citizens,
like Qadri, who will accept any authority that can restore a semblance
of normal life. Rebel-held areas of Aleppo, by comparison, remain racked
with food shortages and crime. But there is a darker side to Islamic
rule, with public executions and strict social codes that have left many
in this once-tolerant community deeply worried about the future.

\href{https://www.nytimes3xbfgragh.onion/interactive/2014/07/03/world/middleeast/syria-iraq-isis-rogue-state-along-two-rivers.html}{}

\includegraphics{https://static01.graylady3jvrrxbe.onion/images/2014/07/03/world/middleeast/syria-iraq-isis-rogue-state-along-two-rivers-1404368464154/syria-iraq-isis-rogue-state-along-two-rivers-1404368464154-videoLarge.png}

\hypertarget{a-rogue-state-along-two-rivers}{%
\subsection{A Rogue State Along Two
Rivers}\label{a-rogue-state-along-two-rivers}}

The victories gained by the militant group calling itself the Islamic
State in Iraq and Syria were built on months of maneuvering along the
Tigris and Euphrates Rivers.

In the city of Raqqa, traffic police officers keep intersections clear,
crime is rare, and tax collectors issue receipts. But statues like the
landmark lions in Al Rasheed Park have been destroyed because they were
considered blasphemous. Public spaces like Al Amasy Square, where young
men and women once hung out and flirted in the evenings, have been
walled off with heavy metal fences topped with the black flags of ISIS.
People accused of stealing have lost their hands in public amputations.

``What I see in Raqqa proves that the Islamic State has a clear vision
to establish a state in the real meaning of the word,'' said a retired
teacher in the city of Raqqa. ``It is not a joke.''

How ISIS rules in Raqqa offers insight into what it is trying to do as
it moves to consolidate its grip in territories spanning the
Syrian-Iraqi border. An employee of The New York Times recently spent
six days in Raqqa and interviewed a dozen residents. The employee and
those interviewed are not being identified to protect them from
retaliation by the extremists who have hunted down and killed those
perceived as opposing their project.

To those entering Raqqa, ISIS makes clear, immediately, who is in
charge.

At the southern entrance to the city, visitors were once greeted by a
towering mosaic of President Bashar al-Assad and Haroun al-Rasheed, the
caliph who ruled the Islamic world from Raqqa in the ninth century. Now
there is a towering black billboard that pays homage to ISIS and to the
so-called martyrs who died fighting for its cause.

Raqqa's City Hall houses the Islamic Services Commission. The former
office of the Finance Ministry contains the Shariah court and the
criminal police. The traffic police are based in the
\href{https://twitter.com/al_khansaa2/status/465938930878398465/photo/1}{First
Shariah High School}. Raqqa's Credit Bank is now the tax authority,
where employees collect \$20 every two months from shop owners for
electricity, water and security. Many said that they had received
official receipts stamped with the ISIS logo and that the fees were less
than they used to pay in bribes to Mr. Assad's government.

``I feel like I am dealing with a respected state, not thugs,'' said a
Raqqa goldsmith in his small shop as a woman shopped for gold pieces
with cash sent from abroad by her husband.

Raqqa is a test case for ISIS, which imposed itself as the ultimate
authority in this city on the Euphrates River early this year. The group
has already proved its military prowess, routing other militias in Syria
as well as the Iraqi military. But it is here in this agricultural hub
that it has had the most time to turn its ideology into reality, a
project that appears unlikely to end soon given the lack of a military
force able to displace it.

An aid worker who travels to Raqqa said the ranks of ISIS were filled
with volatile young men, many of them foreigners more interested in
violence than governance. To keep things running, it has paid or
threatened skilled workers to remain in their posts while putting
loyalist supervisors over them to ensure compliance with Islamic rules.

``They can't fire all the staff and bring new people to run a hospital,
so they change the manager to someone who will enforce their rules and
regulations,'' the aid worker said, speaking on the condition of
anonymity so as not to endanger his work.

Raqqa's three churches, once home to an active Christian minority, have
all been shuttered. After capturing the largest, the Armenian Catholic
Martyrs Church, ISIS removed its crosses, hung black flags from its
facade and converted it into an
\href{http://www.alquds.co.uk/wp-content/uploads/picdata/2013/12/12-03/03qpt963.jpg}{Islamic
center} that screens videos of battles and suicide operations to recruit
new fighters.

The few Christians who remain pay a minority tax of a few dollars per
month. When ISIS's religious police officers patrol to make sure shops
close during Muslim prayers, the Christians must obey, too.

The religious police have banned public smoking of cigarettes and water
pipes --- a move that has dampened the city's social life, forcing cafes
to close. They also make sure that women cover their hair and faces in
public.

\includegraphics{https://static01.graylady3jvrrxbe.onion/images/2014/07/24/world/24raqqa2/24raqqa2-articleLarge.jpg?quality=75\&auto=webp\&disable=upscale}

A university professor from Raqqa said ISIS gunmen recently stopped a
bus heading to Damascus when they found one woman on board
insufficiently covered. They held the bus up for an hour and a half
until she went home and changed, the professor said.

More pragmatically, ISIS has managed to keep food in markets, and
bakeries and gas stations functioning. But it has had more trouble with
drinking water and electricity, which is out for as much as 20 hours a
day.

Perhaps realizing that the young extremists most attracted to its
sectarian violence lack professional skills, the leader of ISIS, Abu
Bakr al-Baghdadi, asked in a recent
\href{https://pietervanostaeyen.wordpress.com/2014/07/01/islamic-state-message-to-the-mujahidin-and-the-islamic-ummah-in-the-month-of-ramadan/}{audio
address} for doctors and engineers to travel to places like Raqqa to
help build his
\href{http://www.nytimes3xbfgragh.onion/2014/07/06/world/asia/iraq-abu-bakr-al-baghdadi-sermon-video.html?_r=0}{newly
declared} Islamic State. ``Their migration is an obligation so that they
can answer the dire need of the Muslims,'' Mr. Baghdadi said.

Hints of this international mobilization are already apparent in Raqqa,
where gunmen at checkpoints are often Saudi, Egyptian, Tunisian or
Libyan. Raqqa's emir of electricity is Sudanese, and one hospital is run
by a Jordanian who reports to an Egyptian boss, according to Syrians who
work under them.

After ISIS's advance into Iraq last month, the Jordanian went to Mosul
to help organize a hospital there before returning to Raqqa.

``He talked with an eager shine in his eyes, saying that the caliphate
of the Islamic State that began in Raqqa would spread over the whole
region,'' one of his employees said.

Advertisement

\protect\hyperlink{after-bottom}{Continue reading the main story}

\hypertarget{site-index}{%
\subsection{Site Index}\label{site-index}}

\hypertarget{site-information-navigation}{%
\subsection{Site Information
Navigation}\label{site-information-navigation}}

\begin{itemize}
\tightlist
\item
  \href{https://help.nytimes3xbfgragh.onion/hc/en-us/articles/115014792127-Copyright-notice}{©~2020~The
  New York Times Company}
\end{itemize}

\begin{itemize}
\tightlist
\item
  \href{https://www.nytco.com/}{NYTCo}
\item
  \href{https://help.nytimes3xbfgragh.onion/hc/en-us/articles/115015385887-Contact-Us}{Contact
  Us}
\item
  \href{https://www.nytco.com/careers/}{Work with us}
\item
  \href{https://nytmediakit.com/}{Advertise}
\item
  \href{http://www.tbrandstudio.com/}{T Brand Studio}
\item
  \href{https://www.nytimes3xbfgragh.onion/privacy/cookie-policy\#how-do-i-manage-trackers}{Your
  Ad Choices}
\item
  \href{https://www.nytimes3xbfgragh.onion/privacy}{Privacy}
\item
  \href{https://help.nytimes3xbfgragh.onion/hc/en-us/articles/115014893428-Terms-of-service}{Terms
  of Service}
\item
  \href{https://help.nytimes3xbfgragh.onion/hc/en-us/articles/115014893968-Terms-of-sale}{Terms
  of Sale}
\item
  \href{https://spiderbites.nytimes3xbfgragh.onion}{Site Map}
\item
  \href{https://help.nytimes3xbfgragh.onion/hc/en-us}{Help}
\item
  \href{https://www.nytimes3xbfgragh.onion/subscription?campaignId=37WXW}{Subscriptions}
\end{itemize}
