Sections

SEARCH

\protect\hyperlink{site-content}{Skip to
content}\protect\hyperlink{site-index}{Skip to site index}

\href{https://www.nytimes3xbfgragh.onion/section/science}{Science}

\href{https://myaccount.nytimes3xbfgragh.onion/auth/login?response_type=cookie\&client_id=vi}{}

\href{https://www.nytimes3xbfgragh.onion/section/todayspaper}{Today's
Paper}

\href{/section/science}{Science}\textbar{}The Odds, Continually Updated

\url{https://nyti.ms/YIzkqv}

\begin{itemize}
\item
\item
\item
\item
\item
\end{itemize}

Advertisement

\protect\hyperlink{after-top}{Continue reading the main story}

Supported by

\protect\hyperlink{after-sponsor}{Continue reading the main story}

\hypertarget{the-odds-continually-updated}{%
\section{The Odds, Continually
Updated}\label{the-odds-continually-updated}}

\includegraphics{https://static01.graylady3jvrrxbe.onion/images/2014/09/30/science/30BAYES/30BAYES-articleLarge.jpg?quality=75\&auto=webp\&disable=upscale}

By F. D. Flam

\begin{itemize}
\item
  Sept. 29, 2014
\item
  \begin{itemize}
  \item
  \item
  \item
  \item
  \item
  \end{itemize}
\end{itemize}

Statistics may not sound like the most heroic of pursuits. But if not
for statisticians, a Long Island fisherman might have died in the
Atlantic Ocean after falling off his boat early one morning last summer.

The man owes his life to a once obscure field known as Bayesian
statistics --- a set of mathematical rules for using new data to
continuously update beliefs or existing knowledge.

The method was invented in the 18th century by an English Presbyterian
minister named Thomas Bayes --- by some accounts to calculate the
probability of God's existence. In this century, Bayesian statistics has
grown vastly more useful because of the kind of advanced computing power
that did not exist even 20 years ago.

It is proving especially useful in approaching complex problems,
including searches like the one the Coast Guard used in 2013 to find the
\href{http://www.nytimes3xbfgragh.onion/2014/01/05/magazine/a-speck-in-the-sea.html?_r=0}{missing
fisherman}, John Aldridge (though not, so far, in the hunt for Malaysia
Airlines Flight 370).

Now Bayesian statistics are rippling through everything from physics to
cancer research, ecology to psychology. Enthusiasts say they are
allowing scientists to solve problems that would have been considered
impossible just 20 years ago. And lately, they have been thrust into an
intense debate over the reliability of research results.

When people think of statistics, they may imagine lists of numbers ---
batting averages or life-insurance tables. But the current debate is
about how scientists turn data into knowledge, evidence and predictions.
Concern has been growing in recent years that some fields are not doing
a very good job at this sort of inference. In 2012, for example, a team
at the biotech company Amgen
\href{http://www.nature.com/nature/journal/v483/n7391/full/483531a.html}{announced
it had analyzed 53 cancer studies and found it could not replicate 47 of
them}.

Similar follow-up analyses have cast doubt on so many findings in fields
such as
\href{http://www.nature.com/nrn/journal/v14/n5/abs/nrn3475.html}{neuroscience}
and social science that researchers talk about a ``replication crisis''

Some statisticians and scientists are optimistic that Bayesian methods
can improve the reliability of research by allowing scientists to
crosscheck work done with the more traditional or ``classical''
approach, known as frequentist statistics. The two methods approach the
same problems from different angles.

The essence of the frequentist technique is to apply probability to
data. If you suspect your friend has a weighted coin, for example, and
you observe that it came up heads nine times out of 10, a frequentist
would calculate the probability of getting such a result with an
unweighted coin. The answer (about 1 percent) is not a direct measure of
the probability that the coin is weighted; it's a measure of how
improbable the nine-in-10 result is --- a piece of information that can
be useful in investigating your suspicion.

Image

Thomas Bayes

By contrast, Bayesian calculations go straight for the probability of
the hypothesis, factoring in not just the data from the coin-toss
experiment but any other relevant information --- including whether you
have previously seen your friend use a weighted coin.

Scientists who have learned Bayesian statistics often marvel that it
propels them through a different kind of scientific reasoning than they
had experienced using classical methods.

``Statistics sounds like this dry, technical subject, but it draws on
deep philosophical debates about the nature of reality,'' said the
Princeton University astrophysicist Edwin Turner, who has witnessed a
widespread conversion to Bayesian thinking in his field over the last 15
years.

\textbf{Countering Pure Objectivity}

Frequentist statistics became the standard of the 20th century by
promising just-the-facts objectivity, unsullied by beliefs or biases. In
the 2003 statistics primer ``Dicing With Death,'' Stephen Senn traces
the technique's roots to 18th-century England, when a physician named
John Arbuthnot set out to calculate the ratio of male to female births.

Arbuthnot gathered christening records from 1629 to 1710 and found that
in London, a few more boys were recorded every year. He then calculated
the odds that such an 82-year run could occur simply by chance, and
found that it was one in trillions. This frequentist calculation can't
tell them what is causing the sex ratio to be skewed. Arbuthnot proposed
that God skewed the birthrates to balance the higher mortality that had
been observed among boys, but scientists today favor a
\href{http://www.livescience.com/33491-male-female-sex-ratio.html}{biological
explanation}over a theological one.

Later in the 1700s, the mathematician and astronomer Daniel Bernoulli
used a similar technique to investigate the curious geometry of the
solar system, in which planets orbit the sun in a flat, pancake-shaped
plane. If the orbital angles were purely random --- with Earth, say, at
zero degrees, Venus at 45 and Mars at 90 --- the solar system would look
more like a sphere than a pancake. But Bernoulli calculated that all the
planets known at the time orbited within seven degrees of the plane,
known as the ecliptic.

What were the odds of that? Bernoulli's calculations put them at about
one in 13 million. Today, this kind of number is called a p-value, the
probability that an observed phenomenon or one more extreme could have
occurred by chance. Results are usually considered ``statistically
significant'' if the p-value is less than 5 percent.

But there is a danger in this tradition, said Andrew Gelman, a
statistics professor at Columbia. Even if scientists always did the
calculations correctly --- and they don't, he argues --- accepting
everything with a p-value of 5 percent means that one in 20
``statistically significant'' results are nothing but random noise.

The proportion of wrong results published in prominent journals is
probably even higher, he said, because such findings are often
surprising and appealingly counterintuitive, said Dr. Gelman, an
occasional contributor to Science Times.

\textbf{Looking at Other Factors}

Take, for instance,
\href{http://pss.sagepub.com/content/early/2013/04/23/0956797612466416.abstract}{a
study} concluding that single women who were ovulating were 20 percent
more likely to vote for President Obama in 2012 than those who were not.
(In married women, the effect was reversed.)

\includegraphics{https://static01.graylady3jvrrxbe.onion/images/2014/09/30/science/30JPBAYES1/30JPBAYES1-articleLarge.jpg?quality=75\&auto=webp\&disable=upscale}

Dr. Gelman re-evaluated the study using Bayesian statistics. That
allowed him to look at probability not simply as a matter of results and
sample sizes, but in the light of other information that could affect
those results.

He factored in data showing that people rarely change their voting
preference over an election cycle, let alone a menstrual cycle. When he
did, the study's statistical significance evaporated. (The paper's lead
author, Kristina M. Durante of the University of Texas, San Antonio,
said she stood by the finding.)

Dr. Gelman said the results would not have been considered statistically
significant had the researchers used the frequentist method properly. He
suggests using Bayesian calculations not necessarily to replace
classical statistics but to flag spurious results.

A famously counterintuitive puzzle that lends itself to a Bayesian
approach is the
\href{http://www.nytimes3xbfgragh.onion/2008/04/08/science/08tier.html}{Monty
Hall problem}, in which Mr. Hall, longtime host of the game show ``Let's
Make a Deal,'' hides a car behind one of three doors and a goat behind
each of the other two. The contestant picks Door No. 1, but before
opening it, Mr. Hall opens Door No. 2 to reveal a goat. Should the
contestant stick with No. 1 or switch to No. 3, or does it matter?

A Bayesian calculation would start with one-third odds that any given
door hides the car, then update that knowledge with the new data: Door
No. 2 had a goat. The odds that the contestant guessed right --- that
the car is behind No. 1 --- remain one in three. Thus, the odds that she
guessed wrong are two in three. And if she guessed wrong, the car must
be behind Door No. 3. So she should indeed switch.

In other fields, researchers are using Bayesian statistics to tackle
problems of formidable complexity. The New York University
astrophysicist David Hogg credits Bayesian statistics with narrowing
down the age of the universe. As recently as the late 1990s, astronomers
could say only that it was eight billion to 15 billion years; now,
factoring in supernova explosions, the distribution of galaxies and
patterns seen in radiation left over from the Big Bang, they have
concluded with some confidence that the number is 13.8 billion years.

Bayesian reasoning combined with advanced computing power has also
revolutionized the search for planets orbiting distant stars, said Dr.
Turner, the Princeton astrophysicist.

In most cases, astronomers can't see these planets; their light is
drowned out by the much brighter stars they orbit. What the scientists
can see are slight variations in starlight; from these glimmers, they
can judge whether planets are passing in front of a star or causing it
to wobble from their gravitational tug.

Making matters more complicated, the size of the apparent wobbles
depends on whether astronomers are observing a planet's orbit edge-on or
from some other angle. But by factoring in data from a growing list of
known planets, the scientists can deduce the most probable properties of
new planets.

One downside of Bayesian statistics is that it requires prior
information --- and often scientists need to start with a guess or
estimate. Assigning numbers to subjective judgments is ``like
fingernails on a chalkboard,'' said physicist Kyle Cranmer, who helped
develop a frequentist technique to identify the latest new subatomic
particle --- the Higgs boson.

Image

Andrew Gelman, a statistics professor at Columbia, says the Bayesian
method is good for flagging erroneous conclusions.Credit...Jingchen Liu

Others say that in confronting the so-called replication crisis, the
best cure for misleading findings is not Bayesian statistics, but good
frequentist ones. It was frequentist statistics that allowed people to
uncover all the problems with irreproducible research in the first
place, said Deborah Mayo, a philosopher of science at Virginia Tech. The
technique was developed to distinguish real effects from chance, and to
prevent scientists from fooling themselves.

Uri Simonsohn, a psychologist at the University of Pennsylvania, agrees.
Several years ago, he published a paper that exposed common statistical
shenanigans in his field --- logical leaps, unjustified conclusions, and
various forms of unconscious and conscious cheating.

He said he had looked into Bayesian statistics and concluded that if
people misused or misunderstood one system, they would do just as badly
with the other. Bayesian statistics, in short, can't save us from bad
science.

\textbf{At Times a Lifesaver}

Despite its 18th-century origins, the technique is only now beginning to
reveal its power with the advent of 21st-century computing speed.

Some historians say Bayes developed his technique to counter the
philosopher David Hume's contention that most so-called miracles were
likely to be fakes or illusions. Bayes didn't make much headway in that
debate --- at least not directly.

But even Hume might have been impressed last year, when the Coast Guard
used Bayesian statistics to search for Mr. Aldridge, its computers
continually updating and narrowing down his most probable locations.

The Coast Guard has been using Bayesian analysis since the 1970s. The
approach lends itself well to problems like searches, which involve a
single incident and many different kinds of relevant data, said Lawrence
Stone, a statistician for Metron, a scientific consulting firm in
Reston, Va., that works with the Coast Guard.

At first, all the Coast Guard knew about the fisherman was that he fell
off his boat sometime from 9 p.m. on July 24 to 6 the next morning. The
sparse information went into a program called Sarops, for Search and
Rescue Optimal Planning System. Over the next few hours, searchers added
new information --- on prevailing currents, places the search
helicopters had already flown and some additional clues found by the
boat's captain.

The system could not deduce exactly where Mr. Aldridge was drifting, but
with more information, it continued to narrow down the most promising
places to search.

Just before turning back to refuel, a searcher in a helicopter spotted a
man clinging to two buoys he had tied together. He had been in the water
for 12 hours; he was hypothermic and sunburned but alive.

Even in the jaded 21st century, it was considered something of a
miracle.

Advertisement

\protect\hyperlink{after-bottom}{Continue reading the main story}

\hypertarget{site-index}{%
\subsection{Site Index}\label{site-index}}

\hypertarget{site-information-navigation}{%
\subsection{Site Information
Navigation}\label{site-information-navigation}}

\begin{itemize}
\tightlist
\item
  \href{https://help.nytimes3xbfgragh.onion/hc/en-us/articles/115014792127-Copyright-notice}{©~2020~The
  New York Times Company}
\end{itemize}

\begin{itemize}
\tightlist
\item
  \href{https://www.nytco.com/}{NYTCo}
\item
  \href{https://help.nytimes3xbfgragh.onion/hc/en-us/articles/115015385887-Contact-Us}{Contact
  Us}
\item
  \href{https://www.nytco.com/careers/}{Work with us}
\item
  \href{https://nytmediakit.com/}{Advertise}
\item
  \href{http://www.tbrandstudio.com/}{T Brand Studio}
\item
  \href{https://www.nytimes3xbfgragh.onion/privacy/cookie-policy\#how-do-i-manage-trackers}{Your
  Ad Choices}
\item
  \href{https://www.nytimes3xbfgragh.onion/privacy}{Privacy}
\item
  \href{https://help.nytimes3xbfgragh.onion/hc/en-us/articles/115014893428-Terms-of-service}{Terms
  of Service}
\item
  \href{https://help.nytimes3xbfgragh.onion/hc/en-us/articles/115014893968-Terms-of-sale}{Terms
  of Sale}
\item
  \href{https://spiderbites.nytimes3xbfgragh.onion}{Site Map}
\item
  \href{https://help.nytimes3xbfgragh.onion/hc/en-us}{Help}
\item
  \href{https://www.nytimes3xbfgragh.onion/subscription?campaignId=37WXW}{Subscriptions}
\end{itemize}
