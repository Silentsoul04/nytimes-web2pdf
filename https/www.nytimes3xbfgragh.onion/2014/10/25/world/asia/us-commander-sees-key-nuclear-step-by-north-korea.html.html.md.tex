Sections

SEARCH

\protect\hyperlink{site-content}{Skip to
content}\protect\hyperlink{site-index}{Skip to site index}

\href{https://www.nytimes3xbfgragh.onion/section/world/asia}{Asia
Pacific}

\href{https://myaccount.nytimes3xbfgragh.onion/auth/login?response_type=cookie\&client_id=vi}{}

\href{https://www.nytimes3xbfgragh.onion/section/todayspaper}{Today's
Paper}

\href{/section/world/asia}{Asia Pacific}\textbar{}U.S. Commander Sees
Key Nuclear Step by North Korea

\url{https://nyti.ms/ZMXXmr}

\begin{itemize}
\item
\item
\item
\item
\item
\end{itemize}

Advertisement

\protect\hyperlink{after-top}{Continue reading the main story}

Supported by

\protect\hyperlink{after-sponsor}{Continue reading the main story}

\hypertarget{us-commander-sees-key-nuclear-step-by-north-korea}{%
\section{U.S. Commander Sees Key Nuclear Step by North
Korea}\label{us-commander-sees-key-nuclear-step-by-north-korea}}

By \href{http://www.nytimes3xbfgragh.onion/by/david-e-sanger}{David E.
Sanger}

\begin{itemize}
\item
  Oct. 24, 2014
\item
  \begin{itemize}
  \item
  \item
  \item
  \item
  \item
  \end{itemize}
\end{itemize}

WASHINGTON --- The top American military commander in South Korea said
on Friday that he believed North Korea had most likely completed its
yearslong quest to shrink a nuclear weapon to a size that could fit atop
a ballistic missile. His assessment, if correct, could change American
calculations about the vulnerability of the United States and its
allies, and the North's ability to sell nuclear weapons to others.

At a Pentagon news conference, Gen. Curtis M. Scaparrotti cautioned that
the North had not yet tested a miniaturized weapon, and for a weapon
``that complex, without it being tested, the probability of it being
effective is pretty darn low.'' But he made clear that based on all he
had learned, ``they have the capability to have miniaturized the device
at this point.''

That has long been a disputed question. For years, American intelligence
agencies have been scouring the evidence --- from satellite photographs,
human spies, intercepted calls and computer transmissions, and the
tracking of nuclear suppliers --- in an effort to assess when the North
would be capable of marrying its nuclear and missile programs. While the
North has conducted three underground nuclear tests, the first was
considered a dud, and the others may have been large devices that could
not fit atop the Nodong missiles that can reach Japan and South Korea,
or the intercontinental missiles North Korea hopes to develop.

But General Scaparrotti's assessment seemed to suggest that the North
had made progress since last year, when President Obama
\href{http://www.nytimes3xbfgragh.onion/2013/04/17/us/politics/obama-voices-doubts-on-north-korean-nuclear-warhead.html?_r=0\&pagewanted=print}{appeared
to contradict} a Defense Intelligence Agency finding about the country's
ability to put a warhead on a missile.

At that time, the agency issued a report that it had
\href{http://www.nytimes3xbfgragh.onion/2013/04/12/world/asia/north-korea-may-have-nuclear-missile-capability-us-agency-says.html?pagewanted=all\&module=Search\&mabReward=relbias\%3As\%2C\%7B\%221\%22\%3A\%22RI\%3A8\%22\%7D}{``moderate
confidence''} that the North had mastered the technology of building a
weapon that could fit into a missile warhead. That forced James R.
Clapper Jr., the director of national intelligence, to issue a statement
that the agency's position was not the consensus view of 15 other
intelligence agencies. Mr. Obama agreed, saying, ``You know, based on
our current intelligence assessment, we do not think that they have that
capacity.''

On Friday, General Scaparrotti sided with the Defense Intelligence
Agency. ``I believe they have the capability to have miniaturized the
device at this point, and they have the technology to potentially
actually deliver what they say they have,'' he told reporters.

It was unclear if he was basing that on new intelligence. In the 18
months since the president's statement, the United States has focused
intently on gathering new intelligence about the North's capabilities
and the intentions of Kim Jong-un, its young leader who just resurfaced
after a lengthy and still unexplained absence.

But even if General Scaparrotti is correct, it does not mean that the
North is ready to threaten the United States with a nuclear-tipped
missile. While the North has successfully tested its medium-range
missiles, and equipped them with re-entry vehicles, it has not achieved
its goal of successfully test-flying an intercontinental ballistic
missile. Its biggest accomplishment has been popping a tiny satellite
into space. And even if the North could mount a weapon on top of a
missile, experts note, there would be no assurance it could deliver a
warhead to a target.

``Re-entry is a real challenge,'' said Gary Samore, a Harvard scholar
who served as Mr. Obama's top adviser on weapons of mass destruction
during his first term. ``There is a lot of heat, and a lot of
vibration'' as a warhead re-enters the atmosphere, aimed at its target.
``You have to do live testing to see if it works,'' he said. ``It's not
something you can do through simulation.''

But for the North, the missile and nuclear technology may not be
intended as much for military use as for a bargaining chip --- the
leaders presumably understand what would follow if they actually
attacked the United States or one of its treaty allies in the Pacific.
Each nuclear and missile test is meant to show that seven decades of
sanctions and containment have failed. And each one amounts to an
advertisement for the world's most destructive weapons.

Among North Korea's biggest customers for missile technology is Iran.
While there is no evidence that the North has ever sold nuclear
technology to the Tehran government, it supplied a nuclear reactor to
Syria. The reactor was destroyed in a September 2007 attack by Israel.

A warhead, even an untested one, could become the ultimate export for a
starving nation. But it would also be a huge risk for the North;
President George W. Bush, soon after the North's first nuclear test
eight years ago, warned the country that it would be held responsible
for any nuclear incident in which its weapons were used.

Mr. Obama had Robert M. Gates, then the defense secretary, issue a
similar warning. But the administration's strategy has been to largely
ignore the North, refusing to acknowledge it as a nuclear state or
re-engage in negotiations that Mr. Gates warned could amount to ``buying
the same horse again,'' meaning making concessions for another temporary
halt in the nuclear program, or resumed inspections.

``We remain open to dialogue with North Korea, but there is no value in
talks just for the sake of talks,'' Secretary of State John Kerry said
on Friday after meeting with South Korea's foreign minister, Yun
Byung-se, at the State Department. ``North Korea must demonstrate that
it is serious about denuclearization,'' he said.

The North, for its part, has been by turns seeking new talks and issuing
statements that its nuclear capabilities are here to stay, and will be
steadily improved. In the past week alone, it has opened fire along the
demilitarized zone and released one of three Americans being held on
thin charges. ``What they speak and what they do seem to be
inconsistent,'' Mr. Yun said on Friday.

Mr. Kerry has suggested that the United States was looking for ways to
re-engage with the country, though such efforts have always been treated
with skepticism at the White House. That did not deter him on Friday.
``The mere entering into talks is not an invitation to take any actions
regarding troops'' that the North wants removed from South Korea, ``or
anything else at this point,'' Mr. Kerry said. ``The first thing you
have to do is come to a competent, real, authentic set of talks about
denuclearization,'' he said, ``and that is the prerequisite.''

Advertisement

\protect\hyperlink{after-bottom}{Continue reading the main story}

\hypertarget{site-index}{%
\subsection{Site Index}\label{site-index}}

\hypertarget{site-information-navigation}{%
\subsection{Site Information
Navigation}\label{site-information-navigation}}

\begin{itemize}
\tightlist
\item
  \href{https://help.nytimes3xbfgragh.onion/hc/en-us/articles/115014792127-Copyright-notice}{©~2020~The
  New York Times Company}
\end{itemize}

\begin{itemize}
\tightlist
\item
  \href{https://www.nytco.com/}{NYTCo}
\item
  \href{https://help.nytimes3xbfgragh.onion/hc/en-us/articles/115015385887-Contact-Us}{Contact
  Us}
\item
  \href{https://www.nytco.com/careers/}{Work with us}
\item
  \href{https://nytmediakit.com/}{Advertise}
\item
  \href{http://www.tbrandstudio.com/}{T Brand Studio}
\item
  \href{https://www.nytimes3xbfgragh.onion/privacy/cookie-policy\#how-do-i-manage-trackers}{Your
  Ad Choices}
\item
  \href{https://www.nytimes3xbfgragh.onion/privacy}{Privacy}
\item
  \href{https://help.nytimes3xbfgragh.onion/hc/en-us/articles/115014893428-Terms-of-service}{Terms
  of Service}
\item
  \href{https://help.nytimes3xbfgragh.onion/hc/en-us/articles/115014893968-Terms-of-sale}{Terms
  of Sale}
\item
  \href{https://spiderbites.nytimes3xbfgragh.onion}{Site Map}
\item
  \href{https://help.nytimes3xbfgragh.onion/hc/en-us}{Help}
\item
  \href{https://www.nytimes3xbfgragh.onion/subscription?campaignId=37WXW}{Subscriptions}
\end{itemize}
