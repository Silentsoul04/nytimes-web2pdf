Sections

SEARCH

\protect\hyperlink{site-content}{Skip to
content}\protect\hyperlink{site-index}{Skip to site index}

\href{https://www.nytimes3xbfgragh.onion/section/business/economy}{Economy}

\href{https://myaccount.nytimes3xbfgragh.onion/auth/login?response_type=cookie\&client_id=vi}{}

\href{https://www.nytimes3xbfgragh.onion/section/todayspaper}{Today's
Paper}

\href{/section/business/economy}{Economy}\textbar{}Liberal Treasury
Nominee's Wall St. Prowess May Be a Vulnerability

\url{https://nyti.ms/1FwsmWx}

\begin{itemize}
\item
\item
\item
\item
\item
\end{itemize}

Advertisement

\protect\hyperlink{after-top}{Continue reading the main story}

Supported by

\protect\hyperlink{after-sponsor}{Continue reading the main story}

\hypertarget{liberal-treasury-nominees-wall-st-prowess-may-be-a-vulnerability}{%
\section{Liberal Treasury Nominee's Wall St. Prowess May Be a
Vulnerability}\label{liberal-treasury-nominees-wall-st-prowess-may-be-a-vulnerability}}

By \href{http://www.nytimes3xbfgragh.onion/by/jonathan-weisman}{Jonathan
Weisman}

\begin{itemize}
\item
  Nov. 27, 2014
\item
  \begin{itemize}
  \item
  \item
  \item
  \item
  \item
  \end{itemize}
\end{itemize}

\includegraphics{https://static01.graylady3jvrrxbe.onion/images/2014/11/28/business/28WEISS/28WEISS-articleLarge.jpg?quality=75\&auto=webp\&disable=upscale}

WASHINGTON --- In 2012, Antonio F. Weiss took his 15-year-old son, Nico,
from the gilded aerie of their Manhattan apartment on Central Park West
to Cleveland to canvass for President Obama's re-election. Mr. Weiss,
48, was also the co-author of a white paper calling for higher taxes on
the rich and has donated hundreds of thousands of dollars to the
\href{http://topics.nytimes3xbfgragh.onion/top/reference/timestopics/organizations/d/democratic_party/index.html?inline=nyt-org}{Democratic
Party}.

Yet in his Wall Street provenance, Mr. Weiss, President Obama's nominee
to be under secretary of the Treasury for domestic finance, has given
the left an unlikely rallying cry to press for a more aggressively
liberal economic policy agenda.

It is not Mr. Weiss's politics that are in question. It is his résumé.

``I have voted for people who have extensive Wall Street experience,''
said Senator Elizabeth Warren, Democrat of Massachusetts. She is
rallying the opposition to Mr. Weiss, the head of investment banking at
Lazard, a storied but relatively small firm. But, she said, ``the
Antonio Weiss nomination is a mistake, and that's why I'm fighting
back.''

The formal confirmation process, while not likely to get underway until
after the new Congress convenes next year, has become an unexpected
proxy war between the liberal and moderate wings of the Democratic
Party. Its outcome will say a lot about the party's direction as it
regroups for the 2016 presidential campaign, in which Hillary Clinton
will be under pressure to discard some of her ties to Wall Street.

At Lazard, Mr. Weiss was involved in a number of international
mega-mergers, including
\href{http://dealbook.nytimes3xbfgragh.onion/2014/08/26/burger-king-to-buy-tim-hortons-for-11-4-billion}{a
deal} that allowed Burger King to acquire the Canadian fast-food chain
Tim Hortons in a maneuver that gave the combined company a lower tax
liability in the United States. And in doing so, he made a lot of money.

Mr. Weiss's assets are worth between \$54 million and \$203 million,
according to his financial disclosure. In addition to his Manhattan
apartment, he owns a 200-year-old, eight-bedroom farmhouse in
Connecticut and property in the Dominican Republic valued at up to \$1
million.

To Ms. Warren and her allies, Mr. Weiss's nomination this month was
proof that their anti-Wall Street views are still getting no respect
within the Obama administration. While they managed to
\href{http://www.nytimes3xbfgragh.onion/2013/09/16/business/economy/summers-pulls-name-from-consideration-for-fed-chief.html}{derail
Mr. Obama's moves} to nominate Lawrence H. Summers, his former Treasury
secretary and economic adviser, as chairman of the Federal Reserve, they
say Mr. Weiss's confirmation by the Senate would send the wrong signal
about whether Democrats can advance the economic prospects of the
struggling middle class.

``The American people are profoundly disappointed with the fraud they
read about every day coming from Wall Street,'' said Senator Bernie
Sanders, an independent from Vermont who is considering running for
president as a Democrat to encourage the party to move to the left.
``They are disgusted that instead of investing in the American economy,
they are busy trying to avoid paying their fair share of taxes, and the
American people want people in the Treasury Department who are prepared
to hold Wall Street accountable.''

Supporters of Mr. Weiss, both inside and outside the Obama
administration, see the brewing fight as no less consequential. Wall
Street executives lend the Treasury Department real-world expertise to
understand how policy proposals might be gamed by the banks and
investment houses they are aimed at.

If the Elizabeth Warren wing of the party can bring Mr. Weiss down, they
say, prominent financiers may no longer play a significant role in
Democratic administrations, which have turned to them since the Clinton
years to bolster their business bona fides.

\includegraphics{https://static01.graylady3jvrrxbe.onion/images/2014/11/28/business/28jpWEISS/28jpWEISS-articleLarge.jpg?quality=75\&auto=webp\&disable=upscale}

``If the rules post-financial crisis were that the one place you
shouldn't go for help is the private sector, particularly the financial
sector, that would be a pretty dangerous thing,'' a senior Treasury
official said, speaking on condition of anonymity.

Moreover, Mr. Weiss's defenders in and out of the administration say he
is being caricatured as a rapacious banker when he is more Daddy
Warbucks than Gordon Gekko. He combines financial expertise with an
unquestioned liberal outlook and an intellectual panache that led to his
becoming publisher of \href{http://www.theparisreview.org/}{The Paris
Review}.

Neera Tanden, president of the Center for American Progress, a
Democratic research and advocacy group, recalled Mr. Weiss working on an
\href{https://www.americanprogress.org/issues/tax-reform/report/2012/12/04/46689/reforming-our-tax-system-reducing-our-deficit/}{economic
policy paper} for her organization that called for sharply higher taxes
on the wealthy, an overhaul of the corporate tax code that would raise
revenue for deficit reduction and changes to the individual tax code to
make it more progressive.

Gene B. Sperling, a former senior economic policy maker in the Obama and
Clinton White Houses, said: ``He has a good progressive heart. He has
hardheaded practical business experience.''

Mr. Weiss declined to comment for this article, citing his pending
confirmation hearings.

The particulars of Mr. Weiss's background and policy views appear to
matter far less than the optics. Mr. Weiss spent years in Paris as vice
chairman of European investment banking at Lazard, then rose to global
head of mergers and acquisitions. His deal making has included this
year's merger of the tobacco giants
\href{http://dealbook.nytimes3xbfgragh.onion/2014/07/15/reynolds-american-to-buy-lorillard-for-27-4-billion/}{Reynolds
American and Lorillard}, Berkshire Hathaway's
\href{http://dealbook.nytimes3xbfgragh.onion/2013/02/14/berkshire-and-3g-capital-to-buy-heinz-for-23-billion}{swallowing
of H .J. Heinz} last year, Google's 2011
\href{dealbook.nytimes3xbfgragh.onion/2011/08/15/google-to-buy-motorola-mobility/}{takeover
of Motorola Mobility} and InBev's
\href{http://www.nytimes3xbfgragh.onion/2008/07/14/business/worldbusiness/14beer.html}{takeover
of Anheuser-Busch} in 2008.

No deal is causing more trouble for him than Burger King's ``inversion''
merger with Tim Hortons, which came just as the Treasury was proposing
new rules to stop American companies from reincorporating as foreign
entities not subject to United States taxes. Lazard itself gave up its
United States citizenship in 2005 to reincorporate in Bermuda, using a
loophole that the Bush administration later closed to deter copycats.

``On the policy on whether or not companies should move overseas to
avoid U.S. taxation when there's not a core business reason for the
move, that's something we think is wrong,'' Treasury Secretary Jacob J.
Lew said in an interview. ``It's something he thinks is wrong.''

Mr. Weiss's defenders in the administration say the Burger King deal was
not really an inversion, in which a large American company adopts a
foreign headquarters in name only. But it still sticks in Democratic
craws. Senator Richard J. Durbin of Illinois, the Senate's
second-ranking Democrat, cited his work on such deals when he announced
his opposition to Mr. Weiss's confirmation.

Beyond Lazard, there is Mr. Weiss himself. To defenders like Ms. Tanden,
his years in Europe made him acutely aware of the perils of wage
stagnation and the obstacles to upward mobility. He grew up in New York,
in a distinctly middle-class family. Both of his parents were teachers.
He attended Yale and Harvard Business School, while also apprenticing
under George Plimpton, the editor of The Paris Review.

Where supporters see brio, detractors see a fat cat. Last week, the
A.F.L.-C.I.O. president Richard L. Trumka sent a letter to Lazard's
compensation committee chairman, Philip A. Laskawy, via the company's
Bermuda affiliate, questioning his decision to speed the vesting of
equity income to ease Mr. Weiss's transition to public service. If he is
confirmed as the under secretary, Mr. Weiss will receive \$6 million to
\$30 million in stock that would normally accrue to him in 2017 and \$3
million in interest income, according to the Project on Government
Oversight.

But beyond that is the Warren wing's belief that Democrats must realign
their economic policies with the interests of working-class voters,
particularly white men without college degrees, who have flocked to the
Republican Party in recent years. The Democrats' attention should be
focused on raising the minimum wage, funding infrastructure investments
financed by higher taxes on the rich and, Ms. Warren adds, a new push to
divide the big banks from their nonbanking activities.

``We have got to be willing to make the government work for America's
families,'' Ms. Warren said. ``That's the start of everything we do.''

Advertisement

\protect\hyperlink{after-bottom}{Continue reading the main story}

\hypertarget{site-index}{%
\subsection{Site Index}\label{site-index}}

\hypertarget{site-information-navigation}{%
\subsection{Site Information
Navigation}\label{site-information-navigation}}

\begin{itemize}
\tightlist
\item
  \href{https://help.nytimes3xbfgragh.onion/hc/en-us/articles/115014792127-Copyright-notice}{©~2020~The
  New York Times Company}
\end{itemize}

\begin{itemize}
\tightlist
\item
  \href{https://www.nytco.com/}{NYTCo}
\item
  \href{https://help.nytimes3xbfgragh.onion/hc/en-us/articles/115015385887-Contact-Us}{Contact
  Us}
\item
  \href{https://www.nytco.com/careers/}{Work with us}
\item
  \href{https://nytmediakit.com/}{Advertise}
\item
  \href{http://www.tbrandstudio.com/}{T Brand Studio}
\item
  \href{https://www.nytimes3xbfgragh.onion/privacy/cookie-policy\#how-do-i-manage-trackers}{Your
  Ad Choices}
\item
  \href{https://www.nytimes3xbfgragh.onion/privacy}{Privacy}
\item
  \href{https://help.nytimes3xbfgragh.onion/hc/en-us/articles/115014893428-Terms-of-service}{Terms
  of Service}
\item
  \href{https://help.nytimes3xbfgragh.onion/hc/en-us/articles/115014893968-Terms-of-sale}{Terms
  of Sale}
\item
  \href{https://spiderbites.nytimes3xbfgragh.onion}{Site Map}
\item
  \href{https://help.nytimes3xbfgragh.onion/hc/en-us}{Help}
\item
  \href{https://www.nytimes3xbfgragh.onion/subscription?campaignId=37WXW}{Subscriptions}
\end{itemize}
