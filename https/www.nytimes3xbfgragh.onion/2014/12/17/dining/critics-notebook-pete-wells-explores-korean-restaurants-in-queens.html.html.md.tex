Sections

SEARCH

\protect\hyperlink{site-content}{Skip to
content}\protect\hyperlink{site-index}{Skip to site index}

\href{https://www.nytimes3xbfgragh.onion/section/food}{Food}

\href{https://myaccount.nytimes3xbfgragh.onion/auth/login?response_type=cookie\&client_id=vi}{}

\href{https://www.nytimes3xbfgragh.onion/section/todayspaper}{Today's
Paper}

\href{/section/food}{Food}\textbar{}In Queens, Kimchi Is Just the Start

\url{https://nyti.ms/1uV5oRU}

\begin{itemize}
\item
\item
\item
\item
\item
\item
\end{itemize}

Advertisement

\protect\hyperlink{after-top}{Continue reading the main story}

Supported by

\protect\hyperlink{after-sponsor}{Continue reading the main story}

Critic's Notebook

\hypertarget{in-queens-kimchi-is-just-the-start}{%
\section{In Queens, Kimchi Is Just the
Start}\label{in-queens-kimchi-is-just-the-start}}

Slide 1 of 31

1/31

For people who don't speak Korean, language isn't the main barrier to
eating well in Queens. The greater challenge is figuring out where to go
for which foods. Here, raw fluke at Bada Story.

Credit...Evan Sung for The New York Times

\begin{itemize}
\item
  \includegraphics{https://static01.graylady3jvrrxbe.onion/images/2014/12/17/dining/20141217-KOREAN-slide-HPAB/20141217-KOREAN-slide-HPAB-superJumbo.jpg}
\item
  \includegraphics{https://static01.graylady3jvrrxbe.onion/images/2014/12/17/dining/20141217-KOREAN-slide-QFSZ/20141217-KOREAN-slide-QFSZ-superJumbo.jpg}
\item
  \includegraphics{https://static01.graylady3jvrrxbe.onion/images/2014/12/17/dining/20141217-KOREAN-slide-EUKH/20141217-KOREAN-slide-EUKH-superJumbo.jpg}
\item
  \includegraphics{https://static01.graylady3jvrrxbe.onion/images/2014/12/17/dining/20141217-KOREAN-slide-M2S2/20141217-KOREAN-slide-M2S2-superJumbo.jpg}
\item
  \includegraphics{https://static01.graylady3jvrrxbe.onion/images/2014/12/17/dining/20141217-KOREAN-slide-Q3YN/20141217-KOREAN-slide-Q3YN-superJumbo.jpg}
\item
  \includegraphics{https://static01.graylady3jvrrxbe.onion/images/2014/12/17/dining/20141217-KOREAN-slide-WUXD/20141217-KOREAN-slide-WUXD-superJumbo.jpg}
\item
  \includegraphics{https://static01.graylady3jvrrxbe.onion/images/2014/12/17/dining/20141217-KOREAN-slide-4O7B/20141217-KOREAN-slide-4O7B-superJumbo.jpg}
\item
  \includegraphics{https://static01.graylady3jvrrxbe.onion/images/2014/12/17/dining/20141217-KOREAN-slide-KY0U/20141217-KOREAN-slide-KY0U-superJumbo.jpg}
\item
  \includegraphics{https://static01.graylady3jvrrxbe.onion/images/2014/12/17/dining/20141217-KOREAN-slide-6PU5/20141217-KOREAN-slide-6PU5-superJumbo.jpg}
\item
  \includegraphics{https://static01.graylady3jvrrxbe.onion/images/2014/12/17/dining/20141217-KOREAN-slide-80I6/20141217-KOREAN-slide-80I6-superJumbo.jpg}
\item
  \includegraphics{https://static01.graylady3jvrrxbe.onion/images/2014/12/17/dining/20141217-KOREAN-slide-SLCI/20141217-KOREAN-slide-SLCI-superJumbo.jpg}
\item
  \includegraphics{https://static01.graylady3jvrrxbe.onion/images/2014/12/17/dining/20141217-KOREAN-slide-Y6CO/20141217-KOREAN-slide-Y6CO-superJumbo.jpg}
\item
  \includegraphics{https://static01.graylady3jvrrxbe.onion/images/2014/12/17/dining/20141217-KOREAN-slide-ABIG/20141217-KOREAN-slide-ABIG-superJumbo.jpg}
\item
  \includegraphics{https://static01.graylady3jvrrxbe.onion/images/2014/12/17/dining/20141217-KOREAN-slide-DRI2/20141217-KOREAN-slide-DRI2-superJumbo.jpg}
\item
  \includegraphics{https://static01.graylady3jvrrxbe.onion/images/2014/12/17/dining/20141217-KOREAN-slide-OT9H/20141217-KOREAN-slide-OT9H-superJumbo.jpg}
\item
  \includegraphics{https://static01.graylady3jvrrxbe.onion/images/2014/12/17/dining/20141217-KOREAN-slide-K16R/20141217-KOREAN-slide-K16R-superJumbo.jpg}
\item
  \includegraphics{https://static01.graylady3jvrrxbe.onion/images/2014/12/17/dining/20141217-KOREAN-slide-OAZ3/20141217-KOREAN-slide-OAZ3-superJumbo.jpg}
\item
  \includegraphics{https://static01.graylady3jvrrxbe.onion/images/2014/12/17/dining/20141217-KOREAN-slide-2JKX/20141217-KOREAN-slide-2JKX-superJumbo.jpg}
\item
  \includegraphics{https://static01.graylady3jvrrxbe.onion/images/2014/12/17/dining/20141217-KOREAN-slide-XUPA/20141217-KOREAN-slide-XUPA-superJumbo.jpg}
\item
  \includegraphics{https://static01.graylady3jvrrxbe.onion/images/2014/12/17/dining/20141217-KOREAN-slide-NAZP/20141217-KOREAN-slide-NAZP-superJumbo.jpg}
\item
  \includegraphics{https://static01.graylady3jvrrxbe.onion/images/2014/12/17/dining/20141217-KOREAN-slide-67BN/20141217-KOREAN-slide-67BN-superJumbo.jpg}
\item
  \includegraphics{https://static01.graylady3jvrrxbe.onion/images/2014/12/17/dining/20141217-KOREAN-slide-QZPF/20141217-KOREAN-slide-QZPF-superJumbo.jpg}
\item
  \includegraphics{https://static01.graylady3jvrrxbe.onion/images/2014/12/17/dining/20141217-KOREAN-slide-9EYJ/20141217-KOREAN-slide-9EYJ-superJumbo.jpg}
\item
  \includegraphics{https://static01.graylady3jvrrxbe.onion/images/2014/12/17/dining/20141217-KOREAN-slide-T26A/20141217-KOREAN-slide-T26A-superJumbo.jpg}
\item
  \includegraphics{https://static01.graylady3jvrrxbe.onion/images/2014/12/17/dining/20141217-KOREAN-slide-EQ6T/20141217-KOREAN-slide-EQ6T-superJumbo.jpg}
\item
  \includegraphics{https://static01.graylady3jvrrxbe.onion/images/2014/12/17/dining/20141217-KOREAN-slide-229J/20141217-KOREAN-slide-229J-superJumbo.jpg}
\item
  \includegraphics{https://static01.graylady3jvrrxbe.onion/images/2014/12/17/dining/20141217-KOREAN-slide-W8UX/20141217-KOREAN-slide-W8UX-superJumbo.jpg}
\item
  \includegraphics{https://static01.graylady3jvrrxbe.onion/images/2014/12/17/dining/20141217-KOREAN-slide-LYPE/20141217-KOREAN-slide-LYPE-superJumbo.jpg}
\item
  \includegraphics{https://static01.graylady3jvrrxbe.onion/images/2014/12/17/dining/20141217-KOREAN-slide-F9CX/20141217-KOREAN-slide-F9CX-superJumbo.jpg}
\item
  \includegraphics{https://static01.graylady3jvrrxbe.onion/images/2014/12/17/dining/20141217-KOREAN-slide-AYDA/20141217-KOREAN-slide-AYDA-superJumbo.jpg}
\item
  \includegraphics{https://static01.graylady3jvrrxbe.onion/images/2014/12/17/dining/20141217-KOREAN-slide-00QX/20141217-KOREAN-slide-00QX-superJumbo.jpg}
\end{itemize}

By \href{http://www.nytimes3xbfgragh.onion/by/pete-wells}{Pete Wells}

\begin{itemize}
\item
  Dec. 16, 2014
\item
  \begin{itemize}
  \item
  \item
  \item
  \item
  \item
  \item
  \end{itemize}
\end{itemize}

We can blame the IRT. The
\href{http://www.nycsubway.org/wiki/IRT_Flushing_Line}{No. 7} train was
never meant to end at Main Street in Flushing. If the Interborough Rapid
Transit Company had followed through on plans to extend it, then all the
remarkable Korean food of Queens might be as famous as the dosas of
Jackson Heights, the tamales of Corona, the som tums of Elmhurst.

But the last stop on the Flushing Local line is just the beginning of
the Korean strip. It stretches east for about five more miles, following
the Long Island Rail Road tracks and Northern Boulevard all the way into
Nassau County. There are hundreds of restaurants in Murray Hill,
Auburndale, Bayside and beyond, serving famous Korean dishes and obscure
ones: beef barbecue and blood sausage; wheat noodles in deep steaming
bowls and arrowroot noodles in broth chilled with ice crystals; tofu
casseroles and live octopus; Korean-Chinese restaurants and
Korean-French bakeries; beery pubs and studious espresso bars; chicken
fried in a shattering crust of rice flour and chicken boiled whole with
ginseng.

The Queens kimchi belt has got to be the least explored, discussed and
celebrated of the city's great ethnic-food districts. For variety of
dishes and excellence of cooking, the only areas that compete are the
Japanese clusters in the East Village and the East 40s or the city's
three Chinatowns. Koreatown in the West 30s, which was once strong,
doesn't even get on the scoreboard.

``I believe that right now, Queens is the closest you can come to
authentic Korean food,'' said Hooni Kim, the chef of
\href{http://www.nytimes3xbfgragh.onion/2011/08/17/dining/reviews/danji-manhattan-restaurant-review.html?pagewanted=all\&module=Search\&mabReward=relbias\%3Ar\%2C\%7B\%221\%22\%3A\%22RI\%3A10\%22\%7D\&_r=0}{Danji}
and
\href{http://www.nytimes3xbfgragh.onion/2013/04/03/dining/reviews/restaurant-review-hanjan-in-manhattan.html?pagewanted=all\&module=Search\&mabReward=relbias\%3Ar\%2C\%7B\%221\%22\%3A\%22RI\%3A10\%22\%7D}{Hanjan}
in Manhattan and a frequent prowler of Northern Boulevard. Unlike the
restaurants on 32nd Street in Manhattan, Mr. Kim said, ``the kitchens
actually cook for Koreans.'' And while there are excellent Korean places
in and around Fort Lee, N.J., some of which have sibling branches in
Queens, Mr. Kim said the flavors along Northern Boulevard are closer to
what he has tasted on his trips to South Korea.

``A lot of the restaurants in Queens are opened by first-generation
people, who know what's being served in Korea now,'' he said.

For people who don't speak Korean, language isn't the main barrier to
eating well in this part of town. The greater challenge is figuring out
where to go for which foods. When Koreans go out for a meal, they tend
to first decide what they're going to eat, then choose a restaurant that
does it especially well.

If you don't know what the house specialty is when you arrive, menus may
not help. Take the hub of restaurants around the Murray Hill station on
the Long Island Rail Road, as good a place as any to start. People in
the know would head to Mapo Korean BBQ if they were in the mood for
marinated short-ribs, or kalbi.

``Unfortunately, they have another 80 things on the menu to distract
somebody who doesn't know,'' Mr. Kim said.

Signs can be misleading, too. Next door to Mapo is Han Joo Chik Naeng
Myun \& BBQ. From the name outside, you may think it specializes in the
cold noodles called naeng myun. But no, everybody comes here for
barbecued pork belly. For naeng myun, look across the street for Keum
Sung Food. There is a green duck on the awning; ignore it. If it's duck
you want, then your destination should be Sura Chung, around the corner.

I followed the lead of the locals. Rather than taking the full measure
of a menu, as a restaurant critic normally would, I zeroed in on one or
two specialties. I compared them with other competing versions nearby. I
would taste all the claimants on the same day when I could, although I
had to break my fried-chicken safari into two trips.

The short list that follows gives some of my favorite places in the area
broken down by specialty. (Most are in the Flushing neighborhoods of
Murray Hill and Auburndale.) I have a lot of work left to do. I could
never get excited about any version I found of jajangmyeon, the
Chinese-derived noodles in black bean sauce. I am still on the hunt for
argument-ending mandoo and kimbap as well. Consider this both a
beginner's guide and a first shot in a straphanger's campaign: When the
Second Avenue subway is done, let's get back to work on the 7.

1. \textbf{Debasaki}

33-67 Farrington Street (35th Avenue), 718-886-6878

Chicken gyoza (stuffed wings)

When your head fills with visions of drumsticks lined up like the
Rockettes, this cavernous pub is not the place for you. A specialist
among Korean chicken establishments, it fries only wings. Most are very
good, but the stuffed wings earn Debasaki its pin on the map. Called
chicken gyoza, they are boned out and filled with kimchi, chopped shrimp
or other stuffings you would normally find in a dumpling. You hold them
by the wing tips until they are cool enough to eat. The décor
successfully combines birch tree trunks with Korean pop videos.

\textbf{2. Mapo Korean BBQ}

149-24 41st Road (149th Place),718-886-8292

Kalbi (marinated short-rib)

Glowing charcoal, handled by tongs, is deposited inside a grill carved
into tables of cherry-red composite stone. Then deeply slashed slabs of
Black Angus short rib, marinated into an unmistakable tenderness, are
stretched inches above the coals for a long time. A server appears with
scissors to snip the meat into rough squares, leaving them to curl in
the heat waves until the edges go black. This is how Mapo BBQ makes
kalbi, and I don't know another charcoal-grilled steak in New York that
can match it. Don't get so involved in making short-rib lettuce wraps
that you forget to grill the rib itself; there's good meat on that bone.

\textbf{3. Han Joo Chik Naeng Myun \& BBQ,}

41-06 149th Place (41st Avenue) 718-359-6888

Samgyupsal (barbecued pork belly)

The Queens crown for best barbecued pork belly, called samgyupsal,
nearly went to Tong Sam Gyup Gui, a small spot a few miles east where
the meat cooks on top of a traditional ridged metal dome so that the fat
trickles down into sizzling kimchi and bean sprouts, and where you swipe
the belly in a superb tonkatsu sauce augmented with mystery ingredients.
The heat source at Han Joo is a flat slab of crystal, the sauce a
standard spicy soybean paste. Nearly everything else is superior,
though, from the banchan to the belly itself, which comes in four forms:
unseasoned thin slices that crisp up, fatter ones that stay meaty,
strips rubbed with green tea and others marinated in red bean paste. The
meat makes the difference*.*

\href{https://www.nytimes3xbfgragh.onion/interactive/2014/12/16/dining/koreamap.html}{}

\includegraphics{https://static01.graylady3jvrrxbe.onion/images/2014/12/16/dining/koreamap-1418760479438/koreamap-1418760479438-videoLarge.jpg}

\hypertarget{pete-wellss-choices-for-korean-dishes-in-queens}{%
\subsection{Pete Wells's Choices for Korean Dishes in
Queens}\label{pete-wellss-choices-for-korean-dishes-in-queens}}

From the hundreds of Korean restaurants in and near Flushing, Queens,
Pete Wells chose a dozen that do particular dishes particularly well.

\textbf{4. Keum Sung Food}

40-07 149th Place (Roosevelt Avenue), 718-539-4596

Naeng myun (cold noodles)

``My stomach feels so cold, my teeth feel so cold,'' begins a
\href{https://www.youtube.com/watch?v=0zGDQ9r36p4}{hit Korean song} that
compares running into an old summertime fling to eating chilled noodle
soup. The dish that gave its name to the tune, naeng myun, is so
ferociously cold that ice shards shimmer on the surface of the broth.

Several decisions must be made. First, do you want noodles made from
buckwheat (momil) or arrowroot (chik)? Choose arrowroot for a texture
that's unlike anything else, extremely elastic but not rubbery. These
noodles won't surrender to your teeth right away. Second, cut or uncut?
A few snips of the scissors make these fantastically long, thin noodles
easier to eat, but may shorten your life, tradition says. Arrowroot
noodles are all about the texture; flavor comes from the remarkably
clean-tasting and balanced beef broth, which you can adjust with hot
mustard and vinegar. For crunch and additional cooling power, there are
cucumbers and sliced Asian pears. Keum Sung's version can make a New
York summer bearable, but it's worth noting that in Korea, naeng myun
never really goes out of season.

\textbf{5. Mat Baram}

150-40 Northern Boulevard (150th Street), 718-460-2535

Kalguksu and sujebi (fresh noodles)

There are those Korean dishes that challenge you to combat and those
that knit up the raveled sleeve of care. The impressively generous bowls
of steaming noodle soups at Mat Baram are the knitting type. The menu of
this simple restaurant, decorated with black horsehair hats and bamboo
pipes, is simple, complicated only by the choice of noodle. The long,
thin ones, kalguksu, are for slurping. The lightly firm squares, sujebi,
are from dough that is pulled by hand, then cut into pieces with a
knife; they are for chewing.

The chicken-ginseng soup is the most grandmotherly; with its cubes of
carrots it could almost pass for the American version, although the
broth is the milky white of a boiled-bone stock and not the familiar
translucent gold. The most distinctive, though, are the hand-torn
noodles made with green tea in a creamy, speckled sauce of ground
perilla seeds with anchovies, radishes and seaweed. The mild, nutty
taste responds well to a few spoonfuls of the fantastic hot sauce, made
in the kitchen from Thai chiles with tomatoes and pineapple.

\textbf{6. Kang Ho Dong Baekjeong}

152-12 Northern Boulevard (Murray Street),718-886-8645

Barbecued marinated pork collar

In the cartoon mural out front, the man dressed as the Statue of
Liberty, clutching a raw steak instead of a tablet, is Kang Ho-dong
himself. A South Korean wrestling champion, Mr. Kang retired from the
ring and reinvented himself as a comedian, variety-show host and global
barbecue magnate. (His latest Baekjeong opened in Manhattan this month.)
The banchan are somewhat skimpy, but in compensation there are beaten
eggs and corn mixed with grated cheese that cook in long canals, basted
by fat that runs from the grill. The meat is gorgeously marbled, but the
standout is the pork collar in a sweet soy marinade, like kalbi made
from a pig. Condiments are provided, including an intriguing proprietary
sauce, but this cut doesn't need them.

\textbf{7. Bonjuk}

152-26 Northern Boulevard (153rd Street), 718-939-5868

Juk (savory rice porridge)

Every night is a party at many Korean restaurants in Queens, and you
rarely see anyone eating alone. You do at Bonjuk, though. This is where
you go when you stayed at the party too long. The dining room is spare
and calming, like a teahouse inside a spa. A sonata for violin and piano
played softly one recent night. ``Well Being Slow Food,'' reads a sign
on the wall. The slow food in question is juk, usually described as rice
porridge, although porridge sounds like punishment while juk should
taste like a reward, or at least a consolation prize. The flavor of the
juk with mushrooms and oysters was a bit undeveloped, but the
ginseng-chicken variety tasted as if a worried grandmother had fussed
over it all day. The intensely good octopus-kimchi juk is for those days
when nothing will restore you but spicy heat.

\textbf{8. Geo Si Gi}

152-28 Northern Boulevard (153rd Street), 718-888-0001

Kamja tang (spicy pork stew)

A sign in Korean tells the story of the dish nearly everybody orders at
Geo Si Gi, hunks of braised pork loin half-submerged like icebergs in a
fiery-red broth bubbling over a gas burner. It's called kamja tang,
sometimes described as potato stew, although a server insisted that a
more accurate translation was ``very tender meat stew.'' Very, very
tender it is, barely hanging on to its shaft of backbone. There are six
kamja tangs on the menu with varying spice levels and extras (one
go-for-broke version incorporates raw octopus, shrimp and green-lipped
mussels that cook in the stew). The broth has a rounded sweetness and a
many-layered depth that soaks into the fat noodles at the bottom of the
pot.

\textbf{9. Mad for Chicken}

157-18 Northern Boulevard (158th Street), 718-321-3818

Fried chicken

Of the half-dozen or so fried chicken dispensaries along and off
Northern Boulevard, Mad for Chicken is the liveliest, a self-described
``Korean gastro pub'' with hockey and football on TV, American rock and
pop on the speakers, macrobrews on tap and a menu that is what TGI
Friday's would serve if it were taken over by a 25-year-old Korean
hipster. (Should you be struck with a midnight craving for poutine with
corn, Mad for Chicken awaits you.) It also has my favorite chicken; a
puffed-up golden shell hovers just above the flesh, and a spicy garlic
sauce that is a little more savory and less sugary than the
competitors'.

\textbf{10. Bada Story}

161-23 Crocheron Avenue (162nd Street),718-321-9555

Hwe (raw seafood)

The dry-ice fog slinking off each platter of raw fluke at Bada Story may
foreshadow the mist of confusion that will surround anybody who
approaches this restaurant's specialty, hwe, with the idea that it is
Korean sashimi. By the time the Japanese showed up with their very
different take on raw seafood, Korea already had fixed ideas about the
subject. Crunch, chew and snap are prized over melting tenderness.
Instead of being dipped lightly in soy sauce, the sea creature in
question can be subjected to a barrage of fermented soybean paste,
salted sesame oil, vinegared chile paste, chopped jalapeños and garlic.

A Bada Story feast (bring a crowd) might start with some excellent fried
fingers of fish and a first-rate seafood pancake, then move on to a
tableau of marine life not seen on the sashimi special at your corner
sushi bar: chewy sea squirts with their wrinkled orange skin; strips of
sea cucumber that unfurl like streamers from your chopsticks but tense
up again when you bite down; sea worms that look like veins and taste
like not very much but have chewiness to spare. Then, in a swirl of
mist, arrives a whole imported Korean fluke that a few minutes ago was
lounging at the bottom of a pool by the kitchen. The slices, draped over
frosty ceramic cups to keep them firm, are arranged as they were on the
skeleton, with the lightly crunchy hard-working muscles around the fins
showcased at the edge of the platter.

\textbf{11. Bangane}

165-19 Northern Boulevard (165th Street),718-762-2799

Korean black goat

Posters of goats hang on the walls of Bangane, goats with rakishly wavy
black hair and an ambiguous twinkle in their eyes. The twinkle might
mean ``follow me and learn the secret of my goatish vigor'' or it might
mean ``I want to eat your belt.'' It's hard to tell with goats.

The patrons of Bangane, though, have goatish vigor on their minds.
``Korean people eat it when they want to make babies,'' a server said as
he brought out a steamer basket lined with garlic chives on which was
mounded enough boiled goat to fill a maternity ward. Dark and tender,
mild and lean, the meat may not immediately result in a procreative
mood. But the more you dunk it in chile paste mixed with crunchy perilla
seeds, the more of it you eat, the more energy you have for dunking and
eating. This is lucky because as soon as you say you have had enough
goat, the rest of it along with the chives and chile paste are tossed
into a pot set over a gas burner next to the table. Broth from the
kitchen is poured in and in no time the second course is ready:
black-goat soup. There is energy in this, too, combined with the
mind-focusing power of hot chiles and green perilla leaves. There is a
third course, goat fried rice stirred together while the soup comes to a
boil. But it is the first two dishes that send you into the night with
an ambiguous twinkle in your eyes.

\textbf{12. To Soc Chon}

45-30 Bell Boulevard (45th Road), Bayside,347-408-4584

Soondae (blood sausage)

Early one winter night, a young couple having dinner in this bright,
colorful, 24-hour restaurant had brought along their small children,
ready for bed in flannel pajamas. In another neighborhood in another
city, they would have been in a Waffle House. But no Waffle House
currently offers soondae, the springy blood sausage that lures people to
To Soc Chon.

Its sausages are lighter than the dense, more blood-forward ones at
Byeoncheon, another soondae destination a couple of miles west. At To
Soc Chon the pork casings are filled mostly with vermicelli, to which
the blood adds its brooding, rich seasoning. You can snack on a plate,
dipping pieces in chile salt, and still have an appetite for the main
course, an innards soup. It looks milky and mysterious, with bits of pig
ears and tongues bobbing to the surface along with purplish lengths of
soondae. Then you stir in garlic chives, red chile paste and mashed
jalapeños, add chile salt to taste, and this cloudy bowl of boiled
bones, blood and innards tastes potent enough to ward off any illness,
including anemia.

Advertisement

\protect\hyperlink{after-bottom}{Continue reading the main story}

\hypertarget{site-index}{%
\subsection{Site Index}\label{site-index}}

\hypertarget{site-information-navigation}{%
\subsection{Site Information
Navigation}\label{site-information-navigation}}

\begin{itemize}
\tightlist
\item
  \href{https://help.nytimes3xbfgragh.onion/hc/en-us/articles/115014792127-Copyright-notice}{©~2020~The
  New York Times Company}
\end{itemize}

\begin{itemize}
\tightlist
\item
  \href{https://www.nytco.com/}{NYTCo}
\item
  \href{https://help.nytimes3xbfgragh.onion/hc/en-us/articles/115015385887-Contact-Us}{Contact
  Us}
\item
  \href{https://www.nytco.com/careers/}{Work with us}
\item
  \href{https://nytmediakit.com/}{Advertise}
\item
  \href{http://www.tbrandstudio.com/}{T Brand Studio}
\item
  \href{https://www.nytimes3xbfgragh.onion/privacy/cookie-policy\#how-do-i-manage-trackers}{Your
  Ad Choices}
\item
  \href{https://www.nytimes3xbfgragh.onion/privacy}{Privacy}
\item
  \href{https://help.nytimes3xbfgragh.onion/hc/en-us/articles/115014893428-Terms-of-service}{Terms
  of Service}
\item
  \href{https://help.nytimes3xbfgragh.onion/hc/en-us/articles/115014893968-Terms-of-sale}{Terms
  of Sale}
\item
  \href{https://spiderbites.nytimes3xbfgragh.onion}{Site Map}
\item
  \href{https://help.nytimes3xbfgragh.onion/hc/en-us}{Help}
\item
  \href{https://www.nytimes3xbfgragh.onion/subscription?campaignId=37WXW}{Subscriptions}
\end{itemize}
