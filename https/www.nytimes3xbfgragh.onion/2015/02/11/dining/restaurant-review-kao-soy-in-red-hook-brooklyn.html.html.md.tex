Sections

SEARCH

\protect\hyperlink{site-content}{Skip to
content}\protect\hyperlink{site-index}{Skip to site index}

\href{https://www.nytimes3xbfgragh.onion/section/food}{Food}

\href{https://myaccount.nytimes3xbfgragh.onion/auth/login?response_type=cookie\&client_id=vi}{}

\href{https://www.nytimes3xbfgragh.onion/section/todayspaper}{Today's
Paper}

\href{/section/food}{Food}\textbar{}Restaurant Review: Kao Soy in Red
Hook, Brooklyn

\url{https://nyti.ms/1zT2QII}

\begin{itemize}
\item
\item
\item
\item
\item
\item
\end{itemize}

Advertisement

\protect\hyperlink{after-top}{Continue reading the main story}

Supported by

\protect\hyperlink{after-sponsor}{Continue reading the main story}

Restaurants

\hypertarget{restaurant-review-kao-soy-in-red-hook-brooklyn}{%
\section{Restaurant Review: Kao Soy in Red Hook,
Brooklyn}\label{restaurant-review-kao-soy-in-red-hook-brooklyn}}

Slide 1 of 10

1/10

Kanlaya Supachana opened Kao Soy last summer in a small space on Van
Brunt Street in Red Hook, Brooklyn.

Credit...Sally Ryan for The New York Times

\begin{itemize}
\item
  \includegraphics{https://static01.graylady3jvrrxbe.onion/images/2015/02/11/dining/20150211REST-slide-9IGX/20150211REST-slide-9IGX-superJumbo.jpg}
\item
  \includegraphics{https://static01.graylady3jvrrxbe.onion/images/2015/02/11/dining/20150211REST-slide-ONF2/20150211REST-slide-ONF2-superJumbo.jpg}
\item
  \includegraphics{https://static01.graylady3jvrrxbe.onion/images/2015/02/11/dining/20150211REST-slide-G8Q9/20150211REST-slide-G8Q9-superJumbo.jpg}
\item
  \includegraphics{https://static01.graylady3jvrrxbe.onion/images/2015/02/11/dining/20150211REST-slide-XXPN/20150211REST-slide-XXPN-superJumbo.jpg}
\item
  \includegraphics{https://static01.graylady3jvrrxbe.onion/images/2015/02/11/dining/20150211REST-slide-XDT6/20150211REST-slide-XDT6-superJumbo.jpg}
\item
  \includegraphics{https://static01.graylady3jvrrxbe.onion/images/2015/02/11/dining/20150211REST-slide-HHQN/20150211REST-slide-HHQN-superJumbo.jpg}
\item
  \includegraphics{https://static01.graylady3jvrrxbe.onion/images/2015/02/11/dining/20150211REST-slide-W99K/20150211REST-slide-W99K-superJumbo.jpg}
\item
  \includegraphics{https://static01.graylady3jvrrxbe.onion/images/2015/02/11/dining/20150211REST-slide-UA8M/20150211REST-slide-UA8M-superJumbo.jpg}
\item
  \includegraphics{https://static01.graylady3jvrrxbe.onion/images/2015/02/11/dining/20150211REST-slide-3W6O/20150211REST-slide-3W6O-superJumbo.jpg}
\item
  \includegraphics{https://static01.graylady3jvrrxbe.onion/images/2015/02/11/dining/20150211REST-slide-5H0F/20150211REST-slide-5H0F-superJumbo.jpg}
\end{itemize}

\begin{itemize}
\tightlist
\item
  Kao Soy\\
  ★ Closed
\end{itemize}

By \href{http://www.nytimes3xbfgragh.onion/by/pete-wells}{Pete Wells}

\begin{itemize}
\item
  Feb. 10, 2015
\item
  \begin{itemize}
  \item
  \item
  \item
  \item
  \item
  \item
  \end{itemize}
\end{itemize}

Only half of the tables inside Kao Soy were taken, but the delivery guys
were coming and going through the door on Van Brunt Street every few
minutes. The December wind off the harbor that was shaking the little
brick buildings in Red Hook, Brooklyn, was keeping everybody at home,
ordering pad see ew and massaman curry from the new neighborhood Thai
place.

This turned out to be lucky for me. Five minutes later, I was facing
down a platter of shrimp with skinny dried chiles over torn lobes of
sweet pomelo that had a flaky coating of toasted coconut and crushed
peanuts. Soon after that came a big bowl of rice noodles, shrimp and
round Asian eggplants in a mild green curry that seemed to get more and
more appealing. This arrived with a separate plate of dried anchovies,
boiled eggs with almost-solid yolks and a fresh red chile sauce that had
definite intent to harm.

Then the kao soy turned up, the dish I'd come to try. This is a
pale-yellow coconut milk curry from Chiang Mai with submerged noodles
and a chicken leg, and, above the surface, fried egg noodles going off
in all directions with strands of green-papaya fritters. There was
another condiment plate for the kao soy, of course. I was eating all of
this as fast as I could, and as the overall deliciousness began to sink
in I started eating even faster. It wasn't fast enough, though, and soon
the plates covered every inch of available surface area and I needed to
slide an empty table next to mine.

In fact, I ran out of space every time I went to Kao Soy. Some blame
must be given to the plates, which are vast white things with rims
almost as wide as
\href{http://stuffnobodycaresabout.com/2012/11/30/all-new-york-city-sidewalks-are-not-created-equal/}{the
sidewalks on Park Avenue}. Some goes to Thailand, where a meal may start
with half a dozen salads and sticky rice, and where even the condiments
seem to have condiments. Some goes to the tables, which are not built
for this kind of thing. Most, though, goes to Kao Soy's menu for
inducing me to order too much (except that it was never too much) by
offering Thai dishes I rarely see around town, including a few from
northern Thailand.

Those of us who prowl restaurants whose owners come from countries known
for delicious food harbor an irrational belief that one day, if we play
it right, we may get the Other Menu. This is the one the owners save for
their family and friends. It's the one that lists all the foods they
loved back home. It's filled with the dishes everybody says Americans
won't like. Nobody I know has ever seen one of these documents. That
hasn't stopped me from raising my eyebrows and dropping key foreign
phrases in the hopes that a server will say: ``Oh, you want the Other
Menu? Just a moment, sir.''

At Kao Soy, the Other Menu is printed on the same green page that lists
the dishes you know from your local Thai spot: the fried rice, the pad
thai, the curry with choice of chicken, beef, pork or shrimp. Those
items fill the shopping bags that go out the door every few minutes.
They help keep the business floating. But the mission of Kao Soy is to
sell New Yorkers on what Sirichai Sreparplarn, one of the restaurant's
two chefs, calls ``the real food from the north.''

\includegraphics{https://static01.graylady3jvrrxbe.onion/images/2015/02/11/dining/11REST3/11REST3-articleLarge.jpg?quality=75\&auto=webp\&disable=upscale}

Several New York neighborhoods have been wrapped in the fiery clutch of
Thailand's northeastern region, Isan. Dishes from the north, though,
have been found almost exclusively in one place,
\href{http://www.nytimes3xbfgragh.onion/2012/06/27/dining/reviews/pok-pok-ny-brooklyn-restaurant-review.html}{Pok
Pok Ny}, just outside Red Hook. This puts two northern Thai kitchens
less than a mile apart in a borough that is home to only about 12
percent of the city's Thai population and virtually no other Thai
restaurants worth knowing about.

Kao Soy was opened last summer by Kanlaya Supachana, who has worked in
several Thai restaurants around Brooklyn, answering phones and such. She
and her co-chef, Mr. Sreparplarn, close friends for many years, had what
he calls ``the same passion for serving real flavor to New Yorkers,''
but they failed to persuade any existing Thai restaurants to share their
vision. Finally a local contractor, Carlos Padillo, agreed to sign on as
Ms. Supachana's business partner, and they took over a tiny space,
filled it with a few tables, a tiny bar and an electric fireplace.

The northern Thai dishes are scattered around the menu. Most obvious is
the excellent kao soy, made in the style of Ms. Supachana's father back
in Chiang Mai. Beneath the heading Snack \& Salad is a dish called nam
prik ong. This is a platter of steamed cauliflower, long beans, strips
of fried pork rinds curled like Fritos and long surfboards of romaine
lettuce, all meant as garnishes for a dip that's something like a spicy,
slightly sweet pork ragù that went for a swim in the ocean. Or maybe the
dip is the garnish; it was gone before I'd decided.

A related dish, sai oua, is filed under Various. The dip here has been
pounded from garlic, onions and fresh green chiles, hot ones. For
dipping, you get steamed young mustard greens, cucumbers, freshly fried
pork (not rinds but meat this time, and marinated in soy), along with a
house-made pork sausage, the sai oua itself. Deeply perfumed with makrut
lime leaves, coriander roots and lemongrass stalks, this sausage makes a
more complete single-bite introduction to Thai flavors than any plate of
pad thai.

For the time being, we reach the end of the northern Thai menu with the
banana blossom fritter and the kang hung le, a warmly spiced peanut
curry cooked with pork belly and beef. A fried drumstick comes on the
side, and you are meant to pull off some dark meat, pinch it with sticky
rice and then scoop up some of the curry. I didn't understand it, but I
loved it anyway.

But we have not reached the end of the Other Menu, because there are
other less-familiar dishes under the Snack \& Salad or Various headings.
Few are dull, and some are very good, like the shredded mango salad with
cashews and two kinds of dried fish, no longer than a toothpick.

Kao Soy isn't quite finished yet, Ms. Supachana will say. (When it is, I
hope she gets her hands on more of the fresh herbs that add minty,
bitter and even fishy notes to many of the salads at Pok Pok Ny; without
them, some of Kao Soy's cold dishes seem to stop short.)

There are big plans for the little bar, including a full stash of
liquor. Two weeks ago, the liquor license arrived, and already Kao Soy
has some interesting ideas about what to drink with Thai food, like
\href{http://www.povertylaneorchards.com/farnum-hill-ciders/}{Farnum
Hill cider} from New Hampshire, a Belgian-style saison ale from
\href{http://www.allagash.com/distributors/}{Allagash} in Maine, or a
red grenache from \href{http://www.donkeyandgoat.com/}{Donkey \& Goat}
in California. They're going to need bigger tables.

Advertisement

\protect\hyperlink{after-bottom}{Continue reading the main story}

\hypertarget{site-index}{%
\subsection{Site Index}\label{site-index}}

\hypertarget{site-information-navigation}{%
\subsection{Site Information
Navigation}\label{site-information-navigation}}

\begin{itemize}
\tightlist
\item
  \href{https://help.nytimes3xbfgragh.onion/hc/en-us/articles/115014792127-Copyright-notice}{©~2020~The
  New York Times Company}
\end{itemize}

\begin{itemize}
\tightlist
\item
  \href{https://www.nytco.com/}{NYTCo}
\item
  \href{https://help.nytimes3xbfgragh.onion/hc/en-us/articles/115015385887-Contact-Us}{Contact
  Us}
\item
  \href{https://www.nytco.com/careers/}{Work with us}
\item
  \href{https://nytmediakit.com/}{Advertise}
\item
  \href{http://www.tbrandstudio.com/}{T Brand Studio}
\item
  \href{https://www.nytimes3xbfgragh.onion/privacy/cookie-policy\#how-do-i-manage-trackers}{Your
  Ad Choices}
\item
  \href{https://www.nytimes3xbfgragh.onion/privacy}{Privacy}
\item
  \href{https://help.nytimes3xbfgragh.onion/hc/en-us/articles/115014893428-Terms-of-service}{Terms
  of Service}
\item
  \href{https://help.nytimes3xbfgragh.onion/hc/en-us/articles/115014893968-Terms-of-sale}{Terms
  of Sale}
\item
  \href{https://spiderbites.nytimes3xbfgragh.onion}{Site Map}
\item
  \href{https://help.nytimes3xbfgragh.onion/hc/en-us}{Help}
\item
  \href{https://www.nytimes3xbfgragh.onion/subscription?campaignId=37WXW}{Subscriptions}
\end{itemize}
