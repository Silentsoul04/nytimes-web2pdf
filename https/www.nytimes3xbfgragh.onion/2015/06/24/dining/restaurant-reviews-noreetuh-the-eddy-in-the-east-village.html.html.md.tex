Sections

SEARCH

\protect\hyperlink{site-content}{Skip to
content}\protect\hyperlink{site-index}{Skip to site index}

\href{https://www.nytimes3xbfgragh.onion/section/food}{Food}

\href{https://myaccount.nytimes3xbfgragh.onion/auth/login?response_type=cookie\&client_id=vi}{}

\href{https://www.nytimes3xbfgragh.onion/section/todayspaper}{Today's
Paper}

\href{/section/food}{Food}\textbar{}Restaurant Reviews: Noreetuh and the
Eddy in the East Village

\url{https://nyti.ms/1GEWgsN}

\begin{itemize}
\item
\item
\item
\item
\item
\item
\end{itemize}

Advertisement

\protect\hyperlink{after-top}{Continue reading the main story}

Supported by

\protect\hyperlink{after-sponsor}{Continue reading the main story}

\hypertarget{restaurant-reviews-noreetuh-and-the-eddy-in-the-east-village}{%
\section{Restaurant Reviews: Noreetuh and the Eddy in the East
Village}\label{restaurant-reviews-noreetuh-and-the-eddy-in-the-east-village}}

Slide 1 of 12

1/12

With prime ingredients, carefully organized plates and nonviolent menu
prices, Noreetuh's chef, Chung Chow, seems intent on making Hawaiian
classics presentable for their introduction to New York. Here, the tuna
poke.

Credit...Ben Russell for The New York Times

\begin{itemize}
\item
  \includegraphics{https://static01.graylady3jvrrxbe.onion/images/2015/06/24/dining/20150624REST-slide-KDH3/20150624REST-slide-KDH3-superJumbo.jpg}
\item
  \includegraphics{https://static01.graylady3jvrrxbe.onion/images/2015/06/24/dining/20150624REST-slide-TBOC/20150624REST-slide-TBOC-superJumbo.jpg}
\item
  \includegraphics{https://static01.graylady3jvrrxbe.onion/images/2015/06/24/dining/20150624REST-slide-E3HG/20150624REST-slide-E3HG-superJumbo.jpg}
\item
  \includegraphics{https://static01.graylady3jvrrxbe.onion/images/2015/06/24/dining/20150624REST-slide-Y2GR/20150624REST-slide-Y2GR-superJumbo.jpg}
\item
  \includegraphics{https://static01.graylady3jvrrxbe.onion/images/2015/06/24/dining/20150624REST-slide-X3NH/20150624REST-slide-X3NH-superJumbo.jpg}
\item
  \includegraphics{https://static01.graylady3jvrrxbe.onion/images/2015/06/24/dining/20150624REST-slide-1G4C/20150624REST-slide-1G4C-superJumbo.jpg}
\item
  \includegraphics{https://static01.graylady3jvrrxbe.onion/images/2015/06/24/dining/20150624REST-slide-UOEZ/20150624REST-slide-UOEZ-superJumbo.jpg}
\item
  \includegraphics{https://static01.graylady3jvrrxbe.onion/images/2015/06/24/dining/20150624REST-slide-VPEI/20150624REST-slide-VPEI-superJumbo.jpg}
\item
  \includegraphics{https://static01.graylady3jvrrxbe.onion/images/2015/06/24/dining/20150624REST-slide-98WH/20150624REST-slide-98WH-superJumbo.jpg}
\item
  \includegraphics{https://static01.graylady3jvrrxbe.onion/images/2015/06/24/dining/20150624REST-slide-CWTQ/20150624REST-slide-CWTQ-superJumbo.jpg}
\item
  \includegraphics{https://static01.graylady3jvrrxbe.onion/images/2015/06/24/dining/20150624REST-slide-ZD0Q/20150624REST-slide-ZD0Q-superJumbo.jpg}
\item
  \includegraphics{https://static01.graylady3jvrrxbe.onion/images/2015/06/24/dining/20150624REST-slide-MV0R/20150624REST-slide-MV0R-superJumbo.jpg}
\end{itemize}

\begin{itemize}
\tightlist
\item
  Noreetuh\\
  ★ American \$\$\$ 128 First Avenue 646-892-3050
\end{itemize}

\href{http://www.opentable.com/single.aspx?ref=4201\&rid=160957}{Reserve
a Table}

\begin{itemize}
\tightlist
\item
  The Eddy\\
  ★ Closed
\end{itemize}

When you make a reservation at an independently reviewed restaurant
through our site, we earn an affiliate commission.

By \href{http://www.nytimes3xbfgragh.onion/by/pete-wells}{Pete Wells}

\begin{itemize}
\item
  June 23, 2015
\item
  \begin{itemize}
  \item
  \item
  \item
  \item
  \item
  \item
  \end{itemize}
\end{itemize}

Any Hawaiian honeymooner who's been off the plane for more than an hour
can tell you that poke rhymes with O.K. and that tuna poke typically
means raw ahi cut up and mixed with soy sauce, seaweed and so on.

But the people at the next table were stumped when they saw it on the
menu of \href{http://www.noreetuh.com/}{Noreetuh}, which has been
offering tastes of Hawaii in the East Village since March.

``What's poke?'' one of them asked, rhyming it with Coke.

After a silence, a man sitting with her spoke up. ``It's a fish,'' he
said. Pause. ``A kind of tuna.''

This exchange suggested some of the challenges facing Noreetuh's chef,
Chung Chow, along with his partners, Jin Ahn and Gerald San Jose. Island
seafood and produce fill the walk-ins at contemporary Honolulu
restaurants like \href{http://mwrestaurant.com/}{MW} or
\href{http://thepigandthelady.com/}{the Pig and the Lady}, but they
rarely reach the island of Manhattan. A knowledgeable audience can't be
imported, either. While islanders can be counted on to know exactly
which local dish Alan Wong is playing around with, the average New
Yorker has almost no idea
\href{http://www.nytimes3xbfgragh.onion/2001/02/07/dining/en-route-hawaii-where-spam-ruled-a-cuisine-thrives.html}{what
the residents of the 50th state eat}. Don't they like \ldots{} Spam?

Sure, among other things. Mr. Chow, who was raised in Hawaii and cooked
at
\href{http://www.nytimes3xbfgragh.onion/2010/11/24/dining/reviews/24rest.html}{Lincoln
Ristorante}and
\href{http://www.nytimes3xbfgragh.onion/2011/10/12/dining/reviews/per-se-nyc-restaurant-review.html}{Per
Se}, goes out of his way to treat Hormel's arrestingly pink canned meat
product as if it were an heirloom ingredient. Stuffed into supple
agnolotti and accessorized with hon-shimeji mushrooms; semi-juicy,
soy-cured spring almonds; and a mob of bonito flakes (waving their hands
in the air like they just don't care), it could almost pass for
mortadella.

\includegraphics{https://static01.graylady3jvrrxbe.onion/images/2015/06/24/dining/24REST2/24REST2-articleLarge.jpg?quality=75\&auto=webp\&disable=upscale}

It may be the most elegant dish on Mr. Chow's menu, although it gets
stiff competition from the monkfish liver torchon, a pink button
surrounded by jellied passion fruit. Lightly pickled pears soften the
passion fruit's brassy tendencies. Mash the fruit and fish liver
together on a sweet dinner roll from
\href{http://www.kingshawaiian.com/}{King's Hawaiian bakery}, and you
have one of the most exciting tastes to wash up on Manhattan's shores
this year.

With prime ingredients, carefully organized plates and nonviolent menu
prices, Mr. Chow seems intent on making Hawaiian classics presentable
for their introduction to New York. Four or five garlic shrimp stand
side by side on a long column of rice, offering the same pleasurable
assault of garlic-flecked butter that you'd get from a big messy pile
slopped onto a paper plate by one of the shrimp trucks of Oahu. And
however you pronounce it, Mr. Chow's poke is classical, generous and
slightly Japanese-leaning, made from firm, cool bigeye tuna, diced and
slicked down with sesame oil. Frills of seaweed carry the heat of
tobanjan, the spicy Japanese bean paste.

Neatness counts for only so much, though, and some dishes are more
fastidious than flavorful. The pork in panko-crusted croquettes isn't
very emphatic, and the katsu sauce on the side needs more kick, too. A
different poke, with octopus and fingerling potatoes, has only a flicker
of the tuna's personality. White asparagus looks very nice alongside
crumbled Chinese sausage and chopped eggs, but the combination doesn't
go very far.

Noreetuh has plenty of dishes for a successful first visit. Dinner will
be particularly fun for wine lovers because Mr. Ahn has compiled an
overachieving list of German rieslings, Burgundies in both colors,
grower Champagnes and more far-flung treats. Better still, the prices
are low; just by sticking to your budget, you can drink at a higher
level than usual.

Noreetuh may have a trickier time converting new diners into regulars,
though. The two dining rooms, while tasteful enough, don't have anything
you could really call atmosphere. Noreetuh means ``playground'' in
Korean, and the owners seem to want to give a party. But where did they
find their DJ.? A Motown hit parade is followed by ``Yellow Submarine,''
and then by Taylor Swift, Natalie La Rose and, gosh, is that really
Taylor Swift, twice in one night?

At the moment, there's something a little cautious about the whole
enterprise. But the last time I went, I ate a bowl of spaghetti, new to
the menu, that points in the right direction. Richly oily smoked
butterfish was folded in among buttered noodles, bright orange with
fiery little capsules of spicy cod roe. This wasn't a polite exercise in
cultural diplomacy; it was a shut-up-and-eat dish, and if a third Taylor
Swift song had played right at that moment, I wouldn't have cared.

\hypertarget{the-eddy}{%
\subsection{The Eddy}\label{the-eddy}}

Another small, modest looking restaurant called
\href{http://www.theeddynyc.com/}{the Eddy} sits right around the
corner, on East Sixth Street's disappearing row of interchangeable curry
spots. The Eddy is so small and modest looking, in fact, that after it
opened last spring, I almost decided not to review it at all, on the
theory that when a restaurant has just over 30 seats and doesn't seem to
have any trouble filling them, there isn't much point in driving more
people there.

The chef is Brendan McHale, and I wasn't always sold on his cooking. I
would have liked some of his dishes better if they hadn't reminded me of
slightly better things I'd eaten elsewhere, like the burrata served, as
at
\href{http://www.nytimes3xbfgragh.onion/2013/09/18/dining/reviews/restaurant-review-estela-in-nolita.html}{Estela},
in an herbaceous puddle of chlorophyll. Others, like the strangely
bloated gnocchi or the dull, spongy lamb loin served with pellets of
teff spaetzle that looked like rabbit feed, were easy to forget.

But other dishes were wonderful. There were unimprovable roasted
potatoes with strips of rib-eye whose dry-aged intensity was amplified
by the low-key funk of melted Brie. And the soft-shell crab with a
peppery arugula pesto that I kept eating after the crab was gone. And
the jiggly, just-set cardamom panna cotta with crystals of rhubarb
granité on top. And the airy chips of fried beef tendon, spread with
Greek yogurt and dotted by smoky trout roe. Like Mr. McHale's bacon
tater tots, which wear little bright-green berets of puréed sweet peas,
the tendon puffs disappeared with a crunch and left behind a strong
desire for more. I would say that both hors d'oeuvres were smart
cocktail party food, but I have never been invited to a cocktail party
this smart.

And then I thought about how extremely pleasant everybody is at the
Eddy, starting with the bartenders, who look up and smile each time a
new customer walks in, as if they lived in the opening credits of a
sitcom. I remembered the drinks they made, which are as good as the ones
at a dedicated cocktail bar. I recalled the compact well-priced wine
list, and how I never picked a bottle that I didn't enjoy a little more
than I'd expected.

When I went back to the Eddy, I noticed how quickly the people I brought
settled in and got comfortable under the timbered rafters and low
ceiling that make the dining room look like a well-kept old tavern. And
how easy it was to hear one another, even when the place was crowded.

The Eddy, in other words, is one of those restaurants that gets so many
little details right that your main course can be a little shaky and you
can still walk out happy. That's a rare thing, no accident when it
happens, and the reason all those seats are so often full.

Advertisement

\protect\hyperlink{after-bottom}{Continue reading the main story}

\hypertarget{site-index}{%
\subsection{Site Index}\label{site-index}}

\hypertarget{site-information-navigation}{%
\subsection{Site Information
Navigation}\label{site-information-navigation}}

\begin{itemize}
\tightlist
\item
  \href{https://help.nytimes3xbfgragh.onion/hc/en-us/articles/115014792127-Copyright-notice}{©~2020~The
  New York Times Company}
\end{itemize}

\begin{itemize}
\tightlist
\item
  \href{https://www.nytco.com/}{NYTCo}
\item
  \href{https://help.nytimes3xbfgragh.onion/hc/en-us/articles/115015385887-Contact-Us}{Contact
  Us}
\item
  \href{https://www.nytco.com/careers/}{Work with us}
\item
  \href{https://nytmediakit.com/}{Advertise}
\item
  \href{http://www.tbrandstudio.com/}{T Brand Studio}
\item
  \href{https://www.nytimes3xbfgragh.onion/privacy/cookie-policy\#how-do-i-manage-trackers}{Your
  Ad Choices}
\item
  \href{https://www.nytimes3xbfgragh.onion/privacy}{Privacy}
\item
  \href{https://help.nytimes3xbfgragh.onion/hc/en-us/articles/115014893428-Terms-of-service}{Terms
  of Service}
\item
  \href{https://help.nytimes3xbfgragh.onion/hc/en-us/articles/115014893968-Terms-of-sale}{Terms
  of Sale}
\item
  \href{https://spiderbites.nytimes3xbfgragh.onion}{Site Map}
\item
  \href{https://help.nytimes3xbfgragh.onion/hc/en-us}{Help}
\item
  \href{https://www.nytimes3xbfgragh.onion/subscription?campaignId=37WXW}{Subscriptions}
\end{itemize}
