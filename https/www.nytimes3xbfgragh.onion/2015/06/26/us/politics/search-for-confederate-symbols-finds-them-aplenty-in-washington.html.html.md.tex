Sections

SEARCH

\protect\hyperlink{site-content}{Skip to
content}\protect\hyperlink{site-index}{Skip to site index}

\href{https://www.nytimes3xbfgragh.onion/section/politics}{Politics}

\href{https://myaccount.nytimes3xbfgragh.onion/auth/login?response_type=cookie\&client_id=vi}{}

\href{https://www.nytimes3xbfgragh.onion/section/todayspaper}{Today's
Paper}

\href{/section/politics}{Politics}\textbar{}Search for Confederate
Symbols Finds Them Aplenty in Washington

\url{https://nyti.ms/1Lr9YD0}

\begin{itemize}
\item
\item
\item
\item
\item
\end{itemize}

Advertisement

\protect\hyperlink{after-top}{Continue reading the main story}

Supported by

\protect\hyperlink{after-sponsor}{Continue reading the main story}

Congressional Memo

\hypertarget{search-for-confederate-symbols-finds-them-aplenty-in-washington}{%
\section{Search for Confederate Symbols Finds Them Aplenty in
Washington}\label{search-for-confederate-symbols-finds-them-aplenty-in-washington}}

\includegraphics{https://static01.graylady3jvrrxbe.onion/images/2015/06/26/us/26memo-web3/26memo-web3-articleLarge.jpg?quality=75\&auto=webp\&disable=upscale}

By
\href{http://www.nytimes3xbfgragh.onion/by/jennifer-steinhauer}{Jennifer
Steinhauer}

\begin{itemize}
\item
  June 25, 2015
\item
  \begin{itemize}
  \item
  \item
  \item
  \item
  \item
  \end{itemize}
\end{itemize}

WASHINGTON --- Just outside a dining room where senators gather for
lunch hangs a large painting of John C. Calhoun, the 19th-century South
Carolina statesman who once called slavery ``indispensable to the peace
and happiness'' of Americans.

A statue depicting Mr. Calhoun midstride stands in the Capitol along
with figures of Confederate heroes like Jefferson Davis, president of
the Confederate States of America, and Wade Hampton III, once one of the
country's largest slaveholders, who made common cause with violent white
supremacists.

``That Calhoun statue is really strutting along; that's among the most
offensive,'' said Representative Keith Ellison, Democrat of Minnesota.
Yet it is one of many commemorations of the nation's Confederate history
on Capitol Hill, where a movement has begun to retire the statues to
museums and replace them with ones depicting less divisive figures.

Almost overnight, the push to remove Confederate symbols has become a
bit like the rush to support
\href{http://topics.nytimes3xbfgragh.onion/top/reference/timestopics/subjects/s/same_sex_marriage/index.html?inline=nyt-classifier}{same-sex
marriage}. Although one effort stems from a precipitating event and the
other from gradual social forces, in both cases politicians are
increasingly finding that one side of the divide is the least
comfortable place to stand.

There are few places where those symbols are in more abundance than at
the United States Capitol, where millions of tourists flock each year to
take in the living history of America.

Senator Mitch McConnell of Kentucky, the majority leader, said Tuesday
that a Jefferson Davis statue in his state's capital, Frankfort, ought
to be moved, opening the door for similar discussions in Washington. The
talk spread to the Pentagon, where officials said that military bases
named for Confederate officers reflected the ``spirit of reconciliation,
not division.''

!{[}A statue of John C. Calhoun, a 19th century South Carolina statesman
and slavery supporter.

Credit...Zach Gibson/The New York
Times{]}(\url{https://static01.graylady3jvrrxbe.onion/images/2015/06/26/us/26memo-web/26memo-web-articleLarge.jpg?quality=75\&auto=webp\&disable=upscale})

On Wednesday, Representative Kathy Castor, Democrat of Florida, called
for replacing the statue of a Confederate general, Kirby Smith, which
has stood in the Capitol since 1922. At their weekly meeting, House
Democrats discussed a possible relocation of the most prominently placed
Confederate statues.

Black members of Congress have quickly embraced the idea of removing
Confederate symbols.

``I just think the statues don't need to be on government property that
represent Confederate officers,'' said Representative G. K. Butterfield,
Democrat of North Carolina and chairman of the Congressional Black
Caucus. ``It is the moment to have that conversation.''

On Thursday morning, Representative
\href{http://www.rollcall.com/members/280.html}{Bennie Thompson},
Democrat of Mississippi, brought a measure to the House floor calling
for the removal of any state flags that feature the ``Southern Cross''
of the Confederacy from areas maintained by the House where flags are
displayed, saying that ``it is an uncontroverted fact that symbols of
the Confederacy offend and insult many members of the general public who
use the hallways of Congress each day.''

In a speech on the House floor, he added: ``We should not identify with
symbols of hatred and bigotry. I saw what happened in Charleston, S.C.,
last Wednesday. The whole world saw. And they did not like it. So this
is one step toward getting us healed as a nation.''

The House defeated the measure by instead referring it to the committee
that sets House rules for further consideration, with a vote almost
totally divided along party lines.

The matter has been bubbling under the surface for years, largely among
Democrats. When she was the speaker of the House, Representative Nancy
Pelosi of California quietly had the statue of Robert E. Lee moved from
\href{http://www.aoc.gov/capitol-buildings/national-statuary-hall}{National
Statuary Hall} to a more remote area in the Capitol known as the crypt.
In his stead, more or less, is now a statue of Rosa Parks, which Ms.
Pelosi pushed to secure.

\href{https://www.nytimes3xbfgragh.onion/interactive/2015/06/23/us/Calls-to-Cut-Ties-to-Symbols-of-the-South.html}{}

\includegraphics{https://static01.graylady3jvrrxbe.onion/images/2015/06/24/us/24charlestonlisty5/24charlestonlisty5-videoLarge.jpg}

\hypertarget{controversial-confederate-symbols}{%
\subsection{Controversial Confederate
Symbols}\label{controversial-confederate-symbols}}

The Confederate battle flag and other symbols of the Confederacy have
come under renewed criticism. Here are recent protests and calls for the
removal of Confederate symbols from public places.

``It is an issue for members of Congress,'' Ms. Pelosi said. ``I've
tried to do my part: Rosa Parks, Sojourner Truth, Hellen Keller ---
we've tried to add women and minorities when we've had the chance.''

\href{http://www.aoc.gov/capitol-hill/national-statuary-hall-collection/nsh-location.}{Each
state}is allowed two statues, scattered throughout the Capitol. The
collection of state statues --- established in the 1860s, with the first
statue arriving at the Capitol in 1870 --- has long contained figures
from the Civil War, from both the Union and the Confederate sides.

But even when Congress was passing sweeping civil rights laws, over the
fervent opposition of Southern lawmakers, many of them Democrats, there
was never a broad push to remove the statues.

The killings of nine black churchgoers last week in Charleston, S.C.,
which the federal authorities have labeled a hate crime, have provided a
tipping point. ``I think it is something we should consider,'' said
Senator Harry Reid, Democrat of Nevada. ``I think it is something that
should be done in a deliberate fashion.''

Should a state wish to replace one of its statues, a state official,
usually the governor, must make a request to the Architect of the
Capitol to provide a new one.

For many lawmakers from the Deep South, the discussion is both welcome
and painful. ``We cannot eradicate our history,'' said Senator Richard
C. Shelby, Republican of Alabama. His state's statues are for Helen
Keller and Joseph Wheeler, a Confederate general whose statue has a belt
buckle reading C.S.A.

\href{https://www.nytimes3xbfgragh.onion/interactive/2015/06/22/us/Divisive-Symbolism-of-a-Southern-Flag.html}{}

\includegraphics{https://static01.graylady3jvrrxbe.onion/images/2015/06/22/us/Flag-Listy-1/Flag-Listy-1-videoLarge.jpg}

\hypertarget{divisive-symbolism-of-a-southern-flag}{%
\subsection{Divisive Symbolism of a Southern
Flag}\label{divisive-symbolism-of-a-southern-flag}}

Supporters of the flag have said they view the shooting and the flag as
unrelated. But Cornell William Brooks, the national president of the
N.A.A.C.P., has called it an emblem of hate.

Mr. Shelby did not directly call for the removal of the Wheeler statue,
but added, ``We need to focus on the things that unite us.''

Some of the statues depict figures known for particularly venomous
rhetoric. Charles Brantley Aycock, represented on one of North
Carolina's statues, ran for governor on a vow to keep schools segregated
and disenfranchise black voters.

``We solved the Negro problem,'' he once said. ``We have taken him out
of politics.''

But the issue is also complicated, particularly for some Southern
Democrats, whose numbers have been dropping in Congress by the dozens in
recent years. ``I think it's a good discussion to have,'' said Senator
Tim Kaine, Democrat of Virginia, who said that as both mayor of Richmond
and governor of his home state he dealt often with the issue.

``Recognizing history is an important thing, even the unflattering
parts,'' Mr. Kaine said, adding that when old bridges named after
Confederate figures were rebuilt, they were often renamed for civil
rights leaders. ``I don't think you rewrite your history root and
branch, but you don't celebrate'' its worst parts, he said.

Some lawmakers have decided that they no longer support the use of the
Confederate flag, but that statues are different.

``If Ohio wants to put up a statue of William Tecumseh Sherman,'' said
Representative Mick Mulvaney, Republican of South Carolina, ``there
would be plenty of people in my district that would find that
disturbing, given what he did in South Carolina. But that would be
Ohio's right, and I would respect Ohio's decision. I don't think the
conversation about the flag translates to the statues.''

Some statues are well hidden. Three Confederate figures, including Wade
Hampton of South Carolina, who was supported by the so-called Red Shirts
accused of suppressing the black vote, stand tucked away in a remote
area of the Capitol Visitor Center.

On Wednesday, Capitol Hill felt infused with South Carolina. As his
fellow senators stood in silence, Senator Tim Scott, Republican of South
Carolina, at times choking up, introduced a resolution to honor the
Charleston shooting victims and read their names. ``Our future as a
nation has been changed,'' he said. ``It has been changed because one
person decided to murder nine.''

Advertisement

\protect\hyperlink{after-bottom}{Continue reading the main story}

\hypertarget{site-index}{%
\subsection{Site Index}\label{site-index}}

\hypertarget{site-information-navigation}{%
\subsection{Site Information
Navigation}\label{site-information-navigation}}

\begin{itemize}
\tightlist
\item
  \href{https://help.nytimes3xbfgragh.onion/hc/en-us/articles/115014792127-Copyright-notice}{©~2020~The
  New York Times Company}
\end{itemize}

\begin{itemize}
\tightlist
\item
  \href{https://www.nytco.com/}{NYTCo}
\item
  \href{https://help.nytimes3xbfgragh.onion/hc/en-us/articles/115015385887-Contact-Us}{Contact
  Us}
\item
  \href{https://www.nytco.com/careers/}{Work with us}
\item
  \href{https://nytmediakit.com/}{Advertise}
\item
  \href{http://www.tbrandstudio.com/}{T Brand Studio}
\item
  \href{https://www.nytimes3xbfgragh.onion/privacy/cookie-policy\#how-do-i-manage-trackers}{Your
  Ad Choices}
\item
  \href{https://www.nytimes3xbfgragh.onion/privacy}{Privacy}
\item
  \href{https://help.nytimes3xbfgragh.onion/hc/en-us/articles/115014893428-Terms-of-service}{Terms
  of Service}
\item
  \href{https://help.nytimes3xbfgragh.onion/hc/en-us/articles/115014893968-Terms-of-sale}{Terms
  of Sale}
\item
  \href{https://spiderbites.nytimes3xbfgragh.onion}{Site Map}
\item
  \href{https://help.nytimes3xbfgragh.onion/hc/en-us}{Help}
\item
  \href{https://www.nytimes3xbfgragh.onion/subscription?campaignId=37WXW}{Subscriptions}
\end{itemize}
