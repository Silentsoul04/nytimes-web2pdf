Sections

SEARCH

\protect\hyperlink{site-content}{Skip to
content}\protect\hyperlink{site-index}{Skip to site index}

\href{https://www.nytimes3xbfgragh.onion/section/food}{Food}

\href{https://myaccount.nytimes3xbfgragh.onion/auth/login?response_type=cookie\&client_id=vi}{}

\href{https://www.nytimes3xbfgragh.onion/section/todayspaper}{Today's
Paper}

\href{/section/food}{Food}\textbar{}Untitled at the Whitney in the
Meatpacking District

\url{https://nyti.ms/1IhYnmZ}

\begin{itemize}
\item
\item
\item
\item
\item
\item
\end{itemize}

Advertisement

\protect\hyperlink{after-top}{Continue reading the main story}

Supported by

\protect\hyperlink{after-sponsor}{Continue reading the main story}

\href{/column/restaurant-review}{Restaurant Review}

\hypertarget{untitled-at-the-whitney-in-the-meatpacking-district}{%
\section{Untitled at the Whitney in the Meatpacking
District}\label{untitled-at-the-whitney-in-the-meatpacking-district}}

\href{https://www.nytimes3xbfgragh.onion/slideshow/2015/08/05/dining/untitled-at-the-whitney.html}{}

\hypertarget{untitled}{%
\subsection{Untitled}\label{untitled}}

11 Photos

View Slide Show ›

\includegraphics{https://static01.graylady3jvrrxbe.onion/images/2015/08/05/dining/RESTAURANT-slide-WUJW/RESTAURANT-slide-WUJW-articleLarge.jpg?quality=75\&auto=webp\&disable=upscale}

Ben Russell for The New York Times

\begin{itemize}
\tightlist
\item
  Untitled\\
  ★★ American \$\$\$ 99 Gansevoort Street 212-570-3670
\end{itemize}

\href{http://www.opentable.com/single.aspx?ref=4201\&rid=169186}{Reserve
a Table}

When you make a reservation at an independently reviewed restaurant
through our site, we earn an affiliate commission.

By \href{http://www.nytimes3xbfgragh.onion/by/pete-wells}{Pete Wells}

\begin{itemize}
\item
  Aug. 4, 2015
\item
  \begin{itemize}
  \item
  \item
  \item
  \item
  \item
  \item
  \end{itemize}
\end{itemize}

If you can get past the name, a graduate-school groaner that I am
supposed to underline, but won't, there is nothing pretentious about
Untitled.

\href{http://www.untitledatthewhitney.com/}{Untitled} is on the ground
floor of the Whitney Museum of American Art. The Whitney is the second
major art museum in the city to choose Danny Meyer to run a big,
semiformal restaurant --- the Museum of Modern Art is the other --- and
yet Mr. Meyer is resolutely uninterested in arty food. When
\href{http://elevenmadisonpark.com/\#/menus}{Eleven Madison Park} began
its swerve toward the conceptual, he sold it. The remaining holdings in
his Union Square Hospitality Group are polite restaurants where polite
people are in very little danger of being challenged or provoked.
Although Mr. Meyer has worked to make his name synonymous with
hospitality, what his restaurants sell above all is reassurance.

Imagine his relief that the Whitney decided against installing
\href{http://www.chicagomag.com/arts-culture/June-2015/A-Homosocial-Sculpture-by-Charles-Ray-Debuts-at-the-Art-Institute-of-Chicago/}{Charles
Ray's sculpture of a naked Huck Finn}, bending over next to an 8-foot
full-frontal Jim, on the plaza in front of Untitled. (Next door at
Santina, which plays Nicki Minaj to Untitled's Taylor Swift, they would
have rearranged the tables for better views.)

The restaurant is slotted into a narrow quadrangle with glass curtain
walls on three sides, designed, like the rest of the museum, by Renzo
Piano. Untitled treats the architect deferentially. Too deferentially, I
think: In its near-total lack of ornament, the dining room can look like
an espresso shop. If Eero Saarinen's cardinal-red chairs weren't so
comfortable, you might get antsy after an hour. An architect friend
spent the night admiring Mr. Piano's meticulous engineering, but also
pointed out that interiors with curtain walls are only as beautiful as
the building across the street. At Untitled, you look at a hulking
industrial slab across Gansevoort Street. At least it is an improvement
on the view from my old desk at Mr. Piano's New York Times building,
where I peered into the dark soul of the Port Authority Bus Terminal.

All the energy and beauty at Untitled are on the plates. They throb with
color. It's not just decoration, either. The color mostly comes from
fruits and vegetables so ripe they're ready to pop. Michael Anthony, the
chef; Suzanne Cupps, the chef de cuisine; and Miro Uskokovic, the pastry
chef, use market produce flamboyantly, as though they were trying to get
the purse-lipped farm-to-table puritans who solemnly hand you a single
baby zucchini to crack a smile.

\includegraphics{https://static01.graylady3jvrrxbe.onion/images/2015/08/05/dining/05REST3/05REST3-articleLarge.jpg?quality=75\&auto=webp\&disable=upscale}

A slice of the tender, rich poundcake is virtually swallowed up by
ricotta, sabayon, strawberries and violas. More strawberries --- and
nearly every other berry under the summer sun --- ring the gloriously
slouching chamomile panna cotta in a great purple landslide. Around a
towering ship's-prow wedge of cake with sesame brittle and peanut-butter
icing, servers pour a blueberry sauce with the spicy buzz of ginger,
which rockets the dessert right out of PB\&J territory.

The four menu categories aren't labeled, but it's obvious that the third
section is turned over to vegetables. A few are best as side dishes,
like the spoonable potato purée liquefied with melted Cheddar, too salty
for more than a few bites. Most of the vegetables, though, have enough
contrast and sophistication to be appetizers or even main courses.

Pickled wine-colored cherries, sunflower seeds and orange splashes of
carrot vinaigrette make every bite of a kale-and-cabbage salad taste
like a new dish. (If we are sentenced to see raw kale everywhere we go,
every restaurant should dress it as exuberantly as Untitled does.)
Earlier this summer, there were roasted and griddled leeks, as dark as
roasted Japanese eggplants and almost that soft, with an intense,
sweet-sour salsa sauce of citrus and pasilla chiles; I've never wanted
to cheer for a plate of leeks before.

Every taste tells you how carefully (and recently) these chefs have done
their shopping. A dinner companion was convinced that some ingredient
had been injected into the flat beans to make them taste so alluring,
and she didn't think it had anything to do with the baby squid,
hazelnuts and ancho chile sauce on the plate. But no, these were just
excellent beans, grilled quickly so they still had some snap and juice.

You can tell that the rotisserie chicken had great flavor down to its
core before it took its turn on the spit. If you're not sure, try the
fried chicken that comes on the same plate. Under the airy, crackling
crust that owes something to Japan and Korea, there's very fine meat.

If every dish were this good, Untitled might rank up there with Gramercy
Tavern, where Mr. Anthony and Mr. Uskokovic hold the same titles they do
here. But the kitchen isn't there yet.

One night I'd brought along a native of Owensboro, Ky., the smoked
mutton capital of the world. But it didn't take a barbecue authority to
know the smoked pork ribs were tough and undercooked, and coated in a
paste that didn't taste of anything but salt.

It was the only real disaster. In other dishes, the worst you could say
is that the cooks packed too much into their shopping bags. The flavor
of swordfish steaks disappears into a mashed eggplant that was a little
too sharp and salty, and the taste of sea scallops can't hold its head
up in a bowl of sweet watermelon gazpacho with lemon cucumbers and
peaches. At times the plates had so much going on that they left you
with only a blurry impression of deliciousness. But as blurry
impressions go, that one is hard to beat.

At the end of one dinner, Mr. Anthony stopped by my table. (I'd been
spotted long ago.) He was enthusiastic about how the gray limestone
floors and white-oak counter catch the light during the day. He was
slightly less enthusiastic about the kitchen, which is as narrow as a
scallion. When he said he and Ms. Cupps were taking inspiration from the
overflowing planters up on the High Line, I understood the botanical
profusion of their plates.

Maybe the wine director, Eduardo Porto Carreiro, can find some
inspiration up there, too. Native plants grow on the High Line, and
American art fills the Whitney, but Untitled's wine list genuflects
toward Europe. You can drink very well without spending a fortune, but
it could get interesting if Mr. Porto Carreiro took his cues from the
location.

Untitled is not really meant for museum visitors, who are much more
likely to recuperate with an avocado toast at Mr. Meyer's Studio Cafe on
the eighth floor. Instead, the restaurant is one of the attractions the
museum is peddling, part of its multipronged campaign to be seen as a
neighborhood hot spot and not just some boring shed where there's
nothing to do but look at art.

Sometimes I miss those boring sheds, though I don't miss the school
lunchroom smell of their cafeterias. And at least the Whitney's urge to
pump itself up with crowds has given us Untitled.

Advertisement

\protect\hyperlink{after-bottom}{Continue reading the main story}

\hypertarget{site-index}{%
\subsection{Site Index}\label{site-index}}

\hypertarget{site-information-navigation}{%
\subsection{Site Information
Navigation}\label{site-information-navigation}}

\begin{itemize}
\tightlist
\item
  \href{https://help.nytimes3xbfgragh.onion/hc/en-us/articles/115014792127-Copyright-notice}{©~2020~The
  New York Times Company}
\end{itemize}

\begin{itemize}
\tightlist
\item
  \href{https://www.nytco.com/}{NYTCo}
\item
  \href{https://help.nytimes3xbfgragh.onion/hc/en-us/articles/115015385887-Contact-Us}{Contact
  Us}
\item
  \href{https://www.nytco.com/careers/}{Work with us}
\item
  \href{https://nytmediakit.com/}{Advertise}
\item
  \href{http://www.tbrandstudio.com/}{T Brand Studio}
\item
  \href{https://www.nytimes3xbfgragh.onion/privacy/cookie-policy\#how-do-i-manage-trackers}{Your
  Ad Choices}
\item
  \href{https://www.nytimes3xbfgragh.onion/privacy}{Privacy}
\item
  \href{https://help.nytimes3xbfgragh.onion/hc/en-us/articles/115014893428-Terms-of-service}{Terms
  of Service}
\item
  \href{https://help.nytimes3xbfgragh.onion/hc/en-us/articles/115014893968-Terms-of-sale}{Terms
  of Sale}
\item
  \href{https://spiderbites.nytimes3xbfgragh.onion}{Site Map}
\item
  \href{https://help.nytimes3xbfgragh.onion/hc/en-us}{Help}
\item
  \href{https://www.nytimes3xbfgragh.onion/subscription?campaignId=37WXW}{Subscriptions}
\end{itemize}
