Sections

SEARCH

\protect\hyperlink{site-content}{Skip to
content}\protect\hyperlink{site-index}{Skip to site index}

\href{https://www.nytimes3xbfgragh.onion/section/politics}{Politics}

\href{https://myaccount.nytimes3xbfgragh.onion/auth/login?response_type=cookie\&client_id=vi}{}

\href{https://www.nytimes3xbfgragh.onion/section/todayspaper}{Today's
Paper}

\href{/section/politics}{Politics}\textbar{}Hand-Wringing in G.O.P.
After Donald Trump's Remarks on Megyn Kelly

\url{https://nyti.ms/1PddiSf}

\begin{itemize}
\item
\item
\item
\item
\item
\item
\end{itemize}

Advertisement

\protect\hyperlink{after-top}{Continue reading the main story}

Supported by

\protect\hyperlink{after-sponsor}{Continue reading the main story}

\hypertarget{hand-wringing-in-gop-after-donald-trumps-remarks-on-megyn-kelly}{%
\section{Hand-Wringing in G.O.P. After Donald Trump's Remarks on Megyn
Kelly}\label{hand-wringing-in-gop-after-donald-trumps-remarks-on-megyn-kelly}}

\includegraphics{https://static01.graylady3jvrrxbe.onion/images/2015/08/09/us/09repubs-JP/09repubs-JP-articleLarge.jpg?quality=75\&auto=webp\&disable=upscale}

By \href{http://www.nytimes3xbfgragh.onion/by/jonathan-martin}{Jonathan
Martin} and
\href{http://www.nytimes3xbfgragh.onion/by/maggie-haberman}{Maggie
Haberman}

\begin{itemize}
\item
  Aug. 8, 2015
\item
  \begin{itemize}
  \item
  \item
  \item
  \item
  \item
  \item
  \end{itemize}
\end{itemize}

\href{http://www.nytimes3xbfgragh.onion/interactive/2015/06/16/us/elections/donald-trump.html?inline=nyt-per}{Donald
J. Trump}'s suggestion that a Fox News journalist had questioned him
forcefully at the Republican presidential debate because she was
menstruating cost him a speaking slot Saturday night at an influential
gathering of conservatives in Atlanta. It also raised new questions
about how much longer
\href{http://topics.nytimes3xbfgragh.onion/top/reference/timestopics/organizations/r/republican_party/index.html?inline=nyt-org}{Republican
Party} leaders would have to contend with Mr. Trump's disruptive
presence in the primary field.

Continuing his complaints about Megyn Kelly, one of the moderators of
the debate, in an interview on CNN Friday night, Mr. Trump said, ``You
could see there was blood coming out of her eyes, blood coming out of
her wherever.'' The remark prompted Erick Erickson, the leader of
RedState, the conservative group, to disinvite him.

``If your standard-bearer has to resort to that,'' Mr. Erickson told
hundreds of conservative activists in a packed Atlanta hotel ballroom on
Saturday, ``we need a new standard-bearer.''

With Mr. Trump
\href{http://www.nytimes3xbfgragh.onion/2015/08/07/us/politics/rivals-jab-at-donald-trump-as-gop-debate-becomes-testy.html}{at
center stage} Thursday in Cleveland, Fox News shattered television
viewership records for a primary debate: Nearly 24 million people
watched. But any hopes that Mr. Trump, the real estate developer and
television celebrity, would try to reinvent himself as a sober-minded
statesman, or that he would collapse under scrutiny and tough questions,
vaporized in the opening minutes when he refused to rule out running as
an independent candidate for president. His remarks Friday only
furthered the impression that he also had no intention of speaking more
carefully. Mr. Trump denied on Saturday that he had been implying that
Ms. Kelly was menstruating. ``I think only a degenerate would think that
I would have meant that,'' he said in an interview, insisting that he
had been referring to Ms. Kelly's nose and ears.

But as his latest eruption rippled through Republican circles, the
conversation
\href{http://www.nytimes3xbfgragh.onion/2015/08/08/us/politics/candidates-continue-to-plead-their-cases-after-first-republican-debate.html}{turned}
to whether the party, and his rival presidential contenders, should
continue to accommodate his candidacy, quietly hoping that this would be
the moment he burned out --- or whether they should try to run him out
on a rail.

If party leaders saw danger in provoking a breakup --- and no small
advantage to be seized from the ratings bonanza Mr. Trump showed himself
capable of delivering --- there were signs that other influential
Republicans would tolerate only so much of Mr. Trump's behavior.

``Come on,'' Jeb Bush, who has campaigned as the adult among the party's
17 presidential candidates, said in his remarks at the RedState
gathering. ``Give me a break. Do we want to win? Do we want to insult 53
percent of all voters? What Donald Trump said is wrong.''

Carly Fiorina, the former Hewlett-Packard chief executive, who delivered
perhaps the most assertive turn in Thursday's debate among the
candidates
\href{http://www.nytimes3xbfgragh.onion/2015/08/07/us/politics/before-main-republican-debate-bottom-7-contenders-put-on-brave-faces.html}{trailing
in the polls},
\href{https://twitter.com/CarlyFiorina/status/629860026916716545}{posted}
on Twitter: ``Mr. Trump: There. Is. No. Excuse.'' Senator Lindsey Graham
of South Carolina went further, saying, ``Enough already with Mr.
Trump.''

Yet in a sign of the lingering reluctance among some in the field to
anger Mr. Trump's supporters, other candidates, including former Gov.
Mike Huckabee of Arkansas and Senator Ted Cruz of Texas, would not
condemn Mr. Trump's comments.

Senator Marco Rubio of Florida, in an interview to be broadcast Sunday,
praised Ms. Kelly but stopped short of calling on Mr. Trump to
apologize.~``I've made a decision here with Donald Trump, you know, if I
comment on everything he says, my whole campaign will be consumed by
it,'' he said. ``That's all I'll do all day.''

\href{https://www.nytimes3xbfgragh.onion/interactive/2016/us/elections/2016-presidential-candidates.html}{}

\includegraphics{https://static01.graylady3jvrrxbe.onion/images/2015/01/30/us/politics/presidential-candidate-tracker-1422646394170/presidential-candidate-tracker-1422646394170-videoLarge-v8.jpg}

\hypertarget{who-is-running-for-president}{%
\subsection{Who Is Running for
President?}\label{who-is-running-for-president}}

Donald J. Trump officially accepted the Republican party's nomination on
July 22. Hillary Clinton was officially nominated on July 26 at the
Democratic Convention.

Mr. Rubio made the comments to Chuck Todd in a taped interview for NBC's
``Meet the Press.''

Mr. Erickson --- an author with his own track record of inflammatory
remarks, sometimes about women --- announced just before midnight Friday
\href{http://www.redstate.com/2015/08/07/i-have-disinvited-donald-trump-to-the-redstate-gathering/}{on
the RedState website} that he was disinviting Mr. Trump. He wrote that
he admired Mr. Trump for his bluntness and for connecting with ``so much
of the anger in the Republican base.''

``But there are even lines blunt talkers and unprofessional politicians
should not cross,'' he wrote. ``Decency is one of those lines.''

He added, ``I just don't want someone on stage who gets a hostile
question from a lady and his first inclination is to imply it was
hormonal.''

Mr. Trump's campaign shot back at Mr. Erickson, calling him a ``weak and
pathetic leader'' and his decision ``another example of weakness through
being politically correct.''

But~on Saturday, Mr. Trump seemed to recognize that he needed to create
a diversion from the latest controversy and declared that he had fired
Roger Stone, a Republican strategist and mischief-maker who had long
been an adviser. But Mr. Stone said in an interview that he had resigned
because the tenor of the campaign was distracting from Mr. Trump's core
message.

In an interview with The New York Times on Friday afternoon, before he
went on CNN, Mr. Trump said he was irritated by the debate moderators'
questions about a third-party candidacy, saying he wanted to run as a
Republican, but he reiterated his threat to mount one if he is unhappy
with his treatment by party leaders. An independent candidacy would be
complicated and costly, he said, but ``if you're rich, it's doable.''

Some Republicans were trying to determine just who was rallying to Mr.
Trump's side, and how damaging it would be if his supporters left the
party's fold.

``Trump isn't and wasn't going to get the conservative vote,'' Joseph W.
McQuaid, publisher of the Union Leader newspaper in New Hampshire, said
in an email. ``Conservative Republicans are worried about their party,
but it's still their party. Trump isn't philosophically a conservative,
and that will come out.''

``Trump's base is more the people who used to have season tickets to the
Roman Colosseum,'' Mr. McQuaid wrote. ``Not sure that they vote in great
numbers, but they like blood sport.''

But others on the right said the disaffected voters rallying to Mr.
Trump represented a constituency that Republicans would be foolish to
ignore.

``People have to get their minds wrapped around the fact that the
seething fury at the leadership of the Republican Party is real, and
it's going to bubble over somehow with somebody, and right now it's with
Trump,'' said the conservative talk show host Laura Ingraham, noting
that there were ``a lot of ticked-off people out there who are willing
to throw both parties into the fire.''

Mr. Erickson himself got a taste of Mr. Trump's die-hard loyalists,
recounting in his speech at the conference Saturday that he had received
a vitriolic response to his decision to bar Mr. Trump.

``We will not gain the White House,'' said Mr. Erickson about the
inflammatory messages he had received, ``if we're screaming at people,
calling them whores and queer and the N-word.''

Even before Friday night, prominent Republican women said they were
worried about how female voters
\href{http://www.nytimes3xbfgragh.onion/2015/08/08/us/politics/fear-that-debate-could-hurt-gop-in-womens-eyes.html}{would
respond} to Mr. Trump's prominence on the debate stage, where he
defended imprecations like ``fat pigs'' and ``bimbo'' to describe women
--- and his rivals did not chide him.

But Mr. Trump's comment Friday night about Ms. Kelly caused a new bout
of consternation among senior Republican leaders, who saw it as the
latest distraction from the business of choosing a presidential
candidate who can return the White House to the party in 2016.

``We need to nominate somebody who can win, somebody who is substantive
and somebody who knows how to govern,'' said former Senator Judd Gregg
of New Hampshire. ``But we can't have that debate in the full jacket as
long as we're sidetracked off on this Trump exercise. It does undermine
our ability to have a substantive debate. All the substantive arguments
are being muted by his persona.''

Still, Mr. Gregg said that while RedState was wise to bar him from its
event, the party would only ``make him an even larger figure'' by trying
to keep him out of future debates. ``He'd love that,'' Mr. Gregg said.
``He loves when institutional forces take him on. That's part of his
shtick.''

He added: ``The campaign is serious, but his campaign isn't. It's
entertainment. What's the line of decency in the entertainment world?
It's pretty far out there.''

Some in the party have mused privately about using Mr. Trump's refusal
to rule out an independent bid as grounds to bar him from future
debates, but there is deep concern that such a heavy-handed effort would
only prod him into pursuing such a run.

It also appeared unlikely that any network could be persuaded to exclude
him. As Mr. Trump crowed Friday in a telephone interview, ``I'm a
ratings machine.''

In Atlanta, there was resignation to the likelihood that Mr. Trump would
continue to draw support from voters looking to rage against the
political establishment. But some of the grass-roots activists in
attendance described him as a jester, not a threat.

David Pettigrew, a retiree from Milledgeville, Ga., said he knew many
conservatives who regretted voting for Ross Perot in 1992 out of
frustration with President George Bush. (Mr. Perot's third-party
candidacy is widely believed to have helped tilt the election to Bill
Clinton.) But he said he doubted Mr. Trump could win over enough
disgruntled Republicans to undermine the party's nominee.

``Hell, if he wants to run as a third party, have at it,'' Mr. Pettigrew
said.

Mr. Trump, for his part, fired off
\href{https://twitter.com/realDonaldTrump/status/630007166129303552}{a
defiant salute} to the RedState crowd on Twitter Saturday morning: ``I
miss you all, and thanks for all of your support. Political correctness
is killing our country.''

He added in a word what he thought of his critics: ``weakness.''

Advertisement

\protect\hyperlink{after-bottom}{Continue reading the main story}

\hypertarget{site-index}{%
\subsection{Site Index}\label{site-index}}

\hypertarget{site-information-navigation}{%
\subsection{Site Information
Navigation}\label{site-information-navigation}}

\begin{itemize}
\tightlist
\item
  \href{https://help.nytimes3xbfgragh.onion/hc/en-us/articles/115014792127-Copyright-notice}{©~2020~The
  New York Times Company}
\end{itemize}

\begin{itemize}
\tightlist
\item
  \href{https://www.nytco.com/}{NYTCo}
\item
  \href{https://help.nytimes3xbfgragh.onion/hc/en-us/articles/115015385887-Contact-Us}{Contact
  Us}
\item
  \href{https://www.nytco.com/careers/}{Work with us}
\item
  \href{https://nytmediakit.com/}{Advertise}
\item
  \href{http://www.tbrandstudio.com/}{T Brand Studio}
\item
  \href{https://www.nytimes3xbfgragh.onion/privacy/cookie-policy\#how-do-i-manage-trackers}{Your
  Ad Choices}
\item
  \href{https://www.nytimes3xbfgragh.onion/privacy}{Privacy}
\item
  \href{https://help.nytimes3xbfgragh.onion/hc/en-us/articles/115014893428-Terms-of-service}{Terms
  of Service}
\item
  \href{https://help.nytimes3xbfgragh.onion/hc/en-us/articles/115014893968-Terms-of-sale}{Terms
  of Sale}
\item
  \href{https://spiderbites.nytimes3xbfgragh.onion}{Site Map}
\item
  \href{https://help.nytimes3xbfgragh.onion/hc/en-us}{Help}
\item
  \href{https://www.nytimes3xbfgragh.onion/subscription?campaignId=37WXW}{Subscriptions}
\end{itemize}
