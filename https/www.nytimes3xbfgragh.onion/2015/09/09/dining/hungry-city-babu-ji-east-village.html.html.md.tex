Sections

SEARCH

\protect\hyperlink{site-content}{Skip to
content}\protect\hyperlink{site-index}{Skip to site index}

\href{https://www.nytimes3xbfgragh.onion/section/food}{Food}

\href{https://myaccount.nytimes3xbfgragh.onion/auth/login?response_type=cookie\&client_id=vi}{}

\href{https://www.nytimes3xbfgragh.onion/section/todayspaper}{Today's
Paper}

\href{/section/food}{Food}\textbar{}Babu Ji in the East Village Asserts
Its Authority

\url{https://nyti.ms/1QbauFB}

\begin{itemize}
\item
\item
\item
\item
\item
\item
\end{itemize}

Advertisement

\protect\hyperlink{after-top}{Continue reading the main story}

Supported by

\protect\hyperlink{after-sponsor}{Continue reading the main story}

\href{/column/hungry-city}{Hungry City}

\hypertarget{babu-ji-in-the-east-village-asserts-its-authority}{%
\section{Babu Ji in the East Village Asserts Its
Authority}\label{babu-ji-in-the-east-village-asserts-its-authority}}

\href{https://www.nytimes3xbfgragh.onion/slideshow/2015/09/09/dining/babu-ji.html}{}

\hypertarget{babu-ji}{%
\subsection{Babu Ji}\label{babu-ji}}

9 Photos

View Slide Show ›

\includegraphics{https://static01.graylady3jvrrxbe.onion/images/2015/09/09/dining/09HUNGRY-BABU-JI-slide-V4T3/09HUNGRY-BABU-JI-slide-V4T3-articleLarge.jpg?quality=75\&auto=webp\&disable=upscale}

Emon Hassan for The New York Times

\begin{itemize}
\tightlist
\item
  Babu Ji\\
  Indian \$\$ 22 East 13th Street 212-951-1082
\end{itemize}

\href{https://resy.com/cities/ny/babu-ji?utm_source=nyt\&utm_medium=restoprofile\&utm_campaign=affiliates\&aff_id=c1fe784}{Reserve
a Table}

When you make a reservation at an independently reviewed restaurant
through our site, we earn an affiliate commission.

By Ligaya Mishan

\begin{itemize}
\item
  Sept. 3, 2015
\item
  \begin{itemize}
  \item
  \item
  \item
  \item
  \item
  \item
  \end{itemize}
\end{itemize}

Babu Ji stands a half-mile from the East Village's declining Curry Row,
where Indian restaurants once huddled shoulder to shoulder, epauletted
in twinkle lights and nearly indistinguishable.

The distance is calculated. At \href{http://www.babujinyc.com/}{Babu
Ji}, which opened in June, the menu is succinct rather than sprawling,
the dining room a sleek diagram of black floors, white walls and gray
tables, with pops of color from Bollywood films flickering by the bar. A
stuffed peacock guards a self-serve refrigerator stocked with craft
beer. Two Indian men --- one with a woolly mustache and sunglasses
rakishly atilt, another in a hot-pink uniform, cradling a tuba ---
preside in photographs above. (Babu ji is an honorific for a man of
authority, even if that authority extends no farther than the block he
lives on.)

The food is dressed up, too: yogurt croquettes pinned with an orchid
boutonniere against a shocking-pink sunset of beetroot ginger sauce; tea
sandwiches of paneer pressed around crunchy pear and ginger; crisp
bubbled wafers under an upheaval of chickpeas, pomegranate seeds, grapes
and tomatoes, slashed by cumin yogurt, date-tamarind sauce and mint
coriander chutney piqued by lime, a beautiful confusion of sweet and
sour.

``May there be ghee and sugar in your mouth,'' reads a proverb painted
in Hindi script on the wall. It's what you say in India to a bearer of
glad tidings, and suggests a happy promise to diners. But ghee is
notably absent.

``I don't use it,'' Jessi Singh, the chef, said firmly to my table one
evening. He had stopped by to introduce the tasting menu, another
novelty for East Village Indian restaurants, and spoke earnestly of a
subtler approach to Indian cuisine.

Depending on the season, his curries may arrive arrayed in wine-red
amaranth stems and branches of bright red currants like tiny tart
balloons, or tresses of shredded red cabbage, translucent rings of
daikon radish and pink-veined baby Swiss chard. There's no set protocol
for the garnishes; Mr. Singh improvises, and a dish ordered one evening
could undergo a makeover the next.

His attentiveness to fresh ingredients may be traced back to his
childhood in a village with erratic electricity on the outskirts of
Chandigarh in northern India, and later in a series of Australian
country towns, where he was home-schooled and worked the family farm. In
his 20s, he set off to see the world, and in San Francisco met his
future wife (and business partner), Jennifer, a nurse born in Canarsie,
Brooklyn.

They ran a few Indian restaurants in Australia before moving to New York
this year. (A previous iteration of Babu Ji, in Melbourne, is now under
new ownership.) They're still easing in: On one visit, the chutney was
too sweet, a vexing problem since it was reprised in so many dishes. A
Keralan moilee of coconut milk, turmeric, mustard seed and curry leaves
was rich but listless despite the addition of lovely petals of raw sea
scallop. At times I wished the flavors would stop beating around the
bush.

Mr. Singh does some of his finest cooking with classics, like leg of
lamb braised for six hours in a sealed pot, dum pukht style; goat on the
bone in a curry whose layers keep descending; and a black-lentil dal of
smoky, wondrous verging-on-bitterness.

Babu Ji might be more revelatory if there were not already a number of
restaurants that have raised the level of Indian cuisine in New York
City:
\href{http://www.nytimes3xbfgragh.onion/2011/03/30/dining/reviews/30rest.html}{Junoon,
Tulsi} and
\href{http://www.nytimes3xbfgragh.onion/2001/02/16/arts/diner-s-journal.html}{Tamarind}.
Even Curry Row has
\href{http://www.nytimes3xbfgragh.onion/2013/04/24/dining/reviews/hungry-city-malai-marke-in-the-east-village.html?_r=0}{its
worthies}.

Still, it's hard to resist the elegantly feral plates here. Or the
theater of dessert: Kulfi, a cousin of ice cream, suffused with cardamom
and pistachios, is presented still entombed in skinny, tapering tin
molds that you must roll between your palms to warm up.

Occasionally the service reminded me that we weren't so far from Curry
Row. One night, a waiter announced that he wanted to make a mustache out
of a diner's hair. Later, a waitress panicked when she saw that my table
hadn't finished the gol gappa, fragile semolina shells filled with that
steadfast threesome: date-tamarind sauce, chutney and yogurt. ``They'll
be ruined,'' she cried.

We swallowed them whole.

Advertisement

\protect\hyperlink{after-bottom}{Continue reading the main story}

\hypertarget{site-index}{%
\subsection{Site Index}\label{site-index}}

\hypertarget{site-information-navigation}{%
\subsection{Site Information
Navigation}\label{site-information-navigation}}

\begin{itemize}
\tightlist
\item
  \href{https://help.nytimes3xbfgragh.onion/hc/en-us/articles/115014792127-Copyright-notice}{©~2020~The
  New York Times Company}
\end{itemize}

\begin{itemize}
\tightlist
\item
  \href{https://www.nytco.com/}{NYTCo}
\item
  \href{https://help.nytimes3xbfgragh.onion/hc/en-us/articles/115015385887-Contact-Us}{Contact
  Us}
\item
  \href{https://www.nytco.com/careers/}{Work with us}
\item
  \href{https://nytmediakit.com/}{Advertise}
\item
  \href{http://www.tbrandstudio.com/}{T Brand Studio}
\item
  \href{https://www.nytimes3xbfgragh.onion/privacy/cookie-policy\#how-do-i-manage-trackers}{Your
  Ad Choices}
\item
  \href{https://www.nytimes3xbfgragh.onion/privacy}{Privacy}
\item
  \href{https://help.nytimes3xbfgragh.onion/hc/en-us/articles/115014893428-Terms-of-service}{Terms
  of Service}
\item
  \href{https://help.nytimes3xbfgragh.onion/hc/en-us/articles/115014893968-Terms-of-sale}{Terms
  of Sale}
\item
  \href{https://spiderbites.nytimes3xbfgragh.onion}{Site Map}
\item
  \href{https://help.nytimes3xbfgragh.onion/hc/en-us}{Help}
\item
  \href{https://www.nytimes3xbfgragh.onion/subscription?campaignId=37WXW}{Subscriptions}
\end{itemize}
