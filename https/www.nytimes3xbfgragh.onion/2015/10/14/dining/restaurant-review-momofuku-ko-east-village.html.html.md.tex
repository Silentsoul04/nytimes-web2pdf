Sections

SEARCH

\protect\hyperlink{site-content}{Skip to
content}\protect\hyperlink{site-index}{Skip to site index}

\href{https://www.nytimes3xbfgragh.onion/section/food}{Food}

\href{https://myaccount.nytimes3xbfgragh.onion/auth/login?response_type=cookie\&client_id=vi}{}

\href{https://www.nytimes3xbfgragh.onion/section/todayspaper}{Today's
Paper}

\href{/section/food}{Food}\textbar{}Restaurant Review: Momofuku Ko in
the East Village

\url{https://nyti.ms/1jwfj16}

\begin{itemize}
\item
\item
\item
\item
\item
\item
\end{itemize}

Advertisement

\protect\hyperlink{after-top}{Continue reading the main story}

Supported by

\protect\hyperlink{after-sponsor}{Continue reading the main story}

\hypertarget{restaurant-review-momofuku-ko-in-the-east-village}{%
\section{Restaurant Review: Momofuku Ko in the East
Village}\label{restaurant-review-momofuku-ko-in-the-east-village}}

\href{https://www.nytimes3xbfgragh.onion/slideshow/2015/10/14/dining/momofuku-ko.html}{}

\hypertarget{momofuku-ko}{%
\subsection{Momofuku Ko}\label{momofuku-ko}}

13 Photos

View Slide Show ›

\includegraphics{https://static01.graylady3jvrrxbe.onion/images/2015/10/14/dining/14REST-MOMOFUKU-slide-40ZH/14REST-MOMOFUKU-slide-40ZH-articleLarge.jpg?quality=75\&auto=webp\&disable=upscale}

Danny Ghitis for The New York Times

\begin{itemize}
\tightlist
\item
  Momofuku Ko\\
  ★★★ Asian;French \$\$\$\$ 8 Extra Place 212-203-8095
\end{itemize}

\href{https://reservations.momofuku.com/restaurants/4/reserve}{Reserve a
Table}

When you make a reservation at an independently reviewed restaurant
through our site, we earn an affiliate commission.

By \href{http://www.nytimes3xbfgragh.onion/by/pete-wells}{Pete Wells}

\begin{itemize}
\item
  Oct. 13, 2015
\item
  \begin{itemize}
  \item
  \item
  \item
  \item
  \item
  \item
  \end{itemize}
\end{itemize}

If you lived in New York in 2008, taking in an adequate supply of oxygen
and any quantity of food media at all, you knew two things about a new
restaurant called \href{http://momofuku.com/new-york/ko/}{Momofuku Ko}.

One was that its owner, chef and chief iconoclast, David Chang, had
imposed a reservations website that frustrated everybody except coders
who wrote programs to snatch up some of the 12 seats the instant they
went online. (Among the frustrated was my predecessor Frank Bruni, who,
in order to write his three-star review,
\href{http://www.nytimes3xbfgragh.onion/2008/05/07/dining/reviews/07rest.html?pagewanted=all\&_r=0}{``entered
into groveling, Ko-dependent arrangements with tireless friends and
readers.''})

You also knew that Ko served foie gras in a way the city had never seen.
A cook behind the counter would rub a frozen cured brick of it across a
Microplane held above a bowl with pine nut brittle, riesling jelly and
lobes of lychee, showering them with falling pink flakes of airborne
pleasure.

Mr. Chang moved Ko to an alleyway off East First Street late last year,
and it's a testament to how much more polished and accomplished the new
place is that when the foie gras appears toward the end of the night, it
doesn't dazzle the way it used to. In the company of another 14 or so
graceful, beautiful dishes, it is slightly flat-footed, a dancer loping
across the stage in Chuck Taylors while the rest of the corps balances
on pointe shoes.

This is no put-down. There's no shame in being outclassed by a firm,
spiced tartare of Japanese sea bream hanging in a sparkling jelly brewed
from the fish's bones in which capsules of finger lime wait, invisibly,
tiny balloons of sourness ready to pop.

Or in being cast in the shade by a one-bite mille-feuille whose crisp
layers of rye pastry, sprinkled with green tea powder, hold lush
yuzukosho béchamel and briny little globes of trout roe.

Or in being shown up by barely seared, gently smoked lobster tail with
the juicy tenderness of a ripe, warm nectarine. Extraordinary on its
own, the tail sits on a whorl of spaghetti squash that becomes
uncharacteristically exciting as it drinks up a foamy lobster sauce. At
last, we know what spaghetti squash is good for.

Leading the gang of cooks responsible for these plates and others is
Sean Gray, Ko's executive chef since 2014. (Mr. Chang, often seen wedged
behind the counter in 2008, now presumably lurks in an underground lair
somewhere, staring at a wall map with pins marking his dominion in New
York, Toronto and Sydney.)

Momofuku Ko is remembered as a local pioneer of the tasting-counter
format that has spread across the country, but the original ambition was
bolder. Mr. Chang was leading a daring experiment that asked: If you
aspired to serve food as original and refined as anything in an
expensive uptown restaurant but wanted to keep prices down, exactly how
many amenities could you strip out? The answer turned out to be, ``Not
quite \emph{that} many, Dave,'' but the question was the right one at
the right time.

With the move, that experiment officially ended. The new Ko has
amenities galore and a price to match.

The high degree of finesse in the \$175 tasting menus must be due in
part to the greater acreage and better equipment in the kitchen, inside
a U-shaped counter that seats up to 18. As they did before the move, Mr.
Gray and his colleagues take turns delivering dishes. The original crew
on First Avenue rarely made eye contact and often appeared mildly
resentful, an understandable reaction in those sweatshop conditions.
Now, Mr. Gray and his collaborators look relaxed. Sometimes they even
smile.

They're not the only ones whose accommodations have been upgraded. We
diners can settle our tense, crooked frames into the pliant leather
seats and backs of the tall stools arranged around the counter, enjoying
space around and behind us. This may not sound like cause to celebrate
unless you remember the old Ko, where every time somebody squeezed
behind you on the way to the restroom, you had to lean forward and hold
your breath.

Should you arrive early to this palace of luxury, you can pause at the
small and very appealing bar, where the cocktails are highly intelligent
without rubbing your nose in it. After the desserts (if you are in luck,
one will be a bittersweet cookie-like tart shell filled with Fernet
Branca pastry cream and dark chocolate mousse), you can ask for a cup of
hearty and well-rounded barley tea, a remarkable concession from Mr.
Chang, who in the past seemed to regard tea as a sign of weakness.

During the meal, there is wine, in uncannily delicate glasses, from some
of the world's most revered grape stompers. Jordan Salcito, beverage
director for the Momofuku group, gives these winemakers' names and faces
prominent play on her 78-page list. She also tries to show stylistic
resemblances by noting when one producer has influenced another. This is
a more useful, up-to-date way to choose a bottle than the ancient system
of memorizing vintages and appellations. It comes at the cost, though,
of emphasizing cult stars over new finds whose bottles are often
bargains; there could be more choices for under \$75.

Versatile wines do best with the quick shifts of flavor in the
seafood-heavy menu. You don't want anything to fight with the intense
flavor of cured, citrus-glazed mackerel sushi under scallions and grated
fresh ginger strands. The skin is quickly blistered with the same
blowtorch used to gently toast the bottom of the rice, bringing a
wonderfully un-sushi-like crunch to one of the restaurant's most
delicious creations.

Dud dishes are rare. There must be better places for exceptionally soft
Osetra caviar than in a tomato-and-basil salad, for instance. (There
are, and on another night the roe was spooned over an amazing,
near-liquid potato purée.)

Soy milk chawanmushi and a dark blotch of intentionally burned
applesauce would have been puzzling with sea urchin even if the urchin
hadn't tasted metallic and iodine-heavy. This was more surprising
because I'd been served exceptional sea urchin on other nights, laid
alongside a swoosh of chickpea purée that incorporates fermented
chickpea paste. If I told you fermented chickpea paste tastes a little
like cheese, you might not want to try it, but you should. Eaten
together with the urchin, two soft and orange blobs finding each other
inside your mouth, it's like a door cracking open to a new world of
flavor.

A few failures don't necessarily derail a meal that aims this high. But
you might, if you were feeling philosophical, point out that the menu's
sensibility could be more unified. Mr. Chang once called tasting menus
``chefs' novels, their big ideas, their statements of purpose and
intent.'' The menus at the new Ko are more like collections of short
stories, not all by the same author.

Quills of trofie dressed with Tasso ham, poblano peppers and Mimolette
cheese make an odd traveling companion for the mackerel sushi. Elysian
Farms lamb with green tomatoes and Calabrian chiles might turn up in any
number of restaurants, and so could the excellent sourdough bread.
Nobody would say that about the daring chickpea and sea urchin duet.

The unifying theme of the original Ko was: ``Forget your ideas of fancy
dining. Here's something new.'' It was built by and for young, hungry,
probably insomniac rebels who chased their craziest ideas. The new Ko
was built by a mature chef for customers who have grown up, and it
reflects all the comforts and resources and security and extra padding
we accumulate in the middle of our lives. Those things can hem us in,
putting the 4 a.m. howlings of youth out of reach, or at least making
them seem counterproductive. If the new Ko has a message, it is: ``Yep.
This is what fancy dining is like now.'' But those youthful howlings are
still in its bones.

Advertisement

\protect\hyperlink{after-bottom}{Continue reading the main story}

\hypertarget{site-index}{%
\subsection{Site Index}\label{site-index}}

\hypertarget{site-information-navigation}{%
\subsection{Site Information
Navigation}\label{site-information-navigation}}

\begin{itemize}
\tightlist
\item
  \href{https://help.nytimes3xbfgragh.onion/hc/en-us/articles/115014792127-Copyright-notice}{©~2020~The
  New York Times Company}
\end{itemize}

\begin{itemize}
\tightlist
\item
  \href{https://www.nytco.com/}{NYTCo}
\item
  \href{https://help.nytimes3xbfgragh.onion/hc/en-us/articles/115015385887-Contact-Us}{Contact
  Us}
\item
  \href{https://www.nytco.com/careers/}{Work with us}
\item
  \href{https://nytmediakit.com/}{Advertise}
\item
  \href{http://www.tbrandstudio.com/}{T Brand Studio}
\item
  \href{https://www.nytimes3xbfgragh.onion/privacy/cookie-policy\#how-do-i-manage-trackers}{Your
  Ad Choices}
\item
  \href{https://www.nytimes3xbfgragh.onion/privacy}{Privacy}
\item
  \href{https://help.nytimes3xbfgragh.onion/hc/en-us/articles/115014893428-Terms-of-service}{Terms
  of Service}
\item
  \href{https://help.nytimes3xbfgragh.onion/hc/en-us/articles/115014893968-Terms-of-sale}{Terms
  of Sale}
\item
  \href{https://spiderbites.nytimes3xbfgragh.onion}{Site Map}
\item
  \href{https://help.nytimes3xbfgragh.onion/hc/en-us}{Help}
\item
  \href{https://www.nytimes3xbfgragh.onion/subscription?campaignId=37WXW}{Subscriptions}
\end{itemize}
