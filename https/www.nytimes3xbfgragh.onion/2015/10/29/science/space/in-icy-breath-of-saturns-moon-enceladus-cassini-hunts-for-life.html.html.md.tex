Sections

SEARCH

\protect\hyperlink{site-content}{Skip to
content}\protect\hyperlink{site-index}{Skip to site index}

\href{https://www.nytimes3xbfgragh.onion/section/science/space}{Space \&
Cosmos}

\href{https://myaccount.nytimes3xbfgragh.onion/auth/login?response_type=cookie\&client_id=vi}{}

\href{https://www.nytimes3xbfgragh.onion/section/todayspaper}{Today's
Paper}

\href{/section/science/space}{Space \& Cosmos}\textbar{}Cassini Seeks
Insights to Life in Plumes of Enceladus, Saturn's Icy Moon

\url{https://nyti.ms/1WhjT57}

\begin{itemize}
\item
\item
\item
\item
\item
\item
\end{itemize}

Advertisement

\protect\hyperlink{after-top}{Continue reading the main story}

Supported by

\protect\hyperlink{after-sponsor}{Continue reading the main story}

\href{/column/out-there}{Out There}

\hypertarget{cassini-seeks-insights-to-life-in-plumes-of-enceladus-saturns-icy-moon}{%
\section{Cassini Seeks Insights to Life in Plumes of Enceladus, Saturn's
Icy
Moon}\label{cassini-seeks-insights-to-life-in-plumes-of-enceladus-saturns-icy-moon}}

\includegraphics{https://static01.graylady3jvrrxbe.onion/images/2015/10/27/multimedia/outthere-cassini/outthere-cassini-videoSixteenByNine1050.jpg}

By \href{http://www.nytimes3xbfgragh.onion/by/dennis-overbye}{Dennis
Overbye}

\begin{itemize}
\item
  Oct. 28, 2015
\item
  \begin{itemize}
  \item
  \item
  \item
  \item
  \item
  \item
  \end{itemize}
\end{itemize}

Where there is water, is there life?

That's the \$64 billion question now facing NASA and the rest of lonely
humanity. When the New Horizons spacecraft,
\href{http://www.nytimes3xbfgragh.onion/interactive/2015/07/15/science/space/new-horizons-pluto-flyby-photos.html}{cameras
clicking},
\href{http://www.nytimes3xbfgragh.onion/interactive/2015/07/14/science/space/pluto-flyby.html}{sped
past}
\href{http://www.nytimes3xbfgragh.onion/2015/07/15/science/space/nasa-new-horizons-spacecraft-reaches-pluto.html}{Pluto}
in July, it represented an inflection point in the
\href{http://www.nytimes3xbfgragh.onion/video/science/100000003783764/fast-and-light-to-pluto.html}{conquest
of the solar system}. Half a century after the first planetary probe
sailed past Venus, all the planets and would-be planets we have known
and loved, and all the marvelous rocks and snowballs circling them, have
been detected and inspected, reconnoitered.

That part of human history, the astrophysical exploration of the solar
system, is over. The next part, the biological exploration of space, is
just beginning. We have finished counting the rocks in the neighborhood.
It is time to find out if anything is living on them, a job that could
easily take another half century.

NASA's mantra for finding alien life has long been to ``follow the
water,'' the one ingredient essential to our own biochemistry. On
Wednesday, NASA sampled the most available water out there, as the
\href{https://www.nytimes3xbfgragh.onion/2017/09/14/science/cassini-grand-finale-saturn.html}{Cassini}
spacecraft plunged through an icy spray erupting from the little
Saturnian moon Enceladus.

Enceladus is only 300 miles across and whiter than a Bing Crosby
Christmas, reflecting virtually all the sunlight that hits it, which
should make it colder and deader than Scrooge's heart.

But in 2005, shortly after
\href{http://www.nytimes3xbfgragh.onion/2004/07/01/science/01CND-SATU.html}{starting
an 11-year sojourn}at Saturn, Cassini recorded jets of water squirting
from cracks known as
\href{http://www.nasa.gov/mission_pages/cassini/media/cassini-083005.html}{tiger
stripes} near the south pole of Enceladus --- evidence, scientists say,
of an underground ocean kept warm and liquid by tidal flexing of the
little moon as it is stretched and squeezed by Saturn.

And with that, Enceladus leapfrogged to the top of astrobiologists' list
of promising places to look for life. If there is life in its ocean,
alien microbes could be riding those geysers out into space where a
passing spacecraft could grab them. No need to drill through miles of
ice or dig up rocks.

\href{https://www.nytimes3xbfgragh.onion/slideshow/2015/10/30/science/space/nasas-cassini-zooms-in-on-enceladus.html}{}

\hypertarget{nasas-cassini-zooms-in-on-enceladus}{%
\subsection{NASA's Cassini Zooms In on
Enceladus}\label{nasas-cassini-zooms-in-on-enceladus}}

4 Photos

View Slide Show ›

\includegraphics{https://static01.graylady3jvrrxbe.onion/images/2015/10/31/science/30enceladus4/30enceladus4-articleLarge.jpg?quality=75\&auto=webp\&disable=upscale}

NASA/JPL-Caltech/Space Science Institute

As Chris McKay, an astrobiologist at NASA's Ames Research Center, said,
it's as if nature had hung up a sign at Enceladus saying ``Free
Samples.''

Discovering life was not on the agenda when Cassini was designed and
launched two decades ago. Its instruments can't capture microbes or
detect life, but in a couple of dozen passes through the plumes of
Enceladus, it has detected various molecules associated with life: water
vapor, carbon dioxide, methane, molecular nitrogen,
\href{https://en.wikipedia.org/wiki/Propane}{propane},
\href{https://en.wikipedia.org/wiki/Acetylene}{acetylene},
\href{https://en.wikipedia.org/wiki/Formaldehyde}{formaldehyde} and
traces of ammonia.

Wednesday's dive was the deepest Cassini will make through the plumes,
only 30 miles above the icy surface. Scientists are especially
interested in measuring the amount of hydrogen gas in the plume, which
would tell them how much energy and heat are being generated by chemical
reactions in hydrothermal vents at the bottom of the moon's ocean.

It is in such ocean vents that some of the most primordial-looking
life-forms have been found on our own planet. What the Cassini
scientists find out could help set the stage for a return mission with a
spacecraft designed to detect or even bring back samples of life.

These are optimistic, almost sci-fi times. The fact that life was
present on Earth as early as 4.1 billion years ago --- pretty much as
soon as asteroids and leftover planet junk stopped bombarding the new
Earth and let it cool down --- has led astrobiologists to conclude that,
given the right conditions, life will take hold quickly. Not just in our
solar system, but in some of the thousands of planetary systems that
Kepler and other missions squinting at distant stars have uncovered.

And if water is indeed the key, the solar system has had several chances
to get lucky. Besides Enceladus, there is an ocean underneath the ice of
\href{http://www.nytimes3xbfgragh.onion/interactive/2015/08/25/science/space/nasa-next-mission.html}{Jupiter's
moon Europa}, and the Hubble Space Telescope has hinted that it too is
venting into space. NASA has begun planning for a mission next decade to
fly by it.

\includegraphics{https://static01.graylady3jvrrxbe.onion/images/2015/10/29/science/space/29OUTTHERE/29OUTTHERE-articleLarge-v2.jpg?quality=75\&auto=webp\&disable=upscale}

And of course there's Mars, with its dead oceans and intriguing streaks
of damp sand, springboard of a thousand sci-fi invasions of Earth, but
in recent decades the target of robot invasions going the other
direction.

Some scientists even make the case that genesis happened not on Earth
but on Mars. Our biochemical ancestors would then have made the passage
on an asteroid, making us all Martians and perhaps explaining our
curious attraction to the Red Planet.

And then there is Titan, Saturn's largest moon, the only moon in the
solar system with a thick atmosphere and lakes on its surface, except
that in this case the liquid in them is methane and the beaches and
valleys are made of hydrocarbon slush.

NASA's working definition of life, coined by a group of biologists in
1992, is ``a self-sustaining chemical system capable of Darwinian
evolution.''

Any liquid could serve as the medium of this thing, process, whatever it
is. Life on Titan would expand our notions of what is biochemically
possible out there in the rest of the universe.

Our history of exploration suggests that surprise is the nature of the
game. That was the lesson of the Voyager missions: Every world or moon
encountered on that twin-spacecraft odyssey was different, an example of
the laws of physics sculpted by time and circumstance into unique and
weird forms.

\href{https://www.nytimes3xbfgragh.onion/interactive/2015/08/25/science/space/nasa-next-mission.html}{}

\includegraphics{https://static01.graylady3jvrrxbe.onion/images/2015/08/25/science/space/NNpromo/NNpromo-videoLarge-v3.jpg}

\hypertarget{nasas-next-horizon-in-space}{%
\subsection{NASA's Next Horizon in
Space}\label{nasas-next-horizon-in-space}}

Since New Horizons beamed back photos of Pluto, the question has loomed:
What's next? More than 1,600 Times readers shared their ideas.

And so far that is the lesson of the new astronomy of exoplanets ---
thousands of planetary systems, but not a single one that looks like our
own.

The detection of a single piece of pond slime, one alien microbe, on
some other world would rank as one of the greatest discoveries in the
history of science. Why should we expect it to look anything like what
we already know?

That microbe won't come any cheaper than the Higgs boson, the keystone
of modern particle physics, which cost
\href{http://www.nytimes3xbfgragh.onion/2013/03/05/science/chasing-the-higgs-boson-how-2-teams-of-rivals-at-CERN-searched-for-physics-most-elusive-particle.html}{more
than \$10 billion to hunt down over half a century}.

Finding that microbe will involve launching big, complicated chunks of
hardware to various corners of the solar system, and that means work for
engineers, scientists, accountants, welders, machinists, electricians,
programmers and practitioners of other crafts yet to be invented ---
astro-robot-paleontologists, say.

However many billions of dollars it takes to knock on doors and find out
if anybody is at home, it will all be spent here on Earth, on people and
things we all say we want: innovation, education, science, technology.

We've seen this have a happy ending before. It was the kids of the
aerospace industry and the military-industrial complex, especially in
California, who gave us Silicon Valley and general relativity in our
pockets.

In this era, a happy ending could include the news that we are not
alone, that the cosmos is more diverse, again, than we had imagined.

Or not.

In another 50 years the silence from out there could be deafening.

Advertisement

\protect\hyperlink{after-bottom}{Continue reading the main story}

\hypertarget{site-index}{%
\subsection{Site Index}\label{site-index}}

\hypertarget{site-information-navigation}{%
\subsection{Site Information
Navigation}\label{site-information-navigation}}

\begin{itemize}
\tightlist
\item
  \href{https://help.nytimes3xbfgragh.onion/hc/en-us/articles/115014792127-Copyright-notice}{©~2020~The
  New York Times Company}
\end{itemize}

\begin{itemize}
\tightlist
\item
  \href{https://www.nytco.com/}{NYTCo}
\item
  \href{https://help.nytimes3xbfgragh.onion/hc/en-us/articles/115015385887-Contact-Us}{Contact
  Us}
\item
  \href{https://www.nytco.com/careers/}{Work with us}
\item
  \href{https://nytmediakit.com/}{Advertise}
\item
  \href{http://www.tbrandstudio.com/}{T Brand Studio}
\item
  \href{https://www.nytimes3xbfgragh.onion/privacy/cookie-policy\#how-do-i-manage-trackers}{Your
  Ad Choices}
\item
  \href{https://www.nytimes3xbfgragh.onion/privacy}{Privacy}
\item
  \href{https://help.nytimes3xbfgragh.onion/hc/en-us/articles/115014893428-Terms-of-service}{Terms
  of Service}
\item
  \href{https://help.nytimes3xbfgragh.onion/hc/en-us/articles/115014893968-Terms-of-sale}{Terms
  of Sale}
\item
  \href{https://spiderbites.nytimes3xbfgragh.onion}{Site Map}
\item
  \href{https://help.nytimes3xbfgragh.onion/hc/en-us}{Help}
\item
  \href{https://www.nytimes3xbfgragh.onion/subscription?campaignId=37WXW}{Subscriptions}
\end{itemize}
