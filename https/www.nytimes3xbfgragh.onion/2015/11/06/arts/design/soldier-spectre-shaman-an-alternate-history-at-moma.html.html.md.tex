Sections

SEARCH

\protect\hyperlink{site-content}{Skip to
content}\protect\hyperlink{site-index}{Skip to site index}

\href{https://www.nytimes3xbfgragh.onion/section/arts/design}{Art \&
Design}

\href{https://myaccount.nytimes3xbfgragh.onion/auth/login?response_type=cookie\&client_id=vi}{}

\href{https://www.nytimes3xbfgragh.onion/section/todayspaper}{Today's
Paper}

\href{/section/arts/design}{Art \& Design}\textbar{}`Soldier, Spectre,
Shaman,' an Alternate History at MoMA

\url{https://nyti.ms/1kxpgMv}

\begin{itemize}
\item
\item
\item
\item
\item
\end{itemize}

Advertisement

\protect\hyperlink{after-top}{Continue reading the main story}

Supported by

\protect\hyperlink{after-sponsor}{Continue reading the main story}

Art Review

\hypertarget{soldier-spectre-shaman-an-alternate-history-at-moma}{%
\section{`Soldier, Spectre, Shaman,' an Alternate History at
MoMA}\label{soldier-spectre-shaman-an-alternate-history-at-moma}}

\includegraphics{https://static01.graylady3jvrrxbe.onion/images/2015/11/06/arts/06FIG/06FIG-articleLarge.jpg?quality=75\&auto=webp\&disable=upscale}

By \href{https://www.nytimes3xbfgragh.onion/by/jason-farago}{Jason
Farago}

\begin{itemize}
\item
  Nov. 5, 2015
\item
  \begin{itemize}
  \item
  \item
  \item
  \item
  \item
  \end{itemize}
\end{itemize}

In 1936, the first director of the Museum of Modern Art mapped out the
art of his time
\href{http://www.moma.org/learn/resources/archives/archives_highlights_02_1936}{in
a famous chart}, the first taste of the museum's persistent narrative of
Modernism. The doctrine goes something like this: Western art of the
early 20th century develops in a single direction, toward abstraction.
Artists working toward that goal were the true radicals. Those who held
onto representation were aesthetic conservatives (whatever their actual
politics).Conveniently enough, the artists who broke most definitively
with figuration, and who made abstract painting and sculpture into the
One True Art, were living in New York --- the city that, in the words of
the art historian Serge Guilbaut,
``\href{http://www.nytimes3xbfgragh.onion/1984/01/01/books/behind-the-scenes-of-abstract-expressionism.html?pagewanted=all}{stole
the idea of Modern art}'' from Paris after World War II.

That is a fantasy, but the fantasy endures in MoMA's presentation of its
permanent collection, which still stages the art of Pollock, Newman,
Rothko and the rest of the New York School as if they were the acme of
Modernism. But
\href{https://www.moma.org/calendar/exhibitions/1551?locale=en}{``Soldier,
Spectre, Shaman,''} a noteworthy and all too rare exhibition on view on
the museum's third floor, offers a vital corrective to the gospel of
abstract art. Most of the 30 or so artists here are European, and stand
outside the museum's tenacious master narrative. While their American
counterparts were paring down painting, sculpture and other media to
their essences, these artists insisted on the primacy of the figure, and
conceived a new, more downhearted humanism for an inhuman age.

\includegraphics{https://static01.graylady3jvrrxbe.onion/images/2015/11/06/arts/06FIGURATIVEJP1/06FIGURATIVEJP1-articleLarge.jpg?quality=75\&auto=webp\&disable=upscale}

With the end of the war, and the full revelation of the Holocaust, the
human body became a figure of pathos and existential dread, and only
rarely one of possible rebirth and redemption. Francis Bacon, in a lurid
1946 painting featuring a black-suited gent standing before a crucified
cow carcass, updated the Christian trope of the Man of Sorrows (with an
allusion to Rembrandt's 1655 ``Slaughtered Ox'') for an era after the
death of God. So did Alberto Giacometti, represented here by his
``Standing Woman,'' of 1948, an emaciated, etiolated bronze of the type
that Giacometti's friend, Jean-Paul Sartre, compared to ``the fleshless
martyrs of Buchenwald.'' His fellow sculptor, Germaine Richier, a
Frenchwoman who fled to Switzerland as the war began, took a similar
approach to bronze sculpture, casting pockmarked and injured figures
whose bumpy surfaces recall the corpses of Pompeii.

Scarred, pained or totally blasted, the body in European art of the
1940s and `50s testified to both physical wounds and to deeper, internal
traumas. That was especially the case in postwar France, where Jean
Fautrier, an artist and member of the Resistance, far too little known
in the United States, created his ``Otages'' (``Hostages''). Their
flattened and anguished faces were informed by the sounds of the torture
of civilians he heard at night outside a suburban Paris asylum. This
show contains a pair of prints that, unfortunately, only suggest the
sickly force of his plaster-thickened paintings. More compelling French
works here include a spare watercolor of two ectoplasmic figures by
\href{http://www.theguardian.com/books/2002/aug/10/featuresreviews.guardianreview18}{Henri
Michaux}, who is better known as a poet; a bronze head by the Greek-born
\href{http://www.wsj.com/articles/sculptor-takis-bends-natures-laws-for-art-1424374561}{Takis},
its eyes formed from whorls on its scored surface; and a small occult
work by the Romanian émigré Victor Brauner, in which a skeleton is
formed from tallow dripped on wood.

Image

``Untitled,'' a 1946 work by Henri Michaux.Credit...Museum of Modern
Art, 2015 Henri Michaux/Artists Rights Society (ARS), New York / ADAGP,
Paris

New York, too, had its fair share of artists who responded to wartime
suffering in figurative means, often with a Surrealist bent. A 1941
print by David Smith depicts a parched battlefield overrun by hybrid
Amazon-centaurs equipped with wheels instead of hind legs. A frail and
gaunt totem by Louise Bourgeois, who moved from France to New York in
1938, stands between two thin rods that could be arms, or crutches.

Perhaps the most startling work here comes from Jan Müller, a German
refugee in New York who studied under Hans Hofmann. Where most of
Hofmann's students turned to gestural abstraction, Müller espoused an
eerie, anxious figuration with ghoulish details. In his fantastic 1957
tableau of the Walpurgisnacht from Goethe's ``Faust,'' Mephistopheles
has the simplified head of a Halloween jack-o'-lantern, while a witch in
one corner has torn her face right off. Müller died the next year, at
the age of 35. One can only speculate where his painting may have led.

Image

``The Devil with Claws,'' a 1952 work by Germaine Richier.Credit...2015
Germaine Richier/Artists Rights Society (ARS), New York / ADAGP, Paris;
John Wronn/The Museum of Modern Art

``Soldier, Spectre, Shaman'' also glances at Japanese art from the
1950s, and focuses on how bodies were depicted in the years after the
bomb. The most compelling works come from Chimei Hamada (now 97), whose
postwar etchings feature amphibian soldiers left for dead on a barren
battlefield, or hanging from wooden pylons. They are shocking, surreal
nightmares drawn from his experiences in the Sino-Japanese War, as blunt
and as memorable as anything by Goya.

``Soldier, Spectre, Shaman'' has been organized by Lucy Gallun, an
assistant curator in the museum's photography department, and Sarah
Suzuki, an associate curator of drawings and prints. It is small, drawn
entirely from the permanent collection, and it may be a greater
achievement as a riposte to MoMA's own history than as a stand-alone
exhibition. MoMA has no collection of Socialist Realism, for one, and so
the show necessarily ignores Soviet examples of postwar figuration, as
well as significant leftist French artists such as the painter Boris
Taslitzky, who survived Buchenwald. More surprising is the omission of
Jean Dubuffet, whose fraught and scumbled figures testify to wartime
privation and inner anguish. The curators have, on the other hand,
thrown in two panoramas of warrior nymphets by the outsider artist Henry
Darger --- lovely works, but extraneous.

Still, we see these works too infrequently in New York, and this show
offers a corrective to an art-historical fiction that MoMA should keep
trying to transcend. It is a reboot, of sorts, of the museum's 1959
exhibition ``New Images of Man,'' which was pilloried at the time for
its defense of the figure (and, worse, of Europe), but which might offer
some guidance to MoMA's curatorial team. Ahead of its
\href{http://www.nytimes3xbfgragh.onion/2014/01/09/arts/design/a-grand-redesign-of-moma-does-not-spare-a-notable-neighbor.html}{controversial
expansion}, the museum is now planning a substantial rehanging of its
permanent collection that, so they say, will abolish the current
medium-specific galleries. It would do well to think not only past
medium but past dogma, too, and to use its unrivaled collection to
reveal the postwar era in all its plenitude breadth and tragedy.

Advertisement

\protect\hyperlink{after-bottom}{Continue reading the main story}

\hypertarget{site-index}{%
\subsection{Site Index}\label{site-index}}

\hypertarget{site-information-navigation}{%
\subsection{Site Information
Navigation}\label{site-information-navigation}}

\begin{itemize}
\tightlist
\item
  \href{https://help.nytimes3xbfgragh.onion/hc/en-us/articles/115014792127-Copyright-notice}{©~2020~The
  New York Times Company}
\end{itemize}

\begin{itemize}
\tightlist
\item
  \href{https://www.nytco.com/}{NYTCo}
\item
  \href{https://help.nytimes3xbfgragh.onion/hc/en-us/articles/115015385887-Contact-Us}{Contact
  Us}
\item
  \href{https://www.nytco.com/careers/}{Work with us}
\item
  \href{https://nytmediakit.com/}{Advertise}
\item
  \href{http://www.tbrandstudio.com/}{T Brand Studio}
\item
  \href{https://www.nytimes3xbfgragh.onion/privacy/cookie-policy\#how-do-i-manage-trackers}{Your
  Ad Choices}
\item
  \href{https://www.nytimes3xbfgragh.onion/privacy}{Privacy}
\item
  \href{https://help.nytimes3xbfgragh.onion/hc/en-us/articles/115014893428-Terms-of-service}{Terms
  of Service}
\item
  \href{https://help.nytimes3xbfgragh.onion/hc/en-us/articles/115014893968-Terms-of-sale}{Terms
  of Sale}
\item
  \href{https://spiderbites.nytimes3xbfgragh.onion}{Site Map}
\item
  \href{https://help.nytimes3xbfgragh.onion/hc/en-us}{Help}
\item
  \href{https://www.nytimes3xbfgragh.onion/subscription?campaignId=37WXW}{Subscriptions}
\end{itemize}
