Sections

SEARCH

\protect\hyperlink{site-content}{Skip to
content}\protect\hyperlink{site-index}{Skip to site index}

\href{https://www.nytimes3xbfgragh.onion/section/world/asia}{Asia
Pacific}

\href{https://myaccount.nytimes3xbfgragh.onion/auth/login?response_type=cookie\&client_id=vi}{}

\href{https://www.nytimes3xbfgragh.onion/section/todayspaper}{Today's
Paper}

\href{/section/world/asia}{Asia Pacific}\textbar{}Japan and South Korea
Settle Dispute Over Wartime `Comfort Women'

\url{https://nyti.ms/1kn6lTY}

\begin{itemize}
\item
\item
\item
\item
\item
\item
\end{itemize}

Advertisement

\protect\hyperlink{after-top}{Continue reading the main story}

Supported by

\protect\hyperlink{after-sponsor}{Continue reading the main story}

\hypertarget{japan-and-south-korea-settle-dispute-over-wartime-comfort-women}{%
\section{Japan and South Korea Settle Dispute Over Wartime `Comfort
Women'}\label{japan-and-south-korea-settle-dispute-over-wartime-comfort-women}}

\includegraphics{https://static01.graylady3jvrrxbe.onion/images/2015/12/29/world/29korea/29korea-videoSixteenByNine1050-v2.jpg}

By \href{http://www.nytimes3xbfgragh.onion/by/choe-sang-hun}{Choe
Sang-Hun}

\begin{itemize}
\item
  Dec. 28, 2015
\item
  \begin{itemize}
  \item
  \item
  \item
  \item
  \item
  \item
  \end{itemize}
\end{itemize}

SEOUL, South Korea --- More than 70 years after the end of
\href{http://topics.nytimes3xbfgragh.onion/top/reference/timestopics/subjects/w/world_war_ii_/index.html?inline=nyt-classifier}{World
War II},
\href{http://topics.nytimes3xbfgragh.onion/top/news/international/countriesandterritories/southkorea/index.html?inline=nyt-geo}{South
Korea} and
\href{http://topics.nytimes3xbfgragh.onion/top/news/international/countriesandterritories/japan/index.html?inline=nyt-geo}{Japan}
reached a landmark agreement on Monday to resolve their dispute over
Korean women who were forced to serve as sex slaves for Japan's Imperial
Army.

The agreement, in which Japan made an apology and promised an \$8.3
million payment that would provide care for the women, was intended to
remove one of the most intractable logjams in relations between South
Korea and Japan, both crucial allies to the United States. The so-called
comfort women have been the most painful legacy of Japan's colonial rule
of Korea, which lasted from 1910 until Japan's defeat in 1945.

The Japanese and South Korean foreign ministers, announcing the
agreement in Seoul, said each side considered it a ``final and
irreversible resolution'' of the issue.

The apology and the payment, which, unlike a previous fund, will come
directly from the Japanese government, represent a compromise for
Japan's prime minister, Shinzo Abe, who has often been reluctant to
offer contrition for his country's militarist past.

The deal won praise from the governing party of President
\href{http://topics.nytimes3xbfgragh.onion/top/reference/timestopics/people/p/park_geunhye/index.html?inline=nyt-per}{Park
Geun-hye} of South Korea and from Secretary of State John Kerry, but it
was immediately criticized as insufficient by opposition politicians in
South Korea, where anti-Japanese sentiments still run deep, and by some
of the former sex slaves themselves.

``We are not craving for money,'' said Lee Yong-soo, 88, one of the
women. ``What we demand is that Japan make official reparations for the
crime it had committed.''

\href{https://www.nytimes3xbfgragh.onion/interactive/2015/08/13/world/asia/japan-ww2-shinzo-abe.html}{}

\includegraphics{https://static01.graylady3jvrrxbe.onion/images/2015/04/30/world/30ABE/30ABE-videoLarge.jpg}

\hypertarget{japans-apologies-for-world-war-ii}{%
\subsection{Japan's Apologies for World War
II}\label{japans-apologies-for-world-war-ii}}

Here is a look at major statements on Japan's war legacy by monarchs and
senior officials since its defeat in 1945.

The United States has repeatedly urged Japan and South Korea to resolve
the dispute, a stumbling block in American efforts to strengthen a joint
front with its Asian allies to confront China's growing assertiveness in
the region, as well as North Korea's attempt to build a nuclear arsenal.

Both Ms. Park and Mr. Abe were eager to forge an agreement this year,
the 50th anniversary of the treaty that normalized relations between
their two nations and the 70th anniversary of the end of the war.

``The issue of `comfort women' was a matter which, with the involvement
of the military authorities of the day, severely injured the honor and
dignity of many women,'' the foreign minister of Japan, Fumio Kishida,
said on Monday, as he read from the agreement at a news conference in
Seoul. ``In this regard, the government of Japan painfully acknowledges
its responsibility.''

Mr. Kishida also said that Mr.
\href{http://topics.nytimes3xbfgragh.onion/top/reference/timestopics/people/a/shinzo_abe/index.html?inline=nyt-per}{Abe}
``expresses anew sincere apologies and remorse from the bottom of his
heart to all those who suffered immeasurable pain and incurable physical
and psychological wounds as `comfort women.'~''

Mr. Abe later called Ms. Park to deliver the same apologies, Ms. Park's
office said.

``I hope that the two countries will cooperate closely to build trust
based on this agreement and open a new relationship,'' she was quoted as
telling Mr. Abe. Ms. Park, who had refused to hold a summit meeting with
Mr. Abe until
\href{http://www.nytimes3xbfgragh.onion/2015/11/02/world/asia/japan-south-korea-summit-park-geun-hye-shinzo-abe.html}{last
month}, had repeatedly urged Japan to address the grievances of the
women before relations could improve.

Although Japan had previously apologized, including in
\href{http://www.nytimes3xbfgragh.onion/1993/08/05/world/japan-admits-army-forced-women-into-war-brothels.html}{a
1993 statement} that acknowledged responsibility for the practice, the
agreement on Monday signaled something of a shift for Mr. Abe.

Image

A statue symbolizing Korean sex slaves in front of the Japanese Embassy
in Seoul.Credit...Chung Sung-Jun/Getty Images

As recently as last year, under pressure from his right wing to scrap
the apology, Mr. Abe and his allies agreed to
\href{http://www.nytimes3xbfgragh.onion/2014/03/01/world/asia/japan-to-review-apology-made-to-wwii-comfort-women.html}{review
the evidence} that led to it.

Under the agreement, the Japanese government will give the \$8.3 million
to a foundation that the South Korean government will establish to offer
medical, nursing and other services to the women. Japan initially
offered considerably less, according to news reports in both countries.
Officials said the women would most likely not receive any cash
payments.

That Tokyo will provide money from the national budget is a departure.
The fund created after the 1993 apology relied on private donors and was
\href{http://www.nytimes3xbfgragh.onion/2007/04/25/world/asia/25japan.html}{never
fully accepted} in South Korea. Although 60 South Korean women had
received financial aid from the fund, many others refused to accept it.

Japan also won an important concession from Seoul, a promise not to
criticize Tokyo over the issue again.

Historians say that at least tens of thousands of women, many of them
Korean, were lured or coerced to work in brothels from the early 1930s
until 1945. The Korean women who survived the war lived mostly in
silence because of the stigma, and many never married. Only in the early
1990s did some of them begin speaking out.

A total of 238 women have come forward in South Korea, but only 46 are
still living. Initial reactions to the resolution from the women were
far from welcoming.

\includegraphics{https://static01.graylady3jvrrxbe.onion/images/2015/12/29/world/29Korea-web/29Korea-web-articleLarge.jpg?quality=75\&auto=webp\&disable=upscale}

``The agreement does not reflect the views of former comfort women,''
Ms. Lee said at a news conference. ``I will ignore it completely.''

She said that the accord fell far short of the women's longstanding
demand that Japan admit legal responsibility and offer formal
reparations.

She said she also opposed the removal of
\href{http://www.nytimes3xbfgragh.onion/2011/12/16/world/asia/statute-in-seoul-becomes-focal-point-of-dispute-between-south-korea-and-japan.html}{a
statue of a girl} symbolizing comfort women that a civic group installed
in front of the Japanese Embassy in Seoul in 2011. During negotiations,
Japan insisted that the statue be removed, and South Korea said on
Monday that it would discuss the matter with the women.

A civic group, the Korean Council for the Women Drafted for Military
Sexual Slavery in Japan, called the deal ``shocking.''

``It's a humiliating diplomacy for South Korea to give a bushel only to
get a peck,'' the group said in a statement. ``The agreement is nothing
but a diplomatic collusion that thoroughly betrayed the wishes of
comfort women and the South Korean people.''

In a statement, Ms. Park appealed to South Koreans to accept the
agreement in the broader context of the need to improve ties with Japan,
a neighbor and important trading partner, adding that her government
wanted to seal a deal before the women died.

Japan has maintained that all legal issues stemming from its colonial
rule of Korea were resolved with the 1965 treaty. Negotiators from both
nations worked out a compromise with the vaguely worded agreement on
Monday, which did not clarify whether the responsibility that Japan
acknowledged was legal or moral. Mr. Kishida made it clear on Monday
that the money was not legal reparation.

The agreement also did not address a lingering debate over whether
coercion was a policy of imperial Japan.

The initial reaction in Japan was generally positive. Former Prime
Minister Tomiichi Murayama, who made a historic apology in 1995 for
Japan's role in World War II that many conservatives opposed, said that
Mr. Abe had ``decided well.''

Tomomi Inada, a right-wing member of Mr. Abe's Liberal Democratic Party,
suggested that the deal would be worthwhile if it put the dispute to
rest.

The Democratic Party of Japan, the largest opposition party, welcomed
the accord but cautioned Mr. Abe's government that any future support
for revisionist causes could undermine it.

Tsuneo Watanabe, a senior fellow at the Tokyo Foundation, a research
group, said Mr. Abe had chosen a pragmatic approach that elevated
economic and security ties over the bristly historical revisionism that
he has sometimes championed.

Stable relations with South Korea, he added, were vital to Mr. Abe's
most cherished foreign policy goal: nurturing alliances to counter the
growing power of China. ``Ultimately, Abe believes in the balance of
power.''

Hiroka Shoji, a researcher on East Asia at Amnesty International, said
the agreement should not be the end in securing justice for the former
sex slaves.

``The women were missing from the negotiation table, and they must not
be sold short in a deal that is more about political expediency than
justice,'' she said.

Advertisement

\protect\hyperlink{after-bottom}{Continue reading the main story}

\hypertarget{site-index}{%
\subsection{Site Index}\label{site-index}}

\hypertarget{site-information-navigation}{%
\subsection{Site Information
Navigation}\label{site-information-navigation}}

\begin{itemize}
\tightlist
\item
  \href{https://help.nytimes3xbfgragh.onion/hc/en-us/articles/115014792127-Copyright-notice}{©~2020~The
  New York Times Company}
\end{itemize}

\begin{itemize}
\tightlist
\item
  \href{https://www.nytco.com/}{NYTCo}
\item
  \href{https://help.nytimes3xbfgragh.onion/hc/en-us/articles/115015385887-Contact-Us}{Contact
  Us}
\item
  \href{https://www.nytco.com/careers/}{Work with us}
\item
  \href{https://nytmediakit.com/}{Advertise}
\item
  \href{http://www.tbrandstudio.com/}{T Brand Studio}
\item
  \href{https://www.nytimes3xbfgragh.onion/privacy/cookie-policy\#how-do-i-manage-trackers}{Your
  Ad Choices}
\item
  \href{https://www.nytimes3xbfgragh.onion/privacy}{Privacy}
\item
  \href{https://help.nytimes3xbfgragh.onion/hc/en-us/articles/115014893428-Terms-of-service}{Terms
  of Service}
\item
  \href{https://help.nytimes3xbfgragh.onion/hc/en-us/articles/115014893968-Terms-of-sale}{Terms
  of Sale}
\item
  \href{https://spiderbites.nytimes3xbfgragh.onion}{Site Map}
\item
  \href{https://help.nytimes3xbfgragh.onion/hc/en-us}{Help}
\item
  \href{https://www.nytimes3xbfgragh.onion/subscription?campaignId=37WXW}{Subscriptions}
\end{itemize}
