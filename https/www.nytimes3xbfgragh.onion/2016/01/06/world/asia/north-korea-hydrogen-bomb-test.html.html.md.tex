Sections

SEARCH

\protect\hyperlink{site-content}{Skip to
content}\protect\hyperlink{site-index}{Skip to site index}

\href{https://www.nytimes3xbfgragh.onion/section/world/asia}{Asia
Pacific}

\href{https://myaccount.nytimes3xbfgragh.onion/auth/login?response_type=cookie\&client_id=vi}{}

\href{https://www.nytimes3xbfgragh.onion/section/todayspaper}{Today's
Paper}

\href{/section/world/asia}{Asia Pacific}\textbar{}North Korea Says It
Has Detonated Its First Hydrogen Bomb

\url{https://nyti.ms/1kKrbwW}

\begin{itemize}
\item
\item
\item
\item
\item
\item
\end{itemize}

Advertisement

\protect\hyperlink{after-top}{Continue reading the main story}

Supported by

\protect\hyperlink{after-sponsor}{Continue reading the main story}

\hypertarget{north-korea-says-it-has-detonated-its-first-hydrogen-bomb}{%
\section{North Korea Says It Has Detonated Its First Hydrogen
Bomb}\label{north-korea-says-it-has-detonated-its-first-hydrogen-bomb}}

\includegraphics{https://static01.graylady3jvrrxbe.onion/images/2016/01/06/world/06korea_web2/06korea_web2-articleLarge.jpg?quality=75\&auto=webp\&disable=upscale}

By \href{http://www.nytimes3xbfgragh.onion/by/david-e-sanger}{David E.
Sanger} and
\href{http://www.nytimes3xbfgragh.onion/by/choe-sang-hun}{Choe Sang-Hun}

\begin{itemize}
\item
  Jan. 5, 2016
\item
  \begin{itemize}
  \item
  \item
  \item
  \item
  \item
  \item
  \end{itemize}
\end{itemize}

WASHINGTON ---
\href{http://www.nytimes3xbfgragh.onion/2016/01/07/world/asia/north-korea-hydrogen-bomb-claim-reactions.html}{North
Korea} declared on Tuesday that it had detonated its first hydrogen
bomb.

The assertion, if true, would dramatically escalate the nuclear
challenge from one of the world's most isolated and dangerous states.

In an announcement, North Korea said that the test had been a ``complete
success.'' But it was difficult to tell whether the statement was true.
North Korea has made repeated claims about its nuclear capabilities that
outside analysts have greeted with skepticism.

``This is the self-defensive measure we have to take to defend our right
to live in the face of the nuclear threats and blackmail by the United
States and to guarantee the security of the Korean Peninsula,'' a female
North Korean announcer said, reading the statement on Central
Television, the state-run network.

The North's announcement came about an hour after detection devices
around the world had picked up a 5.1 seismic event along the country's
northeast coast.

It may be weeks or longer before detectors sent aloft by the United
States and other powers can determine what kind of test was conducted.
Ned Price, a spokesman for the White House National Security Council,
said in a statement that American officials ``cannot confirm these
claims at this time.''

\includegraphics{https://static01.graylady3jvrrxbe.onion/images/2016/01/06/world/06korea_web3/06korea_web3-articleLarge.jpg?quality=75\&auto=webp\&disable=upscale}

But he said the White House expected ``North Korea to abide by its
international obligations and commitments.''

The tremors occurred at or near the Punggye-ri nuclear test site, where
three previous tests have been conducted over the past nine years.

In recent weeks, the North's aggressive young leader, Kim Jong-un, has
boasted that the country has finally developed the technology to build a
thermonuclear weapon --- far more powerful than the low-yield devices
tested first in 2006, then in different configurations months after
President Obama took office in 2009 and again in 2013.

The North Korean announcement said the test had been personally ordered
by Mr. Kim, only three days after he signed an order on Sunday for North
Korean engineers to press ahead with the attempt.

The announcer added that for the North to give up its nuclear weapons
while Washington's ``hostile policy'' continued would be ``as foolish as
for a hunter to lay down his rifle while a ferocious wolf is charging at
him.''

Satellite photographs analyzed by 38 North, a Washington research
institute that follows the North's nuclear activity closely, showed
evidence of a new tunnel being dug in recent weeks.

Another test by itself would not be that remarkable. The North is
believed to have enough plutonium for eight to 12 weapons, and several
years ago it revealed a new program to enrich uranium, the other fuel
for a nuclear weapon.

But if the North Korean claim about a hydrogen bomb is true, this test
was of a different, and significantly more threatening, nature.

In recent weeks, Mr. Kim, believed to be in his early 30s and determined
to accelerate the nuclear weapons program that his grandfather and his
father promoted to give the broken country leverage and influence,
boasted that North Korea had finally developed the technology to build a
thermonuclear weapon.

When Mr. Kim first made the claim, in December, the White House
expressed considerable skepticism, and several other experts say that
the accomplishment would be a stretch, though not impossible.

Image

Japan's defense minister, Gen Nakatani, center, running up to Prime
Minister Shinzo Abe's residence in Tokyo on Wednesday after reports of a
tremor near a North Korean nuclear test site.Credit...Kyodo News, via
Associated Press

Outside analysts took the claim as the latest of several hard-to-verify
assertions that the isolated country has made about its nuclear
capabilities. But some also said that although North Korea did not yet
have H-bomb capability, it might be developing and preparing to test a
boosted fission bomb, more powerful than a traditional nuclear weapon.

Weapon designers can easily boost the destructive power of an atom bomb
by putting at its core a small amount of tritium, a radioactive form of
hydrogen.

Lee Sang-cheol, the top nonproliferation official at the South Korean
Defense Ministry, told a forum in Seoul last month that although Mr.
Kim's hydrogen bomb boasts might be propaganda for his domestic
audience, there was a ``high likelihood'' that North Korea might have
been developing such a boosted fission weapon.

And according to a paper obtained by the South Korean news agency Yonhap
last week, the Chemical, Biological and Radiological Command of the
South Korean military ``did not rule out the possibility'' of a boosted
fission bomb test by the North, although it added it ``does not believe
it is yet capable of directly testing hydrogen bombs.''

For the Obama administration, which only six months ago defused the
Iranian nuclear threat with an agreement to limit its capabilities for
at least a decade, the announcement rekindles another major nuclear
challenge --- one that the administration has never found a way to
manage.

The North has refused to enter the kind of negotiations that Iran did.
Unlike Iran, which denies it has interest in nuclear weapons, the North
has forged ahead with tests and told the West and China it would never
give them up.

Mr. Obama, determined not to give the country new concessions, has
neither acknowledged that North Korea is now a nuclear power nor
negotiated with it. The White House has said that it would only restart
talks with the North if the goal --- agreed to by all parties --- was a
``denuclearized Korean Peninsula.''

China has also failed in its efforts to rein in Mr. Kim. He has never
been invited to Beijing since his father's death, and Chinese officials
are fairly open in their expressions of contempt for him. But they have
not abandoned him, or cut off the aid that keeps the country afloat.

With the test conducted Tuesday night --- Wednesday in North Korea ---
three of the North's four explosions will have occurred during Mr.
Obama's time in office.

Combined with the North's gradually increasing missile technology, its
nuclear program poses a growing threat to the region --- though it is
still not clear the North knows how to mount a nuclear weapon on one of
its missiles.

The test is bound to figure in the American presidential campaign, where
several candidates have already cited the North's nuclear
experimentation as evidence of American weakness --- though they have
not prescribed alternative strategies for choking off the program.

The United States did not develop its first thermonuclear weapons ---
commonly known as hydrogen bombs --- until 1952, seven years after the
first and only use of nuclear weapons in wartime, the weapons dropped on
Hiroshima and Nagasaki. Russia, China and other powers soon followed
suit.

Advertisement

\protect\hyperlink{after-bottom}{Continue reading the main story}

\hypertarget{site-index}{%
\subsection{Site Index}\label{site-index}}

\hypertarget{site-information-navigation}{%
\subsection{Site Information
Navigation}\label{site-information-navigation}}

\begin{itemize}
\tightlist
\item
  \href{https://help.nytimes3xbfgragh.onion/hc/en-us/articles/115014792127-Copyright-notice}{©~2020~The
  New York Times Company}
\end{itemize}

\begin{itemize}
\tightlist
\item
  \href{https://www.nytco.com/}{NYTCo}
\item
  \href{https://help.nytimes3xbfgragh.onion/hc/en-us/articles/115015385887-Contact-Us}{Contact
  Us}
\item
  \href{https://www.nytco.com/careers/}{Work with us}
\item
  \href{https://nytmediakit.com/}{Advertise}
\item
  \href{http://www.tbrandstudio.com/}{T Brand Studio}
\item
  \href{https://www.nytimes3xbfgragh.onion/privacy/cookie-policy\#how-do-i-manage-trackers}{Your
  Ad Choices}
\item
  \href{https://www.nytimes3xbfgragh.onion/privacy}{Privacy}
\item
  \href{https://help.nytimes3xbfgragh.onion/hc/en-us/articles/115014893428-Terms-of-service}{Terms
  of Service}
\item
  \href{https://help.nytimes3xbfgragh.onion/hc/en-us/articles/115014893968-Terms-of-sale}{Terms
  of Sale}
\item
  \href{https://spiderbites.nytimes3xbfgragh.onion}{Site Map}
\item
  \href{https://help.nytimes3xbfgragh.onion/hc/en-us}{Help}
\item
  \href{https://www.nytimes3xbfgragh.onion/subscription?campaignId=37WXW}{Subscriptions}
\end{itemize}
