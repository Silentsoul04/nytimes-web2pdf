Sections

SEARCH

\protect\hyperlink{site-content}{Skip to
content}\protect\hyperlink{site-index}{Skip to site index}

\href{https://www.nytimes3xbfgragh.onion/section/food}{Food}

\href{https://myaccount.nytimes3xbfgragh.onion/auth/login?response_type=cookie\&client_id=vi}{}

\href{https://www.nytimes3xbfgragh.onion/section/todayspaper}{Today's
Paper}

\href{/section/food}{Food}\textbar{}Benoit in Midtown Is the Bistro That
Will Take You to Paris

\url{https://nyti.ms/1SqxKnx}

\begin{itemize}
\item
\item
\item
\item
\item
\item
\end{itemize}

Advertisement

\protect\hyperlink{after-top}{Continue reading the main story}

Supported by

\protect\hyperlink{after-sponsor}{Continue reading the main story}

\href{/column/restaurant-review}{Restaurant Review}

\hypertarget{benoit-in-midtown-is-the-bistro-that-will-take-you-to-paris}{%
\section{Benoit in Midtown Is the Bistro That Will Take You to
Paris}\label{benoit-in-midtown-is-the-bistro-that-will-take-you-to-paris}}

\href{https://www.nytimes3xbfgragh.onion/slideshow/2016/02/03/dining/benoit.html}{}

\hypertarget{benoit}{%
\subsection{Benoit}\label{benoit}}

13 Photos

View Slide Show ›

\includegraphics{https://static01.graylady3jvrrxbe.onion/images/2016/02/03/dining/03REST-BENOIT-slide-89FQ/03REST-BENOIT-slide-89FQ-articleLarge.jpg?quality=75\&auto=webp\&disable=upscale}

Evan Sung for The New York Times

\begin{itemize}
\tightlist
\item
  Benoit\\
  ★★ French \$\$\$\$ 60 West 55th Street 646-943-7373
\end{itemize}

\href{http://www.opentable.com/single.aspx?ref=4201\&rid=21025}{Reserve
a Table}

When you make a reservation at an independently reviewed restaurant
through our site, we earn an affiliate commission.

By \href{http://www.nytimes3xbfgragh.onion/by/pete-wells}{Pete Wells}

\begin{itemize}
\item
  Feb. 2, 2016
\item
  \begin{itemize}
  \item
  \item
  \item
  \item
  \item
  \item
  \end{itemize}
\end{itemize}

Like many New Yorkers, I have convinced myself that a rickety fire
escape platform is a terrace and that a segment of the Empire State
Building's antenna seen through a sliver of window constitutes a
panoramic skyline view. The same capacity for wishful self-delusion must
be what keeps us going to bistros that are nothing like anything in
France.

The chalkboards in painstaking cursive that misspell half the menu; the
sad frites and scrawny mussels and refrigerated cheeses; the old-guard
dishes put through modern contortions until they end up looking like
Gérard Depardieu wearing Jeggings --- we put up with these signs of
forgery because we want the real thing so badly.

To get it, we need to go to France. But right now, the most perfect
substitute bistro in New York is
\href{http://www.benoitny.com/}{Benoit}.

It took Alain Ducasse, who owns the century-old original in Paris, a few
years to get its Midtown incarnation right. ``It's odd the way Benoit
does some dishes so well but misses the bull's-eye with mainstays that
should get the most finicky attention,'' Frank Bruni wrote shortly after
it opened in 2008, pinning one star on
\href{http://www.nytimes3xbfgragh.onion/2008/07/09/dining/reviews/09rest.html}{his
review in The Times}. A year later, in
\href{http://www.nytimes3xbfgragh.onion/2009/06/17/dining/reviews/17brief-002.html}{an
unstarred update}, Julia Moskin found a menu that was coming into focus
but still carried an occasional ``hint of airline food.'' She also wrote
that the servers ``seem to hope that dinner customers will leave early
and stay away forever.''

Philippe Bertineau isn't cooking airline food. He is the restaurant's
third chef. With any luck, Mr. Ducasse won't need a fourth for a long
time.

The menu works like a season at the Metropolitan Opera. A core of
preposterously old-fashioned classics is rounded out with a few modern
compositions to keep everybody on their toes.

To understand the point of keeping rafter-rattling war horses in the
repertoire, all I had to do was eat Benoit's quenelles de brochet. These
two little footballs of happiness are improbably smooth, almost but not
quite fluffy, filled with the freshwater richness of pike. Each one is
enveloped in a thick brown gravy of Nantua sauce that is 10 times
richer, made from crayfish and lobster. I tasted it, and the whole
chorus marched out on the stage at once, the orchestra pounding and the
fat lady throwing her head back and letting it rip. A standing ovation
would have followed if Benoit's quenelles hadn't made me feel a bit like
the fat lady myself.

Even more inimical to hopes of mobility was Mr. Bertineau's cassoulet. A
gleefully debilitating arsenal of duck confit, duck sausage and cured
pork, it is also a magnificent pot of tender baked beans soaked in
garlic and fat. Either the meat or the beans could slow your speed.
Together they act like an anchor.

The calf's liver, an oblong lobe shaped like Argentina, was seared to a
dark gold in the copper sauté pan that escorted it to the table. A step
beyond the medium-rare the server had suggested, it was still very good.
It had no overcooked livery flavor, just the ripe tang of organ meat
that dissolved into very soft Lyonnaise potatoes and onions.

To begin, there is darkly concentrated onion soup, choked with melted
Gruyère that seems never to run out. Snails in the divots of a heavy
glazed dish have a wafer of parsley crust on top that's almost as fun to
eat as a rag of bread dunked in their garlic butter.

Obviously, like so much of Benoit's menu, these appetizers predate the
electrocardiogram. It is possible to start with something lighter.
Regulars often tick off choices on the hors d'oeuvres list, a separate
sheet of paper that comes with a golf pencil: a poached and
oil-marinated sardine; crisp sticks of celery root in rémoulade that may
need more mustard; ham with flageolets and a handful of other dried
beans, each cooked until just tender; the egg mayo, on the menu since
opening day. Five of these little dishes served together make an
excellent \$19 lunch, a kind of Lyonnaise bento box.

The greatest bistros in France are magnetic because they are invariable;
you know the classics will be made the same way they were 15 or 45 years
ago. Benoit is not always this kind of bistro, though this may be a
symptom of its distance from Lyon. The pâté en croûte, from an 1892
recipe, can be dry and underseasoned. The filet mignon au poivre can be
dullish, although the pepper-cream sauce puts up a strong fight.

In compensation, though, Mr. Bertineau cooks almost as well outside the
bistro canon as inside it. He shows this with seafood especially.

One night, my guests, elbow deep in a wild boar stew, threw me looks of
pity: poor guy, stuck with king salmon, butternut squash and
black-trumpet mushrooms. I didn't let on, but I was perfectly content.
The fish, roasted gently, was crazily tender and the white-wine sauce so
appealing that I spooned way more than I needed from the sauceboat left
on the table. (The right French sauce can be a dish in itself.)

The baba is not as spectacular at it was at Alain Ducasse New York
before that restaurant closed. The cake's halves can be too sober,
depending on whether the server applies the Armagnac as if he's putting
out a house fire or baptizing a baby. But other vintage desserts can
restore the faith of any New Yorker who's been brought down by bland
mille-feuilles or watery crème caramels or confused tarte Tatins.

The servers on my recent visits showed no impatience with lingerers;
they simply went about their business with that straight-spined,
matter-of-fact French-waiter style. This is sometimes mistaken for
rudeness but is closer to pride and a confidence that everybody,
including the customers, knows how to act.

It's strange that it took Mr. Ducasse to give New York its most
convincing version of this very un-cheflike style of cooking. That a
highly traditional bistro survived while the other, more inventive
restaurants with which he hoped to conquer Manhattan did not is
peculiar, too --- a missed opportunity for the chef and New Yorkers, who
love French food even when it doesn't love them back.

Advertisement

\protect\hyperlink{after-bottom}{Continue reading the main story}

\hypertarget{site-index}{%
\subsection{Site Index}\label{site-index}}

\hypertarget{site-information-navigation}{%
\subsection{Site Information
Navigation}\label{site-information-navigation}}

\begin{itemize}
\tightlist
\item
  \href{https://help.nytimes3xbfgragh.onion/hc/en-us/articles/115014792127-Copyright-notice}{©~2020~The
  New York Times Company}
\end{itemize}

\begin{itemize}
\tightlist
\item
  \href{https://www.nytco.com/}{NYTCo}
\item
  \href{https://help.nytimes3xbfgragh.onion/hc/en-us/articles/115015385887-Contact-Us}{Contact
  Us}
\item
  \href{https://www.nytco.com/careers/}{Work with us}
\item
  \href{https://nytmediakit.com/}{Advertise}
\item
  \href{http://www.tbrandstudio.com/}{T Brand Studio}
\item
  \href{https://www.nytimes3xbfgragh.onion/privacy/cookie-policy\#how-do-i-manage-trackers}{Your
  Ad Choices}
\item
  \href{https://www.nytimes3xbfgragh.onion/privacy}{Privacy}
\item
  \href{https://help.nytimes3xbfgragh.onion/hc/en-us/articles/115014893428-Terms-of-service}{Terms
  of Service}
\item
  \href{https://help.nytimes3xbfgragh.onion/hc/en-us/articles/115014893968-Terms-of-sale}{Terms
  of Sale}
\item
  \href{https://spiderbites.nytimes3xbfgragh.onion}{Site Map}
\item
  \href{https://help.nytimes3xbfgragh.onion/hc/en-us}{Help}
\item
  \href{https://www.nytimes3xbfgragh.onion/subscription?campaignId=37WXW}{Subscriptions}
\end{itemize}
