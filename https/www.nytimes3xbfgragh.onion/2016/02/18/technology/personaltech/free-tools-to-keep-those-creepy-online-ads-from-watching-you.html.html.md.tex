Sections

SEARCH

\protect\hyperlink{site-content}{Skip to
content}\protect\hyperlink{site-index}{Skip to site index}

\href{https://www.nytimes3xbfgragh.onion/section/technology/personaltech}{Personal
Tech}

\href{https://myaccount.nytimes3xbfgragh.onion/auth/login?response_type=cookie\&client_id=vi}{}

\href{https://www.nytimes3xbfgragh.onion/section/todayspaper}{Today's
Paper}

\href{/section/technology/personaltech}{Personal Tech}\textbar{}Free
Tools to Keep Those Creepy Online Ads From Watching You

\href{https://nyti.ms/1mHBRx6}{https://nyti.ms/1mHBRx6}

\begin{itemize}
\item
\item
\item
\item
\item
\end{itemize}

Advertisement

\protect\hyperlink{after-top}{Continue reading the main story}

Supported by

\protect\hyperlink{after-sponsor}{Continue reading the main story}

\href{/column/tech-fix}{Tech Fix}

\hypertarget{free-tools-to-keep-those-creepy-online-ads-from-watching-you}{%
\section{Free Tools to Keep Those Creepy Online Ads From Watching
You}\label{free-tools-to-keep-those-creepy-online-ads-from-watching-you}}

\includegraphics{https://static01.graylady3jvrrxbe.onion/images/2016/09/14/nytnow/14smliving-ads/14smliving-ads-articleLarge-v2.jpg?quality=75\&auto=webp\&disable=upscale}

By \href{http://www.nytimes3xbfgragh.onion/by/brian-x-chen}{Brian X.
Chen} and
\href{http://www.nytimes3xbfgragh.onion/by/natasha-singer}{Natasha
Singer}

\begin{itemize}
\item
  Feb. 17, 2016
\item
  \begin{itemize}
  \item
  \item
  \item
  \item
  \item
  \end{itemize}
\end{itemize}

SAY you're doing a web search on something like the flu. The next thing
you know, an ad for a flu remedy pops up on your web browser, or your
video streaming service starts playing a commercial for Tylenol.

The content of those ads is no coincidence. Digital ads are able to
follow people around the Internet because advertisers often place
invisible trackers on the websites you visit. Their goal is to collect
details on everywhere you go on the Internet and use that data to serve
targeted ads to your computer, smartphone and connected television.

This global commercial surveillance of consumers is poised to become
more extensive as tech companies expand into the Internet of Things, a
category that includes wearable computers and connected home appliances
like smart thermostats and refrigerators. Amazon, eBay, Facebook and
Google can already follow users from device to device because people log
in to their services with the same IDs on various gadgets.

For other marketing companies, tracking people on multiple
Internet-connected devices has become a holy grail. The process is
complex, because some lack the direct relationship with people that the
giant tech companies already have. Only about 6 percent of marketers can
reliably track a customer on all of that customer's devices, according
to the research firm eMarketer. But advertisers are
\href{http://www.emarketer.com/Webinar/Cross-Device-TargetingWhat-Watch-2016/4000126}{working
toward}the goal.

``Our privacy is completely under assault with all these connected
devices,'' said Jeremiah Grossman, the founder of WhiteHat Security, a
web security firm.

So what better time to get a head start on defending yourself against
web snoops (as if
\href{http://www.nytimes3xbfgragh.onion/2015/11/19/technology/personaltech/foiling-electronic-snoops-in-email.html}{email
trackers}, which this column covered last year, weren't annoying enough
already)? Many companies offer tools to help obscure your digital
footprints while you're browsing the web. We researched and tested four
tracker blockers and found their results varied widely. In the end, the
app Disconnect became our anti-tracking tool of choice.

Here's how web tracking works: In general, targeting individuals with
digital ads involves a sophisticated ecosystem of third parties --- like
online advertising networks, data brokers and analytics companies ---
that compile information on consumers.

When you visit websites, these companies typically pick out your browser
or phone using technologies like cookies, which contain unique
alphanumeric identification tags that can enable trackers to identify
your activities as you move from site to site. To sell ads delivered to
certain categories of consumers, like suburban singles looking for
romance, companies may sync these ID tags to pinpoint individuals.

The downside is, your browsing history may contain sensitive information
about your health concerns, political affiliations, family problems,
religious beliefs or sexual habits.

``More than just being creepy, it's a huge violation of privacy,'' said
\href{https://www.eff.org/about/staff/cooper-quintin}{Cooper Quintin}, a
privacy advocate for the Electronic Frontier Foundation, a digital
rights nonprofit that also offers the anti-tracking tool Privacy Badger.
``People need to be able to read things and do things and talk about
things without having to worry that they're being watched or recorded
somewhere.''

We took a close look at four free privacy tools:
\href{https://www.ghostery.com/}{Ghostery},
\href{https://disconnect.me/}{Disconnect},
\href{https://redmorph.com/}{RedMorph} and
\href{https://www.eff.org/privacybadger}{Privacy Badger}. We tested them
with the Google Chrome browser on the top 20 news websites, including
Yahoo News, CNN, The Huffington Post and The New York Times.

The tracker busters generally work in similar ways. You download and
install an add-on for a web browser like Chrome or Mozilla Firefox. The
anti-tracking companies each compile a list of known web domains that
serve trackers or show patterns of tracking services. Then when someone
connects to a website, the tools prevent the browser from loading any
element that matches their blacklist.

\href{https://www.ghostery.com/}{Ghostery}, a popular tracker blocker,
was the most difficult to set up. When you install it, it asks you to
manually select the trackers you want to block. Our problem with that
approach is that there are hundreds of trackers, and most consumers
probably won't recognize most of them, putting the onus on users to
research which specific services they might wish to block.

Scott Meyer, the chief executive of Ghostery, said this had been a
deliberate design choice. When trackers are blocked, parts of websites
may not function, so it is less confusing to let users experiment and
decide which ones to block on their own, he said.

``We block nothing by default,'' he said. ``That's in direct contrast to
other companies who are saying, `We're turning everything off and let
you turn whatever you want back on.' That's way too complex for users.''

The tracker blocking tool \href{https://redmorph.com/}{RedMorph} takes
the opposite approach. It blocks every tracking signal it can detect and
lets people decide which ones to allow. For parents concerned about
their children's Internet use, RedMorph also offers a service to filter
out certain sites or block certain swear words or other language they
find inappropriate.

``When you go home, you lock the door and you may pull down the shades
at night,'' said Abhay Edlabadkar, the chief executive of RedMorph.
``You should have the same level of privacy control over your Internet
activities.''

In our tests, RedMorph was the most thorough with blocking trackers. It
blocked 22 of them on USAToday.com, whereas Privacy Badger blocked
seven, Disconnect blocked eight and Ghostery detected eight.

But in the process, RedMorph caused the most collateral damage. It
blocked some videos on the websites for CNN, USA Today, Bleacher Report,
The New York Times and The Daily News. It also broke the recommended
reading list on Business Insider and a Twitter box on BuzzFeed. For
people who run into issues loading websites, the company offers an
``Easy Fix'' button to stop blocking a website's trackers, but that's
hardly an ideal solution when it causes so many websites to malfunction.
Mr. Edlabadkar of RedMorph said the tool was blocking some videos or
recommended reading lists because they were loading only after a tracker
had been loaded first.

That leaves \href{https://www.eff.org/privacybadger}{Privacy Badger} and
\href{https://disconnect.me/}{Disconnect}. Privacy Badger detects
third-party domains that users are connecting with when they're loading
a website and blocks those domains only if they are determined to be
tracking you. Its widget shows sliding bars of trackers it has detected.
The ones in red are blocked and the green ones are allowed.

Disconnect takes a similarly nuanced approach. The company said some
tracking was fair and necessary for a website to work properly --- for
example, if a site like The New York Times is using analytics to collect
information about readers, as it describes in its
\href{http://www.nytimes3xbfgragh.onion/content/help/rights/privacy/policy/privacy-policy.html}{privacy
policy}. However, Disconnect will block trackers from third parties that
are collecting, retaining or sharing user data. On its website, it
\href{https://disconnect.me/trackerprotection\#trackers-we-block}{publishes
lists} of trackers it blocks and those it allows, along with
explanations of its policy.

``We really focus on privacy rather than blocking ads that are done in a
respectful way,'' said Casey Oppenheim, the chief executive of
Disconnect. ``It's important we have the ability for publishers to
survive and make money. I think there's a middle ground.''

In the end, we picked Disconnect as our favorite tool because it was the
easiest to understand. It organizes the types of tracking requests it is
blocking into different categories: advertising, analytics, social media
and content.

Mr. Grossman of WhiteHat Security also tested tracking blockers and
chose Disconnect for similar reasons. He breaks his online activities
into two separate web browsers to make himself more difficult to track:
On one browser, he does everyday tasks like reading news articles; on
the other browser he logs into accounts that are linked to his personal
identity, like online banking sites and Amazon.

But Mr. Grossman said that in the broad arms race between consumers and
advertisers, the advertisers always find some way to
\href{https://blog.whitehatsec.com/the-ad-blocking-wars-ad-blockers-vs-ad-tech/}{outmaneuver
us}.

``We're talking megabillion-dollar industries totally designed to track
you online,'' he said. ``That's their mission in life.''

Advertisement

\protect\hyperlink{after-bottom}{Continue reading the main story}

\hypertarget{site-index}{%
\subsection{Site Index}\label{site-index}}

\hypertarget{site-information-navigation}{%
\subsection{Site Information
Navigation}\label{site-information-navigation}}

\begin{itemize}
\tightlist
\item
  \href{https://help.nytimes3xbfgragh.onion/hc/en-us/articles/115014792127-Copyright-notice}{©~2020~The
  New York Times Company}
\end{itemize}

\begin{itemize}
\tightlist
\item
  \href{https://www.nytco.com/}{NYTCo}
\item
  \href{https://help.nytimes3xbfgragh.onion/hc/en-us/articles/115015385887-Contact-Us}{Contact
  Us}
\item
  \href{https://www.nytco.com/careers/}{Work with us}
\item
  \href{https://nytmediakit.com/}{Advertise}
\item
  \href{http://www.tbrandstudio.com/}{T Brand Studio}
\item
  \href{https://www.nytimes3xbfgragh.onion/privacy/cookie-policy\#how-do-i-manage-trackers}{Your
  Ad Choices}
\item
  \href{https://www.nytimes3xbfgragh.onion/privacy}{Privacy}
\item
  \href{https://help.nytimes3xbfgragh.onion/hc/en-us/articles/115014893428-Terms-of-service}{Terms
  of Service}
\item
  \href{https://help.nytimes3xbfgragh.onion/hc/en-us/articles/115014893968-Terms-of-sale}{Terms
  of Sale}
\item
  \href{https://spiderbites.nytimes3xbfgragh.onion}{Site Map}
\item
  \href{https://help.nytimes3xbfgragh.onion/hc/en-us}{Help}
\item
  \href{https://www.nytimes3xbfgragh.onion/subscription?campaignId=37WXW}{Subscriptions}
\end{itemize}
