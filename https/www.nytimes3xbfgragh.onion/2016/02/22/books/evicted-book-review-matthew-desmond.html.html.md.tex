Sections

SEARCH

\protect\hyperlink{site-content}{Skip to
content}\protect\hyperlink{site-index}{Skip to site index}

\href{https://www.nytimes3xbfgragh.onion/section/books}{Books}

\href{https://myaccount.nytimes3xbfgragh.onion/auth/login?response_type=cookie\&client_id=vi}{}

\href{https://www.nytimes3xbfgragh.onion/section/todayspaper}{Today's
Paper}

\href{/section/books}{Books}\textbar{}Review: In `Evicted,' Home Is an
Elusive Goal for America's Poor

\url{https://nyti.ms/1oVxEry}

\begin{itemize}
\item
\item
\item
\item
\item
\end{itemize}

Advertisement

\protect\hyperlink{after-top}{Continue reading the main story}

Supported by

\protect\hyperlink{after-sponsor}{Continue reading the main story}

\hypertarget{review-in-evicted-home-is-an-elusive-goal-for-americas-poor}{%
\section{Review: In `Evicted,' Home Is an Elusive Goal for America's
Poor}\label{review-in-evicted-home-is-an-elusive-goal-for-americas-poor}}

\includegraphics{https://static01.graylady3jvrrxbe.onion/images/2016/02/22/arts/22BOOKDESMOND/22BOOKDESMOND-articleLarge-v2.jpg?quality=75\&auto=webp\&disable=upscale}

By \href{http://www.nytimes3xbfgragh.onion/by/jennifer-senior}{Jennifer
Senior}

\begin{itemize}
\item
  Feb. 21, 2016
\item
  \begin{itemize}
  \item
  \item
  \item
  \item
  \item
  \end{itemize}
\end{itemize}

One of the most heartbreaking moments in Matthew Desmond's ``Evicted:
Poverty and Profit in the American City''--- and there's a shameful
assortment to choose from --- is when 13-year-old Ruby Hinkston takes
refuge in the public library. She's come to use the computer. It turns
out that she's been slowly building her dream house with a free online
game, and she wants to visit it again.

``It had clean, light-reflecting floors,'' Mr. Desmond writes, ``a bed
with sheets \emph{and} pillowcases, and a desk for doing schoolwork.''

This cheerful vision in pixels forms an almost unbearable contrast to
the filth of Ruby's own apartment. The kitchen sink is stopped up, as is
the bathtub and toilet. There are mattresses everywhere, their exposed
innards revealing humming burrows of cockroaches --- and the mattresses
may be the least terrifying of their redoubts. They also fill the
kitchen drawers and erupt from the nonworking drains.

Living in extreme poverty in the United States means waging an almost
gladiatorial battle for creature comforts that luckier people take for
granted. And of all those comforts, perhaps the most important is a
stable, dignified home. Yet as a culture, notes Mr. Desmond, we have
somehow failed to commit ourselves to providing this most fundamental
and obvious necessity.

``Every year in this country,'' he writes, ``families are evicted from
their homes not by the tens of thousands or even the hundreds of
thousands, but by the millions.''

``Evicted'' is a regal hybrid of ethnography and policy reporting. It
follows the lives of eight families in Milwaukee, some black and some
white, all several leagues below the poverty line. Mr. Desmond, a
sociologist and a co-director of the Justice and Poverty Project at
Harvard, lived among them in 2008 and 2009 --- first in the poor, white
\href{http://www.jsonline.com/news/milwaukee/29569949.html}{College
Mobile Home Park}, a dark hole of vanished ambitions and drug abuse (one
woman is ``Heroin Susie,'' not to be confused with ``Office Susie'');
and then in a rooming house run by the landlords Sherrena and Quentin,
who eventually introduced him to many of their black tenants in other
properties. One of those units was Ruby's, with the volcanic cockroach
problem.

The result is an exhaustively researched, vividly realized and, above
all, unignorable book --- after ``Evicted,'' it will no longer be
possible to have a serious discussion about poverty without having a
serious discussion about housing. Like Jonathan Kozol's
``\href{http://www.nytimes3xbfgragh.onion/1991/09/25/books/books-of-the-times-shortchanging-the-nation-s-children.html}{Savage
Inequalities},'' or Barbara Ehrenreich's
``\href{http://barbaraehrenreich.com/nickel-and-dimed-by-barbara-ehrenreich/}{Nickel
and Dimed},'' or Michelle Alexander's
``\href{http://newjimcrow.com/praise-for-the-new-jim-crow}{The New Jim
Crow,}'' this sweeping, yearslong project makes us consider inequality
and economic justice in ways we previously had not. It's sure to capture
the attention of politicians. (\emph{Hillary, what are you reading this
summer?}) Through data and analysis and storytelling, it issues a call
to arms without ever once raising its voice.

What makes ``Evicted'' so eye-opening and original is its emphasis. Most
examinations of the poorest poor look at those in public housing, not
those who've been brutally cast into the private rental market. Yet this
is precisely where most of the impoverished must live. Sixty-seven
percent of poor renting families received no federal assistance for
housing at all in 2013 --- there simply weren't enough vouchers or
subsidized apartments to go around. The very people least capable of
spending 70 to 80 percent of their incomes on rent are exactly the ones
forced to do so.

``If incarceration had come to define the lives of men from impoverished
black neighborhoods,'' Mr. Desmond writes, ``eviction was shaping the
lives of women. Poor black men were locked up. Poor black women were
locked out.''

Image

Matthew DesmondCredit...Amir Levy for The New York Times

With vacancy rates for cheap housing in the single digits, the moment is
ripe for exploitation. It's a landlord's market. So exploit they do.

They keep the rents at punishingly high levels --- if their latest
tenants can't pay, they can always evict them, pocket the security
deposit, and move on to the next desperate soul. They rent units that
run afoul of the property code, which is perfectly legal in Milwaukee as
long as tenants are told in advance --- \emph{caveat rentor.} They deny
their tenants basic appliances, which the law also, amazingly, permits
--- not just in Wisconsin, but in most states.

``If you didn't include a stove or a refrigerator,'' explains Mr.
Desmond, ``you didn't have to fix it when it broke.''

The greatest perversity of all? The ghastliest hovels, like Ruby's
house, often yield the highest returns. They cost nothing to maintain,
because you owe nothing to a tenant who's behind on rent. Evictions are
cheaper than making repairs; the mortgage payments on these places are
minimal; if all goes to hell, you can always stop paying taxes and
surrender the place back to the city. It's win-win. As Sherrena, Ruby's
landlord, likes to say, ``The 'hood is good.''

``Evicted'' is filled with such infuriating paradoxes and demon's loops.
Mr. Desmond explains them one by one, sometimes in the main text and
sometimes in the footnotes, which make for an engrossing reading
experience all on their own. (Think of them as the director's cut.)
They're filled with history, theory and original tenant survey data.
There's even additional dialogue.

But ``Evicted'' is most memorable for its characters, rendered in such
high-resolution detail that their ghost images linger if you shut your
eyes. To respect their privacy, Mr. Desmond has given them pseudonyms,
but their voices are as distinct as fingerprints, their plights
impossible to invent. (If you doubt they're real, read Mr. Desmond's
last chapter, ``About This Project,'' and come to your own conclusions.)

There's Doreen, Ruby's mother, who presides over a three-generation
household of eight. There's Doreen's neighbor, Lamar, a black single
father, who's facing eviction and can't collect disability, even though
he has no legs. There's Arleen, who's evicted or forced out of her house
so many times over the course of the book that I lost count, at one
point calling on 90 landlords before finding another home --- only to be
kicked out a few days later.

The portrait of Sherrena is priceless. She loves her gambling, loves her
red Camaro, loves her annual trips to Jamaica. ``If you ever thinking
about becoming a landlord,'' she earnestly tells Arleen, after taking
her to eviction court, ``don't. It's a bad deal. Get the short end of
the stick every time.''

And then there are Ned and Pam, a pair of crack addicts with five
evictions and a scroll of a rap sheet between them, who still have an
easier time finding housing than any of the black men and women. ``They
were white,'' Mr. Desmond explains.

As these people brokenly shuttle from pillar to post, their lives
inevitably decline. How can you hang on to a job, send your child to
school, or build roots in a community if you are constantly changing
homes, each one more dilapidated and dangerously located than the next?

``Just my soul is messed up,'' Arleen says after her being thrown back
out on the street. Her children can see it. They're the emotional
seismographs in ``Evicted,'' detecting every tremor in their mother's
mood. Her son, Jori, dreams of making things right. He wants to become a
carpenter one day. He wants to build her a house.

Advertisement

\protect\hyperlink{after-bottom}{Continue reading the main story}

\hypertarget{site-index}{%
\subsection{Site Index}\label{site-index}}

\hypertarget{site-information-navigation}{%
\subsection{Site Information
Navigation}\label{site-information-navigation}}

\begin{itemize}
\tightlist
\item
  \href{https://help.nytimes3xbfgragh.onion/hc/en-us/articles/115014792127-Copyright-notice}{©~2020~The
  New York Times Company}
\end{itemize}

\begin{itemize}
\tightlist
\item
  \href{https://www.nytco.com/}{NYTCo}
\item
  \href{https://help.nytimes3xbfgragh.onion/hc/en-us/articles/115015385887-Contact-Us}{Contact
  Us}
\item
  \href{https://www.nytco.com/careers/}{Work with us}
\item
  \href{https://nytmediakit.com/}{Advertise}
\item
  \href{http://www.tbrandstudio.com/}{T Brand Studio}
\item
  \href{https://www.nytimes3xbfgragh.onion/privacy/cookie-policy\#how-do-i-manage-trackers}{Your
  Ad Choices}
\item
  \href{https://www.nytimes3xbfgragh.onion/privacy}{Privacy}
\item
  \href{https://help.nytimes3xbfgragh.onion/hc/en-us/articles/115014893428-Terms-of-service}{Terms
  of Service}
\item
  \href{https://help.nytimes3xbfgragh.onion/hc/en-us/articles/115014893968-Terms-of-sale}{Terms
  of Sale}
\item
  \href{https://spiderbites.nytimes3xbfgragh.onion}{Site Map}
\item
  \href{https://help.nytimes3xbfgragh.onion/hc/en-us}{Help}
\item
  \href{https://www.nytimes3xbfgragh.onion/subscription?campaignId=37WXW}{Subscriptions}
\end{itemize}
