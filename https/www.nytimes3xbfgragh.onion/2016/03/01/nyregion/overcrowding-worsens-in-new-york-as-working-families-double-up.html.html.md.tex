Sections

SEARCH

\protect\hyperlink{site-content}{Skip to
content}\protect\hyperlink{site-index}{Skip to site index}

\href{https://www.nytimes3xbfgragh.onion/section/nyregion}{New York}

\href{https://myaccount.nytimes3xbfgragh.onion/auth/login?response_type=cookie\&client_id=vi}{}

\href{https://www.nytimes3xbfgragh.onion/section/todayspaper}{Today's
Paper}

\href{/section/nyregion}{New York}\textbar{}When the Kitchen Is Also a
Bedroom: Overcrowding Worsens in New York

\url{https://nyti.ms/1LRdW5e}

\begin{itemize}
\item
\item
\item
\item
\item
\end{itemize}

Advertisement

\protect\hyperlink{after-top}{Continue reading the main story}

Supported by

\protect\hyperlink{after-sponsor}{Continue reading the main story}

\hypertarget{when-the-kitchen-is-also-a-bedroom-overcrowding-worsens-in-new-york}{%
\section{When the Kitchen Is Also a Bedroom: Overcrowding Worsens in New
York}\label{when-the-kitchen-is-also-a-bedroom-overcrowding-worsens-in-new-york}}

\includegraphics{https://static01.graylady3jvrrxbe.onion/images/2016/02/27/nyregion/00OVERCROWDING1/00OVERCROWDING1-articleLarge.jpg?quality=75\&auto=webp\&disable=upscale}

By \href{http://www.nytimes3xbfgragh.onion/by/kirk-semple}{Kirk Semple}

\begin{itemize}
\item
  Feb. 29, 2016
\item
  \begin{itemize}
  \item
  \item
  \item
  \item
  \item
  \end{itemize}
\end{itemize}

Rafael's housing situation is an exercise in tolerance and creative
space management. He lives with four other people in an overstuffed
apartment in Jackson Heights, Queens, that measures less than 500 square
feet.

He shares a bedroom with his mother. Two men sublet a second, smaller
bedroom that Rafael created by subdividing the living room with drywall.
His brother sleeps in the kitchen on a mat that he rolls up every
morning and wedges in the corner, opening a path to the front door.

There are belongings occupying just about every square inch of space.
Clean dishes and cookware are piled on the countertop because there is
nowhere else to store them. Clothes are jammed in bags along the kitchen
wall. Shelves overflow with bills, magazines, pill bottles and dog food
(for the resident Chihuahua). Bicycles are stacked on top of a cupboard.

``That's the life here in New York,'' said Rafael, 51, a Mexican
immigrant who works as a waiter and insisted that his last name not be
used to avoid tipping off the landlord to his illegal subletters. ``It's
not easy for anyone.''

Rafael's cramped living arrangement is becoming increasingly common in
many neighborhoods as New York City's housing crisis forces more and
more people to squeeze into crowded apartments.

According to the latest Census Bureau data, about 9 percent of all
households --- or nearly 280,000 units --- in New York City have more
than one person per room,
\href{https://www.huduser.gov/publications/pdf/measuring_overcrowding_in_hsg.pdf}{a
common government measure of crowding}. A decade ago, the rate was 8
percent. The change represents nearly a 13 percent increase. By
comparison, the national crowding rate is 3.4 percent.

The crowding problem in New York worsens considerably in specific
neighborhoods, particularly those with large working-class and immigrant
populations where it is not unusual for two families to cram into
apartments intended for one, and laborers to sleep two, three or more to
a room.

While the Bronx and Brooklyn have the highest percentage of crowded
units among the boroughs, according to Census Bureau data from 2014, the
community district with the highest rate of crowding is in Queens and
encompasses Elmhurst and southern Corona. There, a quarter of all
residential units are considered crowded. A
\href{http://streeteasy.com/blog/doubled-up-crowding-in-nyc/}{study
published in February} by StreetEasy.com found that five of the top 10
most crowded neighborhoods in the city were in central Queens.

``Given the affordable housing crisis in our city, where do we go?''
said Aniqa Nawabi, the resource development manager at
\href{http://chhayacdc.org/}{Chhaya}, a community development group in
Jackson Heights that works closely with the South Asian immigrant
population in Queens. ``The prices keep going up and up and up.''

Residents forced to endure crowded living conditions in order to make
ends meet often do so at great risk to personal health and safety. Reams
of research have linked overcrowding with poor health and the spread of
infectious diseases, including respiratory infections and asthma as well
as anxiety and other mental health problems.

Crowded conditions have also been shown to hurt cognitive and behavioral
development in children and impinge on studying and sleep, leading to
problems that endure throughout life.
\href{http://www.ncbi.nlm.nih.gov/pmc/articles/PMC3805127/}{A 2012
study} in the journal Social Science Research called the effects of
crowding on children ``large and pervasive.''

``Children raised in crowded homes may take their educational,
behavioral and physical health disadvantages with them throughout their
lives,'' the study said.

\includegraphics{https://static01.graylady3jvrrxbe.onion/images/2016/02/27/nyregion/00OVERCROWDING2/00OVERCROWDING2-articleLarge.jpg?quality=75\&auto=webp\&disable=upscale}

Many crowded dwellings in New York City, like Rafael's, also include
illegal room partitions, which impede the movement of firefighters and
increase those units' vulnerability to fire.

Studies and surveys have also established overcrowding as a major
trigger of homelessness.

The city's comptroller, Scott M. Stringer, recently sought to draw
attention to this dimension of the city's housing challenges in
\href{http://comptroller.nyc.gov/wp-content/uploads/documents/Hidden_Households.pdf}{a
report called ``Hidden Households.''}

``The problem of crowding is stubbornly increasing,'' he declared in a
statement accompanying the report, published in October. ``And while we
could all use a little more room to breathe, we must give special
attention to those who are most at risk for the negative effects of
crowding, including bad health, diminished economic opportunity and
increased risk of homelessness.''

The study found that crowding was concentrated in larger, older
buildings and in rental units rather than in owner-occupied units. More
than half of all working-age occupants in crowded dwellings were
unemployed; about 70 percent of crowded units had immigrant heads of
household, a majority Latino; and nearly 40 percent of all residents in
crowded units were under 18 years old.

The study also found that the percentage of units with at least 1.5
people per room --- a measure of ``severe crowding'' used by the city's
Housing Preservation and Development Department --- increased nearly 45
percent between 2005 and 2013, from 2.3 percent to more than 3.3
percent.

The overcrowded dwelling typically finds its most severe expression in
illegally converted basement and cellar apartments, most of them
overcrowded and riddled with safety hazards such as shoddy construction,
dangerous wiring and improper means of exit in the case of a fire.

In recent years, several of these units have emerged into public view
when occupants died in fires and landlords were criminally prosecuted.

According to
\href{http://prattcenter.net/sites/default/files/housing_underground_0.pdf}{a
2008 report} by Chhaya and the Pratt Center for Community Development,
Queens has by far the most illegal units of any borough, about 48,000.

The two groups are part of a coalition that has been lobbying the city
to change the building and zoning laws to permit the legalization of
some of those cellar and basement units. They want to create a pilot
program that would allow certain units to be brought up to code and be
permitted as formal housing.

The coalition estimated that informal units, most of them in cellars and
basements, accounted for nearly 40 percent of all new housing created in
the city between 1990 and 2005 and now number more than 100,000
throughout the city.

León, a Colombian immigrant, lives in a 300-square-foot, two-bedroom
basement apartment in Astoria, Queens, with five other people: his wife,
two daughters, a son-in-law and a granddaughter. They pay a monthly rent
of \$1,200.

``It's a bad situation,'' he said, insisting that his last name be
withheld from publication because, like Rafael, his landlord was unaware
that so many people were living in such close quarters. ``We need
dignity, man. We need privacy. We need a more feasible life, a life in
which it's easier to be human.''

But León was trying to look on the bright side of a grim situation.

``I'm happy,'' he said, ``because I have my family next to me.''

Advertisement

\protect\hyperlink{after-bottom}{Continue reading the main story}

\hypertarget{site-index}{%
\subsection{Site Index}\label{site-index}}

\hypertarget{site-information-navigation}{%
\subsection{Site Information
Navigation}\label{site-information-navigation}}

\begin{itemize}
\tightlist
\item
  \href{https://help.nytimes3xbfgragh.onion/hc/en-us/articles/115014792127-Copyright-notice}{©~2020~The
  New York Times Company}
\end{itemize}

\begin{itemize}
\tightlist
\item
  \href{https://www.nytco.com/}{NYTCo}
\item
  \href{https://help.nytimes3xbfgragh.onion/hc/en-us/articles/115015385887-Contact-Us}{Contact
  Us}
\item
  \href{https://www.nytco.com/careers/}{Work with us}
\item
  \href{https://nytmediakit.com/}{Advertise}
\item
  \href{http://www.tbrandstudio.com/}{T Brand Studio}
\item
  \href{https://www.nytimes3xbfgragh.onion/privacy/cookie-policy\#how-do-i-manage-trackers}{Your
  Ad Choices}
\item
  \href{https://www.nytimes3xbfgragh.onion/privacy}{Privacy}
\item
  \href{https://help.nytimes3xbfgragh.onion/hc/en-us/articles/115014893428-Terms-of-service}{Terms
  of Service}
\item
  \href{https://help.nytimes3xbfgragh.onion/hc/en-us/articles/115014893968-Terms-of-sale}{Terms
  of Sale}
\item
  \href{https://spiderbites.nytimes3xbfgragh.onion}{Site Map}
\item
  \href{https://help.nytimes3xbfgragh.onion/hc/en-us}{Help}
\item
  \href{https://www.nytimes3xbfgragh.onion/subscription?campaignId=37WXW}{Subscriptions}
\end{itemize}
