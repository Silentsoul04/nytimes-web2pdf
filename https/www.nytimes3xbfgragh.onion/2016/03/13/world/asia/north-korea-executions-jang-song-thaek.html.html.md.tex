Sections

SEARCH

\protect\hyperlink{site-content}{Skip to
content}\protect\hyperlink{site-index}{Skip to site index}

\href{https://www.nytimes3xbfgragh.onion/section/world/asia}{Asia
Pacific}

\href{https://myaccount.nytimes3xbfgragh.onion/auth/login?response_type=cookie\&client_id=vi}{}

\href{https://www.nytimes3xbfgragh.onion/section/todayspaper}{Today's
Paper}

\href{/section/world/asia}{Asia Pacific}\textbar{}In Hail of Bullets and
Fire, North Korea Killed Official Who Wanted Reform

\url{https://nyti.ms/22bHK5Z}

\begin{itemize}
\item
\item
\item
\item
\item
\end{itemize}

Advertisement

\protect\hyperlink{after-top}{Continue reading the main story}

Supported by

\protect\hyperlink{after-sponsor}{Continue reading the main story}

\hypertarget{in-hail-of-bullets-and-fire-north-korea-killed-official-who-wanted-reform}{%
\section{In Hail of Bullets and Fire, North Korea Killed Official Who
Wanted
Reform}\label{in-hail-of-bullets-and-fire-north-korea-killed-official-who-wanted-reform}}

\includegraphics{https://static01.graylady3jvrrxbe.onion/images/2016/03/08/world/uncle-web1/uncle-web1-articleLarge.jpg?quality=75\&auto=webp\&disable=upscale}

By \href{http://www.nytimes3xbfgragh.onion/by/choe-sang-hun}{Choe
Sang-Hun}

\begin{itemize}
\item
  March 12, 2016
\item
  \begin{itemize}
  \item
  \item
  \item
  \item
  \item
  \end{itemize}
\end{itemize}

SEOUL, South Korea --- In late 2013, Jang Song-thaek, an uncle of
\href{http://topics.nytimes3xbfgragh.onion/top/reference/timestopics/people/k/kim_jongun/index.html?inline=nyt-per}{Kim
Jong-un}, the North Korean leader, was taken to the Gang Gun Military
Academy in a Pyongyang suburb.

Hundreds of officials were gathered there to witness the execution of
Mr. Jang's two trusted deputies in the administrative department of the
ruling Workers' Party.

The two men, Ri Ryong-ha and Jang Su-gil, were torn apart by
antiaircraft machine guns, according to South Korea's National
Intelligence Service. The executioners then incinerated their bodies
with flamethrowers.

Jang Song-thaek, widely considered the second-most powerful figure in
the North, fainted during the ordeal, according to a new book published
in South Korea that offers a rare glimpse into the secretive Pyongyang
regime.

``Son-in-Law of a Theocracy,'' by Ra Jong-yil, a former deputy director
of the National Intelligence Service, is a rich biography of Mr. Jang,
the most prominent victim of the purges his young nephew has conducted
since assuming power in 2011.

Mr. Jang was convicted of treason in 2013. He was
\href{http://www.nytimes3xbfgragh.onion/2013/12/13/world/asia/north-korea-says-uncle-of-executed.html?_r=0}{executed}
at the same place and in the same way as his deputies, the South Korean
intelligence agency said.

The book asserts that although he was a fixture of the North Korean
political elite for decades, he dreamed of reforming his country. ``With
his execution,
\href{http://topics.nytimes3xbfgragh.onion/top/news/international/countriesandterritories/northkorea/index.html?inline=nyt-geo}{North
Korea} lost virtually the only person there who could have helped the
country introduce reform and openness,'' Mr. Ra said during a recent
interview.

Mr. Ra, who is also a professor of political science and a former South
Korean ambassador to Japan and Britain, mined existing publications but
also interviewed sources in South Korea, Japan and China, including
high-ranking defectors from the North who spoke on the condition of
anonymity.

Mr. Jang met one of the daughters of North Korea's founder, Kim Il-sung,
while both attended Kim Il-sung University in the mid-1960s. The
daughter, Kim Kyong-hee, developed a crush on Mr. Jang, who was tall and
humorous --- and sang and played the accordion.

\includegraphics{https://static01.graylady3jvrrxbe.onion/images/2016/03/08/world/uncle-web2/uncle-web2-articleLarge.jpg?quality=75\&auto=webp\&disable=upscale}

Her father transferred the young man to a provincial college to keep the
two apart. But Ms. Kim hopped in her Soviet Volga sedan to see Mr. Jang
each weekend.

Once they married in 1972, Mr. Jang's career took off under the
patronage of Kim Jong-il, his brother-in-law and the designated
successor of the regime.

In his memoir, a Japanese sushi chef for Kim Jong-il from 1988 to 2001
who goes by the alias
\href{http://www.nytimes3xbfgragh.onion/2012/08/25/world/asia/kim-family-chefs-redemption-suggests-a-softening-north-korea.html}{Kenji
Fujimoto} remembered Mr. Jang as a fun-loving prankster who was a
regular at banquets that could last until morning or even stretch a few
days. A key feature of the events was a ``pleasure squad'' of young,
attractive women who would dance the cancan, sing American country songs
or perform a striptease, according to the book and accounts by
defectors.

Mr. Jang also mobilized North Korean diplomats abroad to import Danish
dairy products, Black Sea caviar, French cognac and Japanese electronics
--- gifts Mr. Kim handed out during his parties to keep his elites
loyal.

But North Korean diplomats who have defected to South Korea also said
that during his frequent trips overseas to shop for Mr. Kim, Mr. Jang
would drink heavily and speak dejectedly about people dying of hunger
back home.

Few benefited more than Mr. Jang from the regime he loyally served. But
he was never fully embraced by the Kim family because he was not blood
kin. This ``liminal existence'' enabled him to see the absurdities of
the regime more clearly than any other figure within it, Mr. Ra wrote.

Mr. Ra said Hwang Jang-yop, a North Korean party secretary who defected
to Seoul in 1997 and lived here until his death in 2010, shared a
conversation he once had with Mr. Jang. When told that the North's
economy was cratering, Mr. Jang responded sarcastically: ``How can an
economy already at the bottom go further down?''

Mr. Jang's frequent partying with the ``pleasure squad'' strained his
marriage. Senior defectors from the North said it was an open secret
among the Pyongyang elite that the couple both had extramarital affairs.

Their only child, Jang Kum-song, killed herself in Paris in 2006. She
overdosed on sleeping pills after the Pyongyang government caught wind
of her dating a Frenchman and summoned her home.

Still, the marriage endured. When Kim Jong-il banished Mr. Jang three
times for overstepping his authority, his wife intervened on his behalf.

After Mr. Kim suffered a stroke in 2008 and
\href{http://www.nytimes3xbfgragh.onion/2011/12/19/world/asia/Kim-Jong-il-Dictator-Who-Turned-North-Korea-Into-a-Nuclear-State-Dies.html?pagewanted=all}{died
in 2011}, Mr. Jang helped his young nephew, Kim Jong-un, establish
himself as successor. At the same time, he vastly expanded his own
influence --- and ambition.

He wrested the lucrative right of exporting coal to China from the
military and gave it to his administrative department. He purged his
rivals, including
\href{http://www.nytimes3xbfgragh.onion/2012/07/16/world/asia/north-korea-removes-army-chief.html}{Ri
Yong-ho}, the chief of the military's general staff, and U Dong-chuk, a
deputy director at the Ministry of State Security, the North's secret
police.

Mr. Jang's campaign for more influence was apparently aimed at pushing
for the kind of economic overhaul that China has introduced, Mr. Ra
wrote. But he underestimated how unpalatable the idea was to Kim
Jong-un, whose totalitarian rule would be undermined by such reform.

Mr. Ra said it was impossible to establish the exact sequence of events
that led to Mr. Jang's downfall. But it was clear his hubris played a
role. At the height of his power, photographs in the North Korean media
showed Mr. Jang leaning on an armrest, looking almost bored, while his
nephew spoke.

Announcing his execution, North Korea said Mr. Jang, ``human scum worse
than a dog,'' had betrayed the Kim family by plotting to overthrow the
younger Mr. Kim, using economic collapse as a pretext, and to rule the
country himself as premier and ``reformer.''

He was accused of planting his followers in key posts and profiteering
from minerals exports. His indictment pointedly noted that Mr. Jang had
stood up and clapped only ``halfheartedly'' when Mr. Kim was being
upheld as supreme leader.

In 2013, Mr. Kim, after hearing complaints about Mr. Jang's expansion of
power, ordered his department to relinquish the management of a fishing
farm and a condensed milk factory. But officials loyal to their
``Comrade No. 1,'' Mr. Jang, blocked those who arrived to carry out Mr.
Kim's orders from entering their premises.

It was probably the last straw for Mr. Kim, still unsure about himself
and extremely sensitive about any challenge to his supposedly monolithic
leadership. Meanwhile, Mr. Jang's enemies in the secret police were
eager to go after him.

``There was no indication that he had a lawyer or was allowed to speak
for himself during his trial,'' Mr. Ra said. ``It was not a trial but a
murder.''

Mr. Jang's name has been expurgated from all official records in the
North. Hundreds of his associates were purged. His wife is alive but
sickly, according to the South Korean intelligence agency.

But some people in Pyongyang still remember his role in the tall
apartment buildings, water parks and other showpiece projects he once
zealously promoted to glorify his nephew's nascent leadership.

Advertisement

\protect\hyperlink{after-bottom}{Continue reading the main story}

\hypertarget{site-index}{%
\subsection{Site Index}\label{site-index}}

\hypertarget{site-information-navigation}{%
\subsection{Site Information
Navigation}\label{site-information-navigation}}

\begin{itemize}
\tightlist
\item
  \href{https://help.nytimes3xbfgragh.onion/hc/en-us/articles/115014792127-Copyright-notice}{©~2020~The
  New York Times Company}
\end{itemize}

\begin{itemize}
\tightlist
\item
  \href{https://www.nytco.com/}{NYTCo}
\item
  \href{https://help.nytimes3xbfgragh.onion/hc/en-us/articles/115015385887-Contact-Us}{Contact
  Us}
\item
  \href{https://www.nytco.com/careers/}{Work with us}
\item
  \href{https://nytmediakit.com/}{Advertise}
\item
  \href{http://www.tbrandstudio.com/}{T Brand Studio}
\item
  \href{https://www.nytimes3xbfgragh.onion/privacy/cookie-policy\#how-do-i-manage-trackers}{Your
  Ad Choices}
\item
  \href{https://www.nytimes3xbfgragh.onion/privacy}{Privacy}
\item
  \href{https://help.nytimes3xbfgragh.onion/hc/en-us/articles/115014893428-Terms-of-service}{Terms
  of Service}
\item
  \href{https://help.nytimes3xbfgragh.onion/hc/en-us/articles/115014893968-Terms-of-sale}{Terms
  of Sale}
\item
  \href{https://spiderbites.nytimes3xbfgragh.onion}{Site Map}
\item
  \href{https://help.nytimes3xbfgragh.onion/hc/en-us}{Help}
\item
  \href{https://www.nytimes3xbfgragh.onion/subscription?campaignId=37WXW}{Subscriptions}
\end{itemize}
