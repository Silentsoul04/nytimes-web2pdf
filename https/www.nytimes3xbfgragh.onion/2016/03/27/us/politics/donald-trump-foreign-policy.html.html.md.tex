Sections

SEARCH

\protect\hyperlink{site-content}{Skip to
content}\protect\hyperlink{site-index}{Skip to site index}

\href{https://www.nytimes3xbfgragh.onion/section/politics}{Politics}

\href{https://myaccount.nytimes3xbfgragh.onion/auth/login?response_type=cookie\&client_id=vi}{}

\href{https://www.nytimes3xbfgragh.onion/section/todayspaper}{Today's
Paper}

\href{/section/politics}{Politics}\textbar{}In Donald Trump's Worldview,
America Comes First, and Everybody Else Pays

\url{https://nyti.ms/22OcUR1}

\begin{itemize}
\item
\item
\item
\item
\item
\item
\end{itemize}

Advertisement

\protect\hyperlink{after-top}{Continue reading the main story}

Supported by

\protect\hyperlink{after-sponsor}{Continue reading the main story}

\hypertarget{in-donald-trumps-worldview-america-comes-first-and-everybody-else-pays}{%
\section{In Donald Trump's Worldview, America Comes First, and Everybody
Else
Pays}\label{in-donald-trumps-worldview-america-comes-first-and-everybody-else-pays}}

\includegraphics{https://static01.graylady3jvrrxbe.onion/images/2016/03/27/us/27trumpforeign-JP-01/27trumpforeign-JP-01-articleLarge.jpg?quality=75\&auto=webp\&disable=upscale}

By \href{http://www.nytimes3xbfgragh.onion/by/david-e-sanger}{David E.
Sanger} and
\href{http://www.nytimes3xbfgragh.onion/by/maggie-haberman}{Maggie
Haberman}

\begin{itemize}
\item
  March 26, 2016
\item
  \begin{itemize}
  \item
  \item
  \item
  \item
  \item
  \item
  \end{itemize}
\end{itemize}

Donald J. Trump, the Republican presidential front-runner, said that if
elected, he might halt purchases of oil from Saudi Arabia and other Arab
allies unless they commit ground troops to the fight against the Islamic
State or ``substantially reimburse'' the United States for combating the
militant group, which threatens their stability.

``If Saudi Arabia was without the cloak of American protection,'' Mr.
Trump said during a 100-minute interview on foreign policy, spread over
two phone calls on Friday, ``I don't think it would be around.''

He also said he would be open to allowing Japan and South Korea to build
their own nuclear arsenals rather than depend on the American nuclear
umbrella for their protection against North Korea and China. If the
United States ``keeps on its path, its current path of weakness, they're
going to want to have that anyway, with or without me discussing it,''
Mr. Trump said.

And he said he would be willing to withdraw United States forces from
both Japan and South Korea if they did not substantially increase their
contributions to the costs of housing and feeding those troops. ``Not
happily, but the answer is yes,'' he said.

Mr. Trump also said he would seek to renegotiate many fundamental
treaties with American allies, possibly including a 56-year-old security
pact with Japan, which he described as one-sided.

In Mr. Trump's worldview, the United States has become a diluted power,
and the main mechanism by which he would re-establish its central role
in the world is economic bargaining. He approached almost every current
international conflict through the prism of a negotiation, even when he
was imprecise about the strategic goals he sought. He again faulted the
Obama administration's handling of the negotiations with Iran last year
--- ``It would have been so much better if they had walked away a few
times,'' he said --- but offered only one new idea about how he would
change its content: Ban Iran's trade with North Korea.

Mr. Trump struck similar themes when he discussed the future of NATO,
which he called ``unfair, economically, to us,'' and said he was open to
an alternative organization focused on counterterrorism. He argued that
the best way to halt China's placement of military airfields and
antiaircraft batteries
\href{http://www.nytimes3xbfgragh.onion/interactive/2015/07/30/world/asia/what-china-has-been-building-in-the-south-china-sea-2016.html}{on
reclaimed islands} in the South China Sea was to threaten its access to
American markets.

``We have tremendous economic power over China,'' he argued. ``And
that's the power of trade.'' He did not mention Beijing's ability for
economic retaliation.

Mr. Trump's views, as he explained them, fit nowhere into the recent
history of the Republican Party: He is not in the internationalist camp
of President George Bush, nor does he favor President George W. Bush's
call to make it the United States' mission to spread democracy around
the world. He agreed with a suggestion that his ideas might be summed up
as ``America First.''

``Not isolationist, but I am America First,'' he said. ``I like the
expression.'' He said he was willing to reconsider traditional American
alliances if partners were not willing to pay, in cash or troop
commitments, for the presence of American forces around the world. ``We
will not be ripped off anymore,'' he said.

In the past week, the
\href{http://www.nytimes3xbfgragh.onion/2016/03/23/world/europe/brussels-airport-explosions.html}{bombings
in Brussels} and an accelerated war
\href{http://www.nytimes3xbfgragh.onion/2016/03/26/world/middleeast/abd-al-rahman-mustafa-al-qaduli-isis-reported-killed-in-syria.html}{against
the Islamic State} have shifted the focus of the campaign trail
conversation back to questions of how the candidates would defend the
United States and what kind of diplomacy they would pursue around the
world.

Mr. Trump explained his thoughts in concrete and easily digestible
terms, but they appeared to reflect little consideration for potential
consequences. Much the same way he treats political rivals and
interviewers, he personalized how he would engage foreign nations,
suggesting his approach would depend partly on ``how friendly they've
been toward us,'' not just on national interests or alliances.

At no point did he express any belief that American forces deployed on
military bases around the world were by themselves valuable to the
United States, though Republican and Democratic administrations have for
decades argued that they are essential to deterring military
adventurism, protecting commerce and gathering intelligence.

Like Richard M. Nixon, Mr. Trump emphasized the importance of
``unpredictability'' for an American president, arguing that the
country's traditions of democracy and openness had made its actions too
easy for adversaries and allies alike to foresee.

``I wouldn't want them to know what my real thinking is,'' he said of
how far he was willing to take the confrontation over the islands in the
South China Sea, which are remote and lightly inhabited but
\href{http://www.nytimes3xbfgragh.onion/2016/03/09/world/asia/south-china-sea-militarization.html}{extend
China's control} over a major maritime thoroughfare. But, he added, ``I
would use trade, absolutely, as a bargaining chip.''

Asked when he thought American power had been at its peak, Mr. Trump
reached back 116 years to the turn of the 20th century, the era of
another unconventional Republican, Theodore Roosevelt, who ended up
leaving the party. His favorite figures in American history, he said,
include two generals, Douglas MacArthur and George S. Patton --- though
he said that, unlike MacArthur, he would not advocate using
\href{http://topics.nytimes3xbfgragh.onion/top/news/science/topics/atomic_weapons/index.html?inline=nyt-classifier}{nuclear
weapons} except as a last resort. (He suggested MacArthur had pressed
during the Korean War to use them against China as a means ``to
negotiate,'' adding, ``He played the nuclear card, but he didn't use
it.'')

Mr. Trump denied that he had had trouble finding top members of the
foreign policy establishment to advise him. ``Many of them are tied up
with contracts working for various networks,'' he said, like Fox or CNN.

He named three advisers in addition to
\href{http://www.nytimes3xbfgragh.onion/2016/03/23/us/politics/donald-trump-foreign-policy-advisers.html}{five
he announced} earlier in the week: retired Maj. Gen. Gary L. Harrell,
Maj. Gen. Bert K. Mizusawa and retired Rear Adm. Charles R. Kubic. They
reflected a continuing bias toward former military officers, rather than
diplomats or academics with foreign policy experience. General Harrell,
a Special Forces veteran, was a commander in the failed ``Black Hawk
Down'' mission in Somalia in 1993. Admiral Kubic, now president of an
engineering firm, has been a sharp critic of President Obama's handling
of the attack on Libya that helped
\href{http://www.nytimes3xbfgragh.onion/2011/10/21/world/africa/qaddafi-killed-as-hometown-falls-to-libyan-rebels.html}{oust
Col. Muammar el-Qaddafi}.

Asked about the briefings he receives and books he has read on foreign
policy, Mr. Trump said his main information source was newspapers,
``including yours.''

Until recently, his foreign policy pronouncements have largely come
through slogans: ``Take the oil,'' ``Build a wall'' and
\href{http://www.nytimes3xbfgragh.onion/politics/first-draft/2015/12/07/donald-trump-calls-for-banning-muslims-from-entering-u-s/}{ban
Muslim immigrants} and visitors, at least temporarily. But as he pulls
closer to the nomination, he has been called on to elaborate.

Pressed about his call to ``take the oil'' controlled by the Islamic
State in the Middle East, Mr. Trump acknowledged that this would require
deploying ground troops, something he does not favor. ``We should've
taken it, and we would've had it,'' he said, referring to the years in
which the United States occupied Iraq. ``Now we have to destroy the
oil.''

He did not rule out spying on American allies, including leaders like
Angela Merkel, the German chancellor, whose cellphone
\href{http://www.nytimes3xbfgragh.onion/2013/10/25/world/europe/allegation-of-us-spying-on-merkel-puts-obama-at-crossroads.html}{was
apparently a target} of the National Security Agency. Mr. Obama said the
agency would no longer target her phone but made no such commitments
about the rest of Germany, or Europe.

``I'm not sure that I would want to be talking about that,'' Mr. Trump
said. ``You understand what I mean by that.''

Mr. Trump was not impressed with Ms. Merkel's handling of the migrant
crisis, however: ``Germany is being destroyed by Merkel's naïveté, or
worse,'' he said. He suggested that Germany and the Gulf nations should
pay for the ``safe zones'' he wants to set up in Syria for refugees, and
for protecting them once built.

Throughout the two conversations, Mr. Trump painted a bleak picture of
the United States as a diminished force in the world, an opinion he has
held since the late 1980s, when he placed ads in The New York Times and
other newspapers calling for Japan and Saudi Arabia to spend more money
on their own defense.

Mr. Trump's new threat to cut off oil purchases from the Saudis was part
of a broader complaint about the United States' Arab allies, which many
in the Obama administration share: that they often look to the United
States to police the Middle East, without putting their own troops at
risk. ``We defend everybody,'' he said. ``When in doubt, come to the
United States. We'll defend you. In some cases free of charge.''

But his rationale for abandoning the region was that ``the reason we're
in the Middle East is for oil, and all of a sudden we're finding out
that there's less reason to be there now.'' He made no mention of the
risks of withdrawal --- that it would encourage Iran to dominate the
Gulf, that the presence of American troops is part of Israel's defense,
and that American air and naval bases in the region are key collection
points for intelligence and bases for drones and Special Operations
forces.

Mr. Trump seemed less comfortable on some topics than others. He called
the United States ``obsolete'' in terms of cyberweaponry, although the
nation's capabilities are generally considered on the cutting edge.

In the morning interview, asked if he would seek a two-state or a
one-state solution in a peace accord between the Israelis and the
\href{http://topics.nytimes3xbfgragh.onion/top/reference/timestopics/subjects/p/palestinians/index.html?inline=nyt-classifier}{Palestinians},
he said: ``I'm not saying anything. What I'm going to do is, you know, I
specifically don't want to address the issue because I would love to see
if a deal could be made.''

But in the evening, saying he had been rushed earlier, he went back to a
position
\href{http://www.nytimes3xbfgragh.onion/politics/first-draft/2016/03/21/donald-trump-calls-himself-lifelong-supporter-of-israel/}{outlined
Monday} to the American Israel Public Affairs Committee, the pro-Israel
group. ``Basically, I support a two-state solution on Israel,'' he said.
``But the Palestinian Authority has to recognize Israel's right to exist
as a Jewish state.''

In discussing nuclear weapons --- which he said he had learned about
from an uncle, John G. Trump, who was on the M.I.T. faculty --- he
seemed fixated on the large stockpiles amassed in the Cold War. While he
referred briefly to North Korean and Pakistani arsenals, he said nothing
about a danger that is a cause of great consternation among global
leaders: small nuclear weapons that could be fashioned by terrorists.

In criticizing the
\href{http://www.nytimes3xbfgragh.onion/2015/07/15/world/middleeast/iran-nuclear-deal-is-reached-after-long-negotiations.html}{Iran
nuclear deal}, he expressed particular outrage at how the roughly \$150
billion released to Iran (by his estimate; the number is in dispute) was
being spent. ``Did you notice they're buying from everybody but the
United States?'' he said.

Told that sanctions under United States law still bar most American
companies from doing business with Iran, he said: ``So, how stupid is
that? We give them the money and we now say, `Go buy Airbus instead of
Boeing,' right?''

But Mr. Trump, who has been pushed to demonstrate a basic command of
international affairs, insisted that voters should not doubt his foreign
policy fluency. ``I do know my subject,'' he said.

Advertisement

\protect\hyperlink{after-bottom}{Continue reading the main story}

\hypertarget{site-index}{%
\subsection{Site Index}\label{site-index}}

\hypertarget{site-information-navigation}{%
\subsection{Site Information
Navigation}\label{site-information-navigation}}

\begin{itemize}
\tightlist
\item
  \href{https://help.nytimes3xbfgragh.onion/hc/en-us/articles/115014792127-Copyright-notice}{©~2020~The
  New York Times Company}
\end{itemize}

\begin{itemize}
\tightlist
\item
  \href{https://www.nytco.com/}{NYTCo}
\item
  \href{https://help.nytimes3xbfgragh.onion/hc/en-us/articles/115015385887-Contact-Us}{Contact
  Us}
\item
  \href{https://www.nytco.com/careers/}{Work with us}
\item
  \href{https://nytmediakit.com/}{Advertise}
\item
  \href{http://www.tbrandstudio.com/}{T Brand Studio}
\item
  \href{https://www.nytimes3xbfgragh.onion/privacy/cookie-policy\#how-do-i-manage-trackers}{Your
  Ad Choices}
\item
  \href{https://www.nytimes3xbfgragh.onion/privacy}{Privacy}
\item
  \href{https://help.nytimes3xbfgragh.onion/hc/en-us/articles/115014893428-Terms-of-service}{Terms
  of Service}
\item
  \href{https://help.nytimes3xbfgragh.onion/hc/en-us/articles/115014893968-Terms-of-sale}{Terms
  of Sale}
\item
  \href{https://spiderbites.nytimes3xbfgragh.onion}{Site Map}
\item
  \href{https://help.nytimes3xbfgragh.onion/hc/en-us}{Help}
\item
  \href{https://www.nytimes3xbfgragh.onion/subscription?campaignId=37WXW}{Subscriptions}
\end{itemize}
