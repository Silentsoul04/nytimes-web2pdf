Sections

SEARCH

\protect\hyperlink{site-content}{Skip to
content}\protect\hyperlink{site-index}{Skip to site index}

\href{https://www.nytimes3xbfgragh.onion/section/food}{Food}

\href{https://myaccount.nytimes3xbfgragh.onion/auth/login?response_type=cookie\&client_id=vi}{}

\href{https://www.nytimes3xbfgragh.onion/section/todayspaper}{Today's
Paper}

\href{/section/food}{Food}\textbar{}At Lilia in Brooklyn, Missy Robbins
Is Cooking Pasta Again

\url{https://nyti.ms/22XZMsG}

\begin{itemize}
\item
\item
\item
\item
\item
\item
\end{itemize}

Advertisement

\protect\hyperlink{after-top}{Continue reading the main story}

Supported by

\protect\hyperlink{after-sponsor}{Continue reading the main story}

\href{/column/restaurant-review}{Restaurant Review}

\hypertarget{at-lilia-in-brooklyn-missy-robbins-is-cooking-pasta-again}{%
\section{At Lilia in Brooklyn, Missy Robbins Is Cooking Pasta
Again}\label{at-lilia-in-brooklyn-missy-robbins-is-cooking-pasta-again}}

\href{https://www.nytimes3xbfgragh.onion/slideshow/2016/03/30/dining/lilia-restaurant-brooklyn.html}{}

\hypertarget{lilia}{%
\subsection{Lilia}\label{lilia}}

12 Photos

View Slide Show ›

\includegraphics{https://static01.graylady3jvrrxbe.onion/images/2016/03/30/dining/30REST-LILIA-slide-31ZE/30REST-LILIA-slide-31ZE-articleLarge.jpg?quality=75\&auto=webp\&disable=upscale}

Devin Yalkin for The New York Times

\begin{itemize}
\tightlist
\item
  Lilia\\
  ★★★ Italian \$\$\$ 567 Union Avenue 718-576-3095
\end{itemize}

\href{https://resy.com/cities/ny/lilia?utm_source=nyt\&utm_medium=restoprofile\&utm_campaign=affiliates\&aff_id=c1fe784}{Reserve
a Table}

When you make a reservation at an independently reviewed restaurant
through our site, we earn an affiliate commission.

By \href{http://www.nytimes3xbfgragh.onion/by/pete-wells}{Pete Wells}

\begin{itemize}
\item
  March 29, 2016
\item
  \begin{itemize}
  \item
  \item
  \item
  \item
  \item
  \item
  \end{itemize}
\end{itemize}

My one-sentence review of Lilia for the too-long-didn't-read crowd:
Missy Robbins is cooking pasta again.

That is all you need to know before you begin plotting a night at this
casual Brooklyn restaurant with Manhattan-level underpinnings.
\href{http://www.lilianewyork.com/menu/}{Lilia's menu} has many other
very good things, but pasta made by Ms. Robbins is a direct route to
happiness that has been shut off to New Yorkers since she left the two
\href{http://www.nytimes3xbfgragh.onion/2009/11/25/dining/reviews/25rest.html?pagewanted=all}{A
Voce}restaurants in 2013.

Slip a fork into the pappardelle with veal Bolognese. Shiny with just
enough herb-flecked sauce that one noodle peels away from the rest as
you lift, they are rolled so thin that they're almost weightless. Taste
them, and you notice their delicacy along with the naked simplicity of
the chopped veal gently cooked into tenderness with dark and meaty dried
porcini. There is no milk in this Bolognese and no tomatoes apart from
some juice, but nothing is missing.

Stab an agnolotto, bring a parcel of sheep's milk ricotta and feta in
gorgeously supple yellow dough to your mouth. It's been rolled in melted
butter with honey and saffron threads. The perfume is subtle but
insistent, and when it threatens to be too much, you bite into a soft
dried tomato. Why did everybody turn against dried tomatoes? At the
right time, they're wonderful, and this is one of those times.

The ricotta gnocchi are tender cheese dumplings that seem to be held
together by nothing but the force of Ms. Robbins's will. Their sauce is
a crunchy pesto of broccoli and pistachios, spring-green and exciting.

The ruffled edges of long belts called malfadini carry melted butter and
Parmigiano-Reggiano with the peppermint tickle of pink peppercorns. Some
sources call this shape ``mafaldine'' and say it is named after a
\href{http://jenniwiltz.com/the-princess-in-the-concentration-camp/}{Princess
Mafalda}. Our server said the word meant ``badly cut,'' but this can't
be right. There is no badly made pasta at Lilia.

\includegraphics{https://static01.graylady3jvrrxbe.onion/images/2016/03/30/dining/30RESTAURANT/30RESTAURANT-articleLarge.jpg?quality=75\&auto=webp\&disable=upscale}

Like
\href{http://www.nytimes3xbfgragh.onion/2008/01/09/dining/reviews/09rest.html}{Barbuto}
in the West Village, Lilia inhabits the shell of an old garage. This
one, in Williamsburg, is full of bleached wood and white furniture that
seem chosen to reflect the sun coming through the windows and skylights
in the early evening.

And as with Barbuto, the food is Italian, confident and coursing with
energy. The intent of every dish seems to be making you glad you're
there. Ms. Robbins gets across her points using only a few ingredients
that haven't been contorted out of shape. She has the assurance of a
chef who has been at it long enough that Barack Obama was a fan of her
cooking at \href{http://www.spiaggiarestaurant.com/}{Spiaggia} in
Chicago before he was the president.

Now proprietor of her own shop, she stands each night at a bend in the
counter separating Lilia's dining room and kitchen. To her right is a
wood-burning grill that runs as hot as a forge. The cooks stationed in
front of it look as if they are baking in their jackets, like potatoes.
They gulp water from plastic quarts, enduring the heat while out at the
tables, eyes go wide over the flavors coming off the flames.

One night, there was a quietly smoky chicken leg under spicy olives
mashed with capers and brightened with mint leaves. There were thin
stems of broccoli rabe, too, tasting unusually sweet against the olives.
You could eat this once a week, and it would not stop tasting new.

Another night brought a flank steak of veal wearing fresh herbs and
olive oil. It had been grilled to the soft pink of medium-rare pork, but
its flavor was more precise and nuanced. I wish more restaurants would
find veal of this quality and cook it so simply and well.

But meat is a minor motif. Grilled seafood is a major one, second in
importance only to pasta. The menu separates it into ``little fish'' of
appetizer size and ``big fish,'' which aren't all that big; they're
normal main courses.

One of these, grilled black bass with exceptional little roasted
potatoes, was spread with a really energetic salsa verde that got extra
lift from tarragon. Raisin- and sage-fueled Marsala (remember Marsala?)
had the right amount of sweetness and zip for a juicy white hunk of
swordfish.

Under little fish, my favorite turned out to be sardines given a
semisweet and citrusy cure, then laid over grilled crostini spread with
soft butter. Sardines and butter are a great combination, and here they
got along even better than usual.

Some grilled scallops got a bit lost in an excess of yogurt-walnut dip.
But tender grilled baby squid held their own on top of whole sweet
tomatoes, cured and marinated, and so did tiny grilled clams in a tangy
dressing of Calabrian chiles.

The antipasti don't always look like antipasti. Most of them are
vegetables, some presented like salads (grilled fennel with blood
oranges) and others like side dishes (roasted squash with a sticky
caramelized surface bathed in brown butter).

The bagna cauda, vegetables crowded around a warm dip that, excellently,
contains a huge dose of anchovies, is a more traditional antipasto. So
is the prosciutto listed under ``Cocktail snacks.'' You drape it on
bread that you spread with butter that's been creamed with grated
Parmigiano-Reggiano. Then you dab it with pickled mustard seeds soaked
in balsamic vinegar. Then you shake your head a little at how smart Ms.
Robbins is about augmenting classic Italian flavors without stepping all
over them.

Desserts follow that template. The olive oil cake is luxuriously rich,
with a satisfyingly browned crust. A tart of crisp sliced apples is
excellent, too. I wasn't expecting much from soft-serve gelato with
mix-and-match toppings, but a swirl of vanilla under a spoonful of
walnuts preserved in lemon syrup tasted exactly right.

Lilia is the first and, for now, only restaurant owned and operated by
Ms. Robbins. This makes it something of a rarity: the sole focus of a
midcareer chef who has the maturity to stay away from trends and
novelties, the experience to know what will be delicious, the managerial
skill to pull it off, and the time to make it work.

It's hard to begrudge her peers' decisions to leverage success into
sprawling empires. Then when you eat at Lilia, it's hard not to wonder
if the rest of us are losing something in the bargain.

Advertisement

\protect\hyperlink{after-bottom}{Continue reading the main story}

\hypertarget{site-index}{%
\subsection{Site Index}\label{site-index}}

\hypertarget{site-information-navigation}{%
\subsection{Site Information
Navigation}\label{site-information-navigation}}

\begin{itemize}
\tightlist
\item
  \href{https://help.nytimes3xbfgragh.onion/hc/en-us/articles/115014792127-Copyright-notice}{©~2020~The
  New York Times Company}
\end{itemize}

\begin{itemize}
\tightlist
\item
  \href{https://www.nytco.com/}{NYTCo}
\item
  \href{https://help.nytimes3xbfgragh.onion/hc/en-us/articles/115015385887-Contact-Us}{Contact
  Us}
\item
  \href{https://www.nytco.com/careers/}{Work with us}
\item
  \href{https://nytmediakit.com/}{Advertise}
\item
  \href{http://www.tbrandstudio.com/}{T Brand Studio}
\item
  \href{https://www.nytimes3xbfgragh.onion/privacy/cookie-policy\#how-do-i-manage-trackers}{Your
  Ad Choices}
\item
  \href{https://www.nytimes3xbfgragh.onion/privacy}{Privacy}
\item
  \href{https://help.nytimes3xbfgragh.onion/hc/en-us/articles/115014893428-Terms-of-service}{Terms
  of Service}
\item
  \href{https://help.nytimes3xbfgragh.onion/hc/en-us/articles/115014893968-Terms-of-sale}{Terms
  of Sale}
\item
  \href{https://spiderbites.nytimes3xbfgragh.onion}{Site Map}
\item
  \href{https://help.nytimes3xbfgragh.onion/hc/en-us}{Help}
\item
  \href{https://www.nytimes3xbfgragh.onion/subscription?campaignId=37WXW}{Subscriptions}
\end{itemize}
