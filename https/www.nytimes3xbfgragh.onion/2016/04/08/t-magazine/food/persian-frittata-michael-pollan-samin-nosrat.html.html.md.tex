Sections

SEARCH

\protect\hyperlink{site-content}{Skip to
content}\protect\hyperlink{site-index}{Skip to site index}

\href{https://myaccount.nytimes3xbfgragh.onion/auth/login?response_type=cookie\&client_id=vi}{}

\href{https://www.nytimes3xbfgragh.onion/section/todayspaper}{Today's
Paper}

An Herby Persian Frittata From Michael Pollan's Chef Teacher

\url{https://nyti.ms/1qchs5y}

\begin{itemize}
\item
\item
\item
\item
\item
\end{itemize}

Advertisement

\protect\hyperlink{after-top}{Continue reading the main story}

Supported by

\protect\hyperlink{after-sponsor}{Continue reading the main story}

\href{/column/food-matters}{Food Matters}

\hypertarget{an-herby-persian-frittata-from-michael-pollans-chef-teacher}{%
\section{An Herby Persian Frittata From Michael Pollan's Chef
Teacher}\label{an-herby-persian-frittata-from-michael-pollans-chef-teacher}}

\includegraphics{https://static01.graylady3jvrrxbe.onion/images/2016/04/08/t-magazine/07tmag-nosrat01/07tmag-nosrat01-articleLarge.jpg?quality=75\&auto=webp\&disable=upscale}

By Charlotte Druckman

\begin{itemize}
\item
  April 8, 2016
\item
  \begin{itemize}
  \item
  \item
  \item
  \item
  \item
  \end{itemize}
\end{itemize}

In the culinary industry, the teacher and writer Samin Nosrat is a go-to
resource for matching the correct techniques with the best ingredients.
The San Diego-born Iranian-American is at her best in the new Netflix
documentary series ``Cooked,'' teaching Michael Pollan how to braise a
pork shoulder. Pollan may be recognized for addressing the omnivore's
dilemma, but he turned to Nosrat to actually learn to cook ---
efficiently, properly and well. ``The formula is liquid, meat and
vegetables --- it's what really makes it delicious,'' she tells him, in
the second episode, before opening the oven door so her student can
slide his covered clay pot into the oven. ``It's all about giving it the
gentle heat it needs \ldots{} the more tender and luscious, and
melt-in-your-mouth rich, velvety yumminess. Yeah.'' As she describes the
tantalizing outcome, she breaks out into a hearty laugh, because she
knows she's just gotten \emph{that} excited about a simple piece of
pork.

It's been nearly 16 years since the U.C. Berkeley undergraduate decided
to pursue a culinary career, and this excitement about discovering ---
and sharing --- has not abated. It is her driving force. It propelled
her, during her sophomore year, to ask for a job at Alice Waters's famed
Berkeley restaurant Chez Panisse, where for four years she learned not
only how to cook professionally, but also to teach ``the young kids
coming in, wide-eyed'' to do the same. And it led her afterward to
apprenticeships with the legendary Tuscan butcher Dario Cecchini and the
revered chef Benedetta Vitali in Florence.

When she was back from Italy, working as a sous chef at Eccolo, she
noticed Pollan in the now-shuttered Berkeley restaurant's dining room
--- and wrote him a note requesting to audit his graduate seminar on
food reporting. In it, students took turns bringing in snacks, and
explaining how they were made. ``I, because I'm insane,'' Nosrat says,
``brought an oven into class and set the table, and made lasagna.''
Later, Pollan asked her to be his kitchen tutor --- while he worked on a
book about cooking. (Around the same time, she began running
\href{http://popupgeneralstore.blogspot.com/}{Pop-Up General Store}, a
sporadically and spontaneously occurring food bazaar in Oakland with the
Chez Panisse alum and the chef/owner of Eccolo,
\href{http://oldfashionedbutcher.blogspot.com/}{Christopher Lee}. When
it rapidly outgrew its customer base, Nosrat shut it down, but not
before passing along everything she'd learned to the people behind the
successful organic grocery delivery service
\href{http://www.nytimes3xbfgragh.onion/2014/08/05/t-magazine/good-eggs-company-lunches-tomato-tart-corn-soup-fruit-butterhead-lettuce-recipes.html}{Good
Eggs}.)

``When I was younger, it was people I wanted to apprentice with,'' she
explains. ``But now, it's about being a partner. I see someone who's
doing something amazing or inspiring and I want to make something in the
world with them, even if it's not apparent what that thing can be.''
Nosrat's relationship with Pollan is the result of that earlier yearning
to find mentors. He encouraged her to turn her organized theory of
cooking into a book, but not without testing it in the classroom first.
She's been teaching for the last six years, ``facing people who didn't
know how to hold a knife, who didn't know what an emulsion was.'' From
these classes, held mostly at \href{https://18reasons.org/}{18 Reasons}
in San Francisco's Mission District, she learned what worked for her
tutees and has been able to incorporate that knowledge into the text.
``I feel like I'm cutting off my arm and I'm giving it to the world,''
she says of ``Salt, Fat, Acid, Heat,'' which publishes next spring.

Her book collaborator is the graphic journalist
\href{http://tmagazine.blogs.nytimes3xbfgragh.onion/2014/03/17/bookshelf-an-illustrator-captures-the-spirit-of-san-francisco-and-an-april-bloomfield-recipe/}{Wendy
MacNaughton}, and it's one in a string of partnerships. There's also a
top-secret editorial project with the Brooklyn-based floral designer
\href{http://tmagazine.blogs.nytimes3xbfgragh.onion/2015/07/13/sarah-ryhanen-saipua-worlds-end/}{Sarah
Ryhanen} that began, in typical Nosrat fashion, with an offer to cook a
meal for her flower arranging class, promising, ``it will be as
beautiful as your flowers.'' Concurrently, she's brainstorming ideas for
her umpteenth side project: a podcast about food culture with Wendy
Dorr, a former producer of ``This American Life'' who's now at
\href{https://gimletmedia.com}{Gimlet Media}.

This week finds her on the road with ``Pop-Up Magazine,'' the
``publication'' whose stories are told on a stage, instead of printed in
a book. On Tuesday night at BAM, she took the stage to talk about the
``taste of conflict'' --- and the flavors and ingredients we miss or
lose due to the politics of war. When she gets back home to Berkeley,
she'll return to spring and doing what she loves best: ``peeling and
turning an artichoke perfectly,'' gathering greens to make an
herb-filled Persian frittata (kuku sabzi, the recipe for which she
shares below) in the style of her mother and telling anyone who will
listen to brine their poultry the day before cooking it. At the rate
she's going, before long, there will be salted chicken in every pot.

\hypertarget{kuku-sabzi}{%
\subsection{Kuku Sabzi}\label{kuku-sabzi}}

\emph{Nosrat grew up eating traditional Persian dishes, like the Iranian
frittata known as kuku sabzi; her mother's version was greener than
anyone else's. ``She was a health food freak and grew up on a farm and
believed in packing as many greens and herbs in there as possible. I've
never seen a kuku as full of green things as hers.'' The chef's version
is at least as verdant. ``It's so insanely green, and healthy, and
spring, and fresh-tasting and different. And it reminds me of my mom,''
she says, enthusiastically. You can make it with whatever fresh herbs
and leafy stems you can get your hands on --- Nosrat has friends who
make it with lettuce.}

Extra-Virgin Olive Oil\\
2 bunches green chard, washed, or 2 pounds wild nettles or spinach,
picked and washed\\
Salt\\
6 tablespoons butter\\
1 large leek, sliced thinly and washed, including green top\\
2 cups roughly chopped dill leaves and tender stems\\
4 cups roughly chopped cilantro leaves and tender stems\\
9 large eggs

1. Preheat the oven to 350 degrees if you do not want to flip your kuku
partway through cooking.

2. If using chard, strip the leaves: Gripping at the base of each stem
with one hand, pinch the stem with the other hand and pull upward to
strip the leaf. Repeat with remaining chard.

3. Gently heat a large cast iron or nonstick frying pan over medium heat
and add 2 tablespoons of olive oil. Add in the chard leaves, or other
greens, and season with salt. Cook, stirring occasionally, until the
leaves are wilted, about 4 to 5 minutes. Remove from the heat, set aside
and allow to cool.

4. If using chard, thinly slice the stems, discarding any tough bits at
the base.

5. Return the pan to the stove and heat over a medium flame. Add 2
tablespoons each of butter and olive oil. When the butter begins to
foam, add the sliced leeks and chard stems, along with a pinch of salt.
Cook until tender and translucent, 15 to 20 minutes. Stir from time to
time, and if needed, add a splash of water, reduce the flame, or cover
with a lid or a piece of parchment paper to entrap steam and keep color
from developing.

6. In the meantime, squeeze the cooked chard (or nettles or spinach)
leaves dry, then chop them roughly. Put them in a large bowl with the
cilantro and dill. When the leeks and chard stems are cooked, add them
to the greens. Use your hands to mix everything up evenly. Taste the
mixture and season generously with salt, knowing you're about to add a
bunch of eggs to the mixture.

7. Add the eggs in, one at a time, until the mixture is just barely
bound with egg --- you might not need to use all nine eggs, depending on
how wet your greens were and how large your eggs are. It should seem
like a ridiculous amount of greens! I usually taste and adjust the
mixture for salt at this point, but if you don't want to taste raw egg,
you can cook up a little test piece of kuku and adjust salt if needed.

8. Wipe out and reheat your pan over medium-high heat. (This is an
important step to prevent the kuku from sticking.) Add 4 tablespoons of
butter and 2 tablespoons of olive oil, then stir to combine. When the
butter begins to foam, carefully pack the kuku mixture into the pan.

9. To help the kuku cook evenly, in the first few minutes of cooking,
use a rubber spatula to gently pull the edges of the frittata into the
center as they set. After about two minutes of this, reduce the heat to
medium and let the kuku cook without touching it. You'll know the pan is
hot enough as long as the oil is gently bubbling up the sides of the
kuku.

10. Because this kuku is so thick, it'll take a while for the center to
set. The key here is to not let the crust burn before the center sets.
Peek at the crust by lifting the kuku with a rubber spatula, and if it's
getting too dark, too soon, then reduce the heat. Rotate the pan a
quarter turn every 3 or 4 minutes to ensure even browning.

11. After about 10 minutes, gather all of your courage and prepare to
flip the kuku. First, tip out as much of the cooking fat as you can into
a bowl to prevent burning yourself, then flip the kuku onto a pizza pan
or the back of a cookie sheet, or into another large frying pan. Add 2
tablespoons olive oil into the hot pan and slide the kuku back in to
cook the second side. Cook for another 10 minutes, rotating the pan
every 3 or 4 minutes.

12. If something goes awry when you try to flip, don't freak out! It's
only lunch! Just do your best to flip the \emph{kuku}, add a little more
oil into the pan, and get it back into the pan in one piece. If you
prefer not to flip, then slip the whole pan into the 350-degree oven and
bake until the center is fully set, about 10 to 12 minutes. I like to
cook it until it is \emph{just} set. Check for doneness using a
toothpick, or just by checking for a faint jiggle at the top of the
frittata.

13. Remove from the oven when done and carefully flip out of the pan
onto a plate. Eat warm, at room temperature, or cold.

Advertisement

\protect\hyperlink{after-bottom}{Continue reading the main story}

\hypertarget{site-index}{%
\subsection{Site Index}\label{site-index}}

\hypertarget{site-information-navigation}{%
\subsection{Site Information
Navigation}\label{site-information-navigation}}

\begin{itemize}
\tightlist
\item
  \href{https://help.nytimes3xbfgragh.onion/hc/en-us/articles/115014792127-Copyright-notice}{©~2020~The
  New York Times Company}
\end{itemize}

\begin{itemize}
\tightlist
\item
  \href{https://www.nytco.com/}{NYTCo}
\item
  \href{https://help.nytimes3xbfgragh.onion/hc/en-us/articles/115015385887-Contact-Us}{Contact
  Us}
\item
  \href{https://www.nytco.com/careers/}{Work with us}
\item
  \href{https://nytmediakit.com/}{Advertise}
\item
  \href{http://www.tbrandstudio.com/}{T Brand Studio}
\item
  \href{https://www.nytimes3xbfgragh.onion/privacy/cookie-policy\#how-do-i-manage-trackers}{Your
  Ad Choices}
\item
  \href{https://www.nytimes3xbfgragh.onion/privacy}{Privacy}
\item
  \href{https://help.nytimes3xbfgragh.onion/hc/en-us/articles/115014893428-Terms-of-service}{Terms
  of Service}
\item
  \href{https://help.nytimes3xbfgragh.onion/hc/en-us/articles/115014893968-Terms-of-sale}{Terms
  of Sale}
\item
  \href{https://spiderbites.nytimes3xbfgragh.onion}{Site Map}
\item
  \href{https://help.nytimes3xbfgragh.onion/hc/en-us}{Help}
\item
  \href{https://www.nytimes3xbfgragh.onion/subscription?campaignId=37WXW}{Subscriptions}
\end{itemize}
