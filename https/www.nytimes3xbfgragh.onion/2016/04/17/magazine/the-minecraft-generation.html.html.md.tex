The Minecraft Generation

\url{https://nyti.ms/1Wtm4RB}

\begin{itemize}
\item
\item
\item
\item
\item
\item
\end{itemize}

\includegraphics{https://static01.graylady3jvrrxbe.onion/images/2016/04/17/magazine/17mag-minecraft-1/17mag-minecraft-1-articleLarge.gif?quality=75\&auto=webp\&disable=upscale}

Sections

\protect\hyperlink{site-content}{Skip to
content}\protect\hyperlink{site-index}{Skip to site index}

Feature

\hypertarget{the-minecraft-generation}{%
\section{The Minecraft Generation}\label{the-minecraft-generation}}

How a clunky Swedish computer game is teaching millions of children to
master the digital world.

Credit...Illustration by Christoph Niemann

Supported by

\protect\hyperlink{after-sponsor}{Continue reading the main story}

By Clive Thompson

\begin{itemize}
\item
  April 14, 2016
\item
  \begin{itemize}
  \item
  \item
  \item
  \item
  \item
  \item
  \end{itemize}
\end{itemize}

Jordan wanted to build an unpredictable trap.

An 11-year-old in dark horn-­rimmed glasses, Jordan is a devotee of
Minecraft, the computer game in which you make things out of virtual
blocks, from dizzying towers to entire cities. He recently read ``The
Maze Runner,'' a sci-fi thriller in which teenagers live inside a
booby-­trapped labyrinth, and was inspired to concoct his own version
--- something he then would challenge his friends to navigate.

Jordan built a variety of obstacles, including a deluge of water and
walls that collapsed inward, Indiana Jones-style. But what he really
wanted was a trap that behaved unpredictably. That would really throw
his friends off guard. How to do it, though? He obsessed over the
problem.

Then it hit him: the animals! Minecraft contains a menagerie of virtual
creatures, some of which players can kill and eat (or tame, if they want
pets). One, a red-and-white cowlike critter called a mooshroom, is known
for moseying about aimlessly. Jordan realized he could harness the
animal's movement to produce randomness. He built a pen out of gray
stones and installed ``pressure plates'' on the floor that triggered a
trap inside the maze. He stuck the mooshroom inside, where it would
totter on and off the plates in an irregular pattern.

Presto: Jordan had used the cow's weird behavior to create, in effect, a
random-­number generator inside Minecraft. It was an ingenious bit of
problem-­solving, something most computer engineers I know would regard
as a great hack --- a way of coaxing a computer system to do something
new and clever.

When I visited Jordan at his home in New Jersey, he sat in his family's
living room at dusk, lit by a glowing iMac screen, and mused on
Minecraft's appeal. ``It's like the earth, the world, and you're the
creator of it,'' he said. On-screen, he steered us over to the entrance
to the maze, and I peered in at the contraptions chugging away. ``My art
teacher always says, `No games are creative, except for the people who
create them.' But she said, `The only exception that I have for that is
Minecraft.' '' He floated over to the maze's exit, where he had posted a
sign for the survivors: \emph{The journey matters more than what you get
in the end.}

Since its release seven years ago, Minecraft has become a global
sensation, captivating a generation of children. There are over 100
million registered players, and it's now the third-best-­selling video
game in history, after Tetris and Wii Sports. In 2014, Microsoft bought
Minecraft --- and Mojang, the Swedish game studio behind it --- for
\$2.5 billion.

There have been blockbuster games before, of course. But as Jordan's
experience suggests --- and as parents peering over their children's
shoulders sense --- Minecraft is a different sort of phenomenon.

For one thing, it doesn't really feel like a \emph{game}. It's more like
a destination, a technical tool, a cultural scene, or all three put
together: a place where kids engineer complex machines, shoot videos of
their escapades that they post on YouTube, make art and set up servers,
online versions of the game where they can hang out with friends. It's a
world of trial and error and constant discovery, stuffed with byzantine
secrets, obscure text commands and hidden recipes. And it runs
completely counter to most modern computing trends. Where companies like
Apple and Microsoft and Google want our computers to be easy to
manipulate --- designing point-and-click interfaces under the assumption
that it's best to conceal from the average user how the computer works
--- Minecraft encourages kids to get under the hood, break things, fix
them and turn mooshrooms into random-­number generators. It invites them
to tinker.

\includegraphics{https://static01.graylady3jvrrxbe.onion/images/2016/04/17/magazine/17mag-minecraft-2/17mag-minecraft-2-articleLarge-v3.gif?quality=75\&auto=webp\&disable=upscale}

In this way, Minecraft culture is a throwback to the heady early days of
the digital age. In the late '70s and '80s, the arrival of personal
computers like the Commodore 64 gave rise to the first generation of
kids fluent in computation. They learned to program in Basic, to write
software that they swapped excitedly with their peers. It was a playful
renaissance that eerily parallels the embrace of Minecraft by today's
youth. As Ian Bogost, a game designer and professor of media studies at
Georgia Tech, puts it, Minecraft may well be this generation's personal
computer.

At a time when even the president is urging kids to learn to code,
Minecraft has become a stealth gateway to the fundamentals, and the
pleasures, of computer science. Those kids of the '70s and '80s grew up
to become the architects of our modern digital world, with all its
allures and perils. What will the Minecraft generation become?

\textbf{``Children,'' the social} critic Walter Benjamin wrote in 1924,
``are particularly fond of haunting any site where things are being
visibly worked on. They are irresistibly drawn by the detritus generated
by building, gardening, housework, tailoring or carpentry.''

Playing with blocks, it turns out, has deep cultural roots in Europe.
Colin Fanning, a curatorial fellow at the Philadelphia Museum of Art,
points out that European philosophers have long promoted block-­based
games as a form of ``good'' play that cultivates abstract thought. A
recent paper Fanning wrote with Rebecca Mir traces the tradition to the
English political philosopher John Locke, who was an early advocate of
alphabet blocks. A century later, Friedrich Froebel --- often called the
inventor of kindergarten --- developed block-­based toys that he claimed
would illustrate the spiritual connectedness of all things. Children
would start with simple blocks, build up to more complex patterns, then
begin to see these patterns in the world around them. Educators like
Maria Montessori picked up on this concept and pioneered the teaching of
math through wooden devices.

During the political cataclysms of the 20th century, European thinkers
regarded construction-­play not merely as a way to educate children but
also as a means to heal their souls. The Danish landscape architect Carl
Theodor Sorensen urged that areas in cities ruined by World War II be
turned into ``junk playgrounds,'' where children would be given
pickaxes, hammers and saws and allowed to shape the detritus into a new
civilization, at child scale. (Several were in fact created in Europe
and were quite popular.) In Sweden, educators worried that
industrialization and the mechanization of society were causing children
to lose touch with physical skills; they began teaching \emph{sloyd}, or
woodcrafting, a practice that continues today.

When Fanning first saw Minecraft, he felt a jolt of recognition. Nearly
all these historical ­impulses were evident in the game. ``It's striking
to me how much this mirrors the appeal and the critical reception of
Minecraft,'' he says. ``In Scandinavian toys, the material of wood has
had a really long association with notions of timelessness and quality
and craftsmanship.'' In Minecraft, as he notes, wood is one of the first
resources new players gather upon entering the game: chopping trees with
their avatar's hand produces blocks of wood, and from those they begin
to build a civilization. Children are turned loose with tools to
transform a hostile environment into something they can live in.

Block-play was, in the European tradition, regarded as a particularly
``wholesome'' activ­ity; it's not hard to draw a line from that to many
parents' belief that Minecraft is the ``good'' computer game in a world
full of anxiety about too much ``screen time.'' In this way, Minecraft
has succeeded Lego as the respectable creative toy. When it was first
sold in the postwar period, Lego presented itself as the heir to the
heritage of playing with blocks. (One ad read: ``It's a pleasure to see
children playing with Lego --- Lego play is quiet and stimulating.
Children learn to grapple with major tasks and solve them together.'')
Today many cultural observers argue that Lego has moved away from that
open-­ended engagement, because it's so often sold in branded kits:
\href{http://lego.brickinstructions.com/m/lego_instructions/set/4842/Hogwarts_Castle_}{the
Hogwarts castle} from ``Harry Potter,''
\href{http://lego.brickinstructions.com/lego_instructions/set/7146/TIE_Fighter}{the
TIE fighter} from ``Star Wars.''

``It's `Buy the box, open the box, turn to the instruction sheet, make
the model, stick it on the shelf, buy the next box,' '' the veteran
­game designer Peter Molyneux says in a 2012 documentary about
Minecraft. ``Lego used to be just a big box of bricks, and you used to
take the bricks, pour them on the carpet and then make stuff. And that's
exactly what Minecraft is.''

As a Swede, Markus Persson, who invented Minecraft and founded Mojang,
grew up amid such cultural influences and probably encountered
\emph{sloyd} in school himself. In Minecraft, Persson created what
Fanning calls ``a sort of digital \emph{sloyd}.''

Persson, now 36, was a child of the '80s computer scene who learned to
program when he was 7 on his father's Commodore 128. By the time he was
in his 20s, he was working for an online photo-­album site and
programming games in his spare time at home, an apartment littered with
game CDs and soda bottles. He released the first version of Minecraft in
2009. The basic play is fairly simple: Each time you start a new game,
Minecraft generates a unique world filled with hills, forests and lakes.
Whatever the player chops at or digs into yields building blocks ---
trees provide wood, the earth dirt and stone. Blocks can be attached to
one another to quickly produce structures. Players can also combine
blocks to ``craft'' new items. Take some stone blocks, add a few pieces
of wood, and you make a pickax, which then helps you dig more quickly
and deeper, till you reach precious materials like gold, silver and
diamond. ``Mobs,'' the game's creatures (``mob'' is short for
``mobile''), can be used for crafting, too. Kill a spider, and you get
spider silk, handy for making bows and arrows.

In its first year, Minecraft found popularity mostly among adult nerds.
But sometime in late 2011, according to Alex Leavitt, a Ph.D. candidate
at the University of Southern California, children discovered it, and
sales of the game exploded. Today it costs \$27 and sells 10,000 copies
a day. (It's still popular across all age groups; according to
Microsoft, the average player is between 28 and 29, and women make up
nearly 40 percent of all players.) Persson frequently added new features
to the game, like a ``survival mode,'' in which every 20 minutes evening
falls and monsters attack --- skeletons shooting arrows, ``creepers''
blowing themselves up when they get close to you --- forcing players to
build protective shelters. (``Creative mode'' is just about making
things.)

Persson also made it possible for players to share their works. You
could package your world as a ``map'' and post it online for others to
download and move around in. Even more sophisticated players could
modify Minecraft's code, creating new types of blocks and creatures, and
then put these ``mods'' online for others to use. Further developments
included a server version of Minecraft that lets people play together on
the Internet inside the same world. These days, kids can pay as little
as \$5 a month to rent such a server. They can also visit much larger
commercial servers capable of hosting hundreds or thousands of players
simultaneously. There is no single, central server: Thousands exist
worldwide.

The game was a hit. But Persson became unsettled by his fame, as well as
the incessant demands of his increasingly impassioned fans --- who
barraged him with emails, tweets and forum posts, imploring him to add
new elements to Minecraft, or complaining when he updated the game and
changed something. By 2014, he'd had enough. After selling Minecraft to
Microsoft, he hunkered down in a \$70 million mansion in Beverly Hills
and now refuses to talk about Minecraft any more.

I wanted to know whether the European tradition of block-play had
influenced him, but Persson politely declined to be interviewed. Via a
public reply to me on Twitter, he explained that he ``sold Minecraft to
get away from it.''

\textbf{Nearly everyone} who plays Minecraft, or even watches someone
else do so, remarks on its feeling of freedom: All those blocks,
infinities of them! Build anything you want! Players have re­created the
Taj Mahal, the U.S.S. Enterprise from ``Star Trek,'' the entire capital
city from ``Game of Thrones.'' It's the most obvious appeal of the game.
But I first started to glimpse how complex Minecraft culture can be when
I saw what kids were doing with what's called ``redstone,'' the game's
virtual wiring. My two sons had begun using it: Zev, who is 8, showed me
an automated ``piston door'' and stone gateway he built. Gabriel, who is
10, had created a ``minigame'' whose actions included a mechanism that
dropped anvils from a height, which players on the ground had to dodge.

Redstone transports energy between blocks, like an electrical
connection. Attach a block that contains power --- a redstone ``torch,''
for example, which looks like a forearm-size matchstick --- to one end
of a trail of redstone, and anything connected to the other end will
receive power. Hit a button \emph{here}, and another block shifts
position over \emph{there}. Persson ingeniously designed redstone in a
way that mimics real-world electronics. Switches and buttons and levers
turn the redstone on and off, enabling players to build what computer
scientists call ``logic gates.'' Place two Minecraft switches next to
each other, connect them to redstone and suddenly you have what's known
as an ``AND'' gate: If Switch 1 and Switch 2 are both thrown, energy
flows through the redstone wire. You can also rig an ``OR'' gate,
whereby flipping either lever energizes the wire.

These AND and OR gates are, in virtual form, the same as the circuitry
you'd find inside a computer chip. They're also like the Boolean logic
that programmers employ every day in their code. Together, these simple
gates let Minecraft players construct machines of astonishing
complexity.

One day this winter, I met Sebastian, a 14-year-old, at his home in New
Jersey, where he showed off his redstone devices. One was a huge
``trading post,'' a contraption that allows players on either side of a
large wall to trade items through an automated chute. It required a
large cluster of AND gates, he said, and took him several days to figure
out.

\includegraphics{https://static01.graylady3jvrrxbe.onion/images/2016/04/13/multimedia/17mag-minecraft-final/17mag-minecraft-final-videoSixteenByNine1050.jpg}

``Hop down here,'' he said, moving down into a subterranean pit beneath
the apparatus and looking around. (In Minecraft, you see the world from
the viewpoint of your in-game avatar.) It was like being in the bowels
of a factory: the redstone sprawled in all directions. He pointed out
different parts of the wiring, rattling off components like an architect
at a construction site. ``Coming in from these two wires are the lever
inputs from the side --- and from over here, the other side. And what
these do is, when they're both on, they power a piston, which pairs
redstone to this block up into this tower dispenser.''

Mastering redstone requires rigorously logical thinking, as well as a
great deal of debugging: When your device isn't working, you have to
carefully go over its circuitry to figure out what's wrong. One fifth
­grader I visited, Natalie, was assembling a redstone door on her iPad
while I watched. But nothing happened when she flicked the ``on'' lever.
``I did that wrong,'' she said with a frown, and began tracing her way
through the circuit. Eventually the problem emerged: A piece of redstone
was angled incorrectly, sending the current in the wrong direction.

This is what computer scientists call computational thinking, and it
turns out to be one of Minecraft's powerful, if subtle, effects. The
game encourages kids to regard logic and if-then statements as fun
things to mess around with. It teaches them what computer coders know
and wrestle with every day, which is that programs rarely function at
first: The work isn't so much in writing a piece of software but in
debugging it, figuring out what you did wrong and coming up with a fix.

Minecraft is thus an almost perfect game for our current educational
moment, in which policy makers are eager to increase kids' interest in
the ``STEM'' disciplines --- science, technology, engineering and math.
Schools and governments have spent millions on ``let's get kids coding''
initiatives, yet it may well be that Minecraft's impact will be greater.
This is particularly striking given that the game was not designed with
any educational purpose in mind. ``We have never done things with that
sort of intent,'' says Jens Bergensten, the lead Minecraft developer at
Mojang and Persson's first hire. ``We always made the game for
ourselves.''

Other Minecraft features resemble the work of software engineers even
more closely. For example, programmers frequently write code and control
their computers through a bare-bones interface known as the ``command
line,'' typing abstruse, text-­based commands rather than pointing and
clicking. Many programmers I know complain that while the
point-and-click world has made computers easier to use for everyday
people, it has also dumbed us down; kids don't learn the command line
the way they would have back when personal-­computer use emerged in the
'70s and '80s. This is partly why newcomers can find programming
alienating: They're not accustomed to controlling a computer using only
text.

But Minecraft, rather audaciously, includes a command line and requires
players to figure it out. Type ``t'' or ``/'' while playing the game,
and a space appears where you can chat with other players or issue
commands that alter the environment. For example, typing ``/time set 0''
instantly changes the time of day inside the game to daybreak; the sun
suddenly appears on the horizon. Complex commands require a player to
master chains of sophisticated command-­line syntax.

One day last fall, I visited Gus, a seventh ­grader in Brooklyn. He was
online with friends on a server they share together, engaging in
boisterous gladiatorial combat. I watched as he typed a command to endow
himself with a better weapon: ``/give AdventureNerd bow 1 0
\{Unbreakable:1,ench:{[}\{id:51,lvl:1\}{]},display:\{Name:``Destiny''\}\}.''
What the command did was give a bow-­and-­arrow weapon to AdventureNerd,
Gus's avatar; make the bow unbreakable; endow it with magic; and name
the weapon Destiny, displayed in a tag floating over the weapon. Gus had
plastered virtual sticky-­notes all over his Mac's desktop listing the
text commands he uses most often. Several commands can be packed into a
``command block,'' so that clicking on the block activates them, much as
clicking on a piece of software launches it.

Mimi Ito, a cultural anthropologist at the University of California,
Irvine, and a ­founder of Connected Camps, an online program where kids
play Minecraft together, has closely studied gamers and learning. Ito
points out that when kids delve into this hackerlike side of the game
--- concocting redstone devices or creating command blocks --- they
often wind up consulting discussion forums online, where they get advice
from adult Minecraft players. These folks are often full-time
programmers who love the game, and so younger kids and teenagers wind up
in conversation with professionals.

``It's one of the places where young people are engaging with more
expert people who are much older than them,'' Ito says. These
connections are transformative: Kids get a glimpse of a professional
path that their schoolwork never illuminates. ``An adult mentor opens up
these new worlds that wouldn't be open to them,'' she adds. Of course,
critics might worry about kids interacting with adults online in this
way, but as Ito notes, when there's a productive task at hand, it's
similar to how guilds have passed on knowledge for ages: knowledgeable
adults mentoring young people.

Image

Credit...Illustration by Christoph Niemann

Ito has also found that kids' impulse to tinker with Minecraft pushes
them to master real-world technical skills. One 15-year-old boy I
interviewed, Eli, became interested in making ``texture packs.'' These
are the external shells that wrap around 3-D objects in the game, like a
drape thrown over a table: Change the pattern on the drape, and you can
change what the object looks like. Designing texture packs prompted Eli
to develop sophisticated Photoshop skills. He would talk to other
texture-­pack designers on Minecraft forums and get them to send him
their Photoshop files so he could see how they did things. He also began
teaching himself to draw. ``I'd be downloading the mod,'' he says,
``looking at the original texture and saying, `O.K., how can I make this
a little more cartoony?' '' Then he would put his own designs up on the
forums to get feedback, which, he discovered, was usually very polite
and constructive. ``The community,'' he says, ``is very helpful.''

While Minecraft rewards this sort of involvement, it can also be
frustrating: Mojang updates Minecraft weekly, and sometimes new updates
aren't compatible with an older version. Players complained to me about
waking up to discover that their complex contraptions no longer worked.
One player spent weeks assembling a giant roller coaster whose carts
were powered by redstone tracks only to have an update change the way
rails functioned, and the entire roller-­coaster mechanism never worked
again. Others ruefully described spending months crafting cities on
their own multiplayer servers, only to have a server crash and destroy
everything.

For Ito, this is all a culturally useful part of the experience: Kids
become more resilient, both practically and philosophically. ``Minecraft
is busted, and you're constantly fixing it,'' she says. ``It's that
home-brew aesthetic. It's kind of broken all the time. It's laggy. The
kids get used to the idea that it's broken and you have to mess with it.
You're not complaining to get the corporate overlord to fix it --- you
just have to fix it yourself.'' This is a useful corrective to other
software. ``IPhone apps are kind of at the opposite end,'' Ito says.
``And the way that kids react when things are broken in the Apple
ecosystem versus the Minecraft ecosystem is totally different. With
{[}Apple{]} it's, `Why are they broken?' Whereas with Minecraft it's
like --- `Oh, they messed with something again, it's broken, we have to
go figure out what they changed.' There's a sort of resignation to that
the fact that you're tinkering all the time.''

Because Minecraft is now seven years old, Ian Bogost will soon have
students at Georgia Tech who grew up playing the game. The prospect
intrigues him. ``I'm very curious to see what their attitude to
technology is,'' he says.

\textbf{Two years ago}, Ava, a fifth grader who lives on Long Island,
whom I met through her aunt, a friend of mine, tried Minecraft for the
first time. She started a ``survival'' world and marveled at the jagged
hills receding into the distance. But like most new players, she had no
idea what to do. Night fell, mobs arrived and a skeleton staggered
toward her. She mistakenly assumed it was friendly. ``I was like, Oh,
hi, how are you?'' Ava says. ``And I died after that.''

Minecraft is an incredibly complex game, but it's also --- at first ---
inscrutable. When you begin, no pop-ups explain what to do; there isn't
even a ``help'' section. You just have to figure things out yourself.
(The exceptions are the Xbox and Play­Station versions, which in
December added tutorials.) This unwelcoming air contrasts with most
large games these days, which tend to come with elaborate training
sessions on how to move, how to aim, how to shoot. In Minecraft, nothing
explains that skeletons will kill you, or that if you dig deep enough
you might hit lava (which will also kill you), or even that you can
craft a pickax.

This ``you're on your own'' ethos resulted from early financial
limitations: Working alone, Persson had no budget to design tutorials.
That omission turned out be an inadvertent stroke of genius, however,
because it engendered a significant feature of Minecraft culture, which
is that new players have to learn \emph{how} to play. Minecraft, as the
novelist and technology writer Robin Sloan has observed, is ``a game
about secret knowledge.'' So like many modern mysteries, it has inspired
extensive information-­­sharing. Players excitedly pass along tips or
strategies at school. They post their discoveries in forums and detail
them on wikis. (The biggest one, hosted at the site Gamepedia, has
nearly 5,000 articles; its entry on Minecraft's ``horses,'' for
instance, is about 3,600 words long.) Around 2011, publishers began
issuing handbooks and strategy guides for the game, which became runaway
best sellers; one book on redstone has outsold literary hits like ``The
Goldfinch,'' by Donna Tartt.

``In Minecraft, knowledge becomes social currency,'' says Michael
Dezuanni, an associate professor of digital media at Queensland
University of Technology in Australia. Dezuanni has studied how
middle-­school girls play the game, watching as they engaged in nuanced,
Talmudic breakdowns of a particular creation. This is, he realized, a
significant part of the game's draw: It offers many opportunities to
display expertise, when you uncover a new technique or strategy and
share it with peers.

The single biggest tool for learning Minecraft lore is YouTube. The site
now has more than 70 million Minecraft videos, many of which are
explicitly tutorial. To make a video, players use ``screencasting''
software (some of which is free, some not) that records what's happening
on-screen while they play; they usually narrate their activity in
voice-­over. The problems and challenges you face in Minecraft are, as
they tend to be in construction or architecture, visual and
three-­dimensional. This means, as many players told me, that video
demonstrations have a particularly powerful explanatory force: It's
easiest to learn something by seeing someone else do it. In this sense,
the game points to the increasing role of video as a rhetorical tool.
(``Minecraft'' is the second-­most-­searched-­for term on YouTube, after
``music.'')

Image

Credit...Illustration by Christoph Niemann

That includes Ava on Long Island --- who, after being killed by
skeletons, began watching ``survival mode'' videos to learn how to stay
alive. Soon she had mastered that, and also discovered the huge number
of YouTube videos in which players review ``minigames,'' little
challenges that some Minecraft devotees design and load onto servers for
others to play. (In one popular minigame, for example, players are shown
a sculpture made of blocks and then try to copy it exactly in 30
seconds.) For young Minecraft fans, these videos are a staple of their
media diet, crowding out TV. Ava's mother is genially baffled by this.
``I don't understand it,'' she told her daughter when I visited them
last fall. ``Why are you watching other people play the game? Why don't
you just play?''

Ava had recently started her own YouTube channel with her friends Aaron
and Patrick, where they play and review minigames. Her father set up a
high-­quality microphone on a telescoping arm bolted to the computer
desk; her sister drew Ava a white sign that says: ``RECORDING.'' (Its
back says: ``NOT RECORDING JUST WANT YOU TO BE QUIET.'') As the family's
gray cat wandered around Ava's keyboard, she dialed up Patrick on a
Skype video call.

When they record a video, they improvise freestyle banter while playing,
and simply start all over again if something goes awry. (Which, Patrick
said dryly, ``happens often.'') So far they have 19 subscribers and have
posted 21 videos.

She played a recent video for me, in which they tried to navigate a
difficult map filled with lethal, flowing lava. Their conversation is
loose and funny; it's like listening to two talk-­radio hosts, or
perhaps the commentary over a game of basketball --- if the commentary
were delivered by the athletes themselves, while they play.

Considered as a genre, YouTube Minecraft videos are quite strange. They
take elements of ``how to'' TV --- a cooking show, a home-­renovation
show --- and blend them with the vocal style of podcasting, while mixing
in a dash of TV shows like ``Orange County Choppers,'' where ingenious
mechanics parade their creations.

``I don't even know that I know how to properly classify them,'' says
Ryan Wyatt, the head of gaming content for YouTube. Minecraft videos
offer a glimpse of the blurring of the line between consumers and
creators. Probably two-thirds of the kids I interviewed had started
their own Minecraft channels on YouTube. Most of them were happy when
even a handful of friends and family watched their videos.

Some Minecraft broadcasters have become genuinely famous, though, and
earn a good living from their work. These superstars aren't children,
generally; they're young adults, like Joseph Garrett, known as
\href{https://www.youtube.com/channel/UCj5i58mCkAREDqFWlhaQbOw?nohtml5=False}{Stampy
Cat}, a 25-year-old Briton with seven million YouTube subscribers. One
of my children's favorite Minecraft broadcasters is a user named
\href{https://www.youtube.com/user/ThatMumboJumbo?nohtml5=False}{Mumbo
Jumbo}, another Briton, whose real name is Oliver Brotherhood, known for
his instructional videos on using redstone. He is 20 and began posting
his videos online when he was 16, he says. At first he did it for fun,
until one video --- which showcases 20 complex opening-­door devices ---
became an unexpected hit, netting him one million views. ``It's not the
next `Gangnam Style,' but it was pretty good,'' Brotherhood says. As
more fans found him, he began posting daily and now spends 50 hours a
week shooting videos and replying to fans. Brotherhood delivered
newspapers while in school, but a year ago his YouTube ad revenue
outstripped it.

``I told my mom, `I'm quitting my paper round,' and she said, `Why?' And
I said, `I do a YouTube channel, and it's earning me more.' '' When his
mother looked at his channel, she saw it had more than 40,000
subscribers and more monthly traffic than the corporate newspaper sites
she consults for.

Next year he plans to study computer science in college. ``In the
redstone community,'' he says, ``a lot of people around me are
programmers.'' Teaching himself coding is much like learning Minecraft,
he found; you experiment, ask questions on Internet forums. He described
his YouTube channel on his college application, and that, too, ``seems
to have helped,'' he says. The university accepted him without even
seeing his final school grades.

Image

Credit...Illustration by Christoph Niemann

\textbf{Last year, London,} a 12-year-old in Washington State, set up a
server so he could play Minecraft with friends. He left it public, open
to anyone --- which led to chaos when some strangers logged on one day
to start ``griefing,'' blowing up his and his friends' creations with
TNT. He shut down the server and, a bit wiser now, started a new one
with some strict rules. This one included a ``whitelist,'' so only
players pre­approved by London can log in, and a plug-in --- a piece of
code that changes how the server works --- that prevents players from
destroying what others have made.

Most online games don't require kids to manage the technical aspects of
how gamers interact. A hugely popular commercial game like World of
Warcraft, for example, is played on a server run by its owner, Blizzard
Entertainment. Game companies usually set the rules of what is and isn't
allowed in their games; if you grief others, you might be banned by a
corporate overlord. Or the opposite might happen: Abuse might be ignored
or policed erratically.

But Minecraft is unusual because Microsoft doesn't control all the
servers where players gather online. There is no single Minecraft server
that everyone around the world logs onto. Sometimes kids log onto a
for-­profit server to play mini­games; sometimes they rent a server for
themselves and their friends. (Microsoft and Mojang run one such rental
service.) Or sometimes they do it free at home: If you and I are in the
same room and we both have tablets running Minecraft, I can invite you
into my Minecraft world through Wi-Fi.

What this means is that kids are constantly negotiating what are, at
heart, questions of governance. Will their world be a free-for-all, in
which everyone can create and destroy everything? What happens if
someone breaks the rules? Should they, like London, employ plug-ins to
prevent damage, in effect using software to enforce property rights?
There are now hundreds of such governance plug-ins.

Seth Frey, a postdoctoral fellow in computational social science at
Dartmouth College, has studied the behavior of thousands of youths on
Minecraft servers, and he argues that their interactions are,
essentially, teaching civic literacy. ``You've got these kids, and
they're creating these worlds, and they think they're just playing a
game, but they have to solve some of the hardest problems facing
humanity,'' Frey says. ``They have to solve the tragedy of the
commons.'' What's more, they're often anonymous teenagers who, studies
suggest, are almost 90 percent male (online play attracts far fewer
girls and women than single-­player mode). That makes them ``what I like
to think of as possibly the worst human beings around,'' Frey adds, only
half-­jokingly. ``So this shouldn't work. And the fact that this works
is astonishing.''

Frey is an admirer of Elinor Ostrom, the Nobel Prize-­winning political
economist who analyzed the often-­unexpected ways that everyday people
govern themselves and manage resources. He sees a reflection of her work
in Minecraft: Running a server becomes a crash course in how to
compromise, balance one another's demands and resolve conflict.

Three years ago, the public library in Darien, Conn., decided to host
its own Minecraft server. To play, kids must acquire a library card.
More than 900 kids have signed up, according to John Blyberg, the
library's assistant director for innovation and user experience. ``The
kids are really a community,'' he told me. To prevent conflict, the
library installed plug-ins that give players a chunk of land in the game
that only they can access, unless they explicitly allow someone else to
do so. Even so, conflict arises. ``I'll get a call saying, `This is
Dasher80, and someone has come in and destroyed my house,' '' Blyberg
says. Sometimes library administrators will step in to adjudicate the
dispute. But this is increasingly rare, Blyberg says. ``Generally, the
self-­governing takes over. I'll log in, and there'll be 10 or 15
messages, and it'll start with, `So-and-so stole this,' and each message
is more of this,'' he says. ``And at the end, it'll be: `It's O.K., we
worked it out! Disregard this message!' ''

Several parents and academics I interviewed think Minecraft servers
offer children a crucial ``third place'' to mature, where they can
gather together outside the scrutiny and authority at home and school.
Kids have been using social networks like Instagram or Snapchat as a
digital third place for some time, but Minecraft imposes different
social demands, because kids have to figure out how to respect one
another's virtual space and how to collaborate on real projects.

``We're increasingly constraining youth's ability to move through the
world around them,'' says Barry Joseph, the associate director for
digital learning at the American Museum of Natural History. Joseph is in
his 40s. When he was young, he and his friends roamed the neighborhood
unattended, where they learned to manage themselves socially. Today's
fearful parents often restrict their children's wanderings, Joseph notes
(himself included, he adds). Minecraft serves as a new free-­ranging
realm.

Joseph's son, Akiva, is 9, and before and after school he and his school
friend Eliana will meet on a Minecraft server to talk and play. His son,
Joseph says, is ``at home but still getting to be with a friend using
technology, going to a place where they get to use pickaxes and they get
to use shovels and they get to do that kind of building. I wonder how
much Minecraft is meeting that need --- that need that all children
have.'' In some respects, Minecraft can be as much social network as
game.

Just as Minecraft propels kids to master Photoshop or video-­editing,
server life often requires kids to acquire complex technical skills. One
13-year-old girl I interviewed, Lea, was a regular on a server called
Total Freedom but became annoyed that its administrators weren't
clamping down on griefing. So she asked if she could become an
administrator, and the owners said yes.

For a few months, Lea worked as a kind of cop on that beat. A software
tool called ``command spy'' let her observe records of what players had
done in the game; she teleported miscreants to a sort of virtual ``time
out'' zone. She was eventually promoted to the next rank --- ``telnet
admin,'' which allowed her to log directly into the server via telnet, a
command-­line tool often used by professionals to manage servers. Being
deeply involved in the social world of Minecraft turned Lea into
something rather like a professional systems administrator. ``I'm
supposed to take charge of anybody who's breaking the rules,'' she told
me at the time.

Not everyone has found the online world of Minecraft so hospitable. One
afternoon while visiting the offices of Mouse, a nonprofit organization
in Manhattan that runs high-tech programs for kids, I spoke with Tori.
She's a quiet, dry-­witted 17-year-old who has been playing Minecraft
for two years, mostly in single-­player mode; a recent castle-­building
competition with her younger sister prompted some bickering after Tori
won. But when she decided to try an online server one day, other players
--- after discovering she was a girl --- spelled out ``BITCH'' in
blocks.

She hasn't gone back. A group of friends sitting with her in the Mouse
offices, all boys, shook their heads in sympathy; they've seen this
behavior ``everywhere,'' one said. I have been unable to find solid
statistics on how frequently harassment happens in Minecraft. In the
broader world of online games, though, there is more evidence: An
academic study of online players of Halo, a shoot-'em-up game, found
that women were harassed twice as often as men, and in an unscientific
poll of 874 self-­described online gamers, 63 percent of women reported
``sex-­based taunting, harassment or threats.'' Parents are sometimes
more fretful than the players; a few told me they didn't let their
daughters play online. Not all girls experience harassment in Minecraft,
of course --- Lea, for one, told me it has never happened to her --- and
it is easy to play online without disclosing your gender, age or name.
In-game avatars can even be animals.

\textbf{How long will Minecraft's} popularity endure? It depends very
much on Microsoft's stewardship of the game. Company executives have
thus far kept a reasonably light hand on the game; they have left major
decisions about the game's development to Mojang and let the team remain
in Sweden. But you can imagine how the game's rich grass-roots culture
might fray. Microsoft could, for example, try to broaden the game's
appeal by making it more user-­friendly --- which might attenuate its
rich tradition of information-­sharing among fans, who enjoy the opacity
and mystery. Or a future update could tilt the game in a direction kids
don't like. (The introduction of a new style of combat this spring led
to lively debate on forums --- some enjoyed the new layer of strategy;
others thought it made Minecraft too much like a typical hack-and-slash
game.) Or an altogether new game could emerge, out-­Minecrafting
Minecraft.

But for now, its grip is strong. And some are trying to strengthen it
further by making it more accessible to lower-­income children. Mimi Ito
has found that the kids who acquire real-world skills from the game ---
learning logic, administering servers, making YouTube channels --- tend
to be upper middle class. Their parents and after-­school programs help
them shift from playing with virtual blocks to, say, writing code. So
educators have begun trying to do something similar, bringing Minecraft
into the classroom to create lessons on everything from math to history.
Many libraries are installing Minecraft on their computers.

One recent afternoon, I visited the Bronx Library Center, a sleek,
recently renovated building in a low-­income part of the borough. A
librarian named Katie Fernandez had set up regular Minecraft days for
youths, and I watched four boys play together on the library's server.
Fernandez had given them a challenge: Erect a copy of the Arc de
Triomphe in Paris in 45 minutes. Three of them began collaborating on
one version; a younger boy worked on his own design. The three gently
teased one another about their skills. ``No, no, stop!'' shouted one,
when he noticed another building a foot of the Arc too wide. ``Ryan,
this --- like this!'' They debated whether command blocks would speed
things up. As the 45th minute approached, they hadn't quite finished
their Arc, so they gleefully stuffed the interior with TNT, detonated it
and hopped onto different games.

Over in the corner, the fourth boy continued to labor away at his Arc.
He told me he often stays up late playing Minecraft with friends; they
have built the Statue of Liberty, 1 World Trade Center and even a copy
of the very library he was sitting in. His fingers clicked in a blur as
he placed angled steps, upside-­down, to mimic the Arc's beveled top. He
sat back to admire his work. ``I haven't blinked for over --- I don't
know how many minutes,'' he said. The model was complete, and remarkably
realistic.

``I'm actually pretty proud of that,'' he said with a smile.

Advertisement

\protect\hyperlink{after-bottom}{Continue reading the main story}

\hypertarget{site-index}{%
\subsection{Site Index}\label{site-index}}

\hypertarget{site-information-navigation}{%
\subsection{Site Information
Navigation}\label{site-information-navigation}}

\begin{itemize}
\tightlist
\item
  \href{https://help.nytimes3xbfgragh.onion/hc/en-us/articles/115014792127-Copyright-notice}{©~2020~The
  New York Times Company}
\end{itemize}

\begin{itemize}
\tightlist
\item
  \href{https://www.nytco.com/}{NYTCo}
\item
  \href{https://help.nytimes3xbfgragh.onion/hc/en-us/articles/115015385887-Contact-Us}{Contact
  Us}
\item
  \href{https://www.nytco.com/careers/}{Work with us}
\item
  \href{https://nytmediakit.com/}{Advertise}
\item
  \href{http://www.tbrandstudio.com/}{T Brand Studio}
\item
  \href{https://www.nytimes3xbfgragh.onion/privacy/cookie-policy\#how-do-i-manage-trackers}{Your
  Ad Choices}
\item
  \href{https://www.nytimes3xbfgragh.onion/privacy}{Privacy}
\item
  \href{https://help.nytimes3xbfgragh.onion/hc/en-us/articles/115014893428-Terms-of-service}{Terms
  of Service}
\item
  \href{https://help.nytimes3xbfgragh.onion/hc/en-us/articles/115014893968-Terms-of-sale}{Terms
  of Sale}
\item
  \href{https://spiderbites.nytimes3xbfgragh.onion}{Site Map}
\item
  \href{https://help.nytimes3xbfgragh.onion/hc/en-us}{Help}
\item
  \href{https://www.nytimes3xbfgragh.onion/subscription?campaignId=37WXW}{Subscriptions}
\end{itemize}
