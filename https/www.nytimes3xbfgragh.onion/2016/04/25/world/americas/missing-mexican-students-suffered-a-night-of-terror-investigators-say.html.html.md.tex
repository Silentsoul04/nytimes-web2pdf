Sections

SEARCH

\protect\hyperlink{site-content}{Skip to
content}\protect\hyperlink{site-index}{Skip to site index}

\href{https://www.nytimes3xbfgragh.onion/section/world/americas}{Americas}

\href{https://myaccount.nytimes3xbfgragh.onion/auth/login?response_type=cookie\&client_id=vi}{}

\href{https://www.nytimes3xbfgragh.onion/section/todayspaper}{Today's
Paper}

\href{/section/world/americas}{Americas}\textbar{}Missing Mexican
Students Suffered a Night of `Terror,' Investigators Say

\url{https://nyti.ms/1SYsySA}

\begin{itemize}
\item
\item
\item
\item
\item
\end{itemize}

Advertisement

\protect\hyperlink{after-top}{Continue reading the main story}

Supported by

\protect\hyperlink{after-sponsor}{Continue reading the main story}

\hypertarget{missing-mexican-students-suffered-a-night-of-terror-investigators-say}{%
\section{Missing Mexican Students Suffered a Night of `Terror,'
Investigators
Say}\label{missing-mexican-students-suffered-a-night-of-terror-investigators-say}}

\includegraphics{https://static01.graylady3jvrrxbe.onion/images/2016/04/25/world/americas/25IGUALA-1/25IGUALA-1-articleLarge.jpg?quality=75\&auto=webp\&disable=upscale}

By \href{http://www.nytimes3xbfgragh.onion/by/kirk-semple}{Kirk Semple}

\begin{itemize}
\item
  April 24, 2016
\item
  \begin{itemize}
  \item
  \item
  \item
  \item
  \item
  \end{itemize}
\end{itemize}

MEXICO CITY --- Municipal police officers encircled the bus, detonated
tear gas, punctured the tires and forced the college students who were
onboard to get off.

``We're going to kill all of you,'' the officers warned, according to
the bus driver. A policeman approached the driver and pointed a pistol
at his chest. ``You, too,'' the officer said.

With a military intelligence official looking on and state and federal
police officers in the immediate vicinity, witnesses said, the students
were put into police vehicles and taken away. They have not been seen
since.

They were among the 43 students who vanished in the city of Iguala one
night in September 2014 amid violent, chaotic circumstances laid bare by
an international panel of investigators who have been examining the
matter for more than a year. The reason for the students' abduction
remains a mystery.

Despite apparent
\href{http://www.nytimes3xbfgragh.onion/2016/04/23/world/americas/investigators-say-mexico-has-thwarted-efforts-to-solve-students-disappearance.html?_r=0}{stonewalling
by the Mexican government} in recent months, the panel's two reports on
the case, the most recent of which was released on Sunday, provide the
fullest accounting of the events surrounding the students'
disappearance, which also left six other people dead, including three
students, and scores wounded.

The reports describe a night of confusion and terror for the students
and city residents, and a seemingly clinical, coordinated harvest by
Mexican law enforcement officials and other gunmen operating in and
around Iguala, in Guerrero, one of Mexico's poorest and most violent
states.

The government said 123 people, including 73 municipal police officials,
had been detained on organized-crime charges in relation to the night's
events, and the Mexican authorities have linked the Iguala police force
to a powerful drug gang.

\includegraphics{https://static01.graylady3jvrrxbe.onion/images/2016/04/25/world/25IGUALA-2/25IGUALA-2-videoSixteenByNineJumbo1600.jpg}

The 43 students were undergraduates at Escuela Normal Rural Raúl Isidro
Burgos, a teachers college, in Ayotzinapa, with a history of activism.

They were among about 100 students who headed out on the evening of
Sept. 26, 2014, with a plan to steal some buses. This was a tradition
that students at the school had done for many years: They would take the
buses, use them to transport their peers to an event and then return
them when they were done. The bus companies and the authorities mostly
tolerated it.

The plan for the outing that evening was to secure several buses to
carry students to a march in Mexico City several days later to
commemorate a student massacre that had occurred in 1968.

Riding in two buses they had commandeered on earlier occasions, they
stationed themselves on a main road on the outskirts of Iguala, planning
to intercept a few buses.

``All of us were happy, having a blast, relaxed, happy with the drivers,
playing,'' a student later testified, according to the panel's first
report. It relied on testimony from survivors, government security
officials and other witnesses as well as reports from an interagency
government command center.

But the region's security forces were already onto the students' plans.
The federal police stepped up patrols near the buses, and the command
center linking local, state and federal police forces, as well as the
military, kept tabs on the students.

At 8:15 p.m., the students made their first strike, boarding a bus that
had stopped in front of a restaurant. The driver knew the drill; bus
companies generally instruct drivers that in the event of a student
hijacking, they should remain with the buses to ensure their safe
return.

\includegraphics{https://static01.graylady3jvrrxbe.onion/images/2016/04/26/world/25Mexico-video/25Mexico-video-videoSixteenByNineJumbo1600.jpg}

The bus driver said he needed to make a pit stop at Iguala's central bus
station. At the station, the driver surprised the students and locked
them in the bus.

Around 9:15 p.m., the students in the two other buses arrived at the
station and freed their classmates. The group commandeered three more
buses, leaving behind one that had no driver. The five buses then left
for Ayotzinapa, three heading toward Iguala's northern beltway, two
toward the southern beltway.

Then the shooting began.

Several police cars pursuing the three northbound buses started firing
warning shots into the air. But the threat of violence did not deter the
students.

A group of them left the buses and started throwing rocks at a police
car that had blocked their path until the car drove away. At another
point, a student sneaked up behind a police officer and tried to disarm
him. As other police officers came to their colleague's aid, the student
ran away, and a police bullet ricocheted and struck him, lightly
wounding him.

As the convoy resumed its northward course through the city, police
bullets hit the buses. The students threw themselves flat on the floor
but ordered the drivers to keep going.

Near the beltway, however, the police had blocked the road with a
vehicle. Several students got off the buses and tried to lift the
cruiser out of the roadway, but officers posted on the highway opened
fire on the group, forcing the students to seek cover behind the buses.
Investigators later counted 30 bullet holes in one of the buses.

As bullets flew and windows shattered, one of the students, Aldo
Gutiérrez, was shot in the head. The first call to an emergency dispatch
number was received at 9:48 p.m. Police officers shot at students who
tried to rush to Mr. Gutiérrez's aid.

\includegraphics{https://static01.graylady3jvrrxbe.onion/images/2016/04/25/world/JP-IGUALA/JP-IGUALA-articleLarge.jpg?quality=75\&auto=webp\&disable=upscale}

Another student was shot in a hand; the bullet sheared off several
fingers. He sought shelter behind a truck, where two police officers ran
over to him, and kicked and punched him. A third student was struck in
an arm by a bullet. Ambulance crews managed to retrieve the three
wounded students and take them to a hospital, along with a fourth
student who suffered an asthma attack.

``They all felt confusion, terror and helplessness,'' wrote the panel,
five lawyers and human rights experts from around Latin America.

At one point, the police made a group of students who were hiding in the
third bus disembark and lie on the ground. About 10:50 p.m., they were
taken away in six or seven patrol cars. They are among the 43 students
who disappeared.

Meanwhile, the two buses that took the southerly route had also run into
trouble. About 9:40 p.m., just as the three-bus convoy was intercepted
near the northern beltway, the police cut off one of the southbound
buses, shattered its windows with tree branches and shot tear gas inside
to flush out the passengers.

The passengers were pulled from the bus and taken away: the rest of the
43 missing students.

Elsewhere in the city, the police had stopped the other southbound bus.
The students on board, who had received word by telephone of the other
attacks, got off the bus and fled into woods.

In a measure of the violent pandemonium that overcame Iguala that night,
another bus and several other civilian vehicles came under attack even
though they had nothing to do with the students.

Los Avispones, a soccer team of high schoolers from the city of
Chilpancingo, had played a match that night against a local team in
Iguala. By 11:15 p.m., the players were aboard their bus and heading
home. Their route out of Iguala took them through a state police
roadblock where they were rerouted because of the confrontation between
the students and the police, witnesses said.

Image

About seven miles outside Iguala, gunmen fired on the bus, killing a
soccer player and the driver, and wounding seven other passengers. The
attackers also fired at other passing cars, killing a 40-year-old woman
who was riding in a taxi.

Witnesses said the gunmen had included police officers, and ballistic
tests found that some of the weapons used in the attack belonged to the
Iguala municipal police department.

``The most probable hypothesis is that the bus had been confused for one
of those carrying the student teachers,'' the investigators wrote.

Some soccer players, including one who had been wounded in the eye and
was bleeding profusely, managed to drive to a nearby army battalion but
were offered no help. ``They indicated that they couldn't do anything
because it wasn't in their jurisdiction,'' a witness testified.

Elsewhere, on routes leading from Iguala to Ayotzinapa, at least two
roadblocks were set up by unidentified gunmen, and one by police
officers from the city of Huitzuco. Two civilians were wounded by
gunfire at one of the roadblocks.

The expert panel concluded that ``the joint action shows a coordinated
modus operandi to stop the flight of the buses.''

Meanwhile, at the entrance to the northern beltway, students who had
survived the police fusillade against the three-bus convoy began to
emerge from their hiding places and regroup at the scene around 11 p.m.
The police had left by then, and the students sought to record the
evidence of the attack while trying to communicate with their classmates
in the other buses.

Image

Relatives of the students, with signs that read, ``We are missing 43.''
The students have not been seen since 2014.Credit...Marco
Ugarte/Associated Press

Journalists, as well as some teachers, began to show up, and by midnight
an impromptu news conference was taking shape in the middle of the road.

About 12:30 a.m., a white sport utility vehicle and a black car drove
by, their occupants taking photos of the gathering. Some were wearing
bulletproof vests and hoods. Some witnesses said they also had seen a
police car in the area.

Fifteen minutes later, the vehicles returned, and three men jumped out
and fired on the news conference from close range. Two young men were
killed, and other people, including students and teachers, were wounded.

The survivors fled into the surrounding blocks. A teacher and several
students ran to a clinic to find help for the wounded. No doctor was
present, but despite their appeals to emergency dispatchers and to
military personnel who appeared at the clinic, an ambulance did not
arrive for more than an hour.

As late as 3 a.m., the bodies of the two young men still lay in the
street, uncovered, in the pouring rain.

By dawn, the situation had calmed down, and the surviving students who
had been hiding across the city received word by telephone that it was
safe to come out. Over the course of the morning, they gathered at the
local offices of the attorney general, where they met with the
authorities.

That morning, the authorities also found the body of another student,
Julio César Mondragón, who had been at the news conference. He had fled
when the shooting began and had become separated from the group.

His facial skin and muscles had been torn away from his head, his skull
was fractured in several places, and his internal organs were ruptured.
His condition, the investigators wrote, ``shows the level of atrocities
committed that night.''

Advertisement

\protect\hyperlink{after-bottom}{Continue reading the main story}

\hypertarget{site-index}{%
\subsection{Site Index}\label{site-index}}

\hypertarget{site-information-navigation}{%
\subsection{Site Information
Navigation}\label{site-information-navigation}}

\begin{itemize}
\tightlist
\item
  \href{https://help.nytimes3xbfgragh.onion/hc/en-us/articles/115014792127-Copyright-notice}{©~2020~The
  New York Times Company}
\end{itemize}

\begin{itemize}
\tightlist
\item
  \href{https://www.nytco.com/}{NYTCo}
\item
  \href{https://help.nytimes3xbfgragh.onion/hc/en-us/articles/115015385887-Contact-Us}{Contact
  Us}
\item
  \href{https://www.nytco.com/careers/}{Work with us}
\item
  \href{https://nytmediakit.com/}{Advertise}
\item
  \href{http://www.tbrandstudio.com/}{T Brand Studio}
\item
  \href{https://www.nytimes3xbfgragh.onion/privacy/cookie-policy\#how-do-i-manage-trackers}{Your
  Ad Choices}
\item
  \href{https://www.nytimes3xbfgragh.onion/privacy}{Privacy}
\item
  \href{https://help.nytimes3xbfgragh.onion/hc/en-us/articles/115014893428-Terms-of-service}{Terms
  of Service}
\item
  \href{https://help.nytimes3xbfgragh.onion/hc/en-us/articles/115014893968-Terms-of-sale}{Terms
  of Sale}
\item
  \href{https://spiderbites.nytimes3xbfgragh.onion}{Site Map}
\item
  \href{https://help.nytimes3xbfgragh.onion/hc/en-us}{Help}
\item
  \href{https://www.nytimes3xbfgragh.onion/subscription?campaignId=37WXW}{Subscriptions}
\end{itemize}
