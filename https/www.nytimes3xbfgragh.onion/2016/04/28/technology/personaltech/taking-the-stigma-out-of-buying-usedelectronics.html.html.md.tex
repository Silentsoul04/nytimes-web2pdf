Sections

SEARCH

\protect\hyperlink{site-content}{Skip to
content}\protect\hyperlink{site-index}{Skip to site index}

\href{https://www.nytimes3xbfgragh.onion/section/technology/personaltech}{Personal
Tech}

\href{https://myaccount.nytimes3xbfgragh.onion/auth/login?response_type=cookie\&client_id=vi}{}

\href{https://www.nytimes3xbfgragh.onion/section/todayspaper}{Today's
Paper}

\href{/section/technology/personaltech}{Personal Tech}\textbar{}Taking
the Stigma Out of Buying Used Electronics

\href{https://nyti.ms/1N0v1jc}{https://nyti.ms/1N0v1jc}

\begin{itemize}
\item
\item
\item
\item
\item
\end{itemize}

Advertisement

\protect\hyperlink{after-top}{Continue reading the main story}

Supported by

\protect\hyperlink{after-sponsor}{Continue reading the main story}

\href{/column/tech-fix}{Tech Fix}

\hypertarget{taking-the-stigma-out-of-buying-used-electronics}{%
\section{Taking the Stigma Out of Buying Used
Electronics}\label{taking-the-stigma-out-of-buying-used-electronics}}

\includegraphics{https://static01.graylady3jvrrxbe.onion/images/2016/04/28/business/28TECHFIX/28TECHFIX-articleLarge.jpg?quality=75\&auto=webp\&disable=upscale}

By \href{http://www.nytimes3xbfgragh.onion/by/brian-x-chen}{Brian X.
Chen}

\begin{itemize}
\item
  April 27, 2016
\item
  \begin{itemize}
  \item
  \item
  \item
  \item
  \item
  \end{itemize}
\end{itemize}

WHEN you sell a used video game console to the retailer GameStop, it
goes on a long journey before winding up in the pre-owned section inside
a store.

The product travels to one of GameStop's repair facilities, like its
enormous refurbishment operations center in Grapevine, Tex., where it
undergoes a litany of tests. A used PlayStation, for example, would be
checked for every component, from the disc-loading tray to the laser
that reads the games. Any malfunctioning component would be replaced,
and with the console inspected, cleared of personal data and cleaned, it
becomes offered for sale.

``We have to somewhat be like a doctor and ask a lot of questions,''
said Bruce Kulp, GameStop's senior vice president for supply chain and
refurbishment. ``The worst thing in our world is to have something go
out that's been pre-owned and a consumer has an issue with it.''

GameStop's refurbishment of video game consoles underlines how a used
electronic sold by a reputable brand can often be as good as buying new.
While a used product may lack the original packaging or there might be
some scuffs on it, the quality of many of the devices remains high and
people who buy the gadgets do the world a favor by putting more use into
the energy, metals, plastics and human labor invested in creating the
product, said Carole Mars, the senior research lead for the
\href{https://www.sustainabilityconsortium.org/}{Sustainability
Consortium}, which studies the sustainability of consumer goods.

Those who buy used gadgets are also part of a trend of anti-consumerism,
which includes maintaining electronics to get
\href{http://www.nytimes3xbfgragh.onion/2016/04/21/technology/personaltech/choosing-to-skipthe-upgrade-and-care-for-the-gadget-youve-got.html}{more
use out of them for a longer period}, rather than discarding and
constantly upgrading to the latest products.

There are no firm numbers indicating how many Americans buy used
electronics. About 25 percent of GameStop's revenue last year consisted
of sales of used products, which was flat compared with the previous
year. Amazon said that more customers are shopping in its used products
store, though it declined to provide numbers. Many organizations dealing
with used electronics sell to those that cannot easily afford
technology, like public schools or economically disadvantaged countries.

Yet used electronics often face a bad rap --- people may lack trust in a
pre-owned product because it has been used by someone other than
themselves. To see whether such stigma is warranted, I compared the
pre-owned products programs of three retailers: Amazon, GameStop and
Gazelle. My takeaway is that you can buy pre-owned products from
reputable brands with as much confidence as you might buy a used car
from a certified dealer.

\hypertarget{the-programs}{%
\subsection{The Programs}\label{the-programs}}

Amazon's in-house program for pre-owned products is called
\href{http://www.warehousedeals.com}{Warehouse Deals}. The giant online
retailer sells used products in 25 categories, including televisions,
cameras, computers, kitchen gadgets and cellphones. Many of the items
come from customers who opened the packaging or used the goods and
returned them to Amazon.

Glenn Ramsdell, director of Amazon's Warehouse Deals, said every item
was checked by hand for its physical and functional condition. A
wireless speaker, for example, would be tested for its connectivity
features and checked for included accessories; repairs are made if
necessary.

Then the items get a grade. ``Like new'' means it was probably untouched
and in perfect condition; ``very good'' describes an item that was well
cared for and lightly used; a ``good'' item might show signs of wear and
tear but works perfectly; and ``acceptable'' would be something that has
cosmetic issues like scratches and dents but otherwise works.

The discounts vary in Warehouse Deals, but with a bit of time, people
can scout out some good deals. An Amazon 6-inch Kindle sells for
\href{http://www.amazon.com/Kindle-Glare-Free-Touchscreen-Display-Wi-Fi/dp/B00I15SB16}{\$60
brand new}, but in ``good'' condition it sells for about \$43.50, a
discount of almost 28 percent.

\href{http://www.gazelle.com}{Gazelle} offers cash for consumers'
pre-owned smartphones, Apple laptops and iPads. Before the products are
listed for sale, they go through a rigorous testing program similar to
GameStop's. The items are shipped to a processing center in Louisville,
Ky., and undergo what Gazelle calls a 30-point inspection, testing
everything from a phone's camera lens to its wireless connections.
Smartphone batteries need to have at least 80 percent of their capacity
remaining, otherwise they are replaced with new ones, the company said.

Dave Maquera, the president of Gazelle, said that similar to the
inspection programs used by certified pre-owned car dealers, Gazelle's
process creates a new level of confidence in buying used phones and
computers. Though he declined to provide specific numbers, he noted that
sales of used devices to consumers had jumped a double-digit percentage
compared to last year.

At GameStop, the product testing gets intense. Mr. Kulp, the supply
chain executive, said the company takes up to 100 game consoles a month,
refurbishes them and puts them through stress tests, running them for
thousands of hours to see if its repairs are long-lasting. These test
units are never sold to consumers.

``It's just like the way a car company would do a crash test,'' he said.

As for used video games, the company buffs out any light scratches from
the discs, but if the game is so deeply scratched that it becomes
unplayable, it heads to the shredder.

The trade-offs for buying used games at GameStop are fairly obvious. For
one, after a new game releases, you will have to wait awhile to buy it
used: In other words, you will be a late adopter. For another, the
product will probably lack its original packaging and there might be
light cosmetic wear. But for the average consumer, all that matters is
that your gaming experience will be exactly the same as if you had
bought the product new.

What if something goes wrong? Amazon and GameStop give customers 30 days
to return used products, the same amount of time they allow for returns
of new products. Gazelle offers a free 30-day warranty for each device
and the option to buy an extended warranty.

\hypertarget{bottom-line}{%
\subsection{Bottom Line}\label{bottom-line}}

Consumers should always consider checking out the used section of
retailers for most electronics, including smartphones, laptops and
desktop computers, said Dr. Mars of the Sustainability Consortium. In my
personal experience buying used video games, computers and even home
gadgets like vacuum cleaners from GameStop, Gazelle and Amazon, I have
never regretted buying used items.

But Dr. Mars said to beware of buying used televisions and computer
monitors, since a lot more can go wrong with larger screens than with
computer equipment. The jury is also still out on buying used wearable
devices like Fitbit trackers because they make up a fairly new category,
she added.

Even if you choose not to buy used, the best thing you can do is sell
your gadget as soon as you stop using it so that someone else can give
it some love. (I tested several
\href{https://www.google.com/webhp?sourceid=chrome-instant\&ion=1\&espv=2\&ie=UTF-8\#q=brian\%20chen\%20used\%20electronics}{trade-in
and recycling services} last year, and found all of them to be
headache-free.) If you procrastinate, the product can get too old,
meaning the retailers won't be able to resell it, so it may end up in a
shredder, Dr. Mars said.

``Turn over that old device so that somebody can get a second life out
of it,'' she said. ``There's no reason for it to go into a drawer.''

Advertisement

\protect\hyperlink{after-bottom}{Continue reading the main story}

\hypertarget{site-index}{%
\subsection{Site Index}\label{site-index}}

\hypertarget{site-information-navigation}{%
\subsection{Site Information
Navigation}\label{site-information-navigation}}

\begin{itemize}
\tightlist
\item
  \href{https://help.nytimes3xbfgragh.onion/hc/en-us/articles/115014792127-Copyright-notice}{©~2020~The
  New York Times Company}
\end{itemize}

\begin{itemize}
\tightlist
\item
  \href{https://www.nytco.com/}{NYTCo}
\item
  \href{https://help.nytimes3xbfgragh.onion/hc/en-us/articles/115015385887-Contact-Us}{Contact
  Us}
\item
  \href{https://www.nytco.com/careers/}{Work with us}
\item
  \href{https://nytmediakit.com/}{Advertise}
\item
  \href{http://www.tbrandstudio.com/}{T Brand Studio}
\item
  \href{https://www.nytimes3xbfgragh.onion/privacy/cookie-policy\#how-do-i-manage-trackers}{Your
  Ad Choices}
\item
  \href{https://www.nytimes3xbfgragh.onion/privacy}{Privacy}
\item
  \href{https://help.nytimes3xbfgragh.onion/hc/en-us/articles/115014893428-Terms-of-service}{Terms
  of Service}
\item
  \href{https://help.nytimes3xbfgragh.onion/hc/en-us/articles/115014893968-Terms-of-sale}{Terms
  of Sale}
\item
  \href{https://spiderbites.nytimes3xbfgragh.onion}{Site Map}
\item
  \href{https://help.nytimes3xbfgragh.onion/hc/en-us}{Help}
\item
  \href{https://www.nytimes3xbfgragh.onion/subscription?campaignId=37WXW}{Subscriptions}
\end{itemize}
