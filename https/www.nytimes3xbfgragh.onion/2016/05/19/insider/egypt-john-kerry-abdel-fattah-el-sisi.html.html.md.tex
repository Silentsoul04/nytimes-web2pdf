Sections

SEARCH

\protect\hyperlink{site-content}{Skip to
content}\protect\hyperlink{site-index}{Skip to site index}

\href{https://www.nytimes3xbfgragh.onion/section/reader-center}{Times
Insider}

\href{https://myaccount.nytimes3xbfgragh.onion/auth/login?response_type=cookie\&client_id=vi}{}

\href{https://www.nytimes3xbfgragh.onion/section/todayspaper}{Today's
Paper}

\href{/section/reader-center}{Times Insider}\textbar{}Kerry Meets
Egypt's Leader, and Where Are Reporters? Corralled at the Airport

\url{https://nyti.ms/250xY82}

\begin{itemize}
\item
\item
\item
\item
\item
\item
\end{itemize}

Advertisement

\protect\hyperlink{after-top}{Continue reading the main story}

Supported by

\protect\hyperlink{after-sponsor}{Continue reading the main story}

\hypertarget{kerry-meets-egypts-leader-and-where-are-reporters-corralled-at-the-airport}{%
\section{Kerry Meets Egypt's Leader, and Where Are Reporters? Corralled
at the
Airport}\label{kerry-meets-egypts-leader-and-where-are-reporters-corralled-at-the-airport}}

By \href{http://www.nytimes3xbfgragh.onion/by/david-e-sanger}{David E.
Sanger}

\begin{itemize}
\item
  May 18, 2016
\item
  \begin{itemize}
  \item
  \item
  \item
  \item
  \item
  \item
  \end{itemize}
\end{itemize}

\href{http://www.nytimes3xbfgragh.onion/section/insider}{\emph{Times
Insider}} \emph{delivers behind-the-scenes insights into how news,
features and opinion come together at The New York Times. In this piece,
David E. Sanger, a national security correspondent, tells why he didn't
travel to Egypt to cover Secretary of State John Kerry's visit with
President Abdel Fattah el-Sisi.}

\includegraphics{https://static01.graylady3jvrrxbe.onion/images/2016/05/19/world/middleeast/19Egypt-web/19Egypt-web-articleLarge.jpg?quality=75\&auto=webp\&disable=upscale}

VIENNA --- When Secretary of State John Kerry traveled on Wednesday to
the presidential palace in Egypt, ostensibly the United States' closest
Arab ally and the recipient of billions of dollars in American aid, the
reporters traveling with him were left behind at the airport.

The Egyptian authorities made clear, State Department officials said,
that reporters like myself would not be welcome at the presidential
palace, apparently out of fear we might ask President Abdel Fattah
el-Sisi a question or two about the dissidents he has jailed.

Instead, only photographers and videographers would be permitted inside,
to record the ritual handshakes between Mr. Kerry and Mr. Sisi ---
images that will suggest all is right between the authoritarian former
general who leads Egypt and the nation that provides him with F-16s,
helicopters and tanks, as well as riot control equipment that human
rights groups say may be used to facilitate Egypt's crackdown against
activists.

In fact, this meeting comes against a backdrop of increasing tension
over Washington's \$1.3 billion in annual military aid to Egypt, which
in recent years has been dwarfed by inflows to the country from Saudi
Arabia and the United Arab Emirates.

The State Department has lately been unable to obtain answers to its
questions about whether any equipment bought with American dollars has
been used by Egyptians to violate human rights, according to a
blistering 77-page \href{http://www.gao.gov/products/GAO-16-435}{report}
issued last week by the Government Accountability Office.

After more than two decades of traveling with American presidents and
chief diplomats --- on visits to places that have included some of the
world's most repressive nations --- I am used to watching leaders
disappear behind closed doors. But not even being allowed to see the
doors close sets something of a new standard.

State Department officials said that once they learned reporters would
be barred from the palace, they never even requested visas for us so we
could, as is customary, follow Mr. Kerry into the capital for his two-
to three-hour visit. Instead, those who made the long trip were stuck in
a lounge at the airport.

I decided to stay behind in Vienna and meet up with the secretary's
entourage this weekend in Myanmar, where he is meeting with Daw Aung San
Suu Kyi, who has emerged from a long house arrest to become a leader of
the new government. (There have been hints that we will have some
access.)

The contrast between Myanmar, once one of the world's most closed
societies, and Egypt made me revisit my recent travels with Mr. Kerry in
terms of what restrictions were placed on us journalists.

In November, Mr. Kerry zipped through Central Asia on a tour of some of
the world's most repressive states, including Turkmenistan, whose
leadership shares Mr. Sisi's approach to anyone who utters a thought the
government finds distasteful. Still, President Gurbanguly
Berdymukhammedov allowed international
\href{http://www.nytimes3xbfgragh.onion/2015/11/04/world/asia/john-kerry-confronts-human-rights-as-he-zips-through-central-asia.html?_r=0}{reporters}
to record his encounters with Mr. Kerry, though local Turkmen
journalists were kept at a far remove.

The King of Bahrain, who knows a thing or two about clearing the streets
of critics,
\href{http://www.nytimes3xbfgragh.onion/2016/04/08/world/middleeast/year-after-iran-nuclear-deal-kerry-confronts-concerns-of-arab-states.html}{invited
reporters in} for the start of his meeting with Mr. Kerry last month,
and, with a deep understanding of how to keep them docile, fed them at
the palace before they were packed off.

Even China's leaders routinely let the news media pool in, though they
do their best to ignore them.

Egypt used to do the same --- in what now looks, by comparison, like the
days of openness when Hosni Mubarak was still president.

Mr. Mubarak was not known for tolerating much criticism before he was
deposed, and would never have won an award from Transparency
International. But he let reporters record his meeting with President
Obama in 2009, when Mr. Obama made a trip to Cairo for a landmark speech
about the future of the Arab world that reads today like a trail of
broken dreams.

Even as the aging Mr. Mubarak fell from power --- pushed partly by Mr.
Obama in phone calls during the Tahrir Square protests --- he showed
some savvy about the Western news media. During one of his last visits
to Washington, Mr. Mubarak invited me and afew other reporters to his
hotel suite to talk about the Middle East's future --- and about his
differences with the Obama administration.

For those of us trying to divine the day-to-day workings of American
foreign policy, there are certain accepted rituals that make up the
everyday fabric of diplomatic reporting.

The journalists on Mr. Kerry's small plane --- an older Boeing 757 that
he despises --- usually depart from the rear door, hop into a van that
joins the back of his motorcade, and catch a glimpse of the start of
most meetings. Sometimes, after these private chats, we are ferried back
into a ceremonial room to ask questions, even uncomfortable ones. On
rare occasions, we even receive answers.

Tuesday in Vienna, where Mr. Kerry held talks on Syria, was typical: Two
old American adversaries, happily or not, subjected themselves to our
inquiries.

First came Foreign Minister Sergey V. Lavrov of Russia, who answered,
and avoided, questions at a half-hour news conference. Then Mr. Kerry
traveled to the Coburg Palace, where he had negotiated the final stages
of the Iran nuclear deal for a month last summer, to see Iran's foreign
minister, Mohammad Javad Zarif. I arrived there before the secretary,
and Mr. Zarif came over for a handshake and a few jokes. He then
answered a question about Iran's complaints that companies and banks
around the world are not returning to Tehran now that many sanctions
have been lifted.

Mr. Sisi took power in Egypt in 2013, after what was viewed by many
around the world as a coup against Mohamed Morsi --- who had been
elected after Mr. Mubarak's ouster --- and led a crackdown in which more
than 1,000 Egyptians were killed and thousands more imprisoned.

After the meeting on Wednesday, Mr. Kerry's spokesman, Mark Toner,
emailed State Department reporters with an official ``readout.'' The two
men had ``discussed a range of bilateral and regional issues, including
recent developments in Libya and Syria,'' the statement said. Mr. Kerry
had ``also stressed the importance of Egypt's role as a regional partner
and reiterated U.S. commitment to help Egypt fight terrorism, increase
economic growth, strengthen democratic institutions and bolster regional
security.''

Such anodyne descriptions of these meetings, making no mention of the
hard questions in the relationship, are exactly why reporters are
expected to press American and foreign officials.

But Mr. Sisi's government seems to be avoiding questions even from its
American benefactors.

In its report, the Government Accountability Office said the Defense
Department had been unable to conduct ``end-use monitoring'' on missiles
and night-vision devices before 2015.

And even though the checks were done in 2015, the report said, the State
Department had been unable to vet the training and equipping of the
military for possible human rights violations.

``State deemed G.A.O.'s estimate of the percentage of Egyptian security
forces that were not vetted to be sensitive but unclassified
information, which is excluded from this public report,'' it said. That
made it difficult to know if the military aid the United States was now
giving Egypt violated the so-called Leahy amendment, which links human
rights performance to the delivery of American security equipment and
training.

I would have loved to ask Mr. Kerry and Mr. Sisi about that estimate at
the presidential palace on Wednesday. If only reporters had been allowed
in.

Advertisement

\protect\hyperlink{after-bottom}{Continue reading the main story}

\hypertarget{site-index}{%
\subsection{Site Index}\label{site-index}}

\hypertarget{site-information-navigation}{%
\subsection{Site Information
Navigation}\label{site-information-navigation}}

\begin{itemize}
\tightlist
\item
  \href{https://help.nytimes3xbfgragh.onion/hc/en-us/articles/115014792127-Copyright-notice}{©~2020~The
  New York Times Company}
\end{itemize}

\begin{itemize}
\tightlist
\item
  \href{https://www.nytco.com/}{NYTCo}
\item
  \href{https://help.nytimes3xbfgragh.onion/hc/en-us/articles/115015385887-Contact-Us}{Contact
  Us}
\item
  \href{https://www.nytco.com/careers/}{Work with us}
\item
  \href{https://nytmediakit.com/}{Advertise}
\item
  \href{http://www.tbrandstudio.com/}{T Brand Studio}
\item
  \href{https://www.nytimes3xbfgragh.onion/privacy/cookie-policy\#how-do-i-manage-trackers}{Your
  Ad Choices}
\item
  \href{https://www.nytimes3xbfgragh.onion/privacy}{Privacy}
\item
  \href{https://help.nytimes3xbfgragh.onion/hc/en-us/articles/115014893428-Terms-of-service}{Terms
  of Service}
\item
  \href{https://help.nytimes3xbfgragh.onion/hc/en-us/articles/115014893968-Terms-of-sale}{Terms
  of Sale}
\item
  \href{https://spiderbites.nytimes3xbfgragh.onion}{Site Map}
\item
  \href{https://help.nytimes3xbfgragh.onion/hc/en-us}{Help}
\item
  \href{https://www.nytimes3xbfgragh.onion/subscription?campaignId=37WXW}{Subscriptions}
\end{itemize}
