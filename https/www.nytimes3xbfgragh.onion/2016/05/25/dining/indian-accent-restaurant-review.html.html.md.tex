Sections

SEARCH

\protect\hyperlink{site-content}{Skip to
content}\protect\hyperlink{site-index}{Skip to site index}

\href{https://www.nytimes3xbfgragh.onion/section/food}{Food}

\href{https://myaccount.nytimes3xbfgragh.onion/auth/login?response_type=cookie\&client_id=vi}{}

\href{https://www.nytimes3xbfgragh.onion/section/todayspaper}{Today's
Paper}

\href{/section/food}{Food}\textbar{}Indian Accent Has a Cosmopolitan
Twang

\url{https://nyti.ms/1s7m6TN}

\begin{itemize}
\item
\item
\item
\item
\item
\item
\end{itemize}

Advertisement

\protect\hyperlink{after-top}{Continue reading the main story}

Supported by

\protect\hyperlink{after-sponsor}{Continue reading the main story}

\href{/column/restaurant-review}{Restaurant Review}

\hypertarget{indian-accent-has-a-cosmopolitan-twang}{%
\section{Indian Accent Has a Cosmopolitan
Twang}\label{indian-accent-has-a-cosmopolitan-twang}}

\href{https://www.nytimes3xbfgragh.onion/slideshow/2016/05/25/dining/indian-accent-restaurant.html}{}

\hypertarget{indian-accent}{%
\subsection{Indian Accent}\label{indian-accent}}

11 Photos

View Slide Show ›

\includegraphics{https://static01.graylady3jvrrxbe.onion/images/2016/05/25/dining/INDIAN-ACCENT-crab/25REST-INDIANACCENTS-slide-WKU5-articleLarge.jpg?quality=75\&auto=webp\&disable=upscale}

Benjamin Petit for The New York Times

\begin{itemize}
\tightlist
\item
  Indian Accent\\
  ★★ Indian \$\$\$\$ 123 West 56th Street 212-842-8070
\end{itemize}

\href{http://www.opentable.com/single.aspx?ref=4201\&rid=193822}{Reserve
a Table}

When you make a reservation at an independently reviewed restaurant
through our site, we earn an affiliate commission.

By \href{http://www.nytimes3xbfgragh.onion/by/pete-wells}{Pete Wells}

\begin{itemize}
\item
  May 24, 2016
\item
  \begin{itemize}
  \item
  \item
  \item
  \item
  \item
  \item
  \end{itemize}
\end{itemize}

The emails from New Delhi began last spring.

A new one arrived every few months, informing me that a branch of a
restaurant in that city named
\href{http://www.indianaccent.com/newyork/index.html}{Indian Accent} was
about to open in Midtown, and urging me to check it out.

``Dear and Most Revered Mr. Pete,'' one began. ``Just wanted to let you
know about the greatest Indian restaurant in the World.''

Another praised
\href{http://www.indianaccent.com/newdelhi/index.html}{the original
location}'s kulchas, or stuffed flatbreads, particularly one filled with
duck and hoisin. ``It's as Indian as
\href{http://www.bravotv.com/people/padma-lakshmi}{Padma Lakshmi},'' my
correspondent wrote. ``But definitely more delicious.''

These notes were my first exposure to the cult inspired by
\href{https://www.youtube.com/watch?v=M-lu29IkZ1w}{Manish Mehrotra}.
Since opening Indian Accent in a small hotel on the outskirts of New
Delhi in 2009, he has become one of the most admired chefs in India. Mr.
Mehrotra's style is lightened by modern influences from abroad (he has
trained in other Asian countries and worked in London) but anchored by
the culture he grew up in; at his New Delhi restaurant, a scoop of kulfi
may be served in a toy pressure cooker, an appliance roughly as common
in Indian homes as microwaves are in American ones.

In India, Mr. Mehrotra's cooking feeds a desire to see traditional food
bent, twisted and played with until it tastes new again. New York could
use more of that, too.
\href{http://www.nytimes3xbfgragh.onion/2011/03/30/dining/reviews/30rest.html?version=meter+at+0\&module=meter-Links\&pgtype=Blogs\&contentId=\&mediaId=\&referrer=http\%3A\%2F\%2Fquery.nytimes3xbfgragh.onion\%2Fsearch\%2Fsitesearch\%2F\%3Faction\%3Dclick\%26contentCollection\%26region\%3DTopBar\%26WT.nav\%3DsearchWidget\%26module\%3DSearchSubmit\%26pgtype\%3DHomepage\&priority=true\&action=click\&contentCollection=meter-links-click}{Junoon}
and
\href{http://www.nytimes3xbfgragh.onion/2010/08/04/dining/reviews/04rest.html}{Tamarind
Tribeca} are on the short list of comfortable, modern Indian restaurants
where you can get dressed up for food that looks pretty on the plate.
But surprises, ladled out by an active imagination working in an Indian
mode, have been scarce since
\href{http://dinersjournal.blogs.nytimes3xbfgragh.onion/2010/09/30/tabla-is-closing/}{Tabla
closed} in 2010.

Indian Accent, which finally opened in February at the foot of
\href{http://www.parkermeridien.com/home/}{Le Parker Meridien Hotel},
has surprises. It also has all the comforts of a restaurant where the
prix fixe menu offers three courses at \$75 (or \$90 for four). Except
for one gold wall at the far end of the restaurant, the interior stays
away from anything that may play into preconceptions of how Indian
restaurants are supposed to look.

At the sharp-edged marble bar, cocktails are infused with tea and spices
and made with a care that overcomes goofy names like Transcendental
Medication. Tables are set with stiffly pressed napkins and thin-stemmed
wineglasses --- a cue to pay attention to Daniel Beedle's list, which is
well stocked with aromatic whites and savory reds that know how to get
along with layered spices.

Mr. Mehrotra shares the title of chef with Vivek Rana, who will
eventually take charge of the kitchen. Duck kulcha isn't on their menu,
but the kulcha stuffed with pastrami and mustard helped me see why my
New Delhi informants had been so excited.

``The chef got the idea for this dish from Peking duck,'' our server
said as he set down a long wooden tray with a copper pan of shredded
lamb at one end, a stack of thin griddled flatbreads called rumali roti
in the other, with cucumber spears and four kinds of chutney in the
middle. One of those was hoisin sauce subcontinentalized with tamarind,
which did bring Peking within shouting distance. But as I folded a roti
around some soft spiced lamb and spooned on one chutney or all four ---
they were great separately and great together --- I kept thinking about
tacos.

Letting China, Mexico and the Carnegie Deli barge into an Indian
restaurant without having the dinner turn into a chaotic grab bag is an
impressive feat. And fried squid sprinkled with puffed rice and
chickpea-flour threads to turn up the crunch factor? Well, why not,
especially when the squid is dusted with a spice blend that could make a
tongue depressor taste good.

I don't know how to categorize the ``sweet pickle ribs.'' They are not,
in fact, pickled, but, in their tart mango sauce with strips of dried
mango on top, these tender baby backs are so good I'd eat them under any
name.

The soy keema is a marvelous thing, too. This version is less like the
original keema, a stew of ground lamb and peas, than it is like an
energetically spiced filling for a vegetarian sloppy Joe. I'm seeing a
takeout window with lines outside on 57th Street. It's possible I got
the whole idea from the delicious tiny rolls, perfumed with lime leaves,
that accompany the keema on a skewer, looking like marshmallows ready
for a campfire.

Mixed in with the outstanding dishes are some that won't inspire many
international emails. Kolhapuri chicken, a chile-fueled curry from
western India, appears as an appetizer of cold chicken salad that you
could feed to the least adventurous eater you know. Fried shiso leaves
looked impressive standing upright in a pile of potatoes and water
chestnuts drizzled with chutney, but the batter was too thick for the
herb to have much impact.

Indian Accent is a young restaurant, still learning to transplant an
approach that worked in New Delhi to Manhattan. It can take time for
chefs to sort out their suppliers in a new region. Fillets of sea bass
glazed with tamarind had a muddy, bottom-feeder taste that we don't
usually seek in wild saltwater fish.

The menu is less confusing than it was at first, although the
crossbreeding of a prix fixe setup with à la carte supplements can still
be awkward. The lamb, for instance, costs an extra \$38. The price is
fair enough for a platter that is bigger than many main courses around
town, but I wish it didn't have to be stapled to the already
considerable cost of a meal.

The complexity of deciding what or how to order is one reason servers
are prone to waitsplaining. Dinner requires a lot of patient listening
and smiling in these early months. True, the menu is paved with terms
some New Yorkers will trip over. I was happy to learn that the sweet
potato shakarkandi are tender cubes stacked over shaved kohlrabi in one
of the best small starters.

But I'm always impatient to start eating, especially if what's on the
way is a dessert as wonderful as the makhan malai. Traditionally a
street snack, here it is a fluffy mound of aerated saffron milk
sprinkled with rose petals, almonds and palm sugar.

The fun of the dessert is in the way these crystallized toppings
transform the unsweetened saffron milk once everything meets inside your
mouth. It's the kind of happy collision that few restaurants in town can
deliver as well as Indian Accent.

Advertisement

\protect\hyperlink{after-bottom}{Continue reading the main story}

\hypertarget{site-index}{%
\subsection{Site Index}\label{site-index}}

\hypertarget{site-information-navigation}{%
\subsection{Site Information
Navigation}\label{site-information-navigation}}

\begin{itemize}
\tightlist
\item
  \href{https://help.nytimes3xbfgragh.onion/hc/en-us/articles/115014792127-Copyright-notice}{©~2020~The
  New York Times Company}
\end{itemize}

\begin{itemize}
\tightlist
\item
  \href{https://www.nytco.com/}{NYTCo}
\item
  \href{https://help.nytimes3xbfgragh.onion/hc/en-us/articles/115015385887-Contact-Us}{Contact
  Us}
\item
  \href{https://www.nytco.com/careers/}{Work with us}
\item
  \href{https://nytmediakit.com/}{Advertise}
\item
  \href{http://www.tbrandstudio.com/}{T Brand Studio}
\item
  \href{https://www.nytimes3xbfgragh.onion/privacy/cookie-policy\#how-do-i-manage-trackers}{Your
  Ad Choices}
\item
  \href{https://www.nytimes3xbfgragh.onion/privacy}{Privacy}
\item
  \href{https://help.nytimes3xbfgragh.onion/hc/en-us/articles/115014893428-Terms-of-service}{Terms
  of Service}
\item
  \href{https://help.nytimes3xbfgragh.onion/hc/en-us/articles/115014893968-Terms-of-sale}{Terms
  of Sale}
\item
  \href{https://spiderbites.nytimes3xbfgragh.onion}{Site Map}
\item
  \href{https://help.nytimes3xbfgragh.onion/hc/en-us}{Help}
\item
  \href{https://www.nytimes3xbfgragh.onion/subscription?campaignId=37WXW}{Subscriptions}
\end{itemize}
