Sections

SEARCH

\protect\hyperlink{site-content}{Skip to
content}\protect\hyperlink{site-index}{Skip to site index}

\href{https://www.nytimes3xbfgragh.onion/section/fashion}{Fashion}

\href{https://myaccount.nytimes3xbfgragh.onion/auth/login?response_type=cookie\&client_id=vi}{}

\href{https://www.nytimes3xbfgragh.onion/section/todayspaper}{Today's
Paper}

\href{/section/fashion}{Fashion}\textbar{}The End of the Office Dress
Code

\url{https://nyti.ms/25gjYr1}

\begin{itemize}
\item
\item
\item
\item
\item
\end{itemize}

Advertisement

\protect\hyperlink{after-top}{Continue reading the main story}

Supported by

\protect\hyperlink{after-sponsor}{Continue reading the main story}

\href{/column/unbuttoned}{Unbuttoned}

\hypertarget{the-end-of-the-office-dress-code}{%
\section{The End of the Office Dress
Code}\label{the-end-of-the-office-dress-code}}

\includegraphics{https://static01.graylady3jvrrxbe.onion/images/2016/05/26/fashion/26UNBUTTONED--WEB1/26UNBUTTONED--WEB1-articleLarge.jpg?quality=75\&auto=webp\&disable=upscale}

By \href{https://www.nytimes3xbfgragh.onion/by/vanessa-friedman}{Vanessa
Friedman}

\begin{itemize}
\item
  May 25, 2016
\item
  \begin{itemize}
  \item
  \item
  \item
  \item
  \item
  \end{itemize}
\end{itemize}

Over the weekend an exhibition opened at the Museum at the Fashion
Institute of Technology in New York. Entitled ``Uniformity,'' it
displays 71 pieces from the museum's collection of (surprise) uniforms,
divided into four categories --- military, work, school, sports --- as
well as a select group of the fashion looks they influenced, like
Geoffrey Beene's 1967 sequined football jersey gown and Rei Kawakubo's
1998 military vest and pleated skirt for Comme des Garçons.

``I was interested in the inherent dichotomy between uniforms and
fashion,'' said Emma McClendon, assistant curator of costume, and the
organizer of the exhibition, ``because while they should be antithetical
to one another --- the first is about conformity, the second about
creativity --- they are also deeply interrelated. It's ironic.''

But not as ironic as the fact that the show opens just as a number of
recent disputes have underscored a somewhat different, and disruptive,
reality. We live in a moment in which the notion of a uniform is
increasingly out of fashion, at least when it comes to the implicit
codes of professional and public life. Indeed, the museum may be the
only place they now make sense.

If once upon a time Melanie Griffith's character in ``Working Girl''
could manipulate viewers' assumptions about her job and background
simply by swapping leather jackets and minidresses for greige suits,
today it would be impossible. ``We are in a very murky period,'' Ms.
McClendon said.

Just before the museum's show opened, for example, Britain was
momentarily distracted from discussions over Brexit (leaving the
European Union) by the news that Nicola Thorp, a temp worker, had been
sent home from her receptionist job at PricewaterhouseCoopers for
refusing to wear heels, as dictated by the dress code of her agency,
Portico.

She took her cause public, starting a
\href{https://petition.parliament.uk/petitions/129823}{petition} for a
parliamentary hearing titled ``Make it illegal for a company to require
women to wear high heels at work.'' If you get more than 100,000
signatures, Parliament will consider the petition, and as of Tuesday
afternoon she had 140,712.

\includegraphics{https://static01.graylady3jvrrxbe.onion/images/2016/05/26/fashion/26UNBUTTONED--WEB2/26UNBUTTONED--WEB2-articleLarge.jpg?quality=75\&auto=webp\&disable=upscale}

Almost immediately, ITV, the British television network, conducted a
\href{https://twitter.com/itvnews/status/730622484815314945}{poll} on
whether employers should be allowed to require women to wear heels;
\href{https://www.facebookcorewwwi.onion/StylistMagazine/videos/10155275808519572/?pnref=story}{social
media} freaked out; and Portico announced it had changed its
\href{http://www.bbc.com/news/uk-england-london-36272893}{policy}: Flats
were now acceptable for women (men, of course, could always wear them).

A few days later,
\href{http://www.nytimes3xbfgragh.onion/2016/05/17/style/meteorologist-ktla-dress-cover-up.html}{sweatergate}
broke out in the United States when a weather forecaster on KTLA-TV in
Los Angeles was handed a gray sweater to cover up a tank dress she was
wearing on the air. She said it was a joke, courtesy of her co-anchor,
but Twitter took offense, perceiving it as an attempt to control what
women wear.

All of this follows famous dress code brouhahas like the
\href{http://www.bbc.com/news/business-12207296}{UBS scandal} of 2010
when the Internet discovered that the Swiss bank had issued a 44-page
booklet of guidelines for employee dress that included instructions on
shoulder width and underwear shade.

Then there was the ``flat shoe'' uproar of 2015, when two women were
supposedly barred from the red carpet in Cannes for not wearing heels.
(The festival director denied the report on Twitter.)

And earlier this year, Kansas State Senator Mitch Holmes was forced to
issue a public apology for having included, in his
\href{http://cjonline.com/news/2016-01-21/kansas-senate-chairmans-rules-block-female-witnesses-revealing-attire}{guidelines}
for the Senate Ethics and Elections Committee, which he chaired, a rule
for those appearing before the state panel that read: ``Conferees should
be dressed in professional attire. For ladies, low-cut necklines and
miniskirts are inappropriate.'' No such specific guidelines were issued
for men. Oops. This did not sit well with many.

``I have decided to retract the conferee guidelines,'' he said later in
a statement, which also noted, ``My failure to clearly specify that all
conferees, regardless of gender, should strive to present themselves
professionally is unacceptable.''

Image

Mary Tyler Moore is shown as TV news producer Mary Richards in a scene
from the ``The Mary Tyler Moore Show'' in the 1970's.Credit...Associated
Press

The slippery slope may have started as a gentle incline way back in the
1970s, and become a bit steeper during the Casual Friday movement of the
1990s and the success of the Facebook I.P.O. in 2012 with its
hoodie-wearing billionaires. But today, we are speeding down it at
breakneck pace, partly thanks to the hot-button conversation around
gender equality, and fluidity.

``There has been a dramatic change very recently,'' said Susan Scafidi,
a law professor at Fordham University and founder of the Fashion Law
Institute.

She noted that last December the New York City Commission on Human
Rights announced
\href{http://www1.nyc.gov/office-of-the-mayor/news/961-15/nyc-commission-human-rights-strong-protections-city-s-transgender-gender}{new
guidelines} for the municipal human rights law that expressly prohibited
``enforcing dress codes, uniforms, and grooming standards that impose
different requirements based on sex or gender.''

As a result, no employer may require men to wear ties unless they also
require women to wear ties, or ask that heels be worn unless both sexes
have to wear them. And though this applies only to ``official'' dress
codes, the trickle-down effect is inevitable.

``Dress is now open to the interpretation of the individual, rather than
an institution,'' Professor Scafidi said.

This has created an even greater tension in the more ambiguous areas of
office dress, especially as the boundaries between home and work become
ever blurrier.

Image

Calista Flockhart and Gil Bellows in the Fox television series, ``Ally
McBeal''.Credit...Larry Watson/Fox, via Associated Press

``There's a strain of thought that says an employee represents a
company, and thus dress is not about personal expression, but company
expression,'' Professor Scafidi said. ``But there's a counterargument
that believes because we identify so much with our careers, we should be
able to be ourselves at work.''

And that has led to all sorts of complications. One person's
``appropriate'' can easily be another's ``disgraceful,'' and words like
``professional,'' when used to describe dress requirements, can seem so
vague as to be almost meaningless. Kanye West wearing ripped jeans and a
jeweled Balmain jacket at the Met Gala: cool or rude? Julia Roberts at
the premiere of ``Money Monster'' at Cannes this year in bare feet: red
carpet pioneer or a step too far?

At The New York Times, Michael Golden, the vice chairman, told me: ``We
have customer-facing jobs and those that are principally internal. We
ask employees to dress appropriately for the interactions planned for
their day.'' But that can have broad interpretations. In the newsroom,
people show up in everything from double-breasted suits to shorts; from
sneakers and Birkenstocks to platform heels.

All of which leaves us where? Confused, mostly. And fast trying to
create our own codes, or parse those of the offices around us. Mark
Zuckerberg, for example, is on the record as saying he wears the same
gray T-shirt every day so that he can focus his energy on other
decisions.

Ms. McClendon acknowledges that she tends to wear ``all black, pretty
much every day, and sculptural shapes --- it's the museum uniform.''
Professor Scafidi said, ``My business uniform is a black jacket with a
fitted, knee-length sheath, classic 100-millimeter single-sole pumps,
and usually our logo pin --- my equivalent of armor, arms and insignia,
respectively.''

Indeed, according to Ms. McClendon, uniforms evolved for a reason:
``They fulfill a need to identify your place in the world,'' for the
wearer and the observer. At least when they are easy to read. And part
of the idea behind the F.I.T. show, she said, was to ``put visitors in
the mind-set to consider uniform dressing more broadly, and how it
impacts their own lives.''

In other words, to live an examined life when it comes to your wardrobe
and your workplace. Because these issues are only going to get more
complicated.

``We are moving into an era where personal expression is going to trump
the desire to create a corporate identity,'' Professor Scafidi said.
``It's a huge power shift.'' And it has already begun.

Advertisement

\protect\hyperlink{after-bottom}{Continue reading the main story}

\hypertarget{site-index}{%
\subsection{Site Index}\label{site-index}}

\hypertarget{site-information-navigation}{%
\subsection{Site Information
Navigation}\label{site-information-navigation}}

\begin{itemize}
\tightlist
\item
  \href{https://help.nytimes3xbfgragh.onion/hc/en-us/articles/115014792127-Copyright-notice}{©~2020~The
  New York Times Company}
\end{itemize}

\begin{itemize}
\tightlist
\item
  \href{https://www.nytco.com/}{NYTCo}
\item
  \href{https://help.nytimes3xbfgragh.onion/hc/en-us/articles/115015385887-Contact-Us}{Contact
  Us}
\item
  \href{https://www.nytco.com/careers/}{Work with us}
\item
  \href{https://nytmediakit.com/}{Advertise}
\item
  \href{http://www.tbrandstudio.com/}{T Brand Studio}
\item
  \href{https://www.nytimes3xbfgragh.onion/privacy/cookie-policy\#how-do-i-manage-trackers}{Your
  Ad Choices}
\item
  \href{https://www.nytimes3xbfgragh.onion/privacy}{Privacy}
\item
  \href{https://help.nytimes3xbfgragh.onion/hc/en-us/articles/115014893428-Terms-of-service}{Terms
  of Service}
\item
  \href{https://help.nytimes3xbfgragh.onion/hc/en-us/articles/115014893968-Terms-of-sale}{Terms
  of Sale}
\item
  \href{https://spiderbites.nytimes3xbfgragh.onion}{Site Map}
\item
  \href{https://help.nytimes3xbfgragh.onion/hc/en-us}{Help}
\item
  \href{https://www.nytimes3xbfgragh.onion/subscription?campaignId=37WXW}{Subscriptions}
\end{itemize}
