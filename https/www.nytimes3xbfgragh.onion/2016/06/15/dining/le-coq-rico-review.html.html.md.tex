Sections

SEARCH

\protect\hyperlink{site-content}{Skip to
content}\protect\hyperlink{site-index}{Skip to site index}

\href{https://www.nytimes3xbfgragh.onion/section/food}{Food}

\href{https://myaccount.nytimes3xbfgragh.onion/auth/login?response_type=cookie\&client_id=vi}{}

\href{https://www.nytimes3xbfgragh.onion/section/todayspaper}{Today's
Paper}

\href{/section/food}{Food}\textbar{}Poultry Has a Pedigree at Le Coq
Rico

\url{https://nyti.ms/1UsGvgw}

\begin{itemize}
\item
\item
\item
\item
\item
\item
\end{itemize}

Advertisement

\protect\hyperlink{after-top}{Continue reading the main story}

Supported by

\protect\hyperlink{after-sponsor}{Continue reading the main story}

\href{/column/restaurant-review}{Restaurant Review}

\hypertarget{poultry-has-a-pedigree-at-le-coq-rico}{%
\section{Poultry Has a Pedigree at Le Coq
Rico}\label{poultry-has-a-pedigree-at-le-coq-rico}}

\href{https://www.nytimes3xbfgragh.onion/slideshow/2016/06/15/dining/le-coq-rico-nyc-restaurant-review.html}{}

\hypertarget{at-le-coq-rico-fowl-takes-center-stage}{%
\subsection{At Le Coq Rico, Fowl Takes Center
Stage}\label{at-le-coq-rico-fowl-takes-center-stage}}

12 Photos

View Slide Show ›

\includegraphics{https://static01.graylady3jvrrxbe.onion/images/2016/06/15/dining/15REST-LECOQRICO-slide-LQBK/15REST-LECOQRICO-slide-LQBK-articleLarge.jpg?quality=75\&auto=webp\&disable=upscale}

Sasha Maslov for The New York Times

\begin{itemize}
\tightlist
\item
  Le Coq Rico\\
  ★★ French \$\$\$ 30 East 20th Street 212-267-7426
\end{itemize}

\href{http://www.opentable.com/single.aspx?ref=4201\&rid=168787}{Reserve
a Table}

When you make a reservation at an independently reviewed restaurant
through our site, we earn an affiliate commission.

By \href{http://www.nytimes3xbfgragh.onion/by/pete-wells}{Pete Wells}

\begin{itemize}
\item
  June 14, 2016
\item
  \begin{itemize}
  \item
  \item
  \item
  \item
  \item
  \item
  \end{itemize}
\end{itemize}

I'm almost always the one who gets the chicken.

For review meals, I try to let my guests choose their own food. Then I
pick something nobody else wants. Often, this is the chicken. I think
this can be explained by a remark a friend made recently. ``I never eat
chicken in restaurants,'' he said. ``I can make chicken at home.''

So can I, but every time I go to a new restaurant, I hope the kitchen
may know things about poultry that I don't. Every once in a while, I'm
right.

At \href{http://www.lecoqriconyc.com/}{Le Coq Rico,} a three-month-old
restaurant on East 20th Street, I was right. In fact, I think I have
finally found the perfect restaurant to take people who think they can
make a better chicken at home.

We'd ordered an old New England breed of chicken called Plymouth Rock
for \$95, along with a guinea fowl that cost a dollar more. Carved and
fit back together, each bird was placed in the center of the table in
its own iron roasting pan. Our eyes locked in on the bronzed skin and
tapering curves of drumsticks with fixed and purposeful stares that, if
we had not been humans looking at poultry, I would call lust.

The meat had all the things I wanted and none of the things I didn't. It
was moist but not drippy or briny; compact and muscular but not tough;
long on deep, rounded flavor that didn't seem to rely on salt or sugar.

Some of my guests preferred the chicken, calling the guinea fowl
``sinewy.'' It was a bit stringy at the joints, but once disentangled,
the flesh had a flavor I found highly persuasive. Even the white meat
tasted like dark meat.

The menu suggested that one bird would feed up to four people. We nearly
demolished twice as many, along with a macaroni gratin and a bundle of
stout, dark fries that we dunked into a small pitcher of jus. After we
were too full to go on, we noticed an untouched chicken leg. One of my
guests ate that, too, down to the bone.

True, he did it on a bet. But I suspect he would have done it for
nothing.

The chef and owner is \href{https://twitter.com/ChefWestermann}{Antoine
Westermann}. His distillations of the cuisine of his native Alsace at
\href{http://www.buerehiesel.fr/EN_index.asp}{Le Buerehiesel} in
Strasbourg were widely praised before he handed it over to his son and
moved to Paris. One of the places Mr. Westermann opened there is the
original \href{http://en.lecoqrico.com/}{Coq Rico}, a theme restaurant
of sorts. Poultry (eggs, organs, broth) turns up in almost everything
the kitchen makes, most spectacularly in the whole chickens that are
first braised in chicken stock and then threaded on to a rotisserie.

He gave Le Coq Rico a subtitle that is both irresistible and accurate:
``The Bistro of Beautiful Birds.''

By March, when he opened a Manhattan branch a few doors down from
Gramercy Tavern, he had spent several months scouting for beautiful
American birds. Mr. Westermann is particular about how long they spend
pecking and strutting before they land on his rotisserie. The menu lists
the age at slaughter --- our server seemed to prefer ``harvest'' --- for
each breed on the menu, from 90 days for the Plymouth Rock to 130 days
for the guinea fowl.

``The industry standard is 40 days,'' the server said, apparently used
to fielding questions about these numbers. (Two more chicken breeds
listed on the menu, New Hampshire and Cornish, weren't available when I
went; they are still enjoying country living.)

A whole Brune Landaise (110 days), a French breed with milder flavor
than the Plymouth Rock, is simmered in a clay casserole with potatoes,
tomatoes, onions, artichokes, stock and a bottle of riesling. This, Mr.
Westermann's version of an Alsatian baeckeoffe, was terrific, with
bewitching undercurrents of spice in a sauce that had body but no
visible fat. The price, though, \$120, made me queasy.

A roasted quarter of the same breed, for \$24, is Le Coq Rico's offering
for the solo chicken eater. The half-breast and thigh I tried were moist
and flavorful but slightly inert, as if warmed over. If I came back
alone, maybe to sit at the counter facing the kitchen in a narrow alley
off the main dining room, I'd want the squab. Unwrapping the cabbage
leaf around it, cutting into the handsomely roasted bird surrounded by
foie gras stuffing, and dipping slices of excellent house-baked baguette
into the glossy dark sauce is a sure route to Francophilia.

Shopping for chickens, Mr. Westermann found eggs. Some go into Matthieu
Simon's desserts, one of which is awe-inspiring: the floating island, a
single grapefruit-size ball of meringue with a crackling sugar glaze on
top, resting in a pond of crème anglaise. The rhubarb soufflé is another
good, if more routine, showpiece for egg whites. I'd choose either over
the double-wide raspberry mille-feuille whose puff pastry wasn't crisp
enough to shatter.

A Coq Rico meal could start with eggs, too --- or, as the menu calls
this group of appetizers, ``Eggz.'' It's rare to find oeuf en meurette,
in all its rich, spoon-coating, lardon-studded glory, made this well in
Manhattan.

Served with a heap of mâche and spicy pink hummus, sautéed chicken
livers were glorious, combining the rich creaminess of a soft cheese
with the metallic tang of organ meat. I expected great things from the
chicken gizzards served with artichokes à la Barigoule, too, but their
flavor seemed to have leaked out somewhere.

From his days at Le Buerehiesel, Mr. Westermann has brought along the
recipe for foie gras terrine baked inside a soft pastry crust. I'm not
convinced this single slice of terrine is worth \$32, but I don't know
anywhere in New York to get a better, cheaper version, either.

The architect Pascal Desprez has slotted Le Coq Rico into a somewhat
awkward U-shaped space. By the entrance is a bar where the cocktail menu
promises, somewhat menacingly, ``the dark side of Le Coq Rico.''

There always seemed to be mild confusion at the host podium, and each
time I was led to my seat, I was afraid I'd be deposited at one of the
tables crammed into the narrow passageway. This hall opens in the back
into a stylishly monochromatic dining room.

Unless you have been to the Paris original, also designed by Mr.
Desprez, the only things about the atmosphere that will strike you as
French may be the peculiar pop music and, on some nights, Mr. Westermann
himself, with wire-framed glasses on a head full of thoughts about
chicken.

Advertisement

\protect\hyperlink{after-bottom}{Continue reading the main story}

\hypertarget{site-index}{%
\subsection{Site Index}\label{site-index}}

\hypertarget{site-information-navigation}{%
\subsection{Site Information
Navigation}\label{site-information-navigation}}

\begin{itemize}
\tightlist
\item
  \href{https://help.nytimes3xbfgragh.onion/hc/en-us/articles/115014792127-Copyright-notice}{©~2020~The
  New York Times Company}
\end{itemize}

\begin{itemize}
\tightlist
\item
  \href{https://www.nytco.com/}{NYTCo}
\item
  \href{https://help.nytimes3xbfgragh.onion/hc/en-us/articles/115015385887-Contact-Us}{Contact
  Us}
\item
  \href{https://www.nytco.com/careers/}{Work with us}
\item
  \href{https://nytmediakit.com/}{Advertise}
\item
  \href{http://www.tbrandstudio.com/}{T Brand Studio}
\item
  \href{https://www.nytimes3xbfgragh.onion/privacy/cookie-policy\#how-do-i-manage-trackers}{Your
  Ad Choices}
\item
  \href{https://www.nytimes3xbfgragh.onion/privacy}{Privacy}
\item
  \href{https://help.nytimes3xbfgragh.onion/hc/en-us/articles/115014893428-Terms-of-service}{Terms
  of Service}
\item
  \href{https://help.nytimes3xbfgragh.onion/hc/en-us/articles/115014893968-Terms-of-sale}{Terms
  of Sale}
\item
  \href{https://spiderbites.nytimes3xbfgragh.onion}{Site Map}
\item
  \href{https://help.nytimes3xbfgragh.onion/hc/en-us}{Help}
\item
  \href{https://www.nytimes3xbfgragh.onion/subscription?campaignId=37WXW}{Subscriptions}
\end{itemize}
