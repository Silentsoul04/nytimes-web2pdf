Sections

SEARCH

\protect\hyperlink{site-content}{Skip to
content}\protect\hyperlink{site-index}{Skip to site index}

\href{https://www.nytimes3xbfgragh.onion/section/us}{U.S.}

\href{https://myaccount.nytimes3xbfgragh.onion/auth/login?response_type=cookie\&client_id=vi}{}

\href{https://www.nytimes3xbfgragh.onion/section/todayspaper}{Today's
Paper}

\href{/section/us}{U.S.}\textbar{}Supreme Court Strikes Down Texas
Abortion Restrictions

\url{https://nyti.ms/298q4o8}

\begin{itemize}
\item
\item
\item
\item
\item
\item
\end{itemize}

Advertisement

\protect\hyperlink{after-top}{Continue reading the main story}

Supported by

\protect\hyperlink{after-sponsor}{Continue reading the main story}

\hypertarget{supreme-court-strikes-down-texas-abortion-restrictions}{%
\section{Supreme Court Strikes Down Texas Abortion
Restrictions}\label{supreme-court-strikes-down-texas-abortion-restrictions}}

\includegraphics{https://static01.graylady3jvrrxbe.onion/images/2016/06/28/us/28SCOTUS/28SCOTUS-videoSixteenByNine3000.jpg}

By \href{http://www.nytimes3xbfgragh.onion/by/adam-liptak}{Adam Liptak}

\begin{itemize}
\item
  June 27, 2016
\item
  \begin{itemize}
  \item
  \item
  \item
  \item
  \item
  \item
  \end{itemize}
\end{itemize}

WASHINGTON --- The Supreme Court on Monday reaffirmed and strengthened
constitutional protections for abortion rights,
\href{http://www.supremecourt.gov/opinions/15pdf/15-274_p8k0.pdf}{striking
down} parts of a restrictive Texas law that could have drastically
reduced the number of abortion clinics in the state, leaving them only
in the largest metropolitan areas.

The 5-to-3 decision was the court's most sweeping statement on abortion
since Planned Parenthood v. Casey in 1992, which reaffirmed the
constitutional right to abortion established in 1973 in Roe v. Wade. It
found that Texas' restrictions --- requiring doctors to have admitting
privileges at nearby hospitals and clinics to meet the standards of
ambulatory surgical centers --- violated Casey's prohibition on placing
an ``undue burden'' on the ability to obtain an abortion.

If Casey limited the right established in Roe, allowing states to
regulate abortion in ways Roe had barred, Monday's decision effectively
expanded that right. It means that similar requirements in other states
are most likely also unconstitutional, and it imperils many other kinds
of restrictions on abortion. It is also sure to energize anti-abortion
forces and make abortion a central issue in the presidential campaign.

The decision concerned two parts of a
\href{http://www.nytimes3xbfgragh.onion/2013/07/19/us/perry-signs-texas-abortion-restrictions-into-law.html}{law
that imposed strict requirements} on abortion providers in Texas signed
into law in July 2013 by Rick Perry, the governor at the time.

One required all clinics in the state to meet the standards for
ambulatory surgical centers, including regulations concerning buildings,
equipment and staffing. The other required doctors performing abortions
to have admitting privileges at a nearby hospital.

``We conclude,'' Justice Stephen G. Breyer wrote for the majority,
``that neither of these provisions offers medical benefits sufficient to
justify the burdens upon access that each imposes. Each places a
substantial obstacle in the path of women seeking a previability
abortion, each constitutes an undue burden on abortion access, and each
violates the federal Constitution.''

Justices Anthony M. Kennedy, Ruth Bader Ginsburg, Sonia Sotomayor and
Elena Kagan joined the majority opinion. Chief Justice John G. Roberts
Jr. and Justices Clarence Thomas and Samuel A. Alito Jr. dissented.

Justice Kennedy's vote was the crucial one, and it came as a relief to
abortion rights groups, which have long viewed his thinking on the issue
as a contradictory muddle.

In the Casey decision, he joined Justices Sandra Day O'Connor and David
H. Souter in a joint opinion that reaffirmed the core of
\href{http://caselaw.lp.findlaw.com/scripts/getcase.pl?court=US\&vol=410\&invol=113}{Roe
v. Wade}. But Justice Kennedy's reputation as an abortion rights
champion had otherwise been undeserved, said David S. Cohen, a law
professor at Drexel University, as Casey was the only case in which he
had found an abortion restriction unconstitutional in his 28 years on
the Supreme Court.

Professor Cohen said Justice Kennedy's vote in Monday's case was a
puzzle. He may have been swayed by the burdens placed on women having to
drive hundreds of miles to obtain abortions, Professor Cohen said, or by
the lack of medical evidence justifying the restrictions --- or both.

Many states have enacted restrictions in recent years that test the
limits of the constitutional right to abortion, and the ruling in the
new case, Whole Woman's Health v. Hellerstedt, No. 15-274, enunciated
principles that will apply to all of the ones said to be justified by a
concern for women's health.

In a
\href{https://twitter.com/WhiteHouse/status/747470028635774976?lang=en}{message}
posted on Twitter, President Obama said he was ``pleased to see the
Supreme Court reaffirm'' that ``every woman has a constitutional right
to make her own reproductive choices.''

Ken Paxton, Texas' attorney general, said, ``The court is becoming a
default medical board for the nation, with no deference being given to
state law.''

The Texas law was passed in 2013 by the Republican-dominated Texas
Legislature and turned a Democratic state senator, Wendy Davis, who
conducted an 11-hour filibuster against the law, into a national
political star.

Last June, the United States Court of Appeals for the Fifth Circuit, in
New Orleans,
\href{http://www.ca5.uscourts.gov/opinions\%5Cpub\%5C14/14-50928-CV0.pdf}{largely
upheld} the contested provisions of the Texas law, saying it had to
accept lawmakers' assertions about the health benefits of abortion
restrictions. The appeals court ruled that the law, with minor
exceptions, did not place an undue burden on the right to abortion.

Justice Breyer said the appeals court's approach was at odds with the
proper application of the undue-burden standard. The Casey decision, he
said, ``requires that courts consider the burdens a law imposes on
abortion access together with the benefits those laws confer.''

\href{https://www.nytimes3xbfgragh.onion/interactive/2016/02/29/us/why-the-abortion-clinics-have-closed.html}{}

\includegraphics{https://static01.graylady3jvrrxbe.onion/images/2016/02/29/us/why-the-abortion-clinics-have-closed-1456762996200/why-the-abortion-clinics-have-closed-1456762996200-videoLarge.png}

\hypertarget{how-the-supreme-courts-decision-will-affect-access-to-abortion}{%
\subsection{How the Supreme Court's Decision Will Affect Access to
Abortion}\label{how-the-supreme-courts-decision-will-affect-access-to-abortion}}

The case centers on a Texas law that has resulted in half its clinics
closing.

In dissent, Justice Thomas said the majority opinion ``reimagines the
undue-burden standard,'' creating a ``benefits-and-burdens balancing
test.'' He said courts should resolve conflicting positions by deferring
to legislatures.

``Today's opinion,'' Justice Thomas wrote, ``does resemble Casey in one
respect: After disregarding significant aspects of the court's prior
jurisprudence, the majority applies the undue-burden standard in a way
that will surely mystify lower courts for years to come.''

The majority opinion considered whether the claimed benefits of the
restrictions outweighed the burdens they placed on a constitutional
right. Justice Breyer wrote that there was no evidence that the
admitting-privileges requirement ``would have helped even one woman
obtain better treatment.''

At the same time, he wrote, there was good evidence that the
admitting-privileges requirement caused the number of abortion clinics
in Texas to drop from 40 to 20.

In a second dissent, Justice Alito, joined by Chief Justice Roberts and
Justice Thomas, said the causal link between the law and the closings
was unproven. Withdrawal of state funds, a decline in the demand for
abortions and doctors' retirements may have played a role, Justice Alito
wrote.

Justice Breyer wrote that the requirement that abortion clinics meet the
demanding and elaborate standards for ambulatory surgical centers also
did more harm than good.

``Abortions taking place in an abortion facility are safe --- indeed,
safer than numerous procedures that take place outside hospitals and to
which Texas does not apply its surgical-center requirements,'' he wrote,
reviewing the evidence. ``Nationwide, childbirth is 14 times more likely
than abortion to result in death, but Texas law allows a midwife to
oversee childbirth in the patient's own home.''

In dissent, Justice Alito said there was good reason to think that the
restrictions were meant to protect women. ``The law was one of many
enacted by states in the wake of the Kermit Gosnell scandal, in which a
physician who ran an abortion clinic in Philadelphia was convicted for
the first degree murder of three infants who were born alive and for the
manslaughter of a patient,'' he wrote.

\href{https://www.nytimes3xbfgragh.onion/interactive/2016/02/14/us/politics/how-scalias-death-could-affect-major-supreme-court-cases-in-the-2016-term.html}{}

\includegraphics{https://static01.graylady3jvrrxbe.onion/images/2016/02/14/us/politics/how-scalias-death-could-affect-major-supreme-court-cases-in-the-2016-term-1455500687480/how-scalias-death-could-affect-major-supreme-court-cases-in-the-2016-term-1455500687480-articleLarge.png}

\hypertarget{how-a-vacancy-on-the-supreme-court-affected-cases-in-the-2015-16-term}{%
\subsection{How a Vacancy on the Supreme Court Affected Cases in the
2015-16
Term}\label{how-a-vacancy-on-the-supreme-court-affected-cases-in-the-2015-16-term}}

The empty seat left by Justice Antonin Scalia's death leaves the court
with two basic options for cases left on the docket this term if the
justices are deadlocked at 4 to 4.

Justice Breyer acknowledged that ``Gosnell's behavior was terribly
wrong.''

``But,'' he added, ``there is no reason to believe that an extra layer
of regulation would have affected that behavior.''

The clinics challenging the law said it had already caused about
\href{http://www.nytimes3xbfgragh.onion/interactive/2014/08/04/us/shrinking-number-of-abortion-clinics-in-texas.html}{half
the state's 41 abortion clinics to close}. If the contested provisions
had taken full effect, they said, the number of clinics would again be
cut in half.

The Supreme Court's decision rippled through the presidential campaign,
with Democrats and Republicans looking to rally voters with reminders
that the future of the court is at stake.

The next president will have at least one and potentially several
vacancies to fill, and Hillary Clinton and Donald J. Trump have both
warned that the fate of laws on immigration, guns and abortion will most
likely be determined by who gets to fill those openings.

Mrs. Clinton, the presumptive Democratic nominee, seized on the court's
ruling to warn that Mr. Trump, her Republican opponent, poses a threat
to women. She recalled his suggestion this year that abortion should be
banned and that women who violate that ban should be penalized. She also
said that with other states also seeking to restrict access to abortions
and with Republicans seeking to defund Planned Parenthood, proponents of
abortions rights could not afford to let up.

``We've seen a concerted, persistent attack on women's health and rights
at the federal level,'' Mrs. Clinton said in a statement. ``Meanwhile,
Donald Trump has said women should be punished for having abortions.''

Mr. Trump has since retracted his assertion that women should be
punished for having abortions, but the re-emergence of the issue is
likely to put him on the defensive because of his previous support of
abortion rights.

Mr. Trump made no direct public comments on Monday's decision.

Still, for many Republicans, the decision added urgency to their desire
to keep Mrs. Clinton from winning the presidency.

``Today's disappointing decision is another reminder of what's at stake
in this election and why we can't afford to let Hillary Clinton win,''
Reince Priebus, the chairman of the Republican National Committee, said.

Advertisement

\protect\hyperlink{after-bottom}{Continue reading the main story}

\hypertarget{site-index}{%
\subsection{Site Index}\label{site-index}}

\hypertarget{site-information-navigation}{%
\subsection{Site Information
Navigation}\label{site-information-navigation}}

\begin{itemize}
\tightlist
\item
  \href{https://help.nytimes3xbfgragh.onion/hc/en-us/articles/115014792127-Copyright-notice}{©~2020~The
  New York Times Company}
\end{itemize}

\begin{itemize}
\tightlist
\item
  \href{https://www.nytco.com/}{NYTCo}
\item
  \href{https://help.nytimes3xbfgragh.onion/hc/en-us/articles/115015385887-Contact-Us}{Contact
  Us}
\item
  \href{https://www.nytco.com/careers/}{Work with us}
\item
  \href{https://nytmediakit.com/}{Advertise}
\item
  \href{http://www.tbrandstudio.com/}{T Brand Studio}
\item
  \href{https://www.nytimes3xbfgragh.onion/privacy/cookie-policy\#how-do-i-manage-trackers}{Your
  Ad Choices}
\item
  \href{https://www.nytimes3xbfgragh.onion/privacy}{Privacy}
\item
  \href{https://help.nytimes3xbfgragh.onion/hc/en-us/articles/115014893428-Terms-of-service}{Terms
  of Service}
\item
  \href{https://help.nytimes3xbfgragh.onion/hc/en-us/articles/115014893968-Terms-of-sale}{Terms
  of Sale}
\item
  \href{https://spiderbites.nytimes3xbfgragh.onion}{Site Map}
\item
  \href{https://help.nytimes3xbfgragh.onion/hc/en-us}{Help}
\item
  \href{https://www.nytimes3xbfgragh.onion/subscription?campaignId=37WXW}{Subscriptions}
\end{itemize}
