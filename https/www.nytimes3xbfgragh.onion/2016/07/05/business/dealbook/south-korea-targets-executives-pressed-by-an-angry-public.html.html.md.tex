Sections

SEARCH

\protect\hyperlink{site-content}{Skip to
content}\protect\hyperlink{site-index}{Skip to site index}

\href{https://myaccount.nytimes3xbfgragh.onion/auth/login?response_type=cookie\&client_id=vi}{}

\href{https://www.nytimes3xbfgragh.onion/section/todayspaper}{Today's
Paper}

\href{/section/business/dealbook}{DealBook}\textbar{}South Korea Targets
Executives, Pressed by an Angry Public

\url{https://nyti.ms/29qFaYM}

\begin{itemize}
\item
\item
\item
\item
\item
\end{itemize}

Advertisement

\protect\hyperlink{after-top}{Continue reading the main story}

Supported by

\protect\hyperlink{after-sponsor}{Continue reading the main story}

DealBook Business and Policy

\hypertarget{south-korea-targets-executives-pressed-by-an-angry-public}{%
\section{South Korea Targets Executives, Pressed by an Angry
Public}\label{south-korea-targets-executives-pressed-by-an-angry-public}}

\includegraphics{https://static01.graylady3jvrrxbe.onion/images/2016/07/05/business/05SKWHITE/05SKWHITE-articleLarge.jpg?quality=75\&auto=webp\&disable=upscale}

By \href{http://www.nytimes3xbfgragh.onion/by/choe-sang-hun}{Choe
Sang-Hun}

\begin{itemize}
\item
  July 4, 2016
\item
  \begin{itemize}
  \item
  \item
  \item
  \item
  \item
  \end{itemize}
\end{itemize}

SEOUL, South Korea --- Looking for answers after the deaths of scores of
children and pregnant women from a mysterious lung ailment, a group of
families in South Korea began to focus on a potential cause: a cleaner
called Oxy.

In 2011, South Korean officials suggested that toxic chemicals in Oxy
--- used to sanitize humidifiers and sold by the British consumer goods
maker Reckitt Benckiser --- and similar products were responsible for
the deaths. Ninety-five have been confirmed by the government, which is
also reviewing hundreds of additional cases reported by families, who
claim more than 460 fatalities. The government's punishment for Reckitt
Benckiser: a \$45,000 fine for falsely advertising Oxy as safe for
humans.

Five years later, simmering anger over the deaths has hit Reckitt
Benckiser --- and prompted widening hostility to white-collar crime that
is directed at foreign and local companies alike.

South Korean prosecutors in May arrested three local Reckitt Benckiser
employees and charged them with professional negligence resulting in
deaths. When a Reckitt Benckiser executive publicly apologized, a
relative of a victim jumped onstage and slapped him in the back of the
neck.

South Korean prosecutors are also considering bringing criminal charges
against local Volkswagen executives stemming from investigations into
the automaker's
\href{https://www.nytimes3xbfgragh.onion/interactive/2015/business/international/vw-diesel-emissions-scandal-explained.html}{cheating
in emissions tests}, while lawmakers have significantly raised fines for
violating emissions rules. Officials last month raided the homes and
offices of top executives of Lotte, a major South Korean conglomerate,
to collect evidence of alleged embezzlement. Lotte has said it is
cooperating.

To outsiders, South Korea's rash of criminal investigations and
prosecutions against corporate officials make it look unusually
aggressive in pursuing white-collar crime. But South Korean analysts and
critics say the tough actions show the opposite: Government officials
have little power to brandish big fines or other civil penalties, as
their counterparts in the United States and Europe do.

``When people are outraged, the government has few things to show to
them except asking prosecutors to get involved,'' said Kim Pil-soo, a
professor of automotive engineering at Daelim University College, who
has followed the Volkswagen scandal.

For corporate scofflaws, South Korea can be a surprisingly forgiving
environment. Fines are modest. Until recently, courts routinely
suspended the sentences of tycoons convicted of bribery, embezzlement or
tax evasion, citing the potential impact on their corporate empires ---
and, by extension, the country's economy. Class-action lawsuits and
court awards are limited.

The soft treatment is a legacy of South Korea's hard-charging economic
past. In the decades after the end of the Korean War in 1953, military
dictators in South Korea favored businesses --- especially a handful of
corporate families known as chaebol --- with tax benefits, cheap
electricity and bank loans and brutal crackdowns on labor activists.

But there are signs that South Koreans' patience is running thin. In a
study a year ago commissioned by the government's Korea Legislation
Research Institute, over half of about 3,000 people queried said they
\href{http://www.koreatimes.co.kr/www/news/nation/2016/04/116_201742.html}{did
not believe} South Korean businesses abided by the laws, while more than
two-thirds said business regulations against pollution should be
strengthened.

In the newly elected National Assembly, where President Park Geun-hye's
pro-business governing party no longer holds a majority, lawmakers are
pushing to allow plaintiffs to ask for punitive damages --- large
financial penalties used in the United States and elsewhere to punish
white-collar crimes --- in more types of cases.

One South Korean group that monitors businesses, the People's Solidarity
for Participatory Democracy, called the introduction of punitive damages
one of South Korea's most urgently needed reforms.

``As a result of our decades-long national strategy focusing on economic
development, our legal system is too much geared toward protecting
industries,'' said Kim Hyun, a former president of the Seoul Bar
Association, who recently collected signatures from a thousand lawyers
supporting punitive damages.

Business groups say that punitive damages could victimize businesses.

``As in the United States, we would see lawyers encouraging lawsuits
against companies, costing them time and money in fighting these
questionable lawsuits,'' said Lee Cheol-haeng, the chief of business
policy studies at the Federation of Korean Industries, which represents
big businesses in South Korea.

Until recently, the cost of violating South Korean law could be modest,
as the Volkswagen case shows.

While the Ministry of Environment ordered Volkswagen Korea to recall
125,000 cars sold in the country, it could fine the company only \$12.3
million. Under a law devised to protect local carmakers, the government
can impose fines up to 1 billion South Korean won, or \$867,000, per
model that violates its clean-air law, no matter how many individual
cars have been sold. The new law, enacted at the end of last year,
allows up to 10 times that amount, or about \$8.7 million.

By contrast, the United States can impose civil penalties of up to
\$37,500 per noncompliant vehicle or engine under the Clean Air Act.
Volkswagen
\href{http://www.nytimes3xbfgragh.onion/2016/06/28/business/volkswagen-settlement-diesel-scandal.html}{has
agreed to pay \$14.7 billion} in fines in the United States.

When the Ministry of Environment asked prosecutors to seek criminal
charges against Volkswagen Korea executives earlier this year, Hong
Dong-gon, a ministry official, said they were meant partly as ``a tool
of pressure'' to wrest a more satisfactory recall and compensation
package from Volkswagen.

The biggest example of resistance to white-collar crime has become
Reckitt Benckiser's local subsidiary. In addition to arrests there, some
workers at local companies that made or sold rival products have been
arrested on similar charges. Two university professors have also been
arrested on charges of manipulating data on Oxy's toxicity in return for
bribes from Reckitt Benckiser Korea.

The families fighting Oxy did not gain national attention until this
year. As the scandal threatened to become a major political burden for
her government, which championed pro-business deregulation, Ms. Park
called for a thorough investigation, and prosecutors began summoning
company officials.

In May, Reckitt Benckiser Korea apologized and acknowledged
responsibility. It also promised to double a humanitarian fund it had
founded for victims to \$8.7 million, following a common practice among
South Korean businesses in legal trouble to make large charitable
donations while seeking lenience in court.

``Although we understand nothing can completely ease the pain of those
affected, we are working to make amends as best we can,'' Reckitt
Benckiser Korea
\href{http://www.rb.com/media/news/2016/may/oxy-rb-and-humidifier-sterilizers-in-korea/}{said
in a statement}.

Some of the families say that is not enough.

``It has been as if there were only victims but no perpetrators,'' said
Kang Chan-ho, whose daughter struggles with lung damage after his family
used one of the toxic disinfectants. ``We have been ignored both by the
government and by the businesses.''

Advertisement

\protect\hyperlink{after-bottom}{Continue reading the main story}

\hypertarget{site-index}{%
\subsection{Site Index}\label{site-index}}

\hypertarget{site-information-navigation}{%
\subsection{Site Information
Navigation}\label{site-information-navigation}}

\begin{itemize}
\tightlist
\item
  \href{https://help.nytimes3xbfgragh.onion/hc/en-us/articles/115014792127-Copyright-notice}{©~2020~The
  New York Times Company}
\end{itemize}

\begin{itemize}
\tightlist
\item
  \href{https://www.nytco.com/}{NYTCo}
\item
  \href{https://help.nytimes3xbfgragh.onion/hc/en-us/articles/115015385887-Contact-Us}{Contact
  Us}
\item
  \href{https://www.nytco.com/careers/}{Work with us}
\item
  \href{https://nytmediakit.com/}{Advertise}
\item
  \href{http://www.tbrandstudio.com/}{T Brand Studio}
\item
  \href{https://www.nytimes3xbfgragh.onion/privacy/cookie-policy\#how-do-i-manage-trackers}{Your
  Ad Choices}
\item
  \href{https://www.nytimes3xbfgragh.onion/privacy}{Privacy}
\item
  \href{https://help.nytimes3xbfgragh.onion/hc/en-us/articles/115014893428-Terms-of-service}{Terms
  of Service}
\item
  \href{https://help.nytimes3xbfgragh.onion/hc/en-us/articles/115014893968-Terms-of-sale}{Terms
  of Sale}
\item
  \href{https://spiderbites.nytimes3xbfgragh.onion}{Site Map}
\item
  \href{https://help.nytimes3xbfgragh.onion/hc/en-us}{Help}
\item
  \href{https://www.nytimes3xbfgragh.onion/subscription?campaignId=37WXW}{Subscriptions}
\end{itemize}
