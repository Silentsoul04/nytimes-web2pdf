Sections

SEARCH

\protect\hyperlink{site-content}{Skip to
content}\protect\hyperlink{site-index}{Skip to site index}

\href{https://www.nytimes3xbfgragh.onion/section/politics}{Politics}

\href{https://myaccount.nytimes3xbfgragh.onion/auth/login?response_type=cookie\&client_id=vi}{}

\href{https://www.nytimes3xbfgragh.onion/section/todayspaper}{Today's
Paper}

\href{/section/politics}{Politics}\textbar{}Donald Trump's Deals Rely on
Being Creative With the Truth

\url{https://nyti.ms/29Xuf6E}

\begin{itemize}
\item
\item
\item
\item
\item
\item
\end{itemize}

Advertisement

\protect\hyperlink{after-top}{Continue reading the main story}

Supported by

\protect\hyperlink{after-sponsor}{Continue reading the main story}

\hypertarget{donald-trumps-deals-rely-on-being-creative-with-the-truth}{%
\section{Donald Trump's Deals Rely on Being Creative With the
Truth}\label{donald-trumps-deals-rely-on-being-creative-with-the-truth}}

\includegraphics{https://static01.graylady3jvrrxbe.onion/images/2016/07/17/us/17TRUMPLIES1/17TRUMPLIES1-articleLarge.jpg?quality=75\&auto=webp\&disable=upscale}

By \href{https://www.nytimes3xbfgragh.onion/by/david-barstow}{David
Barstow}

\begin{itemize}
\item
  July 16, 2016
\item
  \begin{itemize}
  \item
  \item
  \item
  \item
  \item
  \item
  \end{itemize}
\end{itemize}

There was the time Donald J. Trump told Larry King that he had been paid
more than \href{http://www.cnn.com/TRANSCRIPTS/0603/09/lkl.01.html}{\$1
million to give a speech} about his business acumen when in fact he was
paid \$400,000. Or the time he sought a
\href{http://www.nytimes3xbfgragh.onion/2016/05/24/business/dealbook/donald-trump-relationship-bankers.html}{bank
loan} claiming a net worth of \$3.5 billion in 2004, four times as much
as what the bank found when it checked his math. Or the time he boasted
that
\href{https://books.google.com/books?id=Fo_l25-54sQC\&pg=PA46\&lpg=PA46\&dq=Trump+101:+How+to+Get+Rich+Membership+costs+$300,000.+I+think+it\%E2\%80\%99s+a+bargain\&source=bl\&ots=MgDzJ-CmYk\&sig=pwudPB88zsdU4fhWo-8-V3_-AC8\&hl=en\&sa=X\&ved=0ahUKEwiHip7V1PPNAhWDQCYKHYCDBvYQ6AEIHjAA\#v=onepage\&q=Trump\%20101\%3A\%20How\%20to\%20Get\%20Rich\%20Membership\%20costs\%20\%24300\%2C000.\%20I\%20think\%20it\%E2\%80\%99s\%20a\%20bargain\&f=false}{membership}
to Trump National Golf Club in Westchester County, N.Y., cost \$300,000
when the actual initiation fee was \$200,000. Or the time he bragged on
CNBC about his new Trump International Hotel and Tower in Las Vegas,
claiming, ``We have 1,282 units, and they sold out in less than a
week.'' As Mr. Trump knew, more than 300 units had not been sold.

Confronted in a court case about this last untruth, Mr. Trump was
anything but chagrined. ``I'm talking to a television station,'' he
said. ``We do want to put the best spin on the property.''

As Mr. Trump prepares to claim the Republican nomination for president
this week, he and his supporters are sure to laud his main calling card
--- his long, operatic record as a swaggering business tycoon. And
without question, there will be successes aplenty to highlight, from his
gleaming golden high-rises to his well-regarded golf resorts, hit TV
shows and best-selling books.

But a survey of Mr. Trump's four decades of wheeling and dealing also
reveals an equally operatic record of dissembling and deception, some of
it unabashedly confirmed by Mr. Trump himself, who nearly 30 years ago
first extolled the business advantages of ``truthful hyperbole.''
Indeed, based on the mountain of court records churned out over the span
of Mr. Trump's career, it is hard to find a project he touched that did
not produce allegations of broken promises, blatant lies or outright
fraud.

Under the intense scrutiny of a presidential election, many of those
allegations have already become familiar campaign fodder: the
\href{http://www.nytimes3xbfgragh.onion/2016/06/01/us/politics/donald-trump-university.html}{Trump
University} students and Trump
\href{http://www.nytimes3xbfgragh.onion/2016/04/06/us/politics/donald-trump-soho-settlement.html}{condo
buyers} who say they were fleeced; the public servants from New Jersey
to Scotland who now say they rue the zoning approvals, licenses or tax
breaks they gave based on Mr. Trump's promises; the small-time
contractors who say Mr. Trump concocted complaints about their work to
avoid paying them; the infuriated business partners who say Mr. Trump
concealed profits or ignored contractual obligations; the business
journalists and stock analysts who say Mr. Trump smeared them for
critical coverage.

Taken as a whole, though, an examination of Mr. Trump's business career
reveals persistent patterns in the way Mr. Trump bends or breaks the
truth --- patterns that may already feel familiar to those watching his
campaign.

First and foremost is Mr. Trump's tendency toward the self-aggrandizing
fib --- as if it were not impressive enough to be paid \$400,000 for a
speech. What also emerges is a nearly reflexive habit of telling his
target audience precisely what he thinks it wants to hear --- such as
promising Trump University students they will learn all his real estate
secrets from his ``handpicked'' instructors. And finally, there is the
pattern already deeply familiar to his political opponents --- making
spurious claims against adversaries under Mr. Trump's oft-stated theory
that the best defense is a scorched-earth offense.

Equally striking is his Houdiniesque ability to wiggle away from all but
the most skilled and determined efforts to corner him in an apparent
lie. In interviews, lawyers who have tangled with Mr. Trump in court
cases are sometimes reduced to sputtering, astonished rage, calling him
``borderline pathological'' and ``the Michelangelo of deception'' as
they attempt to describe the ease with which Mr. Trump weaves his own
versions of reality.

``He's a bully, and bullies aren't known for their veracity,'' said
Richard C. Seltzer, a retired senior partner at the law firm Kaye
Scholer who confronted Mr. Trump in three real estate lawsuits.

In a telephone interview on Friday, Mr. Trump defended his integrity as
a businessman --- ``I shoot very straight'' --- and argued that those
who accuse him of acting in bad faith are often the same people he has
outmaneuvered in deals.

``What, you're going to quote people that I've beat? Are you going to
quote people that I out-dealt?'' he asked, adding, ``I'll give you
hundreds of names of people that have dealt with me that say I'm very
honest.''

Hillary Clinton, meanwhile, is already hard at work making the case that
Mr. Trump's truth-challenged business record is a harbinger of how he
would mislead from the Oval Office. Her campaign has even put up a
none-too-subtle website:
\href{http://www.artofthesteal.biz/}{www.artofthesteal.biz}.

Mr. Trump's business record may help explain why various fact-checkers
have barely been able to keep pace with his false claims on the campaign
trail.
\href{http://www.politifact.com/personalities/donald-trump/}{PolitiFact}
has labeled 34 of Mr. Trump's assertions ``Pants on Fire'' lies. As of
July 1,
\href{https://www.washingtonpost.com/news/fact-checker/wp/2016/06/10/fact-checker-video-donald-trumps-most-outrageous-four-pinocchio-claims/}{The
Washington Post} had fact-checked 46 statements by Mr. Trump. It gave 70
percent of them its worst rating, four Pinocchios --- a record so
abysmal that the newspaper recently compiled a video of what it called
``Donald Trump's most outrageous four-Pinocchio claims.''

The taxonomy of Mr. Trump's business deceptions has been the subject of
legal and journalistic scrutiny for decades. A Fortune magazine article
from 2000 memorably described Mr. Trump's ``astonishing ability to
prevaricate'' this way: ``But when Trump says he owns 10 percent of the
Plaza Hotel, understand that what he actually means is that he has the
right to 10 percent of the profit if it's ever sold. When he says he's
building a `90-story building' next to the U.N., he means a 72-story
building that has extra-high ceilings. And when he says his casino
company is the `largest employer in the state of New Jersey,' he
actually means to say it is the eighth largest.''

The casino magnate Steve Wynn, a sometimes friend and sometimes foe of
Mr. Trump's, took up the subject of Mr. Trump's honesty in an interview
with New York magazine. ``His statements to people like you, whether
they concern us and our projects or our motivations or his own reality
or his own future or his own present you have seen over the years have
no relation to truth or fact,'' Mr. Wynn said.

\includegraphics{https://static01.graylady3jvrrxbe.onion/images/2016/07/17/us/17TRUMP2/17TRUMPLIES2-articleLarge.jpg?quality=75\&auto=webp\&disable=upscale}

\hypertarget{truthful-hyperbole}{%
\subsection{`Truthful Hyperbole'}\label{truthful-hyperbole}}

Some of the earliest documented examples of Mr. Trump's deceptive
business tactics come from none other than Mr. Trump, who in books and
in interviews sometimes seems to delight in describing the brazen bluffs
and well-timed trickery he used to claw his way to the upper echelons of
New York City's cutthroat real estate world.

``You have to understand where I was coming from,'' Mr. Trump wrote in
his 1987 best-seller,
\href{http://www.nytimes3xbfgragh.onion/2015/09/22/upshot/10-things-i-learned-about-donald-trump-in-the-art-of-the-deal.html}{``The
Art of the Deal.''} ``While there are certainly honorable people in the
real estate business, I was more accustomed to the sort of people with
whom you don't want to waste the effort of a handshake because you know
it's meaningless.''

Mr. Trump was particularly proud of a stratagem he employed in 1982,
when he was trying to entice Holiday Inn to invest in a casino he was
building in Atlantic City. The board of directors decided to visit
Atlantic City, which worried Mr. Trump because he had precious little
actual construction to show off. So Mr. Trump ordered his construction
supervisor to cram every bulldozer and dump truck he could find into the
nearly vacant construction site.

``What the bulldozers and dump trucks did wasn't important, I said, so
long as they did a lot of it. If they got some actual work accomplished,
all the better, but if necessary, he should have the bulldozers dig up
dirt from one side of the site and dump it on the other.''

A week later, when Mr. Trump escorted the Holiday Inn executives to the
site, one board member wanted to know why a worker was filling a hole he
had just dug. ``This was difficult for me to answer, but fortunately,
this board member was more curious than he was skeptical,'' Mr. Trump
wrote, boasting that weeks later Holiday Inn agreed to invest in his
casino.

``That's called `business,''' Mr. Trump said on Friday of the episode.

In court cases against Mr. Trump --- USA Today counted 3,500 lawsuits
involving Mr. Trump, and Mr. Trump estimates he has testified more than
100 times --- plaintiffs' lawyers frequently return to the same two
paragraphs from ``The Art of the Deal.''

``I call it truthful hyperbole. It's an innocent form of exaggeration
--- and a very effective form of promotion.''

In depositions, lawyers have repeatedly probed for the limits of Mr.
Trump's ``truthful hyperbole,'' or, as one lawyer framed it, the
distinction Mr. Trump makes between ``innocent exaggeration'' and
``guilty exaggeration.''

Image

The now-defunct Trump University has left a long trail of customers
saying that they were defrauded.Credit...Thos Robinson/Getty Images

For example, in the now-infamous
\href{http://www.nytimes3xbfgragh.onion/2016/06/01/us/politics/donald-trump-university.html}{Trump
University} litigation, Mr. Trump was asked in a deposition about a
script that had been prepared for Trump University instructors.
According to the script, the instructors were supposed to tell their
students the following: ``I remember one time Mr. Trump said to us over
dinner, he said, `Real estate is the only market that, when there's a
sale going on, people run from the store.' You don't want to run from
the store.''

No such dinners ever took place, Mr. Trump acknowledged. In fact, Mr.
Trump struggled to identify a single one of the instructors he claimed
to have handpicked, even after he was shown their photographs.
Nonetheless, Mr. Trump was not bothered by the script's false
insinuation of real estate secrets shared over chummy dinners. Asked if
this example constituted ``innocent exaggeration,'' Mr. Trump replied,
``Yes, I'd say that's an innocent exaggeration.''

On Friday, Mr. Trump argued that the script might fall under the legal
concept of ``puffery'' --- which many legal dictionaries define as an
exaggeration or statement that ``no reasonable person'' would take as
factual. And in any event, he continued, the true sinners in the Trump
University case are the students who sued him even after giving rave
reviews in their written evaluations of the seminars. ``I think that's
dishonest,'' he said.

Mr. Trump has been repeatedly accused of bringing false legal claims to
avoid paying debts and evade contractual obligations. As far back as
1983, a New York City housing court judge ruled that Mr. Trump filed a
``spurious'' lawsuit to harass a tenant into vacating a Trump building.

Then there was the case Mr. Trump brought against
\href{http://topics.nytimes3xbfgragh.onion/top/reference/timestopics/people/c/barbara_corcoran/index.html}{Barbara
Corcoran}, the real estate broker best known for her appearances on
``Shark Tank.'' In the mid-1990s, Mr. Trump owed millions of dollars to
Ms. Corcoran for helping him secure financing for a development. But
when New York magazine published a cover story about the troubled
project ---
\href{https://books.google.com/books?id=VeMCAAAAMBAJ\&lpg=PP1\&dq=trump\&pg=PP1\#v=onepage\&q\&f=false}{``Trump's
Near-Death Experience''} --- Mr. Trump sued Ms. Corcoran, accusing her
and her associates of sharing damaging information with the magazine and
thus violating a confidentiality agreement. He refused to pay her the
millions he owed, claiming her breach had gravely damaged his business.

At trial, Mr. Trump was unable to produce a single document showing harm
to his business. But his certitude never wavered, even after Ms.
Corcoran's lawyer, Mr. Seltzer, confronted him with article after
article in which Mr. Trump himself had discussed with reporters much of
the same ``confidential'' information he accused Ms. Corcoran's team of
divulging.

``There is something very belligerent about the way he presents facts,
as if he thinks nobody will have the balls to stand up to him,'' Mr.
Seltzer said in an interview. (In dismissing Mr. Trump's suit against
Ms. Corcoran, the judge said the only damages he could identify were to
Mr. Trump's ``bruised ego.'')

Image

The Trump National Golf Club in Westchester County. Mr. Trump
embellished the cost of a membership.Credit...Mike Segar/Reuters

\hypertarget{well-timed-memory-lapses}{%
\subsection{Well-Timed Memory Lapses}\label{well-timed-memory-lapses}}

In Friday's interview, Mr. Trump denied filing frivolous court cases,
insisting, ``I've won a massive majority of the litigation I've been
involved in.'' He pointed to the USA Today survey of his 3,500 legal
cases. Although the newspaper could not determine who had prevailed in
the vast majority of the cases, it did find Mr. Trump the clear winner
in 450 suits and the clear loser in 38.

And, indeed, for all of the litigation Mr. Trump has attracted or
spawned, for all of the times he has been accused of ruinous dishonesty,
the legal and regulatory record is surprisingly bare of official
findings by judges, juries or regulators that Mr. Trump engaged in
perjury or improper deception or actual fraud.

A rare exception came after Mr. Trump decided to demolish a department
store to make way for his Trump Tower in Midtown Manhattan. Mr. Trump's
demolition contractor hired about 200 unauthorized Polish laborers,
paying them as little as \$4 an hour to work 12 hours a day, seven days
a week. The case ended up in federal court after some workers were
shortchanged even these wages.

Mr. Trump protested that he knew nothing about the use of unauthorized
workers --- even though workers testified that they saw him visiting the
site and some witnesses said that Mr. Trump and the executive he
assigned to oversee the demolition were well aware of what was going on.
In 1991, a federal judge, Charles E. Stewart Jr., ruled that despite Mr.
Trump's denials, there was ``strong evidence'' that he and his
subordinates and his contractor had conspired to hire the Polish workers
and deprive them of employment benefits. He awarded them \$325,415 in
damages.

\href{https://www.nytimes3xbfgragh.onion/interactive/2016/us/elections/polls.html}{}

\includegraphics{https://static01.graylady3jvrrxbe.onion/images/2016/06/15/us/elections/polls-1466014214178/polls-1466014214178-articleLarge-v2.jpg}

\hypertarget{latest-election-polls-2016}{%
\subsection{Latest Election Polls
2016}\label{latest-election-polls-2016}}

Get the latest national and state polls on the presidential election
between Hillary Clinton and Donald J. Trump.

But in case after case, Mr. Trump has displayed a special talent for
turning what should be cold hard facts into semantic mush. Perhaps the
most famous example of this skill came when Mr. Trump was asked under
oath a seemingly straightforward question: Had he ever lied about his
net worth? Mr. Trump responded, ``My net worth fluctuates and it goes up
and down with markets and with attitudes and with feelings, even my own
feelings.''

So, he explained in a deposition, when he said
\href{https://books.google.com/books?id=Fo_l25-54sQC\&pg=PA46\&lpg=PA46\&dq=Trump+101:+How+to+Get+Rich+Membership+costs+$300,000.+I+think+it\%E2\%80\%99s+a+bargain\&source=bl\&ots=MgDzJ-CmYk\&sig=pwudPB88zsdU4fhWo-8-V3_-AC8\&hl=en\&sa=X\&ved=0ahUKEwiHip7V1PPNAhWDQCYKHYCDBvYQ6AEIHjAA\#v=onepage\&q=Trump\%20101\%3A\%20How\%20to\%20Get\%20Rich\%20Membership\%20costs\%20\%24300\%2C000.\%20I\%20think\%20it\%E2\%80\%99s\%20a\%20bargain\&f=false}{membership
costs \$300,000} to his Westchester golf club, that included the
\$200,000 initiation fee plus every cent he guessed that a member might
spend on annual dues over the next 20 or 30 years. In other words, ``The
way I say it is more accurate.'' And when he
\href{http://www.cnn.com/TRANSCRIPTS/0603/09/lkl.01.html}{told Larry
King he was paid more than \$1 million} for a speech, it was not his
fault if viewers failed to realize he was including not just his
\$400,000 speaking fee but also the hundreds of thousands of dollars he
assumed must have been spent promoting his appearance.

Part of what makes Mr. Trump such an elusive target is that his paper
trail is often minimal. Mr. Trump has repeatedly testified that he does
not use computers. He says he also throws away his day planner each
month, and just last year he testified that he did not own a smartphone.
``Unlike Hillary Clinton, I'm not a big email fan,'' he said, leaving
open the question of how he posts to Twitter.

Mr. Trump is also adept at deflecting blame to his staff. In two of his
books, Mr. Trump made the startling and, as it turned out, bogus claim
that he had once performed the remarkable feat of climbing out from
under more than \$9 billion in debt. Mr. Trump blamed his ghostwriter
for the mistake. Asked if he reads his books before publication, Mr.
Trump said, ``I read it as quickly as I can because of time
constraints.''

Mr. Trump is also the beneficiary of miraculously well-timed memory
lapses. In suit after suit, the man who claims to possess one of world's
best memories suddenly seems to have chronic memory loss when asked
about critical facts or events.

Such was the case when Mr. Trump filed a libel lawsuit against Timothy
L. O'Brien, the author of ``TrumpNation: The Art of Being the Donald.''
Among other things, Mr. Trump asserted that ``TrumpNation'' cost him a
``deal made in heaven'' with a group of Italian investors, men he had
met and who were on the brink of signing a business partnership that
would have made him hundreds of millions of dollars. Their names? He
could not recall. ``TrumpNation'' also cost him a hotel deal with
Russian investors, he said. He could not remember their names, either.
He was certain the book also ruined a deal with Turkish investors.
Again, he could not recall any names. Polish investors also got cold
feet after they read Mr. O'Brien's book. Their names escaped him, too.
The book also scared off investors from Ukraine. Alas, he could not
think of their names either.

Mr. Trump's lawsuit was dismissed.

Advertisement

\protect\hyperlink{after-bottom}{Continue reading the main story}

\hypertarget{site-index}{%
\subsection{Site Index}\label{site-index}}

\hypertarget{site-information-navigation}{%
\subsection{Site Information
Navigation}\label{site-information-navigation}}

\begin{itemize}
\tightlist
\item
  \href{https://help.nytimes3xbfgragh.onion/hc/en-us/articles/115014792127-Copyright-notice}{©~2020~The
  New York Times Company}
\end{itemize}

\begin{itemize}
\tightlist
\item
  \href{https://www.nytco.com/}{NYTCo}
\item
  \href{https://help.nytimes3xbfgragh.onion/hc/en-us/articles/115015385887-Contact-Us}{Contact
  Us}
\item
  \href{https://www.nytco.com/careers/}{Work with us}
\item
  \href{https://nytmediakit.com/}{Advertise}
\item
  \href{http://www.tbrandstudio.com/}{T Brand Studio}
\item
  \href{https://www.nytimes3xbfgragh.onion/privacy/cookie-policy\#how-do-i-manage-trackers}{Your
  Ad Choices}
\item
  \href{https://www.nytimes3xbfgragh.onion/privacy}{Privacy}
\item
  \href{https://help.nytimes3xbfgragh.onion/hc/en-us/articles/115014893428-Terms-of-service}{Terms
  of Service}
\item
  \href{https://help.nytimes3xbfgragh.onion/hc/en-us/articles/115014893968-Terms-of-sale}{Terms
  of Sale}
\item
  \href{https://spiderbites.nytimes3xbfgragh.onion}{Site Map}
\item
  \href{https://help.nytimes3xbfgragh.onion/hc/en-us}{Help}
\item
  \href{https://www.nytimes3xbfgragh.onion/subscription?campaignId=37WXW}{Subscriptions}
\end{itemize}
