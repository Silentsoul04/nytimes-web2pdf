Sections

SEARCH

\protect\hyperlink{site-content}{Skip to
content}\protect\hyperlink{site-index}{Skip to site index}

\href{https://myaccount.nytimes3xbfgragh.onion/auth/login?response_type=cookie\&client_id=vi}{}

\href{https://www.nytimes3xbfgragh.onion/section/todayspaper}{Today's
Paper}

Cooking for Artists at the Robert Rauschenberg Foundation

\url{https://nyti.ms/2a49CWt}

\begin{itemize}
\item
\item
\item
\item
\item
\end{itemize}

Advertisement

\protect\hyperlink{after-top}{Continue reading the main story}

Supported by

\protect\hyperlink{after-sponsor}{Continue reading the main story}

Travel Diary

\hypertarget{cooking-for-artists-at-the-robert-rauschenberg-foundation}{%
\section{Cooking for Artists at the Robert Rauschenberg
Foundation}\label{cooking-for-artists-at-the-robert-rauschenberg-foundation}}

\href{https://www.nytimes3xbfgragh.onion/slideshow/2016/07/14/t-magazine/shuna-lydon-cooking-for-artists.html}{}

\hypertarget{shuna-lydon-cooking-for-artists}{%
\subsection{Shuna Lydon, Cooking for
Artists}\label{shuna-lydon-cooking-for-artists}}

19 Photos

View Slide Show ›

\includegraphics{https://static01.graylady3jvrrxbe.onion/images/2016/07/14/t-magazine/14tmag-shuna-slide-DWID/14tmag-shuna-slide-DWID-articleLarge.jpg?quality=75\&auto=webp\&disable=upscale}

Courtesy of the Robert Rauschenberg Foundation

By Charlotte Druckman

\begin{itemize}
\item
  July 18, 2016
\item
  \begin{itemize}
  \item
  \item
  \item
  \item
  \item
  \end{itemize}
\end{itemize}

Shuna Lydon has just finished cleaning up the weekly Friday brunch she
prepares for 23 people and is off to walk Suzy the Dog, her sole
companion at the Robert Rauschenberg Foundation in Captiva, Florida. For
the last three months, she has been installed as that institution's
first chef-in-residence, cooking for the staff and visiting
artists-in-residence. She's also there to answer a personal question:
``What is it that feeds me about being a professional cook?'' Her tenure
will last through the end of the year, but she's already begun to find
answers. ``It's a magnificent opportunity,'' the pastry chef and
restaurant veteran says, ``to get back to the real basics of cooking;
you're making lunch and dinner. It's not about being fancy. Some of
these artists are doing really physical work in their field ---
sometimes it's like feeding a bunch of workmen. They're \emph{hungry}.''

Most rewarding is ``listening to and interacting with artists from a
myriad of practices'' and seeing how they all ``think and work
differently.'' She gets to work differently, too, pushed out of her
comfort zone to focus on savory food. Leaving New York City, where she
was the pastry chef at Bakeri in Brooklyn, has enhanced the experience.
For 40 years, Rauschenberg called this island --- a splinter of land
along the Sunshine State's southwest coast --- home. And four years
after his death in 2008, his property was turned into an artists'
community in the tradition of
\href{http://www.nytimes3xbfgragh.onion/2015/03/19/arts/artsspecial/in-the-spirit-of-black-mountain-college-an-avant-garde-incubator.html}{Black
Mountain College}, where Bob, as he's referred to by everyone on-site,
spent his early years. The building Lydon cooks in, the Weeks House, is
a communal space, and although Rauschenberg didn't live in it, he did
``transport a fantastically massive restaurant stove from the Bowery,''
so, Lydon says, ``cooking here is like cooking in a chef's home
kitchen!''

The location does have its challenges. The first is the lack of an ``OFF
button,'' which makes the situation ``a little like the French
Laundry,'' where Lydon was the pastry sous chef, and ``the given was you
worked from dark to dark, seven days a week.'' But she knew she wasn't
signing up for the regular hours of a restaurant or a country club.
``It's like nothing else; it defies definition --- it's a brand new
thing, so the people who work it are writing it.'' There are also
limited grocery options, so she drives to ``between half a dozen and a
dozen stores a week to buy ingredients.'' Lydon assumed, due to the
large Jewish population of places like Boca Raton and Palm Beach, or the
vibrant Cuban community of Miami, that their respective foodstuffs would
be readily available where she was. But there are, she says, ``no
East-Coast ingredients here.'' On top of that, she has a seasonal
setback: The local farmers stop planting in April and don't start again
until the fall. Nothing grows near Captiva in the summer. And on
Captiva, it's always summer.

The constraints, though, have only fueled Lydon's creativity --- they,
along with the lack of restaurant pressures, allow for what she calls
``a more pure cooking experience.'' She will whip up a Southern-style
brunch of biscuits and gravy with peach galettes one day, and a
Thai-inspired meal of smoky charred eggplant with yuzu and soy with
peanut-topped sesame noodles the next. She can pick the mangoes from
Bob's trees and use them in a black-rice salad, or turn a local calabash
squash into a warmly spiced, coconutty soup. There's a reason, she
notes, that artists' residencies are always in remote places: ``You can
go where there's big wide open spaces so that no matter what your
practice is, you can focus inward, you can really get to your core with
as little or as few distractions as possible \ldots{} It's like my soul
is getting fed by being in the place.''

Above, she shares some photos from her experience; and below, some
recipes for two dishes she's prepared for the artists.

\includegraphics{https://static01.graylady3jvrrxbe.onion/images/2016/07/14/t-magazine/14tmag-shuna-slide-6Z0C/14tmag-shuna-slide-6Z0C-articleLarge.jpg?quality=75\&auto=webp\&disable=upscale}

\hypertarget{pine-island-botanicals-calabash-coconut-soup}{%
\subsection{Pine Island Botanicals Calabash-Coconut
Soup}\label{pine-island-botanicals-calabash-coconut-soup}}

Lydon loves ``to serve this orange-yellow soup with black or red rice
and a green vegetable, like pan-roasted okra or nettles or kale --- to
offset the soup's sweet nature, and to give the bowl a stark visual
complement.''

\emph{Yield: 20 servings, sized for hungry artists}

5-7 pounds calabash squash (kabocha is a fine substitute)\\
2-3 leeks\\
2 large yellow onions\\
¾ cup organic coconut oil\\
Sea salt, to taste\\
¼ teaspoon white peppercorns\\
1 teaspoon
\href{https://www.amazon.com/CHIEF-Indian-Head-Curry-Powder/dp/B003SDIWU6}{Chief
brand} Trinidadian curry powder\\
1½ teaspoons
\href{https://www.amazon.com/Chief-Roasted-Geera-3oz/dp/B000QTAH1M}{geera}
(roasted ground cumin seeds)\\
½ teaspoon (regular) cumin seeds\\
1 tablespoon coriander seeds\\
½ tablespoon fenugreek seeds\\
3-4 stems of curry leaves\\
¼ cup chopped fresh galangal\\
1 thumb-sized piece of fresh ginger, peeled and chopped\\
1 stalk lemongrass, pounded and chopped\\
1 tablespoon thinly sliced fresh turmeric\\
2 bird's eye chilies, sliced into rounds\\
8 kaffir lime leaves, sliced into thin strips\\
3 cups fresh grated coconut (best found in the frozen section of your
local Indian grocery)\\
10 cups vegetable stock or water\\
Powdered coconut milk, to taste*\\
Jaggery (palm or coconut sugar), to taste

1. Preheat the oven to 400º F.

2. Prep the calabash squash: Peel the squash and split it in half
lengthwise. Scrape the seeds out and soak it in cold water. (Roasting
the seeds makes for a lovely garnish if you feel so inclined.) Chop the
squash into large pieces. Put it in a roasting pan with 1 to 2 inches of
cold water, cover tightly with aluminum foil, and bake for 30 minutes,
or until squash is fork tender. Using a spoon, scrape the flesh away
from the skin, discarding the latter. Place the flesh in a bowl and set
aside. (You want 10-12 cups of cooked squash; if you have a little more
or less, that's okay.)

2. Prep the leeks: Cut off the smallest amount of the leek's root end
and discard. As if you are turning the top of the leek into a spear, or
an arrow, cut the top of each one at a sharp angle, so as to make a
triangular pointy tip. Slice the green leek tops into ¼-inch pieces and
cut the white part of the leeks into ½-inch rounds (you should have
about 10 cups of sliced leeks). Put all the sliced leeks into a basin of
cold water, and swirl twice in 10 minutes to loosen sand and dirt. Lift
the leeks out and drain in a colander.

3. Peel and slice the onions, and add them to the bowl of leeks.

4. Place a large, heavy-bottomed stockpot over medium heat on the stove.
When the bottom of the pan is warm, pour in the coconut oil. In 2 to 3
minutes, pour in all cleaned leeks and sliced onions, and reduce the
heat down to medium-low. Stir with a wooden spoon intermittently until
the leeks and onions are starting to relax and become translucent ---
you do not want to add color here, just sweat and wilt. When the mixture
is sufficiently soft, increase the heat to medium and add a dash of
salt, white peppercorns, curry powder, geera, cumin, coriander and
fenugreek seeds and curry leaves. Stir intermittently for about 5 to 10
minutes, until the flavors have deepened and married and emit a notably
fragrant steam.

5. Add the galangal, ginger, lemongrass, turmeric, red chilies and lime
leaves to the mixture. Stir intermittently for about 10 minutes to
thoroughly incorporate the aromatics and heat them through. Add the
shredded coconut and steamed calabash squash, and stir to combine. Add
the stock, and stir to loosen the soup, then increase heat and bring it
up to a slow boil, stirring here and there to keep the solids from
sticking to the bottom of the pot.

6. Reduce the heat to a simmer, and cook the soup for 30 to 45 minutes,
stirring every 10 or 15 minutes. Turn the heat off, allowing the soup to
cool down to warm or slightly hot. Throw in a dash of salt, and purée in
a blender until smooth.

7. Using a large whisk to combine, season the soup with the powdered
coconut milk, jaggery and more salt, tasting as you go and adjusting as
needed.

*Lydon recommends getting coconut products ``where the only ingredient
on the packaging is coconut.'' Fresh, grated coconut is unsweetened and
sold frozen. It can be found at your local Indian grocery. Coconut-milk
powder is available there, too, as well as at most Southeast Asian
markets and health food stores, and via amazon.com.

\hypertarget{key-lime-piecicles}{%
\subsection{Key Lime Piecicles}\label{key-lime-piecicles}}

Lydon likes her key-lime flavor pronounced and tart, and suggests that
if you want it sweeter, you use less citrus juice. This recipe yields 13
treats. ``That one odd-numbered piecicle is for you,'' the pastry chef
notes. ``You need to make sure it tastes O.K. enough to share.''

\emph{Yield: a baker's dozen}

2 14-ounce cans sweetened condensed milk (preferably organic)\\
13 ounces fresh key-lime juice (1½ cups plus a splash)\\
1--2 pinches fine sea salt\\
4 cinnamon graham crackers (preferably organic), smashed

1. Whisk the sweetened condensed milk with the key lime juice to
combine. Add the salt, to taste, and whisk again. Pour about an inch of
this mixture into 13
\href{https://food52.com/shop/products/1272-ice-pop-molds}{ice-pop
molds}.

2. Spoon in about ½ inch of smashed graham crackers. Repeat once more,
but don't fill mold to the brim. Add one
\href{http://www.craftysticks.com/Standard-Food-Safe-Sticks-Natural_p_74.html}{ice-pop
stick} to each mold (there should be at least ¼ inch between the top lip
of the mold when your stick goes in). Freeze for at least 24 hours.

Advertisement

\protect\hyperlink{after-bottom}{Continue reading the main story}

\hypertarget{site-index}{%
\subsection{Site Index}\label{site-index}}

\hypertarget{site-information-navigation}{%
\subsection{Site Information
Navigation}\label{site-information-navigation}}

\begin{itemize}
\tightlist
\item
  \href{https://help.nytimes3xbfgragh.onion/hc/en-us/articles/115014792127-Copyright-notice}{©~2020~The
  New York Times Company}
\end{itemize}

\begin{itemize}
\tightlist
\item
  \href{https://www.nytco.com/}{NYTCo}
\item
  \href{https://help.nytimes3xbfgragh.onion/hc/en-us/articles/115015385887-Contact-Us}{Contact
  Us}
\item
  \href{https://www.nytco.com/careers/}{Work with us}
\item
  \href{https://nytmediakit.com/}{Advertise}
\item
  \href{http://www.tbrandstudio.com/}{T Brand Studio}
\item
  \href{https://www.nytimes3xbfgragh.onion/privacy/cookie-policy\#how-do-i-manage-trackers}{Your
  Ad Choices}
\item
  \href{https://www.nytimes3xbfgragh.onion/privacy}{Privacy}
\item
  \href{https://help.nytimes3xbfgragh.onion/hc/en-us/articles/115014893428-Terms-of-service}{Terms
  of Service}
\item
  \href{https://help.nytimes3xbfgragh.onion/hc/en-us/articles/115014893968-Terms-of-sale}{Terms
  of Sale}
\item
  \href{https://spiderbites.nytimes3xbfgragh.onion}{Site Map}
\item
  \href{https://help.nytimes3xbfgragh.onion/hc/en-us}{Help}
\item
  \href{https://www.nytimes3xbfgragh.onion/subscription?campaignId=37WXW}{Subscriptions}
\end{itemize}
