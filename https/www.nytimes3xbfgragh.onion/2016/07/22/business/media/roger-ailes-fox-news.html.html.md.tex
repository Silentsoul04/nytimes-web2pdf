Sections

SEARCH

\protect\hyperlink{site-content}{Skip to
content}\protect\hyperlink{site-index}{Skip to site index}

\href{https://www.nytimes3xbfgragh.onion/section/business/media}{Media}

\href{https://myaccount.nytimes3xbfgragh.onion/auth/login?response_type=cookie\&client_id=vi}{}

\href{https://www.nytimes3xbfgragh.onion/section/todayspaper}{Today's
Paper}

\href{/section/business/media}{Media}\textbar{}Roger Ailes Leaves Fox
News, and Rupert Murdoch Steps In

\url{https://nyti.ms/2ac7lZn}

\begin{itemize}
\item
\item
\item
\item
\item
\item
\end{itemize}

Advertisement

\protect\hyperlink{after-top}{Continue reading the main story}

Supported by

\protect\hyperlink{after-sponsor}{Continue reading the main story}

\hypertarget{roger-ailes-leaves-fox-news-and-rupert-murdoch-steps-in}{%
\section{Roger Ailes Leaves Fox News, and Rupert Murdoch Steps
In}\label{roger-ailes-leaves-fox-news-and-rupert-murdoch-steps-in}}

\includegraphics{https://static01.graylady3jvrrxbe.onion/images/2016/07/22/business/22AILES/22AILES-articleLarge.jpg?quality=75\&auto=webp\&disable=upscale}

By \href{https://www.nytimes3xbfgragh.onion/by/john-koblin}{John
Koblin}, \href{https://www.nytimes3xbfgragh.onion/by/emily-steel}{Emily
Steel} and \href{http://www.nytimes3xbfgragh.onion/by/jim-rutenberg}{Jim
Rutenberg}

\begin{itemize}
\item
  July 21, 2016
\item
  \begin{itemize}
  \item
  \item
  \item
  \item
  \item
  \item
  \end{itemize}
\end{itemize}

In the dark for days, Fox News staffers finally got word on Thursday
about the future of their network.

The news was delivered in person by Rupert Murdoch, the 85-year-old
media mogul who started Fox News with Roger Ailes 20 years ago.

It was an unexpected visit, and with stunned employees listening in
Fox's Midtown Manhattan headquarters, Mr. Murdoch announced that Mr.
Ailes was out as chairman and chief executive. Mr. Murdoch himself would
be taking over Fox News in the interim.

Mr. Ailes was not there. Mr. Murdoch had barred him from the building
starting on Wednesday, according to one person briefed on the matter.
The person said Fox News's parent company, 21st Century Fox, had learned
Mr. Ailes was trying to get some of his on-air stars to criticize those
who cooperated with investigators looking into accusations of sexual
harassment against him.

The announcement was the culmination of an unsettling 15-day stretch for
the network that began on July 6, when Gretchen Carlson, a former Fox
anchor, filed a lawsuit accusing Mr. Ailes of sexual harassment. That
led to an internal investigation by 21st Century Fox.

It was a stunning fall for one of the most powerful people in the media
industry, who built Fox News into a ratings juggernaut and a hugely
influential platform for Republican politics.

Mr. Ailes will walk away with about \$40 million as part of a settlement
agreement, according to two people briefed on the matter, which
essentially amounts to the remainder of his existing employment contract
through 2018. As part of the agreement, Mr. Ailes cannot start a
competitor to Fox News. He will continue to make himself available as an
adviser to Mr. Murdoch on an interim basis, the two people said, though
he will not be directly involved with Fox News or 21st Century Fox.

In a statement, Mr. Murdoch praised Mr. Ailes, 76, and his ``remarkable
contribution'' to the company, without making mention of the sexual
harassment scandal that felled him.

``Roger shared my vision of a great and independent television
organization and executed it brilliantly over 20 great years,'' Mr.
Murdoch said in a statement. ``Fox News has given voice to those who
were ignored by the traditional networks and has been one of the great
commercial success stories of modern media.''

Among those who cooperated with investigators looking into the
allegations against Mr. Ailes was one of his on-air stars, Megyn Kelly.
She had been among a small group of employees who resisted a campaign to
rally support for Mr. Ailes, which came to be viewed as a ``loyalty
test,'' according to several staff members, who declined to be
identified.

Ms. Kelly told investigators that she received repeated, unwanted
advances from Mr. Ailes, which she rejected, according to two people
briefed on her account. The entreaties, which happened in the early part
of her career at Fox, bothered Ms. Kelly to the point that she retained
a lawyer because she worried that her rejections would jeopardize her
job, though they ultimately did not.

In a statement earlier this week, Mr. Ailes's lawyer said he never
sexually harassed Ms. Kelly. During the investigation, led by the law
firm Paul, Weiss, Rifkind, Wharton \& Garrison, around 10 women have
come forward with stories of inappropriate conduct from Mr. Ailes while
at Fox News, according to a person briefed on the investigation.

In a letter to Mr. Murdoch on Thursday, Mr. Ailes wrote: ``Having spent
20 years building this historic business, I will not allow my presence
to become a distraction from the work that must be done every day to
ensure that Fox News and Fox Business continue to lead our industry.''

A copy of the letter was provided by Mr. Ailes's lawyer, Susan Estrich.
She did not respond to further requests for comment.

Though Mr. Ailes made no mention of the investigation into his workplace
behavior or the sexual harassment lawsuit, he said, pointedly: ``I take
particular pride in the role that I have played advancing the careers of
the many women I have promoted to executive and on-air positions. Many
of these talented journalists have deservedly become household names
known for their intelligence and strength, whether reporting the news,
fair and balanced, and offering exciting opinions on our opinion
programs.''

The terms of Mr. Ailes's departure were negotiated over several chaotic
days that transfixed the media world and spurred intense coverage. Mr.
Murdoch, on vacation in the French Riviera with his wife, Jerry Hall,
had been working in tandem with his sons, James and Lachlan, with whom
he leads 21st Century Fox, but it was not until he returned to New York
that a deal was reached.

Mr. Murdoch will assume the role of chairman and will be an interim
chief executive of the Fox News channel and Fox Business Network until a
permanent replacement for Mr. Ailes is found. His interim role is
intended to ensure stability during the rest of the presidential race,
and to be taken as a signal that the network is not on the verge of a
wide-ranging overhaul, said a person briefed on the matter. Mr. Murdoch
plans to be ``extremely engaged'' and had already been attending some
news meetings because Mr. Ailes has had health issues recently, the
person said.

In the same statement as their father, James and Lachlan Murdoch praised
Mr. Ailes but alluded to the trouble at Fox News, saying they were
committed ``to maintaining a work environment based on trust and
respect.'' Lachlan joined his father in the newsroom on Thursday, but
James was not present because of a previously scheduled business trip in
Europe.

Mr. Ailes's position atop Fox News was thrown into doubt two weeks ago
after
\href{http://www.nytimes3xbfgragh.onion/2016/07/13/business/media/gretchen-carlson-fox-news-interview.html}{Ms.
Carlson} filed a sexual harassment suit against him. Mr. Ailes denied
the accusations, but 21st Century Fox began the internal review and,
earlier this week, determined that he could no longer remain in the job.

The campaign to rally support for Mr. Ailes ultimately became a problem
for him. It included declarations casting doubt on Ms. Carlson's charges
from hosts including Greta Van Susteren, Jeanine Pirro and Neil Cavuto,
who in an op-ed described the accusations as ``sick.''

Several female staff members had said on Wednesday that they feared that
campaign was making younger female staff members with their own stories
to tell too frightened to speak with investigators --- something the
investigators feared as well, people briefed on their inquiry said this
week.

A friend of Ms. Kelly, who spoke on condition of anonymity, said that
Ms. Kelly resisted pressure to support Mr. Ailes, and cooperated with
the investigation so that those other staff members would ``feel more
comfortable coming forward to tell the truth.'' Ms. Kelly has not spoken
publicly about the matter.

On Thursday night, Kirsten Powers, a Fox contributor for 11 years, said:
``While I understand loyalty, I was disappointed that so many senior
members of Fox's on-air team rushed to defend Roger in a way that seemed
to prejudge an investigation into sexual harassment. I would hope that
in 2016 people would know that just because you weren't harassed, or
didn't witness harassment by a certain man, that doesn't mean it didn't
happen.''

For hundreds of Fox employees gathered in Cleveland for the Republican
National Convention, the week had been surreal. Those who left New York
just days ago will return this weekend to a workplace turned
upside-down.

``There are people in tears,'' said Chris Wallace, the host of ``Fox
News Sunday.'' ``I shed mine a couple of days ago.'' During an
interview, Mr. Wallace's anchor baritone occasionally grew faint; Bret
Baier, his colleague, was also emotional.

Notably, Mr. Baier and his fellow Fox anchors Brit Hume and Ms. Van
Susteren declined to say definitively whether they would remain at Fox,
although Mr. Baier, after some hesitation, said: ``I couldn't be
happier.'' Several of the network's most recognizable faces ---
including Bill O'Reilly, Ms. Kelly and Mr. Baier --- are known to have
contract clauses that allow them to leave the network if Mr. Ailes is
not in charge.

The lawyer for Ms. Carlson, Nancy Erika Smith, released a statement that
her client's ``extraordinary courage has caused a seismic shift in the
media world.''

She added, ``We hope that all businesses now understand that women will
no longer tolerate sexual harassment, and reputable companies will no
longer shield those who abuse women.''

Advertisement

\protect\hyperlink{after-bottom}{Continue reading the main story}

\hypertarget{site-index}{%
\subsection{Site Index}\label{site-index}}

\hypertarget{site-information-navigation}{%
\subsection{Site Information
Navigation}\label{site-information-navigation}}

\begin{itemize}
\tightlist
\item
  \href{https://help.nytimes3xbfgragh.onion/hc/en-us/articles/115014792127-Copyright-notice}{©~2020~The
  New York Times Company}
\end{itemize}

\begin{itemize}
\tightlist
\item
  \href{https://www.nytco.com/}{NYTCo}
\item
  \href{https://help.nytimes3xbfgragh.onion/hc/en-us/articles/115015385887-Contact-Us}{Contact
  Us}
\item
  \href{https://www.nytco.com/careers/}{Work with us}
\item
  \href{https://nytmediakit.com/}{Advertise}
\item
  \href{http://www.tbrandstudio.com/}{T Brand Studio}
\item
  \href{https://www.nytimes3xbfgragh.onion/privacy/cookie-policy\#how-do-i-manage-trackers}{Your
  Ad Choices}
\item
  \href{https://www.nytimes3xbfgragh.onion/privacy}{Privacy}
\item
  \href{https://help.nytimes3xbfgragh.onion/hc/en-us/articles/115014893428-Terms-of-service}{Terms
  of Service}
\item
  \href{https://help.nytimes3xbfgragh.onion/hc/en-us/articles/115014893968-Terms-of-sale}{Terms
  of Sale}
\item
  \href{https://spiderbites.nytimes3xbfgragh.onion}{Site Map}
\item
  \href{https://help.nytimes3xbfgragh.onion/hc/en-us}{Help}
\item
  \href{https://www.nytimes3xbfgragh.onion/subscription?campaignId=37WXW}{Subscriptions}
\end{itemize}
