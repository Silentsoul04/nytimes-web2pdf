Sections

SEARCH

\protect\hyperlink{site-content}{Skip to
content}\protect\hyperlink{site-index}{Skip to site index}

\href{https://www.nytimes3xbfgragh.onion/section/politics}{Politics}

\href{https://myaccount.nytimes3xbfgragh.onion/auth/login?response_type=cookie\&client_id=vi}{}

\href{https://www.nytimes3xbfgragh.onion/section/todayspaper}{Today's
Paper}

\href{/section/politics}{Politics}\textbar{}Donald Trump Proposes to
Double Hillary Clinton's Spending on Infrastructure

\url{https://nyti.ms/2aKWpmR}

\begin{itemize}
\item
\item
\item
\item
\item
\end{itemize}

Advertisement

\protect\hyperlink{after-top}{Continue reading the main story}

Supported by

\protect\hyperlink{after-sponsor}{Continue reading the main story}

\hypertarget{donald-trump-proposes-to-double-hillary-clintons-spending-on-infrastructure}{%
\section{Donald Trump Proposes to Double Hillary Clinton's Spending on
Infrastructure}\label{donald-trump-proposes-to-double-hillary-clintons-spending-on-infrastructure}}

\includegraphics{https://static01.graylady3jvrrxbe.onion/images/2016/08/03/us/03fd-trumpspend/03fd-trumpspend-articleLarge.jpg?quality=75\&auto=webp\&disable=upscale}

By \href{https://www.nytimes3xbfgragh.onion/by/alan-rappeport}{Alan
Rappeport}

\begin{itemize}
\item
  Aug. 2, 2016
\item
  \begin{itemize}
  \item
  \item
  \item
  \item
  \item
  \end{itemize}
\end{itemize}

Donald J. Trump took a step to Hillary Clinton's left on Tuesday, saying
that he would like to spend at least twice as much as his Democratic
opponent has proposed to invest in new infrastructure as part of his
plan to stimulate the United States' economy.

The idea takes a page out of the progressive playbook and is another
indication that the Republican presidential nominee is prepared to break
with the fiscal conservatism that his party has evangelized over the
past eight years.

``We have bridges that are falling down,'' Mr. Trump said on the Fox
Business Network. ``We have many, many bridges that are in danger of
falling.''

Mrs. Clinton has called for
\href{http://www.nytimes3xbfgragh.onion/politics/first-draft/2015/11/30/hillary-clinton-unveils-sprawling-and-expensive-infrastructure-investment-plan/}{\$275
billion in infrastructure spending} over five years. That would include
the creation of a national infrastructure bank, which would be given
\$25 billion to support loans and loan guarantees. In sum, the plan
would support about \$500 billion in spending on infrastructure.

Senator Bernie Sanders of Vermont, who competed against Mrs. Clinton for
the Democratic nomination,
\href{https://berniesanders.com/issues/creating-jobs-rebuilding-america/}{also
wanted to outspend her} on infrastructure, calling for a \$1 trillion
investment over five years.

Asked how he would pay for \$800 billion to \$1 trillion in
infrastructure spending, Mr. Trump described a strategy that has been
favored by liberal economists over the years. He said he would create an
infrastructure fund that would be supported by government bonds that
investors and citizens could purchase.

``We're going to go out with a fund,'' he said. ``We'll get a fund, make
a phenomenal deal with low interest rates and rebuild our
infrastructure.''

He added, ``We'd do infrastructure bonds from the country, from the
United States.''

If Mr. Trump's call for more spending sounds familiar, that could be
because Lawrence H. Summers, who was President Bill Clinton's Treasury
secretary and the director of President Obama's National Economic
Council, has been saying the same thing. At a Democratic National
Convention round table last week in Philadelphia, he said the United
States should invest between \$1 trillion and \$2 trillion in
infrastructure over the next 10 years.

\includegraphics{https://static01.graylady3jvrrxbe.onion/images/2016/08/03/us/03trumpheart/03trumpheart-videoSixteenByNineJumbo1600.jpg}

``It's the right thing for our children, it's the right thing for our
workers, and it's the right thing for our growth,''
\href{http://fortune.com/2016/07/27/larry-summers-2-trillion-for-infrastructure/}{Mr.
Summers said}.

Conservative critics of Mr. Trump expressed concern that the idea would
put the country deeper in debt and that it sounded alarmingly similar to
Mr. Obama's 2009 stimulus program.

``I'm old enough to remember when the people who supported Trump hated
this idea when Obama proposed it,'' said Leon Wolf, a commentator on the
conservative website Red State.

Part of that plan, which Republicans opposed, was the issuance of
``Build America Bonds,'' which allowed a broader group of investors to
help finance infrastructure projects in states and cities.

While Mr. Trump was not specific about his plan, the proposal of a
federal infrastructure bank is not new and has been supported by liberal
think tanks such as the
\href{https://www.brookings.edu/2012/07/16/what-would-an-infrastructure-bank-really-do/}{Brookings
Institution} as a way to improve ports, dams, water treatment facilities
and large urban redevelopment projects.

Right-leaning groups such as the
\href{https://www.aei.org/publication/fantasy-funding-financing-infrastructure/}{American
Enterprise Institute} have been more skeptical, arguing that users of
roads and bridges should pay for them through tolls.

Although Mr. Trump has said he would use his prowess as a builder to
redevelop infrastructure more efficiently than others could, an
investment of \$1 trillion could conflict with Mr. Trump's promises to
reduce the deficit.

``You'd be increasing the deficit for a significant period of time if
you were deploying that money,'' said Donald Marron, who was a member of
President George W. Bush's Council of Economic Advisers.

The Clinton campaign rejected the idea that Mr. Trump has serious ideas
for rebuilding American infrastructure. ``Donald Trump's only actual
infrastructure proposal is to build a giant wall on the Mexican border
and have Mexico pay for it,'' said Brian Fallon, a spokesman for Mrs.
Clinton. ``Only one candidate in this campaign has put forward a
specific and credible plan for investing in our nation's infrastructure,
and that's Hillary Clinton.''

Advertisement

\protect\hyperlink{after-bottom}{Continue reading the main story}

\hypertarget{site-index}{%
\subsection{Site Index}\label{site-index}}

\hypertarget{site-information-navigation}{%
\subsection{Site Information
Navigation}\label{site-information-navigation}}

\begin{itemize}
\tightlist
\item
  \href{https://help.nytimes3xbfgragh.onion/hc/en-us/articles/115014792127-Copyright-notice}{©~2020~The
  New York Times Company}
\end{itemize}

\begin{itemize}
\tightlist
\item
  \href{https://www.nytco.com/}{NYTCo}
\item
  \href{https://help.nytimes3xbfgragh.onion/hc/en-us/articles/115015385887-Contact-Us}{Contact
  Us}
\item
  \href{https://www.nytco.com/careers/}{Work with us}
\item
  \href{https://nytmediakit.com/}{Advertise}
\item
  \href{http://www.tbrandstudio.com/}{T Brand Studio}
\item
  \href{https://www.nytimes3xbfgragh.onion/privacy/cookie-policy\#how-do-i-manage-trackers}{Your
  Ad Choices}
\item
  \href{https://www.nytimes3xbfgragh.onion/privacy}{Privacy}
\item
  \href{https://help.nytimes3xbfgragh.onion/hc/en-us/articles/115014893428-Terms-of-service}{Terms
  of Service}
\item
  \href{https://help.nytimes3xbfgragh.onion/hc/en-us/articles/115014893968-Terms-of-sale}{Terms
  of Sale}
\item
  \href{https://spiderbites.nytimes3xbfgragh.onion}{Site Map}
\item
  \href{https://help.nytimes3xbfgragh.onion/hc/en-us}{Help}
\item
  \href{https://www.nytimes3xbfgragh.onion/subscription?campaignId=37WXW}{Subscriptions}
\end{itemize}
