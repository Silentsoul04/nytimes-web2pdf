Sections

SEARCH

\protect\hyperlink{site-content}{Skip to
content}\protect\hyperlink{site-index}{Skip to site index}

\href{https://www.nytimes3xbfgragh.onion/section/food}{Food}

\href{https://myaccount.nytimes3xbfgragh.onion/auth/login?response_type=cookie\&client_id=vi}{}

\href{https://www.nytimes3xbfgragh.onion/section/todayspaper}{Today's
Paper}

\href{/section/food}{Food}\textbar{}Gramercy Tavern: A Classic Still on
the Move

\url{https://nyti.ms/2bkGJHo}

\begin{itemize}
\item
\item
\item
\item
\item
\item
\end{itemize}

Advertisement

\protect\hyperlink{after-top}{Continue reading the main story}

Supported by

\protect\hyperlink{after-sponsor}{Continue reading the main story}

\href{/column/restaurant-review}{Restaurant Review}

\hypertarget{gramercy-tavern-a-classic-still-on-the-move}{%
\section{Gramercy Tavern: A Classic Still on the
Move}\label{gramercy-tavern-a-classic-still-on-the-move}}

\href{https://www.nytimes3xbfgragh.onion/slideshow/2016/08/17/dining/gramercy-tavern-nyc.html}{}

\hypertarget{gramercy-tavern}{%
\subsection{Gramercy Tavern}\label{gramercy-tavern}}

12 Photos

View Slide Show ›

\includegraphics{https://static01.graylady3jvrrxbe.onion/images/2016/08/17/dining/17REST-GRAMERCY-slide-H7AU/17REST-GRAMERCY-slide-H7AU-articleLarge.jpg?quality=75\&auto=webp\&disable=upscale}

Francesco Sapienza for The New York Times

\begin{itemize}
\tightlist
\item
  Gramercy Tavern\\
  ★★★ American \$\$\$\$ 42 East 20th Street 212-477-0777
\end{itemize}

\href{http://www.opentable.com/single.aspx?ref=4201\&rid=942}{Reserve a
Table}

When you make a reservation at an independently reviewed restaurant
through our site, we earn an affiliate commission.

By \href{http://www.nytimes3xbfgragh.onion/by/pete-wells}{Pete Wells}

\begin{itemize}
\item
  Aug. 16, 2016
\item
  \begin{itemize}
  \item
  \item
  \item
  \item
  \item
  \item
  \end{itemize}
\end{itemize}

Long before it turned 22 last month,
\href{http://www.gramercytavern.com/}{Gramercy Tavern} had settled into
classic status. It is a classic, but there's a downside to putting a
restaurant in that category. The more familiar something is, the harder
it is to see.

Like the other restaurants in Danny Meyer's
\href{http://www.ushgnyc.com/}{Union Square Hospitality Group}, Gramercy
Tavern is virtually synonymous with a certain style of service.
Cheerful, wholesome, engaging, proactive, the servers are the sort you'd
be comfortable leaving young children with. (Mr. Meyer could run a great
babysitting company.) Whether we walk into the easygoing front tavern or
reserve a table in the more expensive and ceremonious dining room in the
back, we take for granted that the service will be very good.

When that's all it is, and if we forget that Gramercy Tavern on a very
good night is still better than most restaurants when they try their
hardest, it's possible to be mildly disappointed. The staff seems to
understand that high expectations work against the place, though, and
sometimes you can be treated to next-level hospitality.

One recent night, our table had interrupted the march from main courses
to desserts by sharing a small cheese plate. A cheese interruption is
never a bad idea at Gramercy Tavern.

What I really went wild for, though, was a simple condiment that arrived
with the cheeses, a little dish of sour-cherry preserves. I ate it on a
wedge of aged goat, I ate it on crackers, and after they were gone, I
dropped red jam onto my finger from a spoon and slipped the finger into
my mouth.

To make space for the desserts, the cheese plate was soon cleared away.
Not the jam dish, though, because I had covered it with my hand and
smiled in a way that said, ``You'll have to kill me first.''

This being Gramercy Tavern, I knew I'd win without a fight. But I didn't
expect a manager to swing by a few minutes later and say: ``I told Miro
how much you enjoyed the cherry jam, and he was thrilled. It's his
grandmother's recipe, from Serbia, and he loves making it.''

As if by magic, my piggishness had been turned into a compliment to the
pastry chef, Miro Uskokovic, and my happiness had become the
restaurant's. This is what's meant by the cliché ``It's our pleasure to
serve you.'' But it takes in-the-moment intelligence to bring that
phrase to life. Gramercy Tavern can do that.

Maybe I paid a little more attention to Mr. Uskokovic's desserts from
then on, noticing how gingered pineapple and crumbled pralines
brightened the wonderful carrot cake, appreciating the innocent
sweetness that angel food croutons brought to a cheesecake topped with
dark, soft strawberries. Maybe I also enjoyed the desserts more because
I was in a good mood. They had the openness and generosity of spirit
that
\href{https://www.nytimes3xbfgragh.onion/2015/08/05/dining/restaurant-review-untitled-at-the-whitney-in-the-meatpacking-district.html}{I'd
admired at Untitled}, where Mr. Uskokovic is also the pastry chef.

The executive chef at both restaurants is Michael Anthony. In 2006, he
took over at Gramercy Tavern from Tom Colicchio, who had run the kitchen
from the beginning.

As the torch passed, it sputtered. In
\href{http://www.nytimes3xbfgragh.onion/2007/06/06/dining/reviews/06rest.html}{Gramercy
Tavern's last review in The Times}, nine years ago, Frank Bruni wrote
about ``forgettable'' meals at the end of Mr. Colicchio's tenure and a
``clumsy, laughable one'' at the start of Mr. Anthony's.

The new team finally hit its stride, and Mr. Bruni's appraisal carried
three stars. But the kitchen's sensibility changed much more
significantly than a restaurant trying to remain classic would normally
allow.

Where Mr. Colicchio would often focus on a star ingredient, Mr.
Anthony's dishes tend to be ensemble efforts. It's as if he has fallen
in love with everything he puts on the plate. He's like somebody who
goes to the farmers' market hungry, ends up buying everything he sees,
then figures out a way to make it all come together.

A bin of small hot red peppers must have jumped into his shopping bag in
early August, when he made them into an unusually forceful and perfectly
harmonious foil for corn and very tender poached lobster. What to do
with red currants, so pretty and so resistant to the spotlight? Mr.
Anthony scattered them over a chilled marigold-yellow soup of summer
squash, which needed their jolt of juicy tartness.

Mr. Anthony's keen sense of what to do with produce was on full display
in the \$110 vegetable tasting menu, of course. But I also saw it in the
\$125 tasting, where blackberries helped out a braised pork shoulder
paved with crackling hazelnuts and where a supple hunk of halibut sat
over warm summer tomatoes given a briny, oceanic intensity by scraps of
kombu. Even over three courses at a fixed price of \$98, Mr. Anthony
gives you a sense of what's going on that week in the vegetable patch
that few chefs can match.

These are the menus offered in the back dining room. But the keen
responsiveness to the season still comes across if you order à la carte
at the long, perennially busy bar or one of the tables next to it. I
loved the taste of corn kernels and green tomatoes, sliced into
see-through wheels, on a flatbread dotted with lamb sausage, and the way
Mr. Anthony made room for cool peaches in a tomato-basil salad.

For the Tavern's dessert menu, Mr. Uskokovic bakes a pie that has
pastry-crust stars on top of a dark field of wild blueberries --- an
edible flag. It would be hokey if it weren't so good.

The barroom is more than a low-priced alternative; it's integral to the
restaurant's personality. For example, it sets the tone for the way
people drink in both rooms. Gramercy Tavern's wine list is justly
celebrated for its catholic embrace of diverse regions and styles, but
how many other restaurants make cider or a glass of beer (on draft or
from one of the vintage bottles) seem like a natural choice with a
dinner that costs \$100 or more?

Although it's not as plush as the dining room, the tavern is more
attractive. The colors of the fruits and vegetables in Robert Kushner's
scrolling mural above the bar still leap out at you after all these
years. In the back rooms, the collision of Italianate arcades and ye
olde timbered beams is more clearly a pastiche than it once seemed.

The appeal of Gramercy Tavern transcends design, though. It springs from
a sense of trustworthiness that is kept vital by acts of imagination
like the manager's report about the cherry jam and Mr. Anthony's
constant adjustments to the season.

You can tell that people value that combination if you study the clothes
they wear to eat in the front room. With a wait for tables that may be
right next to the crowd at the bar, the setup there is not too different
from
\href{http://www.nytimes3xbfgragh.onion/2016/04/20/dining/salvation-burger-spotted-pig-review.html}{the
Spotted Pig}'s, but most dress as if they were going to one of the
nicest restaurants in New York. And they're right.

Advertisement

\protect\hyperlink{after-bottom}{Continue reading the main story}

\hypertarget{site-index}{%
\subsection{Site Index}\label{site-index}}

\hypertarget{site-information-navigation}{%
\subsection{Site Information
Navigation}\label{site-information-navigation}}

\begin{itemize}
\tightlist
\item
  \href{https://help.nytimes3xbfgragh.onion/hc/en-us/articles/115014792127-Copyright-notice}{©~2020~The
  New York Times Company}
\end{itemize}

\begin{itemize}
\tightlist
\item
  \href{https://www.nytco.com/}{NYTCo}
\item
  \href{https://help.nytimes3xbfgragh.onion/hc/en-us/articles/115015385887-Contact-Us}{Contact
  Us}
\item
  \href{https://www.nytco.com/careers/}{Work with us}
\item
  \href{https://nytmediakit.com/}{Advertise}
\item
  \href{http://www.tbrandstudio.com/}{T Brand Studio}
\item
  \href{https://www.nytimes3xbfgragh.onion/privacy/cookie-policy\#how-do-i-manage-trackers}{Your
  Ad Choices}
\item
  \href{https://www.nytimes3xbfgragh.onion/privacy}{Privacy}
\item
  \href{https://help.nytimes3xbfgragh.onion/hc/en-us/articles/115014893428-Terms-of-service}{Terms
  of Service}
\item
  \href{https://help.nytimes3xbfgragh.onion/hc/en-us/articles/115014893968-Terms-of-sale}{Terms
  of Sale}
\item
  \href{https://spiderbites.nytimes3xbfgragh.onion}{Site Map}
\item
  \href{https://help.nytimes3xbfgragh.onion/hc/en-us}{Help}
\item
  \href{https://www.nytimes3xbfgragh.onion/subscription?campaignId=37WXW}{Subscriptions}
\end{itemize}
