Sections

SEARCH

\protect\hyperlink{site-content}{Skip to
content}\protect\hyperlink{site-index}{Skip to site index}

\href{https://www.nytimes3xbfgragh.onion/section/us}{U.S.}

\href{https://myaccount.nytimes3xbfgragh.onion/auth/login?response_type=cookie\&client_id=vi}{}

\href{https://www.nytimes3xbfgragh.onion/section/todayspaper}{Today's
Paper}

\href{/section/us}{U.S.}\textbar{}`Shadow Brokers' Leak Raises Alarming
Question: Was the N.S.A. Hacked?

\url{https://nyti.ms/2blOYD8}

\begin{itemize}
\item
\item
\item
\item
\item
\item
\end{itemize}

Advertisement

\protect\hyperlink{after-top}{Continue reading the main story}

Supported by

\protect\hyperlink{after-sponsor}{Continue reading the main story}

\hypertarget{shadow-brokers-leak-raises-alarming-question-was-the-nsa-hacked}{%
\section{`Shadow Brokers' Leak Raises Alarming Question: Was the N.S.A.
Hacked?}\label{shadow-brokers-leak-raises-alarming-question-was-the-nsa-hacked}}

\includegraphics{https://static01.graylady3jvrrxbe.onion/images/2016/08/17/us/17hack/17hack-articleInline.jpg?quality=75\&auto=webp\&disable=upscale}

By \href{http://www.nytimes3xbfgragh.onion/by/david-e-sanger}{David E.
Sanger}

\begin{itemize}
\item
  Aug. 16, 2016
\item
  \begin{itemize}
  \item
  \item
  \item
  \item
  \item
  \item
  \end{itemize}
\end{itemize}

The release on websites this week of what appears to be top-secret
computer code that the National Security Agency has used to break into
the networks of foreign governments and other espionage targets has
caused deep concern inside American intelligence agencies, raising the
question of whether America's own elite operatives have been hacked and
their methods revealed.

Most outside experts who examined the posts, by a group calling itself
the Shadow Brokers, said they contained what appeared to be genuine
samples of the code --- though somewhat outdated --- used in the
production of the N.S.A.'s custom-built malware.

Most of the code was designed to break through network firewalls and get
inside the computer systems of competitors like Russia, China and Iran.
That, in turn, allows the N.S.A. to place ``implants'' in the system,
which can lurk unseen for years and be used to monitor network traffic
or enable a debilitating computer attack.

According to these experts, the coding resembled a series of
``products'' developed inside the N.S.A.'s highly classified Tailored
Access Operations unit, some of which were described in general terms in
documents stolen three years ago by Edward J. Snowden, the former N.S.A.
contractor now living in Russia.

But the code does not appear to have come from Mr. Snowden's archive,
which was mostly composed of PowerPoint files and other documents that
described N.S.A. programs. The documents released by Mr. Snowden and his
associates contained no actual source code used to break into the
networks of foreign powers.

Whoever obtained the source code apparently broke into either the
top-secret, highly compartmentalized computer servers of the N.S.A. or
other servers around the world that the agency would have used to store
the files. The code that was published on Monday dates to mid-2013,
when, after Mr. Snowden's disclosures, the agency shuttered many of its
existing servers and moved code to new ones as a security measure.

By midday Tuesday Mr. Snowden himself, in a Twitter message from his
exile in Moscow, declared that ``circumstantial evidence and
conventional wisdom indicates Russian responsibility'' for publication,
which he interpreted as a warning shot to the American government in
case it was thinking of imposing sanctions against Russia in the
cybertheft of documents from the Democratic National Committee.

``Why did they do it?'' Mr. Snowden asked. ``No one knows, but I suspect
this is more diplomacy than intelligence, related to the escalation
around the DNC hack.''

Around the same time, WikiLeaks declared that it had a full set of the
files --- it did not say how it had obtained them --- and would release
them all in the future. The ``Shadow Brokers'' had said they would
auction them off to the highest bidder.

``I think it's Snowden-era stuff, repackaged for resale now,'' said
James A. Lewis, a computer expert at the Center for Strategic and
International Studies, a Washington think tank. ``This is probably some
Russian mind game, down to the bogus accent'' of some of the messages
sent to media organizations by the Shadow Brokers group, delivered in
broken English that seemed right out of a bad spy movie.

The N.S.A. would say nothing on Tuesday about whether the coding
released was real or where it came from. Its public affairs office did
not respond to inquiries.

``It certainly feels all real,'' said Bruce Schneier, a leading
authority on state-sponsored breaches. ``The question is why would
someone steal it in 2013 and release it this week? That's what is making
people think this is likely the work of Russian intelligence.''

There are other theories, including one that some unknown group was
trying to impersonate hackers working for Russian or other intelligence
agencies. Impersonation is relatively easy on the internet, and it could
take considerable time to determine who is behind the release of the
code.

The Shadow Brokers first emerged online on Saturday, creating accounts
on sites like Twitter and Tumblr and announcing plans for an auction.
The group said that ``we give you some Equation Group files free'' and
that it would auction the best ones. The Equation Group is a code name
that Kaspersky Labs, a Russian cybersecurity firm, has given to the
N.S.A.

While still widely considered the most talented group of state-sponsored
hackers in the world, the N.S.A. is still recovering from Mr. Snowden's
disclosures; it has spent hundreds of millions of dollars reconfiguring
and locking down its systems.

Mr. Snowden revealed plans, code names and some operations, including
against targets like China. The Shadow Brokers disclosures are much more
detailed, the actual code and instructions for breaking into foreign
systems as of three summers ago.

``From an operational standpoint, this is not a catastrophic leak,''
Nicholas Weaver, a researcher at the International Computer Science
Institute in Berkeley, Calif.,
\href{https://www.lawfareblog.com/very-bad-monday-nsa-0}{wrote on the
Lawfare blog} on Tuesday.

But he added that ``the big picture is a far scarier one.'' In the weeks
after Mr. Snowden fled Hawaii, landing in Hong Kong before ultimately
going to Russia, it appears that someone obtained that source code.
That, he suggested, would be an even bigger security breach for the
N.S.A. than Mr. Snowden's departure with his trove of files.

However, the fact that the code is dated from 2013 suggests that the
hackers' access was cut off around then, perhaps because the agency
imposed new security measures.

The attack on the Democratic National Committee has raised questions
about whether the Russian government is trying to influence the American
election. If so, it is unclear how --- or whether --- President Obama
will respond. A response could be public or private, and it could
involve sanctions, diplomatic warnings or even a counterattack.

``The real problem for us is that the Russians seem to have taken the
gloves off in the cyberdomain,'' said Mr. Lewis, of the Center for
Strategic and International Studies, ``and we don't know how to
respond.''

Advertisement

\protect\hyperlink{after-bottom}{Continue reading the main story}

\hypertarget{site-index}{%
\subsection{Site Index}\label{site-index}}

\hypertarget{site-information-navigation}{%
\subsection{Site Information
Navigation}\label{site-information-navigation}}

\begin{itemize}
\tightlist
\item
  \href{https://help.nytimes3xbfgragh.onion/hc/en-us/articles/115014792127-Copyright-notice}{©~2020~The
  New York Times Company}
\end{itemize}

\begin{itemize}
\tightlist
\item
  \href{https://www.nytco.com/}{NYTCo}
\item
  \href{https://help.nytimes3xbfgragh.onion/hc/en-us/articles/115015385887-Contact-Us}{Contact
  Us}
\item
  \href{https://www.nytco.com/careers/}{Work with us}
\item
  \href{https://nytmediakit.com/}{Advertise}
\item
  \href{http://www.tbrandstudio.com/}{T Brand Studio}
\item
  \href{https://www.nytimes3xbfgragh.onion/privacy/cookie-policy\#how-do-i-manage-trackers}{Your
  Ad Choices}
\item
  \href{https://www.nytimes3xbfgragh.onion/privacy}{Privacy}
\item
  \href{https://help.nytimes3xbfgragh.onion/hc/en-us/articles/115014893428-Terms-of-service}{Terms
  of Service}
\item
  \href{https://help.nytimes3xbfgragh.onion/hc/en-us/articles/115014893968-Terms-of-sale}{Terms
  of Sale}
\item
  \href{https://spiderbites.nytimes3xbfgragh.onion}{Site Map}
\item
  \href{https://help.nytimes3xbfgragh.onion/hc/en-us}{Help}
\item
  \href{https://www.nytimes3xbfgragh.onion/subscription?campaignId=37WXW}{Subscriptions}
\end{itemize}
