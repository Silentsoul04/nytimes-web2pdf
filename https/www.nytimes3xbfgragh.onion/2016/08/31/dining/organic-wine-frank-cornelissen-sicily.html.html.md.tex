Sections

SEARCH

\protect\hyperlink{site-content}{Skip to
content}\protect\hyperlink{site-index}{Skip to site index}

\href{https://www.nytimes3xbfgragh.onion/section/food}{Food}

\href{https://myaccount.nytimes3xbfgragh.onion/auth/login?response_type=cookie\&client_id=vi}{}

\href{https://www.nytimes3xbfgragh.onion/section/todayspaper}{Today's
Paper}

\href{/section/food}{Food}\textbar{}The Evolution of a Natural Winemaker

\url{https://nyti.ms/2bJ72a9}

\begin{itemize}
\item
\item
\item
\item
\item
\end{itemize}

Advertisement

\protect\hyperlink{after-top}{Continue reading the main story}

Supported by

\protect\hyperlink{after-sponsor}{Continue reading the main story}

\href{/column/the-pour}{The Pour}

\hypertarget{the-evolution-of-a-natural-winemaker}{%
\section{The Evolution of a Natural
Winemaker}\label{the-evolution-of-a-natural-winemaker}}

\includegraphics{https://static01.graylady3jvrrxbe.onion/images/2016/08/31/dining/31POUR1/31pour1-articleInline.jpg?quality=75\&auto=webp\&disable=upscale}

By \href{http://www.nytimes3xbfgragh.onion/by/eric-asimov}{Eric Asimov}

\begin{itemize}
\item
  Aug. 25, 2016
\item
  \begin{itemize}
  \item
  \item
  \item
  \item
  \item
  \end{itemize}
\end{itemize}

SOLICCHIATA, Sicily --- When a writer comes calling, most winemakers
would consider it in their best interest to offer their proudest efforts
for sampling. But Frank Cornelissen has never been like most winemakers.

Instead Mr. Cornelissen, who has a reputation as the most unyielding of
natural winemakers, thinks it's more instructive to taste his failures.
This explains how we came to drink his 2006 Magma Rosso, made from old
nerello mascalese vines grown organically in the foothills of Mount
Etna, as we sat in a restaurant in June outside this small town on the
north face of Etna.

Magma is his top wine, made virtually by hand and aged in terra-cotta
amphorae buried in volcanic rock. It's made without additives, filtering
or sulfur dioxide, the preservative used in almost all conventional
wines. New vintages sell for around \$250.

The 2006 vintage was very warm on Etna, and the wine showed the effects
of the heat. It was overpowering, above 17 percent alcohol, and though
it was indisputably complex and concentrated, with the aromas and
flavors of dried cherries and minerals, the heat of the alcohol made it
hard to enjoy. It also showed quite a bit of volatile acidity. Not a
failure, maybe, as the wine has
\href{http://www.wineanorak.com/frank_cornelissen.htm}{received positive
reviews} in the past, but not a long-term success, either.

``I'm better in cold vintages,'' said Mr. Cornelissen, who has lived and
worked in the area since arriving from Belgium in 2000. ``I just can't
get hot vintages right. It's hard for me to pick before complete
ripeness, and 2006 was just heavy.''

The uncharitable may suggest that Mr. Cornelissen has had plenty of
opportunity to taste his mistakes. He may even agree.

Certainly during his early days on Etna, when Mr. Cornelissen was
experimenting and finding his way, some bottles were oxidized or had
other flaws. When I first started to follow his wines (with the 2007
vintage), I was struck by their unpredictability. One bottle could be
astonishingly beautiful, while the next could be odd and almost devoid
of pleasure.

For our dinner, he also brought a 2004 Contadino, a sort of translucent
dark rosé in which 30 percent of the blend is white grapes. This wine,
an entry-level Cornelissen that sells for \$25 to \$30 nowadays, seemed
a bit tired at first. You wouldn't think it would age. Unexpectedly,
after a few minutes in the glass, it became joyous and exuberant, with
the beguiling aroma of pressed flowers. Mr. Cornelissen noted it was a
cooler vintage.

Perhaps as surprising as this modest wine, Mr. Cornelissen over the
years has also evolved, and his winemaking has matured. Far from the
dogmatic winemaker many imagine, he has let experience dictate which
methods work and which do not. He is no less dedicated to making wines
without artifice that he believes will convey the essence of Etna. But
he has become more precise in the vineyard and the cellar, and his wines
have grown more consistent and expressive of where the grapes are grown.

Nonetheless, his reputation lingers. Earlier this year,
\href{http://www.vogue.com/13427836/natural-wine-industry-chemical-free/}{Vogue
referred to him} as ``a Belgian fanatic,'' and said his wines ``taste
like the aftermath of a volcanic eruption.''

``Today, there's no comparison between my current wines and what I was
doing back then,'' he said. ``It's the accumulation of small details
over 15 years. In Japan, there's a saying: A lot of dust particles make
a mountain.''

His ideals remain the same. ``Liquid rock,'' he said. ``That is my
vision.''

His 2013 Magma, from another cool vintage, could not have been more
different from the 2006. It was elegant, subtle and refined, with great
finesse, purity and character, and the aromas and flavors of flowers,
red fruit and minerals. The alcohol level was at least 15 percent, but I
did not feel the heat during our dinner.

Earlier in the day we had walked the vineyard in Contrada Barbabecchi
where the grapes for Magma are produced. Originally it was planted
around 1910 on silty, sandy soil in a sort of amphitheater facing north
and east, about 3,000 feet in elevation near the limit at which nerello
mascalese can ripen. Higher up are chestnut and olive trees.

``There were no roads up here until 1970,'' Mr. Cornelissen said.

Before that, workers walked miles from the villages below, sometimes
with their mules. They would spend nights in the small stone shacks that
dot the higher-elevation vineyards.

Looking north across the valley toward the Tyrrhenian Sea, the view is
blocked by the Nebrodi mountains.

``The special thing is not Etna,'' he said. ``It's those mountains. They
form a barrier that protects against rain and wind.''

As Mr. Cornelissen has advanced, so has the Etna region. As recently as
2000, coinciding with his arrival, most of the wine produced on Etna was
sold in bulk. Mr. Cornelissen, along with others like Andrea Franchetti
of \href{http://www.passopisciaro.com/home/}{Passopisciaro} and Marco de
Grazia of \href{http://www.tenutaterrenere.com/en/}{Tenuta delle Terre
Nere}, were among the wave of newcomers that drew attention to the wines
of Etna.

In their own idiosyncratic ways, they made the case that Etna, ignored
by the world of fine wines, and with many old vineyards that had largely
been neglected or abandoned by the locals, could be a source of
beautiful, distinctive wines.

Mr. Cornelissen came to Etna as a winemaking novice because of those old
vineyards and the relative absence of a wine establishment and
infrastructure. That gave him the freedom to find his own way.

``I refused to work in another winery,'' he said. ``I didn't want to
work with a biased mind. You learn what works in one place, but it
doesn't work somewhere else.''

After the vineyard, we visited his spotless white cellar to taste. Along
with the amphorae, which he uses for only his most tannic wines, he
employs fiberglass tanks for fermentation and aging. He scoffed when I
asked about using an unnatural material like plastic.

``Archaic wines are not my goal,'' he said. ``I use the best of the old
and the best of the new. I don't mind high tech, as long as it helps me
reach my goal of making wine with cultural baggage.''

Today Mr. Cornelissen waits much longer before bottling his wines. His
2014 \href{http://www.frankcornelissen.it/eng_production.htm}{MunJebel
reds}, his midlevel wines that retail for \$45 to \$75, are still in the
tanks. They will be bottled as single-vineyard or contrada wines,
reflecting the particular vineyard or regional terroirs. While all are
made of nerello mascalese, they are strikingly different.

The Monte Colla, from a steep vineyard that Mr. Cornelissen said is not
on volcanic soil, was the most unusual, tasting like blackberries,
chocolate and earth, with mint and mineral overtones. The Feudo di Mezzo
was tight, dark and smelled like violets, while the Chiusa Spagnolo was
earthier, denser and more tannic, also with the aroma of violets. We
also tasted the disarmingly pretty 2014 Magma, which was simultaneously
structured, tannic and complex.

Ultimately, Mr. Cornelissen hopes his wines will show density, finesse
and profundity while evolving over time. But he concedes that he --- and
Etna --- still have a way to go.

``I'm not talking about holding power but gaining complexity and harmony
over 20 years,'' he said. ``I don't think we have arrived yet.''

Advertisement

\protect\hyperlink{after-bottom}{Continue reading the main story}

\hypertarget{site-index}{%
\subsection{Site Index}\label{site-index}}

\hypertarget{site-information-navigation}{%
\subsection{Site Information
Navigation}\label{site-information-navigation}}

\begin{itemize}
\tightlist
\item
  \href{https://help.nytimes3xbfgragh.onion/hc/en-us/articles/115014792127-Copyright-notice}{©~2020~The
  New York Times Company}
\end{itemize}

\begin{itemize}
\tightlist
\item
  \href{https://www.nytco.com/}{NYTCo}
\item
  \href{https://help.nytimes3xbfgragh.onion/hc/en-us/articles/115015385887-Contact-Us}{Contact
  Us}
\item
  \href{https://www.nytco.com/careers/}{Work with us}
\item
  \href{https://nytmediakit.com/}{Advertise}
\item
  \href{http://www.tbrandstudio.com/}{T Brand Studio}
\item
  \href{https://www.nytimes3xbfgragh.onion/privacy/cookie-policy\#how-do-i-manage-trackers}{Your
  Ad Choices}
\item
  \href{https://www.nytimes3xbfgragh.onion/privacy}{Privacy}
\item
  \href{https://help.nytimes3xbfgragh.onion/hc/en-us/articles/115014893428-Terms-of-service}{Terms
  of Service}
\item
  \href{https://help.nytimes3xbfgragh.onion/hc/en-us/articles/115014893968-Terms-of-sale}{Terms
  of Sale}
\item
  \href{https://spiderbites.nytimes3xbfgragh.onion}{Site Map}
\item
  \href{https://help.nytimes3xbfgragh.onion/hc/en-us}{Help}
\item
  \href{https://www.nytimes3xbfgragh.onion/subscription?campaignId=37WXW}{Subscriptions}
\end{itemize}
