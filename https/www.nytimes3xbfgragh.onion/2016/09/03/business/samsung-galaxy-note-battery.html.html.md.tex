Sections

SEARCH

\protect\hyperlink{site-content}{Skip to
content}\protect\hyperlink{site-index}{Skip to site index}

\href{https://www.nytimes3xbfgragh.onion/section/business}{Business}

\href{https://myaccount.nytimes3xbfgragh.onion/auth/login?response_type=cookie\&client_id=vi}{}

\href{https://www.nytimes3xbfgragh.onion/section/todayspaper}{Today's
Paper}

\href{/section/business}{Business}\textbar{}Samsung to Recall 2.5
Million Galaxy Note 7s Over Battery Fires

\url{https://nyti.ms/2c6mVYs}

\begin{itemize}
\item
\item
\item
\item
\item
\end{itemize}

Advertisement

\protect\hyperlink{after-top}{Continue reading the main story}

Supported by

\protect\hyperlink{after-sponsor}{Continue reading the main story}

\hypertarget{samsung-to-recall-25-million-galaxy-note-7s-over-battery-fires}{%
\section{Samsung to Recall 2.5 Million Galaxy Note 7s Over Battery
Fires}\label{samsung-to-recall-25-million-galaxy-note-7s-over-battery-fires}}

\includegraphics{https://static01.graylady3jvrrxbe.onion/images/2016/09/03/world/03Samsung-web/03Samsung-web-videoSixteenByNineJumbo1600.jpg}

By \href{https://www.nytimes3xbfgragh.onion/by/paul-mozur}{Paul Mozur}
and Su-Hyun Lee

\begin{itemize}
\item
  Sept. 2, 2016
\item
  \begin{itemize}
  \item
  \item
  \item
  \item
  \item
  \end{itemize}
\end{itemize}

HONG KONG --- A ubiquitous source of power in most modern technology,
lithium-ion batteries keep cellphones, laptops, electric cars and
airplanes running. They are also the source of many problems, with some
overheating, catching fire and even exploding.

In a potentially damaging episode, Samsung, the world's biggest maker of
smartphones, announced on Friday that it would recall its Galaxy Note 7
model after discovering a flaw in the battery cell that could result in
fires. The company will replace 2.5 million phones sent to stores and
consumers, in one of the industry's largest recalls.

The recall puts Samsung, which has been trying to match the success of
the Apple iPhone, in a precarious position.

The smartphone industry is grappling with slowing demand and intense
competition. Samsung was regaining swagger with its high-end phone
models, like the Note 7, in which the screens appear to spill off the
side.

But the battery fires threaten to undermine Samsung's efforts, giving an
edge to Apple. The recall comes just days before Apple is expected to
unveil the latest version of its iPhone.

The ultimate scale of the damage to Samsung's reputation and finances
will depend on how quickly the company deals with the issues and how
costly they turn out to be. Along with the expense of fixing the phones,
Samsung could face lost sales if consumers grow wary of its products.

Samsung said it expected that manufacturing replacement phones would
take two weeks. Consumers who have already bought the phones will
receive replacements before new phones go on sale, the company said.
Samsung did not indicate the cost of the recall.

``If you look at previous instances in tech history where there have
been recalls, as long as it doesn't drag on to the point that the
company becomes the butt of a joke, then it should be minor,'' said
Bryan Ma, an analyst at IDC, a technology research firm.

``If it becomes like a Pinto, where you don't want to buy it because it
explodes, that would be a bad situation,'' he said, referring to the
1971 Ford car that became famous for erupting in flames after rear-end
collisions. ``But I think they'll get past it.''

Samsung said that, so far, 35 battery episodes involving the Note 7 had
been registered. Reports of the problem first started to emerge online,
as consumers posted photographs and videos of the charred remains of
phones they said had burst into flames, usually while being recharged.

\href{https://www.youtube.com/watch?v=9rScIV7m24s}{In one video}, a
customer shows a half-melted phone, and explains: ``Came home after
work, put it to charge for a little bit before I had class, went to put
it on my waist, and it caught fire. Yep, brand new phone, not even two
weeks old.''

Samsung said it thought the problem came from a ``minute flaw'' in the
production of the batteries. Samsung would not name the supplier
involved.

The recall covers 10 countries where the phones have been sold. Samsung
said the recall would not affect China, since the models sold there used
a battery from a different supplier.

``We acknowledge the inconvenience this may cause in the market, but
this is to ensure that Samsung continues to deliver the highest-quality
products to our customers,'' the company said in a statement. ``We are
working closely with our partners to ensure the replacement experience
is as convenient and efficient as possible.''

The recall is a major blow to Samsung, which had just started to regain
its competitive footing. The company faces pressures across its product
line, with Apple on the high end and new Chinese brands on the lower
end.

Samsung was gaining traction with the latest Galaxy phones. The phones'
smooth, tapered edges, which make them more comfortable to hold, have
been a hit with consumers.

In the second quarter, Samsung's global smartphone sales rose 5.5
percent from a year earlier while Apple's fell 15 percent, giving
Samsung 22.4 percent of the market compared to Apple's 11.8 percent,
according to IDC. While some of that growth came from Samsung's
lower-end phones, IDC said that a significant part had also come from
new demand for its higher-end phones, which contributed a
disproportionate amount of profit.

In a year that Samsung originally warned could be tough, the company has
performed surprisingly well. In the most recent second quarter, the
company said its
\href{http://www.nytimes3xbfgragh.onion/2016/07/08/business/samsung-profit-rises-on-new-smartphone-screen-technology.html}{operating
profit rose} 15 percent from a year earlier, to about \$7 billion.

Samsung had high expectations for the latest Galaxy phone, which was
released last month, to help continue the momentum.

The Note 7 is 5.7 inches from corner to corner, making it a large phone
sometimes referred to as a phablet --- a combination of phone and
tablet. It sells in the United States for about \$900 to \$1,000 without
subsidies from a wireless carrier.

The phones were released just ahead of Apple's traditional release time
in autumn, before the important holiday shopping season. Apple is set to
show off its latest iPhone on Sept. 7. With the new iPhone, Apple is
expected to make major upgrades to both the hardware and software, as it
generally does every two years.

``You have to applaud Samsung for moving quickly,'' said Ben Wood,
mobile analyst at CCS Insight in Berlin. ``But they can't afford to miss
the run up to the holiday season, so they have to fix this problem
fast.''

Lightweight and powerful, lithium-ion batteries are the go-to for
technology, since they don't take up much room and can quickly recharge
repeatedly without wearing out. But they are also far from perfect.

The batteries, which include volatile and flammable chemical compounds,
can become unstable if overheated or punctured. If that happens, the
battery can burst into flames or explode.

Dell in 2006 recalled more than four million batteries for its notebook
computers. American aviation authorities in 2013 reviewed the design of
Boeing's 787 Dreamliner after
\href{http://www.nytimes3xbfgragh.onion/2013/04/06/business/boeing-completes-test-787-flight.html}{a
number of incidents} involving lithium-ion batteries and the plane's
electrical system. Tesla in 2014 had to reinforce the area protecting
the batteries in its Model S after it became apparent that road debris
striking the bottom of the car
\href{http://www.nytimes3xbfgragh.onion/2013/10/04/business/car-fire-a-test-for-high-flying-tesla.html}{could
cause a fire}.

More recently, battery problems have cropped up with increasing
frequency among lower-end devices. In America, airline companies began
banning hoverboards from flights after it became apparent that some of
the products would occasionally burst into flames. There have also been
increased reports of e-cigarette batteries spontaneously detonating.

Analysts, in part, attribute the issues to the low standards and few
regulations in the global electronics supply chain that sprawls across
China. As smaller Chinese companies have jumped at opportunities to make
their own devices, some have cut corners, leading to a number of
problems, including the occasionally combusting battery.

For Samsung, the recall strikes at the heart of what has long been
considered its greatest strength: its management of the supply chain.

Samsung owns the facilities that produce many of the components in its
smartphones --- including screens, chips and batteries. The system
allows it to keep closer tabs on production.

Yet much like Apple, Samsung also manages a large network of suppliers.

That means it must coordinate the production of a high number of parts
that come from factories in disparate places run on tight margins. It
must also double-check the quality of each product, which can be
exceptionally challenging given that production processes are complex
and factories are sometimes known to sacrifice quality for profit.

``What's surprising is this comes from Samsung, who have such prowess
and competence in manufacturing and supply chain,'' Mr. Ma of IDC said.
``You would think this wouldn't happen to a company like that, but
somehow it slipped through.''

Advertisement

\protect\hyperlink{after-bottom}{Continue reading the main story}

\hypertarget{site-index}{%
\subsection{Site Index}\label{site-index}}

\hypertarget{site-information-navigation}{%
\subsection{Site Information
Navigation}\label{site-information-navigation}}

\begin{itemize}
\tightlist
\item
  \href{https://help.nytimes3xbfgragh.onion/hc/en-us/articles/115014792127-Copyright-notice}{©~2020~The
  New York Times Company}
\end{itemize}

\begin{itemize}
\tightlist
\item
  \href{https://www.nytco.com/}{NYTCo}
\item
  \href{https://help.nytimes3xbfgragh.onion/hc/en-us/articles/115015385887-Contact-Us}{Contact
  Us}
\item
  \href{https://www.nytco.com/careers/}{Work with us}
\item
  \href{https://nytmediakit.com/}{Advertise}
\item
  \href{http://www.tbrandstudio.com/}{T Brand Studio}
\item
  \href{https://www.nytimes3xbfgragh.onion/privacy/cookie-policy\#how-do-i-manage-trackers}{Your
  Ad Choices}
\item
  \href{https://www.nytimes3xbfgragh.onion/privacy}{Privacy}
\item
  \href{https://help.nytimes3xbfgragh.onion/hc/en-us/articles/115014893428-Terms-of-service}{Terms
  of Service}
\item
  \href{https://help.nytimes3xbfgragh.onion/hc/en-us/articles/115014893968-Terms-of-sale}{Terms
  of Sale}
\item
  \href{https://spiderbites.nytimes3xbfgragh.onion}{Site Map}
\item
  \href{https://help.nytimes3xbfgragh.onion/hc/en-us}{Help}
\item
  \href{https://www.nytimes3xbfgragh.onion/subscription?campaignId=37WXW}{Subscriptions}
\end{itemize}
