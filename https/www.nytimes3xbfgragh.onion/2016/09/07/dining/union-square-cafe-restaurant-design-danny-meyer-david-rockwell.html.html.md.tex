Sections

SEARCH

\protect\hyperlink{site-content}{Skip to
content}\protect\hyperlink{site-index}{Skip to site index}

\href{https://www.nytimes3xbfgragh.onion/section/food}{Food}

\href{https://myaccount.nytimes3xbfgragh.onion/auth/login?response_type=cookie\&client_id=vi}{}

\href{https://www.nytimes3xbfgragh.onion/section/todayspaper}{Today's
Paper}

\href{/section/food}{Food}\textbar{}The Same, Only Different: Designing
a New Union Square Cafe

\url{https://nyti.ms/2cpzTy3}

\begin{itemize}
\item
\item
\item
\item
\item
\item
\end{itemize}

Advertisement

\protect\hyperlink{after-top}{Continue reading the main story}

Supported by

\protect\hyperlink{after-sponsor}{Continue reading the main story}

\hypertarget{the-same-only-different-designing-a-new-union-square-cafe}{%
\section{The Same, Only Different: Designing a New Union Square
Cafe}\label{the-same-only-different-designing-a-new-union-square-cafe}}

\href{https://www.nytimes3xbfgragh.onion/slideshow/2016/09/07/dining/the-new-union-square-cafe.html}{}

\hypertarget{the-new-union-square-cafe}{%
\subsection{The New Union Square Cafe}\label{the-new-union-square-cafe}}

10 Photos

View Slide Show ›

\includegraphics{https://static01.graylady3jvrrxbe.onion/images/2016/09/07/dining/07UNION-SQUARE-CAFE-slide-3XUU/07UNION-SQUARE-CAFE-slide-3XUU-articleLarge.jpg?quality=75\&auto=webp\&disable=upscale}

Chad Batka for The New York Times

By \href{http://www.nytimes3xbfgragh.onion/by/michael-kimmelman}{Michael
Kimmelman}

\begin{itemize}
\item
  Sept. 6, 2016
\item
  \begin{itemize}
  \item
  \item
  \item
  \item
  \item
  \item
  \end{itemize}
\end{itemize}

On a clear February morning, executives from
\href{https://www.google.com/search?q=Union+Square+Hospitality+Group\&ie=utf-8\&oe=utf-8}{Union
Square Hospitality Group} and several architects sipped hot coffee and
milled around a table in the partly demolished remains of a shuttered
restaurant, waiting for Danny Meyer.

Winter light through immense, half-papered-over windows cast a pall on
the stained, brown-and-yellow-tiled walls. A sculpture of a red crab
dangled above the front door. Architectural drawings were spread across
the table. Mr. Meyer arrived and everybody turned toward them like
physicians attending to a critical patient.

When Union Square Cafe reboots next month in its new location at 19th
Street and Park Avenue South --- the bygone home of City Crab \& Seafood
Company --- Mr. Meyer's first, signature restaurant will be engaged in a
quasiscientific enterprise. Take one of the most cherished high-end
establishments in the city, a longtime regular atop the Zagat
most-popular list. Keep the chef, the staff, even the old favorites from
the menu, but (isolating the variable) move three blocks.

Then determine: Precisely how much, in a business built on food and
service, does ``place'' (more specifically, architecture) matter? And is
it possible to uproot a classic without destroying its essence?

It's hard to think of many attempts that have worked. The man in charge
of this experiment is the architect and restaurant whisperer David
Rockwell, whose design credits include
\href{http://www.nytimes3xbfgragh.onion/live/2016-tony-awards/best-scenic-design-of-a-musical/}{a
Tony} for the musical ``She Loves Me,'' hotels from the
\href{http://www.wparisopera.com/}{W in Paris} to
\href{https://www.cosmopolitanlasvegas.com/}{the Cosmopolitan} in Las
Vegas, and restaurants like
\href{http://www.noburestaurants.com/fifty-seven/experience/}{Nobu Fifty
Seven} in Midtown Manhattan and \href{http://gatonyc.com/}{Gato}
downtown.

For the last year or so, Mr. Rockwell and his colleagues at
\href{http://www.rockwellgroup.com/}{Rockwell Group} have been laboring
to extract the ambient DNA of the old Union Square Cafe and implant it
into a new, very different body.

\href{https://www.nytimes3xbfgragh.onion/interactive/2016/09/07/dining/new-york-restaurants-fall-openings.html}{}

\includegraphics{https://static01.graylady3jvrrxbe.onion/images/2016/09/07/dining/07REST-PREVIEW-ICON/07REST-PREVIEW-ICON-largeHorizontalJumbo.jpg}

\hypertarget{the-2016-fall-restaurant-preview}{%
\subsection{The 2016 Fall Restaurant
Preview}\label{the-2016-fall-restaurant-preview}}

A look at the new restaurants, bars and markets opening in New York this
season.

Part of the experiment has involved mixing two detail-crazy, yet
dissimilar auteurs.

``David is allergic to the status quo, while Danny arrives at a project
with about 15 design elements he already wants, which come from places
he's been, things he's found,'' said Richard Coraine, Union Square's
chief development officer, who has worked with both of them for years.

``Danny will say, `I've been to this place where you can see into the
wine cellar,' and David will pull out a piece of tracing paper and pick
up a red pen in his left hand, waving it around. I've told him he's got
the best left arm since Sandy Koufax. `What if the wine rack looked like
an apothecary cabinet?' he'll ask Danny, and then start sketching.''

That exchange produced the wine room at
\href{http://www.nytimes3xbfgragh.onion/2010/01/20/dining/reviews/20rest.html}{Maialino},
Mr. Meyer's Italian restaurant on Gramercy Park.

For Union Square Cafe, Mr. Meyer didn't want ``another Rockwell,'' as he
put it to me last fall, when I started dropping in from time to time on
planning sessions, where the two men sometimes approximated an old
married couple.

Mr. Rockwell would propose some floor tile or light fixture. Then he
would nod and purse his lips without actually agreeing while Mr. Meyer
would make a countersuggestion, claiming to defer to Mr. Rockwell's
expertise but expecting to get what he wanted.

What he wanted was Union Square Cafe, only different.

``We're usually called on to invent things at a much more visible,
bigger scale,'' Mr. Rockwell said. ``The invention here took place more
on a micro scale.''

So did meetings, where Mr. Meyer's team, including Sam Lipp, Union
Square Cafe's director of operations, and Carmen Quagliata, the
restaurant's executive chef, would suddenly deliver a dissertation on
the turning radius a barista needs behind the coffee station, or the
effects of citrus on different types of wood.

\includegraphics{https://static01.graylady3jvrrxbe.onion/images/2016/09/07/dining/07CAFE-WEB2/07CAFE-WEB2-articleLarge.jpg?quality=75\&auto=webp\&disable=upscale}

Mr. Meyer himself spent the better part of one session musing on the
width of the mullions in the new windows. He wanted them to recall the
ones in the old restaurant so that, from the sidewalk, even before
entering the restaurant, as Mr. Coraine put it, ``people will feel
they've come home.''

Mr. Rockwell elaborated: ``The best restaurant designs are made up of
large decisions and lots of tiny details. But ultimately, what comes
from long experience, Danny's and mine, involves what I'd call `in
between' things: Designing for when a restaurant isn't full. Ensuring
there's easy eye contact from every table with the waiters' stations, so
diners feel they're always in control, never swallowed up in a big
space. Designing spaces so people want to explore the room, because a
restaurant shouldn't be a one-liner.

``A restaurant should unfold, like a series of snapshots,'' he added.
``That was part of what defined the original Union Square Cafe. It was
episodic.''

The original occupied a side street that dead-ends into Union Square
Park. It was what Mr. Rockwell calls a ``slow'' locale. The new address,
Park Avenue South, is almost the reverse, choked with traffic and lined
with singles bars. Mr. Meyer and Mr. Rockwell agreed from the start that
the new entrance should be on 19th Street, not Park.

That was the easy move. At 10,000 square feet, the new place is as
cavernous as the original, 6,300-square-foot Union Square Cafe was a
deteriorating rabbit warren. Faucets leaked, the wiring dated back to
the Truman administration. Repairs would cost a fortune.

Then, in 2014, as the expiration of the lease loomed, the landlord
proposed
\href{http://www.nytimes3xbfgragh.onion/2014/06/24/dining/union-square-cafe-joins-other-victims-of-new-york-citys-rising-rents.html}{more
than doubling the rent}. ``The numbers just didn't add up,'' said Mr.
Meyer, declining to be specific. ``Suffice it to say, we would have been
working for the landlord instead of giving raises and promotions to our
staff.''

Briefly, he thought about ending a great run, then decided that ``Union
Square Cafe had to go on,'' he said. ``It's my firstborn. It's a New
York institution. For our company, it's the mother yeast of every loaf
we've ever baked.''

His team started scouring for a new home not more than a six-minute
hand-truck ride (not seven or eight minutes --- six) from the
\href{http://www.grownyc.org/greenmarket/manhattan-union-square-m}{Union
Square Greenmarket}, whose fresh food provided the restaurant's
lifeblood from Day 1.

Three decades ago, Union Square was a derelict magnet for drug dealers
and crime, and the newly opened cafe was a pioneer. It became a victim
of the very prosperity it helped bring to the neighborhood --- and today
it is following the northward creep of restaurants from the square that
it also set off.

Usefully, Mr. Rockwell, 60, had been a regular at the original. He knew
its pluses and minuses, including the narrow corridors and terrible
circulation, the back room that was Siberia, the kitchen that was a
dungeon.

And yet almost despite itself, the restaurant achieved what Mr. Meyer,
now 58 and then a 27-year-old novice, and
\href{http://www.nytimes3xbfgragh.onion/2011/06/30/dining/larry-bogdanow-architect-dies-at-64.html}{Larry
Bogdanow}, the original architect, set out to accomplish in 1985: to
blur the line between going out and coming home. The architecture, or
really the apparent lack of any, reinforced the cafe's laid-back,
service-first, vaguely Midwestern mood. The fuzzy aesthetic blending
Italian trattoria, San Francisco bistro and Shaker Meeting House echoed
the borderless cuisine.

The layout's flaws only amplified loyalists' affection, as did the
strategically selected hand-me-down furniture and art collection; the
umbrella stand in the vestibule; the long mahogany bar, with its shelves
of wine bottles and ample room to eat, and the jazzy Judy Rifka murals
of gamboling figures, ``Satyricon'' lite.

Image

The old Union Square Cafe at lunchtime.Credit...John Marshall Mantel for
The New York Times

Mr. Lipp sat down to chat one early December afternoon, just before the
old place closed. ``This restaurant has been a club to many of our
regular diners,'' he said. ``Any time you move an iconic brand, even a
few blocks, it's a big concern, in this case not just because Park
Avenue isn't 16th Street. People by their nature are habitual.'' He
cited the business management book
\href{http://www.penguin.com/book/who-moved-my-cheese-by-spencer-johnson/9780399147241}{``Who
Moved My Cheese?''}

``Like Danny says, we want to make this move with our regulars, not do
it \emph{to} them,'' Mr. Lipp said. ``At the end of the day, I can buy
the same chicken Tom Keller buys and cook it the same way he does. But
he can't create the feeling evoked at Union Square Cafe. That's why
David is so key.''

Mr. Meyer put it this way: ``You don't hire David Rockwell if you just
need a cover band.''

The new layout looks airy, sunny, logical. It doesn't make any big
statements or break any new architectural ground, because that wasn't
the assignment. The design manages to bring a lofty space with immense
windows down to human scale by breaking it up. Not counting the addition
of a private dining room (it has its own staff), the restaurant will
grow by only about eight seats, to 138.

As in the old place, a vestibule directs diners past a few cafe tables
toward the bar. The main room is lined with wainscoting, its dark-green
color custom-tweaked to update the original green wainscoting. A big
staircase becomes the main architectural focus and spatial divider,
implying a separation of rooms.

A scrim of lights, at nine feet, aligned with the mezzanine, creates a
layer of intimacy, slicing the restaurant's lofty height in half.
Discreet spotlights in the ceiling focus pools of light around clusters
of tables. On the mezzanine, booths look down onto the main floor; like
opera boxes, these may well become the best seats in the house.

There's a second bar upstairs for a half-dozen or so guests, a remnant
salvaged from the original cafe, illuminated by some of Mr. Bogdanow's
old glass fixtures. Wide-plank cherry wood floors, like the ones on 16th
Street but now variously patterned, demarcate spaces to underscore the
episodic layout.

It's a kind of homage, the same but not, updated and, for a celebrity
designer, self-effacing, with nods that may register only subliminally,
like the main bar, which is the exact same length as the old bar (27
feet 1 inch); the concrete floor tiles next to it, which are the same
width as the terra-cotta ones at 16th Street; the brass light fixtures
over the bar, hung at the same height.

``Union Square Cafe had its own community,'' Mr. Rockwell said. ``It was
where many other people came to celebrate. It's near and dear to me.
Restaurants were my introduction to New York, places like
\href{http://www.nytimes3xbfgragh.onion/2008/06/29/realestate/29scap.html}{Schrafft's}.
I know this sounds kind of heavy, but having lost my dad when I was 3,
moving a lot, then my mom passing away when I was 15, places that marked
celebration and connection, they've made a big impact on me.

``Something as simple but complicated as getting a meal, it's one of the
things that makes a city so vital,'' Mr. Rockwell said. ``This was a
rare opportunity, to take a seminal place for New Yorkers and prove that
the city remains alive.''

Advertisement

\protect\hyperlink{after-bottom}{Continue reading the main story}

\hypertarget{site-index}{%
\subsection{Site Index}\label{site-index}}

\hypertarget{site-information-navigation}{%
\subsection{Site Information
Navigation}\label{site-information-navigation}}

\begin{itemize}
\tightlist
\item
  \href{https://help.nytimes3xbfgragh.onion/hc/en-us/articles/115014792127-Copyright-notice}{©~2020~The
  New York Times Company}
\end{itemize}

\begin{itemize}
\tightlist
\item
  \href{https://www.nytco.com/}{NYTCo}
\item
  \href{https://help.nytimes3xbfgragh.onion/hc/en-us/articles/115015385887-Contact-Us}{Contact
  Us}
\item
  \href{https://www.nytco.com/careers/}{Work with us}
\item
  \href{https://nytmediakit.com/}{Advertise}
\item
  \href{http://www.tbrandstudio.com/}{T Brand Studio}
\item
  \href{https://www.nytimes3xbfgragh.onion/privacy/cookie-policy\#how-do-i-manage-trackers}{Your
  Ad Choices}
\item
  \href{https://www.nytimes3xbfgragh.onion/privacy}{Privacy}
\item
  \href{https://help.nytimes3xbfgragh.onion/hc/en-us/articles/115014893428-Terms-of-service}{Terms
  of Service}
\item
  \href{https://help.nytimes3xbfgragh.onion/hc/en-us/articles/115014893968-Terms-of-sale}{Terms
  of Sale}
\item
  \href{https://spiderbites.nytimes3xbfgragh.onion}{Site Map}
\item
  \href{https://help.nytimes3xbfgragh.onion/hc/en-us}{Help}
\item
  \href{https://www.nytimes3xbfgragh.onion/subscription?campaignId=37WXW}{Subscriptions}
\end{itemize}
