Sections

SEARCH

\protect\hyperlink{site-content}{Skip to
content}\protect\hyperlink{site-index}{Skip to site index}

\href{https://www.nytimes3xbfgragh.onion/section/world/middleeast}{Middle
East}

\href{https://myaccount.nytimes3xbfgragh.onion/auth/login?response_type=cookie\&client_id=vi}{}

\href{https://www.nytimes3xbfgragh.onion/section/todayspaper}{Today's
Paper}

\href{/section/world/middleeast}{Middle East}\textbar{}Straightforward
Answers to Basic Questions About Syria's War

\url{https://nyti.ms/2cQRi5J}

\begin{itemize}
\item
\item
\item
\item
\item
\end{itemize}

Advertisement

\protect\hyperlink{after-top}{Continue reading the main story}

Supported by

\protect\hyperlink{after-sponsor}{Continue reading the main story}

\href{/column/the-interpreter}{The Interpreter}

\hypertarget{straightforward-answers-to-basic-questions-about-syrias-war}{%
\section{Straightforward Answers to Basic Questions About Syria's
War}\label{straightforward-answers-to-basic-questions-about-syrias-war}}

\includegraphics{https://static01.graylady3jvrrxbe.onion/images/2016/09/19/world/19int-syria-1/19int-syria-1-articleLarge.jpg?quality=75\&auto=webp\&disable=upscale}

By \href{https://www.nytimes3xbfgragh.onion/by/max-fisher}{Max Fisher}

\begin{itemize}
\item
  Sept. 18, 2016
\item
  \begin{itemize}
  \item
  \item
  \item
  \item
  \item
  \end{itemize}
\end{itemize}

You could be forgiven, after five years of Syria's war dominating front
pages, for feeling lost.

It is easy to track the war's toll: It has killed 400,000 people,
displaced millions, opened space for the Islamic State, and sucked in
foreign powers, including the United States. It is harder to keep track
of the how and why. The basics can seem even more confusing than the
day-to-day details.

But those basics are crucial to understanding Syria's war --- and they
are far more complex than they might initially seem. As last week's
truce appears shaky after American planes bombed Syrian troops, here are
straightforward answers to some of the fundamental questions about the
conflict: an attempt to explain its origins, the broader context and how
it relates to the refugee crisis and the rise of the Islamic State.

\hypertarget{1-what-is-the-syrian-civil-war}{%
\subsection{\texorpdfstring{\textbf{1. What is the Syrian civil
war?}}{1. What is the Syrian civil war?}}\label{1-what-is-the-syrian-civil-war}}

The war makes more sense if you think of it as four overlapping
conflicts.

The core conflict is between forces loyal to President Bashar al-Assad
and the rebels who oppose him. Over time, both sides
\href{http://warontherocks.com/2016/08/the-decay-of-the-syrian-regime-is-much-worse-than-you-think/}{fractured}
into multiple militias, including local and foreign fighters, but their
fundamental disagreement is over whether Mr. Assad's government should
stay in power.

This opened a second conflict: Syria's ethnic Kurdish minority took up
arms amid the chaos. The Kurds carved out a de facto ministate and have
gradually taken territory they see as Kurdish --- sometimes with backing
from the United States, which sees the Kurds as an ally against jihadist
groups. While Mr. Assad has not focused on fighting the Kurdish groups,
they are opposed by neighboring Turkey, which is
\href{http://www.nytimes3xbfgragh.onion/2016/06/30/world/middleeast/turkeys-twin-terrorist-threats-explained.html?rref=collection\%2Fcolumn\%2Fthe-interpreter\&action=click\&contentCollection=world\&region=stream\&module=stream_unit\&version=latest\&contentPlacement=20\&pgtype=collection}{in
conflict} with its own Kurdish minority.

The third conflict involves the Islamic State, also known as ISIS or
ISIL, which emerged out of infighting among jihadist groups. In 2014,
the Islamic State seized large parts of Syria and Iraq, and it declared
that territory its caliphate. The group has no allies and is at war with
all other actors in the conflict.

The fourth, and most complex, dynamic may be the crisscrossing foreign
interventions, which have grown steadily. Mr. Assad receives vital
support from Iran and Russia, as well as the Lebanese militant group
Hezbollah. The rebels are backed by the United States and oil-rich Arab
states like Saudi Arabia. These foreign powers have different agendas,
but all pursue them by ramping up Syria's violence, helping to
\href{http://www.nytimes3xbfgragh.onion/2016/08/27/world/middleeast/syria-civil-war-why-get-worse.html?hp\&action=click\&pgtype=Homepage\&clickSource=story-heading\&module=first-column-region\&region=top-news\&WT.nav=top-news\&_r=2}{perpetuate}
the war.

\includegraphics{https://static01.graylady3jvrrxbe.onion/images/2016/09/19/world/19int-syria-2/19int-syria-2-articleLarge.jpg?quality=75\&auto=webp\&disable=upscale}

\hypertarget{2-how-did-the-war-happen}{%
\subsection{\texorpdfstring{\textbf{2. How did the war
happen?}}{2. How did the war happen?}}\label{2-how-did-the-war-happen}}

On the surface, the conflict began in 2011 with the
\href{http://www.nytimes3xbfgragh.onion/slideshow/2011/12/25/sunday-review/25YIP_ARABSPRING.html?action=click\&contentCollection=Opinion\&module=RelatedCoverage\&region=EndOfArticle\&pgtype=article}{Arab
Spring}. Syrians, like other peoples across the region, rose up
peacefully against their authoritarian government. Mr. Assad cracked
down violently. Communities took up arms to defend themselves, then
fought back in what became a civil war. Some soldiers joined the rebels,
but not enough to win.

But that alone does not explain Syria's disintegration. It is now clear
that the state was weak in ways that made it inherently unstable and
prone to violence.

The government was
\href{http://www.nytimes3xbfgragh.onion/2012/06/10/world/middleeast/syrian-alawites-divided-by-assads-response-to-unrest.html?_r=0}{dominated}
by a minority group. Over decades, Syria's religious and ethnic divides
had taken on greater political importance, making the ruling minority
fearful and reactive. Mr. Assad had strong
\href{http://www.vox.com/2015/11/5/9671746/syria-assad-military-loyal}{support}
among the military and security services, but not the broader
population, making violence more tempting. The instability was deepened
by the fact that rural Syrians had moved to cities in large numbers in
recent years, driven in part by
\href{http://www.nytimes3xbfgragh.onion/2015/03/03/science/earth/study-links-syria-conflict-to-drought-caused-by-climate-change.html}{droughts}
linked to climate change.

Fighting, once it began, was worsened by several external factors. A
decade of war in neighboring Iraq had produced battle-hardened extremist
groups that now flowed into Syria. Iraq's political troubles in 2011 and
2012 helped open space for the Islamic State. During this time, Syria
was sucked into the regional power struggle between Iran and Saudi
Arabia.

\hypertarget{3-which-countries-are-involved-and-why}{%
\subsection{\texorpdfstring{\textbf{3. Which countries are involved and
why?}}{3. Which countries are involved and why?}}\label{3-which-countries-are-involved-and-why}}

Five countries are playing a major role in Syria, each with different
agendas. Their interventions have locked the war into an ever-worsening
stalemate.

Iran was first, sending supplies and soldiers to prop up Mr. Assad. Iran
sees Syria as crucial to its regional strategy: It provides access to
Lebanon and therefore Hezbollah, a group Tehran uses for regional
influence and as a counterweight to Israel, whose nuclear weapons it
fears.

Saudi Arabia supported Syria's rebels in the hopes of replacing Mr.
Assad with a friendlier government and of countering Iran's influence.
Saudi Arabia and Iran have been rivals for decades, fighting something
like a cold war for regional dominance. (Other Arab states like Jordan,
Qatar and the United Arab Emirates have also backed the rebels.)

Their struggle has escalated for several reasons: Iran's growing power;
the regional power vacuum that opened with the fall of Saddam Hussein in
2003 in Iraq; more political vacuums opened by the Arab Spring; a
hawkish new king in Saudi Arabia; and Saudi fears that the United States
is becoming less hostile toward Iran.

The United States funnels weapons to Syria's rebels. It did so initially
out of opposition to Mr. Assad, a longtime enemy, and later to encourage
those groups to fight the Islamic State. The United States has also
armed Kurdish groups against the Islamic State.

Turkey
\href{http://www.nytimes3xbfgragh.onion/2011/10/28/world/europe/turkey-is-sheltering-antigovernment-syrian-militia.html}{sheltered}
Syrian rebels and
\href{http://www.nytimes3xbfgragh.onion/2015/03/10/world/europe/despite-crackdown-path-to-join-isis-often-winds-through-porous-turkish-border.html}{ushered
in} foreign recruits, seeking to undermine and perhaps topple Mr. Assad.
Later, the country also acted to
\href{http://www.nytimes3xbfgragh.onion/2016/06/30/world/middleeast/turkeys-twin-terrorist-threats-explained.html?rref=collection\%2Fcolumn\%2Fthe-interpreter\&action=click\&contentCollection=world\&region=stream\&module=stream_unit\&version=latest\&contentPlacement=20\&pgtype=collection}{counter}
Syrian Kurdish groups, fearing that they could strengthen Kurdish
insurgents in Turkey.

Russia has backed Mr. Assad from the beginning, selling him arms and
providing diplomatic cover at the United Nations. Syria is one of
Russia's
\href{http://www.nytimes3xbfgragh.onion/2012/07/11/world/middleeast/russia-sends-warships-on-maneuvers-near-syria.html}{last
remaining allies}, and it is where Moscow maintains its only military
bases outside the former Soviet Union. Russian forces intervened in
2015, at a time when Mr. Assad appeared to be losing ground.

\hypertarget{4-why-is-the-war-so-bloody}{%
\subsection{\texorpdfstring{\textbf{4. Why is the war so
bloody?}}{4. Why is the war so bloody?}}\label{4-why-is-the-war-so-bloody}}

There have been atrocities on all sides, but forces loyal to Mr. Assad
have committed by far the most. Because his government is so weak ---
its support base is small and its military has suffered heavy defections
--- Mr. Assad seems to believe he can regain control only by violently
coercing Syrians into submission. That has included using chemical
weapons, barrel bombs and starvation.

Because neither Mr. Assad nor the rebels are strong enough to win, the
battle lines push back and forth, rolling across communities in waves of
destruction that kill thousands but accomplish little else.

Foreign interventions have made those shifting front lines even bloodier
and have
\href{http://www.nytimes3xbfgragh.onion/2016/08/27/world/middleeast/syria-civil-war-why-get-worse.html?hp\&action=click\&pgtype=Homepage\&clickSource=story-heading\&module=first-column-region\&region=top-news\&WT.nav=top-news\&_r=2}{deepened}
the stalemate. As a result, the overall violence kills more Syrians
without altering the conflict's underlying dynamics.

Image

A man pointed a flashlight at the body of a Syrian man killed by
shelling by the Syrian Army at a graveyard in Aleppo, Syria, in
2012.Credit...Manu Brabo/Associated Press

The years of chaos have destroyed basic order in Syria. As often happens
in lengthy civil wars, militias have filled the vacuum. Their leaders
often behave more as warlords, forcibly extracting resources from local
communities. This practice has been carried out by rebel militias and
some that support the government.

The rise of the Islamic State has worsened all of these trends. The
jihadist group has provided another set of shifting battle lines,
introduced more warlords, compelled more foreign interventions and, most
of all, put communities under its tyrannical, fanatical rule.

\hypertarget{5-how-did-the-war-become-divided-by-religion}{%
\subsection{\texorpdfstring{\textbf{5. How did the war become divided by
religion?}}{5. How did the war become divided by religion?}}\label{5-how-did-the-war-become-divided-by-religion}}

There is nothing innately religious about Syria's war, but its broader
political forces have played out along religious lines. To understand
why, it helps to start about 100 years ago.

After World War I, France took control of the territory of the defeated
Ottoman Empire that is now Syria. France
\href{http://thelede.blogs.nytimes3xbfgragh.onion/2011/06/14/syrias-ruling-alawite-sect/}{ruled
through} minority groups that would be too small to hold power without
outside support. That included Alawites, followers of a branch of Shiite
Islam, who joined the military in large numbers. The last French troops
left in 1946, and a long period of turmoil followed. Syria's military
consolidated power in a 1970 coup led by
\href{http://www.nytimes3xbfgragh.onion/topic/person/hafez-alassad?8qa}{Hafez
al-Assad}, an Alawite general and the father of Bashar al-Assad.

Syria's authoritarian government favored Alawites and other minorities,
widening social and political divides along sectarian lines. A sectarian
civil war next door in Lebanon and the rise of Sunni religious politics
widened them further, and Alawites continued to cluster in positions of
power. The country's Sunni Arab majority came to feel, at times, that
they were underserved.

Minority governments like Syria's tend to be unstable. They sometimes
fear discrimination or worse should they lose power, and can see the
majority group as a potential threat rather than a base of support. This
can make them more willing to use violence to hold on to power --- as
Mr. Assad did when his forces opened fire on peaceful protesters in
2011.

As the war has worsened, many Syrians have based their allegiance on
sectarian identity. But this is not because they are motived primarily
by religious or ethnic concerns. Rather, it is defensive. They fear that
the other side will target them for their background, so they feel safe
only with their own people. This contributes to atrocities: If Alawites
are seen as innately pro-Assad, then Sunni militias could conclude that
all Alawite civilians are a threat and treat them accordingly, which
prompts more defensive sorting.

Image

Kurdish militia fighters at a forward position on the eastern banks of
the Euphrates River in October 2015.Credit...Tyler Hicks/The New York
Times

At the same time, the Iran-Saudi Arabia proxy war is also playing out
along sectarian lines, with the Saudis backing Sunnis and Iran backing
Shiites across the region. For both countries, sectarianism is a tool by
which they can cultivate proxy forces and stir up fear of the other
side.

\hypertarget{6-where-did-the-islamic-state-come-from}{%
\subsection{\texorpdfstring{\textbf{6. Where did the Islamic State come
from?}}{6. Where did the Islamic State come from?}}\label{6-where-did-the-islamic-state-come-from}}

The group has its roots in two earlier wars and the foreign occupations
that followed: the 1979 Soviet invasion of Afghanistan and the
American-led invasion of Iraq in 2003. In the first, Sunni Arab
volunteers fought alongside Afghan rebels, later forming the global
jihadist movement, including Al Qaeda. In the second, Al Qaeda and other
Sunni groups flooded to Iraq to fight both the Americans and Iraq's
Shiite majority.

A key name is Abu Musab al-Zarqawi, a Jordanian extremist who fought in
Afghanistan in the 1990s and Iraq in the 2000s. Mr. Zarqawi's views and
methods were even more extreme and theatrical than Al Qaeda's. He
flourished in Iraq's war, using tactics now associated with the Islamic
State: videotaped beheadings, mass killings of fellow Muslims deemed
nonbelievers and attacks meant to incite a Sunni-Shiite war.

Al Qaeda invited Mr. Zarqawi to rebrand his group as Al Qaeda in Iraq,
but the two factions argued over strategy and ideology, setting them up
for conflict a decade later in Syria.

Mr. Zarqawi was killed in 2006, and his group declined as Sunni Iraqis
turned against it. Later, Iraq's Shiite-dominated government grew
increasingly authoritarian and sectarian, alienating the minority Sunni.
It also purged many experienced military and security officers,
replacing them with political loyalists.

The successor to Mr. Zarqawi's group, then calling itself the Islamic
State in Iraq, exploited these conditions in 2011 and 2012 to
reconstitute itself, for example by breaking extremists out of Iraqi
prisons. Its leader, Abu Bakr al-Baghdadi, combined Mr. Zarqawi's views
with an
\href{http://www.politico.com/magazine/story/2015/08/isis-jihad-121525}{apocalypticism}
taking hold amid the region's upheaval.

Mr. Baghdadi sent a top officer into Syria's war to set up a new Al
Qaeda franchise: the Nusra Front, now known as the Levant Conquest
Front. In 2013, Mr. Baghdadi declared himself commander of all Al Qaeda
forces in Iraq and Syria. After years of tense partnership with Al
Qaeda, the groups finally split. Mr. Baghdadi --- his force now
rebranded as the Islamic State --- invaded Syria to fight his former
Qaeda allies.

Image

Kurdish refugees on their way to shelters in Turkey as the Islamic State
attacked the city of Kobani, Syria, in September 2014.Credit...Bryan
Denton for The New York Times

The Islamic State carved out a ministate in Syria's chaos, then used it
as a base to invade Iraq in 2014. It repeated Mr. Zarqawi's worst
tactics on a far larger scale, committing acts of genocide and mass
murder in the Middle East and abroad, and attracting foreign recruits
from rich and poor countries alike.

\hypertarget{7-why-is-the-refugee-crisis-so-severe}{%
\subsection{\texorpdfstring{\textbf{7. Why is the refugee crisis so
severe?}}{7. Why is the refugee crisis so severe?}}\label{7-why-is-the-refugee-crisis-so-severe}}

The war in Syria has produced nearly
\href{http://data.unhcr.org/syrianrefugees/regional.php}{five million}
refugees. The exodus has created three sets of problems, all dire: a
humanitarian crisis for the refugees themselves, a potential crisis for
the countries that host them and a political crisis in Europe over what
to do.

Syrian refugees face disease and
\href{http://www.unicef.org/emergencies/lebanon_72711.html}{malnutrition}.
Host countries often bar them from working, meaning that families cannot
provide for themselves. Many Syrian children are
\href{http://www.nytimes3xbfgragh.onion/2016/09/15/world/middleeast/refugee-children-school-united-nations.html}{deprived}
of education, a problem that could hinder them for life.

Most Syrian refugees are in Jordan, Lebanon and Turkey, neighboring
countries that lack the necessary resources to help them. The influx
could be destabilizing, particularly in Jordan and Lebanon where Syrian
refugees now make up a large share of the population.

Many refugees, unable to tolerate life in the camps, have braved the
dangerous journey to Europe. But European voters have largely
\href{https://www.theguardian.com/world/2016/sep/01/alan-kurdi-death-one-year-on-compassion-towards-refugees-fades?CMP=share_btn_tw}{rejected}
them, supporting extreme measures to keep out Syrians and other
migrants.

European leaders at one point
\href{http://www.nytimes3xbfgragh.onion/2015/04/25/world/europe/europes-migration-crisis-cannot-be-solved-at-sea-analysts-say.html}{suspended}
search-and-rescue missions in the Mediterranean, partly in response to
complaints that saving refugees' lives might encourage more to make the
journey. Leaders of the campaign to get Britain to leave the European
Union based their argument partly on opposition to accepting Syrian
refugees.

Europe's attitude appears driven by a combination of economic downturn;
hostility toward the European Union, which allows unlimited migration
among member states; and
\href{http://www.nytimes3xbfgragh.onion/2016/08/19/world/europe/frances-burkini-bans-are-about-more-than-religion-or-clothing.html?rref=collection\%2Fcolumn\%2Fthe-interpreter\&action=click\&contentCollection=world\&region=stream\&module=stream_unit\&version=latest\&contentPlacement=8\&pgtype=collection}{demographic
anxiety} rooted in longer-term trends that have made populations more
diverse.

As a result, many refugees are stuck in camps in Italy and Greece. Many
others die trying to reach Europe. European countries, along with the
United States and Canada, have absorbed thousands of refugees, but not
nearly enough to alter the underlying crisis.

Advertisement

\protect\hyperlink{after-bottom}{Continue reading the main story}

\hypertarget{site-index}{%
\subsection{Site Index}\label{site-index}}

\hypertarget{site-information-navigation}{%
\subsection{Site Information
Navigation}\label{site-information-navigation}}

\begin{itemize}
\tightlist
\item
  \href{https://help.nytimes3xbfgragh.onion/hc/en-us/articles/115014792127-Copyright-notice}{©~2020~The
  New York Times Company}
\end{itemize}

\begin{itemize}
\tightlist
\item
  \href{https://www.nytco.com/}{NYTCo}
\item
  \href{https://help.nytimes3xbfgragh.onion/hc/en-us/articles/115015385887-Contact-Us}{Contact
  Us}
\item
  \href{https://www.nytco.com/careers/}{Work with us}
\item
  \href{https://nytmediakit.com/}{Advertise}
\item
  \href{http://www.tbrandstudio.com/}{T Brand Studio}
\item
  \href{https://www.nytimes3xbfgragh.onion/privacy/cookie-policy\#how-do-i-manage-trackers}{Your
  Ad Choices}
\item
  \href{https://www.nytimes3xbfgragh.onion/privacy}{Privacy}
\item
  \href{https://help.nytimes3xbfgragh.onion/hc/en-us/articles/115014893428-Terms-of-service}{Terms
  of Service}
\item
  \href{https://help.nytimes3xbfgragh.onion/hc/en-us/articles/115014893968-Terms-of-sale}{Terms
  of Sale}
\item
  \href{https://spiderbites.nytimes3xbfgragh.onion}{Site Map}
\item
  \href{https://help.nytimes3xbfgragh.onion/hc/en-us}{Help}
\item
  \href{https://www.nytimes3xbfgragh.onion/subscription?campaignId=37WXW}{Subscriptions}
\end{itemize}
