Sections

SEARCH

\protect\hyperlink{site-content}{Skip to
content}\protect\hyperlink{site-index}{Skip to site index}

\href{https://www.nytimes3xbfgragh.onion/section/politics}{Politics}

\href{https://myaccount.nytimes3xbfgragh.onion/auth/login?response_type=cookie\&client_id=vi}{}

\href{https://www.nytimes3xbfgragh.onion/section/todayspaper}{Today's
Paper}

\href{/section/politics}{Politics}\textbar{}Transcript of the First
Debate

\url{https://nyti.ms/2d5AtSu}

\begin{itemize}
\item
\item
\item
\item
\item
\end{itemize}

Advertisement

\protect\hyperlink{after-top}{Continue reading the main story}

Supported by

\protect\hyperlink{after-sponsor}{Continue reading the main story}

\hypertarget{transcript-of-the-first-debate}{%
\section{Transcript of the First
Debate}\label{transcript-of-the-first-debate}}

\includegraphics{https://static01.graylady3jvrrxbe.onion/images/2016/09/26/us/26DEBATE-diptych/26DEBATE-diptych-videoSixteenByNine3000.jpg}

Sept. 27, 2016

\begin{itemize}
\item
\item
\item
\item
\item
\end{itemize}

\emph{Following is a transcript of the first presidential debate on
Monday, as transcribed by the Federal News Service.}

\textbf{HOLT:} Good evening from Hofstra University in Hempstead, New
York. I'm Lester Holt, anchor of ``NBC Nightly News.'' I want to welcome
you to the first presidential debate.

The participants tonight are Donald Trump and Hillary Clinton. This
debate is sponsored by the Commission on Presidential Debates, a
nonpartisan, nonprofit organization. The commission drafted tonight's
format, and the rules have been agreed to by the campaigns.

The 90-minute debate is divided into six segments, each 15 minutes long.
We'll explore three topic areas tonight: Achieving prosperity; America's
direction; and securing America. At the start of each segment, I will
ask the same lead-off question to both candidates, and they will each
have up to two minutes to respond. From that point until the end of the
segment, we'll have an open discussion.

The questions are mine and have not been shared with the commission or
the campaigns. The audience here in the room has agreed to remain silent
so that we can focus on what the candidates are saying.

I will invite you to applaud, however, at this moment, as we welcome the
candidates: Democratic nominee for president of the United States,
Hillary Clinton, and Republican nominee for president of the United
States, Donald J. Trump.

(APPLAUSE)

\textbf{CLINTON:} How are you, Donald?

(APPLAUSE)

\textbf{HOLT:} Good luck to you.

(APPLAUSE)

Well, I don't expect us to cover all the issues of this campaign
tonight, but I remind everyone, there are two more presidential debates
scheduled. We are going to focus on many of the issues that voters tell
us are most important, and we're going to press for specifics. I am
honored to have this role, but this evening belongs to the candidates
and, just as important, to the American people.

Candidates, we look forward to hearing you articulate your policies and
your positions, as well as your visions and your values. So, let's
begin.

We're calling this opening segment ``Achieving Prosperity.'' And central
to that is jobs. There are two economic realities in America today.
There's been a record six straight years of job growth, and new census
numbers show incomes have increased at a record rate after years of
stagnation. However, income inequality remains significant, and nearly
half of Americans are living paycheck to paycheck.

Beginning with you, Secretary Clinton, why are you a better choice than
your opponent to create the kinds of jobs that will put more money into
the pockets of American works?

\textbf{CLINTON:} Well, thank you, Lester, and thanks to Hofstra for
hosting us.

The central question in this election is really what kind of country we
want to be and what kind of future we'll build together. Today is my
granddaughter's second birthday, so I think about this a lot. First, we
have to build an economy that works for everyone, not just those at the
top. That means we need new jobs, good jobs, with rising incomes.

I want us to invest in you. I want us to invest in your future. That
means jobs in infrastructure, in advanced manufacturing, innovation and
technology, clean, renewable energy, and small business, because most of
the new jobs will come from small business. We also have to make the
economy fairer. That starts with raising the national minimum wage and
also guarantee, finally, equal pay for women's work.

\textbf{CLINTON:} I also want to see more companies do profit-sharing.
If you help create the profits, you should be able to share in them, not
just the executives at the top.

And I want us to do more to support people who are struggling to balance
family and work. I've heard from so many of you about the difficult
choices you face and the stresses that you're under. So let's have paid
family leave, earned sick days. Let's be sure we have affordable child
care and debt-free college.

How are we going to do it? We're going to do it by having the wealthy
pay their fair share and close the corporate loopholes.

Finally, we tonight are on the stage together, Donald Trump and I.
Donald, it's good to be with you. We're going to have a debate where we
are talking about the important issues facing our country. You have to
judge us, who can shoulder the immense, awesome responsibilities of the
presidency, who can put into action the plans that will make your life
better. I hope that I will be able to earn your vote on November 8th.

\textbf{HOLT:} Secretary Clinton, thank you.

Mr. Trump, the same question to you. It's about putting money --- more
money into the pockets of American workers. You have up to two minutes.

\textbf{TRUMP:} Thank you, Lester. Our jobs are fleeing the country.
They're going to Mexico. They're going to many other countries. You look
at what China is doing to our country in terms of making our product.
They're devaluing their currency, and there's nobody in our government
to fight them. And we have a very good fight. And we have a winning
fight. Because they're using our country as a piggy bank to rebuild
China, and many other countries are doing the same thing.

So we're losing our good jobs, so many of them. When you look at what's
happening in Mexico, a friend of mine who builds plants said it's the
eighth wonder of the world. They're building some of the biggest plants
anywhere in the world, some of the most sophisticated, some of the best
plants. With the United States, as he said, not so much.

So Ford is leaving. You see that, their small car division leaving.
Thousands of jobs leaving Michigan, leaving Ohio. They're all leaving.
And we can't allow it to happen anymore. As far as child care is
concerned and so many other things, I think Hillary and I agree on that.
We probably disagree a little bit as to numbers and amounts and what
we're going to do, but perhaps we'll be talking about that later.

But we have to stop our jobs from being stolen from us. We have to stop
our companies from leaving the United States and, with it, firing all of
their people. All you have to do is take a look at Carrier air
conditioning in Indianapolis. They left --- fired 1,400 people. They're
going to Mexico. So many hundreds and hundreds of companies are doing
this.

\textbf{TRUMP:} We cannot let it happen. Under my plan, I'll be reducing
taxes tremendously, from 35 percent to 15 percent for companies, small
and big businesses. That's going to be a job creator like we haven't
seen since Ronald Reagan. It's going to be a beautiful thing to watch.

Companies will come. They will build. They will expand. New companies
will start. And I look very, very much forward to doing it. We have to
renegotiate our trade deals, and we have to stop these countries from
stealing our companies and our jobs.

\textbf{HOLT:} Secretary Clinton, would you like to respond?

\textbf{CLINTON:} Well, I think that trade is an important issue. Of
course, we are 5 percent of the world's population; we have to trade
with the other 95 percent. And we need to have smart, fair trade deals.

We also, though, need to have a tax system that rewards work and not
just financial transactions. And the kind of plan that Donald has put
forth would be trickle-down economics all over again. In fact, it would
be the most extreme version, the biggest tax cuts for the top percent of
the people in this country than we've ever had.

I call it trumped-up trickle-down, because that's exactly what it would
be. That is not how we grow the economy.

We just have a different view about what's best for growing the economy,
how we make investments that will actually produce jobs and rising
incomes.

I think we come at it from somewhat different perspectives. I understand
that. You know, Donald was very fortunate in his life, and that's all to
his benefit. He started his business with \$14 million, borrowed from
his father, and he really believes that the more you help wealthy
people, the better off we'll be and that everything will work out from
there.

I don't buy that. I have a different experience. My father was a
small-businessman. He worked really hard. He printed drapery fabrics on
long tables, where he pulled out those fabrics and he went down with a
silkscreen and dumped the paint in and took the squeegee and kept going.

And so what I believe is the more we can do for the middle class, the
more we can invest in you, your education, your skills, your future, the
better we will be off and the better we'll grow. That's the kind of
economy I want us to see again.

\textbf{HOLT:} Let me follow up with Mr. Trump, if you can. You've
talked about creating 25 million jobs, and you've promised to bring back
millions of jobs for Americans. How are you going to bring back the
industries that have left this country for cheaper labor overseas? How,
specifically, are you going to tell American manufacturers that you have
to come back?

\textbf{TRUMP:} Well, for one thing --- and before we start on that ---
my father gave me a very small loan in 1975, and I built it into a
company that's worth many, many billions of dollars, with some of the
greatest assets in the world, and I say that only because that's the
kind of thinking that our country needs.

Our country's in deep trouble. We don't know what we're doing when it
comes to devaluations and all of these countries all over the world,
especially China. They're the best, the best ever at it. What they're
doing to us is a very, very sad thing.

So we have to do that. We have to renegotiate our trade deals. And,
Lester, they're taking our jobs, they're giving incentives, they're
doing things that, frankly, we don't do.

Let me give you the example of Mexico. They have a VAT tax. We're on a
different system. When we sell into Mexico, there's a tax. When they
sell in --- automatic, 16 percent, approximately. When they sell into
us, there's no tax. It's a defective agreement. It's been defective for
a long time, many years, but the politicians haven't done anything about
it.

Now, in all fairness to Secretary Clinton --- yes, is that OK? Good. I
want you to be very happy. It's very important to me.

But in all fairness to Secretary Clinton, when she started talking about
this, it was really very recently. She's been doing this for 30 years.
And why hasn't she made the agreements better? The NAFTA agreement is
defective. Just because of the tax and many other reasons, but just
because of the fact...

\textbf{HOLT:} Let me interrupt just a moment, but...

\textbf{TRUMP:} Secretary Clinton and others, politicians, should have
been doing this for years, not right now, because of the fact that we've
created a movement. They should have been doing this for years. What's
happened to our jobs and our country and our economy generally is ---
look, we owe \$20 trillion. We cannot do it any longer, Lester.

\textbf{HOLT:} Back to the question, though. How do you bring back ---
specifically bring back jobs, American manufacturers? How do you make
them bring the jobs back?

\textbf{TRUMP:} Well, the first thing you do is don't let the jobs
leave. The companies are leaving. I could name, I mean, there are
thousands of them. They're leaving, and they're leaving in bigger
numbers than ever.

And what you do is you say, fine, you want to go to Mexico or some other
country, good luck. We wish you a lot of luck. But if you think you're
going to make your air conditioners or your cars or your cookies or
whatever you make and bring them into our country without a tax, you're
wrong.

And once you say you're going to have to tax them coming in, and our
politicians never do this, because they have special interests and the
special interests want those companies to leave, because in many cases,
they own the companies. So what I'm saying is, we can stop them from
leaving. We have to stop them from leaving. And that's a big, big
factor.

\textbf{HOLT:} Let me let Secretary Clinton get in here.

\textbf{CLINTON:} Well, let's stop for a second and remember where we
were eight years ago. We had the worst financial crisis, the Great
Recession, the worst since the 1930s. That was in large part because of
tax policies that slashed taxes on the wealthy, failed to invest in the
middle class, took their eyes off of Wall Street, and created a perfect
storm.

In fact, Donald was one of the people who rooted for the housing crisis.
He said, back in 2006, ``Gee, I hope it does collapse, because then I
can go in and buy some and make some money.'' Well, it did collapse.

\textbf{TRUMP:} That's called business, by the way.

\textbf{CLINTON:} Nine million people --- nine million people lost their
jobs. Five million people lost their homes. And \$13 trillion in family
wealth was wiped out.

Now, we have come back from that abyss. And it has not been easy. So
we're now on the precipice of having a potentially much better economy,
but the last thing we need to do is to go back to the policies that
failed us in the first place.

Independent experts have looked at what I've proposed and looked at what
Donald's proposed, and basically they've said this, that if his tax
plan, which would blow up the debt by over \$5 trillion and would in
some instances disadvantage middle-class families compared to the
wealthy, were to go into effect, we would lose 3.5 million jobs and
maybe have another recession.

They've looked at my plans and they've said, OK, if we can do this, and
I intend to get it done, we will have 10 million more new jobs, because
we will be making investments where we can grow the economy. Take clean
energy. Some country is going to be the clean- energy superpower of the
21st century. Donald thinks that climate change is a hoax perpetrated by
the Chinese. I think it's real.

\textbf{TRUMP:} I did not. I did not. I do not say that.

\textbf{CLINTON:} I think science is real.

\textbf{TRUMP:} I do not say that.

\textbf{CLINTON:} And I think it's important that we grip this and deal
with it, both at home and abroad. And here's what we can do. We can
deploy a half a billion more solar panels. We can have enough clean
energy to power every home. We can build a new modern electric grid.
That's a lot of jobs; that's a lot of new economic activity.

So I've tried to be very specific about what we can and should do, and I
am determined that we're going to get the economy really moving again,
building on the progress we've made over the last eight years, but never
going back to what got us in trouble in the first place.

\textbf{HOLT:} Mr. Trump?

\textbf{TRUMP:} She talks about solar panels. We invested in a solar
company, our country. That was a disaster. They lost plenty of money on
that one.

Now, look, I'm a great believer in all forms of energy, but we're
putting a lot of people out of work. Our energy policies are a disaster.
Our country is losing so much in terms of energy, in terms of paying off
our debt. You can't do what you're looking to do with \$20 trillion in
debt.

The Obama administration, from the time they've come in, is over 230
years' worth of debt, and he's topped it. He's doubled it in a course of
almost eight years, seven-and-a-half years, to be semi- exact.

So I will tell you this. We have to do a much better job at keeping our
jobs. And we have to do a much better job at giving companies incentives
to build new companies or to expand, because they're not doing it.

And all you have to do is look at Michigan and look at Ohio and look at
all of these places where so many of their jobs and their companies are
just leaving, they're gone.

And, Hillary, I'd just ask you this. You've been doing this for 30
years. Why are you just thinking about these solutions right now? For 30
years, you've been doing it, and now you're just starting to think of
solutions.

\textbf{CLINTON:} Well, actually...

\textbf{TRUMP:} I will bring --- excuse me. I will bring back jobs. You
can't bring back jobs.

\textbf{CLINTON:} Well, actually, I have thought about this quite a bit.

\textbf{TRUMP:} Yeah, for 30 years.

\textbf{CLINTON:} And I have --- well, not quite that long. I think my
husband did a pretty good job in the 1990s. I think a lot about what
worked and how we can make it work again...

\textbf{TRUMP:} Well, he approved NAFTA...

(CROSSTALK)

\textbf{CLINTON:} ... million new jobs, a balanced budget...

\textbf{TRUMP:} He approved NAFTA, which is the single worst trade deal
ever approved in this country.

\textbf{CLINTON:} Incomes went up for everybody. Manufacturing jobs went
up also in the 1990s, if we're actually going to look at the facts.

When I was in the Senate, I had a number of trade deals that came before
me, and I held them all to the same test. Will they create jobs in
America? Will they raise incomes in America? And are they good for our
national security? Some of them I voted for. The biggest one, a
multinational one known as CAFTA, I voted against. And because I hold
the same standards as I look at all of these trade deals.

But let's not assume that trade is the only challenge we have in the
economy. I think it is a part of it, and I've said what I'm going to do.
I'm going to have a special prosecutor. We're going to enforce the trade
deals we have, and we're going to hold people accountable.

When I was secretary of state, we actually increased American exports
globally 30 percent. We increased them to China 50 percent. So I know
how to really work to get new jobs and to get exports that helped to
create more new jobs.

\textbf{HOLT:} Very quickly...

\textbf{TRUMP:} But you haven't done it in 30 years or 26 years or any
number you want to...

\textbf{CLINTON:} Well, I've been a senator, Donald...

\textbf{TRUMP:} You haven't done it. You haven't done it.

\textbf{CLINTON:} And I have been a secretary of state...

\textbf{TRUMP:} Excuse me.

\textbf{CLINTON:} And I have done a lot...

\textbf{TRUMP:} Your husband signed NAFTA, which was one of the worst
things that ever happened to the manufacturing industry.

\textbf{CLINTON:} Well, that's your opinion. That is your opinion.

\textbf{TRUMP:} You go to New England, you go to Ohio, Pennsylvania, you
go anywhere you want, Secretary Clinton, and you will see devastation
where manufacture is down 30, 40, sometimes 50 percent. NAFTA is the
worst trade deal maybe ever signed anywhere, but certainly ever signed
in this country.

And now you want to approve Trans-Pacific Partnership. You were totally
in favor of it. Then you heard what I was saying, how bad it is, and you
said, I can't win that debate. But you know that if you did win, you
would approve that, and that will be almost as bad as NAFTA. Nothing
will ever top NAFTA.

\textbf{CLINTON:} Well, that is just not accurate. I was against it once
it was finally negotiated and the terms were laid out. I wrote about
that in...

\textbf{TRUMP:} You called it the gold standard.

(CROSSTALK)

\textbf{TRUMP:} You called it the gold standard of trade deals. You said
it's the finest deal you've ever seen.

\textbf{CLINTON:} No.

\textbf{TRUMP:} And then you heard what I said about it, and all of a
sudden you were against it.

\textbf{CLINTON:} Well, Donald, I know you live in your own reality, but
that is not the facts. The facts are --- I did say I hoped it would be a
good deal, but when it was negotiated...

\textbf{TRUMP:} Not.

\textbf{CLINTON:} ... which I was not responsible for, I concluded it
wasn't. I wrote about that in my book...

\textbf{TRUMP:} So is it President Obama's fault?

\textbf{CLINTON:} ... before you even announced.

\textbf{TRUMP:} Is it President Obama's fault?

\textbf{CLINTON:} Look, there are differences...

\textbf{TRUMP:} Secretary, is it President Obama's fault?

\textbf{CLINTON:} There are...

\textbf{TRUMP:} Because he's pushing it.

\textbf{CLINTON:} There are different views about what's good for our
country, our economy, and our leadership in the world. And I think it's
important to look at what we need to do to get the economy going again.
That's why I said new jobs with rising incomes, investments, not in more
tax cuts that would add \$5 trillion to the debt.

\textbf{TRUMP:} But you have no plan.

\textbf{CLINTON:} But in --- oh, but I do.

\textbf{TRUMP:} Secretary, you have no plan.

\textbf{CLINTON:} In fact, I have written a book about it. It's called
``Stronger Together.'' You can pick it up tomorrow at a bookstore...

\textbf{TRUMP:} That's about all you've...

(CROSSTALK)

\textbf{HOLT:} Folks, we're going to...

\textbf{CLINTON:} ... or at an airport near you.

\textbf{HOLT:} We're going to move to...

\textbf{CLINTON:} But it's because I see this --- we need to have strong
growth, fair growth, sustained growth. We also have to look at how we
help families balance the responsibilities at home and the
responsibilities at business.

So we have a very robust set of plans. And people have looked at both of
our plans, have concluded that mine would create 10 million jobs and
yours would lose us 3.5 million jobs, and explode the debt which would
have a recession.

\textbf{TRUMP:} You are going to approve one of the biggest tax cuts in
history. You are going to approve one of the biggest tax increases in
history. You are going to drive business out. Your regulations are a
disaster, and you're going to increase regulations all over the place.

And by the way, my tax cut is the biggest since Ronald Reagan. I'm very
proud of it. It will create tremendous numbers of new jobs. But
regulations, you are going to regulate these businesses out of
existence.

When I go around --- Lester, I tell you this, I've been all over. And
when I go around, despite the tax cut, the thing --- the things that
business as in people like the most is the fact that I'm cutting
regulation. You have regulations on top of regulations, and new
companies cannot form and old companies are going out of business. And
you want to increase the regulations and make them even worse.

I'm going to cut regulations. I'm going to cut taxes big league, and
you're going to raise taxes big league, end of story.

\textbf{HOLT:} Let me get you to pause right there, because we're going
to move into --- we're going to move into the next segment. We're going
to talk taxes...

\textbf{CLINTON:} That can't --- that can't be left to stand.

\textbf{HOLT:} Please just take 30 seconds and then we're going to go
on.

\textbf{CLINTON:} I kind of assumed that there would be a lot of these
charges and claims, and so...

\textbf{TRUMP:} Facts.

\textbf{CLINTON:} So we have taken the home page of my website,
HillaryClinton.com, and we've turned it into a fact-checker. So if you
want to see in real-time what the facts are, please go and take a look.
Because what I have proposed...

\textbf{TRUMP:} And take a look at mine, also, and you'll see.

\textbf{CLINTON:} ... would not add a penny to the debt, and your plans
would add \$5 trillion to the debt. What I have proposed would cut
regulations and streamline them for small businesses. What I have
proposed would be paid for by raising taxes on the wealthy, because they
have made all the gains in the economy. And I think it's time that the
wealthy and corporations paid their fair share to support this country.

\includegraphics{https://static01.graylady3jvrrxbe.onion/images/2016/09/27/us/politics/27DEBATE/27DEBATE-videoSixteenByNineJumbo1600.jpg}

\textbf{HOLT:} Well, you just opened the next segment.

\textbf{TRUMP:} Well, could I just finish --- I think I...

(CROSSTALK)

\textbf{HOLT:} I'm going to give you a chance right here...

\textbf{TRUMP:} I think I should --- you go to her website, and you take
a look at her website.

\textbf{HOLT:} ... with a new 15-minute segment...

\textbf{TRUMP:} She's going to raise taxes \$1.3 trillion.

\textbf{HOLT:} Mr. Trump, I'm going to...

\textbf{TRUMP:} And look at her website. You know what? It's no
difference than this. She's telling us how to fight ISIS. Just go to her
website. She tells you how to fight ISIS on her website. I don't think
General Douglas MacArthur would like that too much.

\textbf{HOLT:} The next segment, we're continuing...

\textbf{CLINTON:} Well, at least I have a plan to fight ISIS.

\textbf{HOLT:} ... achieving prosperity...

\textbf{TRUMP:} No, no, you're telling the enemy everything you want to
do.

\textbf{CLINTON:} No, we're not. No, we're not.

\textbf{TRUMP:} See, you're telling the enemy everything you want to do.
No wonder you've been fighting --- no wonder you've been fighting ISIS
your entire adult life.

\textbf{CLINTON:} That's a --- that's --- go to the --- please, fact
checkers, get to work.

\textbf{HOLT:} OK, you are unpacking a lot here. And we're still on the
issue of achieving prosperity. And I want to talk about taxes. The
fundamental difference between the two of you concerns the wealthy.

Secretary Clinton, you're calling for a tax increase on the wealthiest
Americans. I'd like you to further defend that. And, Mr. Trump, you're
calling for tax cuts for the wealthy. I'd like you to defend that. And
this next two-minute answer goes to you, Mr. Trump.

\textbf{TRUMP:} Well, I'm really calling for major jobs, because the
wealthy are going create tremendous jobs. They're going to expand their
companies. They're going to do a tremendous job.

I'm getting rid of the carried interest provision. And if you really
look, it's not a tax --- it's really not a great thing for the wealthy.
It's a great thing for the middle class. It's a great thing for
companies to expand.

And when these people are going to put billions and billions of dollars
into companies, and when they're going to bring \$2.5 trillion back from
overseas, where they can't bring the money back, because politicians
like Secretary Clinton won't allow them to bring the money back, because
the taxes are so onerous, and the bureaucratic red tape, so what --- is
so bad.

So what they're doing is they're leaving our country, and they're,
believe it or not, leaving because taxes are too high and because some
of them have lots of money outside of our country. And instead of
bringing it back and putting the money to work, because they can't work
out a deal to --- and everybody agrees it should be brought back.

Instead of that, they're leaving our country to get their money, because
they can't bring their money back into our country, because of
bureaucratic red tape, because they can't get together. Because we have
--- we have a president that can't sit them around a table and get them
to approve something.

And here's the thing. Republicans and Democrats agree that this should
be done, \$2.5 trillion. I happen to think it's double that. It's
probably \$5 trillion that we can't bring into our country, Lester. And
with a little leadership, you'd get it in here very quickly, and it
could be put to use on the inner cities and lots of other things, and it
would be beautiful.

But we have no leadership. And honestly, that starts with Secretary
Clinton.

\textbf{HOLT:} All right. You have two minutes of the same question to
defend tax increases on the wealthiest Americans, Secretary Clinton.

\textbf{CLINTON:} I have a feeling that by, the end of this evening, I'm
going to be blamed for everything that's ever happened.

\textbf{TRUMP:} Why not?

\textbf{CLINTON:} Why not? Yeah, why not?

(LAUGHTER)

You know, just join the debate by saying more crazy things. Now, let me
say this, it is absolutely the case...

\textbf{TRUMP:} There's nothing crazy about not letting our companies
bring their money back into their country.

\textbf{HOLT:} This is --- this is Secretary Clinton's two minutes,
please.

\textbf{TRUMP:} Yes.

\textbf{CLINTON:} Yeah, well, let's start the clock again, Lester. We've
looked at your tax proposals. I don't see changes in the corporate tax
rates or the kinds of proposals you're referring to that would cause the
repatriation, bringing back of money that's stranded overseas. I happen
to support that.

\textbf{TRUMP:} Then you didn't read it.

\textbf{CLINTON:} I happen to --- I happen to support that in a way that
will actually work to our benefit. But when I look at what you have
proposed, you have what is called now the Trump loophole, because it
would so advantage you and the business you do. You've proposed an
approach that has a...

\textbf{TRUMP:} Who gave it that name? The first I've --- who gave it
that name?

(CROSSTALK)

\textbf{HOLT:} Mr. Trump, this is Secretary Clinton's two minutes.

\textbf{CLINTON:} ... \$4 billion tax benefit for your family. And when
you look at what you are proposing...

\textbf{TRUMP:} How much? How much for my family? CLINTON: ... it is...

\textbf{TRUMP:} Lester, how much?

\textbf{CLINTON:} ... as I said, trumped-up trickle-down. Trickle-down
did not work. It got us into the mess we were in, in 2008 and 2009.
Slashing taxes on the wealthy hasn't worked.

And a lot of really smart, wealthy people know that. And they are
saying, hey, we need to do more to make the contributions we should be
making to rebuild the middle class.

\textbf{CLINTON:} I don't think top-down works in America. I think
building the middle class, investing in the middle class, making college
debt-free so more young people can get their education, helping people
refinance their --- their debt from college at a lower rate. Those are
the kinds of things that will really boost the economy. Broad-based,
inclusive growth is what we need in America, not more advantages for
people at the very top.

\textbf{HOLT:} Mr. Trump, we're...

\textbf{TRUMP:} Typical politician. All talk, no action. Sounds good,
doesn't work. Never going to happen. Our country is suffering because
people like Secretary Clinton have made such bad decisions in terms of
our jobs and in terms of what's going on.

Now, look, we have the worst revival of an economy since the Great
Depression. And believe me: We're in a bubble right now. And the only
thing that looks good is the stock market, but if you raise interest
rates even a little bit, that's going to come crashing down.

We are in a big, fat, ugly bubble. And we better be awfully careful. And
we have a Fed that's doing political things. This Janet Yellen of the
Fed. The Fed is doing political --- by keeping the interest rates at
this level. And believe me: The day Obama goes off, and he leaves, and
goes out to the golf course for the rest of his life to play golf, when
they raise interest rates, you're going to see some very bad things
happen, because the Fed is not doing their job. The Fed is being more
political than Secretary Clinton.

\textbf{HOLT:} Mr. Trump, we're talking about the burden that Americans
have to pay, yet you have not released your tax returns. And the reason
nominees have released their returns for decades is so that voters will
know if their potential president owes money to --- who he owes it to
and any business conflicts. Don't Americans have a right to know if
there are any conflicts of interest?

\textbf{TRUMP:} I don't mind releasing --- I'm under a routine audit.
And it'll be released. And --- as soon as the audit's finished, it will
be released.

But you will learn more about Donald Trump by going down to the federal
elections, where I filed a 104-page essentially financial statement of
sorts, the forms that they have. It shows income --- in fact, the income
--- I just looked today --- the income is filed at \$694 million for
this past year, \$694 million. If you would have told me I was going to
make that 15 or 20 years ago, I would have been very surprised.

But that's the kind of thinking that our country needs. When we have a
country that's doing so badly, that's being ripped off by every single
country in the world, it's the kind of thinking that our country needs,
because everybody --- Lester, we have a trade deficit with all of the
countries that we do business with, of almost \$800 billion a year. You
know what that is? That means, who's negotiating these trade deals?

We have people that are political hacks negotiating our trade deals.

\textbf{HOLT:} The IRS says an audit...

\textbf{TRUMP:} Excuse me.

\textbf{HOLT:} ... of your taxes --- you're perfectly free to release
your taxes during an audit. And so the question, does the public's right
to know outweigh your personal...

\textbf{TRUMP:} Well, I told you, I will release them as soon as the
audit. Look, I've been under audit almost for 15 years. I know a lot of
wealthy people that have never been audited. I said, do you get audited?
I get audited almost every year.

And in a way, I should be complaining. I'm not even complaining. I don't
mind it. It's almost become a way of life. I get audited by the IRS. But
other people don't.

I will say this. We have a situation in this country that has to be
taken care of. I will release my tax returns --- against my lawyer's
wishes --- when she releases her 33,000 e-mails that have been deleted.
As soon as she releases them, I will release.

(APPLAUSE)

I will release my tax returns. And that's against --- my lawyers, they
say, ``Don't do it.'' I will tell you this. No --- in fact, watching
shows, they're reading the papers. Almost every lawyer says, you don't
release your returns until the audit's complete. When the audit's
complete, I'll do it. But I would go against them if she releases her
e-mails.

\textbf{HOLT:} So it's negotiable?

\textbf{TRUMP:} It's not negotiable, no. Let her release the e-mails.
Why did she delete 33,000...

\textbf{HOLT:} Well, I'll let her answer that. But let me just admonish
the audience one more time. There was an agreement. We did ask you to be
silent, so it would be helpful for us. Secretary Clinton?

\textbf{CLINTON:} Well, I think you've seen another example of bait-and-
switch here. For 40 years, everyone running for president has released
their tax returns. You can go and see nearly, I think, 39, 40 years of
our tax returns, but everyone has done it. We know the IRS has made
clear there is no prohibition on releasing it when you're under audit.

So you've got to ask yourself, why won't he release his tax returns? And
I think there may be a couple of reasons. First, maybe he's not as rich
as he says he is. Second, maybe he's not as charitable as he claims to
be.

\textbf{CLINTON:} Third, we don't know all of his business dealings, but
we have been told through investigative reporting that he owes about
\$650 million to Wall Street and foreign banks. Or maybe he doesn't want
the American people, all of you watching tonight, to know that he's paid
nothing in federal taxes, because the only years that anybody's ever
seen were a couple of years when he had to turn them over to state
authorities when he was trying to get a casino license, and they showed
he didn't pay any federal income tax.

\textbf{TRUMP:} That makes me smart.

\textbf{CLINTON:} So if he's paid zero, that means zero for troops, zero
for vets, zero for schools or health. And I think probably he's not all
that enthusiastic about having the rest of our country see what the real
reasons are, because it must be something really important, even
terrible, that he's trying to hide.

And the financial disclosure statements, they don't give you the tax
rate. They don't give you all the details that tax returns would. And it
just seems to me that this is something that the American people deserve
to see. And I have no reason to believe that he's ever going to release
his tax returns, because there's something he's hiding.

And we'll guess. We'll keep guessing at what it might be that he's
hiding. But I think the question is, were he ever to get near the White
House, what would be those conflicts? Who does he owe money to? Well, he
owes you the answers to that, and he should provide them.

\textbf{HOLT:} He also --- he also raised the issue of your e-mails. Do
you want to respond to that?

\textbf{CLINTON:} I do. You know, I made a mistake using a private e-
mail. TRUMP: That's for sure.

\textbf{CLINTON:} And if I had to do it over again, I would, obviously,
do it differently. But I'm not going to make any excuses. It was a
mistake, and I take responsibility for that.

\textbf{HOLT:} Mr. Trump?

\textbf{TRUMP:} That was more than a mistake. That was done purposely.
OK? That was not a mistake. That was done purposely. When you have your
staff taking the Fifth Amendment, taking the Fifth so they're not
prosecuted, when you have the man that set up the illegal server taking
the Fifth, I think it's disgraceful. And believe me, this country thinks
it's --- really thinks it's disgraceful, also.

As far as my tax returns, you don't learn that much from tax returns.
That I can tell you. You learn a lot from financial disclosure. And you
should go down and take a look at that.

The other thing, I'm extremely underleveraged. The report that said
\$650 --- which, by the way, a lot of friends of mine that know my
business say, boy, that's really not a lot of money. It's not a lot of
money relative to what I had.

The buildings that were in question, they said in the same report, which
was --- actually, it wasn't even a bad story, to be honest with you, but
the buildings are worth \$3.9 billion. And the \$650 isn't even on that.
But it's not \$650. It's much less than that.

But I could give you a list of banks, I would --- if that would help
you, I would give you a list of banks. These are very fine institutions,
very fine banks. I could do that very quickly.

I am very underleveraged. I have a great company. I have a tremendous
income. And the reason I say that is not in a braggadocios way. It's
because it's about time that this country had somebody running it that
has an idea about money.

When we have \$20 trillion in debt, and our country's a mess, you know,
it's one thing to have \$20 trillion in debt and our roads are good and
our bridges are good and everything's in great shape, our airports. Our
airports are like from a third world country.

You land at LaGuardia, you land at Kennedy, you land at LAX, you land at
Newark, and you come in from Dubai and Qatar and you see these
incredible --- you come in from China, you see these incredible
airports, and you land --- we've become a third world country.

So the worst of all things has happened. We owe \$20 trillion, and we're
a mess. We haven't even started. And we've spent \$6 trillion in the
Middle East, according to a report that I just saw. Whether it's 6 or 5,
but it looks like it's 6, \$6 trillion in the Middle East, we could have
rebuilt our country twice.

And it's really a shame. And it's politicians like Secretary Clinton
that have caused this problem. Our country has tremendous problems.
We're a debtor nation. We're a serious debtor nation. And we have a
country that needs new roads, new tunnels, new bridges, new airports,
new schools, new hospitals. And we don't have the money, because it's
been squandered on so many of your ideas.

\textbf{HOLT:} We'll let you respond and we'll move on to the next
segment.

\textbf{CLINTON:} And maybe because you haven't paid any federal income
tax for a lot of years. (APPLAUSE)

And the other thing I think is important...

\textbf{TRUMP:} It would be squandered, too, believe me.

\textbf{CLINTON:} ... is if your --- if your main claim to be president
of the United States is your business, then I think we should talk about
that. You know, your campaign manager said that you built a lot of
businesses on the backs of little guys.

And, indeed, I have met a lot of the people who were stiffed by you and
your businesses, Donald. I've met dishwashers, painters, architects,
glass installers, marble installers, drapery installers, like my dad
was, who you refused to pay when they finished the work that you asked
them to do.

We have an architect in the audience who designed one of your clubhouses
at one of your golf courses. It's a beautiful facility. It immediately
was put to use. And you wouldn't pay what the man needed to be paid,
what he was charging you to do...

\textbf{TRUMP:} Maybe he didn't do a good job and I was unsatisfied with
his work...

\textbf{CLINTON:} Well, to...

\textbf{TRUMP:} Which our country should do, too.

\textbf{CLINTON:} Do the thousands of people that you have stiffed over
the course of your business not deserve some kind of apology from
someone who has taken their labor, taken the goods that they produced,
and then refused to pay them?

I can only say that I'm certainly relieved that my late father never did
business with you. He provided a good middle-class life for us, but the
people he worked for, he expected the bargain to be kept on both sides.

And when we talk about your business, you've taken business bankruptcy
six times. There are a lot of great businesspeople that have never taken
bankruptcy once. You call yourself the King of Debt. You talk about
leverage. You even at one time suggested that you would try to negotiate
down the national debt of the United States.

\textbf{TRUMP:} Wrong. Wrong.

\textbf{CLINTON:} Well, sometimes there's not a direct transfer of
skills from business to government, but sometimes what happened in
business would be really bad for government.

\textbf{HOLT:} Let's let Mr. Trump...

\textbf{CLINTON:} And we need to be very clear about that.

\textbf{TRUMP:} So, yeah, I think --- I do think it's time. Look, it's
all words, it's all sound bites. I built an unbelievable company. Some
of the greatest assets anywhere in the world, real estate assets
anywhere in the world, beyond the United States, in Europe, lots of
different places. It's an unbelievable company.

But on occasion, four times, we used certain laws that are there. And
when Secretary Clinton talks about people that didn't get paid, first of
all, they did get paid a lot, but taken advantage of the laws of the
nation.

Now, if you want to change the laws, you've been there a long time,
change the laws. But I take advantage of the laws of the nation because
I'm running a company. My obligation right now is to do well for myself,
my family, my employees, for my companies. And that's what I do.

But what she doesn't say is that tens of thousands of people that are
unbelievably happy and that love me. I'll give you an example. We're
just opening up on Pennsylvania Avenue right next to the White House, so
if I don't get there one way, I'm going to get to Pennsylvania Avenue
another.

But we're opening the Old Post Office. Under budget, ahead of schedule,
saved tremendous money. I'm a year ahead of schedule. And that's what
this country should be doing.

We build roads and they cost two and three and four times what they're
supposed to cost. We buy products for our military and they come in at
costs that are so far above what they were supposed to be, because we
don't have people that know what they're doing.

When we look at the budget, the budget is bad to a large extent because
we have people that have no idea as to what to do and how to buy. The
Trump International is way under budget and way ahead of schedule. And
we should be able to do that for our country.

\textbf{HOLT:} Well, we're well behind schedule, so I want to move to
our next segment. We move into our next segment talking about America's
direction. And let's start by talking about race.

The share of Americans who say race relations are bad in this country is
the highest it's been in decades, much of it amplified by shootings of
African-Americans by police, as we've seen recently in Charlotte and
Tulsa. Race has been a big issue in this campaign, and one of you is
going to have to bridge a very wide and bitter gap.

So how do you heal the divide? Secretary Clinton, you get two minutes on
this.

\textbf{CLINTON:} Well, you're right. Race remains a significant
challenge in our country. Unfortunately, race still determines too much,
often determines where people live, determines what kind of education in
their public schools they can get, and, yes, it determines how they're
treated in the criminal justice system. We've just seen those two tragic
examples in both Tulsa and Charlotte.

And we've got to do several things at the same time. We have to restore
trust between communities and the police. We have to work to make sure
that our police are using the best training, the best techniques, that
they're well prepared to use force only when necessary. Everyone should
be respected by the law, and everyone should respect the law.

\textbf{CLINTON:} Right now, that's not the case in a lot of our
neighborhoods. So I have, ever since the first day of my campaign,
called for criminal justice reform. I've laid out a platform that I
think would begin to remedy some of the problems we have in the criminal
justice system.

But we also have to recognize, in addition to the challenges that we
face with policing, there are so many good, brave police officers who
equally want reform. So we have to bring communities together in order
to begin working on that as a mutual goal. And we've got to get guns out
of the hands of people who should not have them.

The gun epidemic is the leading cause of death of young African-
American men, more than the next nine causes put together. So we have to
do two things, as I said. We have to restore trust. We have to work with
the police. We have to make sure they respect the communities and the
communities respect them. And we have to tackle the plague of gun
violence, which is a big contributor to a lot of the problems that we're
seeing today.

\textbf{HOLT:} All right, Mr. Trump, you have two minutes. How do you
heal the divide?

\textbf{TRUMP:} Well, first of all, Secretary Clinton doesn't want to
use a couple of words, and that's law and order. And we need law and
order. If we don't have it, we're not going to have a country.

And when I look at what's going on in Charlotte, a city I love, a city
where I have investments, when I look at what's going on throughout
various parts of our country, whether it's --- I mean, I can just keep
naming them all day long --- we need law and order in our country.

I just got today the, as you know, the endorsement of the Fraternal
Order of Police, we just --- just came in. We have endorsements from, I
think, almost every police group, very --- I mean, a large percentage of
them in the United States.

We have a situation where we have our inner cities, African- Americans,
Hispanics are living in he'll because it's so dangerous. You walk down
the street, you get shot.

In Chicago, they've had thousands of shootings, thousands since January
1st. Thousands of shootings. And I'm saying, where is this? Is this a
war-torn country? What are we doing? And we have to stop the violence.
We have to bring back law and order. In a place like Chicago, where
thousands of people have been killed, thousands over the last number of
years, in fact, almost 4,000 have been killed since Barack Obama became
president, over --- almost 4,000 people in Chicago have been killed. We
have to bring back law and order.

Now, whether or not in a place like Chicago you do stop and frisk, which
worked very well, Mayor Giuliani is here, worked very well in New York.
It brought the crime rate way down. But you take the gun away from
criminals that shouldn't be having it.

We have gangs roaming the street. And in many cases, they're illegally
here, illegal immigrants. And they have guns. And they shoot people. And
we have to be very strong. And we have to be very vigilant.

\href{https://www.nytimes3xbfgragh.onion/interactive/2016/09/26/us/elections/donald-trump-hillary-clinton-debate.html}{}

\includegraphics{https://static01.graylady3jvrrxbe.onion/images/2016/09/27/us/elections/27liveblogpromo3/27liveblogpromo3-square640.jpg}

\hypertarget{first-clinton-and-trump-debate-analysis}{%
\subsection{First Clinton and Trump Debate:
Analysis}\label{first-clinton-and-trump-debate-analysis}}

The Times analyzed the first presidential debate between Hillary Clinton
and Donald J. Trump in real time.

We have to be --- we have to know what we're doing. Right now, our
police, in many cases, are afraid to do anything. We have to protect our
inner cities, because African-American communities are being decimated
by crime, decimated.

\textbf{HOLT:} Your two --- your two minutes expired, but I do want to
follow up. Stop-and-frisk was ruled unconstitutional in New York,
because it largely singled out black and Hispanic young men.

\textbf{TRUMP:} No, you're wrong. It went before a judge, who was a very
against-police judge. It was taken away from her. And our mayor, our new
mayor, refused to go forward with the case. They would have won an
appeal. If you look at it, throughout the country, there are many places
where it's allowed.

\textbf{HOLT:} The argument is that it's a form of racial profiling.

\textbf{TRUMP:} No, the argument is that we have to take the guns away
from these people that have them and they are bad people that shouldn't
have them.

These are felons. These are people that are bad people that shouldn't be
--- when you have 3,000 shootings in Chicago from January 1st, when you
have 4,000 people killed in Chicago by guns, from the beginning of the
presidency of Barack Obama, his hometown, you have to have
stop-and-frisk.

You need more police. You need a better community, you know, relation.
You don't have good community relations in Chicago. It's terrible. I
have property there. It's terrible what's going on in Chicago.

But when you look --- and Chicago's not the only --- you go to Ferguson,
you go to so many different places. You need better relationships. I
agree with Secretary Clinton on this.

\textbf{TRUMP:} You need better relationships between the communities
and the police, because in some cases, it's not good.

But you look at Dallas, where the relationships were really studied, the
relationships were really a beautiful thing, and then five police
officers were killed one night very violently. So there's some bad
things going on. Some really bad things.

\textbf{HOLT:} Secretary Clinton...

\textbf{TRUMP:} But we need --- Lester, we need law and order. And we
need law and order in the inner cities, because the people that are most
affected by what's happening are African-American and Hispanic people.
And it's very unfair to them what our politicians are allowing to
happen.

\textbf{HOLT:} Secretary Clinton?

\textbf{CLINTON:} Well, I've heard --- I've heard Donald say this at his
rallies, and it's really unfortunate that he paints such a dire negative
picture of black communities in our country.

\textbf{TRUMP:} Ugh.

\textbf{CLINTON:} You know, the vibrancy of the black church, the black
businesses that employ so many people, the opportunities that so many
families are working to provide for their kids. There's a lot that we
should be proud of and we should be supporting and lifting up.

But we do always have to make sure we keep people safe. There are the
right ways of doing it, and then there are ways that are ineffective.
Stop-and-frisk was found to be unconstitutional and, in part, because it
was ineffective. It did not do what it needed to do.

Now, I believe in community policing. And, in fact, violent crime is
one-half of what it was in 1991. Property crime is down 40 percent. We
just don't want to see it creep back up. We've had 25 years of very good
cooperation.

But there were some problems, some unintended consequences. Too many
young African-American and Latino men ended up in jail for nonviolent
offenses. And it's just a fact that if you're a young African-American
man and you do the same thing as a young white man, you are more likely
to be arrested, charged, convicted, and incarcerated. So we've got to
address the systemic racism in our criminal justice system. We cannot
just say law and order. We have to say --- we have to come forward with
a plan that is going to divert people from the criminal justice system,
deal with mandatory minimum sentences, which have put too many people
away for too long for doing too little.

We need to have more second chance programs. I'm glad that we're ending
private prisons in the federal system; I want to see them ended in the
state system. You shouldn't have a profit motivation to fill prison
cells with young Americans. So there are some positive ways we can work
on this.

And I believe strongly that commonsense gun safety measures would assist
us. Right now --- and this is something Donald has supported, along with
the gun lobby --- right now, we've got too many military- style weapons
on the streets. In a lot of places, our police are outgunned. We need
comprehensive background checks, and we need to keep guns out of the
hands of those who will do harm.

And we finally need to pass a prohibition on anyone who's on the
terrorist watch list from being able to buy a gun in our country. If
you're too dangerous to fly, you are too dangerous to buy a gun. So
there are things we can do, and we ought to do it in a bipartisan way.

\textbf{HOLT:} Secretary Clinton, last week, you said we've got to do
everything possible to improve policing, to go right at implicit bias.
Do you believe that police are implicitly biased against black people?

\textbf{CLINTON:} Lester, I think implicit bias is a problem for
everyone, not just police. I think, unfortunately, too many of us in our
great country jump to conclusions about each other. And therefore, I
think we need all of us to be asking hard questions about, you know, why
am I feeling this way?

But when it comes to policing, since it can have literally fatal
consequences, I have said, in my first budget, we would put money into
that budget to help us deal with implicit bias by retraining a lot of
our police officers.

I've met with a group of very distinguished, experienced police chiefs a
few weeks ago. They admit it's an issue. They've got a lot of concerns.
Mental health is one of the biggest concerns, because now police are
having to handle a lot of really difficult mental health problems on the
street.

\textbf{CLINTON:} They want support, they want more training, they want
more assistance. And I think the federal government could be in a
position where we would offer and provide that.

\textbf{HOLT:} Mr. Trump...

\textbf{TRUMP:} I'd like to respond to that.

\textbf{HOLT:} Please.

\textbf{TRUMP:} First of all, I agree, and a lot of people even within
my own party want to give certain rights to people on watch lists and
no- fly lists. I agree with you. When a person is on a watch list or a
no-fly list, and I have the endorsement of the NRA, which I'm very proud
of. These are very, very good people, and they're protecting the Second
Amendment.

But I think we have to look very strongly at no-fly lists and watch
lists. And when people are on there, even if they shouldn't be on there,
we'll help them, we'll help them legally, we'll help them get off. But I
tend to agree with that quite strongly.

I do want to bring up the fact that you were the one that brought up the
words super-predator about young black youth. And that's a term that I
think was a --- it's --- it's been horribly met, as you know. I think
you've apologized for it. But I think it was a terrible thing to say.

And when it comes to stop-and-frisk, you know, you're talking about
takes guns away. Well, I'm talking about taking guns away from gangs and
people that use them. And I don't think --- I really don't think you
disagree with me on this, if you want to know the truth.

I think maybe there's a political reason why you can't say it, but I
really don't believe --- in New York City, stop-and-frisk, we had 2,200
murders, and stop-and-frisk brought it down to 500 murders. Five hundred
murders is a lot of murders. It's hard to believe, 500 is like supposed
to be good?

But we went from 2,200 to 500. And it was continued on by Mayor
Bloomberg. And it was terminated by current mayor. But stop-and- frisk
had a tremendous impact on the safety of New York City. Tremendous
beyond belief. So when you say it has no impact, it really did. It had a
very, very big impact.

\textbf{CLINTON:} Well, it's also fair to say, if we're going to talk
about mayors, that under the current mayor, crime has continued to drop,
including murders. So there is...

\textbf{TRUMP:} No, you're wrong. You're wrong.

\textbf{CLINTON:} No, I'm not.

\textbf{TRUMP:} Murders are up. All right. You check it.

\textbf{CLINTON:} New York --- New York has done an excellent job. And I
give credit --- I give credit across the board going back two mayors,
two police chiefs, because it has worked. And other communities need to
come together to do what will work, as well.

Look, one murder is too many. But it is important that we learn about
what has been effective. And not go to things that sound good that
really did not have the kind of impact that we would want. Who disagrees
with keeping neighborhoods safe?

But let's also add, no one should disagree about respecting the rights
of young men who live in those neighborhoods. And so we need to do a
better job of working, again, with the communities, faith communities,
business communities, as well as the police to try to deal with this
problem.

\textbf{HOLT:} This conversation is about race. And so, Mr. Trump, I
have to ask you for five...

\textbf{TRUMP:} I'd like to just respond, if I might.

\textbf{HOLT:} Please --- 20 seconds.

\textbf{TRUMP:} I'd just like to respond.

\textbf{HOLT:} Please respond, then I've got a quick follow-up for you.

\textbf{TRUMP:} I will. Look, the African-American community has been
let down by our politicians. They talk good around election time, like
right now, and after the election, they said, see ya later, I'll see you
in four years.

The African-American community --- because --- look, the community
within the inner cities has been so badly treated. They've been abused
and used in order to get votes by Democrat politicians, because that's
what it is. They've controlled these communities for up to 100 years.

\textbf{HOLT:} Mr. Trump, let me...

(CROSSTALK)

\textbf{CLINTON:} Well, I --- I do think...

\textbf{TRUMP:} And I will tell you, you look at the inner cities ---
and I just left Detroit, and I just left Philadelphia, and I just ---
you know, you've seen me, I've been all over the place. You decided to
stay home, and that's OK. But I will tell you, I've been all over. And
I've met some of the greatest people I'll ever meet within these
communities. And they are very, very upset with what their politicians
have told them and what their politicians have done.

\textbf{HOLT:} Mr. Trump, I...

\textbf{CLINTON:} I think --- I think --- I think Donald just criticized
me for preparing for this debate. And, yes, I did. And you know what
else I prepared for? I prepared to be president. And I think that's a
good thing.

(APPLAUSE)

\textbf{HOLT:} Mr. Trump, for five years, you perpetuated a false claim
that the nation's first black president was not a natural-born citizen.
You questioned his legitimacy. In the last couple of weeks, you
acknowledged what most Americans have accepted for years: The president
was born in the United States. Can you tell us what took you so long?

\textbf{TRUMP:} I'll tell you very --- well, just very simple to say.
Sidney Blumenthal works for the campaign and close --- very close friend
of Secretary Clinton. And her campaign manager, Patti Doyle, went to ---
during the campaign, her campaign against President Obama, fought very
hard. And you can go look it up, and you can check it out.

\textbf{TRUMP:} And if you look at CNN this past week, Patti Solis Doyle
was on Wolf Blitzer saying that this happened. Blumenthal sent
McClatchy, highly respected reporter at McClatchy, to Kenya to find out
about it. They were pressing it very hard. She failed to get the birth
certificate.

When I got involved, I didn't fail. I got him to give the birth
certificate. So I'm satisfied with it. And I'll tell you why I'm
satisfied with it.

\textbf{HOLT:} That was...

(CROSSTALK)

\textbf{TRUMP:} Because I want to get on to defeating ISIS, because I
want to get on to creating jobs, because I want to get on to having a
strong border, because I want to get on to things that are very
important to me and that are very important to the country.

\textbf{HOLT:} I will let you respond. It's important. But I just want
to get the answer here. The birth certificate was produced in 2011.
You've continued to tell the story and question the president's
legitimacy in 2012, '13, '14, '15...

\textbf{TRUMP:} Yeah.

\textbf{HOLT:} .... as recently as January. So the question is, what
changed your mind?

\textbf{TRUMP:} Well, nobody was pressing it, nobody was caring much
about it. I figured you'd ask the question tonight, of course. But
nobody was caring much about it. But I was the one that got him to
produce the birth certificate. And I think I did a good job.

Secretary Clinton also fought it. I mean, you know --- now, everybody in
mainstream is going to say, oh, that's not true. Look, it's true. Sidney
Blumenthal sent a reporter --- you just have to take a look at CNN, the
last week, the interview with your former campaign manager. And she was
involved. But just like she can't bring back jobs, she can't produce.

\textbf{HOLT:} I'm sorry. I'm just going to follow up --- and I will let
you respond to that, because there's a lot there. But we're talking
about racial healing in this segment. What do you say to Americans,
people of color who...

(CROSSTALK)

\textbf{TRUMP:} Well, it was very --- I say nothing. I say nothing,
because I was able to get him to produce it. He should have produced it
a long time before. I say nothing.

But let me just tell you. When you talk about healing, I think that I've
developed very, very good relationships over the last little while with
the African-American community. I think you can see that.

And I feel that they really wanted me to come to that conclusion. And I
think I did a great job and a great service not only for the country,
but even for the president, in getting him to produce his birth
certificate.

\textbf{HOLT:} Secretary Clinton?

\textbf{CLINTON:} Well, just listen to what you heard.

(LAUGHTER)

And clearly, as Donald just admitted, he knew he was going to stand on
this debate stage, and Lester Holt was going to be asking us questions,
so he tried to put the whole racist birther lie to bed.

But it can't be dismissed that easily. He has really started his
political activity based on this racist lie that our first black
president was not an American citizen. There was absolutely no evidence
for it, but he persisted, he persisted year after year, because some of
his supporters, people that he was trying to bring into his fold,
apparently believed it or wanted to believe it.

But, remember, Donald started his career back in 1973 being sued by the
Justice Department for racial discrimination because he would not rent
apartments in one of his developments to African-Americans, and he made
sure that the people who worked for him understood that was the policy.
He actually was sued twice by the Justice Department.

So he has a long record of engaging in racist behavior. And the birther
lie was a very hurtful one. You know, Barack Obama is a man of great
dignity. And I could tell how much it bothered him and annoyed him that
this was being touted and used against him.

But I like to remember what Michelle Obama said in her amazing speech at
our Democratic National Convention: When they go low, we go high. And
Barack Obama went high, despite Donald Trump's best efforts to bring him
down.

\textbf{HOLT:} Mr. Trump, you can respond and we're going to move on to
the next segment.

\textbf{TRUMP:} I would love to respond. First of all, I got to watch in
preparing for this some of your debates against Barack Obama. You
treated him with terrible disrespect. And I watched the way you talk now
about how lovely everything is and how wonderful you are. It doesn't
work that way. You were after him, you were trying to --- you even sent
out or your campaign sent out pictures of him in a certain garb, very
famous pictures. I don't think you can deny that.

But just last week, your campaign manager said it was true. So when you
tried to act holier than thou, it really doesn't work. It really
doesn't.

Now, as far as the lawsuit, yes, when I was very young, I went into my
father's company, had a real estate company in Brooklyn and Queens, and
we, along with many, many other companies throughout the country --- it
was a federal lawsuit --- were sued. We settled the suit with zero ---
with no admission of guilt. It was very easy to do.

\textbf{TRUMP:} I notice you bring that up a lot. And, you know, I also
notice the very nasty commercials that you do on me in so many different
ways, which I don't do on you. Maybe I'm trying to save the money.

But, frankly, I look --- I look at that, and I say, isn't that amazing?
Because I settled that lawsuit with no admission of guilt, but that was
a lawsuit brought against many real estate firms, and it's just one of
those things.

I'll go one step further. In Palm Beach, Florida, tough community, a
brilliant community, a wealthy community, probably the wealthiest
community there is in the world, I opened a club, and really got great
credit for it. No discrimination against African- Americans, against
Muslims, against anybody. And it's a tremendously successful club. And
I'm so glad I did it. And I have been given great credit for what I did.
And I'm very, very proud of it. And that's the way I feel. That is the
true way I feel.

\textbf{HOLT:} Our next segment is called ``Securing America.'' We want
to start with a 21st century war happening every day in this country.
Our institutions are under cyber attack, and our secrets are being
stolen. So my question is, who's behind it? And how do we fight it?

Secretary Clinton, this answer goes to you.

\textbf{CLINTON:} Well, I think cyber security, cyber warfare will be
one of the biggest challenges facing the next president, because clearly
we're facing at this point two different kinds of adversaries. There are
the independent hacking groups that do it mostly for commercial reasons
to try to steal information that they can use to make money.

But increasingly, we are seeing cyber attacks coming from states, organs
of states. The most recent and troubling of these has been Russia.
There's no doubt now that Russia has used cyber attacks against all
kinds of organizations in our country, and I am deeply concerned about
this. I know Donald's very praiseworthy of Vladimir Putin, but Putin is
playing a really...

(CROSSTALK)

\textbf{CLINTON:} ... tough, long game here. And one of the things he's
done is to let loose cyber attackers to hack into government files, to
hack into personal files, hack into the Democratic National Committee.
And we recently have learned that, you know, that this is one of their
preferred methods of trying to wreak havoc and collect information. We
need to make it very clear --- whether it's Russia, China, Iran or
anybody else --- the United States has much greater capacity. And we are
not going to sit idly by and permit state actors to go after our
information, our private-sector information or our public-sector
information.

And we're going to have to make it clear that we don't want to use the
kinds of tools that we have. We don't want to engage in a different kind
of warfare. But we will defend the citizens of this country.

And the Russians need to understand that. I think they've been treating
it as almost a probing, how far would we go, how much would we do. And
that's why I was so --- I was so shocked when Donald publicly invited
Putin to hack into Americans. That is just unacceptable. It's one of the
reasons why 50 national security officials who served in Republican
information --- in administrations...

\textbf{HOLT:} Your two minutes have expired.

\textbf{CLINTON:} ... have said that Donald is unfit to be the
commander- in-chief. It's comments like that that really worry people
who understand the threats that we face.

\textbf{HOLT:} Mr. Trump, you have two minutes and the same question.
Who's behind it? And how do we fight it?

\textbf{TRUMP:} I do want to say that I was just endorsed --- and more
are coming next week --- it will be over 200 admirals, many of them here
--- admirals and generals endorsed me to lead this country. That just
happened, and many more are coming. And I'm very proud of it.

In addition, I was just endorsed by ICE. They've never endorsed anybody
before on immigration. I was just endorsed by ICE. I was just recently
endorsed --- 16,500 Border Patrol agents.

So when Secretary Clinton talks about this, I mean, I'll take the
admirals and I'll take the generals any day over the political hacks
that I see that have led our country so brilliantly over the last 10
years with their knowledge. OK? Because look at the mess that we're in.
Look at the mess that we're in.

As far as the cyber, I agree to parts of what Secretary Clinton said. We
should be better than anybody else, and perhaps we're not. I don't think
anybody knows it was Russia that broke into the DNC. She's saying
Russia, Russia, Russia, but I don't --- maybe it was. I mean, it could
be Russia, but it could also be China. It could also be lots of other
people. It also could be somebody sitting on their bed that weighs 400
pounds, OK?

\textbf{TRUMP:} You don't know who broke in to DNC.

But what did we learn with DNC? We learned that Bernie Sanders was taken
advantage of by your people, by Debbie Wasserman Schultz. Look what
happened to her. But Bernie Sanders was taken advantage of. That's what
we learned.

Now, whether that was Russia, whether that was China, whether it was
another country, we don't know, because the truth is, under President
Obama we've lost control of things that we used to have control over.

We came in with the Internet, we came up with the Internet, and I think
Secretary Clinton and myself would agree very much, when you look at
what ISIS is doing with the Internet, they're beating us at our own
game. ISIS.

So we have to get very, very tough on cyber and cyber warfare. It is ---
it is a huge problem. I have a son. He's 10 years old. He has computers.
He is so good with these computers, it's unbelievable. The security
aspect of cyber is very, very tough. And maybe it's hardly doable.

But I will say, we are not doing the job we should be doing. But that's
true throughout our whole governmental society. We have so many things
that we have to do better, Lester, and certainly cyber is one of them.

\textbf{HOLT:} Secretary Clinton?

\textbf{CLINTON:} Well, I think there are a number of issues that we
should be addressing. I have put forth a plan to defeat ISIS. It does
involve going after them online. I think we need to do much more with
our tech companies to prevent ISIS and their operatives from being able
to use the Internet to radicalize, even direct people in our country and
Europe and elsewhere.

But we also have to intensify our air strikes against ISIS and
eventually support our Arab and Kurdish partners to be able to actually
take out ISIS in Raqqa, end their claim of being a Caliphate.

We're making progress. Our military is assisting in Iraq. And we're
hoping that within the year we'll be able to push ISIS out of Iraq and
then, you know, really squeeze them in Syria.

But we have to be cognizant of the fact that they've had foreign
fighters coming to volunteer for them, foreign money, foreign weapons,
so we have to make this the top priority.

And I would also do everything possible to take out their leadership. I
was involved in a number of efforts to take out Al Qaida leadership when
I was secretary of state, including, of course, taking out bin Laden.
And I think we need to go after Baghdadi, as well, make that one of our
organizing principles. Because we've got to defeat ISIS, and we've got
to do everything we can to disrupt their propaganda efforts online.

\textbf{HOLT:} You mention ISIS, and we think of ISIS certainly as over
there, but there are American citizens who have been inspired to commit
acts of terror on American soil, the latest incident, of course, the
bombings we just saw in New York and New Jersey, the knife attack at a
mall in Minnesota, in the last year, deadly attacks in San Bernardino
and Orlando. I'll ask this to both of you. Tell us specifically how you
would prevent homegrown attacks by American citizens, Mr. Trump?

\textbf{TRUMP:} Well, first I have to say one thing, very important.
Secretary Clinton is talking about taking out ISIS. ``We will take out
ISIS.'' Well, President Obama and Secretary Clinton created a vacuum the
way they got out of Iraq, because they got out --- what, they shouldn't
have been in, but once they got in, the way they got out was a disaster.
And ISIS was formed.

So she talks about taking them out. She's been doing it a long time.
She's been trying to take them out for a long time. But they wouldn't
have even been formed if they left some troops behind, like 10,000 or
maybe something more than that. And then you wouldn't have had them.

Or, as I've been saying for a long time, and I think you'll agree,
because I said it to you once, had we taken the oil --- and we should
have taken the oil --- ISIS would not have been able to form either,
because the oil was their primary source of income. And now they have
the oil all over the place, including the oil --- a lot of the oil in
Libya, which was another one of her disasters.

\textbf{HOLT:} Secretary Clinton?

\textbf{CLINTON:} Well, I hope the fact-checkers are turning up the
volume and really working hard. Donald supported the invasion of Iraq.

\textbf{TRUMP:} Wrong.

\textbf{CLINTON:} That is absolutely proved over and over again.

\textbf{TRUMP:} Wrong. Wrong.

\textbf{CLINTON:} He actually advocated for the actions we took in Libya
and urged that Gadhafi be taken out, after actually doing some business
with him one time.

\textbf{CLINTON:} But the larger point --- and he says this constantly
--- is George W. Bush made the agreement about when American troops
would leave Iraq, not Barack Obama.

And the only way that American troops could have stayed in Iraq is to
get an agreement from the then-Iraqi government that would have
protected our troops, and the Iraqi government would not give that.

But let's talk about the question you asked, Lester. The question you
asked is, what do we do here in the United States? That's the most
important part of this. How do we prevent attacks? How do we protect our
people?

And I think we've got to have an intelligence surge, where we are
looking for every scrap of information. I was so proud of law
enforcement in New York, in Minnesota, in New Jersey. You know, they
responded so quickly, so professionally to the attacks that occurred by
Rahami. And they brought him down. And we may find out more information
because he is still alive, which may prove to be an intelligence
benefit.

So we've got to do everything we can to vacuum up intelligence from
Europe, from the Middle East. That means we've got to work more closely
with our allies, and that's something that Donald has been very
dismissive of.

We're working with NATO, the longest military alliance in the history of
the world, to really turn our attention to terrorism. We're working with
our friends in the Middle East, many of which, as you know, are Muslim
majority nations. Donald has consistently insulted Muslims abroad,
Muslims at home, when we need to be cooperating with Muslim nations and
with the American Muslim community.

They're on the front lines. They can provide information to us that we
might not get anywhere else. They need to have close working cooperation
with law enforcement in these communities, not be alienated and pushed
away as some of Donald's rhetoric, unfortunately, has led to.

\textbf{HOLT:} Mr. Trump...

\textbf{TRUMP:} Well, I have to respond.

\textbf{HOLT:} Please respond. TRUMP: The secretary said very strongly
about working with --- we've been working with them for many years, and
we have the greatest mess anyone's ever seen. You look at the Middle
East, it's a total mess. Under your direction, to a large extent.

But you look at the Middle East, you started the Iran deal, that's
another beauty where you have a country that was ready to fall, I mean,
they were doing so badly. They were choking on the sanctions. And now
they're going to be actually probably a major power at some point pretty
soon, the way they're going.

But when you look at NATO, I was asked on a major show, what do you
think of NATO? And you have to understand, I'm a businessperson. I did
really well. But I have common sense. And I said, well, I'll tell you. I
haven't given lots of thought to NATO. But two things.

Number one, the 28 countries of NATO, many of them aren't paying their
fair share. Number two --- and that bothers me, because we should be
asking --- we're defending them, and they should at least be paying us
what they're supposed to be paying by treaty and contract.

And, number two, I said, and very strongly, NATO could be obsolete,
because --- and I was very strong on this, and it was actually covered
very accurately in the New York Times, which is unusual for the New York
Times, to be honest --- but I said, they do not focus on terror. And I
was very strong. And I said it numerous times.

And about four months ago, I read on the front page of the Wall Street
Journal that NATO is opening up a major terror division. And I think
that's great. And I think we should get --- because we pay approximately
73 percent of the cost of NATO. It's a lot of money to protect other
people. But I'm all for NATO. But I said they have to focus on terror,
also.

And they're going to do that. And that was --- believe me --- I'm sure
I'm not going to get credit for it --- but that was largely because of
what I was saying and my criticism of NATO.

I think we have to get NATO to go into the Middle East with us, in
addition to surrounding nations, and we have to knock the hell out of
ISIS, and we have to do it fast, when ISIS formed in this vacuum created
by Barack Obama and Secretary Clinton. And believe me, you were the ones
that took out the troops. Not only that, you named the day. They
couldn't believe it. They sat back probably and said, I can't believe
it. They said...

\textbf{CLINTON:} Lester, we've covered...

\textbf{TRUMP:} No, wait a minute.

\textbf{CLINTON:} We've covered this ground.

\textbf{TRUMP:} When they formed, when they formed, this is something
that never should have happened. It should have never happened. Now,
you're talking about taking out ISIS. But you were there, and you were
secretary of state when it was a little infant. Now it's in over 30
countries. And you're going to stop them? I don't think so.

\textbf{HOLT:} Mr. Trump, a lot of these are judgment questions. You had
supported the war in Iraq before the invasion. What makes your...

\textbf{TRUMP:} I did not support the war in Iraq.

\textbf{HOLT:} In 2002...

\textbf{TRUMP:} That is a mainstream media nonsense put out by her,
because she --- frankly, I think the best person in her campaign is
mainstream media.

\textbf{HOLT:} My question is, since you supported it...

\textbf{TRUMP:} Just --- would you like to hear...

\textbf{HOLT:} ... why is your --- why is your judgment...

\textbf{TRUMP:} Wait a minute. I was against the war in Iraq. Just so
you put it out.

\textbf{HOLT:} The record shows otherwise, but why --- why was...

\textbf{TRUMP:} The record does not show that.

\textbf{HOLT:} Why was --- is your judgment any...

\textbf{TRUMP:} The record shows that I'm right. When I did an interview
with Howard Stern, very lightly, first time anyone's asked me that, I
said, very lightly, I don't know, maybe, who knows? Essentially. I then
did an interview with Neil Cavuto. We talked about the economy is more
important. I then spoke to Sean Hannity, which everybody refuses to call
Sean Hannity. I had numerous conversations with Sean Hannity at Fox. And
Sean Hannity said --- and he called me the other day --- and I spoke to
him about it --- he said you were totally against the war, because he
was for the war.

\textbf{HOLT:} Why is your judgment better than...

\textbf{TRUMP:} And when he --- excuse me. And that was before the war
started. Sean Hannity said very strongly to me and other people --- he's
willing to say it, but nobody wants to call him. I was against the war.
He said, you used to have fights with me, because Sean was in favor of
the war.

And I understand that side, also, not very much, because we should have
never been there. But nobody called Sean Hannity. And then they did an
article in a major magazine, shortly after the war started. I think in
'04. But they did an article which had me totally against the war in
Iraq.

And one of your compatriots said, you know, whether it was before or
right after, Trump was definitely --- because if you read this article,
there's no doubt. But if somebody --- and I'll ask the press --- if
somebody would call up Sean Hannity, this was before the war started. He
and I used to have arguments about the war. I said, it's a terrible and
a stupid thing. It's going to destabilize the Middle East. And that's
exactly what it's done. It's been a disaster.

\textbf{HOLT:} My reference was to what you had said in 2002, and my
question was...

\textbf{TRUMP:} No, no. You didn't hear what I said.

\textbf{HOLT:} Why is your judgment --- why is your judgment any
different than Mrs. Clinton's judgment?

\textbf{TRUMP:} Well, I have much better judgment than she does. There's
no question about that. I also have a much better temperament than she
has, you know?

(LAUGHTER)

I have a much better --- she spent --- let me tell you --- she spent
hundreds of millions of dollars on an advertising --- you know, they get
Madison Avenue into a room, they put names --- oh, temperament, let's go
after --- I think my strongest asset, maybe by far, is my temperament. I
have a winning temperament. I know how to win. She does not have a...

\textbf{HOLT:} Secretary Clinton?

\textbf{TRUMP:} Wait. The AFL-CIO the other day, behind the blue screen,
I don't know who you were talking to, Secretary Clinton, but you were
totally out of control. I said, there's a person with a temperament
that's got a problem.

\textbf{HOLT:} Secretary Clinton?

\textbf{CLINTON:} Whew, OK.

(LAUGHTER)

Let's talk about two important issues that were briefly mentioned by
Donald, first, NATO. You know, NATO as a military alliance has something
called Article 5, and basically it says this: An attack on one is an
attack on all. And you know the only time it's ever been invoked? After
9/11, when the 28 nations of NATO said that they would go to Afghanistan
with us to fight terrorism, something that they still are doing by our
side.

With respect to Iran, when I became secretary of state, Iran was weeks
away from having enough nuclear material to form a bomb. They had
mastered the nuclear fuel cycle under the Bush administration. They had
built covert facilities. They had stocked them with centrifuges that
were whirling away.

And we had sanctioned them. I voted for every sanction against Iran when
I was in the Senate, but it wasn't enough. So I spent a year-and-a-half
putting together a coalition that included Russia and China to impose
the toughest sanctions on Iran.

And we did drive them to the negotiating table. And my successor, John
Kerry, and President Obama got a deal that put a lid on Iran's nuclear
program without firing a single shot. That's diplomacy. That's
coalition-building. That's working with other nations.

The other day, I saw Donald saying that there were some Iranian sailors
on a ship in the waters off of Iran, and they were taunting American
sailors who were on a nearby ship. He said, you know, if they taunted
our sailors, I'd blow them out of the water and start another war.
That's not good judgment.

\textbf{TRUMP:} That would not start a war.

\textbf{CLINTON:} That is not the right temperament to be commander-in-
chief, to be taunted. And the worst part...

\textbf{TRUMP:} No, they were taunting us.

\textbf{CLINTON:} ... of what we heard Donald say has been about nuclear
weapons. He has said repeatedly that he didn't care if other nations got
nuclear weapons, Japan, South Korea, even Saudi Arabia. It has been the
policy of the United States, Democrats and Republicans, to do everything
we could to reduce the proliferation of nuclear weapons. He even said,
well, you know, if there were nuclear war in East Asia, well, you know,
that's fine...

\textbf{TRUMP:} Wrong.

\textbf{CLINTON:} ... have a good time, folks.

\textbf{TRUMP:} It's lies.

\textbf{CLINTON:} And, in fact, his cavalier attitude about nuclear
weapons is so deeply troubling. That is the number-one threat we face in
the world. And it becomes particularly threatening if terrorists ever
get their hands on any nuclear material. So a man who can be provoked by
a tweet should not have his fingers anywhere near the nuclear codes, as
far as I think anyone with any sense about this should be concerned.

\textbf{TRUMP:} That line's getting a little bit old, I must say. I
would like to...

\textbf{CLINTON:} It's a good one, though. It well describes the
problem.

(LAUGHTER)

\textbf{TRUMP:} It's not an accurate one at all. It's not an accurate
one. So I just want to give a lot of things --- and just to respond. I
agree with her on one thing. The single greatest problem the world has
is nuclear armament, nuclear weapons, not global warming, like you think
and your --- your president thinks.

Nuclear is the single greatest threat. Just to go down the list, we
defend Japan, we defend Germany, we defend South Korea, we defend Saudi
Arabia, we defend countries. They do not pay us. But they should be
paying us, because we are providing tremendous service and we're losing
a fortune. That's why we're losing --- we're losing --- we lose on
everything. I say, who makes these --- we lose on everything. All I
said, that it's very possible that if they don't pay a fair share,
because this isn't 40 years ago where we could do what we're doing. We
can't defend Japan, a behemoth, selling us cars by the million...

\textbf{HOLT:} We need to move on.

\textbf{TRUMP:} Well, wait, but it's very important. All I said was,
they may have to defend themselves or they have to help us out. We're a
country that owes \$20 trillion. They have to help us out.

\textbf{HOLT:} Our last...

\textbf{TRUMP:} As far as the nuclear is concerned, I agree. It is the
single greatest threat that this country has.

\textbf{HOLT:} Which leads to my next question, as we enter our last
segment here (inaudible) the subject of securing America. On nuclear
weapons, President Obama reportedly considered changing the nation's
longstanding policy on first use. Do you support the current policy? Mr.
Trump, you have two minutes on that.

\textbf{TRUMP:} Well, I have to say that, you know, for what Secretary
Clinton was saying about nuclear with Russia, she's very cavalier in the
way she talks about various countries. But Russia has been expanding
their --- they have a much newer capability than we do. We have not been
updating from the new standpoint.

I looked the other night. I was seeing B-52s, they're old enough that
your father, your grandfather could be flying them. We are not --- we
are not keeping up with other countries. I would like everybody to end
it, just get rid of it. But I would certainly not do first strike.

I think that once the nuclear alternative happens, it's over. At the
same time, we have to be prepared. I can't take anything off the table.
Because you look at some of these countries, you look at North Korea,
we're doing nothing there. China should solve that problem for us. China
should go into North Korea. China is totally powerful as it relates to
North Korea.

And by the way, another one powerful is the worst deal I think I've ever
seen negotiated that you started is the Iran deal. Iran is one of their
biggest trading partners. Iran has power over North Korea.

And when they made that horrible deal with Iran, they should have
included the fact that they do something with respect to North Korea.
And they should have done something with respect to Yemen and all these
other places.

And when asked to Secretary Kerry, why didn't you do that? Why didn't
you add other things into the deal? One of the great giveaways of all
time, of all time, including \$400 million in cash. Nobody's ever seen
that before. That turned out to be wrong. It was actually \$1.7 billion
in cash, obviously, I guess for the hostages. It certainly looks that
way.

So you say to yourself, why didn't they make the right deal? This is one
of the worst deals ever made by any country in history. The deal with
Iran will lead to nuclear problems. All they have to do is sit back 10
years, and they don't have to do much.

\textbf{HOLT:} Your two minutes is expired.

\textbf{TRUMP:} And they're going to end up getting nuclear. I met with
Bibi Netanyahu the other day. Believe me, he's not a happy camper.

\textbf{HOLT:} All right. Mrs. Clinton, Secretary Clinton, you have two
minutes.

\textbf{CLINTON:} Well, let me --- let me start by saying, words matter.
Words matter when you run for president. And they really matter when you
are president. And I want to reassure our allies in Japan and South
Korea and elsewhere that we have mutual defense treaties and we will
honor them.

It is essential that America's word be good. And so I know that this
campaign has caused some questioning and worries on the part of many
leaders across the globe. I've talked with a number of them. But I want
to --- on behalf of myself, and I think on behalf of a majority of the
American people, say that, you know, our word is good.

It's also important that we look at the entire global situation. There's
no doubt that we have other problems with Iran. But personally, I'd
rather deal with the other problems having put that lid on their nuclear
program than still to be facing that.

And Donald never tells you what he would do. Would he have started a
war? Would he have bombed Iran? If he's going to criticize a deal that
has been very successful in giving us access to Iranian facilities that
we never had before, then he should tell us what his alternative would
be. But it's like his plan to defeat ISIS. He says it's a secret plan,
but the only secret is that he has no plan.

So we need to be more precise in how we talk about these issues. People
around the word follow our presidential campaigns so closely, trying to
get hints about what we will do. Can they rely on us? Are we going to
lead the world with strength and in accordance with our values? That's
what I intend to do. I intend to be a leader of our country that people
can count on, both here at home and around the world, to make decisions
that will further peace and prosperity, but also stand up to bullies,
whether they're abroad or at home.

We cannot let those who would try to destabilize the world to interfere
with American interests and security...

\textbf{HOLT:} Your two minutes is...

\textbf{CLINTON:} ... to be given any opportunities at all.

\textbf{HOLT:} ... is expired.

\textbf{TRUMP:} Lester, one thing I'd like to say.

\textbf{HOLT:} Very quickly. Twenty seconds.

\textbf{TRUMP:} I will go very quickly. But I will tell you that Hillary
will tell you to go to her website and read all about how to defeat
ISIS, which she could have defeated by never having it, you know, get
going in the first place. Right now, it's getting tougher and tougher to
defeat them, because they're in more and more places, more and more
states, more and more nations.

\textbf{HOLT:} Mr. Trump...

\textbf{TRUMP:} And it's a big problem. And as far as Japan is
concerned, I want to help all of our allies, but we are losing billions
and billions of dollars. We cannot be the policemen of the world. We
cannot protect countries all over the world...

\textbf{HOLT:} We have just...

\textbf{TRUMP:} ... where they're not paying us what we need.

\textbf{HOLT:} We have just a few final questions...

\textbf{TRUMP:} And she doesn't say that, because she's got no business
ability. We need heart. We need a lot of things. But you have to have
some basic ability. And sadly, she doesn't have that. All of the things
that she's talking about could have been taken care of during the last
10 years, let's say, while she had great power. But they weren't taken
care of. And if she ever wins this race, they won't be taken care of.

\textbf{HOLT:} Mr. Trump, this year Secretary Clinton became the first
woman nominated for president by a major party. Earlier this month, you
said she doesn't have, quote, ``a presidential look.'' She's standing
here right now. What did you mean by that?

\textbf{TRUMP:} She doesn't have the look. She doesn't have the stamina.
I said she doesn't have the stamina. And I don't believe she does have
the stamina. To be president of this country, you need tremendous
stamina.

\textbf{HOLT:} The quote was, ``I just don't think she has the
presidential look.''

\textbf{TRUMP:} You have --- wait a minute. Wait a minute, Lester. You
asked me a question. Did you ask me a question?

You have to be able to negotiate our trade deals. You have to be able to
negotiate, that's right, with Japan, with Saudi Arabia. I mean, can you
imagine, we're defending Saudi Arabia? And with all of the money they
have, we're defending them, and they're not paying? All you have to do
is speak to them. Wait. You have so many different things you have to be
able to do, and I don't believe that Hillary has the stamina.

\textbf{HOLT:} Let's let her respond. CLINTON: Well, as soon as he
travels to 112 countries and negotiates a peace deal, a cease-fire, a
release of dissidents, an opening of new opportunities in nations around
the world, or even spends 11 hours testifying in front of a
congressional committee, he can talk to me about stamina.

(APPLAUSE)

\textbf{TRUMP:} The world --- let me tell you. Let me tell you. Hillary
has experience, but it's bad experience. We have made so many bad deals
during the last --- so she's got experience, that I agree.

(APPLAUSE)

But it's bad, bad experience. Whether it's the Iran deal that you're so
in love with, where we gave them \$150 billion back, whether it's the
Iran deal, whether it's anything you can --- name --- you almost can't
name a good deal. I agree. She's got experience, but it's bad
experience. And this country can't afford to have another four years of
that kind of experience.

\textbf{HOLT:} We are at --- we are at the final question.

(APPLAUSE)

\textbf{CLINTON:} Well, one thing. One thing, Lester.

\textbf{HOLT:} Very quickly, because we're at the final question now.

\textbf{CLINTON:} You know, he tried to switch from looks to stamina.
But this is a man who has called women pigs, slobs and dogs, and someone
who has said pregnancy is an inconvenience to employers, who has said...

\textbf{TRUMP:} I never said that.

\textbf{CLINTON:} .... women don't deserve equal pay unless they do as
good a job as men.

\textbf{TRUMP:} I didn't say that.

\textbf{CLINTON:} And one of the worst things he said was about a woman
in a beauty contest. He loves beauty contests, supporting them and
hanging around them. And he called this woman ``Miss Piggy.'' Then he
called her ``Miss Housekeeping,'' because she was Latina. Donald, she
has a name.

\textbf{TRUMP:} Where did you find this? Where did you find this?

\textbf{CLINTON:} Her name is Alicia Machado.

\textbf{TRUMP:} Where did you find this?

\textbf{CLINTON:} And she has become a U.S. citizen, and you can bet...

\textbf{TRUMP:} Oh, really? CLINTON: ... she's going to vote this
November.

\textbf{TRUMP:} OK, good. Let me just tell you...

(APPLAUSE)

\textbf{HOLT:} Mr. Trump, could we just take 10 seconds and then we ask
the final question...

\textbf{TRUMP:} You know, Hillary is hitting me with tremendous
commercials. Some of it's said in entertainment. Some of it's said ---
somebody who's been very vicious to me, Rosie O'Donnell, I said very
tough things to her, and I think everybody would agree that she deserves
it and nobody feels sorry for her.

But you want to know the truth? I was going to say something...

\textbf{HOLT:} Please very quickly.

\textbf{TRUMP:} ... extremely rough to Hillary, to her family, and I
said to myself, ``I can't do it. I just can't do it. It's inappropriate.
It's not nice.'' But she spent hundreds of millions of dollars on
negative ads on me, many of which are absolutely untrue. They're untrue.
And they're misrepresentations.

And I will tell you this, Lester: It's not nice. And I don't deserve
that.

But it's certainly not a nice thing that she's done. It's hundreds of
millions of ads. And the only gratifying thing is, I saw the polls come
in today, and with all of that money...

\textbf{HOLT:} We have to move on to the final question.

\textbf{TRUMP:} ... \$200 million is spent, and I'm either winning or
tied, and I've spent practically nothing.

(APPLAUSE)

\textbf{HOLT:} One of you will not win this election. So my final
question to you tonight, are you willing to accept the outcome as the
will of the voters? Secretary Clinton?

\textbf{CLINTON:} Well, I support our democracy. And sometimes you win,
sometimes you lose. But I certainly will support the outcome of this
election.

And I know Donald's trying very hard to plant doubts about it, but I
hope the people out there understand: This election's really up to you.
It's not about us so much as it is about you and your families and the
kind of country and future you want. So I sure hope you will get out and
vote as though your future depended on it, because I think it does.

\textbf{HOLT:} Mr. Trump, very quickly, same question. Will you accept
the outcome as the will of the voters? TRUMP: I want to make America
great again. We are a nation that is seriously troubled. We're losing
our jobs. People are pouring into our country.

The other day, we were deporting 800 people. And perhaps they passed the
wrong button, they pressed the wrong button, or perhaps worse than that,
it was corruption, but these people that we were going to deport for
good reason ended up becoming citizens. Ended up becoming citizens. And
it was 800. And now it turns out it might be 1,800, and they don't even
know.

\textbf{HOLT:} Will you accept the outcome of the election?

\textbf{TRUMP:} Look, here's the story. I want to make America great
again. I'm going to be able to do it. I don't believe Hillary will. The
answer is, if she wins, I will absolutely support her.

(APPLAUSE)

\textbf{HOLT:} All right. Well, that is going to do it for us. That
concludes our debate for this evening, a spirit one. We covered a lot of
ground, not everything as I suspected we would.

The next presidential debates are scheduled for October 9th at
Washington University in St. Louis and October 19th at the University of
Nevada Las Vegas. The conversation will continue.

A reminder. The vice presidential debate is scheduled for October 4th at
Longwood University in Farmville, Virginia. My thanks to Hillary Clinton
and to Donald Trump and to Hofstra University for hosting us tonight.
Good night, everyone.

END

Advertisement

\protect\hyperlink{after-bottom}{Continue reading the main story}

\hypertarget{site-index}{%
\subsection{Site Index}\label{site-index}}

\hypertarget{site-information-navigation}{%
\subsection{Site Information
Navigation}\label{site-information-navigation}}

\begin{itemize}
\tightlist
\item
  \href{https://help.nytimes3xbfgragh.onion/hc/en-us/articles/115014792127-Copyright-notice}{©~2020~The
  New York Times Company}
\end{itemize}

\begin{itemize}
\tightlist
\item
  \href{https://www.nytco.com/}{NYTCo}
\item
  \href{https://help.nytimes3xbfgragh.onion/hc/en-us/articles/115015385887-Contact-Us}{Contact
  Us}
\item
  \href{https://www.nytco.com/careers/}{Work with us}
\item
  \href{https://nytmediakit.com/}{Advertise}
\item
  \href{http://www.tbrandstudio.com/}{T Brand Studio}
\item
  \href{https://www.nytimes3xbfgragh.onion/privacy/cookie-policy\#how-do-i-manage-trackers}{Your
  Ad Choices}
\item
  \href{https://www.nytimes3xbfgragh.onion/privacy}{Privacy}
\item
  \href{https://help.nytimes3xbfgragh.onion/hc/en-us/articles/115014893428-Terms-of-service}{Terms
  of Service}
\item
  \href{https://help.nytimes3xbfgragh.onion/hc/en-us/articles/115014893968-Terms-of-sale}{Terms
  of Sale}
\item
  \href{https://spiderbites.nytimes3xbfgragh.onion}{Site Map}
\item
  \href{https://help.nytimes3xbfgragh.onion/hc/en-us}{Help}
\item
  \href{https://www.nytimes3xbfgragh.onion/subscription?campaignId=37WXW}{Subscriptions}
\end{itemize}
