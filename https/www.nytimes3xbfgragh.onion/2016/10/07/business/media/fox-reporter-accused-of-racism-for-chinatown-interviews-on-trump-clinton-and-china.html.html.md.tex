Sections

SEARCH

\protect\hyperlink{site-content}{Skip to
content}\protect\hyperlink{site-index}{Skip to site index}

\href{https://www.nytimes3xbfgragh.onion/pages/business/media/index.html}{Media}

\href{https://myaccount.nytimes3xbfgragh.onion/auth/login?response_type=cookie\&client_id=vi}{}

\href{https://www.nytimes3xbfgragh.onion/section/todayspaper}{Today's
Paper}

\href{/pages/business/media/index.html}{Media}\textbar{}Protest Against
Fox Correspondent Accused of Racism for Chinatown Interviews

\url{https://nyti.ms/2dAt6TD}

\begin{itemize}
\item
\item
\item
\item
\item
\end{itemize}

Advertisement

\protect\hyperlink{after-top}{Continue reading the main story}

Supported by

\protect\hyperlink{after-sponsor}{Continue reading the main story}

\hypertarget{protest-against-fox-correspondent-accused-of-racism-for-chinatown-interviews}{%
\section{Protest Against Fox Correspondent Accused of Racism for
Chinatown
Interviews}\label{protest-against-fox-correspondent-accused-of-racism-for-chinatown-interviews}}

\includegraphics{https://static01.graylady3jvrrxbe.onion/images/2016/10/07/us/07xp-watters_web1/07xp-watters_web1-articleInline-v2.jpg?quality=75\&auto=webp\&disable=upscale}

By \href{http://www.nytimes3xbfgragh.onion/by/liam-stack}{Liam Stack}

\begin{itemize}
\item
  Oct. 6, 2016
\item
  \begin{itemize}
  \item
  \item
  \item
  \item
  \item
  \end{itemize}
\end{itemize}

Elected officials and activists
\href{https://twitter.com/petersterne/status/784129892942278658}{staged
a protest outside} the Manhattan headquarters of Fox News on Thursday
over a segment in which a correspondent conducted a series of mocking
interviews of Asian-Americans in New York City's Chinatown that critics
said trafficked in stereotypes and veered into racism.

The correspondent, Jesse Watters, who has been accused of stalking and
harassment for his ambush-style interviews on the street, expressed
``regret'' late Wednesday after provoking a storm of criticism for the
segment that was broadcast on Monday.

\begin{quote}
Photo of the small protest outside Fox News HQ:
\href{https://t.co/MyXXAyEKNc}{pic.twitter.com/MyXXAyEKNc}

--- Peter Sterne🌹 (@petersterne)
\href{https://twitter.com/petersterne/status/784129892942278658?ref_src=twsrc\%5Etfw}{October
6, 2016}
\end{quote}

Mayor Bill de Blasio
\href{https://twitter.com/BilldeBlasio/status/783830241672331264}{called
the segment ``vile.''} And Councilman Peter Koo said in a statement:
``Passing off this blatantly racist television segment as `gentle fun'
not only validates racist stereotypes, it encourages them. The entire
segment smacks of willful ignorance by buying into the perpetual
foreigner syndrome.

``How is it, that in New York City in 2016, this is still O.K.? Short
answer: It's not, and it is unfortunate that Fox News needs to be
reminded of that.''

Fox broadcast
\href{https://www.youtube.com/watch?v=PJmnLzw8NA4\&feature=youtu.be}{the
interviews} as part of ``Watters' World,'' a recurring segment on ``The
O'Reilly Factor,'' the network's top-rated show. The host, Bill
O'Reilly, introduced the piece by saying it had been inspired by how
frequently China was mentioned during the first presidential debate
between Hillary Clinton and Donald J. Trump.

But the nearly five-minute video was interspersed with references to
martial arts and scenes of Mr. Watters getting a foot massage, playing
with nunchucks and asking loaded questions that some residents appeared
not to understand or couldn't answer. Clips from well-known movies were
sprinkled throughout the segment, including ``The Karate Kid'' and
``Chinatown.''

Mr. Watters begins the piece with an instrumental version of the Carl
Douglas song ``Kung Fu Fighting'' playing softly in the background. He
asks two young women, ``Am I supposed to bow to say hello?'' He asks a
street vendor if his wares were stolen: ``I like these watches --- are
they hot?''

When he asks some passers-by their opinion of Mrs. Clinton and Mr.
Trump, the two men answer in accented English, and their answers are
displayed in subtitles at the bottom of the screen.

``Trump has been beating up on China; how does that make you feel?'' he
asks an older woman. He peppers others with questions like ``Is it the
year of the dragon ... rabbit?'' ``Is everything made in China now?''
``Do they call Chinese food in China just food?''

And at one point, when another young woman says she really doesn't want
to vote for Mr. Trump so her choice was Mrs. Clinton, he opines, ``So
China can keep ripping us off.''

The segment provoked an uproar among social media users, and
Asian-American groups denounced it. The Asian-American Journalists
Association said it was ``outraged and shocked'' and demanded an apology
from the network.

``We should be far beyond tired, racist stereotypes and targeting an
ethnic group for humiliation and objectification on the basis of their
race,'' the group said in a
\href{http://www.aaja.org/watters-world/}{statement}. ``Sadly, Fox News
proves it has a long way to go in reporting on communities of color in a
respectful and fair manner.''

The influential blog Angry Asian Man, founded by Phil Yu, a
Korean-American, described the segment in
\href{http://blog.angryasianman.com/2016/10/fox-news-airs-appallingly-racist-anti.html}{a
post} as ``a new low, even for Fox News.''

``Jesse Watters went for a holy-crap-that's-so-racist-man-on-the-street
approach,'' the post said.

State Senator Daniel L. Squadron, whose district includes Chinatown,
\href{https://www.nysenate.gov/newsroom/articles/daniel-l-squadron/squadron-condemns-mocking-chinatown-oreilly-factor-segment}{condemned
the segment} for ``stereotyping, mockery and a thinly veiled disdain for
immigrants.''

In addition to Mr. Koo, Comptroller Scott M. Stringer, Representatives
Grace Meng and Nydia Velasquez, and Assemblymen Ron Kim and Walter T.
Mosley attended the rally on Thursday, according to a statement from the
New York State Black, Puerto Rican, Hispanic and Asian Legislative
Caucus.

Mr. Watters, who responded to his critics on Twitter on Wednesday, said
he considered himself ``a political humorist'' and regretted that he had
upset people. He said his interviews were meant to be taken as a
lighthearted joke.

Mr. Watters and Mr. O'Reilly, however, appeared to anticipate that the
interviews would cause a stir when the segment was broadcast on Monday.

``I know we're going to get letters,'' Mr. O'Reilly said. ``It's
inevitable.'' The Fox host added that he was surprised, considering how
``insulated'' he believed the residents of Chinatown were, that many
seemed to be aware of what was going on politically.

Mr. Watters said one man who had responded negatively to him was ``one
of many'' who ``hated'' him. ``They're such a polite people --- they
won't walk away or tell me to get out of here,'' he said, laughing.

``They're patient, they're patient,'' Mr. O'Reilly replied.

Renee Tajima-Peña, a professor of Asian-American studies at the
University of California, Los Angeles, said the segment captured a
longstanding and distinct feature of anti-Asian sentiment in the United
States.

``They mock the Chinese and Chinese-Americans, yet the backhanded
compliments --- he said these people were so polite,'' Professor
Tajima-Peña said. ``That kind of duality of the perception of Asians has
been there since time immemorial and the beginning of the republic.''

``We are either perpetual foreigners or we are the favored model
minority,'' she added. ``We are a threat or we are docile.''

Mr. Watters has been at the center of controversy before. He became
known for
\href{http://www.nytimes3xbfgragh.onion/2009/04/16/arts/television/16ambush.html}{street
interviews} that sometimes seemed to serve little purpose save for
bothering critics of Fox News or Mr. O'Reilly. In 2009, Amanda Terkel,
then an editor at the liberal website Think Progress,
\href{https://thinkprogress.org/i-was-followed-harassed-and-ambushed-by-bill-oreilly-s-producer-2849ba2cb056\#.82cvh2izm}{wrote}
that she had been ``accosted'' by Mr. Watters while on vacation in a
town two hours from where she lived.

She said she had been ``followed, harassed and ambushed,'' and referred
to him as ``O'Reilly's top hit man.''

That incident reared its head years later, when Mr. Watters found
himself in
\href{https://www.washingtonpost.com/news/reliable-source/wp/2016/05/01/nerdy-fight-breaks-out-at-whcd-afterparty-between-fox-news-and-huffington-post-reporters/}{a
brawl} at the United States Institute of Peace during an after-party for
the annual White House Correspondents Dinner.

The fight began when Ryan Grim, a reporter at The Huffington Post, where
Ms. Terkel is currently employed, tried to film Mr. Watters with an
iPhone. Fisticuffs soon followed.

``Ambush guy can't take getting ambushed,'' Mr. Grim told
\href{https://www.washingtonpost.com/news/reliable-source/wp/2016/05/01/nerdy-fight-breaks-out-at-whcd-afterparty-between-fox-news-and-huffington-post-reporters/}{The
Washington Post}. ``Maybe he should think about his life choices.''

Advertisement

\protect\hyperlink{after-bottom}{Continue reading the main story}

\hypertarget{site-index}{%
\subsection{Site Index}\label{site-index}}

\hypertarget{site-information-navigation}{%
\subsection{Site Information
Navigation}\label{site-information-navigation}}

\begin{itemize}
\tightlist
\item
  \href{https://help.nytimes3xbfgragh.onion/hc/en-us/articles/115014792127-Copyright-notice}{©~2020~The
  New York Times Company}
\end{itemize}

\begin{itemize}
\tightlist
\item
  \href{https://www.nytco.com/}{NYTCo}
\item
  \href{https://help.nytimes3xbfgragh.onion/hc/en-us/articles/115015385887-Contact-Us}{Contact
  Us}
\item
  \href{https://www.nytco.com/careers/}{Work with us}
\item
  \href{https://nytmediakit.com/}{Advertise}
\item
  \href{http://www.tbrandstudio.com/}{T Brand Studio}
\item
  \href{https://www.nytimes3xbfgragh.onion/privacy/cookie-policy\#how-do-i-manage-trackers}{Your
  Ad Choices}
\item
  \href{https://www.nytimes3xbfgragh.onion/privacy}{Privacy}
\item
  \href{https://help.nytimes3xbfgragh.onion/hc/en-us/articles/115014893428-Terms-of-service}{Terms
  of Service}
\item
  \href{https://help.nytimes3xbfgragh.onion/hc/en-us/articles/115014893968-Terms-of-sale}{Terms
  of Sale}
\item
  \href{https://spiderbites.nytimes3xbfgragh.onion}{Site Map}
\item
  \href{https://help.nytimes3xbfgragh.onion/hc/en-us}{Help}
\item
  \href{https://www.nytimes3xbfgragh.onion/subscription?campaignId=37WXW}{Subscriptions}
\end{itemize}
