Sections

SEARCH

\protect\hyperlink{site-content}{Skip to
content}\protect\hyperlink{site-index}{Skip to site index}

\href{https://www.nytimes3xbfgragh.onion/section/world}{World}

\href{https://myaccount.nytimes3xbfgragh.onion/auth/login?response_type=cookie\&client_id=vi}{}

\href{https://www.nytimes3xbfgragh.onion/section/todayspaper}{Today's
Paper}

\href{/section/world}{World}\textbar{}Rise of Saudi Prince Shatters
Decades of Royal Tradition

\url{https://nyti.ms/2e6PDHT}

\begin{itemize}
\item
\item
\item
\item
\item
\item
\end{itemize}

Advertisement

\protect\hyperlink{after-top}{Continue reading the main story}

Supported by

\protect\hyperlink{after-sponsor}{Continue reading the main story}

\hypertarget{rise-of-saudi-prince-shatters-decades-of-royal-tradition}{%
\section{Rise of Saudi Prince Shatters Decades of Royal
Tradition}\label{rise-of-saudi-prince-shatters-decades-of-royal-tradition}}

\includegraphics{https://static01.graylady3jvrrxbe.onion/images/2016/10/16/world/JP-SAUDIPRINCE2/JP-SAUDIPRINCE2-articleInline.jpg?quality=75\&auto=webp\&disable=upscale}

By \href{http://www.nytimes3xbfgragh.onion/by/mark-mazzetti}{Mark
Mazzetti} and
\href{http://www.nytimes3xbfgragh.onion/by/ben-hubbard}{Ben Hubbard}

\begin{itemize}
\item
  Oct. 15, 2016
\item
  \begin{itemize}
  \item
  \item
  \item
  \item
  \item
  \item
  \end{itemize}
\end{itemize}

He has slashed the state budget, frozen government contracts and reduced
the pay of civil employees, all part of drastic austerity measures as
the Kingdom of Saudi Arabia is buffeted by low oil prices.

But last year, Mohammed bin Salman, Saudi Arabia's deputy crown prince,
saw a yacht he couldn't resist.

While vacationing in the south of France, Prince bin Salman spotted a
440-foot yacht floating off the coast. He dispatched an aide to buy the
ship, the Serene, which was owned by Yuri Shefler, a Russian vodka
tycoon. The deal was done within hours, at a price of approximately 500
million euros (roughly \$550 million today), according to an associate
of Mr. Shefler and a Saudi close to the royal family. The Russian moved
off the yacht the same day.

It is the paradox of the brash, 31-year-old Prince bin Salman: a man who
is trying to overturn tradition, reinvent the economy and consolidate
power --- while holding tight to his royal privilege. In less than two
years, he has emerged as the most dynamic royal in the Arab world's
wealthiest nation, setting up a potential rivalry for the throne.

He has a hand in nearly all elements of Saudi policy --- from
\href{http://www.nytimes3xbfgragh.onion/2016/08/25/world/middleeast/yemen-saudi-arabia-hospital-bombing.html}{a
war in Yemen} that has cost the kingdom billions of dollars and led to
international criticism over civilian deaths, to a push domestically to
\href{http://www.nytimes3xbfgragh.onion/2015/12/29/world/middleeast/squeezed-by-low-oil-prices-saudi-arabia-cuts-spending-to-shrink-deficit.html}{restrain
Saudi Arabia's free-spending habits} and to break
\href{http://www.nytimes3xbfgragh.onion/2016/04/26/world/middleeast/saudi-prince-shares-planto-cut-oil-dependency-and-energize-the-economy.html}{its
``addiction'' to oil}. He has begun to loosen social
\href{http://www.nytimes3xbfgragh.onion/2015/05/23/world/middleeast/saudi-arabia-youths-cellphone-apps-freedom.html}{restrictions
that grate on young people}.

The
\href{http://www.nytimes3xbfgragh.onion/2015/06/07/world/middleeast/surprising-saudi-rises-as-a-prince-among-princes.html}{rise
of Prince bin Salman} has shattered decades of tradition in the royal
family, where respect for seniority and power-sharing among branches are
time-honored traditions. Never before in Saudi history has so much power
been wielded by the deputy crown prince, who is second in line to the
throne. That centralization of authority has angered many of his
relatives.

His seemingly boundless ambitions have led many Saudis and foreign
officials to suspect that his ultimate goal is not just to transform the
kingdom, but also to shove aside the current crown prince, his
57-year-old cousin, Mohammed bin Nayef, to become the next king. Such a
move could further upset his relatives and --- if successful --- give
the country what it has never seen: a young king who could rule the
kingdom for many decades.

Crown Prince bin Nayef, the interior minister and longtime
counterterrorism czar, has deep ties to Washington and the support of
many of the older royals. Deciphering the dynamics of the family can be
like trying to navigate a hall of mirrors, but many Saudi and American
officials say Prince bin Salman has made moves aimed at reaching into
Prince bin Nayef's portfolios and weakening him.

This has left officials in Washington hedging their bets by building
relationships with both men, unsure who will end up on top. The White
House got an early sign of the ascent of the young prince in late 2015,
when --- breaking protocol --- Prince bin Salman delivered a soliloquy
about the failures of American foreign policy during a meeting between
his father, King Salman, and President Obama.

Many young Saudis admire him as an energetic representative of their
generation who has addressed some of the country's problems with
uncommon bluntness. The kingdom's news media have built his image as a
hardworking, businesslike leader less concerned than his predecessors
with the trappings of royalty.

Others see him as a power-hungry upstart who is risking instability by
changing too much, too fast.

Months of interviews with Saudi and American officials, members of the
royal family and their associates, and diplomats focused on Saudi
affairs reveal a portrait of a prince in a hurry to prove that he can
transform Saudi Arabia. Prince bin Salman declined multiple interview
requests for this article.

But the question many raise --- and cannot yet answer --- is whether the
energetic leader will succeed in charting a new path for the kingdom, or
whether his impulsiveness and inexperience will destabilize the Arab
world's largest economy at a time of turbulence in the Middle East.

\includegraphics{https://static01.graylady3jvrrxbe.onion/images/2016/10/16/world/16SAUDIPRINCE/16SAUDIPRINCE-articleInline-v2.jpg?quality=75\&auto=webp\&disable=upscale}

\hypertarget{tension-at-the-top}{%
\subsection{Tension at the Top}\label{tension-at-the-top}}

Early this year, Crown Prince bin Nayef left the kingdom for his
family's villa in Algeria, a sprawling compound an hour's drive north of
Algiers. Although he has long taken annual hunting vacations there, many
who know him said that this year was different. He stayed away for
weeks, largely incommunicado and often refusing to respond to messages
from Saudi officials and close associates in Washington. Even John O.
Brennan, the C.I.A. director, whom he has known for decades, had
difficulty reaching him.

The crown prince has diabetes, and suffers from the lingering effects of
\href{http://www.nytimes3xbfgragh.onion/2009/08/29/world/middleeast/29saudi.html}{an
assassination attempt} in 2009 by a jihadist who detonated a bomb he had
hidden in his rectum.

But his lengthy absence at a time of low oil prices, turmoil in the
Middle East and a foundering Saudi-led war in Yemen led several American
officials to conclude that the crown prince was fleeing frictions with
his younger cousin and that the prince was worried his chance to ascend
the throne was in jeopardy.

Since King Salman
\href{http://www.nytimes3xbfgragh.onion/2015/01/23/world/middleeast/salman-ascends-throne-to-become-saudi-king.html}{ascended
to the throne in January 2015}, new powers had been flowing to his son,
some of them undermining the authority of the crown prince. King Salman
collapsed the crown prince's court into his own, giving Prince bin
Salman control over access to the king. Prince bin Salman also hastily
announced the formation of a military alliance of Islamic countries to
fight terrorism. Counterterrorism had long been the domain of Prince bin
Nayef, but the new plan gave no role to him or his powerful Interior
Ministry.

The exact personal relationship between the two men is unclear, fueling
discussion in Saudi Arabia and in foreign capitals about who is
ascendant. Obscuring the picture are the stark differences in the men's
public profiles. Prince bin Nayef has largely stayed in the shadows,
although he did visit New York last month to address the United Nations
General Assembly before heading to Turkey for a state visit.

His younger cousin, meanwhile, has worked to remain in the spotlight,
touring world capitals, speaking with foreign journalists, being
\href{https://twitter.com/Bandaralgaloud/status/745653648563249152}{photographed}
with the Facebook chairman Mark Zuckerberg and presenting himself as a
face of a new Saudi Arabia.

``There is no topic that is more important than succession matters,
especially now,'' said Joseph A. Kechichian, a senior fellow at the King
Faisal Center for Research and Islamic Studies in Riyadh, who has
extensive contacts in the Saudi royal family. ``This matters for
monarchy, for the regional allies and for the kingdom's international
partners.''

Among the most concrete initiatives so far of Prince bin Salman, who
serves as minister of defense, is
\href{http://www.nytimes3xbfgragh.onion/2016/08/10/world/middleeast/yemen-sana-airstrikes.html}{the
Saudi-led war in Yemen}, which since it was begun last year has failed
to dislodge the Shiite Houthi rebels and their allies from the Yemeni
capital. The war has driven much of Yemen toward famine and killed
thousands of civilians while costing the Saudi government tens of
billions of dollars.

Image

Saudi troops along the country's border with Yemen last year. The war in
Yemen has cost the kingdom billions and led to international
criticism.Credit...Tomas Munita for The New York Times

The prosecution of the war by a prince with no military experience has
exacerbated tensions between him and his older cousins, according to
American officials and members of the royal family. Three of Saudi
Arabia's main security services are run by princes. Although all agreed
that the kingdom had to respond when the Houthis seized the Yemeni
capital and forced the government into exile, Prince bin Salman took the
lead, launching the war in March 2015 without full coordination across
the security services.

The head of the National Guard, Prince Mutaib bin Abdullah, had not been
informed and was out of the country when the first strikes were carried
out, according to a senior National Guard officer.

The National Guard is now holding much of the Yemeni border.

American officials, too, were put off when, just as the Yemen campaign
was escalating, Prince bin Salman took a vacation in the Maldives, the
island archipelago off the coast of India. Several American officials
said Defense Secretary Ashton B. Carter had trouble reaching him for
days during one part of the trip.

The prolonged war has also heightened tensions between Prince bin Salman
and Prince bin Nayef, who won the respect of Saudis and American
officials for dismantling Al Qaeda in the kingdom nearly a decade ago
and now sees it taking advantage of chaos in Yemen, according to several
American officials and analysts.

``If Mohammed bin Nayef wanted to be seen as a big supporter of this
war, he's had a year and a half to do it,'' said Bruce Riedel, a former
Middle East analyst at the C.I.A. and a fellow at the Brookings
Institution.

Near the start of the war, Prince bin Salman was a forceful public
advocate for the campaign and was often photographed visiting troops and
meeting with military leaders. But as the campaign has stalemated, such
appearances have grown rare.

The war underlines the plans of Prince bin Salman for a brawny foreign
policy for the kingdom, one less reliant on Western powers like the
United States for its security. He has criticized the thawing of
America's relations with Iran and comments by Mr. Obama during
\href{http://www.theatlantic.com/magazine/archive/2016/04/the-obama-doctrine/471525/}{an
interview} this year that Saudi Arabia must ``share the neighborhood''
with Iran.

This is part of what analysts say is Prince bin Salman's attempt to
foster a sense of Saudi national identity that has not existed since the
kingdom's founding in 1932.

``There has been a surge of Saudi nationalism since the campaign in
Yemen began, with the sense that Saudi Arabia is taking independent
collective action,'' said Andrew Bowen, a Saudi expert at the Wilson
Center in Washington.

Still, Mr. Bowen said support among younger Saudis could diminish the
longer the conflict dragged on. Diplomats say the death toll for Saudi
troops is higher than the government has publicly acknowledged, and a
recent deadly airstrike on a funeral in the Yemeni capital has renewed
calls by human rights groups and some American lawmakers to block or
delay weapons sales to the kingdom.

People who have met Prince bin Salman said he insisted that Saudi Arabia
must be more assertive in shaping events in the Middle East and
confronting Iran's influence in the region --- whether in Yemen, Syria,
Iraq or Lebanon.

Brian Katulis, a Middle East expert at the Center for American Progress
in Washington, who met the prince this year in Riyadh, said his agenda
was clear.

``His main message is that Saudi Arabia is a force to be reckoned
with,'' Mr. Katulis said.

Image

Prince bin Salman at a news conference in April for Vision 2030, his
plan to transform Saudi life by diversifying its economy away from oil,
increasing Saudi employment and improving education, health and other
government services.Credit...Fayex Nureldine/Agence France-Presse ---
Getty Images

\hypertarget{a-swift-ascent}{%
\subsection{A Swift Ascent}\label{a-swift-ascent}}

Saudi Arabia is one of the world's few remaining absolute monarchies,
which means that Prince bin Salman was given all of his powers by a vote
of one: his own father.

The prince's rise began in early 2015, after
\href{http://www.nytimes3xbfgragh.onion/2015/01/23/world/middleeast/king-abdullah-who-nudged-saudi-arabia-forward-dies-at-90.html}{King
Abdullah died} of lung cancer and King Salman ascended to the throne. In
a series of royal decrees, the new king restructured the government and
shook up the order of succession in the royal family in ways that
invested tremendous power in his son.

He was named defense minister and head of a powerful new council to
oversee the Saudi economy as well as put in charge of the governing body
of
\href{http://www.nytimes3xbfgragh.onion/2016/01/09/business/dealbook/saudi-aramco-ipo-prospect-reflects-kingdom-looking-beyond-oil.html}{Saudi
Aramco, the state oil company} and the primary engine of the Saudi
economy.

More important, the king decreed a new order of succession, overturning
the wishes of King Abdullah and replacing his designated crown prince,
Muqrin bin Abdulaziz, with Prince bin Nayef.

While all previous Saudi kings and crown princes had been sons of the
kingdom's founder, Prince bin Nayef was the first of the founder's
grandsons to be put in line. Many hailed the move because of the
prince's success at fighting Al Qaeda and because he has only daughters,
leading many to hope he would choose a successor based on merit rather
than paternity.

The bigger surprise was that the king named Prince bin Salman deputy
crown prince. He was 29 years old at the time and virtually unknown to
the kingdom's closest allies.

This effectively scrapped the political aspirations of his older
relatives, many of whom had decades of experience in public life and in
key sectors like defense and oil policy. Some are still angry ---
although only in private, out of deference to the 80-year-old king.

Since then, Prince bin Salman has moved quickly to build his public
profile and market himself to other nations as the point man for the
kingdom.

Domestically, his focus has been on an ambitious plan for the future of
the kingdom, called \href{http://vision2030.gov.sa/en}{Vision 2030}. The
plan, released in April, seeks to transform Saudi life by diversifying
its economy
\href{http://www.nytimes3xbfgragh.onion/2016/04/26/world/middleeast/saudi-prince-shares-planto-cut-oil-dependency-and-energize-the-economy.html}{away
from oil}, increasing Saudi employment and improving education, health
and other government services. A National Transformation Plan, laying
out targets for improving government ministries, came shortly after.

Read in one way, the documents are an ambitious blueprint to change the
Saudi way of life. Read in another, they are a scathing indictment of
how poorly the kingdom has been run by Prince bin Salman's elders.

Official government development plans going back decades have called for
reducing the dependence on oil and increasing Saudi employment --- to
little effect. And in calling for transparency and accountability, the
plan acknowledges that both have been in short supply. Diplomats and
economists say much about the Saudi economy remains opaque, including
the cost of generous perks and stipends for members of the royal family.

The need for change is greater now, with global oil prices less than
half of what they were in 2014 and hundreds of thousands of young Saudis
entering the job market yearly. Prince bin Salman has called for a new
era of fiscal responsibility, and over the last year, fuel, water and
electricity prices have gone up while the take-home pay of some public
sector employees
\href{https://www.nytimes3xbfgragh.onion/2016/09/27/world/middleeast/saudi-arabia-cuts-salaries-oil.html}{has
been cut} --- squeezing the budgets of average Saudis. He has also said
the government will sell shares of Saudi Aramco, believed to be the
world's most valuable company.

Many Saudis say his age and ambition are benefits at a time when old
ways of thinking must be changed.

``He is speaking in the language of the youth,'' said Hoda al-Helaissi,
a member of the kingdom's advisory Shura Council, which is appointed by
the king. ``The country for too long has been looking through the lenses
of the older generation, and we need to look at who is going to carry
the torch to the next generation.''

Some of his initiatives have appeared ham-handed. In December, he held
his first news conference to announce the formation of
\href{http://www.reuters.com/article/us-saudi-security-idUSKBN0TX2PG20151215}{a
military alliance of Islamic countries} to fight terrorism. But a number
of countries that he said were involved soon responded that they knew
nothing about it or were still waiting for information before deciding
whether to join.

Others have been popular. After Prince bin Salman called for more
entertainment options for families and young people, who often flee the
country on their vacations, the cabinet passed regulations restricting
the powers of the religious police. An Entertainment Authority he
established has planned its first activities, which include comedy
shows, pro wrestling events and monster truck rallies.

Image

The Serene, a 440-foot yacht Prince Mohammed bin Salman spotted while
vacationing last year. He dispatched an aide to buy it; the deal was
done within hours, at a price of about 500 million euros (roughly \$550
million today).Credit...Phil Walter/Getty Images

The prince has kept his distance from the Council of Senior Scholars,
the mostly elderly clerics who set official religious policy and often
release religious opinions that
\href{http://www.nytimes3xbfgragh.onion/2016/01/22/world/middleeast/saudi-arabias-top-cleric-forbids-chess-but-players-maneuver.html}{young
Saudis mock} as being out of touch with modern life.

Instead, he has sought the favor of younger clerics who boast millions
of followers on social media. After the release of Vision 2030, Prince
bin Salman held a reception for Saudi journalists and academics that
included a number of younger, tech-savvy clerics who have gone forth to
praise the plan.

Prince bin Salman's prominence today was difficult to predict during his
early years, spent largely below the radar of Western officials who keep
track of young Saudi royals who might one day rule the kingdom.

Several of King Salman's other sons, who studied overseas to perfect
foreign languages and earn advanced degrees, built impressive résumés.
One became the first Arab astronaut, another a deputy oil minister, yet
another the governor of Medina Province.

Prince bin Salman stayed in Saudi Arabia and does not speak fluent
English, although he appears to understand it. After a private school
education, he studied law at King Saud University in Riyadh, reportedly
graduating fourth in his class. Another prince of the same generation
said he had gotten to know him during high school, when one of their
uncles hosted regular dinners for the younger princes at his palace. He
recalled Prince bin Salman being one of the crowd, saying he liked to
play bridge and admired Margaret Thatcher.

King Salman is said to see himself in his favorite son, the latest in
the lineage of a family that has ruled most of the Arabian Peninsula for
eight decades.

In 2007, when the United States ambassador dropped in on King Salman,
then a prince and the governor of Riyadh Province, to say farewell at
the end of his posting, the governor asked for help circumventing
America's stringent visa procedures. His wife could not get a visa to
see her doctor, and although his other children were willing to submit
to the visa hurdles, ``his son, Prince Mohammed, refused to go to the
U.S. Embassy to be fingerprinted `like some criminal,''' according to a
\href{https://wikileaks.org/plusd/cables/07RIYADH651_a.html}{State
Department cable} at the time.

Prince bin Salman graduated from the university that year and continued
to work for his father, who was named defense minister in 2011, while
dabbling in real estate and business.

Many members of the royal family remain wary of the young prince's
projects and ultimate ambitions. Some mock him as the ``Prince of the
Vision'' and complain about his army of well-paid foreign consultants
and image-makers.

Other are annoyed by the media cell he created inside the royal court to
promote his initiatives, both foreign and domestic. Called the Center
for Studies and Media Affairs, the group has focused on promoting a
positive story about the Yemen war in Washington and has hired numerous
Washington lobbying and public affairs firms to assist in the effort.

Inside the kingdom, the government has largely succeeded in keeping
criticism --- and even open discussion --- of the prince and his
projects out of the public sphere. His family holds sway over the parent
company of many Saudi newspapers, which have breathlessly covered his
initiatives, and prominent Saudi editors and journalists who have
accompanied him on foreign trips have been given up to \$100,000 in
cash, according to two people who have traveled with the prince's
delegation.

Meanwhile, Saudi journalists deemed too critical have been quietly
silenced through phone calls informing them that they are barred from
publishing, and sometimes from traveling abroad.

In June, a Saudi journalist, Sultan al-Saad al-Qahtani, published
\href{http://riyadhpost.live/7848}{an article} in Arabic on his website,
The Riyadh Post, in which he addressed the lack of discussion about
Prince bin Salman's rise.

``You can buy tens of newspapers and hundreds of journalists, but you
can't buy the history that will be written about you,'' he wrote.

He said that the prince's popularity among Saudis was based on a
``sweeping desire for great change'' and that they loved him based on
the hope that he would ``turn their dreams into reality.''

In that lay the risk, Mr. Qahtani wrote: ``If you fail, this love
withers quickly, as if it never existed, and is replaced by a deep
feeling of frustration and hatred.''

The site was blocked the next day, Mr. Qahtani said, for the third time
in 13 months. (It is now back up, at \href{http://riyadhpost.live/}{a
new address}.)

Image

President Obama welcoming Prince Mohammed bin Nayef, center, and Prince
bin Salman to the White House in May 2015. Officials in Washington have
been hedging their bets by building relationships with both men, unsure
who will end up on top.Credit...Chip Somodevilla/Getty Images

\hypertarget{the-future}{%
\subsection{The Future}\label{the-future}}

As sweeping and long-term as Prince bin Salman's initiatives are, they
may hang by the tenuous thread of his link to his father, who has memory
lapses, according to foreign officials who have met with him. Even the
prince's supporters acknowledge that they are not sure he will retain
his current roles after his father dies.

In the meantime, he is racing against time to establish his reputation
and cement his place in the kingdom's power structure.

His fast ascent, and his well-publicized foreign trips to Washington,
Europe, the Middle East and elsewhere in Asia, have led senior Obama
administration officials to consider the prospect that he could step
over Prince bin Nayef and become Saudi Arabia's next king.

This has led to a balancing act for American officials who want to build
a relationship with him while not being used as leverage in any rivalry
with Prince bin Nayef. Obama administration officials say relations with
Prince bin Salman have generally improved, but only after a rocky start
when he would routinely lecture senior Americans --- even the president.

In November, during a Group of 20 summit meeting at a luxury resort on
the Turkish coast, Prince bin Salman gave what American officials
described as a lengthy speech about what he saw as the failure of
American foreign policy in the Middle East --- from the Obama
administration's restraint in Syria to its efforts to improve relations
with Iran, Saudi Arabia's bitter enemy.

Personal relationships have long been the bedrock of American-Saudi
relations, yet the Obama administration has struggled to find someone to
develop a rapport with the prince. The job has largely fallen to
Secretary of State John Kerry, who has hosted the prince several times
at his home in Georgetown. In June, the two men shared an iftar dinner,
breaking the Ramadan fast. In September 2015, dinner at Mr. Kerry's
house ended with Prince bin Salman playing Beethoven on the piano for
the secretary of state and the other guests.

In May, the prince invited Mr. Kerry for a meeting on the Serene*,* the
luxury yacht he bought from the Russian billionaire.

His desire to reimagine the Saudi state is reflected in his admiration
--- some even call it envy --- for the kingdom's more modern and
progressive neighbor in the Persian Gulf, the United Arab Emirates.

He has influential supporters in this effort, particularly the crown
prince of Abu Dhabi, Sheikh Mohammed bin Zayed Al Nahyan, who for more
than a year has been promoting Prince bin Salman in the Middle East and
in Washington.

Crown Prince bin Zayed, the United Arab Emirates' de facto ruler, is a
favorite among Obama administration officials, who view him as a
reliable ally and a respected voice in the Sunni world. But he also has
a history of personal antipathy toward Prince bin Nayef, adding a
particular urgency to his support for the chief rival of the Saudi crown
prince.

In April of last year, Mr. Obama's national security adviser, Susan E.
Rice, led a small delegation of top White House officials to visit
Prince bin Zayed at his home in McLean, Va. During the meeting,
according to several officials who attended, the prince urged the
Americans to develop a relationship with Prince bin Salman.

But all questions about Prince bin Salman's future are likely to depend
on how long his father lives, according to diplomats who track Saudi
Arabia.

If he died soon, Prince bin Nayef would become king and could dismiss
his younger cousin as a gesture to his fellow royals. In fact, it was
King Salman who set the precedent for such moves by dismissing the crown
prince named by his predecessor.

``If the king's health starts to deteriorate, Mohammed bin Salman is
very likely to try to get Mohammed bin Nayef out of the picture,'' said
Mr. Riedel, the former C.I.A. analyst.

But the longer King Salman reigns, foreign officials said, the longer
the young prince has to consolidate his power --- or to convince Prince
bin Nayef that he is worth keeping around if Prince bin Nayef becomes
king.

Most Saudi watchers do not expect any struggles within the family to
spill into the open, as all the royals understand how much they have to
lose from such fissures becoming public or destabilizing their grip on
the kingdom.

``I am persuaded as someone who focuses on this topic that the ruling
family of Saudi Arabia above all else puts the interest of the family
first and foremost,'' said Mr. Kechichian, the analyst who knows many
royals.

``Not a single member of the family will do anything to hurt the
family.''

Advertisement

\protect\hyperlink{after-bottom}{Continue reading the main story}

\hypertarget{site-index}{%
\subsection{Site Index}\label{site-index}}

\hypertarget{site-information-navigation}{%
\subsection{Site Information
Navigation}\label{site-information-navigation}}

\begin{itemize}
\tightlist
\item
  \href{https://help.nytimes3xbfgragh.onion/hc/en-us/articles/115014792127-Copyright-notice}{©~2020~The
  New York Times Company}
\end{itemize}

\begin{itemize}
\tightlist
\item
  \href{https://www.nytco.com/}{NYTCo}
\item
  \href{https://help.nytimes3xbfgragh.onion/hc/en-us/articles/115015385887-Contact-Us}{Contact
  Us}
\item
  \href{https://www.nytco.com/careers/}{Work with us}
\item
  \href{https://nytmediakit.com/}{Advertise}
\item
  \href{http://www.tbrandstudio.com/}{T Brand Studio}
\item
  \href{https://www.nytimes3xbfgragh.onion/privacy/cookie-policy\#how-do-i-manage-trackers}{Your
  Ad Choices}
\item
  \href{https://www.nytimes3xbfgragh.onion/privacy}{Privacy}
\item
  \href{https://help.nytimes3xbfgragh.onion/hc/en-us/articles/115014893428-Terms-of-service}{Terms
  of Service}
\item
  \href{https://help.nytimes3xbfgragh.onion/hc/en-us/articles/115014893968-Terms-of-sale}{Terms
  of Sale}
\item
  \href{https://spiderbites.nytimes3xbfgragh.onion}{Site Map}
\item
  \href{https://help.nytimes3xbfgragh.onion/hc/en-us}{Help}
\item
  \href{https://www.nytimes3xbfgragh.onion/subscription?campaignId=37WXW}{Subscriptions}
\end{itemize}
