Sections

SEARCH

\protect\hyperlink{site-content}{Skip to
content}\protect\hyperlink{site-index}{Skip to site index}

\href{https://www.nytimes3xbfgragh.onion/section/world/europe}{Europe}

\href{https://myaccount.nytimes3xbfgragh.onion/auth/login?response_type=cookie\&client_id=vi}{}

\href{https://www.nytimes3xbfgragh.onion/section/todayspaper}{Today's
Paper}

\href{/section/world/europe}{Europe}\textbar{}Ecuador Cuts Internet of
Julian Assange, WikiLeaks' Founder

\url{https://nyti.ms/2eh4aRf}

\begin{itemize}
\item
\item
\item
\item
\item
\item
\end{itemize}

Advertisement

\protect\hyperlink{after-top}{Continue reading the main story}

Supported by

\protect\hyperlink{after-sponsor}{Continue reading the main story}

\hypertarget{ecuador-cuts-internet-of-julian-assange-wikileaks-founder}{%
\section{Ecuador Cuts Internet of Julian Assange, WikiLeaks'
Founder}\label{ecuador-cuts-internet-of-julian-assange-wikileaks-founder}}

\includegraphics{https://static01.graylady3jvrrxbe.onion/images/2016/10/19/world/19Assange-web/19Assange-web-articleLarge.jpg?quality=75\&auto=webp\&disable=upscale}

By \href{http://www.nytimes3xbfgragh.onion/by/steven-erlanger}{Steven
Erlanger} and
\href{http://www.nytimes3xbfgragh.onion/by/david-e-sanger}{David E.
Sanger}

\begin{itemize}
\item
  Oct. 18, 2016
\item
  \begin{itemize}
  \item
  \item
  \item
  \item
  \item
  \item
  \end{itemize}
\end{itemize}

LONDON --- Ecuador said Tuesday that it had cut off
\href{http://www.nytimes3xbfgragh.onion/topic/person/julian-assange?8qa}{Julian
Assange}'s access to the internet in his exile in the country's London
embassy, making clear that it feared being sucked into an effort to
``interfere in electoral processes'' in the United States by the
activities of the
\href{http://www.nytimes3xbfgragh.onion/topic/organization/wikileaks?8qa}{WikiLeaks}
founder.

Ecuador said that it was not evicting Mr. Assange from its embassy,
where he sought asylum four years ago. It said that its ``temporary
restriction'' of internet services to Mr. Assange ``does not prevent the
WikiLeaks organization from carrying out its journalistic activities.''

But it was clearly intended to keep the embassy from being the control
center for that leaking operation. ``The government of Ecuador respects
the principle of nonintervention in the affairs of other countries,'' it
said in a statement, ``and it does not interfere in the electoral
processes in support of any candidate in particular.''

The internet cutoff was the latest twist in the odd tale of Mr.
Assange's self-imposed exile, which began in 2012 when he sought refuge
from a Swedish rape investigation that he said was a cover for an
American effort to extradite him. Since then, his world has shrunk to a
single apartment inside the small diplomatic compound in central London.
He has communicated through the embassy's internet connections, visitors
and, presumably, cellphones that would give him another form of internet
access.

Ecuador's decision was the first sign that the government in Quito was
beginning to wonder if its guest in London was overstaying his welcome.

It doubtless was considering the possibility that, should Hillary
Clinton prevail in the United States election next month, it would have
to explain its role as host to the man who, by remote control, appears
to have coordinated the publication of emails purloined from people
close to Mrs. Clinton, along with those of the Democratic National
Committee and other organizations.

The announcement came a day after WikiLeaks said that Mr. Assange's
connection to the internet had been severed shortly after the
organization published
\href{http://www.nytimes3xbfgragh.onion/2016/10/16/us/politics/wikileaks-hack-hillary-clinton-emails.html}{speeches
that Hillary Clinton gave to Goldman Sachs}, the global investment firm.
The transcripts, the latest in a series of disclosures, appear to have
come from the hacked email account of John D. Podesta, the chairman of
her campaign and a White House chief of staff when Mrs. Clinton's
husband was president.

The statement clearly sought to separate Ecuador from the decision by
WikiLeaks to publish Mr. Podesta's emails and, before that, those hacked
from the national committee and elsewhere. In recent weeks, Mr. Assange,
once the hero of the American left for exposing classified State
Department and Pentagon documents,
\href{http://www.nytimes3xbfgragh.onion/2016/10/13/us/politics/wikileaks-hillary-clinton-emails.html}{has
been hailed by Donald J. Trump and his advisers for disclosures from
Mrs. Clinton's campaign}, which Mr. Trump has used almost daily to fuel
his attacks on her.

\href{https://www.nytimes3xbfgragh.onion/interactive/2019/world/julian-assange-wikileaks.html}{}

\includegraphics{https://static01.graylady3jvrrxbe.onion/images/2016/02/05/world/05ASSANGE-web2/05ASSANGE-web2-videoLarge.jpg}

\hypertarget{how-julian-assange-and-wikileaks-became-targets-of-the-us-government}{%
\subsection{How Julian Assange and WikiLeaks Became Targets of the U.S.
Government}\label{how-julian-assange-and-wikileaks-became-targets-of-the-us-government}}

Here are key points in his case since WikiLeaks burst onto the scene in
2010.

American intelligence agencies have said that the D.N.C. hack was the
work of the Russian government and had to be approved at the highest
levels of the Kremlin. But it is unclear how the documents made it to
WikiLeaks, which has never said where the emails came from, if it knows.

Only hours before Ecuador's announcement, WikiLeaks charged that
Secretary of State John Kerry quietly urged the Ecuadorean government,
in a meeting late last month, to stop Mr. Assange from publishing the
emails or interfering in the election. The State Department issued a
statement declaring that the reports were untrue.

Ecuador's action, experts inside and outside the United States
government say, is not likely to slow the flow of leaked emails. Those
emails are routed through servers around the globe, and if the United
States wanted to shut them down covertly, that presumably would have
happened years ago.

In fact, American officials have said, turning off the flow of WikiLeaks
data is a legally complicated issue, especially if American citizens or
American-based firms are involved. The Obama administration, they say,
does not want to be accused of suppressing unwelcome speech --- in the
manner of the Russians and the Chinese.

Efforts to reach WikiLeaks on Tuesday were unsuccessful. A sometimes
spokesman, Kristinn Hrafnsson, did not return messages, and a telephone
message and an email message to Sunshine Press, which represents Mr.
Assange, were also unanswered.

Mr. Assange has insisted he does not know the source of the WikiLeaks
material, though he has made no secret of his distaste for Mrs. Clinton.
The United States government has said that much of the hacking was the
work of Russian intelligence and was part of a broad effort to influence
the election. So far, the White House has not announced how it will
respond, though several options have been discussed with President
Obama, according to administration officials.

On Sunday, in a taped interview broadcast on NBC's ``Meet the Press,''
Vice President Joseph R. Biden Jr., in what was either a warning or an
effort at psychological warfare, said that
\href{http://www.nytimes3xbfgragh.onion/2016/10/16/us/politics/biden-hints-at-us-response-to-cyberattacks-blamed-on-russia.html}{``we're
sending a message'' to the Russians ``at a time and place of our
choosing''} and that President Vladimir V. Putin will ``know it'' when
the message arrives. That seemed to suggest some kind of covert action,
perhaps a cyberstrike, in retaliation for what the American intelligence
community has described as a broad and unprecedented effort by a foreign
power to influence American voters.

It is possible that Ecuador feared that, because of its decision to give
exile to Mr. Assange, it risked becoming a witting or unwitting
participant in an effort at voter manipulation.

WikiLeaks provided no evidence to support
\href{https://twitter.com/wikileaks/status/788369924175441920}{its
claim} that Mr. Kerry had pressured Ecuadorean officials, during a
private meeting in Colombia last month, to clamp down on Mr. Assange,
and the State Department's spokesman, John Kirby, immediately denied the
accusation. ``Reports that Secretary Kerry had conversations with
Ecuadorean officials about this are simply untrue. Period,'' he said.

\includegraphics{https://static01.graylady3jvrrxbe.onion/images/2016/10/05/world/05assange-web1/05assange-web1-videoSixteenByNineJumbo1600-v2.jpg}

The president of Ecuador, Rafael Correa, is a man of the left, and he
recently told the Kremlin-backed broadcaster RT that he would support
Mrs. Clinton.

At the same time, he suggested in the interview that a victory by Mr.
Trump, who has made no secret of his admiration for Mr. Putin, would be
good for Latin America because it would, paradoxically, bolster leftist
parties.

``I sincerely believe that it would be better for Latin America if Trump
won,''
\href{https://www.rt.com/news/361312-ecuador-president-interview-trump/}{Mr.
Correa said}. ``When did progressive governments come to power in Latin
America? During the Bush administration. His primitive policies were
rejected so much that it caused reaction in Latin America. Trump would
do the same.''

Questions to the Ecuadorean Embassy on Tuesday were met with a reference
to
\href{http://reinounido.embajada.gob.ec/ecuador-ratifies-the-validity-of-the-asylum-granted-to-julian-assange/}{the
embassy's website} and a brief statement.

``In view of recent speculations, the government of Ecuador reaffirms
the validity of the asylum granted four years ago to Julian Assange,''
the statement said.

Mr. Assange is the subject of an arrest warrant in Sweden, which wanted
to question him about allegations of rape and sexual abuse dating to
2010, to decide whether or not to bring charges.

Mr. Assange, saying that he feared extradition to the United States on
espionage charges stemming from the publication by WikiLeaks of secret
documents given to the website by the former Army analyst Chelsea
Manning, broke bail and took refuge in the Ecuadorean Embassy in June
2012. He has been in the tiny embassy since.

Given the statute of limitations, the one allegation Mr. Assange still
faces in Sweden is rape. He is wanted for questioning but has not been
charged.

There is no public indictment in the United States of Mr. Assange; if
Sweden chose not to press charges, he would presumably be free to leave
the embassy.

Advertisement

\protect\hyperlink{after-bottom}{Continue reading the main story}

\hypertarget{site-index}{%
\subsection{Site Index}\label{site-index}}

\hypertarget{site-information-navigation}{%
\subsection{Site Information
Navigation}\label{site-information-navigation}}

\begin{itemize}
\tightlist
\item
  \href{https://help.nytimes3xbfgragh.onion/hc/en-us/articles/115014792127-Copyright-notice}{©~2020~The
  New York Times Company}
\end{itemize}

\begin{itemize}
\tightlist
\item
  \href{https://www.nytco.com/}{NYTCo}
\item
  \href{https://help.nytimes3xbfgragh.onion/hc/en-us/articles/115015385887-Contact-Us}{Contact
  Us}
\item
  \href{https://www.nytco.com/careers/}{Work with us}
\item
  \href{https://nytmediakit.com/}{Advertise}
\item
  \href{http://www.tbrandstudio.com/}{T Brand Studio}
\item
  \href{https://www.nytimes3xbfgragh.onion/privacy/cookie-policy\#how-do-i-manage-trackers}{Your
  Ad Choices}
\item
  \href{https://www.nytimes3xbfgragh.onion/privacy}{Privacy}
\item
  \href{https://help.nytimes3xbfgragh.onion/hc/en-us/articles/115014893428-Terms-of-service}{Terms
  of Service}
\item
  \href{https://help.nytimes3xbfgragh.onion/hc/en-us/articles/115014893968-Terms-of-sale}{Terms
  of Sale}
\item
  \href{https://spiderbites.nytimes3xbfgragh.onion}{Site Map}
\item
  \href{https://help.nytimes3xbfgragh.onion/hc/en-us}{Help}
\item
  \href{https://www.nytimes3xbfgragh.onion/subscription?campaignId=37WXW}{Subscriptions}
\end{itemize}
