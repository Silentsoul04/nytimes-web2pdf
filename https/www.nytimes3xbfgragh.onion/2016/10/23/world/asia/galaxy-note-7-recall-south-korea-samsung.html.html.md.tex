Sections

SEARCH

\protect\hyperlink{site-content}{Skip to
content}\protect\hyperlink{site-index}{Skip to site index}

\href{https://www.nytimes3xbfgragh.onion/section/world/asia}{Asia
Pacific}

\href{https://myaccount.nytimes3xbfgragh.onion/auth/login?response_type=cookie\&client_id=vi}{}

\href{https://www.nytimes3xbfgragh.onion/section/todayspaper}{Today's
Paper}

\href{/section/world/asia}{Asia Pacific}\textbar{}Galaxy Note 7 Recall
Dismays South Korea, the `Republic of Samsung'

\url{https://nyti.ms/2exvBXa}

\begin{itemize}
\item
\item
\item
\item
\item
\end{itemize}

Advertisement

\protect\hyperlink{after-top}{Continue reading the main story}

Supported by

\protect\hyperlink{after-sponsor}{Continue reading the main story}

\hypertarget{galaxy-note-7-recall-dismays-south-korea-the-republic-of-samsung}{%
\section{Galaxy Note 7 Recall Dismays South Korea, the `Republic of
Samsung'}\label{galaxy-note-7-recall-dismays-south-korea-the-republic-of-samsung}}

\includegraphics{https://static01.graylady3jvrrxbe.onion/images/2016/10/23/world/23KOREA/23KOREA-articleLarge.jpg?quality=75\&auto=webp\&disable=upscale}

By \href{http://www.nytimes3xbfgragh.onion/by/choe-sang-hun}{Choe
Sang-Hun}

\begin{itemize}
\item
  Oct. 22, 2016
\item
  \begin{itemize}
  \item
  \item
  \item
  \item
  \item
  \end{itemize}
\end{itemize}

SEOUL, South Korea --- A former South Korean teacher, Kim Jeong-min was
at Narita Airport in Japan this month when he watched a television news
report that Samsung Electronics's Galaxy Note 7 smartphone was banned on
airplanes because it was prone to
\href{http://www.nytimes3xbfgragh.onion/2016/09/03/business/samsung-galaxy-note-battery.html?_r=0}{catching
fire}.

Mr. Kim, 58, said he had felt humiliated, as if the non-Koreans in the
airport lounge were looking at him.

Though he does not own a Galaxy Note 7, his reaction was typical of the
intense feelings South Koreans hold toward Samsung, the most dramatic
corporate success story to emerge from the country's transformation from
a war-torn agrarian nation to a global economic powerhouse.

``Whether we like it or not, Samsung is to the global market what our
national team is in the Olympics,'' Mr. Kim said.

Calling Samsung this country's biggest and most profitable company
hardly describes its special (but not always favorable) place in minds
here. Some South Koreans say they live in the ``Republic of Samsung.''

Life can literally begin and end with Samsung: One can be born in a
Samsung hospital; attend a Samsung university; honeymoon in a Samsung
hotel; stock a Samsung-built apartment with Samsung appliances bought
with a Samsung credit card; take children to Samsung amusement parks;
and have one's body, upon death, taken to a Samsung funeral center.

For South Koreans, the company's progression from an assembler of clunky
transistor radio sets to the world's leading producer of flat-panel
television sets, computer chips and smartphones is a source of national
pride. Last year, Samsung accounted for 20 percent of South Korea's
\$527 billion in exports. That pride was dented, and economic unease
deepened, when Samsung recalled more than three million Note 7
smartphones globally and
\href{http://www.nytimes3xbfgragh.onion/2016/10/12/business/international/samsung-galaxy-note7-terminated.html}{decided
not to produce} any more because some devices heated up and burst into
flames.

``This is not just Samsung's trouble. It's trouble for the entire
economy,'' the opposition leader Moon Jae-in, a potential contender in
next year's presidential election, said this month, referring to the
Note 7 crisis. ``Because people take pride in Samsung as a brand
representing South Korea, it is their trouble, too.''

On Thursday, President Park Geun-hye voiced concern about the Galaxy
Note 7 recall's impact on exports. The economy has taken recent hits
from rising unemployment rates and
\href{http://www.nytimes3xbfgragh.onion/2016/09/16/business/dealbook/lack-of-planning-hampers-hanjin-shipping-bankruptcy.html}{the
bankruptcy of Hanjin}, a major shipping company. Its shipyards, among
the world's largest, are laying off thousands after posting huge losses
because of shrinking orders and competition from lower-cost rivals in
China.

Samsung is the best-known brand name South Korea has ever produced,
ranking seventh in the
\href{http://interbrand.com/best-brands/best-global-brands/2016/ranking/}{100
best global brands} compiled by Interbrand, a brand consultancy. Its
Galaxy smartphones have lifted its --- and by extension South Korea's
--- high-tech image more than any other Korean product.

Having already overtaken Sony and other Japanese companies it once
mimicked, Samsung has grown powerful enough to challenge Apple, an icon
of American innovation.

To many South Koreans, the Note 7 recall, the biggest ever in the mobile
phone industry, is just
\href{http://www.nytimes3xbfgragh.onion/2016/10/13/business/international/samsung-galaxy-note7-profit-battery-fires.html}{another
painful lesson for Samsung} to learn from and pay for --- the recall is
estimated to cost it \$6.2 billion --- in its quest to dominate yet
another industry.

``All manufacturing companies, including the American and Japanese, make
mistakes,'' said Park Bo-yeon, 29, who was recently browsing in a
handset shop in downtown Seoul where a notice urged customers to hand in
Note 7s. ``What matters is whether you can learn from them and move on.
Samsung always has.''

Ms. Park suspected that the Note 7 fiasco had been overblown by the
American news media, which she said looked down on Samsung. She said she
was disappointed that Samsung had failed to explain why some Note 7s
heated up and caught fire. But she was equally impressed by Samsung's
``courageous decision to terminate the Note 7 before anyone died.''

Among South Koreans, though, the name Samsung also evokes greed and
secrecy. They often describe the company as a predator that makes
profits not so much through innovation as by ruthlessly squeezing its
numerous domestic parts suppliers.

And Samsung has never shaken off its image as an imitator, though a
highly efficient one. (Last year, it was ordered to pay \$548 million in
damages to Apple for infringing on its iPhone design patents,
\href{http://www.nytimes3xbfgragh.onion/2016/03/22/technology/supreme-court-to-hear-samsung-appeal-on-apple-patent-award.html}{a
case that is now} at the United States Supreme Court.)

The Note 7 disaster raised more doubt about Samsung's reputation. It
also reminded South Koreans that their export-driven economy depended so
heavily on Samsung and a handful of other family-controlled
conglomerates, or chaebol, that they often feel it is held hostage to
them.

``The saying that Samsung's good and bad luck is our country's good and
bad luck is propaganda manufactured by Samsung and media and politicians
beholden to it,'' said Kim Sang-gyun, 32, who was visiting the same shop
as Ms. Park. ``Why should I worry about Samsung's trouble unless I owned
a Samsung share or Note 7? And I don't.''

Samsung is the most successful among the chaebol, which spearheaded
South Korea's industrialization by copying foreign competitors' products
but making them cheaper, better and faster.

China is now using the same model to threaten South Korea in the
industries it has dominated through the ``fast follower'' strategy:
shipbuilding, semiconductor and smartphones. South Korea sees itself as
in a constant race to catch up with innovators like Apple while
struggling to keep a step ahead of Chinese rivals.

The Note 7 humiliation left many South Koreans wondering whether Samsung
--- and South Korea in general --- is stumbling in that race.

``They say Samsung is the strongest among our country's businesses,''
said Mr. Kim, the former teacher. ``That's why its Note 7 failure
worries me. It kind of shows our limit.''

Advertisement

\protect\hyperlink{after-bottom}{Continue reading the main story}

\hypertarget{site-index}{%
\subsection{Site Index}\label{site-index}}

\hypertarget{site-information-navigation}{%
\subsection{Site Information
Navigation}\label{site-information-navigation}}

\begin{itemize}
\tightlist
\item
  \href{https://help.nytimes3xbfgragh.onion/hc/en-us/articles/115014792127-Copyright-notice}{©~2020~The
  New York Times Company}
\end{itemize}

\begin{itemize}
\tightlist
\item
  \href{https://www.nytco.com/}{NYTCo}
\item
  \href{https://help.nytimes3xbfgragh.onion/hc/en-us/articles/115015385887-Contact-Us}{Contact
  Us}
\item
  \href{https://www.nytco.com/careers/}{Work with us}
\item
  \href{https://nytmediakit.com/}{Advertise}
\item
  \href{http://www.tbrandstudio.com/}{T Brand Studio}
\item
  \href{https://www.nytimes3xbfgragh.onion/privacy/cookie-policy\#how-do-i-manage-trackers}{Your
  Ad Choices}
\item
  \href{https://www.nytimes3xbfgragh.onion/privacy}{Privacy}
\item
  \href{https://help.nytimes3xbfgragh.onion/hc/en-us/articles/115014893428-Terms-of-service}{Terms
  of Service}
\item
  \href{https://help.nytimes3xbfgragh.onion/hc/en-us/articles/115014893968-Terms-of-sale}{Terms
  of Sale}
\item
  \href{https://spiderbites.nytimes3xbfgragh.onion}{Site Map}
\item
  \href{https://help.nytimes3xbfgragh.onion/hc/en-us}{Help}
\item
  \href{https://www.nytimes3xbfgragh.onion/subscription?campaignId=37WXW}{Subscriptions}
\end{itemize}
