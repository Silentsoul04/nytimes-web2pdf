Sections

SEARCH

\protect\hyperlink{site-content}{Skip to
content}\protect\hyperlink{site-index}{Skip to site index}

\href{https://www.nytimes3xbfgragh.onion/section/food}{Food}

\href{https://myaccount.nytimes3xbfgragh.onion/auth/login?response_type=cookie\&client_id=vi}{}

\href{https://www.nytimes3xbfgragh.onion/section/todayspaper}{Today's
Paper}

\href{/section/food}{Food}\textbar{}Omakase, on a Deadline, at Sushi on
Jones

\url{https://nyti.ms/2ephCTn}

\begin{itemize}
\item
\item
\item
\item
\item
\item
\end{itemize}

Advertisement

\protect\hyperlink{after-top}{Continue reading the main story}

Supported by

\protect\hyperlink{after-sponsor}{Continue reading the main story}

\href{/column/hungry-city}{Hungry City}

\hypertarget{omakase-on-a-deadline-at-sushi-on-jones}{%
\section{Omakase, on a Deadline, at Sushi on
Jones}\label{omakase-on-a-deadline-at-sushi-on-jones}}

\href{https://www.nytimes3xbfgragh.onion/slideshow/2016/10/26/dining/sushi-on-jones.html}{}

\hypertarget{sushi-on-jones}{%
\subsection{Sushi on Jones}\label{sushi-on-jones}}

8 Photos

View Slide Show ›

\includegraphics{https://static01.graylady3jvrrxbe.onion/images/2016/10/26/dining/23HUNGRY-GREAT-JONES-SUSH-slide-5L7H/23HUNGRY-GREAT-JONES-SUSH-slide-5L7H-articleLarge.jpg?quality=75\&auto=webp\&disable=upscale}

An Rong Xu for The New York Times

\begin{itemize}
\tightlist
\item
  Sushi on Jones\\
  Japanese;Sushi \$\$\$ 348 Bowery 917-270-1815
\end{itemize}

By Ligaya Mishan

\begin{itemize}
\item
  Oct. 20, 2016
\item
  \begin{itemize}
  \item
  \item
  \item
  \item
  \item
  \item
  \end{itemize}
\end{itemize}

It rained the first night I went to \href{http://sushionjones.com}{Sushi
on Jones}, a tiny open-air sushi bar on a NoHo lot once home to an auto
repair shop. Someone had rigged two umbrellas and a tarp over the
four-seat counter. The rain rustled on either side, like curtains.

Sushi on Jones is one of five kiosks in the
\href{http://www.thebowerymarket.com/}{Bowery Market food court}, which
opened in July. Since that rainy night, an awning has been installed,
and David Bouhadana, an owner and the head chef, promises to put in a
heater when blizzards come.

Still, don't expect to linger. Sushi on Jones is modeled after the
no-frills sushi counters found in subway stations in Japan. ``We allow
30 minutes of dining,'' I was informed via text before my arrival.
``Clock starts at reservation time.''

Once there, I spotted a timer, ticking inexorably. Later, I overheard
the host warn two dawdling guests, ``The bell has rung.'' (They were
given a grace period to finish their last pieces.)

The deadline may seem ruthless. But even customers at the celebrated
\href{http://www.sushi-jiro.jp/dining-at-jiro/}{Sukiyabashi Jiro} in
Tokyo are allotted only a half-hour to down 20 pieces of sushi, a
privilege for which they pay 30,000 yen, around \$290.

At Sushi on Jones, the price is \$50 for 12 pieces, which gives you two
and a half minutes to contemplate and devour each, a more leisurely pace
than Jiro's. If the sushi is not quite as transcendent, it may be the
best ever made in the equivalent of a tollbooth.

Mr. Bouhadana uses only seafood that's fresh year-round, so the menu
reads like a classic-rock playlist. First up is hamachi, reliably
blushing and so buttery it's almost louche. (It made me miss kanpachi,
its leaner cousin, brighter and springier in texture.)

A few pieces (snow crab, botan shrimp) feel like placeholders. But there
are always two representatives of the salmon family --- king and arctic
char, gradations of creaminess --- and two types of tuna, perhaps meaty
albacore followed by bluefin cut from along the spine, or toro, sweet
and unresisting, from the belly.

All are presented with neither theatrics nor reverence. Reggae lilts
from the speakers. One night, I was served tea in a 7-Eleven cup with a
Sushi on Jones sticker slapped over it, showing a fish playing the
saxophone.

The rotating chefs --- Kenny Jiang, Weiqi Lin and Tim Lin (not related)
--- keep the ratio of fish to rice at the midpoint between jewel-like
and unwieldy. Anointments are mostly familiar and restrained: daub of
yuzu, gloss of ponzu, pulped daikon, hyphen of ginger. A surprise is
scallop with black salt, the hue from activated charcoal, with its
distant tremor of smoke.

The climax of the meal is washugyu, a crossbreed of wagyu and Black
Angus from the Japanese butcher shop across the street, a long tongue of
it torched, with glimmers of truffle salt. Then comes uni, from Santa
Barbara, Calif., when Mr. Bouhadana can get it, tasting like the ocean
distilled into ice cream, and, in a crazed finale, uni over wagyu ---
layers of cool and warm, brine and earth.

It's a pity that the omakase ends with unagi in a sauce as cloying and
perfunctory as at any middling sushi bar across America. Better to add
an optional \$12 hand roll, which furls wagyu and uni into a tighter
embrace.

Mr. Bouhadana, 30, was born in France to a Moroccan Jewish father and a
French Catholic mother who eventually settled in Florida, lured by the
vision of ``Miami Vice.'' He started training at 18 at a sushi bar in
Boca Raton, plying customers with sake bombs; apprenticed in Japan; put
in time at Morimoto, Hatsuhana and 15 East; and briefly ran the kitchens
at Sushi Uo and Sushi Dojo.

This winter, he and a partner, Derek Feldman, plan to open Sushi
Bouhadana on the Lower East Side. It will have a roof.

Sushi on Jones is not the only place in town to find a reasonably priced
sushi omakase. Elsewhere, for maybe \$15 more, you get such perks as a
moment to breathe and shelter from the storm.

But the precariousness is part of the charm. Once, a customer, tilting
on her stool, dropped a piece of uni on the sidewalk. The sushi chef,
unbidden, silently handed her a replacement.

Another night, three diners sat squeezed together at the smaller window
facing Great Jones Street. ``I can see the sunset,'' one said. To the
west hung a few red wisps, the sky starting to burn itself down.

Advertisement

\protect\hyperlink{after-bottom}{Continue reading the main story}

\hypertarget{site-index}{%
\subsection{Site Index}\label{site-index}}

\hypertarget{site-information-navigation}{%
\subsection{Site Information
Navigation}\label{site-information-navigation}}

\begin{itemize}
\tightlist
\item
  \href{https://help.nytimes3xbfgragh.onion/hc/en-us/articles/115014792127-Copyright-notice}{©~2020~The
  New York Times Company}
\end{itemize}

\begin{itemize}
\tightlist
\item
  \href{https://www.nytco.com/}{NYTCo}
\item
  \href{https://help.nytimes3xbfgragh.onion/hc/en-us/articles/115015385887-Contact-Us}{Contact
  Us}
\item
  \href{https://www.nytco.com/careers/}{Work with us}
\item
  \href{https://nytmediakit.com/}{Advertise}
\item
  \href{http://www.tbrandstudio.com/}{T Brand Studio}
\item
  \href{https://www.nytimes3xbfgragh.onion/privacy/cookie-policy\#how-do-i-manage-trackers}{Your
  Ad Choices}
\item
  \href{https://www.nytimes3xbfgragh.onion/privacy}{Privacy}
\item
  \href{https://help.nytimes3xbfgragh.onion/hc/en-us/articles/115014893428-Terms-of-service}{Terms
  of Service}
\item
  \href{https://help.nytimes3xbfgragh.onion/hc/en-us/articles/115014893968-Terms-of-sale}{Terms
  of Sale}
\item
  \href{https://spiderbites.nytimes3xbfgragh.onion}{Site Map}
\item
  \href{https://help.nytimes3xbfgragh.onion/hc/en-us}{Help}
\item
  \href{https://www.nytimes3xbfgragh.onion/subscription?campaignId=37WXW}{Subscriptions}
\end{itemize}
