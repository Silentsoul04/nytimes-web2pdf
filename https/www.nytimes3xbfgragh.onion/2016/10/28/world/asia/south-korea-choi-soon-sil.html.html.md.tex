Sections

SEARCH

\protect\hyperlink{site-content}{Skip to
content}\protect\hyperlink{site-index}{Skip to site index}

\href{https://www.nytimes3xbfgragh.onion/section/world/asia}{Asia
Pacific}

\href{https://myaccount.nytimes3xbfgragh.onion/auth/login?response_type=cookie\&client_id=vi}{}

\href{https://www.nytimes3xbfgragh.onion/section/todayspaper}{Today's
Paper}

\href{/section/world/asia}{Asia Pacific}\textbar{}A Presidential
Friendship Has Many South Koreans Crying Foul

\url{https://nyti.ms/2dQpqLC}

\begin{itemize}
\item
\item
\item
\item
\item
\end{itemize}

Advertisement

\protect\hyperlink{after-top}{Continue reading the main story}

Supported by

\protect\hyperlink{after-sponsor}{Continue reading the main story}

\hypertarget{a-presidential-friendship-has-many-south-koreans-crying-foul}{%
\section{A Presidential Friendship Has Many South Koreans Crying
Foul}\label{a-presidential-friendship-has-many-south-koreans-crying-foul}}

\includegraphics{https://static01.graylady3jvrrxbe.onion/images/2016/10/27/world/27KOREA-web1/27KOREA-web1-articleLarge.jpg?quality=75\&auto=webp\&disable=upscale}

By \href{http://www.nytimes3xbfgragh.onion/by/choe-sang-hun}{Choe
Sang-Hun}

\begin{itemize}
\item
  Oct. 27, 2016
\item
  \begin{itemize}
  \item
  \item
  \item
  \item
  \item
  \end{itemize}
\end{itemize}

SEOUL, South Korea --- South Koreans have been riveted for weeks by a
scandal involving the president and a shadowy adviser accused of being a
``shaman fortuneteller'' by opposition politicians.

The elusive figure,
\href{http://news.naver.com/main/read.nhn?mode=LSD\&mid=sec\&sid1=103\&oid=144\&aid=0000457447}{Choi
Soon-sil}, is a private citizen with no security clearance, yet she had
remarkable influence over President Park Geun-hye: She was allowed to
edit some of Ms. Park's most important speeches.

The news channel Chosun showed video of presidential aides kowtowing to
her after she apparently gave them orders. She apparently had an advance
copy of the president's itinerary for an overseas trip, the TV station
said.

She even had power over the president's wardrobe, overseeing the design
of her dresses and telling her what colors to wear on certain days.

These may not seem like the makings of a major scandal. But as Ms. Park
nears her last year in office, the revelations have sent her polling
numbers to new lows, and a prominent member of her party has called on
her to resign from it, while some South Koreans want her impeached.

In part, the accusations have resonated because they feed into
longstanding criticism that the president is a disconnected leader who
relies only on a trusted few.

But for most South Koreans, the real drama is that Ms. Choi is the
daughter of a religious figure whose relationship with Ms. Park had long
been the subject of lurid rumors. The figure, Choi Tae-min, was often
compared to Rasputin here, and now critics say his daughter is playing
the same role.

Mr. Choi was the founder of an obscure sect called the Church of Eternal
Life. He befriended Ms. Park, 40 years his junior, soon after her mother
was assassinated in 1974. According to a report by the Korean
intelligence agency from the 1970s that was published by a South Korean
newsmagazine in 2007, Mr. Choi initially approached Ms. Park by telling
her that her mother had appeared in his dreams, asking him to help her.

Image

Choi Soon-sil, who Ms. Park described as an old friend, in a photo taken
from an online news report.Credit...Jeon Heon-Kyun/European Pressphoto
Agency

Mr. Choi was a former police officer who had also been a Buddhist monk
and a convert to Roman Catholicism. (He also used seven different names
and was married six times by the time he died in 1994 at the age of 82.)
He became a mentor to Ms. Park, helping her run a pro-government
volunteer group called Movement for a New Mind. Ms. Choi became a youth
leader in that group.

According to the report by the KCIA, as the country's intelligence
agency was then called, Mr. Choi was a ``pseudo pastor'' who had used
his connection to Ms. Park to secure bribes.

Ms. Park's father, Park Chung-hee, the former military dictator, was
assassinated in 1979 by Kim Jae-gyu, the director of the KCIA. Mr. Kim
told a court that one of the reasons he killed Mr. Park was what he
called the president's failure to stop Mr. Choi's corrupt activities and
keep him away from his daughter.

Ms. Park has said that her father once personally questioned her and Mr.
Choi about the accusations of corruption but found no wrongdoing. Mr.
Choi was never charged with a crime in connection with the allegations;
in a newspaper interview in 2007, Ms. Park called him a patriot and said
she was grateful for his counsel and comfort during ``difficult times.''

But gossip about their relationship --- vehemently denied by Ms. Park
--- has haunted her since. In a 2007 diplomatic cable made public
through WikiLeaks, the American Embassy in Seoul reported rumors that
Mr. Choi ``had complete control over Park's body and soul during her
formative years and that his children accumulated enormous wealth as a
result.'' One such tale held that Ms. Park, who has never married, had
his child. (She has denied that.)

In a televised address to the nation on Tuesday, Ms. Park
\href{http://www.nytimes3xbfgragh.onion/aponline/2016/10/25/world/asia/ap-as-skorea-politics.html}{acknowledged
that she had let Ms. Choi edit} some of her most important speeches.

``I deeply apologize to the people,'' Ms. Park said. She described Ms.
Choi as an old friend who had stood by her through painful times, like
the years after the killings of her mother and father.

On Wednesday, prosecutors raided homes belonging to Ms. Choi and some of
her associates, as well as the offices of two foundations she controls,
in connection with allegations that she had used her ties with Ms. Park
to pressure businesses into donating \$69 million to the foundations.

Ms. Choi, who has not been charged with a crime, had traveled to
Germany, where she told a journalist that she was innocent but that she
would not come home to face investigators.

When local news media first reported allegations that Ms. Choi had
edited the president's speeches, Ms. Park's office dismissed them as
``nonsense.'' But those denials crumbled this week, after the cable
channel JTBC reported that it had obtained a discarded tablet computer
once owned by Ms. Choi.

Files discovered there included drafts of 44 speeches and other
statements that Ms. Park had given from 2012 to 2014, as a presidential
candidate and later as president. The computer's log showed that Ms.
Choi had received them hours or days before Ms. Park delivered the
speeches. Many passages were marked in red.

Among the speeches was one that Ms. Park delivered in Dresden, Germany,
in 2014. Widely billed as one of her most important policy statements,
it set out her vision for eventual reunification with North Korea.

It is not clear how extensive Ms. Choi's changes to Ms. Park's speeches
were. Ms. Park said Tuesday that Ms. Choi had offered ``personal
opinions and thoughts'' and helped with ``phrasing and other things.''

Ms. Choi's close relationship with the president has long been
suspected, as people close to her have worked in Ms. Park's
administration.

She and her ex-husband, who was Ms. Park's chief of staff when she was a
lawmaker, have been accused in the past of improperly profiting from
their influence, allegations that Ms. Park dismissed as ``slander'' and
attempts to ``disrupt the national order.'' Officials who investigated
the allegations were fired. But none of that raised the kind of furor
seen in recent weeks.

Barely a day has passed without someone accusing Ms. Choi of influence
peddling, greed or simply arrogance. Last week, the president of Ewha
Womans University in Seoul, a leading university in the nation, resigned
amid accusations that the school had given Ms. Choi's daughter, a
student there, favorable treatment.

This week, a daily newspaper, Hankyoreh, quoted a former employee of one
of Ms. Choi's foundations, Lee Seong-han, as saying that copies of
reports written for Ms. Park had been brought daily to Ms. Choi for
review.

Mr. Lee said that Ms. Choi called Ms. Park ``sister'' and had her own
teams of advisers who meddled in critical government decisions,
including the appointment of cabinet ministers and the closing of the
Kaesong industrial park, a joint project of North and South Korea, after
the North's nuclear test in January.

``Ms. Choi effectively told the president to do this and do that,'' the
newspaper quoted Mr. Lee as saying. ``There was nothing the president
could decide alone.'' Ms. Park's office did not comment on the report.

Advertisement

\protect\hyperlink{after-bottom}{Continue reading the main story}

\hypertarget{site-index}{%
\subsection{Site Index}\label{site-index}}

\hypertarget{site-information-navigation}{%
\subsection{Site Information
Navigation}\label{site-information-navigation}}

\begin{itemize}
\tightlist
\item
  \href{https://help.nytimes3xbfgragh.onion/hc/en-us/articles/115014792127-Copyright-notice}{©~2020~The
  New York Times Company}
\end{itemize}

\begin{itemize}
\tightlist
\item
  \href{https://www.nytco.com/}{NYTCo}
\item
  \href{https://help.nytimes3xbfgragh.onion/hc/en-us/articles/115015385887-Contact-Us}{Contact
  Us}
\item
  \href{https://www.nytco.com/careers/}{Work with us}
\item
  \href{https://nytmediakit.com/}{Advertise}
\item
  \href{http://www.tbrandstudio.com/}{T Brand Studio}
\item
  \href{https://www.nytimes3xbfgragh.onion/privacy/cookie-policy\#how-do-i-manage-trackers}{Your
  Ad Choices}
\item
  \href{https://www.nytimes3xbfgragh.onion/privacy}{Privacy}
\item
  \href{https://help.nytimes3xbfgragh.onion/hc/en-us/articles/115014893428-Terms-of-service}{Terms
  of Service}
\item
  \href{https://help.nytimes3xbfgragh.onion/hc/en-us/articles/115014893968-Terms-of-sale}{Terms
  of Sale}
\item
  \href{https://spiderbites.nytimes3xbfgragh.onion}{Site Map}
\item
  \href{https://help.nytimes3xbfgragh.onion/hc/en-us}{Help}
\item
  \href{https://www.nytimes3xbfgragh.onion/subscription?campaignId=37WXW}{Subscriptions}
\end{itemize}
