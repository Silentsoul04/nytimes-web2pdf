Sections

SEARCH

\protect\hyperlink{site-content}{Skip to
content}\protect\hyperlink{site-index}{Skip to site index}

\href{https://www.nytimes3xbfgragh.onion/section/world/middleeast}{Middle
East}

\href{https://myaccount.nytimes3xbfgragh.onion/auth/login?response_type=cookie\&client_id=vi}{}

\href{https://www.nytimes3xbfgragh.onion/section/todayspaper}{Today's
Paper}

\href{/section/world/middleeast}{Middle East}\textbar{}Stark Choice for
Syrians in Rebel Areas: `Doom' or the Green Bus

\url{https://nyti.ms/2dQTV8V}

\begin{itemize}
\item
\item
\item
\item
\item
\end{itemize}

Advertisement

\protect\hyperlink{after-top}{Continue reading the main story}

Supported by

\protect\hyperlink{after-sponsor}{Continue reading the main story}

\hypertarget{stark-choice-for-syrians-in-rebel-areas-doom-or-the-green-bus}{%
\section{Stark Choice for Syrians in Rebel Areas: `Doom' or the Green
Bus}\label{stark-choice-for-syrians-in-rebel-areas-doom-or-the-green-bus}}

\includegraphics{https://static01.graylady3jvrrxbe.onion/images/2016/10/30/world/Greenbuses-1030/Greenbuses-1030-videoSixteenByNineJumbo1600.jpg}

By \href{http://www.nytimes3xbfgragh.onion/by/anne-barnard}{Anne
Barnard} and
\href{https://www.nytimes3xbfgragh.onion/by/hwaida-saad}{Hwaida Saad}

\begin{itemize}
\item
  Oct. 29, 2016
\item
  \begin{itemize}
  \item
  \item
  \item
  \item
  \item
  \end{itemize}
\end{itemize}

BEIRUT, Lebanon --- The lime-green buses once ferried Syrians to school,
work and dates at Damascus cafes. Now they pull up at moments of defeat,
when rebel fighters and civilians, besieged and bombarded, give up their
territory to government forces and board the vehicles en route to an
uncertain future.

The buses, once a benign, even beloved feature of the urban landscape,
have become a signature of the Syrian government's starve-or-surrender
strategy. In recent days, government warplanes dropped fliers on the
rebel-held districts of Aleppo, offering a stark choice to the estimated
250,000 people trapped in that strategic city: ``doom,'' represented by
a photo of a bloody body, or ``redemption,'' in the form of
\href{https://twitter.com/CombatChris1/status/789142466624651264}{a
green bus.}

Images of the buses are everywhere: on state television reports and
pro-government websites
\href{https://www.facebookcorewwwi.onion/syrian.reporters/videos/1170350329677205/}{celebrating}
the evacuations, and on
\href{https://www.youtube.com/watch?v=4DfkLfgJy84\&feature=em-uploademail}{opposition
videos} mourning what they call deportations. Women and children, or
fighters with guns, peer from their windows. They cry, chant defiantly
or stare into space as they leave areas that have long symbolized revolt
against President Bashar al-Assad, like the
\href{http://www.nytimes3xbfgragh.onion/2016/08/27/world/middleeast/syria-daraya-falls-symbol-rebellion.html}{recently
emptied Damascus suburb of Daraya}.

Riders are usually offered a choice between two destinations, but as
with so many aspects of the bloody and chaotic Syrian civil war, both
options are bad. They can take the green buses to government territory,
where many fear arrest and conscription, or to another rebel-held area,
where they face continued government airstrikes --- like the ones that
hit a school Wednesday and
\href{http://www.nytimes3xbfgragh.onion/2016/10/27/world/middleeast/syria-school-airstrike.html}{killed
22 children} in Idlib Province.

``Damn the green buses, I'm seeing them in my dreams,'' said Jalal
al-Telawi, 36, a computer technician whose neighbors in Waer, a besieged
district on the outskirts of the central city of Homs, recently debated
whether to board the buses in the latest evacuation offer.

Mr. Telawi was experiencing a certain déjà vu. Two years ago, he took a
green bus with fellow fighters
\href{http://www.nytimes3xbfgragh.onion/2014/05/08/world/middleeast/syria.html}{out
of the Old City of Homs in a deal with the government,}only to face
another siege in Waer. ``We have a phobia --- a `bus complex,''' he
said. ``In our minds, they equal displacement.''

The Chinese-made buses first arrived in Syrian cities with much fanfare
in 2009. Back then, they were a symbol of the modernization promised by
Mr. Assad. At one point painted red and plastered with the logo of a
cellphone company owned by a cousin of the president, they replaced
rickety repurposed school buses and supplemented the small white vans
known as ``servis,'' providing improved, affordable public
transportation for students and workers.

Osama Mohammad Ali, now an antigovernment activist trapped in Waer,
speaks wistfully of riding the bus to law school in Homs on rainy days
alongside people from every sect and walk of life.

``The driver used to play Fairuz'' --- the Lebanese singer and diva ---
``and there was a kind of respect among us,'' he said. ``If I saw an
elderly man standing, I would give him or her my seat.''

But when demonstrations demanding political reform broke out in 2011,
the buses were used to transport camouflage-clad state security officers
or gun-toting militiamen
\href{https://www.youtube.com/watch?v=UMC8W0fGHk4}{through Damascus
traffic} to beat up and arrest protesters.

As the uprising turned to armed conflict, clashes left burned-out buses
rusting in the streets. The steel carcasses sometimes served as barriers
between government and rebel territory.

Then, in a 2014 deal supervised by United Nations officials, the green
buses evacuated the last rebels from the Old City district of
\href{http://www.nytimes3xbfgragh.onion/2014/05/08/world/middleeast/syria.html}{Homs}
as the government took over the area.

``We were all crying, `Is this the end of Homs men?''' Mr. Telawi, the
technician, recalled recently. ``We thought we would be liberating Homs,
but instead our end will be in the green bus.''

The buses have since been used again and again in the local surrender
deals that the government has promoted in place of a national peace
agreement.

The Syrian government and its ally Russia portray the evacuations as an
act of mercy, freeing people they contend are being used as human
shields. Opponents of the government increasingly see the process as
``ethnic cleansing,'' as it has primarily displaced members of the Sunni
majority.

The United Nations has condemned the evacuations as a ``forced
displacement'' of civilians, calling for residents to be allowed to
``return voluntarily, in safety and in dignity.''

But as the stretched Syrian Army continues to expel civilians from
hard-to-control areas, the United Nations and other agencies operating
in the country risk being implicated in ``a dangerous precedent,'' Aron
Lund, an analyst at the Carnegie Middle East Center,
\href{http://carnegie-mec.org/diwan/64498?lang=en}{wrote recently}.
Armed groups, he continued, can ``target and deport civilians with
impunity in Syria, and perhaps elsewhere.''

\includegraphics{https://static01.graylady3jvrrxbe.onion/images/2016/10/30/world/30greenbuses/30greenbuses-articleInline.jpg?quality=75\&auto=webp\&disable=upscale}

When rebel-held towns refuse the deals, bombardments intensify and
sieges tighten. In August, south of Damascus, the last 1,500 people in
Daraya capitulated
\href{http://www.nytimes3xbfgragh.onion/2016/08/26/world/middleeast/daraya-syria-assad-surrender.html}{after
an incendiary bombing of their last hospital}. Some went to
government-held suburbs, others to rebel-held Idlib Province.

Abu Adnan, 50, was one of the bus drivers. He took 20 fighters and their
families to Idlib, where many of them had never been before. They wept
as they crossed the checkpoint on their way out of Daraya.

``Even I started tearing up --- the crying of men is so hard,'' he
recalled in an interview, asking to be identified only by his nickname
to avoid repercussions for expressing sympathy. ``I saw a gunman put
some soil from Daraya in a plastic bag and smell it as if it was soil
from paradise.''

The 200-mile drive took 30 hours, he said, with the bus stopping at many
checkpoints. At some, the rebel passengers threatened to shoot if
security forces clambered aboard. At others, the security men cheered
for Mr. Assad while the passengers cheered for Daraya and revolution.

In September came evacuations from Waer, with hundreds departing for
rebel-held territory farther north.

Mr. Ali, the law student, chose to remain under the blockade. ``I can't
stand to see these buses now,'' he said, but he noted that some people
saw them as a means of rescue, to ``take them from hell to start a new
life,'' perhaps fleeing to Turkey.

Next, buses --- white this time --- took rebels and some civilians from
the Damascus suburb of Qudsaya, where remaining residents chanted their
support for the government as security officials welcomed the town
``back into the lap of the country.''

And on Oct. 19, rebels and activists boarded green buses bound for Idlib
from the Damascus suburb of Moadhamiyeh, where chemical attacks killed
hundreds in 2013. Among them was a doctor named Muhannad, who helped
collect evidence for the United Nations investigation into those attacks
and who spoke on the condition that he be identified by only his first
name, for safety.

He, his wife and their son were hoping to smuggle themselves to Austria,
where the doctor once worked. Staying home was not an option: He was
wanted by 16 security branches, on charges of treating wounded fighters.

``I'm inside Bus No. M-09,'' he said by phone. Asked about the
atmosphere, he said, ``One word: crying.''

More than 2,600 people died in Moadhamiyeh during the war, he said. ``We
will never have the chance to read the Fatiha over our people,'' he
added, referring to a prayer of mourning.

The green buses were also offered --- or offered as a threat --- on Oct.
21 to people in the besieged rebel-held eastern sections of Aleppo, the
ancient city split since 2012 between government and rebel territory.

After the government encircled the rebel side over the summer, a
pro-government reporter, Shadi Helweh, taunted the rebels on state
television. ``I swear to God, the green buses will be here,''
\href{https://www.facebookcorewwwi.onion/syrian.reporters/videos/vb.254488777930036/1114638088581763/?type=2\&theater}{he
said}. ``We will take selfies with those mercenaries while they leave
like rats.''

People in eastern Aleppo have much to flee: government and Russian
airstrikes on hospitals, apartment buildings and schools; dwindling
supplies of food.

But the green buses idled. Virtually no one came out, though Russia had
declared a unilateral halt to airstrikes and promoted evacuations.

The government and Russia accused rebels of blocking civilians from
leaving. Rebels said no one should be evacuated unless humanitarian aid
was also allowed in --- and told residents that it was not safe to take
the buses without international supervision. The United Nations called
for a more comprehensive humanitarian pause --- with aid deliveries and
medical evacuations --- but that did not come to pass.

It is hard to know how many residents of Aleppo and other trouble spots
would leave if they had reliable safety guarantees. Many, including
older residents, say they just want to stay in their homes.

Mr. Telawi, the technician who rode the green bus out of Homs in 2014
but refused to board again in Waer this year, remembered that when one
of his friends had gotten on the bus, he had taken out an old ticket,
stamped it with the self-service machine as if for a normal ride, and
kept it.

``He said he'll use it again,'' Mr. Telawi recalled. ``On the way
back.''

Advertisement

\protect\hyperlink{after-bottom}{Continue reading the main story}

\hypertarget{site-index}{%
\subsection{Site Index}\label{site-index}}

\hypertarget{site-information-navigation}{%
\subsection{Site Information
Navigation}\label{site-information-navigation}}

\begin{itemize}
\tightlist
\item
  \href{https://help.nytimes3xbfgragh.onion/hc/en-us/articles/115014792127-Copyright-notice}{©~2020~The
  New York Times Company}
\end{itemize}

\begin{itemize}
\tightlist
\item
  \href{https://www.nytco.com/}{NYTCo}
\item
  \href{https://help.nytimes3xbfgragh.onion/hc/en-us/articles/115015385887-Contact-Us}{Contact
  Us}
\item
  \href{https://www.nytco.com/careers/}{Work with us}
\item
  \href{https://nytmediakit.com/}{Advertise}
\item
  \href{http://www.tbrandstudio.com/}{T Brand Studio}
\item
  \href{https://www.nytimes3xbfgragh.onion/privacy/cookie-policy\#how-do-i-manage-trackers}{Your
  Ad Choices}
\item
  \href{https://www.nytimes3xbfgragh.onion/privacy}{Privacy}
\item
  \href{https://help.nytimes3xbfgragh.onion/hc/en-us/articles/115014893428-Terms-of-service}{Terms
  of Service}
\item
  \href{https://help.nytimes3xbfgragh.onion/hc/en-us/articles/115014893968-Terms-of-sale}{Terms
  of Sale}
\item
  \href{https://spiderbites.nytimes3xbfgragh.onion}{Site Map}
\item
  \href{https://help.nytimes3xbfgragh.onion/hc/en-us}{Help}
\item
  \href{https://www.nytimes3xbfgragh.onion/subscription?campaignId=37WXW}{Subscriptions}
\end{itemize}
