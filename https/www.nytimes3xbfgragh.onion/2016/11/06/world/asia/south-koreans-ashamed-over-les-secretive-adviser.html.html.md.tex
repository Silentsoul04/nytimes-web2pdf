Sections

SEARCH

\protect\hyperlink{site-content}{Skip to
content}\protect\hyperlink{site-index}{Skip to site index}

\href{https://www.nytimes3xbfgragh.onion/section/world/asia}{Asia
Pacific}

\href{https://myaccount.nytimes3xbfgragh.onion/auth/login?response_type=cookie\&client_id=vi}{}

\href{https://www.nytimes3xbfgragh.onion/section/todayspaper}{Today's
Paper}

\href{/section/world/asia}{Asia Pacific}\textbar{}South Koreans
`Ashamed' Over Leader's Secretive Adviser

\url{https://nyti.ms/2ep05bD}

\begin{itemize}
\item
\item
\item
\item
\item
\end{itemize}

Advertisement

\protect\hyperlink{after-top}{Continue reading the main story}

Supported by

\protect\hyperlink{after-sponsor}{Continue reading the main story}

\hypertarget{south-koreans-ashamed-over-leaders-secretive-adviser}{%
\section{South Koreans `Ashamed' Over Leader's Secretive
Adviser}\label{south-koreans-ashamed-over-leaders-secretive-adviser}}

\includegraphics{https://static01.graylady3jvrrxbe.onion/images/2016/11/06/world/06KOREA1/06KOREA1-videoSixteenByNine3000.jpg}

By \href{http://www.nytimes3xbfgragh.onion/by/choe-sang-hun}{Choe
Sang-Hun}

\begin{itemize}
\item
  Nov. 5, 2016
\item
  \begin{itemize}
  \item
  \item
  \item
  \item
  \item
  \end{itemize}
\end{itemize}

SEOUL, South Korea --- In 1973, a newspaper advertisement in the South
Korean city of Daejeon announced the arrival of ``The Messenger From the
Spiritual World,'' a messiah who embodied the best of Buddhism,
Christianity and other religions.

``Religious leaders of all beliefs, come and learn from him!'' it read.
``Those in distress or afflicted with incurable diseases, come and seek
his counsel!''

The Messenger was Won Ja-kyong, one of several aliases used by Choi
Tae-min --- a man who, more than 20 years after his death, is at the
root of a scandal that has roiled this country and sent President Park
Geun-hye's approval rating to the lowest point of any modern South
Korean leader, according to Gallup Korea. Tens of thousands of people
marched in Seoul, the capital, on Saturday, demanding Ms. Park's
resignation.

The scandal's primary figure is not Mr. Choi but his daughter, Choi
Soon-sil, who has inherited his role as a secretive adviser for the
president. This past week, Ms. Choi was arrested and charged with using
her influence with Ms. Park to extort large sums from Korean companies.

Ms. Park has admitted letting Ms. Choi, who has no government job or
background in policy, edit some of her speeches, and a cascade of news
reports have alleged that Ms. Choi had considerable sway in the
presidential Blue House and other government agencies.

But what infuriates many South Koreans about the Choi affair is not
merely that Ms. Park had a secret adviser, or even the possibility that
the adviser turned a profit from the relationship. It is the notion that
their president has been in thrall for decades to a family of religious
charlatans --- a shameful throwback, in their view, to ancient stories
of Korean kings and queens brought to ruin by deceitful monks or
fortunetelling shamans.

``I am so embarrassed that if a foreigner asks me where I am from, I
feel like saying I am Chinese or Japanese,'' Oh Yoo-jeong said at a
recent antigovernment rally in Seoul. ``Until now, I had thought this
kind of thing happened only in historical dramas on TV.''

Hand-held signs at demonstrations have read, ``I'm ashamed to call this
our country!''

There is no evidence that Ms. Choi has continued the colorful religious
practices of her father --- the founder of an obscure sect called the
Church of Eternal Life --- or that Ms. Park, who befriended both Chois
in the 1970s, was ever his spiritual follower. Still, in protest
rallies, on social media and even in newspaper editorials, Ms. Park is
\href{https://www.youtube.com/watch?v=JmwlA7O96NM}{depicted as a
puppet}, manipulated while a young woman by Mr. Choi and while president
by his daughter.

\includegraphics{https://static01.graylady3jvrrxbe.onion/images/2016/11/06/world/06KOREA2/06KOREA2-articleLarge.jpg?quality=75\&auto=webp\&disable=upscale}

Rumors to that effect are rampant. After a cable news channel discovered
that Ms. Choi had advised Ms. Park on what color to wear during an
overseas trip, some suggested that she had based her choices on an
astrological system. Others have raised the possibility that Ms. Choi
inserted mystic symbolism into decorations at Ms. Park's inauguration
ceremony in 2013. Some people have gone so far as to call Ms. Choi a
``sorcerer regent.''

The ties between Ms. Park and the Chois, which have been a matter of
public record for decades, have long generated gossip of that kind, but
it was never taken seriously by large numbers of people until the recent
revelations of Ms. Choi's influence in the Blue House.

Her father's activities in the 1970s, around the time he met Ms. Park,
were described in journals by Tak Myeong-hwan, a researcher into fringe
religious figures. He describes visiting the ``Messenger'' known as Won
Ja-kyong in 1973, and watching him and his followers chant incantations
in front of brightly colored circles drawn on a wall. Shamans, the
practitioners of traditional spiritual rituals who still thrive in
Korea, came to see him and cowered before him, Mr. Tak wrote.

When the two met again in 1975, the Messenger drove a Jeep, had a
different name --- Choi Tae-min --- and called himself a Christian
pastor, leading a group he called Crusaders to Save the Nation.
According to Mr. Tak, Mr. Choi said he worked for ``the esteemed
daughter Geun-hye'' and had free access to the Blue House.

South Korea was then led by the dictator Park Chung-hee, the father of
Ms. Park, whom Mr. Choi had befriended. In 1975, according to a report
written by South Korea's intelligence agency, Mr. Choi had written to
Ms. Park asking to meet her, saying he had communicated with her mother,
who was killed by a pro-North Korean assassin the previous year. (Ms.
Park and Mr. Choi later denied that he had made such a claim.)

Though Park Chung-hee would prove skeptical of Mr. Choi, the Crusaders
to Save the Nation were an idea the unpopular Cold War dictator could
hardly refuse: pro-government pastors, organized in a crusade against
Communism. Members donned officers' uniforms and underwent
military-style training.

Newspaper articles from the time say Ms. Park attended the group's
inaugural ceremony and was made its honorary president.
\href{http://newstapa.org/35182}{Government television footage} from the
time shows Ms. Park, escorted by a young Choi Soon-sil, attending a
gathering of volunteers and receiving adulation reminiscent of the
personality cult surrounding the Kim family in North Korea.

After Park Chung-hee was assassinated in 1979, Ms. Park retreated into
seclusion, and Mr. Choi remained a trusted confidant in the years that
followed, according to a 1989 government intelligence report that was
recently obtained by the Chosun Ilbo newspaper. The report said he had
provided her with daily necessities and predicted that she would one day
be a ``queen.''

It also said he had embezzled funds from a foundation run by Ms. Park.
An earlier intelligence report had accused Mr. Choi of using his ties
with Ms. Park to extort money from businesses, a charge echoed in the
current allegations against his daughter.

In 1990, Ms. Park's sister and brother wrote to President Roh Tae-woo,
asking him to stop what they called Mr. Choi's continuing extortion and
to rescue their sister from what they called his manipulation. (Ms. Park
later called such accusations ``slander.'') On Friday, Ms. Park
acknowledged that she had cut ties with her siblings since becoming
president to prevent influence-peddling by relatives.

After Mr. Choi died in 1994, his daughter appears to have taken a
prominent role in Ms. Park's life. When Jeon Yeo-ok, a former television
journalist, met the two in the mid-1990s, she said, they were ``just
like a princess and her lady in waiting.''

``It was more like she was
\href{http://news.naver.com/main/read.nhn?mode=LSD\&mid=sec\&oid=023\&aid=0003223740\&sid1=001}{part
of the Choi family},'' Ms. Jeon wrote.

Today, there is no sign of the religious sect Mr. Choi once operated,
and Ms. Park and the Chois have denied practicing cultlike or
shamanistic rituals. But Ms. Choi's scandal has revived old rumors about
her father, which in turn have reinforced speculation about Ms. Choi, in
whom many people purport to see a new version of him.

Korea, a land frequently torn by wars and deprivations, has been a rich
ground for fringe religious groups. When the ferry Sewol sank in 2014,
killing more than 300 people, South Koreans were shocked to learn that
the ship
\href{http://www.nytimes3xbfgragh.onion/2014/07/27/world/asia/in-ferry-deaths-a-south-korean-tycoons-downfall.html}{belonged
to a cult leader}.

\href{http://www.nytimes3xbfgragh.onion/2007/07/07/world/asia/07korea.html}{Shamanism
remains a strong force} in the country, despite past governments'
attempts to discourage it. Thousands of its practitioners, known as
mudang, still operate, almost all of them women who claim to be able to
communicate with the dead. Most of their clients are also women.

Because of this, when Ms. Choi is called a shaman --- as she often is
--- some sense a faint misogyny at play. Some conservative politicians
have criticized Ms. Park for entrusting state affairs to a mere
``ajumma,'' or housewife. Men joke that the scandal has dimmed the
chances of South Korea's having another female president.

``President Park and Ms. Choi made female leadership a matter of
ridicule,'' wrote Yang Sun-hee, a female editorial writer at the
JoongAng Ilbo newspaper.

Even amid the anger over the Choi scandal, many are taking delight in
it. After Ms. Choi lost a shoe during a media scrum this past week, the
resulting headline was
``\href{http://www.kyeonggi.com/?mod=news\&act=articleView\&idxno=1262640}{The
Shaman Wears Prada}.'' A running joke is that Ms. Park can do nothing
without asking ``Soon Siri,'' a play on Ms. Choi's name and the iPhone
voice-command function.

Bloggers have also labeled her Soonderella, a mix of her name and
Cinderella. But as the protests continue, the story seems unlikely to
have a Cinderella ending.

``From now on,'' Ms. Park said on Friday, ``I will completely cut off
any private relationships in my life.''

Advertisement

\protect\hyperlink{after-bottom}{Continue reading the main story}

\hypertarget{site-index}{%
\subsection{Site Index}\label{site-index}}

\hypertarget{site-information-navigation}{%
\subsection{Site Information
Navigation}\label{site-information-navigation}}

\begin{itemize}
\tightlist
\item
  \href{https://help.nytimes3xbfgragh.onion/hc/en-us/articles/115014792127-Copyright-notice}{©~2020~The
  New York Times Company}
\end{itemize}

\begin{itemize}
\tightlist
\item
  \href{https://www.nytco.com/}{NYTCo}
\item
  \href{https://help.nytimes3xbfgragh.onion/hc/en-us/articles/115015385887-Contact-Us}{Contact
  Us}
\item
  \href{https://www.nytco.com/careers/}{Work with us}
\item
  \href{https://nytmediakit.com/}{Advertise}
\item
  \href{http://www.tbrandstudio.com/}{T Brand Studio}
\item
  \href{https://www.nytimes3xbfgragh.onion/privacy/cookie-policy\#how-do-i-manage-trackers}{Your
  Ad Choices}
\item
  \href{https://www.nytimes3xbfgragh.onion/privacy}{Privacy}
\item
  \href{https://help.nytimes3xbfgragh.onion/hc/en-us/articles/115014893428-Terms-of-service}{Terms
  of Service}
\item
  \href{https://help.nytimes3xbfgragh.onion/hc/en-us/articles/115014893968-Terms-of-sale}{Terms
  of Sale}
\item
  \href{https://spiderbites.nytimes3xbfgragh.onion}{Site Map}
\item
  \href{https://help.nytimes3xbfgragh.onion/hc/en-us}{Help}
\item
  \href{https://www.nytimes3xbfgragh.onion/subscription?campaignId=37WXW}{Subscriptions}
\end{itemize}
