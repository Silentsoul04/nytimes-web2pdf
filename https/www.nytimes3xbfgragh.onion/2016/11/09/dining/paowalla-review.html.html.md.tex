Sections

SEARCH

\protect\hyperlink{site-content}{Skip to
content}\protect\hyperlink{site-index}{Skip to site index}

\href{https://www.nytimes3xbfgragh.onion/section/food}{Food}

\href{https://myaccount.nytimes3xbfgragh.onion/auth/login?response_type=cookie\&client_id=vi}{}

\href{https://www.nytimes3xbfgragh.onion/section/todayspaper}{Today's
Paper}

\href{/section/food}{Food}\textbar{}An Indian Chef Comes Home Again at
Paowalla in SoHo

\url{https://nyti.ms/2eBliiR}

\begin{itemize}
\item
\item
\item
\item
\item
\item
\end{itemize}

Advertisement

\protect\hyperlink{after-top}{Continue reading the main story}

Supported by

\protect\hyperlink{after-sponsor}{Continue reading the main story}

\href{/column/restaurant-review}{Restaurant Review}

\hypertarget{an-indian-chef-comes-home-again-at-paowalla-in-soho}{%
\section{An Indian Chef Comes Home Again at Paowalla in
SoHo}\label{an-indian-chef-comes-home-again-at-paowalla-in-soho}}

\href{https://www.nytimes3xbfgragh.onion/slideshow/2016/11/09/dining/paowalla-restaurant-floyd-cardoz.html}{}

\hypertarget{paowalla}{%
\subsection{Paowalla}\label{paowalla}}

12 Photos

View Slide Show ›

\includegraphics{https://static01.graylady3jvrrxbe.onion/images/2016/11/09/dining/09REST-PAOWALLA-slide-87QR/09REST-PAOWALLA-slide-87QR-articleLarge.jpg?quality=75\&auto=webp\&disable=upscale}

Jean Schwarzwalder for The New York Times

\begin{itemize}
\tightlist
\item
  Paowalla\\
  ★★ Indian \$\$\$ 195 Spring Street 212-235-1098
\end{itemize}

\href{https://resy.com/cities/ny/paowalla?utm_source=nyt\&utm_medium=restoprofile\&utm_campaign=affiliates\&aff_id=c1fe784}{Reserve
a Table}

When you make a reservation at an independently reviewed restaurant
through our site, we earn an affiliate commission.

By \href{http://www.nytimes3xbfgragh.onion/by/pete-wells}{Pete Wells}

\begin{itemize}
\item
  Nov. 8, 2016
\item
  \begin{itemize}
  \item
  \item
  \item
  \item
  \item
  \item
  \end{itemize}
\end{itemize}

After
\href{http://events.nytimes3xbfgragh.onion/mem/nycreview.html?res=9C03EFDA103DF937A15751C0A96F958260}{Tabla},
Floyd Cardoz doesn't owe New York anything. In his 11 years as the chef
of that restaurant across from Madison Square Park, he used the language
of Indian cuisine to say things we had never heard before.

A native of Mumbai, he showed us that the Indian spice cabinet can
perform many other tricks besides curry. Upstairs at Tabla, he helped
end the era in which the fine-dining wing of the restaurant business
operated as a club to which cuisines of non-European descent need not
apply. At the bottom of the winding staircase, in the little lounge
known as Bread Bar, he showed us that a serious chef could pull off a
more affordable and accessible menu without pandering.

Restaurants that change people's perceptions don't come along very
often, though. So in assessing Mr. Cardoz's new restaurant,
\href{http://www.paowalla.com/}{Paowalla}, it's more productive to view
the things it does well (and there are a lot) as a bonus, without
getting caught up in the ways in which it doesn't quite live up to
Tabla.

The next two jobs Mr. Cardoz held in New York after Tabla closed in 2010
took him away from Indian cooking. In theory,
\href{http://www.nytimes3xbfgragh.onion/2015/01/28/dining/restaurant-review-blue-smoke-and-north-end-grill.html?_r=0}{North
End Grill} and White Street could have set him free. In practice,
neither gave free rein to his talents.

Luckily for him and for us, he is his own boss at Paowalla, and he's
cooking Indian food again. Reaching into his back catalog, Mr. Cardoz is
serving some dishes and even a couple of cocktails, like the
mind-clearing cucumber-gin highball called the Kachumber Cooler, that
are flashbacks to Tabla and Bread Bar.

He found a corner spot in SoHo that used to be an Italian restaurant.
The wood-fueled pizza oven in the middle of the dining room was a stroke
of luck, fitting in with the theme that also led to the name. Paowalla
means bread seller, and breads lead the menu. They come in many shapes
and sizes, from the familiar standards like naans and rotis to more
exotic forms.

Pao are yeasted dinner rolls, a Portuguese-derived specialty of Goa with
a slight, charming sweetness. (If you grew up with Parker house rolls,
you want to drop a pat of butter on them.) The white spirals called
tingmo are Tibetan steamed buns, which Mr. Cardoz treats like cinnamon
rolls with chile paste where the cinnamon should be.

Rosemary naan had almost no rosemary flavor, and naan stuffed with
minced bacon wasn't as irresistible as it should have been. But the
kulcha filled with Cheddar was improbably wonderful.

I understand why Mr. Cardoz has put the breads at the top of the batting
order on his menu. But I think his policy of automatically serving them
first, with nothing but chutneys to accompany them, is a mistake. These
breads deserve to be on the table in the company of other food. They're
a little lonely on their own.

A lot of the menu is made up of small plates, many of them snacklike and
many served with bread, to the benefit of both bread and snack. The
melting roasted bone marrow is terrific spread on sliced pao and
drizzled with a thrilling chutney of fresh herbs and curry leaves.
Potato fritters are sandwiched between halves of pao in a fun evocation
of wada pao, a Mumbai street snack.

A curried shrimp turnover is a homage to one that Mr. Cardoz remembers
eating as a child on the beaches in Goa. I'd remember a hand pie this
good, too. Another street snack he's toying around with is chaat; the
night I had it, he was making it with sweet potatoes and apples, an
untraditional assemblage that made perfect sense.

One likably eccentric aspect of Mr. Cardoz's menu at North End Grill was
the raft of egg dishes. Eggs make a more limited, but still memorable,
appearance at Paowalla. Scrambled eggs patia are sultry with ginger and
very, very soft; like the marrow, they come with sliced pao. Eggs
kejriwal, meanwhile, is simple and totally likable: a sunny-side-up egg
on toast ratcheted up by a green-chile chutney.

With the larger plates, the playfulness and street-food allusions are
mostly left behind. The results can be mixed.

Mr. Cardoz has an excellent idea for getting New Yorkers to eat dal: He
drops a globe of burrata in the middle of the plate. It's one of the
only large plates without meat or fish; in general, it's a little
tricker to eat a vegetarian meal at Paowalla than it should be.

I wanted more flavor from the pork in baby pig vindaloo and a higher
sour-to-sweet ratio from the sauce. And I wanted the bacon-and-chorizo
biryani to be different in any number of ways: less saturated with
smoke, more aromatic with spices, and probably cheaper than the \$57
Paowalla is charging for it, although I might not whine about the price
of this family-size portion if I had liked the way it tasted.

The seafood main courses, though, tended to be very good, including the
dogfish rubbed with a mildly spicy coating of coconut and tomatoes and
then steamed inside a banana-leaf packet, or the whole sea bream roasted
with a thick vein of vinegared spice paste inside.

The menu sprawls, offering more than enough reasons to try Paowalla
once. Whether you will come back depends on how much you mind the
peculiarly disjointed dining space and service that can be
unintentionally maddening.

At first glance, Paowalla looks like a cozy neighborhood retreat, but
several design decisions make it peculiarly uncomfortable. A row of
stools facing a window puts anyone who sits there up for the inspection
of pedestrians on Sullivan Street. The back end of the dining room is
separated by a steel-and-glass partition, turning one medium-size space
into two cramped, awkward ones and doing a real number on the acoustics.
The most comfortable and, oddly enough, most peaceful evening I had at
Paowalla was spent at a sidewalk table.

That night was also free of a hovering, mosquitolike service style that
I encountered inside. Once I had been spotted, somebody was always
leaning in to clear a glass, or fill one, or wipe the table, or make
room for another plate. When a new course arrived, before anybody had a
chance to try any of it, I heard a question that was new to me.

``First impressions?''

It's bad enough to be asked, as happens more and more often lately,
``How are the first few bites tasting?'' If we have to give opinions on
how the food looks without tasting it, we may as well stay home looking
at Instagram.

Advertisement

\protect\hyperlink{after-bottom}{Continue reading the main story}

\hypertarget{site-index}{%
\subsection{Site Index}\label{site-index}}

\hypertarget{site-information-navigation}{%
\subsection{Site Information
Navigation}\label{site-information-navigation}}

\begin{itemize}
\tightlist
\item
  \href{https://help.nytimes3xbfgragh.onion/hc/en-us/articles/115014792127-Copyright-notice}{©~2020~The
  New York Times Company}
\end{itemize}

\begin{itemize}
\tightlist
\item
  \href{https://www.nytco.com/}{NYTCo}
\item
  \href{https://help.nytimes3xbfgragh.onion/hc/en-us/articles/115015385887-Contact-Us}{Contact
  Us}
\item
  \href{https://www.nytco.com/careers/}{Work with us}
\item
  \href{https://nytmediakit.com/}{Advertise}
\item
  \href{http://www.tbrandstudio.com/}{T Brand Studio}
\item
  \href{https://www.nytimes3xbfgragh.onion/privacy/cookie-policy\#how-do-i-manage-trackers}{Your
  Ad Choices}
\item
  \href{https://www.nytimes3xbfgragh.onion/privacy}{Privacy}
\item
  \href{https://help.nytimes3xbfgragh.onion/hc/en-us/articles/115014893428-Terms-of-service}{Terms
  of Service}
\item
  \href{https://help.nytimes3xbfgragh.onion/hc/en-us/articles/115014893968-Terms-of-sale}{Terms
  of Sale}
\item
  \href{https://spiderbites.nytimes3xbfgragh.onion}{Site Map}
\item
  \href{https://help.nytimes3xbfgragh.onion/hc/en-us}{Help}
\item
  \href{https://www.nytimes3xbfgragh.onion/subscription?campaignId=37WXW}{Subscriptions}
\end{itemize}
