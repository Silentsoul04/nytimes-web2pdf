Sections

SEARCH

\protect\hyperlink{site-content}{Skip to
content}\protect\hyperlink{site-index}{Skip to site index}

\href{https://www.nytimes3xbfgragh.onion/section/world/asia}{Asia
Pacific}

\href{https://myaccount.nytimes3xbfgragh.onion/auth/login?response_type=cookie\&client_id=vi}{}

\href{https://www.nytimes3xbfgragh.onion/section/todayspaper}{Today's
Paper}

\href{/section/world/asia}{Asia Pacific}\textbar{}South Korean
President's Leadership Style Is Seen as Factor in Scandal

\url{https://nyti.ms/2epM3vj}

\begin{itemize}
\item
\item
\item
\item
\item
\end{itemize}

Advertisement

\protect\hyperlink{after-top}{Continue reading the main story}

Supported by

\protect\hyperlink{after-sponsor}{Continue reading the main story}

\hypertarget{south-korean-presidents-leadership-style-is-seen-as-factor-in-scandal}{%
\section{South Korean President's Leadership Style Is Seen as Factor in
Scandal}\label{south-korean-presidents-leadership-style-is-seen-as-factor-in-scandal}}

\includegraphics{https://static01.graylady3jvrrxbe.onion/images/2016/11/12/world/12PARK/12PARK-articleLarge.jpg?quality=75\&auto=webp\&disable=upscale}

By \href{http://www.nytimes3xbfgragh.onion/by/choe-sang-hun}{Choe
Sang-Hun}

\begin{itemize}
\item
  Nov. 11, 2016
\item
  \begin{itemize}
  \item
  \item
  \item
  \item
  \item
  \end{itemize}
\end{itemize}

SEOUL, South Korea --- A police detective who worked in the South Korean
president's office filed a report in 2014 accusing relatives and
associates of an unofficial presidential adviser of meddling in state
affairs.

He was promptly reassigned.

That was just the start of his troubles. After a newspaper reported some
of his findings, the detective, Park Kwan-cheon, who worked as an
anti-graft watchdog, was charged with leaking government documents.
President Park Geun-hye, who is not related to Mr. Park, accused him of
``undermining national discipline.'' He was convicted and spent 16
months in prison.

To opponents of the president, the case confirms that she is just like
her father, the military dictator Park Chung-hee: an isolated,
authoritarian leader who uses state power against critics while shielded
by a small coterie of advisers.

Mr. Park is not the only official who paid for raising alarms about the
adviser, Choi Soon-sil, a longtime friend of the president who is at the
center of the
\href{http://www.nytimes3xbfgragh.onion/2016/11/06/world/asia/south-koreans-ashamed-over-les-secretive-adviser.html}{scandal}
crippling her administration. Other officials were demoted or forced to
resign. At least two people, including a journalist, were prosecuted for
spreading rumors that Ms. Park had a relationship with Ms. Choi's
ex-husband.

As the scandal grows, even many older South Koreans who revere Ms.
Park's father --- and who were crucial to her election victory in 2012
--- have turned against her. Her approval numbers have dropped to record
lows, and crowds of protesters have called on her to resign. A large
demonstration in Seoul was expected on Saturday.

``In the end, she turned out to be a dunderhead who couldn't even
separate public affairs from private friendships,'' said Kim Ky-baek,
64, who runs a nationalist website, Minjokcorea. ``What so disappointed
conservatives like me is that she tainted her father's name, rather than
honoring it.''

Prosecutors have charged Ms. Choi with leveraging her ties to Ms. Park
to extort millions from South Korean businesses; they have also charged
one aide to Ms. Park with helping her to do so. News reports have said
that Ms. Choi held considerable sway in the presidential Blue House and
the Ministry of Culture, Sports and Tourism, despite having no official
post or background in policy. Ms. Park has said only that Ms. Choi
edited some of her speeches.

In addition, Ms. Choi's background --- her father, who was also close to
Ms. Park, led a fringe religious sect --- has led many South Koreans to
conclude that Ms. Choi wielded a sort of cultlike control over the
president. Ms. Park denied this.

Such colorful accusations aside, the notion that Ms. Park relies too
heavily on a few trusted aides --- one of whom might one day betray her,
like the intelligence chief who assassinated her father in 1979 --- has
been part of South Korean political discussion for years. A former
cabinet minister recently compared her advisers to cockroaches, saying
that they operated in the shadows.

Ms. Park's detached leadership style may have encouraged such
speculation. She holds just one news conference a year. Even after
apologizing on Nov. 4 for the Choi scandal and agreeing to be questioned
by investigators if asked, she took no questions from reporters. Some of
her senior presidential aides said that they have never had a one-on-one
policy meeting with Ms. Park.

Her government's zealous pursuit of ideological opponents has also
invited comparisons to her father's rule. In 2014, it forced a small
left-wing party to disband, on the grounds that it subscribed to North
Korean ideology. A performance artist was indicted over graffiti
directed toward Ms. Park that read ``sayonara,'' the Japanese word for
goodbye.

In 2014, a Japanese reporter, Tatsuya Kato, was charged with defamation
for reporting rumors that Ms. Park and Ms. Choi's husband, himself a
former parliamentary aide for Ms. Park, had been engaged in a romantic
liaison during the sinking of a ferry that killed hundreds of students.
\href{http://www.nytimes3xbfgragh.onion/2015/12/18/world/asia/south-korea-park-geun-hye-defamation-verdict.html}{Mr.
Kato was later acquitted}, but in 2015, a South Korean activist was
imprisoned for
\href{http://www.nytimes3xbfgragh.onion/2016/03/06/world/asia/defamation-laws-south-korea-critics-press-freedom.html}{scattering
leaflets that carried the same rumor}.

And officials like Mr. Park, the former police officer, have paid a
price for investigating Ms. Choi or her family. In 2013, two officials
at the culture and sports ministry who pursued accusations that her
family interfered in the affairs of an equestrian association --- Ms.
Choi's daughter is an equestrian --- were banished to obscure positions
and later resigned.

This summer, Lee Seok-su, a senior government auditor appointed by Ms.
Park to monitor the president's relatives and associates, was forced to
resign after looking into corruption allegations involving Ms. Choi and
presidential aides. Several aides sued journalists in 2014 for reporting
similar allegations involving them and Ms. Choi's husband. One of those
aides was recently arrested on charges of passing on classified
presidential documents to Ms. Choi.

Prosecutors are
\href{http://www.nytimes3xbfgragh.onion/2016/11/04/world/asia/south-korea-park-geun-hye-investigation.html}{being
pressured to expand their inquiry} to include Ms. Park. In July last
year, she invited 17 senior South Korean executives to the Blue House,
and it has been suggested in domestic news media that she may have asked
them to donate to foundations controlled by Ms. Choi.

Ms. Park's office denied any wrongdoing tied to the meetings. It said it
could not comment on matters under prosecutors' investigation but added
that many news reports were speculative.

Ms. Park has apologized twice for the Choi scandal in televised
speeches, saying that she had let her guard down with a trusted friend.
But she did not say whether she knew about Ms. Choi's alleged extortion.

On the day of her second speech, however, new evidence emerged that her
administration had put heavy-handed pressure on businesses in the past.
MBN, a cable news channel, broadcast a recording of a 2013 telephone
conversation in which a presidential aide told an executive at CJ, a
food and entertainment conglomerate, that Ms. Park wanted its vice
chairwoman to resign for reasons he did not specify. ``We want her to
quit,'' the aide said. ``What more explanation do you need?''

That recording, too, raised memories of South Korea's authoritarian
past. One of the military dictators who succeeded Ms. Park's father,
Chun Doo-hwan, forced businesses to donate to a foundation under his
control in the 1980s. Big business in South Korea remain vulnerable to
political manipulation because of their murky corporate governance, said
Kim Sang-jo, an economist at Hansung University in Seoul.

``What people find so outrageous and anachronistic is that a similar
thing is still happening in South Korea 30 years later,'' Mr. Kim said.

In a bid to regain public trust, Ms. Park recently agreed to
\href{http://www.nytimes3xbfgragh.onion/2016/11/08/world/asia/south-korea-park-choi-scandal-parliament.html}{cede
some power to a prime minister} chosen by the opposition-dominated
Parliament. But such moves have failed to defuse the scandal. Large
protests denouncing Ms. Park have been held in central Seoul on a weekly
basis.

``Poetic justice is what comes to mind,'' Mr. Park, the former police
officer who investigated Ms. Choi's family in 2014, recently told
reporters.

Advertisement

\protect\hyperlink{after-bottom}{Continue reading the main story}

\hypertarget{site-index}{%
\subsection{Site Index}\label{site-index}}

\hypertarget{site-information-navigation}{%
\subsection{Site Information
Navigation}\label{site-information-navigation}}

\begin{itemize}
\tightlist
\item
  \href{https://help.nytimes3xbfgragh.onion/hc/en-us/articles/115014792127-Copyright-notice}{©~2020~The
  New York Times Company}
\end{itemize}

\begin{itemize}
\tightlist
\item
  \href{https://www.nytco.com/}{NYTCo}
\item
  \href{https://help.nytimes3xbfgragh.onion/hc/en-us/articles/115015385887-Contact-Us}{Contact
  Us}
\item
  \href{https://www.nytco.com/careers/}{Work with us}
\item
  \href{https://nytmediakit.com/}{Advertise}
\item
  \href{http://www.tbrandstudio.com/}{T Brand Studio}
\item
  \href{https://www.nytimes3xbfgragh.onion/privacy/cookie-policy\#how-do-i-manage-trackers}{Your
  Ad Choices}
\item
  \href{https://www.nytimes3xbfgragh.onion/privacy}{Privacy}
\item
  \href{https://help.nytimes3xbfgragh.onion/hc/en-us/articles/115014893428-Terms-of-service}{Terms
  of Service}
\item
  \href{https://help.nytimes3xbfgragh.onion/hc/en-us/articles/115014893968-Terms-of-sale}{Terms
  of Sale}
\item
  \href{https://spiderbites.nytimes3xbfgragh.onion}{Site Map}
\item
  \href{https://help.nytimes3xbfgragh.onion/hc/en-us}{Help}
\item
  \href{https://www.nytimes3xbfgragh.onion/subscription?campaignId=37WXW}{Subscriptions}
\end{itemize}
