Sections

SEARCH

\protect\hyperlink{site-content}{Skip to
content}\protect\hyperlink{site-index}{Skip to site index}

\href{https://www.nytimes3xbfgragh.onion/section/us}{U.S.}

\href{https://myaccount.nytimes3xbfgragh.onion/auth/login?response_type=cookie\&client_id=vi}{}

\href{https://www.nytimes3xbfgragh.onion/section/todayspaper}{Today's
Paper}

\href{/section/us}{U.S.}\textbar{}Amid Division, a March in Washington
Seeks to Bring Women Together

\url{https://nyti.ms/2eNEWNF}

\begin{itemize}
\item
\item
\item
\item
\item
\end{itemize}

Advertisement

\protect\hyperlink{after-top}{Continue reading the main story}

Supported by

\protect\hyperlink{after-sponsor}{Continue reading the main story}

\hypertarget{amid-division-a-march-in-washington-seeks-to-bring-women-together}{%
\section{Amid Division, a March in Washington Seeks to Bring Women
Together}\label{amid-division-a-march-in-washington-seeks-to-bring-women-together}}

\includegraphics{https://static01.graylady3jvrrxbe.onion/images/2016/11/18/nytnow/18xp-MARCH/18xp-MARCH-articleInline.jpg?quality=75\&auto=webp\&disable=upscale}

By \href{http://www.nytimes3xbfgragh.onion/by/katie-rogers}{Katie
Rogers}

\begin{itemize}
\item
  Nov. 18, 2016
\item
  \begin{itemize}
  \item
  \item
  \item
  \item
  \item
  \end{itemize}
\end{itemize}

A movement is growing to bring together women across race, creed and
political beliefs by luring them off social media and arranging for them
to meet in person.

It's a nice idea, but there's one catch:
\href{https://www.facebookcorewwwi.onion/events/2169332969958991/}{The
Women's March on Washington} is being organized on Facebook, the
nation's preferred platform to battle over race, gender, politics and
just about everything else.

The timing of the event, which organizers began planning the morning
after the election but are careful not to call a protest, is aimed at
the coming administration of President-elect Donald J. Trump. More than
\href{https://www.facebookcorewwwi.onion/events/2169332969958991/?active_tab=discussion}{100,000
people have said on Facebook} that they will travel to the capital to
participate. The plan is to walk from the Lincoln Memorial to the White
House on Jan. 21, 2017, the morning after Mr. Trump's inauguration.

``We're doing it his very first day in office because we are making a
statement,'' one organizer, Breanne Butler, said. ``The marginalized
groups you attacked during your campaign? We are here and we are
watching. And, like, `Welcome to the White House.' ''

Since Election Day, there has been momentum around supporting groups
that are opposed to Mr. Trump's espoused views on women and minority
groups. Nonprofit organizations, including the Planned Parenthood
Federation of America and the American Civil Liberties Union, have
reported a
\href{http://www.nytimes3xbfgragh.onion/2016/11/18/us/politics/nonprofit-donations-trump.html}{surge
in donations} after the election. But the election taught Americans that
women are deeply divided along party lines, education level and race:
\href{http://www.nytimes3xbfgragh.onion/2016/12/01/us/politics/white-women-helped-elect-donald-trump.html}{53
percent of white women voted for Mr. Trump}, according to exit poll
data.

On the march group's Facebook page, it is easy to see how complicated
the idea of the
\href{http://www.nytimes3xbfgragh.onion/2016/11/13/opinion/the-myth-of-female-solidarity.html}{``women's
vote,''} an already
\href{http://www.nytimes3xbfgragh.onion/2016/11/15/magazine/the-dream-and-the-myth-of-the-womens-vote.html}{mythological
concept}, has become, and how difficult it might be for organizers to
fulfill their aim of gathering women who remain fiercely divided on
reproductive rights, gun control, same-sex marriage and immigration,
among other issues.

Not everyone on the page believes, for instance, that Hillary Clinton
would have made a good president, or that Stephen K. Bannon, a chief
strategist under Mr. Trump,
\href{http://www.nytimes3xbfgragh.onion/2016/11/15/us/politics/stephen-bannon-white-house-trump.html}{holds
divisive views about minorities}. Debates over both have sprung up in
recent days. Bob Bland, one of the march organizers, said in an email
that organizers in Maryland had to change a Facebook page from public to
private to protect the safety of women who want to attend.

Evvie Harmon, a yoga teacher from Greenville, S.C., who is helping
state-based efforts to organize for the march, said the group had nixed
a possible idea for a slogan --- ``Human rights are women's rights, and
women's rights are human rights'' ---
\href{http://www.nytimes3xbfgragh.onion/politics/first-draft/2015/09/05/20-years-later-hillary-clintons-beijing-speech-on-women-resonates/}{because
it was something} that Mrs. Clinton once said.

``This is not an anti-Trump protest,'' Ms. Harmon said. ``This is the
reaction of women and minorities across the world who are very disturbed
by the rhetoric that was said over the last year and a half.''

Aside from dueling political views, organizers are trying to take
feedback from a cacophony of voices in real time as they try to assemble
a network of state volunteers, plan programming and arrange
transportation and lodging for the event. Ms. Butler, a chef who is
organizing the event in her spare time, said the march had no official
means of funding yet.

There are women on the page who have said that the march is not
inclusive enough, and that they don't want an event organized by white
women. Ms. Butler acknowledged the criticism but stressed that the women
who are organizing are from different racial and religious backgrounds.

(There was even controversy over the original name: Organizers have
changed the name from Million Woman March to the Women's March on
Washington because observers took issue with the fact that the original
name \href{http://www.blackpast.org/aah/million-woman-march-1997}{echoed
a black women's march} held in Philadelphia in 1997.)

Ms. Butler, 27, said the greater concern would be helping local groups
raise money to help women who can't afford to travel to Washington.

``The reality is that it's incredibly expensive to fly to D.C. on
inauguration weekend,'' Ms. Butler said. ``We don't want only an
upper-middle class of people at this march because no one else can
afford to go.''

Tabitha St. Bernard-Jacobs, 34, who plans to help sign up attendees by
visiting churches, synagogues and community centers in New York City,
said she had been working to include all types of people --- including
those who have not been on Facebook lately.

``I am a woman of color and I am an immigrant,'' said Ms. St.
Bernard-Jacobs, who lives in Brooklyn. She said of the march: ``For me,
it has been completely inclusive.''

This is all plenty of pressure for a days-old grass-roots movement
without a concrete path to funding itself, but organizers are optimistic
as they look ahead to January.

According to Ms. Butler, the group's request for a permit to march is
still pending. On Friday, Michael Litterst, a spokesman for the National
Park Service, said in an email that the group's request to march is one
of at least 13 requests currently under review for areas the agency
administers in the nation's capital. Those also include rallies and
demonstrations.

Mr. Litterst said the Park Service was also reviewing five requests for
official inauguration events.

Advertisement

\protect\hyperlink{after-bottom}{Continue reading the main story}

\hypertarget{site-index}{%
\subsection{Site Index}\label{site-index}}

\hypertarget{site-information-navigation}{%
\subsection{Site Information
Navigation}\label{site-information-navigation}}

\begin{itemize}
\tightlist
\item
  \href{https://help.nytimes3xbfgragh.onion/hc/en-us/articles/115014792127-Copyright-notice}{©~2020~The
  New York Times Company}
\end{itemize}

\begin{itemize}
\tightlist
\item
  \href{https://www.nytco.com/}{NYTCo}
\item
  \href{https://help.nytimes3xbfgragh.onion/hc/en-us/articles/115015385887-Contact-Us}{Contact
  Us}
\item
  \href{https://www.nytco.com/careers/}{Work with us}
\item
  \href{https://nytmediakit.com/}{Advertise}
\item
  \href{http://www.tbrandstudio.com/}{T Brand Studio}
\item
  \href{https://www.nytimes3xbfgragh.onion/privacy/cookie-policy\#how-do-i-manage-trackers}{Your
  Ad Choices}
\item
  \href{https://www.nytimes3xbfgragh.onion/privacy}{Privacy}
\item
  \href{https://help.nytimes3xbfgragh.onion/hc/en-us/articles/115014893428-Terms-of-service}{Terms
  of Service}
\item
  \href{https://help.nytimes3xbfgragh.onion/hc/en-us/articles/115014893968-Terms-of-sale}{Terms
  of Sale}
\item
  \href{https://spiderbites.nytimes3xbfgragh.onion}{Site Map}
\item
  \href{https://help.nytimes3xbfgragh.onion/hc/en-us}{Help}
\item
  \href{https://www.nytimes3xbfgragh.onion/subscription?campaignId=37WXW}{Subscriptions}
\end{itemize}
