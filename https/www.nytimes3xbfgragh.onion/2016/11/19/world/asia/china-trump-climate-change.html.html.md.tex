Sections

SEARCH

\protect\hyperlink{site-content}{Skip to
content}\protect\hyperlink{site-index}{Skip to site index}

\href{https://www.nytimes3xbfgragh.onion/section/world/asia}{Asia
Pacific}

\href{https://myaccount.nytimes3xbfgragh.onion/auth/login?response_type=cookie\&client_id=vi}{}

\href{https://www.nytimes3xbfgragh.onion/section/todayspaper}{Today's
Paper}

\href{/section/world/asia}{Asia Pacific}\textbar{}Trump Has Called
Climate Change a Chinese Hoax. Beijing Says It Is Anything But.

\url{https://nyti.ms/2eMOXdC}

\begin{itemize}
\item
\item
\item
\item
\item
\item
\end{itemize}

Advertisement

\protect\hyperlink{after-top}{Continue reading the main story}

Supported by

\protect\hyperlink{after-sponsor}{Continue reading the main story}

\href{/column/sinosphere}{Sinosphere}

\hypertarget{trump-has-called-climate-change-a-chinese-hoax-beijing-says-it-is-anything-but}{%
\section{Trump Has Called Climate Change a Chinese Hoax. Beijing Says It
Is Anything
But.}\label{trump-has-called-climate-change-a-chinese-hoax-beijing-says-it-is-anything-but}}

\includegraphics{https://static01.graylady3jvrrxbe.onion/images/2016/11/19/world/19CHINACLIMATE-1/19CHINACLIMATE-1-articleLarge.jpg?quality=75\&auto=webp\&disable=upscale}

By \href{http://www.nytimes3xbfgragh.onion/by/edward-wong}{Edward Wong}

\begin{itemize}
\item
  Nov. 18, 2016
\item
  \begin{itemize}
  \item
  \item
  \item
  \item
  \item
  \item
  \end{itemize}
\end{itemize}

Despite
\href{http://dotearth.blogs.nytimes3xbfgragh.onion/2016/09/21/trumps-stance-on-the-paris-climate-agreement-is-criticized-by-375-scientists/}{overwhelming
scientific evidence} that the world's climate is changing, the
president-elect of the United States, Donald J. Trump, has long been on
the side of the deniers.

In 2012, he
\href{https://twitter.com/realdonaldtrump/status/265895292191248385?lang=en}{posted
on Twitter} a couple of messages that asserted that climate change was a
hoax that China had devised to secure an unfair trade advantage,
presumably because the Obama administration was seeking to curb coal
consumption in the United States.

``The concept of global warming was created by and for the Chinese in
order to make U.S. manufacturing non-competitive,'' Mr. Trump wrote.
That message has been reshared more than 104,000 times and ``liked''
nearly 66,000 times.

On Wednesday, a deputy foreign minister of China, Liu Zhenmin, told
reporters at a climate conference in Marrakesh, Morocco, that starting
from the 1980s, the administrations of Mr. Trump's Republican
predecessors --- Presidents Ronald Reagan and George Bush --- supported
climate change negotiations under a United Nations panel.

That was apparently an important moment in China's realization of the
onset of climate change.

Mr. Liu said that President Xi Jinping brought up the issue in
\href{http://www.nytimes3xbfgragh.onion/2016/11/15/world/asia/trump-china-xi-jinping.html}{his
call with Mr. Trump} on Monday, saying that China would continue its
struggle to curb climate change ``whatever the circumstances,''
\href{http://www.bloomberg.com/news/articles/2016-11-16/china-tells-trump-that-climate-change-is-no-hoax-it-invented}{according
to Bloomberg News}.

China's lecturing the United States on the need to fight climate change
is a reversal from the usual roles and a sign that, with the United
States governed by Mr. Trump, China may have to take the leadership
position in the global campaign.

Image

Liu Zhenmin, a deputy foreign minister of China.Credit...How Hwee
Young/European Pressphoto Agency

Under President Obama, the United States government persuaded China to
announce important pledges in the fight against climate change. In 2014,
Mr. Xi stood next to Mr. Obama in Beijing and said that China would
ensure that its greenhouse gas emissions
\href{http://www.nytimes3xbfgragh.onion/2014/11/12/world/asia/china-us-xi-obama-apec.html}{peaked
by 2030} and that 20 percent of its energy would come from non-fossil
fuel sources by that year. Mr. Obama pledged to greatly reduce coal use
by 2025.

Diplomats and climate negotiators have been meeting in Marrakesh this
week to discuss steps needed to carry out the Paris climate agreement,
which was negotiated last year and which China and the United States
signed this year. There has been
\href{http://www.nytimes3xbfgragh.onion/2016/11/16/world/united-nations-climate-change-trump.html?_r=0}{much
anxiety} over whether Mr. Trump might try to withdraw the United States
from the pact after he takes office.

But the agreement now has enough countries as signatories to make it
legally binding, and Mr. Trump may have a hard time extricating the
United States from the deal. In addition, other countries have said they
intend to go ahead with the plan on their own.

``It is a new world order,'' Erik Solheim, the executive director of the
United Nations Environment Program, said in Marrakesh. ``Leadership on
climate change policy has now gone to the developing countries, China
among them.''

However, having the United States on board would be hugely helpful in
trying to meet the ambitious goal of keeping the future increase in
global temperatures \href{http://www.cop21.gouv.fr/en/why-2c/}{below 3.6
degrees Fahrenheit}. The United States is the second-leading emitter of
greenhouse gases, after China.

Researchers say greenhouse gas emissions have leveled off for at least
three years now, mostly because of a drop in coal consumption in China.
A major reason is the slowing Chinese economy. But it is still unclear
whether China's emissions have
\href{http://www.nytimes3xbfgragh.onion/2016/04/04/world/asia/china-climate-change-peak-carbon-emissions.html}{hit
a peak}, well before the 2030 deadline.

China is already grappling with the direct consequences of climate
change, like
\href{http://www.nytimes3xbfgragh.onion/2015/12/09/world/asia/chinese-glaciers-retreat-signals-trouble-for-asian-water-supply.html?_r=0}{melting
glaciers} in western mountain ranges and
\href{http://www.nytimes3xbfgragh.onion/interactive/2016/10/25/world/asia/china-climate-change-resettlement.html}{expanding
deserts}.

Advertisement

\protect\hyperlink{after-bottom}{Continue reading the main story}

\hypertarget{site-index}{%
\subsection{Site Index}\label{site-index}}

\hypertarget{site-information-navigation}{%
\subsection{Site Information
Navigation}\label{site-information-navigation}}

\begin{itemize}
\tightlist
\item
  \href{https://help.nytimes3xbfgragh.onion/hc/en-us/articles/115014792127-Copyright-notice}{©~2020~The
  New York Times Company}
\end{itemize}

\begin{itemize}
\tightlist
\item
  \href{https://www.nytco.com/}{NYTCo}
\item
  \href{https://help.nytimes3xbfgragh.onion/hc/en-us/articles/115015385887-Contact-Us}{Contact
  Us}
\item
  \href{https://www.nytco.com/careers/}{Work with us}
\item
  \href{https://nytmediakit.com/}{Advertise}
\item
  \href{http://www.tbrandstudio.com/}{T Brand Studio}
\item
  \href{https://www.nytimes3xbfgragh.onion/privacy/cookie-policy\#how-do-i-manage-trackers}{Your
  Ad Choices}
\item
  \href{https://www.nytimes3xbfgragh.onion/privacy}{Privacy}
\item
  \href{https://help.nytimes3xbfgragh.onion/hc/en-us/articles/115014893428-Terms-of-service}{Terms
  of Service}
\item
  \href{https://help.nytimes3xbfgragh.onion/hc/en-us/articles/115014893968-Terms-of-sale}{Terms
  of Sale}
\item
  \href{https://spiderbites.nytimes3xbfgragh.onion}{Site Map}
\item
  \href{https://help.nytimes3xbfgragh.onion/hc/en-us}{Help}
\item
  \href{https://www.nytimes3xbfgragh.onion/subscription?campaignId=37WXW}{Subscriptions}
\end{itemize}
