\href{/section/world/americas}{Americas}\textbar{}Fidel Castro, Cuban
Revolutionary Who Defied U.S., Dies at 90

\url{https://nyti.ms/2g1IaMO}

\begin{itemize}
\item
\item
\item
\item
\item
\item
\end{itemize}

\includegraphics{https://static01.graylady3jvrrxbe.onion/images/2015/06/20/world/Fidel-Castro-obituary-slide-P9CB/Fidel-Castro-obituary-slide-P9CB-articleLarge-v6.jpg?quality=75\&auto=webp\&disable=upscale}

Sections

\protect\hyperlink{site-content}{Skip to
content}\protect\hyperlink{site-index}{Skip to site index}

\hypertarget{fidel-castro-cuban-revolutionary-who-defied-us-dies-at-90}{%
\section{Fidel Castro, Cuban Revolutionary Who Defied U.S., Dies at
90}\label{fidel-castro-cuban-revolutionary-who-defied-us-dies-at-90}}

Mr. Castro brought the Cold War to the Western Hemisphere, bedeviled 11
American presidents and briefly pushed the world to the brink of nuclear
war.

Credit...Jack Manning/The New York Times

Supported by

\protect\hyperlink{after-sponsor}{Continue reading the main story}

By \href{http://www.nytimes3xbfgragh.onion/by/anthony-depalma}{Anthony
DePalma}

\begin{itemize}
\item
  Nov. 26, 2016
\item
  \begin{itemize}
  \item
  \item
  \item
  \item
  \item
  \item
  \end{itemize}
\end{itemize}

\href{http://www.nytimes3xbfgragh.onion/es/2016/11/26/fidel-castro-lider-de-la-revolucion-cubana-y-simbolo-de-la-izquierda-muere-a-los-90-anos/}{Leer
en español}

Fidel Castro, the fiery apostle of revolution who brought the Cold War
to the Western Hemisphere in 1959 and then defied the United States for
nearly half a century as Cuba's maximum leader, bedeviling 11 American
presidents and briefly pushing the world to the brink of nuclear war,
died on Friday. He was 90.

Cuban state television announced the death but gave no other details.

In declining health for several years, Mr. Castro had orchestrated what
he hoped would be the continuation of his Communist revolution,
\href{http://www.nytimes3xbfgragh.onion/2006/08/14/world/americas/14cuba.html?_r=0}{stepping
aside in 2006} when a serious illness felled him. He provisionally ceded
much of his power to his younger brother Raúl, now 85, and two years
later
\href{http://www.nytimes3xbfgragh.onion/2008/02/20/world/americas/20cuba.html}{formally
resigned as president}. Raúl Castro, who had fought alongside Fidel
Castro from the earliest days of the insurrection and remained minister
of defense and his brother's closest confidant, has ruled Cuba since
then, although he has told the Cuban people he intends to resign in
2018.

Fidel Castro had held on to power longer than any other living national
leader except Queen Elizabeth II. He became a towering international
figure whose importance in the 20th century far exceeded what might have
been expected from the head of state of a Caribbean island nation of 11
million people.

He dominated his country with strength and symbolism from the day he
triumphantly
\href{http://timesmachine.nytimes3xbfgragh.onion/timesmachine/1959/01/09/89104354.html?action=click\&contentCollection=Archives\&module=ArticleEndCTA\&region=ArchiveBody\&pgtype=article\&pageNumber=1}{entered
Havana} on Jan. 8, 1959, and completed his overthrow of
\href{http://www.nytimes3xbfgragh.onion/1973/08/07/archives/batista-excuban-dictator-dies-in-spain-unending-exile-succession-of.html}{Fulgencio
Batista} by delivering his first major speech in the capital before tens
of thousands of admirers at the vanquished dictator's military
headquarters.

A spotlight shone on him as he swaggered and spoke with passion until
dawn. Finally, white doves were released to signal Cuba's new peace.
When one landed on Mr. Castro, perching on a shoulder, the crowd
erupted, chanting: ``Fidel! Fidel!'' To the war-weary Cubans gathered
there and those watching on television, it was an electrifying sign that
their young, bearded guerrilla leader was destined to be their savior.

Most people in the crowd had no idea what Mr. Castro planned for Cuba. A
master of image and myth, Mr. Castro believed himself to be the messiah
of his fatherland, an indispensable force with authority from on high to
control Cuba and its people.

He wielded power like a tyrant, controlling every aspect of the island's
existence. He was Cuba's ``Máximo Lider.'' From atop a Cuban Army tank,
he directed his country's defense at the
\href{http://learning.blogs.nytimes3xbfgragh.onion/2012/04/17/april-17-1961-the-bay-of-pigs-invasion-against-castro/}{Bay
of Pigs}. Countless details fell to him, from selecting the color of
uniforms that Cuban soldiers wore in Angola to overseeing a program to
produce a superbreed of milk cows. He personally set the goals for sugar
harvests. He personally sent countless men to prison.

But it was more than repression and fear that kept him and his
totalitarian government in power for so long. He had both admirers and
detractors in Cuba and around the world. Some saw him as a ruthless
despot who trampled rights and freedoms; many others hailed him as the
crowds did that first night, as a revolutionary hero for the ages.

Even when he fell ill and
\href{http://www.nytimes3xbfgragh.onion/2007/01/16/world/americas/16fidel.html}{was
hospitalized} with diverticulitis in the summer of 2006, giving up most
of his powers for the first time, Mr. Castro tried to dictate the
details of his own medical care and orchestrate the continuation of his
Communist revolution, engaging a plan as old as the revolution itself.

By handing power to his brother, Mr. Castro once more raised the ire of
his enemies in Washington. United States officials condemned the
transition, saying it prolonged a dictatorship and again denied the
long-suffering Cuban people a chance to control their own lives.

But in December 2014, President Obama used his executive powers to dial
down the decades of antagonism between Washington and Havana by moving
to exchange prisoners and
\href{http://www.nytimes3xbfgragh.onion/2014/12/18/world/americas/us-cuba-relations.html}{normalize
diplomatic relations} between the two countries, a deal worked out with
the help of Pope Francis and after 18 months of secret talks between
representatives of both governments.

\includegraphics{https://static01.graylady3jvrrxbe.onion/images/2016/03/30/world/americas/castro-video/castro-video--videoSixteenByNine3000.jpg}

Though increasingly frail and rarely seen in public, Mr. Castro even
then made clear his enduring mistrust of the United States. A few days
after Mr. Obama's highly publicized
\href{http://www.nytimes3xbfgragh.onion/2016/03/21/world/americas/obama-arrives-in-cuba.html}{visit
to Cuba} in 2016 --- the first by a sitting American president in 88
years --- Mr. Castro penned
\href{http://www.nytimes3xbfgragh.onion/2016/03/29/world/americas/fidel-castro-criticizes-barack-obamas-efforts-to-change-cuba.html}{a
cranky response} denigrating Mr. Obama's overtures of peace and
insisting that Cuba did not need anything the United States was
offering.

To many, Fidel Castro was a self-obsessed zealot whose belief in his own
destiny was unshakable, a chameleon whose economic and political colors
were determined more by pragmatism than by doctrine. But in his chest
beat the heart of a true rebel. ``Fidel Castro,'' said
\href{http://www.nytimes3xbfgragh.onion/1978/03/09/archives/henry-wriston-dies-brown-head-193755-educator-88-helped-to-draw-up.html}{Henry
M. Wriston}, the president of the Council on Foreign Relations in the
1950s and early '60s, ``was everything a revolutionary should be.''

Mr. Castro was perhaps the most important leader to emerge from Latin
America since the wars of independence in the early 19th century. He was
decidedly the most influential shaper of Cuban history since his own
hero,
\href{http://academic.eb.com/EBchecked/topic/366828/Jose-Julian-Marti}{José
Martí}, struggled for Cuban independence in the late 19th century. Mr.
Castro's revolution transformed Cuban society and had a longer-lasting
impact throughout the region than that of any other 20th-century Latin
American insurrection, with the possible exception of the
\href{http://www.nytimes3xbfgragh.onion/1988/03/13/books/history-out-of-chaos.html?pagewanted=all}{1910
Mexican Revolution}.

His legacy in Cuba and elsewhere has been a mixed record of social
progress and abject poverty, of racial equality and political
persecution, of medical advances and a degree of misery comparable to
the conditions that existed in Cuba when he entered Havana as a
victorious guerrilla commander in 1959.

That image made him a symbol of revolution throughout the world and an
inspiration to many imitators.
\href{http://www.nytimes3xbfgragh.onion/2013/03/06/world/americas/hugo-chavez-venezuelas-polarizing-leader-dies-at-58.html}{Hugo
Chávez} of Venezuela considered Mr. Castro his ideological godfather.
\href{http://www.nytimes3xbfgragh.onion/1994/02/21/world/how-a-revolution-survived-a-mexican-chronicle.html}{Subcommander
Marcos} began a revolt in the mountains of southern Mexico in 1994,
using many of the same tactics. Even Mr. Castro's spotty performance as
an aging autocrat in charge of a foundering economy could not undermine
his image.

But beyond anything else, it was Mr. Castro's obsession with the United
States, and America's obsession with him, that shaped his rule. After he
embraced Communism, Washington portrayed him as a devil and a tyrant and
repeatedly tried to remove him from power through an ill-fated invasion
at the Bay of Pigs in 1961, an
\href{http://www.treasury.gov/resource-center/sanctions/Programs/Pages/cuba.aspx}{economic
embargo} that has lasted decades,
\href{http://washington.blogs.nytimes3xbfgragh.onion/2007/06/26/a-plot-to-assassinate-castro-was-approved-by-cia-director-allen-dulles/}{assassination
plots} and even bizarre plans to undercut his prestige by making his
beard fall out.

Mr. Castro's defiance of American power made him a beacon of resistance
in Latin America and elsewhere, and his bushy beard, long Cuban cigar
and green fatigues became universal symbols of rebellion.

Mr. Castro's understanding of the power of images, especially on
television, helped him retain the loyalty of many Cubans even during the
harshest periods of deprivation and isolation when he routinely blamed
America and its embargo for many of Cuba's ills. And his mastery of
words in thousands of speeches, often lasting hours, imbued many Cubans
with his own hatred of the United States by keeping them on constant
watch for an invasion --- military, economic or ideological --- from the
north.

\href{https://www.nytimes3xbfgragh.onion/interactive/2016/11/26/world/americas/fidel-castro-cuban-posters.html}{}

\includegraphics{https://static01.graylady3jvrrxbe.onion/images/2016/05/04/world/americas/fidel-poster-promo/fidel-poster-promo-square640-v3.jpg}

\hypertarget{castros-revolution-illustrated}{%
\subsection{Castro's Revolution,
Illustrated}\label{castros-revolution-illustrated}}

Thousands of posters were commissioned by the government to promote his
vision of a socialist society.

Over many years Mr. Castro gave hundreds of interviews and retained the
ability to twist the most compromising question to his favor. In
\href{http://www.playboy.com/articles/50-years-of-the-playboy-interview-fidel-castro}{a
1985 interview} in Playboy magazine, he was asked how he would respond
to President Ronald Reagan's description of him as a ruthless military
dictator. ``Let's think about your question,'' Mr. Castro said, toying
with his interviewer. ``If being a dictator means governing by decree,
then you might use that argument to accuse the pope of being a
dictator.''

He turned the question back on Reagan: ``If his power includes something
as monstrously undemocratic as the ability to order a thermonuclear war,
I ask you, who then is more of a dictator, the president of the United
States or I?''

After leading his guerrillas against a repressive Cuban dictator, Mr.
Castro, in his early 30s, aligned Cuba with the Soviet Union and used
Cuban troops to support revolution in Africa and throughout Latin
America.

His willingness to allow the Soviets to build missile-launching sites in
Cuba led to a
\href{http://topics.nytimes3xbfgragh.onion/top/reference/timestopics/subjects/c/cuban_missile_crisis/index.html}{harrowing
diplomatic standoff} between the United States and the Soviet Union in
the fall of 1962, one that could have escalated into a nuclear exchange.
The world remained tense until the confrontation was defused 13 days
after it began, and the launching pads were dismantled.

With the
\href{http://www.nytimes3xbfgragh.onion/1991/09/06/world/soviet-turmoil-soviet-congress-yields-rule-republics-avoid-political-economic.html}{dissolution
of the Soviet Union} in 1991, Mr. Castro faced one of his biggest
challenges: surviving without huge Communist subsidies. He defied
predictions of his political demise. When threatened, he fanned
antagonism toward the United States. And when the Cuban economy neared
collapse, he
\href{http://www.nytimes3xbfgragh.onion/1994/08/28/world/flight-from-cuba-in-havana-dollars-define-cuba-s-haves-and-have-nots.html}{legalized
the United States dollar}, which he had railed against since the 1950s,
only to
\href{http://www.nytimes3xbfgragh.onion/2004/11/28/international/americas/28cuba.html}{ban
dollars} again a few years later when the economy stabilized.

Mr. Castro continued to taunt American presidents for a half-century,
frustrating all of Washington's attempts to contain him. After nearly
five decades as a pariah of the West, even when his once booming voice
had withered to an old man's whisper and his beard had turned gray, he
remained defiant.

He often told interviewers that he identified with Don Quixote, and like
Quixote he struggled against threats both real and imagined, preparing
for decades, for example, for another invasion that never came. As the
leaders of every other nation of the hemisphere gathered in Quebec City
in April 2001 for the third
\href{http://www.nytimes3xbfgragh.onion/2001/04/23/world/talks-tie-trade-in-the-americas-to-democracy.html}{Summit
of the Americas}, an uninvited Mr. Castro, then 74, fumed in Havana,
presiding over ceremonies commemorating the embarrassing defeat of
C.I.A.-backed exiles at the Bay of Pigs in 1961. True to character, he
portrayed his exclusion as a sign of strength, declaring that Cuba ``is
the only country in the world that does not need to trade with the
United States.''

\href{https://www.nytimes3xbfgragh.onion/slideshow/2016/11/26/world/americas/fidel-castro-cuban-revolutionary.html}{}

\hypertarget{fidel-castro-cuban-revolutionary}{%
\subsection{Fidel Castro, Cuban
Revolutionary}\label{fidel-castro-cuban-revolutionary}}

21 Photos

View Slide Show ›

\includegraphics{https://static01.graylady3jvrrxbe.onion/images/2015/06/20/world/Fidel-Castro-obituary-slide-HTON/Fidel-Castro-obituary-slide-HTON-articleLarge.jpg?quality=75\&auto=webp\&disable=upscale}

Paul Hosefros/The New York Times

\hypertarget{personal-powers}{%
\subsection{Personal Powers}\label{personal-powers}}

Fidel Alejandro Castro Ruz was born on Aug. 13, 1926 --- 1927 in some
reports --- in what was then the eastern Cuban province of Oriente, the
son of a plantation owner, Ángel Castro, and one of his maids,
\href{http://timesmachine.nytimes3xbfgragh.onion/timesmachine/1963/08/08/81821103.html?pageNumber=NaN\&zoom=16}{Lina
Ruz González}, who became his second wife and had seven children. The
father was a Spaniard who had arrived in Cuba under mysterious
circumstances. One account, supported by Mr. Castro himself, was that
his father had agreed to take the place of a Spanish aristocrat who had
been drafted into the Spanish Army in the late 19th century to fight
against Cuban independence and American hegemony.

Other versions suggest that Ángel Castro went penniless to Cuba but
eventually established a plantation and did business with the despised,
American-owned United Fruit Company. By the time Fidel was a youngster,
his father was a major landholder.

Fidel was a boisterous young student who was sent away to study with the
Jesuits at the Colegio de Dolores in Santiago de Cuba and later to the
Colegio de Belén, an exclusive Jesuit high school in Havana. Cuban lore
has it that he was headstrong and fanatical even as a boy. In one
account, Fidel was said to have bicycled head-on into a wall to make a
point to his friends about the strength of his will.

In another often-repeated tale, young Fidel and his class were led on a
mountain hike by a priest. The priest slipped in a fast-moving stream
and was in danger of drowning until Fidel pulled him to shore, then both
knelt in prayers of thanks for their good fortune.

A sense of destiny accompanied Mr. Castro as he entered the University
of Havana's law school in 1945 and almost immediately immersed himself
in radical politics. He took part in an invasion of the Dominican
Republic that unsuccessfully tried to oust the dictator
\href{http://www.britannica.com/EBchecked/topic/607139/Rafael-Trujillo}{Rafael
Trujillo}. He became increasingly obsessed with Cuban politics and led
student protests and demonstrations even when he was not enrolled in the
university.

Mr. Castro's university days earned him the image of rabble-rouser and
seemed to support the view that he had had Communist leanings all along.
But in an interview in 1981, quoted in
\href{http://www.nytimes3xbfgragh.onion/2001/05/22/world/tad-szulc-74-dies-times-correspondent-who-uncovered-bay-of-pigs-imbroglio.html}{Tad
Szulc}'s 1986 biography,
``\href{http://www.nytimes3xbfgragh.onion/1986/11/30/books/power-unshared-and-total.html}{Fidel},''
Mr. Castro said that he had flirted with Communist ideas but did not
join the party.

``I had entered into contact with Marxist literature,'' Mr. Castro said.
``At that time, there were some Communist students at the University of
Havana, and I had friendly relations with them, but I was not in the
Socialist Youth, I was not a militant in the Communist Party.''

He acknowledged that radical philosophy had influenced his character:
``I was then acquiring a revolutionary conscience; I was active; I
struggled, but let us say I was an independent fighter.''

After receiving his law degree, Mr. Castro briefly represented the poor,
often bartering his services for food. In 1952, he ran for Congress as a
candidate for the opposition Orthodox Party. But the election was
scuttled because of
\href{http://timesmachine.nytimes3xbfgragh.onion/timesmachine/1952/03/11/84244252.html?pageNumber=1}{a
coup} staged by Mr. Batista.

Mr. Castro's initial response to the Batista government was to challenge
it with a legal appeal, claiming that Mr. Batista's actions had violated
the Constitution. Even as a symbolic act, the attempt was futile.

\includegraphics{https://static01.graylady3jvrrxbe.onion/images/2015/10/09/world/castro-obit-web-3/castro-obit-web-3-articleInline.jpg?quality=75\&auto=webp\&disable=upscale}

His core group of radical students gained followers, and on
\href{http://query.nytimes3xbfgragh.onion/mem/archive-free/pdf?res=9806E6D81F3DE03ABC4F51DFB1668388649EDE}{July
26, 1953}, Mr. Castro led them in an attack on the Moncada barracks in
Santiago de Cuba. Many of the rebels were killed. The others were
captured, as were Mr. Castro and his brother Raúl. At his trial, Mr.
Castro defended the attack. Mr. Batista had issued an order not to
discuss the proceedings, but six Cuban journalists who had been allowed
in the courtroom recorded Mr. Castro's defense.

``As for me, I know that jail will be as hard as it has ever been for
anyone, filled with threats, with vileness and cowardly brutality,'' Mr.
Castro
\href{https://www.marxists.org/history/cuba/archive/castro/1953/10/16.htm}{declared}.
``I do not fear this, as I do not fear the fury of the miserable tyrant
who snuffed out the life of 70 brothers of mine. Condemn me, it does not
matter. History will absolve me.''

Mr. Castro was sentenced to 15 years in prison. Mr. Batista then made
what turned out to be a huge strategic error. Believing that the rebels'
energy had been spent, and under pressure from civic leaders to show
that he was not a dictator, he released Mr. Castro and his followers in
an amnesty after the 1954 presidential election.

Mr. Castro went into exile in Mexico, where he plotted his return to
Cuba. He tried to buy a used American PT boat to carry his band to Cuba,
but the deal fell through. Then he caught sight of a beat-up 61-foot
wooden yacht named Granma, once owned by an American who lived in Mexico
City.

The Granma remains on display in Havana, encased in glass.

Image

Mr. Castro with other rebel leaders at a secret base in June 1957
including Che Guevara, the guerrillas' physician, second from left, and
Mr. Castro's brother Raúl, kneeling in the foreground.Credit...United
Press International

\hypertarget{man-of-the-mountains}{%
\subsection{Man of the Mountains}\label{man-of-the-mountains}}

During Mr. Castro's long rule, his character and image underwent several
transformations, beginning with his days as a revolutionary in the
Sierra Maestra of eastern Cuba. After arriving on the coast in the
overloaded yacht with
\href{http://timesmachine.nytimes3xbfgragh.onion/timesmachine/1967/10/11/80715565.html?pageNumber=18\&rpm=true}{Che
Guevara} and 80 of their comrades in December 1956, Mr. Castro took on
the role of freedom fighter. He engaged in a campaign of harassment and
guerrilla warfare that infuriated Mr. Batista, who had seized power in a
1952 garrison revolt, ending a brief period of democracy.

Although his soldiers and weapons vastly outnumbered Mr. Castro's, Mr.
Batista grew fearful of the young guerrilla's mesmerizing oratory. He
ordered government troops not to rest until they had killed Mr. Castro,
and the army frequently reported that it had done so. Newspapers around
the world reported his death in the December 1956 landing. But three
months later, Mr. Castro was interviewed for a series of articles that
would revive his movement and thus change history.

The escapade began when Castro loyalists contacted a correspondent and
editorial writer for The New York Times, Herbert L. Matthews, and
arranged for him to interview Mr. Castro. A few Castro supporters took
Mr. Matthews into the mountains disguised as a wealthy American planter.

Drawing on his reporting, Mr. Matthews
\href{http://www.nytimes3xbfgragh.onion/packages/html/books/matthews/matthews022457.pdf}{wrote
sympathetically} of both the man and his movement, describing Mr.
Castro, then 30, parting the jungle leaves and striding into a clearing
for the interview.

``This was quite a man --- a powerful six-footer, olive-skinned,
full-faced, with a straggly beard,'' Mr. Matthews wrote.

The three articles, which began in The Times on Sunday, Feb. 24, 1957,
presented a Castro that Americans could root for. ``The personality of
the man is overpowering,'' Mr. Matthews wrote. ``Here was an educated,
dedicated fanatic, a man of ideals, of courage and of remarkable
qualities of leadership.''

\href{http://timesmachine.nytimes3xbfgragh.onion/timesmachine/1959/03/08/89158388.html?pageNumber=288\&rpm=true\&zoom=16}{The
articles} repeated Mr. Castro's assertions that Cuba's future was
anything but a Communist state. ``He has strong ideas of liberty,
democracy, social justice, the need to restore the Constitution, to hold
elections,'' Mr. Matthews wrote. When asked about the United States, Mr.
Castro replied, ``You can be sure we have no animosity toward the United
States and the American people.''

\href{https://www.nytimes3xbfgragh.onion/interactive/2016/11/26/world/americas/castro-archive-promo.html}{}

\includegraphics{https://static01.graylady3jvrrxbe.onion/images/2015/08/20/world/americas/castro-archive-promo-1459200755162/castro-archive-promo-1459200755162-largeHorizontalJumbo.png}

\hypertarget{the-timess-coverage-of-fidel-castro}{%
\subsection{The Times's Coverage of Fidel
Castro}\label{the-timess-coverage-of-fidel-castro}}

Articles include an exclusive Times interview with the young guerrilla
leader in 1957 and coverage of the revolution, the Bay of Pigs invasion
and the 1962 missile crisis.

The Cuban government denounced Mr. Matthews and called the articles
fabrications. But the news that he had survived the landing breathed
life into Mr. Castro's movement. His small band of irregulars skirmished
with government troops, and each encounter increased their support in
Cuba and around the world, even though other insurgent forces in the
cities were also fighting to overthrow the Batista government.

It was the symbolic strength of his movement, not the armaments under
Mr. Castro's control, that overwhelmed the government. By the time
\href{http://timesmachine.nytimes3xbfgragh.onion/timesmachine/1959/01/02/83432639.html?action=click\&contentCollection=Archives\&module=LedeAsset\&region=ArchiveBody\&pgtype=article\&pageNumber=1\&rpm=true}{Mr.
Batista fled} from a darkened Havana airport just after midnight on New
Year's Day 1959, Mr. Castro was already a legend. Competing opposition
groups were unable to seize power.

Events over the next few months became the catalyst for another
transformation in Mr. Castro's public image. More than 500 Batista-era
officials were brought before courts-martial and special tribunals,
summarily convicted and shot to death. The grainy black-and-white images
of the executions broadcast on American television horrified viewers.

Mr. Castro defended the executions as necessary to solidify the
revolution. He complained that the United States had raised not a
whimper when Mr. Batista had tortured and executed thousands of
opponents.

But to wary observers in the United States, the executions were a signal
that Mr. Castro was not the democratic savior he had seemed. In May
1959, he began
\href{http://timesmachine.nytimes3xbfgragh.onion/timesmachine/1959/05/14/80577418.html?action=click\&contentCollection=Archives\&module=ArticleEndCTA\&region=ArchiveBody\&pgtype=article\&pageNumber=10\&rpm=true}{confiscating
privately owned agricultural land}, including land
\href{http://query.nytimes3xbfgragh.onion/gst/abstract.html?res=9F02E5DB1F3CE63BBC4052DFB1668382649EDE}{owned
by Americans}, openly provoking the United States government.

In the spring of 1960, Mr. Castro ordered American and British
refineries in Cuba to
\href{http://query.nytimes3xbfgragh.onion/mem/archive-free/pdf?res=9402E5D9153DE333A25757C2A9639C946191D6CF}{accept
oil from the Soviet Union}. Under pressure from Congress, President
Dwight D. Eisenhower
\href{http://timesmachine.nytimes3xbfgragh.onion/timesmachine/1960/05/24/105436278.html?pageNumber=5\&rpm=true\&zoom=16}{cut
the American sugar quota} from Cuba, forcing Mr. Castro to look for new
markets. He turned to the Soviet Union for economic aid and political
support. Thus began a half-century of American antagonism toward Cuba.

Finally, in 1961, he gave the United States 48 hours to reduce the staff
of its embassy in Havana to 18 from 60. A frustrated Eisenhower
\href{http://query.nytimes3xbfgragh.onion/gst/abstract.html?res=9F02E2D81E31EE32A25757C0A9679C946091D6CF}{broke
off diplomatic relations} with Cuba and closed the embassy on the Havana
seacoast. The diplomatic stalemate lasted until 2015, when
\href{http://www.nytimes3xbfgragh.onion/2015/07/21/world/americas/cuba-us-embassy-diplomatic-relations.html}{embassies
were finally reopened} in both Havana and Washington.

During his two years in the mountains, Mr. Castro had sketched a social
revolution whose aim, at least on the surface, seemed to be to restore
the democracy that Mr. Batista's coup had stifled. Mr. Castro promised
free elections and vowed to end American domination of the economy and
the working-class oppression that he said it had caused.

Despite having a law degree, Mr. Castro had no real experience in
economics or government. Beyond improving education and reducing Cuba's
dependence on sugar and the United States, his revolution began without
a clear sense of the new society he planned, except that it would be
different from what had existed under Mr. Batista.

At the time, Cuba was a playground for rich American tourists and
gangsters where glaring disparities of wealth persisted, although the
country was one of the most economically advanced in the Caribbean.

After taking power in 1959, Mr. Castro put together a cabinet of
moderates, but it did not last long. He named
\href{http://www.nytimes3xbfgragh.onion/2001/03/09/world/felipe-pazos-88-economist-cuban-split-early-with-castro.html}{Felipe
Pazos}, an economist, president of the Banco Nacional de Cuba, Cuba's
central bank. But when Mr. Pazos openly criticized Mr. Castro's growing
tolerance of Communists and his failure to restore democracy, he was
dismissed. In place of Mr. Pazos, Mr. Castro named Che Guevara, an
Argentine doctor who knew nothing about monetary policy but whose
revolutionary credentials were unquestioned.

Opposition to the Castro government began to grow in Cuba, leading
peasants and anti-Communist insurgents to take up arms against it. The
\href{http://www.nytimes3xbfgragh.onion/1964/05/21/exiles-proclaim-anticastro-war-and-urge-revolt.html}{Escambray
Revolt}, as it was called, lasted from 1959 to 1965, when it was crushed
by Mr. Castro's army.

As the first waves of Cuban exiles arrived in Miami and northern New
Jersey after the revolution, many were intent on overthrowing the man
they had once supported. Their number would eventually total a million,
many from what had been, proportionately, the largest middle class in
Latin America.

The Central Intelligence Agency helped train an exile army to retake
Cuba by force. The army was to make a beachhead at the Bay of Pigs, a
remote spot on Cuba's southern coast, and instigate a popular
insurrection.

Mr. Szulc, then a correspondent for The Times, had picked up information
about the invasion, and had written an article about it.
\href{http://www.nytimes3xbfgragh.onion/times-insider/2014/12/26/1961-the-c-i-a-readies-a-cuban-invasion-and-the-times-blinks/}{But
The Times}, at the request of the Kennedy administration, withheld some
of what Mr. Szulc had found, including information that an attack was
imminent. Specific references to the C.I.A. were also omitted.

Ten days later, on April 17, 1961, 1,500
\href{http://query.nytimes3xbfgragh.onion/gst/abstract.html?res=9406E6D7133DE733A2575BC1A9629C946091D6CF}{Cuban
fighters landed} at the Bay of Pigs. Mr. Castro was waiting for them.
The invasion was badly planned and by all accounts doomed. Most of the
invaders were either captured or killed. Promised American air support
never arrived. The historian
\href{http://www.nytimes3xbfgragh.onion/2006/02/22/national/22DRAPER.html}{Theodore
Draper} called the
\href{http://www.nytimes3xbfgragh.onion/2001/03/23/world/bay-of-pigs-enemies-finally-sit-down-together.html}{botched
operation} ``a perfect failure,'' and the invasion aroused a distrust of
the United States that Mr. Castro exploited for political gain for the
rest of his life.

Image

Mr. Castro with Mr. Guevara in Havana in January 1959.Credit...Roberto
Salas

\hypertarget{declaration-or-deception}{%
\subsection{Declaration or Deception?}\label{declaration-or-deception}}

The C.I.A., fighting the Cold War, had acted out of worries about Mr.
Castro's increasingly open Communist connections. As he consolidated
power, even some of his most faithful supporters grew concerned. One
break had taken place as early as 1959.
\href{http://www.nytimes3xbfgragh.onion/2014/03/01/world/americas/huber-matos-comrade-of-castro-then-adversary-dies-at-95.html}{Huber
Matos}, who had fought alongside Mr. Castro in the Sierra Maestra,
resigned as military governor of Camagüey Province to protest the
Communists' growing influence as well as the appointment of Raúl Castro,
whose Communist sympathies were well known, as commander of Cuba's armed
forces. Suspecting an antirevolutionary plot, Fidel Castro had Mr. Matos
arrested and charged with treason.

Within two months, Mr. Matos was tried, convicted and sentenced to 20
years
\href{http://query.nytimes3xbfgragh.onion/gst/abstract.html?res=9500EFDE1F39E63BBC4F52DFB767838E669EDE}{in
prison}. When he was released in 1979, Mr. Matos, nearly blind, went
into exile in the United States, where he lived until his death in 2014.
Shortly after arriving in Miami and joining the legions of Castro
opponents there, Mr. Matos
\href{http://worldview.carnegiecouncil.org/archive/worldview/1980/04/3361.html/_res/id=sa_File1/v23_i004_a002.pdf}{told
Worldview magazine}: ``I differed from Fidel Castro because the original
objective of our revolution was `Freedom or Death.' Once Castro had
power, he began to kill freedom.''

It was not until just before the Bay of Pigs invasion that Mr. Castro
declared publicly that his revolution was socialist. A few months later,
on Dec. 2, 1961, he removed any lingering doubt about his loyalties when
he affirmed in a long speech, ``I am a Marxist-Leninist.''

Many Cubans who had willingly accepted great sacrifice for what they
believed would be a democratic revolution were dismayed. They broke
ranks with Mr. Castro, putting themselves and their families at risk.
Others, from the safety of the United States, publicly accused Mr.
Castro of betraying the revolution and called him a tyrant. Even his
family began to raise doubts about his intentions.

``As I listened, I thought that surely he must be a superb actor,'' Mr.
Castro's sister Juanita wrote in an account in Life magazine in 1964,
referring to the December 1961 speech. ``He had fooled not only so many
of his friends, but his family as well.'' She recalled his upbringing as
the son of a well-to-do landowner in eastern Cuba who had sent him to
exclusive Jesuit schools. In 1948, after Mr. Castro married Mirta
Díaz-Balart, whose family had ties to the Batista government, his father
gave them a three-month honeymoon in the United States.

``How could Fidel, who had been given the best of everything, be a
Communist?'' Juanita Castro wrote. ``This was the riddle which paralyzed
me and so many other Cubans who refused to believe that he was leading
our country into the Communist camp.''

\href{https://www.nytimes3xbfgragh.onion/interactive/2016/11/26/world/americas/fidel-castro-timeline.html}{}

\includegraphics{https://static01.graylady3jvrrxbe.onion/images/2015/01/09/world/09castro1_hp/09castro1_hp-videoLarge.jpg}

\hypertarget{fidel-castro-1926-2016}{%
\subsection{Fidel Castro: 1926-2016}\label{fidel-castro-1926-2016}}

A master of image and myth, Mr. Castro believed himself to be the
messiah of his fatherland, an indispensable force with authority from on
high to control Cuba and its people.

Although the young Fidel was deeply involved in a radical student
movement at the University of Havana, his early allegiance to Communist
doctrine was uncertain at best. Some analysts believed that the
obstructionist attitudes of American officials had pushed Mr. Castro
toward the Soviet Union.

Indeed, although Mr. Castro pursued ideologically communist policies, he
never established a purely Communist state in Cuba, nor did he adopt
orthodox Communist Party ideology. Rather, what developed in Cuba was
less doctrinaire, a tropical form of communism that suited his needs. He
centralized the economy and flattened out much of the traditional
hierarchy of Cuban society, improving education and health care for many
Cubans, while depriving them of free speech and economic opportunity.

But unlike other Communist countries, Cuba was never governed by a
functioning politburo; Mr. Castro himself, and later his brother Rául,
filled all the important positions in the party, the government and the
army, ruling Cuba as its maximum leader.

``The Cuban regime turns out to be simply the case of a third-world
dictator seizing a useful ideology in order to employ its wealth against
his enemies,'' wrote the columnist Georgie Anne Geyer, whose critical
\href{http://www.nytimes3xbfgragh.onion/1991/02/10/books/the-last-stalinist.html}{biography
of Mr. Castro} was published in 1991.

In this view of Mr. Castro, he was above all an old-style Spanish
caudillo, one of a long line of Latin American strongmen who endeared
themselves to people searching for leaders. The analyst Alvaro Vargas
Llosa of the \href{http://www.independent.org/}{Independent Institute}
in Washington called him ``the ultimate 20th-century caudillo.''

In Cuba, through good times and bad, Mr. Castro's supporters referred to
themselves not as Communists but as Fidelistas. He remained personally
popular among segments of Cuban society even after his economic policies
created severe hardship. As Mr. Castro consolidated power, eliminated
his enemies and grew increasingly autocratic, the Cuban people referred
to him simply as Fidel. To say ``Castro'' was considered disloyal,
although in later decades Cubans would commonly say just that and mean
it. Or they would invoke his overwhelming presence by simply bringing a
hand to their chins, as if to stroke a beard.

Image

A celebration in Santiago de Cuba on July 26, 1964, the anniversary of
the attack on the Moncada military barracks that started the Cuban
revolution.Credit...Grey Villet/The LIFE Images Collection, via Getty
Images

\hypertarget{global-brinkmanship}{%
\subsection{Global Brinkmanship}\label{global-brinkmanship}}

Mr. Castro's alignment with the Soviet Union meant that the Cold War
between the world's superpowers, and the ideological battle between
democracy and communism, had erupted in the United States' sphere of
influence. A clash was all but inevitable, and it came in
\href{http://www.nytimes3xbfgragh.onion/slideshow/2012/10/19/world/americas/cuban-missilecrisis-.html}{October
1962}. American spy planes took reconnaissance photos suggesting that
the Soviets had exploited their new alliance to build bases in Cuba for
intermediate-range nuclear missiles capable of reaching North America.

Mr. Castro allowed the bases to be constructed, but once they were
discovered, he became a bit player in the ensuing drama, overshadowed by
President John F. Kennedy and the Soviet leader,
\href{http://query.nytimes3xbfgragh.onion/gst/abstract.html?res=9C06E0D61E3FE63ABC4A52DFBF66838A669EDE}{Nikita
S. Khrushchev}. Kennedy put United States military forces on alert and
ordered a naval blockade of Cuba. The two sides were at a stalemate for
13 tense days, and the world held its breath.

Finally, after receiving assurances that the United States would remove
American missiles from Turkey and not invade Cuba, the Soviets withdrew
the missiles and dismantled the bases.

But the Soviet presence in Cuba continued to grow. Soviet troops,
technicians and engineers streamed in, eventually producing a generation
of blond Cubans with names like Yuri, Alexi and Vladimiro. The Soviets
were willing to buy all the sugar Cuba could produce. Even as other
Caribbean nations diversified, Cuba decided to stick with one major
crop, sugar, and one major buyer.

Image

Mr. Castro at an experimental cattle-breeding station in 1964. He tried
to develop a Cuban supercow that could produce milk at prodigious
rates.Credit...Jack Manning/The New York Times

But after forcing the entire nation into a failed effort to reach a
record 10-million-ton sugar harvest in 1970, Mr. Castro recognized the
need to break the cycle of dependence on the Soviets and sugar. Once
more, he relied on his belief in himself and his revolution for
solutions. One unlikely consequence was his effort to develop a Cuban
supercow. Although he had no training in animal husbandry, Mr. Castro
decided to crossbreed humpbacked
\href{http://www.britannica.com/EBchecked/topic/77082/Brahman}{Asian
Zebus} with standard Holsteins to create a new breed that could produce
milk at prodigious rates.

Decades later, the Zebus could still be found grazing in pastures across
the island, symbols of Mr. Castro's micromanagement. A few of the
hybrids did give more milk, and one that set a milk production record
was stuffed and placed in a museum. But most were no better producers
than their parents.

As the Soviets settled in Cuba in the 1960s, hundreds of Cuban students
were sent to Moscow, Prague and other cities of the Soviet bloc to study
science and medicine. Admirers from around the world, including some
Americans, were impressed with the way that health care and literacy in
Cuba had improved. A reshaping of Cuban society was underway.

Cuba's tradition of racial segregation was turned upside down as
peasants from the countryside, many of them dark-skinned descendants of
Africans enslaved by the Spaniards centuries before, were invited into
Havana and other cities that had been overwhelmingly white. They were
given the keys to the elegant homes and spacious apartments of the
middle-class Cubans who had fled to the United States. Rents came to be
little more than symbolic, and basic foods like milk and eggs were sold
in government stores at below production cost.

Mr. Castro's early overhauls also changed Cuba in ways that were less
than utopian. Foreign-born priests were exiled, and local clergy were
harassed so much that many closed their churches. The Roman Catholic
Church excommunicated Mr. Castro for violating a 1949 papal decree
against supporting Communism. He established a sinister system of local
\href{http://query.nytimes3xbfgragh.onion/gst/abstract.html?res=9E06E6DC1F38E73ABC4A52DFB667838E669EDE}{Committees
for the Defense of the Revolution,} which set neighbors to informing on
neighbors. Thousands of dissidents and homosexuals were rounded up and
sentenced to either prison or forced labor. And although blacks were
welcomed into the cities, Mr. Castro's government remained
overwhelmingly white.

Mr. Castro regularly fanned the flames of revolution with his oratory.
In marathon speeches, he incited the Cuban people by laying out what he
considered the evils of capitalism in general and of the United States
in particular. For decades, the regime controlled all publications and
broadcasting outlets and restricted access to goods and information in
ways that would not have been possible if Cuba were not an island.

His revolution established at home, Mr. Castro looked to export it.
Thousands of Cuban soldiers were sent to Africa to fight in
\href{http://query.nytimes3xbfgragh.onion/gst/abstract.html?res=9507E4DD1E3AEF33A25755C1A9679C946790D6CF}{Angola},
\href{http://query.nytimes3xbfgragh.onion/gst/abstract.html?res=9507E4DD1E3AEF33A25755C1A9679C946790D6CF}{Mozambique}
and
\href{http://query.nytimes3xbfgragh.onion/gst/abstract.html?res=9400E6DA1530E632A25756C2A9649C946990D6CF}{Ethiopia}
in support of Communist insurgents. The strain on Cuba's treasury and
its society was immense, but Mr. Castro insisted on being a global
player in the Communist struggle.

As potential threats to his rule were eliminated, Mr. Castro tightened
his grip.
\href{http://query.nytimes3xbfgragh.onion/mem/archive/pdf?res=9F0CE1DE1531EE3BBC4953DFB7678382649EDE}{Camilo
Cienfuegos}, who had led a division in the insurrection and was
immensely popular in Cuba, was killed in a plane crash days after going
to arrest Huber Matos in Camagüey on Mr. Castro's orders. His body was
never found. Che Guevara, who had become hostile toward the Soviet
Union, broke with Mr. Castro before going off to Bolivia, where he was
captured and killed in 1967 for trying to incite a revolution there.

Despite the fiery rhetoric from Mr. Castro in the early years of the
revolution, Washington did
\href{http://www.nytimes3xbfgragh.onion/2014/12/18/upshot/when-jfk-secretly-reached-out-to-castro.html}{attempt
a reconciliation}. By some accounts, in the weeks before he was
assassinated in 1963, Kennedy had aides look at mending fences,
providing Mr. Castro was willing to break with the Soviets.

Image

Marines helping a Cuban child off a boat in Key West, Fla., in May 1980.
Mr. Castro tried to defuse domestic discontent by allowing about 125,000
Cubans to flee in boats, rafts and inner tubes. He emptied prisons of
criminals and people with mental illnesses, forcing them to join the
exodus.Credit...Fernando Yovera/Associated Press

But with Kennedy's assassination, and suspicions that Mr. Castro and the
Cubans were somehow involved, the 90 miles separating Cuba from the
United States became a gulf of antagonism and mistrust. The C.I.A. tried
several times to eliminate Mr. Castro or undermine his authority. One
plot involved exposing him to a chemical that would cause his beard to
fall out, and another using a poison pen to kill him. Mr. Castro often
boasted of how many times he had escaped C.I.A. plots to kill him, and
he ordered information about the foiled attempts to be put on display at
a Havana museum.

Relations between the United States and Cuba
\href{http://query.nytimes3xbfgragh.onion/gst/abstract.html?res=9C07E1D61139E334BC4953DFB066838C669EDE}{briefly
thawed} in the 1970s during the administration of President Jimmy
Carter. For the first time, Cuban-Americans were allowed to visit family
in Havana under strict guidelines. But that fleeting détente ended in
1980 when Mr. Castro tried to defuse growing domestic discontent by
allowing about
\href{http://www.history.com/this-day-in-history/castro-announces-mariel-boatlift}{125,000
Cubans to flee} in boats, makeshift rafts and inner tubes, departing
from the beach at Mariel. He used the opportunity to empty Cuban prisons
of criminals and people with mental illnesses and force them to join the
Mariel boatlift. Mr. Carter's successor, Reagan, slammed shut the door
that Mr. Carter had opened.

In 1989, when frustrated veterans from Cuba's African ventures began
rallying around Gen. Arnaldo Ochoa, who led Cuban forces on the
continent, Mr. Castro effectively got rid of a potential rival by
bringing the general and some of his supporters
\href{http://www.nytimes3xbfgragh.onion/1989/06/27/world/raul-castro-adds-sparks-to-cuban-trial.html}{to
trial} on drug charges. General Ochoa and several other high-ranking
officers
\href{http://www.nytimes3xbfgragh.onion/1989/07/14/world/cuban-general-and-three-others-executed-for-sending-drugs-to-us.html}{were
executed} on the orders of Raúl Castro, who was then the minister of
defense.

The United States economic embargo, imposed by Eisenhower and widened by
Kennedy, has continued for more than five decades. But its effectiveness
was undermined by the Soviet Union, which gave Cuba \$5 billion a year
in subsidies, and later by Venezuela, which sent Cuba badly needed oil
and long-term economic support. Most other countries, including close
United States allies like Canada, maintained relations with Cuba
throughout the decades and continued trading with the island. In recent
years, successive American presidents have punched big holes in the
embargo, allowing a broad range of economic activity, though maintaining
the ban on tourism.

Image

Mr. Castro, speaking on July 26, 2003, lived to rule a country where the
overwhelming majority of people had never known any other
leader.Credit...Sven Creutzmann/Polaris

\hypertarget{end-of-an-empire}{%
\subsection{End of an Empire}\label{end-of-an-empire}}

``I faced my greatest challenge after I turned 60,'' Mr. Castro
\href{http://www.vanityfair.com/news/1994/03/fidel-castro-exclusive-interview}{said
in an interview with Vanity Fair magazine in 1994}. He was referring to
the collapse of the Soviet empire, which brought an end to the subsidies
that had kept his government afloat for so long. He had also lost a
steady source of oil and a reliable buyer for Cuban sugar.

Abandoned, isolated, facing increasing dissent at home, Mr. Castro
seemed to have come to the end of his line. Cuba's collapse appeared
imminent, and Mr. Castro's final hours in power were widely anticipated.
Miami exiles began making elaborate preparations for a triumphant
return.

But Mr. Castro, defying predictions, fought on. He chose an unlikely
weapon: the hated American dollar, which he had long condemned as the
corrupt symbol of capitalism. In the summer of 1993, he
\href{http://www.nytimes3xbfgragh.onion/1993/08/16/business/cuba-s-new-money-law.html}{made
it legal} for Cubans to hold American dollars spent by tourists or sent
by exiled family members. That policy eventually led to a dual currency
system that has fostered resentment and hampered economic development in
Cuba.

Mr. Castro, the self-proclaimed Marxist-Leninist, was also willing to
experiment with capitalism and free enterprise, at least for a time.
Encouraged by his brother Raúl, he allowed farmers to sell excess
produce at market rates, and he ordered officials to turn a blind eye to
small, family-run kitchens and restaurants, called paladares, that
charged market prices. Under Rául Castro, those reforms were broadened
considerably, though they were sometimes met with public grumbling from
his older brother.

But despite his apparent distaste for capitalism, and lingering memories
of the 1950s Cuba that preceded his rule, Fidel Castro continued to
foster Cuba's tourism industry. He allowed Spanish, Italian and Canadian
companies to develop resort hotels and vacation properties, usually in
association with an arm of the Cuban military.

For many years, the resorts were off limits to most Cubans. They
generated hard cash, but a new generation of struggling young Cuban
women were lured into prostitution by the tourists' money.

For a time, Mexican and Canadian investors poured money into the
decrepit telephone company (owned by ITT until it was nationalized by
Mr. Castro in 1960), mining operations and other enterprises, which
helped keep Cuba's economy from collapsing. He declared an emergency
during which he expected the Cuban people to tighten their belts. He
called the United States embargo genocide.

All his efforts were not enough to keep dissent from sprouting in
Havana, Santiago de Cuba and other urban areas during this period of
hardship. Despite worldwide condemnation of his actions, Mr. Castro
clamped down on a fledgling democracy movement, jailing anyone who dared
to call for free elections. He also cracked down on the nucleus of an
independent press, imprisoning or harassing Cuban reporters and editors.

In 1994, for the first time, demonstrators
\href{http://www.nytimes3xbfgragh.onion/1994/08/06/world/protesters-battle-police-in-havana-castro-warns-us.html}{took
to the streets of Havana} to express their anger over the failed
promises of the revolution. Mr. Castro had to personally appeal for
calm. Then, in early 1996, he seized an opportunity to rebuild his
support by again demonizing the United States.

Image

In 1994, demonstrators took to the streets of Havana to express their
anger over the failed promises of the revolution.Credit...Prensa Latina,
via Associated Press

A South Florida group,
\href{http://www.nytimes3xbfgragh.onion/1996/02/26/world/pilots-group-firm-foe-of-castro-ignored-risks.html}{Brothers
to the Rescue}, had been flying three civilian planes toward the Cuban
coast when two were shot down by Cuban military jets. Four men on board
were killed. Mr. Castro raged against Washington, maintaining that the
planes had violated Cuban airspace. American officials condemned the
attack.

Until then, President Bill Clinton had been moving discreetly but
steadily toward easing the United States embargo and re-establishing
some relations with Cuba. But in the wake of the attack, and the
virulent reaction from Cuban-Americans in Florida --- a state Mr.
Clinton considered important to his re-election bid --- he reluctantly
signed the
\href{http://www.nytimes3xbfgragh.onion/1995/09/22/world/bill-tighten-economic-embargo-cuba-passed-with-strong-support-house.html}{Helms-Burton
law}, which allowed the United States to punish foreign companies that
were using confiscated American property in Cuba.

The State Department's first warnings under the new law went to a
Canadian mining company that had taken over a huge nickel mine, and a
Mexican investment group that had purchased the Cuban telephone company.
Despite
\href{http://www.nytimes3xbfgragh.onion/1996/06/13/world/canada-and-mexico-join-to-oppose-us-law-on-cuba.html}{protests
from American allies}, the United States maintained the Helms-Burton law
as a weapon against Mr. Castro, although all its provisions have never
been carried out.

But in Cuba, the American actions reinforced Mr. Castro's complaints
about American arrogance and helped channel domestic dissent toward
Washington. One of his strengths as a communicator --- he considered
Reagan his only worthy competitor in that regard --- had always been to
transform his anger toward the United States into a rallying cry for the
Cuban people.

``We are left with the honor of being one of the few adversaries of the
United States,'' Mr. Castro told Maria Shriver of NBC in a 1998
interview. When Ms. Shriver asked him if that truly was an honor, he
answered, ``Of course.''

``For such a small country as Cuba to have such a gigantic country as
the United States live so obsessed with this island,'' he said, ``it is
an honor for us.''

Image

Mr. Castro in June 2006. His impact on Latin America and the Western
Hemisphere has the earmarks of lasting indefinitely.Credit...Alejandro
Ernesto/European Pressphoto Agency

\hypertarget{parallel-lives}{%
\subsection{Parallel Lives}\label{parallel-lives}}

As he grew older and grayer, Mr. Castro could no longer be easily linked
to the intense guerrilla fighter who had come out of the Sierra Maestra.
He rambled incoherently in his long speeches. He was rumored to be
suffering from various diseases. After 40 years, the revolution he
started no longer held promise, and Cubans by the thousands, including
many who had never known any other life but under Mr. Castro, risked
their lives trying to reach the United States on rafts, inner tubes and
even old trucks outfitted with floats.

Although the revolution lost its luster, what never diminished was Mr.
Castro's ability to confound American officials and to create situations
to seize the advantage of a particular moment.

That was evident early in 1998 when
\href{http://www.nytimes3xbfgragh.onion/1998/01/22/world/the-pope-in-cuba-the-overview-castro-and-cheering-crowds-greet-pope.html}{Pope
John Paul II visited Havana} and met with Mr. Castro. The meeting was
widely expected to be seen as a rebuke and an embarrassment to Mr.
Castro. The aging anti-Communist pontiff stood beside the aging
Communist leader, who had abandoned his military uniform for the
occasion in favor of a dark suit. The pope talked about human rights and
the lack of basic freedoms in Cuba. But he also called Washington's
embargo ``unjust and ethically unacceptable,'' allowing Mr. Castro to
\href{http://www.nytimes3xbfgragh.onion/1998/01/22/world/the-pope-in-cuba-the-implications-castro-s-spin-pope-is-on-our-side.html}{claim
a political if not a moral victory}.

Image

Pope John Paul II visited Havana in 1998. The pope spoke about the lack
of basic freedoms in Cuba, but also called the United States embargo
``unjust and ethically unacceptable.''Credit...Pool photo by Paul Hanna

The next year, Mr. Castro converted another conflict into an opportunity
to bolster his standing among his own people while infuriating the
United States. A young woman and her 5-year-old son were among more than
a dozen Cubans who had set out for Florida in a 17-foot aluminum boat.
The boat capsized and the woman drowned, but the boy,
\href{http://topics.nytimes3xbfgragh.onion/top/reference/timestopics/people/g/elian_gonzalez/index.html?8qa}{Elián
González}, survived two days in an inner tube before being picked up by
the United States Coast Guard and taken to Miami, where he was united
with relatives.

Later, however, the relatives
\href{http://www.nytimes3xbfgragh.onion/2000/04/14/us/elian-gonzalez-case-overview-cuban-s-family-defies-reno-court-issues-stay.html}{refused
to release} the boy when his father, in Cuba, demanded his return. The
standoff between the family and United States officials created the kind
of emotional and political drama that Mr. Castro had become a master at
manipulating for his own purposes.

Mr. Castro made the boy another symbol of American oppression, which
diverted attention from the deteriorating conditions in Cuba. After
several months, American agents seized the boy from his Miami relatives
and returned him to his father in Cuba, where he was greeted by Mr.
Castro.

\href{http://www.nytimes3xbfgragh.onion/2000/04/27/us/elian-gonzalez-case-counting-cost-castro-emerges-conflict-s-clear-winner.html}{That
episode} carried great significance for Mr. Castro in the way it echoed
one in his personal life.

Mr. Castro and his wife, Mirta Díaz-Balart, divorced in 1955, six years
after the birth of their son, Fidelito.

In 1956, when Mr. Castro and Ms. Díaz-Balart were both in Mexico, Mr.
Castro arranged to have the boy visit him before embarking on what he
said would be a dangerous voyage, which turned out to be his invasion of
Cuba. He promised to bring the boy back in two weeks, but it was a
trick. At the end of that period, Mr. Castro placed Fidelito in the
custody of a friend in Mexico City. He then sailed for Cuba with his
fellow rebels on the yacht Granma.

The boy's mother, with the help of her family and the Cuban Embassy in
Mexico City, found a team of professional kidnappers, who ambushed the
boy and his guardians in a park and carried him off. Ms. Díaz-Balart
took Fidelito to New York and enrolled him in a local school for a year.
But after Mr. Castro entered Havana and grabbed control of the
government, he persuaded his former wife to send the boy back. The
younger Mr. Castro lived in Cuba until, years later, he was sent to the
Soviet Union to study. He became a physicist, married a Russian woman
and eventually returned to Cuba, where
\href{http://www.nytimes3xbfgragh.onion/1981/11/20/nyregion/notes-on-people-a-castro-reappears-after-long-silence.html}{he
was named} head of Cuba's nuclear power program.

Details of Mr. Castro's personal life were always murky. He had no
formal home but lived in many different houses and estates in and around
Havana. He had relationships with several women, and only in his later
years was he willing to acknowledge that he had a relationship of more
than 40 years with Dalia Soto del Valle, who had rarely been seen in
public. (Whether they were legally married was not clear.)

The two had five sons --- Alexis, Alexander, Alejandro, Antonio and
Ángel --- all of whom live in Cuba. Mr. Castro also has a daughter,
Alina, a radio host in Miami, who bitterly
\href{http://www.nytimes3xbfgragh.onion/2002/03/22/us/on-the-air-in-miami-castro-s-rebellious-daughter.html}{attacked
her father on the air} for years.

Mr. Castro had stormy relations with many of his relatives both in Cuba
and the United States. He remained close to Celia Sánchez, a woman who
was with him in the Sierra Maestra and who looked after his schedule and
his archives devotedly, until she died in 1980. A sister, Ángela Castro,
died at 88 in Havana in February 2012, according to The Associated
Press, quoting her sister Juanita. And his elder brother Ramón died in
February 2016 at 91.

\href{https://www.nytimes3xbfgragh.onion/interactive/2016/03/19/world/americas/cuba-on-the-edge-of-change-photo-essay.html}{}

\includegraphics{https://static01.graylady3jvrrxbe.onion/images/2016/03/15/world/americas/cuba-template-slide-CH7H/cuba-template-slide-CH7H-videoLarge.jpg}

\hypertarget{cuba-on-the-edge-of-change}{%
\subsection{Cuba on the Edge of
Change}\label{cuba-on-the-edge-of-change}}

Photographs from a land of endless waiting and palpable erosion --- but
also, an uncanny openness among everyday people.

Outlasting all his enemies, Mr. Castro lived to rule a country where the
overwhelming majority of people had never known any other leader. Hardly
anyone talked openly of a time without him until the day, in 2001, when
he appeared to faint while giving a speech. Then, in 2004, he stumbled
while leaving a platform,
\href{http://www.nytimes3xbfgragh.onion/2004/10/23/international/americas/23cuba.html}{breaking
a kneecap} and reminding Cubans again of his mortality and forcing them
to confront their future.

As Mr. Castro and his revolution aged, Cuban dissidents grew bolder.
\href{http://www.nytimes3xbfgragh.onion/2003/05/20/world/dissident-accuses-cuba-of-manipulating-fear-of-us-invasion.html}{Oswaldo
Payá}, using a clause in the Cuban Constitution, collected thousands of
signatures in a petition demanding a referendum on free speech and other
political freedoms. (Mr. Payá died in a car crash in 2012.) Bloggers
wrote disparagingly of Mr. Castro and the regime, although most of their
missives could not be read in Cuba, where internet access was strictly
limited.

A group of Cuban women who called themselves the
\href{http://www.nytimes3xbfgragh.onion/video/multimedia/100000003389586/ladies-in-white-march-in-havana.html}{Ladies
in White} rallied on Sundays to protest the imprisonment of their
fathers, husbands and sons, whose pictures they carried on posters
inscribed with the number of years to which they were sentenced as
political prisoners.

After being made his brother's successor, Raúl Castro tried to control
the fragments of the revolution that remained after Fidel Castro fell
ill, including a close association with President Hugo Chávez of
Venezuela, who modeled himself after Fidel. (Mr. Chávez died in 2013.)

Image

With his brother Raúl Castro, then Cuba's defense minister in December
2003. Fidel Castro ceded much of his power to his brother in
2006.Credit...Adalberto Roque/Agence France-Presse --- Getty Images

Never as popular as his brother, Raúl Castro was considered a better
manager, and in some ways was seen as more conscious of the everyday
needs of the Cuban people, despite his reputation as the revolution's
executioner. One of his first moves as leader was to replace the grossly
overcrowded city buses, known as ``camels,'' with new ones, many
imported from China. He opened up the economy somewhat, allowing
entrepreneurs to start businesses, and he eased restrictions on
traveling, access to cellphones, computers and other personal items, and
the buying and selling of property.

Still, Raúl Castro came under mounting pressure from Cubans demanding
even more economic and political opportunity. He took more steps to open
the economy and, in so doing, dismantled parts of the socialist state
that his brother had defended for so long.

Lurking in the background as Raúl Castro embarked on that new course was
the brooding visage of Fidel, whose revolution has been seen as a
rebellion of one man. When President Obama and Raúl Castro
simultaneously
\href{http://www.nytimes3xbfgragh.onion/video/world/americas/100000003332576/raul-castro-on-restoration-of-diplomacy.html}{went
on TV} in their countries in 2014 to announce a prisoner exchange and
the first steps toward normalizing relations, Cubans and Americans alike
expected to hear Fidel either accepting or condemning the moves.

Six weeks after the deal was announced, Mr. Castro, or someone writing
in his name, finally
\href{http://www.nytimes3xbfgragh.onion/2015/01/28/world/americas/fidel-castro-breaks-silence-over-thaw-in-us-cuba-relations.html}{reacted}
in a way that combined his own bluster and his brother's new approach.

``I do not trust the politics of the United States, nor have I exchanged
a word with them, but this is not, in any way, a rejection of a peaceful
solution to conflicts,'' Mr. Castro wrote near the end of a rambling
letter to students on the commemoration of the 70th anniversary of his
own time at the University of Havana.

Sounding more like his brother than his old self, he backed any peaceful
attempts to resolve the problems between the two countries. He then took
one final swipe at his old nemesis.

\includegraphics{https://static01.graylady3jvrrxbe.onion/images/2016/04/20/world/20cuba-video/20cuba-video-videoSixteenByNine3000.jpg}

``The grave dangers that threaten humanity today have to give way to
norms that are compatible with human dignity,'' the letter said. ``No
country is excluded from such rights. With this spirit I have fought,
and will continue fighting, until my last breath.''

In April 2016, a frail Mr. Castro made what many thought would be his
last public appearance, at the Seventh Congress of the Cuban Communist
Party. Dressed in an incongruous blue tracksuit jacket, his hands at
times quivering and his once powerful voice reduced to a tinny squawk,
he expressed surprise at having survived to almost 90, and he
\href{http://www.nytimes3xbfgragh.onion/2016/04/20/world/americas/in-farewell-fidel-castro-urges-party-to-fulfill-vision-for-cuba.html}{bade
farewell to the party}, the political system and the revolutionary Cuba
he had created.

``Soon I will be like everybody else,'' Mr. Castro said. ``Our turn
comes to us all, but the ideas of Cuban communism will endure.''

No one is sure if the force of the revolution will dissipate without Mr.
Castro and, eventually, his brother. But Fidel Castro's impact on Latin
America and the Western Hemisphere has the earmarks of lasting
indefinitely. The power of his personality remains inescapable, for
better or worse, not only in Cuba but also throughout Latin America.

``We are going to live with Fidel Castro and all he stands for while he
is alive,'' wrote Mr. Matthews of The Times, whose own fortunes were
dimmed considerably by his connection to Mr. Castro, ``and with his
ghost when he is dead.''

Advertisement

\protect\hyperlink{after-bottom}{Continue reading the main story}

\hypertarget{site-index}{%
\subsection{Site Index}\label{site-index}}

\hypertarget{site-information-navigation}{%
\subsection{Site Information
Navigation}\label{site-information-navigation}}

\begin{itemize}
\tightlist
\item
  \href{https://help.nytimes3xbfgragh.onion/hc/en-us/articles/115014792127-Copyright-notice}{©~2020~The
  New York Times Company}
\end{itemize}

\begin{itemize}
\tightlist
\item
  \href{https://www.nytco.com/}{NYTCo}
\item
  \href{https://help.nytimes3xbfgragh.onion/hc/en-us/articles/115015385887-Contact-Us}{Contact
  Us}
\item
  \href{https://www.nytco.com/careers/}{Work with us}
\item
  \href{https://nytmediakit.com/}{Advertise}
\item
  \href{http://www.tbrandstudio.com/}{T Brand Studio}
\item
  \href{https://www.nytimes3xbfgragh.onion/privacy/cookie-policy\#how-do-i-manage-trackers}{Your
  Ad Choices}
\item
  \href{https://www.nytimes3xbfgragh.onion/privacy}{Privacy}
\item
  \href{https://help.nytimes3xbfgragh.onion/hc/en-us/articles/115014893428-Terms-of-service}{Terms
  of Service}
\item
  \href{https://help.nytimes3xbfgragh.onion/hc/en-us/articles/115014893968-Terms-of-sale}{Terms
  of Sale}
\item
  \href{https://spiderbites.nytimes3xbfgragh.onion}{Site Map}
\item
  \href{https://help.nytimes3xbfgragh.onion/hc/en-us}{Help}
\item
  \href{https://www.nytimes3xbfgragh.onion/subscription?campaignId=37WXW}{Subscriptions}
\end{itemize}
