Sections

SEARCH

\protect\hyperlink{site-content}{Skip to
content}\protect\hyperlink{site-index}{Skip to site index}

\href{https://www.nytimes3xbfgragh.onion/section/world/americas}{Americas}

\href{https://myaccount.nytimes3xbfgragh.onion/auth/login?response_type=cookie\&client_id=vi}{}

\href{https://www.nytimes3xbfgragh.onion/section/todayspaper}{Today's
Paper}

\href{/section/world/americas}{Americas}\textbar{}Can Oil Help Mexico
Withstand Trump's Attack on Trade? It's Hard to See How

\begin{itemize}
\item
\item
\item
\item
\item
\end{itemize}

Advertisement

\protect\hyperlink{after-top}{Continue reading the main story}

Supported by

\protect\hyperlink{after-sponsor}{Continue reading the main story}

\hypertarget{can-oil-help-mexico-withstand-trumps-attack-on-trade-its-hard-to-see-how}{%
\section{Can Oil Help Mexico Withstand Trump's Attack on Trade? It's
Hard to See
How}\label{can-oil-help-mexico-withstand-trumps-attack-on-trade-its-hard-to-see-how}}

\includegraphics{https://static01.graylady3jvrrxbe.onion/images/2016/11/28/world/28MEXICO1/28MEXICO1-articleLarge.jpg?quality=75\&auto=webp\&disable=upscale}

By
\href{https://www.nytimes3xbfgragh.onion/by/elisabeth-malkin}{Elisabeth
Malkin}

\begin{itemize}
\item
  Nov. 27, 2016
\item
  \begin{itemize}
  \item
  \item
  \item
  \item
  \item
  \end{itemize}
\end{itemize}

CIUDAD DEL CARMEN, Mexico --- The town that oil built is emptying out.

``For Sale'' signs are plastered on concrete-block houses and
sun-bleached bungalows alike. The idled oil workers who used to cluster
in the main square, hoping to pick up odd jobs, have moved on.

Here in Ciudad del Carmen, on the gulf coast of Mexico, even the
ironclad union positions are slipping away. Some roughnecks on the
offshore rigs of the national oil company, Pemex, have not worked in
months, and their voices are filled with anxiety.

``What do you think is going to happen?'' some ask.

Pemex has been
\href{http://www.nytimes3xbfgragh.onion/2013/12/13/world/americas/mexico-oil.html}{limping
along for years}, bleeding billions of dollars annually, saddled with
debt and struggling to maintain production as its giant oil fields in
the Gulf of Mexico run dry. Next year, it will pump less than two
million barrels a day, the lowest output since 1980.

Fixing the oil company was already at the top of Mexico's list of
priorities, the focus of a
\href{http://www.nytimes3xbfgragh.onion/2013/12/13/world/americas/mexico-oil.html}{long
debate over the fate} of one of its most important --- and troubled ---
national institutions.

Now, that mammoth undertaking has become all the more critical with the
United States' election of Donald J. Trump. As Mexicans steel themselves
for an American president who made upending his nation's relationship
with Mexico a cornerstone of his campaign, officials on this side of the
border have hastened to reassure the country that Mexico's economy is
sound.

If Mr. Trump goes through with his promises to renegotiate the North
American Free Trade Agreement, deport migrants and tax remittances to
pay for his border wall, Mexico will face severe economic shocks,
particularly to the vibrant manufacturing base, whose products replaced
oil as the country's main export years ago.

The Mexican peso remains at record lows. The central bank raised
interest rates this month, citing ``heightened uncertainty.'' And last
week, it cut growth forecasts for this year and next. The bank's
governor, Agustín Carstens, told a local radio station that
understanding the Trump administration's policies was ``like trying to
put a jigsaw puzzle together without having all the pieces.''

Many are pessimistic that the government can come up with a backup plan.
``Mexico lacks a credible Plan B to offset the anti-trade wave,''
analysts at Morgan Stanley warned in a recent note to investors.

The damage that Mr. Trump could inflict on the busy factories that ship
cars and computers to the United States has given a sharp urgency to
Mexico's efforts to jump-start parts of the economy that do not rely on
Nafta --- particularly its dilapidated oil industry.

\includegraphics{https://static01.graylady3jvrrxbe.onion/images/2016/11/28/world/28MEXICO2/28MEXICO2-articleLarge.jpg?quality=75\&auto=webp\&disable=upscale}

To that end, when José Antonio Meade, the finance minister, lists the
Mexican economy's strengths, he singles out the importance of the new
energy laws that broke the 75-year monopoly held by Pemex.

The laws, part of a package of economic overhauls that President Enrique
Peña Nieto pushed through Congress three years ago, allow for private
investment in Mexico's oil sector for the first time since foreign
companies were expelled in 1938.

Only days before the American presidential election, Pemex's chief
executive, José Antonio González Anaya, presented a timetable of
projects he expected to offer to potential partners and promised to
begin returning the national oil company to solvency.

He and the finance minister met with investors in New York this month to
argue that Mexico's economy was solid and that ``the oil sector will
continue to be an engine of national economic growth,'' according to a
joint statement from Pemex and the Finance Ministry. The pair followed
up with a visit to London.

Mr. Peña Nieto has pushed through other overhauls, including
\href{http://www.nytimes3xbfgragh.onion/2016/06/27/world/americas/mexico-teachers-protests-enrique-pena-nieto.html}{changes
in education},
\href{http://www.nytimes3xbfgragh.onion/2016/08/10/world/americas/mexicos-carlos-slim-helu.html?_r=0}{telecommunications},
taxes, electricity and finance, but they have yet to generate
significant economic growth. Most economists project that the economy
will expand by just over 2 percent this year.

The most radical of all these overhauls, though, was ending the monopoly
of Pemex, the country's largest company, and allowing it to seek capital
and technology from private companies. The measure struck at Mexico's
most enduring
\href{http://www.nytimes3xbfgragh.onion/2013/12/13/world/americas/mexico-oil.html}{symbol
of national sovereignty}, rejecting the long-held conviction that it
could develop its most valuable natural resource on its own.

``The only way to bring back production in the next five, six years is
to bring more investment to Pemex,'' said Juan Carlos Zepeda, the
president of the National Hydrocarbons Commission, Mexico's oil
regulator. ``There is no other way.''

But after the new energy laws were approved, the company stalled, the
promised joint ventures did not happen, and oil prices plunged.

Pemex reeled as its debt soared and production dropped. Falling oil
revenue means oil funds less than 20 percent of the government budget,
down from as much as 40 percent when prices were at their peak.

Image

Members of the marine wildlife conservation organization Sea Shepherd
monitored the fuel tanker Burgos as it continued to burn on Sept. 25, a
day after it erupted in flames off the coast of the port city of Boca
del Rio, Mexico.Credit...Felix Marquez/Associated Press

``The government was never prepared for sustained low oil prices,'' said
John Padilla, a managing director of IPD Latin America, an energy
consulting firm. ``They never saw a Pemex implosion in the way it
occurred.''

The president chose Mr. González Anaya, a Harvard-educated economist
known for efficiency, to take over at Pemex in February. He quickly
announced the first joint venture proposal: a deepwater oil field just
south of United States waters.

Experts believe that Mexico's untapped deepwater oil fields are its next
great prize. But they are risky and expensive, a concern at a time when
low oil prices have forced international oil companies to scrap many
planned investments.

Still, major companies like BP, Exxon Mobil, Chevron and Shell have
qualified to bid in a deepwater auction in December.

In an interview in his office at the top of the Pemex tower in Mexico
City, Mr. González Anaya warned not to expect too much.

``Some people have said to me, `Look, Pemex won't go back to producing
three million barrels.' Well, no,'' he said. ``That's a shame --- but
no. What I can say and demonstrate is the company's solidity.''

Not everyone is sure that companies will jump at the chance to team up
with Pemex.

``Two years ago, everybody wanted to partner with Pemex,'' Mr. Padilla
said. ``They were being courted like the homecoming queen. Fast-forward
two years later, and how can you go to your board and say, `Pemex is
good for the money'?''

Another question is whether the government can speed up the
transformation as a defense against Mr. Trump's promised policies. Even
if the government attracts private investment, the effect on production
could take years to materialize.

``They're not going to turn the economy around on energy reform,'' said
Jeremy M. Martin, an energy expert at the Institute of the Americas in
San Diego.

Image

Part of the vast network of installations at the Miguel Hidalgo
refinery, one of the largest refineries in Mexico.Credit...Janet Jarman
for The New York Times

In the meantime, Mexico must fix the company's many problems.

Pemex's rusty refineries operate at about 60 percent capacity, forcing
the country to import more than half of its gasoline. The company loses
billions of dollars every quarter, and it owes almost \$100 billion in
debt and an additional \$68 billion in pension liabilities. Budget cuts
have halted exploration for next year.

An explosion on a fuel tanker in September was the latest in a series of
fiery and often fatal accidents. Gangs routinely tap Pemex's pipelines
to steal gasoline, tipped off from inside the company.

The government continues to tax Pemex heavily, and the oil workers'
union --- an ally of Mr. Peña Nieto's Institutional Revolutionary Party
--- remains powerful.

Mr. González Anaya's first action after arriving at Pemex was to slash
the budget by 22 percent, halting expensive projects and cutting waste.

``We haven't finished,'' he said. Referring to his effort to rein in
overspending, he added, ``We continue, continue, continue.''

For decades, Pemex made many people very rich. The company granted
inflated contracts to local business executives who cultivated political
connections, according to interviews with contractors in Ciudad del
Carmen. Mayors in oil states demanded Pemex cash for public works.

``The budget has been converted into plunder,'' said Mariano Ruiz Funes,
a former Pemex chief of staff.

Analysts argue that Pemex may have to sell off parts of the company.

``We will see a much smaller Pemex in the years to come,'' Mr. Ruiz
Funes said, predicting a ``long and painful'' adjustment. ``Politically
it will be difficult.''

Mr. González Anaya is not prepared to make that decision.

Pemex ``is not just any company,'' he said. ``You can't ask a national
oil company to be Exxon.''

But in Ciudad del Carmen, the riches of that national oil company are
long gone. The city has lost about 23,000 jobs since the end of 2014.

``What we're living through in Carmen, we have never lived through
something like this in contemporary Mexico,'' said José Domingo
Berzunza, the economic development secretary for Campeche, the
surrounding state.

Advertisement

\protect\hyperlink{after-bottom}{Continue reading the main story}

\hypertarget{site-index}{%
\subsection{Site Index}\label{site-index}}

\hypertarget{site-information-navigation}{%
\subsection{Site Information
Navigation}\label{site-information-navigation}}

\begin{itemize}
\tightlist
\item
  \href{https://help.nytimes3xbfgragh.onion/hc/en-us/articles/115014792127-Copyright-notice}{©~2020~The
  New York Times Company}
\end{itemize}

\begin{itemize}
\tightlist
\item
  \href{https://www.nytco.com/}{NYTCo}
\item
  \href{https://help.nytimes3xbfgragh.onion/hc/en-us/articles/115015385887-Contact-Us}{Contact
  Us}
\item
  \href{https://www.nytco.com/careers/}{Work with us}
\item
  \href{https://nytmediakit.com/}{Advertise}
\item
  \href{http://www.tbrandstudio.com/}{T Brand Studio}
\item
  \href{https://www.nytimes3xbfgragh.onion/privacy/cookie-policy\#how-do-i-manage-trackers}{Your
  Ad Choices}
\item
  \href{https://www.nytimes3xbfgragh.onion/privacy}{Privacy}
\item
  \href{https://help.nytimes3xbfgragh.onion/hc/en-us/articles/115014893428-Terms-of-service}{Terms
  of Service}
\item
  \href{https://help.nytimes3xbfgragh.onion/hc/en-us/articles/115014893968-Terms-of-sale}{Terms
  of Sale}
\item
  \href{https://spiderbites.nytimes3xbfgragh.onion}{Site Map}
\item
  \href{https://help.nytimes3xbfgragh.onion/hc/en-us}{Help}
\item
  \href{https://www.nytimes3xbfgragh.onion/subscription?campaignId=37WXW}{Subscriptions}
\end{itemize}
