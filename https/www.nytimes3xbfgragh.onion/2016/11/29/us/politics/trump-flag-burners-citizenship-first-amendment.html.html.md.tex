Sections

SEARCH

\protect\hyperlink{site-content}{Skip to
content}\protect\hyperlink{site-index}{Skip to site index}

\href{https://www.nytimes3xbfgragh.onion/section/politics}{Politics}

\href{https://myaccount.nytimes3xbfgragh.onion/auth/login?response_type=cookie\&client_id=vi}{}

\href{https://www.nytimes3xbfgragh.onion/section/todayspaper}{Today's
Paper}

\href{/section/politics}{Politics}\textbar{}Trump Calls for Revoking
Flag Burners' Citizenship. Court Rulings Forbid It.

\url{https://nyti.ms/2ggoDIL}

\begin{itemize}
\item
\item
\item
\item
\item
\item
\end{itemize}

Advertisement

\protect\hyperlink{after-top}{Continue reading the main story}

Supported by

\protect\hyperlink{after-sponsor}{Continue reading the main story}

\hypertarget{trump-calls-for-revoking-flag-burners-citizenship-court-rulings-forbid-it}{%
\section{Trump Calls for Revoking Flag Burners' Citizenship. Court
Rulings Forbid
It.}\label{trump-calls-for-revoking-flag-burners-citizenship-court-rulings-forbid-it}}

\includegraphics{https://static01.graylady3jvrrxbe.onion/images/2016/11/30/us/30flag/30flag-articleLarge.jpg?quality=75\&auto=webp\&disable=upscale}

By \href{http://www.nytimes3xbfgragh.onion/by/charlie-savage}{Charlie
Savage}

\begin{itemize}
\item
  Nov. 29, 2016
\item
  \begin{itemize}
  \item
  \item
  \item
  \item
  \item
  \item
  \end{itemize}
\end{itemize}

WASHINGTON --- Since the terrorist attacks of Sept. 11, 2001,
politicians have periodically
\href{http://www.nytimes3xbfgragh.onion/2010/05/07/world/07rights.html}{announced
with fanfare} that they would
\href{http://www.cruz.senate.gov/files/documents/Bills/20140905_Expatriate_Terrorist_Act.pdf}{introduce
a bill} to strip the citizenship of Americans accused of terrorism. The
idea tends to attract brief attention, but fades away, in part because
the Supreme Court long ago ruled that the Constitution does not permit
the government to take a person's citizenship against his or her will.

But on Tuesday, President-elect Donald J. Trump revived the idea and
took it much further than the extreme case of a suspected terrorist. He
proposed that Americans who protest government policies by burning the
flag could lose their citizenship --- meaning, among other things, their
right to vote --- as punishment.

\begin{quote}
Nobody should be allowed to burn the American flag - if they do, there
must be consequences - perhaps loss of citizenship or year in jail!

--- Donald J. Trump (@realDonaldTrump)
\href{https://twitter.com/realDonaldTrump/status/803567993036754944?ref_src=twsrc\%5Etfw}{November
29, 2016}
\end{quote}

Mr. Trump wrote the post shortly after Fox News
aired\href{http://mms.tveyes.com/PlaybackPortal.aspx?SavedEditID=7e878902-48dd-4bc5-9726-266f8d700d30}{a
segment} about a
\href{http://www.nytimes3xbfgragh.onion/2016/11/28/us/hampshire-college-flag-veterans-protest.html}{dispute
at Hampshire College in Massachusetts}, which removed the American flag
from its campus flagpole after protests over his election victory;
during one demonstration, someone burned a flag.

Even if Mr. Trump were to persuade Congress to enact a criminal statute,
a dramatic shift in the balance between government power and individual
freedom, anyone convicted and sentenced could point to clear Supreme
Court precedents to make the case for a constitutional violation.

The obstacles include the precedent that the Constitution does not allow
the government to expatriate Americans against their will, through a
landmark 1967 case,
\href{https://scholar.google.com/scholar_case?case=2521246303796542623\&hl=en\&as_sdt=6\&as_vis=1\&oi=scholarr}{Afroyim
v. Rusk}. They also include a 1989 decision,
\href{https://www.law.cornell.edu/supremecourt/text/491/397}{Texas v.
Johnson}, in which the court struck down criminal laws banning flag
burning, ruling that the act was a form of political expression
protected by the First Amendment.

\href{https://www.law.georgetown.edu/faculty/cole-david-d.cfm}{David D.
Cole}, a Georgetown University law professor who co-wrote the Supreme
Court briefs in the flag-burning case and who is about to become
national legal director at the American Civil Liberties Union, said he
wondered if Mr. Trump's strategy was to goad people into burning flags
in order to ``marginalize'' the protests against him. But he also called
Mr. Trump's proposal ``beyond the pale.''

``To me it is deeply troubling that the person who is going to become
the most powerful government official in the United States doesn't
understand the first thing about the First Amendment --- which is you
can't punish people for expressing dissent --- and also doesn't seem to
understand that citizenship is a constitutional right that cannot be
taken away, period, under any circumstances,'' he said.

The 1967 case involving the stripping of citizenship traces back to a
1940 law that automatically revoked the citizenship of Americans who
took actions like voting in a foreign country's election or joining its
military.

The case centered on a man who had been born in Poland, became a
naturalized American citizen, and later went to Israel and voted in an
election there. When he subsequently tried to renew his American
passport, the State Department refused, saying he was no longer an
American citizen, and he sued.

\href{https://www.nytimes3xbfgragh.onion/interactive/2016/us/politics/donald-trump-administration.html}{}

\includegraphics{https://static01.graylady3jvrrxbe.onion/images/2016/11/11/us/politics/donald-trump-administration-1478905372015/donald-trump-administration-1478905372015-square640.jpg}

\hypertarget{donald-trumps-cabinet-is-complete-heres-the-full-list}{%
\subsection{Donald Trump's Cabinet Is Complete. Here's the Full
List.}\label{donald-trumps-cabinet-is-complete-heres-the-full-list}}

A list of appointees and nominees for top posts in the new
administration.

In a 5-to-4 ruling, the Supreme Court called citizenship and the rights
that stem from it ``no light trifle to be jeopardized any moment'' by
politicians' attempts to curtail it. The court said that the 14th
Amendment, which guarantees due process of law, does not empower the
government to ``rob'' someone's citizenship. Americans, the ruling
explained, can only lose their citizenship by voluntarily renouncing it.

``The very nature of our free government makes it completely incongruous
to have a rule of law under which a group of citizens temporarily in
office can deprive another group of citizens of their citizenship,''
Justice Hugo L. Black wrote.

In a case in 1980,
\href{https://supreme.justia.com/cases/federalhttps:/supreme.justia.com/cases/federal/us/444/252/case.html/us/444/252/case.html}{Vance
v. Terrazas}, the Supreme Court extended that precedent by a vote of 6
to 3. That case concerned a man who was born with both American and
Mexican citizenship, and who as a student took an oath of allegiance to
Mexico, renouncing his American citizenship in order to obtain a Mexican
citizenship document.

When the State Department said he had thus surrendered his American
citizenship, he sued. The court majority said he was still a citizen
because the government had to prove that he specifically intended to
relinquish it, rather than having said those words with a different
motivation, like fulfilling his desire to obtain the certificate.

The 1989 flag-burning case was also decided by a vote of 5 to 4. It
centered on a protester who had burned a flag outside the 1984
Republican National Convention in Dallas as part of a political
demonstration against Reagan administration policies. The protester,
Gregory Johnson, was charged under a state law that criminalized
desecrating the flag and appealed his conviction.

\href{https://www.nytimes3xbfgragh.onion/interactive/2016/11/11/us/politics/what-trump-wants-to-change.html}{}

\includegraphics{https://static01.graylady3jvrrxbe.onion/images/2016/11/11/us/politics/what-trump-wants-to-change-1479009739985/what-trump-wants-to-change-1479009739985-largeHorizontalJumbo.png}

\hypertarget{20-things-donald-trump-said-he-wanted-to-get-rid-of-as-president}{%
\subsection{20 Things Donald Trump Said He Wanted to Get Rid of as
President}\label{20-things-donald-trump-said-he-wanted-to-get-rid-of-as-president}}

Some of the parts of the government that Mr. Trump promised to dismantle
if he was elected.

The majority ruled that Mr. Johnson's act was symbolic speech protected
by the Constitution, effectively striking down state laws against flag
desecration across the country. In response, Congress swiftly enacted a
federal law against such desecration, but in 1990 the same five-justice
majority
\href{https://www.law.cornell.edu/supremecourt/text/496/310}{struck it
down}, too.

Just one of the justices who participated in the flag-burning cases,
Justice Anthony M. Kennedy, is still on the court today; he sided with
the majority that struck down the bans. Justice Antonin Scalia, who died
in February and whose seat Mr. Trump will get to fill because Republican
senators refused to hold a hearing for President Obama's nominee for the
vacancy, was also in the majority.

After the 1989 decision, supporters of a flag-burning ban tried to enact
an amendment to the Constitution to make an exception to the First
Amendment, but it twice
\href{http://www.nytimes3xbfgragh.onion/1989/10/20/us/senate-rejects-amendment-outlawing-flag-desecration.html}{fell
short in the Senate}.

The issue flared again a decade ago. In 2005, Hillary Clinton, a senator
from New York at that time, co-sponsored the
\href{https://www.congress.gov/bill/109th-congress/senate-bill/1911/cosponsors}{Flag
Protection Act}. Arguing that desecration of the symbol ``may amount to
fighting words or a direct threat to the physical and emotional
well-being'' of onlookers, the bill would have banned flag burning if
abusing the symbol was ``intended to incite a violent response rather
than make a political statement.''

The crafters of that bill sought to frame it as a compromise and an
alternative to an amendment, saying ``the Bill of Rights is a guarantee
of those freedoms and should not be amended in a manner that could be
interpreted to restrict freedom, a course that is regularly resorted to
by authoritarian governments which fear freedom and not by free and
democratic nations.''

But Congress did not act on the legislation. The following year, when
the Senate again tried to approve a constitutional amendment to empower
Congress to ban flag desecration and it
\href{http://www.nytimes3xbfgragh.onion/2006/06/27/washington/27cnd-flag.html}{fell
one vote short} of the necessary two-thirds majority, Mrs. Clinton was
among those who
\href{http://www.nytimes3xbfgragh.onion/2006/06/28/washington/28hillary.html}{voted
against that measure}.

Advertisement

\protect\hyperlink{after-bottom}{Continue reading the main story}

\hypertarget{site-index}{%
\subsection{Site Index}\label{site-index}}

\hypertarget{site-information-navigation}{%
\subsection{Site Information
Navigation}\label{site-information-navigation}}

\begin{itemize}
\tightlist
\item
  \href{https://help.nytimes3xbfgragh.onion/hc/en-us/articles/115014792127-Copyright-notice}{©~2020~The
  New York Times Company}
\end{itemize}

\begin{itemize}
\tightlist
\item
  \href{https://www.nytco.com/}{NYTCo}
\item
  \href{https://help.nytimes3xbfgragh.onion/hc/en-us/articles/115015385887-Contact-Us}{Contact
  Us}
\item
  \href{https://www.nytco.com/careers/}{Work with us}
\item
  \href{https://nytmediakit.com/}{Advertise}
\item
  \href{http://www.tbrandstudio.com/}{T Brand Studio}
\item
  \href{https://www.nytimes3xbfgragh.onion/privacy/cookie-policy\#how-do-i-manage-trackers}{Your
  Ad Choices}
\item
  \href{https://www.nytimes3xbfgragh.onion/privacy}{Privacy}
\item
  \href{https://help.nytimes3xbfgragh.onion/hc/en-us/articles/115014893428-Terms-of-service}{Terms
  of Service}
\item
  \href{https://help.nytimes3xbfgragh.onion/hc/en-us/articles/115014893968-Terms-of-sale}{Terms
  of Sale}
\item
  \href{https://spiderbites.nytimes3xbfgragh.onion}{Site Map}
\item
  \href{https://help.nytimes3xbfgragh.onion/hc/en-us}{Help}
\item
  \href{https://www.nytimes3xbfgragh.onion/subscription?campaignId=37WXW}{Subscriptions}
\end{itemize}
