Sections

SEARCH

\protect\hyperlink{site-content}{Skip to
content}\protect\hyperlink{site-index}{Skip to site index}

\href{https://www.nytimes3xbfgragh.onion/section/world/middleeast}{Middle
East}

\href{https://myaccount.nytimes3xbfgragh.onion/auth/login?response_type=cookie\&client_id=vi}{}

\href{https://www.nytimes3xbfgragh.onion/section/todayspaper}{Today's
Paper}

\href{/section/world/middleeast}{Middle East}\textbar{}`We Are Dead
Either Way': Agonizing Choices for Syrians in Aleppo

\url{https://nyti.ms/2hapqeB}

\begin{itemize}
\item
\item
\item
\item
\item
\end{itemize}

Advertisement

\protect\hyperlink{after-top}{Continue reading the main story}

Supported by

\protect\hyperlink{after-sponsor}{Continue reading the main story}

\hypertarget{we-are-dead-either-way-agonizing-choices-for-syrians-in-aleppo}{%
\section{`We Are Dead Either Way': Agonizing Choices for Syrians in
Aleppo}\label{we-are-dead-either-way-agonizing-choices-for-syrians-in-aleppo}}

\includegraphics{https://static01.graylady3jvrrxbe.onion/images/2016/12/11/world/ALEPPO/ALEPPO-articleInline.jpg?quality=75\&auto=webp\&disable=upscale}

By \href{http://www.nytimes3xbfgragh.onion/by/anne-barnard}{Anne
Barnard} and
\href{https://www.nytimes3xbfgragh.onion/by/hwaida-saad}{Hwaida Saad}

\begin{itemize}
\item
  Dec. 10, 2016
\item
  \begin{itemize}
  \item
  \item
  \item
  \item
  \item
  \end{itemize}
\end{itemize}

BEIRUT, Lebanon --- He had long been one of the more optimistic
anti-government residents of the besieged, rebel-held section of Aleppo,
trying to buoy others' spirits even as loyalist forces closed in. But as
the Syrian Army and allied militias took more and more territory in an
apparently decisive offensive during the week, Yasser Hmeish, an
accountant for the local medical council, grew frantic.

Soldiers seized his neighborhood on Wednesday while he worked at a
clinic blocks away. Several of Mr. Hmeish's neighbors were brought into
the clinic wounded, but died before he could ask what had happened to
his family.

``I don't know anything, anything about them,'' Mr. Hmeish said in an
audio message, in one of scores of exchanges we had with people inside
east Aleppo as the offensive unfolded. ``We are about to die or be
arrested.''

After years of bombing and months under siege, rebels had lost more than
three-quarters of their territory in eastern Aleppo by the end of the
week, throwing thousands of civilians and fighters into chaos. We
followed the events in real time from Beirut, monitoring social media
and talking via WhatsApp, Skype, telephone and other media with doctors,
fighters, housewives, local council members, antigovernment activists,
aid workers and others, including on the government side. All were
people we had gotten to know through years of covering Syria's bloody
civil war.

They confided their dilemmas over whether to flee to government-held
areas or stay put until the bitter end, a choice that split up many
families. They described escalating bombardment and attacks that have
killed women and children trying to reach safe ground. They revealed
deepening rifts among rebel groups, and between fighters and civilians,
over whether and how to surrender.

Some told us of men who, upon reaching government territory, were pulled
aside and detained --- in a country where torture is common --- or sent
to fight in the Army. Some said rebel fighters were stopping people from
leaving, while others said insurgents were helping them cross front
lines. Many simply begged us for help.

``This is our last S O S,'' said Mohammad al-Ahmad, a radiology nurse
whose hospital was taken over by government forces. He sent us text
messages from a makeshift basement clinic where, he said, only first aid
is available and serious injuries mean death.

``I saw doctors standing in front of victims crying,'' he said.

As a turning point in the war approaches, countries like the United
States, which have long demanded that President Bashar al-Assad step
down, are reduced to wrangling with Russia, Mr. Assad's main backer,
over how to protect or evacuate the civilians --- no one is sure how
many --- trapped in the shrinking enclave.

Last month, proposed evacuation deals fell through and government forces
began an all-out aerial and ground attack. At least 40,000 people have
fled to government areas during the offensive, but some have been unable
to, and others have fled deeper into rebel territory, fearing arrest on
the other side. Now, Russia has declared that anyone who does not leave
will be considered a terrorist and ``destroyed.''

The end game is starkly testing the competing narratives at the heart of
the conflict in Aleppo. Many on the eastern, rebel-held side say that
they have stayed so long because they reject and fear the government and
that its indiscriminate attacks are punishment for dissent. Government
officials and supporters say the offensive is liberating residents held
hostage by the insurgents, who indiscriminately shell government
districts.

Many people --- estimates range from Russia's tens of thousands to the
United Nations' 250,000 --- stayed for years when they could have left.
They built underground schools, hospitals and playgrounds; distributed
food; and organized a local council of rebels and civilians.

But as the government
\href{https://www.nytimes3xbfgragh.onion/2016/12/07/world/middleeast/syria-aleppo.html}{advanced}
this past week, all semblance of order was shattered. Even some die-hard
supporters of the rebellion fled, including hundreds of fighters. Three
local council members crossed to government territory and were denounced
by colleagues as traitors, only to be arrested, with pro-government
websites trumpeting the capture of members of a ``terrorist council.''

One woman said that she and 30 relatives had left their homes in the
Qaterji neighborhood, but that one of her sons, a fighter, had stayed
behind. She spoke on the condition that she be identified only by her
nickname, Umm Faisal, out of concern for her safety.

Local rebels, Umm Faisal said by phone, guided the group to avoid
snipers. At the front line, shelling erupted.

``We started running,'' she recalled. ``People left their luggage on the
ground, holding children's hands and running. Some were killed and
wounded.''

They made contact with government soldiers, who led hundreds of people
on foot through holes in walls, over sand and barrels and rubble.

On the way, Umm Faisal said, she saw soldiers loading trucks with goods
looted from houses, who smiled and greeted the travelers: ``Thank God
for your safety.''

At a reception center in the Jibreen district, Umm Faisal and her family
received blankets, bread and medicine while the authorities took their
identification. Men --- whether dutifully or sincerely --- chanted for
Mr. Assad and the army. Still, she said, many of them were detained, and
six of her relatives, men in their 20s, 30s and 40s, were held for Army
service.

Later, Umm Faisal said, she, her husband and two children took a taxi to
the rebel-held town of Marea, near the Turkish border. Along the
twisting route, they bribed guards at checkpoints and talked their way
out of detention by Kurdish militias.

Before she left the refugee center, Umm Faisal refused to appear on
state television, she said, because ``I couldn't lie on TV.''

Others, though, have been filmed thanking Mr. Assad, receiving aid bags
marked ``Help from Russia'' and saying rebels had earlier prevented them
from leaving.

One aid worker, who spoke on the condition of anonymity for fear of
repercussions, said it was unclear how freely civilians in government
territory were speaking. But the worker said one man had told her in
compelling detail that the rebels refused to let him leave for medical
treatment for his young daughter, who was rail thin from a stomach
ailment.

Separately, a nurse in eastern Aleppo, who spoke on the condition that
she not be named, told us that rebels had stopped her parents and others
from leaving the Bustan al-Qasr neighborhood. First, the fighters warned
their group that government troops would shoot them, the nurse said.
Then they threatened the travelers, asking why they wanted to ``go to
those who are bombing you.''

``We are dead either way,'' one civilian had replied, the nurse said.

Aid workers say the reclaimed areas of east Aleppo are largely destroyed
and empty: Very few people stayed as troops moved in.

The army took several hundred people back to one district, Hanano,
telling them to move into abandoned apartments. But with no electricity,
heat or water and many homes looted, some have asked to return to
shelters despite the cold, uncomfortable conditions there as aid
agencies scramble to catch up.

Those who have decided to stay inside eastern Aleppo face continued
airstrikes and shelling as people crowd into shrinking areas without
medical care.

Modar Shekho, a nurse, and his family ran from their house under fire,
several of his friends said. As they searched for shelter, a shell
killed his brother. As they looked for a place to bury him, the friends
said, another explosion killed his father, a teacher who had helped
arrange for students to take their state exams despite the siege.

It was the second time that Mr. Shekho had lost two members of his
family in a single day, according to friends; a brother and sister, a
doctor and nurse, died in a hospital bombing in 2013.

Given such suffering, rebel groups are arguing. A hard-line minority,
including the Qaeda-linked Levant Conquest Front, wants to continue the
fight, and the rest, including United States-backed groups, want to make
a deal to get civilians out. Some question whether they have already
waited too long.

Zohair al-Shimali, a longtime activist, said he and others who had led
street protests and later supported rebels had failed to deliver.

``People started to hate their lives because of us,'' he wrote in a text
message. ``Everything we've done is for nothing, we lost everything.''

He said that while some rebels were counseling ``patience,'' he was
ready for any deal, either for rebels to leave in exchange for ending
the bombing or to bus civilians out.

``People have the right to take their children wherever they want, we
can't take them as hostages,'' he said. ``We couldn't meet the promises
we made them.''

Infighting and mistrust among rebels have hastened their collapse.
During the week, Qaeda-linked fighters attacked United States-backed
groups and took their supplies.

Dr. Farida, a gynecologist who spoke on the condition that her last name
not be used, checked off the people around her who planned to leave.

``My only neighbor, my only aunt and uncle, my secretary, my maid, and
even my right hand at work are going to the regime areas,'' she wrote in
a text message. ``They are neutral and just want to live and eat and
drink, they don't care about the revolution and religion.

``I feel like I lost a piece of me,'' she wrote. ``I can't stop
crying.''

Advertisement

\protect\hyperlink{after-bottom}{Continue reading the main story}

\hypertarget{site-index}{%
\subsection{Site Index}\label{site-index}}

\hypertarget{site-information-navigation}{%
\subsection{Site Information
Navigation}\label{site-information-navigation}}

\begin{itemize}
\tightlist
\item
  \href{https://help.nytimes3xbfgragh.onion/hc/en-us/articles/115014792127-Copyright-notice}{©~2020~The
  New York Times Company}
\end{itemize}

\begin{itemize}
\tightlist
\item
  \href{https://www.nytco.com/}{NYTCo}
\item
  \href{https://help.nytimes3xbfgragh.onion/hc/en-us/articles/115015385887-Contact-Us}{Contact
  Us}
\item
  \href{https://www.nytco.com/careers/}{Work with us}
\item
  \href{https://nytmediakit.com/}{Advertise}
\item
  \href{http://www.tbrandstudio.com/}{T Brand Studio}
\item
  \href{https://www.nytimes3xbfgragh.onion/privacy/cookie-policy\#how-do-i-manage-trackers}{Your
  Ad Choices}
\item
  \href{https://www.nytimes3xbfgragh.onion/privacy}{Privacy}
\item
  \href{https://help.nytimes3xbfgragh.onion/hc/en-us/articles/115014893428-Terms-of-service}{Terms
  of Service}
\item
  \href{https://help.nytimes3xbfgragh.onion/hc/en-us/articles/115014893968-Terms-of-sale}{Terms
  of Sale}
\item
  \href{https://spiderbites.nytimes3xbfgragh.onion}{Site Map}
\item
  \href{https://help.nytimes3xbfgragh.onion/hc/en-us}{Help}
\item
  \href{https://www.nytimes3xbfgragh.onion/subscription?campaignId=37WXW}{Subscriptions}
\end{itemize}
