Sections

SEARCH

\protect\hyperlink{site-content}{Skip to
content}\protect\hyperlink{site-index}{Skip to site index}

\href{https://www.nytimes3xbfgragh.onion/section/us}{U.S.}

\href{https://myaccount.nytimes3xbfgragh.onion/auth/login?response_type=cookie\&client_id=vi}{}

\href{https://www.nytimes3xbfgragh.onion/section/todayspaper}{Today's
Paper}

\href{/section/us}{U.S.}\textbar{}Abortion Foes, Emboldened by Trump,
Promise `Onslaught' of Tough Restrictions

\url{https://nyti.ms/2he5gQT}

\begin{itemize}
\item
\item
\item
\item
\item
\item
\end{itemize}

Advertisement

\protect\hyperlink{after-top}{Continue reading the main story}

Supported by

\protect\hyperlink{after-sponsor}{Continue reading the main story}

\hypertarget{abortion-foes-emboldened-by-trump-promise-onslaught-of-tough-restrictions}{%
\section{Abortion Foes, Emboldened by Trump, Promise `Onslaught' of
Tough
Restrictions}\label{abortion-foes-emboldened-by-trump-promise-onslaught-of-tough-restrictions}}

\includegraphics{https://static01.graylady3jvrrxbe.onion/images/2016/12/12/us/12abortion2/12abortion2-articleInline.jpg?quality=75\&auto=webp\&disable=upscale}

By \href{http://www.nytimes3xbfgragh.onion/by/sabrina-tavernise}{Sabrina
Tavernise} and
\href{http://www.nytimes3xbfgragh.onion/by/sheryl-gay-stolberg}{Sheryl
Gay Stolberg}

\begin{itemize}
\item
  Dec. 11, 2016
\item
  \begin{itemize}
  \item
  \item
  \item
  \item
  \item
  \item
  \end{itemize}
\end{itemize}

COLUMBUS, Ohio --- Christina Hagan, the youngest woman in the Ohio
Legislature, received a surprise last week. The toughest piece of
abortion legislation in the country --- a bill she had championed for
years --- suddenly passed.

The measure, which would ban abortions after a heartbeat is detected, as
early as six weeks, was long presumed dead. But now that Donald J. Trump
is headed to the White House, the political winds have changed, and it
passed with overwhelming majorities.

So did a bill banning abortions after 20 weeks. Neither contains
exceptions for rape and incest. Now Gov. John Kasich --- a Republican
who is an ardent abortion opponent and onetime challenger to Mr. Trump
--- is weighing whether to sign one or both.

``President-elect Trump has drastically shifted the dynamics,'' said Ms.
Hagan, 28, a Republican who has served in the State House since 2011.
``I honestly could not have foreseen this victory a week or a month
ago.''

The effects of Mr. Trump's victory are only beginning to be felt. But
one of the biggest changes is playing out in abortion politics. From the
composition of the Supreme Court (Mr. Trump has promised to nominate
staunchly anti-abortion justices), to efforts on Capitol Hill to enact a
permanent ban on taxpayer-financed abortions, to emboldened Republican
statehouses like the one in Ohio, combatants on both sides see legalized
abortion imperiled as it has not been for decades.

That includes, they agree, the possibility of overturning Roe v. Wade,
the landmark 1973 abortion decision, during the Trump presidency.

``This is the strongest the pro-life movement has been since 1973,''
said Marjorie Dannenfelser, the president of the Susan B. Anthony List,
an anti-abortion group, and the chairwoman of a coalition of abortion
opponents that worked to elect Mr. Trump. ``We are dealing now with a
president who has not been playing the game in the way that other
presidents, including Republicans, have.''

Mr. Trump was elected after a decade of anti-abortion gains in state
legislatures; Ohio is the 18th state to adopt a 20-week abortion ban,
though two such bills, in Arizona and Idaho, did not survive
constitutional challenges in federal court. States that preserve rights,
like New York and California, are increasingly rare.

``I think we are standing on the precipice of a really dark time,'' said
Ilyse Hogue, the president of Naral Pro-Choice America. She said that
Mr. Trump had ``zero mandate'' to roll back Roe, and that her group
would fight back hard; its fund-raising and membership are up.

On Monday, the anti-abortion group \href{http://www.aul.org/}{Americans
United for Life} is scheduled to release a 135-page report describing
what it calls ``horrific abortion clinic conditions'' in 32 states.
Clarke Forsythe, the group's acting president, said the report was
intended to be ``an inspiration to state legislators'' to enact new
restrictions, and as a ``rebuke to the Supreme Court's tragic decision''
to strike down a far-reaching Texas anti-abortion law in June.

\includegraphics{https://static01.graylady3jvrrxbe.onion/images/2016/12/12/us/12abortion1/12abortion1-articleInline.jpg?quality=75\&auto=webp\&disable=upscale}

In Texas, where abortion foes are still bruised by
\href{https://www.nytimes3xbfgragh.onion/2016/06/28/us/supreme-court-texas-abortion.html}{that
ruling, S}tate Representative Jonathan Stickland has vowed
\href{https://www.texastribune.org/2016/06/27/whats-next-texas-abortion-restrictions/}{``an
absolute onslaught of pro-life legislation''} in 2017. He said Texas
also might adopt a heartbeat bill.

Four states --- Louisiana, Mississippi, North Dakota and South Dakota
--- have adopted ``trigger bans'' that would automatically make abortion
a crime if the Supreme Court overturned Roe v. Wade, leaving it to the
states to decide on the legality of abortions. Mr. Strickland predicted
that states would start ``filling up the pipeline'' with anti-abortion
bills.

Others are more cautious.

``Instead of getting swept up in Trumpmania, let's be realistic and
continue the incremental approach,'' said Michael Gonidakis, president
of Ohio Right to Life, who pushed for Ohio's 20-week ban. ``Just because
a Republican president appoints a justice doesn't mean that the court
will fall our way.''

In Tennessee, a grand jury recently issued
\href{http://www.nytimes3xbfgragh.onion/2016/11/29/us/tennessee-woman-accused-of-coat-hanger-abortion-faces-new-charges.html}{new
felony charges} against a woman charged with trying to abort her
24-week-old fetus with a coat hanger --- a case that abortion-rights
advocates are citing as a throwback to the era of back-alley abortions.

Only four of the state's 95 counties have abortion clinics. Now, said
Allison Glass, state director of Healthy and Free Tennessee, a coalition
of reproductive rights groups, advocates fear ``a really extreme
abortion ban.''

Mr. Trump's emergence as a potent ally of the anti-abortion movement is
a surprise. Ms. Dannenfelser said she opposed him during the Republican
primaries, but mobilized 800 volunteers to work for Mr. Trump in the
general election after he committed in writing to her group's four top
priorities ---
\href{https://www.donaldjtrump.com/press-releases/trump-campaign-announces-national-co-chairs-of-pro-life-coalition}{a
letter} she called ``probably the most valuable piece of paper we'll
ever have.''

Those priorities include putting anti-abortion justices on the Supreme
Court; passing a national 20-week ban like Ohio's; eliminating federal
money for Planned Parenthood as long as its clinics perform abortions;
and making permanent the Hyde Amendment, passed annually by Congress to
ban taxpayer-funded abortions.

``This is an attack on every single front,'' said Representative Jan
Schakowsky, Democrat of Illinois, a longtime defender of abortion
rights. She predicted a ``fierce battle,'' beginning the day after Mr.
Trump's inauguration, when thousands of women are expected to march on
Washington.

In Ohio, just about everyone seems surprised that the so-called
heartbeat bill passed. It would ban abortions after a fetal heartbeat
was detected, usually between six and eight weeks, which is often before
a woman even knows she is pregnant and is far earlier than any law on
the books in the United States.

Mainstream anti-abortion advocates opposed it on the grounds that it
would never survive a court challenge (similar laws were struck down in
North Dakota and Arkansas) and could mean a setback for their broader
strategy of weakening one of the pillars of American abortion
jurisprudence: the definition of medical viability of the fetus,
currently understood to be around 24 weeks.

``Every political adviser I spoke to told me not to carry this,'' Ms.
Hagan said on Thursday, the last day of the state's legislative session.
``They said: `That's a lightning rod. Don't touch it.'''

Image

The check-in area of Preterm, in suburban Cleveland, the largest
abortion provider in Ohio. Its phones have been ringing with people
asking questions about the state's abortion rules.Credit...Ty Wright for
The New York Times

For years it foundered. Janet Porter, president of Faith 2 Action, an
anti-abortion group, tried many things to get it passed. She sent
lawmakers teddy bears, roses and heart-shaped balloons. She reserved
parts of the Statehouse for prayer meetings and rallies. She rented a
small plane to fly overhead and drag a banner urging passage. Nothing
worked.

But on Tuesday, the bill popped up on the schedule, folded into other
legislation at the last minute. Opponents were shocked.

``There were no hearings. No legislative process at all,'' said Gabriel
Mann, the communications manager for Naral Pro-Choice Ohio. ``Just all
of a sudden, hey, this bill is back. And boom, it passed.''

Even supporters were surprised.

``All of a sudden we discovered that, yes, lo and behold, we have
life,'' Timothy E. Ginter, a Republican state representative, said. The
first he heard the bill would be considered, he said, was the day it
passed. ``The heartbeat bill had a heartbeat,'' he added.

Mr. Kasich has signed 17 anti-abortion measures since taking office in
2011, Mr. Mann said. During that time, the number of abortion providers
in the state has been cut to nine from 16.

Both bills, though overlapping, in theory could be signed into law. But
many abortion supporters say the heartbeat provision is less likely to
survive an immediate legal challenge, and thus a harder sell for Mr.
Kasich.

Clinics are worried. On a gray Friday outside Preterm, the largest
abortion provider in Ohio, which is in suburban Cleveland, a protester
yelled at people going inside. A few placards with pictures of babies
were propped up against metal lawn chairs dusted with snow.

``Our phones have been ringing off the hook, primarily with the simple
question: Is abortion still legal in Ohio?'' said Nancy R. Starner,
director of development and communications for Preterm. She said the
calls had started the day after the heartbeat bill passed.

Angel Rucker, director of clinical services for Preterm, said 20 weeks
was roughly the time when ultrasounds determined fetal anomalies.

``It's very scary,'' she said of the new legislation. ``I'm quite
concerned for women.''

After winning election, Mr. Trump told
\href{http://www.nytimes3xbfgragh.onion/2016/11/14/us/politics/donald-trump-twitter-white-house.html}{``60
Minutes''} that if Roe were overturned, some women seeking abortions
``might have to go to another state.'' But Peggy B. Lehner, a Republican
state senator in Ohio and the principal sponsor of the legislation
calling for a 20-week abortion ban, said that would ``create chaos.''
She wants abortion banned outright. But the closer that moment comes,
she said, the fiercer the abortion rights resistance will get.

``Abortion has been part of our culture for the last 50 years,'' Ms.
Lehner said. ``It's not just going to go away.''

Advertisement

\protect\hyperlink{after-bottom}{Continue reading the main story}

\hypertarget{site-index}{%
\subsection{Site Index}\label{site-index}}

\hypertarget{site-information-navigation}{%
\subsection{Site Information
Navigation}\label{site-information-navigation}}

\begin{itemize}
\tightlist
\item
  \href{https://help.nytimes3xbfgragh.onion/hc/en-us/articles/115014792127-Copyright-notice}{©~2020~The
  New York Times Company}
\end{itemize}

\begin{itemize}
\tightlist
\item
  \href{https://www.nytco.com/}{NYTCo}
\item
  \href{https://help.nytimes3xbfgragh.onion/hc/en-us/articles/115015385887-Contact-Us}{Contact
  Us}
\item
  \href{https://www.nytco.com/careers/}{Work with us}
\item
  \href{https://nytmediakit.com/}{Advertise}
\item
  \href{http://www.tbrandstudio.com/}{T Brand Studio}
\item
  \href{https://www.nytimes3xbfgragh.onion/privacy/cookie-policy\#how-do-i-manage-trackers}{Your
  Ad Choices}
\item
  \href{https://www.nytimes3xbfgragh.onion/privacy}{Privacy}
\item
  \href{https://help.nytimes3xbfgragh.onion/hc/en-us/articles/115014893428-Terms-of-service}{Terms
  of Service}
\item
  \href{https://help.nytimes3xbfgragh.onion/hc/en-us/articles/115014893968-Terms-of-sale}{Terms
  of Sale}
\item
  \href{https://spiderbites.nytimes3xbfgragh.onion}{Site Map}
\item
  \href{https://help.nytimes3xbfgragh.onion/hc/en-us}{Help}
\item
  \href{https://www.nytimes3xbfgragh.onion/subscription?campaignId=37WXW}{Subscriptions}
\end{itemize}
