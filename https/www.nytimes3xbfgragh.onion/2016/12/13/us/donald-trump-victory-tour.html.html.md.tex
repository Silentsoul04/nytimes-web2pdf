Sections

SEARCH

\protect\hyperlink{site-content}{Skip to
content}\protect\hyperlink{site-index}{Skip to site index}

\href{https://www.nytimes3xbfgragh.onion/section/us}{U.S.}

\href{https://myaccount.nytimes3xbfgragh.onion/auth/login?response_type=cookie\&client_id=vi}{}

\href{https://www.nytimes3xbfgragh.onion/section/todayspaper}{Today's
Paper}

\href{/section/us}{U.S.}\textbar{}From Kanye West to the Midwest, Donald
Trump's Presidential Reality Show Rolls On

\url{https://nyti.ms/2hDbcE4}

\begin{itemize}
\item
\item
\item
\item
\item
\end{itemize}

Advertisement

\protect\hyperlink{after-top}{Continue reading the main story}

Supported by

\protect\hyperlink{after-sponsor}{Continue reading the main story}

\hypertarget{from-kanye-west-to-the-midwest-donald-trumps-presidential-reality-show-rolls-on}{%
\section{From Kanye West to the Midwest, Donald Trump's Presidential
Reality Show Rolls
On}\label{from-kanye-west-to-the-midwest-donald-trumps-presidential-reality-show-rolls-on}}

\includegraphics{https://static01.graylady3jvrrxbe.onion/images/2016/12/14/us/14TRUMP/14TRUMP-articleInline.jpg?quality=75\&auto=webp\&disable=upscale}

By \href{http://www.nytimes3xbfgragh.onion/by/mark-landler}{Mark
Landler}

\begin{itemize}
\item
  Dec. 13, 2016
\item
  \begin{itemize}
  \item
  \item
  \item
  \item
  \item
  \end{itemize}
\end{itemize}

WEST ALLIS, Wis. --- President-elect Donald J. Trump said on Tuesday
that his nominee for secretary of state, Rex W. Tillerson, would reverse
``years of foreign policy blunders and disasters,'' and that his close
ties to antagonistic foreign leaders --- a potential hurdle to his
confirmation by the Senate --- was one of the reasons he had picked him.

``Rex is friendly with many of the leaders in the world who we don't get
along with,'' Mr. Trump said to a boisterous crowd here. ``And some
people don't like that. They don't want him to be friendly.'' But he
added, ``That's why I'm doing the deal with Rex, because I like what
this is all about.''

Mr. Trump did not refer specifically to Mr. Tillerson's longstanding
ties to President Vladimir V. Putin of Russia, a relationship he
cultivated as chief executive of Exxon Mobil. But the president-elect
described Mr. Tillerson as a ``great diplomat, a strong man, a tough
man,'' noting that he had been endorsed by Republican foreign-policy
elders including James A. Baker III, Robert M. Gates and Condoleezza
Rice.

``People are looking at this résumé and honestly, they've never seen a
résumé like this before,'' he said.

Mr. Trump's visit to Wisconsin was the latest stop on his ``thank-you''
tour of battleground states, and one of the most resonant, given that
his slim victory in this previously reliable Democratic bastion was so
unexpected, and for Hillary Clinton, so devastating.

During his remarks, Mr. Trump savored the results of a recount that
reaffirmed his victory over Mrs. Clinton by close to 23,000 votes. ``I
refuse to say it was a scam tonight,'' he said, before saying that
Democrats were behind the recount campaign.

The rally also showed how victory can heal rifts: Gov. Scott Walker, a
bitter primary rival of Mr. Trump's who offered only a tepid endorsement
and steered clear of him during the campaign, introduced the
president-elect as a stalwart leader who had nominated Gen. James N.
Mattis to serve as defense secretary.

The House speaker, Paul D. Ryan, who initially held off on endorsing Mr.
Trump, thanked him for the Republican victory here. Mr. Trump said he
looked forward to working with Mr. Ryan to pass his legislative agenda,
but added rather pointedly, ``We're going to build the wall, Paul.''

Mr. Trump also lashed out at a familiar litany of foes, including the
news media and Mrs. Clinton, drawing cries of ``CNN sucks'' and ``Lock
her up,'' neither of which he tried very hard to dampen.

The rally came after a theatrical day of meetings at Trump Tower in
Midtown Manhattan, with visitors including Kanye West, Bill Gates and
the retired football stars Jim Brown and Ray Lewis. There were cameos by
familiar bit players: the Naked Cowboy and a man who paces through the
lobby reading aloud from books like ``Night,'' by Elie Wiesel.

The scene showcased yet again how Mr. Trump has turned his transition
into a kind of political reality show, with twists and turns, special
guests and a running narrative built around his own celebrity.

\includegraphics{https://static01.graylady3jvrrxbe.onion/images/2016/12/14/watching/14trump2/14trump2-articleInline.jpg?quality=75\&auto=webp\&disable=upscale}

Since soon after he was elected, he has summoned people to interview for
cabinet jobs --- a public spectacle that includes a ritual parade before
a phalanx of reporters and cameras. But on Tuesday, with Mr. Tillerson's
nomination, he ended one of the longest-running personnel dramas.

With many of the major cabinet posts filled, Mr. Trump's guest list has
broadened to include people who are simply there to advise him, talk
about big ideas or perhaps run a project by him. Sometimes, as with Mr.
West, the purpose of the visit is not all that clear.

Shortly before 10 a.m., the polymorphous hip-hop star and the
president-elect emerged from the gold elevators in the building's lobby
to pose for photographers and to engage in a stilted question-and-answer
session with the scrum of waiting reporters.

Asked the reason for their meeting, Mr. Trump said: ``Just friends, just
friends. He's a good man. Long time. Friends for a long time.'' Mr. West
stood silently next to him, arms crossed over his chest. The two
continued to pose --- Mr. Trump smiling; Mr. West glowering --- while
reporters tried to extract a comment from the normally opinionated
musician.

``I'm just here to take a picture,'' Mr. West said, before cracking a
smile and offering a thumbs-up gesture, as Mr. Trump grabbed his hand
and patted him on the shoulders to say goodbye.

Hours later, Mr. West wrote on Twitter that he had come to discuss
``multicultural issues'' with Mr. Trump, believing ``it is important to
have a direct line of communication with our future President if we
truly want change.'' Mr. Trump had characterized their conversation as
being about ``life.''

Mr. Gates, the billionaire philanthropist who arrived at noon for his
meeting, avoided such awkwardness by talking to reporters alone, and
offering them only crumbs. ``It was a good time,'' he said of his
encounter with Mr. Trump. The two talked about the ``power of
innovation,'' Mr. Gates said before heading toward the exit.

It was a different story a few minutes earlier when Mr. Lewis and Mr.
Brown emerged. They were accompanied by Darrell C. Scott, a Cleveland
pastor and supporter of Mr. Trump. He wanted the news media to know
about their sales pitch to Mr. Trump: that he throw his administration's
weight behind Amer-I-Can, a program co-founded by Mr. Brown to help
young people caught up in gang violence.

``We couldn't have had a better meeting,'' said Mr. Brown, the legendary
fullback who held a cane engraved with the name of his pro football
team, the Cleveland Browns. ``The graciousness, the intelligence; the
reception we got was fantastic.''

Mr. Scott said Mr. Trump had made a verbal commitment to ``put the
government behind it, put the Trump administration behind it.'' Mr.
Lewis, a former Baltimore Ravens linebacker, said the men had appealed
to him by promoting the program, which has helped 30,000 to 40,000 gang
members transform their lives, as a job-creation machine.

``Urban development and job creation are everything,'' Mr. Lewis said.

When a reporter asked the men whom they had voted for, they became less
chatty.

``The election is over,'' Mr. Scott said, as the Naked Cowboy sauntered
behind him, swaddled in a stars-and-stripes boxing robe.

Advertisement

\protect\hyperlink{after-bottom}{Continue reading the main story}

\hypertarget{site-index}{%
\subsection{Site Index}\label{site-index}}

\hypertarget{site-information-navigation}{%
\subsection{Site Information
Navigation}\label{site-information-navigation}}

\begin{itemize}
\tightlist
\item
  \href{https://help.nytimes3xbfgragh.onion/hc/en-us/articles/115014792127-Copyright-notice}{©~2020~The
  New York Times Company}
\end{itemize}

\begin{itemize}
\tightlist
\item
  \href{https://www.nytco.com/}{NYTCo}
\item
  \href{https://help.nytimes3xbfgragh.onion/hc/en-us/articles/115015385887-Contact-Us}{Contact
  Us}
\item
  \href{https://www.nytco.com/careers/}{Work with us}
\item
  \href{https://nytmediakit.com/}{Advertise}
\item
  \href{http://www.tbrandstudio.com/}{T Brand Studio}
\item
  \href{https://www.nytimes3xbfgragh.onion/privacy/cookie-policy\#how-do-i-manage-trackers}{Your
  Ad Choices}
\item
  \href{https://www.nytimes3xbfgragh.onion/privacy}{Privacy}
\item
  \href{https://help.nytimes3xbfgragh.onion/hc/en-us/articles/115014893428-Terms-of-service}{Terms
  of Service}
\item
  \href{https://help.nytimes3xbfgragh.onion/hc/en-us/articles/115014893968-Terms-of-sale}{Terms
  of Sale}
\item
  \href{https://spiderbites.nytimes3xbfgragh.onion}{Site Map}
\item
  \href{https://help.nytimes3xbfgragh.onion/hc/en-us}{Help}
\item
  \href{https://www.nytimes3xbfgragh.onion/subscription?campaignId=37WXW}{Subscriptions}
\end{itemize}
