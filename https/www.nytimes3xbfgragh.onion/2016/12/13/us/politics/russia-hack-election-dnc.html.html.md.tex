Sections

SEARCH

\protect\hyperlink{site-content}{Skip to
content}\protect\hyperlink{site-index}{Skip to site index}

\href{https://www.nytimes3xbfgragh.onion/section/politics}{Politics}

\href{https://myaccount.nytimes3xbfgragh.onion/auth/login?response_type=cookie\&client_id=vi}{}

\href{https://www.nytimes3xbfgragh.onion/section/todayspaper}{Today's
Paper}

\href{/section/politics}{Politics}\textbar{}The Perfect Weapon: How
Russian Cyberpower Invaded the U.S.

\url{https://nyti.ms/2hBJis3}

\begin{itemize}
\item
\item
\item
\item
\item
\item
\end{itemize}

Advertisement

\protect\hyperlink{after-top}{Continue reading the main story}

Supported by

\protect\hyperlink{after-sponsor}{Continue reading the main story}

\hypertarget{the-perfect-weapon-how-russian-cyberpower-invaded-the-us}{%
\section{The Perfect Weapon: How Russian Cyberpower Invaded the
U.S.}\label{the-perfect-weapon-how-russian-cyberpower-invaded-the-us}}

\includegraphics{https://static01.graylady3jvrrxbe.onion/images/2016/12/14/us/14hack-top1-sub/14hack-top1-sub-articleLarge.jpg?quality=75\&auto=webp\&disable=upscale}

By \href{https://www.nytimes3xbfgragh.onion/by/eric-lipton}{Eric
Lipton},
\href{https://www.nytimes3xbfgragh.onion/by/david-e-sanger}{David E.
Sanger} and
\href{https://www.nytimes3xbfgragh.onion/by/scott-shane}{Scott Shane}

\begin{itemize}
\item
  Dec. 13, 2016
\item
  \begin{itemize}
  \item
  \item
  \item
  \item
  \item
  \item
  \end{itemize}
\end{itemize}

\href{https://www.nytimes3xbfgragh.onion/2016/12/21/world/russia-hack-presidential-election.html}{Читать
статью по-русски}

WASHINGTON --- When Special Agent Adrian Hawkins of the Federal Bureau
of Investigation called the Democratic National Committee in September
2015 to pass along some troubling news about its computer network, he
was transferred, naturally, to the help desk.

His message was brief, if alarming. At least one computer system
belonging to the D.N.C. had been compromised by hackers federal
investigators had named ``the
\href{https://labsblog.f-secure.com/2015/09/17/the-dukes-7-years-of-russian-cyber-espionage/}{Dukes},''
a
\href{https://www.nytimes3xbfgragh.onion/2020/07/16/us/politics/vaccine-hacking-russia.html}{cyberespionage
team} linked to the Russian government.

The F.B.I. knew it well: The bureau had spent the last few years trying
to kick the Dukes out of the unclassified email systems of the White
House, the State Department and even the Joint Chiefs of Staff, one of
the government's best-protected networks.

Yared Tamene, the tech-support contractor at the D.N.C. who fielded the
call, was no expert in cyberattacks. His first moves were to check
Google for ``the Dukes'' and conduct a cursory search of the D.N.C.
computer system logs to look for hints of such a cyberintrusion. By his
own account, he did not look too hard even after Special Agent Hawkins
called back repeatedly over the next several weeks --- in part because
he wasn't certain the caller was a real F.B.I. agent and not an
impostor.

``I had no way of differentiating the call I just received from a prank
call,'' Mr. Tamene wrote in an internal memo, obtained by The New York
Times, that detailed his contact with the F.B.I.

It was the cryptic first sign of a cyberespionage and
information-warfare campaign devised to disrupt the 2016 presidential
election, the first such attempt by a foreign power in American history.
What started as an information-gathering operation, intelligence
officials believe, ultimately morphed into an effort to harm one
candidate, Hillary Clinton, and tip the election to her opponent, Donald
J. Trump.

Like another famous American election scandal, it started with a
break-in at the D.N.C. The first time, 44 years ago at the committee's
old offices in the Watergate complex, the burglars planted listening
devices and jimmied a filing cabinet. This time, the burglary was
conducted from afar, directed by the Kremlin, with spear-phishing emails
and zeros and ones.

An examination by The Times of the Russian operation --- based on
interviews with dozens of players targeted in the attack, intelligence
officials who investigated it and Obama administration officials who
deliberated over the best response --- reveals a series of missed
signals, slow responses and a continuing underestimation of the
seriousness of the cyberattack.

The D.N.C.'s fumbling encounter with the F.B.I. meant the best chance to
halt the Russian intrusion was lost. The failure to grasp the scope of
the attacks undercut efforts to minimize their impact. And the White
House's reluctance to respond forcefully meant the Russians have not
paid a heavy price for their actions, a decision that could prove
critical in deterring future cyberattacks.

The low-key approach of the F.B.I. meant that Russian hackers could roam
freely through the committee's network for nearly seven months before
top D.N.C. officials were alerted to the attack and hired cyberexperts
to protect their systems. In the meantime, the hackers moved on to
targets outside the D.N.C., including Mrs. Clinton's campaign chairman,
John D. Podesta, whose private email account was hacked months later.

Even Mr. Podesta, a savvy Washington insider who had written a
\href{https://www.whitehouse.gov/sites/default/files/docs/big_data_privacy_report_may_1_2014.pdf}{2014
report on cyberprivacy for President Obama}, did not truly understand
the gravity of the hacking.

\includegraphics{https://static01.graylady3jvrrxbe.onion/images/2016/12/14/us/14HACK-tear1/14HACK1-articleLarge.jpg?quality=75\&auto=webp\&disable=upscale}

By last summer, Democrats watched in helpless fury as their private
emails and confidential documents appeared online day after day ---
procured by Russian intelligence agents, posted on WikiLeaks and other
websites, then eagerly reported on by the American media, including The
Times. Mr. Trump gleefully cited many of the purloined emails on the
campaign trail.

The fallout included
\href{https://www.nytimes3xbfgragh.onion/2016/07/25/us/politics/debbie-wasserman-schultz-dnc-wikileaks-emails.html}{the
resignations of Representative Debbie Wasserman Schultz of Florida}, the
chairwoman of the D.N.C., and most of her top party aides. Leading
Democrats were sidelined at the height of the campaign, silenced by
revelations of embarrassing emails or consumed by the scramble to deal
with the hacking. Though little-noticed by the public,
\href{https://guccifer2.wordpress.com/2016/08/15/dccc-internal-docs-on-primaries-in-florida/}{confidential
documents taken by the Russian hackers from the D.N.C.'s sister
organization}, the Democratic Congressional Campaign Committee, turned
up in congressional races in a dozen states,
\href{http://www.nytimes3xbfgragh.onion/2016/12/13/us/politics/house-democrats-hacking-dccc.html}{tainting
some of them with accusations of scandal}.

Image

President Vladimir V. Putin of Russia during a reception last week at
the Kremlin in Moscow.Credit...Pool photo by Alexei Nikolsky

In recent days, a skeptical president-elect, the nation's intelligence
agencies and the two major parties have become embroiled in an
extraordinary public dispute over what evidence exists that President
Vladimir V. Putin of Russia moved beyond mere espionage to deliberately
try to subvert American democracy and pick the winner of the
presidential election.

Many of Mrs. Clinton's closest aides believe that the Russian assault
had a profound impact on the election, while conceding that other
factors --- Mrs. Clinton's weaknesses as a candidate; her private email
server; the public statements of the F.B.I. director, James B. Comey,
about her handling of classified information --- were also important.

While there's no way to be certain of the ultimate impact of the hack,
this much is clear: A low-cost, high-impact weapon that Russia had
test-fired in elections from Ukraine to Europe was trained on the United
States, with devastating effectiveness. For Russia, with an enfeebled
economy and a nuclear arsenal it cannot use short of all-out war,
cyberpower proved the perfect weapon: cheap, hard to see coming, hard to
trace.

\href{https://www.nytimes3xbfgragh.onion/interactive/2016/07/27/us/politics/trail-of-dnc-emails-russia-hacking.html}{}

\includegraphics{https://static01.graylady3jvrrxbe.onion/images/2016/07/27/us/politics/trail-of-dnc-emails-russia-hacking-1469656463301/trail-of-dnc-emails-russia-hacking-1469656463301-thumbLarge-v6.png}

\hypertarget{following-the-links-from-russian-hackers-to-the-us-election}{%
\subsection{Following the Links From Russian Hackers to the U.S.
Election}\label{following-the-links-from-russian-hackers-to-the-us-election}}

How U.S. intelligence officials have connected the Russian government to
an attempt to disrupt the 2016 presidential election.

``There shouldn't be any doubt in anybody's mind,'' Adm. Michael S.
Rogers, the director of the National Security Agency and commander of
United States Cyber Command, said at a postelection conference. ``This
was not something that was done casually, this was not something that
was done by chance, this was not a target that was selected purely
arbitrarily,'' he said. ``This was a conscious effort by a nation-state
to attempt to achieve a specific effect.''

For the people whose emails were stolen, this new form of political
sabotage has left a trail of shock and professional damage. Neera
Tanden, president of the Center for American Progress and a key Clinton
supporter, recalls walking into the busy Clinton transition offices,
humiliated to see her face on television screens as pundits
\href{http://www.cnn.com/2016/10/18/politics/clinton-staffers-frustrated-hillary-clinton-bill-clinton-chelsea-clinton/}{discussed
a leaked email in which she had called Mrs. Clinton's instincts
``suboptimal.''}

``It was just a sucker punch to the gut every day,'' Ms. Tanden said.
``It was the worst professional experience of my life.''

The United States, too, has carried out cyberattacks, and in decades
past the C.I.A. tried to subvert foreign elections. But the Russian
attack is increasingly understood across the political spectrum as an
ominous historic landmark --- with one notable exception: Mr. Trump has
rejected the findings of the intelligence agencies he will soon oversee
as ``ridiculous,'' insisting that the hacker may be American, or
Chinese, but that ``they have no idea.''

Mr. Trump cited the reported disagreements between the agencies about
whether Mr. Putin intended to help elect him. On Tuesday, a Russian
government spokesman echoed Mr. Trump's scorn.

``This tale of `hacks' resembles a banal brawl between American security
officials over spheres of influence,'' Maria Zakharova, the spokeswoman
for the Russian Foreign Ministry, wrote on Facebook.

Over the weekend, four prominent senators --- two Republicans and two
Democrats --- joined forces to pledge an investigation while pointedly
ignoring Mr. Trump's skeptical claims.

``Democrats and Republicans must work together, and across the
jurisdictional lines of the Congress, to examine these recent incidents
thoroughly and devise comprehensive solutions to deter and defend
against further cyberattacks,'' said Senators John McCain, Lindsey
Graham, Chuck Schumer and Jack Reed.

``This cannot become a partisan issue,'' they said. ``The stakes are too
high for our country.''

\hypertarget{a-target-for-break-ins}{%
\subsection{A Target for Break-Ins}\label{a-target-for-break-ins}}

Sitting in the basement of the Democratic National Committee
headquarters, below a wall-size 2012 portrait of a smiling Barack Obama,
is a 1960s-era filing cabinet missing the handle on the bottom drawer.
Only a framed newspaper story hanging on the wall hints at the
importance of this aged piece of office furniture.

\href{http://www.washingtonpost.com/wp-srv/local/longterm/tours/scandal/watergat.htm}{``GOP
Security Aide Among 5 Arrested in Bugging Affair,''} reads the headline
from the front page of The Washington Post on June 19, 1972, with the
bylines of Bob Woodward and Carl Bernstein.

Andrew Brown, 37, the technology director at the D.N.C., was born after
that famous break-in. But as he began to plan for this year's election
cycle, he was well aware that the D.N.C. could become a break-in target
again.

There were aspirations to ensure that the D.N.C. was well protected
against cyberintruders --- and then there was the reality, Mr. Brown and
his bosses at the organization acknowledged: The D.N.C. was a nonprofit
group, dependent on donations, with a fraction of the security budget
that a corporation its size would have.

``There was never enough money to do everything we needed to do,'' Mr.
Brown said.

The D.N.C. had a standard email spam-filtering service, intended to
block phishing attacks and malware created to resemble legitimate email.
But when Russian hackers started in on the D.N.C., the committee did not
have the most advanced systems in place to track suspicious traffic,
internal D.N.C. memos show.

Mr. Tamene, who reports to Mr. Brown and fielded the call from the
F.B.I. agent, was not a full-time D.N.C. employee; he works for a
Chicago-based contracting firm called
\href{https://web.archive.org/web/20160507204851/http://www.misdepartment.com/staff}{The
MIS Department}. He was left to figure out, largely on his own, how to
respond --- and even whether the man who had called in to the D.N.C.
switchboard was really an F.B.I. agent.

``The F.B.I. thinks the D.N.C. has at least one compromised computer on
its network and the F.B.I. wanted to know if the D.N.C. is aware, and if
so, what the D.N.C. is doing about it,'' Mr. Tamene wrote in an internal
memo about his contacts with the F.B.I. He added that ``the Special
Agent told me to look for a specific type of malware dubbed `Dukes' by
the U.S. intelligence community and in cybersecurity circles.''

Part of the problem was that Special Agent Hawkins did not show up in
person at the D.N.C. Nor could he email anyone there, as that risked
alerting the hackers that the F.B.I. knew they were in the system.

Image

An internal memo by Yared Tamene, a tech-support contractor at the
D.N.C., expressed uncertainty about the identity of Special Agent Adrian
Hawkins of the F.B.I., who called to inform him of the breach.

Mr. Tamene's initial scan of the D.N.C. system --- using his
less-than-optimal tools and incomplete targeting information from the
F.B.I. --- found nothing. So when Special Agent Hawkins called
repeatedly in October, leaving voice mail messages for Mr. Tamene,
urging him to call back, ``I did not return his calls, as I had nothing
to report,'' Mr. Tamene explained in his memo.

In November, Special Agent Hawkins called with more ominous news. A
D.N.C. computer was ``calling home, where home meant Russia,'' Mr.
Tamene's memo says, referring to software sending information to Moscow.
``SA Hawkins added that the F.B.I. thinks that this calling home
behavior could be the result of a state-sponsored attack.''

Mr. Brown knew that Mr. Tamene, who declined to comment, was fielding
calls from the F.B.I. But he was tied up on a different problem:
evidence suggesting that the campaign of Senator Bernie Sanders of
Vermont, Mrs. Clinton's main Democratic opponent, had improperly gained
access to her campaign data.

Ms. Wasserman Schultz, then the D.N.C.'s chairwoman, and Amy Dacey, then
its chief executive, said in interviews that neither of them was
notified about the early reports that the committee's system had likely
been compromised.

Shawn Henry, who once led the F.B.I.'s cyber division and is now
president of CrowdStrike Services, the cybersecurity firm retained by
the D.N.C. in April, said he was baffled that the F.B.I. did not call a
more senior official at the D.N.C. or send an agent in person to the
party headquarters to try to force a more vigorous response.

``We are not talking about an office that is in the middle of the woods
of Montana,'' Mr. Henry said. ``We are talking about an office that is
half a mile from the F.B.I. office that is getting the notification.''

``This is not a mom-and-pop delicatessen or a local library. This is a
critical piece of the U.S. infrastructure because it relates to our
electoral process, our elected officials, our legislative process, our
executive process,'' he added. ``To me it is a high-level, serious
issue, and if after a couple of months you don't see any results,
somebody ought to raise that to a higher level.''

The F.B.I. declined to comment on the agency's handling of the hack.
``The F.B.I. takes very seriously any compromise of public and private
sector systems,'' it said in a statement, adding that agents ``will
continue to share information'' to help targets ``safeguard their
systems against the actions of persistent cybercriminals.''

By March, Mr. Tamene and his team had met at least twice in person with
the F.B.I. and concluded that Agent Hawkins was really a federal
employee. But then the situation took a dire turn.

A second team of Russian-affiliated hackers began to target the D.N.C.
and other players in the political world, particularly Democrats. Billy
Rinehart, a former D.N.C. regional field director who was then working
for Mrs. Clinton's campaign, got an odd email warning from Google.

``Someone just used your password to try to sign into your Google
account,''
\href{http://www.documentcloud.org/documents/3237163-Rinehart-Hacking-Email.html}{the
March 22 email said}, adding that the sign-in attempt had occurred in
Ukraine. ``Google stopped this sign-in attempt. You should change your
password immediately.''

Mr. Rinehart was in Hawaii at the time. He remembers checking his email
at 4 a.m. for messages from East Coast associates. Without thinking much
about the notification, he clicked on the ``change password'' button and
half asleep, as best he can remember, he typed in a new password.

Image

A screenshot of the phishing email that Billy Rinehart clicked on,
unknowingly giving Russian hackers access to his account. The New York
Times has redacted Mr. Rinehart's email address.

What he did not know until months later is that he had just given the
Russian hackers access to his email account.

Hundreds of similar phishing emails were being sent to American
political targets, including an identical email sent on March 19 to Mr.
Podesta, chairman of the Clinton campaign. Given how many emails Mr.
Podesta received through this personal email account, several aides also
had access to it, and one of them noticed the warning email, sending it
to a computer technician to make sure it was legitimate before anyone
clicked on the ``change password'' button.

``This is a legitimate email,'' Charles Delavan,
\href{https://wikileaks.org/podesta-emails/emailid/34899}{a Clinton
campaign aide, replied} to another of Mr. Podesta's aides, who had
noticed the alert. ``John needs to change his password immediately.''

With another click, a decade of emails that Mr. Podesta maintained in
his Gmail account --- a total of about 60,000 --- were unlocked for the
Russian hackers. Mr. Delavan, in an interview, said that his bad advice
was a result of a typo: He knew this was a phishing attack, as the
campaign was getting dozens of them. He said he had meant to type that
it was an ``illegitimate'' email, an error that he said has plagued him
ever since.

Image

Mr. Podesta, center, with Huma Abedin, Hillary Clinton's closest aide,
in Brooklyn the day after the election. Hackers gained access to tens of
thousands of Mr. Podesta's emails.Credit...Dave Sanders for The New York
Times

During this second wave, the hackers also gained access to the
Democratic Congressional Campaign Committee, and then, through a virtual
private network connection, to the main computer network of the D.N.C.

The F.B.I. observed this surge of activity as well, again reaching out
to Mr. Tamene to warn him. Yet Mr. Tamene still saw no reason to be
alarmed: He found copies of the phishing emails in the D.N.C.'s spam
filter. But he had no reason, he said, to believe that the computer
systems had been infiltrated.

One bit of progress had finally been made by the middle of April: The
D.N.C., seven months after it had first been warned, finally installed a
``robust set of monitoring tools,'' Mr. Tamene's internal memo says.

\hypertarget{honing-stealthy-tactics}{%
\subsection{Honing Stealthy Tactics}\label{honing-stealthy-tactics}}

Image

The headquarters of the Russian F.S.B., the main successor to the
Soviet-era K.G.B., in Moscow.Credit...Pavel Golovkin/Associated Press

The United States had two decades of warning that Russia's intelligence
agencies were trying to break into America's most sensitive computer
networks. But the Russians have always managed to stay a step ahead.

Their first major attack was detected on Oct. 7, 1996, when a computer
operator at the Colorado School of Mines discovered some nighttime
computer activity he could not explain. The school had a major contract
with the Navy, and the operator warned his contacts there. But as
happened two decades later at the D.N.C., at first ``everyone was unable
to connect the dots,'' said
\href{http://www.kcl.ac.uk/sspp/departments/warstudies/people/professors/rid.aspx}{Thomas
Rid}, a scholar at King's College in London who has studied the attack.

Investigators gave it a name ---
\href{https://medium.com/@chris_doman/the-first-sophistiated-cyber-attacks-how-operation-moonlight-maze-made-history-2adb12cc43f7\#.ghd5tn5cf}{Moonlight
Maze} --- and spent two years, often working day and night, tracing how
it hopped from the Navy to the Department of Energy to the Air Force and
NASA. In the end, they concluded that the total number of files stolen,
if printed and stacked, would be taller than the Washington Monument.

Whole weapons designs were flowing out the door, and it was a first
taste of what was to come: an escalating campaign of cyberattacks around
the world.

But for years, the Russians stayed largely out of the headlines, thanks
to the Chinese --- who took bigger risks, and often got caught. They
stole the designs for the F-35 fighter jet, corporate secrets for
rolling steel, even the blueprints for gas pipelines that supply much of
the United States. And during the 2008 presidential election cycle,
Chinese intelligence hacked into the campaigns of Mr. Obama and Mr.
McCain, making off with internal position papers and communications. But
they didn't publish any of it.

The Russians had not gone away, of course. ``They were just a lot more
stealthy,'' said
\href{https://www.fireeye.com/company/leadership.html}{Kevin Mandia}, a
former Air Force intelligence officer who spent most of his days
fighting off Russian cyberattacks before founding Mandiant, a
cybersecurity firm that is now a division of FireEye --- and the company
the Clinton campaign brought in to secure its own systems.

The Russians were also quicker to turn their attacks to political
purposes. A 2007 cyberattack on Estonia, a former Soviet republic that
had joined NATO, sent a message that Russia could paralyze the country
without invading it. The next year cyberattacks were used during
Russia's war with Georgia.

But American officials did not imagine that the Russians would dare try
those techniques inside the United States. They were largely focused on
preventing what former Defense Secretary Leon E. Panetta warned was an
approaching ``cyber Pearl Harbor'' --- a shutdown of the power grid or
cellphone networks.

But in 2014 and 2015, a Russian hacking group began systematically
targeting the State Department, the White House and the Joint Chiefs of
Staff. ``Each time, they eventually met with some form of success,''
Michael Sulmeyer, a former cyberexpert for the secretary of defense, and
Ben Buchanan, now both of the
\href{http://belfercenter.ksg.harvard.edu/project/69/cyber_security_project.html}{Harvard
Cyber Security Project}, wrote recently in a soon-to-be published paper
for the Carnegie Endowment.

The Russians grew stealthier and stealthier, tricking government
computers into sending out data while disguising the electronic
``command and control'' messages that set off alarms for anyone looking
for malicious actions. The State Department was so crippled that it
repeatedly closed its systems to throw out the intruders. At one point,
officials traveling to Vienna with Secretary of State John Kerry for the
Iran nuclear negotiations had to set up commercial Gmail accounts just
to communicate with one another and with reporters traveling with them.

Mr. Obama was briefed regularly on all this, but he made a decision that
many in the White House now regret: He did not name Russians publicly,
or issue sanctions. There was always a reason: fear of escalating a
cyberwar, and concern that the United States needed Russia's cooperation
in negotiations over Syria.

``We'd have all these circular meetings,'' one senior State Department
official said, ``in which everyone agreed you had to push back at the
Russians and push back hard. But it didn't happen.''

So the Russians escalated again --- breaking into systems not just for
espionage, but to publish or broadcast what they found, known as
``doxing'' in the cyberworld.

It was a brazen change in tactics, moving the Russians from espionage to
influence operations. In February 2014,
\href{https://www.nytimes3xbfgragh.onion/2014/02/08/world/europe/ukraine.html}{they
broadcast an intercepted phone call} between Victoria Nuland, the
assistant secretary of state who handles Russian affairs and has a
contentious relationship with Mr. Putin, and Geoffrey Pyatt, the United
States ambassador to Ukraine. Ms. Nuland was heard describing a
little-known American effort to broker a deal in Ukraine, then in
political turmoil.

They were not the only ones on whom the Russians used the steal-and-leak
strategy. The Open Society Foundation, run by George Soros, was a major
target, and \href{http://soros.dcleaks.com/}{when its documents were
released}, some turned out to have been altered to make it appear as if
the foundation was financing Russian opposition members.

Last year, the attacks became more aggressive. Russia hacked a major
French television station, frying critical hardware. Around Christmas,
it attacked part of the power grid in Ukraine, dropping a portion of the
country into darkness, killing backup generators and taking control of
generators. In retrospect, it was a warning shot.

The attacks ``were not fully integrated military operations,'' Mr.
Sulmeyer said. But they showed an increasing boldness.

\hypertarget{cozy-bear-and-fancy-bear}{%
\subsection{Cozy Bear and Fancy Bear}\label{cozy-bear-and-fancy-bear}}

Image

Supporters of President-elect Donald J. Trump at a ``thank you'' rally
last week in Des Moines.Credit...Doug Mills/The New York Times

The day before the White House Correspondents' Association dinner in
April, Ms. Dacey, the D.N.C.'s chief executive, was preparing for a
night of parties when she got an urgent phone call.

With the new monitoring system in place, Mr. Tamene had examined
administrative logs of the D.N.C.'s computer system and found something
very suspicious: An unauthorized person, with administrator-level
security status, had gained access to the D.N.C.'s computers.

``Not sure it is related to what the F.B.I. has been noticing,'' said
one internal D.N.C. email sent on April 29. ``The D.N.C. may have been
hacked in a serious way this week, with password theft, etc.''

No one knew just how bad the breach was --- but it was clear that a lot
more than a single filing cabinet worth of materials might have been
taken. A secret committee was immediately created, including Ms. Dacey,
Ms. Wasserman Schultz, Mr. Brown and
\href{https://www.perkinscoie.com/en/professionals/michael-sussmann.html}{Michael
Sussmann}, a former cybercrimes prosecutor at the Department of Justice
who now works at Perkins Coie, the Washington law firm that handles
D.N.C. political matters.

``Three most important questions,'' Mr. Sussmann wrote to his clients
the night the break-in was confirmed. ``1) What data was accessed? 2)
How was it done? 3) How do we stop it?''

Mr. Sussmann instructed his clients not to use D.N.C. email because they
had just one opportunity to lock the hackers out --- an effort that
could be foiled if the hackers knew that the D.N.C. was on to them.

``You only get one chance to raise the drawbridge,'' Mr. Sussmann said.
``If the adversaries know you are aware of their presence, they will
take steps to burrow in, or erase the logs that show they were
present.''

Image

Michael Sussmann, a Washington lawyer and former cybercrime prosecutor
at the Justice Department, received an email in late April confirming
that the D.N.C.'s computer system had been compromised.

The D.N.C. immediately hired
\href{https://www.crowdstrike.com/}{CrowdStrike}, a cybersecurity firm,
to scan its computers, identify the intruders and build a new computer
and telephone system from scratch. Within a day, CrowdStrike confirmed
that the intrusion had originated in Russia, Mr. Sussmann said.

The work that such companies do is a computer version of old-fashioned
crime scene investigation, with fingerprints, bullet casings and DNA
swabs replaced by an electronic trail that can be just as incriminating.
And just as police detectives learn to identify the telltale methods of
a veteran burglar, so CrowdStrike investigators recognized the
distinctive handiwork of Cozy Bear and Fancy Bear.

Those are CrowdStrike's nicknames for the two Russian hacking groups
that the firm found at work inside the D.N.C. network. Cozy Bear --- the
group also known as the Dukes or A.P.T. 29, for ``advanced persistent
threat'' --- may or may not be associated with the F.S.B., the main
successor to the Soviet-era K.G.B., but it is widely believed to be a
Russian government operation. It made its first appearance in 2014, said
Dmitri Alperovitch, CrowdStrike's co-founder and chief technology
officer.

It was Cozy Bear, CrowdStrike concluded, that first penetrated the
D.N.C. in the summer of 2015, by sending spear-phishing emails to a long
list of American government agencies, Washington nonprofits and
government contractors. Whenever someone clicked on a phishing message,
the Russians would enter the network, ``exfiltrate'' documents of
interest and stockpile them for intelligence purposes.

``Once they got into the D.N.C., they found the data valuable and
decided to continue the operation,'' said Mr. Alperovitch, who was born
in Russia and moved to the United States as a teenager.

Only in March 2016 did Fancy Bear show up --- first penetrating the
computers of the Democratic Congressional Campaign Committee, and then
jumping to the D.N.C., investigators believe. Fancy Bear, sometimes
called A.P.T. 28 and believed to be directed by the G.R.U., Russia's
military intelligence agency, is an older outfit, tracked by Western
investigators for nearly a decade. It was Fancy Bear that got hold of
Mr. Podesta's email.

Attribution, as the skill of identifying a cyberattacker is known, is
more art than science. It is often impossible to name an attacker with
absolute certainty. But over time, by accumulating a reference library
of hacking techniques and targets, it is possible to spot repeat
offenders. Fancy Bear, for instance, has gone after military and
political targets in Ukraine and Georgia, and at NATO installations.

That largely rules out cybercriminals and most countries, Mr.
Alperovitch said. ``There's no plausible actor that has an interest in
all those victims other than Russia,'' he said. Another clue: The
Russian hacking groups tended to be active during working hours in the
Moscow time zone.

To their astonishment, Mr. Alperovitch said, CrowdStrike experts found
signs that the two Russian hacking groups had not coordinated their
attacks. Fancy Bear, apparently not knowing that Cozy Bear had been
rummaging in D.N.C. files for months, took many of the same documents.

In the six weeks after CrowdStrike's arrival, in total secrecy, the
computer system at the D.N.C. was replaced. For a weekend, email and
phones were shut off; employees were told it was a system upgrade. All
laptops were turned in and the hard drives wiped clean, with the
uninfected information on them imaged to new drives.

Though D.N.C. officials had learned that the Democratic Congressional
Campaign Committee had been infected, too, they did not notify their
sister organization, which was in the same building, because they were
afraid that it would leak.

All of this work took place as the bitter contest for the Democratic
nomination continued to play out between Mrs. Clinton and Mr. Sanders,
and it was already causing a major distraction for Ms. Wasserman Schultz
and the D.N.C.'s chief executive.

``This was not a bump in the road --- bumps in the road happen all the
time,'' she said in an interview. ``Two different Russian spy agencies
had hacked into our network and stolen our property. And we did not yet
know what they had taken. But we knew they had very broad access to our
network. There was a tremendous amount of uncertainty. And it was
chilling.''

The D.N.C. executives and their lawyer had their first formal meeting
with senior F.B.I. officials in mid-June, nine months after the bureau's
first call to the tech-support contractor. Among the early requests at
that meeting, according to participants: that the federal government
make a quick ``attribution'' formally blaming actors with ties to
Russian government for the attack to make clear that it was not routine
hacking but foreign espionage.

``You have a presidential election underway here and you know that the
Russians have hacked into the D.N.C.,'' Mr. Sussmann said, recalling the
message to the F.B.I. ``We need to tell the American public that. And
soon.''

\hypertarget{the-medias-role}{%
\subsection{The Media's Role}\label{the-medias-role}}

Image

Supporters of Senator Bernie Sanders's presidential campaign protested
at the Democratic National Convention in Philadelphia in
July.Credit...Ruth Fremson/The New York Times

In mid-June, on Mr. Sussmann's advice, D.N.C. leaders decided to take a
bold step. Concerned that word of the hacking might leak, they decided
to go public in The Washington Post with the
\href{https://www.washingtonpost.com/world/national-security/russian-government-hackers-penetrated-dnc-stole-opposition-research-on-trump/2016/06/14/cf006cb4-316e-11e6-8ff7-7b6c1998b7a0_story.html}{news}
that the committee had been attacked. That way, they figured, they could
get ahead of the story, win a little sympathy from voters for being
victimized by Russian hackers and refocus on the campaign.

But the very next day, a new, deeply unsettling shock awaited them.
Someone calling himself Guccifer 2.0
\href{https://guccifer2.wordpress.com/2016/06/15/dnc/}{appeared} on the
web, claiming to be the D.N.C. hacker --- and he posted a confidential
committee document detailing Mr. Trump's record and half a dozen other
documents to prove his bona fides.

``And it's just a tiny part of all docs I downloaded from the Democrats
networks,'' he wrote. Then something more ominous: ``The main part of
the papers, thousands of files and mails, I gave to WikiLeaks. They will
publish them soon.''

It was bad enough that Russian hackers had been spying inside the
committee's network for months. Now the public release of documents had
turned a conventional espionage operation into something far more
menacing: political sabotage, an unpredictable, uncontrollable menace
for Democratic campaigns.

Guccifer 2.0 borrowed the moniker of an earlier hacker, a Romanian who
called himself Guccifer and was jailed for breaking into the personal
computers of former President George W. Bush, former Secretary of State
Colin L. Powell and other notables. This new attacker seemed intent on
showing that the D.N.C.'s cyberexperts at CrowdStrike were wrong to
blame Russia. Guccifer 2.0 called himself a ``lone hacker'' and mocked
CrowdStrike for calling the attackers ``sophisticated.''

But online investigators quickly undercut his story. On a whim, Lorenzo
Franceschi-Bicchierai, a writer for Motherboard, the tech and culture
site of Vice, tried to contact Guccifer 2.0 by direct message on
Twitter.

``Surprisingly, he answered right away,'' Mr. Franceschi-Bicchierai
said. But whoever was on the other end seemed to be mocking him. ``I
asked him why he did it, and he said he wanted to expose the Illuminati.
He called himself a Gucci lover. And he said he was Romanian.''

That gave Mr. Franceschi-Bicchierai an idea. Using Google Translate, he
sent the purported hacker some questions in Romanian. The answers came
back in Romanian. But when he was offline, Mr. Franceschi-Bicchierai
checked with a couple of native speakers, who told him Guccifer 2.0 had
apparently been using Google Translate as well --- and was clearly not
the Romanian he claimed to be.

Cyberresearchers found other clues pointing to Russia. Microsoft Word
documents posted by Guccifer 2.0 had been edited by someone calling
himself, in Russian, Felix Edmundovich --- an obvious nom de guerre
honoring the founder of the Soviet secret police,
\href{http://www.newworldencyclopedia.org/entry/Felix_Dzerzhinsky}{Felix
Edmundovich Dzerzhinsky}. Bad links in the texts were marked by warnings
in Russian, generated by what was clearly a Russian-language version of
Word.

When Mr. Franceschi-Bicchierai managed to engage Guccifer 2.0 over a
period of weeks, he found that his interlocutor's tone and manner
changed. ``At first he was careless and colloquial. Weeks later, he was
curt and more calculating,'' he said. ``It seemed like a group of
people, and a very sloppy attempt to cover up.''

Computer experts drew the same conclusion about
\href{http://dcleaks.com/}{DCLeaks.com}, a site that sprang up in June,
claiming to be the work of ``hacktivists'' but posting more stolen
documents. It, too, seemed to be a clumsy front for the same Russians
who had stolen the documents. Notably, the website was registered in
April, suggesting that the Russian hacking team planned well in advance
to make public what it stole.

In addition to what Guccifer 2.0 published on his site, he provided
material directly on request to some
\href{http://hellofla.com/2016/08/12/guccifer-2-0-strikes-fl-18/}{bloggers}
and
\href{https://theintercept.com/2016/10/09/exclusive-new-email-leak-reveals-clinton-campaigns-cozy-press-relationship/}{publications}.
The steady flow of Guccifer 2.0 documents constantly undercut Democratic
messaging efforts. On July 6, 12 days before the Republican National
Convention began in Cleveland, Guccifer released the D.N.C.'s battle
plan and budget for countering it. For Republican operatives, it was
insider gold.

Then WikiLeaks, a far more established outlet, began to publish the
hacked material --- just as Guccifer 2.0 had promised. On July 22, three
days before the start of the Democratic National Convention in
Philadelphia, \href{https://wikileaks.org/dnc-emails/}{WikiLeaks dumped
out 44,053 D.N.C. emails with 17,761 attachments}. Some of the messages
made clear that some D.N.C. officials favored Mrs. Clinton over her
progressive challenger, Mr. Sanders.

That was no shock; Mr. Sanders, after all, had been an independent
socialist, not a Democrat, during his long career in Congress, while
Mrs. Clinton had been one of the party's stars for decades. But the
emails, some of them crude or insulting, infuriated Sanders delegates as
they arrived in Philadelphia. Ms. Wasserman Schultz resigned under
pressure on the eve of the convention where she had planned to preside.

Mr. Trump, by now the Republican nominee, expressed delight at the
continuing jolts to his opponent, and he began to use Twitter and his
stump speeches to highlight the WikiLeaks releases. On July 25, he sent
out a lighthearted
\href{https://twitter.com/realDonaldTrump/status/757538729170964481}{tweet}:
``The new joke in town,'' he wrote, ``is that Russia leaked the
disastrous D.N.C. e-mails, which should never have been written
(stupid), because Putin likes me.''

But WikiLeaks was far from finished. On Oct. 7, a month before the
election, the site began the serial publication of
\href{https://wikileaks.org/podesta-emails/}{thousands of private emails
to and from Mr. Podesta}, Mrs. Clinton's campaign manager.

The same day, the United States formally accused the Russian government
of being behind the hackings, in
\href{https://www.dhs.gov/news/2016/10/07/joint-statement-department-homeland-security-and-office-director-national}{a
joint statement} by the director of national intelligence and the
Department of Homeland Security, and Mr. Trump suffered his worst blow
to date, with the release of a recording in which he bragged about
sexually assaulting women.

The Podesta emails were nowhere near as sensational as the Trump video.
But, released by WikiLeaks day after day over the last month of the
campaign, they provided material for countless news reports. They
disclosed the contents of Mrs. Clinton's speeches to large banks, which
she had refused to release. They exposed tensions inside the campaign,
including disagreements over donations to the Clinton Foundation that
staff members thought might look bad for the candidate and Ms. Tanden's
complaint that Mrs. Clinton's instincts were ``suboptimal.''

``I was just mortified,'' Ms. Tanden said in an interview. Her emails
were released on the eve of one of the presidential debates, she
recalled. ``I put my hands over my head and said, `I can't believe this
is happening to me.''' Though she had regularly appeared on television
to support Mrs. Clinton, she canceled her appearances because all the
questions were about what she had said in the emails.

Ms. Tanden, like other Democrats whose messages became public, said it
was obvious to her that WikiLeaks was trying its best to damage the
Clinton campaign. ``If you care about transparency, you put all the
emails out at once,'' she said. ``But they wanted to hurt her. So they
put them out 1,800 to 3,000 a day.''

The Trump campaign knew in advance about WikiLeaks' plans. Days before
the Podesta email release began, Roger Stone, a Republican operative
working with the Trump campaign, sent out an excited tweet about what
was coming.

But in an interview, Mr. Stone said he had no role in the leaks; he had
just heard from an American with ties to WikiLeaks that damning emails
were coming.

Julian Assange, the WikiLeaks founder and editor, has resisted the
conclusion that his site became a pass-through for Russian hackers
working for Mr. Putin's government or that he was deliberately trying to
undermine Mrs. Clinton's candidacy. But the evidence on both counts
appears compelling.

In a series of email exchanges, Mr. Assange refused to say anything
about WikiLeaks' source for the hacked material. He denied that he had
made his animus toward Mrs. Clinton clear in public statements (``False.
But what is this? Junior high?'') or that the site had timed the
releases for maximum negative effect on her campaign. ``WikiLeaks makes
its decisions based on newsworthiness, including for its recent epic
scoops,'' he wrote.

Mr. Assange disputed the conclusion of the Oct. 7 statement from the
intelligence agencies that the leaks were ``intended to interfere with
the U.S. election process.''

``This is false,'' he wrote. ``As the disclosing party we know that this
was not the intent. Publishers publishing newsworthy information during
an election is part of a free election.''

Image

Julian Assange, the WikiLeaks founder and editor, disputed intelligence
agencies' conclusion that the email leaks were ``intended to interfere
with the U.S. election process.''Credit...Steffi Loos/Agence
France-Presse --- Getty Images

But asked whether he believed the leaks were one reason for Mr. Trump's
election, Mr. Assange seemed happy to take credit. ``Americans
extensively engaged with our publications,'' he wrote. ``According to
Facebook statistics WikiLeaks was the most referenced political topic
during October.''

Though Mr. Assange did not say so, WikiLeaks' best defense may be the
conduct of the mainstream American media. Every major publication,
including The Times, published multiple stories citing the D.N.C. and
Podesta emails posted by WikiLeaks, becoming a de facto instrument of
Russian intelligence.

Mr. Putin, a student of martial arts, had turned two institutions at the
core of American democracy --- political campaigns and independent media
--- to his own ends. The media's appetite for the hacked material, and
its focus on the gossipy content instead of the Russian source,
disturbed some of those whose personal emails were being reposted across
the web.

``What was really surprising to me?'' Ms. Tanden said. ``I could not
believe that reporters were covering it.''

\hypertarget{devising-a-government-response}{%
\subsection{Devising a Government
Response}\label{devising-a-government-response}}

Image

The D.N.C. headquarters in Washington.Credit...Justin T. Gellerson for
The New York Times

Inside the White House, as Mr. Obama's advisers debated their response,
their conversation turned to North Korea.

In late 2014, hackers working for Kim Jong-un, the North's young and
unpredictable leader, had carried out a well-planned attack on Sony
Pictures Entertainment intended to stop the Christmastime release of a
comedy about a C.I.A. plot to kill Mr. Kim.

\href{https://www.nytimes3xbfgragh.onion/2014/12/31/business/media/sony-attack-first-a-nuisance-swiftly-grew-into-a-firestorm-.html?_r=0}{In
that case, embarrassing emails had also been released.} But the real
damage was done to Sony's own systems: More than 70 percent of its
computers melted down when a particularly virulent form of malware was
released. Within weeks, intelligence agencies
\href{https://www.nytimes3xbfgragh.onion/2014/12/31/business/media/sony-attack-first-a-nuisance-swiftly-grew-into-a-firestorm-.html?_r=0}{traced
the attack back to the North and its leadership}. Mr. Obama called North
Korea out in public, and
\href{https://www.nytimes3xbfgragh.onion/2015/01/03/us/in-response-to-sony-attack-us-levies-sanctions-on-10-north-koreans.html}{issued
some not-very-effective sanctions}. The Chinese even cooperated, briefly
cutting off the North's internet connections.

As the first Situation Room meetings on the Russian hacking began in
July, ``it was clear that Russia was going to be a much more complicated
case,'' said one participant. The Russians clearly had a more
sophisticated understanding of American politics, and they were masters
of
``\href{http://www.nytimes3xbfgragh.onion/2016/12/09/world/europe/vladimir-putin-russia-fake-news-hacking-cybersecurity.html}{kompromat},''
their term for compromising information.

But a formal ``attribution report'' still had not been forwarded to the
president.

``It took forever,'' one senior administration official said,
complaining about the pace at which the intelligence assessments moved
through the system.

In August a group that called itself the
\href{https://www.nytimes3xbfgragh.onion/2016/08/17/us/shadow-brokers-leak-raises-alarming-question-was-the-nsa-hacked.html}{``Shadow
Brokers''} published a set of software tools that looked like what the
N.S.A. uses to break into foreign computer networks and install
``implants,'' malware that can be used for surveillance or attack. The
code came from the Tailored Access Operations unit of the N.S.A., a
secretive group that mastered the arts of surveillance and cyberwar.

The assumption --- still unproved --- was that the code was put out in
the open by the Russians as a warning: Retaliate for the D.N.C., and
there are a lot more secrets, from the hackings of the State Department,
the White House and the Pentagon, that might be spilled as well. One
senior official compared it to the scene in ``The Godfather'' where the
head of a favorite horse is left in a bed, as a warning.

The N.S.A. said nothing. But by late August, Admiral Rogers, its
director, was pressing for a more muscular response to the Russians. In
his role as director of the Pentagon's Cyber Command, he proposed a
series of potential counter-cyberstrikes.

Image

Adm. Michael S. Rogers, the director of the National Security Agency and
commander of United States Cyber Command, pressed for a more muscular
response to the Russians.Credit...Jim Wilson/The New York Times

While officials will not discuss them in detail, the possible
counterstrikes reportedly included operations that would turn the tables
on Mr. Putin, exposing his financial links to Russia's oligarchs, and
punching holes in the Russian internet to allow dissidents to get their
message out. Pentagon officials judged the measures too unsubtle and
ordered up their own set of options.

But in the end, none of those were formally presented to the president.

In a series of ``deputies meetings'' run by
\href{https://www.whitehouse.gov/the-press-office/2014/12/18/statement-president-selection-avril-haines-deputy-national-security-advi}{Avril
Haines}, the deputy national security adviser and a former deputy
director of the C.I.A., several officials warned that an overreaction by
the administration would play into Mr. Putin's hands.

``If we went to Defcon 4,'' one frequent participant in Ms. Haines's
meetings said, using a phrase from the Cold War days of warnings of war,
``we would be saying to the public that we didn't have confidence in the
integrity of our voting system.''

Even something seemingly straightforward --- using the president's
executive powers, bolstered after the Sony incident, to place economic
and travel sanctions on cyberattackers --- seemed too risky.

``No one was all that eager to impose costs before Election Day,'' said
another participant in the classified meeting. ``Any retaliatory
measures were seen through the prism of what would happen on Election
Day.''

Instead, when Mr. Obama's national security team reconvened after summer
vacation, the focus turned to a crash effort to secure the nation's
voting machines and voter-registration rolls from hacking. The scenario
they discussed most frequently --- one that turned out not to be an
issue --- was a narrow vote in favor of Mrs. Clinton, followed by a
declaration by Mr. Trump that the vote was ``rigged'' and more leaks
intended to undercut her legitimacy.

Donna Brazile, the interim chairwoman of the D.N.C., became increasingly
frustrated as the clock continued to run down on the presidential
election --- and still there was no broad public condemnation by the
White House, or Republican Party leaders, of the attack as an act of
foreign espionage.

Ms. Brazile even reached out to Reince Priebus, the chairman of the
Republican National Committee, urging him twice in private conversations
and in a
\href{https://www.documentcloud.org/documents/3237225-2016-10-18-DNC-RNC-Letter.html}{letter}
to join her in condemning the attacks --- an offer he declined to take
up.

``We just kept hearing the government would respond, the government
would respond,'' she said. ``Once upon a time, if a foreign government
interfered with our election we would respond as a nation, not as a
political party.''

But Mr. Obama did decide that he would deliver a warning to Mr. Putin in
person at a Group of 20 summit meeting in Hangzhou, China, the last time
they would be in the same place while Mr. Obama was still in office.
When the two men met for a tense pull-aside, Mr. Obama explicitly warned
Mr. Putin of a strong American response if there was continued effort to
influence the election or manipulate the vote, according to White House
officials who were not present for the one-on-one meeting.

Later that day, Mr. Obama made a rare reference to America's own
offensive cybercapacity, which he has almost never talked about.
``Frankly, both offensively and defensively, we have more capacity,'' he
told reporters.

But when it came time to make a public assertion of Russia's role in
early October, it was made in a written statement from the director of
national intelligence and the secretary of homeland security. It was far
less dramatic than the president's appearance in the press room two
years before to directly accuse the North Koreans of attacking Sony.

The reference in the statement to hackings on ``political
organizations,'' officials now say, encompassed a hacking on data stored
by the Republicans as well. Two senior officials say the forensic
evidence was accompanied by ``human and technical'' sources in Russia,
which appears to mean that the United States' implants or taps in
Russian computer and phone networks helped confirm the country's role.

But that may not be known for decades, until the secrets are
declassified.

A week later Vice President Joseph R. Biden Jr. was sent out to transmit
a public warning to Mr. Putin: The United States will retaliate ``at the
time of our choosing. And under the circumstances that have the greatest
impact.''

Later, after Mr. Biden said he was not concerned that Russia could
``fundamentally alter the election,'' he was asked whether the American
public would know if the message to Mr. Putin had been sent.

``Hope not,'' Mr. Biden responded.

Some of his former colleagues think that was the wrong answer. An
American counterstrike, said Michael Morell, the former deputy director
of the C.I.A. under Mr. Obama, has ``got to be overt. It needs to be
seen.''

A covert response would significantly limit the deterrence effect, he
added. ``If you can't see it, it's not going to deter the Chinese and
North Koreans and Iranians and others.''

The Obama administration says it still has more than 30 days to do
exactly that.

\hypertarget{the-next-target}{%
\subsection{The Next Target}\label{the-next-target}}

Image

President Obama and Vice President Joseph R. Biden Jr. walked back
toward the White House after delivering remarks about the election
results last month.Credit...Al Drago/The New York Times

As the year draws to a close, it now seems possible that there will be
multiple investigations of the Russian hacking --- the intelligence
review Mr. Obama has ordered completed by Jan. 20, the day he leaves
office, and one or more congressional inquiries. They will wrestle with,
among other things, Mr. Putin's motive.

Did he seek to mar the brand of American democracy, to forestall
anti-Russian activism for both Russians and their neighbors? Or to
weaken the next American president, since presumably Mr. Putin had no
reason to doubt American forecasts that Mrs. Clinton would win easily?
Or was it, as the C.I.A. concluded last month, a deliberate attempt to
elect Mr. Trump?

In fact, the Russian hack-and-dox scheme accomplished all three goals.

What seems clear is that Russian hacking, given its success, is not
going to stop. Two weeks ago, the German intelligence chief, Bruno Kahl,
\href{http://www.sueddeutsche.de/politik/bruno-kahl-im-interview-stoerversuche-aus-russland-1.3270241}{warned}
that Russia might target elections in Germany next year.
\href{http://www.nytimes3xbfgragh.onion/aponline/2016/11/29/world/europe/ap-eu-germany-cyberattacks.html}{``The
perpetrators have an interest to delegitimize the democratic process as
such,'' Mr. Kahl said.} Now, he added, ``Europe is in the focus of these
attempts of disturbance, and Germany to a particularly great extent.''

But Russia has by no means forgotten its American target. On the day
after the presidential election, the cybersecurity company Volexity
\href{https://www.volexity.com/blog/2016/11/09/powerduke-post-election-spear-phishing-campaigns-targeting-think-tanks-and-ngos/}{reported}
five new waves of phishing emails, evidently from Cozy Bear, aimed at
think tanks and nonprofits in the United States.

One of them purported to be from Harvard University, attaching a fake
paper. Its title: ``Why American Elections Are Flawed.''

Advertisement

\protect\hyperlink{after-bottom}{Continue reading the main story}

\hypertarget{site-index}{%
\subsection{Site Index}\label{site-index}}

\hypertarget{site-information-navigation}{%
\subsection{Site Information
Navigation}\label{site-information-navigation}}

\begin{itemize}
\tightlist
\item
  \href{https://help.nytimes3xbfgragh.onion/hc/en-us/articles/115014792127-Copyright-notice}{©~2020~The
  New York Times Company}
\end{itemize}

\begin{itemize}
\tightlist
\item
  \href{https://www.nytco.com/}{NYTCo}
\item
  \href{https://help.nytimes3xbfgragh.onion/hc/en-us/articles/115015385887-Contact-Us}{Contact
  Us}
\item
  \href{https://www.nytco.com/careers/}{Work with us}
\item
  \href{https://nytmediakit.com/}{Advertise}
\item
  \href{http://www.tbrandstudio.com/}{T Brand Studio}
\item
  \href{https://www.nytimes3xbfgragh.onion/privacy/cookie-policy\#how-do-i-manage-trackers}{Your
  Ad Choices}
\item
  \href{https://www.nytimes3xbfgragh.onion/privacy}{Privacy}
\item
  \href{https://help.nytimes3xbfgragh.onion/hc/en-us/articles/115014893428-Terms-of-service}{Terms
  of Service}
\item
  \href{https://help.nytimes3xbfgragh.onion/hc/en-us/articles/115014893968-Terms-of-sale}{Terms
  of Sale}
\item
  \href{https://spiderbites.nytimes3xbfgragh.onion}{Site Map}
\item
  \href{https://help.nytimes3xbfgragh.onion/hc/en-us}{Help}
\item
  \href{https://www.nytimes3xbfgragh.onion/subscription?campaignId=37WXW}{Subscriptions}
\end{itemize}
