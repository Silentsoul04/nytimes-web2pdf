Sections

SEARCH

\protect\hyperlink{site-content}{Skip to
content}\protect\hyperlink{site-index}{Skip to site index}

\href{https://www.nytimes3xbfgragh.onion/section/politics}{Politics}

\href{https://myaccount.nytimes3xbfgragh.onion/auth/login?response_type=cookie\&client_id=vi}{}

\href{https://www.nytimes3xbfgragh.onion/section/todayspaper}{Today's
Paper}

\href{/section/politics}{Politics}\textbar{}G.O.P. Plans to Replace
Health Care Law With `Universal Access'

\url{https://nyti.ms/2hMach1}

\begin{itemize}
\item
\item
\item
\item
\item
\end{itemize}

Advertisement

\protect\hyperlink{after-top}{Continue reading the main story}

Supported by

\protect\hyperlink{after-sponsor}{Continue reading the main story}

\hypertarget{gop-plans-to-replace-health-care-law-with-universal-access}{%
\section{G.O.P. Plans to Replace Health Care Law With `Universal
Access'}\label{gop-plans-to-replace-health-care-law-with-universal-access}}

\includegraphics{https://static01.graylady3jvrrxbe.onion/images/2016/12/16/us/16health/09transitionbriefing3-articleInline.jpg?quality=75\&auto=webp\&disable=upscale}

By \href{https://www.nytimes3xbfgragh.onion/by/robert-pear}{Robert Pear}
and \href{http://www.nytimes3xbfgragh.onion/by/thomas-kaplan}{Thomas
Kaplan}

\begin{itemize}
\item
  Dec. 15, 2016
\item
  \begin{itemize}
  \item
  \item
  \item
  \item
  \item
  \end{itemize}
\end{itemize}

WASHINGTON --- House Republicans, responding to criticism that repealing
the Affordable Care Act would leave millions without health insurance,
said on Thursday that their goal in replacing President Obama's health
law was to guarantee ``universal access'' to health care and coverage,
not necessarily to ensure that everyone actually has insurance.

In defending the Affordable Care Act, the Obama administration,
congressional Democrats and advocacy groups have focused on the 20
million people covered by the law, which has pushed the percentage of
Americans without health insurance to record lows. The American Medical
Association recently said that ``any new reform proposal should not
cause individuals currently covered to become uninsured.''

But House Republicans, preparing for a rapid legislative strike on the
law next month, emphasize a different measure of success.

``Our goal here is to make sure that everybody can buy coverage or find
coverage if they choose to,'' a House leadership aide told journalists
on the condition of anonymity at a health care briefing organized by
Republican leaders.

Republicans have an ``ironclad commitment'' to repeal the law, the aide
said, as lawmakers moved to discredit predictions that many people would
lose coverage.

``There's a lot of scare tactics out there on this,'' said
Representative Kevin Brady, Republican of Texas and chairman of the Ways
and Means Committee. ``We can reassure the American public that the plan
they are in right now, the Obamacare plans, will not end on Jan. 20,''
the day Donald J. Trump will be inaugurated.

The suggestion that 20 million people will lose coverage is a ``big
lie,'' Mr. Brady said, after meeting here with Republican members of his
committee.

``Republicans,'' he said, ``will provide an adequate transition period
to give people peace of mind that they will have those options available
to them as we work through this solution.''

Republicans have not settled on the details or the timing of their
replacement plan. The House speaker, Paul D. Ryan of Wisconsin, portrays
repeal of the law not as an ideological crusade, but as a form of
urgently needed relief.

``Insurance markets are collapsing,'' Mr. Ryan said this week.
``Premiums are soaring. Patients' choices are dwindling.''

The House leadership aide said that repealing major provisions of the
law was a priority for the first 100 days of the Trump administration.
But, he said, the date that those provisions would actually disappear
would be delayed, allowing a transition period as short as two years or
as long as three or four years. During that time, Republicans plan to
pass one or more replacement bills.

By giving people the choice to buy insurance, Republicans could end up
dangerously skewing the health insurance market, Obama administration
officials and insurance executives say. Sick people are more likely to
sign up, they say, and there may not be enough healthy people paying
premiums to cover the costs for those who are less healthy.

Under the Affordable Care Act, people who go without insurance are
subject to tax penalties. The Internal Revenue Service says that more
than eight million tax returns included penalty payments for people who
went without coverage in 2014.

The House leadership aide said that lowering the cost of insurance was a
much better way to encourage people to opt in.

``We would like to get to a point where we have what we call universal
access, where everybody is able to access coverage to some degree or
another,'' the aide said. ``Over the past six years, if you look at the
experience we've had with the A.C.A. rollout, chasing coverage doesn't
necessarily yield great outcomes. You can have people going into an
exchange, finding out that their pediatrician is no longer available to
them.''

The aide said House Republicans had not decided on the future of
cost-sharing subsidies that are paid by the federal government to
insurance companies. Such subsidies are intended to reduce out-of-pocket
costs for millions of low-income people buying insurance under the
Affordable Care Act.

A federal district judge, responding to a lawsuit filed by the House,
ruled in May that the Obama administration had paid billions of dollars
to insurers since January 2014 even though Congress had not appropriated
money for such ``cost-sharing reductions,'' and that the payments
therefore violated the Constitution. Without that money, estimated at
\$130 billion over 10 years, insurers would increase premiums or pull
out of the insurance exchanges, creating chaos for consumers, some
health policy experts say.

But now that the House leadership has won a legal victory, Republicans
have not decided how to proceed. The aide declined on Thursday to say if
Republicans would seek an immediate halt to the cost-sharing subsidy
payments. He did not rule out the possibility that a
Republican-controlled Congress might keep the money flowing for a
transition period, to stabilize the market while Republicans develop
alternatives to the health law.

``It's an open question,'' he said.

Republicans said they were more interested in vindicating Congress's
constitutional power of the purse.

In any event, the House leadership aide said, Republicans do not intend
to pull the rug out from people who have gained insurance under the
Affordable Care Act.

Advertisement

\protect\hyperlink{after-bottom}{Continue reading the main story}

\hypertarget{site-index}{%
\subsection{Site Index}\label{site-index}}

\hypertarget{site-information-navigation}{%
\subsection{Site Information
Navigation}\label{site-information-navigation}}

\begin{itemize}
\tightlist
\item
  \href{https://help.nytimes3xbfgragh.onion/hc/en-us/articles/115014792127-Copyright-notice}{©~2020~The
  New York Times Company}
\end{itemize}

\begin{itemize}
\tightlist
\item
  \href{https://www.nytco.com/}{NYTCo}
\item
  \href{https://help.nytimes3xbfgragh.onion/hc/en-us/articles/115015385887-Contact-Us}{Contact
  Us}
\item
  \href{https://www.nytco.com/careers/}{Work with us}
\item
  \href{https://nytmediakit.com/}{Advertise}
\item
  \href{http://www.tbrandstudio.com/}{T Brand Studio}
\item
  \href{https://www.nytimes3xbfgragh.onion/privacy/cookie-policy\#how-do-i-manage-trackers}{Your
  Ad Choices}
\item
  \href{https://www.nytimes3xbfgragh.onion/privacy}{Privacy}
\item
  \href{https://help.nytimes3xbfgragh.onion/hc/en-us/articles/115014893428-Terms-of-service}{Terms
  of Service}
\item
  \href{https://help.nytimes3xbfgragh.onion/hc/en-us/articles/115014893968-Terms-of-sale}{Terms
  of Sale}
\item
  \href{https://spiderbites.nytimes3xbfgragh.onion}{Site Map}
\item
  \href{https://help.nytimes3xbfgragh.onion/hc/en-us}{Help}
\item
  \href{https://www.nytimes3xbfgragh.onion/subscription?campaignId=37WXW}{Subscriptions}
\end{itemize}
