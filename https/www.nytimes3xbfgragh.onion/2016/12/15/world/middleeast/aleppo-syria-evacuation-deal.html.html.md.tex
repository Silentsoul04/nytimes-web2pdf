Sections

SEARCH

\protect\hyperlink{site-content}{Skip to
content}\protect\hyperlink{site-index}{Skip to site index}

\href{https://www.nytimes3xbfgragh.onion/section/world/middleeast}{Middle
East}

\href{https://myaccount.nytimes3xbfgragh.onion/auth/login?response_type=cookie\&client_id=vi}{}

\href{https://www.nytimes3xbfgragh.onion/section/todayspaper}{Today's
Paper}

\href{/section/world/middleeast}{Middle East}\textbar{}Aleppo Evacuation
Effort Restarts, and Assad Calls It History in the Making

\url{https://nyti.ms/2hJ0FHg}

\begin{itemize}
\item
\item
\item
\item
\item
\item
\end{itemize}

Advertisement

\protect\hyperlink{after-top}{Continue reading the main story}

Supported by

\protect\hyperlink{after-sponsor}{Continue reading the main story}

\hypertarget{aleppo-evacuation-effort-restarts-and-assad-calls-it-history-in-the-making}{%
\section{Aleppo Evacuation Effort Restarts, and Assad Calls It History
in the
Making}\label{aleppo-evacuation-effort-restarts-and-assad-calls-it-history-in-the-making}}

\includegraphics{https://static01.graylady3jvrrxbe.onion/images/2016/12/16/world/16Syria3/16Syria3-articleInline.jpg?quality=75\&auto=webp\&disable=upscale}

By \href{http://www.nytimes3xbfgragh.onion/by/ben-hubbard}{Ben Hubbard}
and \href{https://www.nytimes3xbfgragh.onion/by/hwaida-saad}{Hwaida
Saad}

\begin{itemize}
\item
  Dec. 15, 2016
\item
  \begin{itemize}
  \item
  \item
  \item
  \item
  \item
  \item
  \end{itemize}
\end{itemize}

BEIRUT, Lebanon --- After months of fierce bombardment and failed
diplomacy, the Syrian government began removing residents from the last
rebel-held districts in the city of Aleppo on Thursday, a process that
solidifies President Bashar al-Assad's control over the country's
largest city.

Cold, hungry and carrying satchels and children, about 1,000 people,
some of them wounded, boarded
\href{https://www.nytimes3xbfgragh.onion/2016/10/30/world/middleeast/once-propelled-by-hope-for-a-modern-syria-green-buses-now-run-on-tears.html}{green
buses} and ambulances that carried the first batch of evacuees out of
the rebel enclave. A second group of more than 1,000 people departed
later, and a third left after nightfall.

Mr. Assad hailed the evacuation in
\href{https://www.youtube.com/watch?v=oLP6-jilHgU\&feature=youtu.be}{a
video released by his office}, saying the ``liberation'' of Aleppo would
serve as a historical watershed, like the birth of Christ, the
revelation of the Quran, the collapse of the Soviet Union and the two
world wars.

``I want to confirm that what is happening today is history that is
being written by every Syrian citizen,'' Mr. Assad said, smiling in a
dark blue suit. ``Its writing did not start today. It started about six
years ago when the crisis and the war
on\href{https://www.nytimes3xbfgragh.onion/topic/destination/syria?8qa}{Syria}
began.''

\includegraphics{https://static01.graylady3jvrrxbe.onion/images/2016/12/16/world/16Syria2/16Syria2-videoSixteenByNineJumbo1600.jpg}

It remained unclear how many people, both civilians and fighters,
remained in eastern Aleppo, but aid workers guessed it would take
several days to get them out.

If successful, the evacuation will return the whole city to Mr. Assad's
control, signaling a major turning point in the nearly six-year
conflict. Since early in the war, Aleppo, once the country's industrial
center, has been split, with the government holding the west and rebels
holding neighborhoods in the east.

But vast military aid for the Syrian government from Iran, Russia and
the Lebanese militant group Hezbollah turned the tide in the
government's favor, allowing it to surround the rebels.
The\href{https://www.nytimes3xbfgragh.onion/2016/12/13/world/middleeast/syria-aleppo-civilians.html?rref=collection\%2Fbyline\%2Fanne-barnard\&action=click\&contentCollection=undefined\&region=stream\&module=stream_unit\&version=latest\&contentPlacement=3\&pgtype=collection}{evacuation
deal} was reached between Russia and Turkey, which backs the rebels,
after months of heavy shelling and airstrikes that left entire
neighborhoods in ruins and killed hundreds of people.

Rebels in east Aleppo had also frequently shelled government-held areas,
killing civilians there, too. Many people on that side of the city
considered the rebels dangerous militants and were glad to see them
ousted.

\includegraphics{https://static01.graylady3jvrrxbe.onion/images/2016/12/16/world/16Syria2/16Syria2-videoSixteenByNineJumbo1600.jpg}

The United States, which has supported the rebels along with Turkey,
Saudi Arabia and other Gulf states, has struggled and failed to reach an
accord to end the war, and the United States did not help broker the
evacuation agreement.

``There is absolutely no justification whatsoever for the indiscriminate
and savage brutality against civilians shown by the regime and by its
Russian and Iranian allies over the past few weeks, or indeed over the
past five years,'' Secretary of State John Kerry said. ``The Assad
regime is actually carrying out nothing short of a massacre.''

The loss of a foothold in Aleppo would be a major blow to the
opposition, which would then hold sway in only one of Syria's provincial
capitals, Idlib, where the Syrian branch of Al Qaeda is a major force.
The less territory the rebels hold, the harder it is for the United
States and its allies to get the Syrian government to negotiate with
them, the Obama administration's goal.

Residents reached in eastern Aleppo on Thursday said they felt the
gravity of leaving places where they had lived and suffered for so long.
Some bid farewell to places and things they would miss, while others
pondered why the opposition had failed.

Realizing he would be leaving soon, Hisham Skeif said he had visited his
family's home and sat by an old tree, where he thought about Native
Americans and Palestinians --- other peoples forced from their homes.

``I never understood such feelings until today,'' he said. ``This is my
great-grandfather's land. It is not only my memories, but theirs as
well.''

While he had participated in the uprising against Mr. Assad that started
the conflict, he felt that it had gone wrong by resorting to arms.
``Today, hours before we leave, I say that our biggest mistake was
carrying weapons,'' he said.

Others assumed that as declared enemies of the Assad government, they
would never be allowed to return and left messages for those who would
come later.

\includegraphics{https://static01.graylady3jvrrxbe.onion/images/2016/12/14/world/video-aleppo2/video-aleppo2-videoSixteenByNineJumbo1600.jpg}

Salem Abualnaser, a dentist,
\href{https://www.facebookcorewwwi.onion/photo.php?fbid=1722860838030708\&set=a.1439006866416108.1073741829.100009204267784\&type=3\&theater}{left
a handwritten note on a whiteboard} near the door of a small community
center for children.

``Warning! Do not destroy! There are things here that your children may
benefit from,'' it read. Reached in a Facebook call on Thursday, Mr.
Abualnaser said he had put the sign up because he had heard that
government troops had looted other neighborhoods after seizing them from
the rebels.

``The situation here is better today and people are relaxed a little bit
because all the shelling and airstrikes that we had yesterday have
stopped,'' he said.

Others appeared to be covering their tracks as they left east Aleppo.
Television images of the rebel enclave showed rising columns of smoke,
which activists said were because fighters had set fire to their
headquarters and ammunition stores.

\includegraphics{https://static01.graylady3jvrrxbe.onion/images/2016/12/16/world/16Syria5/16Syria5-articleInline.jpg?quality=75\&auto=webp\&disable=upscale}

The evacuation's start
\href{https://www.nytimes3xbfgragh.onion/2016/12/14/world/middleeast/aleppo-syria-evacuation-deal.html?rref=collection\%2Fbyline\%2Fanne-barnard\&action=click\&contentCollection=undefined\&region=stream\&module=stream_unit\&version=latest\&contentPlacement=2\&pgtype=collection}{was
delayed} twice, once Wednesday morning when militia fighters unhappy
with the deal fired on vehicles trying to bring people out, and on
Thursday, when gunfire targeted rescue workers clearing rubble from a
road, killing one and wounding three, according to Maan al-Shanan, an
antigovernment activist.

The \href{http://www.syriahr.com/en/}{Syrian Observatory for Human
Rights}, which opposes the government and tracks the conflict from
Britain, said that the presence of 250 foreign fighters among those to
be evacuated also complicated the deal because the Syrian government
wanted to detain them for interrogation.

It remained unclear how that issue was resolved.

But evacuation efforts restarted Thursday morning, with apparent
modifications to the plan.

Syrian state news media reported that buses and ambulances had begun
evacuating residents from two Shiite villages in neighboring Idlib
Province that have long been surrounded by Sunni rebels.

\href{https://www.nytimes3xbfgragh.onion/interactive/2016/12/14/world/middleeast/aleppo-siege-audio-video.html}{}

\includegraphics{https://static01.graylady3jvrrxbe.onion/images/2016/12/14/world/14aleppoteacher/14aleppoteacher-thumbLarge.png}

\hypertarget{one-mans-view-into-the-last-days-in-rebel-held-aleppo}{%
\subsection{One Man's View Into the Last Days in Rebel-Held
Aleppo}\label{one-mans-view-into-the-last-days-in-rebel-held-aleppo}}

Messages posted to a WhatsApp group by an English teacher in Aleppo
describe what he saw in the shrinking rebel-held territory, as Syrian
forces and their allies waged an assault to retake the city.

The two villages, Fua and Kfraya, were not originally part of the
evacuation deal. The moves to evacuate people from the villages
suggested that they had been added to the deal to ensure that people in
eastern Aleppo would be allowed to leave.

Live images on Syrian state television showed the line of green buses
and ambulances emerging from the rebel enclave into a government-held
area, and antigovernment activists filmed their arrival in rebel-held
territory further west, where many evacuees broke into tears and ate
apples and cookies handed to them.

Under the agreement, civilians evacuated from eastern Aleppo can stay in
government-controlled sectors or continue to rebel-held areas. Rebels
may only go to other rebel-held areas.

Most were likely to end up in Idlib Province to the west of Aleppo,
where a rebel alliance that includes the Syrian affiliate of Al Qaeda
dominates. Some aid workers feared that moving the Aleppo residents
there would merely lay the groundwork for a future siege.

Image

A bus carried families out of Aleppo on Thursday. The evacuation deal
was reached between Russia, which backs the Syrian government, and
Turkey, which supports the opposition.Credit...Karam Al-Masri/Agence
France-Presse --- Getty Images

``I don't know what will happen in Idlib, but if there is no cease-fire
or political accord then it will become the next Aleppo,'' said Staffan
de Mistura, the United Nations envoy for Syria.

When east Aleppo was besieged by government forces and their allies, the
United Nations reported that more than 250,000 people were trapped
there.

But on Thursday, Lt. Gen. Viktor Poznikhir of the Russian military's
general staff said that 900 militants had been killed in the recent
offensive against the rebel-held neighborhoods and that more than
108,000 civilians had fled the area, including more than 3,000 rebels.

Speaking to journalists in Geneva, Jan Egeland, the United Nations
humanitarian adviser for Syria, said that an estimated 50,000 people had
fled eastern Aleppo.

The discrepancy in the numbers could not immediately be resolved.

The United Nations has not been able to provide protection for those
fleeing, he said, because its workers were not given access.

``We have not been witnesses to atrocities that we know have been
committed by all sides in this horrific war,'' Mr. Egeland added.

The plight of those trapped in eastern Aleppo as supplies have dwindled
and destruction has spread has caused anguish among humanitarian
workers.

``It took 4,000 years to build Aleppo, hundreds of generations, yet one
generation managed to tear it down in four years,'' Mr. Egeland said.

Advertisement

\protect\hyperlink{after-bottom}{Continue reading the main story}

\hypertarget{site-index}{%
\subsection{Site Index}\label{site-index}}

\hypertarget{site-information-navigation}{%
\subsection{Site Information
Navigation}\label{site-information-navigation}}

\begin{itemize}
\tightlist
\item
  \href{https://help.nytimes3xbfgragh.onion/hc/en-us/articles/115014792127-Copyright-notice}{©~2020~The
  New York Times Company}
\end{itemize}

\begin{itemize}
\tightlist
\item
  \href{https://www.nytco.com/}{NYTCo}
\item
  \href{https://help.nytimes3xbfgragh.onion/hc/en-us/articles/115015385887-Contact-Us}{Contact
  Us}
\item
  \href{https://www.nytco.com/careers/}{Work with us}
\item
  \href{https://nytmediakit.com/}{Advertise}
\item
  \href{http://www.tbrandstudio.com/}{T Brand Studio}
\item
  \href{https://www.nytimes3xbfgragh.onion/privacy/cookie-policy\#how-do-i-manage-trackers}{Your
  Ad Choices}
\item
  \href{https://www.nytimes3xbfgragh.onion/privacy}{Privacy}
\item
  \href{https://help.nytimes3xbfgragh.onion/hc/en-us/articles/115014893428-Terms-of-service}{Terms
  of Service}
\item
  \href{https://help.nytimes3xbfgragh.onion/hc/en-us/articles/115014893968-Terms-of-sale}{Terms
  of Sale}
\item
  \href{https://spiderbites.nytimes3xbfgragh.onion}{Site Map}
\item
  \href{https://help.nytimes3xbfgragh.onion/hc/en-us}{Help}
\item
  \href{https://www.nytimes3xbfgragh.onion/subscription?campaignId=37WXW}{Subscriptions}
\end{itemize}
