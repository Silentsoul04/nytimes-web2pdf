Sections

SEARCH

\protect\hyperlink{site-content}{Skip to
content}\protect\hyperlink{site-index}{Skip to site index}

\href{https://www.nytimes3xbfgragh.onion/section/movies}{Movies}

\href{https://myaccount.nytimes3xbfgragh.onion/auth/login?response_type=cookie\&client_id=vi}{}

\href{https://www.nytimes3xbfgragh.onion/section/todayspaper}{Today's
Paper}

\href{/section/movies}{Movies}\textbar{}Review: `Hidden Figures' Honors
3 Black Women Who Helped NASA Soar

\url{https://nyti.ms/2hdUycH}

\begin{itemize}
\item
\item
\item
\item
\item
\end{itemize}

Advertisement

\protect\hyperlink{after-top}{Continue reading the main story}

Supported by

\protect\hyperlink{after-sponsor}{Continue reading the main story}

\hypertarget{review-hidden-figures-honors-3-black-women-who-helped-nasa-soar}{%
\section{Review: `Hidden Figures' Honors 3 Black Women Who Helped NASA
Soar}\label{review-hidden-figures-honors-3-black-women-who-helped-nasa-soar}}

\includegraphics{https://static01.graylady3jvrrxbe.onion/images/2016/12/23/arts/23COVERMOVIES1/23COVERMOVIES1-articleInline.jpg?quality=75\&auto=webp\&disable=upscale}

\begin{itemize}
\tightlist
\item
  Hidden Figures\\
  Directed by Theodore Melfi Biography, Drama, History PG 2h 7m
\end{itemize}

By \href{https://www.nytimes3xbfgragh.onion/by/a-o--scott}{A.O. Scott}

\begin{itemize}
\item
  Dec. 22, 2016
\item
  \begin{itemize}
  \item
  \item
  \item
  \item
  \item
  \end{itemize}
\end{itemize}

``Hidden Figures'' takes us back to 1961, when racial segregation and
workplace sexism were widely accepted facts of life and the word
``computer'' referred to a person, not a machine. Though a gigantic IBM
mainframe does appear in the movie --- big enough to fill a room and
probably less powerful than the phone in your pocket --- the most
important computers are three African-American women who work at NASA
headquarters in Hampton, Va. Assigned to data entry jobs and denied
recognition or promotion, they would go on to play crucial roles in the
American space program.

Based on
\href{http://margotleeshetterly.com/hidden-figures-nasas-african-american-computers/}{Margot
Lee Shetterly's} nonfiction book of the same title, the film, directed
by Theodore Melfi (who wrote the script with Allison Schroeder), turns
the entwined careers of Katherine Goble (later Johnson), Mary Jackson
and Dorothy Vaughan into a rousing celebration of merit rewarded and
perseverance repaid. Like many movies about the overcoming of racism, it
offers belated acknowledgment of bravery and talent and an overdue
reckoning with the sins of the past. And like most movies about
real-world breakthroughs,
\href{https://www.nytimes3xbfgragh.onion/2016/05/22/movies/taraji-p-henson-octavia-spencer-hidden-figures-rocket-science-and-race.html}{``Hidden
Figures''} is content to stay within established conventions. The story
may be new to most viewers, but the manner in which it's told will be
familiar to all but the youngest.

\includegraphics{https://static01.graylady3jvrrxbe.onion/images/2017/01/03/arts/hidden-figures-anatomy-image/08PHARRELL-figures-videoSixteenByNine3000-v2.jpg}

This is not necessarily a bad thing. There is something to be said for a
well-told tale with a clear moral and a satisfying emotional payoff. Mr.
Melfi, whose previous film was the heart-tugging, borderline-treacly
Bill Murray vehicle
\href{https://www.youtube.com/watch?v=9dP5lJnJHXg}{``St. Vincent,''}
knows how to push our emotional buttons without too heavy a hand. He
trusts his own skill, the intrinsic interest of the material and ---
above all --- the talent and dedication of the cast. From one scene to
the next, you may know more or less what is coming, but it is never less
than delightful to watch these actors at work.

Start with the three principals, whose struggles at NASA take place as
the agency is scrambling to send an astronaut into orbit. Katherine
Goble is the central hidden figure, a mathematical prodigy played with
perfect nerd charisma by Taraji P. Henson. Katherine is plucked from the
computing room and assigned to a team that will calculate the launch
coordinates and trajectory for an Atlas rocket. She receives a cold
welcome --- particularly from an engineer named Paul Stafford (Jim
Parsons) --- and is not spared the indignities facing a black woman in a
racially segregated, gender-stratified workplace. The only bathroom she
is allowed to use is in a distant building, and she horrifies her new
co-workers when she helps herself to a cup of coffee.

Dorothy (Octavia Spencer) and Mary (Janelle Monáe) also face
discrimination. Dorothy, who is in charge of several dozen computers, is
repeatedly denied promotion to supervisor and treated with condescension
by her immediate boss (Kirsten Dunst). The Polish-born engineer (Olek
Krupa) with whom Mary works is more enlightened, but Mary runs into the
brick wall of Virginia's Jim Crow laws when she tries to take
graduate-level physics courses.

\includegraphics{https://static01.graylady3jvrrxbe.onion/images/2016/12/23/arts/hidden-movies/hidden-movies-videoSixteenByNineJumbo1600.jpg}

``Hidden Figures'' effectively conveys the poisonous normalcy of white
supremacy, and the main characters' determination to pursue their
ambitions in spite of it and to live normal lives in its shadow. The
racism they face does not depend on the viciousness or virtue of
individual white people, and for the most part the white characters are
not treated as heroes for deciding, at long last, to behave decently.
Two of them, however, are singled out for commendation:
\href{http://www.nytimes3xbfgragh.onion/2016/12/08/us/john-glenn-dies.html}{John
Glenn}, portrayed by Glen Powell as a natural democrat with no time for
racial hierarchies; and Al Harrison, the head of Katherine's group, for
whom the success of the mission is more important than color.

Kevin Costner, who plays Al, is an actor almost uniquely capable of
upstaging through understatement. He is also one of the great
gum-chewers in American cinema, a habit that, along with the flattop
haircut and heavy-framed glasses, gives Al an aura of midcentury
no-nonsense masculine competence. He desegregates the NASA bathrooms
with a sledgehammer and stands up for Katherine in quieter but no less
emphatic ways when her qualifications are challenged.

It's a bit much, maybe, but Mr. Costner, as usual, does what he can to
give the white men of America a good name. The movie, meanwhile, expands
the schoolbook chronicle of the conquest of space beyond the usual
heroes, restoring some of its idealism and grandeur in the process. It
also embeds that history in daily life, departing from the televised
spectacle of liftoffs and landings and the public drama of the civil
rights movement to spend time with its heroines and their families at
home and in church. The sweetest subplot involves the romance between
Katherine, a widow with three daughters, and a handsome military officer
played by Mahershala Ali.

``Hidden Figures'' makes a fascinating and timely companion to
\href{https://www.nytimes3xbfgragh.onion/2016/11/04/movies/loving-review-joel-edgerton-ruth-negga.html}{``Loving,''}
Jeff Nichols's film about the Virginia couple who challenged their
state's law against interracial marriage, which was struck down by the
Supreme Court in 1967. The two movies take place in the same state in
the same era, and focus on the quiet dramas that move history forward.
They introduce you to real people you might wish you had known more
about earlier. They can fill you with outrage at the persistence of
injustice and gratitude toward those who had the grit to stand up
against it.

Advertisement

\protect\hyperlink{after-bottom}{Continue reading the main story}

\hypertarget{site-index}{%
\subsection{Site Index}\label{site-index}}

\hypertarget{site-information-navigation}{%
\subsection{Site Information
Navigation}\label{site-information-navigation}}

\begin{itemize}
\tightlist
\item
  \href{https://help.nytimes3xbfgragh.onion/hc/en-us/articles/115014792127-Copyright-notice}{©~2020~The
  New York Times Company}
\end{itemize}

\begin{itemize}
\tightlist
\item
  \href{https://www.nytco.com/}{NYTCo}
\item
  \href{https://help.nytimes3xbfgragh.onion/hc/en-us/articles/115015385887-Contact-Us}{Contact
  Us}
\item
  \href{https://www.nytco.com/careers/}{Work with us}
\item
  \href{https://nytmediakit.com/}{Advertise}
\item
  \href{http://www.tbrandstudio.com/}{T Brand Studio}
\item
  \href{https://www.nytimes3xbfgragh.onion/privacy/cookie-policy\#how-do-i-manage-trackers}{Your
  Ad Choices}
\item
  \href{https://www.nytimes3xbfgragh.onion/privacy}{Privacy}
\item
  \href{https://help.nytimes3xbfgragh.onion/hc/en-us/articles/115014893428-Terms-of-service}{Terms
  of Service}
\item
  \href{https://help.nytimes3xbfgragh.onion/hc/en-us/articles/115014893968-Terms-of-sale}{Terms
  of Sale}
\item
  \href{https://spiderbites.nytimes3xbfgragh.onion}{Site Map}
\item
  \href{https://help.nytimes3xbfgragh.onion/hc/en-us}{Help}
\item
  \href{https://www.nytimes3xbfgragh.onion/subscription?campaignId=37WXW}{Subscriptions}
\end{itemize}
