Sections

SEARCH

\protect\hyperlink{site-content}{Skip to
content}\protect\hyperlink{site-index}{Skip to site index}

\href{https://www.nytimes3xbfgragh.onion/section/world/middleeast}{Middle
East}

\href{https://myaccount.nytimes3xbfgragh.onion/auth/login?response_type=cookie\&client_id=vi}{}

\href{https://www.nytimes3xbfgragh.onion/section/todayspaper}{Today's
Paper}

\href{/section/world/middleeast}{Middle East}\textbar{}Rebuffing Israel,
U.S. Allows Censure Over Settlements

\url{https://nyti.ms/2hgkJPX}

\begin{itemize}
\item
\item
\item
\item
\item
\item
\end{itemize}

Advertisement

\protect\hyperlink{after-top}{Continue reading the main story}

Supported by

\protect\hyperlink{after-sponsor}{Continue reading the main story}

\hypertarget{rebuffing-israel-us-allows-censure-over-settlements}{%
\section{Rebuffing Israel, U.S. Allows Censure Over
Settlements}\label{rebuffing-israel-us-allows-censure-over-settlements}}

\includegraphics{https://static01.graylady3jvrrxbe.onion/images/2016/12/24/world/24NATIONS/24NATIONS-videoSixteenByNineJumbo1600.jpg}

By \href{http://www.nytimes3xbfgragh.onion/by/somini-sengupta}{Somini
Sengupta} and
\href{https://www.nytimes3xbfgragh.onion/by/rick-gladstone}{Rick
Gladstone}

\begin{itemize}
\item
  Dec. 23, 2016
\item
  \begin{itemize}
  \item
  \item
  \item
  \item
  \item
  \item
  \end{itemize}
\end{itemize}

UNITED NATIONS --- Defying extraordinary pressure from President-elect
Donald J. Trump and furious lobbying by Israel, the Obama administration
on Friday allowed the \href{http://www.un.org/en/sc/}{United Nations
Security Council} to adopt a resolution that condemned Israeli
settlement construction.

The administration's decision not to veto the measure reflected its
accumulated frustration over Israeli settlements. The American
abstention on the vote also broke a longstanding policy of shielding
Israel from action at the United Nations that described the settlements
as illegal.

While the resolution is not expected to have any practical impact on the
ground, it is regarded as a major rebuff to Israel, one that could
increase its isolation over the paralyzed peace process with Israel's
Palestinian neighbors, who have sought to establish their own state on
territory held by Israel.

Applause broke out in the 15-member Security Council's chambers after
the vote on the measure, which passed 14 to 0, with the United States
ambassador, Samantha Power, raising her hand as the lone abstention.
Israel's ambassador, Danny Danon, denounced the measure, and castigated
the council members who had approved it.

``Would you ban the French from building in Paris?'' he told them.

The resolution describes the settlement building as a ``major obstacle''
to peace and demands that Israel stop the construction, which most the
world regards as illegal.

Prime Minister Benjamin Netanyahu of Israel, who had scrambled in recent
days to stop the measure from coming to a vote, issued a blistering
denunciation afterward.

``Israel rejects this shameful anti-Israel resolution at the U.N. and
will not abide by its terms,'' Mr. Netanyahu said in a statement. ``At a
time when the Security Council does nothing to stop the slaughter of
half a million people in Syria, it disgracefully gangs up on the one
true democracy in the Middle East, Israel, and calls the Western Wall
`occupied territory.' ''

Mr. Netanyahu immediately retaliated against two of the countries that
sponsored the resolution. He ordered Israel's ambassadors to New Zealand
and Senegal to return home for consultations, canceled a planned visit
to Israel next month by Senegal's foreign minister and cut off all aid
programs to Senegal.

The vote came a day after
\href{https://www.nytimes3xbfgragh.onion/2016/12/22/world/middleeast/donald-trump-united-nations-israel-settlements.html}{Mr.
Trump personally intervened} to keep the measure, which had been
originally proposed by Egypt, from coming up for a vote on Thursday, as
scheduled. Mr. Trump's aides said he had spoken to Mr. Netanyahu. Both
men also spoke to the Egyptian president, Abdel Fattah el-Sisi. Egypt
postponed the vote under what that country's United Nations ambassador
called intense pressure.

But in a show of mounting exasperation, four other countries on the
Security Council --- Malaysia, New Zealand, Senegal and Venezuela ---
all of them relatively powerless temporary members with rotating
two-year seats, snatched the resolution away from Egypt and put it up
for a vote Friday.

\href{https://www.nytimes3xbfgragh.onion/interactive/2016/12/23/world/middleeast/document-security-council-draft-resolution-israel.html}{}

\includegraphics{https://static01.graylady3jvrrxbe.onion/images/2016/12/23/world/middleeast/image-MEPP-Draft-21-12/image-MEPP-Draft-21-12-square640.gif}

\hypertarget{un-security-council-draft-resolution-on-the-middle-east-peace-process}{%
\subsection{U.N. Security Council Draft Resolution on the Middle East
Peace
Process}\label{un-security-council-draft-resolution-on-the-middle-east-peace-process}}

A United Nations Security Council draft resolution condemning Israel's
settlement building.

The Obama administration has been highly critical of Israel's settlement
building, describing it as an impediment to a two-state solution in the
Israeli-Palestinian conflict that has long been the official United
States position, regardless of the party in power.

Mr. Trump, who had urged the administration to veto the resolution, has
made clear that he will take a far more sympathetic approach to Israel
when his administration assumes office on Jan. 20.

Mr. Trump's comments on the resolution amounted to his most direct
intervention on United States foreign policy during his transition to
power. Minutes after the Security Council vote was announced, Mr. Trump
made his anger known in a
\href{https://twitter.com/realDonaldTrump/status/812390964740427776}{Twitter
posting}, saying: ``As to the U.N., things will be different after Jan.
20th.''

A range of senators and congressmen from both parties also denounced the
resolution, a reflection of the deep loyalty to Israel shared by
Democrats and Republicans. Senator Chuck Schumer of New York said, ``It
is extremely frustrating, disappointing and confounding that the
administration has failed to veto this resolution.''

\includegraphics{https://static01.graylady3jvrrxbe.onion/images/2016/12/23/world/23Nations2/23Nations2-articleLarge.jpg?quality=75\&auto=webp\&disable=upscale}

Senator Lindsey Graham of South Carolina, who oversees a subcommittee
that oversees United Nations funding by the United States,
\href{http://www.lgraham.senate.gov/public/index.cfm/press-releases?ID=3D8C552F-EE98-4A25-942E-5B389F2695F2}{threatened}
to take steps that could ``suspend or significantly reduce'' that
financing.

Reaction to the resolution also illustrated fissures among American Jews
regarding Israeli policy. Some, like the World Jewish Congress and
\href{http://www.ajc.org/site/apps/nlnet/content3.aspx?c=7oJILSPwFfJSG\&b=9302337\&ct=14976765\&notoc=1}{American
Jewish Committee}, called the resolution a one-sided measure that would
not help the peace process. Ronald S. Lauder, president of the World
Jewish Congress, said in a
\href{http://www.worldjewishcongress.org/en/news/wjc-president-lauder-dismayed-by-un-security-council-resolution-on-israeli-settlements-calls-us-choice-not-to-veto-disconcerting-an-unfortunate-12-5-2016}{statement}:
``It is also disconcerting and unfortunate that the United States,
Israel's greatest ally, chose to abstain rather than veto this
counterproductive text.''

Other groups that have grown increasingly critical of the Israeli
government's approach to the peace process applauded the resolution and
the Obama administration's decision not to block it.

J Street, a Washington-based organization that advocates a two-state
solution,
\href{http://jstreet.org/press-releases/j-street-welcomes-us-abstention-unsc-resolution/\#.WF2d7X35PZw}{said}
the resolution ``conveys the overwhelming support of the international
community, including Israel's closest friends and allies, for the
two-state solution, and their deep concern over the deteriorating status
quo between Israelis and Palestinians and the lack of meaningful
progress toward peace.''

Ms. Power, the United States ambassador, portrayed the abstention as
consistent with the American disapproval of settlement-building, but she
also criticized countries at the United Nations for treating Israel
unfairly. She said the United States remained committed to its
``steadfast support'' for Israel and reminded the council that Israel
received an enormous amount of American military aid.

Ms. Power said the United States chose not to veto the resolution, as it
had done to a similar measure
\href{http://www.nytimes3xbfgragh.onion/2011/02/19/world/middleeast/19nations.html}{under
Mr. Obama in 2011}, because settlement building had accelerated so much
that it had put the two-state solution in jeopardy, and because the
peace process had gone nowhere.

``Today the Security Council reaffirmed its established consensus that
settlements have no legal validity,'' she said. ``The United States has
been sending a message that settlements must stop privately and publicly
for nearly five decades.''

She also rebuked Palestinian leaders for ``too often'' failing to
condemn violence against Israeli civilians. But she directed a portion
of her remarks to Mr. Netanyahu, whose relations with the Obama
administration have never been warm.

``One cannot simultaneously champion expanding Israeli settlements and
champion a viable two-state solution that would end the conflict,'' she
said, arguing that the settlements have undermined Israel's security.

Israel's ambassador, Mr. Danon, who had exhorted the American delegation
to block the measure, expressed his anger in a statement that looked
forward to a change in policy under Mr. Trump.

``It was to be expected that Israel's greatest ally would act in
accordance with the values that we share and that they would have vetoed
this disgraceful resolution,'' he said.

Riyad Mansour, the Palestinian ambassador to the United Nations,
welcomed the resolution's adoption but tempered his approval with a
warning. ``In reality, today's action may be too little too late,'' he
said. ``After years of allowing the law to be trampled and the situation
to spiral downward, today's resolution may rightly be seen as a last
attempt to preserve the two-state solution and revive the path for
peace.''

The resolution condemned Israeli housing construction in East Jerusalem
and the occupied West Bank as a ``flagrant violation under international
law'' that was ``dangerously imperiling the viability'' of a future
peace settlement establishing a
\href{http://topics.nytimes3xbfgragh.onion/top/reference/timestopics/subjects/p/palestinians/index.html?inline=nyt-classifier}{Palestinian}
state.

The resolution also included a nod to Israel and its backers by
condemning ``all acts of violence against civilians, including acts of
terror, as well as all acts of provocation, incitement and
destruction.'' That language is diplomatic scolding aimed at Palestinian
leaders, whom Israel accuses of encouraging attacks on Israeli
civilians.

Hamas, the Palestinian group that controls the Gaza Strip and is deemed
a terrorist organization by the United States and Israel, expressed
appreciation to the Security Council. ``We praise the countries that
voted for the resolution,'' said Hazem Kassem, a spokesman for the
group. ``We emphasize the need to turn such a resolution into action,
not only to halt settlements but to eradicate Israel's occupation in all
its forms.''

Advertisement

\protect\hyperlink{after-bottom}{Continue reading the main story}

\hypertarget{site-index}{%
\subsection{Site Index}\label{site-index}}

\hypertarget{site-information-navigation}{%
\subsection{Site Information
Navigation}\label{site-information-navigation}}

\begin{itemize}
\tightlist
\item
  \href{https://help.nytimes3xbfgragh.onion/hc/en-us/articles/115014792127-Copyright-notice}{©~2020~The
  New York Times Company}
\end{itemize}

\begin{itemize}
\tightlist
\item
  \href{https://www.nytco.com/}{NYTCo}
\item
  \href{https://help.nytimes3xbfgragh.onion/hc/en-us/articles/115015385887-Contact-Us}{Contact
  Us}
\item
  \href{https://www.nytco.com/careers/}{Work with us}
\item
  \href{https://nytmediakit.com/}{Advertise}
\item
  \href{http://www.tbrandstudio.com/}{T Brand Studio}
\item
  \href{https://www.nytimes3xbfgragh.onion/privacy/cookie-policy\#how-do-i-manage-trackers}{Your
  Ad Choices}
\item
  \href{https://www.nytimes3xbfgragh.onion/privacy}{Privacy}
\item
  \href{https://help.nytimes3xbfgragh.onion/hc/en-us/articles/115014893428-Terms-of-service}{Terms
  of Service}
\item
  \href{https://help.nytimes3xbfgragh.onion/hc/en-us/articles/115014893968-Terms-of-sale}{Terms
  of Sale}
\item
  \href{https://spiderbites.nytimes3xbfgragh.onion}{Site Map}
\item
  \href{https://help.nytimes3xbfgragh.onion/hc/en-us}{Help}
\item
  \href{https://www.nytimes3xbfgragh.onion/subscription?campaignId=37WXW}{Subscriptions}
\end{itemize}
