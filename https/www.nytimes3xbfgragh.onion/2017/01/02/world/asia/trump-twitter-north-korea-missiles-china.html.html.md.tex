Sections

SEARCH

\protect\hyperlink{site-content}{Skip to
content}\protect\hyperlink{site-index}{Skip to site index}

\href{https://www.nytimes3xbfgragh.onion/section/world/asia}{Asia
Pacific}

\href{https://myaccount.nytimes3xbfgragh.onion/auth/login?response_type=cookie\&client_id=vi}{}

\href{https://www.nytimes3xbfgragh.onion/section/todayspaper}{Today's
Paper}

\href{/section/world/asia}{Asia Pacific}\textbar{}`It Won't Happen,'
Donald Trump Says of North Korean Missile Test

\url{https://nyti.ms/2hLWbyo}

\begin{itemize}
\item
\item
\item
\item
\item
\end{itemize}

Advertisement

\protect\hyperlink{after-top}{Continue reading the main story}

Supported by

\protect\hyperlink{after-sponsor}{Continue reading the main story}

\hypertarget{it-wont-happen-donald-trump-says-of-north-korean-missile-test}{%
\section{`It Won't Happen,' Donald Trump Says of North Korean Missile
Test}\label{it-wont-happen-donald-trump-says-of-north-korean-missile-test}}

\includegraphics{https://static01.graylady3jvrrxbe.onion/images/2017/01/04/world/03TRUMPKOREA-1/03TRUMPKOREA-1-videoSixteenByNineJumbo1600.jpg}

By \href{http://www.nytimes3xbfgragh.onion/by/maggie-haberman}{Maggie
Haberman} and
\href{http://www.nytimes3xbfgragh.onion/by/david-e-sanger}{David E.
Sanger}

\begin{itemize}
\item
  Jan. 2, 2017
\item
  \begin{itemize}
  \item
  \item
  \item
  \item
  \item
  \end{itemize}
\end{itemize}

\href{http://cn.nytimes3xbfgragh.onion/asia-pacific/20170104/trump-twitter-north-korea-missiles-china/}{阅读简体中文版}

Faced with a threat from North Korea that it might soon test an
intercontinental ballistic missile, President-elect Donald J. Trump
\href{https://twitter.com/realDonaldTrump/status/816057920223846400?lang=en}{took
to Twitter} on Monday to declare bluntly, ``It won't happen!''

Mr. Trump made his post on Twitter, where he often tests out his first
thoughts on developing issues in the United States and abroad, a day
after North Korea's young leader, Kim Jong-un,
\href{https://www.nytimes3xbfgragh.onion/2017/01/01/world/asia/north-korea-intercontinental-ballistic-missile-test-kim-jong-un.html}{declared}
that the ``final stage in preparations'' was underway for a test of such
a missile. Mr. Kim offered no time frame. North Korea has routinely
tested short- and medium-range missiles, with some successes and many
failures, but it has so far stopped short of testing a long-range
missile, which could reach Guam or the West Coast of the United States.

``North Korea just stated that it is in the final stages of developing a
nuclear weapon capable of reaching parts of the U.S.,'' Mr. Trump wrote,
somewhat misstating Mr. Kim's warning. Pyongyang has already tested
nuclear weapons underground; the latest threat concerned what Mr. Kim
called a ``test launch of an intercontinental ballistic missile.'' But
Mr. Kim also boasted last year that the North had conducted ``the first
H-bomb test,'' and experts say there is
\href{https://www.nytimes3xbfgragh.onion/2016/01/07/world/asia/north-korea-hydrogen-bomb-q-a.html}{no
evidence} for that claim.

After his first Twitter message,
\href{https://twitter.com/realDonaldTrump/status/816068355555815424}{Mr.
Trump added}: ``China has been taking out massive amounts of money \&
wealth from the U.S. in totally one-sided trade, but won't help with
North Korea. Nice!'' That appeared to reflect briefings Mr. Trump has
received about how Chinese leaders fear instability and collapse in the
North more than the status quo.

Mr. Trump takes office in less than three weeks, and a test by North
Korea, if it demonstrated that the missile could in fact reach American
shores, would present one of the first big national security tests for
his administration.

A spokesman for the Chinese Ministry of Foreign Affairs rejected Mr.
Trump's criticisms and appeared to suggest that such comments could
inflame tensions with North Korea.

``We hope that all sides avoid using words and actions that lead to
escalating tensions,'' the spokesman, Geng Shuang, said at a regular
news briefing in Beijing on Tuesday when asked about Mr. Trump's
messages. Mr. Geng said that China was committed to using negotiations
to defuse the standoff over North Korea's nuclear weapons.

China, he said, ``has made tremendous efforts to promote a peaceful and
effective solution to the North Korean nuclear issue.''

There was no immediate comment from either North or South Korea on Mr.
Trump's latest remarks.

North Korea conducted a nuclear test in the first months of the Obama
administration, turning many White House officials against the country
--- and against the concept of negotiating with it. Early in the
presidential campaign, Mr. Trump said he was willing to sit down with
Mr. Kim and perhaps have a hamburger with him. But negotiating with the
North would be anathema to many Republicans, and even Mr. Obama, who was
willing to talk with the leaders of Cuba and Myanmar, refused to enter
negotiations with the North unless it acknowledged that the endpoint of
the talks would be a nuclear-free Korean Peninsula.

In his New Year's Day speech, Mr. Kim said he would continue his
country's efforts to build a nuclear-strike capability unless the United
States abandoned its ``hostile'' policy toward the North.

Some analysts have predicted that North Korea will conduct a weapons
test in the coming months, taking advantage of leadership changes in
both the United States and South Korea.

How Mr. Trump would respond to such a provocation is a matter of great
concern for South Koreans, who are also struggling with uncertainty in
their domestic politics. South Korea's Parliament
\href{https://www.nytimes3xbfgragh.onion/2016/12/09/world/asia/south-korea-president-park-geun-hye-impeached.html}{voted
on Dec. 9 to impeach} President Park Geun-hye over a corruption scandal.
If the nation's Constitutional Court decides to formally remove her from
office, the country will have a new election within months.

Advertisement

\protect\hyperlink{after-bottom}{Continue reading the main story}

\hypertarget{site-index}{%
\subsection{Site Index}\label{site-index}}

\hypertarget{site-information-navigation}{%
\subsection{Site Information
Navigation}\label{site-information-navigation}}

\begin{itemize}
\tightlist
\item
  \href{https://help.nytimes3xbfgragh.onion/hc/en-us/articles/115014792127-Copyright-notice}{©~2020~The
  New York Times Company}
\end{itemize}

\begin{itemize}
\tightlist
\item
  \href{https://www.nytco.com/}{NYTCo}
\item
  \href{https://help.nytimes3xbfgragh.onion/hc/en-us/articles/115015385887-Contact-Us}{Contact
  Us}
\item
  \href{https://www.nytco.com/careers/}{Work with us}
\item
  \href{https://nytmediakit.com/}{Advertise}
\item
  \href{http://www.tbrandstudio.com/}{T Brand Studio}
\item
  \href{https://www.nytimes3xbfgragh.onion/privacy/cookie-policy\#how-do-i-manage-trackers}{Your
  Ad Choices}
\item
  \href{https://www.nytimes3xbfgragh.onion/privacy}{Privacy}
\item
  \href{https://help.nytimes3xbfgragh.onion/hc/en-us/articles/115014893428-Terms-of-service}{Terms
  of Service}
\item
  \href{https://help.nytimes3xbfgragh.onion/hc/en-us/articles/115014893968-Terms-of-sale}{Terms
  of Sale}
\item
  \href{https://spiderbites.nytimes3xbfgragh.onion}{Site Map}
\item
  \href{https://help.nytimes3xbfgragh.onion/hc/en-us}{Help}
\item
  \href{https://www.nytimes3xbfgragh.onion/subscription?campaignId=37WXW}{Subscriptions}
\end{itemize}
