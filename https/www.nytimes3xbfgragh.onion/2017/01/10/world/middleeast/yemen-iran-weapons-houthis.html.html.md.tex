Sections

SEARCH

\protect\hyperlink{site-content}{Skip to
content}\protect\hyperlink{site-index}{Skip to site index}

\href{https://www.nytimes3xbfgragh.onion/section/world/middleeast}{Middle
East}

\href{https://myaccount.nytimes3xbfgragh.onion/auth/login?response_type=cookie\&client_id=vi}{}

\href{https://www.nytimes3xbfgragh.onion/section/todayspaper}{Today's
Paper}

\href{/section/world/middleeast}{Middle East}\textbar{}Arms Seized Off
Coast of Yemen Appear to Have Been Made in Iran

\url{https://nyti.ms/2igU0U1}

\begin{itemize}
\item
\item
\item
\item
\item
\end{itemize}

Advertisement

\protect\hyperlink{after-top}{Continue reading the main story}

Supported by

\protect\hyperlink{after-sponsor}{Continue reading the main story}

\hypertarget{arms-seized-off-coast-of-yemen-appear-to-have-been-made-in-iran}{%
\section{Arms Seized Off Coast of Yemen Appear to Have Been Made in
Iran}\label{arms-seized-off-coast-of-yemen-appear-to-have-been-made-in-iran}}

By \href{http://www.nytimes3xbfgragh.onion/by/c-j-chivers}{C. J.
Chivers} and
\href{http://www.nytimes3xbfgragh.onion/by/eric-schmitt}{Eric Schmitt}

\begin{itemize}
\item
  Jan. 10, 2017
\item
  \begin{itemize}
  \item
  \item
  \item
  \item
  \item
  \end{itemize}
\end{itemize}

Photographs recently released by the Australian government show that
light anti-armor weapons seized from a smuggling vessel near
\href{https://www.nytimes3xbfgragh.onion/topic/destination/yemen?8qa}{Yemen}'s
coast appear to have been manufactured in
\href{https://www.nytimes3xbfgragh.onion/topic/destination/iran?8qa}{Iran},
further suggesting that Tehran has had a hand in a high-seas gunrunning
operation to the Horn of Africa and the Arabian Peninsula.

The weapons, a selection of at least nine rocket-propelled grenade
launchers, were among thousands of weapons seized by an Australian
warship, the Darwin, in February from an Iranian dhow that was sailing
under the name Samer. The photographs of the weapons, a sample of the
much larger quantity of arms, were obtained by the
\href{http://www.smallarmssurvey.org/}{Small Arms Survey}, a
Geneva-based international research center, after a long open-records
dispute with the Australian military.

Iran has been repeatedly accused of providing arms helping to fuel one
side of the war in Yemen, in which rebels from the country's north,
known as the Houthis, ousted the government from the capital, Sana, in
2014. The United States and other Western governments have provided vast
quantities of weapons, and other forms of military support, to the
embattled government and its allies in a
\href{https://www.nytimes3xbfgragh.onion/2016/10/10/world/middleeast/yemen-saudi-arabia-military.html?action=click\&contentCollection=Middle\%20East\&module=RelatedCoverage\&region=Marginalia\&pgtype=article}{coalition
led by Saudi Arabia}, contributing to violence that the United Nations
said last year had caused more than 10,000 civilian casualties.

Matthew Schroeder, an analyst for the survey, said a study of the
weapons' characteristics and factory markings had showed that they match
Iranian-made rocket-propelled grenade launchers previously documented in
Iraq in 2008 and 2015, and in Ivory Coast in 2014 and 2015.

That finding follows
\href{https://www.washingtonpost.com/news/checkpoint/wp/2016/11/30/how-iranian-weapons-are-ending-up-in-yemen/?utm_term=.f5126d97f1e7}{a
report late last year} by Conflict Armament Research, a private arms
consultancy, that said the available evidence pointed to an apparent
``weapon pipeline, extending from Iran to Somalia and Yemen, which
involves the transfer, by dhow, of significant quantities of
Iranian-manufactured weapons and weapons that plausibly derive from
Iranian stockpiles.''

For years, Iran has been under a series of international sanctions
prohibiting it from exporting arms. The United States has frequently
claimed that Tehran has violated the sanctions in support of proxy
forces in many conflicts, including in Iraq, Syria, Yemen and the
Palestinian territories.

\includegraphics{https://static01.graylady3jvrrxbe.onion/images/2017/01/11/world/11Yemen1/11Yemen1-articleLarge.jpg?quality=75\&auto=webp\&disable=upscale}

The grenade launchers that were the subject of Mr. Schroeder's analysis
are the central component of a reusable weapon system commonly called
RPG-7s.

They were among 81 launchers seized on the Samer by Australian sailors,
part of a hidden cargo that included 1,968 Kalashnikov assault rifles,
49 PK machine guns, 41 spare machine-gun barrels and 20 60-millimeter
mortar tubes --- enough weapons to arm a potent ground force.

Although the evidence was not conclusive, Mr. Schroeder said, ``the
seizure appears to be yet another example of Iranian weapons being
shipped abroad despite longstanding U.N. restrictions on arms transfers
from Iran.''

With Iran observing three days of mourning following the death of
\href{https://www.nytimes3xbfgragh.onion/2017/01/10/world/middleeast/iran-rafsanjani-funeral-protests.html?ref=world}{Ayatollah
Ali Akbar Hashemi Rafsanjani,} it was not possible to contact the
government for comment. But on previous occasions, Iran has refused to
respond to inquiries about the smuggling.

The Samer episode was one of four interdictions of Iranian dhows from
September 2015 through March 2016 that yielded, in total, more than 80
antitank guided missiles and 5,000 Kalashnikov rifles as well as sniper
rifles, machine guns and almost 300 RPG launchers, according to data
provided by the United States Navy.

In 2013, the Navy
\href{http://www.nytimes3xbfgragh.onion/2013/03/03/world/middleeast/seized-arms-off-yemen-raise-alarm-over-iran.html}{stopped
another dhow} off the Yemeni coast and found it to be carrying
shoulder-fired antiaircraft missiles and launchers, rifle and
machine-gun cartridges, C4 plastic explosives, night-vision equipment
and other military items.

In an interview in Bahrain, Vice Adm. Kevin M. Donegan, the commander of
the Navy's Fifth Fleet, suggested that these seizures were part of a
larger effort by Iran to move weapons to
\href{https://www.nytimes3xbfgragh.onion/2016/11/26/world/middleeast/houthi-rebels-yemen.html?rref=collection\%2Ftimestopic\%2FYemen\&action=click\&contentCollection=world\&region=stream\&module=stream_unit\&version=latest\&contentPlacement=10\&pgtype=collection}{the
Houthis}.

Image

A grenade launcher that was part of the weapons
cache.Credit...Australian Department of Defense, via Small Arms Survey

``Absolutely it's not everything,'' he said of the four seizures in 2015
and 2016. ``These are the ones that I know of because we were able to
interdict them.''

Admiral Donegan noted, however, that the captains operating the vessels
are typically ``out-of-work fishermen, smugglers; they're not
necessarily working for the government'' of Iran. He added that the
evidence of Iran's hand in the shipments, while strong, was not
ironclad.

This echoed the report by Conflict Armament Research, which said that
antitank weapons apparently seized in Yemen have matched lot numbers for
the same class of weapons seized on Iranian dhows but stopped short of
claiming to have clear proof of an Iranian government hand.

The consultancy also documented weapons manufactured by China, Russia,
Romania, Bulgaria and perhaps in North Korea in seizures from the dhows.

The consultancy also did not suggest that the evidence indicated a
direct handoff of weapons from the dhows to Houthi forces. Rather, it
said, the weapons appear to be offloaded in Somalia and transferred to
smaller vessels for smuggling into southern Yemen.

Weapons from Iranian dhows would not be alone in reaching the conflict,
which has been fueled in part by extensive arms transfers by outside
governments.

Western governments, including those of the United States, Britain and
Canada, have provided billions of dollars worth of weapons and military
equipment, as well as intelligence and logistics support, to the
Saudi-led coalition, which has been waging an extensive bombing campaign
against the Houthis.

Among
\href{https://www.nytimes3xbfgragh.onion/2016/11/14/world/middleeast/yemen-saudi-bombing-houthis-hunger.html?_r=0}{the
American-provided weapons} have been GBU-series guided bombs and cluster
munitions, both of which have been linked by human rights groups and
journalists to attacks on Yemeni factories and civilian deaths.

Advertisement

\protect\hyperlink{after-bottom}{Continue reading the main story}

\hypertarget{site-index}{%
\subsection{Site Index}\label{site-index}}

\hypertarget{site-information-navigation}{%
\subsection{Site Information
Navigation}\label{site-information-navigation}}

\begin{itemize}
\tightlist
\item
  \href{https://help.nytimes3xbfgragh.onion/hc/en-us/articles/115014792127-Copyright-notice}{©~2020~The
  New York Times Company}
\end{itemize}

\begin{itemize}
\tightlist
\item
  \href{https://www.nytco.com/}{NYTCo}
\item
  \href{https://help.nytimes3xbfgragh.onion/hc/en-us/articles/115015385887-Contact-Us}{Contact
  Us}
\item
  \href{https://www.nytco.com/careers/}{Work with us}
\item
  \href{https://nytmediakit.com/}{Advertise}
\item
  \href{http://www.tbrandstudio.com/}{T Brand Studio}
\item
  \href{https://www.nytimes3xbfgragh.onion/privacy/cookie-policy\#how-do-i-manage-trackers}{Your
  Ad Choices}
\item
  \href{https://www.nytimes3xbfgragh.onion/privacy}{Privacy}
\item
  \href{https://help.nytimes3xbfgragh.onion/hc/en-us/articles/115014893428-Terms-of-service}{Terms
  of Service}
\item
  \href{https://help.nytimes3xbfgragh.onion/hc/en-us/articles/115014893968-Terms-of-sale}{Terms
  of Sale}
\item
  \href{https://spiderbites.nytimes3xbfgragh.onion}{Site Map}
\item
  \href{https://help.nytimes3xbfgragh.onion/hc/en-us}{Help}
\item
  \href{https://www.nytimes3xbfgragh.onion/subscription?campaignId=37WXW}{Subscriptions}
\end{itemize}
