Sections

SEARCH

\protect\hyperlink{site-content}{Skip to
content}\protect\hyperlink{site-index}{Skip to site index}

\href{https://www.nytimes3xbfgragh.onion/section/us}{U.S.}

\href{https://myaccount.nytimes3xbfgragh.onion/auth/login?response_type=cookie\&client_id=vi}{}

\href{https://www.nytimes3xbfgragh.onion/section/todayspaper}{Today's
Paper}

\href{/section/us}{U.S.}\textbar{}Defiant Voices Flood U.S. Cities as
Women Rally for Rights

\url{https://nyti.ms/2jLxyDE}

\begin{itemize}
\item
\item
\item
\item
\item
\item
\end{itemize}

Advertisement

\protect\hyperlink{after-top}{Continue reading the main story}

Supported by

\protect\hyperlink{after-sponsor}{Continue reading the main story}

\hypertarget{defiant-voices-flood-us-cities-as-women-rally-for-rights}{%
\section{Defiant Voices Flood U.S. Cities as Women Rally for
Rights}\label{defiant-voices-flood-us-cities-as-women-rally-for-rights}}

\includegraphics{https://static01.graylady3jvrrxbe.onion/images/2017/01/22/us/22womena-march-videostill/22womena-march-videostill-videoSixteenByNine3000.jpg}

By Susan Chira and
\href{http://www.nytimes3xbfgragh.onion/by/yamiche-alcindor}{Yamiche
Alcindor}

\begin{itemize}
\item
  Jan. 21, 2017
\item
  \begin{itemize}
  \item
  \item
  \item
  \item
  \item
  \item
  \end{itemize}
\end{itemize}

WASHINGTON --- The day after what many had assumed would be the
inauguration of the first female president, hundreds of thousands of
women flooded the streets of Washington, and many more marched in cities
across the country, in defiant, jubilant rallies against the man who
defeated her.

Protesters jammed the streets near the Capitol for the main
demonstration, packed so tightly at times that they could barely move.
In Chicago, the size of a rally so quickly outgrew early estimates that
the official march that was scheduled to follow was canceled for safety,
though many paraded through downtown, anyway.

In Manhattan, Fifth Avenue became a tide of signs and symbolic pink
hats, while in downtown Los Angeles, shouts of ``love trumps hate''
echoed along a one-mile route leading to City Hall, with many
demonstrators spilling over into adjacent streets in a huge,
festival-like atmosphere.

The marches were the kickoff for what their leaders hope will be a
sustained campaign of protest in a polarized nation, riven by an
election that raised unsettling questions about American values,
out-of-touch elites and barriers to women's ambitions.

\includegraphics{https://static01.graylady3jvrrxbe.onion/images/2017/01/23/us/23MARCH/23MARCH-videoSixteenByNineJumbo1600-v2.jpg}

On successive days, two parallel and separate Americas were on display
in virtually the same location. First there was
\href{https://www.nytimes3xbfgragh.onion/2017/01/20/us/politics/trump-inauguration-day.html}{President
Trump's inauguration}, his message of an ailing society he would restore
to greatness aimed at the triumphant supporters who thronged Washington
on Friday.

Then on Saturday, in what amounted to a counterinauguration, the
speakers, performers and marchers proclaimed allegiance to a profoundly
different vision of the nation. They voiced determination to protect an
array of rights that they believe Mr. Trump threatens, and that they
thought only recently were secure.

``Thank you for understanding that sometimes we must put our bodies
where our beliefs are,'' Gloria Steinem, the feminist icon and an
honorary chairwoman of the march, told those gathered in Washington.
``Pressing `send' is not enough.''

To mobilize a progressive movement reeling from Hillary Clinton's
defeat, organizers
\href{https://www.womensmarch.com/principles/}{broadened the platform}
beyond longstanding women's issues such as abortion, equal pay and
sexual assault to include immigrant rights, police brutality, mass
incarceration, voter suppression and environmental protection.

\includegraphics{https://static01.graylady3jvrrxbe.onion/images/2017/01/22/us/22MARCH-02a/22MARCH-02a-articleInline.jpg?quality=75\&auto=webp\&disable=upscale}

But the march's origins were in the
\href{https://www.nytimes3xbfgragh.onion/2016/12/30/opinion/sunday/feminism-lost-now-what.html}{outrage
and despair} of many women after an election that placed gender in the
spotlight as never before.

Mrs. Clinton assertively claimed the mantle of history, offering herself
as the champion of women and families, and calling out her opponent for
\href{https://www.nytimes3xbfgragh.onion/2016/10/08/us/politics/donald-trump-women.html}{boasting
of forcing himself on women} in a recording that prompted a national
conversation about sexual assault. In a sly allusion to the crude
remarks Mr. Trump made on the tape, many marchers, women and men alike,
wore pink ``pussy hats'' sporting cat ears.

In Washington, demonstrators old and young pushed strollers and hoisted
children onto their shoulders or guided elderly parents through the
milling crowds. They waved handmade signs: ``Hate Does Not Make America
Great,'' ``I Will Not Go Back Quietly to the 1950s'' and ``I'm 17 ---
Fear Me!'' They chanted, ``This is what democracy looks like.'''

Emma Wendt, 13, came with a large group of family members and
schoolmates from Kensington, Md., for a simple reason: ``being part of
history.''

\href{https://www.nytimes3xbfgragh.onion/interactive/2017/01/21/world/womens-march-pictures.html}{}

\includegraphics{https://static01.graylady3jvrrxbe.onion/images/2017/01/21/us/womens-march/womens-march-square640.png}

\hypertarget{pictures-from-the-2017-womens-marches-on-every-continent}{%
\subsection{Pictures From the 2017 Women's Marches on Every
Continent}\label{pictures-from-the-2017-womens-marches-on-every-continent}}

Crowds in hundreds of cities around the world gathered in conjunction
with the Women's March on Washington.

The marchers were confronting a president who has appointed just a
handful of women to his cabinet and inner circle, and who has pledged to
nominate a Supreme Court justice who opposes abortion rights and to
dismantle a health care act that covers contraception. His appointees
have track records of voting to cut funding for anti-domestic violence
programs, opposing increases in the minimum wage and restructuring
Medicaid --- moves that disproportionately affect women and minorities.

Crowd estimates were not available in some locations, but a city
official in Washington said that participation there likely surpassed
half a million, according to The Associated Press. Added to the more
than 400,000 that Mayor Bill de Blasio's office said had marched in New
York City, hundreds of thousands more in Chicago and Los Angeles, and
those who showed up at many other marches nationwide, the total
attendance easily surpassed one million in the United States. Marches
also took place in a number of cities abroad, including Berlin, Paris,
Rome, Amsterdam and Cape Town.

In Boston, where the crowd swelled to 175,000, Senator Elizabeth Warren
looked out at the admiring throngs and conjured up the image of Mr.
Trump's being sworn in the day before.

``The sight is now burned into my eyes forever,'' Ms. Warren said,
adding, ``We will use that vision to fight harder.''

\href{https://www.nytimes3xbfgragh.onion/interactive/2017/01/22/us/politics/womens-march-trump-crowd-estimates.html}{}

\includegraphics{https://static01.graylady3jvrrxbe.onion/images/2017/01/22/us/politics/womens-march-trump-crowd-estimates-1485071976042/womens-march-trump-crowd-estimates-1485071976042-square640-v2.jpg}

\hypertarget{crowd-scientists-say-womens-march-in-washington-had-3-times-as-many-people-as-trumps-inauguration}{%
\subsection{Crowd Scientists Say Women's March in Washington Had 3 Times
as Many People as Trump's
Inauguration}\label{crowd-scientists-say-womens-march-in-washington-had-3-times-as-many-people-as-trumps-inauguration}}

Estimates by crowd scientists of attendance at events on Friday and
Saturday and how they calculated it.

Yet women did not protest --- or vote --- as a bloc. About 53 percent of
white women voted for Mr. Trump, according to exit polls, and many said
his demeaning comments about women mattered less to them than their
belief that he had the independence and business experience to bring
about change, restore well-paying jobs and protect America's borders.

``The women's march clearly doesn't represent all women,'' Alex Smith,
the national chairwoman of the College Republicans, said in an email.
She noted
\href{https://www.nytimes3xbfgragh.onion/2017/01/18/us/womens-march-abortion.html}{the
exclusion} of anti-abortion women's groups from the event. ``It is
precisely this type of dogmatic intransigence that voters rejected.''

The marches came a day after confrontations between anti-Trump
protesters and the police led to more than 200 arrests in Washington.
But Saturday's demonstrations were peaceful, and counterprotests were
few. In St. Paul, one man was arrested after marchers reported he had
``sprayed irritants'' into the crowd, the police said.

Though the Washington march ended within sight of the White House, and
some demonstrators passed by his recently opened hotel, Mr. Trump did
not cross paths with the crowd. But on Sunday morning, Mr. Trump
acknowledged the demonstrations on Twitter,
\href{https://twitter.com/realDonaldTrump/status/823150055418920960}{questioning}
whether the protesters had voted.

Image

A woman wore a United States flag as a hijab during a protest in front
of the Brandenburg Gate in Berlin.Credit...Gregor Fischer/DPA, via
Agence France-Presse --- Getty Images

A little later, Sunday, Mr. Trump
\href{https://twitter.com/realDonaldTrump/status/823174199036542980}{added
on Twitter} that he supported the right of peaceable assembly.

Among those celebrity performers, were some who had appeared at campaign
events for Mrs. Clinton, including Madonna, who gave a speech, said
toward the end of of the march. ``I have thought a lot about blowing up
the White House. But I know that this will not change anything,'' she
said. (The Secret Service declined to comment on the remark, though an
investigation seemed unlikely.)

After attending the inauguration on Friday, Mrs. Clinton herself was not
seen at the march. She did, however, acknowledge the moment on Twitter.

``Thanks for standing, speaking \& marching for our values
@womensmarch,'' she wrote.

The marches captured the potential and the perils for the progressive
movement --- whether it can frame its message to appeal to new
generations and whether it can translate protests into action locally
and nationally.

Plans for Saturday's march in Washington began as Facebook posts just
after the election by a retired lawyer in Hawaii and a fashion designer
in New York, both of whom are white and had no experience organizing
protests. Soon, protests flooded the feeds urging them to diversify. In
the end,
\href{https://www.nytimes3xbfgragh.onion/2017/01/09/us/womens-march-on-washington-opens-contentious-dialogues-about-race.html}{a
triumvirate of African-American, Latina and Muslim women joined the
leadership team}.

Image

In a sly allusion to crude remarks made by Mr. Trump about sexual
assault, many marchers wore hats sporting cat ears.Credit...Hilary Swift
for The New York Times

The march's initial struggles echoed broader debates in the movement
about whether the courting of new demographic groups alienated the white
working-class voters who had carried Mr. Trump to victory, or whether
white women had betrayed gender solidarity by voting for him. Yet on
Saturday, these tensions did not deter a multiracial, multigenerational
turnout. Mothers marched with daughters and granddaughters; whole
families, including husbands and sons, marched arm in arm.

Mikhael Tara Garver, 37, of Brooklyn, who marched with her mother,
recalled how her family had reacted after the election: ``We were all
calling my great-aunts because we all knew how important Hillary was to
them and how important surviving to see that moment was for them.''

Another family came from Baltimore. ``We have to get away from fear,''
said Lureen Grace Wiggins, 49. Her daughter, Eden, 17, was exhilarated
by the size of the crowd: ``When you're out here and people see you,
they know you care.''

The march was rich in historical allusions --- most deliberately, the
1963 march led by the Rev. Dr. Martin Luther King Jr. But it echoed many
other marches, including those in the 1970s that brought hundreds of
thousands of women to the streets championing an Equal Rights Amendment
that was ultimately defeated, and those from the late 1990s and on for
abortion rights, culminating in a 2004 March for Women's Lives that
organizers said drew more than one million to the capital.

Saturday's march happened to come just six days before quite a different
one: the annual March for Life by opponents of abortion.

But perhaps the most apt analogy, said Ellen Fitzpatrick, the author of
``The Highest Glass Ceiling,'' was to the 1913 suffragists' march on
Washington, timed to coincide with the inauguration of President Woodrow
Wilson. Led by the renowned suffragist Alice Paul, it featured a lawyer,
Inez Milholland, riding a white horse down Pennsylvania Avenue, with 24
floats, nine marching bands and luminaries like Helen Keller. The women
were hooted and jeered at and roughed up by the police, prompting
congressional hearings and generating public sympathy. They won the vote
seven years later.

Faye Wattleton, the former president of Planned Parenthood, said that
women have always had to regroup, even after they thought battles were
won. ``This is not new,'' she said. ``We have to go back to the
battlefield and re-fight the wars against women.''

Advertisement

\protect\hyperlink{after-bottom}{Continue reading the main story}

\hypertarget{site-index}{%
\subsection{Site Index}\label{site-index}}

\hypertarget{site-information-navigation}{%
\subsection{Site Information
Navigation}\label{site-information-navigation}}

\begin{itemize}
\tightlist
\item
  \href{https://help.nytimes3xbfgragh.onion/hc/en-us/articles/115014792127-Copyright-notice}{©~2020~The
  New York Times Company}
\end{itemize}

\begin{itemize}
\tightlist
\item
  \href{https://www.nytco.com/}{NYTCo}
\item
  \href{https://help.nytimes3xbfgragh.onion/hc/en-us/articles/115015385887-Contact-Us}{Contact
  Us}
\item
  \href{https://www.nytco.com/careers/}{Work with us}
\item
  \href{https://nytmediakit.com/}{Advertise}
\item
  \href{http://www.tbrandstudio.com/}{T Brand Studio}
\item
  \href{https://www.nytimes3xbfgragh.onion/privacy/cookie-policy\#how-do-i-manage-trackers}{Your
  Ad Choices}
\item
  \href{https://www.nytimes3xbfgragh.onion/privacy}{Privacy}
\item
  \href{https://help.nytimes3xbfgragh.onion/hc/en-us/articles/115014893428-Terms-of-service}{Terms
  of Service}
\item
  \href{https://help.nytimes3xbfgragh.onion/hc/en-us/articles/115014893968-Terms-of-sale}{Terms
  of Sale}
\item
  \href{https://spiderbites.nytimes3xbfgragh.onion}{Site Map}
\item
  \href{https://help.nytimes3xbfgragh.onion/hc/en-us}{Help}
\item
  \href{https://www.nytimes3xbfgragh.onion/subscription?campaignId=37WXW}{Subscriptions}
\end{itemize}
