Sections

SEARCH

\protect\hyperlink{site-content}{Skip to
content}\protect\hyperlink{site-index}{Skip to site index}

\href{https://www.nytimes3xbfgragh.onion/section/politics}{Politics}

\href{https://myaccount.nytimes3xbfgragh.onion/auth/login?response_type=cookie\&client_id=vi}{}

\href{https://www.nytimes3xbfgragh.onion/section/todayspaper}{Today's
Paper}

\href{/section/politics}{Politics}\textbar{}Trump Abandons Trans-Pacific
Partnership, Obama's Signature Trade Deal

\url{https://nyti.ms/2jQSDwo}

\begin{itemize}
\item
\item
\item
\item
\item
\item
\end{itemize}

Advertisement

\protect\hyperlink{after-top}{Continue reading the main story}

Supported by

\protect\hyperlink{after-sponsor}{Continue reading the main story}

\hypertarget{trump-abandons-trans-pacific-partnership-obamas-signature-trade-deal}{%
\section{Trump Abandons Trans-Pacific Partnership, Obama's Signature
Trade
Deal}\label{trump-abandons-trans-pacific-partnership-obamas-signature-trade-deal}}

\includegraphics{https://static01.graylady3jvrrxbe.onion/images/2017/01/24/us/24trump-1/24trump-1-videoSixteenByNineJumbo1600.jpg}

By \href{http://www.nytimes3xbfgragh.onion/by/peter-baker}{Peter Baker}

\begin{itemize}
\item
  Jan. 23, 2017
\item
  \begin{itemize}
  \item
  \item
  \item
  \item
  \item
  \item
  \end{itemize}
\end{itemize}

WASHINGTON --- President Trump upended America's traditional, bipartisan
trade policy on Monday as he formally abandoned the ambitious, 12-nation
Trans-Pacific Partnership brokered by his predecessor and declared an
end to the era of multinational trade agreements that defined global
economics for decades.

With the stroke of a pen on his first full weekday in office, Mr. Trump
signaled that he plans to follow through on promises to take a more
aggressive stance against foreign competitors as part of his ``America
First'' approach. In doing so, he demonstrated that he would not follow
old rules, effectively discarding longstanding Republican orthodoxy that
expanding global trade was good for the world and America --- and that
the United States should help write the rules of international commerce.

Although the Trans-Pacific Partnership had not been approved by
Congress, Mr. Trump's decision to withdraw not only doomed former
President Barack Obama's signature trade achievement, but also carried
broad geopolitical implications in a fast-growing region. The deal,
which was to link a dozen nations from Canada and Chile to Australia and
Japan in a complex web of trade rules, was sold as a way to permanently
tie the United States to East Asia and create an economic bulwark
against a rising China.

Instead, Mr. Trump said American workers would be protected against
competition from low-wage countries like Vietnam and Malaysia, also
parties to the deal.

But some in both parties worry that China will move to fill the economic
vacuum as America looks inward, and will expand its sway over Asia and
beyond.

Monday was a busy day for the new president. In addition to abandoning
the trade deal, he ordered a freeze on federal government hiring, except
for the military and other security agencies. He
\href{https://www.nytimes3xbfgragh.onion/2017/01/23/world/trump-ban-foreign-aid-abortions.html}{reinstituted
a ban} on federal funding for overseas family planning groups that
assist or counsel women seeking abortions. He met with congressional,
labor and business leaders. And he promised to cut up to 75 percent of
federal regulations.

Mr. Trump's decision to scrap the Trans-Pacific Partnership, or T.P.P.,
reversed a free-trade strategy adopted by presidents of both parties
dating back to the Cold War, and aligned him more with the political
left. When he told a meeting of union leaders at the White House on
Monday that he had just terminated the pact, they broke into applause.

\href{https://www.nytimes3xbfgragh.onion/interactive/2016/business/tpp-explained-what-is-trans-pacific-partnership.html}{}

\includegraphics{https://static01.graylady3jvrrxbe.onion/images/2016/08/04/business/04TPPEXPLAINER/04TPPEXPLAINER-videoLarge.jpg}

\hypertarget{what-is-tpp-behind-the-trade-deal-that-died}{%
\subsection{What Is TPP? Behind the Trade Deal That
Died}\label{what-is-tpp-behind-the-trade-deal-that-died}}

On his first full workday in office, President Trump delivered on a
campaign promise by abandoning the enormous trade deal that had became a
flashpoint in American politics.

``We're going to stop the ridiculous trade deals that have taken
everybody out of our country and taken companies out of our country, and
it's going to be reversed,'' Mr. Trump told them, saying that from now
on, the United States would sign trade deals only with individual
allies. ``I think you're going to have a lot of companies come back to
our country.''

Mr. Trump may also move quickly to renegotiate the
\href{https://www.nytimes3xbfgragh.onion/2016/10/04/upshot/donald-trump-trashes-nafta-but-unwinding-it-would-come-at-a-huge-cost.html}{North
American Free Trade Agreement}. He is scheduling meetings with the
leaders of Canada and Mexico, the two main partners in that pact, which
was negotiated by President George Bush and pushed through Congress by
President Bill Clinton. While Nafta has been a major driver of American
trade for nearly two decades, it has long been divisive, with critics
blaming it for lost jobs and lower wages.

But free-trade advocates said that in canceling the Pacific pact, Mr.
Trump lost an agreement that had already renegotiated Nafta under more
modern rules governing intellectual property, internet access and
agriculture, since both Mexico and Canada were signatories. He also
undercut Mr. Obama's so-called pivot to Asia and, critics said,
essentially ceded the field to China, which was not part of the
agreement.

``There's no doubt that this action will be seen as a huge, huge win for
China,'' Michael B. Froman, the trade representative who negotiated the
pact for Mr. Obama, said in an interview. ``For the Trump
administration, after all this talk about being tough on China, for
their first action to basically hand the keys to China and say we're
withdrawing from our leadership position in this region is
geostrategically damaging.''

Some Republicans agreed, but only a few would publicly challenge the
president. Senator John McCain of Arizona called the decision ``a
serious mistake'' that would hurt America. ``It will send a troubling
signal of American disengagement in the Asia-Pacific region at a time we
can least afford it,'' he said in a statement.

The Obama administration negotiated the trade pact for nearly eight
years. Speaker Paul D. Ryan and other congressional Republicans worked
with Mr. Obama to pass legislation granting so-called fast-track
authority to negotiate it over Democratic objections. But Mr. Obama
never submitted the final agreement for approval amid vocal opposition.

The agreement, the largest regional trade accord ever,
\href{https://www.nytimes3xbfgragh.onion/2015/06/15/world/asia/the-trans-pacific-trade-deal-and-a-presidents-legacy.html}{brought
together} the United States and 11 other nations in a free-trade zone
for about 40 percent of the world's economy. It was intended to lower
tariffs while establishing rules for resolving trade disputes, setting
patents and protecting intellectual property.

Obama officials argued that it benefited the United States by opening
markets while giving up very little in return. In particular, it finally
brought the United States and Japan, the world's largest and
third-largest economies, together in a free-trade pact.

Mr. Trump's decision was crushing for Japan, where Prime Minister Shinzo
Abe spent considerable political capital to get the agreement through
Parliament, which ratified it Friday. Just hours before Mr. Trump
dispensed with it, Mr. Abe told Parliament that Tokyo would lobby the
new administration on the merits of the deal.

Japan was the last to join the pact, which would give its manufacturers
tariff-free access to export markets in the United States and other
Asian countries, but would bring its automakers into competition with
lower-wage countries like Mexico. Mr. Abe became a strong enthusiast
after making politically painful concessions on agricultural imports
that the United States had sought.

China, by contrast, welcomed Mr. Trump's move, although its leaders will
probably relish the moment quietly. Given Mr. Trump's harsh attacks on
China and
\href{https://www.nytimes3xbfgragh.onion/2016/12/21/us/politics/peter-navarro-carl-icahn-trump-china-trade.html}{his
appointment} of a leading China critic, Peter Navarro, to the new post
of trade council director, Beijing is bracing for a potentially
combative relationship.

Victor Shih, an expert on China's political economy at the University of
California, San Diego, said withdrawing from the T.P.P. would alter
America's image in the region. ``The U.S. will be seen as an unreliable
partner both economically and perhaps even in the security arena,'' he
said. ``While some countries in Asia have no choice but to be close to
the U.S., others may begin to look to China.''

China has already sought to capitalize by making a push to complete an
alternative pact, the Regional Comprehensive Economic Partnership, which
aims to unite 10 members of the Association of Southeast Asian Nations
with Japan, South Korea, Australia, New Zealand and India.

Australia's trade minister, Steven Ciobo, said on Monday that other
members of the trade pact were exploring whether to create a ``T.P.P.
minus one,'' without the United States.

``The T.P.P. offers very material benefits for all parties that signed
up for the agreement,'' he said in an interview. ``It would be a great
shame to lose those benefits. Notwithstanding President Trump's
decision, there's still a lot of merits to capturing those gains.''

If Mr. Trump scrambled coalitions overseas, he did so at home, too.
Democrats and labor groups praised his move. James P. Hoffa, general
president of the Teamsters union, said Mr. Trump had ``taken the first
step toward fixing 30 years of bad trade policies.'' Lori Wallach,
director of Public Citizen's Global Trade Watch, said it would ``bury
the moldering corpse'' of the Pacific deal, though she expressed concern
about how Nafta would be renegotiated.

Some people emerging from the union meeting with Mr. Trump, who won
surprising victories in Midwestern labor strongholds, expressed
enthusiasm for both his trade action and his promise to build new roads,
bridges and other infrastructure.

``We just had probably the most incredible meeting of our careers,''
Sean McGarvey, president of North America's Building Trades Unions,
said. ``We will work with him and his administration to help him
implement his plans on infrastructure, trade and energy policy, so we
really do put America back to work.''

Advertisement

\protect\hyperlink{after-bottom}{Continue reading the main story}

\hypertarget{site-index}{%
\subsection{Site Index}\label{site-index}}

\hypertarget{site-information-navigation}{%
\subsection{Site Information
Navigation}\label{site-information-navigation}}

\begin{itemize}
\tightlist
\item
  \href{https://help.nytimes3xbfgragh.onion/hc/en-us/articles/115014792127-Copyright-notice}{©~2020~The
  New York Times Company}
\end{itemize}

\begin{itemize}
\tightlist
\item
  \href{https://www.nytco.com/}{NYTCo}
\item
  \href{https://help.nytimes3xbfgragh.onion/hc/en-us/articles/115015385887-Contact-Us}{Contact
  Us}
\item
  \href{https://www.nytco.com/careers/}{Work with us}
\item
  \href{https://nytmediakit.com/}{Advertise}
\item
  \href{http://www.tbrandstudio.com/}{T Brand Studio}
\item
  \href{https://www.nytimes3xbfgragh.onion/privacy/cookie-policy\#how-do-i-manage-trackers}{Your
  Ad Choices}
\item
  \href{https://www.nytimes3xbfgragh.onion/privacy}{Privacy}
\item
  \href{https://help.nytimes3xbfgragh.onion/hc/en-us/articles/115014893428-Terms-of-service}{Terms
  of Service}
\item
  \href{https://help.nytimes3xbfgragh.onion/hc/en-us/articles/115014893968-Terms-of-sale}{Terms
  of Sale}
\item
  \href{https://spiderbites.nytimes3xbfgragh.onion}{Site Map}
\item
  \href{https://help.nytimes3xbfgragh.onion/hc/en-us}{Help}
\item
  \href{https://www.nytimes3xbfgragh.onion/subscription?campaignId=37WXW}{Subscriptions}
\end{itemize}
