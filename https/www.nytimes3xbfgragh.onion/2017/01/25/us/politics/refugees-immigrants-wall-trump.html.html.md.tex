Sections

SEARCH

\protect\hyperlink{site-content}{Skip to
content}\protect\hyperlink{site-index}{Skip to site index}

\href{https://www.nytimes3xbfgragh.onion/section/politics}{Politics}

\href{https://myaccount.nytimes3xbfgragh.onion/auth/login?response_type=cookie\&client_id=vi}{}

\href{https://www.nytimes3xbfgragh.onion/section/todayspaper}{Today's
Paper}

\href{/section/politics}{Politics}\textbar{}Trump Orders Mexican Border
Wall to Be Built and Plans to Block Syrian Refugees

\url{https://nyti.ms/2ktUvwa}

\begin{itemize}
\item
\item
\item
\item
\item
\end{itemize}

Advertisement

\protect\hyperlink{after-top}{Continue reading the main story}

Supported by

\protect\hyperlink{after-sponsor}{Continue reading the main story}

\hypertarget{trump-orders-mexican-border-wall-to-be-built-and-plans-to-block-syrian-refugees}{%
\section{Trump Orders Mexican Border Wall to Be Built and Plans to Block
Syrian
Refugees}\label{trump-orders-mexican-border-wall-to-be-built-and-plans-to-block-syrian-refugees}}

\includegraphics{https://static01.graylady3jvrrxbe.onion/images/2017/01/26/us/26trump-sub/26trump-sub-videoSixteenByNineJumbo1600.jpg}

By
\href{https://www.nytimes3xbfgragh.onion/by/julie-hirschfeld-davis}{Julie
Hirschfeld Davis}

\begin{itemize}
\item
  Jan. 25, 2017
\item
  \begin{itemize}
  \item
  \item
  \item
  \item
  \item
  \end{itemize}
\end{itemize}

WASHINGTON --- President Trump on Wednesday began a sweeping crackdown
on illegal immigration, ordering the immediate construction of a border
wall with Mexico and aggressive efforts to find and deport unauthorized
immigrants. He planned additional actions to cut back on legal
immigration, including barring Syrian refugees from entering the United
States.

At the headquarters of the Department of Homeland Security, Mr. Trump
signed a pair of executive orders that paved the way for a border wall
and called for a newly expanded force to sweep up immigrants who are in
the country illegally. He revived programs that allow the federal
government to work with local and state law enforcement agencies to
arrest and detain unauthorized immigrants with criminal records and to
share information to help track and deport them.

He also planned to clamp down on legal immigration in another action
expected as early as Thursday. An eight-page draft of that executive
order, obtained by The New York Times, would indefinitely block Syrian
refugees from entering the United States and bar all refugees from the
rest of the world for at least 120 days.

When the refugee program resumes, it would be much smaller, with the
total number of refugees resettled in the United States this year more
than halved, to 50,000 from 110,000.

It would also suspend any immigration for at least 30 days from a number
of predominantly Muslim countries --- Iran, Iraq, Libya, Somalia, Sudan,
Syria and Yemen --- while the government toughened its already stringent
screening procedures to weed out potential terrorists.

White House officials declined to comment on the coming plan, but in a
wide-ranging interview that aired Wednesday on ABC, Mr. Trump
acknowledged that it aimed to erect formidable barriers for those
seeking refuge in the United States.

``It's going to be very hard to come in,'' Mr. Trump said. ``Right now,
it's very easy to come in.''

He also said his administration would ``absolutely do safe zones in
Syria'' to discourage refugees from seeking safety in other countries,
and chided Europe and Germany in particular for accepting millions of
immigrants. ``It's a disaster, what's happening there,'' Mr. Trump said.

\href{https://www.nytimes3xbfgragh.onion/interactive/2017/01/25/us/politics/document-Trump-EO-Draft-on-Refugees.html}{}

\includegraphics{https://static01.graylady3jvrrxbe.onion/images/2017/01/25/us/politics/image-Trump-EO-Draft-on-Refugees/image-Trump-EO-Draft-on-Refugees-square640-v3.gif}

\hypertarget{trump-drafts-executive-order-on-refugees}{%
\subsection{Trump Drafts Executive Order on
Refugees}\label{trump-drafts-executive-order-on-refugees}}

An early draft of an executive order that President Donald J. Trump is
expected to issue as early as Thursday outlines his plans to
indefinitely block Syrian refugees from entering the United States and
institute a temporary halt on all refugees from the rest of the world.

Taken together, the moves would turn the full weight of the federal
government to fortifying the United States border, rounding up some of
the 11 million people who are in the country illegally and targeting
refugees, who are often among the world's most vulnerable people. It is
an aggressive use of presidential power that follows through on the
nationalistic vision Mr. Trump presented during his presidential
campaign.

``A nation without borders is not a nation,'' Mr. Trump said Wednesday
at the Department of Homeland Security, where he signed the orders
alongside the newly sworn-in secretary, John F. Kelly. ``Beginning
today, the United States of America gets back control of its borders.''

The plans are a sharp break with former President Barack Obama's
approach and what was once a bipartisan consensus to devise a path to
citizenship for some of the nation's illegal immigrants. Mr. Obama,
however, angered many immigrant groups by deporting millions of
unauthorized workers, largely during his first term.

But Mr. Trump, whose campaign rallies were filled with chants from his
supporters of ``build the wall,'' has vowed to go much further. He has
often described unauthorized immigrants as criminals who must be found
and forcibly removed from the United States, as he did again on
Wednesday.

``We are going to get the bad ones out --- the criminals and the drug
dealers and gangs and gang members,'' Mr. Trump said. ``The day is over
when they can stay in our country and wreak havoc. We are going to get
them out, and we are going to get them out fast.''

The president had invited the families of people killed by unauthorized
immigrants to watch him sign the orders alongside Homeland Security
employees, and he asked each of them to stand in turn, telling of the
deaths of their relatives, which he said had inspired his policies.

``We hear you, we see you, and you will never, ever be ignored again,''
Mr. Trump said, contending that they had been ``victimized by open
borders.''

\href{https://www.nytimes3xbfgragh.onion/interactive/2017/01/25/us/politics/trump-refugee-plan.html}{}

\includegraphics{https://static01.graylady3jvrrxbe.onion/images/2017/01/24/us/politics/trump-refugee-plan-1485372260359/trump-refugee-plan-1485372260359-thumbLarge-v3.jpg}

\hypertarget{how-trumps-executive-order-will-affect-the-us-refugee-program}{%
\subsection{How Trump's Executive Order Will Affect the U.S. Refugee
Program}\label{how-trumps-executive-order-will-affect-the-us-refugee-program}}

The order cuts the number of refugees to the U.S. in half and bars those
from Syria.

The immigration orders drew furious condemnation from civil rights and
religious groups as well as immigrant advocacy organizations. The groups
described them as meanspirited, counterproductive and costly and said
the new policies would raise constitutional concerns while undermining
the American tradition of welcoming people from around the world.

``They're setting out to unleash this deportation force on steroids, and
local police will be able to run wild, so we're tremendously concerned
about the impact that could have on immigrants and families across the
country,'' said Joanne Lin, senior legislative counsel at the American
Civil Liberties Union. ``After today's announcement, the fear quotient
is going to go up exponentially.''

Lynn Tramonte, the deputy director of America's Voice Education Fund, an
immigration advocacy group, said Mr. Trump was ``wasting no time taking
a wrecking ball to the Statue of Liberty.'' She called the orders ``a
dramatic, radical and extreme assault on immigrants and the values of
our country.''

The orders also rankled officials in countries around the world.
President Enrique Peña Nieto of Mexico, who had planned to travel to
Washington next week to meet with Mr. Trump, let it be known that
\href{https://www.nytimes3xbfgragh.onion/2017/01/25/world/americas/trump-mexico-border-wall.html}{he
was considering canceling his trip}, senior Mexican officials said.

Mr. Trump has claimed that Mexico will ultimately pay for the wall, but
officials there have repeatedly said they have no intention of doing so.

Conservative organizations in the United States and some Republican
lawmakers praised Mr. Trump's moves, saying they would usher in overdue
enforcement of crucial homeland security laws that Mr. Obama had refused
to carry out.

``This looks like a return to enforcing the immigration laws, which is
something that President Obama strayed from and has not been prioritized
in a very long time,'' said Tommy Binion, the director of policy
outreach at the conservative-aligned Heritage Foundation. ``To have
President Trump focus on the problems immigration is bringing us as a
nation is a relief. Finally, we have a government that recognizes the
tragedies that we're facing.''

Mr. Trump will not be able to accomplish the goals laid out in the
immigration orders by himself. Congress would have to appropriate new
funding for the construction of a wall, which some have estimated could
cost tens of billions of dollars. Nonetheless, Mr. Trump directed
federal agencies to use existing funds as a start to the wall and
formally called for the hiring of an additional 5,000 Border Patrol
agents and 10,000 immigration officers.

The order would threaten the nation's roughly three dozen sanctuary
cities --- jurisdictions that limit their cooperation with federal
authorities seeking to detain unauthorized immigrants --- with losing
federal grant money if they do not comply with such requests.

At the same time, Mr. Trump is reviving a program called Secure
Communities, ended by the Obama administration, in which federal
officials use digital fingerprints shared by local law enforcement
departments to find and deport immigrants who commit crimes.

The provisions are chilling to many immigration advocates, who argued
that they could sweep up unauthorized immigrants beyond the criminals
Mr. Trump says he wants to target. Among those listed as priorities for
removal are those who have ``engaged in fraud or willful
misrepresentation in connection with any official matter or application
before a governmental agency,'' which would essentially include any
undocumented worker who has signed an employment agreement in the United
States.

The order also includes a section that directs federal agencies to
adjust their privacy policies to exclude unauthorized immigrants, in
effect allowing the sharing of their personal identifying information,
which could be used to track and apprehend them.

``With today's sweeping and constitutionally suspect executive actions,
the president is turning his back on both our history and our values as
a proud nation of immigrants,'' said Representative Nancy Pelosi of
California, the Democratic leader. ``Wasting billions of taxpayer
dollars on a border wall Mexico will never pay for, and punishing cities
that do not want their local police forces forced to serve as President
Trump's deportation dragnet, does nothing to fix our immigration system
or keep Americans safe.''

The order on refugees is in line with a Muslim ban that Mr. Trump
proposed during the campaign, though it does not single out any
particular religion. It orders the secretary of state and the secretary
of Homeland Security to prioritize those who are persecuted members of
religious minorities, essentially ensuring that Christians living in
predominantly Muslim countries would be at the top of the list.

``In order to protect Americans,'' the order states, ``we must ensure
that those admitted to this country do not bear hostile attitudes toward
our country and its founding principles.''

It says that for the time being, admitting anyone from Iraq, Syria,
Iran, Sudan, Libya, Somalia or Yemen is ``detrimental to the interests
of the United States.''

Advertisement

\protect\hyperlink{after-bottom}{Continue reading the main story}

\hypertarget{site-index}{%
\subsection{Site Index}\label{site-index}}

\hypertarget{site-information-navigation}{%
\subsection{Site Information
Navigation}\label{site-information-navigation}}

\begin{itemize}
\tightlist
\item
  \href{https://help.nytimes3xbfgragh.onion/hc/en-us/articles/115014792127-Copyright-notice}{©~2020~The
  New York Times Company}
\end{itemize}

\begin{itemize}
\tightlist
\item
  \href{https://www.nytco.com/}{NYTCo}
\item
  \href{https://help.nytimes3xbfgragh.onion/hc/en-us/articles/115015385887-Contact-Us}{Contact
  Us}
\item
  \href{https://www.nytco.com/careers/}{Work with us}
\item
  \href{https://nytmediakit.com/}{Advertise}
\item
  \href{http://www.tbrandstudio.com/}{T Brand Studio}
\item
  \href{https://www.nytimes3xbfgragh.onion/privacy/cookie-policy\#how-do-i-manage-trackers}{Your
  Ad Choices}
\item
  \href{https://www.nytimes3xbfgragh.onion/privacy}{Privacy}
\item
  \href{https://help.nytimes3xbfgragh.onion/hc/en-us/articles/115014893428-Terms-of-service}{Terms
  of Service}
\item
  \href{https://help.nytimes3xbfgragh.onion/hc/en-us/articles/115014893968-Terms-of-sale}{Terms
  of Sale}
\item
  \href{https://spiderbites.nytimes3xbfgragh.onion}{Site Map}
\item
  \href{https://help.nytimes3xbfgragh.onion/hc/en-us}{Help}
\item
  \href{https://www.nytimes3xbfgragh.onion/subscription?campaignId=37WXW}{Subscriptions}
\end{itemize}
