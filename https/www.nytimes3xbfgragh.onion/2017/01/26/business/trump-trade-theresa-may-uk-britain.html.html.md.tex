Sections

SEARCH

\protect\hyperlink{site-content}{Skip to
content}\protect\hyperlink{site-index}{Skip to site index}

\href{https://www.nytimes3xbfgragh.onion/section/business}{Business}

\href{https://myaccount.nytimes3xbfgragh.onion/auth/login?response_type=cookie\&client_id=vi}{}

\href{https://www.nytimes3xbfgragh.onion/section/todayspaper}{Today's
Paper}

\href{/section/business}{Business}\textbar{}Ahead of Theresa May's
Visit, Talk of Trade Deal Is Clouded by Politics

\url{https://nyti.ms/2jBrFpP}

\begin{itemize}
\item
\item
\item
\item
\item
\end{itemize}

Advertisement

\protect\hyperlink{after-top}{Continue reading the main story}

Supported by

\protect\hyperlink{after-sponsor}{Continue reading the main story}

\hypertarget{ahead-of-theresa-mays-visit-talk-of-trade-deal-is-clouded-by-politics}{%
\section{Ahead of Theresa May's Visit, Talk of Trade Deal Is Clouded by
Politics}\label{ahead-of-theresa-mays-visit-talk-of-trade-deal-is-clouded-by-politics}}

\includegraphics{https://static01.graylady3jvrrxbe.onion/images/2017/01/27/business/27TRANSATLANTICPIC1/27TRANSATLANTIC1-articleInline.jpg?quality=75\&auto=webp\&disable=upscale}

By \href{http://www.nytimes3xbfgragh.onion/by/peter-s-goodman}{Peter S.
Goodman}

\begin{itemize}
\item
  Jan. 26, 2017
\item
  \begin{itemize}
  \item
  \item
  \item
  \item
  \item
  \end{itemize}
\end{itemize}

In the world according to President Trump, the United States and Britain
are about to make their special relationship great again. Having cast
off decades of global niceties --- the United States by putting him in
the White House, Britain by voting to leave the European Union --- they
can now forge a magnificent future.

Mr. Trump is readying proof. As he prepares to host Prime Minister
Theresa May of Britain at the White House on Friday, he intends to begin
discussions on a free-trade agreement that would deepen commercial ties
between their countries.

A trade agreement would make for useful political theater on both sides
of the Atlantic, particularly as
\href{https://www.nytimes3xbfgragh.onion/2016/04/23/world/europe/obama-britain-visit.html}{President
Barack Obama had pointedly warned} that a deal would not easily happen.
But as a spur to economic expansion and hiring, a deal would almost
certainly amount to very little.

Until Britain
\href{https://www.nytimes3xbfgragh.onion/2017/01/24/world/europe/theresa-may-brexit-vote-article-50.html}{actually
leaves Europe} --- a step most likely at least two years away --- it is
legally barred from entering a trade agreement with any country. If a
deal was quietly negotiated to take effect after Britain's departure,
its economic punch would probably be modest. Tariffs between the
countries are already minimal, meaning there are relatively few trade
impediments left to clear away.

What areas of business could be opened to increased competition tend to
be the most politically sensitive: agriculture, government procurement
and defense. Legions of lobbyists would mount ferocious battles to
preserve the privileges of now-favored industries.

American federal government contracts would attract keen interest from
British defense and aerospace companies, especially as Mr. Trump has
\href{https://www.nytimes3xbfgragh.onion/2016/09/08/us/politics/donald-trump-speech.html}{vowed
to rebuild the military}. But opening such business to foreign
competitors in a trade agreement would directly collide with Mr. Trump's
``America First'' mantra.

``If President Trump genuinely means `buy American, hire American' above
all else, then there's going to be very little left to offer other
countries,'' said Adam S. Posen, a former member of the rate-setting
committee at the Bank of England and now president of the Peterson
Institute for International Economics in Washington. ``There is no way
in hell this is a massive trade deal. It would take massive effort to
create a trade deal of even minimal effect.''

Trade, it is worth noting, tends to involve trade-offs. Negotiations are
politically fraught, because they expose domestic industries to the
prospect of greater competition with overseas companies. In exchange,
domestic consumers gain access to more and cheaper products.

British steak aficionados may cheer the shrinking of steep tariffs that
limit the influx of American beef. The British beef industry would
probably respond less enthusiastically. It would seek protection,
particularly as Britain's exit from the bloc would eliminate European
subsidies for British agriculture. Any proposed opening of agriculture
would probaly provoke British opposition to genetically modified crops.

British insurance companies may be eager for a greater crack at the
enormous American market, but that would be outside the scope of a
potential deal. In the United States, the insurance industry is
regulated at the state level, meaning the Trump administration would
lack the authority to engineer an opening.

The biggest impediment to a substantial deal between Britain and the
United States is the similarity of the two economies, and especially the
importance of financial services on both shores.

Opening finance to greater competition would require that British
bankers pledge fealty to American regulations. Either that, or the rules
would have to be opened and renegotiated by both sides. That might
happen in the same way that Mr. Trump might perhaps renounce shiny ties
and golf in favor of tie-dye and Zen meditation.

For American companies, the appeal of giving something up to gain access
to the British market has been diminished by the very thing that makes
such a trade permissible: Britain's looming exit from Europe, known as
Brexit.

\includegraphics{https://static01.graylady3jvrrxbe.onion/images/2017/01/27/business/27TRANSATLANTIC2/27TRANSATLANTIC2-articleInline.jpg?quality=75\&auto=webp\&disable=upscale}

Britain's inclusion in the European Union and its enormous single market
has made Britain a prized center for manufacturing and corporate
headquarters for multinational companies. American investment banks have
been able to set up shop in London and use these bases to serve clients
from Ireland to Romania, as if operating in one country.

But once Britain leaves Europe, these benefits would almost certainly
disappear, raising endless questions about the future terms of trade
across the English Channel.

Already, major banks are publicly outlining plans to shift some
operations to cities in the European Union, exploring Dublin, Paris and
Luxembourg. Already, major multinational companies are expressing
reservations about sinking the next investment in Britain.

All of this makes a trade deal less compelling, diminishing the stakes,
leaving politics as the most meaningful context.

Last year, before Britain's referendum, Mr. Obama visited the country
and urged voters to opt for remaining in Europe. He sought to undercut a
key aspiration of those urging a break from the bloc --- the prospect
that a more independent Britain could strike a free-trade deal with the
United States.

Washington had little interest in striking deals with single countries,
Mr. Obama said, preferring to focus on giant multilateral trade
arrangements like the sprawling Trans-Pacific Partnership (a deal
\href{https://www.nytimes3xbfgragh.onion/2017/01/23/us/politics/tpp-trump-trade-nafta.html?_r=0}{Mr.
Trump just renounced}). If it sought a trade deal with the United
States, Britain would land ``in the back of the queue,'' Mr. Obama
warned.

With Mr. Trump now in the White House, Britain's place in the queue has
effectively been upgraded, a handy development for Mrs. May as she
presses ahead with Brexit. She is under pressure to demonstrate that
Britain, once liberated from Europe, can harvest bounty via trade with
faster-growing, more innovative parts of the globe. A trade deal with
the largest economy on earth would presumably check that box.

``It is clearly in the British government's political interest to show
that it can negotiate a free-trade agreement with the United States,''
said Stephen Woolcock, a trade expert at the London School of Economics.
``The question is what the economic costs and benefits of such an
agreement would be? These would seem to be less clear cut.''

In much of the world, Britain's decision to abandon the European Union
has been construed as a retrenchment of British influence and a
disengagement from global affairs. In recent weeks, Mrs. May has labored
to displace that notion by describing post-Brexit Britain as a land of
dauntless capitalists scouring the planet for fresh opportunities.

In a speech at the World Economic Forum in Davos, Switzerland, last
week, she described a British exit as ``a vote to take control and make
decisions for ourselves and, crucially, to become even more global and
internationalist in action and in spirit, too. Because that is who we
are as a nation.''

A trade deal with the United States could be considered evidence to
further that case.

Mr. Trump could post a deal with Britain on Twitter as proof he is not,
despite popular perceptions, a nativist protectionist leading the United
States on a retreat from global concerns.

Mr. Trump has
\href{https://www.nytimes3xbfgragh.onion/2016/12/01/business/economy/trump-carrier-pence-jobs.html}{strong-armed
Carrier} and
\href{https://www.nytimes3xbfgragh.onion/2016/11/19/business/ford-move-cited-as-victory-by-trump-has-no-effect-on-us-jobs.html}{Ford
Motor into} altering plans to shift production from American factories
to Mexico, cashing in on photo opportunities that allow him to claim he
is engineering a renaissance in industrial America, even as the big
picture remains unchanged.

In a similar vein, he could use a trade deal with Britain --- even a
narrow, economically inconsequential one --- to claim vindication for
his encouragement of leaving the bloc. That could resonate with the sort
of voters who make up Mr. Trump's base as well as supporters of Brexit
in Britain --- that is, those inclined to blame immigrants for economic
afflictions.

``Trump can play to his constituents,'' Mr. Posen of the Peterson
Institute said, people ``who believe that trade deals among white rich
people are particularly desirable.''

Advertisement

\protect\hyperlink{after-bottom}{Continue reading the main story}

\hypertarget{site-index}{%
\subsection{Site Index}\label{site-index}}

\hypertarget{site-information-navigation}{%
\subsection{Site Information
Navigation}\label{site-information-navigation}}

\begin{itemize}
\tightlist
\item
  \href{https://help.nytimes3xbfgragh.onion/hc/en-us/articles/115014792127-Copyright-notice}{©~2020~The
  New York Times Company}
\end{itemize}

\begin{itemize}
\tightlist
\item
  \href{https://www.nytco.com/}{NYTCo}
\item
  \href{https://help.nytimes3xbfgragh.onion/hc/en-us/articles/115015385887-Contact-Us}{Contact
  Us}
\item
  \href{https://www.nytco.com/careers/}{Work with us}
\item
  \href{https://nytmediakit.com/}{Advertise}
\item
  \href{http://www.tbrandstudio.com/}{T Brand Studio}
\item
  \href{https://www.nytimes3xbfgragh.onion/privacy/cookie-policy\#how-do-i-manage-trackers}{Your
  Ad Choices}
\item
  \href{https://www.nytimes3xbfgragh.onion/privacy}{Privacy}
\item
  \href{https://help.nytimes3xbfgragh.onion/hc/en-us/articles/115014893428-Terms-of-service}{Terms
  of Service}
\item
  \href{https://help.nytimes3xbfgragh.onion/hc/en-us/articles/115014893968-Terms-of-sale}{Terms
  of Sale}
\item
  \href{https://spiderbites.nytimes3xbfgragh.onion}{Site Map}
\item
  \href{https://help.nytimes3xbfgragh.onion/hc/en-us}{Help}
\item
  \href{https://www.nytimes3xbfgragh.onion/subscription?campaignId=37WXW}{Subscriptions}
\end{itemize}
