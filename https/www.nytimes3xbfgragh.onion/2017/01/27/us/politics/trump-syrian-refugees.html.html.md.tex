Sections

SEARCH

\protect\hyperlink{site-content}{Skip to
content}\protect\hyperlink{site-index}{Skip to site index}

\href{https://www.nytimes3xbfgragh.onion/section/politics}{Politics}

\href{https://myaccount.nytimes3xbfgragh.onion/auth/login?response_type=cookie\&client_id=vi}{}

\href{https://www.nytimes3xbfgragh.onion/section/todayspaper}{Today's
Paper}

\href{/section/politics}{Politics}\textbar{}Trump Bars Refugees and
Citizens of 7 Muslim Countries

\url{https://nyti.ms/2kcXwk4}

\begin{itemize}
\item
\item
\item
\item
\item
\end{itemize}

Advertisement

\protect\hyperlink{after-top}{Continue reading the main story}

Supported by

\protect\hyperlink{after-sponsor}{Continue reading the main story}

\hypertarget{trump-bars-refugees-and-citizens-of-7-muslim-countries}{%
\section{Trump Bars Refugees and Citizens of 7 Muslim
Countries}\label{trump-bars-refugees-and-citizens-of-7-muslim-countries}}

\includegraphics{https://static01.graylady3jvrrxbe.onion/images/2017/01/28/us/28military-vid/28military-vid-videoSixteenByNine3000.jpg}

By \href{http://www.nytimes3xbfgragh.onion/by/michael-d-shear}{Michael
D. Shear} and
\href{http://www.nytimes3xbfgragh.onion/by/helene-cooper}{Helene Cooper}

\begin{itemize}
\item
  Jan. 27, 2017
\item
  \begin{itemize}
  \item
  \item
  \item
  \item
  \item
  \end{itemize}
\end{itemize}

WASHINGTON --- President Trump on Friday closed the nation's borders to
refugees from around the world, ordering that families fleeing the
slaughter in Syria be indefinitely blocked from entering the United
States, and temporarily suspending immigration from several
predominantly Muslim countries.

In an executive order that he said was part of an extreme vetting plan
to keep out ``radical Islamic terrorists,'' Mr. Trump also established a
religious test for refugees from Muslim nations: He ordered that
Christians and others from minority religions be granted priority over
Muslims.

``We don't want them here,'' Mr. Trump said of Islamist terrorists
during a signing ceremony at the Pentagon. ``We want to ensure that we
are not admitting into our country the very threats our soldiers are
fighting overseas. We only want to admit those into our country who will
support our country, and love deeply our people.''

Earlier in the day, Mr. Trump explained to an interviewer for the
\href{http://www1.cbn.com/home}{Christian Broadcasting Network} that
Christians in Syria were ``horribly treated'' and alleged that under
previous administrations, ``if you were a Muslim you could come in, but
if you were a Christian, it was almost impossible.''

``I thought it was very, very unfair. So we are going to help them,''
the president said.

In fact, the United States accepts tens of thousands of Christian
refugees. According to the Pew Research Center, almost as many Christian
refugees (37,521) were admitted as Muslim refugees (38,901) in the 2016
fiscal year.

The executive order suspends the entry of refugees into the United
States for 120 days and directs officials to determine additional
screening ''to ensure that those approved for refugee admission do not
pose a threat to the security and welfare of the United States.''

The order also stops the admission of refugees from Syria indefinitely,
and bars entry into the United States for 90 days from seven
predominantly Muslim countries linked to concerns about terrorism. Those
countries are Iraq, Syria, Iran, Sudan, Libya, Somalia and Yemen.

Additionally, Mr. Trump signed a memorandum on Friday directing what he
called ``a great rebuilding of the armed services,'' saying it would
call for budget negotiations to acquire new planes, new ships and new
resources for the nation's military.

\href{https://www.nytimes3xbfgragh.onion/interactive/2017/01/25/us/politics/trump-refugee-plan.html}{}

\includegraphics{https://static01.graylady3jvrrxbe.onion/images/2017/01/24/us/politics/trump-refugee-plan-1485372260359/trump-refugee-plan-1485372260359-thumbLarge-v3.jpg}

\hypertarget{how-trumps-executive-order-will-affect-the-us-refugee-program}{%
\subsection{How Trump's Executive Order Will Affect the U.S. Refugee
Program}\label{how-trumps-executive-order-will-affect-the-us-refugee-program}}

The order cuts the number of refugees to the U.S. in half and bars those
from Syria.

``Our military strength will be questioned by no one, but neither will
our dedication to peace,'' Mr. Trump said.

Announcing his ``extreme vetting'' plan, the president invoked the
specter of the Sept. 11, 2001, attacks. Most of the 19 hijackers on the
planes that crashed into the World Trade Center, the Pentagon and a
field in Shanksville, Pa., were from Saudi Arabia. The rest were from
the United Arab Emirates, Egypt and Lebanon. None of those countries are
on Mr. Trump's visa ban list.

Human rights activists roundly condemned Mr. Trump's actions, describing
them as officially sanctioned religious persecution dressed up to look
like an effort to make the United States safer.

The International Rescue Committee called it ``harmful and hasty.'' The
American Civil Liberties Union described it as a ``euphemism for
discriminating against Muslims.'' Raymond Offensheiser, the president of
Oxfam America, said the order would harm families around the world who
are threatened by authoritarian governments.

``The refugees impacted by today's decision are among the world's most
vulnerable people --- women, children, and men --- who are simply trying
to find a safe place to live after fleeing unfathomable violence and
loss,'' Mr. Offensheiser said.

The president signed the executive order shortly after issuing a
statement noting that Friday was International Holocaust Remembrance
Day, an irony that many of his critics highlighted on Twitter. The
statement did not mention Jews, although it
\href{https://www.whitehouse.gov/the-press-office/2017/01/27/statement-president-international-holocaust-remembrance-day}{cited}
the ``depravity and horror inflicted on innocent people by Nazi
terror.''

Mr. Trump's actions came during a swearing-in ceremony for Secretary of
Defense Jim Mattis, a former Marine general. Standing in the Hall of
Heroes at the Pentagon, Mr. Trump hailed the members of America's
military as ``the backbone of this country'' and described Mr. Mattis as
a ``man of action.'' The president mistakenly referred to Mr. Mattis as
a ``soldier,'' a term abhorred by Marines.

Mr. Trump has been deferential to Mr. Mattis, who has quickly
established himself as a top aide whose advice the president is willing
to take. On Friday, Mr. Trump said he would let Mr. Mattis ``override''
him by banning torture during terror interrogations even though Mr.
Trump believes the tactics do work in getting information from suspects.

In a remarkable show of deference to his own subordinate, Mr. Trump said
during an earlier news conference Friday morning with Theresa May, the
British prime minister, that he would let Mr. Mattis decide about
whether to use torture in interrogations. Mr. Mattis has said he does
not believe torture is effective.

``I don't necessarily agree, but I will tell you that he will override
because I'm giving him that power,'' Mr. Trump said. ``I'm going to rely
on him. I happen to feel that it does work.''

Mr. Trump appeared to be struggling with the issue even as he spoke,
returning several times to his own belief in the effectiveness of
torture even as he stated that he would let Mr. Mattis decide.

``But I'm going with our leaders,'' he said. ``We are going to win, with
or without.''

Then he added, ``But I do disagree.''

Mr. Mattis spent his first week as defense secretary trying to reassure
not only American allies, but also military rank and file, that the
United States will not abandon a national security structure that has
stood in place since the end of World War II. He has told officials in
the Pentagon building that at an uncertain time, he intends, as defense
secretary, to provide an even-keeled, measured approach to national
security issues.

Before the signing ceremony, Mr. Trump met with Mr. Mattis and his
military chiefs for about an hour. The meeting --- which took place in a
Pentagon secure room known as ``the tank'' --- included introductions
for Mr. Trump to his military chiefs of staff. The meeting was attended
by Michael Flynn, the national security adviser; Gen. Joseph Dunford,
chairman of the Joint Chiefs of Staff; and the chiefs of the four
services and the National Guard.

The men discussed how to accelerate the fight against the Islamic State
and North Korea and how to deal with a host of global challenges, said a
defense official who was not authorized to talk publicly about the
internal talks. The leaders also discussed how to improve military
readiness.

The newly sworn-in secretary of defense also gave Mr. Trump a little of
what the president has been asking --- or tweeting --- for. On Thursday,
Mr. Mattis ordered a review of the controversial F-35 Joint Strike
Fighter program, which has been criticized by Mr. Trump for its cost
overruns.

Mr. Mattis also ordered that plans for a new Air Force One --- another
project that has come under fire from Mr. Trump --- should be reviewed,
``with the specific objective of identifying means to substantially
reduce the program's costs while delivering needed capabilities.''

The F-35 review, Mr. Mattis said in a memo, will also look at how to
reduce costs while still meeting requirements set out for the fighter
jet program.

During his confirmation hearings this month, Mr. Mattis defended Twitter
messages from Mr. Trump criticizing the F-35 program. Mr. Mattis said at
the time that Mr. Trump had ``in no way shown a lack of support for the
program,'' adding, ``He just wants more bang for the buck.''

The cost of building the F-35 next-generation fighter jet has been an
issue at the Pentagon for several years. At an estimated \$400 billion
over 15 years for 2,443 planes, the fighter jet is the military's
largest weapons project.

Advertisement

\protect\hyperlink{after-bottom}{Continue reading the main story}

\hypertarget{site-index}{%
\subsection{Site Index}\label{site-index}}

\hypertarget{site-information-navigation}{%
\subsection{Site Information
Navigation}\label{site-information-navigation}}

\begin{itemize}
\tightlist
\item
  \href{https://help.nytimes3xbfgragh.onion/hc/en-us/articles/115014792127-Copyright-notice}{©~2020~The
  New York Times Company}
\end{itemize}

\begin{itemize}
\tightlist
\item
  \href{https://www.nytco.com/}{NYTCo}
\item
  \href{https://help.nytimes3xbfgragh.onion/hc/en-us/articles/115015385887-Contact-Us}{Contact
  Us}
\item
  \href{https://www.nytco.com/careers/}{Work with us}
\item
  \href{https://nytmediakit.com/}{Advertise}
\item
  \href{http://www.tbrandstudio.com/}{T Brand Studio}
\item
  \href{https://www.nytimes3xbfgragh.onion/privacy/cookie-policy\#how-do-i-manage-trackers}{Your
  Ad Choices}
\item
  \href{https://www.nytimes3xbfgragh.onion/privacy}{Privacy}
\item
  \href{https://help.nytimes3xbfgragh.onion/hc/en-us/articles/115014893428-Terms-of-service}{Terms
  of Service}
\item
  \href{https://help.nytimes3xbfgragh.onion/hc/en-us/articles/115014893968-Terms-of-sale}{Terms
  of Sale}
\item
  \href{https://spiderbites.nytimes3xbfgragh.onion}{Site Map}
\item
  \href{https://help.nytimes3xbfgragh.onion/hc/en-us}{Help}
\item
  \href{https://www.nytimes3xbfgragh.onion/subscription?campaignId=37WXW}{Subscriptions}
\end{itemize}
