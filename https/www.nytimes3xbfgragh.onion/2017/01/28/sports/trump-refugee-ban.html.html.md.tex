Sections

SEARCH

\protect\hyperlink{site-content}{Skip to
content}\protect\hyperlink{site-index}{Skip to site index}

\href{https://www.nytimes3xbfgragh.onion/section/sports}{Sports}

\href{https://myaccount.nytimes3xbfgragh.onion/auth/login?response_type=cookie\&client_id=vi}{}

\href{https://www.nytimes3xbfgragh.onion/section/todayspaper}{Today's
Paper}

\href{/section/sports}{Sports}\textbar{}Trump's Immigration Order Could
Have a Big Impact on Sports

\url{https://nyti.ms/2jCIhjp}

\begin{itemize}
\item
\item
\item
\item
\item
\end{itemize}

Advertisement

\protect\hyperlink{after-top}{Continue reading the main story}

Supported by

\protect\hyperlink{after-sponsor}{Continue reading the main story}

\hypertarget{trumps-immigration-order-could-have-a-big-impact-on-sports}{%
\section{Trump's Immigration Order Could Have a Big Impact on
Sports}\label{trumps-immigration-order-could-have-a-big-impact-on-sports}}

\includegraphics{https://static01.graylady3jvrrxbe.onion/images/2017/01/29/sports/29muslimban-spts/29muslimban-spts-articleInline.jpg?quality=75\&auto=webp\&disable=upscale}

By \href{https://www.nytimes3xbfgragh.onion/by/jere-longman}{Jeré
Longman}

\begin{itemize}
\item
  Jan. 28, 2017
\item
  \begin{itemize}
  \item
  \item
  \item
  \item
  \item
  \end{itemize}
\end{itemize}

President Trump's ban on visitors from seven predominantly Muslim
nations could have a wide impact on international sports, including
jeopardizing a warm relationship between the United States and Iran in
wrestling competitions and threatening the chances of Los Angeles
hosting the 2024 Summer Olympics and of the United States securing
soccer's 2026 World Cup.

On Saturday, sports officials struggled to understand the implications
of Mr. Trump's executive order, including the question of whether
athletes from the targeted nations could enter the United States to
compete, especially in the initial 90-day period of the ban.

``We are working closely with the administration to understand the new
rules and how we best navigate them as it pertains to visiting
athletes,'' Patrick Sandusky, a spokesman for the United States Olympic
Committee, said in an email. ``We know they are supportive of the
Olympic movement and our bid and believe we will have a good working
relationship with them to ensure our success in hosting and attending
events.''

At least one International Olympic Committee delegate criticized Mr.
Trump's decision. The delegate, Richard Peterkin of St. Lucia, said
Saturday that he considered the executive order ``very, very
disappointing.''

He also called the United States a ``haven'' for many athletes hoping to
train outside their home countries.

``Most of our athletes from St. Lucia train in the States,'' Mr.
Peterkin said. ``We don't have an issue because we're not on that list
of seven, but if we were --- there go hopes and dreams.''

On Saturday evening,
\href{https://www.nytimes3xbfgragh.onion/2017/01/28/us/refugees-detained-at-us-airports-prompting-legal-challenges-to-trumps-immigration-order.html}{a
federal judge blocked} part of the executive order, but that decision
appeared unrelated to the elements of the ban that troubled sports
officials.

The most immediate effect may come in wrestling, given that one of the
nations affected by Mr. Trump's ban is Iran, which has long had a
congenial relationship with the United States in that sport. Iran said
on Saturday that it would stop American citizens from entering the
country, in retaliation for Mr. Trump's order.

The United States freestyle wrestling team is scheduled to participate
in a World Cup competition in Iran on Feb. 8. Steve Fraser, the chief
fund-raiser for U.S.A. Wrestling and a 1984 Greco-Roman Olympic
champion, said on Saturday that the president of Iran's wrestling
federation was scheduled to meet this weekend with government officials
there in an attempt to make sure the Americans would still be invited to
the meet.

``There's some nervousness by us that the Iranian government might say,
`We can't get visas to go there, so no Americans can come here,
either,''' Mr. Fraser said.

While Olympic boycotts have resulted from tense political differences
between nations, opposing countries have also long found common ground
on playing fields, on the track and in sports arenas. One of the most
celebrated examples is the so-called Ping-Pong diplomacy that helped
foster the relationship between the United States and China in the
1970s.

In wrestling, the United States, Iran, Cuba and Russia banded together
in 2013 to persuade the I.O.C. to keep the sport in the Summer Games.
American wrestlers and officials are warmly welcomed in Iran, and
Iranian wrestlers compete regularly in the United States. They may be
invited to meets in May in New York and in June in Los Angeles, Mr.
Fraser said. There is uncertainty now, however, about whether they would
be granted P1 visas, commonly known as sports visas, to compete.

In 2014, Christina Kelley, the chief international ambassador for U.S.A.
Wrestling, became one of the few women allowed into a wrestling arena in
Iran since the 1979 Islamic Revolution. She said on Saturday that she
was frustrated by Mr. Trump's decision.

``I don't think our current president has any clue what the State
Department and what sports diplomats and cultural exchanges do for our
country and for the safety of our people around the world,'' Ms. Kelley
said.

The ban on visitors from the seven nations --- Iran, Iraq, Libya,
Somalia, Sudan, Syria and Yemen --- comes at a delicate time for the
U.S.O.C. Los Angeles is seeking to host the 2024 Summer Games, and it
will learn in September whether it, Paris or Budapest will get the
Games.

(There is some speculation that the I.O.C. will award the 2024 Games to
Paris and the 2028 Games to Los Angeles, but the U.S.O.C. remains
committed to the bid for the 2024 Games.)

David Wallechinsky, an American member of the I.O.C.'s cultural and
heritage commission and the president of the International Society of
Olympic Historians, said the election of Mr. Trump in November had hurt
Los Angeles's bid with I.O.C. delegates because Mr. Trump was perceived
as being ``anti-Muslim, anti-woman and anti-Latino.''

``This is worse,'' Mr. Wallechinsky said of the Muslim ban, adding, ``I
would consider it a blow to the Los Angeles bid --- not fatal but a
blow.''

At a meeting at the I.O.C.'s headquarters in Lausanne, Switzerland,
several days after the American presidential election, Mr. Wallechinsky
was asked repeatedly, ``What is wrong with your country?'' he said.

He said he sought to assure I.O.C. officials by explaining that
three-quarters of the voters in Los Angeles had voted against Mr. Trump,
describing the city to them as a ``a multicultural, Trump-free zone.''

The United States is expected to bid to host the world's other major
sporting event, the World Cup, in 2026. In June, Sunil Gulati, president
of the United States Soccer Federation, told reporters that a Trump
presidency could complicate an American bid, especially if it were a
joint bid with Mexico, given Mr. Trump's plans to build a wall across
America's southern border.

``I think a co-hosted World Cup with Mexico would be trickier if
Secretary Clinton isn't in the White House,'' Mr. Gulati said at the
time, in a reference to Hillary Clinton, who lost the election to Mr.
Trump.

After Mr. Trump won the election, Mr. Gulati modified his remarks,
saying, ``It's not going to dissuade us or persuade us to bid.''
International perceptions of the Trump administration ``matter, for
sure,'' Mr. Gulati said, ``but I think those will be developed in the
months to come.''

U.S. Soccer said Saturday it would have no immediate comment as it
examined Mr. Trump's order.

Many questions remained unanswered about the ability of a number of
athletes to travel. Two N.B.A. players, Thon Maker and Luol Deng, were
born in Sudan, one of the seven countries listed in Mr. Trump's
executive order. (Both, however, were born in Wau, now part of South
Sudan, which gained its independence from Sudan in 2011.)

Mr. Maker's family fled Sudan when he was 5 and eventually settled in
Australia. Mr. Maker, who plays for the Milwaukee Bucks, moved to the
United States to play high school basketball in Louisiana, eventually
moved to Canada and is an Australian citizen who holds a passport from
that country. It was unclear how people with dual citizenship would be
treated under the order.

Mr. Deng, a forward for the Los Angeles Lakers, has lived in the United
States for 17 years. His family fled to Egypt when he was 5 to escape
the Sudanese civil war. Mr. Deng came to the United States when he was
14 and attended high school in New Jersey, and he later became a British
citizen.

The N.B.A. also holds an annual Basketball Without Borders camp, and it
is expected to be held in New Orleans during the league's All-Star
weekend in February. While rosters have not been released, last year's
camp involved players from 25 countries, including Amir Reza Shah-Ravesh
from Iran.

Major League Soccer has two American-born players with familial ties to
two of the nations facing bans. Steve Beitashour of Toronto has played
for Iran's national team, and Justin Meram of Columbus has played for
Iraq. League officials were looking into the matter on Saturday.

Mo Farah of Britain, who was born in Somalia and has won four Olympic
gold medals on the track at 5,000 meters and 10,000 meters, is a
Nike-sponsored athlete coached by Alberto Salazar.

Early Sunday morning Mr. Farah's \href{https://t.co/RUqwoTahZq}{Facebook
page had a post} that, among other things, said: ``On 1st January this
year, Her Majesty The Queen made me a Knight of the Realm. On 27th
January, President Donald Trump seems to have made me an alien. I am a
British citizen who has lived in America for the past six years -
working hard, contributing to society, paying my taxes and bringing up
our four children in the place they now call home. Now, me and many
others like me are being told that we may not be welcome.''

Abdi Abdirahman, a four-time Olympian for the United States who finished
third in the 2016 New York City Marathon, was also born in Somalia, and
the race regularly attracts runners from around the globe. The 2016 race
featured three runners from Iran, nine from Syria and one from Sudan in
the field of about 50,000 runners.

``Our goal is always to recruit the world's best and cleanest athletes,
regardless of where they're from,'' Chris Weiller, a spokesman for New
York Road Runners, which organizes the marathon, said in an email.

Jackie Brock-Doyle, a spokeswoman for track and field's world governing
body, the International Association of Athletics Federations, said, ``We
clearly need to understand the implications of this new U.S. immigration
policy and will be seeking assurances that it will not adversely
affect'' the sport's world championships, scheduled to be held in
Eugene, Ore., in 2021.

Phil Andrews, the chief executive of U.S.A. Weightlifting, said
officials were trying to figure out the impact of Mr. Trump's ban on the
world weight-lifting championships, scheduled for November in Anaheim,
Calif., and on the American team's participation at a competition in
Iran.

``Our view is that politics and sport should be separate,'' Mr. Andrews
said, stressing sports diplomacy among nations. ``We sincerely hope to
peacefully welcome these seven nations to Anaheim this November. It is
unimaginable to be able to host a true world event without their
participation.''

Advertisement

\protect\hyperlink{after-bottom}{Continue reading the main story}

\hypertarget{site-index}{%
\subsection{Site Index}\label{site-index}}

\hypertarget{site-information-navigation}{%
\subsection{Site Information
Navigation}\label{site-information-navigation}}

\begin{itemize}
\tightlist
\item
  \href{https://help.nytimes3xbfgragh.onion/hc/en-us/articles/115014792127-Copyright-notice}{©~2020~The
  New York Times Company}
\end{itemize}

\begin{itemize}
\tightlist
\item
  \href{https://www.nytco.com/}{NYTCo}
\item
  \href{https://help.nytimes3xbfgragh.onion/hc/en-us/articles/115015385887-Contact-Us}{Contact
  Us}
\item
  \href{https://www.nytco.com/careers/}{Work with us}
\item
  \href{https://nytmediakit.com/}{Advertise}
\item
  \href{http://www.tbrandstudio.com/}{T Brand Studio}
\item
  \href{https://www.nytimes3xbfgragh.onion/privacy/cookie-policy\#how-do-i-manage-trackers}{Your
  Ad Choices}
\item
  \href{https://www.nytimes3xbfgragh.onion/privacy}{Privacy}
\item
  \href{https://help.nytimes3xbfgragh.onion/hc/en-us/articles/115014893428-Terms-of-service}{Terms
  of Service}
\item
  \href{https://help.nytimes3xbfgragh.onion/hc/en-us/articles/115014893968-Terms-of-sale}{Terms
  of Sale}
\item
  \href{https://spiderbites.nytimes3xbfgragh.onion}{Site Map}
\item
  \href{https://help.nytimes3xbfgragh.onion/hc/en-us}{Help}
\item
  \href{https://www.nytimes3xbfgragh.onion/subscription?campaignId=37WXW}{Subscriptions}
\end{itemize}
