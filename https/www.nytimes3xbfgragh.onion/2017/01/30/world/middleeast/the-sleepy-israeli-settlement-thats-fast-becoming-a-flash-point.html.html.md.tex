Sections

SEARCH

\protect\hyperlink{site-content}{Skip to
content}\protect\hyperlink{site-index}{Skip to site index}

\href{https://www.nytimes3xbfgragh.onion/section/world/middleeast}{Middle
East}

\href{https://myaccount.nytimes3xbfgragh.onion/auth/login?response_type=cookie\&client_id=vi}{}

\href{https://www.nytimes3xbfgragh.onion/section/todayspaper}{Today's
Paper}

\href{/section/world/middleeast}{Middle East}\textbar{}Israel's
Hard-Liners Want to `Go Big': Annex a Settlement

\url{https://nyti.ms/2jLwL51}

\begin{itemize}
\item
\item
\item
\item
\item
\end{itemize}

Advertisement

\protect\hyperlink{after-top}{Continue reading the main story}

Supported by

\protect\hyperlink{after-sponsor}{Continue reading the main story}

\hypertarget{israels-hard-liners-want-to-go-big-annex-a-settlement}{%
\section{Israel's Hard-Liners Want to `Go Big': Annex a
Settlement}\label{israels-hard-liners-want-to-go-big-annex-a-settlement}}

By \href{http://www.nytimes3xbfgragh.onion/by/ian-fisher}{Ian Fisher}

\begin{itemize}
\item
  Jan. 30, 2017
\item
  \begin{itemize}
  \item
  \item
  \item
  \item
  \item
  \end{itemize}
\end{itemize}

\includegraphics{https://static01.graylady3jvrrxbe.onion/images/2017/01/31/world/middleeast/31SETTLEMENT-SS-slide-PMQ0/31SETTLEMENT-SS-slide-PMQ0-articleLarge.jpg?quality=75\&auto=webp\&disable=upscale}

MA'ALE ADUMIM, West Bank --- The first babies of Ma'ale Adumim, a hilly
city on the eastern outskirts of Jerusalem, are now middle-aged. A
cemetery finally opened last year, and 40 residents are buried there,
most dead of natural causes after long and peaceful lives.

That is to say, there is nothing temporary about this place, one of the
closest settlements to Jerusalem in the occupied West Bank, which Israel
seized from Jordan 50 years ago. ``It's part of Jerusalem,'' said Sima
Weiss, 58, who has lived here 30 years, raised three children and works
a cleaning job just 20 minutes away by bus in the holy city proper. ``I
don't feel like a settler.''

The world has focused more critically recently on Israel's settlements
in occupied territory, after last month's United Nations declaration ---
which the United States tacitly supported --- that they are killing the
dream of one state for Jews, one for Palestinians.

Many Israelis argue that Ma'ale Adumim --- a city of 41,000 with filled
schools, a largely secular civic pride and skittish stray cats --- is a
special case: Its closeness to Jerusalem has put it near the top of the
list of settlements Israelis say they could swap for other land in a
peace deal.

Yet Ma'ale Adumim has become a flash point of the conflict between
Palestinians and Israelis. Right-wing politicians, emboldened by a more
sympathetic Trump administration, want to annex it to Israel proper ---
the first formal annexation of a settlement. Supporters of the move
argue that in the long absence of negotiations, Israel cannot stand
still, and Ma'ale Adumim would likely be a part of Israel in any case.

Image

A view onto a balcony in Ma'ale Adumim, one of the closest settlements
to Jerusalem in the occupied West Bank.Credit...Dan Balilty for The New
York Times

``Clearly it's time for a quantum change,'' Naftali Bennett, the
education minister, who plans to introduce the annexation bill, said in
an interview. ``The incremental approach has not worked. We have to
understand it's a new reality. We have to go big, bold and fast.''

The Parliament seems poised to approve a law that few thought had any
chance of passage just a few months ago: It would ultimately legalize
settlement homes built illegally on private Palestinian land. Critics
call this yet another form of creeping annexation.

Many Palestinians agree this is a critical moment. They fear Ma'ale
Adumim will be just the beginning of the annexation of settlements in
the West Bank, now home to roughly 400,000 Jews, and the end of the
two-state dream.

``We believe in two states for two nations, but if they took that'' ---
Ma'ale Adumim --- ``there will be no longer two states,'' said Yousef
Mostafa Mkhemer, chairman of the Organization of Jerusalem
Steadfastness, which focuses on issues like Muslim holy sites, refugee
camps and Israeli settlements. ``There will be one state called
Israel.''

Many Palestinians and peace activists argue that the line has already
been crossed --- that any annexation of Ma'ale Adumim, after so many
years, would be a technicality.

Image

Palestinian laborers working at a construction site in Ma'ale Adumim
last week. Prime Minister Benjamin Netanyahu has allotted 100 new
building units to the city.Credit...Dan Balilty for The New York Times

``We are living in one state now,'' said Ziad Abu Zayyad, a Palestinian
lawyer and writer. Mr. Zayyad, a former Palestinian minister, said that
unlike most Palestinians he supported Donald J. Trump for president, in
part because he felt his apparently greater sympathy for Israel would
begin to provide a clarity to a long-stuck conflict.

``I want to see a change,'' he said. ``I'm fed up.''

``He could be a big devil. He could be something good. My point is he
will make a change, for the good or for the bad.''

There are signs, in fact, that the conflict here is already shifting,
with Ma'ale Adumim near the center, no matter how quiet and workaday its
residents think themselves (70 percent of residents commute to Jerusalem
proper for their jobs).

After eight years of little building, Prime Minister Benjamin Netanyahu
just allotted 100 new building units to Ma'ale Adumim, part of 2,500 new
proposed housing units around the West Bank settlements, and another 560
in East Jerusalem. Mr. Netanyanu has proclaimed this as just the
beginning of a new wave of building.

Image

After eight years of little building, Israel is planning 2,500 new
housing units around the West Bank settlements, and another 560 in East
Jerusalem.Credit...Dan Balilty for The New York Times

Less than a month after the United Nations resolution, the city's mayor,
Benny Kashriel, and another settlement leader proudly attended Mr.
Trump's inauguration. That would have been unthinkable for past incoming
American presidents, out of fear it could be interpreted as an
endorsement of settlements, which most of the world considers illegal.

``It's a different policy,'' Mr. Kashriel, mayor for 25 years, said just
a day back from Washington. He believes that the new administration sees
places like Ma'ale Adumim more benignly than did former President Barack
Obama, whose administration blocked much building here and in the nearby
E1, an especially contentious area closer to Jerusalem.

``We didn't steal the land from anybody,'' he said. ``It was built on
empty hills. You can see there --- the desert, rocks and sand. Now you
have a living city.''

Much of the outside world's attention has focused on more religious
settlements deeper into the West Bank or on land with Palestinian titles
in more direct conflict with Palestinians. But here, scrutiny has been
intense on Ma'ale Adumim.

It is partly symbolic: Israel has not annexed 1967 land beyond East
Jerusalem and the Golan Heights. Opponents of the move fear it would be
the start of a process that would not end until politicians like Mr.
Bennett achieved their dreams of annexing large swaths of the West Bank
and leaving the Palestinians with what Mr. Netanyahu recently called ``a
state-minus.''

Image

A shopping mall in Ma'ale Adumim. The city also has a library, a
theater, 15 schools and 78 kindergartens.Credit...Dan Balilty for The
New York Times

It is partly strategic: The settlement is at the heart of entrenched
plans to expand Jerusalem, linking it to the city proper, along with
other nearby settlements that also function in practice as Jerusalem
suburbs. One issue with Ma'ale Adumim, critics argue, is its place in
the West Bank, between north and south, that combined with other
building plans could both hamper transit of Palestinians and threaten
the contiguous borders of any future Palestinian state.

The area is also not as empty as Ma'ale Adumim's supporters say.

Eid Abu Khamis, the leader of some 8,000 Bedouins in the area, says
harassment by Israel has increased recently. More of their makeshift
housing has been torn down and land for their goats and sheep --- they
sell meat, yogurt and cheese to survive --- declared off-limits.

Most of the Bedouins live in the E1 area, which is technically a part of
Ma'ale Adumim and is slated for some 3,700 new housing units. The Obama
administration staunchly opposed any development in E1 as a possible
point of no return for a viable Palestinian state.

Image

Palestinian children playing in the street Sunday in the E1 section of
Ma'ale Adumim, an especially contentious area where some 8,000 Bedouins
live. It is slated for some 3,700 new housing units.Credit...Dan Balilty
for The New York Times

``If the Bedouin are kicked out of this land where we have lived for 30
years, it will be the end of negotiations with the State of Israel,''
Mr. Khamis said.

Many Palestinians argue that the annexation could ignite another round
of violent revolt. A Palestinian flag was recently planted in a park in
Ma'ale Adumim here, a worrying sign for residents that the less
expensive, less harried life in their suburb may change.

``I didn't come because I believe we should take all the land between
the Mediterranean and the Jordan River,'' said a 71-year-old resident, a
driver who would give his name only as Max S. ``It was cheap. That's why
I came. If I could change my apartment to an apartment in Tel Aviv, in a
minute I would do it. ``

People here are proud of what they have built since it was founded in
1975, with 15 religious Jews. There is a library, a theater, 15 schools
and 78 kindergartens. It is mostly secular, but gets along with its more
religious residents, about a quarter of the population. An industrial
park, while the occasional target of the worldwide campaign to boycott
goods made in settlements, is thriving and employs some 4,000
Palestinians --- at much higher wages, the mayor notes, than they could
earn in Palestinian areas.

Image

The Israeli settlement of Ma'ale Adumim was founded in 1975, with 15
religious Jews. Today it is a city of 41,000.Credit...Dan Balilty for
The New York Times

Ronit Jackov, 55, who works in the local mall (which is getting a new
floor with five cinemas), said she favored annexation, largely so the
city can begin to grow again after years of a building freeze.

She said she would never move back to Jerusalem, in part because the
city has become too religious and rigid. ``I'm not comfortable in a
place where people tell you how to live,'' she said. ``People want to
live their lives.''

And she places much hope in Mr. Trump, that he will side more forcefully
with Israelis.

``I'm not a very political person,'' she said. ``But I'm waiting for him
to carry out what he said, and I'll say, `You are great.' Because the
whole world is against us. We need someone on our side.''

Image

An Israeli man looking over Ma'ale Adumim last week. Supporters of the
annexation of the West Bank settlement see it as a near
inevitability.Credit...Dan Balilty for The New York Times

Advertisement

\protect\hyperlink{after-bottom}{Continue reading the main story}

\hypertarget{site-index}{%
\subsection{Site Index}\label{site-index}}

\hypertarget{site-information-navigation}{%
\subsection{Site Information
Navigation}\label{site-information-navigation}}

\begin{itemize}
\tightlist
\item
  \href{https://help.nytimes3xbfgragh.onion/hc/en-us/articles/115014792127-Copyright-notice}{©~2020~The
  New York Times Company}
\end{itemize}

\begin{itemize}
\tightlist
\item
  \href{https://www.nytco.com/}{NYTCo}
\item
  \href{https://help.nytimes3xbfgragh.onion/hc/en-us/articles/115015385887-Contact-Us}{Contact
  Us}
\item
  \href{https://www.nytco.com/careers/}{Work with us}
\item
  \href{https://nytmediakit.com/}{Advertise}
\item
  \href{http://www.tbrandstudio.com/}{T Brand Studio}
\item
  \href{https://www.nytimes3xbfgragh.onion/privacy/cookie-policy\#how-do-i-manage-trackers}{Your
  Ad Choices}
\item
  \href{https://www.nytimes3xbfgragh.onion/privacy}{Privacy}
\item
  \href{https://help.nytimes3xbfgragh.onion/hc/en-us/articles/115014893428-Terms-of-service}{Terms
  of Service}
\item
  \href{https://help.nytimes3xbfgragh.onion/hc/en-us/articles/115014893968-Terms-of-sale}{Terms
  of Sale}
\item
  \href{https://spiderbites.nytimes3xbfgragh.onion}{Site Map}
\item
  \href{https://help.nytimes3xbfgragh.onion/hc/en-us}{Help}
\item
  \href{https://www.nytimes3xbfgragh.onion/subscription?campaignId=37WXW}{Subscriptions}
\end{itemize}
