Sections

SEARCH

\protect\hyperlink{site-content}{Skip to
content}\protect\hyperlink{site-index}{Skip to site index}

\href{https://www.nytimes3xbfgragh.onion/section/food}{Food}

\href{https://myaccount.nytimes3xbfgragh.onion/auth/login?response_type=cookie\&client_id=vi}{}

\href{https://www.nytimes3xbfgragh.onion/section/todayspaper}{Today's
Paper}

\href{/section/food}{Food}\textbar{}The New Chumley's Raises the
Culinary Bar

\url{https://nyti.ms/2jRueGH}

\begin{itemize}
\item
\item
\item
\item
\item
\item
\end{itemize}

Advertisement

\protect\hyperlink{after-top}{Continue reading the main story}

Supported by

\protect\hyperlink{after-sponsor}{Continue reading the main story}

\href{/column/restaurant-review}{Restaurant Review}

\hypertarget{the-new-chumleys-raises-the-culinary-bar}{%
\section{The New Chumley's Raises the Culinary
Bar}\label{the-new-chumleys-raises-the-culinary-bar}}

\href{https://www.nytimes3xbfgragh.onion/slideshow/2017/01/31/dining/chumleys.html}{}

\hypertarget{chumleys}{%
\subsection{Chumley's}\label{chumleys}}

11 Photos

View Slide Show ›

\includegraphics{https://static01.graylady3jvrrxbe.onion/images/2017/02/01/dining/01REST-CHUMLEYS-slide-5W2C/01REST-CHUMLEYS-slide-5W2C-articleLarge.jpg?quality=75\&auto=webp\&disable=upscale}

Ben Sklar for The New York Times

\begin{itemize}
\tightlist
\item
  Chumley's\\
  ★★ American \$\$\$ 86 Bedford Street 212-675-2081
\end{itemize}

\href{http://www.opentable.com/single.aspx?ref=4201\&rid=266758}{Reserve
a Table}

When you make a reservation at an independently reviewed restaurant
through our site, we earn an affiliate commission.

By \href{http://www.nytimes3xbfgragh.onion/by/pete-wells}{Pete Wells}

\begin{itemize}
\item
  Jan. 31, 2017
\item
  \begin{itemize}
  \item
  \item
  \item
  \item
  \item
  \item
  \end{itemize}
\end{itemize}

If you heard that \href{http://chumleysnewyork.com/}{Chumley's} is open
again, you were misinformed. The dim, spare, beer-scented hideaway in
the West Village is gone, torn down, not coming back. At its old address
is a restaurant that has nothing in common with the original except a
name, a door, an archway and framed photographs of, and jackets of books
by, writers who used to drink there. Most of them wouldn't be able to
afford a cocktail there now, let alone dinner.

Chumley's first quietly and selectively opened its unmarked door, inside
a hidden courtyard at the end of an alley, during Prohibition. Flappers,
sailors, actors, Wobblies, writers heading for fame and bohemians
heading for nowhere ate and drank in its windowless rooms. Simone de
Beauvoir, who tumbled into Chumley's in the 1950s on a tip from Richard
Wright, wrote that it was ``utterly simple, with its little tables lined
against the walls, but it has something so rare in America ---
atmosphere.''

On busy nights in later years, the atmosphere could be like that of a
college rathskeller in a state with a low drinking age. But if you
caught Chumley's at the right hour, when there were empty booths and a
log burning in the fireplace, it could be a shelter from the city and an
ideal embodiment of it at the same time.

All this time, the building, almost a century old when Chumley's opened,
was falling apart --- slowly at first and then, in 2007, in a landslide
of bricks. It was so far gone that it had to be torn down and
\href{http://www.nytimes3xbfgragh.onion/2013/01/01/nyregion/chumleys-ex-speakeasy-in-greenwich-village-seems-on-mend.html}{rebuilt}.
The work went so slowly that Chumley's nearly lost its chance to renew
its liquor license.

Then Alessandro Borgognone, who owns
\href{http://www.nytimes3xbfgragh.onion/2013/12/11/dining/reviews/restaurant-review-sushi-nakazawa-in-the-west-village.html}{Sushi
Nakazawa} around the corner, was made a partner in the business; he must
have hit the construction crew with a dose of wasabi, because the
build-out zoomed along, the liquor license came through just in time,
and the restaurant began serving in October.

\includegraphics{https://static01.graylady3jvrrxbe.onion/images/2017/02/01/dining/01REST1/01REST1-articleInline.jpg?quality=75\&auto=webp\&disable=upscale}

The chef is
\href{https://www.nytimes3xbfgragh.onion/2016/09/07/dining/chumleys-bar-restaurant.html}{Victoria
Blamey}, who has cooked under
\href{http://www.nytimes3xbfgragh.onion/2013/06/05/dining/paul-liebrandt-exacting-chef-set-to-open-the-elm-in-williamsburg.html}{Paul
Liebrandt},
\href{https://www.nytimes3xbfgragh.onion/2015/02/16/dining/matthew-lightner-will-leave-atera.html}{Matthew
Lightner} and
\href{https://www.nytimes3xbfgragh.onion/2015/01/14/dining/restaurant-review-upland-on-park-avenue-south.html?mtrref=www.google.com\&gwh=3DDE5EAB4BAB624808D2D2A0314C0F17\&gwt=pay}{Justin
Smillie}.
\href{http://chumleysnewyork.com/wp-content/uploads/2016/11/chumleys-drinks.pdf}{Her
menu} has far-flung influences --- the old regulars would have taken one
look at it and called for another shot --- but she doesn't let the
dishes get wispy or abstract. Some of them even look like bar food, sort
of. And she loads them with more excitement than you'll come across in
other new restaurants that are getting far more attention.

Starting with a warm pretzel and French onion dip, you first notice just
how light the pretzel is. It is to the street-cart variety what a
hummingbird is to a Butterball turkey. The crust is salt-flecked and
rough with toasted onion powder. Orange dots of salmon roe sit on top of
the dip, which tastes pure and chemical free --- cooked garlic and
shallots and onions are stirred into a base of soft cream cheese,
cloumage and crème fraîche.

Chumley's beef tartare goes a few places other tartares don't. Instead
of mustard and capers, it takes its sharpness from confit tomatoes and
gratings of a sheep's milk cheese from Catalonia. It's meant to be
spread on a wavy, blistered puff of fried beef tendon. Crunchier and
airier than a cracker, the tendon has a very clean flavor and doesn't
soak up the taste of raw steak the way toast does.

There is a terrine that would be right at home in one of those casually
excellent bistros run by young chefs in Paris. Foie gras and shredded
ham hock are pressed between leaves of savoy cabbage; the cross section
you get on a vintage china plate is served with a tart, fruity puddle of
apple cider reduced to a gastrique. It's radically simple, and makes
something memorable out of ham hocks and cabbage. (To be fair, the foie
gras does its part, too.)

New York may not need another burger, but any place called Chumley's
definitely does. The one Ms. Blamey has provided is like an erotic poem
on the theme of fat.

It is a double-decker, both of its patties buried under American cheese
and soaked with bone marrow that's been hit with a blowtorch. Liquid
marrow falls from the burger; so do a few dark and crunchy fried
shallots and orange blobs of cheese. On the side are very crisp and
skinny fries tossed, after their second trip to the deep fryer, in
melted dry-aged beef fat.

I imagine the burger is pretty great after a few drinks. I ate it while
I was as sober as a mechanical pencil, and it was all I could do not to
paint my face with grease until I looked like Martin Sheen at the end of
``Apocalypse Now.''

Any qualms I had about the price, \$25, vanished while I was looking
around the floor for my napkin. I had also rolled my eyes at first when
I saw a \$43 potpie on the menu of a former speakeasy. The thing is, it
is an amazing potpie.

The body of a Dungeness crab is stuffed, all the way out to its spiky
corners, with a crab meat stew. Ms. Blamey based it on an abalone dish
from Chile, where she grew up, and it gets its kick from fresh ají dulce
peppers along with ground Aleppo pepper. The richness that comes from
the fatty innards of the crab helps the stew cling like cream to the
tender puff pastry crust, which is dotted with sesame seeds.

There are two desserts: a scoop of vanilla ice cream with pistachio
streusel and parsnip purée (it's good, I swear), and a gold-foil-wrapped
sandwich of ice cream made with Luxardo maraschino cherries pressed
between square almond-cocoa cookies that are as black as Oreos.

Image

Framed photographs of, and jackets of books by, writers who used to
drink at Chumley's.Credit...Ben Sklar for The New York Times

Along with cocktails, Chumley's drinks menu has a series of variations
on the Scotch and soda, a terrific idea. Highballs deserve more
attention from restaurants; they're more refreshing and more amenable to
food than more concentrated medicine like the manhattan.

Ms. Blamey works in a galley in the back. Across from her confined
territory are the restrooms, where you can visit framed slices of the
original wood tables. They were deeply carved over the years, and now
the names and initials are displayed behind glass like frescoes from
Pompeii.

Nobody at Chumley's is likely to take a switchblade to the tables now,
not after having to provide credit-card numbers to make a reservation.
It's possible to get a table without one, but you wouldn't know that
from the way the hosts greet new arrivals with one finger poised over an
iPad and an expression that says, ``Are you on the list?''

There is something a little too clubby, too, about the way cards that
read RESERVED are placed on the empty tables and in front of unoccupied
bar stools, even when the restaurant is winding down for the night.
Closing time is midnight now, a concession to neighbors who fought the
liquor license.

Chumley's is emphatically worth going to as long as Ms. Blamey remains
content to work in that kitchen. What happens after that is anyone's
guess. The reservations policy seems to keep people from just dropping
by for a drink, and casual drinking was part of the atmosphere that de
Beauvoir liked so much.

Now, instead of atmosphere, Chumley's has décor; the book jackets and
photographs are elements in a haunted house attraction featuring the
ghosts of Hemingway and Kerouac. The neighbors sleep better, but the
neighborhood isn't as interesting.

Advertisement

\protect\hyperlink{after-bottom}{Continue reading the main story}

\hypertarget{site-index}{%
\subsection{Site Index}\label{site-index}}

\hypertarget{site-information-navigation}{%
\subsection{Site Information
Navigation}\label{site-information-navigation}}

\begin{itemize}
\tightlist
\item
  \href{https://help.nytimes3xbfgragh.onion/hc/en-us/articles/115014792127-Copyright-notice}{©~2020~The
  New York Times Company}
\end{itemize}

\begin{itemize}
\tightlist
\item
  \href{https://www.nytco.com/}{NYTCo}
\item
  \href{https://help.nytimes3xbfgragh.onion/hc/en-us/articles/115015385887-Contact-Us}{Contact
  Us}
\item
  \href{https://www.nytco.com/careers/}{Work with us}
\item
  \href{https://nytmediakit.com/}{Advertise}
\item
  \href{http://www.tbrandstudio.com/}{T Brand Studio}
\item
  \href{https://www.nytimes3xbfgragh.onion/privacy/cookie-policy\#how-do-i-manage-trackers}{Your
  Ad Choices}
\item
  \href{https://www.nytimes3xbfgragh.onion/privacy}{Privacy}
\item
  \href{https://help.nytimes3xbfgragh.onion/hc/en-us/articles/115014893428-Terms-of-service}{Terms
  of Service}
\item
  \href{https://help.nytimes3xbfgragh.onion/hc/en-us/articles/115014893968-Terms-of-sale}{Terms
  of Sale}
\item
  \href{https://spiderbites.nytimes3xbfgragh.onion}{Site Map}
\item
  \href{https://help.nytimes3xbfgragh.onion/hc/en-us}{Help}
\item
  \href{https://www.nytimes3xbfgragh.onion/subscription?campaignId=37WXW}{Subscriptions}
\end{itemize}
