Sections

SEARCH

\protect\hyperlink{site-content}{Skip to
content}\protect\hyperlink{site-index}{Skip to site index}

\href{https://www.nytimes3xbfgragh.onion/section/world/middleeast}{Middle
East}

\href{https://myaccount.nytimes3xbfgragh.onion/auth/login?response_type=cookie\&client_id=vi}{}

\href{https://www.nytimes3xbfgragh.onion/section/todayspaper}{Today's
Paper}

\href{/section/world/middleeast}{Middle East}\textbar{}Trump Embraces
Pillars of Obama's Foreign Policy

\url{https://nyti.ms/2k0nMLb}

\begin{itemize}
\item
\item
\item
\item
\item
\item
\end{itemize}

Advertisement

\protect\hyperlink{after-top}{Continue reading the main story}

Supported by

\protect\hyperlink{after-sponsor}{Continue reading the main story}

\hypertarget{trump-embraces-pillars-of-obamas-foreign-policy}{%
\section{Trump Embraces Pillars of Obama's Foreign
Policy}\label{trump-embraces-pillars-of-obamas-foreign-policy}}

\includegraphics{https://static01.graylady3jvrrxbe.onion/images/2017/02/03/world/03DIPLO-1486081166907/03DIPLO-1486081166907-articleInline.jpg?quality=75\&auto=webp\&disable=upscale}

By \href{http://www.nytimes3xbfgragh.onion/by/mark-landler}{Mark
Landler}, \href{http://www.nytimes3xbfgragh.onion/by/peter-baker}{Peter
Baker} and
\href{http://www.nytimes3xbfgragh.onion/by/david-e-sanger}{David E.
Sanger}

\begin{itemize}
\item
  Feb. 2, 2017
\item
  \begin{itemize}
  \item
  \item
  \item
  \item
  \item
  \item
  \end{itemize}
\end{itemize}

WASHINGTON --- President Trump, after promising a radical break with the
foreign policy of Barack Obama, is embracing some key pillars of the
former administration's strategy, including warning Israel to curb
settlement construction, demanding that Russia withdraw from Crimea and
threatening Iran with sanctions for ballistic missile tests.

In the most startling shift, the White House issued an unexpected
statement appealing to the Israeli government not to expand the
construction of Jewish settlements beyond their current borders in East
Jerusalem and the West Bank. Such expansion, it said, ``may not be
helpful in achieving'' the goal of peace.

At the United Nations, Ambassador Nikki R. Haley declared that the
United States would not lift sanctions against Russia until it stopped
destabilizing Ukraine and pulled troops out of Crimea.

On Iran, the administration is preparing economic sanctions similar to
those the Obama administration imposed just over a year ago. The White
House has also shown no indication that it plans to rip up Mr. Obama's
landmark nuclear deal, despite Mr. Trump's withering criticism of it
during the presidential campaign.

New administrations often fail to change the foreign policies of their
predecessors as radically as they promised, in large part because
statecraft is so different from campaigning. And of course, today's
positions could shift over time. There is no doubt the Trump
administration has staked out new ground on trade and immigration,
upending relations with Mexico and large parts of the Muslim world in
the process.

But the administration's reversals were particularly stark because they
came after days of tempestuous phone calls between Mr. Trump and foreign
leaders, in which he gleefully challenged diplomatic orthodoxy and
appeared to jeopardize one relationship after another.

Mr. Trump made warmer relations with Russia the centerpiece of his
foreign policy during the campaign, and European leaders had been
steeling for him to lift sanctions they and Mr. Obama imposed on
President Vladimir V. Putin after he annexed Crimea. But on Thursday,
Mr. Trump's United Nations ambassador, Ms. Haley, sounded a lot like her
predecessor, Samantha Power.

``We do want to better our relations with Russia,'' she said in her
first remarks to an open session of the United Nations Security Council.
``However, the dire situation in eastern Ukraine is one that demands
clear and strong condemnation of Russian actions.''

Similarly, Mr. Trump presented himself during the campaign as a stalwart
supporter of Israel and
\href{https://www.nytimes3xbfgragh.onion/2016/12/22/world/middleeast/donald-trump-united-nations-israel-settlements.html}{criticized}
the Obama administration for
\href{https://www.nytimes3xbfgragh.onion/2016/12/23/world/middleeast/israel-settlements-un-vote.html}{allowing
the passage of a Security Council resolution} in December that condemned
Israel for its expansion of settlements.

``While we don't believe the existence of settlements is an impediment
to peace,'' his press secretary, Sean Spicer, said in a statement, ``the
construction of new settlements or the expansion of existing settlements
beyond their current borders may not be helpful in achieving that
goal.''

The White House noted that the president ``has not taken an official
position on settlement activity.'' It said he would discuss the issue
with Prime Minister Benjamin Netanyahu of Israel when they meet Feb. 15,
in effect telling Mr. Netanyahu to wait until then. Emboldened by Mr.
Trump's support, Israel
\href{https://www.nytimes3xbfgragh.onion/2017/02/01/world/middleeast/israel-3000-homes-west-bank.html}{has
announced} more than 5,000 new homes in the West Bank since his
inauguration.

Mr. Trump shifted his policy after he met briefly with King Abdullah II
of Jordan on the sidelines of the National Prayer Breakfast --- an
encounter that put the king, one of the most respected leaders of the
Arab world, ahead of Mr. Netanyahu in seeing the new president. Jordan,
with its large Palestinian population, has been steadfastly critical of
settlements.

The administration's abrupt turnaround also coincided with Secretary of
State Rex W. Tillerson's first day at the State Department and the
arrival of Defense Secretary Jim Mattis in South Korea
\href{https://www.nytimes3xbfgragh.onion/2017/02/02/world/asia/james-mattis-us-korea-thaad.html}{on
his first official trip}. Both men are viewed as potentially capable of
exerting a moderating influence on the president and his cadre of White
House advisers, though it was unclear how much they had to do with the
shifts.

With Iran, Mr. Trump has indisputably taken a harder line than his
predecessor. While the Obama administration often looked for ways to
avoid confrontation with Iran in its last year, Mr. Trump seems equally
eager to challenge what he has said is an Iranian expansion across the
region, especially in Iraq and Yemen.

In an early morning Twitter post on Thursday, Mr. Trump was bombastic on
Iran. ``Iran has been formally PUT ON NOTICE for firing a ballistic
missile,'' he wrote. ``Should have been thankful for the terrible deal
the U.S. made with them!'' In a second post, he said wrongly, ``Iran was
on its last legs and ready to collapse until the U.S. came along and
gave it a life-line in the form of the Iran Deal: \$150 billion.''

Still, the administration has been careful not to specify what the
national security adviser, Michael T. Flynn, meant when he said on
Wednesday that Iran had been put ``on notice'' for its missile test and
for its arming and training of the Houthi rebels in Yemen.

The new sanctions could be announced as soon as Friday. But most experts
have said they will have little practical effect, because the companies
that supply missile parts rarely have direct business with the United
States, and allies have usually been reluctant to reimpose sanctions
after many were lifted as part of the 2015 nuclear accord.

Ali Akbar Velayati, an adviser to Iran's supreme leader, replied, ``This
is not the first time that an inexperienced person has threatened
Iran,'' according to the semiofficial Fars news agency. ``The American
government will understand that threatening Iran is useless.''

Some analysts said they worried that the administration did not have
tools, short of military action, to back up its warning.

``Whether the Trump administration intended it or not, they have created
their own red line,'' said Aaron David Miller, a senior fellow at the
Woodrow Wilson International Center for Scholars. ``When Iran tests
again, the administration will have no choice but to put up or shut
up.''

Mr. Netanyahu will cheer Mr. Trump's tough tone with Iran. But the
statement on settlements may force him to change course on a delicate
domestic issue. His coalition government seemed to take Mr. Trump's
inauguration as a starting gun in a race to increase construction in
occupied territory.

After Mr. Trump was sworn in, Israel announced that it would authorize
another 2,500 homes in areas already settled in the West Bank, and then
followed that this week by announcing 3,000 more. On Wednesday, Mr.
Netanyahu took it a step further, vowing to build the first new
settlement in the West Bank in many years.

For Mr. Netanyahu, the settlement spree reflects a sense of liberation
after years of constraints from Washington, especially under Mr. Obama,
who, like other presidents, viewed settlement construction as an
impediment to negotiating a final peace settlement. It is also an effort
to deflect criticism from Israel's political right for Mr. Netanyahu's
compliance with a court order to
\href{https://www.nytimes3xbfgragh.onion/video/world/middleeast/100000004906367/jewish-settlers-resist-outpost-evacuation.html}{force
several dozen families} out of an illegal West Bank outpost, Amona.

The ``beyond their current borders'' phrase in the White House statement
hinted at a return to a policy President George W. Bush outlined to
Prime Minister Ariel Sharon in 2004, which acknowledged that it was
unrealistic to expect Israel to give up its major settlements in a final
deal, although they would be offset by mutually agreed-upon land swaps.

Mr. Trump had also promised
\href{https://www.nytimes3xbfgragh.onion/2017/01/19/world/middleeast/donald-trump-jerusalem-embassy-israel-palestinians.html}{to
move the American Embassy} from Tel Aviv to Jerusalem. But the White
House has slowed down the move, in part out of fear of a violent
response.

The policy shifts came after a turbulent week in which Mr. Trump also
clashed with
\href{https://www.nytimes3xbfgragh.onion/2017/02/02/world/australia/donald-trump-malcolm-turnbull-refugees.html}{the
leaders of Australia} and
\href{https://www.nytimes3xbfgragh.onion/2017/01/26/world/americas/mexico-pena-nieto-donald-trump.html}{Mexico}
over one of the most fraught issues of his new presidency: immigration.
He defended the tense exchanges as an overdue display of toughness by a
United States that has been exploited ``by every nation in the world,
virtually.''

``They're tough; we have to be tough. It's time we're going to be a
little tough, folks,'' he said at the prayer breakfast Thursday. ``It's
not going to happen anymore.''

Yet later in the day, the White House felt obliged to put a more
diplomatic gloss on events. Mr. Spicer said
\href{https://www.nytimes3xbfgragh.onion/2017/02/02/us/politics/us-australia-trump-turnbull.html}{Mr.
Trump's call} with Prime Minister Malcolm Turnbull of Australia had been
``very cordial,'' even if Mr. Trump bitterly opposed an agreement
negotiated by the Obama administration for the United States to accept
the transfer of 1,250 refugees from an Australian detention camp.

A senior administration official disputed a report that Mr. Trump had
threatened to send troops to Mexico to deal with its ``bad hombres.''
The official said that the conversation with President Enrique Peña
Nieto had been ``actually very friendly,'' and that Mr. Trump had been
speaking in jest.

Advertisement

\protect\hyperlink{after-bottom}{Continue reading the main story}

\hypertarget{site-index}{%
\subsection{Site Index}\label{site-index}}

\hypertarget{site-information-navigation}{%
\subsection{Site Information
Navigation}\label{site-information-navigation}}

\begin{itemize}
\tightlist
\item
  \href{https://help.nytimes3xbfgragh.onion/hc/en-us/articles/115014792127-Copyright-notice}{©~2020~The
  New York Times Company}
\end{itemize}

\begin{itemize}
\tightlist
\item
  \href{https://www.nytco.com/}{NYTCo}
\item
  \href{https://help.nytimes3xbfgragh.onion/hc/en-us/articles/115015385887-Contact-Us}{Contact
  Us}
\item
  \href{https://www.nytco.com/careers/}{Work with us}
\item
  \href{https://nytmediakit.com/}{Advertise}
\item
  \href{http://www.tbrandstudio.com/}{T Brand Studio}
\item
  \href{https://www.nytimes3xbfgragh.onion/privacy/cookie-policy\#how-do-i-manage-trackers}{Your
  Ad Choices}
\item
  \href{https://www.nytimes3xbfgragh.onion/privacy}{Privacy}
\item
  \href{https://help.nytimes3xbfgragh.onion/hc/en-us/articles/115014893428-Terms-of-service}{Terms
  of Service}
\item
  \href{https://help.nytimes3xbfgragh.onion/hc/en-us/articles/115014893968-Terms-of-sale}{Terms
  of Sale}
\item
  \href{https://spiderbites.nytimes3xbfgragh.onion}{Site Map}
\item
  \href{https://help.nytimes3xbfgragh.onion/hc/en-us}{Help}
\item
  \href{https://www.nytimes3xbfgragh.onion/subscription?campaignId=37WXW}{Subscriptions}
\end{itemize}
