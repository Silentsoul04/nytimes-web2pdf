Sections

SEARCH

\protect\hyperlink{site-content}{Skip to
content}\protect\hyperlink{site-index}{Skip to site index}

\href{https://www.nytimes3xbfgragh.onion/section/business}{Business}

\href{https://myaccount.nytimes3xbfgragh.onion/auth/login?response_type=cookie\&client_id=vi}{}

\href{https://www.nytimes3xbfgragh.onion/section/todayspaper}{Today's
Paper}

\href{/section/business}{Business}\textbar{}Lockheed Lowers Price on
F-35 Fighters, After Prodding by Trump

\url{https://nyti.ms/2k5A58N}

\begin{itemize}
\item
\item
\item
\item
\item
\end{itemize}

Advertisement

\protect\hyperlink{after-top}{Continue reading the main story}

Supported by

\protect\hyperlink{after-sponsor}{Continue reading the main story}

\hypertarget{lockheed-lowers-price-on-f-35-fighters-after-prodding-by-trump}{%
\section{Lockheed Lowers Price on F-35 Fighters, After Prodding by
Trump}\label{lockheed-lowers-price-on-f-35-fighters-after-prodding-by-trump}}

\includegraphics{https://static01.graylady3jvrrxbe.onion/images/2017/02/04/business/04FIGHTER/04FIGHTER-articleInline.jpg?quality=75\&auto=webp\&disable=upscale}

By
\href{http://www.nytimes3xbfgragh.onion/by/christopher-drew}{Christopher
Drew}

\begin{itemize}
\item
  Feb. 3, 2017
\item
  \begin{itemize}
  \item
  \item
  \item
  \item
  \item
  \end{itemize}
\end{itemize}

After weeks of pressure from President Trump, Lockheed Martin agreed on
Friday to a somewhat larger price cut on its F-35 fighters than it had
on the last few orders, and finally brought the cost of the main version
below \$100 million for each jet.

The Pentagon will buy 90 of the radar-evading planes under the new
contract for \$8.2 billion.

The F-35 is by far the Pentagon's largest program; it has plans to
eventually build more than 2,400 of them for the Air Force, Navy and
Marines, and hundreds more for allies. Mr. Trump began to criticize the
much-delayed project shortly after he was elected and met with
Lockheed's chief executive, Marillyn A. Hewson, twice to try to bring
down the price.

Officials with the military and with Lockheed said the president's
intervention helped speed up the negotiations and contributed to the
cost savings. But some savings will simply come from increased
manufacturing efficiencies as the production rate grows, making it
difficult to quantify how much can be attributed to Mr. Trump's
prodding.

The price of the Air Force version of the plane, for instance, will drop
7.3 percent to \$94.6 million from \$102 million in the previous batch
that the government ordered before the November election. That price was
5.5 percent lower than the \$108 million that Lockheed had charged for
the planes in the previous production lot.

Lt. Gen. Christopher C. Bogdan, the Air Force officer in charge of the
F-35 program, predicted in mid-December that the prices would drop by 6
to 7 percent in the contract that was announced on Friday.

Mr. Trump got out in front of the contract news on Monday, telling
reporters at the White House that negotiators for the Pentagon ``were
having a lot of difficulty, there was no movement, and I was able to get
\$600 million approximately off those planes.''

Lowering the price of the Air Force version to less than \$100 million
is a long-awaited milestone for the program, which began 15 years ago as
a way to create a versatile but relatively inexpensive fighter compared
with the more acrobatic F-22. Lockheed has built around 200 of the
planes so far, including a version for the Marines that lands and takes
off like a helicopter and a Navy model that flies from aircraft
carriers.

The complexity involved in building three different versions initially
caused the program to spin out of control, raising the expected cost to
\$400 billion for 2,443 planes. The Pentagon restructured the project in
2011, and the price of each plane has gradually dropped since then.

Advertisement

\protect\hyperlink{after-bottom}{Continue reading the main story}

\hypertarget{site-index}{%
\subsection{Site Index}\label{site-index}}

\hypertarget{site-information-navigation}{%
\subsection{Site Information
Navigation}\label{site-information-navigation}}

\begin{itemize}
\tightlist
\item
  \href{https://help.nytimes3xbfgragh.onion/hc/en-us/articles/115014792127-Copyright-notice}{©~2020~The
  New York Times Company}
\end{itemize}

\begin{itemize}
\tightlist
\item
  \href{https://www.nytco.com/}{NYTCo}
\item
  \href{https://help.nytimes3xbfgragh.onion/hc/en-us/articles/115015385887-Contact-Us}{Contact
  Us}
\item
  \href{https://www.nytco.com/careers/}{Work with us}
\item
  \href{https://nytmediakit.com/}{Advertise}
\item
  \href{http://www.tbrandstudio.com/}{T Brand Studio}
\item
  \href{https://www.nytimes3xbfgragh.onion/privacy/cookie-policy\#how-do-i-manage-trackers}{Your
  Ad Choices}
\item
  \href{https://www.nytimes3xbfgragh.onion/privacy}{Privacy}
\item
  \href{https://help.nytimes3xbfgragh.onion/hc/en-us/articles/115014893428-Terms-of-service}{Terms
  of Service}
\item
  \href{https://help.nytimes3xbfgragh.onion/hc/en-us/articles/115014893968-Terms-of-sale}{Terms
  of Sale}
\item
  \href{https://spiderbites.nytimes3xbfgragh.onion}{Site Map}
\item
  \href{https://help.nytimes3xbfgragh.onion/hc/en-us}{Help}
\item
  \href{https://www.nytimes3xbfgragh.onion/subscription?campaignId=37WXW}{Subscriptions}
\end{itemize}
