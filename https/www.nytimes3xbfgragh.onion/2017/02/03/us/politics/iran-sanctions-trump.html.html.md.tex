Sections

SEARCH

\protect\hyperlink{site-content}{Skip to
content}\protect\hyperlink{site-index}{Skip to site index}

\href{https://www.nytimes3xbfgragh.onion/section/world/middleeast}{Middle
East}

\href{https://myaccount.nytimes3xbfgragh.onion/auth/login?response_type=cookie\&client_id=vi}{}

\href{https://www.nytimes3xbfgragh.onion/section/todayspaper}{Today's
Paper}

\href{/section/world/middleeast}{Middle East}\textbar{}U.S. Imposes New
Sanctions on Iran Over Missile Test

\url{https://nyti.ms/2ka4Lbt}

\begin{itemize}
\item
\item
\item
\item
\item
\end{itemize}

Advertisement

\protect\hyperlink{after-top}{Continue reading the main story}

Supported by

\protect\hyperlink{after-sponsor}{Continue reading the main story}

\hypertarget{us-imposes-new-sanctions-on-iran-over-missile-test}{%
\section{U.S. Imposes New Sanctions on Iran Over Missile
Test}\label{us-imposes-new-sanctions-on-iran-over-missile-test}}

\includegraphics{https://static01.graylady3jvrrxbe.onion/images/2017/02/04/world/04trump2/04trump2-articleInline.jpg?quality=75\&auto=webp\&disable=upscale}

By \href{http://www.nytimes3xbfgragh.onion/by/david-e-sanger}{David E.
Sanger}

\begin{itemize}
\item
  Feb. 3, 2017
\item
  \begin{itemize}
  \item
  \item
  \item
  \item
  \item
  \end{itemize}
\end{itemize}

WASHINGTON --- New sanctions that the Trump administration imposed on
Friday to punish Tehran's latest ballistic missile test marked the
beginning of what officials called the end of an era in which the United
States was ``too tolerant of Iran's bad behavior.''

In what was described as the first in a series of efforts to confront
Iran around the globe, the ban on banking transfers was levied against
25 Iranians and companies that officials said assisted in Tehran's
ballistic missile program and support of terrorist groups.

The immediate trigger for the sanctions, which drew from a list of
targets drawn up last year by the Obama administration, was Iran's
missile test last Sunday. The exact details of the test remain shrouded
in considerable mystery. But the way the two countries jabbed at each
other --- with the White House saying it would ``no longer tolerate
Iran's provocations that threaten our interests,'' and Iranian state
news media vowing retaliation --- had distinct echoes of the darkest
days before the July 2015 nuclear accord was reached.

In striking that deal 19 months ago, the Obama administration was
gambling that, over time, Washington and Tehran would learn how to
manage their differences and cooperate on one or two discrete projects,
starting with eliminating the Islamic State. But that era never arrived.
And with the announcements on Friday, it became clearer than ever that
leaders in both countries now see an advantage in taking a hard line ---
each betting that the other does not have the stomach for a risky,
expensive confrontation.

``The danger is that this is the first stage in an escalation that could
culminate in a military confrontation between Iran and the United
States, or Iran and Israel,'' said Karim Sadjadpour of the Carnegie
Endowment for International Peace. ``The entire eight years of the Obama
administration was an example of unprecedented but largely
unreciprocated overtures for cooperation with Iran in the Middle East.
The Iranians weren't interested. And now, the Iranians sense the rest of
the world would not line up with the Trump administration.''

\includegraphics{https://static01.graylady3jvrrxbe.onion/images/2017/02/04/us/04XP-TRUMP/04XP-TRUMP-articleInline.jpg?quality=75\&auto=webp\&disable=upscale}

The sanctions themselves are unlikely to have a significant effect on
Iranian action. They strike at specific companies and arms traders from
Iran to Lebanon and China. Mr. Obama took similar steps a year ago,
after another Iranian missile test. But by and large, his administration
tried to de-escalate tensions --- and at one point even assured European
banks that, under the nuclear deal, they were free to resume
transactions with Iran without fear of American retaliation.

In announcing the new sanctions, the White House made clear that it
planned to call out every violation, and respond. The Treasury
Department took the unusual step of describing the inner workings of
three networks that produce technology for Iran around the globe, in an
effort to expose front companies and signal a new level of pressure on
Tehran.

``The international community has been too tolerant of Iran's bad
behavior,'' said Michael T. Flynn, the president's national security
adviser. ``The ritual of convening a United Nations Security Council in
an emergency meeting and issuing a strong statement is not enough. The
Trump administration will no longer tolerate Iran's provocations that
threaten our interests.''

Kate Bauer, a former Treasury official who is now at the Washington
Institute for Near East Policy, said the sanctions and the announcements
surrounding it were ``a way to take back the narrative, to declare that
this is not a `post-sanctions era.'''

``By providing so much public detail about the networks that feed Iran's
missile program,'' she said, ``they will cause significant disruption.''

Even inside the White House it is unclear how much further, beyond
sanctions, President Trump is willing to take the confrontation. While
he suggested during his campaign that he might scrap the nuclear deal,
which he described as a ``disaster,'' both his defense secretary, Jim
Mattis, and his secretary of state, Rex W. Tillerson, made clear during
their confirmation hearings that the world was better off with the
accord, for the next decade at least, because of its prohibitions on
Iran amassing enough enriched uranium or separated plutonium to
manufacture even a single nuclear weapon.

The Iranians have largely complied with every provision of the deal,
according to the International Atomic Energy Agency, which conducts
regular inspections of the nuclear facilities. When small violations
have been found, the Iranians have quickly rectified them, including by
shipping fuel out of the country.

But nothing in the nuclear agreement deals with Iran's support of
Hezbollah or other terrorist groups, or its missile testing. A United
Nations Security Council resolution, also negotiated in Vienna as the
nuclear accord was being completed, calls on Iran to show restraint in
testing, and prohibits test flights of a missile that could carry a
nuclear warhead. Iran maintains that none of its missiles are designed
for that purpose, though outside experts note it would be fairly easy to
alter one to fit a warhead.

It is unclear exactly what Iran was testing last weekend. Its missile
traveled about 600 miles before its re-entry vehicle exploded. That may
have been accidental, or an intentional detonation. Reports in Germany
have suggested that a cruise missile --- harder to strike with missile
defenses --- was also launched, but American officials have not
confirmed that.

But in both Washington and Tehran, the test itself was clearly less
important than the symbolism of the moment. Mr. Trump wanted to
demonstrate he would not tolerate even minor infractions of Iran's
commitments. For their part, the Iranians wanted to demonstrate that
they would continue any activity not specifically prohibited by the
nuclear accord, and would not be intimidated. On Friday, hours after the
sanctions were announced, the Foreign Ministry in Tehran promised to
impose ``legal restrictions'' on an unspecified number of American
individuals and entities --- in effect, a retaliatory blacklist.

Since Americans are already prohibited from doing business in Iran, it
was far from clear what they had in mind.

In a statement carried on state television, the ministry said the
identities of the American targets would be announced later, and that
those targeted ``were involved in helping and founding regional
terrorist groups.''

That appeared to be a response to a part of the sanctions aimed at
Iran's support for various proxy forces in the region, including the
Houthi rebels in Yemen.

A senior administration official called Iran's moves ``destabilizing.''
Asked whether the administration believed Iran controlled everything
that Houthi rebels were doing in Yemen, he conceded that Tehran may not
make every tactical decision but said it arms and supports the rebels.

He said that the sanctions were ``initial steps in response to Iranian
provocative behavior.'' The official spoke at a briefing for reporters
under rules, set by the administration, that prohibited naming those
conducting it.

Democrats did not criticize the sanctions, and even some former members
of the Obama administration said they saw value in pushing back against
the Iranians. But Senator Mark Warner, Democrat of Virginia, who serves
on the Senate Intelligence Committee, warned against provoking Iran into
further action.

``I urge the administration to bring clarity to their overall strategy
towards Iran, and to refrain from ambiguous rhetoric --- or provocative
tweets --- that will exacerbate efforts to confront those challenges.''

Advertisement

\protect\hyperlink{after-bottom}{Continue reading the main story}

\hypertarget{site-index}{%
\subsection{Site Index}\label{site-index}}

\hypertarget{site-information-navigation}{%
\subsection{Site Information
Navigation}\label{site-information-navigation}}

\begin{itemize}
\tightlist
\item
  \href{https://help.nytimes3xbfgragh.onion/hc/en-us/articles/115014792127-Copyright-notice}{©~2020~The
  New York Times Company}
\end{itemize}

\begin{itemize}
\tightlist
\item
  \href{https://www.nytco.com/}{NYTCo}
\item
  \href{https://help.nytimes3xbfgragh.onion/hc/en-us/articles/115015385887-Contact-Us}{Contact
  Us}
\item
  \href{https://www.nytco.com/careers/}{Work with us}
\item
  \href{https://nytmediakit.com/}{Advertise}
\item
  \href{http://www.tbrandstudio.com/}{T Brand Studio}
\item
  \href{https://www.nytimes3xbfgragh.onion/privacy/cookie-policy\#how-do-i-manage-trackers}{Your
  Ad Choices}
\item
  \href{https://www.nytimes3xbfgragh.onion/privacy}{Privacy}
\item
  \href{https://help.nytimes3xbfgragh.onion/hc/en-us/articles/115014893428-Terms-of-service}{Terms
  of Service}
\item
  \href{https://help.nytimes3xbfgragh.onion/hc/en-us/articles/115014893968-Terms-of-sale}{Terms
  of Sale}
\item
  \href{https://spiderbites.nytimes3xbfgragh.onion}{Site Map}
\item
  \href{https://help.nytimes3xbfgragh.onion/hc/en-us}{Help}
\item
  \href{https://www.nytimes3xbfgragh.onion/subscription?campaignId=37WXW}{Subscriptions}
\end{itemize}
