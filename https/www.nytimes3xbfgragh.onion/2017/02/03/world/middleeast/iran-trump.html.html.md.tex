Sections

SEARCH

\protect\hyperlink{site-content}{Skip to
content}\protect\hyperlink{site-index}{Skip to site index}

\href{https://www.nytimes3xbfgragh.onion/section/world/middleeast}{Middle
East}

\href{https://myaccount.nytimes3xbfgragh.onion/auth/login?response_type=cookie\&client_id=vi}{}

\href{https://www.nytimes3xbfgragh.onion/section/todayspaper}{Today's
Paper}

\href{/section/world/middleeast}{Middle East}\textbar{}Iran Treads
Cautiously With Trump. So Far.

\url{https://nyti.ms/2kxMjuw}

\begin{itemize}
\item
\item
\item
\item
\item
\end{itemize}

Advertisement

\protect\hyperlink{after-top}{Continue reading the main story}

Supported by

\protect\hyperlink{after-sponsor}{Continue reading the main story}

\hypertarget{iran-treads-cautiously-with-trump-so-far}{%
\section{Iran Treads Cautiously With Trump. So
Far.}\label{iran-treads-cautiously-with-trump-so-far}}

!{[}Iranian Hard-liners have been critical of President Hassan Rouhani.
With President Trump in power, some dismiss him as a figure of the past,
the right answer in the Obama era but the wrong one now.

Credit...Abedin Taherkenareh/European Pressphoto
Agency{]}(\url{https://static01.graylady3jvrrxbe.onion/images/2017/02/04/world/04Iran/04Iran-articleInline.jpg?quality=75\&auto=webp\&disable=upscale})

By \href{http://www.nytimes3xbfgragh.onion/by/thomas-erdbrink}{Thomas
Erdbrink}

\begin{itemize}
\item
  Feb. 3, 2017
\item
  \begin{itemize}
  \item
  \item
  \item
  \item
  \item
  \end{itemize}
\end{itemize}

TEHRAN --- Until President Trump's inauguration, Iran's clerics felt
comfortable leading worshipers in a chorus of ``Death to America'' while
simultaneously signing
\href{https://www.nytimes3xbfgragh.onion/2016/12/11/world/middleeast/iran-boeing-airplane-deal.html}{a
\$16.6 billion deal} with Boeing. Now, the establishment is treading
carefully, with even most hard-liners concerned that the smallest
provocation could lead to military conflict.

But some question how long their caution will last in the face of a
Trump administration that has brought a new level of hostility and
confrontation to a relationship that under President Barack Obama was
stable, if brittle.

Tensions flared this week after Iran confirmed that it had
\href{https://www.nytimes3xbfgragh.onion/2017/01/30/world/middleeast/iran-missile-test.html?_r=0}{conducted
a missile test} and the national security adviser, Michael T. Flynn,
\href{https://www.nytimes3xbfgragh.onion/2017/02/01/world/middleeast/iran-missile-test.html}{warned
Iran} on Wednesday that it had been put ``on notice.''

On Friday, the administration
\href{https://www.nytimes3xbfgragh.onion/2017/02/03/us/politics/iran-sanctions-trump.html}{slapped
economic sanctions on 25 people and entities} in response to the missile
test and suggested that might be just the beginning. Iran responded by
announcing a retaliatory blacklist of unspecified American people and
entities who would be subject to ``legal restrictions.''

Mr. Trump added his own blustery Twitter message as a prelude to the
American actions:

Iran's foreign minister, Mohammad Javad Zarif, the American-educated
diplomat who has been the country's most prominent spokesman to the
West, publicly played down the Trump administration's threats on Friday
in a restrained rejoinder via his own Twitter account:

Other Iranian officials have also responded mildly to the Trump
administration.

After
\href{https://www.nytimes3xbfgragh.onion/2017/01/27/us/politics/refugee-muslim-executive-order-trump.html}{Mr.
Trump's Jan. 27 executive order} barring citizens from seven
Muslim-majority countries, including Iran, the Iranian government
expressed anger. But so far it has reciprocated only by banning
\href{https://www.nytimes3xbfgragh.onion/2017/02/03/sports/iran-american-wrestling-team-world-cup.html}{a
visit by American wrestlers}.

A Foreign Ministry spokesman, Bahram Qassemi, said on Thursday, ``It is
a shame that the U.S. government, instead of thanking the Iranian nation
for their continued fight against terrorism, keeps repeating unfounded
claims and adopts unwise policies that are effectively helping terrorist
groups.''

Mr. Flynn also made clear that the challenge to Iran extended beyond the
missile test, holding Tehran responsible for an attack on a Saudi
warship by Houthi rebels in Yemen, who the United States says are
supported by Iran. Iran denies that, but the remark was taken as a
veiled warning about Iran's support of regional proxies like Hezbollah
in Lebanon and Shiite militias in Iraq.

``The Trump administration condemns actions by Iran that undermine
security, prosperity and stability throughout the Middle East and place
American lives at risk,'' Mr. Flynn said on Wednesday.

Later that day, Mr. Trump
\href{https://twitter.com/realDonaldTrump/status/826990079738540033}{said
on Twitter} that ``Iran is rapidly taking over Iraq,'' its neighbor,
even after ``the U.S. squandered three trillion dollars there. Obvious
long ago.''

During the presidential campaign, Mr. Trump spoke favorably of
\href{https://www.nytimes3xbfgragh.onion/2016/04/28/us/politics/donald-trump-foreign-policy-speech.html}{unpredictability
in foreign policy}, pointing to the Reagan presidency as an example of
the benefits of keeping opponents off balance. Since taking office, he
has been good to his word, and Iranians have noticed.

``Trump is not predictable for Americans, not for Europeans and not for
us,'' said Nader Karimi Joni, an analyst close to the government of
President Hassan Rouhani. ``He and his team are not trustworthy. They
will not honor any agreement. Nothing good is coming from this.''

Certainly not for Mr. Rouhani, a moderate who came to power promising to
ratchet down tensions with the West, cinch a nuclear deal and get Iran's
economy moving again. Now, all those goals are in jeopardy, and Mr.
Rouhani's re-election this spring is far from assured.

On Thursday, an aide to Iran's supreme leader, Ayatollah Ali Khamenei,
called Mr. Trump's remarks ``hollow rants'' and said that they ``would
bring losses for his country's national interest.'' And a former foreign
minister, Ali Akbar Velayati, said that Iran would continue to test
missiles.

Mr. Rouhani has called Mr. Trump
\href{http://www.reuters.com/article/us-usa-trump-immigration-iran-idUSKBN15G3NK}{a
political novice}. But there is little doubt that the clerics have been
thrown off balance. One analyst with access to government deliberations
said that hard-liners in Iran were confused and did not know how to deal
with the situation. Some in the establishment are opting for the same
rhetoric and tactics they used under Mr. Obama, but in reality, this is
uncharted territory, he said.

Mr. Trump has filled his foreign policy team with advisers like Mr.
Flynn and the secretary of defense, Jim Mattis, who consider Iran to be
the greatest cause of instability in the Middle East, if not the world,
and who say the Obama administration treated Tehran with kid gloves.

When Iran test-fired a missile on Sunday, it did not announce the
launching on state television, something the country's leadership
usually likes to do to send a message to its adversaries. Missile tests
are commonplace in Iran, but Sunday's muted approach was clearly meant
to test the Trump administration's reaction.

Mr. Flynn did not elaborate on what he meant by saying Iran was on
notice. But in Iran, many agreed that whatever it meant, it did not bode
well.

``When they say we are put on notice, it means we should be very
careful,'' said Soroush Farhadian, a reformist journalist. ``We need
self-restraint. Our military men need to be prevented from making
blunders. One incident in the Persian Gulf can lead to conflict right
now.''

In recent months, speedboats belonging to the Islamic Revolutionary
Guards Corps have made a practice of aggressively approaching United
States naval vessels, then veering off. During the campaign, Mr. Trump
vowed that if they tried that under his watch, he would ``blow them out
of the water.'' There have been no reported incidents since Mr. Trump
took office.

Ayatollah Khamenei was uncharacteristically quiet during a visit on
Thursday to the martyr's graves, usually a moment for brief remarks
about current affairs. Mr. Rouhani's position is more delicate. He has
been promoting ties with the United States. And though he executed the
nuclear agreement with the blessing of Ayatollah Khamenei, the supreme
leader has also been critical of the deal.

Having tied his political future to the nuclear agreement and having
promised to normalize relations with the West, Mr. Rouhani is rapidly
losing influence, analysts say. He now finds himself faced with an
American travel ban for his citizens and an American president who
thinks the nuclear pact ``is a really, really bad deal.''

Hard-liners are deeply critical of Mr. Rouhani and are increasingly
dismissing him as a figure of the past, a man who may have been the
right answer in the Obama era but is the wrong one now. Many expect the
next president to be a far more combative figure, in the mold of the
former president, Mahmoud Ahmadinejad.

``For this Trump, we need to talk and act tough,'' said Hamidreza
Taraghi, a hard-line analyst. ``Mr. Rouhani speaks beautiful words, but
they are empty. We can deal with Mr. Trump. He is a businessman, but we
should not compromise.''

Advertisement

\protect\hyperlink{after-bottom}{Continue reading the main story}

\hypertarget{site-index}{%
\subsection{Site Index}\label{site-index}}

\hypertarget{site-information-navigation}{%
\subsection{Site Information
Navigation}\label{site-information-navigation}}

\begin{itemize}
\tightlist
\item
  \href{https://help.nytimes3xbfgragh.onion/hc/en-us/articles/115014792127-Copyright-notice}{©~2020~The
  New York Times Company}
\end{itemize}

\begin{itemize}
\tightlist
\item
  \href{https://www.nytco.com/}{NYTCo}
\item
  \href{https://help.nytimes3xbfgragh.onion/hc/en-us/articles/115015385887-Contact-Us}{Contact
  Us}
\item
  \href{https://www.nytco.com/careers/}{Work with us}
\item
  \href{https://nytmediakit.com/}{Advertise}
\item
  \href{http://www.tbrandstudio.com/}{T Brand Studio}
\item
  \href{https://www.nytimes3xbfgragh.onion/privacy/cookie-policy\#how-do-i-manage-trackers}{Your
  Ad Choices}
\item
  \href{https://www.nytimes3xbfgragh.onion/privacy}{Privacy}
\item
  \href{https://help.nytimes3xbfgragh.onion/hc/en-us/articles/115014893428-Terms-of-service}{Terms
  of Service}
\item
  \href{https://help.nytimes3xbfgragh.onion/hc/en-us/articles/115014893968-Terms-of-sale}{Terms
  of Sale}
\item
  \href{https://spiderbites.nytimes3xbfgragh.onion}{Site Map}
\item
  \href{https://help.nytimes3xbfgragh.onion/hc/en-us}{Help}
\item
  \href{https://www.nytimes3xbfgragh.onion/subscription?campaignId=37WXW}{Subscriptions}
\end{itemize}
