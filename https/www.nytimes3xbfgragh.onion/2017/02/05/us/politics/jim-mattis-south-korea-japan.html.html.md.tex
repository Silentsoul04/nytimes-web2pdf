Sections

SEARCH

\protect\hyperlink{site-content}{Skip to
content}\protect\hyperlink{site-index}{Skip to site index}

\href{https://www.nytimes3xbfgragh.onion/section/politics}{Politics}

\href{https://myaccount.nytimes3xbfgragh.onion/auth/login?response_type=cookie\&client_id=vi}{}

\href{https://www.nytimes3xbfgragh.onion/section/todayspaper}{Today's
Paper}

\href{/section/politics}{Politics}\textbar{}Jim Mattis Seeks to Soothe
Tensions in Japan and South Korea

\url{https://nyti.ms/2kbNagY}

\begin{itemize}
\item
\item
\item
\item
\item
\end{itemize}

Advertisement

\protect\hyperlink{after-top}{Continue reading the main story}

Supported by

\protect\hyperlink{after-sponsor}{Continue reading the main story}

\hypertarget{jim-mattis-seeks-to-soothe-tensions-in-japan-and-south-korea}{%
\section{Jim Mattis Seeks to Soothe Tensions in Japan and South
Korea}\label{jim-mattis-seeks-to-soothe-tensions-in-japan-and-south-korea}}

\includegraphics{https://static01.graylady3jvrrxbe.onion/images/2017/02/06/world/06military/06military-articleInline-v2.jpg?quality=75\&auto=webp\&disable=upscale}

By \href{http://www.nytimes3xbfgragh.onion/by/michael-r-gordon}{Michael
R. Gordon} and
\href{http://www.nytimes3xbfgragh.onion/by/choe-sang-hun}{Choe Sang-Hun}

\begin{itemize}
\item
  Feb. 5, 2017
\item
  \begin{itemize}
  \item
  \item
  \item
  \item
  \item
  \end{itemize}
\end{itemize}

\href{http://cn.nytimes3xbfgragh.onion/asia-pacific/20170206/jim-mattis-south-korea-japan/}{阅读简体中文版}

WASHINGTON --- President Trump relishes referring to his new defense
secretary, Jim Mattis, as ``Mad Dog.'' Mr. Mattis has never liked the
nickname, which he argues is a news media invention. But to the
commander in chief, it signifies a man who knows how to take the fight
to the enemy and win.

During his recent alliance-mending mission in Asia, however, Mr. Mattis
was more loyal friend than attack dog, hailed as a welcome voice of
sober restraint.

``He appeared calm and considerate to allies despite his nickname,''
said Kookmin Ilbo, a South Korean daily newspaper. ``Thoughtful and
diligent'' was the verdict of the Mainichi Shimbun, a major Japanese
newspaper. Even the daily Tokyo Sports weighed in, describing Mr. Mattis
as the ``only sensible person in the Trump administration.''

The mission, the first overseas trip by a top member of the Trump
administration's national security team, was intended to shore up
relations with two essential allies, South Korea and Japan. It required
a tricky balance: standing firm against North Korea's saber rattling and
China's territorial ambitions in the South China Sea --- stances that
come naturally to a defense secretary --- while also reassuring allies
made nervous by Mr. Trump's campaign talk about pulling back from
American security commitments in Asia.

The task was all the more challenging given the administration's mixed
signals on foreign policy and Mr. Mattis's testimony to Congress that he
was reluctant to repeat the Obama administration's language about
``rebalancing'' or pivoting to Asia, because it implied that the United
States was turning away from its defense obligations elsewhere.

South Korea, which remains technically at war with the nuclear-armed
North Korea, cherishes its strong alliance with the United States. But
in a nation still scarred by the Korean War, many are concerned that a
hawkish American administration might escalate tensions with the North.
They fear that could disrupt their export-driven economy and even lead
to an armed conflict with the North under its unpredictable young
leader, Kim Jong-un.

During his two-day visit to Seoul, which was his first stop, Mr. Mattis
\href{https://www.nytimes3xbfgragh.onion/2017/02/02/world/asia/james-mattis-us-korea-thaad.html}{pushed
to deploy} an antimissile system known as Thaad, short for Terminal High
Altitude Area Defense, which would be used to intercept North Korea's
medium-range missiles. In a stark warning to Pyongyang, he said that any
use of nuclear weapons by North Korea would be met with an
``overwhelming'' response.

But Mr. Mattis also sought to remind South Koreans of the United States'
past sacrifices for their country and its commitment to their defense.

During a meeting with his South Korean counterpart, Defense Minister Han
Min-koo, Mr. Mattis recalled how he had come to South Korea for training
in the 1970s when he was a young Marine lieutenant based in Okinawa,
Japan. He fondly remembered a Sergeant Chung, a South Korean marine who
shared some kimchi with him. He also noted that he had commanded the
First Marine Division, which had fought in 1950 in the Chosin Reservoir
battle of the Korean War.

Such comments clearly resonated with Mr. Han and top Defense Ministry
officials, all of whom are retired or serving military officers and who
have had close interactions with American troops. Mr. Mattis is a
retired four-star general.

```Mad Dog' Mattis in South Korea was unexpectedly soft,'' read a
headline in OhMyNews, a widely read online newspaper. It observed that
Mr. Mattis, despite his nickname, was considered the most prudent among
Trump administration officials when it came to military action because
he had seen what war was like.

In Japan, Mr. Mattis sought to carry out a similar balancing act. During
the campaign, Mr. Trump threatened to walk away from the mutual defense
pact unless the Japanese did more to reimburse the United States more
for defending their territory.

But speaking at a joint news conference with his Japanese counterpart on
Saturday, Mr. Mattis said that the United States
\href{https://www.nytimes3xbfgragh.onion/2017/02/03/world/asia/us-japan-mattis-abe-defense.html}{stood
by the pact}, reiterating that the American defense commitment extended
to disputed islands in the East China Sea, known in Japan as the Senkaku
and in China as the Diaoyu. Mr. Mattis also described Japan as ``a model
of cost sharing'' and praised the administration of Prime Minister
Shinzo Abe for increasing spending on the military.

Strikingly, Mr. Mattis expressed caution about using military force. In
contrast, Rex W. Tillerson, Mr. Trump's new secretary of state,
suggested during his confirmation hearing that the United States should
be prepared to block China's access to the islands that it has claimed
in the South China Sea and built up with airfields, ports and weapons.

``We're going to have to send China a clear signal that first, the
island-building stops, and second, your access to those islands is also
not going to be allowed,'' Mr. Tillerson told the Senate Foreign
Relations Committee last month.

Mr. Mattis has long argued that diplomacy should be backed up by
military might, but that force should not be the first recourse. In the
case of the South China Sea, he said, it is the diplomats who should be
carrying the ball.

``There is no need right now at this time for military maneuvers or
something like that,'' said Mr. Mattis, who described the dispute as
``something that's best solved by the diplomats.''

With the Trump administration in flux, and the potential for surprises
from North Korea and China, it seems likely that there will be fresh
challenges. But for now, Mr. Mattis appears to have succeeded in his
reassurance mission in Seoul and Tokyo.

``Words matter enormously over there,'' said Michael O'Hanlon, a
military expert at the Brookings Institution. ``Not only did Mattis say
all the right things on issues ranging from Thaad to the Senkaku/Diaoyu
to the strength of alliances to the need for a firm but steady and
nondramatic U.S. approach to the South China Sea, he also went with a
listening ear and little bravado. Things are definitely better, at least
for the moment.''

Advertisement

\protect\hyperlink{after-bottom}{Continue reading the main story}

\hypertarget{site-index}{%
\subsection{Site Index}\label{site-index}}

\hypertarget{site-information-navigation}{%
\subsection{Site Information
Navigation}\label{site-information-navigation}}

\begin{itemize}
\tightlist
\item
  \href{https://help.nytimes3xbfgragh.onion/hc/en-us/articles/115014792127-Copyright-notice}{©~2020~The
  New York Times Company}
\end{itemize}

\begin{itemize}
\tightlist
\item
  \href{https://www.nytco.com/}{NYTCo}
\item
  \href{https://help.nytimes3xbfgragh.onion/hc/en-us/articles/115015385887-Contact-Us}{Contact
  Us}
\item
  \href{https://www.nytco.com/careers/}{Work with us}
\item
  \href{https://nytmediakit.com/}{Advertise}
\item
  \href{http://www.tbrandstudio.com/}{T Brand Studio}
\item
  \href{https://www.nytimes3xbfgragh.onion/privacy/cookie-policy\#how-do-i-manage-trackers}{Your
  Ad Choices}
\item
  \href{https://www.nytimes3xbfgragh.onion/privacy}{Privacy}
\item
  \href{https://help.nytimes3xbfgragh.onion/hc/en-us/articles/115014893428-Terms-of-service}{Terms
  of Service}
\item
  \href{https://help.nytimes3xbfgragh.onion/hc/en-us/articles/115014893968-Terms-of-sale}{Terms
  of Sale}
\item
  \href{https://spiderbites.nytimes3xbfgragh.onion}{Site Map}
\item
  \href{https://help.nytimes3xbfgragh.onion/hc/en-us}{Help}
\item
  \href{https://www.nytimes3xbfgragh.onion/subscription?campaignId=37WXW}{Subscriptions}
\end{itemize}
