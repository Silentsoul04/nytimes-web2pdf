Sections

SEARCH

\protect\hyperlink{site-content}{Skip to
content}\protect\hyperlink{site-index}{Skip to site index}

\href{https://www.nytimes3xbfgragh.onion/section/world/asia}{Asia
Pacific}

\href{https://myaccount.nytimes3xbfgragh.onion/auth/login?response_type=cookie\&client_id=vi}{}

\href{https://www.nytimes3xbfgragh.onion/section/todayspaper}{Today's
Paper}

\href{/section/world/asia}{Asia Pacific}\textbar{}Kim Jong-un's Half
Brother Is Reported Assassinated in Malaysia

\url{https://nyti.ms/2lf2oFg}

\begin{itemize}
\item
\item
\item
\item
\item
\end{itemize}

Advertisement

\protect\hyperlink{after-top}{Continue reading the main story}

Supported by

\protect\hyperlink{after-sponsor}{Continue reading the main story}

\hypertarget{kim-jong-uns-half-brother-is-reported-assassinated-in-malaysia}{%
\section{Kim Jong-un's Half Brother Is Reported Assassinated in
Malaysia}\label{kim-jong-uns-half-brother-is-reported-assassinated-in-malaysia}}

By \href{http://www.nytimes3xbfgragh.onion/by/choe-sang-hun}{Choe
Sang-Hun} and
\href{https://www.nytimes3xbfgragh.onion/by/rick-gladstone}{Rick
Gladstone}

\begin{itemize}
\item
  Feb. 14, 2017
\item
  \begin{itemize}
  \item
  \item
  \item
  \item
  \item
  \end{itemize}
\end{itemize}

\includegraphics{https://static01.graylady3jvrrxbe.onion/images/2017/02/15/world/15Kim/15Kim-articleLarge.jpg?quality=75\&auto=webp\&disable=upscale}

SEOUL, South Korea --- He seemed like an ordinary passenger in the
departure hall of the airport for Malaysia's capital, awaiting a
four-hour flight to Macau. Moments later, he felt dizzy and was carried
out on a stretcher, apparently dying from poisoned-needle punctures or
perhaps a toxic liquid splashed on his face by two women who ran away.

The ruckus caused by the man's death on Monday at the international
airport for Kuala Lumpur was minor news until a thunderbolt from the
South Korean and Malaysian news media a day later: The victim was Kim
Jong-nam, 45, the estranged older half brother of
\href{https://www.nytimes3xbfgragh.onion/2017/01/25/world/asia/north-korea-defector.html}{Kim
Jong-un}, the unpredictable and ruthless leader of
\href{https://www.nytimes3xbfgragh.onion/topic/destination/north-korea?8qa}{North
Korea}.

The death immediately turned into an international assassination
intrigue connected to the opaque regime of the Kim family, which has
ruled North Korea for more than 60 years.

It came as Kim Jong-un, 33, who has ordered scores of subordinates
executed when he questioned their fealty, has further shaken up the
ranks of his closest aides,
\href{https://www.nytimes3xbfgragh.onion/2017/02/03/world/asia/north-korea-purge-kim-jong-un-kim-won-hong.html?_r=0}{purging
the chief of the secret police}less than two weeks ago. In addition, Kim
Jong-un has stoked a new international crisis with a
\href{https://www.nytimes3xbfgragh.onion/2017/02/13/world/asia/north-korea-missile-launch-success.html}{ballistic
missile launching} and threats of more nuclear weapons tests.

The South Korean news channel TV Chosun said that two women had stabbed
Kim Jong-nam with poisoned needles and fled in a taxi and that the local
police were searching for them.
\href{http://www.thestar.com.my/news/nation/2017/02/14/police-confirm-kim-jong-nam-killed-at-klia/}{The
Star, a Malaysian newspaper}, quoted the police as saying the victim had
sought help from a departure hall receptionist after someone ``grabbed
him from behind and splashed liquid on his face.''

He died as medics rushed him to a hospital.

Political experts on North Korea's politics immediately speculated that
Kim Jong-un had ordered the assassination of his older half sibling, who
at one time had been the heir apparent and had been favored by China,
the country's ally and principal benefactor.

``Maybe Kim Jong-nam was about to do something drastic that would either
compromise the regime or the family,'' said Jae H. Ku, director of the
U.S.-Korea Institute at the Johns Hopkins School of Advanced
International Studies. ``By the nature of things in North Korea, the
fact that he is in the bloodline represented a threat.''

Others were even more emphatic in their suspicion that Kim Jong-un had
been responsible, partly because Kim Jong-nam had been publicly critical
of the transfer of power that made Kim Jong-un the top leader after the
death of their father, Kim Jong-il, in 2011.

``The apparent murder today of Kim Jong-nam in Malaysia by agents of his
brother is the latest explosive turn in Pyongyang's vicious palace
intrigue,'' said Nicholas Eberstadt, a political economist who
specializes in North and South Korea at the American Enterprise
Institute in Washington. ``The question remains: Do these deadly
measures secure his rule or serve to undermine it?''

There also was speculation that Kim Jong-un might have ordered Kim
Jong-nam killed because China might have been planning to support him as
a replacement for Kim Jong-un, who has angered Chinese leaders with his
provocative weapons and missile tests.

``Kim Jong-nam reportedly has been Beijing's favorite, which may mean
one day the Chinese Communist Party may overthrow Kim Jong-un and
install Kim Jong-nam,'' said Lee Sung-yoon, a North Korea expert at
Tufts University's Fletcher School of Law and Diplomacy.

The Royal Malaysia Police identified the dead man as Kim Chol, an alias
that South Korean officials said had been used by Kim Jong-nam. A police
statement said the cause of death was under investigation.

In Seoul on Wednesday, Prime Minister Hwang Kyo-ahn, who is serving as
acting president during the impeachment trial of President Park
Geun-hye, called a meeting of security-related cabinet ministers and
urged his government to work closely with the Malaysian authorities to
help uncover who killed Kim Jong-nam.

``If he was killed by the Kim Jong-un regime, it will be an example of
its cruelty and inhumaneness,'' Mr. Hwang said.

North Korea's state-run media has said nothing about the reports.

Kim Jong-nam, the eldest son of
\href{http://www.nytimes3xbfgragh.onion/2011/12/19/world/asia/kim-jong-il-is-dead.html?pagewanted=all}{Kim
Jong-il}, had been widely considered next in line to succeed him until
2001, when he was caught trying to take his son to
\href{http://www.nytimes3xbfgragh.onion/2001/05/04/world/japan-deports-man-said-to-be-north-korean-leader-s-son.html}{Tokyo
Disneyland with a fake visa}. He was detained for several days, then
deported to China.

Other analysts in South Korea say Kim Jong-nam fell out of the
succession race after his mother, Sung Hae-rim, was rejected by the
North Korean leader, who favored Kim Jong-un's mother, Ko Young-hee. Ms.
Ko and Kim Jong-il had another son, Kim Jong-chol, who was seen at an
Eric Clapton concert in London in 2015.

North Korea began grooming Kim Jong-un as heir after his father
\href{http://www.nytimes3xbfgragh.onion/2008/12/12/world/asia/12kim.html}{had
a stroke in 2008}. As his youngest brother consolidated power, Kim
Jong-nam lived in semi-exile abroad. Until recently, he was sometimes
seen in Macau. TV Chosun said he had also been visiting Singapore and
Malaysia, where he had girlfriends.

Kim Jong-nam's son, Kim Han-sol, had once studied in Bosnia and later in
France. In an interview with a European television channel in 2012, the
son said he did not know how his uncle, Kim Jong-un, ``became a
dictator.''

Kim Jong-nam was once questioned in Macau by a reporter about the
likelihood that his half brother would take over, and he seemed to
accept his fate.

``It is my father's decision,'' he said. ``So, once he decides, we have
to support.''

But at other times, he was critical of Kim Jong-un's ascendance.

``I believe that my father originally was against the notion of a third
generation succeeding him,'' Kim Jong-nam, interviewed in November 2010,
told the Japanese journalist Yoji Gomi in the book ``My Father, Kim
Jong-il, and I.'' ``There must have been some internal reasons that made
him change his mind.''

Kim Jong-nam also once predicted doom for his half brother's rule while
talking to reporters from Japan, North Korea's sworn enemy. His
criticism fueled speculation that China and certain generals in
Pyongyang, the North Korean capital, might be protecting him in case
anything should go wrong with Kim Jong-un's rule.

Mr. Gomi said in an interview on Tuesday that the last time he had
contacted Kim Jong-nam, in January 2012, he had said North Korea should
follow China's economic path.

China supported Kim Jong-nam financially for many years because if Kim
Jong-un died, North Koreans, indoctrinated to venerate the Kim family,
would look to Kim Jong-nam to step in as leader, according to Kang
Chunnu, 51, a distant relative of the Kim family who lives in Britain.

``Kim Jong-nam is a person which China can control and the North Korean
people can trust,'' she said by telephone.

There seemed little question in South Korea that Kim Jong-un was behind
his half brother's death.

A spokesman for South Korea's governing Liberty Korea Party, Kim
Myung-yeon, said the killing was a ``naked example of Kim Jong-un's
reign of terror.''

Since taking power, Mr. Kim has executed more than 140 senior party and
military officials deemed a threat to his authority, often ordering them
killed by machine guns and even flamethrowers, according to the
Institute for National Security Strategy, a research group affiliated
with the South's National Intelligence Service.

Thae Yong-ho, who was the North's No. 2 diplomat in London until his
defection to South Korea last summer, said he had fled partly because of
Kim Jong-un's ruthlessness.

In 2015, South Korean officials said that Gen. Hyon Yong-chol, the
defense minister, had been executed with an antiaircraft gun in
Pyongyang after dozing off during military events and second-guessing
Mr. Kim's orders. In August last year, they said Mr. Kim found fault
with a deputy premier's ``disrespectful posture'' during a meeting and
had him executed by firing squad.

Relatives were not spared. An uncle and the country's No. 2 official,
\href{https://www.nytimes3xbfgragh.onion/2016/03/13/world/asia/north-korea-executions-jang-song-thaek.html}{Jang
Song-thaek}, was executed in 2013 on charges of factionalism, corruption
and sedition.

Defectors from North Korea live in fear of retaliation. In 1997, Lee
Han-young, a nephew of Kim Jong-nam's mother, was shot and killed in
Seoul. South Korean officials suspected that a North Korean agent killed
Mr. Lee, who had become a bitter critic of the government in Pyongyang
after defecting to Seoul in 1982.

Cheong Seong-chang, a longtime researcher on the Kim family, said that
the killing of Kim Jong-nam could have been carried out only on the
orders of Kim Jong-un.

Ken E. Gause, a specialist in leadership studies at the CNA Corporation,
a research group in Alexandria, Va., said the assassination also might
have been meant as a warning to all North Korean expatriates.

``Given the recent defections,'' he said, ``Kim Jong-un felt the need to
show that the regime could get to anyone who may be contemplating
opposing the regime.''

Advertisement

\protect\hyperlink{after-bottom}{Continue reading the main story}

\hypertarget{site-index}{%
\subsection{Site Index}\label{site-index}}

\hypertarget{site-information-navigation}{%
\subsection{Site Information
Navigation}\label{site-information-navigation}}

\begin{itemize}
\tightlist
\item
  \href{https://help.nytimes3xbfgragh.onion/hc/en-us/articles/115014792127-Copyright-notice}{©~2020~The
  New York Times Company}
\end{itemize}

\begin{itemize}
\tightlist
\item
  \href{https://www.nytco.com/}{NYTCo}
\item
  \href{https://help.nytimes3xbfgragh.onion/hc/en-us/articles/115015385887-Contact-Us}{Contact
  Us}
\item
  \href{https://www.nytco.com/careers/}{Work with us}
\item
  \href{https://nytmediakit.com/}{Advertise}
\item
  \href{http://www.tbrandstudio.com/}{T Brand Studio}
\item
  \href{https://www.nytimes3xbfgragh.onion/privacy/cookie-policy\#how-do-i-manage-trackers}{Your
  Ad Choices}
\item
  \href{https://www.nytimes3xbfgragh.onion/privacy}{Privacy}
\item
  \href{https://help.nytimes3xbfgragh.onion/hc/en-us/articles/115014893428-Terms-of-service}{Terms
  of Service}
\item
  \href{https://help.nytimes3xbfgragh.onion/hc/en-us/articles/115014893968-Terms-of-sale}{Terms
  of Sale}
\item
  \href{https://spiderbites.nytimes3xbfgragh.onion}{Site Map}
\item
  \href{https://help.nytimes3xbfgragh.onion/hc/en-us}{Help}
\item
  \href{https://www.nytimes3xbfgragh.onion/subscription?campaignId=37WXW}{Subscriptions}
\end{itemize}
