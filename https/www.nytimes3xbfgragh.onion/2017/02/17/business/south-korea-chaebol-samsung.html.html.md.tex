Sections

SEARCH

\protect\hyperlink{site-content}{Skip to
content}\protect\hyperlink{site-index}{Skip to site index}

\href{https://www.nytimes3xbfgragh.onion/section/business}{Business}

\href{https://myaccount.nytimes3xbfgragh.onion/auth/login?response_type=cookie\&client_id=vi}{}

\href{https://www.nytimes3xbfgragh.onion/section/todayspaper}{Today's
Paper}

\href{/section/business}{Business}\textbar{}Money, Power, Family: Inside
South Korea's Chaebol

\url{https://nyti.ms/2lqFAm1}

\begin{itemize}
\item
\item
\item
\item
\item
\end{itemize}

Advertisement

\protect\hyperlink{after-top}{Continue reading the main story}

Supported by

\protect\hyperlink{after-sponsor}{Continue reading the main story}

\hypertarget{money-power-family-inside-south-koreas-chaebol}{%
\section{Money, Power, Family: Inside South Korea's
Chaebol}\label{money-power-family-inside-south-koreas-chaebol}}

\includegraphics{https://static01.graylady3jvrrxbe.onion/images/2017/02/18/world/18chaebol-1/18chaebol-1-articleLarge.jpg?quality=75\&auto=webp\&disable=upscale}

By \href{https://www.nytimes3xbfgragh.onion/by/carlos-tejada}{Carlos
Tejada}

\begin{itemize}
\item
  Feb. 17, 2017
\item
  \begin{itemize}
  \item
  \item
  \item
  \item
  \item
  \end{itemize}
\end{itemize}

HONG KONG --- In South Korea, it's all in the family.

South Korean officials on Friday
\href{https://www.nytimes3xbfgragh.onion/2017/02/16/world/asia/korea-samsung-lee-jae-yong.html}{arrested
Lee Jae-yong}, the vice chairman of Samsung and the scion of an
immensely wealthy and powerful family. Mr. Lee --- known as Jay Y. Lee
in the West --- was accused of paying bribes to a confidante of South
Korea's embattled president. Samsung has defended Mr. Lee, and it said
on Friday that it would work to ensure ``that the truth is revealed.''

Mr. Lee has been widely considered Samsung's de facto leader since his
father, Lee Kun-hee, Samsung's 75-year-old chairman, had a heart attack
in 2014. That put the younger Mr. Lee at the helm of a sprawling
business empire that encompasses gadgets, appliances, engineering,
construction, shipbuilding, insurance and credit cards. Samsung
Electronics --- the maker of televisions and smartphones used around the
world, including the
\href{https://www.nytimes3xbfgragh.onion/2017/01/23/business/samsung-galaxy-note7-fires.html}{doomed
Galaxy Note 7 smartphone} --- by itself accounts for one-fifth of South
Korea's exports.

Samsung is only one of a handful of family-controlled companies, called
chaebol, that dominate economic life in South Korea. Some, like Hyundai,
LG and Samsung, are well known outside their home country. But
domestically, they all wield immense power --- and are
\href{https://www.nytimes3xbfgragh.onion/2017/02/17/business/samsung-heir-arrested-south-korea.html}{coming
under increasing scrutiny}.

\hypertarget{what-are-chaebol}{%
\subsection{What are chaebol?}\label{what-are-chaebol}}

The word comes from the combination of the characters for ``rich'' and
``clan.'' It applies to large groups of interconnected companies that
are usually dominated by a wealthy family. South Korea has several, but
the best known outside the country are Hyundai, LG and Samsung. Others
include Hanjin, Kumho, Lotte and SK Group.

Chaebol are generally conglomerates of affiliated companies. LG, for
example, makes smartphones, televisions, electronic components,
chemicals and fertilizer. It also owns Korean baseball and basketball
teams. Hyundai, which makes the Hyundai and Kia cars that are popular in
the United States and other countries, also makes elevators, provides
logistics services, and runs hotels and department stores.

\hypertarget{how-did-chaebol-come-to-power}{%
\subsection{How did chaebol come to
power?}\label{how-did-chaebol-come-to-power}}

They rose from the ashes of the Korean War. After the conflict ended,
officials steered relief funds and cheap loans to businessmen who
promised to rebuild the country. The government also protected homegrown
industries from foreign competition to help them develop. The recipe
proved to be potent: Chaebol played a major role in South Korea's rise
as an industrial giant in the following decades.

But the recipe also created imbalances, a number of economists
\href{http://econpapers.repec.org/bookchap/iieppress/25.htm}{have
argued}. Money meant for the common people often ended up in the hands
of the wealthy families, creating resentment that lingers to this day.
And government protection and crackdowns on the labor movement allowed
these families to expand their business empires into new areas with
little to fear from potential foreign competition or costly failures.

As a result, chaebol became sprawling businesses that held a
\href{https://www.google.com.hk/url?sa=t\&rct=j\&q=\&esrc=s\&source=web\&cd=1\&cad=rja\&uact=8\&ved=0ahUKEwjNmM630JbSAhWMmJQKHbuDDKoQFggaMAA\&url=https\%3A\%2F\%2Fdocsonline.wto.org\%2Fdol2fe\%2FPages\%2FFE_Search\%2FExportFile.aspx\%3FId\%3D35204\%26filename\%3DQ\%2FWT\%2FTPR\%2FS73-4.pdf\&usg=AFQjCNFsqkTSqiCciRt1XyLtVCGPRvv4qQ\&sig2=mnXQ9DuyOCwMnIEAu3da6A}{nearly
two-thirds market share} in South Korean manufacturing by the end of the
1990s, according to the World Trade Organization. But there is a
deep-seated belief among many South Koreans that their immense wealth
was accumulated at the expense of the public. To those people, recurring
chaebol scandals are particularly galling.

\hypertarget{how-did-they-get-political-power}{%
\subsection{How did they get political
power?}\label{how-did-they-get-political-power}}

South Korea's recipe for growth also fostered tight ties between the
government and businesses.

Take the example of Park Chung-hee, a general and the father of the
current president, who took power in South Korea after a 1961 coup. He
led an effort to rev up the South Korean economy --- and he used many of
the companies that became chaebol to do it. His government steered money
to companies that chased his economic goals, such as emphasizing
exports.

The dynamic shifted somewhat as South Korea transitioned to a democracy
in the 1980s. By then, the chaebol had become so economically powerful
that they held considerable political sway. Politicians began to rely on
the companies' political and financial support to get elected.

\hypertarget{are-chaebol-under-threat}{%
\subsection{Are chaebol under threat?}\label{are-chaebol-under-threat}}

They are. But time will tell whether it will result in change.

Public support of chaebol has gradually waned. The Asian financial
crisis of the late 1990s stirred worries that the cozy relationship
between chaebol member companies could lead to severe damage across
multiple businesses if one failed. As the economy has matured and
created a nation of consumers, an increasing number of South Koreans
worry about the political power and corruption of the chaebol, with many
now saying white-collar crime
\href{https://www.nytimes3xbfgragh.onion/2016/07/05/business/dealbook/south-korea-targets-executives-pressed-by-an-angry-public.html}{is
a major issue}.

Despite those concerns, chaebol executives are widely believed to be
treated with kid gloves. The elder Mr. Lee, Samsung's chairman, has been
pardoned twice after being convicted of white-collar crimes, with the
potential impact to South Korea's economy given as the reason.

Advertisement

\protect\hyperlink{after-bottom}{Continue reading the main story}

\hypertarget{site-index}{%
\subsection{Site Index}\label{site-index}}

\hypertarget{site-information-navigation}{%
\subsection{Site Information
Navigation}\label{site-information-navigation}}

\begin{itemize}
\tightlist
\item
  \href{https://help.nytimes3xbfgragh.onion/hc/en-us/articles/115014792127-Copyright-notice}{©~2020~The
  New York Times Company}
\end{itemize}

\begin{itemize}
\tightlist
\item
  \href{https://www.nytco.com/}{NYTCo}
\item
  \href{https://help.nytimes3xbfgragh.onion/hc/en-us/articles/115015385887-Contact-Us}{Contact
  Us}
\item
  \href{https://www.nytco.com/careers/}{Work with us}
\item
  \href{https://nytmediakit.com/}{Advertise}
\item
  \href{http://www.tbrandstudio.com/}{T Brand Studio}
\item
  \href{https://www.nytimes3xbfgragh.onion/privacy/cookie-policy\#how-do-i-manage-trackers}{Your
  Ad Choices}
\item
  \href{https://www.nytimes3xbfgragh.onion/privacy}{Privacy}
\item
  \href{https://help.nytimes3xbfgragh.onion/hc/en-us/articles/115014893428-Terms-of-service}{Terms
  of Service}
\item
  \href{https://help.nytimes3xbfgragh.onion/hc/en-us/articles/115014893968-Terms-of-sale}{Terms
  of Sale}
\item
  \href{https://spiderbites.nytimes3xbfgragh.onion}{Site Map}
\item
  \href{https://help.nytimes3xbfgragh.onion/hc/en-us}{Help}
\item
  \href{https://www.nytimes3xbfgragh.onion/subscription?campaignId=37WXW}{Subscriptions}
\end{itemize}
