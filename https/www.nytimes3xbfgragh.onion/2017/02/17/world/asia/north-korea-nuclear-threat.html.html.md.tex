Sections

SEARCH

\protect\hyperlink{site-content}{Skip to
content}\protect\hyperlink{site-index}{Skip to site index}

\href{https://www.nytimes3xbfgragh.onion/section/world/asia}{Asia
Pacific}

\href{https://myaccount.nytimes3xbfgragh.onion/auth/login?response_type=cookie\&client_id=vi}{}

\href{https://www.nytimes3xbfgragh.onion/section/todayspaper}{Today's
Paper}

\href{/section/world/asia}{Asia Pacific}\textbar{}How the North Korean
Nuclear Threat Has Grown

\url{https://nyti.ms/2lsw4yT}

\begin{itemize}
\item
\item
\item
\item
\item
\end{itemize}

Advertisement

\protect\hyperlink{after-top}{Continue reading the main story}

Supported by

\protect\hyperlink{after-sponsor}{Continue reading the main story}

\hypertarget{how-the-north-korean-nuclear-threat-has-grown}{%
\section{How the North Korean Nuclear Threat Has
Grown}\label{how-the-north-korean-nuclear-threat-has-grown}}

Feb. 17, 2017

By \href{https://www.nytimes3xbfgragh.onion/by/rick-gladstone}{Rick
Gladstone} and Rogene Jacquette

When North Korea
\href{https://www.nytimes3xbfgragh.onion/2017/02/11/world/asia/north-korea-missile-test-trump.html}{tested
a missile}that fell harmlessly into the sea this month, it was more than
just an attempt by its 33-year-old leader,
\href{https://www.nytimes3xbfgragh.onion/topic/person/kim-jongun}{Kim
Jong-un}, to jolt a new American president. Arms experts observed
something new: solid-fuel technology that makes such missiles easier to
hide and launch quickly. North Korea's nuclear weapons program has
progressed in four areas that bear watching: arsenal size, bomb
strength, missile technology and ability to elude detection.

\hypertarget{arsenal-size-small-but-thought-to-be-growing}{%
\subsubsection{\texorpdfstring{\textbf{Arsenal size: small, but thought
to be
growing}}{Arsenal size: small, but thought to be growing}}\label{arsenal-size-small-but-thought-to-be-growing}}

Knowledge of the weapons stockpile is based on estimates. Experts say
that North Korea has fewer than
\href{https://www.nytimes3xbfgragh.onion/interactive/2016/12/23/world/nuclear-weapon-countries.html}{10
nuclear weapons}. Satellite imagery of North Korea's nuclear complex in
Yongbyon, combined with official North Korean propaganda photos and
recent nuclear tests, suggests that the country could rapidly expand its
arsenal. By one estimate, the country now has enough plutonium and
highly
\href{https://www.nytimes3xbfgragh.onion/2017/01/12/opinion/the-us-must-talk-to-north-korea.html}{enriched
uranium to build 20 to 25 nuclear weapons}.

\includegraphics{https://static01.graylady3jvrrxbe.onion/images/2016/09/10/world/10Korea-2/10Korea-2-articleInline.jpg?quality=75\&auto=webp\&disable=upscale}

\hypertarget{explosive-power-from-one-kiloton-to-10-kilotons-in-10-years}{%
\subsubsection{\texorpdfstring{\textbf{Explosive power: from one kiloton
to 10 kilotons in 10
years}}{Explosive power: from one kiloton to 10 kilotons in 10 years}}\label{explosive-power-from-one-kiloton-to-10-kilotons-in-10-years}}

The explosive force of North Korea's first nuclear
device,~\href{http://www.nytimes3xbfgragh.onion/2006/10/09/world/asia/09korea.html}{tested
in October 2006}, was less than a kiloton, which is 1,000 tons of TNT.
Its second test,
\href{http://www.nytimes3xbfgragh.onion/2009/05/25/world/asia/25nuke.html}{in
2009}, had more than double that force.

By
\href{https://www.nytimes3xbfgragh.onion/2016/01/06/world/asia/north-korea-hydrogen-bomb-test.html}{January
2016}, the country claimed to have exploded a hydrogen bomb in a fourth
test, but outside monitors expressed skepticism. Seismic readings
suggested an explosive force~of~four~to~six kilotons.

Seismic readings of North Korea's fifth test,~in
\href{https://www.nytimes3xbfgragh.onion/2016/09/10/world/asia/north-korea-nuclear-weapons-tests.html}{September
2016}, however, registered a force of approximately 10 kilotons,
according to South Korea's Defense Ministry.

\hypertarget{technology-missiles-could-reach-continental-us-by-2026}{%
\subsubsection{\texorpdfstring{\textbf{Technology: missiles could reach
continental U.S. by
2026}}{Technology: missiles could reach continental U.S. by 2026}}\label{technology-missiles-could-reach-continental-us-by-2026}}

In 1999,
\href{http://www.nytimes3xbfgragh.onion/1999/02/03/world/cia-sees-a-north-korean-missile-threat.html}{George
J. Tenet}, then director of the Central Intelligence Agency, said he
could hardly overstate his concern about North Korea's program, warning
that the Taepodong-1 missile, with a reach of up to 1,243 miles, could
deliver bomb payloads to Alaska and Hawaii.

In the nearly two decades since, the country's investment in becoming a
nuclear weapons power has succeeded despite diplomacy and international
sanctions. In 2016, Mr. Kim launched dozens of missiles for tests and as
shows of military might. Some missiles could be launched from mobile
pads and submarines, making them easier to hide. They could potentially
carry nuclear warheads, according to Siegfried S. Hecker, emeritus
director of the Los Alamos National Laboratory in New Mexico, birthplace
of the atomic bomb.

He and other analysts have said they assume North Korea has designed and
demonstrated nuclear warheads that can be mounted on short-range and
perhaps medium-range missiles. Writing in
\href{http://38north.org/2016/09/shecker091216/}{September 2016}, Dr.
Hecker said, ``Pyongyang will likely develop the capability to reach the
continental United States with a nuclear tipped missile in a decade or
so.''

\hypertarget{covert-capability-smaller-more-mobile-weapons}{%
\subsubsection{\texorpdfstring{\textbf{Covert capability: smaller, more
mobile
weapons}}{Covert capability: smaller, more mobile weapons}}\label{covert-capability-smaller-more-mobile-weapons}}

When he became North Korea's top leader in April 2012,
\href{http://topics.nytimes3xbfgragh.onion/top/reference/timestopics/people/k/kim_jongun/index.html?inline=nyt-per}{Mr.
Kim} said that his ``first, second and third'' priorities were to
strengthen the military, and he declared that superiority in military
technology was ``no longer monopolized by imperialists.'' Less than
three years later, Gen. Curtis M. Scaparrotti, then commander of United
States forces in South Korea, said he believed that North Korea had
\href{https://www.nytimes3xbfgragh.onion/2014/10/25/world/asia/us-commander-sees-key-nuclear-step-by-north-korea.html}{made
a nuclear weapon small enough to fit atop a missile.}

In May 2015, Mr. Kim said North Korea
\href{https://www.nytimes3xbfgragh.onion/2015/05/21/world/asia/north-korea-claims-it-has-built-small-nuclear-warheads.html}{had
the ability to miniaturize nuclear weapons}. That claim was greeted with
skepticism by analysts, but in March 2016 Mr. Kim
\href{http://www.bbc.com/news/world-asia-35760797}{was
photographed}admiring what state media described as a home-built
warhead. In
\href{https://www.nytimes3xbfgragh.onion/2016/08/24/world/asia/north-korea-submarine-missile.html}{August
2016} North Korea test-fired a ballistic missile from a submarine,
demonstrating a significant improvement in its ability to strike enemies
stealthily.

The missile test this month, analysts said, further proved that North
Korea was committed to producing more lethal systems that could be
deployed quickly. ~~``The North Koreans are sincerely paranoid,'' said
Joshua Pollack, a senior research associate at the Middlebury
Institute's James Martin Center for Nonproliferation Studies. ``They're
increasingly very blunt about how they would use these things
preemptively.''

Advertisement

\protect\hyperlink{after-bottom}{Continue reading the main story}

\hypertarget{site-index}{%
\subsection{Site Index}\label{site-index}}

\hypertarget{site-information-navigation}{%
\subsection{Site Information
Navigation}\label{site-information-navigation}}

\begin{itemize}
\tightlist
\item
  \href{https://help.nytimes3xbfgragh.onion/hc/en-us/articles/115014792127-Copyright-notice}{©~2020~The
  New York Times Company}
\end{itemize}

\begin{itemize}
\tightlist
\item
  \href{https://www.nytco.com/}{NYTCo}
\item
  \href{https://help.nytimes3xbfgragh.onion/hc/en-us/articles/115015385887-Contact-Us}{Contact
  Us}
\item
  \href{https://www.nytco.com/careers/}{Work with us}
\item
  \href{https://nytmediakit.com/}{Advertise}
\item
  \href{http://www.tbrandstudio.com/}{T Brand Studio}
\item
  \href{https://www.nytimes3xbfgragh.onion/privacy/cookie-policy\#how-do-i-manage-trackers}{Your
  Ad Choices}
\item
  \href{https://www.nytimes3xbfgragh.onion/privacy}{Privacy}
\item
  \href{https://help.nytimes3xbfgragh.onion/hc/en-us/articles/115014893428-Terms-of-service}{Terms
  of Service}
\item
  \href{https://help.nytimes3xbfgragh.onion/hc/en-us/articles/115014893968-Terms-of-sale}{Terms
  of Sale}
\item
  \href{https://spiderbites.nytimes3xbfgragh.onion}{Site Map}
\item
  \href{https://help.nytimes3xbfgragh.onion/hc/en-us}{Help}
\item
  \href{https://www.nytimes3xbfgragh.onion/subscription?campaignId=37WXW}{Subscriptions}
\end{itemize}
