Sections

SEARCH

\protect\hyperlink{site-content}{Skip to
content}\protect\hyperlink{site-index}{Skip to site index}

\href{https://www.nytimes3xbfgragh.onion/section/business}{Business}

\href{https://myaccount.nytimes3xbfgragh.onion/auth/login?response_type=cookie\&client_id=vi}{}

\href{https://www.nytimes3xbfgragh.onion/section/todayspaper}{Today's
Paper}

\href{/section/business}{Business}\textbar{}With No Frills and No
Commissions, Robinhood App Takes On Big Brokerages

\url{https://nyti.ms/2ltSe3E}

\begin{itemize}
\item
\item
\item
\item
\item
\end{itemize}

Advertisement

\protect\hyperlink{after-top}{Continue reading the main story}

Supported by

\protect\hyperlink{after-sponsor}{Continue reading the main story}

\hypertarget{with-no-frills-and-no-commissions-robinhood-app-takes-on-big-brokerages}{%
\section{With No Frills and No Commissions, Robinhood App Takes On Big
Brokerages}\label{with-no-frills-and-no-commissions-robinhood-app-takes-on-big-brokerages}}

\includegraphics{https://static01.graylady3jvrrxbe.onion/images/2017/02/19/business/19ROBINHOOD1/19ROBINHOOD1-articleLarge.jpg?quality=75\&auto=webp\&disable=upscale}

By Janet Morrissey

\begin{itemize}
\item
  Feb. 18, 2017
\item
  \begin{itemize}
  \item
  \item
  \item
  \item
  \item
  \end{itemize}
\end{itemize}

A start-up called \href{https://www.robinhood.com/company/}{Robinhood
Markets} is taking on the big brokerage firms with its commission-free
trading app, and appears to be making headway. Since its introduction in
December 2014, the app has attracted a million users and executed more
than \$30 billion in trades, up from \$2 billion in 2015.

Despite the app's hype and surging popularity, some industry experts
question if the free-trades business model can survive, or if it will
wind up joining other start-ups that have crashed and burned. The
company currently makes money primarily from interest on customer cash
balances.

At
\href{https://www.nytimes3xbfgragh.onion/2020/07/08/technology/robinhood-risky-trading.html}{Robinhood},
there is no minimum deposit to register an account, and there are no
trading fees for customers who buy and sell United States-listed stocks
and exchange-traded funds. To keep costs down, the company, in Palo
Alto, Calif., takes a no-frills approach. It has no storefront offices.
It does not provide research reports, analytical tools, stock screening
gizmos or options trading on its platform.

``Uber opened our eyes that if you could hail a car off your phone and
watch movies, why can't I trade 10 shares of Facebook or a thousand
shares?'' said Howard Lindzon, a general partner of the Social Leverage
Fund and a founder of StockTwits, a financial communications platform.
His fund holds an equity stake in Robinhood.

Chase Kaye, a 21-year-old Manhattan resident, signed up for Robinhood
over a year ago after he grew tired of paying the \$7 commission on
trades at the Vanguard Group.

``The interface on the phone app is way better,'' said Mr. Kaye, a
marketing management student at Baruch College. ``It's simple, not as
intimidating and very intuitive.'' Today about 70 percent of his assets
are with Robinhood, Mr. Kaye said.

Still, nothing is truly free in the world of finance.

``The illusion of being free encourages frequent trading, which is well
known to be a cause of lousy investment returns,'' said Tad Borek, an
investment adviser at \href{https://www.borekfinancial.com/}{Borek
Financial Management}. ``Is this just handing some free chips to
gamblers?''

Even the name conjures up images of a firm helping the poor, when it may
actually be doing the opposite, he contends.

Robinhood's founders dispute these concerns. Their goal is to make
investing accessible to everyone, not just the superwealthy, which is
why the name Robinhood was chosen, said Vladimir Tenev, 30, a founder.

``We have active investors who are moving their trading over from other
brokerages where they have paid thousands to tens of thousands of
dollars in trading commissions,'' Mr. Tenev said. ``We also have younger
customers who are accumulating stocks and building mini diversified
portfolios over time.''

A customer could buy one share in a stock or ETF today and five more
shares next month to slowly build a portfolio. The fees at larger firms
would make such a strategy cost prohibitive. Big brokerage firms like TD
Ameritrade, Schwab, Fidelity and E-Trade typically charge \$7 to \$10 a
trade.

Robinhood's bare-bones business model has led to complaints about its
customer service, with long waiting times by phone and emails that are
not answered. Internet blogs are also filled with complaints about
lengthy waits --- up to eight days --- to transfer money from a
Robinhood account to a bank or vice versa. Mr. Tenev said the company
was looking to solve these problems.

And some customers are frustrated that the company doesn't offer options
trading, a website trading platform or automated transfers from other
brokerage accounts. Mr. Tenev said that would change. ``If you look down
several years, we'll offer all of these things,'' he said.

Robinhood is the brainchild of Mr. Tenev and Baiju Prafulkumar Bhatt,
Stanford graduates who shared a passion for math and physics. In 2009
they moved to New York, where they began Celeris and Chronos Research,
which made high-frequency trading software for hedge funds and banks.

Inspired by the Occupy Wall Street movement in 2011, they rethought
their priorities. ``The financial services industry should serve all
people, regardless of net worth,'' Mr. Tenev said. In 2012, they moved
back to California, hired engineers and spent the next 18 months
building Robinhood.

\includegraphics{https://static01.graylady3jvrrxbe.onion/images/2017/02/19/business/19ROBINHOOD2/19ROBINHOOD2-articleLarge.jpg?quality=75\&auto=webp\&disable=upscale}

In late 2013, Robinhood set up a beta version and invited people to test
the app. When the sign-up page was posted on Hacker News and Reddit, the
site was flooded with requests.

``We had almost 50,000 people that signed up that first weekend,'' Mr.
Bhatt, 32, recalled. By the time the app began in the App Store in late
2014, it had a wait-list of a million people, most under the age of 30.

The company has raised \$66 million in start-up cash from such
high-flying investors as Andreessen Horowitz, GV, Index Ventures and New
Enterprise Associates, as well as entertainers like Jared Leto, Snoop
Dogg and Nas.

In addition to cash balances, the company makes money from ``payment for
order flow,'' which refers to the money it receives for selling its
orders to market makers to be executed.

``These revenue streams alone are not enough to sustain an online broker
providing \$0 trades --- not even close,'' said Blain Reinkensmeyer, the
head of broker research at StockBrokers.com.

Recently, Robinhood introduced a Gold program, which offers margin
accounts and after-hours trading, for a fee.

``Robinhood Gold is where we anticipate the bulk of the revenues to come
from in the future,'' Mr. Tenev said. Robinhood charges Gold accounts a
fixed monthly fee that works out to 5 to 6 percent interest on margin
accounts, whether the customer uses the credit line or not. At larger
firms, like Schwab, customers are charged interest only on the margin
amount that's used.

``I got the email to sign up for a free month trial of their new Gold
account, but it's not something I'm interested in,'' Mr. Kaye said.
``I'll probably stay with Vanguard for margin accounts.''

Another firm, Zecco, attempted a similar ``free trades'' model in 2006.
Despite raising more than \$35 million in financing, the company
struggled to get off the ground, and eventually merged with TradeKing in
2012. It now charges \$4.95 a trade.

``Zecco couldn't find a path to profitability without charging for
commissions and evolving into a traditional online broker,'' Mr.
Reinkensmeyer said. ``It's amazing how quickly history is forgotten.''

Mr. Lindzon, the Robinhood equity investor, noted that Zecco operated
when smartphone use was in its infancy.

``With a couple billion people now on smartphones, it's a very different
market today,'' said Aaron Levie, a founder of the cloud company Box,
and another of Robinhood's equity investors.

Other start-ups have entered the territory to challenge Robinhood. They
include Ustocktrade, which charges a \$1-a-month membership fee and
offers \$1 trades, and Loyal3, which offers no-fee trades on a limited
number of stocks, partial shares and initial public offerings. There's
also Bank of America's Merrill Edge, which offers 30 free stock and ETF
trades each month for customers with a balance of at least \$50,000 and
100 free trades a month on balances of at least \$100,000.

So far, the big brokerage firms do not appear concerned about Robinhood.
Kim Hillyer, a spokesman for TDAmeritrade, said investors value a
full-service firm that offers free education, webinars, research tools
and investment consultants in addition to a full trading platform that
includes futures and options. She noted that TDAmeritrade accounts have
continued to grow since Robinhood's introduction, rising to seven
million in fiscal 2016 from 6.3 million in 2014. ``The growth in new
accounts is a validation for the value proposition that we have,'' she
said.

Erin Montgomery, a spokeswoman for Schwab, agreed. ``We know that
investors and traders alike are looking for value from their brokerage
firms, and our clients find that here,'' she said. The firm's investment
services and products include financial planning and research as well as
a full trading platform and commission-free ETFs.

Still, the two firms said they are always looking for ways to stay
competitive, and sometimes offer low-fee and no-fee trading promotions.

Advertisement

\protect\hyperlink{after-bottom}{Continue reading the main story}

\hypertarget{site-index}{%
\subsection{Site Index}\label{site-index}}

\hypertarget{site-information-navigation}{%
\subsection{Site Information
Navigation}\label{site-information-navigation}}

\begin{itemize}
\tightlist
\item
  \href{https://help.nytimes3xbfgragh.onion/hc/en-us/articles/115014792127-Copyright-notice}{©~2020~The
  New York Times Company}
\end{itemize}

\begin{itemize}
\tightlist
\item
  \href{https://www.nytco.com/}{NYTCo}
\item
  \href{https://help.nytimes3xbfgragh.onion/hc/en-us/articles/115015385887-Contact-Us}{Contact
  Us}
\item
  \href{https://www.nytco.com/careers/}{Work with us}
\item
  \href{https://nytmediakit.com/}{Advertise}
\item
  \href{http://www.tbrandstudio.com/}{T Brand Studio}
\item
  \href{https://www.nytimes3xbfgragh.onion/privacy/cookie-policy\#how-do-i-manage-trackers}{Your
  Ad Choices}
\item
  \href{https://www.nytimes3xbfgragh.onion/privacy}{Privacy}
\item
  \href{https://help.nytimes3xbfgragh.onion/hc/en-us/articles/115014893428-Terms-of-service}{Terms
  of Service}
\item
  \href{https://help.nytimes3xbfgragh.onion/hc/en-us/articles/115014893968-Terms-of-sale}{Terms
  of Sale}
\item
  \href{https://spiderbites.nytimes3xbfgragh.onion}{Site Map}
\item
  \href{https://help.nytimes3xbfgragh.onion/hc/en-us}{Help}
\item
  \href{https://www.nytimes3xbfgragh.onion/subscription?campaignId=37WXW}{Subscriptions}
\end{itemize}
