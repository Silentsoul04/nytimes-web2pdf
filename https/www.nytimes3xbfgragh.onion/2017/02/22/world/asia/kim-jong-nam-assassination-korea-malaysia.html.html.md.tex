Sections

SEARCH

\protect\hyperlink{site-content}{Skip to
content}\protect\hyperlink{site-index}{Skip to site index}

\href{https://www.nytimes3xbfgragh.onion/section/world/asia}{Asia
Pacific}

\href{https://myaccount.nytimes3xbfgragh.onion/auth/login?response_type=cookie\&client_id=vi}{}

\href{https://www.nytimes3xbfgragh.onion/section/todayspaper}{Today's
Paper}

\href{/section/world/asia}{Asia Pacific}\textbar{}Kim Jong-nam's Death:
A Geopolitical Whodunit

\url{https://nyti.ms/2m93uDV}

\begin{itemize}
\item
\item
\item
\item
\item
\end{itemize}

Advertisement

\protect\hyperlink{after-top}{Continue reading the main story}

Supported by

\protect\hyperlink{after-sponsor}{Continue reading the main story}

\hypertarget{kim-jong-nams-death-a-geopolitical-whodunit}{%
\section{Kim Jong-nam's Death: A Geopolitical
Whodunit}\label{kim-jong-nams-death-a-geopolitical-whodunit}}

\includegraphics{https://static01.graylady3jvrrxbe.onion/images/2017/02/23/world/23kim-1/23kim-1-articleInline.jpg?quality=75\&auto=webp\&disable=upscale}

By
\href{https://www.nytimes3xbfgragh.onion/by/richard-c-paddock}{Richard
C. Paddock} and
\href{http://www.nytimes3xbfgragh.onion/by/choe-sang-hun}{Choe Sang-Hun}

\begin{itemize}
\item
  Feb. 22, 2017
\item
  \begin{itemize}
  \item
  \item
  \item
  \item
  \item
  \end{itemize}
\end{itemize}

\href{http://cn.nytimes3xbfgragh.onion/asia-pacific/20170223/kim-jong-nam-assassination-korea-malaysia/}{阅读简体中文版}

KUALA LUMPUR, Malaysia --- The two young women were what South Korean
intelligence calls ``lizard's tails,'' expendable assets to be cast off
after an operation.

Guided by North Korean agents, they practiced at malls in Kuala Lumpur,
then set their sights on the target: Kim Jong-nam, the estranged elder
brother of North Korea's erratic leader, Kim Jong-un.

With hands doused with toxic liquid, they rubbed the face of their
victim, who was waiting to check in for a flight at Kuala Lumpur
International Airport. Minutes later, their target died on the way to a
hospital. The two women washed their hands and fled.

The suspects were swiftly taken into custody as circumstantial evidence
mounted that North Korea was responsible for the attack.

The very public killing of Mr. Kim appears to be another remarkable
episode in the annals of bizarre North Korean behavior, a whodunit with
geopolitical implications. Speculation swirled that he had been killed
to remove him from the line of succession in North Korea.

In the days since the killing was caught on video, the drama has had an
ever-expanding and multinational cast of characters --- women from
Indonesia and Vietnam accused of carrying out the attack, one of whom
was apparently wearing a white shirt emblazoned with the letters LOL;
\href{https://www.nytimes3xbfgragh.onion/2017/02/15/world/asia/kim-jong-un-brother-assassination-north-korea.html}{a
Malaysian boyfriend}; and others believed to be North Korean agents.

On Wednesday, Malaysia's police chief, Khalid Abu Bakar, said a senior
diplomat at the North Korean Embassy and an employee of the North Korean
state-owned airline, Air Koryo, were also wanted for questioning.
Another North Korean, who was not identified, was also being sought. Mr.
Khalid also said that extra police officers had been sent to the morgue
where Mr. Kim's body was being kept after an attempt to break into the
facility was detected.

Image

Kim Jong-nam was caught entering Japan on a false Dominican Republic
passport in 2001, embarrassing his family. He said he wanted to go to
Tokyo Disneyland.Credit...Shizuo Kambayashi/Associated Press

North Korea has refused to even acknowledge that the dead man was Kim
Jong-nam and has accused Malaysia of carrying out a politically
motivated investigation to placate South Korea and the United States.

North Korea has nonetheless demanded that the body be sent there and, in
a statement on Wednesday, the North Korean Embassy said the two women
were innocent and should be freed.

If the women really had poison on their hands, the embassy statement
said, ``then how is it possible that these female suspects could still
be alive?''

One possible theory is that each woman used a single chemical that
became lethal only when mixed with another. The Malaysian police,
however, said that the substance or substances used in the attack had
not yet been identified.

North Korea has denied any involvement in the killing, which is likely
to anger China, its main ally, which has been seen as a protector of Kim
Jong-nam.

Mr. Kim had long been on a hit list drawn up by his half brother, Kim
Jong-un, according to South Korean intelligence. The younger Mr. Kim,
33, has ordered the execution of scores of senior officials, including
at least one disfavored relative, and may have been prompted to act if
he believed that Beijing saw his half brother as a possible replacement
for him.

Malaysian authorities say the two women arrested, Doan Thi Huong, 28,
and Siti Aisyah, 25, were recruited, trained and equipped by four North
Koreans who have since fled to their home country.

If the attack was a plot by North Korea, it would not be the first time
it had tried to kill Kim Jong-nam.

\includegraphics{https://static01.graylady3jvrrxbe.onion/images/2017/02/23/world/23kim-3/23kim-3-articleInline.jpg?quality=75\&auto=webp\&disable=upscale}

In 2010, according to South Korean investigators, a North Korean agent
based in China received a special order from Pyongyang: ``Terminate''
Kim Jong-nam and bring his body to the North.

That agent, Kim Young-soo, was told that Kim Jong-nam was going to
travel to China from Singapore, where he was then living. The agent's
boss gave him a bundle of cash and ordered him to bribe a taxi driver to
run over Mr. Kim in a fake traffic accident.

The plot was scrapped when Mr. Kim failed to arrive as planned. But it
came to light in 2012, when the agent was caught entering South Korea
and confessed under interrogation.

Since 2011, when Kim Jong-un succeeded his father as North Korea's
ruler, there has been a standing order to assassinate his half brother,
South Korean intelligence officials said last week. There was another
assassination attempt against him in 2012.

Mr. Kim was so afraid that he begged for his life in a letter to his
half brother in 2012.

``Please withdraw the order to punish me and my family,'' Mr. Kim was
quoted as saying in the letter. ``We have nowhere to hide. The only way
to escape is to choose suicide.''

\hypertarget{a-troubled-family}{%
\subsection{A Troubled Family}\label{a-troubled-family}}

The Kim family, which has ruled North Korea since its founding in 1948,
has presided over a Shakespearean nest of internecine plots and family
intrigue. Rival relatives have been sent into exile and occasional
bloody purges have killed off anyone of questionable loyalty and set an
example for others.

Kim Jong-nam was an early dropout in the Kim dynasty's third-generation
power struggle. Sidelined from the race to succeed his father since the
1970s, when his mother was abandoned by his father, he had been
effectively shut out of power and shut off from his father since he was
a teenager. South Korean officials say he never met his half brother Kim
Jong-un.

The final straw for Kim Jong-nam was when he was caught entering Japan
on a false Dominican Republic passport in 2001, embarrassing the family.
He told Japanese officials that he wanted to visit Tokyo Disneyland.

Image

Oceanfront villas in Macau, one of which is believed to belong to Kim
Jong-nam. Mr. Kim lived in exile, mostly in Macau.Credit...Bobby
Yip/Reuters

Mr. Kim lived in exile, mostly in Macau, but enjoyed the affluent life
of a globe-trotting playboy, sometimes traveling with a female
bodyguard. While his father was still alive, the government in Pyongyang
sent him cash allowances.

His uncle, Jang Song-thaek, became a father figure and his main
connection to his country. South Korean officials said Mr. Kim was
thought to have used that connection to conduct business for himself,
like handling contracts involving North Korean minerals.

Mr. Kim often visited Kuala Lumpur, where Mr. Jang's nephew, Jang
Yong-chol, served as North Korean ambassador until 2013.

Mr. Kim sometimes stayed at an embassy guesthouse and sometimes at
five-star hotels, according to Steve Hwang, a restaurant owner who
became a friend.

Mr. Kim often came to the restaurant, Koryo-Won, with his wife, dressed
casually and always wearing a baseball cap. A bodyguard sat outside in
the mall, visible through the window.

``He was very humble, very friendly, a very nice guy,'' Mr. Hwang said.

Mr. Kim never gave his name, but Mr. Hwang, who is from South Korea and
has family in the North, recognized him. To be certain, he said he
collected Mr. Kim's dishes after a meal and sent them to the South
Korean Embassy for fingerprint and DNA analysis, he said. The word came
back that it was indeed Mr. Kim.

When Kim Jong-un took power, he cut off his half brother's allowance. In
2013, he executed their uncle, Mr. Jang, on charges of corruption and
sedition. Mr. Jang's nephew, the ambassador, was recalled the same year
and is thought to have been executed.

Kim Jong-un may have been angered by reports that his half brother had
once considered defecting to South Korea. After Kim Jong-nam's
assassination, some defectors claimed that he had been asked to serve as
head of a government in exile. But Kim Jong-nam never formally proposed
to defect, according to South Korean officials, and he had told
reporters that he had no interest in politics, although he also
criticized the dynastic succession in Pyongyang.

Image

Members of the news media outside the North Korean Embassy in Kuala
Lumpur, Malaysia, after Mr. Kim's death.Credit...Fazry Ismail/European
Pressphoto Agency

\hypertarget{the-setup}{%
\subsection{The Setup}\label{the-setup}}

When Mr. Kim arrived in Kuala Lumpur on Feb. 6, he was using a
diplomatic passport with the name Kim Chol.

By then, it appears, the plot against him was already underway.

Four North Korean men accused of organizing the attack had begun
arriving on Jan. 31, nearly a week before Mr. Kim, the police say. Each
one landed on a different day. The last one arrived Feb. 7, a day after
Mr. Kim.

Unlike most countries, Malaysia allows North Koreans to enter without a
visa and makes it relatively easy for them to work. North Koreans have
established a number of businesses in Malaysia to export products to
other parts of the world and earn foreign currency to send home.

The four North Korean conspirators apparently recruited Ms. Huong and
Ms. Siti from entertainment establishments. Ms. Siti worked as a ``spa
masseuse,'' the police say, and Ms. Huong as an ``entertainment outlet
employee.''

Ms. Huong grew up in a small farming village in Vietnam about three
hours south of Hanoi and studied pharmacy at a community college. Ms.
Siti grew up in a farming village east of the Indonesian capital,
Jakarta. She quit school after sixth grade, was married at 16 and
divorced at 20, before she left for Malaysia.

There were reports that the women were duped, that they had been told
they were participating in a prank. Indonesian officials said they
thought Ms. Siti was tricked into thinking that she was part of a comedy
video involving spraying liquid onto unwitting victims in public.

But Mr. Khalid, the police chief, said they knew what they were doing.
The women had practiced the attack at two malls, he said.

``We strongly believe it is a planned thing and that they are being
trained to do that,'' he said. ``It is not just shooting movies or a
play thing. No way.''

Image

Among the suspects arrested in Mr. Kim's death are Doan Thi Huong, 28,
and Siti Aisyah, 25, who the authorities say were recruited, trained and
equipped by North Koreans.Credit...Royal Malaysian Police

The police say the plotters also brought in Ri Jong Chol, a North Korean
who had been living and working in Kuala Lumpur since at least August.
He was almost certainly a government agent, according to Thae Yong-ho, a
North Korean diplomat who defected to the South last summer, because he
was allowed to live with his family abroad.

On the morning of Feb. 13, Mr. Kim went to the airport to catch his
flight home.

Security videos show him entering the departure hall at Terminal 2
carrying a shoulder bag, checking the departure board and walking toward
the check-in counter for AirAsia, a budget airline.

After his encounter with the women, Mr. Kim approached airport staff and
security officers, waving his hands toward his face repeatedly as he
told them of the attack. They walked with him to the airport clinic one
level down.

Within minutes, he was in an ambulance, but by then the poison was
taking effect. He was dead before he reached the hospital, the police
said.

His last words were,
``\href{http://www.thestar.com.my/news/nation/2017/02/17/jong-nam-last-words-very-painful/\#y1L2cD6yr5HpXtMZ.99}{Very
painful}, very painful. I was sprayed liquid,'' China Press, a Malaysian
Chinese-language newspaper, reported.

The police say the four North Korean conspirators watched the attack
unfold. Soon after, they passed through immigration, had their passports
stamped and left the country before the authorities realized Mr. Kim had
been murdered. All are now believed to be in North Korea.

Mr. Hwang said Mr. Kim had stopped coming to his restaurant around 2014,
after his uncle's execution, and may have fallen on lean times --- which
may explain why he had no bodyguards last week as he prepared to fly
home on a budget carrier.

Mr. Hwang didn't see him during his final trip to Kuala Lumpur and was
surprised by his appearance in the security video. He was wearing a
blazer, instead of his usual T-shirt, and no hat.

It was the first time Mr. Hwang saw that he was bald.

``Nobody could protect him,'' he said.

Advertisement

\protect\hyperlink{after-bottom}{Continue reading the main story}

\hypertarget{site-index}{%
\subsection{Site Index}\label{site-index}}

\hypertarget{site-information-navigation}{%
\subsection{Site Information
Navigation}\label{site-information-navigation}}

\begin{itemize}
\tightlist
\item
  \href{https://help.nytimes3xbfgragh.onion/hc/en-us/articles/115014792127-Copyright-notice}{©~2020~The
  New York Times Company}
\end{itemize}

\begin{itemize}
\tightlist
\item
  \href{https://www.nytco.com/}{NYTCo}
\item
  \href{https://help.nytimes3xbfgragh.onion/hc/en-us/articles/115015385887-Contact-Us}{Contact
  Us}
\item
  \href{https://www.nytco.com/careers/}{Work with us}
\item
  \href{https://nytmediakit.com/}{Advertise}
\item
  \href{http://www.tbrandstudio.com/}{T Brand Studio}
\item
  \href{https://www.nytimes3xbfgragh.onion/privacy/cookie-policy\#how-do-i-manage-trackers}{Your
  Ad Choices}
\item
  \href{https://www.nytimes3xbfgragh.onion/privacy}{Privacy}
\item
  \href{https://help.nytimes3xbfgragh.onion/hc/en-us/articles/115014893428-Terms-of-service}{Terms
  of Service}
\item
  \href{https://help.nytimes3xbfgragh.onion/hc/en-us/articles/115014893968-Terms-of-sale}{Terms
  of Sale}
\item
  \href{https://spiderbites.nytimes3xbfgragh.onion}{Site Map}
\item
  \href{https://help.nytimes3xbfgragh.onion/hc/en-us}{Help}
\item
  \href{https://www.nytimes3xbfgragh.onion/subscription?campaignId=37WXW}{Subscriptions}
\end{itemize}
