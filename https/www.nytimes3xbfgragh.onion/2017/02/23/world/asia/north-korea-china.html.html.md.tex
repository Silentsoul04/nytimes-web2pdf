Sections

SEARCH

\protect\hyperlink{site-content}{Skip to
content}\protect\hyperlink{site-index}{Skip to site index}

\href{https://www.nytimes3xbfgragh.onion/section/world/asia}{Asia
Pacific}

\href{https://myaccount.nytimes3xbfgragh.onion/auth/login?response_type=cookie\&client_id=vi}{}

\href{https://www.nytimes3xbfgragh.onion/section/todayspaper}{Today's
Paper}

\href{/section/world/asia}{Asia Pacific}\textbar{}North Korea Accuses
China of `Mean Behavior' After It Tightens Sanctions

\url{https://nyti.ms/2lBFN41}

\begin{itemize}
\item
\item
\item
\item
\item
\end{itemize}

Advertisement

\protect\hyperlink{after-top}{Continue reading the main story}

Supported by

\protect\hyperlink{after-sponsor}{Continue reading the main story}

\hypertarget{north-korea-accuses-china-of-mean-behavior-after-it-tightens-sanctions}{%
\section{North Korea Accuses China of `Mean Behavior' After It Tightens
Sanctions}\label{north-korea-accuses-china-of-mean-behavior-after-it-tightens-sanctions}}

\includegraphics{https://static01.graylady3jvrrxbe.onion/images/2017/02/25/world/24chinakorea-1/24chinakorea-1-articleLarge.jpg?quality=75\&auto=webp\&disable=upscale}

By \href{http://www.nytimes3xbfgragh.onion/by/choe-sang-hun}{Choe
Sang-Hun}

\begin{itemize}
\item
  Feb. 23, 2017
\item
  \begin{itemize}
  \item
  \item
  \item
  \item
  \item
  \end{itemize}
\end{itemize}

\href{http://cn.nytimes3xbfgragh.onion/asia-pacific/20170224/north-korea-china/}{阅读简体中文版}

SEOUL, South Korea --- North Korea on Thursday criticized China in
unusually bitter language for tightening sanctions, accusing its
powerful Communist neighbor of ``mean behavior'' and ``dancing to the
tune of the U.S.''

The anti-Beijing commentary carried by the North's state-run Korean
Central News Agency, did not name China and was written by a writer
named Jong Phil. While it was not a formal government statement,
commentators in North Korea do not depart from the government's official
position. Mr. Jong left no doubt about his target, referring to ``a
neighboring country, which often claims itself to be a `friendly
neighbor.'''

Mr. Jong's commentary came five days after China
\href{https://www.nytimes3xbfgragh.onion/2017/02/18/world/asia/north-korea-china-coal-imports-suspended.html}{announced
that it was suspending all coal imports} from North Korea for the rest
of the year. China said the ban was part of its efforts to enforce
United Nations sanctions aimed at ending the North's nuclear weapons and
ballistic-missile programs.

``Its recent measures are, in effect, tantamount to the enemies' moves
to bring down the social system in the DPRK,'' the commentary said,
using the acronym for North Korea's official name, Democratic People's
Republic of Korea. ``This country, styling itself a big power, is
dancing to the tune of the U.S.''

China imposed the coal ban after
\href{https://www.nytimes3xbfgragh.onion/2017/02/11/world/asia/north-korea-missile-test-trump.html}{the
North Korean test of a ballistic missile} that the United Nations
Security Council condemned as a violation of its resolutions. The ban
deprived North Korea of one of its most important sources of hard
currency. Coal accounted for nearly 40 percent of its exports in the
past several years, and almost all of it was shipped to China, according
to South Korean government estimates.

Although North Korea's ``juche'' ideology emphasizes the nation's
self-reliance, in reality the country depends on China for 90 percent of
its external trade.

\href{https://www.nytimes3xbfgragh.onion/interactive/2017/02/24/world/asia/north-korea-propaganda-photo.html}{}

\includegraphics{https://static01.graylady3jvrrxbe.onion/images/2017/02/24/international-home/north-korea-propaganda-photo-1487891040109/north-korea-propaganda-photo-1487891040109-thumbLarge.png}

\hypertarget{what-one-photo-tells-us-about-north-koreas-nuclear-program}{%
\subsection{What One Photo Tells Us About North Korea's Nuclear
Program}\label{what-one-photo-tells-us-about-north-koreas-nuclear-program}}

Clues from a single propaganda photo reveal details about North Korea's
expanding weapons programs and internal politics.

That has led officials in South Korea and the United States to argue
that Beijing should use its economic influence to force the North to
suspend its weapons programs. While pushing for ever tighter sanctions,
some officials have suggested that Beijing was losing patience with the
North Korean regime over its continued tests of nuclear weapons and
ballistic missiles.

But other analysts remain skeptical about China's willingness to use its
economic leverage. China and North Korea share a deep bond forged
decades ago when their Communist leaders fought together.

These analysts think that Beijing also fears a destabilized North Korea
more than a nuclear-armed North Korea, and that it considers the country
a vital buffer against the United States military based in South Korea.

They tend to see China's suspension of coal imports as a warning to
North Korea, and as a deft move to blunt Washington's criticism that it
was not doing enough to enforce sanctions. China insists that Washington
engage the North in negotiations to solve the nuclear problem.

On Wednesday, an editorial in the state-controlled Global Times of China
said that despite the coal import ban, China's friendship with North
Korea remained unchanged.

``Chinese sanctions only target at its nuclear weapon program, and we
are firmly opposed to Seoul's political fantasy against Pyongyang,'' it
said.

For its part, North Korea maintained a defiant tone. It warned on
Thursday that it would be ``utterly childish'' to think that the North
would stop building its ``nuclear weapons and intercontinental ballistic
rockets if a few pennies of money is cut off.''

Advertisement

\protect\hyperlink{after-bottom}{Continue reading the main story}

\hypertarget{site-index}{%
\subsection{Site Index}\label{site-index}}

\hypertarget{site-information-navigation}{%
\subsection{Site Information
Navigation}\label{site-information-navigation}}

\begin{itemize}
\tightlist
\item
  \href{https://help.nytimes3xbfgragh.onion/hc/en-us/articles/115014792127-Copyright-notice}{©~2020~The
  New York Times Company}
\end{itemize}

\begin{itemize}
\tightlist
\item
  \href{https://www.nytco.com/}{NYTCo}
\item
  \href{https://help.nytimes3xbfgragh.onion/hc/en-us/articles/115015385887-Contact-Us}{Contact
  Us}
\item
  \href{https://www.nytco.com/careers/}{Work with us}
\item
  \href{https://nytmediakit.com/}{Advertise}
\item
  \href{http://www.tbrandstudio.com/}{T Brand Studio}
\item
  \href{https://www.nytimes3xbfgragh.onion/privacy/cookie-policy\#how-do-i-manage-trackers}{Your
  Ad Choices}
\item
  \href{https://www.nytimes3xbfgragh.onion/privacy}{Privacy}
\item
  \href{https://help.nytimes3xbfgragh.onion/hc/en-us/articles/115014893428-Terms-of-service}{Terms
  of Service}
\item
  \href{https://help.nytimes3xbfgragh.onion/hc/en-us/articles/115014893968-Terms-of-sale}{Terms
  of Sale}
\item
  \href{https://spiderbites.nytimes3xbfgragh.onion}{Site Map}
\item
  \href{https://help.nytimes3xbfgragh.onion/hc/en-us}{Help}
\item
  \href{https://www.nytimes3xbfgragh.onion/subscription?campaignId=37WXW}{Subscriptions}
\end{itemize}
