Sections

SEARCH

\protect\hyperlink{site-content}{Skip to
content}\protect\hyperlink{site-index}{Skip to site index}

\href{https://www.nytimes3xbfgragh.onion/section/technology}{Technology}

\href{https://myaccount.nytimes3xbfgragh.onion/auth/login?response_type=cookie\&client_id=vi}{}

\href{https://www.nytimes3xbfgragh.onion/section/todayspaper}{Today's
Paper}

\href{/section/technology}{Technology}\textbar{}U.S. Fines ZTE of China
\$1.19 Billion for Breaching Sanctions

\url{https://nyti.ms/2mBHfXa}

\begin{itemize}
\item
\item
\item
\item
\item
\end{itemize}

Advertisement

\protect\hyperlink{after-top}{Continue reading the main story}

Supported by

\protect\hyperlink{after-sponsor}{Continue reading the main story}

\hypertarget{us-fines-zte-of-china-119-billion-for-breaching-sanctions}{%
\section{U.S. Fines ZTE of China \$1.19 Billion for Breaching
Sanctions}\label{us-fines-zte-of-china-119-billion-for-breaching-sanctions}}

\includegraphics{https://static01.graylady3jvrrxbe.onion/images/2017/03/08/business/08chinasanctions1/08chinasanctions1-articleInline.jpg?quality=75\&auto=webp\&disable=upscale}

By \href{https://www.nytimes3xbfgragh.onion/by/paul-mozur}{Paul Mozur}
and \href{http://www.nytimes3xbfgragh.onion/by/cecilia-kang}{Cecilia
Kang}

\begin{itemize}
\item
  March 7, 2017
\item
  \begin{itemize}
  \item
  \item
  \item
  \item
  \item
  \end{itemize}
\end{itemize}

\href{http://cn.nytimes3xbfgragh.onion/business/20170308/zte-china-fine/}{阅读简体中文版}

HONG KONG --- As one of China's few truly international technology
companies, ZTE is often held up by Beijing as part of a new generation
of firms that is able to compete beyond Chinese borders.

On Tuesday, the United States government made an example of ZTE in a
different way.

As part of
\href{https://www.nytimes3xbfgragh.onion/2016/03/08/technology/us-restricts-sales-to-zte-saying-it-breached-sanctions.html}{a
settlement for breaking sanctions and selling electronics to Iran} and
North Korea, ZTE agreed to plead guilty and pay \$1.19 billion in fines,
the United States Department of Commerce said in
\href{https://www.commerce.gov/news/press-releases/2017/03/secretary-commerce-wilbur-l-ross-jr-announces-119-billion-penalty}{an
announcement}. The penalty is the largest criminal fine in a United
States sanctions case.

The action is the latest in a series of skirmishes between the United
States and China over technology policy. It also offered a chance for
President Trump's young administration to make a statement about the
seriousness of United States sanctions. In addition to ZTE, the Commerce
Department is also investigating the company's larger Chinese rival,
Huawei, for violating United States sanctions.

``We are putting the world on notice: The games are over,'' said
Commerce Secretary Wilbur L. Ross. ``Those who flout our economic
sanctions and export control laws will not go unpunished --- they will
suffer the harshest of consequences.''

ZTE was found to have breached United States sanctions against Iran by
selling American-made goods to the country last March. At the time, the
Commerce Department said it would force American companies to obtain a
special license to sell to ZTE, which makes smartphones and
telecommunications infrastructure equipment. The restrictions would have
had the potential to cripple ZTE's supply chain.

The ban, however, was never put in place, and instead the Chinese
company was given a series of reprieves.

Still, ZTE, which is China's second-largest maker of telecom equipment,
has not fared well over the past year. Its revenue from the expansion of
China's 4G cellular networks has slowed and its smartphone business has
faced major competition from new Chinese handset makers, as well as
Huawei.

On Tuesday, the Commerce Department said that along with selling
prohibited American electronics to build Iran's telecom networks, ZTE
also made 283 shipments of microprocessors, servers and routers to North
Korea, violating American embargoes in that country as well.

``ZTE engaged in an elaborate scheme to acquire U.S.-origin items, send
the items to Iran and mask its involvement in those exports,'' said the
acting assistant attorney general, Mary B. McCord. ``The plea agreement
alleges that the highest levels of management within the company
approved the scheme.''

She added that ZTE repeatedly lied to and misled federal investigators,
its own lawyers and internal investigators.

In a statement, ZTE said that it had strengthened its compliance
policies and undergone a shake-up of top leaders; the company named a
new chief executive last April.

``ZTE acknowledges the mistakes it made, takes responsibility for them
and remains committed to positive change in the company,'' said Zhao
Xianming, chairman and chief executive of ZTE.

Although China and the United States have occasionally traded barbs over
technology policy and cyberattacks, the actions against ZTE by the
United States government have not had a major impact on the relationship
of the two countries, though Beijing could respond harshly to the new
fine.

It is unclear whether the Commerce Department has completed its
investigations into Chinese telecom equipment makers.

In a rare step accompanying the announcement last March, the Commerce
Department provided two internal ZTE documents.

One, from 2011 and signed by several senior ZTE executives, detailed how
the company had ``ongoing projects in all five major embargoed countries
--- Iran, Sudan, North Korea, Syria and Cuba.'' Another document laid
out in a complex flow chart a method for circumventing United States
export controls.

Citing an unnamed company as a model for circumventing United States
sanctions, that second document seemed to
\href{https://www.nytimes3xbfgragh.onion/2016/03/19/technology/zte-document-raises-questions-about-huawei-and-sanctions.html}{implicate
ZTE's more politically important rival, Huawei}.

The New York Times reported last year that the United States government
was also investigating whether Huawei broke export controls. The
Commerce Department
\href{https://www.nytimes3xbfgragh.onion/2016/06/03/technology/huawei-technologies-subpoena-iran-north-korea.html}{subpoenaed
Huawei}, demanding it turn over all information regarding the export or
re-export of American technology to Cuba, Iran, North Korea, Sudan and
Syria.

Huawei has said it is committed to complying with laws and regulations
where it operates.

Huawei and ZTE are private companies, but they have deep ties to the
Chinese government, in part because they supply much of the equipment
that makes the country's telecom backbone function.

Advertisement

\protect\hyperlink{after-bottom}{Continue reading the main story}

\hypertarget{site-index}{%
\subsection{Site Index}\label{site-index}}

\hypertarget{site-information-navigation}{%
\subsection{Site Information
Navigation}\label{site-information-navigation}}

\begin{itemize}
\tightlist
\item
  \href{https://help.nytimes3xbfgragh.onion/hc/en-us/articles/115014792127-Copyright-notice}{©~2020~The
  New York Times Company}
\end{itemize}

\begin{itemize}
\tightlist
\item
  \href{https://www.nytco.com/}{NYTCo}
\item
  \href{https://help.nytimes3xbfgragh.onion/hc/en-us/articles/115015385887-Contact-Us}{Contact
  Us}
\item
  \href{https://www.nytco.com/careers/}{Work with us}
\item
  \href{https://nytmediakit.com/}{Advertise}
\item
  \href{http://www.tbrandstudio.com/}{T Brand Studio}
\item
  \href{https://www.nytimes3xbfgragh.onion/privacy/cookie-policy\#how-do-i-manage-trackers}{Your
  Ad Choices}
\item
  \href{https://www.nytimes3xbfgragh.onion/privacy}{Privacy}
\item
  \href{https://help.nytimes3xbfgragh.onion/hc/en-us/articles/115014893428-Terms-of-service}{Terms
  of Service}
\item
  \href{https://help.nytimes3xbfgragh.onion/hc/en-us/articles/115014893968-Terms-of-sale}{Terms
  of Sale}
\item
  \href{https://spiderbites.nytimes3xbfgragh.onion}{Site Map}
\item
  \href{https://help.nytimes3xbfgragh.onion/hc/en-us}{Help}
\item
  \href{https://www.nytimes3xbfgragh.onion/subscription?campaignId=37WXW}{Subscriptions}
\end{itemize}
