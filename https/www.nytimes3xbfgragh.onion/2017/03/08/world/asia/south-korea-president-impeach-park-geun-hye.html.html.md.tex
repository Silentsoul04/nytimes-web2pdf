Sections

SEARCH

\protect\hyperlink{site-content}{Skip to
content}\protect\hyperlink{site-index}{Skip to site index}

\href{https://www.nytimes3xbfgragh.onion/section/world/asia}{Asia
Pacific}

\href{https://myaccount.nytimes3xbfgragh.onion/auth/login?response_type=cookie\&client_id=vi}{}

\href{https://www.nytimes3xbfgragh.onion/section/todayspaper}{Today's
Paper}

\href{/section/world/asia}{Asia Pacific}\textbar{}South Korean
President's Fate to Be Decided by Court on Friday

\url{https://nyti.ms/2m1NP5E}

\begin{itemize}
\item
\item
\item
\item
\item
\end{itemize}

Advertisement

\protect\hyperlink{after-top}{Continue reading the main story}

Supported by

\protect\hyperlink{after-sponsor}{Continue reading the main story}

\hypertarget{south-korean-presidents-fate-to-be-decided-by-court-on-friday}{%
\section{South Korean President's Fate to Be Decided by Court on
Friday}\label{south-korean-presidents-fate-to-be-decided-by-court-on-friday}}

\includegraphics{https://static01.graylady3jvrrxbe.onion/images/2017/03/09/world/09impeach-1/09impeach-1-articleInline.jpg?quality=75\&auto=webp\&disable=upscale}

By \href{http://www.nytimes3xbfgragh.onion/by/choe-sang-hun}{Choe
Sang-Hun}

\begin{itemize}
\item
  March 8, 2017
\item
  \begin{itemize}
  \item
  \item
  \item
  \item
  \item
  \end{itemize}
\end{itemize}

SEOUL, South Korea --- The Constitutional Court of South Korea said
Wednesday that it would rule on Friday whether to reinstate President
Park Geun-hye or formally oust her from office on charges of corruption
and abuse of power.

Ms. Park has been suspended from office, with all her presidential
powers frozen, since the National Assembly overwhelmingly
\href{https://www.nytimes3xbfgragh.onion/2016/12/09/world/asia/south-korea-president-park-geun-hye-impeached.html}{voted
to impeach her} in December on charges of abusing power and collecting
bribes from businesses.

If the Constitutional Court supports the vote in its verdict on Friday,
Ms. Park will become the first South Korean president to lose office
through parliamentary impeachment.

The court has held 17 hearings since the impeachment vote, but Ms. Park
has never appeared in court. In a statement read by one of her lawyers
during the last hearing, on Feb. 27, she vehemently denied any
wrongdoing.

``I feel crushed by all these misunderstandings and allegations,'' she
said, calling the accusations groundless.

But Kweon Seong-dong, the lead prosecutor in the impeachment trial,
called Ms. Park and her secretive confidante, Choi Soon-sil, ``enemies
to democracy.''

In the scandal, which has rocked South Korea for months, Ms. Park and
Ms. Choi were accused of conspiring to collect tens of millions of
dollars in bribes from businesses. Ms. Park was also accused of letting
Ms. Choi, who had no experience in policy making, edit her speeches,
install acquaintances as senior government officials and influence state
affairs from the shadows.

In its impeachment motion, the legislature also accused Ms. Park of
undermining freedom of the press by cracking down on her critics and of
shirking her duty to protect citizens' lives by neglecting to respond
efficiently to a
\href{https://www.nytimes3xbfgragh.onion/2014/04/21/world/asia/chaos-ruled-sinking-ferry.html}{ferry
disaster in 2014} that killed more than 300 people.

If at least six members of the nine-judge Constitutional Court vote to
impeach, Ms. Park will be removed from office. South Korea will then
have 60 days to elect a successor, with Prime Minister Hwang Kyo-ahn
acting as president during that time.

If fewer than six judges vote for impeachment, Ms. Park will immediately
be returned to office. Her five-year term ends next February.

The court has been working with only eight judges after one retired in
January, but that will not affect the required number of votes for or
against impeachment.

No South Korean president has been forced from office through
impeachment. The National Assembly voted in 2004 to impeach President
Roh Moo-hyun, but the Constitutional Court reinstated him, ruling that
his violations of election law were too minor to justify ending his
presidency. Mr. Roh did not attend the court's hearings on his
impeachment.

The charges against Ms. Park are considered much more serious than those
that Mr. Roh faced, and they have infuriated the public.
\href{https://www.nytimes3xbfgragh.onion/2016/11/13/world/asia/korea-park-geun-hye-protests.html}{Large
crowds have gathered} in central Seoul for months on Saturdays demanding
an end to her presidency. In recent weeks, however, Ms. Park's
supporters have also
\href{https://www.nytimes3xbfgragh.onion/2017/02/18/world/asia/south-korea-impeached-leader-park-geun-hye.html}{organized
big rallies} calling for her reinstatement.

Regardless of the Constitutional Court's ruling, Ms. Park will most
likely face criminal charges as soon as her presidency ends. While in
office, she is protected from indictment.

On Monday, a special prosecutor
\href{https://www.nytimes3xbfgragh.onion/2017/03/06/world/asia/president-park-geun-hye-bribery-korea.html}{asked
state prosecutors} to indict Ms. Park on bribery charges, saying that
she and Ms. Choi conspired to take \$38 million in bribes from Samsung,
one of the world's largest technology companies. Samsung's vice
chairman, Lee Jae-yong, the third-generation scion of the family who
runs the conglomerate, goes on trial this week
\href{https://www.nytimes3xbfgragh.onion/2017/02/28/world/asia/lee-jae-yong-samsung.html}{on
bribery charges}.

The special prosecutor said that the president should also face a
criminal charge of abusing official power, saying she
\href{https://www.nytimes3xbfgragh.onion/2017/01/12/world/asia/south-korea-president-park-blacklist-artists.html}{conspired
with aides} to blacklist thousands of artists, writers and movie
directors deemed unfriendly to her government and exclude them from
government-funded support programs.

Advertisement

\protect\hyperlink{after-bottom}{Continue reading the main story}

\hypertarget{site-index}{%
\subsection{Site Index}\label{site-index}}

\hypertarget{site-information-navigation}{%
\subsection{Site Information
Navigation}\label{site-information-navigation}}

\begin{itemize}
\tightlist
\item
  \href{https://help.nytimes3xbfgragh.onion/hc/en-us/articles/115014792127-Copyright-notice}{©~2020~The
  New York Times Company}
\end{itemize}

\begin{itemize}
\tightlist
\item
  \href{https://www.nytco.com/}{NYTCo}
\item
  \href{https://help.nytimes3xbfgragh.onion/hc/en-us/articles/115015385887-Contact-Us}{Contact
  Us}
\item
  \href{https://www.nytco.com/careers/}{Work with us}
\item
  \href{https://nytmediakit.com/}{Advertise}
\item
  \href{http://www.tbrandstudio.com/}{T Brand Studio}
\item
  \href{https://www.nytimes3xbfgragh.onion/privacy/cookie-policy\#how-do-i-manage-trackers}{Your
  Ad Choices}
\item
  \href{https://www.nytimes3xbfgragh.onion/privacy}{Privacy}
\item
  \href{https://help.nytimes3xbfgragh.onion/hc/en-us/articles/115014893428-Terms-of-service}{Terms
  of Service}
\item
  \href{https://help.nytimes3xbfgragh.onion/hc/en-us/articles/115014893968-Terms-of-sale}{Terms
  of Sale}
\item
  \href{https://spiderbites.nytimes3xbfgragh.onion}{Site Map}
\item
  \href{https://help.nytimes3xbfgragh.onion/hc/en-us}{Help}
\item
  \href{https://www.nytimes3xbfgragh.onion/subscription?campaignId=37WXW}{Subscriptions}
\end{itemize}
