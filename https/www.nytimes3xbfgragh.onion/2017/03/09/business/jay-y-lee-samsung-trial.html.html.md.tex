Sections

SEARCH

\protect\hyperlink{site-content}{Skip to
content}\protect\hyperlink{site-index}{Skip to site index}

\href{https://www.nytimes3xbfgragh.onion/section/business}{Business}

\href{https://myaccount.nytimes3xbfgragh.onion/auth/login?response_type=cookie\&client_id=vi}{}

\href{https://www.nytimes3xbfgragh.onion/section/todayspaper}{Today's
Paper}

\href{/section/business}{Business}\textbar{}Jay Y. Lee, Samsung Leader
Facing `Trial of the Century,' Denies Charges

\url{https://nyti.ms/2moKtfz}

\begin{itemize}
\item
\item
\item
\item
\item
\end{itemize}

Advertisement

\protect\hyperlink{after-top}{Continue reading the main story}

Supported by

\protect\hyperlink{after-sponsor}{Continue reading the main story}

\hypertarget{jay-y-lee-samsung-leader-facing-trial-of-the-century-denies-charges}{%
\section{Jay Y. Lee, Samsung Leader Facing `Trial of the Century,'
Denies
Charges}\label{jay-y-lee-samsung-leader-facing-trial-of-the-century-denies-charges}}

\includegraphics{https://static01.graylady3jvrrxbe.onion/images/2017/03/10/world/10Samsung/10Samsung-articleInline.jpg?quality=75\&auto=webp\&disable=upscale}

By Jonathan Soble

\begin{itemize}
\item
  March 9, 2017
\item
  \begin{itemize}
  \item
  \item
  \item
  \item
  \item
  \end{itemize}
\end{itemize}

Five executives at Samsung, including the conglomerate's de facto
leader, Lee Jae-yong, formally denied bribery charges against them on
Thursday, in a preliminary hearing for a trial with the potential to
shake South Korea.

Mr. Lee, who also goes by the name Jay Y. Lee, and the other executives
face charges that strike at the heart of the deep ties between the South
Korean government and powerful family-controlled businesses, a source of
growing public resentment. Parliament voted in December to impeach
\href{https://www.nytimes3xbfgragh.onion/2016/12/09/world/asia/south-korea-president-park-geun-hye-impeached.html}{President
Park Geun-hye} over accusations of corruption and other abuses of power,
and she could be
\href{https://www.nytimes3xbfgragh.onion/2017/03/08/world/asia/south-korea-president-impeach-park-geun-hye.html?rref=collection\%2Ftimestopic\%2FPark\%20Geun-hye\&action=click\&contentCollection=timestopics\&region=stream\&module=stream_unit\&version=latest\&contentPlacement=2\&pgtype=collection}{formally
removed from office} soon.

But the related arrest of Mr. Lee, scion of the country's biggest and
most profitable conglomerate, or chaebol, is a
\href{https://www.nytimes3xbfgragh.onion/2017/03/04/business/south-korea-samsung-bribery-lee.html}{momentous
turn}. Chaebol bosses, including Mr. Lee's father, have been convicted
in corruption cases, but punishments have usually been light or
commuted.

Many see Mr. Lee's trial as a test of whether South Korea can change by
abandoning longstanding deference to the business clans that have
dominated the country's glittering economic rise. The chief prosecutor
has said it could be the ``trial of the century.''

Mr. Lee is accused of funneling \$36 million in bribes to a secretive
confidante of Ms. Park's, as well as a range of other crimes:
embezzlement, illegal transfer of property abroad and perjury before
Parliament. He and the other executives, who are accused of aiding Mr.
Lee, did not appear in court for the preliminary hearing on Thursday,
but they denied the charges through lawyers.

Prosecutors say Mr. Lee sought a particularly South Korean favor in
return: Approval for a merger that cemented his family's hold over the
sprawling Samsung group, a vast and complex network of companies whose
interests range from cellphones to shipbuilding.

The merger, of two Samsung affiliates, took place in 2015 and helped Mr.
Lee, now 48, inherit corporate control from his father, Lee Kun-hee, who
remains Samsung's chairman but has been out of the public eye since
falling ill in 2014. The elder Mr. Lee has been convicted of corruption
charges twice but was pardoned both times.

South Koreans are losing patience after years of chaebol-related
corruption scandals. When crowds took to the streets in recent months to
demand Ms. Park's impeachment, they also called for the arrest of Mr.
Lee and other business leaders. That may have emboldened the special
prosecutor,
\href{https://www.nytimes3xbfgragh.onion/2017/03/06/world/asia/president-park-geun-hye-bribery-korea.html}{Park
Young-soo}, who
\href{https://www.nytimes3xbfgragh.onion/2017/01/18/world/asia/samsung-korea-president-impeachment.html}{had
been struggling} to build criminal cases against Mr. Lee and Ms. Park.
Prosecutors have named Ms. Park as an accomplice in Mr. Lee's case, most
recently in a report issued on Monday about Mr. Lee's indictment.

Prosecutors say bribes Samsung made to members of Ms. Park's circle took
many forms, including a \$900,000 horse for the equestrian daughter of
the president's confidante, Choi Soon-sil.

The preliminary hearing on Thursday attracted a modest crowd of
onlookers, reporters and people with grievances against Samsung, a
symbol both of national pride and, for some in South Korea, unfair elite
privilege.

``There is a saying in Korean that goes, `Guilty without money, and not
guilty with money,' '' said Koh Hyun-sook, 53, who added that her
husband was fighting a disability case against a Samsung company.

The hearing largely addressed the kind of evidence that would be allowed
at the trial. Lawyers for Mr. Lee and other executives said prosecutors
had overstepped the bounds of the case by contending that Samsung was
engaged in a long-term, covert effort to ensure the Lee family
succession.

It remains unclear when the main trial will begin. The judge scheduled
the next preliminary hearing for March 23.

Advertisement

\protect\hyperlink{after-bottom}{Continue reading the main story}

\hypertarget{site-index}{%
\subsection{Site Index}\label{site-index}}

\hypertarget{site-information-navigation}{%
\subsection{Site Information
Navigation}\label{site-information-navigation}}

\begin{itemize}
\tightlist
\item
  \href{https://help.nytimes3xbfgragh.onion/hc/en-us/articles/115014792127-Copyright-notice}{©~2020~The
  New York Times Company}
\end{itemize}

\begin{itemize}
\tightlist
\item
  \href{https://www.nytco.com/}{NYTCo}
\item
  \href{https://help.nytimes3xbfgragh.onion/hc/en-us/articles/115015385887-Contact-Us}{Contact
  Us}
\item
  \href{https://www.nytco.com/careers/}{Work with us}
\item
  \href{https://nytmediakit.com/}{Advertise}
\item
  \href{http://www.tbrandstudio.com/}{T Brand Studio}
\item
  \href{https://www.nytimes3xbfgragh.onion/privacy/cookie-policy\#how-do-i-manage-trackers}{Your
  Ad Choices}
\item
  \href{https://www.nytimes3xbfgragh.onion/privacy}{Privacy}
\item
  \href{https://help.nytimes3xbfgragh.onion/hc/en-us/articles/115014893428-Terms-of-service}{Terms
  of Service}
\item
  \href{https://help.nytimes3xbfgragh.onion/hc/en-us/articles/115014893968-Terms-of-sale}{Terms
  of Sale}
\item
  \href{https://spiderbites.nytimes3xbfgragh.onion}{Site Map}
\item
  \href{https://help.nytimes3xbfgragh.onion/hc/en-us}{Help}
\item
  \href{https://www.nytimes3xbfgragh.onion/subscription?campaignId=37WXW}{Subscriptions}
\end{itemize}
