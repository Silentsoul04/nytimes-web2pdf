Sections

SEARCH

\protect\hyperlink{site-content}{Skip to
content}\protect\hyperlink{site-index}{Skip to site index}

\href{https://www.nytimes3xbfgragh.onion/section/world/asia}{Asia
Pacific}

\href{https://myaccount.nytimes3xbfgragh.onion/auth/login?response_type=cookie\&client_id=vi}{}

\href{https://www.nytimes3xbfgragh.onion/section/todayspaper}{Today's
Paper}

\href{/section/world/asia}{Asia Pacific}\textbar{}South Korean Stores
Feel China's Wrath as U.S. Missile System Is Deployed

\url{https://nyti.ms/2m6tpbF}

\begin{itemize}
\item
\item
\item
\item
\item
\item
\end{itemize}

Advertisement

\protect\hyperlink{after-top}{Continue reading the main story}

Supported by

\protect\hyperlink{after-sponsor}{Continue reading the main story}

\hypertarget{south-korean-stores-feel-chinas-wrath-as-us-missile-system-is-deployed}{%
\section{South Korean Stores Feel China's Wrath as U.S. Missile System
Is
Deployed}\label{south-korean-stores-feel-chinas-wrath-as-us-missile-system-is-deployed}}

\includegraphics{https://static01.graylady3jvrrxbe.onion/images/2017/03/10/world/10China1/10China1-articleInline.jpg?quality=75\&auto=webp\&disable=upscale}

By
\href{https://www.nytimes3xbfgragh.onion/by/javier-c-hernandez}{Javier
C. Hernández}, Owen Guo and Ryan Mcmorrow

\begin{itemize}
\item
  March 9, 2017
\item
  \begin{itemize}
  \item
  \item
  \item
  \item
  \item
  \item
  \end{itemize}
\end{itemize}

\href{http://cn.nytimes3xbfgragh.onion/china/20170310/china-lotte-thaad-south-korea/}{阅读简体中文版}

BEIJING --- He usually went there for neon-orange jars of citron tea
with honey, jumbo packets of dried seaweed, and cartons of eggs that
seemed to be perpetually on sale, 15 for eight renminbi, or about a
dollar.

But this time was different. Li Xin, a retired store owner, showed up at
a Beijing branch of Lotte Mart, a South Korean supermarket chain, with a
message.

``Get out of China!'' Mr. Li, 66, a tall man in a fraying green parka,
shouted at the entrance to a Lotte Mart store in central Beijing. ``We
don't want traitors!''

A wave of anti-South Korean sentiment has broken out across China after
the
\href{https://www.nytimes3xbfgragh.onion/2017/03/06/world/asia/north-korea-thaad-missile-defense-us-china.html}{South's
embrace} of an American missile defense system Washington began shipping
this week that China says can be used to spy on its territory.

The government-controlled news media, in brassy editorials, has urged
boycotts of South Korean products. Students, retirees and taxi drivers
have led protests against South Korean businesses. Tourism officials
have ordered several mainland travel agencies to cancel group trips to
South Korea.

Frustrated nationalists have vowed not to eat kimchi or Korean barbecue.
South Korean bands have been denied visas to perform in China, and South
Korean shows have disappeared from Chinese television and streaming
services.

The furor poses a test for President Xi Jinping of China.

On the one hand, Beijing is seeking to pressure South Korea to abandon
the missile defense system, which officials see as a threat to China's
security.

On the other, Chinese leaders are eager to maintain good relations,
especially as South Korea faces political uncertainty over the removal
Friday of its president, Park Geun-hye, and the possibility that a
candidate willing to reconsider the missile system could replace her.

The backlash against South Korean businesses has divided the Chinese.
Some say it is necessary to counter American military might in Asia.
Others warn against such nationalism, arguing that China should find
more amicable ways of engaging South Korea, a close economic ally.

``Peace is most important,'' said Liu Yuanyuan, 25, an employee at a
German pharmaceutical company in Beijing. ``Countries should not
threaten one another.''

Much of the ire against South Korea has focused on Lotte, a conglomerate
that operates 112 stores with some 13,000 employees in mainland China.
The company, which entered China in 2008, has been overwhelmed with
protests and scrutiny by the authorities since it agreed to provide land
in South Korea for the deployment of the American antimissile system.

\includegraphics{https://static01.graylady3jvrrxbe.onion/images/2017/03/10/world/10China2/10China2-articleInline.jpg?quality=75\&auto=webp\&disable=upscale}

As of Thursday, the Chinese authorities had closed nearly half of
Lotte's stores on the mainland, citing safety violations, the company
said in a statement.

One store was fined about \$3,000 for using hand-radios that emitted
``illegal wireless signals.'' On Wednesday, the authorities ordered the
monthlong closing of a chocolate factory jointly owned by Lotte and the
Hershey Company of the United States after the results of a fire
inspection.

Protests erupted again at Lotte stores this week after American
officials announced that they had begun to install the antimissile
technology, known as the Terminal High Altitude Area Defense system, or
Thaad, in South Korea.

In Zibo, a city in the eastern province of Shandong, protesters held
banners demanding that Lotte leave China. ``The safety of our motherland
cannot be violated,'' a 30-foot banner read.

In Xuchang, a city in the central province of Henan, employees at a mall
stood in rows holding banners protesting Lotte and singing the Chinese
national anthem.

And in Beijing, there were scattered voices of discontent, including Mr.
Li's.

He said he had first heard about Lotte's land deal on WeChat, a popular
messaging app, where a petition calling the company a ``traitor and
enemy of the Chinese people'' was circulating.

``I had a lot of hope for the future of China and South Korea,'' he
said. ``Now I worry South Korea is changing.''

In Kunming, a southern city, students at Yunnan Minzu University posted
a sign denouncing Lotte on the door of their dormitory.

``Seoul is tiny and insignificant!'' the sign read, according to a
photograph provided by one of the students, Liu Guomengchen, 21.
``Empower my big China!''

Ms. Liu, who studies environmental design, said she would stop buying
South Korean cosmetics and other goods.

``These things are totally dispensable,'' she said. ``China is becoming
more and more powerful. Countries like South Korea and the U.S. see our
rise as a threat, so they want to work together to weaken us.''

The Chinese news media has played a central role in fueling the
protests. An
\href{http://news.xinhuanet.com/world/2017-02/27/c_1120539249.htm}{opinion
article} by Xinhua, the official news agency, last month suggested that
Lotte was an accomplice in an effort to undermine China and that it was
no longer welcome in the country. An editorial in
\href{http://article.cyol.com/news/content/2017-03/01/content_15679126.htm}{China
Youth Daily} last week urged a boycott.

Image

Barbed wire around a golf course owned by Lotte in Seongju, South Korea,
where the Terminal High Altitude Area Defense system will be
deployed.Credit...Reuters

For all the bombast in the news media, some have urged restraint,
questioning the wisdom of efforts to drum up criticism of South Korea.

Zhang Mengjie, 29, who adores South Korean boy bands like BTS, said
boycotts of South Korean goods and artists were irrational.

``These stars are just there for entertainment; they don't want to
engage in politics,'' Ms. Zhang said. ``They have nothing to do with
it.''

Referring to the people who were protesting against Lotte, she added:
``I don't think this is real patriotism. They just go with the flow, act
impulsively and use extreme rhetoric.''

For Chinese citizens with relatives and friends from South Korea, the
backlash has created anxieties.

Dong Mengmeng, 24, a ski coach from the eastern province of Anhui, is
planning to get married next month to her South Korean fiancé, Jung
Jaeyoon, 27, in Gyeongju, South Korea. But she said that because of the
tensions she had been unable to secure tourist visas for 11 Chinese
relatives to attend the ceremony.

``I've been hijacked by patriotism,'' she said.

Ms. Dong said that she was often in tears and that her mother was afraid
to inquire about the status of her visa application because she was
worried she would be harassed.

\begin{quote}
\end{quote}

This year is the 25th anniversary of the establishment of formal ties
between China and South Korea. But in a sign of the tensions between the
two countries, there is no plan in place yet to celebrate the occasion.

Analysts said that the protests might be short-lived and that Chinese
leaders probably did not want too much animosity with elections looming
in South Korea.

``These initiatives would typically peter out quite quickly,'' said Pal
Nyiri, a professor at Vrije Universiteit Amsterdam who studies Chinese
nationalism. ``The government has been following the same policies ---
fostering nationalism and then using it, but also being wary of it
getting out of hand.''

Many Chinese people are already finding it difficult to uphold a
boycott, given the preponderance of popular Korean goods --- makeup,
face masks, kimchi --- on the shelves of Chinese stores.

Zhang Xin, an electrician, stopped at a Lotte store in Beijing on his
way home, and bought a bag of ribs to make lunch for his wife. He said
he supported the boycott.

``South Korea is always blustering at China; they are arrogant,'' Mr.
Zhang, 49, said. ``In the future, I will shop here less.''

Advertisement

\protect\hyperlink{after-bottom}{Continue reading the main story}

\hypertarget{site-index}{%
\subsection{Site Index}\label{site-index}}

\hypertarget{site-information-navigation}{%
\subsection{Site Information
Navigation}\label{site-information-navigation}}

\begin{itemize}
\tightlist
\item
  \href{https://help.nytimes3xbfgragh.onion/hc/en-us/articles/115014792127-Copyright-notice}{©~2020~The
  New York Times Company}
\end{itemize}

\begin{itemize}
\tightlist
\item
  \href{https://www.nytco.com/}{NYTCo}
\item
  \href{https://help.nytimes3xbfgragh.onion/hc/en-us/articles/115015385887-Contact-Us}{Contact
  Us}
\item
  \href{https://www.nytco.com/careers/}{Work with us}
\item
  \href{https://nytmediakit.com/}{Advertise}
\item
  \href{http://www.tbrandstudio.com/}{T Brand Studio}
\item
  \href{https://www.nytimes3xbfgragh.onion/privacy/cookie-policy\#how-do-i-manage-trackers}{Your
  Ad Choices}
\item
  \href{https://www.nytimes3xbfgragh.onion/privacy}{Privacy}
\item
  \href{https://help.nytimes3xbfgragh.onion/hc/en-us/articles/115014893428-Terms-of-service}{Terms
  of Service}
\item
  \href{https://help.nytimes3xbfgragh.onion/hc/en-us/articles/115014893968-Terms-of-sale}{Terms
  of Sale}
\item
  \href{https://spiderbites.nytimes3xbfgragh.onion}{Site Map}
\item
  \href{https://help.nytimes3xbfgragh.onion/hc/en-us}{Help}
\item
  \href{https://www.nytimes3xbfgragh.onion/subscription?campaignId=37WXW}{Subscriptions}
\end{itemize}
