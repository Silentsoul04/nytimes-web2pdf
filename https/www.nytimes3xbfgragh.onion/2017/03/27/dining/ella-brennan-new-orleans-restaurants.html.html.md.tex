Sections

SEARCH

\protect\hyperlink{site-content}{Skip to
content}\protect\hyperlink{site-index}{Skip to site index}

\href{https://www.nytimes3xbfgragh.onion/section/food}{Food}

\href{https://myaccount.nytimes3xbfgragh.onion/auth/login?response_type=cookie\&client_id=vi}{}

\href{https://www.nytimes3xbfgragh.onion/section/todayspaper}{Today's
Paper}

\href{/section/food}{Food}\textbar{}At 91, Ella Brennan Still Feeds (and
Leads) New Orleans

\url{https://nyti.ms/2nFhDJ0}

\begin{itemize}
\item
\item
\item
\item
\item
\item
\end{itemize}

Advertisement

\protect\hyperlink{after-top}{Continue reading the main story}

Supported by

\protect\hyperlink{after-sponsor}{Continue reading the main story}

\hypertarget{at-91-ella-brennan-still-feeds-and-leads-new-orleans}{%
\section{At 91, Ella Brennan Still Feeds (and Leads) New
Orleans}\label{at-91-ella-brennan-still-feeds-and-leads-new-orleans}}

\includegraphics{https://static01.graylady3jvrrxbe.onion/images/2017/03/29/dining/29Brennan1/29Brennan1-articleInline.jpg?quality=75\&auto=webp\&disable=upscale}

By Brett Anderson

\begin{itemize}
\item
  March 27, 2017
\item
  \begin{itemize}
  \item
  \item
  \item
  \item
  \item
  \item
  \end{itemize}
\end{itemize}

NEW ORLEANS --- A typical evening for Ella Brennan begins with
cocktails. They are delivered from
\href{http://www.commanderspalace.com/}{Commander's Palace}, her
family's restaurant next door to her mansion in this city's historic
Garden District.

Drinks are often followed by dinner, also from Commander's, and wine.
Ms. Brennan favors Champagne and distinguished whites from the Côte de
Beaune because, as she puts it, ``I'm too damn old to drink cheap wine''
and ``because I can.''

In New Orleans, there is little question that Miss Ella, as Ms. Brennan
is widely known here, has earned her sense of entitlement. Her
hometown's civic pride is tightly linked to its reputation for culinary
excellence; Ms. Brennan helped create that reputation, and has made
maintaining it a personal responsibility for more than 60 years.

At 91, she is the matriarch of an extended family of restaurateurs that
employs nearly 1,400 people, both full and part time, and she has
mentored countless talents, including the celebrity chefs Paul Prudhomme
and Emeril Lagasse.

``She was there to embrace and elevate, not just me but the entire
staff,'' said Mr. Lagasse, whom Ms. Brennan plucked from obscurity and
installed, at age 23, as Commander's chef in the early 1980s. ``The list
of people she has impacted in the hospitality industry in America is
endless.''

Ms. Brennan's family has recently set out to ensure that her legacy is
appreciated beyond New Orleans. Those efforts culminated last year with
the release of a memoir,
\href{http://www.gibbs-smith.com/Miss-Ella-of-Commanders-Palace-P1943.aspx}{``Miss
Ella of Commander's Palace,''} written with Ti Martin, her daughter. It
was followed by the documentary
\href{http://www.ellabrennanmovie.com/}{``Ella Brennan: Commanding the
Table,''} directed by Leslie Iwerks. The film, which has been screening
at festivals, will be available on Netflix in May.

The book and film tell the story of a hard-charging, no-nonsense
businesswoman --- her other nickname is Hurricane Ella --- who excelled
in a male-dominated profession decades before gender imbalance in the
restaurant industry became a hot issue.

``When I first started, you would hardly ever see another woman in the
kitchen, much less running the show,'' said Sue Zemanick, the former
chef at \href{http://www.gautreausrestaurant.com/}{Gautreau's}, a ``Top
Chef Masters'' contestant and a line cook at Commander's early in her
career. ``Working at a restaurant with such strong women in charge gave
me hope that I could make it.''

\includegraphics{https://static01.graylady3jvrrxbe.onion/images/2017/03/29/dining/29Brennan2/29Brennan2-articleLarge.jpg?quality=75\&auto=webp\&disable=upscale}

While Ms. Brennan is, as Ms. Martin put it, ``as retired as Mom is
capable of being,'' her New Orleans restaurants, which include
\href{http://www.cafeadelaide.com/}{Café Adelaide} and
\href{http://www.sobounola.com/}{SoBou}, remain an obsession for her;
they are managed by Ms. Martin and Lally Brennan, a niece of Ms.
Brennan.

On a recent afternoon, Ms. Brennan sat next to the unlit fireplace in
her living room. As she moved a sore leg back and forth between an
ottoman and the floor, she suggested that her impulse to empower
employees was a rejection of restaurant industry norms she confronted as
a young woman.

``In those days, no one was paying attention to developing people,'' she
said. ``A restaurateur has to be part of a team to make something
everyone can be proud of.''

Her decision to tap Mr. Prudhomme, a Cajun, to run Commander's kitchen
in the 1970s loosened the grip that European-born chefs had on American
fine dining. Alongside Ms. Brennan, and later on his own, the
media-savvy Mr. Prudhomme helped set the table for a renaissance in
American regional cooking that has yet to abate.

``New Orleans was really the centerpiece of the whole American food
movement,'' the New York restaurateur Drew Nieporent said. ``And Ella
put New Orleans on the map.''

Ms. Brennan entered the hospitality business as a teenager working at
\href{http://www.ruebourbon.com/oldabsinthehouse/}{the Old Absinthe
House}, a Bourbon Street bar owned by her brother Owen, 15 years her
senior. She had dropped out of a local business school after deciding,
as she wrote in her book, that ``I wasn't going to type for any man.''

In 1946, Mr. Brennan bought Vieux Carré, a French Quarter restaurant
that Ms. Brennan described recently, with typical candor, as
``terrible.'' Mr. Brennan hired her to manage the business, looking to
prove that an Irish family could operate a restaurant superior to
established French-Creole restaurants like Arnaud's, Antoine's and
Galatoire's.

``I didn't know anything,'' Ms. Brennan recalled. ``But Owen was a
raconteur. He slept till noon. He got me to do all of the things he
didn't like. So I learned.''

A voracious reader, she pored over books recommended by the restaurant's
two chefs, Jack Eames and Paul Blangé, whose respect she sought and whom
she frequently praised. ``I was amazed at their talent,'' she said.
``They were like surgeons.''

The gregarious Brennan family --- the other siblings were Dick,
Adelaide, Dottie and John --- cultivated a clientele of influential
locals and visitors, including celebrities. All the while, Ms. Brennan
worked with the chefs to elevate Vieux Carré's food. For inspiration,
she traveled frequently to New York City, where she became a regular at
the ``21'' Club. Culinary pioneers like James Beard and Helen McCully,
the food editor of McCall's magazine, took the young visitor under their
wings.

``You had to go to New York in those days,'' Ms. Brennan said. ``I was
trying to get New Orleans to that level.''

In 1955, the Brennans were preparing to relocate Vieux Carré to a larger
space on Royal Street when Owen died of a heart attack at 45. It fell to
Ms. Brennan to carry out her brother's vision for the new restaurant,
which would be called
\href{https://www.nytimes3xbfgragh.onion/2015/01/21/dining/brennans-in-new-orleans-walks-the-tightrope-of-tradition.html}{Brennan's}.

By all accounts, Brennan's ushered in a new age for fine dining in the
South. It became a grand showcase for New Orleans joie de vivre, replete
with a wine cellar, antiques and food that stretched the boundaries of
traditional French-Creole cuisine. The national press took notice,
especially of its boozy, multicourse breakfasts.

``America was coming to appreciate dining as entertainment,'' Ms.
Brennan wrote, ``and all of the newspapers were just beginning to have
columns about food.''

Ms. Brennan helped develop new dishes, including bananas Foster. She
pushed to expand the family's restaurant properties, with spotty
success, aggravating a rift between Owen's heirs and what came to be
known as ``Ella's side'' of the family.

In 1973, Ms. Brennan was dismissed from Brennan's, initiating a bitter,
litigious family split that has never fully healed. ``It was a
tragedy,'' Ms. Brennan said. ``Family is everything to us.''

Image

Ms. Brennan at Commander's Palace. Guided by her, it became known for
its sophisticated blending of south Louisiana and nouvelle cuisines, and
for its celebratory approach to dining.Credit...Edmund D. Fountain for
The New York Times

(Owen Brennan's sons continued to run Brennan's until 2013, when a group
that includes Ralph Brennan, a son of John Brennan, purchased the
debt-riddled property at a sheriff's auction. Brennan's of Houston,
overseen by Ms. Brennan's son Alex, is part of her restaurant group.)

Commander's Palace, which Ms. Brennan acquired in 1969, was a sprawling
property in need of repair. In partnership with her sisters and
brothers, she set about turning it into a restaurant that would eclipse
Brennan's. Early reviews weren't favorable. But over time, Commander's
became renowned for its sophisticated blending of south Louisiana and
nouvelle cuisines --- Ms. Brennan christened the style haute Creole ---
and its warm, celebratory approach to fine dining.

``We want people who eat here to feel important, and we want them to
have fun,'' Ms. Brennan said.

Ms. Brennan never presumed that her management responsibilities ended at
the kitchen door. Alex McCrery, a Commander's line cook in the early
2000s, recalled that Ms. Brennan admonished another cook for proposing
that the kitchen make mustard ice cream.

``She was the type who was like your mom is when she is disappointed in
you,'' Mr. McCrery said. ``She was stern, and you'd feel like you really
messed up.''

Ms. Brennan said she ``gave up men for Lent'' after her divorce in 1970
from Paul Martin. She also never learned to cook (``I don't think she
can boil water,'' Mr. Lagasse said), but that doesn't prevent her from
giving lavish dinner parties in the home she shares with Dottie.

``We don't carry on like we once did,'' she said before a recent meal.
``But I do like to take advantage of living next door to my
restaurant.''

A waiter carried Ms. Brennan's drink as she moved, with the aid of a
walker, from her book-lined parlor to her chandelier-lit dining room.
She sat at the head of the table as Commander's waiters delivered a
series of courses from the restaurant's kitchen.

As she sank her spoon into a steaming sea urchin and stone crab soufflé,
Ms. Brennan explained that entertaining gives her an opportunity to
weigh in on new ideas from Tory McPhail, Commander's current and
longest-tenured chef. It also allows her to partake in the kind of
pleasure she has provided others for so long.

``What we're doing here tonight is letting him experiment on us,'' she
said, referring to Mr. McPhail, who had just introduced a course of lamb
osso buco. Ms. Brennan was as engaged and vivacious as when the evening
began.

As Mr. McPhail turned to leave, she called after him, ``Bring back a
bottle of Champagne, will you?''

Recipe:
\href{http://cooking.nytimes3xbfgragh.onion/recipes/1017148-bananas-foster}{\textbf{Bananas
Foster}}

Advertisement

\protect\hyperlink{after-bottom}{Continue reading the main story}

\hypertarget{site-index}{%
\subsection{Site Index}\label{site-index}}

\hypertarget{site-information-navigation}{%
\subsection{Site Information
Navigation}\label{site-information-navigation}}

\begin{itemize}
\tightlist
\item
  \href{https://help.nytimes3xbfgragh.onion/hc/en-us/articles/115014792127-Copyright-notice}{©~2020~The
  New York Times Company}
\end{itemize}

\begin{itemize}
\tightlist
\item
  \href{https://www.nytco.com/}{NYTCo}
\item
  \href{https://help.nytimes3xbfgragh.onion/hc/en-us/articles/115015385887-Contact-Us}{Contact
  Us}
\item
  \href{https://www.nytco.com/careers/}{Work with us}
\item
  \href{https://nytmediakit.com/}{Advertise}
\item
  \href{http://www.tbrandstudio.com/}{T Brand Studio}
\item
  \href{https://www.nytimes3xbfgragh.onion/privacy/cookie-policy\#how-do-i-manage-trackers}{Your
  Ad Choices}
\item
  \href{https://www.nytimes3xbfgragh.onion/privacy}{Privacy}
\item
  \href{https://help.nytimes3xbfgragh.onion/hc/en-us/articles/115014893428-Terms-of-service}{Terms
  of Service}
\item
  \href{https://help.nytimes3xbfgragh.onion/hc/en-us/articles/115014893968-Terms-of-sale}{Terms
  of Sale}
\item
  \href{https://spiderbites.nytimes3xbfgragh.onion}{Site Map}
\item
  \href{https://help.nytimes3xbfgragh.onion/hc/en-us}{Help}
\item
  \href{https://www.nytimes3xbfgragh.onion/subscription?campaignId=37WXW}{Subscriptions}
\end{itemize}
