\href{/section/business/media}{Media}\textbar{}Bill O'Reilly Thrives at
Fox News, Even as Harassment Settlements Add Up

\url{https://nyti.ms/2om4mG5}

\begin{itemize}
\item
\item
\item
\item
\item
\item
\end{itemize}

\includegraphics{https://static01.graylady3jvrrxbe.onion/images/2017/03/31/business/31FOX1/31FOX1-articleLarge.jpg?quality=75\&auto=webp\&disable=upscale}

Sections

\protect\hyperlink{site-content}{Skip to
content}\protect\hyperlink{site-index}{Skip to site index}

\hypertarget{bill-oreilly-thrives-at-fox-news-even-as-harassment-settlements-add-up}{%
\section{Bill O'Reilly Thrives at Fox News, Even as Harassment
Settlements Add
Up}\label{bill-oreilly-thrives-at-fox-news-even-as-harassment-settlements-add-up}}

About \$13 million has been paid out over the years to address
complaints from women about Mr. O'Reilly's behavior. He denies the
claims have merit.

Bill O'Reilly, Fox News's top-rated host, has faced a series of sexual
harassment allegations going back years.Credit...Robert Wright for The
New York Times

Supported by

\protect\hyperlink{after-sponsor}{Continue reading the main story}

By \href{https://www.nytimes3xbfgragh.onion/by/emily-steel}{Emily Steel}
and
\href{http://www.nytimes3xbfgragh.onion/by/michael-s-schmidt}{Michael S.
Schmidt}

\begin{itemize}
\item
  April 1, 2017
\item
  \begin{itemize}
  \item
  \item
  \item
  \item
  \item
  \item
  \end{itemize}
\end{itemize}

For nearly two decades, Bill O'Reilly has been Fox News's top asset,
building
\href{http://www.adweek.com/tvnewser/the-top-cable-news-programs-of-2016-were/315007}{the
No. 1 program in cable news} for a network that has pulled in billions
of dollars in revenues for its parent company, 21st Century Fox.

Behind the scenes, the company has repeatedly stood by Mr. O'Reilly as
he faced a series of allegations of sexual harassment or other
inappropriate behavior.

An investigation by The New York Times has found a total of five women
who have received payouts from either Mr. O'Reilly or the company in
exchange for agreeing to not pursue litigation or speak about their
accusations against him. The agreements totaled about \$13 million.

Two settlements came after the network's former chairman, Roger Ailes,
was dismissed last summer in the wake of a sexual harassment scandal,
when the company said it did not tolerate behavior that ``disrespects
women or contributes to an uncomfortable work environment.''

The women who made allegations against Mr. O'Reilly either worked for
him or appeared on his show. They have complained about a wide range of
behavior, including verbal abuse, lewd comments, unwanted advances and
phone calls in which it sounded as if Mr. O'Reilly was masturbating,
according to documents and interviews.

The reporting suggests a pattern: As an influential figure in the
newsroom, Mr. O'Reilly would create a bond with some women by offering
advice and promising to help them professionally. He then would pursue
sexual relationships with them, causing some to fear that if they
rebuffed him, their careers would stall.

Of the five settlements, two were previously known --- one for about \$9
million in 2004 with a producer, and another struck last year with a
former on-air personality, which The Times
\href{https://www.nytimes3xbfgragh.onion/2017/01/10/business/media/bill-oreilly-sexual-harassment-fox-news-juliet-huddy.html}{reported
on in January}. The Times has learned new details related to those
cases.

The three other settlements were uncovered by The Times. Two involved
sexual harassment claims against Mr. O'Reilly, and the other was for
verbal abuse related to an episode in which he berated a young producer
in front of newsroom colleagues.

Besides the women who reached settlements, two other women have spoken
of inappropriate behavior by the host. A former regular guest on his
show, Wendy Walsh, told The Times that after she rebuffed an advance
from him he didn't follow through on a verbal offer to secure her a
lucrative position at the network. And a former Fox News host named
Andrea Tantaros said Mr. O'Reilly sexually harassed her
\href{https://www.nytimes3xbfgragh.onion/2016/08/23/business/media/andrea-tantaros-of-fox-news-claims-retaliation-for-harassment-complaints.html}{in
a lawsuit she filed} last summer against the network and Mr. Ailes.

\includegraphics{https://static01.graylady3jvrrxbe.onion/images/2017/03/31/business/31FOX3/31FOX3-articleLarge.jpg?quality=75\&auto=webp\&disable=upscale}

Representatives for 21st Century Fox would not discuss specific
accusations against Mr. O'Reilly, but in a written statement to The
Times the company acknowledged it had addressed the issue with him.

``21st Century Fox takes matters of workplace behavior very seriously,''
the statement said. ``Notwithstanding the fact that no current or former
Fox News employee ever took advantage of the 21st Century Fox hotline to
raise a concern about Bill O'Reilly, even anonymously, we have looked
into these matters over the last few months and discussed them with Mr.
O'Reilly. While he denies the merits of these claims, Mr. O'Reilly has
resolved those he regarded as his personal responsibility. Mr. O'Reilly
is fully committed to supporting our efforts to improve the environment
for all our employees at Fox News.''

According to legal experts, companies occasionally settle disputes that
they believe have little merit because it is less risky than taking the
matters to trial, which can be costly and create a string of
embarrassing headlines.

The revelations about Mr. O'Reilly, 67, come after sexual harassment
accusations against Mr. Ailes led to an
\href{https://www.nytimes3xbfgragh.onion/2016/07/21/business/media/as-an-internal-inquiry-sinks-ailes-questions-about-fox-newss-fate.html}{internal
investigation} that found women at Fox News
\href{https://www.nytimes3xbfgragh.onion/2016/07/24/business/at-fox-news-kisses-innuendo-propositions-and-fears-of-reprisal.html}{faced
harassment}. Current and former Fox News employees told The Times that
they feared making complaints to network executives or the human
resources department.

Mr. Ailes, who has denied the allegations against him, received \$40
million as part of his
\href{https://www.nytimes3xbfgragh.onion/2016/07/22/business/media/roger-ailes-fox-news.html}{exit
package}. The company has reached settlements with at least six women
who accused Mr. Ailes of sexual harassment, according to a person
briefed on the agreements.

At the time of Mr. Ailes's departure, 21st Century Fox's top executives,
James and Lachlan Murdoch, the sons of the executive chairman, Rupert
Murdoch, said the company was committed to ``maintaining a work
environment based on trust and respect.''

Since then, the company has struck two settlements involving Mr.
O'Reilly, and learned of one Mr. O'Reilly reached secretly in 2011.

The company declined to answer questions about whether Mr. O'Reilly had
ever been disciplined.

Mr. O'Reilly has thrived since joining Fox News in 1996. He earns an
annual salary of about \$18 million as the host of ``The O'Reilly
Factor.'' Every weeknight at 8 p.m., he presents a pugnacious,
anti-political-correctness viewpoint and a fervent strain of patriotism
that appeals to conservative viewers.

His value to the company is enormous. From 2014 through 2016, the show
generated more than \$446 million in advertising revenues, according to
the research firm Kantar Media.

This is a sensitive time for Fox News as it continues to deal with the
fallout of the Ailes scandal. The network is facing an
\href{https://www.nytimes3xbfgragh.onion/2017/02/15/business/media/fox-news-sexual-harassment-payments.html}{investigation
by the United States attorney's office} in Manhattan, which is looking
into how the company structured settlements. Fox News has said that
neither it nor 21st Century Fox has received a subpoena but that they
have ``been in communication with the U.S. attorney's office for
months.''

Details on the allegations against Mr. O'Reilly and the company's
handling of them are based on more than five dozen interviews with
current and former employees of Fox News and its former and current
parent companies, News Corporation and 21st Century Fox; representatives
for the network; and people close to Mr. O'Reilly and the women. Most
spoke on the condition of anonymity, citing confidentiality agreements
and fear of retaliation. The Times also examined more than 100 pages of
documents and court filings related to the complaints.

Ms. Walsh, the former guest on Mr. O'Reilly's show, said his offer to
make her a contributor never materialized after she declined an
invitation to go to his hotel suite after a dinner in 2013. ``I feel bad
that some of these old guys are using mating strategies that were
acceptable in the 1950s and are not acceptable now,'' she said. ``I hope
young men can learn from this.''

She said romantic relationships at the workplace ``should never happen
when there is an imbalance of power and colleagues shouldn't unwittingly
be manipulated into obtaining sex for somebody.''

Just over a week ago, Mr. O'Reilly hired the crisis communications
expert Mark Fabiani --- who worked in the Clinton White House --- to
respond to The Times. In a statement, Mr. O'Reilly suggested that his
prominence made him a target.

``Just like other prominent and controversial people,'' the statement
read, ``I'm vulnerable to lawsuits from individuals who want me to pay
them to avoid negative publicity. In my more than 20 years at Fox News
Channel, no one has ever filed a complaint about me with the Human
Resources Department, even on the anonymous hotline.

``But most importantly, I'm a father who cares deeply for my children
and who would do anything to avoid hurting them in any way. And so I
have put to rest any controversies to spare my children.

``The worst part of my job is being a target for those who would harm me
and my employer, the Fox News Channel. Those of us in the arena are
constantly at risk, as are our families and children. My primary efforts
will continue to be to put forth an honest TV program and to protect
those close to me.''

Fredric S. Newman, a lawyer for Mr. O'Reilly, said in a statement Friday
evening, ``We are now seriously considering legal action to defend Mr.
O'Reilly's reputation.''

\hypertarget{lurid-claims-burst-into-view}{%
\subsection{Lurid Claims Burst Into
View}\label{lurid-claims-burst-into-view}}

Fox News has been aware of complaints about inappropriate behavior by
Mr. O'Reilly since at least 2002, when Mr. O'Reilly stormed into the
newsroom and screamed at a young producer, according to current and
former employees, some of whom witnessed the incident.

Shortly thereafter, the woman, Rachel Witlieb Bernstein, left the
network with a payout and bound by a confidentiality agreement, people
familiar with the deal said. The exact amount she was paid is not known,
but it was far less than the other settlements. The case did not involve
sexual harassment.

Two years later, allegations about Mr. O'Reilly entered the public arena
in lurid fashion when a producer on his show, Andrea Mackris, then 33,
\href{http://www.nytimes3xbfgragh.onion/2004/10/14/business/media/accused-of-harassment-fox-star-sues-and-is-sued.html}{filed
a sexual harassment lawsuit} against him. In the suit, she said he had
told her to buy a vibrator, called her at times when it sounded as if he
was masturbating and described sexual fantasies involving her. Ms.
Mackris had recorded some of the conversations, people familiar with the
case said.

Ms. Mackris also said in the suit that Mr. O'Reilly, who was married at
the time (he and his wife divorced in 2011), threatened her, saying he
would make any woman who complained about his behavior ``pay so dearly
that she'll wish she'd never been born.''

Fox News and Mr. O'Reilly adopted an aggressive strategy that served as
a stark warning of what could happen to women if they came forward with
complaints, current and former employees told The Times.

Before Ms. Mackris even filed suit, Fox News and Mr. O'Reilly surprised
her with a pre-emptive suit of their own, asserting she was seeking to
extort \$60 million in return for not going public with ``scandalous and
scurrilous'' claims about him.

``This is the single most evil thing I have ever experienced, and I have
seen a lot,'' he said on his show the day both suits were filed. ``But
these people picked the wrong guy.''

A public relations firm was hired to help shape the narrative in Mr.
O'Reilly's favor, and the private investigator Bo Dietl was retained to
dig up information on Ms. Mackris. The goal was to depict her as a
promiscuous woman, deeply in debt, who was trying to shake down Mr.
O'Reilly, according to people briefed on the strategy. Several
unflattering stories about her appeared in the tabloids.

After two weeks of sensational headlines, the two sides settled, and Mr.
O'Reilly agreed to pay Ms. Mackris about \$9 million, according to
people briefed on the agreement. The parties agreed to issue a public
statement that ``no wrongdoing whatsoever'' had occurred.

Image

Andrea Mackris, a former producer at Fox News, sued Mr. O'Reilly in 2004
for sexual harassment. The case was settled for around \$9
million.Credit...Christopher Gregory for The New York Times

\hypertarget{settling-behind-closed-doors}{%
\subsection{Settling Behind Closed
Doors}\label{settling-behind-closed-doors}}

In the years that followed, Mr. O'Reilly and Fox News dealt with sexual
harassment allegations in private, striking agreements with three more
women.

In 2011, Rebecca Gomez Diamond, who had hosted a show on the Fox
Business Network --- also supervised by Mr. Ailes --- was told the
network was not renewing her contract. Similar to Ms. Mackris, she had
recorded conversations with Mr. O'Reilly, according to people familiar
with the case. Armed with the recordings, her lawyers went to the
company and outlined her complaints against him.

Ms. Diamond left the network, bound by a confidentiality agreement, and
Mr. O'Reilly paid the settlement, two of the people said. The exact
amount of the payout is not known.

Although that deal was made nearly six years ago, Fox News's parent
company, 21st Century Fox, learned of it only in late 2016 when it
conducted an investigation into Fox News under Mr. Ailes's tenure,
according to another person familiar with the matter.

In the aftermath of Mr. Ailes's ouster last summer, as 21st Century Fox
was completing settlements and trying to put the scandal behind it, it
reached deals with two women who had complained about sexual harassment
by Mr. O'Reilly.

One was Laurie Dhue, a Fox News anchor from 2000 to 2008. Though Ms.
Dhue had not raised sexual harassment issues during her tenure or upon
her departure, her lawyers went to the company to outline her harassment
claims against Mr. O'Reilly and Mr. Ailes, according to people briefed
on the complaints. In response, 21st Century Fox reached a settlement
with her for over \$1 million, according to a person briefed on the
agreement.

In September, 21st Century Fox reached a settlement worth \$1.6 million
with Juliet Huddy, who had made regular appearances on Mr. O'Reilly's
show, according to people familiar with the matter. Ms. Huddy's lawyers
had told the company that Mr. O'Reilly pursued a sexual relationship in
2011, at a time he exerted significant influence over her airtime.

Among Ms. Huddy's complaints was that he made inappropriate phone calls,
the lawyers said in correspondence obtained by The Times. The letter
said that when he tried to kiss her, she pulled away and fell to the
ground and he didn't help her up.

When she rebuffed him, he tried to blunt her career prospects, the
letter said.

Ms. Huddy was eventually moved to an early morning show on WNYW, an
affiliate station, where she worked until she left the company in
September.

Image

Roger Ailes, the former chairman of Fox News, was ousted last summer
after Gretchen Carlson, an anchor, accused him of sexual harassment. The
network's parent company later paid \$20 million to settle the
suit.Credit...Reed Saxon/Associated Press

Before Ms. Huddy reached an agreement with 21st Century Fox, Mr. Newman,
Mr. O'Reilly's lawyer, sent a letter to her lawyer outlining some
embarrassing personal issues he said Ms. Huddy had. He stated that she
would ``face significant credibility concerns if she tries to pursue a
claim against Mr. O'Reilly.'' The letter, which was obtained by The
Times, said that if she were to follow through with a claim against Mr.
O'Reilly, he would pursue legal action ``to hold Ms. Huddy, and all who
have assisted her, personally liable for any damage suffered by him or
his family.''

In January, when The Times and others reported on Ms. Huddy's
settlement, representatives for Fox News and Mr. O'Reilly dismissed the
allegations.

Fox News is now in a legal battle with Ms. Tantaros, the former on-air
personality who is suing the network and Mr. Ailes after turning down a
settlement offer of nearly \$1 million. Mr. O'Reilly is not a defendant,
but in the suit Ms. Tantaros said that in early 2016 Mr. O'Reilly had
asked ``her to come to stay with him on Long Island where it would be
`very private,''' and told her ``on more than one occasion that he could
`see {[}her{]} as a wild girl,''' according to court documents.

In an affidavit filed under oath, Ms. Tantaros's psychologist, Michele
Berdy, who treated her from 2013 to 2016, said she recalled ``a number
of occasions when Andrea complained to me
\href{http://www.nydailynews.com/entertainment/tv/tantaros-ex-therapist-backs-ailes-sexual-harassment-claims-article-1.2810182}{about
recurring unwanted advances} from Bill O'Reilly.''

Fox News said it investigated Ms. Tantaros's claims and found them
baseless. The company explained her departure by saying she published a
book that violated company policy. In court papers, the network said
that she ``is not a victim; she is an opportunist'' and that her
allegations bore ``all the hallmarks of the wannabe.''

Ms. Walsh, the former guest on ``The O'Reilly Factor,'' told The Times
she was propositioned by Mr. O'Reilly in 2013 but did not lodge a
complaint because she did not want to harm her career prospects.

Ms. Walsh said that she met Mr. O'Reilly for a dinner, arranged by his
secretary, at the restaurant in the Hotel Bel-Air in Los Angeles. During
the dinner, she said, he told her he was friends with Mr. Ailes, and
promised to make her a network contributor --- a job that can pay
several hundred thousand dollars a year.

After dinner, she said, Mr. O'Reilly invited her to his hotel suite. Ms.
Walsh said she declined. Trying to remain cordial, she suggested that
they go to the hotel bar instead. Once there, she said, he became
hostile, telling her that she could forget any career advice he had
given her and that she was on her own. He also told her that her black
leather purse was ugly.

Ms. Walsh continued to appear on his show for about four months, but she
said she sensed that he had become cold toward her on camera. Then, a
producer for ``The O'Reilly Factor'' told Ms. Walsh that she would no
longer appear on the show. She was never made a contributor.

Image

Mr. O'Reilly at his desk at Fox News studios. He has contended that,
like other prominent people, he has been the target of people seeking
payouts to avoid negative publicity.Credit...Richard Perry/The New York
Times

``I knew my hopes of a career at Fox News were in jeopardy after that
evening,'' said Ms. Walsh, now an adjunct professor of psychology at
California State University, Channel Islands, and a radio host at KFI AM
640 in Los Angeles.

A person briefed on the network's decision said that Ms. Walsh was
removed from the broadcast because the program's ratings declined during
her segments.

\hypertarget{shadowing-anothers-exile}{%
\subsection{Shadowing Another's Exile}\label{shadowing-anothers-exile}}

Ms. Mackris, the producer who sued Mr. O'Reilly in 2004, never worked in
television news again.

In the years after the dispute, she suffered from post-traumatic stress
and spent years seeing a therapist, struggling to figure out how to
create a new life, according to interviews with people close to her at
the time.

Ms. Mackris's settlement prevents her from talking about Fox News and
her dispute with Mr. O'Reilly, according to people briefed on the deal.
But she is allowed to talk about her life now.

Today, Ms. Mackris lives with her cats in an art-filled condo in her
hometown, St. Louis, where she keeps bowls of colorful gumballs on
tabletops. Her family is close by. She has traveled the world,
volunteered, returned to school, discovered prayer and meditation, and
started writing.

She is working on a book she researched and wrote over the past four
years about a woman who fled Romania during World War II.

``A few years ago, I heard about a pair of
\href{http://www.itv.com/news/meridian/2012-04-25/royal-earrings-set-to-fetch-600-000/}{natural
pearl earrings forgotten in a drawer} for 35 years that had just sold
for millions at auction,'' Ms. Mackris said. ``They'd been given to a
woman named Elena Lupescu by the king of Romania who ruled up until
World War II, and I was immediately and completely taken by her story.''

``She lived in exile,'' Ms. Mackris continued. ``She lived in silence.
And I got really curious about three things: How did she live with it
all? Did she forgive them? And was she free?''

At Fox News, Mr. O'Reilly has continued his dominance. In the months
since the presidential election, as the network has pulled in record
ratings, his show has averaged 3.9 million viewers a night, according to
Nielsen. Since September, he has released three books, including one for
children, adding to his growing publishing empire. And in February, Mr.
O'Reilly landed a coveted interview with President Trump before the
Super Bowl.

Mr. O'Reilly was an early defender of Mr. Ailes and Fox News during that
sexual harassment scandal last summer. His support remained resolute
into the fall, after the company had reached agreements to settle the
harassment claims from Ms. Huddy and Ms. Dhue. In November, he chided
Megyn Kelly, his colleague at the time, after she described being
sexually harassed by Mr. Ailes in her memoir.

``If somebody is paying you a wage, you owe that person or company
allegiance,'' \href{https://www.youtube.com/watch?v=w8qvTIAiax0}{he said
on his nightly show}, without mentioning Ms. Kelly by name. ``You don't
like what's happening in the workplace, go to human resources or
leave.''

Advertisement

\protect\hyperlink{after-bottom}{Continue reading the main story}

\hypertarget{site-index}{%
\subsection{Site Index}\label{site-index}}

\hypertarget{site-information-navigation}{%
\subsection{Site Information
Navigation}\label{site-information-navigation}}

\begin{itemize}
\tightlist
\item
  \href{https://help.nytimes3xbfgragh.onion/hc/en-us/articles/115014792127-Copyright-notice}{©~2020~The
  New York Times Company}
\end{itemize}

\begin{itemize}
\tightlist
\item
  \href{https://www.nytco.com/}{NYTCo}
\item
  \href{https://help.nytimes3xbfgragh.onion/hc/en-us/articles/115015385887-Contact-Us}{Contact
  Us}
\item
  \href{https://www.nytco.com/careers/}{Work with us}
\item
  \href{https://nytmediakit.com/}{Advertise}
\item
  \href{http://www.tbrandstudio.com/}{T Brand Studio}
\item
  \href{https://www.nytimes3xbfgragh.onion/privacy/cookie-policy\#how-do-i-manage-trackers}{Your
  Ad Choices}
\item
  \href{https://www.nytimes3xbfgragh.onion/privacy}{Privacy}
\item
  \href{https://help.nytimes3xbfgragh.onion/hc/en-us/articles/115014893428-Terms-of-service}{Terms
  of Service}
\item
  \href{https://help.nytimes3xbfgragh.onion/hc/en-us/articles/115014893968-Terms-of-sale}{Terms
  of Sale}
\item
  \href{https://spiderbites.nytimes3xbfgragh.onion}{Site Map}
\item
  \href{https://help.nytimes3xbfgragh.onion/hc/en-us}{Help}
\item
  \href{https://www.nytimes3xbfgragh.onion/subscription?campaignId=37WXW}{Subscriptions}
\end{itemize}
