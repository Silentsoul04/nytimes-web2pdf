Sections

SEARCH

\protect\hyperlink{site-content}{Skip to
content}\protect\hyperlink{site-index}{Skip to site index}

\href{https://www.nytimes3xbfgragh.onion/section/politics}{Politics}

\href{https://myaccount.nytimes3xbfgragh.onion/auth/login?response_type=cookie\&client_id=vi}{}

\href{https://www.nytimes3xbfgragh.onion/section/todayspaper}{Today's
Paper}

\href{/section/politics}{Politics}\textbar{}Trump, Citing No Evidence,
Suggests Susan Rice Committed Crime

\url{https://nyti.ms/2nYhOgv}

\begin{itemize}
\item
\item
\item
\item
\item
\end{itemize}

Advertisement

\protect\hyperlink{after-top}{Continue reading the main story}

Supported by

\protect\hyperlink{after-sponsor}{Continue reading the main story}

\hypertarget{trump-citing-no-evidence-suggests-susan-rice-committed-crime}{%
\section{Trump, Citing No Evidence, Suggests Susan Rice Committed
Crime}\label{trump-citing-no-evidence-suggests-susan-rice-committed-crime}}

\includegraphics{https://static01.graylady3jvrrxbe.onion/images/2017/04/06/us/06trump1/06trump1-articleLarge.jpg?quality=75\&auto=webp\&disable=upscale}

By \href{http://www.nytimes3xbfgragh.onion/by/maggie-haberman}{Maggie
Haberman},
\href{http://www.nytimes3xbfgragh.onion/by/matthew-rosenberg}{Matthew
Rosenberg} and
\href{https://www.nytimes3xbfgragh.onion/by/glenn-thrush}{Glenn Thrush}

\begin{itemize}
\item
  April 5, 2017
\item
  \begin{itemize}
  \item
  \item
  \item
  \item
  \item
  \end{itemize}
\end{itemize}

WASHINGTON --- President Trump said on Wednesday that Susan E. Rice, the
former national security adviser, may have committed a crime by seeking
to learn the identities of Trump associates swept up in surveillance of
foreign officials by United States spy agencies, repeating an assertion
his allies in the news media have been making since last week.

Mr. Trump gave no evidence to support his claim, and current and former
intelligence officials from both Republican and Democratic
administrations have said they do not believe Ms. Rice's actions were
unusual or unlawful. The president repeatedly rebuffed attempts by two
New York Times reporters to learn more about what led him to the
conclusion, saying he would talk more about it ``at the right time.''

The allegation by a sitting president was a remarkable escalation ---
and, his critics say, the latest effort to change the story at a time
when his nascent administration has been consumed by questions about any
role his associates may have played in a Russian campaign to disrupt
last year's presidential election.

Since March 4, when Mr. Trump posted on Twitter that President Barack
Obama had ``wiretapped'' him at Trump Tower during the campaign, the
president and his allies have repeatedly sought evidence trying to
corroborate that claim, despite flat denials from James B. Comey, the
director of the F.B.I., and other senior intelligence officials.

Wednesday's interview revealed how Mr. Trump seizes on claims made by
the conservative news media, from fringe outlets to Fox News, and gives
them a presidential stamp of approval and also increases their reach.

Last week, some Republican television commentators asserted that Ms.
Rice had improperly leaked the names of Trump associates picked up in
surveillance of foreign officials. On Sunday, a conservative writer and
conspiracy theorist reported, without identifying his sources, that Ms.
Rice had been the one to seek identities of the Trump associates.

Other conservative outlets picked up the report, and the Drudge Report
website, which has been supportive of Mr. Trump, featured the story
prominently. White House officials then accused mainstream news outlets
of not giving the story proper coverage.

The interview with The Times was supposed to be focused on Mr. Trump's
plans for large-scale spending on the nation's infrastructure. But
moments after it began, the president began talking about Ms. Rice.

``I think the Susan Rice thing is a massive story. I think it's a
massive, massive story. All over the world,'' Mr. Trump said.

``It's a bigger story than you know,'' the president added cryptically,
also saying that new information would emerge ``in terms of what other
people have done also.''

``The Russia story is a total hoax. There has been absolutely nothing
coming out of that,'' he said.

Turning the subject to Ms. Rice, the president said: ``What's happened
is terrible. I've never seen people so indignant, including many
Democrats who are friends of mine.''

Through a spokeswoman, Ms. Rice said, ``I'm not going to dignify the
president's ludicrous charge with a comment.'' In an interview with
MSNBC on Tuesday, Ms. Rice said she had done nothing wrong.

``The allegation is that somehow the Obama administration officials
utilized intelligence for political purposes,'' Ms. Rice said. ``That's
absolutely false.''

Normally, when Americans are swept up in surveillance of foreign
officials by intelligence agencies, their identities are supposed to be
obscured. But they can be revealed --- or ``unmasked'' --- for national
security reasons, and intelligence officials say it is a regular
occurrence and completely legal for a national security adviser to
request the identities of Americans who are mentioned in intelligence
reports.

Intelligence officials said any requests that Ms. Rice made would have
had to be granted by the intelligence agency that produced the report.
In most cases, that would likely have been the National Security Agency,
which is responsible for electronic surveillance of foreign officials.

``Requests to learn the identity of a U.S. person were not routine, but
also not uncommon,'' said Stephen Slick, a retired C.I.A. official who
served as the senior director for intelligence at the National Security
Council under President George W. Bush.

Requesting that a name be revealed so that officials could ``make sense
of an intelligence report was a normal part of the daily intelligence
rhythm at the White House, State Department, Defense Department and
other national security agencies,'' added Mr. Slick, who is now the
director of the Intelligence Studies Project at the University of Texas
at Austin.

It could be a crime if Ms. Rice leaked the name of any American wrapped
up in the surveillance net, but she flatly denied doing so in her MSNBC
interview.

``I leaked nothing to nobody, and never have and never would,'' Ms. Rice
said.

The broader issue of how intelligence collected by the national security
apparatus is disseminated and used has long been an animating issue for
civil libertarians, a point that Mr. Trump made in the interview.

Revelations about American programs for intercepting and mining private
data made by Edward J. Snowden, a government contractor, proved deeply
embarrassing for the Obama administration in 2013.

Mr. Trump declined to say whether he would be willing to declassify some
of the information that has been at issue. He also did not explain what
he believed was unlawful in his estimation.

It is not the first time Mr. Trump has made a provocative allegation
without providing supporting evidence. One of the most notorious
instances of this was his yearslong claim that Mr. Obama was not born in
the United States.

Like his statements about Mr. Obama --- which he walked away from late
in the 2016 presidential campaign --- Mr. Trump's claims about Ms. Rice
have taken hold in the conservative news media, where she has been a
target ever since her press appearances after the terrorist attack on a
diplomatic outpost in Libya in September 2012.

Mr. Trump's March 4 Twitter message came after reports in conservative
news outlets --- including Breitbart, the website once run by the
president's chief strategist, Stephen K. Bannon --- claiming that there
had been surveillance of some kind against Mr. Trump when he was a
candidate.

Mr. Trump was widely criticized for the intemperate post, and he began
to ask his advisers about how he might be able to investigate the issue.

Weeks later, Representative Devin Nunes, the top Republican on the House
Intelligence Committee, told reporters that he had learned of new
information that Trump associates may have been surveilled in some way.
He rushed to the White House to brief the president, even though it was
later revealed that the information had come from White House officials.

Representative Adam Schiff of California, the top Democrat on the
Intelligence Committee, cast Mr. Trump's comments as part of a broader
effort by the president to distract from the investigations into
Russia's interference in the election. The committee is running one of
the investigations.

``He began by accusing President Obama of a crime without any
evidence,'' Mr. Schiff said. ``He's now moved on to accusing Susan Rice
of a crime without any evidence, and this is sadly how this president
operates.''

It ``would be a terrible way to do business,'' Mr. Schiff added. ``It's
a worse way to run a country.''

Advertisement

\protect\hyperlink{after-bottom}{Continue reading the main story}

\hypertarget{site-index}{%
\subsection{Site Index}\label{site-index}}

\hypertarget{site-information-navigation}{%
\subsection{Site Information
Navigation}\label{site-information-navigation}}

\begin{itemize}
\tightlist
\item
  \href{https://help.nytimes3xbfgragh.onion/hc/en-us/articles/115014792127-Copyright-notice}{©~2020~The
  New York Times Company}
\end{itemize}

\begin{itemize}
\tightlist
\item
  \href{https://www.nytco.com/}{NYTCo}
\item
  \href{https://help.nytimes3xbfgragh.onion/hc/en-us/articles/115015385887-Contact-Us}{Contact
  Us}
\item
  \href{https://www.nytco.com/careers/}{Work with us}
\item
  \href{https://nytmediakit.com/}{Advertise}
\item
  \href{http://www.tbrandstudio.com/}{T Brand Studio}
\item
  \href{https://www.nytimes3xbfgragh.onion/privacy/cookie-policy\#how-do-i-manage-trackers}{Your
  Ad Choices}
\item
  \href{https://www.nytimes3xbfgragh.onion/privacy}{Privacy}
\item
  \href{https://help.nytimes3xbfgragh.onion/hc/en-us/articles/115014893428-Terms-of-service}{Terms
  of Service}
\item
  \href{https://help.nytimes3xbfgragh.onion/hc/en-us/articles/115014893968-Terms-of-sale}{Terms
  of Sale}
\item
  \href{https://spiderbites.nytimes3xbfgragh.onion}{Site Map}
\item
  \href{https://help.nytimes3xbfgragh.onion/hc/en-us}{Help}
\item
  \href{https://www.nytimes3xbfgragh.onion/subscription?campaignId=37WXW}{Subscriptions}
\end{itemize}
