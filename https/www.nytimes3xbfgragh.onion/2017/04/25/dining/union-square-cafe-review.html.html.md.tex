Sections

SEARCH

\protect\hyperlink{site-content}{Skip to
content}\protect\hyperlink{site-index}{Skip to site index}

\href{https://www.nytimes3xbfgragh.onion/section/food}{Food}

\href{https://myaccount.nytimes3xbfgragh.onion/auth/login?response_type=cookie\&client_id=vi}{}

\href{https://www.nytimes3xbfgragh.onion/section/todayspaper}{Today's
Paper}

\href{/section/food}{Food}\textbar{}Union Square Cafe Returns, With More
Than a Dash of Déjà Vu

\url{https://nyti.ms/2q1FmC2}

\begin{itemize}
\item
\item
\item
\item
\item
\item
\end{itemize}

Advertisement

\protect\hyperlink{after-top}{Continue reading the main story}

Supported by

\protect\hyperlink{after-sponsor}{Continue reading the main story}

\href{/column/restaurant-review}{Restaurant Review}

\hypertarget{union-square-cafe-returns-with-more-than-a-dash-of-duxe9juxe0-vu}{%
\section{Union Square Cafe Returns, With More Than a Dash of Déjà
Vu}\label{union-square-cafe-returns-with-more-than-a-dash-of-duxe9juxe0-vu}}

\href{https://www.nytimes3xbfgragh.onion/slideshow/2017/04/25/dining/union-square-cafe.html}{}

\hypertarget{union-square-cafe}{%
\subsection{Union Square Cafe}\label{union-square-cafe}}

9 Photos

View Slide Show ›

\includegraphics{https://static01.graylady3jvrrxbe.onion/images/2017/04/26/dining/26REST-UNIONSQUARE-slide-LI52/26REST-UNIONSQUARE-slide-LI52-articleLarge.jpg?quality=75\&auto=webp\&disable=upscale}

Francesco Sapienza for The New York Times

\begin{itemize}
\tightlist
\item
  Union Square Cafe\\
  ★★★ American;Italian \$\$\$\$ 101 East 19th Street 212-243-4020
\end{itemize}

\href{https://resy.com/cities/ny/union-square-cafe?utm_source=nyt\&utm_medium=restoprofile\&utm_campaign=affiliates\&aff_id=c1fe784}{Reserve
a Table}

When you make a reservation at an independently reviewed restaurant
through our site, we earn an affiliate commission.

By \href{http://www.nytimes3xbfgragh.onion/by/pete-wells}{Pete Wells}

\begin{itemize}
\item
  April 25, 2017
\item
  \begin{itemize}
  \item
  \item
  \item
  \item
  \item
  \item
  \end{itemize}
\end{itemize}

Sitting at the upstairs bar, one of the best places to eat in the new
\href{http://www.unionsquarecafe.com/}{Union Square Cafe}, I listened in
on a conversation the woman next to me was having with the bartender. In
December, after a year out of operation, Danny Meyer's first and most
formative restaurant had been transferred from its original address on
East 16th Street to the corner of Park Avenue South and East 19th. The
woman had been transferred, too; a regular at the old location, she had
some pointed questions about a certain item on the menu.

``Is it the same?'' she asked. ``\emph{Exactly} the same?''

The bartender assured her that it was. ``If we ever took it off the
menu, we'd have a riot on our hands,'' he said.

The item in question was a salad of red-tipped oakleaf and Bibb
lettuces, with some fresh and springy croutons and whitecaps of grated
Gruyère on top. It is a very nice salad, a model of good behavior. It
does not bring to mind images of mob violence. This is Union Square
Cafe, though, a place that from the beginning has inspired attachments
of a peculiar intensity.

Other restaurants have customers; Union Square Cafe has fans, and knows
it. Almost everything about the new place caters to their memories of
the original, fetishizing the restaurant's own idiosyncrasies to an
amazing degree. It's almost a museum to itself. If you're not part of
the fan base, you might feel a bit like somebody who goes to see ``Rogue
One'' without ever having seen ``Star Wars.''

At the old location, I was more an admirer than a fan. I was never quite
as amazed by the food as it seemed I was supposed to be. But then,
fluctuation is normal in a long-running restaurant. Over the years,
reviews of Union Square Cafe in The Times have ranged from
\href{http://www.nytimes3xbfgragh.onion/1986/01/24/arts/restaurants-693586.html}{two}
to
\href{http://www.nytimes3xbfgragh.onion/1999/09/01/dining/restaurants-even-the-doggy-bags-really-care.html}{three
stars}. In
\href{http://www.nytimes3xbfgragh.onion/2009/08/05/dining/reviews/05rest.html}{the
most recent appraisal}, in 2009, Frank Bruni awarded it two.

\href{http://www.rockwellgroup.com/}{The Rockwell Group}, the
architecture firm, was given
\href{https://www.nytimes3xbfgragh.onion/2016/09/07/dining/union-square-cafe-restaurant-design-danny-meyer-david-rockwell.html}{the
job of evoking the old address} in a space that is much roomier, without
the narrow passageways and sunken dining room. Some of this is done
subliminally: Pendant lamps downstairs hang at the height of the 16th
Street ceilings. I could swear the original downstairs bar was smaller,
but that's probably because I could almost never get a seat there. Mr.
Rockwell swears that the new one is the same length: 27 feet 1 inch.

The upstairs bar top where my cocktail rested was precisely the size of
the old upstairs bar top because it was, in fact, the same piece of wood
with a fresh coat of varnish. The art collection has come along for the
ride, too, and one of the pleasures of the Rockwell layout is the way
you seem to bump into a Frank Stella or a Claes Oldenburg or a Judy
Rifka every time you turn around.

The cellar stocks wines from Italy, France and the United States, and
nowhere else. As Jason Wagner, the wine director, explained one night,
those are the countries that were on the list when Union Square Cafe
opened in 1985.

The menu does not go so far as to point out all the old showstoppers ---
some from Michael Romano's reign in the kitchen and others from that of
the current chef, Carmen Quagliata --- but the servers are happy to
help. Do I even need to say this? It's the most famous thing about Union
Square Cafe: The servers are always happy to help.

I hadn't realized the salad was a cult object, but I did know about the
gnocchi. In this case, I stand with the cult. They are not so much
gnocchi as little cushions of ricotta that have been tricked into
holding their cylindrical shape only for as long as it takes to move
them from the plate to the mouth.

\includegraphics{https://static01.graylady3jvrrxbe.onion/images/2017/04/26/dining/26RESTAURANT2/26RESTAURANT2-articleLarge.jpg?quality=75\&auto=webp\&disable=upscale}

Consider me a fanboy when it comes to the ideally crunchy fritto misto,
and the polenta, too, a \$13 bowl of warm fluff that has absorbed its
weight, and then some, in milk and creamy young cow's milk cheese. With
maitake mushrooms and shiny green pesto, it's so filling and likable
that you could make a meal of it and walk away more content than if
you'd had a 12-course tasting somewhere else.

The risk in churning out old recipes like this is that the kitchen
becomes bored, and the food boring. That's not happening at Union Square
Cafe right now; the cooks seem to have dialed right in on the qualities
that made these dishes favorites in the first place.

Union Square Cafe has always been the offspring of a mixed marriage
between a trattoria and a bistro, with an American bar and grill
somewhere in the family tree. In the current phenotype, the trattoria
genes are dominant.

There is a wonderfully tender and gentle rabbit sugo, tossed with bands
of pappardelle. Mr. Quagliata's pasta work can be just as satisfying
when he carves out new traditions, tossing tubes of rigatoni with
roasted carrots and sizzled scallions and chiles, then brightening the
flavors with a spoonful of yogurt.

A little less thrilling was a lasagna Bolognese that has been rotated
off the menu for the time being; while the pasta sheets were as thin and
light as any Italian could hope for, the flavors in the meat sauce
weren't as developed as they might have been.

Though I split that lasagna with somebody else so we could have
traditional appetizer-primi-secondi meals, the mains are so imposing
that a simple two-course plan is the wiser path. I was nearly done in by
a braised lamb shank, although the bright salsa verde helped spur me
along. Even the choices that sound lighter, like red snapper sandwiched
between a thatch of shaved puntarelle and a bed of chickpea purée, or a
flattened half-chicken, with cumin and dried chiles embedded in its
surface and slabs of white sweet potato peeking out from below, are best
tackled with an undented appetite.

After all: There will be dessert, and it will not be a minor event,
especially if somebody at your table has strong memories of the panna
cotta (beautiful; the pastry chef, Daniel Alvarez, surrounds it with
citrus segments and granola), or the pumpkin bread pudding (more bread
than pudding when I last saw it), or the fantastic tart topped with a
caramelized puff of banana so soft it seems to have blown apart.

Sometimes Union Square Cafe's eagerness to please has something close to
the opposite effect. When a server suggested with wide-eyed eagerness
that I might like ``a very cold glass of milk'' with the
espresso-chocolate cake, chills went down my spine, and not chills of
anticipation. The last time I drank milk with a meal, I was wearing
footed pajamas.

Then I'd remember how happy the service had made one of my guests, who
had shown up before me, but still about 30 seconds too late to grab a
seat at the downstairs bar. She spun around in agitation for a moment,
but a manager spotted her and offered to call upstairs to see if there
were any open bar stools there.

This empathic attunement to disturbances in the Force defines the Meyer
brand of hospitality. While it flows through
\href{http://www.ushgnyc.com/restaurants/}{all his restaurants}, it
functions in its purest and most effective form at Union Square Cafe.

The restaurant is far too nice to be polarizing, but it won't be
everybody's cold glass of milk. Diners from the Instagram generation
might want more crunch, more adventure or more spontaneity. I hope that
as it settles into its new home, it gets a little less self-referential.
But even now, it has a combination of energy and well-honed familiarity
that's rare in Manhattan. It's rare anywhere, and it lifts this
restaurant above fashion, to a plateau that it occupies all by itself.

\href{https://www.facebookcorewwwi.onion/nytfood/}{\emph{Follow NYT Food
on Facebook}}\emph{,}
\href{https://instagram.com/nytfood}{\emph{Instagram}}\emph{,}
\href{https://twitter.com/nytfood}{\emph{Twitter}} \emph{and}
\href{https://www.pinterest.com/nytfood/}{\emph{Pinterest}}\emph{.}
\href{https://www.nytimes3xbfgragh.onion/newsletters/cooking}{\emph{Get
regular updates from NYT Cooking, with recipe suggestions, cooking tips
and shopping advice}}\emph{.}

Advertisement

\protect\hyperlink{after-bottom}{Continue reading the main story}

\hypertarget{site-index}{%
\subsection{Site Index}\label{site-index}}

\hypertarget{site-information-navigation}{%
\subsection{Site Information
Navigation}\label{site-information-navigation}}

\begin{itemize}
\tightlist
\item
  \href{https://help.nytimes3xbfgragh.onion/hc/en-us/articles/115014792127-Copyright-notice}{©~2020~The
  New York Times Company}
\end{itemize}

\begin{itemize}
\tightlist
\item
  \href{https://www.nytco.com/}{NYTCo}
\item
  \href{https://help.nytimes3xbfgragh.onion/hc/en-us/articles/115015385887-Contact-Us}{Contact
  Us}
\item
  \href{https://www.nytco.com/careers/}{Work with us}
\item
  \href{https://nytmediakit.com/}{Advertise}
\item
  \href{http://www.tbrandstudio.com/}{T Brand Studio}
\item
  \href{https://www.nytimes3xbfgragh.onion/privacy/cookie-policy\#how-do-i-manage-trackers}{Your
  Ad Choices}
\item
  \href{https://www.nytimes3xbfgragh.onion/privacy}{Privacy}
\item
  \href{https://help.nytimes3xbfgragh.onion/hc/en-us/articles/115014893428-Terms-of-service}{Terms
  of Service}
\item
  \href{https://help.nytimes3xbfgragh.onion/hc/en-us/articles/115014893968-Terms-of-sale}{Terms
  of Sale}
\item
  \href{https://spiderbites.nytimes3xbfgragh.onion}{Site Map}
\item
  \href{https://help.nytimes3xbfgragh.onion/hc/en-us}{Help}
\item
  \href{https://www.nytimes3xbfgragh.onion/subscription?campaignId=37WXW}{Subscriptions}
\end{itemize}
