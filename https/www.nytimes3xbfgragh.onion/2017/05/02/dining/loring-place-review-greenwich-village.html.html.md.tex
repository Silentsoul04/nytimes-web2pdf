Sections

SEARCH

\protect\hyperlink{site-content}{Skip to
content}\protect\hyperlink{site-index}{Skip to site index}

\href{https://www.nytimes3xbfgragh.onion/section/food}{Food}

\href{https://myaccount.nytimes3xbfgragh.onion/auth/login?response_type=cookie\&client_id=vi}{}

\href{https://www.nytimes3xbfgragh.onion/section/todayspaper}{Today's
Paper}

\href{/section/food}{Food}\textbar{}`Vegetable Forward' Chef Gets His
Own Patch at Loring Place

\url{https://nyti.ms/2quqqwp}

\begin{itemize}
\item
\item
\item
\item
\item
\item
\end{itemize}

Advertisement

\protect\hyperlink{after-top}{Continue reading the main story}

Supported by

\protect\hyperlink{after-sponsor}{Continue reading the main story}

\href{/column/restaurant-review}{Restaurant Review}

\hypertarget{vegetable-forward-chef-gets-his-own-patch-at-loring-place}{%
\section{`Vegetable Forward' Chef Gets His Own Patch at Loring
Place}\label{vegetable-forward-chef-gets-his-own-patch-at-loring-place}}

\href{https://www.nytimes3xbfgragh.onion/slideshow/2017/05/02/dining/loring-place.html}{}

\hypertarget{loring-place}{%
\subsection{Loring Place}\label{loring-place}}

11 Photos

View Slide Show ›

\includegraphics{https://static01.graylady3jvrrxbe.onion/images/2017/05/03/dining/03REST-LORING-slide-WZ7T/03REST-LORING-slide-WZ7T-articleLarge.jpg?quality=75\&auto=webp\&disable=upscale}

Casey Kelbaugh for The New York Times

\begin{itemize}
\tightlist
\item
  Loring Place\\
  ★★ American \$\$\$ 21 West Eighth Street 212-388-1831
\end{itemize}

\href{https://resy.com/cities/ny/loring-place?utm_source=nyt\&utm_medium=restoprofile\&utm_campaign=affiliates\&aff_id=c1fe784}{Reserve
a Table}

When you make a reservation at an independently reviewed restaurant
through our site, we earn an affiliate commission.

By \href{http://www.nytimes3xbfgragh.onion/by/pete-wells}{Pete Wells}

\begin{itemize}
\item
  May 2, 2017
\item
  \begin{itemize}
  \item
  \item
  \item
  \item
  \item
  \item
  \end{itemize}
\end{itemize}

ABC Kitchen signified a small turning point in the appetites of New
Yorkers when it opened seven years ago. We had snuffled through much of
the previous decade in search of pork shoulders, lamb belly, rib-eyes
and marrow bones, and if we didn't exactly wake up on the first day of
2010 doubled over with regret, some of us were starting to wonder why
the bed smelled like bacon.

Then that spring we got
\href{http://www.nytimes3xbfgragh.onion/2010/06/02/dining/reviews/02rest.html}{ABC
Kitchen}, with
\href{https://www.nytimes3xbfgragh.onion/2016/09/07/dining/chef-dan-kluger-restaurant-abc-kitchen.html}{Dan
Kluger} in the kitchen, Jean-Georges Vongerichten whispering in his ear
and a new way of filling our stomachs on the menu.

Mr. Kluger wasn't cooking health food, not with that mayo-spritzed
cheeseburger. But a sizable chunk of the nearly 50 dishes were salads or
saladlike constructions. Vegetables were everywhere, including on top of
the whole-wheat pizzas. The plates jangled with flavor, and very little
of it came from animal fat. We might have called the food ``vegetable
forward'' if that term had been invented.

When diners stood up at the end of the night, you could almost hear them
sighing with relief that they'd had an entire meal in a hot downtown
restaurant without eating lard.

The two chefs teamed up again for
\href{http://www.nytimes3xbfgragh.onion/2013/07/31/dining/reviews/restaurant-review-abc-cocina-in-manhattan.html}{ABC
Cocina}, which opened
\href{http://www.slate.com/blogs/browbeat/2015/07/02/new_york_times_pea_guacamole_controversy_a_deep_analysis_of_the_tweet_that.html}{a
new front on the culture wars} by
\href{https://cooking.nytimes3xbfgragh.onion/recipes/1015047-green-pea-guacamole?smid=tw-nytimes}{mashing
peas into guacamole}, before Mr. Kluger left Mr. Vongerichten's
synchronized-swimming team to splash around in a pool of his own. After
two years of preparation, he opened
\href{https://www.loringplacenyc.com/}{Loring Place} in December in an
expansive space on West Eighth Street, where the
\href{http://lostnewyorkcity.blogspot.com/2011/02/goodbye-shoe-row.html}{shoe
stores used to be}.

\includegraphics{https://static01.graylady3jvrrxbe.onion/images/2017/05/03/dining/03REST1/03REST1listy-articleInline.jpg?quality=75\&auto=webp\&disable=upscale}

There's a long and busy bar to the right of the entrance where the
drinks list offers you very well-made classic cocktails as well as
innovations that taste like classics in the making. On the left are two
dining areas: one that sits on a platform in the window, and another,
bigger one off the kitchen. Designed by
\href{http://www.cycleprojects.com/}{Cycle Projects}, the rooms are full
of handsome expanses of white brick and stained walnut.

Yes, these hard surfaces amplify the noise. Yes, they're trying to fix
it.

Mr. Kluger has said in interviews that at Loring Place he is bringing
back some of the tricks he picked up before he learned his ABC's, while
he was working under
\href{http://www.nytimes3xbfgragh.onion/2011/12/21/dining/floyd-cardoz-the-chef-of-tabla-switches-cuisines-feed-me.html}{Floyd
Cardoz} and
\href{http://www.craftedhospitality.com/tom-colicchio/\#/about}{Tom
Colicchio}. That may be true of the recipes, but the template of the
menu comes unmistakably from ABC Kitchen, all the way down to the
whole-wheat pizzas.

There are more salads at Loring Place, and the saladlike constructions
are even more saladlike. At the end of winter, I loved the combination
of crisp Bosc pears with soft roasted leeks over thick yogurt, and some
sugar-glazed walnuts for ballast. A couple of weeks ago, with pollen in
the air, I appreciated the arrival of sweet sugar snap peas, lightly
charred and served with bright pink breakfast radishes and shreds of
pecorino.

In this herbivorous decade, Loring Place isn't the only restaurant to
try grinding flour from local grains, or the only one to get mixed
results. The emmer crackers, armored with sunflower seeds, are terrific
with a schmear of the excellent and lightly sweet hummus. So are the
breakfast radishes lodged in the surface of the hummus.

A friend who knows her grains looked dubious when I ordered the
house-made whole-wheat spaghetti. I was skeptical, too. (It is almost
never as good as it sounds.) Loring Place gets it right, though, and
lets it be, tossing it with basil, chiles, grated cheese and some
velvety leaves of spinach.

The whole-wheat bread, on the other hand, isn't quite at the point where
Loring Place should be charging \$6 for a loaf smaller than a football.
The crust is impressively sturdy, but the interior is neither tender nor
elastic --- it's dry and crumbly.

I liked the whole-wheat pizza crust enough to wish it had been allowed
to breathe a little more under a mass of shaved asparagus, maitake
mushrooms, jalapeños and three kinds of cheese.

Image

The Loring Place ``grandma crust'' is thicker than in the archetypal
versions of the pan pizza, but flavorful, light and
nongreasy.Credit...Casey Kelbaugh for The New York Times

But I have to admit I had a hard time keeping up my interest in the
other pies at Loring Place once I'd eaten a square of what the menu
calls ``grandma-style pan pizza.'' Like a less doughy Sicilian baked
with canned tomatoes, not sauce,
\href{http://www.bonappetit.com/restaurants-travel/article/find-grandma-pie}{the
grandma} has been slowly expanding its range west of its traditional
breeding grounds in Nassau County. The Loring Place grandma crust is
thicker than in the archetypal versions, but flavorful, light and
nongreasy. I hope Mr. Kluger has stocked up on pizza pans because he
runs a serious risk of selling a grandma pie to every table in the
restaurant.

Its only close rival on the menu is the cheeseburger, which comes with
bacon made on site, a likably combative pickled-pepper aioli and a
cast-iron pan of thick-cut fries that seem to get crunchier as they got
cooler, a neat feature.

I don't imagine Mr. Kluger wants us to think of Loring Place as a place
for cheeseburgers and square pizza, but a lot of the other main courses
aren't as compelling for one reason or another. Halibut with a market's
worth of mixed mushrooms in miso and lemon juice was an ideal example of
how to cook and serve fish, but I can't say the same about the
nondescript king salmon or the sea scallops in a sauce whose naked
acidity was intensified by apples marinated in vinegar.

Grilled chicken breast with carrot barbecue sauce sounded more exciting
than it was; the sauce never kicked in hard enough to rouse the warm
peas and rice that seemed to have settled down under the chicken for a
long winter's nap.

As much as I want to take Loring Place on its own terms, the menu makes
comparisons with ABC Kitchen inevitable, and Loring Place comes off as
slightly less interesting. It's not just that a lot of it feels familiar
now; some of it is also unfocused and busy. The plates have more going
on, but there's not as much happening.

To be fair, very few restaurants are as interesting as ABC Kitchen was
when it opened, and there are solid reasons that reservations at Loring
Place are hard to come by. They include the wine list, which packs a few
nice surprises, not the least of which is the attention it pays to the
\$40 to \$60 range. And they continue through the smartly nostalgic
dessert menu. There's a sly update on the Hostess CupCake, a tin of
cookies, and a vanilla ice-cream sundae with pretzels, toffee and
whatnot, modeled on the Dairy Queen Blizzard.

O.K. It's kind of like
\href{https://www.jamesbeard.org/recipes/salted-caramel-ice-cream-sundae}{the
sundae at ABC Kitchen}. It's still good.

\href{https://www.facebookcorewwwi.onion/nytfood/}{\emph{Follow NYT Food
on Facebook}}\emph{,}
\href{https://instagram.com/nytfood}{\emph{Instagram}}\emph{,}
\href{https://twitter.com/nytfood}{\emph{Twitter}} \emph{and}
\href{https://www.pinterest.com/nytfood/}{\emph{Pinterest}}\emph{.}
\href{https://www.nytimes3xbfgragh.onion/newsletters/cooking}{\emph{Get
regular updates from NYT Cooking, with recipe suggestions, cooking tips
and shopping advice}}\emph{.}

Advertisement

\protect\hyperlink{after-bottom}{Continue reading the main story}

\hypertarget{site-index}{%
\subsection{Site Index}\label{site-index}}

\hypertarget{site-information-navigation}{%
\subsection{Site Information
Navigation}\label{site-information-navigation}}

\begin{itemize}
\tightlist
\item
  \href{https://help.nytimes3xbfgragh.onion/hc/en-us/articles/115014792127-Copyright-notice}{©~2020~The
  New York Times Company}
\end{itemize}

\begin{itemize}
\tightlist
\item
  \href{https://www.nytco.com/}{NYTCo}
\item
  \href{https://help.nytimes3xbfgragh.onion/hc/en-us/articles/115015385887-Contact-Us}{Contact
  Us}
\item
  \href{https://www.nytco.com/careers/}{Work with us}
\item
  \href{https://nytmediakit.com/}{Advertise}
\item
  \href{http://www.tbrandstudio.com/}{T Brand Studio}
\item
  \href{https://www.nytimes3xbfgragh.onion/privacy/cookie-policy\#how-do-i-manage-trackers}{Your
  Ad Choices}
\item
  \href{https://www.nytimes3xbfgragh.onion/privacy}{Privacy}
\item
  \href{https://help.nytimes3xbfgragh.onion/hc/en-us/articles/115014893428-Terms-of-service}{Terms
  of Service}
\item
  \href{https://help.nytimes3xbfgragh.onion/hc/en-us/articles/115014893968-Terms-of-sale}{Terms
  of Sale}
\item
  \href{https://spiderbites.nytimes3xbfgragh.onion}{Site Map}
\item
  \href{https://help.nytimes3xbfgragh.onion/hc/en-us}{Help}
\item
  \href{https://www.nytimes3xbfgragh.onion/subscription?campaignId=37WXW}{Subscriptions}
\end{itemize}
