Sections

SEARCH

\protect\hyperlink{site-content}{Skip to
content}\protect\hyperlink{site-index}{Skip to site index}

\href{https://www.nytimes3xbfgragh.onion/section/world/europe}{Europe}

\href{https://myaccount.nytimes3xbfgragh.onion/auth/login?response_type=cookie\&client_id=vi}{}

\href{https://www.nytimes3xbfgragh.onion/section/todayspaper}{Today's
Paper}

\href{/section/world/europe}{Europe}\textbar{}That Popular Hostel in
Berlin? It's North Korean, and It's Closing

\url{https://nyti.ms/2q48vP8}

\begin{itemize}
\item
\item
\item
\item
\item
\end{itemize}

Advertisement

\protect\hyperlink{after-top}{Continue reading the main story}

Supported by

\protect\hyperlink{after-sponsor}{Continue reading the main story}

\hypertarget{that-popular-hostel-in-berlin-its-north-korean-and-its-closing}{%
\section{That Popular Hostel in Berlin? It's North Korean, and It's
Closing}\label{that-popular-hostel-in-berlin-its-north-korean-and-its-closing}}

\includegraphics{https://static01.graylady3jvrrxbe.onion/images/2017/05/11/world/11hostel/11hostel-articleLarge.jpg?quality=75\&auto=webp\&disable=upscale}

By \href{http://www.nytimes3xbfgragh.onion/by/alison-smale}{Alison
Smale}

\begin{itemize}
\item
  May 10, 2017
\item
  \begin{itemize}
  \item
  \item
  \item
  \item
  \item
  \end{itemize}
\end{itemize}

BERLIN --- The City Hostel Berlin operates out of a large, anonymous
building in what was Communist East Berlin, but it's a short walk from
Checkpoint Charlie and other attractions and has become a popular place
to stay, with good ratings on TripAdvisor and Yelp.

The only tip-off that this hostel differs from others in a city long a
magnet for the world's youth is the dreary embassy next door, where
\href{https://www.nytimes3xbfgragh.onion/topic/destination/north-korea?8qa}{North
Korea}'s flag flaps from a pole near a poorly tended garden and that
country's ruling family, the Kims, is enshrined in a photo display on a
gray metal fence.

The hostel, a former diplomatic quarters, has been earning the Kim
government tens of thousands of euros a month over the past decade, but
it will soon be closed to comply with the latest
\href{https://www.nytimes3xbfgragh.onion/2016/11/30/world/asia/north-korea-un-sanctions.html}{United
Nations sanctions} imposed over North Korea's nuclear tests.

The German government confirmed Wednesday that it was acting ``as
swiftly as possible'' to cut off the currency flow after German news
outlets reported that North Korea was charging an unnamed German
businessman €38,000 a month (about \$41,000) to operate the hostel.

Some years after the fall of the Iron Curtain, North Korea, in a classic
capitalist maneuver, leased the building, which had been the home of
dozens of its diplomats. Its prime location in the center of Berlin made
it popular with backpackers and students on field trips.

Martin Schäfer, a spokesman for
\href{https://www.nytimes3xbfgragh.onion/topic/destination/germany?8qa}{Germany}'s
Foreign Ministry, said the hostel would be closed in compliance with
stiffer sanctions passed in November by the United Nations that
specifically ban any commercial dealings with North Korean embassies or
on their property. The German authorities are acting as fast as they can
within German law, Mr. Schäfer added.

Philipp Lengsfeld, a Berlin deputy for the center-right Christian
Democrats in Parliament, said the North Korean connection ``was an open
secret --- every cabdriver knew that.''

The government plans to close the hostel were first reported by the
investigative reporting unit of the daily Süddeutsche Zeitung and the
public broadcasters WDR and NDR.

The unusual rental arrangement began in the 2000s, according to various
German news outlets. Mr. Lengsfeld, who visited North and South Korea
with a German parliamentary delegation in 2015, declined in a brief
telephone interview to say whether the government of Chancellor Angela
Merkel should have acted sooner to end it.

North Korea has relatively few embassies in Europe, and many of them are
holdovers from Communist times. Its Berlin mission stands on
Glinkastrasse, once a central thoroughfare in the government district of
East Berlin. A middle-aged woman emerged after a reporter rang the bell
during office hours. In halting English, she said diplomats in authority
were not around.

In the hostel, the young German employees were also reticent, referring
reporters to the hostel's
\href{http://www.cityhostel-berlin.com/gb/}{website}, which promises
``cheap accommodation in the city center,'' and even covering their name
badges.

The rooms range from singles and doubles to bunks for four or eight
guests in a room, according to the hostel's website. Prices are listed
as low as €17 a bed (about \$18.50), rising to €59 (about \$64) for a
single room. A man answering the telephone at the hostel said it was
booked on Wednesday and most coming days.

Guests came and went at lunchtime on Wednesday. Three teenagers on a
school trip from Pforzheim in southwestern Germany were smoking in the
courtyard, unaware of the North Korean connection. They, like other
guests, did not seem perturbed when told about it.

The lively reception area boasts a terrace with a beer garden, an
electronic baby grand piano and vivid signs, all of which enlivens the
distinctive uniformity of East German architecture.

But a mural on the back wall might upset the landlords back in
Pyongyang, where rigid Marxism still reigns. In cartoon style, it
depicts a slice of Berlin's Cold War history. Jagged fragments of a
structure litter the picture while a quiet sign off to the left
announces: ``Construction of the Wall, 1961. Wall falls, 1989.''

Advertisement

\protect\hyperlink{after-bottom}{Continue reading the main story}

\hypertarget{site-index}{%
\subsection{Site Index}\label{site-index}}

\hypertarget{site-information-navigation}{%
\subsection{Site Information
Navigation}\label{site-information-navigation}}

\begin{itemize}
\tightlist
\item
  \href{https://help.nytimes3xbfgragh.onion/hc/en-us/articles/115014792127-Copyright-notice}{©~2020~The
  New York Times Company}
\end{itemize}

\begin{itemize}
\tightlist
\item
  \href{https://www.nytco.com/}{NYTCo}
\item
  \href{https://help.nytimes3xbfgragh.onion/hc/en-us/articles/115015385887-Contact-Us}{Contact
  Us}
\item
  \href{https://www.nytco.com/careers/}{Work with us}
\item
  \href{https://nytmediakit.com/}{Advertise}
\item
  \href{http://www.tbrandstudio.com/}{T Brand Studio}
\item
  \href{https://www.nytimes3xbfgragh.onion/privacy/cookie-policy\#how-do-i-manage-trackers}{Your
  Ad Choices}
\item
  \href{https://www.nytimes3xbfgragh.onion/privacy}{Privacy}
\item
  \href{https://help.nytimes3xbfgragh.onion/hc/en-us/articles/115014893428-Terms-of-service}{Terms
  of Service}
\item
  \href{https://help.nytimes3xbfgragh.onion/hc/en-us/articles/115014893968-Terms-of-sale}{Terms
  of Sale}
\item
  \href{https://spiderbites.nytimes3xbfgragh.onion}{Site Map}
\item
  \href{https://help.nytimes3xbfgragh.onion/hc/en-us}{Help}
\item
  \href{https://www.nytimes3xbfgragh.onion/subscription?campaignId=37WXW}{Subscriptions}
\end{itemize}
