Sections

SEARCH

\protect\hyperlink{site-content}{Skip to
content}\protect\hyperlink{site-index}{Skip to site index}

\href{https://www.nytimes3xbfgragh.onion/section/world/asia}{Asia
Pacific}

\href{https://myaccount.nytimes3xbfgragh.onion/auth/login?response_type=cookie\&client_id=vi}{}

\href{https://www.nytimes3xbfgragh.onion/section/todayspaper}{Today's
Paper}

\href{/section/world/asia}{Asia Pacific}\textbar{}How North Korea
Managed to Defy Years of Sanctions

\url{https://nyti.ms/2r82oqt}

\begin{itemize}
\item
\item
\item
\item
\item
\end{itemize}

Advertisement

\protect\hyperlink{after-top}{Continue reading the main story}

Supported by

\protect\hyperlink{after-sponsor}{Continue reading the main story}

\hypertarget{how-north-korea-managed-to-defy-years-of-sanctions}{%
\section{How North Korea Managed to Defy Years of
Sanctions}\label{how-north-korea-managed-to-defy-years-of-sanctions}}

\includegraphics{https://static01.graylady3jvrrxbe.onion/images/2017/05/13/world/13nksanctions-1/13nksanctions-1-articleLarge-v2.jpg?quality=75\&auto=webp\&disable=upscale}

By \href{http://www.nytimes3xbfgragh.onion/by/jane-perlez}{Jane Perlez},
Yufan Huang and
\href{https://www.nytimes3xbfgragh.onion/by/paul-mozur}{Paul Mozur}

\begin{itemize}
\item
  May 12, 2017
\item
  \begin{itemize}
  \item
  \item
  \item
  \item
  \item
  \end{itemize}
\end{itemize}

\href{https://cn.nytimes3xbfgragh.onion/asia-pacific/20170515/north-korea-sanctions-loopholes-china-united-states-garment-industry/}{阅读简体中文版}

DANDONG, China --- As the end of the fashion season approached, and the
suits and dresses arrived in her company's warehouses here in the
Chinese border town of Dandong, the accountant crammed about \$100,000
into a backpack, then boarded a rickety train with several co-workers.

She asked to be identified only by her surname, Lang, given the
sensitivity of their destination:
\href{https://www.nytimes3xbfgragh.onion/topic/destination/north-korea?8qa}{North
Korea}.

After a six-hour journey, she recalled, they arrived at a factory where
hundreds of women using high-end European machines sewed clothes with
``Made in China'' labels. Her boss handed the money to the North Korean
manager, all of it in American bills as required.

Despite seven rounds of
\href{https://www.nytimes3xbfgragh.onion/2016/11/30/world/asia/north-korea-un-sanctions.html}{United
Nations sanctions} over the past 11 years, including a ban on ``bulk
cash'' transfers, large avenues of trade remain open to North Korea,
allowing it to earn foreign currency to sustain its economy and finance
\href{https://www.nytimes3xbfgragh.onion/topic/subject/north-koreas-nuclear-program?8qa}{its
program} to build a nuclear weapon that can strike the United States.

Fraudulent labeling helps support its garment industry, which generated
more than \$500 million for the isolated nation last year, according to
Chinese trade data.

North Korea earned an additional \$1.1 billion selling coal to China
last year using a loophole in the ban on such exports, and researchers
say tens of thousands of North Koreans who work overseas as laborers are
forced
\href{https://www.nytimes3xbfgragh.onion/2015/02/20/world/asia/north-koreans-toil-in-slavelike-conditions-abroad-rights-groups-say.html}{to
send back} as much as \$250 million annually. Diplomats estimate the
country makes \$70 million more selling rights to harvest seafood from
its waters.

China accounts for more than 80 percent of trade with North Korea, and
the Trump administration is counting on Beijing to use that leverage to
pressure it into giving up its nuclear arsenal. The Chinese government
took a big step in February by announcing that it was
\href{https://www.nytimes3xbfgragh.onion/2017/02/18/world/asia/north-korea-china-coal-imports-suspended.html}{suspending
imports of coal} from the country through the end of the year.

But China has a long record of shielding North Korea from more painful
sanctions, because it is afraid of a regime collapse that could send
refugees streaming across the border and leave it with a more hostile
neighbor.

In addition, Beijing now has a sympathetic ear in
\href{https://www.nytimes3xbfgragh.onion/topic/destination/south-korea?8qa}{South
Korea}, whose newly elected president,
\href{https://www.nytimes3xbfgragh.onion/2017/05/09/world/asia/south-korea-election-president-moon-jae-in.html?rref=collection\%2Ftimestopic\%2FSouth\%20Korea\&action=click\&contentCollection=world\&region=stream\&module=stream_unit\&version=latest\&contentPlacement=8\&pgtype=collection}{Moon
Jae-in}, echoes its view that sanctions alone will not be enough to
persuade Pyongyang to abandon its nuclear program.

While North Korea remains impoverished and dependent on food aid, its
economy
\href{https://www.nytimes3xbfgragh.onion/2017/04/30/world/asia/north-korea-economy-marketplace.html}{appears
to be growing}, partly because of a limited embrace of market forces
since its leader, Kim Jong-un, took power more than five years ago.

Foreign trade, primarily with China, has surged, too, more than doubling
since 2000, though it has slipped in the past three years.

In theory, North Korea's greater openness to trade makes it more
vulnerable to sanctions, with new potential targets and pressure points.
But it also highlights the limits of an approach to sanctions ---
defined largely by China at the United Nations --- that aims to punish
North Korea's military and ruling elite while sparing its people. As
trade expands, the lines have blurred.

\includegraphics{https://static01.graylady3jvrrxbe.onion/images/2017/05/13/world/13nksanctions-3/13nksanctions-3-articleLarge.jpg?quality=75\&auto=webp\&disable=upscale}

\hypertarget{north-korean-labor}{%
\subsection{North Korean Labor}\label{north-korean-labor}}

Positioned near the mouth of the Yalu River, Dandong is China's largest
border town, and much of North Korea's trade with the world flows across
its old bridges or through its deepwater port.

Ms. Lang, 33, moved here more than a decade ago to study environmental
protection. She ended up like many with ambition in this city of more
than three million: doing business with North Korea.

She wears exquisite makeup and carries a Louis Vuitton handbag, and she
said her role in the garment trade was straightforward: Orders come in
from Japan, Europe and other parts of China, and she gets the clothes
made.

For those with quick deadlines or detailed specifications, she turns to
Chinese factories in Dandong, where quality control is better. Yet even
these factories employ North Korean laborers, she said.

For decades, North Korea has been accused of sending workers abroad and
confiscating most of their wages, an arrangement that activists liken to
slave labor. Researchers say the practice has expanded since Mr. Kim
took power, with more than 50,000 workers now toiling in up to 40
countries.

In Dandong, the local government boasts that 10,000 North Koreans are
employed in its apparel factories, working 12- to 14-hour shifts, with
just two to four days off each month and a monthly wage of no more than
\$260.

``They are well disciplined and easy to manage,'' says the
\href{http://zhaoshang.dandong.gov.cn/html/80/20172/a9b7ba70783b617e9998dc4dd82eb3c5.html}{website
of the Dandong commerce bureau}, noting that the workers have been
vetted before arrival. ``There is no such thing as absenteeism or
interfering with management, no using illness to shun work or
procrastination and losing work time.''

Ms. Lang sends more-flexible orders to North Korea, where costs are
lower but it is impossible to guarantee delivery dates because of power
failures and a shortage of trucks.

Her company ships fabric, buttons and zippers to factories there, she
said, because the North lacks the materials, and they put ``Made in
China'' labels in garments to make them easier to sell overseas. That
would most likely be considered fraud and a violation of place-of-origin
rules in countries that import the clothes, experts said.

Paul Tjia, managing director of GPI Consultancy, a Dutch company that
offers advice on doing business in North Korea, said that some European
clients had ordered hundreds of thousands of garments and that ``Made in
China'' labels could be justified by additional work put into the
clothes inside China.

But he added: ``I'm not a garment manufacturer. I just make the
introductions.''

Image

A silk mill in Pyongyang, the North Korean capital, in February.
Fraudulent labeling helps support the garment industry, which generated
more than \$500 million for the country last year, according to trade
data.Credit...Ed Jones/Agence France-Presse --- Getty Images

\hypertarget{loopholes-abound}{%
\subsection{Loopholes Abound}\label{loopholes-abound}}

China has kept North Korea's garment sector off the list of industries
targeted by United Nations sanctions, arguing that punishing it would
hurt ordinary people and not military programs. It has protected North
Korea's seafood industry using the same argument.

But it is difficult to say who benefits from this trade, in part because
even private enterprise in North Korea is overseen by state officials
who extract taxes and bribes.

Image

``Whether the proceeds from the textile industry support the nuclear
program is an open question,'' said Joseph M. DeThomas, a professor at
Pennsylvania State University and a former American ambassador involved
in sanctions policy. ``Money is fungible.''

At least one North Korean enterprise controlled by the atomic energy
bureau, the Korea Kumsan Trading Corporation, ran a garment factory that
added embroidery and beading to clothing, according to
\href{http://www.naenara.com.kp/ch/trade/?company+15+31}{a North Korean
government trade website}.

And South Korean officials say the millions paid by Chinese companies to
fish in North Korean waters go primarily to firms controlled by the
North's military.

Sanctions also do not cover the organized export of labor. The United
States has urged countries to eject North Korean workers, saying their
remittances benefit the military, not their families. But China, Russia
and other nations continue to hire them.

American sanctions against North Korea began with a near-total economic
embargo adopted in 1950, at the start of the Korean War. Over the years,
some sanctions were eased and others added, including after the
\href{https://www.nytimes3xbfgragh.onion/2014/12/18/world/asia/us-links-north-korea-to-sony-hacking.html}{cyberattack
on Sony Pictures} in 2014 that Washington attributed to the North.

The United Nations Security Council did not impose sanctions until July
2006, when, after a series of missile tests, it banned countries from
selling material for missiles or weapons of mass destruction to North
Korea.

The North detonated its first nuclear device months later, followed by
additional tests in 2009 and 2013, and two in 2016. The Security Council
tightened sanctions after each test, as well as after a satellite launch
in 2013. It targeted military supplies and luxury goods, shut Pyongyang
out of the international financial system and, most recently, banned a
range of mineral exports.

But loopholes abound. Resolutions called for searches of vessels
carrying cargo to North Korea but have failed to stop its
\href{http://www.nytimes3xbfgragh.onion/2006/10/20/world/asia/20shipping.html}{use
of ships sailing under foreign flags}. And when the Security Council
banned its top export, coal, China insisted on an exception for
transactions judged to be for ``livelihood purposes.''

New measures seek to limit North Korea's ability to make money through
its embassies. In Berlin, for example, the authorities are
\href{https://www.nytimes3xbfgragh.onion/2017/05/10/world/europe/north-korea-germany-berlin-hostel-sanctions.html}{closing
a hostel} run out of former diplomatic quarters. But the North has
responded to such crackdowns by shifting business to countries with
weaker enforcement.

``How much cooperation will the international community get from Cuba,
Russia, Iran or even Pakistan, Bangladesh or Laos?'' asked Stephan
Haggard, an expert on the North Korean economy at the University of
California, San Diego.

The United States has also urged a boycott of Air Koryo, the North
Korean airline, but it still flies to China and Russia. Chinese tourism
to North Korea is booming, said Cha Yong Hyok, whose company, Indprk,
takes groups by train to Pyongyang and will soon use new flights from
Dandong.

The North often circumvents banking sanctions using front companies and
agents overseas, and North Koreans routinely send and receive payments
using Chinese intermediaries who take a commission, despite the ban on
``bulk cash'' transfers.

``We can and should go after these targets, but turning this into a game
of financial cat-and-mouse will never achieve the level of pressure
needed,'' said Daniel L. Glaser, a former Treasury Department official
involved in sanctions enforcement.

Ultimately, he argued, that pressure will come only if China makes a
strategic decision to truly squeeze the North. ``Though China has taken
helpful steps at times,'' he said, ``it has never been willing to go all
in.''

Image

A television made by Konka, a Chinese brand, at a barber shop in
Pyongyang in April. Just about every big appliance maker in China does
business with North Korea.Credit...Ed Jones/Agence France-Presse ---
Getty Images

\hypertarget{in-business-with-the-north}{%
\subsection{In Business With the
North}\label{in-business-with-the-north}}

Many of China's best-known companies have done business with North Korea
even as they have sought customers and investors in the United States or
relied on American-made parts and materials. ZTE, the mobile phone and
electronics manufacturer, for example, shipped about \$15 million of
goods to the North in 2015, according to Chinese customs records viewed
via the global trade database company Panjiva.

\href{https://www.nytimes3xbfgragh.onion/2017/03/07/technology/zte-china-fine.html}{The
company} agreed to pay \$1.19 billion in March for violating American
sanctions against Iran and North Korea, in part by sending 283 shipments
of electronics with American-made components to the North. ZTE has
pledged to improve oversight.

But many Chinese companies sell products to North Korea without such
problems. The electric car and battery maker BYD, in which Warren E.
Buffett's Berkshire Hathaway owns a 10 percent stake, has shipped \$14
million in goods to North Korea since 2012, including rubber products in
January and vehicles in December, customs records show. BYD and
Berkshire Hathaway didn't immediately respond to a request for comment.

Just about every big Chinese appliance maker does business with North
Korea, too, shipping refrigerators, air-conditioners, televisions and
other electronics. The major Chinese automakers sell vehicles to the
North as well.

Even Tsingtao Brewery shows up in customs records, delivering \$20,000
worth of beer in the summer of 2014.

United Nations sanctions prohibit the sale of luxury goods to North
Korea, but countries are generally left to define what that means. The
resolutions list jewelry, luxury automobiles, sports equipment and
snowmobiles but make no mention of televisions, consumer electronics or
home appliances.

In some cases, Chinese companies with access to advanced technology are
doing business with North Korea. Subsidiaries of the defense
manufacturer Norinco made seven shipments, mostly of electronic and
optical goods, worth a total of \$1.5 million, in the second half of
last year, records say. Norinco did not respond to a request for
comment.

Matthew Brazil, a security consultant and former diplomat for the United
States who investigated Chinese trade controls in the 1990s, said it was
often impossible to get China to follow up on leads suggesting Chinese
firms were violating restrictions. ``Three months later, if you're
lucky, the visit is scheduled, and many times, visits weren't scheduled
at all,'' he said.

Mr. Brazil said the problem had persisted, and ``any level of control of
American electronics has completely collapsed because this technology
can be so easily shipped from China to North Korea.''

On AliExpress, an e-commerce platform run by the Chinese internet giant
Alibaba, six of the nine shipping services list North Korea as a
potential destination. Alibaba declined to comment.

The manager of a shipping firm in Dandong who asked to be identified
only by her surname, Li, because of the nature of her work said shipping
a package of electronics to North Korea was straightforward ``as long as
it doesn't have obvious labels'' and meets weight requirements.

In fact, a delivery is more likely to run into problems on the North
Korean side of the border than with customs inspectors in China. ``The
key,'' she said, ``is to make sure everything is fine with the people on
the other side.''

Advertisement

\protect\hyperlink{after-bottom}{Continue reading the main story}

\hypertarget{site-index}{%
\subsection{Site Index}\label{site-index}}

\hypertarget{site-information-navigation}{%
\subsection{Site Information
Navigation}\label{site-information-navigation}}

\begin{itemize}
\tightlist
\item
  \href{https://help.nytimes3xbfgragh.onion/hc/en-us/articles/115014792127-Copyright-notice}{©~2020~The
  New York Times Company}
\end{itemize}

\begin{itemize}
\tightlist
\item
  \href{https://www.nytco.com/}{NYTCo}
\item
  \href{https://help.nytimes3xbfgragh.onion/hc/en-us/articles/115015385887-Contact-Us}{Contact
  Us}
\item
  \href{https://www.nytco.com/careers/}{Work with us}
\item
  \href{https://nytmediakit.com/}{Advertise}
\item
  \href{http://www.tbrandstudio.com/}{T Brand Studio}
\item
  \href{https://www.nytimes3xbfgragh.onion/privacy/cookie-policy\#how-do-i-manage-trackers}{Your
  Ad Choices}
\item
  \href{https://www.nytimes3xbfgragh.onion/privacy}{Privacy}
\item
  \href{https://help.nytimes3xbfgragh.onion/hc/en-us/articles/115014893428-Terms-of-service}{Terms
  of Service}
\item
  \href{https://help.nytimes3xbfgragh.onion/hc/en-us/articles/115014893968-Terms-of-sale}{Terms
  of Sale}
\item
  \href{https://spiderbites.nytimes3xbfgragh.onion}{Site Map}
\item
  \href{https://help.nytimes3xbfgragh.onion/hc/en-us}{Help}
\item
  \href{https://www.nytimes3xbfgragh.onion/subscription?campaignId=37WXW}{Subscriptions}
\end{itemize}
