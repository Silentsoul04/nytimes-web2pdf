Sections

SEARCH

\protect\hyperlink{site-content}{Skip to
content}\protect\hyperlink{site-index}{Skip to site index}

\href{https://www.nytimes3xbfgragh.onion/section/politics}{Politics}

\href{https://myaccount.nytimes3xbfgragh.onion/auth/login?response_type=cookie\&client_id=vi}{}

\href{https://www.nytimes3xbfgragh.onion/section/todayspaper}{Today's
Paper}

\href{/section/politics}{Politics}\textbar{}A Constitutional Puzzle: Can
the President Be Indicted?

\url{https://nyti.ms/2scC27a}

\begin{itemize}
\item
\item
\item
\item
\item
\end{itemize}

Advertisement

\protect\hyperlink{after-top}{Continue reading the main story}

Supported by

\protect\hyperlink{after-sponsor}{Continue reading the main story}

\href{/column/sidebar}{Sidebar}

\hypertarget{a-constitutional-puzzle-can-the-president-be-indicted}{%
\section{A Constitutional Puzzle: Can the President Be
Indicted?}\label{a-constitutional-puzzle-can-the-president-be-indicted}}

\includegraphics{https://static01.graylady3jvrrxbe.onion/images/2017/05/30/us/30bar/30bar-articleInline.jpg?quality=75\&auto=webp\&disable=upscale}

By \href{http://www.nytimes3xbfgragh.onion/by/adam-liptak}{Adam Liptak}

\begin{itemize}
\item
  May 29, 2017
\item
  \begin{itemize}
  \item
  \item
  \item
  \item
  \item
  \end{itemize}
\end{itemize}

WASHINGTON --- The Constitution does not answer every question. It
includes detailed instructions, for instance, about how Congress may
remove a president who has committed serious offenses. But it does not
say whether the president may be criminally prosecuted in the meantime.

The Supreme Court has never answered that question, either. It heard
arguments on the issue in 1974 in a case in
\href{http://caselaw.findlaw.com/us-supreme-court/418/683.html}{which it
ordered President Richard M. Nixon to turn over tape recordings}, but it
did not resolve it.

\href{https://www.nytimes3xbfgragh.onion/2017/05/16/us/politics/james-comey-trump-flynn-russia-investigation.html?_r=0}{Reports}
that President Trump asked James B. Comey, then the F.B.I. director, to
shut down an investigation into his former national security adviser,
Michael T. Flynn, prompted accusations that the president may have
obstructed justice. Robert S. Mueller III, the former F.B.I. director
who has been appointed special counsel to look into ties between the
Trump campaign and Russia, will presumably investigate the matter.

But would the Constitution allow Mr. Mueller to indict Mr. Trump if he
finds evidence of criminal conduct?

The prevailing view among most legal experts is no. They say the
president is immune from prosecution so long as he is in office.

``The framers implicitly immunized a sitting president from ordinary
criminal prosecution,'' said Akhil Reed Amar, a law professor at Yale.

Note the word ``implicitly.'' Professor Amar acknowledged that the text
of the Constitution did not directly answer the question. ``It has to
be,'' he said, ``a structural inference about the uniqueness of the
president himself.''

The closest the Constitution comes to addressing the issue is in this
passage, from
\href{https://www.law.cornell.edu/constitution/articlei\#section3}{Article
I, Section 3}: ``Judgment in cases of impeachment shall not extend
further than to removal from office, and disqualification to hold and
enjoy any office of honor, trust or profit under the United States: but
the party convicted shall nevertheless be liable and subject to
indictment, trial, judgment and punishment, according to law.''

This much seems clear: The president and other federal officials may be
prosecuted after they leave office, and there is no double jeopardy
protection from prosecution if they are removed following impeachment.

However, ``whether the Constitution allows indictment of a sitting
president is debatable,'' Brett M. Kavanaugh, who served on the staff of
Kenneth W. Starr, the independent counsel who investigated President
Bill Clinton, wrote in a 1998 law review article. Mr. Kavanaugh, who is
now a federal appeals court judge, also concluded that impeachment, not
prosecution, was the right way to address a sitting president's crimes.

The most prominent dissenter from the prevailing view is Eric M.
Freedman, a law professor at Hofstra University and the author of
\href{http://scholarlycommons.law.hofstra.edu/cgi/viewcontent.cgi?article=2059\&context=hlr}{a
1999 law review article} that made the case for allowing criminal
prosecution of incumbent presidents.

Professor Freedman demonstrated that the issue had divided the founding
generation and argued that granting sitting presidents immunity from
prosecution was ``inconsistent with the history, structure and
underlying philosophy of our government, at odds with precedent and
unjustified by practical considerations.''

He pointed out that other federal officials who are subject to
impeachment, including judges, have been indicted while in office.
Courts have rejected the argument that impeachment is the sole remedy
for such officials.

But Professor Amar said that presidents were different.

``If you're going to undo a national election, the body that does that
should have a national mandate,'' he said. ``Even a federal prosecution
would follow only from an indictment from a grand jury sitting in one
locality.''

Vice President Spiro T. Agnew, facing a grand jury investigation that
would lead to
\href{http://www.nytimes3xbfgragh.onion/learning/general/onthisday/big/1010.html}{his
resignation in 1973}, argued that he was immune from prosecution while
in office. Impeachment, he said, was the only remedy.

The Justice Department, in
\href{http://query.nytimes3xbfgragh.onion/mem/archive-free/pdf?res=990CE3D71F30E63ABC4E53DFB6678388669EDE}{a
brief signed by Solicitor General Robert H. Bork}, disagreed. But,
though the question was not before the court, Mr. Bork added that
``structural features of the Constitution'' barred prosecutions of
sitting presidents.

Since the president has the power to control federal prosecutions and to
pardon federal offenses, Mr. Bork wrote, it would make no sense to allow
the president to be prosecuted until after he is removed from office and
forfeits those powers. (Mr. Bork would go on to become a federal appeals
court judge and an unsuccessful nominee to the Supreme Court.)

A year later, Leon Jaworski, the Watergate special prosecutor, took a
less categorical position.

``It is an open and substantial question whether an incumbent president
is subject to indictment,'' he
\href{https://drive.google.com/file/d/0B6mkR0LD6dl3aWJHd3pOUkxWeWc/view}{told
the Supreme Court} during his successful quest to obtain the White House
recordings that contributed to Nixon's resignation.

In a series of memorandums, the Justice Department's Office of Legal
Counsel \href{https://www.justice.gov/file/19351/download}{concluded}
that indicting a sitting president would violate the Constitution by
undermining his ability to do his job. Those memos, too, though, said
the answer was a matter of structure and inference.

``Neither the text nor the history of the Constitution ultimately
provided dispositive guidance in determining whether a president is
amenable to indictment or criminal prosecution while in office,'' a 2000
memo said, summarizing an earlier one. ``It therefore based its analysis
on more general considerations of constitutional structure.''

The \href{https://www.law.cornell.edu/cfr/text/28/600.7}{Justice
Department's regulations} require Mr. Mueller, the special counsel, to
follow the department's ``rules, regulations, procedures, practices and
policies.'' If the memos bind Mr. Mueller, it would seem he could not
indict Mr. Trump, no matter what he uncovered.

But Andrew Manuel Crespo, a law professor at Harvard, has questioned
whether the special-counsel regulations should be read that broadly. The
regulations,
\href{https://takecareblog.com/blog/the-road-to-united-states-v-trump-is-paved-with-prosecutorial-discretion}{he
wrote} on \href{https://takecareblog.com/}{Take Care}, a law blog,
``focus more on administrative protocols and procedures than on legal
analyses, arguments or judgments.''

Even if Mr. Mueller has a measure of discretion, Professor Amar said,
the right process for assessing Mr. Trump's conduct, should it come to
that, is the one described in detail in the Constitution: impeachment.

``Much of the recent pontificating about the technical elements of
obstruction of justice is quite beside the point,'' he said. ``Donald
Trump is to be judged by the House and the Senate, who are in turn
judged on Election Day by the American people more generally.''

Advertisement

\protect\hyperlink{after-bottom}{Continue reading the main story}

\hypertarget{site-index}{%
\subsection{Site Index}\label{site-index}}

\hypertarget{site-information-navigation}{%
\subsection{Site Information
Navigation}\label{site-information-navigation}}

\begin{itemize}
\tightlist
\item
  \href{https://help.nytimes3xbfgragh.onion/hc/en-us/articles/115014792127-Copyright-notice}{©~2020~The
  New York Times Company}
\end{itemize}

\begin{itemize}
\tightlist
\item
  \href{https://www.nytco.com/}{NYTCo}
\item
  \href{https://help.nytimes3xbfgragh.onion/hc/en-us/articles/115015385887-Contact-Us}{Contact
  Us}
\item
  \href{https://www.nytco.com/careers/}{Work with us}
\item
  \href{https://nytmediakit.com/}{Advertise}
\item
  \href{http://www.tbrandstudio.com/}{T Brand Studio}
\item
  \href{https://www.nytimes3xbfgragh.onion/privacy/cookie-policy\#how-do-i-manage-trackers}{Your
  Ad Choices}
\item
  \href{https://www.nytimes3xbfgragh.onion/privacy}{Privacy}
\item
  \href{https://help.nytimes3xbfgragh.onion/hc/en-us/articles/115014893428-Terms-of-service}{Terms
  of Service}
\item
  \href{https://help.nytimes3xbfgragh.onion/hc/en-us/articles/115014893968-Terms-of-sale}{Terms
  of Sale}
\item
  \href{https://spiderbites.nytimes3xbfgragh.onion}{Site Map}
\item
  \href{https://help.nytimes3xbfgragh.onion/hc/en-us}{Help}
\item
  \href{https://www.nytimes3xbfgragh.onion/subscription?campaignId=37WXW}{Subscriptions}
\end{itemize}
