Sections

SEARCH

\protect\hyperlink{site-content}{Skip to
content}\protect\hyperlink{site-index}{Skip to site index}

\href{https://www.nytimes3xbfgragh.onion/section/food}{Food}

\href{https://myaccount.nytimes3xbfgragh.onion/auth/login?response_type=cookie\&client_id=vi}{}

\href{https://www.nytimes3xbfgragh.onion/section/todayspaper}{Today's
Paper}

\href{/section/food}{Food}\textbar{}Some Food, a Plate, a Room. That's
Enough at King, in SoHo.

\url{https://nyti.ms/2sP1Mqs}

\begin{itemize}
\item
\item
\item
\item
\item
\item
\end{itemize}

Advertisement

\protect\hyperlink{after-top}{Continue reading the main story}

Supported by

\protect\hyperlink{after-sponsor}{Continue reading the main story}

\href{/column/restaurant-review}{Restaurant Review}

\hypertarget{some-food-a-plate-a-room-thats-enough-at-king-in-soho}{%
\section{Some Food, a Plate, a Room. That's Enough at King, in
SoHo.}\label{some-food-a-plate-a-room-thats-enough-at-king-in-soho}}

\href{https://www.nytimes3xbfgragh.onion/slideshow/2017/06/06/dining/king-restaurant-nyc-review.html}{}

\hypertarget{king}{%
\subsection{King}\label{king}}

10 Photos

View Slide Show ›

\includegraphics{https://static01.graylady3jvrrxbe.onion/images/2017/06/07/dining/07REST-KING-slide-EDRK/07REST-KING-slide-EDRK-articleLarge.jpg?quality=75\&auto=webp\&disable=upscale}

David Williams for The New York Times

\begin{itemize}
\tightlist
\item
  King\\
  ★★ French;Italian \$\$\$ 18 King Street 917-825-1618
\end{itemize}

\href{https://resy.com/cities/ny/the-king?utm_source=nyt\&utm_medium=restoprofile\&utm_campaign=affiliates\&aff_id=c1fe784}{Reserve
a Table}

When you make a reservation at an independently reviewed restaurant
through our site, we earn an affiliate commission.

By \href{http://www.nytimes3xbfgragh.onion/by/pete-wells}{Pete Wells}

\begin{itemize}
\item
  June 6, 2017
\item
  \begin{itemize}
  \item
  \item
  \item
  \item
  \item
  \item
  \end{itemize}
\end{itemize}

A friend who has made herself a regular at the SoHo restaurant
\href{http://kingrestaurant.nyc/}{King} sums up its appeal, with deep
approval, as ``food on a plate in a room.'' This undersells the place
--- King offers much more than that --- but she has a point.

The restaurant, which opened in September, is not a show-off. The dining
room is small, tidy and nearly square, painted the pale color of butter
in the winter. Windows stretching to the ceiling along the north wall
look out on the dormered Federal townhouses of King Street. The facing
wall is broken up with a single abstract painting, a doorway to the
kitchen and a long opening behind which the four people who make up the
entire kitchen staff can be seen at work. It is a room content with
being just a room, and letting you focus on the people you're with and
the food that's on the plate.

There will be papery curls of
\href{http://www.nytimes3xbfgragh.onion/2010/03/24/dining/24mini.html}{carta
di musica} before anything else arrives. Brushed with rosemary, this
Sardinian flatbread shatters when you tap it. It will be gone in a
minute, but it serves as a brief introduction to King's culinary realm.

After this gift from the kitchen, an order of panisse may seem
redundant. It isn't. At King, these chickpea fritters are long, thin
tongues that puff up like pommes soufflés and are scented all over by
fried sage. They're beautiful. They were on the menu last month, but
they may not stay long.

\includegraphics{https://static01.graylady3jvrrxbe.onion/images/2017/06/07/dining/07REST1/07REST1-articleLarge.jpg?quality=75\&auto=webp\&disable=upscale}

Nothing does. King's two chefs, Jess Shadbolt and Clare de Boer, cook a
different slate of dishes each night. The changes tend to be
evolutionary, drawing from a fixed repertoire of ideas. The first time I
had halibut there, it was grilled, and its skin had been deeply charred
without drying out the rich white meat right under it. The plate was
filled out with spinach and white coco beans soft enough to mash with a
spoon.

The next time I ate halibut at King, it was poached, the beans were back
and the greens had been replaced with small artichoke hearts simmered in
white wine and amaranth. I'll eat either dish again in a minute if the
chefs give me a chance.

Their food borrows liberally from the home cooking of those parts of
Italy and France where New Yorkers of a certain generation dreamed of
buying a summer house. This style has gone out lately, but a few years
ago it was the default for the breed of American chefs who led their
staff in fava-shucking parties each spring. Ms. Shadbolt and Ms. de Boer
practice an English variant of the style they learned in the kitchen of
\href{http://www.rivercafe.co.uk/}{the River Cafe} in London.

Anybody who's eaten at that restaurant or has gone to bed with one of
its cookbooks will experience occasional flashbacks at King. Polenta and
almond flour go into a classic River Cafe dessert and into a different
one that recurs at King. If you see it on the menu, especially if it is
weighted with nectarines, pounce.

Image

Halibut with small artichoke hearts and white coco beans.Credit...David
Williams for The New York Times

Chefs can't get far on imitation. They need to understand the how and
why of things, and Ms. Shadbolt and Ms. de Boer do. Once you get past
King's debt to River Cafe, what you really notice is how many little
moves they know that can raise a recipe from good to exceptional.

Olive oil blended with nettles is an excellent sauce for boiled
fingerling potatoes. What makes King's even better is that a few
fingerlings have been crushed into the purée, so from time to time you
bite into a chunk of potato hiding inside the sauce. This was a side
dish that traveled alongside a guinea hen, roasted whole with a bath of
verdicchio in the bottom of the pan. Wilted nettles sprawled over its
crisp skin. With a squeeze of lemon, it was one of the most appealing
guinea hens I had ever come across.

If you eat at King often, you can see the chefs making subtle
adjustments to keep flavors in balance. Ravioli under spring peas and
raw pea shoots were filled with minimally seasoned ricotta one week. The
next, the ricotta inside floppy tortellini got an extra spur from lemon
zest. The citrus might have stepped on the sweetness of the peas, but it
helped the tortellini, because the only competition on the plate was
fresh marjoram.

And when some new seasonal ingredient appears on the scene, you can see
the chefs strike like cobras. Saltwort, the salt-marsh-loving succulent
that Italians call agretti, was the exciting foundation for a May salad
built with wild arugula and raw ovals of asparagus stalks, making one of
their first appearances of the year. Salty goldenrod bottarga was shaved
over everything, even the white rim of the plate.

Image

King's two chefs, Jess Shadbolt, left, and Clare de Boer cook different
dishes each night. The changes tend to be evolutionary, drawing from a
fixed repertoire.Credit...David Williams for The New York Times

To be won over by King, it helps not to expect things you've never seen
before. Even those meeting saltwort for the first time will find that
the rest of the meal looks familiar. What Ms. de Boer and Ms. Shadbolt
offer is not a wild vision of new ways to cook but a solid vision of how
to eat. They put pleasure at the table above gymnastics on the plate.
For reasons I don't want to understand, I associate this trait with
other female chefs around town, including
\href{https://www.nytimes3xbfgragh.onion/2016/04/13/dining/i-sodi-restaurant-review.html}{Rita
Sodi},
\href{https://www.nytimes3xbfgragh.onion/2016/03/30/dining/lilia-restaurant-review.html}{Missy
Robbins},
\href{https://www.nytimes3xbfgragh.onion/2017/05/18/magazine/chowder-soaked-toast-any-chef-would-want-to-claim.html}{Gabrielle
Hamilton},
\href{https://cityroom.blogs.nytimes3xbfgragh.onion/2012/07/03/chefs-picnic-sara-jenkins/}{Sara
Jenkins},
\href{https://www.nytimes3xbfgragh.onion/2016/10/26/dining/beatrice-inn-review.html}{Angie
Mar} and
\href{https://www.nytimes3xbfgragh.onion/2017/02/07/dining/white-gold-butchers-review-april-bloomfield-restaurant.html}{April
Bloomfield}, another River Cafe alumna.

At King, the vision extends to how to drink. Annie Shi, who superintends
the dining room and is a third business partner with the chefs, can
offer guidance with the wine list. She favors French and Italian
producers, many of them not quite famous, whose wines gracefully weave
in and around the cooking.

At the compact bar by the front door, cocktails are put together with
the simplicity and respect for aperitif wines that you find in Italy.
There is a kir and a sbagliato, which is nothing more than Campari and
red vermouth on the rocks topped up with prosecco. These and other
drinks slip into the bloodstream without knocking the palate out of
alignment.

The desserts are cafe style. They don't look like extraterrestrial
landscapes but rather recognizable slices and scoops.

One of the few things at King that didn't make perfect sense was a
tiramisù; it went too heavy on the espresso and too light on the
mascarpone. Every other dessert was just what I wanted, even when I
didn't know I wanted it. Chilled, thickened cream flavored with Pernod?
I'm a fan now. I'm also a new convert to something called the Colonel.
It's a cup of lemon granita served with a tiny pitcher of cold vodka.
You pour one over the other.

I have no idea why the vodka makes the granita taste better, but it
does.

\href{https://www.facebookcorewwwi.onion/nytfood/}{\emph{Follow NYT Food
on Facebook}}\emph{,}
\href{https://instagram.com/nytfood}{\emph{Instagram}}\emph{,}
\href{https://twitter.com/nytfood}{\emph{Twitter}} \emph{and}
\href{https://www.pinterest.com/nytfood/}{\emph{Pinterest}}\emph{.}
\href{https://www.nytimes3xbfgragh.onion/newsletters/cooking}{\emph{Get
regular updates from NYT Cooking, with recipe suggestions, cooking tips
and shopping advice}}\emph{.}

Advertisement

\protect\hyperlink{after-bottom}{Continue reading the main story}

\hypertarget{site-index}{%
\subsection{Site Index}\label{site-index}}

\hypertarget{site-information-navigation}{%
\subsection{Site Information
Navigation}\label{site-information-navigation}}

\begin{itemize}
\tightlist
\item
  \href{https://help.nytimes3xbfgragh.onion/hc/en-us/articles/115014792127-Copyright-notice}{©~2020~The
  New York Times Company}
\end{itemize}

\begin{itemize}
\tightlist
\item
  \href{https://www.nytco.com/}{NYTCo}
\item
  \href{https://help.nytimes3xbfgragh.onion/hc/en-us/articles/115015385887-Contact-Us}{Contact
  Us}
\item
  \href{https://www.nytco.com/careers/}{Work with us}
\item
  \href{https://nytmediakit.com/}{Advertise}
\item
  \href{http://www.tbrandstudio.com/}{T Brand Studio}
\item
  \href{https://www.nytimes3xbfgragh.onion/privacy/cookie-policy\#how-do-i-manage-trackers}{Your
  Ad Choices}
\item
  \href{https://www.nytimes3xbfgragh.onion/privacy}{Privacy}
\item
  \href{https://help.nytimes3xbfgragh.onion/hc/en-us/articles/115014893428-Terms-of-service}{Terms
  of Service}
\item
  \href{https://help.nytimes3xbfgragh.onion/hc/en-us/articles/115014893968-Terms-of-sale}{Terms
  of Sale}
\item
  \href{https://spiderbites.nytimes3xbfgragh.onion}{Site Map}
\item
  \href{https://help.nytimes3xbfgragh.onion/hc/en-us}{Help}
\item
  \href{https://www.nytimes3xbfgragh.onion/subscription?campaignId=37WXW}{Subscriptions}
\end{itemize}
