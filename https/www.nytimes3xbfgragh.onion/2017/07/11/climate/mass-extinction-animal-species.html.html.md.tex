Sections

SEARCH

\protect\hyperlink{site-content}{Skip to
content}\protect\hyperlink{site-index}{Skip to site index}

\href{https://www.nytimes3xbfgragh.onion/section/climate}{Climate}

\href{https://myaccount.nytimes3xbfgragh.onion/auth/login?response_type=cookie\&client_id=vi}{}

\href{https://www.nytimes3xbfgragh.onion/section/todayspaper}{Today's
Paper}

\href{/section/climate}{Climate}\textbar{}Era of `Biological
Annihilation' Is Underway, Scientists Warn

\url{https://nyti.ms/2vaaBMI}

\begin{itemize}
\item
\item
\item
\item
\item
\end{itemize}

\hypertarget{climate-and-environment}{%
\subsubsection{\texorpdfstring{\href{https://www.nytimes3xbfgragh.onion/section/climate?name=styln-climate\&region=TOP_BANNER\&variant=undefined\&block=storyline_menu_recirc\&action=click\&pgtype=Article\&impression_id=52ca3100-e39b-11ea-84b4-99a32c8d6878}{Climate
and
Environment}}{Climate and Environment}}\label{climate-and-environment}}

\begin{itemize}
\tightlist
\item
  \href{https://www.nytimes3xbfgragh.onion/2020/08/17/climate/alaska-oil-drilling-anwr.html?name=styln-climate\&region=TOP_BANNER\&variant=undefined\&block=storyline_menu_recirc\&action=click\&pgtype=Article\&impression_id=52ca3101-e39b-11ea-84b4-99a32c8d6878}{Arctic
  Refuge}
\item
  \href{https://www.nytimes3xbfgragh.onion/interactive/2020/climate/trump-environment-rollbacks.html?name=styln-climate\&region=TOP_BANNER\&variant=undefined\&block=storyline_menu_recirc\&action=click\&pgtype=Article\&impression_id=52ca3102-e39b-11ea-84b4-99a32c8d6878}{Trump's
  Changes}
\item
  \href{https://www.nytimes3xbfgragh.onion/interactive/2020/04/19/climate/climate-crash-course-1.html?name=styln-climate\&region=TOP_BANNER\&variant=undefined\&block=storyline_menu_recirc\&action=click\&pgtype=Article\&impression_id=52ca3103-e39b-11ea-84b4-99a32c8d6878}{Climate
  101}
\item
  \href{https://www.nytimes3xbfgragh.onion/interactive/2018/08/30/climate/how-much-hotter-is-your-hometown.html?name=styln-climate\&region=TOP_BANNER\&variant=undefined\&block=storyline_menu_recirc\&action=click\&pgtype=Article\&impression_id=52ca5810-e39b-11ea-84b4-99a32c8d6878}{Is
  Your Hometown Hotter?}
\end{itemize}

Advertisement

\protect\hyperlink{after-top}{Continue reading the main story}

Supported by

\protect\hyperlink{after-sponsor}{Continue reading the main story}

\hypertarget{era-of-biological-annihilation-is-underway-scientists-warn}{%
\section{Era of `Biological Annihilation' Is Underway, Scientists
Warn}\label{era-of-biological-annihilation-is-underway-scientists-warn}}

\includegraphics{https://static01.graylady3jvrrxbe.onion/images/2017/07/12/world/12CLIMATEPOD-COMBO/CLIMATEPOD-COMBO-articleInline.jpg?quality=75\&auto=webp\&disable=upscale}

By
\href{https://www.nytimes3xbfgragh.onion/by/tatiana-schlossberg}{Tatiana
Schlossberg}

\begin{itemize}
\item
  July 11, 2017
\item
  \begin{itemize}
  \item
  \item
  \item
  \item
  \item
  \end{itemize}
\end{itemize}

\href{https://cn.nytstyle.com/science/20170712/mass-extinction-animal-species}{阅读简体中文版}\href{https://www.nytimes3xbfgragh.onion/es/2017/07/14/aniquilacion-biologica-extincion-especies}{Leer
en español}

From the common barn swallow to the exotic giraffe, thousands of animal
species are in precipitous decline, a sign that an irreversible era of
mass extinction is underway, new research finds.

\href{http://m.pnas.org/content/early/2017/07/05/1704949114.full.pdf}{The
study}, published Monday in the Proceedings of the National Academy of
Sciences, calls the current decline in animal populations a ``global
epidemic'' and part of the ``ongoing
\href{http://www.newyorker.com/magazine/2009/05/25/the-sixth-extinction}{sixth
mass extinction}'' caused in large measure by human destruction of
animal habitats. The previous five extinctions were caused by natural
phenomena.

Gerardo Ceballos, a researcher at the Universidad Nacional Autónoma de
México in Mexico City, acknowledged that the study is written in
unusually alarming tones for an academic research paper. ``It wouldn't
be ethical right now not to speak in this strong language to call
attention to the severity of the problem,'' he said.

Dr. Ceballos emphasized that he and his co-authors,
\href{https://ccb.stanford.edu/paul-r-ehrlich}{Paul R. Ehrlich} and
Rodolfo Dirzo, both professors at Stanford University, are not
alarmists, but are using scientific data to back up their assertions
that significant population decline and possible mass extinction of
species all over the world may be imminent, and that both have been
underestimated by many other scientists.

The study's authors looked at reductions in a species' range --- a
result of factors like habitat degradation, pollution and climate
change, among others --- and extrapolated from that how many populations
have been lost or are in decline, a method that they said
\href{http://www.iucnredlist.org/static/categories_criteria_3_1}{is used
by the International Union for Conservation of Nature}.

\href{https://www.nytimes3xbfgragh.onion/section/climate?action=click\&pgtype=Article\&state=default\&region=MAIN_CONTENT_1\&context=storylines_keepup}{}

\hypertarget{climate-and-environment-}{%
\subsubsection{Climate and Environment
›}\label{climate-and-environment-}}

\hypertarget{keep-up-on-the-latest-climate-news}{%
\paragraph{Keep Up on the Latest Climate
News}\label{keep-up-on-the-latest-climate-news}}

Updated Aug. 18, 2020

Here's what you need to know this week:

\begin{itemize}
\item
  \begin{itemize}
  \tightlist
  \item
    Five automakers
    \href{https://www.nytimes3xbfgragh.onion/2020/08/17/climate/california-automakers-pollution.html?action=click\&pgtype=Article\&state=default\&region=MAIN_CONTENT_1\&context=storylines_keepup}{sealed
    a binding agreement} with California to follow the state's stricter
    tailpipe emissions rules.
  \item
    The Trump
    administration\href{https://www.nytimes3xbfgragh.onion/2020/08/13/climate/trump-methane.html?action=click\&pgtype=Article\&state=default\&region=MAIN_CONTENT_1\&context=storylines_keepup}{eliminated
    a major methane rule}, even as leaks are worsening, in a decision
    that researchers warned ignored science.
  \item
    Climate change leaders said
    \href{https://www.nytimes3xbfgragh.onion/2020/08/12/climate/kamala-harris-environmental-justice.html?action=click\&pgtype=Article\&state=default\&region=MAIN_CONTENT_1\&context=storylines_keepup}{the
    vice-presidential choice of Kamala Harris} signaled that Democrats
    will have a focus on environmental justice.
  \end{itemize}
\end{itemize}

They found that about 30 percent of all land vertebrates --- mammals,
birds, reptiles and amphibians --- are experiencing declines and local
population losses. In most parts of the world, mammal populations are
losing 70 percent of their members because of habitat loss.

In particular, they cite cheetahs, which have declined to around 7,000
members; Borneo and Sumatran orangutans, of which fewer than 5,000
remain; populations of African lions, which have declined by 43 percent
since 1993; pangolins, which have been ``decimated''; and
\href{https://www.nytimes3xbfgragh.onion/2016/09/09/science/a-quadruple-take-on-the-giraffe-its-four-species-not-one.html}{giraffes,
whose four species} now number under 100,000 members.

The study defines populations as the number of individuals in a given
species in a 10,000-square-kilometer unit of habitat, known as a
quadrat.

Jonathan Losos, a biology professor at Harvard, said that he was not
aware of other papers that have used this method, but that it was ``a
reasonable first pass'' at estimating the extent of species decline and
population loss.

Dr. Losos also noted that giving precise estimates of wildlife
populations was difficult, in part because scientists do not always
agree on what defines a population, which makes the question inherently
subjective.

Despite those issues, Dr. Losos said, ``I think it's a very important
and troubling paper that documents that the problems we have with
biodiversity are much greater than commonly thought.''

The authors of the paper suggest that previous estimates of global
extinction rates have been too low, in part because scientists have been
too focused on complete extinction of a species. Two vertebrate species
are estimated to go extinct every year, which the authors wrote ``does
not generate enough public concern,'' and lends the impression that many
species are not severely threatened, or that mass extinction is a
distant catastrophe.

Conservatively, scientists estimate that 200 species have gone extinct
in the past 100 years; the ``normal'' extinction rate over the past two
million years has been that two species go extinct every 100 years
because of evolutionary and other factors.

Rather than extinctions, the paper looks at how populations are doing:
the disappearance of entire populations, and the decrease of the number
of individuals within a population. Over all, they found this phenomenon
is occurring globally, but that tropical regions, which have the
greatest biodiversity, are experiencing the greatest loss in numbers,
and that temperate regions are seeing higher proportions of population
loss. Dr. Ehrlich, who rose to prominence in the 1960s after he wrote
``\href{https://www.nytimes3xbfgragh.onion/2015/06/01/us/the-unrealized-horrors-of-population-explosion.html?mcubz=1}{The
Population Bomb},'' a book that predicted the imminent collapse of
humanity because of overpopulation, said he saw a similar phenomenon in
the animal world as a result of human activity.

``There is only one overall solution, and that is to reduce the scale of
the human enterprise,'' he said. ``Population growth and increasing
consumption among the rich is driving it.''

He and Dr. Ceballos said that habitat destruction --- deforestation for
agriculture, for example --- and pollution were the primary culprits,
but that climate change exacerbates both problems. Accelerating
deforestation and rising carbon pollution are likely to make climate
change worse, which could have disastrous consequences for the ability
of many species to survive on earth.

Dr. Ceballos struck a slightly more hopeful tone, adding that some
species have been able to rebound when some of these pressures are taken
away.

Dr. Ehrlich, however, continued to sound the alarm. ``We're toxifying
the entire planet,'' he said.

When asked about the clear advocacy position the paper has taken, a
rarity in scientific literature, he said, ``Scientists don't give up
their responsibility as citizens to say what they think about the data
that they're gathering.''

Advertisement

\protect\hyperlink{after-bottom}{Continue reading the main story}

\hypertarget{site-index}{%
\subsection{Site Index}\label{site-index}}

\hypertarget{site-information-navigation}{%
\subsection{Site Information
Navigation}\label{site-information-navigation}}

\begin{itemize}
\tightlist
\item
  \href{https://help.nytimes3xbfgragh.onion/hc/en-us/articles/115014792127-Copyright-notice}{©~2020~The
  New York Times Company}
\end{itemize}

\begin{itemize}
\tightlist
\item
  \href{https://www.nytco.com/}{NYTCo}
\item
  \href{https://help.nytimes3xbfgragh.onion/hc/en-us/articles/115015385887-Contact-Us}{Contact
  Us}
\item
  \href{https://www.nytco.com/careers/}{Work with us}
\item
  \href{https://nytmediakit.com/}{Advertise}
\item
  \href{http://www.tbrandstudio.com/}{T Brand Studio}
\item
  \href{https://www.nytimes3xbfgragh.onion/privacy/cookie-policy\#how-do-i-manage-trackers}{Your
  Ad Choices}
\item
  \href{https://www.nytimes3xbfgragh.onion/privacy}{Privacy}
\item
  \href{https://help.nytimes3xbfgragh.onion/hc/en-us/articles/115014893428-Terms-of-service}{Terms
  of Service}
\item
  \href{https://help.nytimes3xbfgragh.onion/hc/en-us/articles/115014893968-Terms-of-sale}{Terms
  of Sale}
\item
  \href{https://spiderbites.nytimes3xbfgragh.onion}{Site Map}
\item
  \href{https://help.nytimes3xbfgragh.onion/hc/en-us}{Help}
\item
  \href{https://www.nytimes3xbfgragh.onion/subscription?campaignId=37WXW}{Subscriptions}
\end{itemize}
