Sections

SEARCH

\protect\hyperlink{site-content}{Skip to
content}\protect\hyperlink{site-index}{Skip to site index}

\href{https://www.nytimes3xbfgragh.onion/section/food}{Food}

\href{https://myaccount.nytimes3xbfgragh.onion/auth/login?response_type=cookie\&client_id=vi}{}

\href{https://www.nytimes3xbfgragh.onion/section/todayspaper}{Today's
Paper}

\href{/section/food}{Food}\textbar{}Flavors Cross Borders at Little Tong
Noodle Shop

\url{https://nyti.ms/2v8TvPf}

\begin{itemize}
\item
\item
\item
\item
\item
\item
\end{itemize}

Advertisement

\protect\hyperlink{after-top}{Continue reading the main story}

Supported by

\protect\hyperlink{after-sponsor}{Continue reading the main story}

\href{/column/restaurant-review}{Restaurant Review}

\hypertarget{flavors-cross-borders-at-little-tong-noodle-shop}{%
\section{Flavors Cross Borders at Little Tong Noodle
Shop}\label{flavors-cross-borders-at-little-tong-noodle-shop}}

\href{https://www.nytimes3xbfgragh.onion/slideshow/2017/07/11/dining/little-tong-noodle-shop-nyc.html}{}

\hypertarget{little-tong-noodle-shop}{%
\subsection{Little Tong Noodle Shop}\label{little-tong-noodle-shop}}

11 Photos

View Slide Show ›

\includegraphics{https://static01.graylady3jvrrxbe.onion/images/2017/07/12/dining/12REST-LITTLETONG-slide-QYJ6/12REST-LITTLETONG-slide-QYJ6-articleLarge.jpg?quality=75\&auto=webp\&disable=upscale}

Ramsay de Give for The New York Times

\begin{itemize}
\tightlist
\item
  Little Tong Noodle Shop\\
  ★★ Chinese \$ 177 First Avenue 929-367-8664
\end{itemize}

By \href{http://www.nytimes3xbfgragh.onion/by/pete-wells}{Pete Wells}

\begin{itemize}
\item
  July 11, 2017
\item
  \begin{itemize}
  \item
  \item
  \item
  \item
  \item
  \item
  \end{itemize}
\end{itemize}

Wheat noodles blur at the edges. At their core, they taste of cooked
flour, or flour and egg, but toward the surface, they take on the flavor
of their sauce. Think of pappardelle allied with a meaty ragù, or spools
of ramen entwined with pork broth, throwing off starch, drinking in
soup. Their boundaries dissolve.

Rice noodles have very little flavor of their own and generally don't
take on much, either. They tend to make a hard break with their
surrounding flavors; the more intense those flavors, the more welcome
the break. They are almost pure texture.

The favorite noodle in the Yunnan Province of southwestern China is a
long, round, spaghetti-like rice noodle called mixian. It is the
specialty at \href{https://www.littletong.com/}{Little Tong Noodle
Shop}, a small, new, inexpensive restaurant in the East Village. Little
Tong's mixian seems to have a barely perceptible sour tang, but it is
the very soft and elusively smooth texture that is most memorable. Just
when your mouth is about to get some purchase on the noodles, away they
slip.

Little Tong Noodle Shop plunges mixian into soups and sauces that make a
definite impression. Yunnan shares borders with Tibet, Vietnam, Myanmar
and Laos; northern Thailand is not too far, either. The characteristic
Southeast Asian tension of saltiness, sourness and chile heat creeps
into these noodle bowls. They are buoyed by green herbs, too, by mint
and cilantro, and by the tendency of the chef, Simone Tong, to twist
contrasting strands of flavor together.

Fundamentally, the banna shrimp mixian is a tomato soup, removed from
Campbell's reach by a whiff of smoke and a slight taste of shrimp and
crab shells. Wheels of pickled green chiles add not just heat but depth;
there are fried shallots, too, and fresh mint, stirred in along with
coconut.

\includegraphics{https://static01.graylady3jvrrxbe.onion/images/2017/07/12/dining/12REST1/12REST1-articleInline.jpg?quality=75\&auto=webp\&disable=upscale}

In the grandma chicken mixian, a dark and grainy oil slick --- a
distillation of garlic and black sesame --- dyes and deepens a chicken
broth that is already so full-bodied it's almost sticky. Resting on top
of the rice noodles is chicken confit, which, in terms of moistness and
flavor, is the exact opposite of the desiccated, used-up meat typically
found in chicken soups. There are spicy red bits of fermented chile,
some salty and sour pickles, an egg boiled in pu-erh tea, and long ivory
chrysanthemum petals. It has to be the most interesting chicken noodle
soup in the city right now.

Another soup, the ``little pot'' mixian, tastes something like a spicy
miso ramen that has shed its excess weight and picked up extra flavors.
Chopped shiitakes and pork belly sit on the surface of the salty, milky
pork broth, along with garlic chives and minced stems of pickled mustard
greens. If this had been ramen, I would have left some noodles behind,
but I finished this bowl, all of it, in an astonished haze, and didn't
regret it.

Little Tong Noodle Shop is Ms. Tong's first restaurant as chef. She is
also an owner. It is a modest place in many ways. The drinks list offers
one red, one white, one rosé and a handful of beers and sakes. (The best
choice may be Yunnanese tea.)

There are 28 indoor seats, some pale wood slats on one wall, and exposed
brick on another. The Noodle Shop does not necessarily invite
comparisons to a more famous Noodle Bar three doors south on First
Avenue, but it does not exactly discourage comparisons to David Chang's
restaurant, either.

Like Mr. Chang, who cooked for
\href{https://www.nytimes3xbfgragh.onion/2017/05/05/well/eat/farmers-market-shopping-with-the-chef-tom-colicchio.html}{Tom
Colicchio} before opening the first Momofuku, Ms. Tong threaded a path
toward Asian noodles that ran through modern New York restaurant
kitchens. Born in Chengdu, China, she went to culinary school in the
United States. She worked as an intern for Masato Shimizu while he was
still at 15 East, and cooked for Wylie Dufresne at
\href{https://www.nytimes3xbfgragh.onion/2014/06/11/dining/chef-says-hell-close-wd-50.html}{WD-50}
and
\href{http://www.nytimes3xbfgragh.onion/2013/07/10/dining/reviews/restaurant-review-alder-in-the-east-village.html}{Alder}.

It's easier to see an American chef's sensibility in the section of the
menu called ``little eats,'' which are free of noodles and give Ms. Tong
a chance to evoke Yunnan with East Coast ingredients. She coaxes
impressive flavors from a small amount of cold Chinese broccoli dressed
in citrus soy, with a salty, crunchy and smoky dusting of grated egg
yolk. The ``mini stir fry,'' a seasonal improvisation, was made recently
with fiddlehead ferns, pine nuts and tender, lime-green needles from the
spring tips of spruce branches.

They eat flowers in Yunnan, and Ms. Tong seems to recognize the value,
in the age of Instagram, of a strategically deployed blossom or two.
(They eat insects in Yunnan, too, but you don't see her garnishing
plates with deep-fried bamboo grubs.) In the case of the excellent egg
drop soup, the stray petals had less presence than the corn kernels and
salmon roe and the rags of cooked egg whites in the translucent broth.

Image

Simone Tong, the chef.Credit...Ramsay de Give for The New York Times

But I liked the peppery bite that chrysanthemum brought to the grandma
chicken mixian and found the rose petals on the Yunnan salad to be more
than just decorative. You will think this salad belongs in a trattoria
when I tell you that it is composed mainly of radicchio and prosciutto,
but no Italian would then go on to add fresh pineapple and boiled beef
tendon.

One night I ate peanut butter cookies for dessert. I've never seen them
again. Where in Yunnan they came from, I don't know; where they've gone,
I can only guess. My latest meal there ended with six flavors of fruit
sorbet in gumball-size scoops topped, of course, with edible flowers.

The noodles cost either \$14 or \$15, except for one special, which is
\$16. Every appetizer is under \$10 except the chopped raw beef, which
is dressed with shallot oil and chile oil and can be spread on flaky,
oily scallion bread with a daub of Sichuan-spiced beef fat or a swipe of
smoked egg yolk. For this I paid \$13 without griping. In fact, because
Little Tong's prices include service, they all seemed too low to be
real.

I wouldn't mind if Ms. Tong asked \$16 or \$17 for noodles, rather than
charging \$1 each for chile oil, chile soy sauce and fermented chiles.
People in Yunnan like to doctor their broth to taste from a whole range
of condiments and spices; it's part of the experience. Ms. Tong has a
precise palate, I think, but there were times when I wanted to
recalibrate the broth.

Little Tong Noodle Shop is strict about not seating incomplete parties,
even when the place is empty. Once you have a table, though, the servers
are quick and sympathetic.

The dining room is overseen by Emmeline Zhao, who has the twin titles of
general manager and panda caretaker. One of the sakes comes in a glass
decorated with painted pandas, like an old Welch's jelly jar. Panda art
hangs in the front window, and if you are ever called on to administer
the Heimlich maneuver to a panda, the custom-drawn choking poster by the
restroom door should provide all the information you need.

\href{https://www.facebookcorewwwi.onion/nytfood/}{\emph{Follow NYT Food
on Facebook}}\emph{,}
\href{https://instagram.com/nytfood}{\emph{Instagram}}\emph{,}
\href{https://twitter.com/nytfood}{\emph{Twitter}} \emph{and}
\href{https://www.pinterest.com/nytfood/}{\emph{Pinterest}}\emph{.}
\href{https://www.nytimes3xbfgragh.onion/newsletters/cooking}{\emph{Get
regular updates from NYT Cooking, with recipe suggestions, cooking tips
and shopping advice}}\emph{.}

Advertisement

\protect\hyperlink{after-bottom}{Continue reading the main story}

\hypertarget{site-index}{%
\subsection{Site Index}\label{site-index}}

\hypertarget{site-information-navigation}{%
\subsection{Site Information
Navigation}\label{site-information-navigation}}

\begin{itemize}
\tightlist
\item
  \href{https://help.nytimes3xbfgragh.onion/hc/en-us/articles/115014792127-Copyright-notice}{©~2020~The
  New York Times Company}
\end{itemize}

\begin{itemize}
\tightlist
\item
  \href{https://www.nytco.com/}{NYTCo}
\item
  \href{https://help.nytimes3xbfgragh.onion/hc/en-us/articles/115015385887-Contact-Us}{Contact
  Us}
\item
  \href{https://www.nytco.com/careers/}{Work with us}
\item
  \href{https://nytmediakit.com/}{Advertise}
\item
  \href{http://www.tbrandstudio.com/}{T Brand Studio}
\item
  \href{https://www.nytimes3xbfgragh.onion/privacy/cookie-policy\#how-do-i-manage-trackers}{Your
  Ad Choices}
\item
  \href{https://www.nytimes3xbfgragh.onion/privacy}{Privacy}
\item
  \href{https://help.nytimes3xbfgragh.onion/hc/en-us/articles/115014893428-Terms-of-service}{Terms
  of Service}
\item
  \href{https://help.nytimes3xbfgragh.onion/hc/en-us/articles/115014893968-Terms-of-sale}{Terms
  of Sale}
\item
  \href{https://spiderbites.nytimes3xbfgragh.onion}{Site Map}
\item
  \href{https://help.nytimes3xbfgragh.onion/hc/en-us}{Help}
\item
  \href{https://www.nytimes3xbfgragh.onion/subscription?campaignId=37WXW}{Subscriptions}
\end{itemize}
