Sections

SEARCH

\protect\hyperlink{site-content}{Skip to
content}\protect\hyperlink{site-index}{Skip to site index}

\href{https://www.nytimes3xbfgragh.onion/section/world/asia}{Asia
Pacific}

\href{https://myaccount.nytimes3xbfgragh.onion/auth/login?response_type=cookie\&client_id=vi}{}

\href{https://www.nytimes3xbfgragh.onion/section/todayspaper}{Today's
Paper}

\href{/section/world/asia}{Asia Pacific}\textbar{}Ruling Threatens Hong
Kong's Independence From China

\url{https://nyti.ms/2ufDHOc}

\begin{itemize}
\item
\item
\item
\item
\item
\end{itemize}

Advertisement

\protect\hyperlink{after-top}{Continue reading the main story}

Supported by

\protect\hyperlink{after-sponsor}{Continue reading the main story}

\hypertarget{ruling-threatens-hong-kongs-independence-from-china}{%
\section{Ruling Threatens Hong Kong's Independence From
China}\label{ruling-threatens-hong-kongs-independence-from-china}}

\includegraphics{https://static01.graylady3jvrrxbe.onion/images/2017/07/15/world/15hongkong-1/15hongkong-1-articleLarge.jpg?quality=75\&auto=webp\&disable=upscale}

By Alan Wong

\begin{itemize}
\item
  July 14, 2017
\item
  \begin{itemize}
  \item
  \item
  \item
  \item
  \item
  \end{itemize}
\end{itemize}

\href{https://cn.nytimes3xbfgragh.onion/china/20170717/hong-kong-court-pro-democracy-lawmakers/}{阅读简体中文版}

HONG KONG --- Nearly three years after sweeping pro-democracy protests
filled the streets of Hong Kong, a local court delivered the struggling
movement a severe blow on Friday,
\href{http://www.nytimes3xbfgragh.onion/2016/12/02/world/asia/hong-kong-lawmakers-democracy.html}{removing
four legislators} from office and assuring China greater influence over
the city's government.

The pro-democracy lawmakers were dismissed from the
\href{http://www.legco.gov.hk/index.html}{Hong Kong Legislative Council}
because they had used unacceptable words or even dubious tones in taking
oaths of office that require declarations of loyalty to China. The
ruling means that democracy advocates in the semiautonomous city's
legislature will no longer have enough votes to block legislation from
their pro-Beijing counterparts.

``Voters entrusted us with the task of monitoring the government,'' said
Leung Kwok-hung, one of those unseated. ``We've lost that power.''

Hong Kong has been rattled by episodes that have raised fears that China
is reaching deeper into the city to enforce its will. A bookseller who
sold lurid titles about China's leaders was
\href{https://www.nytimes3xbfgragh.onion/2016/11/05/world/asia/hong-kong-china-booksellers-pen.html}{abducted
and taken to mainland China}.
\href{https://www.nytimes3xbfgragh.onion/2017/02/10/world/asia/xiao-jianhua-hong-kong-disappearance.html}{Xiao
Jianhua}, a prominent billionaire who grew up in China, was snatched
from a high-end hotel and brought to the mainland.

And when
\href{https://www.nytimes3xbfgragh.onion/2017/07/01/world/asia/hong-kong-china-xi-jinping.html}{President
Xi Jinping of China visited Hong Kong} two weeks ago for the 20th
anniversary of its return to Chinese sovereignty, he mixed reassurances
about the city's special status with an unmistakable warning not to test
Beijing's will.

Since 1997, China's economy has become less dependent on Hong Kong,
while the territory's prosperity has become more entangled with the
mainland. As the political and economic power imbalance grows, and Hong
Kong is drawn tighter into China's orbit, Beijing's leaders are less
willing to offer Hong Kong the same degree of deference they once showed
it.

But through it all, people in Hong Kong have been comforted by the fact
that the city has its own legal system --- based on British common law,
unlike China's --- that is proudly independent. That legal system
remains robust, and it is one reason investment flows to Hong Kong. But
Friday's ruling will deepen worries that Chinese influence is weakening
judicial protections.

Since pro-democracy protesters occupied major streets in Hong Kong for
months in 2014 --- a movement that came to be known as the Umbrella
Revolution --- Mr. Xi's government has sought to strengthen its grip on
the city. But the democratic lawmakers held enough seats in the
legislature to frustrate the city's pro-Beijing administration with
filibustering and veto power over bills introduced by pro-Beijing
lawmakers.

The court ruling was ``a disturbing and ominous development,'' said
Willy Lam, a political analyst and adjunct professor at the Chinese
University of Hong Kong. Like many critics of the decision, he suggested
that the judge had bent over backward to create a decision pleasing to
Beijing.

``It's a direct interference in Hong Kong's internal affairs, a breach
of both its judicial independence and separation of powers,'' Mr. Lam
said.

The ruling could galvanize opposition groups in Hong Kong. On Friday
night, hundreds of protesters gathered outside the Legislative Council,
a concrete and glass edifice near Victoria Harbor, to denounce the
decision.

But for now, Hong Kong's pro-democracy parties have been forced into
retreat after a buoyant showing in local elections last year, and some
of the lawmakers who were removed may privately rue turning their
oath-taking into protests.

Hong Kong returned to Chinese sovereignty from British rule in 1997.
Under terms agreed upon by London and Beijing, Hong Kong retained its
own legal system, as well as the Legislative Council.

When their protest movement in 2014 failed to bring about freer local
elections, Hong Kong's democracy campaigners set their sights on
maintaining enough members in the council to thwart policies they saw as
weakening the city's separate status.

The voters did not disappoint. In September, people turned up at polling
stations in record numbers, electing many of the protesters who led
rallies and spent nights in tents.

It was a triumphant moment for the activists, and the message was clear:
Hong Kong people reject Chinese encroachment on their city's freedoms.
The next month, in the grand chamber of the Legislative Council, the
newly elected legislators took the oath of office.

That's when the troubles began.

First, the authorities came for the separatists. In November, the
\href{https://www.nytimes3xbfgragh.onion/2016/11/08/world/asia/china-hong-kong-sixtus-leung-yau-wai-ching.html}{Chinese
government took the extraordinary step} of blocking
\href{http://www.nytimes3xbfgragh.onion/2016/09/02/world/asia/hong-kong-elections-legco.html}{Sixtus
Leung}, known as Baggio, and
\href{https://www.nytimes3xbfgragh.onion/2016/11/05/world/asia/hong-kong-yau-wai-ching.html}{Yau
Wai-ching}, advocates for an independent Hong Kong, from assuming office
as legislators, ostensibly because they inserted anti-China snubs into
their oaths of office.

It did so by issuing a legal interpretation of the Basic Law, the
charter ensuring that Hong Kong is governed according to a ``one
country, two systems'' principle and that the judiciary remains
independent for at least half a century from when the city returned to
Chinese rule. The interpretation orders that legislators who deliver an
oath in an ``insincere or undignified manner'' must be barred from
office and not be given a chance to do it again.

The purge continued on Friday. The court removed the four additional
legislators based on the interpretation and precedent set in the removal
of Mr. Leung and Ms. Yau, arguing that they, too, had failed to take the
oath properly.

The removed legislators include Nathan Law, a leader of the 2014
protests who later founded the party Demosisto with fellow protester
Joshua Wong.

``It's flagrant political suppression by the government,'' Mr. Law said.
``I had read the oath completely, and the Legislative Council approved
it. It only became an issue after Beijing's interpretation.''

Mr. Law, 24, had begun his oath saying he would ``never serve a regime
that murders its own people'' and read the Cantonese word for ``China''
with an upward inflection, as if asking a question. He was the youngest
person ever to win a legislative seat.

``By adopting a rising intonation, Mr. Law was objectively expressing a
doubt on or disrespect of the status of the People's Republic of China
as Hong Kong's legitimate sovereign country,'' the judgment said.

The three other legislators who were unseated, Leung Kwok-hung, Lau
Siu-lai and Edward Yiu, had delivered their oaths with various displays
of defiance, including by reading extremely slowly, inserting words
calling for democracy and displaying props. Likewise, their oaths were
declared invalid by the court, and they have been asked to pack up in
two weeks.

The four disqualified legislators may not be the last to be removed,
since at least four other pro-democracy legislators used props or made
defiant speeches before or after delivering the oath of office.

``They played with fire and got burned,'' said Priscilla Leung, vice
chairwoman of the pro-Beijing party Business and Professionals Alliance
for Hong Kong.

Ms. Leung told reporters she might introduce a bill to amend legislative
rules to prevent filibustering by opposition legislators, though she
declined to offer a timeline. ``We've been discussing that since I
entered the Legislative Council in 2008, but we hadn't had enough
votes.''

Advertisement

\protect\hyperlink{after-bottom}{Continue reading the main story}

\hypertarget{site-index}{%
\subsection{Site Index}\label{site-index}}

\hypertarget{site-information-navigation}{%
\subsection{Site Information
Navigation}\label{site-information-navigation}}

\begin{itemize}
\tightlist
\item
  \href{https://help.nytimes3xbfgragh.onion/hc/en-us/articles/115014792127-Copyright-notice}{©~2020~The
  New York Times Company}
\end{itemize}

\begin{itemize}
\tightlist
\item
  \href{https://www.nytco.com/}{NYTCo}
\item
  \href{https://help.nytimes3xbfgragh.onion/hc/en-us/articles/115015385887-Contact-Us}{Contact
  Us}
\item
  \href{https://www.nytco.com/careers/}{Work with us}
\item
  \href{https://nytmediakit.com/}{Advertise}
\item
  \href{http://www.tbrandstudio.com/}{T Brand Studio}
\item
  \href{https://www.nytimes3xbfgragh.onion/privacy/cookie-policy\#how-do-i-manage-trackers}{Your
  Ad Choices}
\item
  \href{https://www.nytimes3xbfgragh.onion/privacy}{Privacy}
\item
  \href{https://help.nytimes3xbfgragh.onion/hc/en-us/articles/115014893428-Terms-of-service}{Terms
  of Service}
\item
  \href{https://help.nytimes3xbfgragh.onion/hc/en-us/articles/115014893968-Terms-of-sale}{Terms
  of Sale}
\item
  \href{https://spiderbites.nytimes3xbfgragh.onion}{Site Map}
\item
  \href{https://help.nytimes3xbfgragh.onion/hc/en-us}{Help}
\item
  \href{https://www.nytimes3xbfgragh.onion/subscription?campaignId=37WXW}{Subscriptions}
\end{itemize}
