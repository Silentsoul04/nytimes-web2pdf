Sections

SEARCH

\protect\hyperlink{site-content}{Skip to
content}\protect\hyperlink{site-index}{Skip to site index}

\href{https://www.nytimes3xbfgragh.onion/section/politics}{Politics}

\href{https://myaccount.nytimes3xbfgragh.onion/auth/login?response_type=cookie\&client_id=vi}{}

\href{https://www.nytimes3xbfgragh.onion/section/todayspaper}{Today's
Paper}

\href{/section/politics}{Politics}\textbar{}E.P.A. Promised `a New Day'
for the Agriculture Industry, Documents Reveal

\url{https://nyti.ms/2v9R92m}

\begin{itemize}
\item
\item
\item
\item
\item
\end{itemize}

Advertisement

\protect\hyperlink{after-top}{Continue reading the main story}

Supported by

\protect\hyperlink{after-sponsor}{Continue reading the main story}

\hypertarget{epa-promised-a-new-day-for-the-agriculture-industry-documents-reveal}{%
\section{E.P.A. Promised `a New Day' for the Agriculture Industry,
Documents
Reveal}\label{epa-promised-a-new-day-for-the-agriculture-industry-documents-reveal}}

\includegraphics{https://static01.graylady3jvrrxbe.onion/images/2017/08/19/us/19dc-pesticide/merlin-to-scoop-125733062-141745-articleInline.jpg?quality=75\&auto=webp\&disable=upscale}

By \href{http://www.nytimes3xbfgragh.onion/by/eric-lipton}{Eric Lipton}
and \href{http://www.nytimes3xbfgragh.onion/by/roni-caryn-rabin}{Roni
Caryn Rabin}

\begin{itemize}
\item
  Aug. 18, 2017
\item
  \begin{itemize}
  \item
  \item
  \item
  \item
  \item
  \end{itemize}
\end{itemize}

WASHINGTON --- In the weeks before the Environmental Protection Agency
\href{https://www.nytimes3xbfgragh.onion/2017/03/29/us/politics/epa-insecticide-chlorpyrifos.html?mcubz=3\&_r=0}{decided
to reject its own scientists}' advice to ban a potentially harmful
pesticide, Scott Pruitt, the agency's head, promised farming industry
executives who wanted to keep using the pesticide that it is ``a new
day, and a new future,'' and that he was listening to their pleas.

Details on this meeting and dozens of other meetings in the weeks
leading up to the
\href{https://www.epa.gov/sites/production/files/2017-03/documents/chlorpyrifos3b_order_denying_panna_and_nrdc27s_petitition_to_revoke_tolerances.pdf}{late
March decision by Mr. Pruitt} are contained in more than
\href{https://www.documentcloud.org/documents/3935290-EPA-HQ-2017-005731-Redacted.html}{700
pages of internal agency documents} obtained by The New York Times
through a Freedom of Information request.

Though hundreds of pages describing the deliberations were redacted from
the documents, the internal memos show how the E.P.A.'s new staff,
appointed by President Trump, pushed the agency's career staff to draft
a ruling that would deny
\href{https://www.nrdc.org/sites/default/files/hea_10072201a.pdf}{the
decade-old petition by environmentalists} to ban the pesticide,
\href{https://www.epa.gov/ingredients-used-pesticide-products/chlorpyrifos}{chlorpyrifos}.

Chlorpyrifos is still widely used in agriculture --- on apples, oranges,
strawberries, almonds and many other fruits --- though it was
\href{http://www.nytimes3xbfgragh.onion/2000/06/09/us/epa-citing-risks-to-children-signs-accord-to-limit-insecticide.html}{barred
from residential use in 2000}. The
\href{https://www.regulations.gov/document?D=EPA-HQ-OPP-2015-0653-0001}{E.P.A.'s
scientists have recommended}it be banned from use on farms and produce
because it has
been\href{https://www.mailman.columbia.edu/public-health-now/news/pesticide-chlorpyrifos-linked-childhood-developmental-delays}{linked
to lower I.Q.s and developmental delays} among
\href{http://www.google.com/url?q=http\%3A\%2F\%2Frepository.usfca.edu\%2Fcapstone\%2F529\%2F\&sa=D\&sntz=1\&usg=AFQjCNFl-3uwWNpfUbCJd4f0OuRRVxcvwA}{agricultural
workers and their children}.

\href{https://www.documentcloud.org/documents/3935290-EPA-HQ-2017-005731-Redacted.html\#document/p32/a369099}{At
a March 1 meeting at E.P.A. headquarters with members}of the American
Farm Bureau Federation from Washington State, industry representatives
pressed the E.P.A. not to reduce the number of pesticides available.
They said there were not enough alternative pesticides to chlorpyrifos.
They also said there was a need for ``a reasonable approach to regulate
this pesticide,'' which is widely used in Washington State, and that
they wanted ``the farming community to be more involved in the
process.''

According to the documents, Mr. Pruitt ``stressed that this is a new
day, a new future, for a common-sense approach to environmental
protection.'' He said the new administration ``is looking forward to
working closely with the agricultural community.''

Three days before Donald J. Trump's inauguration, Dow Chemical had
separately
\href{https://www.documentcloud.org/documents/3935257-2017-1-17-Dow-Comments-on-Pesticide-Ban-Urging.html}{submitted
a request to the agency to reject} the petition to ban chlorpyrifos,
calling the scientific link between the childhood health issues and the
pesticide unclear, agency records show.

Amy Graham, an E.P.A. spokeswoman, said the denial of the petition to
ban chlorpyrifos was justified. ``Taking emails out of context doesn't
change the fact that we continue to examine the science surrounding
chlorpyrifos,'' she said in a written statement. She added that the
agency was examining ``scientific concerns with the methodology used by
the previous administration.''

The emails show that as late as March 7,
\href{https://www.epa.gov/aboutepa/about-office-chemical-safety-and-pollution-prevention-ocspp}{Wendy
Cleland-Hamnett}, the acting head of the E.P.A.'s office of chemical
safety, was presenting the top political staff with options for how to
handle the decade-old petition from an environmental group requesting
the ban.

``We would talk about impacts of different options in the briefing,''
Ms. Cleland-Hamnett
\href{https://www.documentcloud.org/documents/3935290-EPA-HQ-2017-005731-Redacted.html\#document/p228/a369116}{wrote
in a March 7 email.} The email raised the possibility of a meeting with
Mr. Pruitt to discuss the pesticide, a decision that the E.P.A.'s
political staff had called a ``hot'' regulatory item, given a
court-ordered deadline of late March to rule on the petition.

The next day, Ryan Jackson, Mr. Pruitt's chief of staff, wrote to
another political appointee that he had ``scared'' the agency's career
staff, suggesting that he had made clear the direction that the
political staff wanted to go --- and given the career staff explicit
verbal orders to prepare documents explaining why the agency had shifted
its position.

``I think I did scare them or surprise them,''
\href{https://www.documentcloud.org/documents/3935290-EPA-HQ-2017-005731-Redacted.html\#document/p530/a369101}{Mr.
Jackson wrote to Samantha Dravis}, Mr. Pruitt's political appointee to
oversee E.P.A. policy. ``They are getting us information for Friday but
they know where this is headed and they are documenting it well.''

As the draft of the order rejecting the ban of the petition was being
written, political staff at the E.P.A. continued to organize meetings
with agriculture industry officials. An
\href{https://www.documentcloud.org/documents/3935290-EPA-HQ-2017-005731-Redacted.html\#document/p317/a369104}{email
on March 10 said:} ``Basic info for meeting. Purpose is to reset
relationship with ag leaders.''

When Ms. Cleland-Hamnett wrote back to the political appointees on March
16 to provide a draft of the order rejecting the ban of the pesticide,
\href{https://www.documentcloud.org/documents/3935290-EPA-HQ-2017-005731-Redacted.html\#document/p301/a369109}{she
told her bosses} that ``I think this version will allow you to see how
we're describing the basis for the denial.''

The emails indicate E.P.A. officials closely coordinated their decision
on chlorpyrifos with the White House and the Department of Agriculture,
which is more closely linked with the agriculture industry and
\href{https://www.documentcloud.org/documents/3935263-2017-1-17-USDA-COMMENT-DONT-APPROVE-PETITION.html}{had
questioned the justification for the ban}.

On March 29, as the E.P.A. was about to publicly announce Mr. Pruitt's
decision to forego the ban, an E.P.A. political employee
\href{https://www.documentcloud.org/documents/3935290-EPA-HQ-2017-005731-Redacted.html\#document/p357/a369105}{asked
in an email}, ``Did you run this by Ray Starling at the White House?''
referring to the
\href{https://www.whitehouse.gov/the-press-office/2017/02/27/white-house-national-economic-council-director-announces-senior-staff}{special
assistant to the president} for agriculture, trade and food assistance.

E.P.A. officials wanted to demonstrate in the
\href{https://www.epa.gov/newsreleases/epa-administrator-pruitt-denies-petition-ban-widely-used-pesticide-0}{news
release} that they had the support of the Agriculture Department and the
White House,
\href{https://www.documentcloud.org/documents/3935290-EPA-HQ-2017-005731-Redacted.html\#document/p141/a369102}{writing
in one email}, ``Do you think we could add `With Support from USDA,
Admin....' Into the headline, to show it's a joint release? Or is that
too much?''

Environmental groups said the emails demonstrate that the E.P.A. under
Mr. Pruitt is doing favors for the industry, even if it means
compromising public health.

``What is clear from these documents is that Administrator Pruitt's
abrupt action to vacate the ban on chlorpyrifos was an ideological ---
not a health-based decision,'' said Melanie Benesh, a legislative
attorney at the \href{http://www.ewg.org/}{Environmental Working Group.}
``In fact, the Pruitt E.P.A. has shown time and time again that it seems
to only be willing to act quickly when it comes to dismantling
health-protective rules like the proposed ban on chlorpyrifos at the
behest of industry.''

Advertisement

\protect\hyperlink{after-bottom}{Continue reading the main story}

\hypertarget{site-index}{%
\subsection{Site Index}\label{site-index}}

\hypertarget{site-information-navigation}{%
\subsection{Site Information
Navigation}\label{site-information-navigation}}

\begin{itemize}
\tightlist
\item
  \href{https://help.nytimes3xbfgragh.onion/hc/en-us/articles/115014792127-Copyright-notice}{©~2020~The
  New York Times Company}
\end{itemize}

\begin{itemize}
\tightlist
\item
  \href{https://www.nytco.com/}{NYTCo}
\item
  \href{https://help.nytimes3xbfgragh.onion/hc/en-us/articles/115015385887-Contact-Us}{Contact
  Us}
\item
  \href{https://www.nytco.com/careers/}{Work with us}
\item
  \href{https://nytmediakit.com/}{Advertise}
\item
  \href{http://www.tbrandstudio.com/}{T Brand Studio}
\item
  \href{https://www.nytimes3xbfgragh.onion/privacy/cookie-policy\#how-do-i-manage-trackers}{Your
  Ad Choices}
\item
  \href{https://www.nytimes3xbfgragh.onion/privacy}{Privacy}
\item
  \href{https://help.nytimes3xbfgragh.onion/hc/en-us/articles/115014893428-Terms-of-service}{Terms
  of Service}
\item
  \href{https://help.nytimes3xbfgragh.onion/hc/en-us/articles/115014893968-Terms-of-sale}{Terms
  of Sale}
\item
  \href{https://spiderbites.nytimes3xbfgragh.onion}{Site Map}
\item
  \href{https://help.nytimes3xbfgragh.onion/hc/en-us}{Help}
\item
  \href{https://www.nytimes3xbfgragh.onion/subscription?campaignId=37WXW}{Subscriptions}
\end{itemize}
