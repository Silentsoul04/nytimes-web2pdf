Sections

SEARCH

\protect\hyperlink{site-content}{Skip to
content}\protect\hyperlink{site-index}{Skip to site index}

Who Will Save These Dying Italian Towns?

\url{https://nyti.ms/2xfWPgJ}

\begin{itemize}
\item
\item
\item
\item
\item
\end{itemize}

\includegraphics{https://static01.graylady3jvrrxbe.onion/images/2017/09/10/t-magazine/10tmag-italian/10tmag-italian-articleLarge.gif?quality=75\&auto=webp\&disable=upscale}

\hypertarget{who-will-save-these-dying-italian-towns}{%
\section{Who Will Save These Dying Italian
Towns?}\label{who-will-save-these-dying-italian-towns}}

Near-empty villages try to hold on to an endangered way of life --- and
some of the country's most important artisanal traditions.

Credit...

Supported by

\protect\hyperlink{after-sponsor}{Continue reading the main story}

By Deborah Needleman

\begin{itemize}
\item
  Sept. 7, 2017
\item
  \begin{itemize}
  \item
  \item
  \item
  \item
  \item
  \end{itemize}
\end{itemize}

THE FIRST THING that must be said about the ancient town of Civita di
Bagnoregio, just two hours away from Rome and Florence, is that it is
beautiful. From a distance, it looks literally otherworldly: The town
sits so high atop a perilously steep pinnacle of eroding volcanic rock
that it seems as if it's perched upon clouds rather than tethered to the
earth. Its very sediment is strafed with 2,500 years of architectural
history: Etruscan caves, ancient remains, medieval dwellings and
Renaissance villas.

Originally a center along ancient trade routes, Civita di Bagnoregio was
prosperous from Roman times through the late Middle Ages. But after a
devastating earthquake in 1695, most residents fled for lower ground,
and so began the city's long decline. By the end of World War II, nearly
all of its inhabitants had left in search of work in cities or abroad.
For the last half century, its population has hovered around 10 or so
full-time residents.

Because the erosion of the hill is so severe (houses have been tumbling
off its sides since the 1700s), Civita di Bagnoregio will eventually be
reclaimed by the landscape. Residents and visitors alike must park at
the base and ascend a steep footbridge to enter through a huge Gothic
archway. Past the backless facade of a Renaissance house, with several
of its windows open to the sky like a stage set, lies a small, dusty
piazza with a church, a fine seventh-century medieval tower, a small bar
and not much else. There is no pharmacy or school, no hospital, none of
the necessities that somehow serve to make a place a place. There are
only a couple of inns, and a few restaurants. Civita is real without
being actual, if that makes any sense.

\includegraphics{https://static01.graylady3jvrrxbe.onion/images/2017/08/23/t-magazine/23tmag-italy-slide-W9PJ/23tmag-italy-slide-W9PJ-articleInline.jpg?quality=75\&auto=webp\&disable=upscale}

FOR ALL THE ANCIENT Italian hill towns and villages that delight the
traveler --- the San Gimignanos, Montepulcianos and Fiesoles --- there
are scores of others (many equally or more beautiful) where few venture
and in which very few reside today. According to a 2016 Italian
environmental association report, there are nearly 2,500 rural Italian
villages that are perilously depopulated, some semi-abandoned and others
virtual ghost towns. A primary narrative of Italy in the 20th century
has been what followed the collision of poverty, urbanization, mass
emigration and natural disaster, a confluence of events that has
devastated many towns that had otherwise managed to thrive, or at least
get by, for centuries. These towns, most of which are in the
historically impoverished south, had already lost tens of millions of
inhabitants in the great waves of migration from the late 19th century
to the mid 1970s; in the last 25 years, they lost another 15 percent.
Now, houses and schools sit empty and fields fallow; shops are
unattended.

These rural places were once intricately tied to the countryside around
them, their inhabitants working as farmers and merchants, craftsmen and
shepherds. But when these towns die, it's not just the population that
suffers: so too do the unique traditions and skills associated with each
place, as well as the landscape that supported them. This phenomenon is,
of course, not unique to Italy: Small towns across the developed world,
including in the United States, are left behind as technologies and
economies change, rendering the industries and the know-how that once
sustained them obsolete, forcing their populations to relocate to urban
centers. What is particular to Italy, however, is the exquisite
architectural character of its hill towns, as well as the quality of the
handiwork and traditions that were born, cultivated and perfected here.
These towns and their craftsmanship are what we think of when we think
of Italy --- as fundamental to the country's identity as its important
cities and grand artistic legacies. It isn't far-fetched to say that
what's at risk of being lost with their obsolescence is nothing less
than Italy's rural soul.

But though these towns may represent the essence of Italian history and
the country's artisanal tradition, the government has done little to
help preserve them, aside from declaring 2017 ``The Year of the
Villages'' in hopes of boosting tourism. It has therefore fallen to
locals --- citizens and mayors --- to try to change their fates, often
through inventive, sometimes ingenious, methods that mingle humor with a
deep sorrow and desperation. One picturesque medieval hamlet in Tuscany,
Pratariccia, sold itself on eBay for \$3.1 million several years ago.
Another, Calsazio, tried to follow, offering itself for only \$333,000,
listing the item's condition as ``used.'' In Calabria, the mayor of
Sellia (population 530) signed a decree banning death and illness in his
town, and recently opened an adventure park with a giant zip line he
thought would lure visitors. Most recently, the mayor of Bormida in
Liguria floated a provisional offer on his Facebook page: \$2,100 to
anyone who moved there in order to keep it populated. (There was so much
interest that he had to delete the post.)

Image

One of Sutera's quiet streets.Credit...Domingo Milella

And then there are towns like Civita di Bagnoregio. Like so many of the
others, it has been preserved by the very forces that doomed it: poverty
and abandonment. Unlike the others, however, Civita was saved by having
been ``discovered'' by fashionable Romans (including Gucci creative
director Alessandro Michele) and expats over the last 20 or so years,
who have made summer houses or weekend places of its exceptionally fine,
deserted buildings, drawn by the romance of Civita's remarkable
situation --- and its proximity to Rome. The restoration of the entire
town is eerily pristine; there's nary a yellowing leaf on the potted
geraniums and colorful hydrangeas that grace the exterior of every
perfectly renovated house.

These days, Civita has become a tourist destination for day trippers,
who arrive by the busload and pay a small fee to enter. Sometimes up to
5,000 people a day wander the town, which at its seasonal height sleeps
only about 100. The effect of all these people --- selfie sticks moving
through the air like antennae --- gives the place the unfortunate air of
a Disney set: a hyper-clean, historically accurate medieval town as
realized on a Universal Studios back lot. There is nothing to mar the
scene --- no pizzerias or Starbucks or even cars. And just as one starts
to wonder what kind of town is one in which there are no children or
families, no banks or offices, dusk starts to fall, and the tourists and
the white heat of the day retreat. Things go quiet, the light glows pink
and the ``locals,'' many from Rome and the U.S., start to appear ---
there are drinks on terraces and quiet dinners in the side streets,
conversations in private gardens among neighbors and friends who know
one another, and who all love and care for this enchanted, imperiled
piece of history.

Image

Some of Italy's virtually abandoned villages are right outside of major
cities.Credit...T magazine

BUT THE FARTHER ONE gets from major cities like Florence or Rome, the
more difficult it is to attract weekend tourists. Deep in Sicily, off a
terrible road whose signs resignedly warn of potholes, lies the isolated
town of Sutera, built around the base of a steep mountain. In 2013, at
the behest of its mayor, the town opened its doors --- and its empty
houses --- to survivors of the catastrophic Lampedusa shipwreck, which
killed more than 360 refugees. Sutera's population had dwindled from
5,000 in 1970 to just 1,500, and the mayor recognized the humanitarian
and economic opportunity the migrants could provide for his moribund
town. To help the refugees, most of whom are from sub-Saharan Africa,
integrate into the community, they are paired with local families, and
required to take Italian lessons, given to them by the town's citizens.
(The European Union provides funding for food, clothing and housing,
which can spur the creation of jobs for both migrants and locals.)
Initially, there was some resistance, but that has disappeared with the
energy these newcomers have brought to the area. Today, one can find
young Nigerians taking their morning espresso alongside the old men, and
local children kicking soccer balls in the street with their new
playmates. And each summer the town hosts a daylong festival featuring
the traditional food, music and dance of the immigrants.

One of the first towns to invite migrants into its walls was Riace, in
Calabria, whose mayor, Domenico Lucano, was named one of Fortune's
``World's 50 Greatest Leaders'' last year. By 1998, when it took in a
group of Kurdish refugees, Riace's population had fallen to around 800
from 2,500 after World War II. Today, its population is 1,500, with
migrants from over 20 countries. Some of these are apprenticing
artisans, learning old skills like embroidery, glass mosaic and pottery
that were themselves dying out, and so helping keep Italian culture
alive. Lucano told the BBC, ``The multiculturalism, the variety of
skills and personal stories which people have brought to Riace, have
revolutionized what was becoming a ghost town.'' Other towns have taken
Riace's lead, too: an act of humanity that has become an act of
self-preservation as well.

\includegraphics{https://static01.graylady3jvrrxbe.onion/images/2017/09/09/insider/08-italy-01/08-italy-01-videoSixteenByNineJumbo1600.jpg}

UNLIKE URBAN CENTERS, hill towns were built to be connected to the
countryside, which provided each its particular raison d'être, from its
subsistence to its commerce. Even physically, the towns appear like
natural outcroppings, terraced along the sides of hills, as if sprouting
from the earth beneath them.

In the region of Abruzzo, surrounded by the high peaks of the Apennines,
the stunning fortified medieval hilltop village of Santo Stefano di
Sessanio sits atop a ridge overlooking a dramatic and lush plateau. Once
a bustling center of agriculture and wool production, it began to shrink
when the Italian wool industry went into decline, crippled by
competition from abroad. By the 1990s, the town had only about 100
full-time residents. Santo Stefano is just two hours from Rome and is
surrounded by countryside that resembles the Austrian hills of ``The
Sound of Music'': expansive fields of wildflowers backed by majestic
snowcapped mountains. It is a sublime place for hiking or bicycling in
summer and skiing or snowshoeing in winter. And yet, Abruzzo, long
considered poor and backward, has never been particularly beloved by
Italians, and consequently, not much considered or well known.

The ancient hill town came as a shock, a revelation really, to Daniele
Kihlgren, the renegade scion of an Italian concrete fortune, when he
came upon it while on a motorcycle ride in the late 1990s. Although
semi-abandoned, its medieval character and architecture were completely
intact --- unruined, ironically, by concrete, the material Kihlgren is
the first to acknowledge has disgraced so much of Italy. How, he
wondered, might places of such distinct and exquisite beauty be
revitalized without wrecking their historic identity? And how might
their local traditions, from food to domestic handicrafts, be
organically preserved? ``We can't compete with China in mass production,
and we can't compete in technology,'' Kihlgren says, ``but we have what
no one else in the world has,'' which is the beauty of these villages
and the cultural history of its people, the stuff he calls Italy's minor
patrimony. ``And if we don't ruin it, it can be what saves southern
Italy.''

Image

In 2013, Sutera's mayor took in survivors of the devastating Lampedusa
shipwreck. In the years since, more refugees have settled here, giving
new life to a once-depleted town.Credit...Domingo Milella

Kihlgren began buying up many of the empty buildings, perhaps a quarter
of the town, and proceeded to create one of the most novel forms of
hospitality anywhere, an \emph{albergo diffuso} (scattered hotel) called
Sextantio. The ``rooms'' of the hotel are in ancient buildings all over
town, and are served by one central reception area, allowing guests to
be immersed in the community. Just as important, it is invisible,
respecting the historic shape of the town and its architectural
integrity. Kihlgren also recognized Santo Stefano as part of a delicate
ecosystem, in which the town, the people, its cultural production and
the countryside are inextricable from one another; as one falters or
languishes, so too do the others. He realized that if he wanted
traditional Abruzzo loom-woven wool blankets for his 60 beds, that meant
he needed artisans to weave them, which required yarn to be spun, which
implied sheep, who need shepherds, and farmland, and farmers. So it
proceeds from the building materials used, to the construction
techniques employed, to the ingredients and recipes served in the
hotel's restaurant down to the ceramic dishes they're served on. This
cycle, which connects land to people, is what keeps Santo Stefano from
becoming a chic version of Colonial Williamsburg.

It is also what has helped revive it. Thanks largely to Sextantio, there
are now new jobs, thousands of tourists annually, nearly two dozen new
bed-and-breakfasts and several restaurants, galleries and shops. But
Kihlgren has sunk his fortune into this project --- as well as into
other villages that he has bought, either outright or partially, with
the intention of resurrecting them using the same model. While his is
primarily a cultural project, he is keenly aware that unless it can make
money and be replicated, it is just a folly in his grand quest to
resuscitate southern Italy. And indeed, there have been many
difficulties and setbacks --- from his own high standards to the
devastating earthquake that rocked the region in 2009. This past July
marked the first month since the project's inception in which he made a
profit.

Image

High atop volcanic sediment, the ancient town of Civita di Bagnoregio
makes for a dazzling sight. Its few remaining residents have marketed it
as a preserved relic of a bygone Italy.Credit...Domingo Milella

Still, Kihlgren remains optimistic --- and how can anyone who loves
Italy not root for his success? ``If I, someone with no business skills
at all, can make this work,'' he says, ``then there is hope --- hope
that the values contained in these small historical places can be the
engine to revitalize them.''

Advertisement

\protect\hyperlink{after-bottom}{Continue reading the main story}

\hypertarget{site-index}{%
\subsection{Site Index}\label{site-index}}

\hypertarget{site-information-navigation}{%
\subsection{Site Information
Navigation}\label{site-information-navigation}}

\begin{itemize}
\tightlist
\item
  \href{https://help.nytimes3xbfgragh.onion/hc/en-us/articles/115014792127-Copyright-notice}{©~2020~The
  New York Times Company}
\end{itemize}

\begin{itemize}
\tightlist
\item
  \href{https://www.nytco.com/}{NYTCo}
\item
  \href{https://help.nytimes3xbfgragh.onion/hc/en-us/articles/115015385887-Contact-Us}{Contact
  Us}
\item
  \href{https://www.nytco.com/careers/}{Work with us}
\item
  \href{https://nytmediakit.com/}{Advertise}
\item
  \href{http://www.tbrandstudio.com/}{T Brand Studio}
\item
  \href{https://www.nytimes3xbfgragh.onion/privacy/cookie-policy\#how-do-i-manage-trackers}{Your
  Ad Choices}
\item
  \href{https://www.nytimes3xbfgragh.onion/privacy}{Privacy}
\item
  \href{https://help.nytimes3xbfgragh.onion/hc/en-us/articles/115014893428-Terms-of-service}{Terms
  of Service}
\item
  \href{https://help.nytimes3xbfgragh.onion/hc/en-us/articles/115014893968-Terms-of-sale}{Terms
  of Sale}
\item
  \href{https://spiderbites.nytimes3xbfgragh.onion}{Site Map}
\item
  \href{https://help.nytimes3xbfgragh.onion/hc/en-us}{Help}
\item
  \href{https://www.nytimes3xbfgragh.onion/subscription?campaignId=37WXW}{Subscriptions}
\end{itemize}
