Sections

SEARCH

\protect\hyperlink{site-content}{Skip to
content}\protect\hyperlink{site-index}{Skip to site index}

\href{https://myaccount.nytimes3xbfgragh.onion/auth/login?response_type=cookie\&client_id=vi}{}

\href{https://www.nytimes3xbfgragh.onion/section/todayspaper}{Today's
Paper}

\href{/section/opinion}{Opinion}\textbar{}How Did Women Fare in China's
Communist Revolution?

\url{https://nyti.ms/2jWp66e}

\begin{itemize}
\item
\item
\item
\item
\item
\item
\end{itemize}

Advertisement

\protect\hyperlink{after-top}{Continue reading the main story}

Supported by

\protect\hyperlink{after-sponsor}{Continue reading the main story}

\href{/section/opinion}{Opinion}

\href{/column/red-century}{Red Century}

\hypertarget{how-did-women-fare-in-chinas-communist-revolution}{%
\section{How Did Women Fare in China's Communist
Revolution?}\label{how-did-women-fare-in-chinas-communist-revolution}}

By Helen Gao

\begin{itemize}
\item
  Sept. 25, 2017
\item
  \begin{itemize}
  \item
  \item
  \item
  \item
  \item
  \item
  \end{itemize}
\end{itemize}

\href{https://cn.nytimes3xbfgragh.onion/opinion/20170926/women-china-communist-revolution/}{阅读简体中文版}\href{https://cn.nytimes3xbfgragh.onion/opinion/20170926/women-china-communist-revolution/zh-hant/}{閱讀繁體中文版}

\includegraphics{https://static01.graylady3jvrrxbe.onion/images/2017/09/25/opinion/25gao/25gao-articleInline.jpg?quality=75\&auto=webp\&disable=upscale}

BEIJING --- My grandmother likes to tell stories from her career as a
journalist in the early decades of the People's Republic of China. She
recalls scrawling down Chairman Mao's latest pronouncements as they came
through loudspeakers and talking with joyous peasants from the newly
collectivized countryside. In what was her career highlight, she turned
an anonymous candy salesman into a national labor hero with glowing
praises for his service to the people.

She had grown up in the central province of Hunan, where her father was
a landlord. She talks about her mother as a glum housewife who resented
her husband for taking a concubine after she had failed to give birth to
a boy.

``The Communists did many terrible things,'' my grandmother always says
at the end of her reminiscences. ``But they made women's lives much
better.''

That often-repeated dictum sums up the popular perception of Mao
Zedong's legacy regarding women in China. As every Chinese schoolchild
learns in history class, the Communists rescued peasant daughters from
urban brothels and ushered cloistered wives into factories, liberating
them from the oppression of Confucian patriarchy and imperialist threat.

But the narrative of an across-the-board elevation of women's status
under Mao contains crucial caveats.

While the Communist revolution brought women more job opportunities, it
also made their interests subordinate to collective goals. Stopping at
the household doorstep, Mao's words and policies did little to alleviate
women's domestic burdens like housework and child care. And by
inundating society with rhetoric blithely celebrating its achievements,
the revolution deprived women of the private language with which they
might understand and articulate their personal experiences.

When historians researched the collectivization of the Chinese
countryside in the 1950s, an event believed to have empowered rural
women by offering them employment, they discovered a complicated
picture. While women indeed contributed enormously to collective
farming, they rarely rose to positions of responsibility; they remained
outsiders in communes organized around their husbands' family and
village relationships. Studies also showed that women routinely
performed physically demanding jobs but earned less than men, since the
lighter, most valued tasks involving large animals or machinery were
usually reserved for men.

The urban workplace was hardly more inspiring. Women were shunted to
collective neighborhood workshops with meager pay and dismal working
conditions, while men were more commonly employed in comfortable
big-industry and state-enterprise jobs. Party cadres' explanations for
this reflected deeply entrenched gender prejudices: Women have a weaker
constitution and gentler temper, rendering them unfit for the strenuous
tasks of operating heavy equipment or manning factory floors.

The party at times paid lip service to the equal sharing of domestic
labor, but in practice it condoned women's continuing subordination in
the home. In posters and speeches, female socialist icons were portrayed
as ``iron women'' who labored heroically in front of steel furnaces
while maintaining a harmonious family. But it was a cherry-picking
approach that focused exclusively on bringing women into the work force
and neglected their experiences in other realms.

Visitors to rural areas saw peasant wives toiling around the clock:
cooking, mending clothes and feeding livestock after finishing a day of
work in the fields. Their plight shocked the urban youth who were sent
down to the countryside during the Cultural Revolution, such that Naihua
Zhang, a sociology professor at Florida Atlantic University who spent
time in the countryside as a young woman in that era, equated rural
marriage with a total erasure of women's identity.

Researchers also observed that after marriage factory women often
experienced slower career advancement than men as they became saddled
with domestic responsibilities that left them with little time to learn
new skills and take on extra work, both prerequisites for promotion.
State services that promised to ease their burden, like public child
care centers, were in reality few and far between. Unlike their
counterparts in developed countries, Chinese women didn't have
labor-saving household appliances, since Mao's economic policies
prioritized heavy industry over the production of consumer products like
washing machines and dishwashers.

Some Western scholars have said these realities amounted to a
``revolution postponed.'' Yet the conclusions of researchers were
contradicted by none other than Chinese women themselves.

During her field study in China in 1970s, Margery Wolf, who was an
anthropology professor at University of Iowa, was surprised by how
effusive Chinese women were about the miracle of female emancipation in
the very presence of their continued oppression.

``It was easy to take gender equality --- an ideal that was widely
promoted --- as the reality and regard problems as reminiscent of old
systems and ideology that would erode with time,'' said Professor Zhang,
the sociologist.

The state rolled out propaganda campaigns aimed at not only enlisting
women in the work force but also shaping their self-perception. Posters,
textbooks and newspapers propagated images and narratives that, devoid
of any particularities of personal experiences, depicted women as men's
equal in outlook, value and achievement. For women in the workplace to
adhere to this narrowly defined acceptable female image meant to see,
understand and speak about their life not as it was, but as what it
ought to be according to the party ideal.

It is a measure of the campaign's success that women who publicly
described their experiences in the Mao era did so exclusively in
official rhetoric. Elisabeth Croll, an anthropologist specializing in
Chinese women, observed that all published accounts of Chinese women's
lives during the early decades of the People's Republic followed the
standard narrative of their rise from mistreated wives and daughters to
independent, socialist workers; it had become the story of practically
every woman.

Forty years after Mao's death, this aspect of his legacy is still
understood through his famous pronouncement on gender equality, ``Women
hold up half the sky.'' It is a slogan my grandmother utters in the same
breath as the chairman's other sins and deeds.

She does not mention the arduous work of managing a household and
raising three children amid tumultuous revolutionary campaigns. Nor does
she complain about how she couldn't join the party because of her
husband's unpopular political affiliations. She gives only a chuckle
when she recalls the exhortations she once received from party superiors
to marry just as her career was taking off.

For all its flaws, the Communist revolution taught Chinese women to
dream big. When it came to advice for my mother, my grandmother
applauded her daughter's decision to go to graduate school and urged her
to find a husband who would be supportive of her career. She still seems
to think that the new market economy --- with its meritocracy and
freedom of choice --- will finally allow women to be masters of their
minds and actions.

After all, she has always said to my mother, ``you have more
opportunities.''

Advertisement

\protect\hyperlink{after-bottom}{Continue reading the main story}

\hypertarget{site-index}{%
\subsection{Site Index}\label{site-index}}

\hypertarget{site-information-navigation}{%
\subsection{Site Information
Navigation}\label{site-information-navigation}}

\begin{itemize}
\tightlist
\item
  \href{https://help.nytimes3xbfgragh.onion/hc/en-us/articles/115014792127-Copyright-notice}{©~2020~The
  New York Times Company}
\end{itemize}

\begin{itemize}
\tightlist
\item
  \href{https://www.nytco.com/}{NYTCo}
\item
  \href{https://help.nytimes3xbfgragh.onion/hc/en-us/articles/115015385887-Contact-Us}{Contact
  Us}
\item
  \href{https://www.nytco.com/careers/}{Work with us}
\item
  \href{https://nytmediakit.com/}{Advertise}
\item
  \href{http://www.tbrandstudio.com/}{T Brand Studio}
\item
  \href{https://www.nytimes3xbfgragh.onion/privacy/cookie-policy\#how-do-i-manage-trackers}{Your
  Ad Choices}
\item
  \href{https://www.nytimes3xbfgragh.onion/privacy}{Privacy}
\item
  \href{https://help.nytimes3xbfgragh.onion/hc/en-us/articles/115014893428-Terms-of-service}{Terms
  of Service}
\item
  \href{https://help.nytimes3xbfgragh.onion/hc/en-us/articles/115014893968-Terms-of-sale}{Terms
  of Sale}
\item
  \href{https://spiderbites.nytimes3xbfgragh.onion}{Site Map}
\item
  \href{https://help.nytimes3xbfgragh.onion/hc/en-us}{Help}
\item
  \href{https://www.nytimes3xbfgragh.onion/subscription?campaignId=37WXW}{Subscriptions}
\end{itemize}
