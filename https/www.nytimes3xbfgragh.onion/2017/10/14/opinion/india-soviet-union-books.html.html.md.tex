Sections

SEARCH

\protect\hyperlink{site-content}{Skip to
content}\protect\hyperlink{site-index}{Skip to site index}

\href{https://myaccount.nytimes3xbfgragh.onion/auth/login?response_type=cookie\&client_id=vi}{}

\href{https://www.nytimes3xbfgragh.onion/section/todayspaper}{Today's
Paper}

\href{/section/opinion}{Opinion}\textbar{}Baba Yaga on the Ganges

\url{https://nyti.ms/2kPIpyj}

\begin{itemize}
\item
\item
\item
\item
\item
\end{itemize}

Advertisement

\protect\hyperlink{after-top}{Continue reading the main story}

Supported by

\protect\hyperlink{after-sponsor}{Continue reading the main story}

\href{/section/opinion}{Opinion}

\href{/column/red-century}{Red Century}

\hypertarget{baba-yaga-on-the-ganges}{%
\section{Baba Yaga on the Ganges}\label{baba-yaga-on-the-ganges}}

By Palash Krishna Mehrotra

\begin{itemize}
\item
  Oct. 14, 2017
\item
  \begin{itemize}
  \item
  \item
  \item
  \item
  \item
  \end{itemize}
\end{itemize}

\includegraphics{https://static01.graylady3jvrrxbe.onion/images/2017/10/14/opinion/14mehrota-nyt/14mehrotaWeb-articleLarge.jpg?quality=75\&auto=webp\&disable=upscale}

Allahabad, INDIA --- I grew up in Allahabad, a city in northern India on
the banks of the Ganges, in the 1980s. Not much happened there, apart
from the two seasonal festivals --- Holi, which marks the end of winter,
and Diwali, the greeter at winter's doorstep.

The rest of the world was far away. There was little world history
taught to us in school; the curriculum was insularly devoted to India's
freedom struggle. We were faintly aware of the Cold War, that India was
friendly with the Soviet Union, that an Indian astronaut had gone into
space with a couple of Russians.

Two cultural events made Allahabad come alive --- the release of a new
Bollywood movie, and the arrival of a Soviet Book Exhibition.

It was the age of benign propaganda and the Soviets were winning. Though
India was proclaimed to be non-aligned throughout the Cold War, it
leaned heavily toward the Soviet Union. India followed the socialist
economic model and the Soviets invested significantly in India --- from
defense to infrastructure.

My father grew up in Bhilai, a town in central India where the Russians
were helping set up a steel plant. One afternoon some Russian officials
arrived, chairs were hurriedly pulled together and photographs taken. No
words were exchanged but everyone smiled for the camera. A photograph
appeared in the next issue of Sovietland magazine, underlining the
bonhomie between Russian and Indian families.

Indians were paranoid about Americans meddling in our internal political
affairs. In Allahabad, the rumor was that the Americans sucked out the
vitamins from the rice before sending it as aid to India.

The Soviets we loved. When printing technology in India was restricted
to black and white, Russia bamboozled us with colorful comics and
magazines. Russian ballets and circuses performed in Indian cities and
were broadcast frequently on Indian state television.

Soviet cultural centers were active social hubs in Indian cities,
offering lessons in music, dance and chess. Russians loved Bollywood.
Raj Kapoor, who made and starred in musicals about simple-hearted
characters smiling in the face of adversity, would be greeted with
rousing receptions in the Soviet Union.

The Soviet Book Exhibition would cram a vast ground in the center of
town with stalls selling books and magazines, for all ages, in English
and various Indian languages. Sovietland magazine was published in 13
Indian languages. It carried pictures of Soviet life, of collective
farming, and Indo-Soviet collaborations on projects like the Bhilai
steel plant.

Soviet books were inexpensive and beautifully produced on glossy art
paper. Their low cost and great production value matched the preferred
phrase of socialist India: ``cheap and best.''

The comparatively meek American cultural effort was largely restricted
to sending complimentary copies of Span magazine to university
professors. My father, a poet who taught English literature at Allahabad
University, received a copy of Span. It had pictures of cheeseburgers
studded with sesame seeds and of geometry-box corn fields.

A typical issue would have Ronald Reagan on the cover, a short story by
John Gardner, photographs by Ansel Adams, a debate on America's nuclear
strategy, and a long piece on America's search for natural gas.

``The Russians are way ahead of the Americans in their hide-and-seek
game... They strike at the grass roots,'' an essay in The Christian
Science Monitor noted in 1982. ``They are in touch with the masses, not
with the elite.''

After the book exhibitions, a traveling van stocked with Russian
classics would snake its way through the lanes of Allahabad and other
small towns across India.

Russian children's literature is believed to have blurred the lines
between propaganda and art, perhaps like no other children's literature.
I don't remember any disreputable titles that glorified electrification
and agriculture. I read numerous folk tales and children's books printed
by Moscow publishers --- Raduga and Progress.

I had begun to read The Hardy Boys series, Nancy Drew and Enid Blyton by
the time the Soviet books arrived. Even as a kid, one could make out
that the American and English books were written to a formula. After a
while one wasn't reading these for the story but for the salivating
descriptions of food --- the exotic buttered scones and ginger beer in
Blyton's family-farm dramas; the ice cream sodas in Archie comics.

Food and propaganda were curiously missing from the Russian books.
Alexander Raskin's ``When Daddy Was a Little Boy'' was first published
in 1966. It is a quietly funny, comically grotesque memoir of growing
up: a boy gets bitten by a dog on both cheeks.

I found some echoes of my Indian childhood in the Russian tales. Reading
obsessively, even under the blanket; disliking and not eating bread;
going for piano or sitar lessons, even when you had no talent; the
emphasis on good handwriting; making silly verses and reciting them
unbidden to every visitor. Arrogance was to be avoided at all costs.

The couples in the folk tales were either childless, or aging with
beautiful unmarried daughters. Some stories centered on grandmothers
living with granddaughters. There was a touch of the macabre: two
sisters kill their third sister out of malice before the redemptive
power of magic revives her. I came across words I didn't understand:
\emph{sarafans}, \emph{kavach}. It didn't matter.

In those throbbing, feral stories, I encountered Baba Yaga, the
quintessential Russian witch. At times, she lived in the middle of the
forest in a rotating cabin; at others, in ``a great house of white stone
with forty-one pillars at the gate.''

Not all folk tales ended with a neat moral. Many were about the
constantly varying kaleidoscope of human nature. The miser borrows a
kopeck from a peasant to give to a beggar. The dogged peasant keeps
after the miser; he wants his kopeck back. Both keep outwitting the
other. By the end of the tale, both men are richer due to the peasant's
cleverness. The naïve and yet not-so-naïve peasant still worries about
the kopeck he lent the miser.

The Russians came to India and distributed their stories virtually for
free. If this was propaganda, no one has bad memories of imbibing it.

My favorite part of the evening, after having shopped for twenty
hard-bound books for ten rupees at the Soviet Book Exhibition, was to go
to Hotstuff, the only American-style joint in town. I would leaf through
the illustrations by Igor Yershov, listen to Wham, drink a strawberry
shake and eat a Big Boy burger.

Advertisement

\protect\hyperlink{after-bottom}{Continue reading the main story}

\hypertarget{site-index}{%
\subsection{Site Index}\label{site-index}}

\hypertarget{site-information-navigation}{%
\subsection{Site Information
Navigation}\label{site-information-navigation}}

\begin{itemize}
\tightlist
\item
  \href{https://help.nytimes3xbfgragh.onion/hc/en-us/articles/115014792127-Copyright-notice}{©~2020~The
  New York Times Company}
\end{itemize}

\begin{itemize}
\tightlist
\item
  \href{https://www.nytco.com/}{NYTCo}
\item
  \href{https://help.nytimes3xbfgragh.onion/hc/en-us/articles/115015385887-Contact-Us}{Contact
  Us}
\item
  \href{https://www.nytco.com/careers/}{Work with us}
\item
  \href{https://nytmediakit.com/}{Advertise}
\item
  \href{http://www.tbrandstudio.com/}{T Brand Studio}
\item
  \href{https://www.nytimes3xbfgragh.onion/privacy/cookie-policy\#how-do-i-manage-trackers}{Your
  Ad Choices}
\item
  \href{https://www.nytimes3xbfgragh.onion/privacy}{Privacy}
\item
  \href{https://help.nytimes3xbfgragh.onion/hc/en-us/articles/115014893428-Terms-of-service}{Terms
  of Service}
\item
  \href{https://help.nytimes3xbfgragh.onion/hc/en-us/articles/115014893968-Terms-of-sale}{Terms
  of Sale}
\item
  \href{https://spiderbites.nytimes3xbfgragh.onion}{Site Map}
\item
  \href{https://help.nytimes3xbfgragh.onion/hc/en-us}{Help}
\item
  \href{https://www.nytimes3xbfgragh.onion/subscription?campaignId=37WXW}{Subscriptions}
\end{itemize}
