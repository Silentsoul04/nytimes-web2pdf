Sections

SEARCH

\protect\hyperlink{site-content}{Skip to
content}\protect\hyperlink{site-index}{Skip to site index}

\href{https://myaccount.nytimes3xbfgragh.onion/auth/login?response_type=cookie\&client_id=vi}{}

\href{https://www.nytimes3xbfgragh.onion/section/todayspaper}{Today's
Paper}

\href{/section/opinion}{Opinion}\textbar{}When New York City Was the
Capital of American Communism

\url{https://nyti.ms/2gUUaz4}

\begin{itemize}
\item
\item
\item
\item
\item
\end{itemize}

Advertisement

\protect\hyperlink{after-top}{Continue reading the main story}

Supported by

\protect\hyperlink{after-sponsor}{Continue reading the main story}

\href{/section/opinion}{Opinion}

\href{/column/red-century}{Red Century}

\hypertarget{when-new-york-city-was-the-capital-of-american-communism}{%
\section{When New York City Was the Capital of American
Communism}\label{when-new-york-city-was-the-capital-of-american-communism}}

By Maurice Isserman

\begin{itemize}
\item
  Oct. 20, 2017
\item
  \begin{itemize}
  \item
  \item
  \item
  \item
  \item
  \end{itemize}
\end{itemize}

\href{http://cn.nytimes3xbfgragh.onion/opinion/20171023/new-york-american-communism/}{阅读简体中文版}\href{http://cn.nytimes3xbfgragh.onion/opinion/20171023/new-york-american-communism/zh-hant/}{閱讀繁體中文版}

\includegraphics{https://static01.graylady3jvrrxbe.onion/images/2017/10/21/opinion/21issermanSub/21issermanSub-articleLarge.jpg?quality=75\&auto=webp\&disable=upscale}

The Brooklyn-born playwright and critic Lionel Abel, who cut his
political teeth in left-wing circles in Greenwich Village in the 1930s,
remarked in his memoirs that during the Depression years, New York City
``went to Russia and spent most of the decade there.'' Leaving aside Mr.
Abel's taste for the mordant, he had a point.

For a few decades --- from the 1930s until Communism's demise as an
effective political force in the 1950s --- New York City was the one
place where American communists came close to enjoying the status of a
mass movement. Party members could live in a milieu where co-workers,
neighbors and the family dentist were fellow Communists; they bought
life insurance policies (excellent value for money) from
party-controlled fraternal organizations; they could even spend their
evenings out in night clubs run by Communist sympathizers (like the
ironically named Café Society on Sheridan Square in Greenwich Village, a
showcase for up-and-coming black performers like Billie Holliday).

What became the Communist Party U.S.A. (its name varied in the early
years) was founded in Chicago in 1919 and, following a period of
underground organization, opened its national headquarters in that city
in 1921. But the bulk of the movement's members were in New York, and in
1927 Communist headquarters were shifted to a party-owned building in
Manhattan, at 35 East 12th Street, two blocks south of Union Square.
(The building still stands, although under new ownership, and in what
has evolved into a considerably less proletarian neighborhood than in
the old days.)

New York would remain the capital city of American Communism from then
on. Leading communists, including such figures as William Z. Foster and
Earl Browder, had their offices on the top floor of the 12th Street
building; accordingly, within the movement, it became the custom to
refer to party leadership as the ``ninth floor.'' (And, for some reason,
even in non- and anti-Communist left-wing circles, ``the party'' was
always understood to refer to the Communists, rather than any rival
organizations.)

Immigrants, many of them of Eastern European Jewish background, provided
the main social base for the party in New York City in the 1920s: As
late as 1931, four-fifths of the Communists living in the city were
foreign-born.

Of course, immigrant radicalism was nothing new in New York. The
socialist leader Morris Hillquit, born in Riga, Latvia, won more than a
fifth of the votes cast in the 1917 mayoral election. Socialists
initially hailed the news of the Bolshevik Revolution, but many of them
--- except for those who left to become Communists --- came in time to
understand and oppose the Soviet regime's abandonment of the left's
traditional democratic and egalitarian ideals.

Neither of the two main rival left-wing parties, Socialists or
Communists, enjoyed much success in the 1920s. But with the onset of the
Great Depression, Socialists were poised once again to become the
dominant party on the left. In the 1932 presidential election, the
Socialist candidate, Norman Thomas, won almost nine times the votes that
the Communist candidate, Mr. Foster, received. (Neither of them had a
fraction of the support of the actual winner, Franklin D. Roosevelt.)

But the balance of power on the left was about to change, and nowhere
would that change make itself felt more dramatically than in New York.
With the Depression spiraling out of control in the early 1930s, the
Soviet Union began to be viewed in a new and more sympathetic light by
millions of people around the world, including many in the United
States. The ``workers' state'' with its planned economy, viewed at a
hazy distance and with a lot of wishful thinking, seemed to offer a
desirable alternative to the cruel irrationality of a failed capitalist
system, with its mass unemployment and widespread social misery.

Marxism-Leninism, Communists proclaimed, was a science, whose practical
application by centralized and disciplined revolutionary parties in
Europe, the Americas and elsewhere, held the key to unifying the workers
of the world. Within a few years of the Nazi seizure of power in Germany
in 1933, Soviet leaders shifted their international strategy from
promoting world revolution to seeking anti-fascist alliances with
Western democratic powers. In the era of the ``popular front,'' as
American Communists stressed the need for anti-fascist unity, they began
to win grudging respect in labor and liberal circles, as useful allies
in the struggle for social change.

Party members did their best to appear less threatening and less
foreign-inspired even as they still praised all things Soviet,
proclaiming that Communism was simply ``20th-century Americanism.''
Communists also reached out to groups they had previously scorned, like
the New Deal Democrats, and to politicians they had previously
denounced, like Mayor Fiorello LaGuardia.

For a while, it worked. In cities around the country, from Detroit to
Seattle to Los Angeles, Communists began to play a visible and effective
role in politics, both local and national. But nowhere were they as
successful as in New York.

By 1938, the party counted 38,000 members in New York State, about half
its national membership, and most of those lived in New York City.
Communists were increasingly native-born (although many were the
children of immigrants). Party-organized mass meetings in the old
Madison Square Garden were packed with as many as 20,000 participants;
the annual May Day parades drew tens of thousands, too.

Some neighborhoods in New York could be likened to the ``red belt''
surrounding Paris: Communist-organized cooperative parties on Allerton
Avenue in the Bronx were a strong base of party support, as were parts
of East Harlem, Brooklyn and the Lower East Side. In Harlem, the party's
strong commitment to fighting racism (still quite rare, even on the
liberal left) helped it to attract the support of African-Americans
across the social spectrum, including some leading artists like actor
and singer Paul Robeson.

Communists were central to spreading the gospel of unionism from the
garment trades to a host of previously unorganized industries and
workplaces, as organizers and officials in the Transport Workers Union,
the National Maritime Union, the Teachers Union and the American
Newspaper Guild, among others. Ben Gold, the president of the Fur
Workers Union, was one of the few labor leaders in the United States who
openly avowed his Communist beliefs. A Communist candidate for the
presidency of the city's board of aldermen received nearly 100,000 votes
in 1938; and during World War II, two open Communists, Peter V.
Cacchione of Brooklyn and Benjamin Davis of Harlem, held seats on the
City Council. At City College, Brooklyn College and Columbia University,
there were hundreds of members of the Young Communist League, and
thousands of students who joined Communist front groups like the
American Youth Congress.

In the end, the decade or so that New York City ``spent'' in Russia came
to nothing. The Communist Party's ties to the Soviet Union, which forced
it into the role of apologist for the worst crimes of the Stalin regime,
from the Moscow Trials to the Nazi-Soviet Pact, limited its appeal even
at the height of its success. With the onset of the Cold War, and of a
second Red Scare more pervasive and longer-lasting than the original,
Communists found themselves persecuted and isolated.

In 1956, with a hard core of 20,000 or so surviving members, the party
was dealt a fatal blow when the Soviet leader, Nikita Khrushchev,
delivered a ``secret speech'' to the 20th party congress in Moscow,
denouncing his predecessor, Stalin, as a bloody mass murderer. The
speech leaked. So did the disillusioned membership of the Communist
Party U.S.A., reduced to a few thousand members by 1958, and never
recovering much beyond that in decades to come. It did, however, survive
the collapse of its political inspiration, the Soviet experiment.

On the 100th anniversary of the Russian Revolution, the national
headquarters of the Communist Party U.S.A. remains in New York City, on
one floor of a party-owned building at 235 West 23rd Street. Party
members are apparently divided over whether to keep the building, which
generates considerable rent revenue, or make a killing on the real
estate market by selling it.

A very capitalist question, in the end, to preoccupy the remaining
comrades.

Advertisement

\protect\hyperlink{after-bottom}{Continue reading the main story}

\hypertarget{site-index}{%
\subsection{Site Index}\label{site-index}}

\hypertarget{site-information-navigation}{%
\subsection{Site Information
Navigation}\label{site-information-navigation}}

\begin{itemize}
\tightlist
\item
  \href{https://help.nytimes3xbfgragh.onion/hc/en-us/articles/115014792127-Copyright-notice}{©~2020~The
  New York Times Company}
\end{itemize}

\begin{itemize}
\tightlist
\item
  \href{https://www.nytco.com/}{NYTCo}
\item
  \href{https://help.nytimes3xbfgragh.onion/hc/en-us/articles/115015385887-Contact-Us}{Contact
  Us}
\item
  \href{https://www.nytco.com/careers/}{Work with us}
\item
  \href{https://nytmediakit.com/}{Advertise}
\item
  \href{http://www.tbrandstudio.com/}{T Brand Studio}
\item
  \href{https://www.nytimes3xbfgragh.onion/privacy/cookie-policy\#how-do-i-manage-trackers}{Your
  Ad Choices}
\item
  \href{https://www.nytimes3xbfgragh.onion/privacy}{Privacy}
\item
  \href{https://help.nytimes3xbfgragh.onion/hc/en-us/articles/115014893428-Terms-of-service}{Terms
  of Service}
\item
  \href{https://help.nytimes3xbfgragh.onion/hc/en-us/articles/115014893968-Terms-of-sale}{Terms
  of Sale}
\item
  \href{https://spiderbites.nytimes3xbfgragh.onion}{Site Map}
\item
  \href{https://help.nytimes3xbfgragh.onion/hc/en-us}{Help}
\item
  \href{https://www.nytimes3xbfgragh.onion/subscription?campaignId=37WXW}{Subscriptions}
\end{itemize}
