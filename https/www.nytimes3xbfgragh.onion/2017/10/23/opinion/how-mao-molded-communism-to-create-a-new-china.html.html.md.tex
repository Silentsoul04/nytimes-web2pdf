Sections

SEARCH

\protect\hyperlink{site-content}{Skip to
content}\protect\hyperlink{site-index}{Skip to site index}

\href{https://myaccount.nytimes3xbfgragh.onion/auth/login?response_type=cookie\&client_id=vi}{}

\href{https://www.nytimes3xbfgragh.onion/section/todayspaper}{Today's
Paper}

\href{/section/opinion}{Opinion}\textbar{}How Mao Molded Communism to
Create a New China

\url{https://nyti.ms/2zuCa5C}

\begin{itemize}
\item
\item
\item
\item
\item
\item
\end{itemize}

Advertisement

\protect\hyperlink{after-top}{Continue reading the main story}

Supported by

\protect\hyperlink{after-sponsor}{Continue reading the main story}

\href{/section/opinion}{Opinion}

\href{/column/red-century}{Red Century}

\hypertarget{how-mao-molded-communism-to-create-a-new-china}{%
\section{How Mao Molded Communism to Create a New
China}\label{how-mao-molded-communism-to-create-a-new-china}}

By Roderick MacFarquhar

\begin{itemize}
\item
  Oct. 23, 2017
\item
  \begin{itemize}
  \item
  \item
  \item
  \item
  \item
  \item
  \end{itemize}
\end{itemize}

\href{https://cn.nytimes3xbfgragh.onion/opinion/20171024/how-mao-molded-communism-to-create-a-new-china/}{阅读简体中文版}\href{https://cn.nytimes3xbfgragh.onion/opinion/20171024/how-mao-molded-communism-to-create-a-new-china/zh-hant/}{閱讀繁體中文版}

\includegraphics{https://static01.graylady3jvrrxbe.onion/images/2017/10/23/opinion/23macfarquharWeb/23macfarquharWeb-articleLarge.jpg?quality=75\&auto=webp\&disable=upscale}

Toward the end of his life, dying of Lou Gehrig's disease, Mao Zedong
claimed two achievements: leading the Communist revolution to victory
and starting the Cultural Revolution. By pinpointing these episodes, he
had underlined the lifelong contradiction in his attitudes toward
revolution and state power.

Mao molded Communism to fit his two personas. To use Chinese parlance,
he was both a tiger and a monkey king.

For the Chinese, the tiger is the king of the jungle. Translated into
human terms, a tiger is a high official. The agency running President Xi
Jinping's anticorruption campaign today likes to boast when it has
brought down another ``tiger.'' By leading the Chinese Communist Party
to victory in 1949, Mao became the top tiger.

The monkey king is an imaginary being with the strength of a superman,
an ability to fly and a predilection for using his immense cudgel for
destructive purposes. He is a sage. Ordinary humans and even spirits
cannot defeat him.

In his earliest writings, Mao seemed to portray himself more as a
Nietzschean superman, or a tiger:

\begin{quote}
The great actions of the hero are his own, are the expression of his
motive power, lofty and cleansing, relying on no precedent. His force is
like that of a powerful wind arising from a deep gorge, like the
irresistible sexual desire for one's lover, a force that will not stop,
that cannot be stopped. All obstacles dissolve before him.
\end{quote}

In his early 20s, roaming the countryside of Hunan Province with a
friend, Mao convinced his companion that he saw himself in the tradition
of the peasant founders of Chinese dynasties, in particular Liu Bang,
founder of the first great Chinese Empire, the Han. By the time he was
42, shortly after the bedraggled survivors of the epic Long March had
reached safety in northwest China, Mao went as far as to look down upon
all the great emperors of the past. In a famous poem, ``Snow,'' Mao
wrote:

\emph{This land so rich in beauty}

\emph{Has made countless heroes bow in homage}

\emph{But alas! Qin Shihuang and Han Wudi}

\emph{Were lacking in literary grace,}

\emph{And Tang zong and Song zu}

\emph{Had little poetry in their souls;}

\emph{And Genghis Khan}

\emph{Proud son of heaven for a day,}

\emph{Knew only shooting eagles, bow outstretched.}

\emph{All are past and gone!}

\emph{For truly great men}

\emph{Look to this age alone.}

But however self-confident Mao's early dreams of glory, his supreme
leadership was far from preordained. On the eve of his coming out as a
Marxist at age 27, he was an unsophisticated provincial nationalist. He
gloomily dismissed the chances of the new Chinese republic surviving,
wondered about Hunan becoming an American state and advocated that all
of the Chinese provinces should become separate countries.

It was only in November 1920 that he admitted defeat: The Hunanese did
not have the vision to appreciate his ideas. He wrote to his activist
friends in the provincial capital to say that he would henceforth be a
socialist. He was just in time.

Communist cells had been organized in Shanghai, Beijing and other
cities, and in mid-1921, the first congress of the Chinese Communist
Party was held. Mao, who had quickly organized a Communist group in
Hunan, achieved the cachet of being one of only 12 delegates to attend.
He was thus an early tiger.

The Soviet agents who funded and masterminded the organization of the
early C.C.P. reported to the Comintern, the agency for spreading Soviet
ideas and influence abroad. With memories of defeat in the
Russo-Japanese War of 1904-5, and competing with Japan for influence in
Manchuria, the Soviets needed a strong China as an ally against Japanese
expansionism.

The fledgling C.C.P. was too weak. The Soviets decided to bolster the
well known revolutionary who had helped bring down the Manchu dynasty
but had then been pushed aside by warlords: Sun Yat-sen.

They provided him with funds, reorganized his Nationalist party, known
as the K.M.T., and helped him to train an army. C.C.P. members were
instructed by Comintern agents to back the K.M.T., and even become
members, but to retain their allegiance to the Communists. The plan was
for the C.C.P. to take over the K.M.T. from within after the Chinese
warlords were conquered by this united front.

Most of the C.C.P. leadership opposed the Comintern policy; they thought
collaboration with the ``bourgeois'' K.M.T. would demoralize their
members. But the piper called the tune, and they joined the K.M.T., few
more readily than Mao.

Two events set Mao off on a new, career-shaping course. The first was
Chiang Kai-shek's attack on the Communist Party. By 1927, after Sun
Yat-sen's death, Chiang Kai-shek had taken over the leadership of the
K.M.T., and he had conquered much of the southern half of the country.
Aware of the Soviets' long-term aim for a C.C.P. takeover of the K.M.T.,
he short-circuited the plan in May 1927 by ordering the slaughter of
Communists, mainly in Shanghai. Communist leaders scattered in flight.

The other event was Mao's experience with peasant power. After the death
of their parents, Mao and his two brothers owned a valuable property
back in their home village that had been built up by their father. The
family had made the transition from poor to rich peasants. And though he
had grown up surrounded by the miseries of rural life, as a fledgling
Communist, Mao had been focusing on the urban proletariat until Moscow,
realizing that China was different, ordered more attention be paid to
the peasantry.

Mao became active in peasant affairs, and his transformative experience
was witnessing and chronicling an uprising in his native Hunan. In a
famous passage, he rejected allegations that the peasants had gone too
far:

\begin{quote}
A revolution is not the same as inviting people to dinner or writing an
essay or painting a picture or embroidering a flower; it cannot be
anything so refined, so calm and gentle.
\end{quote}

Witnessing the bloodshed in the Hunanese countryside, Mao was
discovering his other persona. As the scholar-diplomat Richard Solomon
first pointed out, Mao reveled in ``luan,'' or upheaval. When young, Mao
had written that for change to come about, China must be ``destroyed and
reformed.'' He now realized that only the peasantry could bring that
about. Mao would be the monkey king to lead that destruction.

The primary source for the monkey king is the classic Chinese novel
``The Journey to the West.'' Ostensibly about the famous Chinese monk
Xuan Zang, who made the arduous crossing of the Himalayas to seek out
original Buddhist scriptures in India, ``Journey*''* is a fantastical
tale in which Sun Wukong, the monkey king, plays a major role as the
monk's escort. In the early 1960s, when the C.C.P.'s quarrel with the
Soviet Communist Party was underway, Mao praised the monkey king:

\emph{A thunderstorm burst over the earth,}

\emph{So a devil rose from a heap of white bones.}

\emph{The deluded monk was not beyond the light,}

\emph{But the malignant demon must wreak havoc.}

\emph{The Golden Monkey wrathfully swung his massive cudgel}

\emph{And the jade-like firmament was cleared of dust.}

\emph{Today, a miasmal mist once more rising,}

\emph{We hail Sun Wu-kung, the wonder-worker.}

Mao then rose from guerrilla chief in the late 1920s to a party leader
in the mid-1930s on the Long March, the flight of the C.C.P. from the
southeast to the northwest to escape Chiang Kai-shek's attacks. This was
an epic event in Communist annals because it took a year, covered some
6,000 miles and reduced the 85,000 who had set out to a mere 8,000 by
the time they reached the northwest. He absorbed two lessons: All power
grew out of the barrel of a gun; and most of the time peasants were very
difficult to organize because they had fields to tend and families to
feed.

From the mid-1930s to the mid-1950s, Mao played his tiger role. He led
an increasingly strong and efficient party and army that survived the
anti-Japanese war and then defeated Chiang and the K.M.T. in the civil
war of the late 1940s. From 1949 until 1956, Mao presided over the
installation of the Communist dictatorship in China, rooting out all
opposition, real or imagined, and transforming the ownership of the
means of production from private hands to socialist control.

It was then that he dabbled in the monkey business for the first time.
From the point of view of a dutiful C.C.P. cadre, ``monkey business''
could be defined as any measure that would disrupt the party's standard
operating procedures. Cadres did not appreciate it when Mao in 1956
exhorted intellectuals to ``Let a hundred flowers bloom'' and a year
later again encouraged intellectuals to criticize the conduct of the
party. As members of the ruling elite, the cadres resented being
criticized, and Mao, having promised that the criticisms would only be
like a light rain, quickly wound up the campaigns when they turned into
a typhoon, and purged the critics.

Mao truly became the monkey king by starting the Cultural Revolution in
1966 to dispel the ``miasmal mist'' of Soviet-style ``revisionism'' from
the C.C.P. Now, it was the youth of China, not the peasants, who were to
be his agents of destruction, as major party and government departments
were trashed and their officials humiliated and purged.

For Mao, the Cultural Revolution ended in 1969 with the appointment of a
new, and hopefully more revolutionary, leadership. But though he had
dealt the age-old bureaucratic system of China a terrible blow, he knew
that it could rise again from the ashes. He always emphasized that China
would have to experience regular Cultural Revolutions.

But when Mao's chosen successor, Hua Guofeng, repeated that dictum, he
sealed his fate. Deng Xiaoping and his fellow survivors did not want any
more monkey kings plunging the party and the country into chaos again.

And yet today, China's current ruler, Xi Jinping, with his relentless
anticorruption drive to make officials more honest, has unleashed
another Cultural Revolution against the bureaucracy, albeit one that is
controlled from the center not from the streets.

This is the action of a monkey king. There is no chaos today, but there
surely is widespread fear and resentment as his mighty cudgel claims
more victims.

The 19th Communist Party Congress currently underway will confirm that
Mr. Xi is top tiger, the most powerful ruler since Mao. But Mr. Xi will
have to ensure that his alternate persona as monkey king does not loom
too large. As the revolutionary founder, Mao could never have been
toppled. But as a revolutionary successor, Mr. Xi could be.

Advertisement

\protect\hyperlink{after-bottom}{Continue reading the main story}

\hypertarget{site-index}{%
\subsection{Site Index}\label{site-index}}

\hypertarget{site-information-navigation}{%
\subsection{Site Information
Navigation}\label{site-information-navigation}}

\begin{itemize}
\tightlist
\item
  \href{https://help.nytimes3xbfgragh.onion/hc/en-us/articles/115014792127-Copyright-notice}{©~2020~The
  New York Times Company}
\end{itemize}

\begin{itemize}
\tightlist
\item
  \href{https://www.nytco.com/}{NYTCo}
\item
  \href{https://help.nytimes3xbfgragh.onion/hc/en-us/articles/115015385887-Contact-Us}{Contact
  Us}
\item
  \href{https://www.nytco.com/careers/}{Work with us}
\item
  \href{https://nytmediakit.com/}{Advertise}
\item
  \href{http://www.tbrandstudio.com/}{T Brand Studio}
\item
  \href{https://www.nytimes3xbfgragh.onion/privacy/cookie-policy\#how-do-i-manage-trackers}{Your
  Ad Choices}
\item
  \href{https://www.nytimes3xbfgragh.onion/privacy}{Privacy}
\item
  \href{https://help.nytimes3xbfgragh.onion/hc/en-us/articles/115014893428-Terms-of-service}{Terms
  of Service}
\item
  \href{https://help.nytimes3xbfgragh.onion/hc/en-us/articles/115014893968-Terms-of-sale}{Terms
  of Sale}
\item
  \href{https://spiderbites.nytimes3xbfgragh.onion}{Site Map}
\item
  \href{https://help.nytimes3xbfgragh.onion/hc/en-us}{Help}
\item
  \href{https://www.nytimes3xbfgragh.onion/subscription?campaignId=37WXW}{Subscriptions}
\end{itemize}
