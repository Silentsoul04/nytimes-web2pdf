Sections

SEARCH

\protect\hyperlink{site-content}{Skip to
content}\protect\hyperlink{site-index}{Skip to site index}

\href{https://myaccount.nytimes3xbfgragh.onion/auth/login?response_type=cookie\&client_id=vi}{}

\href{https://www.nytimes3xbfgragh.onion/section/todayspaper}{Today's
Paper}

\href{/section/opinion}{Opinion}\textbar{}A Tale of Two Revolutions

\url{https://nyti.ms/2zBdusj}

\begin{itemize}
\item
\item
\item
\item
\item
\end{itemize}

Advertisement

\protect\hyperlink{after-top}{Continue reading the main story}

Supported by

\protect\hyperlink{after-sponsor}{Continue reading the main story}

\href{/section/opinion}{Opinion}

\href{/column/red-century}{Red Century}

\hypertarget{a-tale-of-two-revolutions}{%
\section{A Tale of Two Revolutions}\label{a-tale-of-two-revolutions}}

\href{http://topics.nytimes3xbfgragh.onion/top/reference/timestopics/people/k/enrique_krauze/index.html}{\includegraphics{https://static01.graylady3jvrrxbe.onion/images/2013/10/04/opinion/Krauze-contributor/Krauze-contributor-thumbLarge-v4.png}}

By
\href{http://topics.nytimes3xbfgragh.onion/top/reference/timestopics/people/k/enrique_krauze/index.html}{Enrique
Krauze}

\begin{itemize}
\item
  Oct. 25, 2017
\item
  \begin{itemize}
  \item
  \item
  \item
  \item
  \item
  \end{itemize}
\end{itemize}

\href{https://www.nytimes3xbfgragh.onion/es/2017/10/17/la-revolucion-domesticada}{Leer
en español}

\includegraphics{https://static01.graylady3jvrrxbe.onion/images/2017/10/25/opinion/25krauzeWeb/25krauzeWeb-articleLarge.jpg?quality=75\&auto=webp\&disable=upscale}

MEXICO CITY --- The Russian Revolution of 1917, and the regime that
governed in its name for most of the 20th century, exerted a powerful
political and ideological influence on Latin America. The revolution put
its stamp on political parties, labor unions, artists, intellectuals and
students, who saw the Soviet Union as an alternative to capitalism, a
bulwark against United States imperialism and an example to emulate.
Although the revelations of the crimes of Stalinist totalitarianism
diminished the luster of the Russian Revolution in the 1950s, the
surprising victory of the Communists in Cuba revived the revolutionary
spirit in Latin America, inspiring guerrilla movements that alarmed
military regimes allied to the United States.

Mexico was a case apart. Few countries had as much success as Mexico in
neutralizing the effects of the Russian Revolution. The reason was
simple. Mexico had undergone its own revolution from 1910 to 1917 and
was advancing on its own revolutionary road. The nationalist and
socialist ideology of the Mexican Revolution triumphed in every
confrontation with the homegrown Marxist-Leninism of the Mexican
Communist Party --- Lenin and Trotsky could never compete with Pancho
Villa and Emiliano Zapata. And the tension between the two revolutions
shaped the Mexican political process for decades to come.

The Mexican muralist movement of the 1920s was as original and dynamic
as Russian Modernism, with which Mexican artists carried on a creative
dialogue. Mexico, in 1924, was the first country in the Western
Hemisphere to establish diplomatic relations with the Soviet Union, a
move frowned upon by the United States, whose government confused
Mexican nationalism with Communism. Facing this apparent rapprochement
between the two revolutions, President Calvin Coolidge seriously
considered military action against ``Soviet Mexico.''

That changed when the banker Dwight Morrow became ambassador to Mexico
in 1927. He helped restructure the Mexican debt, became an adviser to
Mexican political figures and had the brilliant instinct to become a
friend and patron to leftist artists. The most famous among them were of
course Diego Rivera and Frida Kahlo, and many young writers --- among
them the combative poet Octavio Paz --- were Marxists who believed that
the Soviet Union was ``the land of the future.''

Declared illegal in 1929 and repressed, the Mexican Communist Party
gained some influence during the term of President Lázaro Cárdenas
(1934-40), but ``domestication'' once again had an effect. It was
impossible to compete from the left with a government so clearly
revolutionary as that of President Cárdenas, which distributed over 42
million acres of land, nationalized American --- and European --- owned
oil enterprises in 1938 and had the support of the country's main labor
union, the Confederation of Mexican Workers.

Perhaps the most significant proof of Mexican autonomy with respect to
the Russian Revolution came in 1936, when Mr. Cárdenas gave asylum to
Leon Trotsky, at the request of Mr. Rivera. When the Communist Party of
Mexico refused to participate in the assassination of Mr. Trotsky,
carried out in 1940 by a Stalinist agent, it sealed its fate. During the
Cold War, the Institutional Revolutionary Party, or P.R.I., could
present itself openly as a nationalist and progressive alternative to
Communism, while the Communist Party remained quite marginal, supported
mainly by the railroad unions and some prominent cultural figures.

Frida Kahlo, when she died in 1954, received the first official homage
ever accorded to an artist, in Mexico City's Palace of Fine Arts. Her
coffin was covered with a banner of the hammer and sickle. This was
emblematic of a resurgence of Communism in Mexico, not stemming from
parties and unions but from artistic, academic and literary circles,
where Marxism had begun to gain renewed vigor thanks to Jean-Paul
Sartre's writings. Nevertheless, in the arena of politics, the P.R.I.
continued its undisputed reign. At least until the student movement of
1968 (when its dominion over the new middle classes began to crack), the
official party was an all-powerful alliance that ran from right to left,
with only the extremes on both sides excluded.

Not even the Cuban Revolution changed the situation. Showing impressive
political skill, the P.R.I. regime did not condemn Fidel Castro and
abstained from the vote by the Organization of American States to expel
Cuba, yet it also became the buffer between the United States and the
Communist tendencies of the rest of Latin America. In exchange, the
United States accepted a certain degree of nationalist rhetoric by
Mexico.

The compromise with Havana was clear. The expedition led by Mr. Castro
in 1956 had set sail from Mexico, and Mexico would defend Cuba from the
United States through diplomacy. Cuba, for its part, would not sponsor
guerrilla uprisings in Mexico. Although this tacit agreement was no
longer totally functional by the 1970s, guerrilla movements in Mexico
had much less reach and impact than those in Central America. When such
movements were brutally repressed, Havana and Moscow reacted with
indifference. And when Mexican guerrillas seized planes and flew them to
Cuba, Mr. Castro either immediately returned the hijackers or imprisoned
them.

Although the Castro government made its arrangements with the P.R.I.,
the prestige of the Cuban Revolution, among recent generations,
overshadowed the Mexican, which many younger people saw as antiquated
and false. In the 1970s and 1980s, Marxism in all its varieties became a
common language in Mexican public universities, and this cultural and
academic hegemony of Marxism is a key factor in understanding the
paradoxical strengthening of the Mexican left at the very moment of the
fall of the Berlin Wall.

Young people in the universities were the base for the popularity of
Cuauhtémoc Cárdenas, the son of the president, when in 1987 he abandoned
the P.R.I., which had governed nationally since the 1930s. Partisans of
the left welcomed Mr. Cárdenas and his dissident comrades.

By then, the Communist Party had merged into the Mexican Socialist
Party. That party put Mr. Cárdenas up as the candidate of the left in
the presidential elections of 1988. Orchestrated electoral fraud
prevented his victory.

But instead of calling for an armed revolt, Mr. Cárdenas united the
entire left into one party, the Party of the Democratic Revolution.
Although it was defeated in the presidential elections of 1994 and 2000,
the party entered the new century as a consolidated force with a strong
presence in state governments and legislatures and with power in Mexico
City. The city's leader, Andrés Manuel López Obrador, greatly admired
Che Guevara and Mr. Castro but was no Marxist and came, like Cuauhtémoc
Cárdenas, originally from the P.R.I.

Mr. López Obrador would become the populist caudillo of the Mexican
left. In 2006 he ran for president, came within fractions of a percent
of victory and accused the government of electoral fraud. Significantly,
his closest advisers included no Communist politicians of the old guard
but many academics influenced by Marxism as well as various former
politicians of the old P.R.I. of the '70s, '80s and '90s. Yet one more
time, the Mexican Revolution had absorbed and transformed (and
sidelined) the Russian Revolution.

Advertisement

\protect\hyperlink{after-bottom}{Continue reading the main story}

\hypertarget{site-index}{%
\subsection{Site Index}\label{site-index}}

\hypertarget{site-information-navigation}{%
\subsection{Site Information
Navigation}\label{site-information-navigation}}

\begin{itemize}
\tightlist
\item
  \href{https://help.nytimes3xbfgragh.onion/hc/en-us/articles/115014792127-Copyright-notice}{©~2020~The
  New York Times Company}
\end{itemize}

\begin{itemize}
\tightlist
\item
  \href{https://www.nytco.com/}{NYTCo}
\item
  \href{https://help.nytimes3xbfgragh.onion/hc/en-us/articles/115015385887-Contact-Us}{Contact
  Us}
\item
  \href{https://www.nytco.com/careers/}{Work with us}
\item
  \href{https://nytmediakit.com/}{Advertise}
\item
  \href{http://www.tbrandstudio.com/}{T Brand Studio}
\item
  \href{https://www.nytimes3xbfgragh.onion/privacy/cookie-policy\#how-do-i-manage-trackers}{Your
  Ad Choices}
\item
  \href{https://www.nytimes3xbfgragh.onion/privacy}{Privacy}
\item
  \href{https://help.nytimes3xbfgragh.onion/hc/en-us/articles/115014893428-Terms-of-service}{Terms
  of Service}
\item
  \href{https://help.nytimes3xbfgragh.onion/hc/en-us/articles/115014893968-Terms-of-sale}{Terms
  of Sale}
\item
  \href{https://spiderbites.nytimes3xbfgragh.onion}{Site Map}
\item
  \href{https://help.nytimes3xbfgragh.onion/hc/en-us}{Help}
\item
  \href{https://www.nytimes3xbfgragh.onion/subscription?campaignId=37WXW}{Subscriptions}
\end{itemize}
