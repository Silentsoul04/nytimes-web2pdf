Sections

SEARCH

\protect\hyperlink{site-content}{Skip to
content}\protect\hyperlink{site-index}{Skip to site index}

\href{https://www.nytimes3xbfgragh.onion/section/food}{Food}

\href{https://myaccount.nytimes3xbfgragh.onion/auth/login?response_type=cookie\&client_id=vi}{}

\href{https://www.nytimes3xbfgragh.onion/section/todayspaper}{Today's
Paper}

\href{/section/food}{Food}\textbar{}José Andrés Fed Puerto Rico, and May
Change How Aid Is Given

\url{https://nyti.ms/2iMhZgD}

\begin{itemize}
\item
\item
\item
\item
\item
\item
\end{itemize}

Advertisement

\protect\hyperlink{after-top}{Continue reading the main story}

Supported by

\protect\hyperlink{after-sponsor}{Continue reading the main story}

\hypertarget{josuxe9-andruxe9s-fed-puerto-rico-and-may-change-how-aid-is-given}{%
\section{José Andrés Fed Puerto Rico, and May Change How Aid Is
Given}\label{josuxe9-andruxe9s-fed-puerto-rico-and-may-change-how-aid-is-given}}

\includegraphics{https://static01.graylady3jvrrxbe.onion/images/2017/11/01/dining/01JOSE1/01JOSE1-articleLarge-v2.jpg?quality=75\&auto=webp\&disable=upscale}

By \href{http://www.nytimes3xbfgragh.onion/by/kim-severson}{Kim
Severson}

\begin{itemize}
\item
  Oct. 30, 2017
\item
  \begin{itemize}
  \item
  \item
  \item
  \item
  \item
  \item
  \end{itemize}
\end{itemize}

\href{https://www.nytimes3xbfgragh.onion/es/2017/11/03/los-chefs-que-cambian-como-se-entrega-la-ayuda-posdesastre/}{Leer
en español}

SAN JUAN, P.R. --- José Andrés was walking along a dark street in a
stained T-shirt and a ball cap, trying to decompress after another day
of feeding an island that has been largely without electricity since
Hurricane Maria hit a month ago.

He'd gone barely half a block before two women ran over to snag a
selfie. A man shouted out his name from a bar running on a generator and
offered to buy him a rum sour.

The reaction is more subdued in rural mountain communities like Naguabo,
where Mr. Andrés and his crew have been delivering supplies so cooks at
a small Pentecostal church can make 5,000 servings of arroz con pollo
and carne guisada every day. There, people touch his sleeve and whisper,
``Gracias.'' They surround him and pray.

``He's much more than a hero,'' said Jesus R. Rivera, who was inside a
cigar store watching Mr. Andrés pick out one of his daily smokes. ``The
situation is that still some people don't even have food. He is all that
is keeping them from starving.''

It's overwhelming, even for Mr. Andrés, the larger-than-life,
Michelin-starred Spanish chef with a prolific, unfiltered social media
presence,
\href{http://www.npr.org/sections/thetwo-way/2017/04/07/523004201/trump-organization-settles-lawsuit-with-chef-jos-andr-s}{who
got into a legal fight with the Trump Organization} after Donald Trump
made disparaging comments about Mexicans.

\includegraphics{https://static01.graylady3jvrrxbe.onion/images/2017/11/01/dining/01JOSE2/01JOSE2-articleLarge.jpg?quality=75\&auto=webp\&disable=upscale}

``Every day I have this personal anxiety inside,'' Mr. Andrés said
during a Jeep ride through the countryside in late October. ``We only
came here to try to help a few thousand because nobody had a plan to
feed Puerto Rico, and we opened the biggest restaurant in the world in a
week. That's how crazy this is.''

Since he hit the ground five days after the hurricane devastated this
island of 3.4 million on Sept. 20, he has built a network of kitchens,
supply chains and delivery services that as of Monday had served more
than 2.2 million warm meals and sandwiches. No other single agency ---
not the Red Cross, the Salvation Army nor any government entity --- has
fed more people freshly cooked food since the hurricane, or done it in
such a nurturing way.

Mr. Andrés's effort, by all accounts the largest emergency feeding
program ever set up by a group of chefs, has started winding down. But
it illustrates in dramatic fashion the rise of chefs as valuable players
in a realm traditionally left to more-established aid organizations.

With an ability to network quickly, organize kitchens in difficult
circumstances and marshal raw ingredients and equipment, chef-led groups
are creating a model for a more agile, local response to catastrophes.

``It's part of a larger trend we're starting to see with corporations
and individuals who are applying their unique skill sets to solve
problems after a disaster,'' said Bob Ottenhoff, the president and chief
executive of the \href{http://disasterphilanthropy.org/}{Center for
Disaster Philanthropy}, which helps donors make strategic contributions
related to domestic and international emergencies.

Image

At its peak, as many as 500 volunteers were making 30,000 sandwiches a
day under the aegis of World Central Kitchen.Credit...Eric Rojas for The
New York Times

In addition to sending money or showing up to hand out blankets or boxes
of food, companies like UPS and IBM are designing ways to quickly supply
logistical and technical aid.

``Chefs are part of that trend now, too,'' Mr. Ottenhoff said. ``They're
starting to say, `Look, people are in need of not just food but good
food, and we know how to serve large quantities of good food very
quickly.'''

Kimberly Grant, the chief executive of Mr. Andrés's
\href{http://www.thinkfoodgroup.com/}{Think Food Group}, which runs 27
restaurants, put it like this: ``Who else can take raw ingredients that
are seemingly unassociated and make them into delicious food and do it
under extreme pressure?''

Restaurateurs have long offered food when trouble hit their communities.

Kitchens near the World Trade Center in New York served thousands of
meals each day to emergency workers after 9/11. In response to the 2004
earthquake off the coast of
\href{https://en.wikipedia.org/wiki/Sumatra}{Sumatra}, Indonesia, the
celebrity chef Cat Cora started
\href{http://chefsforhumanity.org/}{Chefs for Humanity}. Competition
barbecue teams that headed to Joplin, Mo., after the 2011 tornadoes
organized themselves into
\href{https://operationbbqrelief.org/}{Operation BBQ Relief}, a
nonprofit group that has since responded to
\href{http://www.cnn.com/2017/09/28/us/cnn-hero-stan-hays-operation-bbq-relief/index.html}{more
than 40 disasters}.

Two weeks ago,
\href{http://www.pressdemocrat.com/news/7559619-181/sonoma-family-meal-brings-chefs?artslide=0}{a
food writer in Northern California} tapped the region's best chefs to
provide a steady stream of meals for people who had lost homes to
wildfires. The restaurateur and TV personality Guy Fieri, who had to
evacuate his Santa Rosa residence,
\href{https://www.today.com/food/guy-fieri-feeds-california-wildfire-evacuees-first-responders-t117562}{organized
a team of volunteers} and began serving mashed potatoes and pork loin to
firefighters and others in a parking lot.

Image

Mr. Andrés (in the light green cap) and volunteers from a church consult
a map to decide where to deliver food cooked by congregation members to
the areas around Naguabo.Credit...Eric Rojas for The New York Times

Mr. Andrés helped out after Hurricane Sandy, but his first big lesson in
emergency food relief came in August, when
\href{https://www.washingtonpost.com/news/food/wp/2017/08/30/jose-andres-is-on-the-ground-in-houston-ready-to-cook-for-displaced-residents/?utm_term=.0c209646d1a6}{he
rallied local chefs} in Houston to help feed survivors of Hurricane
Harvey.

Other Houston chefs and caterers started a website called ``I Have Food
I Need Food'' and used social media to
\href{http://www.houstonchronicle.com/local/gray-matters/article/How-a-Midtown-shelter-became-a-kitchen-command-12252653.php}{create
a system}to organize donations, cook food and get it delivered. They
codified their effort in a manual and sent it to chefs in Miami who were
staring down Hurricane Irma, which landed 16 days later.

Mr. Andrés went to Houston in part to study how to expand the scope of
\href{https://www.worldcentralkitchen.org/}{World Central Kitchen}, a
nonprofit association of chefs he established in 2011 after helping
Haiti earthquake victims a year earlier. The idea was to learn how he
and Brian MacNair, World Central Kitchen's executive director, could add
emergency food relief to an agenda that already included building school
kitchens, organizing culinary training and offering other forms of
support in several countries.

But nothing prepared Mr. Andrés for what he faced in Puerto Rico. After
taking one of the first commercial flights to the island after the
storm, he realized that things were worse than anyone knew.

He found his friend Jose Enrique, the chef who has been leading Puerto
Rico's
\href{https://www.nytimes3xbfgragh.onion/2015/02/18/dining/reclaiming-puerto-ricos-food-paradise.html}{farm-to-table
resurgence}. Mr. Enrique had no electricity to run his
\href{http://joseenriquepr.com/}{Restaurant Jose Enrique}, in the
Santurce district of San Juan. Rain poured through the roof. But he had
food in the freezer. Other chefs did, too. Someone had a generator.

Image

Mr. Andrés called food trucks like this one, owned by Xoimar Manning and
Michael Sauri, the ``Delta strike force'' of his feeding operation in
Puerto Rico. Here, he helps serve what will be the only hot meal of the
day for people in the community of Tocones.Credit...Eric Rojas for The
New York Times

``We decided we would just start cooking,'' Mr. Enrique said.

The next morning, Mr. Andrés went to a food distributor and loaded up
his car. ``I was already smart enough to know I would need aluminum
pans, so I bought every aluminum pan I could,'' he said.

They began cooking big pots of the classic island stew called sancocho
on the street in front of Mr. Enrique's small restaurant. Word spread
and the lines grew. They sent food to people waiting in 10-hour lines at
gas stations. They heard that workers at the city's biggest medical
clinic were going hungry, so they added it to what was now a makeshift
delivery schedule. ``Every day it would just double,'' Mr. Enrique said.

Mr. Andrés didn't realize that his was the biggest hot-food game on the
island until a week or so after they started. Someone from the Salvation
Army pulled up and asked for 120 meals.

``In my life I never expected the Salvation Army to be asking me for
food,'' he said. ``If one of the biggest NGOs comes to us for food, who
is actually going to be feeding Puerto Rico? We are. We are it.''

More cooks arrived to help. Partnerships were forged with other aid
groups and large food companies. Sandwiches and fruit were added to
their repertory of rice dishes.

Image

A boy holds a plate of food that volunteers from the Jesucristo Monte
Moriah Pentecostal Church cooked and delivered to places like a public
housing project in Naguabo.Credit...Eric Rojas for The New York Times

The team moved its base of operation to the island's largest arena. To
pay for it all, at least in the beginning, they used Mr. Andrés's credit
cards, or cash from the pockets of the Orvis fly-fishing vest he wore
like a battle jacket.

Mr. Andrés left the island only a few times, the first after 11 days on
the ground. He had lost 25 pounds and became dehydrated.

His team deployed food trucks, like a strike force, to isolated
neighborhoods and towns that needed help. Agents of
\href{https://www.ice.gov/hsi}{Homeland Security Investigations}, a
division of United States Immigration and Customs Enforcement, were
serving as emergency workers, and staying in the same hotel as Mr.
Andrés's crew. The chef persuaded them to load food into their vehicles
every morning as they headed out to patrol.

With limited ability to communicate, the crew organized everything with
satellite phones, WhatsApp and a big paper map of all the feeding
stations on the island, which Mr. Andrés carried like a general at war.

He negotiated with a chain of vocational schools around the island to
let culinary students cook there. During visits to his kitchens, 18 in
all, he admonished volunteers to add more mayonnaise to sandwiches, keep
the temperature up on the pans of rice or serve bigger portions.

Image

Mr. Andrés, holding a rum sour, poses with admirers at La Placita de
Santurce, a club in San Juan.Credit...Erika P. Rodriguez for The New
York Times

The \href{http://www.compass-usa.com/}{Compass Group}, a giant American
food-service operation that Mr. Andrés recently partnered with, sent
someone who understood what it takes to feed several thousand people at
a time.

Mr. Andrés recruited his own chefs, too. David Thomas, accustomed to
making \$540 suckling pigs as the executive chef at Mr. Andrés's
\href{https://slslasvegas.com/restaurants-bars/bazaar-meat-by-jose-andres/}{Bazaar
Meat} restaurant in Las Vegas, suddenly found himself trying to figure
out how to make meals out of donations that might include 5,000 pounds
of lunch meat one day and 17 pallets of yogurt the next.

The operation grew so big that at one point you couldn't find any sliced
cheese in all of Puerto Rico. The team had bought it all up for
sandwiches.

Eventually, the effort would cost World Central Kitchen about \$400,000
a day, paid for by donations from foundations, celebrities and a flood
of smaller donors, as well as two Federal Emergency Management Agency
contracts --- one early on to cover the cost of 140,000 meals, and
another for \$10 million to cover two weeks' worth of meals while Mr.
Andrés's team scaled down the operation.

Mr. Andrés, who often rolls right over regulations and ignores the word
``no,'' clashed more than once with FEMA and other large organizations
that have a more-seasoned and methodical approach. In meetings and
telephone calls, FEMA officials reminded him that he and his people
lacked the experience needed to organize a mass emergency feeding
operation, he said.

Image

Pastor Eliomar Santana, who persuaded Mr. Andrés to let his congregation
open a kitchen as part of Mr. Andrés's food-distribution program, with
the chef during his first visit to the Jesucristo Monte Moria
Pentecostal Church. Church members made a video to honor the chef's work
on the island.Credit...Eric Rojas for The New York Times

``We are not perfect, but that doesn't mean the government is perfect,''
Mr. Andrés said. ``I am doing it without red tape and 100 meetings.''

FEMA officials contacted for this article were quick to point out that
many other groups and agencies besides World Central Kitchen were
feeding Puerto Rico; a spokesman would not publicly discuss Mr. Andrés
or his operation.

Late last week, the system that was serving more than 130,000 meals a
day became much smaller. A core crew will most likely keep things going
until Thanksgiving, with one main kitchen and a handful in some of the
neediest regions.

Mr. Andrés flew home to Washington, D.C., on Thursday. ``This has been
like my little Vietnam, but now I need to go back to normal life,'' he
said.

He never intended to stay as long as he did, he said. Or to feed an
island.

``At the end, I couldn't forgive myself if I didn't try to do what I
thought was right,'' he said. ``We need to think less sometimes and
dream less and just make it happen.''

\href{https://www.facebookcorewwwi.onion/nytfood/}{\emph{Follow NYT Food
on Facebook}}\emph{,}
\href{https://instagram.com/nytfood}{\emph{Instagram}}\emph{,}
\href{https://twitter.com/nytfood}{\emph{Twitter}} \emph{and}
\href{https://www.pinterest.com/nytfood/}{\emph{Pinterest}}\emph{.}
\href{https://www.nytimes3xbfgragh.onion/newsletters/cooking}{\emph{Get
regular updates from NYT Cooking, with recipe suggestions, cooking tips
and shopping advice}}\emph{.}

Advertisement

\protect\hyperlink{after-bottom}{Continue reading the main story}

\hypertarget{site-index}{%
\subsection{Site Index}\label{site-index}}

\hypertarget{site-information-navigation}{%
\subsection{Site Information
Navigation}\label{site-information-navigation}}

\begin{itemize}
\tightlist
\item
  \href{https://help.nytimes3xbfgragh.onion/hc/en-us/articles/115014792127-Copyright-notice}{©~2020~The
  New York Times Company}
\end{itemize}

\begin{itemize}
\tightlist
\item
  \href{https://www.nytco.com/}{NYTCo}
\item
  \href{https://help.nytimes3xbfgragh.onion/hc/en-us/articles/115015385887-Contact-Us}{Contact
  Us}
\item
  \href{https://www.nytco.com/careers/}{Work with us}
\item
  \href{https://nytmediakit.com/}{Advertise}
\item
  \href{http://www.tbrandstudio.com/}{T Brand Studio}
\item
  \href{https://www.nytimes3xbfgragh.onion/privacy/cookie-policy\#how-do-i-manage-trackers}{Your
  Ad Choices}
\item
  \href{https://www.nytimes3xbfgragh.onion/privacy}{Privacy}
\item
  \href{https://help.nytimes3xbfgragh.onion/hc/en-us/articles/115014893428-Terms-of-service}{Terms
  of Service}
\item
  \href{https://help.nytimes3xbfgragh.onion/hc/en-us/articles/115014893968-Terms-of-sale}{Terms
  of Sale}
\item
  \href{https://spiderbites.nytimes3xbfgragh.onion}{Site Map}
\item
  \href{https://help.nytimes3xbfgragh.onion/hc/en-us}{Help}
\item
  \href{https://www.nytimes3xbfgragh.onion/subscription?campaignId=37WXW}{Subscriptions}
\end{itemize}
