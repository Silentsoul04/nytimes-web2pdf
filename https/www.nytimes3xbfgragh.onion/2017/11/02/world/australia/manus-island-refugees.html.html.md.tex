Sections

SEARCH

\protect\hyperlink{site-content}{Skip to
content}\protect\hyperlink{site-index}{Skip to site index}

\href{https://www.nytimes3xbfgragh.onion/section/world/australia}{Australia}

\href{https://myaccount.nytimes3xbfgragh.onion/auth/login?response_type=cookie\&client_id=vi}{}

\href{https://www.nytimes3xbfgragh.onion/section/todayspaper}{Today's
Paper}

\href{/section/world/australia}{Australia}\textbar{}What Is Happening on
Manus Island? The Detainee Crisis Explained

\href{https://nyti.ms/2iVFJyJ}{https://nyti.ms/2iVFJyJ}

\begin{itemize}
\item
\item
\item
\item
\item
\end{itemize}

Advertisement

\protect\hyperlink{after-top}{Continue reading the main story}

Supported by

\protect\hyperlink{after-sponsor}{Continue reading the main story}

\hypertarget{what-is-happening-on-manus-island-the-detainee-crisis-explained}{%
\section{What Is Happening on Manus Island? The Detainee Crisis
Explained}\label{what-is-happening-on-manus-island-the-detainee-crisis-explained}}

\includegraphics{https://static01.graylady3jvrrxbe.onion/images/2017/11/03/world/03manus-1/03manus-1-articleLarge.jpg?quality=75\&auto=webp\&disable=upscale}

By \href{http://www.nytimes3xbfgragh.onion/by/russell-goldman}{Russell
Goldman} and
\href{http://www.nytimes3xbfgragh.onion/by/damien-cave}{Damien Cave}

\begin{itemize}
\item
  Nov. 2, 2017
\item
  \begin{itemize}
  \item
  \item
  \item
  \item
  \item
  \end{itemize}
\end{itemize}

The United Nations warned on Thursday of an ``unfolding humanitarian
emergency'' in Papua New Guinea, as hundreds of asylum seekers
barricaded themselves inside an abandoned detention center on Manus
Island, choosing to remain inside a camp devoid of food and clean water
rather than relocate to alternative facilities in a city nearby.

The camp officially closed on Tuesday, but the detainees have refused to
relocate to temporary accommodations, claiming fears of violent
reprisals by the island's residents.

\hypertarget{who-are-the-detainees}{%
\subsection{Who are the detainees?}\label{who-are-the-detainees}}

By law, Australia will not resettle any migrants who approach the
country by boat, a policy intended to discourage dangerous ocean
crossings and human smuggling. Since 2013, Australia has paid Papua New
Guinea, its closest neighbor, to house hundreds of migrants caught at
sea while trying to reach the continent.

About 600 migrants, all men, and mostly from the Middle East and
Southeast Asia, remain at the center. Most of them have sought status as
refugees or asylum seekers.

Many of the men have already had their asylum claims vetted and approved
by the United States and are awaiting placement there, according to
American officials. But nearly 200 have been rejected, leaving them in
legal limbo.

\hypertarget{what-happened-this-week}{%
\subsection{What happened this week?}\label{what-happened-this-week}}

Tensions over the migrants have grown since the governments of Australia
and Papua New Guinea agreed in April to close the site by Oct. 31.

On Tuesday, water and electricity to the camp were shut off, and
detainees were supposed to move to temporary housing in Lorengau, a city
close by on the island.

Many of the men refused,
\href{https://www.nytimes3xbfgragh.onion/2017/10/25/world/australia/australia-manus-refugees.html}{citing
previous attacks} by residents in Lorengau. Instead the migrants
barricaded themselves inside, using solar power for their phones and
digging wells for water as police cars circled.

``It's very surprising to see it come to this level,'' said Jonathan
Pryke, Pacific Islands program director for the Lowy Institute, an
Australian think tank. ``It just seems like a complete mess.''

By Thursday, conditions in the camp appeared to have declined.

One detainee, Behrouz Boochani, a Kurdish journalist from Iran, said the
men were experiencing dehydration, hunger, anxiety and the fear of
attack and disease.

``Heat, humidity, hunger and incessant mosquitoes are taking their
toll,'' Mr. Boochani wrote on Facebook. ``This is not a hunger strike.
It is a situation that the Australian government has created, forcing
people into starvation and these harsh conditions by refusing to offer a
safe place for resettlement.''

By Friday, detainees said conditions were leading to illness. One asylum
seeker, Imran Mohammad, said in a text message that three diabetic men
had fallen ill because they were unable to get insulin, and the camp was
strewn with garbage.

\includegraphics{https://static01.graylady3jvrrxbe.onion/images/2017/11/03/world/03manus-2/03manus-2-articleLarge.jpg?quality=75\&auto=webp\&disable=upscale}

\hypertarget{who-is-responsible}{%
\subsection{Who is responsible?}\label{who-is-responsible}}

The governments of Australia and Papua New Guinea each claim the other
is responsible for relocating the men until a permanent solution can be
found.

Australia has pledged 250 million Australian dollars, or \$193 million,
for the men's food and security at the facilities in Lorengau for the
next year.

Julie Bishop, Australia's foreign minister, said on Thursday that it
made ``no sense'' for the detainees to remain at the camp. But Nat Jit
Lam, the regional representative of the United Nations' refugee agency,
said the temporary housing was incomplete and unsafe.

``I will not be bringing any refugee there to stay --- not in that
state,'' Mr. Lam told ABC Radio of Australia.

In a statement issued Thursday, the agency said: ``Australia remains
responsible for the well-being of all those moved to Papua New Guinea
until adequate, long-term solutions outside the country are found.''

\hypertarget{what-will-happen-to-the-detainees}{%
\subsection{What will happen to the
detainees?}\label{what-will-happen-to-the-detainees}}

Australia has consistently said it will not accept the men for
resettlement.

They have all been given the option of permanent residency in Papua New
Guinea, or applying to resettle in Cambodia or Nauru, the location of a
second offshore facility run by the Australian government. None of the
men still on Manus have accepted the offer, according to reports.

American officials said dozens if not hundreds of refugees from Manus
and Nauru would be accepted in the coming weeks and months.
\href{https://www.nytimes3xbfgragh.onion/2017/09/20/world/australia/refugees-turnbull-trump.html}{About
50 men} already moved to the United States in September under a deal
brokered by former President Barack Obama.

\href{https://www.nytimes3xbfgragh.onion/2017/10/20/world/asia/jacinda-ardern-new-zealand.html?_r=0}{Jacinda
Ardern}, the new prime minister of New Zealand, also said this week she
would an honor a predecessor's pledge to accept 150 refugees. The
Australian government has been reluctant to allow New Zealand to accept
the men, fearing it would open a legal backdoor to Australia.

If those options do not work out, there is another possibility: Russell
Crowe, the Australian actor,
\href{https://twitter.com/russellcrowe/status/925660477371269120}{offered
on Twitter} Wednesday to provide housing and jobs for six of the men.
Calling Australia's refugee policy the ``nation's shame,'' he added,
``I'm sure there'd be other Australians who would do the same.''

Advertisement

\protect\hyperlink{after-bottom}{Continue reading the main story}

\hypertarget{site-index}{%
\subsection{Site Index}\label{site-index}}

\hypertarget{site-information-navigation}{%
\subsection{Site Information
Navigation}\label{site-information-navigation}}

\begin{itemize}
\tightlist
\item
  \href{https://help.nytimes3xbfgragh.onion/hc/en-us/articles/115014792127-Copyright-notice}{©~2020~The
  New York Times Company}
\end{itemize}

\begin{itemize}
\tightlist
\item
  \href{https://www.nytco.com/}{NYTCo}
\item
  \href{https://help.nytimes3xbfgragh.onion/hc/en-us/articles/115015385887-Contact-Us}{Contact
  Us}
\item
  \href{https://www.nytco.com/careers/}{Work with us}
\item
  \href{https://nytmediakit.com/}{Advertise}
\item
  \href{http://www.tbrandstudio.com/}{T Brand Studio}
\item
  \href{https://www.nytimes3xbfgragh.onion/privacy/cookie-policy\#how-do-i-manage-trackers}{Your
  Ad Choices}
\item
  \href{https://www.nytimes3xbfgragh.onion/privacy}{Privacy}
\item
  \href{https://help.nytimes3xbfgragh.onion/hc/en-us/articles/115014893428-Terms-of-service}{Terms
  of Service}
\item
  \href{https://help.nytimes3xbfgragh.onion/hc/en-us/articles/115014893968-Terms-of-sale}{Terms
  of Sale}
\item
  \href{https://spiderbites.nytimes3xbfgragh.onion}{Site Map}
\item
  \href{https://help.nytimes3xbfgragh.onion/hc/en-us}{Help}
\item
  \href{https://www.nytimes3xbfgragh.onion/subscription?campaignId=37WXW}{Subscriptions}
\end{itemize}
