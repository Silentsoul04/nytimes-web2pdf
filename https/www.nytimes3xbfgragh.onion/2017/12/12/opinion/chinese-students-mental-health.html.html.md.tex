Sections

SEARCH

\protect\hyperlink{site-content}{Skip to
content}\protect\hyperlink{site-index}{Skip to site index}

\href{https://myaccount.nytimes3xbfgragh.onion/auth/login?response_type=cookie\&client_id=vi}{}

\href{https://www.nytimes3xbfgragh.onion/section/todayspaper}{Today's
Paper}

\href{/section/opinion}{Opinion}\textbar{}Chinese, Studying in America,
and Struggling

\href{https://nyti.ms/2jQH4Vh}{https://nyti.ms/2jQH4Vh}

\begin{itemize}
\item
\item
\item
\item
\item
\end{itemize}

Advertisement

\protect\hyperlink{after-top}{Continue reading the main story}

Supported by

\protect\hyperlink{after-sponsor}{Continue reading the main story}

\href{/section/opinion}{Opinion}

\hypertarget{chinese-studying-in-america-and-struggling}{%
\section{Chinese, Studying in America, and
Struggling}\label{chinese-studying-in-america-and-struggling}}

By Helen Gao

\begin{itemize}
\item
  Dec. 12, 2017
\item
  \begin{itemize}
  \item
  \item
  \item
  \item
  \item
  \end{itemize}
\end{itemize}

\href{http://cn.nytimes3xbfgragh.onion/opinion/20171213/chinese-students-mental-health/}{阅读简体中文版}\href{http://cn.nytimes3xbfgragh.onion/opinion/20171213/chinese-students-mental-health/zh-hant/}{閱讀繁體中文版}

\includegraphics{https://static01.graylady3jvrrxbe.onion/images/2017/12/13/opinion/13gaoCOLOR-inyt/13gaoCOLOR-inyt-articleLarge.jpg?quality=75\&auto=webp\&disable=upscale}

BEIJING --- In the fall of 2015, as I began my last year of graduate
school, I sensed something was wrong. I woke up in the mornings with a
pounding heart. In seminar discussions my sentences came out in
faltering fragments, while my classmates' voices reached me as a
cacophony of piercing sounds.

Until that point, I had enjoyed my time on the Harvard campus, whose
maple-shaded Georgian buildings I had first seen in pictures my father
took 20 years earlier on his first trip from China to America. On the
back of one of the photos, he had inscribed in blue ink, ``You will see
it with your own eyes someday.''

But in those days, instead of taking in its beauty, my eyes were fixed
on the ground as I slogged from one academic building to the next,
counting pavement stones to help tame my racing thoughts.

The anxiety attacks took me by surprise. I had spent eight years
studying and working in the United States away from my home in Beijing.
Over time, the isolation of graduate school, the heavy reading load in a
second language and the strain that distance put on relationships with
people in Beijing all began to add up.

Trying to make sense of what happened, I recalled incidents among fellow
Chinese international students that at the time had seemed like only
minor slumps in coping with the demands of student life: missed classes,
complaints of insomnia, months of sudden absence from group events,
lengthy Facebook posts strung with sullen adjectives.

There were
\href{http://www.chinadaily.com.cn/china/2017twosession/2017-03/08/content_28470916.htm}{544,500
Chinese studying abroad in 2016}, and a more recent report
\href{https://www.nytimes3xbfgragh.onion/2017/05/04/us/chinese-students-western-campuses-china-influence.html?_r=0}{said
329,000 are studying in the United States alone}. For those students,
the opportunity is the culmination of uncounted after-school hours
devoted to American standardized test prep lessons, and it means
liberation from the merciless Chinese education system.

Yet those triumphs come with hidden perils. A
\href{https://www.researchgate.net/publication/234103690_Report_of_a_Mental_Health_Survey_Among_Chinese_International_Students_at_Yale_University}{survey
released in 2013 by Yale researchers} found that 45 percent of Chinese
international students on campus reported symptoms of depression, and 29
percent reported symptoms of anxiety. The rates are startling, compared
with the roughly 13 percent for depression and anxiety among the general
population in American universities. Those findings are corroborated by
reports from other American campuses, as well as from
\href{https://www.timeshighereducation.com/news/chinese-students-still-more-stressed-when-abroad}{some
schools in Australia} and Britain with large Chinese student
populations.

The Chinese students acknowledge the usual challenges of living abroad
--- like the language barrier and cultural differences --- but cite
academic pressure as the most likely cause of stress. Despite all they
have heard about a liberal arts education, they are often surprised by
the rigor needed to succeed. The results-oriented mind-set with which
many Chinese tackle their studies doesn't fit well in a system that
emphasizes the analytical process and critical thinking.

As a result, the determination and perseverance that have made Chinese
students winners at home can deepen their sense of frustration abroad,
when a paper outline does not easily emerge from heaps of painstakingly
compiled notecards, or when a history exam asks questions about
hypothetical scenarios rather than the chronology they have committed to
heart.

The feeling is not eased by a frequently cited difficulty in building
productive relationships with academic advisers. In a study that
interviewed 19 Chinese graduate students at a university in the American
Southwest about their sources of stress, many described having trouble
establishing trust with their advisers. Some feared that the language
barrier might lead advisers to doubt their intelligence. Others
confessed to being kept awake at night thinking about communication
blunders such as a bungled conversation or a misphrased email to an
adviser.

Those challenges may seem common enough; many, indeed, are not
unfamiliar to American students. But for the Chinese students, who grew
up imbibing messages that all but equated their life prospects and
self-worth with academic achievements, the setbacks can be profoundly
unnerving. The bright promise of intellectual freedom often ends up
producing an insecurity so consuming that it leaves them unable to
consider failing.

The price of failure is more than imaginary for a majority of those
students. Chinese international students overwhelmingly pay full
tuition. An annual cost of, say, \$50,000 to \$60,000 is about 10 times
the average urban disposable income in China and often requires
working-class families to empty bank accounts or sell properties.
Although parents have not hesitated to make those sacrifices when it
comes to the future of their treasured only child, to the conscientious
American-college freshman from Shenzhen or Changsha struggling to keep
up with academic requirements, that weight can feel like an avalanche
bearing down.

A Chinese student in Chicago (using a pseudonym) voiced a popular
sentiment when she told The Paper, a popular Chinese online news outlet,
of her constant wondering whether her school performance justified the
money her working-class parents spent on her education. It made her more
anxious than she had been during the ``gaokao,'' the notoriously
cutthroat national college entrance exam.

People like that student are unlikely to find consolation in recent
statistics: according to
\href{http://en.ccg.org.cn/report-on-employment-entrepreneurship-of-chinese-returnees-2017/}{a
September report} from the Center for China and Globalization, a
Beijing-based think tank, and Zhilian Zhaopin, a Chinese recruitment
agency, 80.5 percent of Chinese overseas returnees make less than
\$1,500 a month, with their average pay only marginally higher than that
of graduates of mainland colleges.

These shared difficulties have led Chinese students to turn to one
another in moments of distress. The stigma in Chinese culture associated
with mental illnesses is just beginning to lift,
\href{http://www.bjreview.com.cn/quotes/txt/2014-06/16/content_624279.htm}{as
several Chinese celebrities have opened up} about their personal
battles. But because of the heavy shortage of well-trained therapists in
China, therapy remains a hazy concept even for the most worldly
students. The Yale survey found that despite the alarming rate of mental
illness, 27 percent of Chinese students on campus had never heard of the
university's mental health counseling service, and only 4 percent had
ever used it.

Some of those who have tried the services are often left underwhelmed.
In addition to the long wait and limited session time that are common to
the increasingly crowded university mental health counseling centers,
there are thornier problems. How can Chinese students convey the texture
of their thoughts and moods in a foreign language when the language
barrier is a cause of their stress and inhibition in the first place?
How do they communicate their nostalgia for mouthwatering homemade
Chinese dishes when the sympathetic therapist may not have ventured
beyond Panda Express?

A few institutions, such as Purdue University and Ohio State University,
have set up counseling services tailored to Chinese students. More
schools need to follow suit. Hiring Chinese-speaking mental health
counselors may be the ideal solution, although qualified candidates can
be hard to find.

Informal counselor-led support groups and outreach programs, which have
received positive feedback from Asian-American students, could be
extended to Chinese international students. Universities could hire and
train well-acclimated Chinese students to become counselors for their
communities.

Chinese students are the largest international student group on most
American campuses, and their tuition is a major source of much-needed
revenue. College administrators should work harder to meet their mental
health needs --- at least as hard as the students worked to gain
admission.

Advertisement

\protect\hyperlink{after-bottom}{Continue reading the main story}

\hypertarget{site-index}{%
\subsection{Site Index}\label{site-index}}

\hypertarget{site-information-navigation}{%
\subsection{Site Information
Navigation}\label{site-information-navigation}}

\begin{itemize}
\tightlist
\item
  \href{https://help.nytimes3xbfgragh.onion/hc/en-us/articles/115014792127-Copyright-notice}{©~2020~The
  New York Times Company}
\end{itemize}

\begin{itemize}
\tightlist
\item
  \href{https://www.nytco.com/}{NYTCo}
\item
  \href{https://help.nytimes3xbfgragh.onion/hc/en-us/articles/115015385887-Contact-Us}{Contact
  Us}
\item
  \href{https://www.nytco.com/careers/}{Work with us}
\item
  \href{https://nytmediakit.com/}{Advertise}
\item
  \href{http://www.tbrandstudio.com/}{T Brand Studio}
\item
  \href{https://www.nytimes3xbfgragh.onion/privacy/cookie-policy\#how-do-i-manage-trackers}{Your
  Ad Choices}
\item
  \href{https://www.nytimes3xbfgragh.onion/privacy}{Privacy}
\item
  \href{https://help.nytimes3xbfgragh.onion/hc/en-us/articles/115014893428-Terms-of-service}{Terms
  of Service}
\item
  \href{https://help.nytimes3xbfgragh.onion/hc/en-us/articles/115014893968-Terms-of-sale}{Terms
  of Sale}
\item
  \href{https://spiderbites.nytimes3xbfgragh.onion}{Site Map}
\item
  \href{https://help.nytimes3xbfgragh.onion/hc/en-us}{Help}
\item
  \href{https://www.nytimes3xbfgragh.onion/subscription?campaignId=37WXW}{Subscriptions}
\end{itemize}
