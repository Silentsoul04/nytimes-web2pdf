Sections

SEARCH

\protect\hyperlink{site-content}{Skip to
content}\protect\hyperlink{site-index}{Skip to site index}

\href{https://www.nytimes3xbfgragh.onion/section/politics}{Politics}

\href{https://myaccount.nytimes3xbfgragh.onion/auth/login?response_type=cookie\&client_id=vi}{}

\href{https://www.nytimes3xbfgragh.onion/section/todayspaper}{Today's
Paper}

\href{/section/politics}{Politics}\textbar{}How the Russia Inquiry
Began: A Campaign Aide, Drinks and Talk of Political Dirt

\url{https://nyti.ms/2q0nOds}

\begin{itemize}
\item
\item
\item
\item
\item
\item
\end{itemize}

Advertisement

\protect\hyperlink{after-top}{Continue reading the main story}

Supported by

\protect\hyperlink{after-sponsor}{Continue reading the main story}

\hypertarget{how-the-russia-inquiry-began-a-campaign-aide-drinks-and-talk-of-political-dirt}{%
\section{How the Russia Inquiry Began: A Campaign Aide, Drinks and Talk
of Political
Dirt}\label{how-the-russia-inquiry-began-a-campaign-aide-drinks-and-talk-of-political-dirt}}

\includegraphics{https://static01.graylady3jvrrxbe.onion/images/2017/12/31/us/politics/31dc-investigate1/merlin_129497372_bf04b35f-e889-4853-9fae-c3182b845d3c-articleLarge.jpg?quality=75\&auto=webp\&disable=upscale}

By \href{https://www.nytimes3xbfgragh.onion/by/sharon-lafraniere}{Sharon
LaFraniere},
\href{https://www.nytimes3xbfgragh.onion/by/mark-mazzetti}{Mark
Mazzetti} and
\href{https://www.nytimes3xbfgragh.onion/by/matt-apuzzo}{Matt Apuzzo}

\begin{itemize}
\item
  Dec. 30, 2017
\item
  \begin{itemize}
  \item
  \item
  \item
  \item
  \item
  \item
  \end{itemize}
\end{itemize}

WASHINGTON --- During a night of heavy drinking at an upscale London bar
in May 2016, George Papadopoulos, a young foreign policy adviser to the
Trump campaign, made a startling revelation to Australia's top diplomat
in Britain: Russia had political dirt on Hillary Clinton.

About three weeks earlier, Mr. Papadopoulos had been told that Moscow
had thousands of emails that would embarrass Mrs. Clinton, apparently
stolen in an effort to try to damage her campaign.

Exactly how much Mr. Papadopoulos said that night at the Kensington Wine
Rooms with the Australian, Alexander Downer, is unclear. But two months
later,
\href{https://www.nytimes3xbfgragh.onion/2016/07/27/us/politics/assange-timed-wikileaks-release-of-democratic-emails-to-harm-hillary-clinton.html}{when
leaked Democratic emails began appearing online}, Australian officials
passed the information about Mr. Papadopoulos to their American
counterparts, according to four current and former American and foreign
officials with direct knowledge of the Australians' role.

The hacking and the revelation that a member of the Trump campaign may
have had inside information about it were driving factors that led the
F.B.I. to open an investigation in July 2016 into Russia's attempts to
disrupt the election and whether any of President Trump's associates
conspired.

If Mr. Papadopoulos, who
\href{https://www.nytimes3xbfgragh.onion/2017/10/30/us/politics/george-papadopoulos-russia-trump.html}{pleaded
guilty to lying to the F.B.I.} and is now a cooperating witness, was the
improbable match that set off a blaze that has consumed the first year
of the Trump administration, his saga is also a tale of the Trump
campaign in miniature. He was brash, boastful and underqualified, yet he
exceeded expectations. And, like the campaign itself, he proved to be a
tantalizing target for a Russian influence operation.

While some of Mr. Trump's advisers have derided him as an insignificant
campaign volunteer or a
``\href{http://www.cnn.com/2017/10/31/politics/caputo-papadopoulos-coffee-boy-cnntv/index.html}{coffee
boy},'' interviews and new documents show that he stayed influential
throughout the campaign. Two months before the election, for instance,
he helped arrange a New York meeting between Mr. Trump and President
Abdel Fattah el-Sisi of Egypt.

The information that Mr. Papadopoulos gave to the Australians answers
one of the lingering mysteries of the past year: What so alarmed
American officials to provoke the F.B.I. to
\href{https://www.nytimes3xbfgragh.onion/2017/03/20/us/politics/fbi-investigation-trump-russia-comey.html}{open
a counterintelligence investigation} into the Trump campaign months
before the presidential election?

It was not, as Mr. Trump and other politicians have alleged, a dossier
compiled by a former British spy hired by a rival campaign. Instead, it
was firsthand information from one of America's closest intelligence
allies.

Interviews and previously undisclosed documents show that Mr.
Papadopoulos played a critical role in this drama and reveal a Russian
operation that was more aggressive and widespread than previously known.
They add to an emerging portrait, gradually filled in over the past year
in revelations by federal investigators, journalists and lawmakers, of
Russians with government contacts trying to establish secret channels at
various levels of the Trump campaign.

The F.B.I. investigation, which
\href{https://www.nytimes3xbfgragh.onion/2017/05/17/us/politics/robert-mueller-special-counsel-russia-investigation.html}{was
taken over} seven months ago by the special counsel, Robert S. Mueller
III, has cast a shadow over Mr. Trump's first year in office --- even as
he and his aides repeatedly played down the Russian efforts and falsely
denied campaign contacts with Russians.

They have also insisted that Mr. Papadopoulos was a low-level figure.
But spies frequently target peripheral players as a way to gain insight
and leverage.

F.B.I. officials disagreed in 2016 about how aggressively and publicly
to pursue the Russia inquiry before the election. But there was little
debate about what seemed to be afoot. John O. Brennan, who retired this
year after four years as C.I.A. director,
\href{https://www.nytimes3xbfgragh.onion/2017/05/23/us/politics/congress-testimony-john-brennan-russia-budget.html}{told
Congress in May} that he had been concerned about multiple contacts
between Russian officials and Trump advisers.

Russia, he said, had tried to ``suborn'' members of the Trump campaign.

\hypertarget{the-signal-to-meet}{%
\subsection{`The Signal to Meet'}\label{the-signal-to-meet}}

Mr. Papadopoulos, then an ambitious 28-year-old from Chicago, was
working as an energy consultant in London when the Trump campaign,
\href{https://www.nytimes3xbfgragh.onion/2017/10/31/us/trump-foreign-policy-advisers.html}{desperate
to create a foreign policy team}, named him as an adviser in early March
2016. His political experience was limited to two months on Ben Carson's
presidential campaign before it collapsed.

Mr. Papadopoulos had no experience on Russia issues. But during his job
interview with Sam Clovis, a top early campaign aide, he saw an opening.
He was told that improving relations with Russia was one of Mr. Trump's
top foreign policy goals, according to court papers, an account Mr.
Clovis has denied.

Traveling in Italy that March, Mr. Papadopoulos met Joseph Mifsud, a
Maltese professor at a now-defunct London academy who had valuable
contacts with the Russian Ministry of Foreign Affairs. Mr. Mifsud showed
little interest in Mr. Papadopoulos at first.

\includegraphics{https://static01.graylady3jvrrxbe.onion/images/2017/12/31/us/politics/31dc-investigate2/merlin_131203434_7c1c32c9-cd38-417a-bf91-ce6907fe3337-articleLarge.jpg?quality=75\&auto=webp\&disable=upscale}

But when he found out he was a Trump campaign adviser, he latched onto
him, according to court records and emails obtained by The New York
Times. Their joint goal was to arrange a meeting between Mr. Trump and
President Vladimir V. Putin of Russia in Moscow, or between their
respective aides.

In response to questions, Mr. Papadopoulos's lawyers declined to provide
a statement.

Before the end of the month, Mr. Mifsud had arranged a meeting at a
London cafe between Mr. Papadopoulos and Olga Polonskaya, a young woman
from St. Petersburg whom he falsely described as Mr. Putin's niece.
Although Ms. Polonskaya told The Times in a text message that her
English skills are poor, her emails to Mr. Papadopoulos were largely
fluent. ``We are all very excited by the possibility of a good
relationship with Mr. Trump,'' Ms. Polonskaya wrote in one message.

\href{https://www.nytimes3xbfgragh.onion/2017/11/10/us/russia-inquiry-trump.html}{More
important}, Mr. Mifsud connected Mr. Papadopoulos to Ivan Timofeev, a
program director for the prestigious Valdai Discussion Club, a gathering
of academics that meets annually with Mr. Putin. The two men
corresponded for months about how to connect the Russian government and
the campaign. Records suggest that Mr. Timofeev, who has been described
by Mr. Mueller's team as an intermediary for the Russian Foreign
Ministry, discussed the matter with the ministry's former leader, Igor
S. Ivanov, who is widely viewed in the United States as one of Russia's
elder statesmen.

When Mr. Trump's foreign policy team gathered for the first time at the
end of March in Washington, Mr. Papadopoulos said he had the contacts to
set up a meeting between Mr. Trump and Mr. Putin. Mr. Trump listened
intently but apparently deferred to Jeff Sessions, then a senator from
Alabama and head of the campaign's foreign policy team, according to
participants in the meeting.

Mr. Sessions, now the attorney general, initially did not reveal that
discussion to Congress, because, he has said, he did not recall it. More
recently, he said he pushed back against Mr. Papadopoulos's proposal, at
least partly because he did not want someone so unqualified to represent
the campaign on such a sensitive matter.

If the campaign wanted Mr. Papadopoulos to stand down, previously
undisclosed emails obtained by The Times show that he either did not get
the message or failed to heed it. He continued for months to try to
arrange some kind of meeting with Russian representatives, keeping
senior campaign advisers abreast of his efforts. Mr. Clovis ultimately
encouraged him and another foreign policy adviser to travel to Moscow,
but neither went because the campaign would not cover the cost.

Mr. Papadopoulos was trusted enough to edit the outline of Mr. Trump's
\href{https://www.nytimes3xbfgragh.onion/2016/04/28/us/politics/donald-trump-foreign-policy-speech.html}{first
major foreign policy speech} on April 27, an address in which the
candidate said it was possible to improve relations with Russia. Mr.
Papadopoulos flagged the speech to his newfound Russia contacts, telling
Mr. Timofeev that it should be taken as ``the signal to meet.''

``That is a statesman speech,'' Mr. Mifsud agreed. Ms. Polonskaya wrote
that she was pleased that Mr. Trump's ``position toward Russia is much
softer'' than that of other candidates.

Stephen Miller, then a senior policy adviser to the campaign and now a
top White House aide, was eager for Mr. Papadopoulos to serve as a
surrogate, someone who could publicize Mr. Trump's foreign policy views
without officially speaking for the campaign. But Mr. Papadopoulos's
first public attempt to do so was a disaster.

In a May 4, 2016, interview with The Times of London, Mr. Papadopoulos
called on Prime Minister David Cameron to apologize to Mr. Trump for
criticizing his remarks on Muslims as ``stupid'' and divisive. ``Say
sorry to Trump or risk special relationship, Cameron told,''
\href{https://www.thetimes.co.uk/article/say-sorry-to-trump-or-risk-special-relationship-cameron-told-h6ng0r7xj}{the
headline read}. Mr. Clovis, the national campaign co-chairman, severely
reprimanded Mr. Papadopoulos for failing to clear his explosive comments
with the campaign in advance.

From then on, Mr. Papadopoulos was more careful with the press ---
though he never regained the full trust of Mr. Clovis or several other
campaign officials.

Mr. Mifsud proposed to Mr. Papadopoulos that he, too, serve as a
campaign surrogate. He could write op-eds under the guise of a
``neutral'' observer, he wrote in a previously undisclosed email, and
follow Mr. Trump to his rallies as an accredited journalist while
receiving briefings from the inside the campaign.

In late April, at a London hotel, Mr. Mifsud told Mr. Papadopoulos that
he had just learned from high-level Russian officials in Moscow that the
Russians had ``dirt'' on Mrs. Clinton in the form of ``thousands of
emails,'' according to court documents. Although Russian hackers had
been mining data from the Democratic National Committee's computers for
months, that information was not yet public. Even the committee itself
did not know.

Whether Mr. Papadopoulos shared that information with anyone else in the
campaign is one of many unanswered questions. He was mostly in contact
with the campaign over emails. The day after Mr. Mifsud's revelation
about the hacked emails, he told Mr. Miller in an email only that he had
``interesting messages coming in from Moscow'' about a possible trip.
The emails obtained by The Times show no evidence that Mr. Papadopoulos
discussed the stolen messages with the campaign.

Not long after, however, he opened up to Mr. Downer, the Australian
diplomat, about his contacts with the Russians. It is unclear whether
Mr. Downer was fishing for that information that night in May 2016. The
meeting at the bar came about because of a series of connections,
beginning with an Israeli Embassy official who introduced Mr.
Papadopoulos to another Australian diplomat in London.

It is also not clear why, after getting the information in May, the
Australian government waited two months to pass it to the F.B.I. In a
statement, the Australian Embassy in Washington declined to provide
details about the meeting or confirm that it occurred.

``As a matter of principle and practice, the Australian government does
not comment on matters relevant to active investigations,'' the
statement said. The F.B.I. declined to comment.

Image

A House Judiciary Committee session last month at which Attorney General
Jeff Sessions testified. Mr. Sessions was head of the Trump campaign's
foreign policy team.Credit...Al Drago for The New York Times

\hypertarget{a-secretive-investigation}{%
\subsection{A Secretive Investigation}\label{a-secretive-investigation}}

Once the information Mr. Papadopoulos had disclosed to the Australian
diplomat reached the F.B.I., the bureau opened an investigation that
became one of its most closely guarded secrets. Senior agents did not
discuss it at the daily morning briefing, a classified setting where
officials normally speak freely about highly sensitive operations.

Besides the information from the Australians, the investigation was also
propelled by intelligence from other friendly governments, including the
British and Dutch. A trip to Moscow by
\href{https://www.nytimes3xbfgragh.onion/2017/12/18/magazine/what-if-anything-does-carter-page-know.html}{another
adviser, Carter Page}, also raised concerns at the F.B.I.

With so many strands coming in --- about Mr. Papadopoulos, Mr. Page, the
hackers and more --- F.B.I. agents debated how aggressively to
investigate the campaign's Russia ties, according to current and former
officials familiar with the debate. Issuing subpoenas or questioning
people, for example, could cause the investigation to burst into public
view in the final months of a presidential campaign.

It could also tip off the Russian government, which might try to cover
its tracks. Some officials argued against taking such disruptive steps,
especially since the F.B.I. would not be able to unravel the case before
the election.

Others believed that the possibility of a compromised presidential
campaign was so serious that it warranted the most thorough, aggressive
tactics. Even if the odds against a Trump presidency were long, these
agents argued, it was prudent to take every precaution.

That included questioning Christopher Steele, the former British spy who
was
\href{https://www.nytimes3xbfgragh.onion/2017/01/11/us/politics/donald-trump-russia-intelligence.html}{compiling
the dossier} alleging a far-ranging Russian conspiracy to elect Mr.
Trump. A team of F.B.I. agents traveled to Europe to interview Mr.
Steele in early October 2016. Mr. Steele had shown some of his findings
to an F.B.I. agent in Rome three months earlier, but that information
was not part of the justification to start an counterintelligence
inquiry, American officials said.

Ultimately, the F.B.I. and Justice Department decided to keep the
investigation quiet, a decision that Democrats in particular have
criticized. And agents did not interview Mr. Papadopoulos until late
January.

\hypertarget{opening-doors-to-the-top}{%
\subsection{Opening Doors, to the Top}\label{opening-doors-to-the-top}}

He was hardly central to the daily running of the Trump campaign, yet
Mr. Papadopoulos continuously found ways to make himself useful to
senior Trump advisers. In September 2016, with the United Nations
General Assembly approaching and stories circulating that Mrs. Clinton
was going to meet with Mr. Sisi, the Egyptian president, Mr.
Papadopoulos sent a message to Stephen K. Bannon, the campaign's chief
executive, offering to broker a similar meeting for Mr. Trump.

After days of scheduling discussions, the meeting was set and Mr.
Papadopoulos sent a list of talking points to Mr. Bannon, according to
people familiar with those interactions. Asked about his contacts with
Mr. Papadopoulos, Mr. Bannon declined to comment.

Mr. Trump's improbable victory raised Mr. Papadopoulos's hopes that he
might ascend to a top White House job. The election win also prompted a
business proposal from Sergei Millian, a naturalized American citizen
born in Belarus. After he had contacted Mr. Papadopoulos out of the blue
over LinkedIn during the summer of 2016, the two met repeatedly in
Manhattan.

Mr. Millian has bragged of his ties to Mr. Trump --- boasts that the
president's advisers have said are overstated. He headed an obscure
organization called the Russian-American Chamber of Commerce, some of
whose board members and clients are difficult to confirm. Congress is
investigating where he fits into the swirl of contacts with the Trump
campaign, although he has said he is unfairly being scrutinized only
because of his support for Mr. Trump.

Mr. Millian proposed that he and Mr. Papadopoulos form an energy-related
business that would be financed by Russian billionaires ``who are not
under sanctions'' and would ``open all doors for us'' at ``any level all
the way to the top.''

One billionaire, he said, wanted to explore the idea of opening a
Trump-branded hotel in Moscow. ``I know the president will distance
himself from business, but his children might be interested,'' he wrote.

Nothing came of his proposals, partly because Mr. Papadopoulos was
hoping that Michael T. Flynn, then Mr. Trump's pick to be national
security adviser, might give him the energy portfolio at the National
Security Council.

The pair exchanged New Year's greetings in the final hours of 2016.
``Happy New Year, sir,'' Mr. Papadopoulos wrote.

``Thank you and same to you, George. Happy New Year!'' Mr. Flynn
responded, ahead of a year that seemed to hold great promise.

But 2017 did not unfold that way. Within months, Mr. Flynn was fired,
and both men were charged with lying to the F.B.I. And both became
important witnesses in the investigation Mr. Papadopoulos had played a
critical role in starting.

Advertisement

\protect\hyperlink{after-bottom}{Continue reading the main story}

\hypertarget{site-index}{%
\subsection{Site Index}\label{site-index}}

\hypertarget{site-information-navigation}{%
\subsection{Site Information
Navigation}\label{site-information-navigation}}

\begin{itemize}
\tightlist
\item
  \href{https://help.nytimes3xbfgragh.onion/hc/en-us/articles/115014792127-Copyright-notice}{©~2020~The
  New York Times Company}
\end{itemize}

\begin{itemize}
\tightlist
\item
  \href{https://www.nytco.com/}{NYTCo}
\item
  \href{https://help.nytimes3xbfgragh.onion/hc/en-us/articles/115015385887-Contact-Us}{Contact
  Us}
\item
  \href{https://www.nytco.com/careers/}{Work with us}
\item
  \href{https://nytmediakit.com/}{Advertise}
\item
  \href{http://www.tbrandstudio.com/}{T Brand Studio}
\item
  \href{https://www.nytimes3xbfgragh.onion/privacy/cookie-policy\#how-do-i-manage-trackers}{Your
  Ad Choices}
\item
  \href{https://www.nytimes3xbfgragh.onion/privacy}{Privacy}
\item
  \href{https://help.nytimes3xbfgragh.onion/hc/en-us/articles/115014893428-Terms-of-service}{Terms
  of Service}
\item
  \href{https://help.nytimes3xbfgragh.onion/hc/en-us/articles/115014893968-Terms-of-sale}{Terms
  of Sale}
\item
  \href{https://spiderbites.nytimes3xbfgragh.onion}{Site Map}
\item
  \href{https://help.nytimes3xbfgragh.onion/hc/en-us}{Help}
\item
  \href{https://www.nytimes3xbfgragh.onion/subscription?campaignId=37WXW}{Subscriptions}
\end{itemize}
