Sections

SEARCH

\protect\hyperlink{site-content}{Skip to
content}\protect\hyperlink{site-index}{Skip to site index}

\href{https://www.nytimes3xbfgragh.onion/section/business}{Business}

\href{https://myaccount.nytimes3xbfgragh.onion/auth/login?response_type=cookie\&client_id=vi}{}

\href{https://www.nytimes3xbfgragh.onion/section/todayspaper}{Today's
Paper}

\href{/section/business}{Business}\textbar{}A Saucy App Knows China's
Taste in News. The Censors Are Worried.

\url{https://nyti.ms/2DTwhkv}

\begin{itemize}
\item
\item
\item
\item
\item
\end{itemize}

Advertisement

\protect\hyperlink{after-top}{Continue reading the main story}

Supported by

\protect\hyperlink{after-sponsor}{Continue reading the main story}

\hypertarget{a-saucy-app-knows-chinas-taste-in-news-the-censors-are-worried}{%
\section{A Saucy App Knows China's Taste in News. The Censors Are
Worried.}\label{a-saucy-app-knows-chinas-taste-in-news-the-censors-are-worried}}

\includegraphics{https://static01.graylady3jvrrxbe.onion/images/2018/01/03/world/03Toutiao/merlin_131797532_66fb05ff-cf15-498b-8f0f-de8f476f581a-articleLarge.jpg?quality=75\&auto=webp\&disable=upscale}

By \href{https://www.nytimes3xbfgragh.onion/by/raymond-zhong}{Raymond
Zhong}

\begin{itemize}
\item
  Jan. 2, 2018
\item
  \begin{itemize}
  \item
  \item
  \item
  \item
  \item
  \end{itemize}
\end{itemize}

\href{https://cn.nytimes3xbfgragh.onion/business/20180103/china-toutiao-censorship/}{阅读简体中文版}\href{https://cn.nytimes3xbfgragh.onion/business/20180103/china-toutiao-censorship/zh-hant/}{閱讀繁體中文版}

HONG KONG --- One of the world's most valuable start-ups got that way by
using artificial intelligence to satisfy Chinese internet users'
voracious appetite for news and entertainment. Every day, its smartphone
app feeds 120 million people personalized streams of buzzy news stories,
videos of dogs frolicking in snow, GIFs of traffic mishaps and listicles
such as ``The World's Ugliest Celebrities.''

Now the company is discovering the risks involved, under China's
censorship regime, in giving the people exactly what they want.

The makers of the popular news app Jinri Toutiao unveiled moves this
week to allay rising concerns from the authorities. Last week, the
Beijing bureau of China's top internet regulator accused Toutiao of
``spreading pornographic and vulgar information'' and ``causing a
negative impact on public opinion online,'' and it ordered that updates
to several popular sections of the app be halted for 24 hours.

In response, the app's parent company, Beijing Bytedance Technology,
took down or temporarily suspended the accounts of more than 1,100
bloggers that it said had been publishing ``low-quality content'' on the
app. It also replaced Toutiao's ``Society'' section with a new section
called ``New Era,'' which is heavy on state media coverage of government
decisions.

The change was made, the company said, to ``promote the spirit of the
Communist Party congress,'' referring to the
\href{https://www.nytimes3xbfgragh.onion/2017/10/17/world/asia/xi-jinping-communist-party-china.html}{gathering
of top party leaders that took place in Beijing} in October.

The episode points to the fine line that Toutiao's creators must walk.

Despite China's famously strict censorship, online news is a big
business there. More than 610 million people in the country gained
access to some news on the internet in 2016, according to official
statistics.

Toutiao, which says it uses complex algorithms to decide what its users
see, combines China's hunger for media content with its
\href{https://www.nytimes3xbfgragh.onion/2017/05/27/technology/china-us-ai-artificial-intelligence.html}{rising
ambitions in artificial intelligence}. Its daily user base of 120
million people is equivalent to more than one-third of the population of
the United States.

Suan Lin, a 24-year-old private equity analyst in Shanghai, said that
she normally has to search high and low online to find articles about
the Chinese historical dramas she watches on television. But Toutiao
delivers, she said.

``Once you're on it,'' she said, ``you just can't stop.''

In China, however, a strong position in media invites scrutiny from the
government's censorship apparatus. That scrutiny has become heightened
over the past two years as the authorities have looked beyond the
political to crack down on news it sees as degrading to society as a
whole, which can include things as seemingly unsubversive
\href{https://www.nytimes3xbfgragh.onion/2017/06/09/world/asia/china-celebrity-news-wechat.html}{as
celebrity gossip}.

In Toutiao's case, one of the accounts that were suspended this week had
posted a saucy video of a woman in a short skirt. It got 57,000 views.
Another suspended account had recently put up a post titled ``The
World's Ugliest Celebrities, Michael Jackson Is Ranked First, You Won't
Want to Eat After Reading This.''

``Once you have more people watching, then you want to be more
cautious,'' Wei-Ying Ma, who heads Toutiao's artificial intelligence
lab, told \href{http://aideas.toutiao.com/index.html}{a conference in
Beijing} last month.

As Toutiao's popularity has skyrocketed, Bytedance has become a darling
of Silicon Valley investors such as Sequoia Capital. The company, which
is currently valued at \$20 billion, has been in talks with existing
backers to raise new financing that would value the company at more than
\$30 billion, according to a person familiar with the discussions who
spoke on the condition of anonymity because the details are not public.

That price tag would make Bytedance among the most valuable privately
held technology companies in the world, not just in China. Airbnb is
said to be
\href{https://www.nytimes3xbfgragh.onion/2017/03/09/technology/airbnb-1-billion-funding.html}{valued
at around \$30 billion}. SpaceX, the rocket maker founded by Elon Musk,
is
\href{https://www.nytimes3xbfgragh.onion/2017/07/27/technology/spacex-is-now-one-of-the-worlds-most-valuable-privately-held-companies.html}{valued
at \$21 billion}.

Bytedance has big plans for overseas expansion, too. It recently
\href{https://www.nytimes3xbfgragh.onion/2017/11/10/business/dealbook/musically-sold-app-video.html}{spent
between \$800 million and \$1 billion to purchase Musical.ly}, a
video-based social network popular with teenagers in the United States
and Europe. At the Beijing conference last month, a top Bytedance
executive, Liu Zhen, said the company hoped to be earning half its
revenue from outside China within the next five years.

Jinri Toutiao, whose name means ``today's headlines'' in Chinese and is
pronounced JING-er TOE-tee-yow, aggregates content from various sources
and looks much like Facebook's newsfeed. But instead of displaying
articles and videos based on what your friends have shared, the app does
so based on what you have previously read and watched on the app.

If you click on articles about iPhones, then Toutiao will feed you more
tech coverage. After you watch a few cooking videos, the app will fetch
you more clips of people wrapping dumplings and braising chicken's feet.

This approach has helped Toutiao thrive amid China's heavily controlled
environment for social media. Instead of policing the sharing activity
of tens of millions of users, the company needs only to calibrate and
adjust its centralized recommendation software.

But it also needs to make sure the app's content does not cross the
lines of censors. That is a huge task, particularly given that the
overwhelming majority of content on Toutiao is produced by individual
bloggers, not professional news organizations or other institutions. Ms.
Liu said at last month's conference in Beijing that 90 percent of the
app's content comes from blogger accounts. Toutiao has around 1.2
million content-producing accounts in total.

According to Bytedance, every piece of content is automatically screened
to check that it is acceptable before appearing on Toutiao. But once
something has attracted more views, the system applies a more
sophisticated screening algorithm. Certain material is also examined by
humans as a final check.

Bytedance also takes more overt steps to stay on the right side of the
authorities. Important updates from the government are sometimes pinned
to the top of a user's feed. That can lead to awkward juxtapositions ---
between, say, a state media write-up on President Xi Jinping's recent
decisions and a photo slide show on six women who are ``so beautiful
that rich businessmen immediately became attracted to them,'' as the
piece's headline puts it.

Toutiao has come in for official rebuke before. Last June, the Beijing
bureau of the Cyberspace Administration of China
\href{http://news.xinhuanet.com/politics/2017-06/07/c_1121104271.htm}{ordered
around a dozen accounts} on the app shut down, calling on Toutiao and
other news portals to ``actively promote socialist core values'' and
create a ``healthy, uplifting environment for mainstream opinion'' by
eschewing dishy coverage of celebrity scandals.

In September, the website of the People's Daily newspaper, the official
mouthpiece of the Communist Party, published a
\href{http://opinion.people.com.cn/n1/2017/0918/c1003-29540709.html}{series
of opinion articles} strongly criticizing A.I.-based news apps,
including Toutiao, for spreading misinformation and superficial content.

Despite Toutiao's popularity, some in China share that view. Yang Sun, a
26-year-old financial analyst in Shanghai, decried the app's
sensationalist headlines.

``It should absolutely be taken offline,'' Ms. Yang said. ``Totally
deserves it.''

Advertisement

\protect\hyperlink{after-bottom}{Continue reading the main story}

\hypertarget{site-index}{%
\subsection{Site Index}\label{site-index}}

\hypertarget{site-information-navigation}{%
\subsection{Site Information
Navigation}\label{site-information-navigation}}

\begin{itemize}
\tightlist
\item
  \href{https://help.nytimes3xbfgragh.onion/hc/en-us/articles/115014792127-Copyright-notice}{©~2020~The
  New York Times Company}
\end{itemize}

\begin{itemize}
\tightlist
\item
  \href{https://www.nytco.com/}{NYTCo}
\item
  \href{https://help.nytimes3xbfgragh.onion/hc/en-us/articles/115015385887-Contact-Us}{Contact
  Us}
\item
  \href{https://www.nytco.com/careers/}{Work with us}
\item
  \href{https://nytmediakit.com/}{Advertise}
\item
  \href{http://www.tbrandstudio.com/}{T Brand Studio}
\item
  \href{https://www.nytimes3xbfgragh.onion/privacy/cookie-policy\#how-do-i-manage-trackers}{Your
  Ad Choices}
\item
  \href{https://www.nytimes3xbfgragh.onion/privacy}{Privacy}
\item
  \href{https://help.nytimes3xbfgragh.onion/hc/en-us/articles/115014893428-Terms-of-service}{Terms
  of Service}
\item
  \href{https://help.nytimes3xbfgragh.onion/hc/en-us/articles/115014893968-Terms-of-sale}{Terms
  of Sale}
\item
  \href{https://spiderbites.nytimes3xbfgragh.onion}{Site Map}
\item
  \href{https://help.nytimes3xbfgragh.onion/hc/en-us}{Help}
\item
  \href{https://www.nytimes3xbfgragh.onion/subscription?campaignId=37WXW}{Subscriptions}
\end{itemize}
