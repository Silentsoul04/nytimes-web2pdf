Sections

SEARCH

\protect\hyperlink{site-content}{Skip to
content}\protect\hyperlink{site-index}{Skip to site index}

\href{https://www.nytimes3xbfgragh.onion/section/world/asia}{Asia
Pacific}

\href{https://myaccount.nytimes3xbfgragh.onion/auth/login?response_type=cookie\&client_id=vi}{}

\href{https://www.nytimes3xbfgragh.onion/section/todayspaper}{Today's
Paper}

\href{/section/world/asia}{Asia Pacific}\textbar{}Trump Rebuked China
for North Korea's Oil Smuggling. It's More Complicated.

\url{https://nyti.ms/2DgR6qs}

\begin{itemize}
\item
\item
\item
\item
\item
\item
\end{itemize}

Advertisement

\protect\hyperlink{after-top}{Continue reading the main story}

Supported by

\protect\hyperlink{after-sponsor}{Continue reading the main story}

\hypertarget{trump-rebuked-china-for-north-koreas-oil-smuggling-its-more-complicated}{%
\section{Trump Rebuked China for North Korea's Oil Smuggling. It's More
Complicated.}\label{trump-rebuked-china-for-north-koreas-oil-smuggling-its-more-complicated}}

\includegraphics{https://static01.graylady3jvrrxbe.onion/images/2018/01/19/world/19ship-1/18ship-1-articleLarge.jpg?quality=75\&auto=webp\&disable=upscale}

By Chris Horton,
\href{https://www.nytimes3xbfgragh.onion/by/steven-lee-myers}{Steven Lee
Myers} and
\href{https://www.nytimes3xbfgragh.onion/by/michael-schwirtz}{Michael
Schwirtz}

\begin{itemize}
\item
  Jan. 18, 2018
\item
  \begin{itemize}
  \item
  \item
  \item
  \item
  \item
  \item
  \end{itemize}
\end{itemize}

\href{https://cn.nytimes3xbfgragh.onion/china/20180119/north-korea-oil-smuggling/}{阅读简体中文版}\href{https://cn.nytimes3xbfgragh.onion/china/20180119/north-korea-oil-smuggling/zh-hant/}{閱讀繁體中文版}

KAOHSIUNG, Taiwan --- The two ships met in daylight in the middle of the
East China Sea. One was an 11,253-ton oil tanker, the Lighthouse
Winmore, supposedly heading to Taiwan. The other, an aging freighter,
was emblazoned with the red, white and blue flag of North Korea.

In an illicit high-seas exchange captured in photographs taken by an
American spy plane, the Lighthouse Winmore offloaded what officials
later said was 600 tons of oil to the North Korean vessel in violation
of
\href{https://www.nytimes3xbfgragh.onion/2017/12/22/world/asia/north-korea-security-council-nuclear-missile-sanctions.html}{United
Nations sanctions}.

With those sanctions constricting its trade, including the import of
refined petroleum, North Korea has increasingly turned to illegal
clandestine shipments to acquire the fuel it needs, according to
diplomatic officials and documents obtained by The New York Times.

Trafficking on the high seas has become what these officials regard as a
pernicious subversion of the sanctions. Though the frequency of
smuggling is difficult to estimate, they fear it is undermining efforts
to thwart the nuclear weapons ambitions of Kim Jong-un, the North Korean
leader, through economic pressure. The trafficking has also strained
relations between the United States and North Korea's two largest
trading partners, China and Russia.

Last month, the United States tried to persuade other members of the
United Nations Security Council to blacklist 10 ships that it said were
involved in smuggling oil and coal. In addition to the Lighthouse
Winmore, this list, obtained by The New York Times, included four North
Korean-flagged vessels, as well as ships linked to South Korea, Hong
Kong, Taiwan and China.

A United Nations diplomat said the transfers were happening frequently
in the Yellow Sea, the East China Sea and possibly the Sea of Japan.

The diplomat, who spoke on the condition of anonymity to discuss
intelligence materials, said that detecting the smuggling was difficult
and that preventing it would require interdiction of suspect ships at
sea, which could further inflame tensions with North Korea.

Determining who abets the North Koreans has proved difficult. China and
Russia have been blamed for the smuggling, or at least for failing to
counter it, though evidence of direct government involvement is slim.

A day before South Korea announced late last month that it had
\href{https://www.nytimes3xbfgragh.onion/2017/12/29/world/asia/south-korea-ship-seized.html}{impounded
the Lighthouse Winmore} in November, President Trump seemed to directly
implicate China in the smuggling, writing on Twitter that it had been
\href{https://twitter.com/realDonaldTrump/status/946416486054285314}{``caught
RED HANDED''} allowing illicit oil deliveries to North Korea.

Tracing the Winmore's ownership and its contraband oil underscores the
difficulty of identifying who is complicit, despite what the president's
jab suggested.

\includegraphics{https://static01.graylady3jvrrxbe.onion/images/2018/01/19/world/asia/19ship-cover/19ship-cover-videoSixteenByNineJumbo1600.png}

Most of the Lighthouse Winmore's 25 crewmen were Chinese, but other
connections to China were more tenuous. Some links were to places where
the United States may be just as influential.

The ship's flag showed that it was from Hong Kong, which for the last
two decades has been a semiautonomous special administrative region of
China. The oil originated with a multinational commodities trader,
Trafigura Group, and was sold first to a company in Hong Kong, then to a
company in Taiwan, the island that China regards as a breakaway province
and that maintains close but unofficial relations with the United
States.

The ship's owner appears to be a Hong Kong-based company whose director
lives in Guangzhou, China. But the ship was leased by a fishing magnate
from Taiwan, Chen Shih-hsien, whose company, Billions Bunker Group, was
until last month registered in the Marshall Islands, a Pacific Ocean
nation that enjoys American military protection.

The difficulty has been compounded by the opacity of the international
shipping industry, where vessels can be flagged in faraway countries and
ownership is often obscured to limit legal liability.

\hypertarget{covering-their-tracks}{%
\subsection{Covering Their Tracks}\label{covering-their-tracks}}

The North Koreans, too, have become experts at covering their tracks
\href{https://www.nytimes3xbfgragh.onion/2017/05/12/world/asia/north-korea-sanctions-loopholes-china-united-states-garment-industry.html}{to
evade sanctions}. Identification numbers are forged; vessels are
registered under ``flags of convenience'' to mask their North Korean
origins; names and even paint jobs are changed frequently.

The latest
\href{http://www.un.org/ga/search/view_doc.asp?symbol=S/2017/742}{sanctions
report} to the United Nations Security Council in September warned that
``lax enforcement of the sanctions regime coupled with the country's
evolving evasion techniques are undermining the goals of the
resolutions'' intended to halt North Korea's nuclear and missile
development.

South Korea's November seizure of the Lighthouse Winmore was made public
only in late December --- after Mr. Trump's post on Twitter. Two days
later, South Korea disclosed the impounding of another ship, the Koti,
flagged in Panama and operated by Harmonized Resources Shipping
Management Company in Hong Kong.

According to Hong Kong corporate records, the company's owner is an
individual named Ma Guixian, and its headquarters is in China's port of
Dalian, across the bay from North Korea. The phone at the company has
since been disconnected.

After the seizures were publicized, the Chinese government bristled at
suggestions that it had been somehow involved.

``Instead of fixing their eyes on Chinese shipping, they had better ask
their government whether it has fully and comprehensively implemented
the relevant Security Council resolutions,'' a foreign affairs
spokeswoman, Hua Chunying, told journalists. Can any country, she added,
``be 100 percent sure that what should be done is done and not a single
breach will happen?''

\includegraphics{https://static01.graylady3jvrrxbe.onion/images/2018/01/17/world/asia/sub-ship2/sub-ship2-articleLarge.jpg?quality=75\&auto=webp\&disable=upscale}

China has enforced sanctions against North Korea by cutting imports of
coal, seafood and a variety of other products. Mr. Trump, as recently as
this month, praised the Chinese government for ``sharply reducing its
trade with North Korea,'' saying in an official
\href{https://www.whitehouse.gov/briefings-statements/statement-press-secretary-19/}{White
House statement} that ``this action supports the United States-led
global effort.''

As Ms. Hua's remarks made clear, though, China has not ruled out that
some illicit traffic has continued.

Stopping ship-to-ship smuggling on the open seas is complicated. The
suspect area is vast and hard to patrol. Finding smugglers among the
hundreds of ships carrying out oil transfers at sea is almost
impossible. Such ship-to-ship transfers are legal if the recipient is
not North Korea.

Any effort to blockade North Korean ports or inspect North Korean-bound
ships suspected of carrying illicit oil could be considered acts of war
by Mr. Kim and could lead to further escalation.

Determining how much illicit petroleum is reaching North Korea is also
difficult, experts said. Before 2017, North Korea imported about 4.5
million barrels of refined petroleum a year, mostly from Russia and
China. A United Nations resolution adopted last month imposed a cap of
500,000 barrels of refined petroleum a year. (North Korea is also
permitted annual imports of four million barrels of unrefined oil, much
of which comes from China via pipeline.)

Western officials say it is unlikely that North Korea could smuggle in
enough refined oil to make up the difference.

The activities of entrepreneurs like Mr. Chen are providing a lifeline,
though. So far, officials have not said definitively whether Mr. Chen
was a small-time smuggler or a major figure in the illicit oil trade. He
was quoted by several news organizations as saying that he was not aware
that the transferred oil had been bound for North Korea. He could not be
reached for comment.

Born in 1965, he followed his family into the commercial fishing
industry, becoming a prominent figure in the tuna trade as a member of
the Taiwan Tuna Association, based in the southern port city Kaohsiung.

Records show that Mr. Chen also owned two ships, known as bunker
vessels, used for refueling at sea. These vessels, which are like
floating gas stations, are an integral part of the international
shipping system, particularly in the fishing industry, allowing ships to
stay at sea longer. The Lighthouse Winmore, which he leased, was also a
bunker vessel.

\hypertarget{billions-bunker-group}{%
\subsection{Billions Bunker Group}\label{billions-bunker-group}}

Mr. Chen registered a new company in the Marshall Islands, the Billions
Bunker Group, in 2014. The Marshall Islands is one of only 20 countries
with official diplomatic ties with Taiwan. According to the corporate
registry there, the company was annulled on Dec. 29, the day South Korea
made public its seizure of the Lighthouse Winmore.

Image

The Friendship Bridge, which connects the North Korean town of Sinuiju
with the Chinese city of Dandong, is used to transport goods between the
two countries.Credit...Lam Yik Fei for The New York Times

Billions Bunker separately registered a company called Taiwan Group
Corporation in the British Virgin Islands, creating a complex network of
holdings, including others that were disclosed in the
\href{https://www.nytimes3xbfgragh.onion/2016/04/05/world/panama-papers-explainer.html}{Panama
Papers} leaks that revealed hidden wealth in offshore tax havens.

Prosecutors in South Korea said the Lighthouse Winmore's manifest
falsely indicated that it had been headed to Taiwan when it departed the
South Korean port of Yeosu on Oct. 11. Eight days later, an American
naval intelligence aircraft photographed it seemingly transferring oil
to the Sam Jong 2, according to documents presented to the United
Nations Security Council and seen by The Times.

The documents also included photographs of the two ships owned by Mr.
Chen's company: the Billions 18 and the Billions 88. While officials say
that the Billions 18 was involved in smuggling, evidence is insufficient
to say the same about the Billions 88.

The authorities in Taiwan arrested Mr. Chen on Jan. 3, their
investigation seemingly prompted by the South Korean seizure. He was
released on bail for the equivalent of about \$50,000, according to a
statement by the district prosecutor's office in Kaohsiung, which said
that he faced charges of falsifying export documents.

No one answered the door at Mr. Chen's home in Tainan, a city just north
of Kaohsiung.

The Kaohsiung offices of two of Mr. Chen's businesses, Yingjen Fishery
Company and Kao Yang Fishery Company, were shuttered. A neighboring
shopkeeper said they had been frequented until late December.

People who know Mr. Chen expressed shock that he may have been smuggling
to North Korea. At the headquarters of the Taiwan Tuna Association, an
employee, Simon Lee, described him as thoughtful and polite --- not
typical traits in the fishing business.

``I thought to myself, `How could it be him?''' Mr. Lee said.

Since the South Koreans took action, the names of the two ships Mr. Chen
owned have been changed, a common industry practice that makes tracking
ownership more difficult. The Billions 18 became the Kingsway, still
under the flag of Panama, while the Billions 88 became the Twins Bull,
flagged in Palau, according to
\href{https://www.marinetraffic.com/}{Marine Traffic}, a website that
tracks shipping.

On Jan. 12, Taiwan's Ministry of Justice announced sanctions on Mr.
Chen, freezing his assets and forbidding others to do business with him.

Smuggling tied to North Korea will remain difficult to eradicate,
officials and experts said.

As North Korea faces tighter restrictions on trade, those willing to
evade the sanctions can command even higher premiums. That is why the
United States and others have sought to press China and Russia to police
sanctions more vigorously. Officials say they have shown little
inclination.

\hypertarget{photos-and-evidence}{%
\subsection{Photos and Evidence}\label{photos-and-evidence}}

In December, when American diplomats presented Russia and China with the
list of 10 ships it claimed had been involved in smuggling, they also
provided photos and evidence from the ships' electronic monitoring
systems. Included in the evidence were maps showing the routes of three
ships that apparently had illegally delivered North Korean coal to ports
in Russia and Vietnam. Most of the others, including two vessels
connected to Mr. Chen, were involved in illicit ship-to-ship oil
transfers.

The Chinese agreed to blacklist four. The Billions 18 was included; the
Lighthouse Winmore was not.

``Russia and China have never been fully committed to North Korean
sanctions,'' one Security Council diplomat said, speaking on the
condition of anonymity because the topic is so contentious. ``They are
trying to do whatever it takes to get away with it rather than a full
implementation.''

Advertisement

\protect\hyperlink{after-bottom}{Continue reading the main story}

\hypertarget{site-index}{%
\subsection{Site Index}\label{site-index}}

\hypertarget{site-information-navigation}{%
\subsection{Site Information
Navigation}\label{site-information-navigation}}

\begin{itemize}
\tightlist
\item
  \href{https://help.nytimes3xbfgragh.onion/hc/en-us/articles/115014792127-Copyright-notice}{©~2020~The
  New York Times Company}
\end{itemize}

\begin{itemize}
\tightlist
\item
  \href{https://www.nytco.com/}{NYTCo}
\item
  \href{https://help.nytimes3xbfgragh.onion/hc/en-us/articles/115015385887-Contact-Us}{Contact
  Us}
\item
  \href{https://www.nytco.com/careers/}{Work with us}
\item
  \href{https://nytmediakit.com/}{Advertise}
\item
  \href{http://www.tbrandstudio.com/}{T Brand Studio}
\item
  \href{https://www.nytimes3xbfgragh.onion/privacy/cookie-policy\#how-do-i-manage-trackers}{Your
  Ad Choices}
\item
  \href{https://www.nytimes3xbfgragh.onion/privacy}{Privacy}
\item
  \href{https://help.nytimes3xbfgragh.onion/hc/en-us/articles/115014893428-Terms-of-service}{Terms
  of Service}
\item
  \href{https://help.nytimes3xbfgragh.onion/hc/en-us/articles/115014893968-Terms-of-sale}{Terms
  of Sale}
\item
  \href{https://spiderbites.nytimes3xbfgragh.onion}{Site Map}
\item
  \href{https://help.nytimes3xbfgragh.onion/hc/en-us}{Help}
\item
  \href{https://www.nytimes3xbfgragh.onion/subscription?campaignId=37WXW}{Subscriptions}
\end{itemize}
