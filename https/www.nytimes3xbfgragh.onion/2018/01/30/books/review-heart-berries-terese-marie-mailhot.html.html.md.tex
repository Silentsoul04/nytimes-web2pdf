Sections

SEARCH

\protect\hyperlink{site-content}{Skip to
content}\protect\hyperlink{site-index}{Skip to site index}

\href{https://www.nytimes3xbfgragh.onion/section/books}{Books}

\href{https://myaccount.nytimes3xbfgragh.onion/auth/login?response_type=cookie\&client_id=vi}{}

\href{https://www.nytimes3xbfgragh.onion/section/todayspaper}{Today's
Paper}

\href{/section/books}{Books}\textbar{}January's Book Club Pick: `Heart
Berries' Shatters a Pattern of Silence

\url{https://nyti.ms/2FtKZ2G}

\begin{itemize}
\item
\item
\item
\item
\item
\end{itemize}

Advertisement

\protect\hyperlink{after-top}{Continue reading the main story}

Supported by

\protect\hyperlink{after-sponsor}{Continue reading the main story}

\href{/column/books-of-the-times}{Books of The Times}

\hypertarget{januarys-book-club-pick-heart-berries-shatters-a-pattern-of-silence}{%
\section{January's Book Club Pick: `Heart Berries' Shatters a Pattern of
Silence}\label{januarys-book-club-pick-heart-berries-shatters-a-pattern-of-silence}}

By \href{https://www.nytimes3xbfgragh.onion/by/parul-sehgal}{Parul
Sehgal}

\begin{itemize}
\item
  Jan. 30, 2018
\item
  \begin{itemize}
  \item
  \item
  \item
  \item
  \item
  \end{itemize}
\end{itemize}

\includegraphics{https://static01.graylady3jvrrxbe.onion/images/2018/01/31/arts/31bookmailhot1/31bookmailhot1-popup-v3.jpg?quality=75\&auto=webp\&disable=upscale}

Don't be fooled by the title. Terese Marie Mailhot's memoir, published
under the romantic, rather forgettable name ``Heart Berries,'' is a
sledgehammer.

In a book slender enough to slide into your back pocket, Mailhot reckons
with the wages of intergenerational trauma. She grew up on Seabird
Island Indian Reservation in British Columbia. Members of her family had
passed through Canada's brutal residential school system, which
separated indigenous children from their families and cultures, and, in
some cases, subjected them to physical and sexual abuse. Mailhot's
grandmother went to one such school. So many children reportedly starved
to death there, the nuns ran out of places to bury them; their bones
were hidden in the walls of a new boarding school under construction.

These phantoms speak throughout Mailhot's book --- they speak through
her. She began working on it when she had herself committed after a
breakdown. She wrote her way out of the chaos of her past, asking: ``How
could misfortune follow me so well, and why did I choose it every
time?''

Mailhot's early life was pocked with poverty, addiction and abuse. Her
affectionate but absent mother brought home men who preyed upon her
children. Her satanic father, incarcerated after he and four other men
abducted a girl, was so terrifying that Mailhot's maternal grandmother
saved up money to hire a hit man to kill him. He survived, but met a
violent death soon enough, killed in a brawl over a prostitute or, some
say, a cigarette.

Mailhot grew up in foster care for a time, and after she aged out, at
17, she married --- she didn't know what else to do. She had a child,
and lost custody of him as she was giving birth to another. The marriage
collapsed. Mailhot fell into a messy affair with her writing teacher and
a second marriage. (``I wasn't stable, but men don't usually care about
that.'') It's to him that the book is addressed. ``Casey, I want to be
polite and present myself as decent,'' she writes early on before,
happily, jettisoning that ambition.

Image

Terese Marie MailhotCredit...Isaiah Mailhot

``Heart Berries'' has a mixture of vulnerability and rage, sexual
yearning and artistic ambition, swagger and self-mockery that recalls
Chris Kraus's ``I Love Dick.'' Mailhot writes of a friend: ``She thinks
my husband doesn't understand how to communicate love, and I think he's
impotent.'' Later, of Casey: ``I wanted to know what I looked like to
you. A sin committed and a prayer answered, you said. You looked like a
hamburger fried in a donut. You were hairy and large.''

She is unsparing to everyone, especially herself. She makes some
horrifying admissions: ``Isaiah cried all night, and I remembered well
that I held a hand over his mouth, long enough for me to know I am a
horror to my baby.'' She describes blinding rages in which she gave her
husband a black eye, broke every glass in the house and some of the
windows.

In a trice she can shift registers, though, and her candor and keenness
of eye translate surprisingly well to tenderness. ``I wondered if maybe
falling in love looked like a crisis to an observer,'' she writes. ``I
found myself caressing my own face.'' In every revelation --- of joy or
suffering --- there is an unmistakable note of triumph. ``I was the
third generation of the things we didn't talk about,'' she notes; there
is exhilaration in shattering this pattern.

``We function the way ghosts function in ghost stories,'' the Ojibwe
writer David Treuer said in 2006 about how indigenous people are
represented. ``We sort of hover around to admonish people about what
they should be doing, what they're doing wrong, how they're destroying
nature. We're always there, but chained to our own deaths, not really
alive and active and engaged.''

A decade later, however, and we have a wave of acclaimed young
indigenous writers --- Natalie Diaz, Layli Long Soldier, Tommy Pico,
Elissa Washuta, Tommy Orange --- all of them vigorously engaged in
projects that carve out new spaces for self-definition, and creating
characters that do the same. (``We have some strength in numbers now,''
Diaz has said. ``We're actually considered for our work that isn't meant
for natives.'') It might be just the beginning. Sherman Alexie has
projected that the
\href{https://www.nytimes3xbfgragh.onion/2016/11/02/us/north-dakota-oil-pipeline-battle-whos-fighting-and-why.html}{Dakota
Access Pipeline protests}, which brought together members of 300 tribes
in America, are going to catalyze the next generation of indigenous
artists and activists.

If ``Heart Berries'' is any indication, the work to come will not just
surface suppressed stories; it might give birth to new forms. ``The
writers before me seemed to do the work of looking \emph{at} being
indigenous so we could look through it,'' Mailhot writes. Her
experiments with structure and language (some more successful than
others; she can be fatally attracted to a faux lyricism) are in the
service of trying to find new ways to think about the past, trauma,
repetition and reconciliation, which might be a way of saying a new
model for the memoir. ``In white culture, forgiveness is synonymous with
letting go. In my culture, I believe we carry pain until we can
reconcile with it through ceremony. Pain is not framed like a problem
with a solution.'' And heart berries, she tells us, referred to
strawberries in the language; as the lore goes, the first medicine man
learned how to put them to use.

So much of what Mailhot is moving toward here still feels nascent ---
the book wants a tighter weave, more focus. But give me narrative power
and ambition over tidiness any day. ``I wanted as much of the world as I
could take,'' Mailhot writes. ``And I didn't have the conscience to be
ashamed.''

Advertisement

\protect\hyperlink{after-bottom}{Continue reading the main story}

\hypertarget{site-index}{%
\subsection{Site Index}\label{site-index}}

\hypertarget{site-information-navigation}{%
\subsection{Site Information
Navigation}\label{site-information-navigation}}

\begin{itemize}
\tightlist
\item
  \href{https://help.nytimes3xbfgragh.onion/hc/en-us/articles/115014792127-Copyright-notice}{©~2020~The
  New York Times Company}
\end{itemize}

\begin{itemize}
\tightlist
\item
  \href{https://www.nytco.com/}{NYTCo}
\item
  \href{https://help.nytimes3xbfgragh.onion/hc/en-us/articles/115015385887-Contact-Us}{Contact
  Us}
\item
  \href{https://www.nytco.com/careers/}{Work with us}
\item
  \href{https://nytmediakit.com/}{Advertise}
\item
  \href{http://www.tbrandstudio.com/}{T Brand Studio}
\item
  \href{https://www.nytimes3xbfgragh.onion/privacy/cookie-policy\#how-do-i-manage-trackers}{Your
  Ad Choices}
\item
  \href{https://www.nytimes3xbfgragh.onion/privacy}{Privacy}
\item
  \href{https://help.nytimes3xbfgragh.onion/hc/en-us/articles/115014893428-Terms-of-service}{Terms
  of Service}
\item
  \href{https://help.nytimes3xbfgragh.onion/hc/en-us/articles/115014893968-Terms-of-sale}{Terms
  of Sale}
\item
  \href{https://spiderbites.nytimes3xbfgragh.onion}{Site Map}
\item
  \href{https://help.nytimes3xbfgragh.onion/hc/en-us}{Help}
\item
  \href{https://www.nytimes3xbfgragh.onion/subscription?campaignId=37WXW}{Subscriptions}
\end{itemize}
