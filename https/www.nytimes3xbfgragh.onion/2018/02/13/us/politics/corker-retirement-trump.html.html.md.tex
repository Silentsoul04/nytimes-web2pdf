Sections

SEARCH

\protect\hyperlink{site-content}{Skip to
content}\protect\hyperlink{site-index}{Skip to site index}

\href{https://www.nytimes3xbfgragh.onion/section/politics}{Politics}

\href{https://myaccount.nytimes3xbfgragh.onion/auth/login?response_type=cookie\&client_id=vi}{}

\href{https://www.nytimes3xbfgragh.onion/section/todayspaper}{Today's
Paper}

\href{/section/politics}{Politics}\textbar{}Corker Reconsiders
Retirement, but He Must Win Over Trump to Do It

\url{https://nyti.ms/2BWI1Gm}

\begin{itemize}
\item
\item
\item
\item
\item
\end{itemize}

Advertisement

\protect\hyperlink{after-top}{Continue reading the main story}

Supported by

\protect\hyperlink{after-sponsor}{Continue reading the main story}

\hypertarget{corker-reconsiders-retirement-but-he-must-win-over-trump-to-do-it}{%
\section{Corker Reconsiders Retirement, but He Must Win Over Trump to Do
It}\label{corker-reconsiders-retirement-but-he-must-win-over-trump-to-do-it}}

\includegraphics{https://static01.graylady3jvrrxbe.onion/images/2018/02/14/us/politics/14dc-corker/merlin_133783728_1c8faff3-bdff-45c7-92fd-89b3368f8de1-articleLarge.jpg?quality=75\&auto=webp\&disable=upscale}

By \href{http://www.nytimes3xbfgragh.onion/by/jonathan-martin}{Jonathan
Martin}

\begin{itemize}
\item
  Feb. 13, 2018
\item
  \begin{itemize}
  \item
  \item
  \item
  \item
  \item
  \end{itemize}
\end{itemize}

WASHINGTON --- Senator Bob Corker of Tennessee has been reconsidering
his decision to retire this year, but Mr. Corker's hopes for retaining
his seat are running into a potentially insurmountable object: President
Trump.

Just over four months after Mr. Corker, upon
\href{https://www.nytimes3xbfgragh.onion/2017/09/26/us/politics/tennessees-bob-corker-announces-retirement-from-senate.html}{declaring
he would retire}, unleashed a biting series of attacks on Mr. Trump, the
president is refusing to bless
\href{https://www.nytimes3xbfgragh.onion/2017/10/24/us/politics/trump-corker-feud-dog-catcher.html}{his
friend-turned-foe's effort} to re-enter the Republican primary race.

Instead, in a telephone conversation last week, Mr. Trump offered
encouragement to Representative Marsha Blackburn, a conservative
lawmaker and White House ally who has emerged as the favorite to win the
Republican nomination for Mr. Corker's seat, according to three
Republicans familiar with the call.

Mr. Trump's West Wing advisers,
\href{https://www.nytimes3xbfgragh.onion/2017/10/08/us/politics/trump-corker.html}{their
memories still fresh from Mr. Corker's jibes}, are urging the president
to resist entreaties from the senator and a handful of his colleagues
who worry that the seat could slip from Republican hands in November.
They are showing Mr. Trump polling that indicates how steep of a climb
Mr. Corker would face in a primary campaign.

If Mr. Trump does not change course, his silence could effectively doom
any hopes Mr. Corker has for seeking a third term. The president is
highly popular among Tennessee Republicans and may be the only person
who could reverse Mr. Corker's standing with primary voters, who have
soured on him since he portrayed Mr. Trump as a juvenile in need of day
care, whose instability may push the country into World War III.

And without Mr. Trump's direct intervention, Ms. Blackburn is highly
unlikely to bow to Mr. Corker.

``Marsha Blackburn is not getting out of this race regardless of who
gets in,'' said Ward Baker, the Nashville-area lawmaker's chief
strategist.

Further, Senator Mitch McConnell of Kentucky, the majority leader, has
rebuffed Mr. Corker by telling him that he must secure the president's
support to re-enter the race, according to Republicans familiar with the
conversation, a rare act of political deference that suggests he is
uneasy about driving Ms. Blackburn out of the primary race.

But Mr. Corker and some of his Senate allies are aggressively working to
win over the White House, embarking on what one West Wing official
described as a sudden charm offensive. The senator has avoided any
criticism of Mr. Trump in recent weeks and on Monday met with Ivanka
Trump, the president's daughter.

Mr. Trump's political advisers, getting wind of the meeting, scrambled
to brief Ms. Trump and her own staff about Mr. Corker's renewed interest
in running again and his desire for the president's support, according
to a Republican official. An aide to Mr. Corker said Ms. Trump requested
the meeting.

More broadly, Mr. Corker's advisers say he and Mr. Trump have patched up
their relationship and the senator is simply hearing out those who would
like him to remain in the Senate, a decision he technically does not
have to make until Tennessee's filing deadline in early April.

``In recent days, people across Tennessee have reached out to Senator
Corker with concerns about the outcome of this election because they
believe it could determine control of the Senate and the future of our
agenda,'' said Micah Johnson, Mr. Corker's spokeswoman.

Mr. Corker is not the only Republican who could shake up a closely
watched Senate race. Representative Kevin Cramer of North Dakota, who
had been heavily courted by Mr. Trump and other leading Republicans but
decided against a bid, is now signaling that he may mount a bid after
all against Senator Heidi Heitkamp, North Dakota's freshman Democrat.

\includegraphics{https://static01.graylady3jvrrxbe.onion/images/2018/02/14/us/politics/14dc-corker2/merlin_110147333_08293267-9663-4835-921f-44d5e352fb51-articleLarge.jpg?quality=75\&auto=webp\&disable=upscale}

If Mr. Cramer's potential change of heart is lifting Republicans'
spirits, the party is now uneasy about the prospect of having one too
many Republicans in Tennessee.

Mr. Corker's hopes for a rapprochement with Mr. Trump illustrate the
degree to which loyalty to the president is becoming a central litmus
test in Republican politics. And that he might consider a reconciliation
with a senator who only a few months ago portrayed him as an unruly
toddler underscores that there are no permanent friends or enemies to
Mr. Trump.

But party officials overseeing the midterms have little appetite for a
bloody primary campaign in Tennessee.

Democrats have cleared their field for Phil Bredesen, a wealthy former
two-term governor and mayor of Nashville whose prospects in an
increasingly conservative state would markedly improve if he faced a
Republican limping into a general election.

Further, in a season full of disappointments across the Senate
Republican map, Ms. Blackburn has been a bright spot, raising \$2
million in the final quarter of last year, contributing to other
candidates and uniting most of the party's factions.

``She is raising more money and doing more than any other candidate in
country,'' said Josh Holmes, a Republican consultant and adviser to Mr.
McConnell.

Ms. Blackburn, who was already facing former Representative Stephen
Fincher in the primary, is signaling that if she is forced to take on
Mr. Corker she will wield her gender as a political weapon. That would
raise a delicate issue for a Senate Republican caucus that includes just
five women in its ranks.

``Anyone who thinks Marsha Blackburn can't win a general election is
just a plain sexist pig,'' said Andrea Bozek, a spokeswoman for Ms.
Blackburn, adding, ``We aren't worried about these ego-driven, tired old
men.''

That is an unmistakable reference to a bloc of establishment-aligned
Tennessee Republicans who believe Ms. Blackburn's conservative politics
and hard-charging style may turn off some voters and make her vulnerable
against the centrist and low-key Mr. Bredesen.

Tom Ingram, a Republican strategist and a longtime friend of Mr.
Corker's, argued that Republicans could imperil the seat and their
one-seat majority if they do not rally around an incumbent who has
already won the state twice.

``It's a dicey race with an unproven statewide candidate against Phil
Bredesen,'' Mr. Ingram said. ``If Corker is the nominee, it's not in
play.''

Fueling the anxieties of some Tennessee Republicans is the openness some
in the party's donor wing have toward backing Mr. Bredesen. Concerns
about such defections cropped up again this week when an invitation to a
\$5,400-per-couple Bredesen fund-raiser hosted by the widow of the
former Nashville Republican powerhouse Ted Welch began circulating among
Republicans.

Such talk induces eye-rolling among a younger generation of Republicans
in Tennessee.

``Every Republican in Tennessee outside the walls of the Belle Meade
Country Club believes Marsha is a heavy favorite against anybody who
might get in that primary,'' said Brad Todd, a Republican consultant
raised in the state who is not aligned in the Senate race.

Advertisement

\protect\hyperlink{after-bottom}{Continue reading the main story}

\hypertarget{site-index}{%
\subsection{Site Index}\label{site-index}}

\hypertarget{site-information-navigation}{%
\subsection{Site Information
Navigation}\label{site-information-navigation}}

\begin{itemize}
\tightlist
\item
  \href{https://help.nytimes3xbfgragh.onion/hc/en-us/articles/115014792127-Copyright-notice}{©~2020~The
  New York Times Company}
\end{itemize}

\begin{itemize}
\tightlist
\item
  \href{https://www.nytco.com/}{NYTCo}
\item
  \href{https://help.nytimes3xbfgragh.onion/hc/en-us/articles/115015385887-Contact-Us}{Contact
  Us}
\item
  \href{https://www.nytco.com/careers/}{Work with us}
\item
  \href{https://nytmediakit.com/}{Advertise}
\item
  \href{http://www.tbrandstudio.com/}{T Brand Studio}
\item
  \href{https://www.nytimes3xbfgragh.onion/privacy/cookie-policy\#how-do-i-manage-trackers}{Your
  Ad Choices}
\item
  \href{https://www.nytimes3xbfgragh.onion/privacy}{Privacy}
\item
  \href{https://help.nytimes3xbfgragh.onion/hc/en-us/articles/115014893428-Terms-of-service}{Terms
  of Service}
\item
  \href{https://help.nytimes3xbfgragh.onion/hc/en-us/articles/115014893968-Terms-of-sale}{Terms
  of Sale}
\item
  \href{https://spiderbites.nytimes3xbfgragh.onion}{Site Map}
\item
  \href{https://help.nytimes3xbfgragh.onion/hc/en-us}{Help}
\item
  \href{https://www.nytimes3xbfgragh.onion/subscription?campaignId=37WXW}{Subscriptions}
\end{itemize}
