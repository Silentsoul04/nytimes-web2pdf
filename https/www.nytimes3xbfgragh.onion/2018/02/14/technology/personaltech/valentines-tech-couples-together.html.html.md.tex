Sections

SEARCH

\protect\hyperlink{site-content}{Skip to
content}\protect\hyperlink{site-index}{Skip to site index}

\href{https://www.nytimes3xbfgragh.onion/section/technology/personaltech}{Personal
Tech}

\href{https://myaccount.nytimes3xbfgragh.onion/auth/login?response_type=cookie\&client_id=vi}{}

\href{https://www.nytimes3xbfgragh.onion/section/todayspaper}{Today's
Paper}

\href{/section/technology/personaltech}{Personal Tech}\textbar{}This
Valentine's Day, Considering Tech That Keeps Couples Together

\url{https://nyti.ms/2Bus2y3}

\begin{itemize}
\item
\item
\item
\item
\item
\end{itemize}

Advertisement

\protect\hyperlink{after-top}{Continue reading the main story}

Supported by

\protect\hyperlink{after-sponsor}{Continue reading the main story}

\href{/column/tech-we-are-using}{Tech We're Using}

\hypertarget{this-valentines-day-considering-tech-that-keeps-couples-together}{%
\section{This Valentine's Day, Considering Tech That Keeps Couples
Together}\label{this-valentines-day-considering-tech-that-keeps-couples-together}}

\includegraphics{https://static01.graylady3jvrrxbe.onion/images/2018/02/15/business/15TECHUSING-1/15TECHUSING-1-articleLarge.jpg?quality=75\&auto=webp\&disable=upscale}

By The New York Times

\begin{itemize}
\item
  Feb. 14, 2018
\item
  \begin{itemize}
  \item
  \item
  \item
  \item
  \item
  \end{itemize}
\end{itemize}

\emph{How do New York Times journalists use technology in their jobs and
in their personal lives? For Valentine's Day, Daniel Jones, who edits}
\href{https://www.nytimes3xbfgragh.onion/column/modern-love}{\emph{the
Modern Love column}} \emph{for The Times, discussed the tech he's
using.}

\textbf{As the editor of Modern Love, you are constantly reading stories
about people's relationships, including submissions that never make it
to publication. With that bird's-eye view, how do you think tech has
transformed relationships?}

In looking for love, tech gives us access, protection and the power of
curation. Meaning we can shop for love from the safety of our homes
while presenting only the best parts of ourselves to potential mates. So
much control! How could anything go wrong?

But actually, as with so much of tech, it's mostly the illusion of
control. Because we can't really choose anyone even if it seems that we
can. We can't protect our hearts with the glass of our phones. And we
can't keep a potential mate from discovering who we truly are; we just
increase the odds of their ultimately being disappointed by keeping them
hooked to our ``best'' self.

\textbf{How has tech}
\href{https://www.nytimes3xbfgragh.onion/2017/09/21/well/family/the-love-lives-of-digital-natives.html}{\textbf{changed
dating}}\textbf{? Is online dating making romance better or worse?}

I think we view technology in all areas as being a shortcut. We can
solve problems faster, maximize time, use less energy while doing more.
We can work smarter. And we can love smarter, too, right?
\href{https://www.nytimes3xbfgragh.onion/2017/12/21/style/instagram-thirst-traps-dating-breakups.html?rref=collection\%2Ftimestopic\%2FDating\%20and\%20Courtship\&action=click\&contentCollection=timestopics\&region=stream\&module=stream_unit\&version=latest\&contentPlacement=2\&pgtype=collection}{With
online dating}, we can find the perfect person more efficiently, make
contact more easily, rule that person in or out faster.

But that whole approach is fueled by an expectation that our search for
love is like finding a needle in a haystack. And from what I've seen,
love doesn't often work that way. Having too many choices is in itself
inefficient and fantasy inducing.

\includegraphics{https://static01.graylady3jvrrxbe.onion/images/2018/02/15/business/15TECHUSING-2/15TECHUSING-2-articleLarge.jpg?quality=75\&auto=webp\&disable=upscale}

In my view, love is best found not in a haystack but in a pea pod. Here
are your five peas. Or maybe nine peas. The anthropologist Helen Fisher
claims that a human brain can reasonably consider only nine or 10
choices; beyond that, they turn to noise. So if you were to limit
yourself to nine people and get to know them, give each a chance, you'd
probably fall in love with one.

Even so, I'm not one of those people who moan about how dating apps make
things worse. It's just different, much the way online shopping differs
from brick-and-mortar.

When shopping online, you buy more stuff because it's so easy. Then you
probably fantasize a little more about your new shoes or flying drone as
you await their arrival, which means you're going to be disappointed
more often because the shoes don't fit or the drone is poorly made. So
you return them and start over --- same as with the average online date.
In the store, meanwhile, all of that trying on and rejecting would have
happened in real time, much like getting to know and rejecting people at
the bar or a party happens in real time, leaving you little time to
fantasize.

What's particularly misleading about online dating is how everyone on
the site is trying to seduce you. They're trying to seduce everyone.
They're all saying, ``Date me!''

When you
\href{http://www.nytimes3xbfgragh.onion/2012/08/22/technology/the-new-high-tech-dating-technology-meet-in-a-bar.html}{walk
into a bar}, does everyone rush up to you with a big smile and a glowing
résumé? No. Everyone ignores you. But online, a little piece of you
believes. Because you're being pitched, you allow yourself to believe in
these people's vulnerability and desire. Which is why meeting people
online tends to heighten our fantasies and deepen our disappointment.

\textbf{Our smartphones now include photos, chat transcripts, contacts
and more. Is it good for relationships to have so much of our lives
public?}

What I've noticed about young people who've grown up with social media
is the degree to which they bring a sense of audience to nearly
everything they do. They see their own life as a performance, one that
is constantly ``reviewed'' by friends, family, aunts, cousins,
strangers.

But what's refreshing to me is how goofy a lot of it is. It's curated
but not always to show perfection; in fact, imperfection seems to be the
goal, something to laugh at, an ugly or embarrassing moment or
expression or bad experience. They splash it all out there as a way of
saying, ``This is me.'' I think a lot of us who didn't grow up with that
are horrified about having unflattering pictures of ourselves out there
or embarrassing stories. We see that kind of online vulnerability as
risky, and they often don't --- they celebrate it.

\textbf{How has tech affected your job editing Modern Love?}

Tech --- which for me mostly consists of my laptop and phone --- has
made me both happier and more stressed out, more engaged with the world
and more isolated from my immediate surroundings, more in touch with my
friends and family while I see them less and less. In my job, tech
allows me incredible freedom --- to do my work at The Times or from the
beach, and whether I'm feeling good or am sick in bed. I'm writing this
now on a train from New York City to Massachusetts on a brisk, beautiful
Saturday in February. It's glorious, and at the same time my work and
personal life have almost no boundary.

\textbf{What tech are you obsessed with in your personal life?}

I could not navigate the world without Google Maps, and I could not
fully appreciate the world without Google Earth. I still don't get how
either works. They are miraculous.

\textbf{Recently there has been a lot of talk about}
\href{https://www.nytimes3xbfgragh.onion/2018/01/08/technology/apple-tech-children-jana-calstrs.html}{\textbf{smartphone
addiction}}\textbf{. What's your advice for couples who struggle to put
down their phones and want to remain present with each other?}

I'm a bad person to ask about that because I'm on my phone all the time,
too. I'm just glad these things didn't exist when I had small children,
because I know I'd be like those parents you see at the playground,
staring at their phones as their children try to get their attention.

My immediate family is scattered --- kids in college, my wife and I
frequently in different cities --- and these days that texting
connection outweighs the negative of being on our phones when we're
together. But just barely. With phone addiction, it's all about
willpower. These days I'm without my phone only in the shower.

So I guess my parting advice to couples who want to remain present with
each other: Spend more time together in the shower.

Advertisement

\protect\hyperlink{after-bottom}{Continue reading the main story}

\hypertarget{site-index}{%
\subsection{Site Index}\label{site-index}}

\hypertarget{site-information-navigation}{%
\subsection{Site Information
Navigation}\label{site-information-navigation}}

\begin{itemize}
\tightlist
\item
  \href{https://help.nytimes3xbfgragh.onion/hc/en-us/articles/115014792127-Copyright-notice}{©~2020~The
  New York Times Company}
\end{itemize}

\begin{itemize}
\tightlist
\item
  \href{https://www.nytco.com/}{NYTCo}
\item
  \href{https://help.nytimes3xbfgragh.onion/hc/en-us/articles/115015385887-Contact-Us}{Contact
  Us}
\item
  \href{https://www.nytco.com/careers/}{Work with us}
\item
  \href{https://nytmediakit.com/}{Advertise}
\item
  \href{http://www.tbrandstudio.com/}{T Brand Studio}
\item
  \href{https://www.nytimes3xbfgragh.onion/privacy/cookie-policy\#how-do-i-manage-trackers}{Your
  Ad Choices}
\item
  \href{https://www.nytimes3xbfgragh.onion/privacy}{Privacy}
\item
  \href{https://help.nytimes3xbfgragh.onion/hc/en-us/articles/115014893428-Terms-of-service}{Terms
  of Service}
\item
  \href{https://help.nytimes3xbfgragh.onion/hc/en-us/articles/115014893968-Terms-of-sale}{Terms
  of Sale}
\item
  \href{https://spiderbites.nytimes3xbfgragh.onion}{Site Map}
\item
  \href{https://help.nytimes3xbfgragh.onion/hc/en-us}{Help}
\item
  \href{https://www.nytimes3xbfgragh.onion/subscription?campaignId=37WXW}{Subscriptions}
\end{itemize}
