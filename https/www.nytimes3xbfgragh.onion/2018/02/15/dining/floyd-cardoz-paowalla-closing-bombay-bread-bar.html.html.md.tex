Sections

SEARCH

\protect\hyperlink{site-content}{Skip to
content}\protect\hyperlink{site-index}{Skip to site index}

\href{https://www.nytimes3xbfgragh.onion/section/food}{Food}

\href{https://myaccount.nytimes3xbfgragh.onion/auth/login?response_type=cookie\&client_id=vi}{}

\href{https://www.nytimes3xbfgragh.onion/section/todayspaper}{Today's
Paper}

\href{/section/food}{Food}\textbar{}Floyd Cardoz Pivots Again, for an
Elusive Ingredient: Fun

\url{https://nyti.ms/2BushsR}

\begin{itemize}
\item
\item
\item
\item
\item
\item
\end{itemize}

Advertisement

\protect\hyperlink{after-top}{Continue reading the main story}

Supported by

\protect\hyperlink{after-sponsor}{Continue reading the main story}

\hypertarget{floyd-cardoz-pivots-again-for-an-elusive-ingredient-fun}{%
\section{Floyd Cardoz Pivots Again, for an Elusive Ingredient:
Fun}\label{floyd-cardoz-pivots-again-for-an-elusive-ingredient-fun}}

\includegraphics{https://static01.graylady3jvrrxbe.onion/images/2018/02/21/dining/21Cardoz/merlin_133655340_88e46ade-864e-43c5-b65c-4fe608eeb8da-articleLarge.jpg?quality=75\&auto=webp\&disable=upscale}

By Priya Krishna

\begin{itemize}
\item
  Feb. 15, 2018
\item
  \begin{itemize}
  \item
  \item
  \item
  \item
  \item
  \item
  \end{itemize}
\end{itemize}

There's one memory the chef Floyd Cardoz keeps coming back to: He's a
17-year-old student at St. Andrew's High School in Bombay, it's exam
week and late one night, to escape the stress and the studying, he has
biked to a nearby food cart.

He spots a street vendor stuffing pockets of naan with peppery lamb,
spiced potatoes, fresh chiles and lime juice, then toasting them on a
tawa, or griddle. He leans against his bike and devours the messy
sandwich, savoring the deeply comforting blend of textures and spices.

That's how Mr. Cardoz envisioned the experience at his Indian
restaurant,
\href{https://www.nytimes3xbfgragh.onion/2016/11/09/dining/paowalla-review.html}{Paowalla},
which opened in SoHo in 2016: accessible, craveable, nostalgic and meant
for casual grazing. But the restaurant he actually opened turned out
more rigid and formal, and not popular enough.

``We were only doing 150 covers on Friday and Saturday nights,'' he
said. ``I wanted a line out the door.''

So he's making a course correction: Paowalla will close after service on
Saturday and soon reopen in the same location as the Bombay Bread Bar.
It will be, he says, a more colorful, stripped-down and affordable
place, where Mr. Cardoz is hoping he can do for Indian cuisine in New
York what restaurants like
\href{https://www.nytimes3xbfgragh.onion/2017/11/28/dining/ugly-baby-review-thai-brooklyn.html}{Ugly
Baby} and
\href{http://www.nytimes3xbfgragh.onion/2013/07/03/dining/reviews/restaurant-review-uncle-boons-in-nolita.html}{Uncle
Boons} have done for Thai food --- present authentic, regional cuisine
in a way that New Yorkers viscerally respond to.

For Mr. Cardoz, 57, who gained renown as the chef of Tabla --- the
pioneering Indian restaurant from Danny Meyer's
\href{https://www.ushgnyc.com/}{Union Square Hospitality Group} that was
open from 1998 to 2010 --- Paowalla represented a long-awaited return to
Indian food, after intervening years at the seafood-focused
\href{http://www.nytimes3xbfgragh.onion/2012/03/21/dining/reviews/north-end-grill-in-battery-park-city.html}{North
End Grill}, and a high-end TriBeCa spot, White Street.

Mr. Cardoz opened Paowalla with the intention of recapturing the
conviviality of Tabla's popular downstairs bread bar, with fluffy
Cheddar-bacon kulchas baked fresh in a wood-burning oven, sticky-sweet
pork ribs vindaloo and a fiercely spicy Indo-Chinese three-chile
chicken. But that goal proved elusive, he said.

\includegraphics{https://static01.graylady3jvrrxbe.onion/images/2018/02/21/dining/21Cardoz3/21Cardoz3-articleLarge.jpg?quality=75\&auto=webp\&disable=upscale}

Mr. Meyer agreed. ``There was something sexy about the bread bar and the
way it touched all the senses that was missing from Paowalla,'' he said.

In October, Mr. Cardoz was in Mumbai (formerly Bombay) to open
\href{http://opedromumbai.com/}{O Pedro}, a casual Goan restaurant to
accompany his popular cafe there,
\href{http://thebombaycanteen.com/}{the Bombay Canteen}. ``They were
both these informal, experiential places where you could have cocktails
and a bunch of bites and not feel like you had to sit down for an entire
three-course meal,'' he said.

Paowalla desperately needed that lively, unfiltered spirit, he realized.
His investors agreed to put in additional capital to transform it into
the Bombay Bread Bar. The name is an amalgamation of the two restaurants
that have best represented his culinary identity: the Bombay Canteen and
Tabla's Bread Bar.

The new menu is dominated by dishes that are personal to Mr. Cardoz, who
grew up in Bombay and Goa. The breads, and favorites like the vindaloo,
three-chile chicken and Eggs Kejriwal are staying put. But there are new
snacks like onion ring bhajias, or fritters, served with his mother's
cinnamon and clove-spiked ketchup; and bhel puri, a tangy, crunchy,
chutney-coated puffed rice snack typical of Bombay street vendors.

A short list of larger dishes includes his grandmother's Goan fish
curry, dressed in coconut and chilies; and short-rib nihari, a
slow-cooked Hyderabadi stew spiced with garam masala.

To match the restaurant's breezier fare, the walls will be bathed in
bright pops of blue and yellow, and the dark wood tables will be covered
with loud floral tablecloths.

The most promising news of all? Mr. Cardoz has decided to re-create that
braised lamb sandwich from his school days. And this time, he won't
modify the recipe just to make New Yorkers comfortable with it.

``People still don't recognize Indian food for what it is,'' he said.
``They're not celebrating it like they should. So I'm willing to take
that chance here.''

\href{https://www.facebookcorewwwi.onion/nytfood/}{\emph{Follow NYT Food
on Facebook}}\emph{,}
\href{https://instagram.com/nytfood}{\emph{Instagram}}\emph{,}
\href{https://twitter.com/nytfood}{\emph{Twitter}} \emph{and}
\href{https://www.pinterest.com/nytfood/}{\emph{Pinterest}}\emph{.}
\href{https://www.nytimes3xbfgragh.onion/newsletters/cooking}{\emph{Get
regular updates from NYT Cooking, with recipe suggestions, cooking tips
and shopping advice}}\emph{.}

Advertisement

\protect\hyperlink{after-bottom}{Continue reading the main story}

\hypertarget{site-index}{%
\subsection{Site Index}\label{site-index}}

\hypertarget{site-information-navigation}{%
\subsection{Site Information
Navigation}\label{site-information-navigation}}

\begin{itemize}
\tightlist
\item
  \href{https://help.nytimes3xbfgragh.onion/hc/en-us/articles/115014792127-Copyright-notice}{©~2020~The
  New York Times Company}
\end{itemize}

\begin{itemize}
\tightlist
\item
  \href{https://www.nytco.com/}{NYTCo}
\item
  \href{https://help.nytimes3xbfgragh.onion/hc/en-us/articles/115015385887-Contact-Us}{Contact
  Us}
\item
  \href{https://www.nytco.com/careers/}{Work with us}
\item
  \href{https://nytmediakit.com/}{Advertise}
\item
  \href{http://www.tbrandstudio.com/}{T Brand Studio}
\item
  \href{https://www.nytimes3xbfgragh.onion/privacy/cookie-policy\#how-do-i-manage-trackers}{Your
  Ad Choices}
\item
  \href{https://www.nytimes3xbfgragh.onion/privacy}{Privacy}
\item
  \href{https://help.nytimes3xbfgragh.onion/hc/en-us/articles/115014893428-Terms-of-service}{Terms
  of Service}
\item
  \href{https://help.nytimes3xbfgragh.onion/hc/en-us/articles/115014893968-Terms-of-sale}{Terms
  of Sale}
\item
  \href{https://spiderbites.nytimes3xbfgragh.onion}{Site Map}
\item
  \href{https://help.nytimes3xbfgragh.onion/hc/en-us}{Help}
\item
  \href{https://www.nytimes3xbfgragh.onion/subscription?campaignId=37WXW}{Subscriptions}
\end{itemize}
