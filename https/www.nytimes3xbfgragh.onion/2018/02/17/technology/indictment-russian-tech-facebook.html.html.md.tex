Sections

SEARCH

\protect\hyperlink{site-content}{Skip to
content}\protect\hyperlink{site-index}{Skip to site index}

\href{https://www.nytimes3xbfgragh.onion/section/technology}{Technology}

\href{https://myaccount.nytimes3xbfgragh.onion/auth/login?response_type=cookie\&client_id=vi}{}

\href{https://www.nytimes3xbfgragh.onion/section/todayspaper}{Today's
Paper}

\href{/section/technology}{Technology}\textbar{}To Stir Discord in 2016,
Russians Turned Most Often to Facebook

\href{https://nyti.ms/2BBBbET}{https://nyti.ms/2BBBbET}

\begin{itemize}
\item
\item
\item
\item
\item
\item
\end{itemize}

Advertisement

\protect\hyperlink{after-top}{Continue reading the main story}

Supported by

\protect\hyperlink{after-sponsor}{Continue reading the main story}

\hypertarget{to-stir-discord-in-2016-russians-turned-most-often-to-facebook}{%
\section{To Stir Discord in 2016, Russians Turned Most Often to
Facebook}\label{to-stir-discord-in-2016-russians-turned-most-often-to-facebook}}

\includegraphics{https://static01.graylady3jvrrxbe.onion/images/2018/02/18/business/18FACEBOOK1/merlin_127567715_777d954c-1ff4-4fb0-82ef-73cd1532acf9-articleLarge.jpg?quality=75\&auto=webp\&disable=upscale}

By \href{https://www.nytimes3xbfgragh.onion/by/sheera-frenkel}{Sheera
Frenkel} and
\href{http://www.nytimes3xbfgragh.onion/by/katie-benner}{Katie Benner}

\begin{itemize}
\item
  Feb. 17, 2018
\item
  \begin{itemize}
  \item
  \item
  \item
  \item
  \item
  \item
  \end{itemize}
\end{itemize}

SAN FRANCISCO --- In 2014, Russians working for a shadowy firm called
the Internet Research Agency started gathering American followers in
online groups focused on issues like religion and immigration. Around
mid-2015, the Russians began buying digital ads to spread their
messages. A year later, they tapped their followers to help organize
political rallies across the United States.

Their digital instrument of choice for all of these actions? Facebook
and its
\href{https://www.nytimes3xbfgragh.onion/2017/12/17/technology/instagram-russian-trolls.html}{photo-sharing
site Instagram}.

The social network, more than any other technology tool, was singled out
on Friday by the Justice Department when prosecutors
\href{https://www.nytimes3xbfgragh.onion/2018/02/16/us/politics/russians-indicted-mueller-election-interference.html?hp\&action=click\&pgtype=Homepage\&clickSource=story-heading\&module=a-lede-package-region\&region=top-news\&WT.nav=top-news}{charged
13 Russians and three companies} for executing a scheme to subvert the
2016 election and support Donald J. Trump's presidential campaign. In a
37-page indictment, officials detailed how the Russians repeatedly
turned to Facebook and Instagram, often using stolen identities to pose
as Americans, to sow discord among the electorate by creating Facebook
groups, distributing divisive ads and posting inflammatory images.

While the indictment does not accuse Facebook of any wrongdoing, it
provided the first comprehensive account from the authorities of how
critical the company's platforms had been to the Russian campaign to
disrupt the 2016 election. Facebook and Instagram were mentioned 41
times, while other technology that the Russians used was featured far
less. Twitter was referred to nine times, YouTube once and the
electronic payments company PayPal 11 times.

It is unprecedented for an American technology company to be so central
to what the authorities say was a foreign scheme to commit election
fraud in the United States. The indictment further batters Facebook's
image after it has spent months grappling with questions about how it
was misused and why it did not act earlier to prevent that activity.

Jonathan Albright, research director at Columbia University's Tow Center
for Digital Journalism, said the indictment laid bare how effectively
Facebook could be turned against the country.

\includegraphics{https://static01.graylady3jvrrxbe.onion/images/2018/02/18/business/18FACEBOOK2/merlin_133997369_a0fea109-ec66-457c-9f8f-07b7bc53a484-articleLarge.jpg?quality=75\&auto=webp\&disable=upscale}

``Facebook built incredibly effective tools which let Russia profile
citizens here in the U.S. and figure out how to manipulate us,'' Mr.
Albright said. ``Facebook, essentially, gave them everything they
needed.'' He added that many of the tools that the Russians used,
including those that allow ads to be targeted and that show how
widespread an ad becomes, still pervade Facebook.

Facebook, with more than two billion members on the social network
alone, has long struggled with what its sites show and the kind of
illicit activity it may enable, from
\href{https://www.nytimes3xbfgragh.onion/2016/01/30/technology/facebook-gun-sales-ban.html}{selling
unlicensed guns} to
\href{https://www.nytimes3xbfgragh.onion/2017/04/25/world/asia/thailand-phuket-facebook-killing-daughter.html}{broadcasting
live killings}. The company's business depends on people being highly
engaged with what is posted on its sites, which in turn helps make it a
marquee destination for advertisers.

When suggestions first arose after the 2016 election that Facebook may
have influenced the outcome, Mark Zuckerberg, the company's chief
executive, dismissed the concerns. But by last September,
\href{https://www.nytimes3xbfgragh.onion/2017/09/06/technology/facebook-russian-political-ads.html}{Facebook
had disclosed} that the Internet Research Agency had bought divisive ads
on hot-button issues through the company. It later said 150 million
Americans had seen the Russian propaganda on the social network and
Instagram.

The resulting firestorm has damaged Facebook's reputation. Company
officials, along with executives from Google and YouTube, were
\href{https://www.nytimes3xbfgragh.onion/2017/10/31/us/politics/facebook-twitter-google-hearings-congress.html}{grilled
by lawmakers} last fall. Facebook has since hired thousands of people to
help monitor content and has worked with Robert S. Mueller III, the
special counsel leading the investigation into Russian election
interference. It has also changed its advertising policy so that any ad
that mentions a candidate's name goes through a more stringent vetting
process. Mr. Zuckerberg has vowed to not let Facebook be abused by bad
actors.

Yet Facebook's multiple mentions in Friday's indictment renew questions
of why the world's biggest social media company didn't catch the Russian
activity earlier or do more to stop it. How effective the company's new
efforts to reduce foreign manipulation have been is also unclear.

Rob Goldman, Facebook's vice president of advertising, waded into the
discussion on Friday with a series of tweets that argued that Russia's
goal was to sow chaos among the electorate rather than to force a
certain outcome in the election. On Saturday, President Trump cited
those tweets as evidence that Russia's disinformation campaign was not
aimed at handing him a victory.

In Silicon Valley, where Facebook has its headquarters, some critics
pilloried the company after the indictment became public.

``Mueller's indictment underscores the central role of Facebook and
other platforms in the Russian interference in 2016,'' said Roger
McNamee, a Silicon Valley venture capitalist who had invested early in
Facebook. ``In its heyday, television brought the country together,
giving viewers a shared set of facts and experiences. Facebook does just
the opposite, enabling every user to have a unique set of facts, driving
the country apart for profit.''

Joel Kaplan, Facebook's vice president of global policy, said in a
statement that the company was grateful the government was taking action
``against those who abused our service and exploited the openness of our
democratic process.''

He added that Facebook was working with the Federal Bureau of
Investigation ahead of this year's midterm elections to ensure that a
similar manipulation campaign would not take place. ``We know we have
more to do to prevent against future attacks,'' he said.

Facebook has previously questioned whether law enforcement should be
more involved in helping to stop the threat from nation state actors.
Facebook said it worked closely with the special counsel's
investigation.

YouTube did not respond to a request for comment, while Twitter declined
to comment. PayPal said in a statement that it has worked closely with
law enforcement and ``is intensely focused on combating and preventing
the illicit use of our services.''

According to the indictment, the Internet Research Agency, created in
2014 in St. Petersburg and employing about 80 people, was given the job
of interfering with elections and political processes.

The group began using American social media to achieve those aims in
2014, when it started making Facebook pages dedicated to social issues
like race and religion. Over the next two years, the indictment said,
the Russians stole the identities of real Americans to create fake
personas and fake accounts on social media. The group then used those to
populate and promote Facebook pages like United Muslims of America,
Blacktivist and Secured Borders.

By 2016, the indictment said, the size of some of these
Russian-controlled Facebook groups had ballooned to hundreds of
thousands of followers.

\href{https://www.nytimes3xbfgragh.onion/interactive/2018/02/16/us/politics/russia-propaganda-election-2016.html}{}

\includegraphics{https://static01.graylady3jvrrxbe.onion/images/2018/02/16/us/russia-propaganda-election-2016-promo/russia-propaganda-election-2016-promo-articleLarge.jpg}

\hypertarget{the-propaganda-tools-used-by-russians-to-influence-the-2016-election}{%
\subsection{The Propaganda Tools Used by Russians to Influence the 2016
Election}\label{the-propaganda-tools-used-by-russians-to-influence-the-2016-election}}

Thirteen Russian nationals have been charged with illegally trying to
disrupt the American political process through inflammatory social media
posts and organized political rallies.

Image

Facebook was also used by the Russians to organize political rallies in
the United States.

The Russians then used these groups to push various messages, including
telling Americans not to vote in the 2016 election for either Mr. Trump
or his opponent, Hillary Clinton. In October 2016, according to the
indictment, one Russian-controlled Instagram account called Woke Blacks
posted a message saying: ``Hatred for Trump is misleading the people and
forcing Blacks to vote Killary. We cannot resort to the lesser of two
devils. Then we'd surely be better off without voting AT ALL.''

Around 2015, according to the indictment, the Russians also started
purchasing ads on Facebook and other social media sites like Twitter,
targeting specific communities within the United States. The group used
stolen PayPal accounts to pay for the ads and to promote posts, spending
hundreds of thousands of dollars on the outreach.

In one ad,
\href{https://www.nytimes3xbfgragh.onion/2017/11/01/us/politics/russia-2016-election-facebook.html}{published}
to promote a Facebook event called ``Down with Hillary,'' an image of
Mrs. Clinton was shown with a black ``X'' painted across her face. The
text read, ``Hillary Clinton is the co-author of Obama's anti-police and
anti-Constitutional propaganda.''

By mid-2016, according to the indictment, the Russians were using their
fake Facebook personas to organize political rallies in the United
States. That June, for example, posing as the United Muslims of America
on Facebook, they promoted a rally called ``Support Hillary. Save
American Muslims.'' For an August 2016 event organized through Facebook,
the Russians also paid for a cage to be built that was large enough to
hold an actress depicting Mrs. Clinton in a prison uniform.

At every step, the Russians used Facebook's own tools to make sure their
propaganda was as effective as possible. Those tools allowed them to get
real-time results on which types of ad campaigns were reaching their
target audience or which posts were getting the most engagement with
viewers.

Researchers said that those tools are still widely available and that
while the company has worked to remove fake accounts and stem the flow
of disinformation, it has refused to let outside researchers examine the
data on how Russian actors used the platform so effectively.

``They're taking steps to fix this, but there's no easy solution,''
Anton Vuljaj, a Republican media strategist who has advised campaigns
and media groups, said of Facebook and other social media companies.
``This also shows that the public needs to be more vigilant about what
is real and what is not online.''

Advertisement

\protect\hyperlink{after-bottom}{Continue reading the main story}

\hypertarget{site-index}{%
\subsection{Site Index}\label{site-index}}

\hypertarget{site-information-navigation}{%
\subsection{Site Information
Navigation}\label{site-information-navigation}}

\begin{itemize}
\tightlist
\item
  \href{https://help.nytimes3xbfgragh.onion/hc/en-us/articles/115014792127-Copyright-notice}{©~2020~The
  New York Times Company}
\end{itemize}

\begin{itemize}
\tightlist
\item
  \href{https://www.nytco.com/}{NYTCo}
\item
  \href{https://help.nytimes3xbfgragh.onion/hc/en-us/articles/115015385887-Contact-Us}{Contact
  Us}
\item
  \href{https://www.nytco.com/careers/}{Work with us}
\item
  \href{https://nytmediakit.com/}{Advertise}
\item
  \href{http://www.tbrandstudio.com/}{T Brand Studio}
\item
  \href{https://www.nytimes3xbfgragh.onion/privacy/cookie-policy\#how-do-i-manage-trackers}{Your
  Ad Choices}
\item
  \href{https://www.nytimes3xbfgragh.onion/privacy}{Privacy}
\item
  \href{https://help.nytimes3xbfgragh.onion/hc/en-us/articles/115014893428-Terms-of-service}{Terms
  of Service}
\item
  \href{https://help.nytimes3xbfgragh.onion/hc/en-us/articles/115014893968-Terms-of-sale}{Terms
  of Sale}
\item
  \href{https://spiderbites.nytimes3xbfgragh.onion}{Site Map}
\item
  \href{https://help.nytimes3xbfgragh.onion/hc/en-us}{Help}
\item
  \href{https://www.nytimes3xbfgragh.onion/subscription?campaignId=37WXW}{Subscriptions}
\end{itemize}
