Sections

SEARCH

\protect\hyperlink{site-content}{Skip to
content}\protect\hyperlink{site-index}{Skip to site index}

\href{https://myaccount.nytimes3xbfgragh.onion/auth/login?response_type=cookie\&client_id=vi}{}

\href{https://www.nytimes3xbfgragh.onion/section/todayspaper}{Today's
Paper}

\href{/section/opinion}{Opinion}\textbar{}Back to School, at 64

\url{https://nyti.ms/2GiHu0G}

\begin{itemize}
\item
\item
\item
\item
\item
\item
\end{itemize}

Advertisement

\protect\hyperlink{after-top}{Continue reading the main story}

Supported by

\protect\hyperlink{after-sponsor}{Continue reading the main story}

\href{/section/opinion}{Opinion}

\href{/column/on-campus}{On Campus}

\hypertarget{back-to-school-at-64}{%
\section{Back to School, at 64}\label{back-to-school-at-64}}

By Anne Rudig

\begin{itemize}
\item
  March 12, 2018
\item
  \begin{itemize}
  \item
  \item
  \item
  \item
  \item
  \item
  \end{itemize}
\end{itemize}

\includegraphics{https://static01.graylady3jvrrxbe.onion/images/2018/03/12/opinion/12oncampus/merlin_135201306_142dc61a-0899-4c7f-a5f6-2b245d19be0e-articleLarge.jpg?quality=75\&auto=webp\&disable=upscale}

``Will I be the oldest person here?'' I asked the chairman of the
nonfiction writing program at Columbia University, shortly after I was
accepted. He laughed and said no. There were two or three older, but not
by much.

``It takes a lifetime to make a writer,'' he continued.

And I've had the lifetime.

I was 64 when I entered graduate school. I had just left the work force
--- not retired, just tired. Tired of hitting the glass ceiling and of
policies that failed to protect employees from abuse. As disenchantment
with my job grew, writing became a healthy distraction.

I applied to Columbia with no expectation of acceptance and no idea how
to pay for it. My husband and I had just finished paying for our kids'
education and we didn't have enough for retirement. But then one snowy
February evening, the head of the writing program called as I stirred a
pot of soup, and said, ``You're in, and there will be a scholarship.'' I
dropped my wooden spoon.

Sure, I'd written copy at ad agencies for 20 years, started a memoir and
taken some online classes. But did that make me a writer? And the Ivy
League part scared me. I imagined I wasn't smart enough. But I also
realized I had some things to say.

I sat in workshops and seminars during my first semester at Columbia,
ashamed of my age and surrounded by brilliant young people. They were
polite, but I assumed my work held little interest for them. I don't
swipe for sex. I haven't pulled an all-nighter in decades. I thought
they'd laugh at my flapping upper arms and my wrinkly neck, even though
many of them admire Nora Ephron.

Some of my instructors were close in age to me, winners of Pulitzers
with decades of writing and publishing experience. Some were just a
little older than my children. I wondered, was it too late for me to do
this?

My peers wrote about online dating, strange roommates, odd living
situations, first big loves and rejections. Others wrote memoirs about
life with a meth-addicted father or what it means to be biracial. We all
struggled to be emotionally honest on the page. For me, that meant
finding restraint and perspective by stepping back, seeing my own
complicity in much of what's happened in my life.

My classmates transported me back to my own 20s and 30s. I attended the
University of California, Berkeley, during the second wave of feminism.
In the 1970s, solidarity with the antiwar movement meant cooking for the
male demonstrators. Now things are different. Gender identification is
not assumed. Respect for difference is. I've been asked not only for my
name but also for my preferred pronouns.

Inevitably, I blundered. One icy January morning I dug out the warmest
coat I owned. On my way into class, shivering vegans jeered at me. What
was I thinking? My kids are vegan. The coat, trimmed in rabbit fur at
the neckline, now hangs at the back of my closet where my cat
occasionally tries to kill it or have sex with it.

After several workshops in which everyone felt overexposed, I realized
no one cared how old I was. Workshops are like the painting critiques I
endured as an undergraduate in the art department at Berkeley. But
they're different now --- criticism no longer terrifies me. My
defensiveness has been worn away. I'm better able to hear the helpful
parts and understand what I would have taken personally years ago. I'm
finally a good student. And I don't have to worry that what I write will
make things awkward with my parents at Christmas.

I have fewer distractions than my younger classmates. Dating and the
insecurity that went with it are memories. Figuring out my sexuality and
experimenting with illegal substances no longer consume my weekends.

One professor said, ``You have to write badly before you can write
well.'' Along with my fear of being too old, my first semester held
plenty of writing badly. But I kept going, figuring they'd admitted me
and provided a scholarship for a reason.

Now I'm in my final semester. I'm amazed by my talented peers and by
their easygoing acceptance and candor. Despite the age difference, we've
fed one another in many ways --- emotionally, intellectually and
literally (I've brought tomatoes from my garden, while another student
brought freshly baked bread). Although my professors have been stellar,
much of my education has come from reading and listening to my fellow
students --- on a grandmother in India; an Armenian family's path,
post-genocide; sorority life in 2017; and what it means to be
transgender within an authoritarian regime.

I've been warned by my professors that a degree in writing is unlikely
to bring riches. That's okay. I've been enriched beyond measure. And
like my peers, I've got some stories to tell. A lifetime of them.

Advertisement

\protect\hyperlink{after-bottom}{Continue reading the main story}

\hypertarget{site-index}{%
\subsection{Site Index}\label{site-index}}

\hypertarget{site-information-navigation}{%
\subsection{Site Information
Navigation}\label{site-information-navigation}}

\begin{itemize}
\tightlist
\item
  \href{https://help.nytimes3xbfgragh.onion/hc/en-us/articles/115014792127-Copyright-notice}{©~2020~The
  New York Times Company}
\end{itemize}

\begin{itemize}
\tightlist
\item
  \href{https://www.nytco.com/}{NYTCo}
\item
  \href{https://help.nytimes3xbfgragh.onion/hc/en-us/articles/115015385887-Contact-Us}{Contact
  Us}
\item
  \href{https://www.nytco.com/careers/}{Work with us}
\item
  \href{https://nytmediakit.com/}{Advertise}
\item
  \href{http://www.tbrandstudio.com/}{T Brand Studio}
\item
  \href{https://www.nytimes3xbfgragh.onion/privacy/cookie-policy\#how-do-i-manage-trackers}{Your
  Ad Choices}
\item
  \href{https://www.nytimes3xbfgragh.onion/privacy}{Privacy}
\item
  \href{https://help.nytimes3xbfgragh.onion/hc/en-us/articles/115014893428-Terms-of-service}{Terms
  of Service}
\item
  \href{https://help.nytimes3xbfgragh.onion/hc/en-us/articles/115014893968-Terms-of-sale}{Terms
  of Sale}
\item
  \href{https://spiderbites.nytimes3xbfgragh.onion}{Site Map}
\item
  \href{https://help.nytimes3xbfgragh.onion/hc/en-us}{Help}
\item
  \href{https://www.nytimes3xbfgragh.onion/subscription?campaignId=37WXW}{Subscriptions}
\end{itemize}
