Sections

SEARCH

\protect\hyperlink{site-content}{Skip to
content}\protect\hyperlink{site-index}{Skip to site index}

\href{https://myaccount.nytimes3xbfgragh.onion/auth/login?response_type=cookie\&client_id=vi}{}

\href{https://www.nytimes3xbfgragh.onion/section/todayspaper}{Today's
Paper}

\href{/section/opinion}{Opinion}\textbar{}The Tragedy of Hubert Humphrey

\url{https://nyti.ms/2G6DSxY}

\begin{itemize}
\item
\item
\item
\item
\item
\item
\end{itemize}

Advertisement

\protect\hyperlink{after-top}{Continue reading the main story}

Supported by

\protect\hyperlink{after-sponsor}{Continue reading the main story}

\href{/section/opinion}{Opinion}

\href{/column/vietnam-67}{Vietnam '67}

\hypertarget{the-tragedy-of-hubert-humphrey}{%
\section{The Tragedy of Hubert
Humphrey}\label{the-tragedy-of-hubert-humphrey}}

By Michael Brenes

\begin{itemize}
\item
  March 23, 2018
\item
  \begin{itemize}
  \item
  \item
  \item
  \item
  \item
  \item
  \end{itemize}
\end{itemize}

\includegraphics{https://static01.graylady3jvrrxbe.onion/images/2018/03/24/opinion/24Vietnam-Brenes/24Vietnam-Brenes-articleLarge.jpg?quality=75\&auto=webp\&disable=upscale}

On Feb. 17, 1965, Vice President Hubert Humphrey sent President Lyndon
B. Johnson a memorandum stating the United States must begin an exit
strategy in Vietnam: ``It is always hard to cut losses. But the Johnson
administration is in a stronger position to do so now than any
administration in this century.'' Johnson had trounced Barry Goldwater
in the 1964 election --- and thus, no longer had to prove he was tough
on Communism --- and the conflict had not developed into a full-blown
war. ``Nineteen sixty-five is the year of minimum political risk,''
Humphrey wrote.

Humphrey gave Johnson the opportunity to change the course of history:
By pulling out of Vietnam, he could have avoided opposition from his own
party and seeing his vision for the Great Society jeopardized by a
foreign war and his aspirations for nuclear disarmament between the
Soviet Union and the United States thwarted.

Johnson ignored Humphrey's advice. In fact, he was described as
infuriated with the vice president; the day after receiving the memo,
Johnson told his national security adviser, McGeorge Bundy, that
Humphrey should ``stay out of the peacekeeping and negotiating field''
on Vietnam.

The president went further, and more or less banned him from the Oval
Office for the remainder of 1965. Humphrey lost his responsibilities in
the administration on civil rights --- the subject that elevated him to
the Senate in 1948, when he told the Democrats at their national
convention they needed to ``get out of the shadow of states' rights and
to walk forthrightly into the bright sunshine of human rights.''

Humphrey, who had long been the most prominent and productive liberal in
the Senate --- and the Democrat (other than Johnson) most responsible
for the passage of the 1964 Civil Rights Act, seemingly vanished from
the public eye overnight, In August 1965, the comedian and musician Tom
Lehrer sang to a raucous audience, ``Whatever Became of You, Hubert?'':

\begin{quote}
\emph{Whatever became of you, Hubert?}\\
\emph{We miss you, so tell us, please:}\\
\emph{Are you sad? Are you cross?}\\
\emph{Are you gathering moss}\\
\emph{While you wait for the boss to sneeze?}
\end{quote}

Vietnam destined Humphrey to a miserable four years as Johnson's vice
president. For his dissent against the war (his ``disloyalty''),
Humphrey suffered the brunt of Johnson's unpredictable wrath. Humphrey's
advisers felt Johnson's intimidating, dismissive treatment was the
reason Humphrey reversed his position on Vietnam a year later: why he
defended the war as a necessary fight against Communism that provided
jobs, hope and prosperity to suffering Vietnamese. It was his only way
back into his boss's good graces.

Humphrey's support for the war condemned him in history as a supporting
player in the tragedy of Vietnam. The war alienated Humphrey from
liberals, civil rights activists and young Americans --- the same people
who, for decades, had loved Humphrey for his support of racial justice,
full employment and the labor movement --- and ultimately cost him the
presidency in 1968. Voters thought Humphrey meant continued war, while
Richard Nixon promised ``an honorable end to the war in Vietnam.''

But given what we now know the history of the Vietnam War after 1968,
Hubert Humphrey --- both his life and political career --- deserves
re-examination. Humphrey forces us to consider the history that might
have been: the possibility of ending the Vietnam War before 1973, an
expansion of the Great Society in the 1970s, a different America.
Without Vietnam (and his being Johnson's vice president), Humphrey might
have won in 1968. The country --- and the world --- would be drastically
different.

Hubert Humphrey arrived in the Senate in 1949 as a liberal in an
illiberal institution. Southerners held the reins of power in Congress,
and they hated Humphrey for his opposition to Jim Crow segregation and
``that speech'' at the Democratic National Convention.

While he was determined in his quest for social justice, his legislation
often stalled in committee. He gravitated toward the one man who could
help him: Lyndon Johnson. By 1954, Johnson needed Humphrey too ---
Johnson had become Senate majority leader and wanted liberals to fall
behind his leadership; Johnson concluded Humphrey was the brightest and
most pragmatic of them. It was a devil's bargain: Johnson helped
Humphrey with his relationships with Southerners, and Humphrey vowed to
keep the liberals in line.

The partnership between Johnson and Humphrey was as close as that of two
antagonists could be. When Johnson became president in November 1963,
Humphrey ensured that the Civil Rights Act overcame the Senate
filibuster the following summer. Johnson recognized Humphrey's talents
as a legislator and orator (``There are so many ways I envy you,''
Johnson said in 1951), and chose Humphrey as his vice president in 1964
--- but not before asking Humphrey for his backing (``unswerving
loyalty,'' as Humphrey recalled) on all his decisions. When Mississippi
civil rights activists tried to force the Democratic Party to recognize
them over the state's official, segregationist delegation at the 1964
national convention, it was Humphrey who, on Johnson's orders, made them
back down.

Once in office, Humphrey tried to keep his commitment to Johnson, but on
Vietnam his convictions conflicted with his promises. Humphrey had been
suspicious of American involvement in Vietnam since the mid-1950s, but
became more incredulous of the war's success after meeting with the
veteran intelligence officer Edward Lansdale in 1964, who argued that a
political solution to the war was possible. Humphrey sent several memos
to Johnson in 1964 implying Johnson should pull back on the conflict,
and that he meet with Lansdale. Johnson dismissed each one.

Then, on Feb. 7, 1965, American forces were attacked at Pleiku and nine
Americans were killed. Bundy, the national security adviser, sent
panicked cables to Johnson demanding the United States retaliate. When
Johnson asked Humphrey his thoughts on bombing North Vietnam, Humphrey
responded, ``Mr. President, I don't think we should.'' Johnson ordered
the bombing anyway. Then Humphrey wrote his Feb. 17 memo, and his fate
was sealed for 1965.

But Johnson gave Humphrey one last chance to prove his loyalty, sending
him to South Vietnam in February 1966 (almost one year to the date of
his memo). On that trip, after meeting with Gen. William Westmoreland,
American and Vietnamese soldiers, and South Vietnamese civilians,
Humphrey convinced himself of the truth he wanted to believe: Vietnam
was winnable; it was a war for democracy; it represented a global
mission for peace and prosperity.

Humphrey's adviser Thomas Hughes recalled that Humphrey returned from
Vietnam ``saying things that were crazy'' about the virtues of the war.
In a meeting of the National Security Council in June 1966, Humphrey
said, ``I have come around reluctantly to accepting the wider bombing
program.''

For two years, Humphrey seemed to genuinely believe that Vietnam was a
necessary war, that it represented a fight against global poverty and
Communist tyranny. Humphrey convinced Johnson he believed this, that he
had changed, and was welcomed back into Johnson's good graces. (After
Humphrey encouraged Johnson's staff members to send the president his
speeches supporting the war, Humphrey was admitted to the president's
luncheons on Vietnam.)

But as he promoted the war to the American people (his main task after
1966), Humphrey was increasingly taunted by the antiwar movement. When
Humphrey emerged as the Democratic candidate in 1968 --- after the
assassination of Robert Kennedy and the upheaval at the Democratic
National Convention --- ``Dump the Hump'' became a common motto. Signs
with slogans such as ``Killer of Babies'' and ``Humphrey's Johnson's War
Salesman'' regularly greeted him on the campaign trail.

The protests agonized Humphrey. ``All I had ever been as a liberal
spokesman seemed lost, all that I had accomplished in significant
programs was ignored. I felt robbed of my personal history,'' he
recalled.

On Sept. 30, 1968, Humphrey had enough of Johnson and his war, and in a
speech in Salt Lake City he demanded a halt to the bombing. Humphrey
called Johnson to warn him of the speech hours before. Johnson reacted
coldly: ``I take it you are not asking for my advice. You're going to
give the speech anyway.'' Johnson then shunned Humphrey for the
remainder of 1968 --- indeed, the question remains whether Johnson
favored Richard Nixon over Humphrey in the election, and whether
Johnson's hatred of Humphrey led to his loss.

But what if Humphrey had not been Johnson's vice president --- what if
Humphrey remained in the Senate? What if Eugene McCarthy received the
vice-presidential nomination in 1964 as he wanted? McCarthy would have
become Humphrey: forced to defend America's policy in Vietnam, and
painted as a patsy for Johnson's War. Humphrey would be the skeptic on
Vietnam, and eventual vociferous critic --- but also more palatable to
the party establishment than McCarthy ever was. Divisions within the
party would be united under a Humphrey candidacy in 1968, the wounds
Vietnam opened among ``New Democrats'' healed by a Cold War liberal.

Humphrey could have won in 1968 under these circumstances. Would
Humphrey have faced the same pressure as Nixon to end the war with
``peace through honor?'' Most likely, and certainly during his first
term. But Humphrey would have immediately searched for a political
solution to the war --- for the conflict to end peacefully, and without
further military commitment. Needless to say, he also would have
continued to expand the Great Society, and not begin its long
demolition, as Nixon did.

For these reasons, Humphrey represents the possibilities for a different
history for the United States after 1968, particularly for Democrats
looking today to rebuild their party and understand the mistakes of the
past. Vietnam turned America's leading liberal into a personification of
liberalism's failures. This is the tragedy of Hubert Humphrey and his
Vietnam War --- one that shapes Americans today.

Advertisement

\protect\hyperlink{after-bottom}{Continue reading the main story}

\hypertarget{site-index}{%
\subsection{Site Index}\label{site-index}}

\hypertarget{site-information-navigation}{%
\subsection{Site Information
Navigation}\label{site-information-navigation}}

\begin{itemize}
\tightlist
\item
  \href{https://help.nytimes3xbfgragh.onion/hc/en-us/articles/115014792127-Copyright-notice}{©~2020~The
  New York Times Company}
\end{itemize}

\begin{itemize}
\tightlist
\item
  \href{https://www.nytco.com/}{NYTCo}
\item
  \href{https://help.nytimes3xbfgragh.onion/hc/en-us/articles/115015385887-Contact-Us}{Contact
  Us}
\item
  \href{https://www.nytco.com/careers/}{Work with us}
\item
  \href{https://nytmediakit.com/}{Advertise}
\item
  \href{http://www.tbrandstudio.com/}{T Brand Studio}
\item
  \href{https://www.nytimes3xbfgragh.onion/privacy/cookie-policy\#how-do-i-manage-trackers}{Your
  Ad Choices}
\item
  \href{https://www.nytimes3xbfgragh.onion/privacy}{Privacy}
\item
  \href{https://help.nytimes3xbfgragh.onion/hc/en-us/articles/115014893428-Terms-of-service}{Terms
  of Service}
\item
  \href{https://help.nytimes3xbfgragh.onion/hc/en-us/articles/115014893968-Terms-of-sale}{Terms
  of Sale}
\item
  \href{https://spiderbites.nytimes3xbfgragh.onion}{Site Map}
\item
  \href{https://help.nytimes3xbfgragh.onion/hc/en-us}{Help}
\item
  \href{https://www.nytimes3xbfgragh.onion/subscription?campaignId=37WXW}{Subscriptions}
\end{itemize}
