Sections

SEARCH

\protect\hyperlink{site-content}{Skip to
content}\protect\hyperlink{site-index}{Skip to site index}

\href{https://myaccount.nytimes3xbfgragh.onion/auth/login?response_type=cookie\&client_id=vi}{}

\href{https://www.nytimes3xbfgragh.onion/section/todayspaper}{Today's
Paper}

\href{/section/opinion}{Opinion}\textbar{}A Pale Smoke

\url{https://nyti.ms/2uyBC14}

\begin{itemize}
\item
\item
\item
\item
\item
\item
\end{itemize}

Advertisement

\protect\hyperlink{after-top}{Continue reading the main story}

Supported by

\protect\hyperlink{after-sponsor}{Continue reading the main story}

\href{/section/opinion}{Opinion}

\href{/column/vietnam-67}{Vietnam '67}

\hypertarget{a-pale-smoke}{%
\section{A Pale Smoke}\label{a-pale-smoke}}

By David Gerstel

\begin{itemize}
\item
  March 30, 2018
\item
  \begin{itemize}
  \item
  \item
  \item
  \item
  \item
  \item
  \end{itemize}
\end{itemize}

\includegraphics{https://static01.graylady3jvrrxbe.onion/images/2018/03/31/opinion/31Vietnam-Gerstel/31Vietnam-Gerstel-articleLarge.jpg?quality=75\&auto=webp\&disable=upscale}

\emph{This is the final article in the Vietnam '67 series. To read
previous articles,}
\href{https://www.nytimes3xbfgragh.onion/column/vietnam-67}{\emph{click
here}}\emph{.}

A month ago, I decided to go to my house in Vermont from my home in
Canada. The house is near Island Pond, about 25 miles below the Quebec
border, in the area called the Northeast Kingdom. The region used to be
largely agrarian, scarcely populated, farms with cattle, smallish in
acreage, poor in cash. A land of subsistence farming, a practice left
over from an earlier century.

There are men I know there, damaged men who raise chickens for eggs and
sell them in the town. I regard it as northern Appalachia, land covered
with forest that once was open land back when the king needed ship's
masts and timber and sheep were important. In the beginning there were a
few families, fewer than 20 in the 1830s. It stayed like that until a
few years ago. Now it is second homes, snowmobiling, four-wheelers and
soft ice cream.

After crossing through immigration and customs I normally stop to pick
up groceries at the Derby Village Store, an old-fashioned market with
linoleum-covered wood floors, narrow aisles and seemingly pre-school-age
cashiers. The choice is reasonable and the prices good. If you don't
care for Walmart, Shop `n Save or Shaw's, you go to the Derby Village
Store.

It's a human-size place, where the carts bump into one another and you
have to maneuver around corners with apologies. It does not sell lattes
or 12 kinds of salami. It offers an assortment of inexpensive wine and
beer. It's a Kellogg's and Campbell's store, with a resident butcher and
a polite staff.

I had picked up cheese, eggs, vegetables, chicken tenders and ice cream.
I stopped by the deli counter for sliced meat when I noticed an older
man, some would say very old, with a cap on his head above a checked
wool jacket too light for the season. His pants were out of fashion,
baggy, rolled up and stained with oil, cinched at the waist with a belt
that was too long and hung down one leg.

He was slender and one shoulder was bent down, his face narrow, with
stubble that showed the years and the tears of time. His hands were
clawed with arthritis and spotted, misshapen nails with dirt under them
that could not be scrubbed away. He was a man you would not notice, or
if you did, not wish to touch as he shuffled behind the cart. He wore a
cap that read in proud gold stitching, ``Vietnam Veteran U.S.N.''

It should be said that seeing Vietnam veterans is not unusual in Vermont
or any other state, but the generation is declining in number and
strength.
\href{https://www.nytimes3xbfgragh.onion/2013/03/26/science/how-many-vietnam-veterans-are-still-alive.html}{About
400 die a day} of the 2.7 million who served. It was a long time ago,
and newer conflicts have taken the stage and headlines. Heroes and
graves.

The veterans of Vietnam are opaque coming to transparent, like onion
paper held to light. Soon they will be crippled relics in homes and
memories, photos on a dusty mantel or side table with a yellowed doily.
You have passed them unseen a hundred times. You never knew their
stories, because they are mostly private and you did not want to ask or
know. Beyond the cap or the coat with the same words, they do not share.
They wear these things for themselves and to say, I am here, and I was
there.

We stood by the counter waiting for our slices of pastrami and teriyaki
chicken. The old man carried in his cart milk, ground coffee, a few cans
of stew and soups. I could see that most of the food was ready to eat,
or add water and stir. He likely lived by himself, wife dead or gone.

During the pause of commerce I asked him where he had served. He
responded by telling me simply that he had been in Danang on
helicopters, and on a warship off the coast. It was vague and
intentional. When he said he had piloted a chopper, I thought most
likely a Huey with that iconic whooping sound of the blades biting the
air and announcing its flight path, a sound that you know for life and
run for the tree line when you hear. He might have flown something
larger, a workhorse Chinook, used to carry anything that would fit or be
hung, from troops to supplies, small artillery. It went where it was
needed; you did what was required.

Between the cutting of the meats and the wrapping of the cheese, each of
us looked at the other. Maybe I had fewer scars in the open. Maybe I had
more hair, a down winter coat and better boots, gloves, a wool scarf. It
looked as if he had had a hard time after the war. The Kingdom offered a
hard life, but easier than what he had done. His boots had uneven soles.
His gait was off. He had probably been wounded.

I did not want to ask. Fate had treated me more kindly. But as we used
to say, no one gets out of here alive. It could have been my
imagination, but our eyes reflected back into ourselves. We were
doubled. He did not smile.

He asked if I had been there. He asked because I had spoken first. He
could not tell if I was a flatlander, from out of state, just curious or
a fool. He wanted to know if he had to explain that place and the cap.
His eyes saw the answer before I spoke. He asked if I knew Danang, Nha
Trang, Dong Ha, the Highlands and the Delta. How many words did I use to
reply? I said two or three, meaning years or tours. How many words do
you need for the rainy season, fighting, the heat, humidity and the
smell, fear?

I said yes, remembering the firefights over the hills and low mountains
at dusk, surrounding the bay at Danang, the jets coming in and dropping
bombs and tumbling napalm. Heard the artillery, a different sound and
cadence for every caliber. Beautiful parts in a violent orchestra, with
strings, wind instruments and drums.

I watched the earth explode and burn, green to black and dead. A
Broadway show sitting on a folding aluminum chair for the afternoon
matinee performance with a beer. He chuckled with his eyes, told me
about rocket-propelled grenade attacks on the airport, running for
cover, strafing runs on the riverbeds and trails, dancing to the music,
killing men in the open. He told me about friends on the wall, the
memorial we thought was going to be singular, unique, though we learned
that was a lie.

I told him about flying supplies into fire bases, coming in hot,
breaking hard, slowing and throwing off ammunition and water, grabbing
the wounded, engines whining to full power and the aircraft shuddering,
lifting off short and hoping that the enemy we hardly knew did not have
the range or luck. Dust choking, blinding. Laughing as we gained
altitude, at the odds beaten, and hoping to go back for another run.
Until next time. Every time we did this, the bets went back to even, the
clock unwound.

Our purchases were put on the glass counter top. He took his, tipped his
head and cap and headed off, nothing else. I glanced as he turned, said,
``So long,'' watched him walk to the cash register, pay with bills, and
out of the store, leaving his cart at the door. He limped and did not
look back. If he had said other than an acknowledgment with his nod, it
would have been more than he wanted to say and more than I needed to
hear.

What was it, two, three minutes of history? Meeting, touching and
knowing that there had been a past and that the present was tinged with
the gone before. Walking on. Shared survival and the pleasure of release
with the conjugation of years. Nothing to be spoken or shared with
clerks, cashiers or office workers. A diminishing private world with
signals and symbols.

Two men, strangers, known from and for a thousand years in kind, history
buried, burned into the circuits of the brain and whatever passes for
the soul. Pushed far back and down, down deep as it is possible to go.
For a second, I thought I saw a young man, looking out the door of an
aircraft at the jungle and rising flashes of light, soaring, swooping
and rocking, smiling, wind in his hair, wind so strong that it made the
eyes water and distorted his cheeks.

What he saw was something else. Maybe nothing.

We had validated ourselves at the meat counter in a small rural town
under a winter sky. For a day I thought about the man that I would never
meet again, except in myself. A man said that in our youths our hearts
were touched by fire, and fire is cleansing. I am not sure. Fire leaves
ash and scar. And a pale smoke that rises into the heavens and is gone.

Advertisement

\protect\hyperlink{after-bottom}{Continue reading the main story}

\hypertarget{site-index}{%
\subsection{Site Index}\label{site-index}}

\hypertarget{site-information-navigation}{%
\subsection{Site Information
Navigation}\label{site-information-navigation}}

\begin{itemize}
\tightlist
\item
  \href{https://help.nytimes3xbfgragh.onion/hc/en-us/articles/115014792127-Copyright-notice}{©~2020~The
  New York Times Company}
\end{itemize}

\begin{itemize}
\tightlist
\item
  \href{https://www.nytco.com/}{NYTCo}
\item
  \href{https://help.nytimes3xbfgragh.onion/hc/en-us/articles/115015385887-Contact-Us}{Contact
  Us}
\item
  \href{https://www.nytco.com/careers/}{Work with us}
\item
  \href{https://nytmediakit.com/}{Advertise}
\item
  \href{http://www.tbrandstudio.com/}{T Brand Studio}
\item
  \href{https://www.nytimes3xbfgragh.onion/privacy/cookie-policy\#how-do-i-manage-trackers}{Your
  Ad Choices}
\item
  \href{https://www.nytimes3xbfgragh.onion/privacy}{Privacy}
\item
  \href{https://help.nytimes3xbfgragh.onion/hc/en-us/articles/115014893428-Terms-of-service}{Terms
  of Service}
\item
  \href{https://help.nytimes3xbfgragh.onion/hc/en-us/articles/115014893968-Terms-of-sale}{Terms
  of Sale}
\item
  \href{https://spiderbites.nytimes3xbfgragh.onion}{Site Map}
\item
  \href{https://help.nytimes3xbfgragh.onion/hc/en-us}{Help}
\item
  \href{https://www.nytimes3xbfgragh.onion/subscription?campaignId=37WXW}{Subscriptions}
\end{itemize}
