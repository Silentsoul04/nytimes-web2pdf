Sections

SEARCH

\protect\hyperlink{site-content}{Skip to
content}\protect\hyperlink{site-index}{Skip to site index}

\href{https://myaccount.nytimes3xbfgragh.onion/auth/login?response_type=cookie\&client_id=vi}{}

\href{https://www.nytimes3xbfgragh.onion/section/todayspaper}{Today's
Paper}

\href{/section/business/dealbook}{DealBook}\textbar{}Jamie Dimon's
Letter --- Handicapping a Run for President in 2020: DealBook Briefing

\url{https://nyti.ms/2ElH5YK}

\begin{itemize}
\item
\item
\item
\item
\item
\end{itemize}

Advertisement

\protect\hyperlink{after-top}{Continue reading the main story}

Supported by

\protect\hyperlink{after-sponsor}{Continue reading the main story}

DealBook Business and Policy

\hypertarget{jamie-dimons-letter--handicapping-a-run-for-president-in-2020-dealbook-briefing}{%
\section{Jamie Dimon's Letter --- Handicapping a Run for President in
2020: DealBook
Briefing}\label{jamie-dimons-letter--handicapping-a-run-for-president-in-2020-dealbook-briefing}}

\includegraphics{https://static01.graylady3jvrrxbe.onion/images/2018/04/05/business/dealbook/05dn-newsletter-dimon/merlin_117646124_9a6628a5-20da-4959-94a1-c65f5633eeb0-articleLarge.jpg?quality=75\&auto=webp\&disable=upscale}

April 5, 2018

\begin{itemize}
\item
\item
\item
\item
\item
\end{itemize}

\textbf{Good Thursday. Here's what we're watching:}

• Jamie Dimon chafes under liquidity rules.

• Mark Zuckerberg has disclosed more on how much Facebook user data may
have been accessed.

• The White House is still talking tough on trade with China.

• Blackstone's infrastructure fund isn't doing so well.

\textbf{Get this in your inbox each morning. Sign up}
\href{https://www.nytimes3xbfgragh.onion/newsletters/dealbook?pgtype=subscriptionspage\&version=new\&contentId=DK\&eventName=signup\&module=newsletter-sign-up\&region=button}{\textbf{here}}\textbf{.}

\hypertarget{is-he-or-isnt-he}{%
\subsection{Is he or isn't he?}\label{is-he-or-isnt-he}}

On Wall Street, there's been a running parlor game about whether Jamie
Dimon, JPMorgan Chase's chief executive, would try to run for president
in 2020. His annual letters, filled with commentary on U.S. policy, have
only increased the chatter, and this year's letter, released Thursday,
is no different. It touches on everything from trade to immigration.
While Mr. Dimon, who once called himself ``barely a Democrat,'' says he
plans to stay at the bank for the next five years, the question among
political prognosticators is whether Mr. Dimon's policy views could ever
find a home among voters in this politically polarized environment.

Mr. Dimon's sensible letter was seemingly supportive of some of
President Trump's policies, while taking the opposing side on others.
Mr. Dimon particularly applauded the corporate tax cut and deregulation
efforts. On China trade, Mr. Dimon writes: ``It is not unreasonable for
the United States to press ahead for more equivalency,'' and that ``one
of the administration's best arguments is that negotiation alone has not
worked.'' However, he said he would like to see more cooperation with
U.S. allies on talks with China and added that the U.S. should ``revisit
the Trans-Pacific Partnership.''

On immigration, he called for tougher border control, writing,
``American citizens have the right to complain that we have not
successfully protected our borders since the last immigration reform in
1986.'' He also said, ``People immigrating to this country should be
taught American history, our language and our principles.'' At the same
time, he also advocated for ``a path to legal status and citizenship''
for ``Dreamers'' and improving ``merit-based immigration'' programs so
that immigrants educated here can stay.

Mr. Dimon made a point of saying that part of the problems facing the
country is that too many people are talking past each other.

``When people argue as if there are binary solutions, the argument is
almost always wrong,'' he wrote, advocating that people try to look at
all sides of a situation. ``I tell my liberal friends to read columnists
like Arthur Brooks and George Will. And I tell my conservative friends
to read writers like Tom Friedman.''

If Mr. Dimon were to run, he would likely have to do so as an
independent. It is hard to see how liberal Democrats would embrace much
of his deregulatory positions. And yet many of his positions -- and
especially some of his social positions not enunciated in his letter ---
may not be far right enough to capture the interest of Republicans.

Maybe the country needs someone in the middle, but history has not been
kind to third-party candidates.

\emph{--- Andrew Ross Sorkin}

\hypertarget{dimon-chafes-under-liquidity-rules}{%
\subsection{Dimon chafes under liquidity
rules.}\label{dimon-chafes-under-liquidity-rules}}

Jamie Dimon, in his annual
\href{https://www.jpmorganchase.com/corporate/investor-relations/document/ceo-letter-to-shareholders-2017.pdf}{letter}
to JPMorgan Chase's shareholders, had much to say about political issues
like trade and immigration.

But he also addressed a policy area much closer to home -- bank
regulation. Mr. Dimon often has expressed skepticism about aspects of
the banking overhaul that took place in the wake of the financial
crisis. But his remarks on regulation arguably carry more weight now.
Congress is closer to modifying the post-crisis rules than at any point
since Dodd-Frank was signed into law in 2010. An important regulatory
requirement Mr. Dimon criticized, therefore, might become a focus for
those wanting to roll back the rules.

Mr. Dimon in this year's letter zeroed in on regulations that require
banks to have a certain amount of liquidity, the financial term for
access to cash or low-risk assets that can be easily sold for cash.

In 2008, some large financial firms were crippled when customers,
depositors and creditors withdrew their money. The Federal Reserve and
taxpayers had to step in to support the banks. To help prevent a repeat
of this, post-crisis rules require that banks hold a certain amount of
liquid assets to cover a theoretical level of outflows. Mr. Dimon is in
favor of banks having more liquidity than before the crisis, but in the
letter, released Thursday, he wrote:

\begin{quote}
Liquidity requirements, while much higher, now have an element of
rigidity built in that did not exist before. Banks will be unable to use
that liquidity when they most need to do so --- to make loans or
intermediate markets. They have a ``red line'' they cannot cross (they
are required to maintain hard and fast liquidity requirements). As
clients demand more liquidity from their banks, the banks essentially
will be unable to provide it.
\end{quote}

Notably,
\href{https://www.americanbanker.com/news/cheat-sheet-inside-crapos-reg-relief-deal-with-democrats}{legislation}
currently in Congress would tweak liquidity rules. If it passes, bankers
may press for more loosening. And, as they have before, bank lobbyists
may deploy Mr. Dimon's main argument here -- that regulation can lead to
less lending. But there are flaws in this line of attack, especially
when it comes to liquidity rules.

First, a bank that lacks liquidity is in no position to lend. When a
bank's creditors and customers fear it doesn't have enough cash in
reserve, they will most likely withdraw their money. This not only
deprives the bank of the money it needs to lend, it also sharply reduces
its chance of survival. Or as Gregg Gelzinis of the left-leaning Center
for American Progress puts it, ``You won't be able to take this action
if you fail.''

Perhaps more importantly, Mr. Dimon leaves out some crucial context.
Congress designed the whole post-crisis overhaul to ensure that banks
have the funds on hand to lend during a downturn. Dodd-Frank's
architects saw that when banks stop lending, it makes a recession far
worse. And that is one reason why banks now have much higher levels of
capital. Regulators expect that they will use up some of that capital to
keep lending through the economic tumult. This is why the Federal
Reserve
\href{https://www.federalreserve.gov/newsevents/pressreleases/bcreg20170622a.htm}{said}
after its last bank stress tests:

\begin{quote}
The nation's largest bank holding companies have strong capital levels
and retain their ability to lend to households and businesses during a
severe recession.
\end{quote}

Andrew Gray, a JPMorgan spokesman, followed up with more material on Mr.
Dimon's thinking on liquidity.

He re-emphasized Mr. Dimon's point about having to tie up funds that
could be used to make loans. Mr. Gray noted that a bank facing outflows
would be expected to keep refilling its pool of liquid assets to be in
compliance with a liquidity regulation. This would leave fewer funds for
lending and making markets, Mr. Gray asserted.

He also addressed the argument that higher capital levels, not
liquidity, will largely determine the extent to which banks will lend
through a downturn. Mr. Gray asserted that liquidity rules play an
important role in ensuring banks have the resources to participate in
wider financial markets. Rules forcing the hoarding of liquidity may,
thus, crimp that activity.

\emph{--- Peter Eavis}

\includegraphics{https://static01.graylady3jvrrxbe.onion/images/2018/04/03/business/03dc-cfpb-1/merlin_133783710_02c7732c-adfd-43e4-b014-c5f191fa45fa-articleLarge.jpg?quality=75\&auto=webp\&disable=upscale}

\hypertarget{you-get-a-raise-you-get-a-raise}{%
\subsection{You get a raise. You get a
raise.}\label{you-get-a-raise-you-get-a-raise}}

Mick Mulvaney, the acting director of the Consumer Financial Protection
Bureau, has complained that the regulator engages in ``wasteful
spending.'' He even submitted a quarterly budget request recently of
\$0.

That attitude apparently didn't apply to two of his recent hires.

Mr. Mulvaney
\href{https://www.nytimes3xbfgragh.onion/2018/04/05/business/cfpb-mick-mulvaney-pay-raises.html}{appointed
two senior staff members who are paid salaries} far above what they had
been earning in their previous government jobs in Washington, according
to agency documents obtained by The New York Times through a public
records request.

• \textbf{Kirsten Sutton} was hired in January to be the agency's chief
of staff. Mr. Mulvaney agreed to pay her \$259,500 a year. Her salary is
the highest allowed under
\href{https://files.consumerfinance.gov/f/documents/cfpb_pay_scales.pdf}{the
consumer bureau's pay scale} and represents a raise of more than 50
percent working for one of the C.F.P.B.'s fiercest critics,
Representative Jeb Hensarling of Texas, the Republican chairman of the
House Financial Services Committee. It is also 22 percent more than her
predecessor.

• \textbf{Brian Johnson} is another new deputy. In late December, Mr.
Mulvaney
\href{https://www.documentcloud.org/documents/4430796-Johnson-Schedule-C-CFPB.html}{signed
paperwork} that awarded Mr. Johnson a salary of \$239,595. Mr. Johnson
previously worked on the House Financial Services Committee. In his most
recent job there, as policy director, he earned about \$170,000 a year,
according to LegiStorm. Of the 20 people at the bureau with the title of
senior adviser, Mr. Johnson is the second-highest paid. (Another
adviser, whose name wasn't disclosed in the public records reviewed by
The Times, is paid \$240,000.)

\textbf{Context}

The consumer bureau, as well as fellow financial regulators like the
Securities and Exchange Commission and the Office of the Comptroller of
the Currency, is allowed under federal law to pay employees
significantly more than other government agencies can pay. The rationale
is that the higher salaries are necessary to recruit skilled employees
who otherwise might land jobs on Wall Street.

As a result, the C.F.P.B. is stocked with employees who earn more than
the average Washington bureaucrat. Of the agency's 1,600 employees, 219
make more than \$200,000, according to the records reviewed by The
Times.

\emph{--- Stacy Cowley}

Image

BlackRock, led by Laurence D. Fink, will allow investors to invest in
market indexes without putting money into manufacturers and retailers of
firearms.Credit...Mike Cohen for The New York Times

\hypertarget{gun-free-funds}{%
\subsection{Gun-free funds.}\label{gun-free-funds}}

BlackRock, the world's largest money manager, plans to offer a series of
new products allowing individuals and institutions to invest in market
indexes without putting money into manufacturers and retailers of
firearms.

They include:

• a new exchange-traded fund that will track the performance of the MSCI
USA Small Cap Extended ESG Focus Index, composed of small stocks with
favorable environmental, social and governance standards, while
specifically excluding producers and large retailers of civilian
firearms.

• a similar credit-focused ETF that eschews the debt of gunmakers and
distributors.

• an option to take the guns out of institutional strategies tracking
major market indexes, including the S\&P 500, the Russell 2000 and three
others.

\textbf{Critic's corner}

\href{https://www.nytimes3xbfgragh.onion/2018/04/05/business/dealbook/blackrock-guns.html}{Rob
Cox of Breakigviews writes:}

\begin{quote}
``BlackRock's moves may spark criticism from gun-rights supporters, who
lashed out at companies like Delta Air Lines that severed ties with the
National Rifle Association. But it's hard to see how offering customers
more choice is ever a bad idea. Moreover, by getting out early,
BlackRock makes clear to those young people lobbying for tighter gun
laws -- who it hopes will be customers for decades to come -- that it
has something for them. Far from virtue signaling, it's simply good
business.''
\end{quote}

Image

Credit...Noah Berger/Associated Press

\hypertarget{that-was-a-huge-mistake-and-it-was-my-mistake}{%
\subsection{`That was a huge mistake, and it was my
mistake'}\label{that-was-a-huge-mistake-and-it-was-my-mistake}}

\href{https://www.nytimes3xbfgragh.onion/2018/04/04/technology/mark-zuckerberg-testify-congress.html?dlbk}{What
emerged from Facebook} yesterday --- in Mark Zuckerberg's conference
call with reporters and in a
\href{https://newsroom.fb.com/news/2018/04/restricting-data-access/}{company
blog post} --- were revelations that users' public data was more
compromised than previously thought.

• Facebook said Cambridge Analytica harvested the data of 87 million
users, not 50 million.

• A vulnerability in search and account recovery functions may have
exposed ``most'' of Facebook's 2 billion users to unauthorized
harvesting of their public profile information.

Mr. Zuckerberg, ahead of his testimony to Congress,
\href{https://www.cnbc.com/2018/04/04/mark-zuckerberg-facebook-user-privacy-issues-my-mistake.html}{struck
a conciliatory tone}. But regulators in Australia, the U.S. and
elsewhere
\href{https://www.wsj.com/articles/mark-zuckerberg-to-testify-before-house-committee-on-april-11-1522844990}{are
investigating} whether the company broke privacy rules.

Facebook also outlined steps to prevent misuse of its platform. Among
the most significant: restriction of access to so-called A.P.I.s that
let developers plug into its network.
\href{https://www.theverge.com/2018/4/4/17199632/facebook-cambridge-analytica-data-collection-87-million-users-api-developer-restrictions?dlbk}{The
Verge's take}: ``Effectively, Facebook has put a nail in the coffin of
its app platform.''

The most immediate effect: Tinder users
\href{http://nymag.com/selectall/2018/04/cant-log-into-tinder-blame-facebook.html?dlbk}{couldn't
log in}.

A brighter note for Facebook: Its shares are up in premarket trading
today, after Mr. Zuckerberg said that \#DeleteFacebook didn't appear to
be having much effect.

\textbf{The Facebook calendar}

• April 9: Alerts should start popping up for users whose information
may have been shared with Cambridge Analytica.

• April 10: Mr. Zuckerberg is scheduled to testify before the Senate's
Commerce and Judiciary committees.

• April 11: He will testify before the House Energy and Commerce
Committee.

\textbf{Elsewhere in tech and regulation:}
\href{https://www.wsj.com/articles/the-woman-who-is-reining-in-americas-technology-giants-1522856428?dlbk}{A
closer look at Margrethe Vestager}, the public face of Europe's tech
clampdown. And
\href{https://www.ft.com/content/86d1ce50-3799-11e8-8eee-e06bde01c544?dlbk}{the
privacy advocate Max Schrems} says the fight over user data has just
begun.

Image

\hypertarget{the-white-house-remains-committed-to-trade-saber-rattling}{%
\subsection{The White House remains committed to trade
saber-rattling}\label{the-white-house-remains-committed-to-trade-saber-rattling}}

\href{https://twitter.com/realDonaldTrump/status/981492087328792577}{President
Trump} and
\href{https://www.youtube.com/watch?time_continue=19\&v=uN1SkixgZ5Y}{the
White House} insist that the U.S. isn't in a trade war with China. But
it hasn't backed off from its tariff threats, and the White House press
secretary, Sarah Huckabee Sanders, said the U.S. may feel
``\href{https://www.politico.com/story/2018/04/04/china-tariffs-us-imports-trump-500163?dlbk}{a
little bit of short-term pain}.''

More
\href{https://www.nytimes3xbfgragh.onion/2018/04/04/business/the-united-states-is-starting-a-trade-war-with-china-now-what.html?dlbk}{from
behind the scenes}, by Ana Swanson and Keith Bradsher of the NYT:

\begin{quote}
People familiar with the negotiations say Steven Mnuchin, the Treasury
secretary, and Wilbur Ross, the commerce secretary, have at times argued
for more dialogue with the Chinese and quicker concessions that would
help diminish the trade deficit. \ldots{} Other top trade advisers,
including longtime China critics like Robert Lighthizer and Peter
Navarro, have taken a tougher stance.
\end{quote}

Calming the markets
\href{https://www.politico.com/story/2018/04/04/kudlow-trump-trade-economy-502446?dlbk}{has
mostly fallen to Larry Kudlow}, the White House's national economic
adviser of less than a week.

\href{https://www.nytimes3xbfgragh.onion/2018/04/04/business/dealbook/china-us-tariffs.html?dlbk}{Peter
Eavis's take}: We can take lessons on the U.S.-China trade skirmish from
Brexit. Looking back, the panicky early reaction to that seems to have
sprung more from a fear of the unknown than any big threats to the
global trading system.

And Greg Ip of the WSJ, who says
\href{https://www.wsj.com/articles/china-started-the-trade-war-not-trump-1521797401?dlbk}{China
started the trade fight}, had
\href{https://twitter.com/greg_ip/status/981585883739709441}{an
informative tweetstorm} about each side's vulnerabilities:

Image

But Neil Irwin of the Upshot warns that China has
\href{https://www.nytimes3xbfgragh.onion/2018/04/05/upshot/us-china-trade-war-unconventional-retaliation.html?dlbk}{other
ways of retaliating}.

\hypertarget{trade-deficit-hits-a-9-12-year-high-in-february}{%
\subsection{Trade deficit hits a 9 1/2 year high in
February.}\label{trade-deficit-hits-a-9-12-year-high-in-february}}

The United States trade deficit rose 1.6 percent to \$57.6 billion, its
highest level since October 2008. The deficit has now risen for six
straight months, the longest such streak since 2000.

The merchandise-trade gap with China narrowed to \$34.7 billion in
February from \$35.5 billion. The Trump administration wants to cut
\$100 billion, or about 25 percent, from the annual deficit with China,
Bloomberg points out. During President Trump's first year in office, the
United States' trade deficit with China rose to \$375.2 billion in 2017,
a record high.

Image

Crown Prince Mohammed bin Salman has talked about diversifying Saudi
industry and investment.Credit...Amir Levy/Reuters

\hypertarget{whither-blackstones-giant-infrastructure-fund}{%
\subsection{Whither Blackstone's giant infrastructure
fund?}\label{whither-blackstones-giant-infrastructure-fund}}

Last May, Blackstone announced with great fanfare that it planned a \$40
billion fund with Saudi Arabia to invest in U.S. infrastructure: up to
\$20 billion from the Saudis, who would match commitments by other
outside partners. But as of now, Blackstone has secured just \$575
million in such commitments --- and missed two fund-raising deadlines.

What's happened, per
\href{https://www.nytimes3xbfgragh.onion/2018/04/04/business/blackstone-infrastructure-fund-saudi.html?dlbk}{Kate
Kelly and Andrew}:

• The Saudis wanted the fund overseen by an investment committee, with a
seat for them. Blackstone refused.

• Many potential investors balked at committing money until an
investment team was in place, a process that took some time.

\textbf{In other infrastructure news:} D.J. Gribbin, the Trump
administration's infrastructure point person,
\href{https://apnews.com/1b0b9aafef6c416a85c1a71d7d80b7c3}{is leaving
the White House}, which looks to have delayed its investment plan until
after the midterms.

\hypertarget{the-political-flyaround}{%
\subsection{The political flyaround}\label{the-political-flyaround}}

• The E.P.A. chief Scott Pruitt, the subject of some unhappy rumblings
from the White House, has begun a media pushback against claims about
his spending.
(\href{https://www.politico.com/story/2018/04/04/scott-pruitt-trump-epa-job-456900?dlbk}{Politico})

• George Nader, the U.A.E.-connected businessman cooperating with Robert
Mueller's investigation, has links to Russia, too.
(\href{https://www.nytimes3xbfgragh.onion/2018/04/04/us/politics/george-nader-russia-uae-special-counsel-investigation.html?dlbk}{NYT})

• President Trump plans to deploy National Guard troops to help guard
the U.S.'s southwestern border. He'd previously said it would be ``our
military.''
(\href{https://www.nytimes3xbfgragh.onion/2018/04/04/us/politics/trump-governors-national-guard-border-mexico.html?dlbk}{NYT})

• John Kasich is eating at diners in New Hampshire, for some reason.
(\href{https://www.nytimes3xbfgragh.onion/2018/04/04/us/politics/john-kasich-trump-new-hampshire.html?dlbk}{NYT})

• Amalgamated Bank became the latest restrict gun sales by its business
customers, following guidelines published by the Mike Bloomberg-backed
Everytown for Gun Safety.
(\href{https://www.amalgamatedbank.com/news/amalgamated-bank-adopts-additional-policies-promote-gun-safety?dlbk}{Amalgamated})

Image

Sundar Pichai, Google's C.E.O.Credit...Jason Lee/Reuters

\hypertarget{google-should-not-be-in-the-business-of-war}{%
\subsection{`Google should not be in the business of
war'}\label{google-should-not-be-in-the-business-of-war}}

More than 3,100 Googlers have signed a petition
\href{https://www.nytimes3xbfgragh.onion/2018/04/04/technology/google-letter-ceo-pentagon-project.html?dlbk}{protesting
the tech colossus's A.I. work with the Defense Department}.

More from Scott Shane and Dai Wakabayashi of the NYT:

\begin{quote}
That kind of idealistic stance, while certainly not shared by all Google
employees, comes naturally to a company whose motto is ``Don't be
evil,'' a phrase invoked in the protest letter. But it is distinctly
foreign to Washington's massive defense industry and certainly to the
Pentagon, where the defense secretary, Jim Mattis, has often said a
central goal is to increase the ``lethality'' of the United States
military.
\end{quote}

The company line: ``We're actively engaged across the company in a
comprehensive discussion of this important topic.''

\textbf{In other defense tech news:} Oracle's Safra Catz reportedly
\href{https://www.bloomberg.com/news/articles/2018-04-04/oracle-s-catz-is-said-to-raise-amazon-contract-fight-with-trump?dlbk}{argued
to Mr. Trump over dinner} Tuesday night that the Pentagon's process for
choosing a cloud services contractor was tilted toward Amazon. (Oracle
is also bidding.)

\textbf{The tech flyaround}

• On balance, the U.S. Postal Service
\href{https://www.nytimes3xbfgragh.onion/2018/04/04/technology/amazon-postal-service-trump.html?dlbk}{probably
does pretty well out of Amazon}. Its real problem, Barry Ritholtz
argues,
\href{https://www.bloomberg.com/view/articles/2018-04-04/congress-not-amazon-messed-up-the-u-s-postal-service}{is
Congress}.

• Spotify shares ended down yesterday at \$144.22, but still up on
private stock trades earlier this year.
\href{https://www.nytimes3xbfgragh.onion/2018/04/04/business/media/as-spotify-goes-public-sony-cashes-in.html?dlbk}{One
big winner was Sony}, which sold 17.2 percent of its stake.

•
\href{https://www.wired.com/story/fin7-carbanak-hacking-group-behind-a-string-of-big-breaches?dlbk}{Meet
Fin7 (or Carbanak or Cobalt Spider)}, the criminal group suspected of
many prominent data thefts. And a
\href{https://www.nytimes3xbfgragh.onion/2018/04/04/business/energy-environment/pipeline-cyberattack.html?dlbk}{cyberattack
on a natural gas pipeline servicer} raised questions about the U.S.'s
energy system.

• Apple is reportedly working on touchless gestures and a curved screen
for iPhones.
(\href{https://www.bloomberg.com/news/articles/2018-04-04/apple-is-said-to-work-on-touchless-control-curved-iphone-screen?dlbk}{Bloomberg})

Image

Credit...YouTube

\hypertarget{the-youtube-suspect-had-complained-about-ad-policies}{%
\subsection{The YouTube suspect had complained about ad
policies}\label{the-youtube-suspect-had-complained-about-ad-policies}}

The woman suspected of shooting three people at YouTube headquarters
before killing herself had become angry with the company's policies.
Police did not say which ones, but she had posted about
\href{https://www.nytimes3xbfgragh.onion/2018/04/04/technology/youtube-attacker-demonetization.html?rref=collection\%2Fsectioncollection\%2Fbusiness}{YouTube
pulling ads from videos} that it said did not meet its standards.

The woman,
\href{https://www.nytimes3xbfgragh.onion/2018/04/04/us/youtube-shooting-nasim-najafi-aghdam.html}{Nasim
Najafi Aghdam}, was a
\href{https://www.nytimes3xbfgragh.onion/video/world/middleeast/100000005833316/youtube-shooter-was-popular-and-ridiculed-in-iran.html}{social
media video star} in Iran and had called YouTube a dictatorship because
she was making less money from her posts.

Also of note: The attack
\href{https://www.wsj.com/articles/youtube-shooting-raises-questions-about-silicon-valleys-open-campus-push-1522879455}{highlighted
the tension} between Silicon Valley giants' love of big, open campuses
and their security needs.

Image

Les Moonves honoring Lynda Carter.Credit...Chris Pizzello/Invision

\hypertarget{the-hot-take-les-moonves-shouldnt-run-cbs-viacom}{%
\subsection{The hot take: Les Moonves shouldn't run
CBS-Viacom}\label{the-hot-take-les-moonves-shouldnt-run-cbs-viacom}}

As the two corporate siblings jostle over a potential union --- Viacom
\href{http://www.latimes.com/business/hollywood/la-fi-ct-viacom-cbs-merger-20180404-story.html?dlbk}{has
reportedly rejected CBS's offer} and is preparing a counter bid --- who
would run the combined company remains a sticking point. Rich Greenfield
of BTIG
\href{https://www.btigresearch.com/wp-login.php?redirect_to=\%2F2018\%2F04\%2F04\%2Fviacom-and-cbs-merger-must-avoid-management-chaos\%2F\%3Fdlbk}{offered
a contrarian case} against it being Mr. Moonves of CBS, one of the
best-regarded media executives around.

``The future of a combined Viacom and CBS is going to be driven by an
increasing focus on international and scaling up,'' he wrote --- an
agenda better suited to Viacom's C.E.O., Bob Bakish.

Image

Credit...Mike Blake/Reuters

\hypertarget{the-deals-flyaround}{%
\subsection{The deals flyaround}\label{the-deals-flyaround}}

• Investor withdrawals at Bill Ackman's Pershing Square Capital
Management are huge.
(\href{https://www.wsj.com/articles/pershing-square-faces-wave-of-investor-redemptions-1522920601?dlbk}{WSJ})

• Starboard Value plans to continue
\href{https://www.wsj.com/articles/starboard-pursuing-proxy-fight-at-newell-brands-despite-deal-with-icahn-1522866118?dlbk}{pressing
its proxy fight} at Newell Brands. Carl Icahn, who has settled on that
one, is off trying to
\href{https://www.wsj.com/articles/carl-icahn-to-push-for-full-board-refresh-at-sandridge-1522884571?dlbk}{replace
the board} at SandRidge Energy.

• The judge overseeing the Justice Department's lawsuit to block AT\&T's
takeover of Time Warner asked about settling it through arbitration.
(\href{https://www.wsj.com/articles/arbitration-questioned-for-u-s-challenge-of-at-t-bid-for-time-warner-1522870327?dlbk}{WSJ})

• Broadcom is officially a U.S. company now. It still can't go after
Qualcomm.
(\href{https://www.prnewswire.com/news-releases/broadcom-completes-redomiciliation-to-the-united-states-300624646.html?dlbk}{Broadcom})

• Some big drug makers are regretting their deals of 2015 and 2016.
(\href{https://www.ft.com/content/1393a7f6-339d-11e8-ac48-10c6fdc22f03?dlbk}{FT})

• J.M. Smucker will pay \$1.9 billion for Ainsworth Pet Nutrition, which
makes Rachael Ray's dog food brand, and will consider selling its U.S.
baking brands like Pillsbury.
(\href{https://www.reuters.com/article/us-ainsworthpet-m-a-smucker/j-m-smucker-to-buy-ainsworth-pet-nutrition-for-1-9-billion-idUSKCN1HB2YG?dlbk}{Reuters})

• SS\&C of the U.S. is reportedly trying to top Temenos's £1.4 billion
bid for the British trading software maker Fidessa.
(\href{https://www.ft.com/content/94c7c1d8-3828-11e8-8b98-2f31af407cc8?dlbk}{FT})

Image

Credit...Reuters/Mike Blake

\hypertarget{quote-of-the-day}{%
\subsection{Quote of the day}\label{quote-of-the-day}}

\begin{quote}
``Retreating from the world is not the solution, nor is burning down the
current system and starting anew.''
\end{quote}

\emph{--- Jamie Dimon of JPMorgan, in his firm's}
\href{https://reports.jpmorganchase.com/investor-relations/2017/ar-ceo-letters.htm?dlbk}{\emph{latest
annual letter}}\emph{, where he also talks taxes, deficits and public
policy.}

\hypertarget{revolving-door}{%
\subsection{Revolving door}\label{revolving-door}}

• \textbf{Michael Doppelt} has rejoined Irving Place Capital as the
investment firm's head of fund-raising and limited partner relations.
(\href{https://www.reuters.com/article/us-irving-place-capital-moves-michael-do/irving-place-capital-names-fundraising-and-lp-relations-head-idUSKCN1HB21Q}{Reuters})

\hypertarget{the-speed-read}{%
\subsection{The speed read}\label{the-speed-read}}

• Britain's requirement that companies with at least 250 staff disclose
measures of gender pay gaps is forcing a reckoning at many companies.
(\href{https://www.nytimes3xbfgragh.onion/2018/04/04/business/britain-gender-pay-gap.html}{NYT})

• David Smith, chairman of Sinclair Broadcast Group, says the media is
getting his company all wrong.
(\href{https://www.nytimes3xbfgragh.onion/2018/04/04/business/media/sinclairs-boss-responds-to-criticism-you-cant-be-serious.html}{NYT})

• Settlement agreements between Bill O'Reilly and two of his accusers
were made public for the first time, offering details about his tactics
against sexual harassment allegations.
(\href{https://www.nytimes3xbfgragh.onion/2018/04/04/business/media/how-bill-oreilly-silenced-his-accusers.html}{NYT})

• Nike's top human resources executive told employees the company had
``failed to gain traction'' in hiring and promoting women and
minorities.
(\href{https://www.wsj.com/articles/nikes-hr-chief-says-company-fails-to-promote-enough-women-minoritiesmemo-1522871805}{WSJ})

• Why bond kings love Southern California.
(\href{https://www.bloomberg.com/news/features/2018-04-04/how-southern-california-became-home-to-bond-kings}{Bloomberg})

• A guide to the long war between Bumble and Tinder.
(\href{https://www.nytimes3xbfgragh.onion/2018/04/04/style/tinder-bumble-lawsuit-explainer.html}{NYT})

\emph{We'd love your feedback. Please email thoughts and suggestions to}
\href{mailto:bizday@NYTimes.com?subject=Newsletter\%20Feedback}{\emph{bizday@NYTimes.com}}\emph{.}

Advertisement

\protect\hyperlink{after-bottom}{Continue reading the main story}

\hypertarget{site-index}{%
\subsection{Site Index}\label{site-index}}

\hypertarget{site-information-navigation}{%
\subsection{Site Information
Navigation}\label{site-information-navigation}}

\begin{itemize}
\tightlist
\item
  \href{https://help.nytimes3xbfgragh.onion/hc/en-us/articles/115014792127-Copyright-notice}{©~2020~The
  New York Times Company}
\end{itemize}

\begin{itemize}
\tightlist
\item
  \href{https://www.nytco.com/}{NYTCo}
\item
  \href{https://help.nytimes3xbfgragh.onion/hc/en-us/articles/115015385887-Contact-Us}{Contact
  Us}
\item
  \href{https://www.nytco.com/careers/}{Work with us}
\item
  \href{https://nytmediakit.com/}{Advertise}
\item
  \href{http://www.tbrandstudio.com/}{T Brand Studio}
\item
  \href{https://www.nytimes3xbfgragh.onion/privacy/cookie-policy\#how-do-i-manage-trackers}{Your
  Ad Choices}
\item
  \href{https://www.nytimes3xbfgragh.onion/privacy}{Privacy}
\item
  \href{https://help.nytimes3xbfgragh.onion/hc/en-us/articles/115014893428-Terms-of-service}{Terms
  of Service}
\item
  \href{https://help.nytimes3xbfgragh.onion/hc/en-us/articles/115014893968-Terms-of-sale}{Terms
  of Sale}
\item
  \href{https://spiderbites.nytimes3xbfgragh.onion}{Site Map}
\item
  \href{https://help.nytimes3xbfgragh.onion/hc/en-us}{Help}
\item
  \href{https://www.nytimes3xbfgragh.onion/subscription?campaignId=37WXW}{Subscriptions}
\end{itemize}
