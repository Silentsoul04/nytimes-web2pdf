Sections

SEARCH

\protect\hyperlink{site-content}{Skip to
content}\protect\hyperlink{site-index}{Skip to site index}

\href{https://www.nytimes3xbfgragh.onion/section/politics}{Politics}

\href{https://myaccount.nytimes3xbfgragh.onion/auth/login?response_type=cookie\&client_id=vi}{}

\href{https://www.nytimes3xbfgragh.onion/section/todayspaper}{Today's
Paper}

\href{/section/politics}{Politics}\textbar{}Mark Zuckerberg Testimony:
Senators Question Facebook's Commitment to Privacy

\url{https://nyti.ms/2GMDpW5}

\begin{itemize}
\item
\item
\item
\item
\item
\item
\end{itemize}

Advertisement

\protect\hyperlink{after-top}{Continue reading the main story}

Supported by

\protect\hyperlink{after-sponsor}{Continue reading the main story}

\hypertarget{mark-zuckerberg-testimony-senators-question-facebooks-commitment-to-privacy}{%
\section{Mark Zuckerberg Testimony: Senators Question Facebook's
Commitment to
Privacy}\label{mark-zuckerberg-testimony-senators-question-facebooks-commitment-to-privacy}}

\includegraphics{https://static01.graylady3jvrrxbe.onion/images/2018/04/11/us/politics/11zuck-highlights/11zuck-highlights-videoSixteenByNineJumbo1600.jpg}

By The New York Times

\begin{itemize}
\item
  April 10, 2018
\item
  \begin{itemize}
  \item
  \item
  \item
  \item
  \item
  \item
  \end{itemize}
\end{itemize}

Facebook's chief executive, Mark Zuckerberg, began the first of two
marathon hearings in Washington on Tuesday afternoon, answering tough
questions on the company's mishandling of data.

This was Mr. Zuckerberg's first appearance before Congress, prompted by
the revelation that Cambridge Analytica, a political consulting firm
linked to the Trump campaign, harvested the data of an estimated 87
million Facebook users to psychologically profile voters during the 2016
election.

Mr. Zuckerberg,
\href{https://www.nytimes3xbfgragh.onion/2018/04/10/fashion/mark-zuckerberg-suit-congress.html}{clad
in a navy suit and bright blue tie}, faced hours of questioning from
lawmakers, who pressed him to account for how third-party partners could
data without users' knowledge. Senator John Thune of South Dakota talked
about the need for Facebook to avoid creating ``a privacy nightmare.''

Lawmakers grilled the 33-year-old executive on the proliferation of
so-called fake news on Facebook, Russian interference during the 2016
presidential election and censorship of conservative media. Among the
highlights:

• Senators warned that they are skeptical that the company can regulate
itself and threatened to enact privacy rules and other regulations. They
said they weren't sure if they could trust a company that has repeatedly
violated its privacy promises.

• There were glimmers of a partisan divide: Senator Ted Cruz, Republican
of Texas, asked about Facebook's handling of conservative media,
including content related to Glenn Beck and a Fox News personality;
Democrats probed Mr. Zuckerberg on how quickly Facebook responded to
Russian meddling.

• Mr. Zuckerberg, surrounded by his top legal and policy executives,
appeared well-coached. He answered questions directly and without
defensiveness as he tried to reiterate the mission of the social network
to better connect the world.

--- \emph{Cecilia Kang}

\hypertarget{i-think-that-may-be-what-this-is-all-about--your-right-to-privacy}{%
\subsection{`I think that may be what this is all about \ldots{} your
right to
privacy.'}\label{i-think-that-may-be-what-this-is-all-about--your-right-to-privacy}}

Senator Richard J. Durbin, Democrat of Illinois, zeroed in on the issue
at the heart of Facebook's troubles, asking Mr. Zuckerberg whether he
would be comfortable sharing the name of the hotel he stayed in last
night or if he would be comfortable sharing the names of the people he
has messaged this week.

``No. I would probably not choose to do that publicly here,'' Mr.
Zuckerberg said.

``I think that may be what this is all about,'' Mr. Durbin said. ``Your
right to privacy. The limits of your right to privacy. And how much you
give away in modern America in the name of, quote, connecting people
around the world.''

--- \emph{Deborah Solomon}

\hypertarget{centering-the-hearing-on-cambridge-analytica}{%
\subsection{Centering the hearing on Cambridge
Analytica}\label{centering-the-hearing-on-cambridge-analytica}}

Much of the hearing so far has centered on Cambridge Analytica. The
hearing was called as a result of
\href{https://www.nytimes3xbfgragh.onion/2018/04/04/us/politics/cambridge-analytica-scandal-fallout.html}{reporting
by The New York Times} on the company's data harvesting. Lawmakers asked
Mr. Zuckerberg what, if anything, he knew about Cambridge's harvesting,
what he was doing to ensure it would not happen again and whether he
knew of other operations that engaged in similar data collection on the
platform.

Mr. Zuckerberg said Facebook would be ``investigating many apps, tens of
thousands of apps, and if we find any suspicious activity, we're going
to conduct a full audit of those apps to understand how they're using
their data and if they're doing anything improper. If we find that
they're doing anything improper, we'll ban them from Facebook and we
will tell everyone affected.''

\emph{--- Matthew Rosenberg}

\includegraphics{https://static01.graylady3jvrrxbe.onion/images/2018/04/11/us/politics/11dc-fblivebrief-overall/merlin_136666506_bd81b6ee-4010-4187-b03a-df046a58f484-articleLarge.jpg?quality=75\&auto=webp\&disable=upscale}

\hypertarget{did-facebook-deceive-its-users}{%
\subsection{Did Facebook deceive its
users?}\label{did-facebook-deceive-its-users}}

Senator Kamala Harris, Democrat of California, zeroed in on whether
Facebook deceived consumers. She pressed Mr. Zuckerberg on whether the
company made a decision not to inform users about the Cambridge
Analytica episode when they learned in 2015 that data was sold by a
researcher to the political consulting firm.

``I'm talking about notification of users. And this relates to the issue
of transparency and the issue of trust: informing users of what you know
in terms of how their personal information was misused,'' Ms. Harris
said.

Mr. Zuckerberg did not admit that the company explicitly decided to
withhold that information from consumers, but he said the company made a
mistake in not informing users.

The question was key to the Federal Trade Commission's investigation of
Facebook's violation of a 2011 consent decree. If the company withheld
information, which would be a deceptive act, the company could face
record fines for violating its promises to the agency.

The tough questions by Ms. Harris, were closely watched because she is
from the San Francisco Bay Area and is seen as a rising political star
within the Democratic Party.

\emph{--- Cecilia Kang}

\hypertarget{democrats-press-on-russian-meddling}{%
\subsection{Democrats press on Russian
meddling}\label{democrats-press-on-russian-meddling}}

Senator Dianne Feinstein, the top Democrat on the Judiciary Committee,
pressed Mr. Zuckerberg on Russia's exploitation of the platform during
the 2016 presidential election.

Mr. Zuckerberg admitted that the company's effort to find and stop the
Russian meddling was ``slow,'' and called that failure ``one of my
greatest regrets.'' He said Facebook was tracking known Russian hacking
groups in real time but took much longer to recognize the inflammatory
posts of the Internet Research Agency, a private company with Kremlin
ties.

``There are people in Russia whose job is to exploit our systems,'' Mr.
Zuckerberg said. ``This is an arms race.''

But the Facebook founder said the company deployed new artificial
intelligence tools to detect malicious activity in elections in France,
Italy and a special Senate race in Alabama. He said he believed the new
technology would help protect the integrity of elections around the
world from manipulation via Facebook.

\emph{--- Scott Shane}

\includegraphics{https://static01.graylady3jvrrxbe.onion/images/2018/04/11/us/politics/11dc-fblivebriefvid/11dc-fblivebriefvid-videoSixteenByNine3000-v2.jpg}

\hypertarget{booker-raised-concerns-about-racial-targeting}{%
\subsection{Booker raised concerns about racial
targeting}\label{booker-raised-concerns-about-racial-targeting}}

Senator Cory Booker, Democrat of New Jersey, questioned Mr. Zuckerberg
over discriminatory uses of Facebook's advertising platform to target
ads to users by race, and tools that law enforcement officials have
reportedly used to surveil activists of color.

Mr. Booker's questioning is notable given that he and Mr. Zuckerberg
have a history of friendly collaboration dating back to 2010, when Mr.
Zuckerberg donated \$100 million to the public school system in Newark,
where Mr. Booker was mayor at the time.

Mr. Booker has long been seen as a tech-friendly lawmaker, and he has
known Mr. Zuckerberg for longer than most lawmakers. His tough stance
today is a sign of how dramatically the political winds around Facebook
have shifted.

--- \emph{Kevin Roose}

\hypertarget{as-zuckerberg-was-being-grilled-facebooks-stock-price-jumped}{%
\subsection{As Zuckerberg was being grilled, Facebook's stock price
jumped}\label{as-zuckerberg-was-being-grilled-facebooks-stock-price-jumped}}

Early impressions of Mr. Zuckerberg's testimony were positive. In his
first appearance before Congress, he appeared confident and answered
questions directly. At first he was grim-faced, looked tired and
serious. But he warmed up after an hour and offered humor about the
company's motto. He insisted on continuing questions when offered a
break, eliciting smiles and laughter from staff sitting behind him.

``This is a different Mark Zuckerberg than the Street was fearing,''
said Daniel Ives, chief strategy officer and head of technology research
for GBH Insights in New York. ``It's a defining 48 hours that will
determine the future of Facebook and so far he has passed with flying
colors and the Street is relieved.''

Investors appeared pleased: Facebook's stock closed up nearly 4.5
percent.

\hypertarget{some-senators-didnt-share-investors-enthusiasm}{%
\subsection{Some senators didn't share investors'
enthusiasm}\label{some-senators-didnt-share-investors-enthusiasm}}

Not all lawmakers left appeased by Mr. Zuckerberg's testimony.

``I was unsatisfied,'' said Senator Richard Blumenthal, Democrat of
Connecticut. ``More of the apology tour,'' he said, ``which we have
heard before.''

Mr. Blumenthal said it was clear to him that Facebook could not and
would not fully regulate itself and that Congress needed to provide a
solution.

``The old saying: There ought to be a law. There has to be a law. Unless
there's a law, their business model is going to continue to maximize
profit over privacy,'' he said.

--- \emph{Nicholas Fandos}

\hypertarget{move-fast-and--what}{%
\subsection{Move fast and \ldots{} what?}\label{move-fast-and--what}}

Senator John Cornyn of Texas, the majority whip, asked if Facebook's
motto is still ``move fast and break things.'' Mr. Zuckerberg said it
has been revised.

``I don't know when we changed it,'' Mr. Zuckerberg replied, ``but the
mantra is currently `move fast with stable infrastructure,' which is a
much less sexy mantra.'' He appeared to be joking, eliciting laughs from
the executives sitting behind him.

--- \emph{Cecilia Kang}

\hypertarget{facebook-is-working-with-robert-mueller}{%
\subsection{Facebook is working with Robert
Mueller}\label{facebook-is-working-with-robert-mueller}}

Mr. Zuckerberg said that Facebook was working with Robert S. Mueller
III, the special counsel investigating Russia's interference in the 2016
election and potential ties to the Trump campaign.

Replying to Senator Patrick Leahy, Democrat of Vermont, Mr. Zuckerberg
initially seemed to confirm that Facebook had been served with subpoenas
along the way, before saying he was not certain.

``I want to be careful here because our work with the special counsel is
confidential,'' he said. ``And I want to make sure that in an open
session I'm not revealing something that's confidential.''

Mr. Zuckerberg said he had not personally spoken with investigators from
Mr. Mueller's team, but that he thought others at the company had.

--- \emph{Nicholas Fandos}

\includegraphics{https://static01.graylady3jvrrxbe.onion/images/2018/04/11/autossell/user/user-videoSixteenByNineJumbo1600.jpg}

\hypertarget{zuckerberg-has-a-long-history-of-apologizing}{%
\subsection{Zuckerberg has a long history of
apologizing}\label{zuckerberg-has-a-long-history-of-apologizing}}

Mr. Zuckerberg has a history of apologizing for the company's mistakes
and promising to do better. Wired Magazine recently noted that Mr.
Zuckerberg has a 14-year history of apologizing. That seems to have
caused some consternation on Capitol Hill, where lawmakers prodded Mr.
Zuckerberg about why, exactly, they should believe his promises now.

``After more than a decade of promises to do better, how is today's
apology different and why should we trust Facebook to make the necessary
changes to ensure user privacy and give people a clearer picture of your
privacy policies?'' Senator John Thune, Republican of South Dakota,
asked.

Mr. Zuckerberg referred again to his company's humble beginnings in his
dorm room at Harvard.

``So we have made a lot of mistakes in running the company. I think it's
pretty much impossible, I believe, to start a company in your dorm room
and then grow it to be at the scale that we're at now without making
some mistakes.''

\hypertarget{will-regulation-of-facebook-be-coming}{%
\subsection{Will regulation of Facebook be
coming?}\label{will-regulation-of-facebook-be-coming}}

Mr. Thune, the chairman of the Senate Commerce Committee, called
Facebook and its role in society ``extraordinary'' and began the hearing
by explaining why Facebook is being singled out and why Mr. Zuckerberg
was asked to appear alone.

He said the Cambridge Analytica situation underscored how Facebook can
be used for nefarious reasons, saying it appeared ``to be the result of
people exploiting the tools you created to manipulate users'
information.''

In an indication that he may support legislation for internet companies,
Mr. Thune said, ``In the past, many of my colleagues on both sides of
the aisle have been willing to defer to tech companies' efforts to
regulate themselves. But this may be changing.''

Senator Charles E. Grassley of Iowa, who chairs the Judiciary Committee,
said the tech industry ``has a responsibility'' to protect its users and
said ``the status quo no longer works.''

\emph{--- Cecilia Kang}

Image

Senator John Thune of South Dakota talked about the need for Facebook to
avoid creating ``a privacy nightmare.''Credit...Gabriella Demczuk for
The New York Times

\hypertarget{a-lingering-question-does-facebook-favor-democrats}{%
\subsection{A lingering question: Does Facebook favor
Democrats?}\label{a-lingering-question-does-facebook-favor-democrats}}

That was the accusation that Senator Ted Cruz, Republican of Texas, was
leveling at Mr. Zuckerberg when he grilled him for several minutes as to
why the social network has been allegedly censoring content from
conservative organizations and Trump supporters.

Mr. Zuckerberg declined to answer whether Facebook is a neutral public
forum or if it is expressing its own views of free speech, avoiding a
complex legal question that Mr. Cruz was posing.

However, Mr. Zuckerberg insisted that the company does not discriminate
against Republican employees and that its definition for what kind of
language should be kept off the platform was rooted in common sense.

``I am very committed to making sure that Facebook is a platform for all
ideas,'' he said after Mr. Cruz ticked off several examples of potential
liberal bias on the social network.

--- \emph{Alan Rappeport}

\hypertarget{is-facebook-a-monopoly}{%
\subsection{Is Facebook a monopoly?}\label{is-facebook-a-monopoly}}

Senator Lindsey Graham, Republican of South Carolina, pressed Mr.
Zuckerberg on whether Facebook is a monopoly, asking him to explain what
other options customers have if they get frustrated with the social
network.

While Mr. Graham, the South Carolina Republican, tried compare his
industry to the car business, where people can switch from Ford to
Chevrolet if they want, Mr. Zuckerberg insisted that in his case, it's
complicated. He noted that Facebook overlaps with companies like Google
and Twitter and that he faces competition from a variety of popular
apps.

``You don't think you have a monopoly?'' Mr. Graham asked.

Mr. Zuckerberg replied: ``It doesn't feel that way to me.''

\hypertarget{a-boost}{%
\subsection{A boost}\label{a-boost}}

Mr. Zuckerberg is clad
\href{https://www.nytimes3xbfgragh.onion/slideshow/2018/04/10/fashion/mark-zuckerbergs-greatest-suits-appearances/s/10OTR-slide-R0LB.html}{in
a suit} and tie at his hearing. He also had a cushion in his seat to
help give his testimony a lift.

\hypertarget{before-the-hearing-zuckerberg-was-welcomed-by-dozens-of-zuckerbergs}{%
\subsection{Before the hearing, Zuckerberg was welcomed by dozens of
Zuckerbergs}\label{before-the-hearing-zuckerberg-was-welcomed-by-dozens-of-zuckerbergs}}

Image

Greeting Mr. Zuckerberg on Tuesday are dozens of cardboard cutouts of
his own image wearing ``fix fakebook.'' in front of the U.S. Capitol
Building.Credit...Lawrence Jackson for The New York Times

Advertisement

\protect\hyperlink{after-bottom}{Continue reading the main story}

\hypertarget{site-index}{%
\subsection{Site Index}\label{site-index}}

\hypertarget{site-information-navigation}{%
\subsection{Site Information
Navigation}\label{site-information-navigation}}

\begin{itemize}
\tightlist
\item
  \href{https://help.nytimes3xbfgragh.onion/hc/en-us/articles/115014792127-Copyright-notice}{©~2020~The
  New York Times Company}
\end{itemize}

\begin{itemize}
\tightlist
\item
  \href{https://www.nytco.com/}{NYTCo}
\item
  \href{https://help.nytimes3xbfgragh.onion/hc/en-us/articles/115015385887-Contact-Us}{Contact
  Us}
\item
  \href{https://www.nytco.com/careers/}{Work with us}
\item
  \href{https://nytmediakit.com/}{Advertise}
\item
  \href{http://www.tbrandstudio.com/}{T Brand Studio}
\item
  \href{https://www.nytimes3xbfgragh.onion/privacy/cookie-policy\#how-do-i-manage-trackers}{Your
  Ad Choices}
\item
  \href{https://www.nytimes3xbfgragh.onion/privacy}{Privacy}
\item
  \href{https://help.nytimes3xbfgragh.onion/hc/en-us/articles/115014893428-Terms-of-service}{Terms
  of Service}
\item
  \href{https://help.nytimes3xbfgragh.onion/hc/en-us/articles/115014893968-Terms-of-sale}{Terms
  of Sale}
\item
  \href{https://spiderbites.nytimes3xbfgragh.onion}{Site Map}
\item
  \href{https://help.nytimes3xbfgragh.onion/hc/en-us}{Help}
\item
  \href{https://www.nytimes3xbfgragh.onion/subscription?campaignId=37WXW}{Subscriptions}
\end{itemize}
