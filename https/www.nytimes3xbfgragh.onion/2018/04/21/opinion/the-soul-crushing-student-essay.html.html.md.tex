Sections

SEARCH

\protect\hyperlink{site-content}{Skip to
content}\protect\hyperlink{site-index}{Skip to site index}

\href{https://www.nytimes3xbfgragh.onion/section/opinion/sunday}{Sunday
Review}

\href{https://myaccount.nytimes3xbfgragh.onion/auth/login?response_type=cookie\&client_id=vi}{}

\href{https://www.nytimes3xbfgragh.onion/section/todayspaper}{Today's
Paper}

\href{/section/opinion/sunday}{Sunday Review}\textbar{}The Soul-Crushing
Student Essay

\url{https://nyti.ms/2F3pPI5}

\begin{itemize}
\item
\item
\item
\item
\item
\end{itemize}

Advertisement

\protect\hyperlink{after-top}{Continue reading the main story}

Supported by

\protect\hyperlink{after-sponsor}{Continue reading the main story}

\href{/section/opinion}{Opinion}

\href{/column/on-campus}{On Campus}

\hypertarget{the-soul-crushing-student-essay}{%
\section{The Soul-Crushing Student
Essay}\label{the-soul-crushing-student-essay}}

By Scott Korb

Mr. Korb teaches writing to first-year college students.

\begin{itemize}
\item
  April 21, 2018
\item
  \begin{itemize}
  \item
  \item
  \item
  \item
  \item
  \end{itemize}
\end{itemize}

\includegraphics{https://static01.graylady3jvrrxbe.onion/images/2018/04/23/opinion/23Korb/23Korb-articleLarge.jpg?quality=75\&auto=webp\&disable=upscale}

Last August, as college started up again, I hadn't quite finished my
beach read, William Finnegan's ``Barbarian Days: A Surfing Life,'' so I
brought it to my freshman writing class. I tried reading a few passages
aloud to break the ice. I thought my students might relate to one in
which the writer first wonders about turning his surfing experiences
into something worth reading:

``Our queer devotions, frustrations, little triumphs, and large
peculiarities, plus a few waterfront characters, plus photos, could
probably keep a blog burbling along.''

``What do you make of that?'' I asked. ``Large peculiarities --- what do
you suppose he means?''

Crickets.

Sand I'd smuggled back from California slid from between the pages to my
desk.

Maybe they were wondering what a blog was.

We expect college freshmen to feel at least as comfortable with
self-expression as the burbling bloggers and writers of yesteryear.
Something beyond stylized selfies must populate their social media
streams, after all.

But every year I find that getting them to admit to feeling devoted or
frustrated, to being peculiar in any way (much less in a large way),
verges on impossible. And as someone who has read thousands of student
essays over the past 10 years, few things are more dispiriting --- and
as the pages mount, soul-crushing --- than those written by 18-year-olds
who can't see themselves as peculiar.

But why can't they?

One reason reveals itself when someone finally asks the clarifying
question: ``Do you mean we can write with the word `I'?''

The class looks up in wonder. This happens every semester.

Somewhere along the way, these young people were told by teachers that
who they are in their writing ought to be divorced from who they are on
their phones, or as the writer Grace Paley may have said, with their
families and on their streets. They know a private ``I'' who experiences
devotion and frustration. I see them text in class and talk and laugh
and sometimes cry in the halls. They wear band T-shirts, often from my
era, so I assume they have taste. I watch them read.

But no matter who they are in private, when I first encounter their
writing, they use only the public passive voice: The text was read. The
test was taken.

It's never: I read the text. I took the test. And it is never ever: I
loved the text with queer devotion!

It's true that a student's writing style isn't everything and that much
of what we call good writing cannot be taught. (Bad writing apparently
has been.) One can be devoted to something --- a band from the '90s,
surfing, YHWH--- without being able to put that devotion into words.

But my experience with students has me worried that years of ``texts
being read'' and ``tests being taken'' have created the sense in them
that whatever they're devoted to doesn't matter much to the rest of us
--- so long as they know the answers to our questions, so long as they
pass the test. Writing so passively and with what they've been taught is
appropriate and ``objective'' distance from topics they often seem
disinterested in, these young people signal to me that they're still
waiting for something important or real to happen to them.

Perhaps they feel that only someone who has lived through something
momentous --- like the teenagers who survived the Parkland, Fla.,
shooting --- has earned the right to be heard. It's hard to imagine any
of those young activists writing, ``The rally was held because Congress
was lobbied and guns were purchased.''

But what about those queer devotions and frustrations, experiences and
ideas that have stirred an individual heart into peculiarity?

A decade teaching young writers has taught me a great deal. First, we
need to value more the complete and complex lives of young people: where
they come from, how they express themselves. They have already lived
lives worthy of our attention and appreciation.

Second, we need to encourage young people to take seriously those lives
they've lived, even as they come to understand --- often through
schooling and just as often not --- that there's a whole lot more we'll
expect of them. Through this, we can help them learn to expect more of
themselves, too.

Some lines from the great writer John McPhee have helped me consolidate
these lessons over the years.
\href{https://www.newyorker.com/magazine/2011/11/14/progression}{Reflecting
in The New Yorker in 2011}, he wrote: ``I once made a list of all the
pieces I had written in maybe 20 or 30 years, and then put a check mark
beside each one whose subject related to things I had been interested in
before I went to college. I checked off more than 90 percent.''

I always tell my students that I find these lines heartening. As a
writer, I've spent more than 20 years reckoning with the joys and
tragedies, the shame and grief, commitments to sports and study, of my
own pre-college years. A good deal of my writing continues to take me to
northern Florida where, when I was young, my father was killed by a
drunken driver; the stories I continue to uncover there --- about
justice and race and addiction --- begin with me at 5 and continue
through my adolescence into this adult life.

Mr. McPhee, and Mr. Finnegan, too --- who at 13,
\href{https://www.newyorker.com/magazine/2015/06/01/off-diamond-head-finnegan}{he
writes}, found in the obliterative sea that ``the frontiers of the
thinkable were quietly, fitfully edging back'' --- tell me that there's
no good reason for me ever to stop going to Florida and attending to
what happened there.

At the start of this semester, I read some passages from Barry Lopez's
wintry classic
\href{https://books.google.com/books?id=Y-WxJYMD5HsC\&pg=PT11\&lpg=PT11\&dq=\%22It+is+as+subtle+in+its+expression+as+turns+of+the+mind\%22\&source=bl\&ots=E13jUzsHZm\&sig=TuD6gkf-AeDo_R4d0fwtYDYSVxo\&hl=en\&sa=X\&ved=0ahUKEwiC5KOQ1bfaAhULd98KHb32CtwQ6AEIKjAB\#v=onepage\&q=\%22It\%20is\%20as\%20subtle\%20in\%20its\%20expression\%20as\%20turns\%20of\%20the\%20mind\%22\&f=false}{``Arctic
Dreams.''} The descriptions are incomparable, even as the setting itself
remains ineffable: ``The physical landscape is baffling in its ability
to transcend whatever we would make of it. It is as subtle in its
expression as turns of the mind, and larger than our grasp; and yet it
is still knowable.''

This has been the lesson for my students this term. Look around at what
baffles you; look in at your peculiar self and how your own frontiers
continue to edge back. Don't worry, you'll never fully grasp how the
world transcends you and your ability to describe it. I surely don't,
and I'm 41! But don't forget: You've been trying to understand and
triumph in the world for as long as you can remember, even as a kid. Now
go and write.

Advertisement

\protect\hyperlink{after-bottom}{Continue reading the main story}

\hypertarget{site-index}{%
\subsection{Site Index}\label{site-index}}

\hypertarget{site-information-navigation}{%
\subsection{Site Information
Navigation}\label{site-information-navigation}}

\begin{itemize}
\tightlist
\item
  \href{https://help.nytimes3xbfgragh.onion/hc/en-us/articles/115014792127-Copyright-notice}{©~2020~The
  New York Times Company}
\end{itemize}

\begin{itemize}
\tightlist
\item
  \href{https://www.nytco.com/}{NYTCo}
\item
  \href{https://help.nytimes3xbfgragh.onion/hc/en-us/articles/115015385887-Contact-Us}{Contact
  Us}
\item
  \href{https://www.nytco.com/careers/}{Work with us}
\item
  \href{https://nytmediakit.com/}{Advertise}
\item
  \href{http://www.tbrandstudio.com/}{T Brand Studio}
\item
  \href{https://www.nytimes3xbfgragh.onion/privacy/cookie-policy\#how-do-i-manage-trackers}{Your
  Ad Choices}
\item
  \href{https://www.nytimes3xbfgragh.onion/privacy}{Privacy}
\item
  \href{https://help.nytimes3xbfgragh.onion/hc/en-us/articles/115014893428-Terms-of-service}{Terms
  of Service}
\item
  \href{https://help.nytimes3xbfgragh.onion/hc/en-us/articles/115014893968-Terms-of-sale}{Terms
  of Sale}
\item
  \href{https://spiderbites.nytimes3xbfgragh.onion}{Site Map}
\item
  \href{https://help.nytimes3xbfgragh.onion/hc/en-us}{Help}
\item
  \href{https://www.nytimes3xbfgragh.onion/subscription?campaignId=37WXW}{Subscriptions}
\end{itemize}
