Sections

SEARCH

\protect\hyperlink{site-content}{Skip to
content}\protect\hyperlink{site-index}{Skip to site index}

\href{https://myaccount.nytimes3xbfgragh.onion/auth/login?response_type=cookie\&client_id=vi}{}

\href{https://www.nytimes3xbfgragh.onion/section/todayspaper}{Today's
Paper}

Six Times Journalists on the Paper's History of Covering AIDS and Gay
Issues

\url{https://nyti.ms/2KjlQLs}

\begin{itemize}
\item
\item
\item
\item
\item
\end{itemize}

Advertisement

\protect\hyperlink{after-top}{Continue reading the main story}

Supported by

\protect\hyperlink{after-sponsor}{Continue reading the main story}

\hypertarget{six-times-journalists-on-the-papers-history-of-covering-aids-and-gay-issues}{%
\section{Six Times Journalists on the Paper's History of Covering AIDS
and Gay
Issues}\label{six-times-journalists-on-the-papers-history-of-covering-aids-and-gay-issues}}

Edited by \href{https://www.nytimes3xbfgragh.onion/by/kurt-soller}{Kurt
Soller}

\begin{itemize}
\item
  April 27, 2018
\item
  \begin{itemize}
  \item
  \item
  \item
  \item
  \item
  \end{itemize}
\end{itemize}

\emph{The New York Times had a spotty record of covering the AIDS
epidemic in the early 1980s --- and gay culture in general. Times
staffers reflect on the paper's past, and what we can learn from it
today.}

Any newspaper must, by definition, aspire to be the ``paper of record,''
and yet when it came to \emph{this} newspaper's coverage of gay people
and AIDS in the early '80s --- when the disease was morphing into a
national crisis, and when rights that had been won a decade earlier,
after the Stonewall Riots, were once again being jeopardized --- The
Times's own record was checkered at best. Information about the spread
of illness was often scant, judgmental or distressingly vague --- even
while reporters on the Science desk were
\href{https://www.nytimes3xbfgragh.onion/2018/04/13/t-magazine/nyt-writers-80s-coverage.html?rref=collection\%2Fsectioncollection\%2Ft-magazine\&action=click\&contentCollection=t-magazine\&region=stream\&module=stream_unit\&version=latest\&contentPlacement=18\&pgtype=sectionfront}{trying
their best with an ever-evolving story}. The social and emotional toll
of AIDS and the resulting queer movement were, when covered, often
buried in the back of the newspaper (on a page called Styles of the
Times), far from national news stories that were deemed important enough
for the front page. Famously, it would take President Ronald Reagan more
than
\href{https://www.nytimes3xbfgragh.onion/1987/10/11/opinion/the-reagan-aids-strategy-in-ruins.html}{four
years to} acknowledge the disease publicly. And it took until 1983 for
The Times to run an article about the disease on Page A1, two years
after
\href{https://www.nytimes3xbfgragh.onion/1981/07/03/us/rare-cancer-seen-in-41-homosexuals.html}{the
first reports of symptoms.}

When T Magazine chose to devote an issue to the cultural relevance of
\href{https://www.nytimes3xbfgragh.onion/interactive/2018/04/17/t-magazine/new-york-1980s-culture.html}{New
York City in the years between 1981 and 1983}, it was inevitable that
gay people and the disease that disproportionately decimated them would
play a major role in our coverage, from the
\href{https://www.nytimes3xbfgragh.onion/2018/04/16/t-magazine/gay-literature-1980s.html}{birth
of gay literature} to the
\href{https://www.nytimes3xbfgragh.onion/2018/04/19/t-magazine/keith-haring-tina-chow-aids-resurrected.html}{hundreds
of creative leaders} who changed the culture only to then
\href{https://www.nytimes3xbfgragh.onion/interactive/2018/04/17/t-magazine/aids-epidemic-deaths-new-york.html}{die
prematurely}. But we also wanted to re-examine the ways in which The New
York Times itself dealt with these issues as they were happening. To
reread these articles today is to be struck by how confused and scared
and defensive many people must have felt, even those acting from a place
of journalistic remove: In Frank Rich's
\href{https://www.nytimes3xbfgragh.onion/1985/04/22/theater/theater-the-normal-heart-by-larry-kramer.html}{laudatory
1985 review} of Larry Kramer's ``The Normal Heart,'' for instance,
there's a strange addendum noting that ``a spokesman from The New York
Times said yesterday that charges in `The Normal Heart' that The Times
suppressed news about AIDS are untrue.'' To reread these articles is
also to be reminded how news coverage shapes perceptions and policies,
particularly when it comes to oppressed communities. So we asked six
members of the L.G.B.T.Q. community who work for The Times (across
various ages and beats) to reflect and report on what the paper got
right and wrong in those years, and what we might learn from that
troubling history today. \emph{--- Kurt Soller}

\includegraphics{https://static01.graylady3jvrrxbe.onion/images/2018/04/27/t-magazine/mini-essays-slide-TBD7/mini-essays-slide-TBD7-articleLarge.jpg?quality=75\&auto=webp\&disable=upscale}

\hypertarget{revisiting-rare-cancer-seen-in-41-homosexuals}{%
\subsection{Revisiting `Rare Cancer Seen in 41
Homosexuals'}\label{revisiting-rare-cancer-seen-in-41-homosexuals}}

\hypertarget{by-jeremy-w-peters-washington-correspondent}{%
\subsubsection{By Jeremy W. Peters, Washington
correspondent}\label{by-jeremy-w-peters-washington-correspondent}}

Like most children of the 1980s, I can't remember a time when I wasn't
hearing about AIDS. I was in my early teens during the height of the
epidemic, and it permeated almost everything in my young consciousness:
school, culture, sports and the evening news. This was when the Indiana
teenager Ryan White was dying, basketball star Magic Johnson announced
he'd contracted H.I.V., Tom Hanks won the Oscar for playing an infected
gay man in 1993's ``Philadelphia'' and the disease became the leading
cause of death among Americans aged 25 to 44. At school, the AIDS
lexicon was drilled into our heads. We knew by then that you couldn't
get it from a toilet seat or a kiss. But we were constantly reminded how
we could get it: ``Unprotected sex'' and ``dirty needles'' would fill in
the blanks on many a health class pop-quiz about ``risk behaviors.''

I think the disease's ubiquity and the involuntary nature of how we came
to know about it was why I was so struck, some 30 years later, when I
first came across the 1981 New York Times article that alerted readers
to a strange new condition.
``\href{https://www.nytimes3xbfgragh.onion/1981/07/03/us/rare-cancer-seen-in-41-homosexuals.html}{Rare
Cancer Seen in 41 Homosexuals},'' read the headline. The story was on
Page A20, not exactly prime print real estate, and ran the day before
the Fourth of July. The article went on to describe the initial
hypotheses doctors were making about this sudden outbreak of a rare
cancer, Kaposi's sarcoma --- hypotheses that would turn out to be
tragically wrong, such as how it might not be contagious or that ``there
was no apparent danger to nonhomosexuals.'' It was a bit disorienting to
read early, inchoate descriptions of symptoms that soon none of us would
mistake. There was the malfunction of T cells that one doctor found in
nine of the victims, which was connected to ``severe defects in their
immunological systems.'' And the descriptions of telltale AIDS
disfigurement: ``It appears in one or more violet-colored spots anywhere
on the body.''

I recently talked to the reporter who wrote that article, my colleague
Lawrence K. Altman, a physician with training in epidemiology, who told
me what a hindrance it was in those first years of AIDS that people
struggled to find the right language. This was true, he said, not just
for the government officials and doctors talking about the disease on
television and in print, but also for the journalists who had to explain
it. In doing so, they often employed the same euphemisms, like ``bodily
fluids,'' to substitute for words like semen, leaving the impression
that saliva from a kiss might infect you.

AIDS has always been scary to me. But this shed light on a different
kind of fear that I hadn't quite contemplated, the kind that gay men
before me must have felt as they watched this plague kill their friends
and were left wondering whether they would be next. For my generation,
it was the known that was so terrifying. We knew that if you somehow
caught the disease it might be a death sentence. But we also knew how
you caught it and therefore how to avoid it. The fear of the unknown, it
dawned on me, would have been far worse.

\begin{center}\rule{0.5\linewidth}{\linethickness}\end{center}

Image

A 1992 flier for a gay pride party thrown by a group of New York Times
employees. Inset: Natalie Kitroeff.

\hypertarget{the-new-york-timess-coming-out-party}{%
\subsection{The New York Times's Coming-Out
Party}\label{the-new-york-timess-coming-out-party}}

\hypertarget{by-natalie-kitroeff-economy-reporter}{%
\subsubsection{By Natalie Kitroeff, economy
reporter}\label{by-natalie-kitroeff-economy-reporter}}

The flier, sleek and disarming, showed up one day on newsroom bulletin
boards in the summer of 1992. It was an invitation in search of a guest
list.

``We're having a party.\\
We'd love to invite you.\\
But we don't know who you are.''

It was the first time that gay people working at The New York Times had
publicly advertised one of their pride parties, which had been happening
for six years. ``The idea that a bunch of gay people from The Times
would get together and have a party was remarkable, after so many years
of hiding,'' said Richard J. Meislin, a former reporter and editor, who
kept his sexuality secret until the 1980s, when other queer people at
the newspaper started to come out.

It wasn't easy to be gay at The Times when Meislin started in 1975 as a
copy boy. Many employees felt that A.M. ``Abe'' Rosenthal, the paper's
editor, was homophobic. And the publisher at the time, Arthur ``Punch''
Sulzberger, had his own blind spots, according to his son. ``Abe was
part of the challenge. It was one of the issues my father struggled with
as well,'' said Arthur Sulzberger Jr., the publisher since 1992, who
recently handed over the reins to his son, A.G. ``The way {[}gay
people{]} were being treated by being forced to be hidden was
antithetical to the values of the company and the values of
journalism.''

For this essay, I spoke with five current and former Times staffers who
were among the first people to come out at the paper in the 1980s and
beyond, including Donna Cartwright, who in 1998 became the first
newsroom employee to come out as transgender. I read Samuel G.
Freedman's
\href{http://www.tabletmag.com/scroll/191965/remembering-jeff-schmalz-a-trailblazing-reporter-who-covered-aids}{oral
history chronicling the work of Jeffrey Schmalz}, who died of
AIDS-related causes in 1993 and transformed the paper's coverage of the
disease. I watched a video of his memorial service.

What emerged was a picture of a newsroom that often forced its L.G.B.T.
journalists to choose between their career ambitions and their desire to
have an openly queer life. The Times spent much of the 1980s figuring
out how to cover gays as real people with newsworthy problems. It was
also deeply unsure of how to deal with its gay employees, whose
sexuality may have seemed, to those in management, to be at odds with
the pedigree the institution wanted to uphold.

Meislin, for instance, never had control of his own story: He was outed
by AIDS, even though he didn't have the disease. After arriving in
Mexico City as a correspondent in 1983, he got sick with a mysterious
ailment that made his joints ache. The paper sent an editor to check on
him, because some people thought he'd contracted H.I.V. He hadn't, and
though Meislin recovered within weeks, word of his sexuality reportedly
got to Rosenthal's desk. Two and a half years later, Meislin was called
back to New York City, cutting short his prestigious stint abroad. ``The
perception in the newsroom was that I was brought back because I was
gay,'' he said. ``And it's one that I shared.''

Cartwright, who started at the Times in 1977, said that episode was one
of many reasons she didn't come to terms with her gender identity
sooner. ``If you lived openly as a gay person you were sticking your
neck out.''

Rosenthal, who died in 2006, denied that Meislin's move had anything to
do with his sexuality in an interview with Michelangelo Signorile for
The Advocate 26 years ago. ``People who are used to being discriminated
against will sometimes take certain acts as being discriminatory when
they're not,'' he told the magazine.

Nancy Lee, who eventually became the paper's picture editor, didn't dare
put a photo of her girlfriend at her desk when she arrived in 1980.
``That was to risk everything,'' said Lee, who is now the executive
editor of The New York Times News Service \& Syndicate.

David W. Dunlap, a former Metro reporter and current Times historian,
said he added a pullout couch to the Washington apartment he was sharing
with his boyfriend during his first job as a clerk to the columnist
James Reston in 1975. The idea, he said, was to give the impression that
the couch was \emph{definitely} the place where his roommate slept.

To me, those stories are unavoidably personal. I started at The Times as
a clerk to a columnist in 2012, while I was figuring out that I was a
lesbian. I never came out to him, because I wasn't sure I wanted him to
know. After I left and returned as a reporter three years later, I put a
photo of my girlfriend at the time next to my computer.

The path to my comfort, I now know, was paved by people like Lee, who
decided to show up to New York's pride parade in the late 1980s with
other closeted Times staffers. ``We weren't wearing any New York Times
gear, but we knew we were at risk,'' she said. ``People were
photographing it. You could end up on the news.'' They had their first
pride party in 1987. Sulzberger Jr., then an editor, had started meeting
with his gay colleagues several years earlier. ``He would take us to
lunch and say `I know you're gay and don't worry about it, you're going
to be fine,''' Dunlap said.

In the fall of 1986, Max Frankel replaced Rosenthal as executive editor.
Months later, Sulzberger Jr. became assistant publisher, signaling the
rise of an avowed progressive on L.G.B.T. issues. The parties doubled,
and then tripled in size. AIDS began to kill Timesmen like Robert
Barrios, a copy editor; J. Russell King, who helped edit the front page;
and Schmalz,
who\href{https://www.nytimes3xbfgragh.onion/1993/11/07/obituaries/jeffrey-schmalz-39-times-writer-on-politics-and-then-aids-dies.html}{wrote
dozens of stories about the epidemic}.

The Times had stepped up its coverage of the crisis, but staffers still
took issue with the language the paper used to talk about gays. The gay
and lesbian caucus, formed by employees in 1993, put out regular
bulletins that occasionally addressed language issues.
\href{https://www.nytimes3xbfgragh.onion/2014/03/23/fashion/gays-lesbians-the-term-homosexual.html}{They
wanted to be called gays, not homosexuals.} (The Times had started to
allow the word gay to be used in this context as an adjective in 1987,
but the caucus said it should also be used as a noun.) They also asked
that the newspaper stop referring to ``admitted homosexuals'' when it
really meant openly gay. ``The term connotes confession, criminality and
shame,'' one bulletin said. The point was to get the company to stop
insulting a chunk of its readers --- and its employees.

Progress happened quickly after the caucus revved up, thanks partly to
Sulzberger Jr.'s determination to get The Times out in front of the
issue. The company began offering spousal benefits to gay couples in
1994. And in 1998, Cartwright judged the newsroom ready to handle her
own coming out, typed and printed on yellow paper and tacked onto more
than a dozen bulletin boards throughout the building. ``I have decided
to resolve a longstanding conflict in my life by beginning to live full
time as a woman,'' the letter said. Cartwright, a former copy editor,
remembers hurrying back to her desk after posting the last copy and
walking by the clump of people reading what she'd written. Someone
looked at her, and smiled. She started showing up to work as Donna six
weeks later.

``The only part that was difficult was getting people to stop calling me
`he' and `him,''' she said. She told someone in human resources, who
reminded staff of Cartwright's pronouns, but it just kept happening. One
day she realized that she was just being too polite: ``One time when
someone referred to me as `he' I got visibly angry about it, and that
worked, so I did it a couple of times.'' She didn't have many problems
after that.

The Times gave Cartwright a key to a bathroom on the 11th floor of the
building. After she had gender reassignment surgery, she began to use
the ladies' room.

When I told her that The Times still doesn't have gender neutral
bathrooms, she caught her breath. ``They may have decided that I was a
unicorn,'' she said. She suggested quietly letting the paper's leaders
know that ``The Times is kind of behind the times on this.'' If asking
nicely doesn't work, Cartwright said, we could always get visibly angry.
It worked for her.

\begin{center}\rule{0.5\linewidth}{\linethickness}\end{center}

Image

AIDS appears on the front page of The New York Times in May 1983 (at
bottom) and June 1983 (at top). Inset: John Koblin.

\hypertarget{what-makes-a-front-page-story}{%
\subsection{What Makes a Front-Page
Story?}\label{what-makes-a-front-page-story}}

\hypertarget{by-john-koblin-media-reporter}{%
\subsubsection{By John Koblin, media
reporter}\label{by-john-koblin-media-reporter}}

In July 1981, The New York Times reported that 41 gay men in New York
and California had been diagnosed with a mysterious cancer. It took
almost two years, right before Memorial Day weekend in 1983, for The
Times to finally dedicate front-page space to the story. Under the
headline,
``\href{https://www.nytimes3xbfgragh.onion/1983/05/25/us/health-chief-calls-aids-battle-no-1-priority.html}{Health
Chief Calls AIDS Battle `No. 1 Priority},''' readers of The Times would
learn about a growing catastrophe. There were 558 dead in the United
States. There were more than 1,400 cases reported. The fatality rate was
through the roof.

The front page of The Times has long held a sacred place in the media.
Back then, seven or eight stories, filed from Washington and around the
world, would set the day's agenda. For decades, TV morning-news
producers have used the front page as a guide to mapping out top
stories. And yet roughly 700 editions of the paper had come and gone
before AIDS, quickly turning into a full-fledged crisis, had earned a
spot on Page One. It was never lost on AIDS activists just how vital the
paper was --- and for how long it did not pay serious attention to the
disease.

``Are you kidding?'' emailed Larry Kramer, the activist and writer.
``The front page of The New York Times is the most important real estate
in the world for getting any issue out. As The Times goes, so will every
other news outlet all over the globe.''

So why was The Times seemingly indifferent to the story for so long?

The paper's Science desk, which was responsible for reporting on
outbreaks at the time, was overtaxed in the early 1980s, and it did not
help that identifying the cause of AIDS was a slow burn. ``Science news
was running as fast and freely as Trump is today,'' said Lawrence K.
Altman, the Times reporter who wrote the rare cancer story and still
works at the paper. He cited stories the desk covered about President
Reagan and Pope John Paul II getting shot, along with advances being
made with the artificial heart.

If the burden was then on other sections at the paper, top editors were
less than enthusiastic about surfacing the story. ``There were strong
messages that you got that were not written on any whiteboard,'' said
David W. Dunlap, a reporter in the Metro section at the time. ``You knew
to avoid it. It was a self-reinforcing edict: Don't write about
queers.''

And then there was Kramer's perspective: ``Every friend I had from those
days is now dead because it was no secret that Abe Rosenthal hated
homosexuals,'' he said, referring to the paper's executive editor at the
time. ``This is the chief reason why I hate The New York Times.''

By the time The Times \emph{did} give AIDS front-page attention, it did
so with a bit of a stiff arm. Though public health officials were now
going on the record to discuss the disease's devastation, there was
reluctance to discuss whom it affected most. In that first front-page
story, it took seven paragraphs --- which appeared after the jump, or
inside the paper --- to mention how hard it was hitting gay men.

Max Frankel, the former editorial page editor at the paper, which
operated separately from the newsroom, said this was in keeping with the
paper's ethos at the time: ``They were being squeamish for some
reason,'' Frankel said, speaking of the newsroom. ``Their squeamishness
was actually damaging to the public understanding of what was going
on.''

Frankel, who would take over the newsroom in 1986 as executive editor,
said the editorial department was quicker than the news columns to
describe ``anal intercourse'' as a means of spreading the disease.
``Whatever their rules were down there, we said we're going to do it our
way,'' he said. ``That was the approach.''

On June 16, 1983, AIDS would land on Page One for a second time, under
the headline
``\href{https://www.nytimes3xbfgragh.onion/1983/06/16/nyregion/homosexuals-confronting-a-time-of-change.html}{Homosexuals
Confronting a Time of Change}.'' The story, which ran longer than 3,000
words, examined a wave of anxiety hitting New York gay men, and took a
broader look at what their lives had been like since the Stonewall Riots
in 1969.

By that point, the death toll was fast approaching 600 people.

\begin{center}\rule{0.5\linewidth}{\linethickness}\end{center}

Image

The obituaries of Alvin Ailey, Michel Foucault and Roy Cohn. Inset:
Wesley Morris.

\hypertarget{a-disease-by-any-other-name}{%
\subsection{A Disease By Any Other
Name}\label{a-disease-by-any-other-name}}

\hypertarget{by-wesley-morris-critic-at-large}{%
\subsubsection{By Wesley Morris, critic at
large}\label{by-wesley-morris-critic-at-large}}

Newspapers can be funny places. Writers know. Take the premium placed on
brevity, concision and economy. Saying more with less, it's called. Good
advice. Maddening rigorousness. Depends on the day. But sometimes it
hits a snag. In order to say ``SYTYCD,'' for instance, it's probably
helpful to use ``So You Think You Can Dance?'' first. The show needs a
proper name before a concise one. For three decades, though, newspapers
--- this newspaper --- made an exception. There's nothing more concise
than AIDS. It names a larger disease: acquired immune deficiency
syndrome. But, really, ``AIDS'' gets the job done. Only, it was an
improper name. And so it often went unused, since it named an
impropriety.

When the choreographer Alvin Ailey died in December of 1989, his
obituary made the front page. The headline was accurate but bloodless:
``\href{https://www.nytimes3xbfgragh.onion/1989/12/02/obituaries/alvin-ailey-a-leading-figure-in-modern-dance-dies-at-58.html}{Leading
figure in modern dance}.'' (Cristal: leading bubbly wine of France.) But
it was the cause of death that was cause for concern: ``Dr. Albert
Knapp, Mr. Ailey's physician, attributed his death to terminal blood
dyscrasia, a rare disorder that affects the bone marrow and red blood
cells.'' AIDS would have been the more economic choice. Ailey died of
AIDS. But terminal blood dyscrasia? Rare disorder? They tell a story ---
of tremendous suffering, certainly, but of intense trepidation, too.

Search for the Times obituary of anyone who we now know died of AIDS
during the 1980s (and beyond). You'll often find they died of something
else. ``Hospitalized earlier this month for a neurological disorder, but
the cause of his death was not immediately disclosed''
(\href{https://archive.nytimes3xbfgragh.onion/www.nytimes3xbfgragh.onion/books/00/12/17/specials/foucault-obit.html?_r=1}{Michel
Foucault, 1984}). ``Died at 1:30 A.M. of viral encephalitis, an
inflammation of the brain, after having slipped into a coma several days
ago''
(\href{https://www.nytimes3xbfgragh.onion/1986/05/31/obituaries/perry-ellis-fashion-designer-dead.html}{Perry
Ellis, 1986}). Saying less with more.

Medically, it's not really AIDS that kills you. AIDS acts as a doorjamb
for other diseases to traipse in and ravage a body. But the average
obituary noted what traipsed but omitted what made the traipsing
possible. ``AIDS-related'' rarely appeared --- nor did coverage of the
deaths of black people and Latinos.

Occasionally, things would be called as they were. When the incandescent
light of underground theater Ethyl Eichelberger
\href{https://www.nytimes3xbfgragh.onion/1990/08/14/obituaries/ethyl-eichelberger-performer-45-creator-of-a-gallery-of-characters.html}{died
in 1990}, one of his leading ladies ``said that he had AIDS and had
committed suicide by slashing his wrists.''

The exceptions to euphemism are notable: Women, for instance, even a
trans woman, like the actress Elizabeth Eden,
\href{https://www.nytimes3xbfgragh.onion/1987/10/01/obituaries/elizabeth-eden-transsexual-who-figured-in-1975-movie.html}{who
died in 1987}. When the art critic, actress and advice columnist Cookie
Mueller
\href{https://www.nytimes3xbfgragh.onion/1989/11/15/obituaries/cookie-mueller-dead-actress-and-writer-40.html}{died
at 40}, in 1989, her brief notice in the paper named the cause as
``pneumonia resulting from AIDS,'' which was the same explanation for
Eden's cause of death.

Newspapers can be funny places. So can people. The omission of AIDS
wasn't necessarily an editorial judgment. It's often up to the families
and friends and physicians of the deceased to state how they died. In a
real power move, the dying could name their cause of death. To see Roy
Cohn die in ``Angels in America'' is to know that. Liver cancer was his
preferred cause of death, not the complications from AIDS that actually
killed him. That was for homosexuals, a label he swore he was too good
for. AIDS killed people. But it also threatened to kill pride. Families,
friends and co-workers died of embarrassment. No one wanted to go to an
AIDS funeral. Hence the euphemisms and the workarounds. There was only
so much they could cover up, of course, since lots of people went to too
many AIDS funerals.

And so a disease of the body spawned a disease of good --- appallingly
good --- manners. The shame of AIDS endured, in part, because we were
too uptight, too judgmental, too fearful to call it by its name. But
this paper --- and to some extent the culture --- has shaken free of
those old misgivings. Being on the other side of a crisis helps with
that. When the theater composer Michael Friedman died of AIDS-related
causes last year at 41, the paper not only said so, it devoted
\href{https://www.nytimes3xbfgragh.onion/2017/10/11/theater/michael-friedman-aids-death-theater.html}{a
long and searching story}, by Michael Paulson, to the matter of how and
why.

Oskar Eustis, the artistic director of the Public Theater, told Paulson
that Friedman's death was ``a real warning shot across the bow for
anybody who thinks this disease isn't deadly any more.'' People were
speaking of distress --- and they were in distress about a disease that
still kills. Writing
\href{https://www.nytimes3xbfgragh.onion/2017/06/06/magazine/americas-hidden-hiv-epidemic.html}{in
The New York Times Magazine} last year, Linda Villarosa detailed how the
disease is still ravaging black gay and bisexual American men at rates
higher than anyplace else on the planet. And when she did, she said
AIDS.

\begin{center}\rule{0.5\linewidth}{\linethickness}\end{center}

Image

The Mine Shaft, closed by the New York City Department of Health in Nov.
1985. Inset: Adam Nagourney.Credit...Yvonne Hemsey/Getty Images

\hypertarget{on-sex-clubs--and-how-to-cover-them}{%
\subsection{On Sex Clubs --- And How to Cover
Them}\label{on-sex-clubs--and-how-to-cover-them}}

\hypertarget{by-adam-nagourney-los-angeles-bureau-chief}{%
\subsubsection{By Adam Nagourney, Los Angeles bureau
chief}\label{by-adam-nagourney-los-angeles-bureau-chief}}

It was a world that thrived in the shadows of New York --- late at
night, hidden on dark streets behind barely-marked doorways, typically
guarded by a man sitting on a stool under a light bulb. What took place
inside ranged from questionable to outright illicit: the full-service
bar that kept serving drinks after 4 a.m., the open use of cocaine and
marijuana and, in many cases, the anonymous and often marathon sexual
encounters among gay men. Sex clubs, bathhouses, back rooms, movie
theaters, after-hours bars, dance clubs --- places like Alex in
Wonderland and the Mine Shaft and the New St. Marks Baths, all part of
New York in the 1970s and early 1980s. This was a culture that was known
to some of gay New York, but arguably not most of it. And it certainly
was not known to most of straight New York.

The AIDS epidemic changed all that. Almost overnight, as panic spread
and Governor Mario M. Cuomo issued regulations in late 1985 to close
down establishments where unsafe sexual activity took place, this
underground world came crashing to the surface. A battery of city
inspectors, police officers and reporters went in undercover to see for
themselves what was going on behind those doors --- and shared their
findings with the public. ``The Case for Closing Bathhouses: Night Visit
by Post Reporter Reveals Shocking Evidence,'' read a New York Post
headline over a 1985 story that described, in considerable detail, what
the reporter saw and heard as he moved through the dark warren of a
bathhouse. It was part of stream of columns and editorials --- not to
mention grainy video from local news cameras that had been secreted into
the clubs --- that was turning up the pressure on public officials.

And while The New York Times was sluggish on reporting the first
stirrings of the AIDS epidemic, it quickly jumped on the train as the
story became something of a frenzy: voyeuristic and yet significant at
the same time. ``Guests paid a \$12 `membership' fee and were asked to
sign a form pledging to engage only in `safe sex' involving no exchange
of bodily fluids,'' read
\href{https://www.nytimes3xbfgragh.onion/1985/11/09/nyregion/at-homosexual-establishments-a-new-climate-of-caution.html}{a
1985 Times report} on a club known as the Hell Fire that catered to a
straight and gay crowd with an interest in sadomasochism. ``Signs on the
walls listed the club rules, which included `no bullwhips, electric
prods or animals' and `no touching without permission.'''

These were uncomfortable times for many of the people drawn into this:
There were the city inspectors who spent hours compiling firsthand
reports that were detailed if at times unnerving. (``Two of the
inspectors said they heard sounds of whipping and moaning, but did not
investigate `for reasons of personal
safety,'\href{https://www.nytimes3xbfgragh.onion/1985/11/08/nyregion/city-closes-bar-frequented-by-homosexuals-citing-sexual-activity-linked-to-aids.html}{a
1985 Times story said}.) There were the elected officials, who had to
explain --- and justify --- evolving policies as they navigated a
politically fraught epidemic transmitted by sexual activity, with
terrifying questions around every corner.

It was challenging for news organizations as well, drawn to the story
for legitimate reasons (a health crisis) and perhaps less noble reasons
(sensationalism) as they struggled with just how explicitly detailed the
reports needed to be. And no less discomfited were many in the city's
gay and lesbian community, concerned that raising the curtain on a world
that most people did not know existed could threaten the gay rights
movement after a decade of progress. Some gay leaders decried the city
crackdown as an assault on privacy. ``Consensual sexual activity by
adults out of public view should always be beyond the eye and the arm of
government,'' said Thomas B. Stoddard, the executive director of the
Lambda Legal Defense and Education Fund, in 1988. Others considered it a
legitimate response to a health emergency. ``Is compulsive sexuality
freedom?''' Jim Fouratt, one of the city's earliest gay activists, said
a few weeks before the city cracked down. ``I would argue it's not. All
we got was a lot more alienated sexually, and a lot more disease.''

Cuomo, whom I covered at the time as a Daily News reporter based in
Albany, would seem at times astonished --- even ashen-faced --- as he
learned about this side of New York from inspectors and aides struggling
to advise the governor about which kind of sexual practices the state
might regulate. Down in the city, Mayor Edward I. Koch was already
facing criticism that he was slow to act in combating the epidemic; gay
political leaders contended that it was because the mayor, who never
married and lived in Greenwich Village, was gay. (At various points in
his life, Koch said he was not, or would not answer the question.)

As the mayor and governor were at odds about what to do (sound
familiar?), Cuomo's health advisers, after much debate, recommended
shutting down establishments that permitted unsafe-sex practices. City
Hall pushed back, as many of Koch's aides argued that a crackdown would
simply push this kind of behavior into darker corners, making it harder
to regulate. Cuomo finally moved, essentially ignoring the mayor as he
issued state regulations that would close clubs that permitted these
kinds of activities, and leaving it to cities to enforce. Koch resisted
at first, contending the order was poorly drafted and difficult to
implement: He was obviously aware of how many gay activists viewed this
as a civil rights issue, a setback at a time when gay rights seemed to
be on the advance.

But Koch's resistance did not last for long: On November 7, 1985, the
city closed down the Mine Shaft, the start of a march of enforcement,
reported with the exquisitely detailed stories about what had led
authorities to act. ``In graphic depositions written by city inspectors,
a portrait emerged of a dark place with black walls, back rooms, open
cubicles without doors and the accouterments of sadomasochism,''
\href{https://www.nytimes3xbfgragh.onion/1985/11/08/nyregion/city-closes-bar-frequented-by-homosexuals-citing-sexual-activity-linked-to-aids.html}{wrote
one story}. ``They reported seeing many patrons engaging in anal
intercourse and fellatio --- the `high risk' sexual practices cited in
the state rules.'' And that was from The Times, which was relatively
restrained compared with the rest of the coverage. New York would never
quite be the same.

\begin{center}\rule{0.5\linewidth}{\linethickness}\end{center}

Image

The first article in Styles addressing AIDS, May 30, 1983.

\hypertarget{in-the-days-before-thursgay-styles}{%
\subsection{In the Days Before `Thursgay'
Styles}\label{in-the-days-before-thursgay-styles}}

\hypertarget{by-denny-lee-senior-staff-editor-for-styles}{%
\subsubsection{By Denny Lee, senior staff editor for
Styles}\label{by-denny-lee-senior-staff-editor-for-styles}}

The Styles section of The New York Times is where many readers first
learned of
\href{https://www.nytimes3xbfgragh.onion/1993/09/26/style/bustin-stereotypes.html}{gay
rodeos}, the drag queen
\href{https://www.nytimes3xbfgragh.onion/1993/04/18/style/paris-has-burned.html}{Angie
Xtravaganza} and the
\href{https://www.nytimes3xbfgragh.onion/1996/11/24/style/right-off-the-runway-and-into-the-salon.html}{propensity
of hair dressers} to wear leather shirts, silk gazar tunics and other
outré garb typically reserved for the catwalk.

In the mid-2000s, the Thursday section was even nicknamed
``\href{http://gawker.com/210347/thursgay-styles-my-what-big-package-you-have}{Thursgay
Styles}'' by the creative underclass, for its effete coverage of men's
fashion and unabashed objectification of the male body. Maybe it was the
mandate to cover fashion, night life and subculture, but by the time the
Styles section ran its first report on a
\href{https://www.nytimes3xbfgragh.onion/2002/08/18/us/times-will-begin-reporting-gay-couples-ceremonies.html}{same-sex
commitment celebration in 2002} (two years before same-sex marriage was
legalized in Massachusetts), its gay-centric reputation was hard to
dispute.

But that wasn't always the case.

At the dawn of the Reagan years, when ``Dynasty'' introduced one of the
first gay characters on prime-time TV, but before Madonna released
``Like a Virgin,'' the Styles of the Times page (yes, it was a single
page back then) ran just three articles between 1981 and 1983 that
examined gay life in general. That's not counting, of course, the (often
closeted) gay characters who flitted in and out of ``Notes on Fashion,''
a weekly column by
\href{https://www.nytimes3xbfgragh.onion/1989/01/23/obituaries/john-duka-39-former-times-writer-on-fashion-and-art.html}{John
Duka}, a reporter who went on to help start the publicity firm KCD,
before dying from AIDS-related complications in 1989.

Using language that betrays the era's ignorance and discomfort with gay
sex, the first Styles piece in that time period, published on May 30,
1983, explored the
``\href{https://www.nytimes3xbfgragh.onion/1983/05/30/style/facing-the-emotional-anguish-of-aids.html}{special
emotional difficulties of AIDS victims}.'' While other sections of the
paper covered the science of the disease and, to a much lesser extent,
the political ramifications, Styles addressed its emotional toll. In
addition to the shock and stigma of a diagnosis, the piece also touched
on the fear that people with H.I.V. could somehow ``spread AIDS to
family, friends or partners.'' A second article covered the second
annual convention of what is now known as PFLAG (the Parents, Families
and Friends of Lesbians and Gays) in New York, ``where
\href{https://www.nytimes3xbfgragh.onion/1983/10/10/style/for-homosexuals-parents-srength-in-community.html}{125
parents of homosexuals} from all over the country gathered'' to combat
prejudice.

But it is a third article,
``\href{https://www.nytimes3xbfgragh.onion/1983/07/18/style/relationships-of-herpes-aids-and-fear-of-sex.html}{Of
Herpes, AIDS and Fear Of Sex},'' that reveals the most about the Style
section's coverage of queer people in the early 1980s. Published on July
18, 1983, under the rubric ``Relationships,'' it begins with a pointed
question: Has the fear of AIDS and herpes led to a decline in
promiscuity? Granted, the thesis does adhere to a certain editorial
logic. It's the kind of ``if true'' story that editors --- such as I (an
editor at Styles) --- might propose based on a hunch: If AIDS has led to
a fear of sex, then it stands to reason that some people, especially gay
men, are having less sex.

The problem was, there was no definitive evidence to support that claim,
only the opinions of public health officials who, when asked, were quick
to poke holes in the story. They invoked the word ``speculation'' three
times. A hospital supervisor in Brooklyn was even quoted saying the
opposite: ``We are continuing to see tremendous amounts of sexual
activity among adolescents.''

So why was this piece published in the first place? One could argue that
The Times, which proudly bills itself as a family newspaper, is
puritanical to a fault. But it also has to do with simple arithmetic: In
1983, there were few, if any, openly gay reporters on staff, and hardly
anyone to challenge stereotypes about gay people. It would take until
the next decade before the floodgates would open at The Times, and
openly gay men and women would assume positions of leadership in the
newsroom. But even that, more than two decades later, hasn't stopped
many gay people from debating what, exactly, Styles --- and the rest of
the Times --- gets right about their lives.

\begin{center}\rule{0.5\linewidth}{\linethickness}\end{center}

\emph{\textbf{Read more:}}

\emph{\href{https://www.nytimes3xbfgragh.onion/interactive/2018/04/17/t-magazine/new-york-1980s-culture.html}{New
York City, 1981-1983: 36 Months That Changed the Culture}}

\emph{\href{https://www.nytimes3xbfgragh.onion/2018/04/19/t-magazine/keith-haring-tina-chow-aids-resurrected.html}{Four
Geniuses, Gone to AIDS, as They Might Be Today}}

\emph{\href{https://www.nytimes3xbfgragh.onion/2018/04/17/t-magazine/24-hours-new-york-city-1980s-life.html}{What
New York Was Like in the Early '80s --- Hour by Hour}}

Advertisement

\protect\hyperlink{after-bottom}{Continue reading the main story}

\hypertarget{site-index}{%
\subsection{Site Index}\label{site-index}}

\hypertarget{site-information-navigation}{%
\subsection{Site Information
Navigation}\label{site-information-navigation}}

\begin{itemize}
\tightlist
\item
  \href{https://help.nytimes3xbfgragh.onion/hc/en-us/articles/115014792127-Copyright-notice}{©~2020~The
  New York Times Company}
\end{itemize}

\begin{itemize}
\tightlist
\item
  \href{https://www.nytco.com/}{NYTCo}
\item
  \href{https://help.nytimes3xbfgragh.onion/hc/en-us/articles/115015385887-Contact-Us}{Contact
  Us}
\item
  \href{https://www.nytco.com/careers/}{Work with us}
\item
  \href{https://nytmediakit.com/}{Advertise}
\item
  \href{http://www.tbrandstudio.com/}{T Brand Studio}
\item
  \href{https://www.nytimes3xbfgragh.onion/privacy/cookie-policy\#how-do-i-manage-trackers}{Your
  Ad Choices}
\item
  \href{https://www.nytimes3xbfgragh.onion/privacy}{Privacy}
\item
  \href{https://help.nytimes3xbfgragh.onion/hc/en-us/articles/115014893428-Terms-of-service}{Terms
  of Service}
\item
  \href{https://help.nytimes3xbfgragh.onion/hc/en-us/articles/115014893968-Terms-of-sale}{Terms
  of Sale}
\item
  \href{https://spiderbites.nytimes3xbfgragh.onion}{Site Map}
\item
  \href{https://help.nytimes3xbfgragh.onion/hc/en-us}{Help}
\item
  \href{https://www.nytimes3xbfgragh.onion/subscription?campaignId=37WXW}{Subscriptions}
\end{itemize}
