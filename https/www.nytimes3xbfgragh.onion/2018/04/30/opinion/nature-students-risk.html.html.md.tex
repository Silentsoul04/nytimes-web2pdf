Sections

SEARCH

\protect\hyperlink{site-content}{Skip to
content}\protect\hyperlink{site-index}{Skip to site index}

\href{https://myaccount.nytimes3xbfgragh.onion/auth/login?response_type=cookie\&client_id=vi}{}

\href{https://www.nytimes3xbfgragh.onion/section/todayspaper}{Today's
Paper}

\href{/section/opinion}{Opinion}\textbar{}Nature Is Risky. That's Why
Students Need It.

\url{https://nyti.ms/2KqAaS7}

\begin{itemize}
\item
\item
\item
\item
\item
\end{itemize}

Advertisement

\protect\hyperlink{after-top}{Continue reading the main story}

Supported by

\protect\hyperlink{after-sponsor}{Continue reading the main story}

\href{/section/opinion}{Opinion}

\href{/column/on-campus}{On Campus}

\hypertarget{nature-is-risky-thats-why-students-need-it}{%
\section{Nature Is Risky. That's Why Students Need
It.}\label{nature-is-risky-thats-why-students-need-it}}

By Heather E. Heying

Dr. Heying is an evolutionary biologist and a former professor at
Evergreen State College.

\begin{itemize}
\item
  April 30, 2018
\item
  \begin{itemize}
  \item
  \item
  \item
  \item
  \item
  \end{itemize}
\end{itemize}

\includegraphics{https://static01.graylady3jvrrxbe.onion/images/2018/04/30/opinion/30heying/merlin_137431521_488c631e-9273-4687-8425-b11d6409ced7-articleLarge.jpg?quality=75\&auto=webp\&disable=upscale}

Nature is unscripted and hard to predict. Having recently discovered
this reality, Penn State has decided that its 98-year-old, student-led
Outing Club shall no longer be allowed to go on outings. Citing the high
risk of remote environments and poor cellphone service, the university
is recommending that the club restrict its offerings to films and
speakers. Students are being funneled into engaging only in previously
vetted human constructions.

The students of the Outing Club
\href{http://sites.psu.edu/outingclub/}{are fighting back} --- and good
for them. Driven to explore both nature and risk, they are well on their
way to adulthood, which means knowing how to resist injunctions that are
more protection against future lawsuits than they are in service of the
students themselves.

Not so long ago universities took on the authority of parents, in loco
parentis. Now that many modern parents have absolved themselves of the
responsibility of raising mature, bold, responsible adults, it seems
universities have followed suit. At Penn State, the Outing Club wasn't
the only one on the chopping block --- caving and scuba diving are
reportedly out as well.

\begin{quote}
\end{quote}

In my 15 years as a professor at Evergreen State College, I led field
trips to Panama and Ecuador that sometimes lasted months. My students
and I explored archipelagos and jungles, coral reefs and colonial
cities. And I experienced and heard tell of many dangerous situations.

On one trip alone, in 2016, which my husband and fellow professor Bret
Weinstein and I led together with 30 undergraduates (and our own two
children), there were life-threatening emergencies involving a tree fall
in the Amazon, a boat accident in Galápagos and, later, a serious
earthquake in coastal Ecuador. Everyone made it home, but why take such
risks? Is studying the politics of land use, the cultures of early
Americans or territoriality in butterflies worth it?

Over the course of several trips, I saw students rise to challenges in
ways that they simply could not at home. I purposefully sought out field
sites that were remote not just because nature is more interesting and
intact in such places --- more lianas climbing their way up to the
light, more vine snakes mimicking those same lianas --- but also because
encountering nature in its least disturbed state often comes at the
``cost'' of having no connection to the outside world. Far from the
virtual eyes that document our every move, people are revealed, to
themselves and to others.

In the field, I watched students descend into their own darkness,
depression gripping them, and I watched as they emerged from it,
stronger and more grounded. Romantic ideas of the jungle disappear with
the reality of constant sweat and biting insects, and the realization
that to see charismatic animals do interesting things, you have to get
out there and fade into the forest, and then wait patiently for it to
come back alive around you.

Some hate it. They cannot abide the lack of control, the discovery that
nature is not a nature documentary. Most, though, find hidden strength
and unanticipated freedom.

One afternoon on a tributary of the Amazon, a river with broad clay
banks, a gloriously messy mud fight broke out between friends, while a
few of us standing on the banks cheered them on. They were filthy and
raucous and there were no books in sight. Students were exploring
boundaries of all sorts, and it looked a lot like education to me.

Another evening, the students tried to give research presentations under
a corrugated metal roof but a squall came up, and the rain was pounding
the roof so noisily that we had to reschedule. We dispersed, some taking
the opportunity to catch up on sleep, some wandering off into the forest
to explore the warm, wet embrace of a tropical jungle at night. If
education is, in part, preparation for an unpredictable and shifting
world, teaching courage and curiosity ought to be a priority.

On domestic field trips in remote locations, my classes did field
exercises and even, sometimes, endured lectures, but we also explored
without explicit goals, cooked and shared meals, sat around campfires
and told stories. In eastern Washington's scablands, high winds can make
standing on mesas dangerous, and climbing up to them, through scree
fields, is a challenge, too. Students unaccustomed to physical exertion,
injured and on crutches, or just born-and-bred in the city and not
familiar with how to navigate a slope of jagged, loose rocks, all faced
the fields and took them on. On the scree fields of eastern Washington,
facing an unpredictable and shifting world is a literal endeavor.

One brave student from the 2016 trip was injured in the boat accident in
the Galápagos. The boat was destroyed, but she soldiered on. Then, three
weeks later, she was nearly crushed when the five-story unreinforced
masonry hotel she was staying in collapsed during a major earthquake.
She was lucky: Almost everyone in the building died. She and another
student dug themselves out of the rubble.

Her recovery was long and painful. She --- a serious ballet dancer ---
was wheelchair-bound for months. After a year of surgeries, crutches and
other frustrations, she caught me off guard. Despite everything, she
said, she would do it all again. The trip had been that important to
her.

In advance of these study-abroad trips, I led long conversations about
risk, how to assess it, what we perceive our own relationship with it to
be. We discussed how risk is different in landscapes that haven't been
rendered safe by liability lawsuits and in which medical help is a very
long way away. We talked about the hidden hazards of the jungle ---
rising water, tree falls --- compared with the familiar ones, like
snakes and big cats, that people are primed to be scared of. In the
tropical lowland rain forest --- the jungle --- you might get stuck in
deep mud and perhaps need help to get out. Look before you reach for a
tree for leverage. Some trees defend themselves with nasty spikes, and a
branch might be crawling with bullet ants, so named for the intense
experience of being stung by one.

But it turns out that risk and potential go hand in hand. We need to let
children, including college students, risk getting hurt. Protection from
pain guarantees weakness, fragility and greater suffering in the future.
The discomfort may be physical, emotional or intellectual --- My ankle!
My feelings! My worldview! --- and all need to be experienced to learn
and grow.

Advertisement

\protect\hyperlink{after-bottom}{Continue reading the main story}

\hypertarget{site-index}{%
\subsection{Site Index}\label{site-index}}

\hypertarget{site-information-navigation}{%
\subsection{Site Information
Navigation}\label{site-information-navigation}}

\begin{itemize}
\tightlist
\item
  \href{https://help.nytimes3xbfgragh.onion/hc/en-us/articles/115014792127-Copyright-notice}{©~2020~The
  New York Times Company}
\end{itemize}

\begin{itemize}
\tightlist
\item
  \href{https://www.nytco.com/}{NYTCo}
\item
  \href{https://help.nytimes3xbfgragh.onion/hc/en-us/articles/115015385887-Contact-Us}{Contact
  Us}
\item
  \href{https://www.nytco.com/careers/}{Work with us}
\item
  \href{https://nytmediakit.com/}{Advertise}
\item
  \href{http://www.tbrandstudio.com/}{T Brand Studio}
\item
  \href{https://www.nytimes3xbfgragh.onion/privacy/cookie-policy\#how-do-i-manage-trackers}{Your
  Ad Choices}
\item
  \href{https://www.nytimes3xbfgragh.onion/privacy}{Privacy}
\item
  \href{https://help.nytimes3xbfgragh.onion/hc/en-us/articles/115014893428-Terms-of-service}{Terms
  of Service}
\item
  \href{https://help.nytimes3xbfgragh.onion/hc/en-us/articles/115014893968-Terms-of-sale}{Terms
  of Sale}
\item
  \href{https://spiderbites.nytimes3xbfgragh.onion}{Site Map}
\item
  \href{https://help.nytimes3xbfgragh.onion/hc/en-us}{Help}
\item
  \href{https://www.nytimes3xbfgragh.onion/subscription?campaignId=37WXW}{Subscriptions}
\end{itemize}
