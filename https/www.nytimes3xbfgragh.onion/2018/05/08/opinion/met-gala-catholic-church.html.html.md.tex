Sections

SEARCH

\protect\hyperlink{site-content}{Skip to
content}\protect\hyperlink{site-index}{Skip to site index}

\href{https://myaccount.nytimes3xbfgragh.onion/auth/login?response_type=cookie\&client_id=vi}{}

\href{https://www.nytimes3xbfgragh.onion/section/todayspaper}{Today's
Paper}

\href{/section/opinion}{Opinion}\textbar{}Make Catholicism Weird Again

\url{https://nyti.ms/2FWOzSP}

\begin{itemize}
\item
\item
\item
\item
\item
\item
\end{itemize}

Advertisement

\protect\hyperlink{after-top}{Continue reading the main story}

Supported by

\protect\hyperlink{after-sponsor}{Continue reading the main story}

\href{/section/opinion}{Opinion}

\hypertarget{make-catholicism-weird-again}{%
\section{Make Catholicism Weird
Again}\label{make-catholicism-weird-again}}

\href{https://www.nytimes3xbfgragh.onion/by/ross-douthat}{\includegraphics{https://static01.graylady3jvrrxbe.onion/images/2018/04/03/opinion/ross-douthat/ross-douthat-thumbLarge.png}}

By \href{https://www.nytimes3xbfgragh.onion/by/ross-douthat}{Ross
Douthat}

Opinion Columnist

\begin{itemize}
\item
  May 8, 2018
\item
  \begin{itemize}
  \item
  \item
  \item
  \item
  \item
  \item
  \end{itemize}
\end{itemize}

\includegraphics{https://static01.graylady3jvrrxbe.onion/images/2018/05/09/opinion/09douthatSUBB/09douthatSUBB-articleLarge.jpg?quality=75\&auto=webp\&disable=upscale}

In 1904, during a debate in France over the anticlerical government's
takeover of church property, a young Marcel Proust wrote
\href{https://rorate-caeli.blogspot.com/2015/01/the-death-of-cathedrals-and-venerable.html}{an
essay for Le Figaro} inviting readers to imagine a future in which the
Catholic Church vanished completely from his country's memory, leaving
only the bones of French cathedrals as its monuments.

Then he further imagined the cultured elites of some future France
rediscovering the texts and chants and rubrics of Catholic liturgy, and
in a spasm of enraptured aestheticism, restoring the cathedrals and
training actors to recreate the Tridentine Rite Mass. In his vision,
like devotees of Wagner making pilgrimage, ``caravans of swells make
their way to \ldots{} Amiens, Chartres, Bourges, Laon, Rheims, Rouen,
Paris,'' and inside France's Gothic churches ``they experience the
feeling they once sought in Bayreuth \ldots{} enjoying a work of art in
the very setting that had been built for it.''

But of course the recreated Catholic liturgy and revived Catholic
aesthetic would never be the real thing; the actors might know their
roles, and the incense might waft thick, but attendees could ``only ever
be curious dilettantes; try as they might, the soul of times past does
not dwell within them.''

Proust's essay, lately translated by Catholic traditionalists, came to
mind while watching the beautiful and blasphemous spectacle at the Met
Gala on Monday night, where a parade of stars and fashionistas swanned
about in costumes inspired by the aesthetics of Catholicism, while a
wide variety of genuinely Catholic articles, from vestments to tiaras,
were displayed in a Met exhibit titled
``\href{https://www.metmuseum.org/exhibitions/listings/2018/heavenly-bodies}{Heavenly
Bodies: Fashion and the Catholic Imagination}.''

Like Proust's
``\href{https://rorate-caeli.blogspot.com/2015/01/the-death-of-cathedrals-and-venerable.html}{caravans
of swells}'' attending liturgical performances, the attendees at the Met
were paying a cultural homage to the aesthetic riches of the Roman
Church --- when, of course, they weren't sexing them up for shock value.
But the spectacle was not exactly Proust's prophecy come to life,
because unlike in his thought experiment, Catholicism today remains a
living faith --- weakened but hardly gone, with as complicated a
relationship to its own traditions as any lapsed-Catholic museum curator
or celebrity dressing up as the
\href{http://wwd.com/eye/lifestyle/zendaya-versace-joan-arc-2018-met-gala-1202667880/}{Maid
of Orleans}.

This complication is apparent in the Catholic response to the Met Gala
itself, which consisted of an institutional blessing for the spectacle
--- not just Cardinal Timothy Dolan opening the museum exhibit, but the
Sistine Chapel Choir performing for the swells and starlets in the
evening --- followed by an angry Catholic social-media backlash against
the evening's various impieties. When a living faith gets treated like a
museum piece, it's hard for its adherents to know whether to treat the
moment as an opportunity for outreach or for outrage.

\href{https://www.nytimes3xbfgragh.onion/newsletters/opiniontoday?action=click\&module=Intentional\&pgtype=Article}{\emph{{[}Receive
the day's most urgent debates right in your inbox by subscribing to the
Opinion Today newsletter.{]}}}

But the complexity runs much deeper, because to the extent that part of
the Proustian prophecy has come true, to the extent that elements of the
Catholic tradition have turned into archaic curiosities to be
rediscovered by aesthetes and donned lewdly by Rihanna, the choices made
by the church's own leaders have played as much of a role as the
anticlericalism of Proust's era.

It was the church's own leadership that decided, in the years following
the Second Vatican Council, that the attachment to the church as culture
had become an impediment to the mission of preaching the gospel in the
modern world. It was the leadership that embraced a different approach,
in which Catholic Christianity would seek to enter more fully into
modern culture, adopting its styles and habits --- modernist and even
brutalist church architecture, casual dress, guitar music, a general
suburban and Protestant affect, etc. --- in order to effectively
transform it from within. It was the leadership that decided that much
of what Proust depicted as Catholicism's cultural glory --- the old Mass
above all, but also a host of customs and costumes and rituals ---
needed to be retired in order to reach people in a more disenchanted
age.

This idea was hardly absurd in theory; from Roman Empire days through
missionary efforts, Christianity had often advanced through
inculturation, importing a consistent religious message into varying
cultural forms.

But Catholicism's attempt to do the same with modern culture since the
1960s has largely seemed to fail. The secular culture welcomed the
church's Protestantization and demystification and even secularization,
praised the bishops and theologians who pursued it, and then simply
pocketed the concessions and ignored the religious ideas those
concessions were supposed to advance. Meanwhile, that same secular world
maintained a consistent fascination, from ``The Exorcist'' down to,
well, the Met Gala, with all the weirder parts of Catholicism that were
supposedly a stumbling block to modernity's conversion.

This failure, and how exactly Catholics should interpret it, helps frame
the debates roiling the church
\href{http://www.simonandschuster.com/books/To-Change-the-Church/Ross-Douthat/9781501146923}{in
the age of Pope Francis}. One theory is that the evidence of the last 50
years suggests that modern culture is inherently anti-religious or
anti-Catholic in some abiding way, which means the attempt to adopt its
cultural forms and ``accompany'' its denizens will inevitably end in
dissolution for the church itself.

Thus the only plausible approach for Catholicism is to offer itself, not
as a chaplaincy within modern liberalism, but as a full alternative
culture in its own right --- one that reclaims the inheritance on
display at the Met, glories in its own weirdness and supernaturalism,
and spurns both accommodations and entangling alliances (including the
ones that
\href{http://theweek.com/articles/771381/what-catholics-have-sacrificed-by-allying-republican-evangelicals}{conservative
Catholics have forged} with libertarian-inflected right-wing political
movements).

The other view is that in fact inculturation has not gone far enough,
that the church may have changed its liturgy and costumes, but it's
still held back by its abstract dogmas and arid legalisms, and that one
final great leap into modernity, a renewed commitment to accompaniment
and understanding and adaptation, is necessary for the church to gain
what it sought when it began its great demystification project 50 years
ago.

As pontiff, Francis has been on both sides of these debates. The
radicalism of his economic and ecological vision, often portrayed as
simply liberal, actually represents a kind of left-leaning pessimism
that arguably points backward to the strenuous critiques of modernity
issued by 19th-century popes. And at times this radicalism has been
matched by his willingness to join conservative members of his flock in
culture war --- as recently in the
\href{https://www.nytimes3xbfgragh.onion/2018/04/28/opinion/sunday/alfie-evans-and-the-experts.html}{Alfie
Evans case in England}, where the pope ended up in a public conflict
with the
\href{https://www.nationalreview.com/2018/05/alfie-evans-catholic-teaching-supports-wishes-of-parents/}{more
culturally accommodating sort of Catholic} over whether to defer to
medical professionals and deprive a brain-damaged toddler of oxygen
because his life was judged no longer worth sustaining.

But only at times; on many other fronts, the Francis era has been a
springtime for accommodation and inculturation, and especially for the
secularizing and Protestantizing German Catholicism that
\href{https://www.wsj.com/articles/germanys-liberal-bishops-gain-influence-under-pope-francis-1525431601}{helped
forge the original revolution of the 1960s}, and whose leaders believe
that only further modernization can refill their empty churches.

Under German influence, but with the pope's implicit blessing, Catholic
rules on divorce and
\href{https://cruxnow.com/global-church/2018/05/03/pope-wants-germans-to-find-unanimous-solution-on-intercommunion/}{now
perhaps intercommunion} may be joining the Latin Mass and meatless
Fridays on the altar of sacrifices to the culture of the modern world.

Meanwhile in the case of the opulent style of Catholic fashion on
display at the Met Gala, it is very clear where Francis stands. As Tara
Isabella Burton points out in an
\href{https://www.vox.com/2018/5/7/17306388/the-real-controversy-at-the-heart-of-catholic-fashion}{astute
piece} for Vox, it's the pope's traditionalist adversaries who are more
likely to don the sort of ``heavenly'' garb being feted and imitated at
the Met --- while from his own simple choice of dress to his constant
digs at overdressed clerics and fancy traditionalists, the pope believes
that baroque Catholicism belongs in a museum or at a costume gala, and
that the church's future lies in the simple, the casual, the austere and
the plain.

For this, as for his doctrine-shaking innovations, Francis has won
admiring press. But as with the last wave of Catholic revolution, there
is little evidence that the modernizing project makes moderns into
Catholics. (The latest
\href{http://news.gallup.com/poll/232226/church-attendance-among-catholics-resumes-downward-slide.aspx}{Gallup
data}, for instance, shows American Mass attendance declining faster in
the Francis era.)

Instead, the quest for accommodation seems to encourage moderns to
divide their sense of what Catholicism represents in two --- into an Old
Church that's frightening and fascinating in equal measure, and a New
Church that's a little more liked but much more easily ignored.

Francis and other would-be modernizers are right, and have always been
right, that Catholic Christianity should not trade on fear. But a
religion that claims to be divinely established cannot persuade without
a lot of fascination, and far too much of that has been given up,
consigned to the museum, as Western Catholicism has traced its slow
decline.

Here the Met Gala should offer the faith from which it took its theme a
little bit of inspiration. The path forward for the Catholic Church in
the modern world is extraordinarily uncertain. But there is no plausible
path that does not involve more of what was displayed and appropriated
and blasphemed against in New York City Monday night, more of what once
made Catholicism both great and weird, and could yet make it both again.

Advertisement

\protect\hyperlink{after-bottom}{Continue reading the main story}

\hypertarget{site-index}{%
\subsection{Site Index}\label{site-index}}

\hypertarget{site-information-navigation}{%
\subsection{Site Information
Navigation}\label{site-information-navigation}}

\begin{itemize}
\tightlist
\item
  \href{https://help.nytimes3xbfgragh.onion/hc/en-us/articles/115014792127-Copyright-notice}{©~2020~The
  New York Times Company}
\end{itemize}

\begin{itemize}
\tightlist
\item
  \href{https://www.nytco.com/}{NYTCo}
\item
  \href{https://help.nytimes3xbfgragh.onion/hc/en-us/articles/115015385887-Contact-Us}{Contact
  Us}
\item
  \href{https://www.nytco.com/careers/}{Work with us}
\item
  \href{https://nytmediakit.com/}{Advertise}
\item
  \href{http://www.tbrandstudio.com/}{T Brand Studio}
\item
  \href{https://www.nytimes3xbfgragh.onion/privacy/cookie-policy\#how-do-i-manage-trackers}{Your
  Ad Choices}
\item
  \href{https://www.nytimes3xbfgragh.onion/privacy}{Privacy}
\item
  \href{https://help.nytimes3xbfgragh.onion/hc/en-us/articles/115014893428-Terms-of-service}{Terms
  of Service}
\item
  \href{https://help.nytimes3xbfgragh.onion/hc/en-us/articles/115014893968-Terms-of-sale}{Terms
  of Sale}
\item
  \href{https://spiderbites.nytimes3xbfgragh.onion}{Site Map}
\item
  \href{https://help.nytimes3xbfgragh.onion/hc/en-us}{Help}
\item
  \href{https://www.nytimes3xbfgragh.onion/subscription?campaignId=37WXW}{Subscriptions}
\end{itemize}
