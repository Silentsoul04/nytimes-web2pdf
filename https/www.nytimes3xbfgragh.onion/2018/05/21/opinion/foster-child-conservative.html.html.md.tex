Sections

SEARCH

\protect\hyperlink{site-content}{Skip to
content}\protect\hyperlink{site-index}{Skip to site index}

\href{https://myaccount.nytimes3xbfgragh.onion/auth/login?response_type=cookie\&client_id=vi}{}

\href{https://www.nytimes3xbfgragh.onion/section/todayspaper}{Today's
Paper}

\href{/section/opinion}{Opinion}\textbar{}Why Being a Foster Child Made
Me a Conservative

\url{https://nyti.ms/2IVWepX}

\begin{itemize}
\item
\item
\item
\item
\item
\item
\end{itemize}

Advertisement

\protect\hyperlink{after-top}{Continue reading the main story}

Supported by

\protect\hyperlink{after-sponsor}{Continue reading the main story}

\href{/section/opinion}{Opinion}

\href{/column/on-campus}{On Campus}

\hypertarget{why-being-a-foster-child-made-me-a-conservative}{%
\section{Why Being a Foster Child Made Me a
Conservative}\label{why-being-a-foster-child-made-me-a-conservative}}

By Rob Henderson

Mr. Henderson
\href{https://news.yale.edu/2018/05/14/senior-arrived-air-force-vet-and-graduates-gates-cambridge-scholar}{served
in the Air Force} before going to Yale, where he majored in psychology.
He graduated~on Monday.

\begin{itemize}
\item
  May 21, 2018
\item
  \begin{itemize}
  \item
  \item
  \item
  \item
  \item
  \item
  \end{itemize}
\end{itemize}

\includegraphics{https://static01.graylady3jvrrxbe.onion/images/2018/05/21/opinion/21on-campus-Henderson/merlin_138208818_0d435da8-2082-4a17-af6b-a6ff660e8718-articleLarge.jpg?quality=75\&auto=webp\&disable=upscale}

NEW HAVEN --- There aren't many conservative students at Yale:
\href{http://features.yaledailynews.com/blog/2016/10/27/election-2016-conservative-views-considered-unwelcome-at-yale/}{fewer
than 12 percent,} according to a survey by our student newspaper. There
are fewer former foster children. I am one of the rare students on
campus who can claim both identities.

My unusual upbringing has shaped my conservatism. My birth mother was
addicted to drugs. As a young child, I spent five years in foster care.
At age 7, I was adopted, but for a long time after that I was raised in
broken homes.

Foster care, broken homes and military service have fashioned my
judgments. My experiences drive me to reflect on what environments are
best for children. Certainly not the ones I came from.

Where I came from can be understood through my name: Robert Kim
Henderson. All three names were taken from different adults.

Robert comes from my supposed biological father. The only information I
have about him is his name from a document provided by a social worker
responsible for my case when I was a foster child.

My middle name, Kim, comes from my biological mother. It was her family
name. She succumbed to drug addiction, rendering her unable to care for
me.

And my last name: Henderson. It comes from my former adoptive father.
After my adoptive mother left him, he severed ties with me in order to
hurt her. He figured that my emotional pain from his desertion would be
transmitted to my adoptive mother. He was right. The three people who
gave me their names have something in common: All abandoned me. None
took responsibility.

Last year, a fellow student told me I was a victim. Yale is the only
place where someone has said this to me. I responded that if someone had
told me I was a victim when I was a kid, I would never have made it to
the Air Force, where I served for eight years, or to Yale. I would have
given up. When I was 10, a teacher told me that if I applied myself, I
could alter my future. This advice changed my life. From my response, my
fellow student inferred that I was not as progressive as him. As our
conversation unfolded, he asked, ``What does it actually mean to be a
conservative?''

For me, the answer is that people who came before us weren't stupid.
They were stunted in many ways. But not in every way. Older people have
insights worthy of our attention.

One piece of inherited wisdom is the value of the two-parent family.
It's not fashionable to talk about this. How people raise their children
is a matter of preference. But it is not really up for debate that the
two-parent home
is,\href{http://www.nytimes3xbfgragh.onion/2012/07/15/us/two-classes-in-america-divided-by-i-do.html}{on
average}, better for children.

First, two parents can provide their children more resources, including
emotional support, encouragement and help with homework. One
conscientious parent, no matter how heroic, cannot do the work of two.
Second, single-parent households have a lower standard of living, which
is associated with lower school grades and test scores.

Here is an example of the success of intact families from one of my
psychology classes. The professor asked students to anonymously respond
to a question about parental background. Out of 25 students, only one
student besides me did not grow up in a traditional two-parent family.
It's no accident that most of my peers at Yale came from intact
families.

Outcomes are worse for foster children. Ten percent of them enroll in
college, and
\href{https://www.sciencedaily.com/releases/2015/04/150419193908.htm}{3
percent graduate}. To my knowledge, among more than 5,000 undergraduates
at Yale my senior year, the number of former foster children was under
10.

Along with taking accumulated wisdom seriously, I understand
conservative philosophy to mean that the role of the individual in
making decisions and undertaking obligations is paramount. Individuals
have rights. But they also have responsibilities.

For instance, when I say parents should prioritize their children over
their careers, there is a sense of unease among my peers. They think I
want to blame individuals rather than a nebulous foe like poverty. They
are mostly right. Many people who come from privilege do not like
placing blame on ordinary people. They prefer to blame ideologies,
institutions, abstractions.

A cynical interpretation of this attitude is that some students want to
keep the competition down. Fewer children raised in good families means
less competition for those at the top.

My skin crawls when people use me as an example of a person who can
shoulder the burdens of a nontraditional upbringing and succeed. They
use my success as an argument for lax attitudes about parenting. But I
am one of the lucky ones.

Many people have asked me how I turned out to be relatively successful,
given my turbulent childhood. My answer is simple: During adolescence, I
had the benefit of two parents, my adoptive mother and her partner, and
I believed I had control of my future.

My adoptive mother and her partner raised me from middle school through
high school in the early to mid-2000s in a rural California town called
Red Bluff. They made a stable home for me. We had dinner together every
weeknight. We talked about minutiae. They would ask me, ``How was school
today?'' And I would respond with the usual ``It was fine.'' They gave
me unsolicited advice. I was sarcastic in response. And we loved one
another.

I experienced a stable family, if only for a few years. Though they
experienced homophobia and struggled financially, they never let it get
in the way of doing the right thing for their son.

Ordinary adults taking responsibility made all the difference for me. I
maintain that the agency of individuals will lead to fewer impoverished
childhoods.

If today that makes me a conservative, great. I take responsibility for
that.

Advertisement

\protect\hyperlink{after-bottom}{Continue reading the main story}

\hypertarget{site-index}{%
\subsection{Site Index}\label{site-index}}

\hypertarget{site-information-navigation}{%
\subsection{Site Information
Navigation}\label{site-information-navigation}}

\begin{itemize}
\tightlist
\item
  \href{https://help.nytimes3xbfgragh.onion/hc/en-us/articles/115014792127-Copyright-notice}{©~2020~The
  New York Times Company}
\end{itemize}

\begin{itemize}
\tightlist
\item
  \href{https://www.nytco.com/}{NYTCo}
\item
  \href{https://help.nytimes3xbfgragh.onion/hc/en-us/articles/115015385887-Contact-Us}{Contact
  Us}
\item
  \href{https://www.nytco.com/careers/}{Work with us}
\item
  \href{https://nytmediakit.com/}{Advertise}
\item
  \href{http://www.tbrandstudio.com/}{T Brand Studio}
\item
  \href{https://www.nytimes3xbfgragh.onion/privacy/cookie-policy\#how-do-i-manage-trackers}{Your
  Ad Choices}
\item
  \href{https://www.nytimes3xbfgragh.onion/privacy}{Privacy}
\item
  \href{https://help.nytimes3xbfgragh.onion/hc/en-us/articles/115014893428-Terms-of-service}{Terms
  of Service}
\item
  \href{https://help.nytimes3xbfgragh.onion/hc/en-us/articles/115014893968-Terms-of-sale}{Terms
  of Sale}
\item
  \href{https://spiderbites.nytimes3xbfgragh.onion}{Site Map}
\item
  \href{https://help.nytimes3xbfgragh.onion/hc/en-us}{Help}
\item
  \href{https://www.nytimes3xbfgragh.onion/subscription?campaignId=37WXW}{Subscriptions}
\end{itemize}
