Sections

SEARCH

\protect\hyperlink{site-content}{Skip to
content}\protect\hyperlink{site-index}{Skip to site index}

\href{https://www.nytimes3xbfgragh.onion/section/technology}{Technology}

\href{https://myaccount.nytimes3xbfgragh.onion/auth/login?response_type=cookie\&client_id=vi}{}

\href{https://www.nytimes3xbfgragh.onion/section/todayspaper}{Today's
Paper}

\href{/section/technology}{Technology}\textbar{}Amazon Pushes Facial
Recognition to Police. Critics See Surveillance Risk.

\url{https://nyti.ms/2IWdonh}

\begin{itemize}
\item
\item
\item
\item
\item
\end{itemize}

Advertisement

\protect\hyperlink{after-top}{Continue reading the main story}

Supported by

\protect\hyperlink{after-sponsor}{Continue reading the main story}

\hypertarget{amazon-pushes-facial-recognition-to-police-critics-see-surveillance-risk}{%
\section{Amazon Pushes Facial Recognition to Police. Critics See
Surveillance
Risk.}\label{amazon-pushes-facial-recognition-to-police-critics-see-surveillance-risk}}

\includegraphics{https://static01.graylady3jvrrxbe.onion/images/2018/05/23/business/23AMAZON-2/merlin_138518058_4c53be0e-9f0c-42dd-8bc6-b4818f5d18aa-articleLarge.jpg?quality=75\&auto=webp\&disable=upscale}

By \href{http://www.nytimes3xbfgragh.onion/by/nick-wingfield}{Nick
Wingfield}

\begin{itemize}
\item
  May 22, 2018
\item
  \begin{itemize}
  \item
  \item
  \item
  \item
  \item
  \end{itemize}
\end{itemize}

SEATTLE --- In late 2016, Amazon
\href{https://aws.amazon.com/blogs/aws/amazon-rekognition-image-detection-and-recognition-powered-by-deep-learning/}{introduced}
a new online service that could help identify faces and other objects in
images, offering it to anyone at a low cost through its giant cloud
computing division, Amazon Web Services.

Not long after, it began pitching the technology to law enforcement
agencies, saying the program could aid criminal investigations by
recognizing suspects in photos and videos. It used a couple of early
customers, like the Orlando Police Department in Florida and the
Washington County Sheriff's Office in Oregon, to encourage other
officials to sign up.

But now that aggressive push is putting the giant tech company at the
center of an increasingly heated debate around the role of facial
recognition in law enforcement. Fans of the technology see a powerful
new tool for catching criminals, but detractors see an instrument of
mass surveillance.

On Tuesday, the American Civil Liberties Union led a group of more than
two dozen civil rights organizations that asked Amazon to stop selling
its image recognition system, called Rekognition, to law enforcement.
The group says that the police could use it to track protesters or
others whom authorities deem suspicious, rather than limiting it to
people committing crimes.

Facial recognition is not new technology, but the organizations appear
to be focusing on Amazon because of its prominence and what they see as
a departure from the company's oft-stated focus on customers.

``Amazon Rekognition is primed for abuse in the hands of governments,''
the group
\href{https://www.aclunc.org/docs/20180522_AR_Coalition_Letter.pdf}{said
in the letter}, which was addressed to Jeff Bezos, Amazon's chief
executive. ``This product poses a grave threat to communities, including
people of color and immigrants, and to the trust and respect Amazon has
worked to build.''

With the letter, the A.C.L.U. released a collection of internal emails
and other documents from law enforcement agencies in Washington County
and Orlando that it obtained through open records requests. The
correspondence between Amazon and law enforcement officials provides an
unusual peek into the company's ambitions with facial recognition tools,
and how it has interacted with some of the officials using its products.

Many of the companies supplying the technology are security contractors
little known to the public, but Amazon is one of the first major tech
companies to actively market technology for conducting facial
recognition to law enforcement. The efforts are still a tiny part of
Amazon's business, with the service one of dozens it offers through
Amazon Web Services. But few companies have Amazon's ability to
effectively push widespread adoption of tech products.

\includegraphics{https://static01.graylady3jvrrxbe.onion/images/2018/05/23/business/23AMAZON/merlin_97580653_a15dec1d-0c99-48aa-9d89-7198f3b8742c-articleLarge.jpg?quality=75\&auto=webp\&disable=upscale}

``The idea that a massive and highly resourced company like Amazon has
moved decisively into this space could mark a sea change for this
technology,'' said Alvaro Bedoya, executive director at the Center on
Privacy \& Technology at the Georgetown University Law Center.

In a statement, a spokeswoman for Amazon Web Services stressed that the
company offered a general image recognition technology that could
automate the process of identifying people, objects and activities. She
said amusement parks had used it to find lost children, and Sky News,
the British broadcaster, used it
\href{http://www.streamingmedia.com/PressRelease/Sky-to-Launch-Live-Whos-Who-innovation-for-Royal-Wedding_47055.aspx}{last
weekend} to automatically identify guests attending the royal wedding.
(The New York Times has also used the technology, including for the
royal wedding.)

The spokeswoman said that, as with all A.W.S. services, the company
requires customers to comply with the law.

The United States military and intelligence agencies have used facial
recognition tools for years in overseas conflicts to identify possible
terrorist suspects. But domestic law enforcement agencies
\href{https://www.nytimes3xbfgragh.onion/2015/08/13/us/facial-recognition-software-moves-from-overseas-wars-to-local-police.html}{are
increasingly using the technology} at home for more routine forms of
policing.

The people who can be identified through facial recognition systems are
not just those with criminal records. More than 130 million American
adults are in facial recognition databases that can be searched in
criminal investigations, the Center on Privacy \& Technology at
Georgetown Law \href{https://www.perpetuallineup.org/}{estimates}.

Facial recognition is showing up in new corners of public life all the
time, often followed by challenges from critics about its efficacy as a
security tool and its impact on privacy. Arenas are using it
\href{https://www.nytimes3xbfgragh.onion/2018/03/13/sports/facial-recognition-madison-square-garden.html}{to
screen for known troublemakers} at events, while the Department of
Homeland Security is using it
\href{https://www.nytimes3xbfgragh.onion/2017/12/21/us/politics/facial-scans-airports-security-privacy.html}{to
identify foreign visitors} who overstay their visas at airports. And in
China, facial recognition is ubiquitous, used to identify customers in
stores and single out jaywalkers.

There are also concerns about the accuracy of facial recognition, with
troubling variations based on gender and race.
\href{https://www.nytimes3xbfgragh.onion/2018/02/09/technology/facial-recognition-race-artificial-intelligence.html}{One
study} by the Massachusetts Institute of Technology showed that the
gender of darker-skinned women was misidentified up to 35 percent of the
time by facial recognition software.

``We have it being used in unaccountable ways and with no regulation,''
said Malkia Cyril, executive director of the Center for Media Justice, a
nonprofit civil rights organization that signed the A.C.L.U.'s letter to
Amazon.

The documents the A.C.L.U. obtained from the Orlando Police Department
show city officials considering using video analysis tools from Amazon
with footage from surveillance cameras, body-worn cameras and drones.

Amazon may have gone a little far in describing what the technology can
do. This month, it published a video of an Amazon official, Ranju Das,
speaking at a company event in Seoul, South Korea, in which he said
Orlando could even use Amazon's Rekognition system to find the
whereabouts of the mayor through cameras around the city.

In a statement, a spokesman for the Orlando Police Department, Sgt.
Eduardo Bernal, said the city was not using Amazon's technology to track
the location of elected officials in its jurisdiction, nor did it have
plans to. He said the department was testing Amazon's service now, but
was not using it in investigations or public spaces.

``We are always looking for new solutions to further our ability to keep
the residents and visitors of Orlando safe,'' he said.

(On Thursday, Amazon added a note on
\href{https://www.youtube.com/watch?time_continue=1628\&v=sUzuJc-xBEE}{the
YouTube page for the video of Mr. Das} that said he ``got confused and
misspoke about the City of Orlando's use of A.W.S. technologies.'')

Early last year, the company began courting the Washington County
Sheriff's Office outside of Portland, Ore., eager to promote how it was
using Amazon's service for recognizing faces, emails obtained by the
A.C.L.U. show. Chris Adzima, a systems analyst in the office, told
Amazon officials that he fed about 300,000 images from the county's mug
shot database into Amazon's system.

Within a week of going live, the system was used to identify and arrest
a suspect who stole more than \$5,000 from local stores, he said, adding
there were no leads before the system identified him. The technology was
also cheap, costing just a few dollars a month after a setup fee of
around \$400.

Mr. Adzima ended up
\href{https://aws.amazon.com/blogs/machine-learning/using-amazon-rekognition-to-identify-persons-of-interest-for-law-enforcement/}{writing
a blog post} for Amazon about how the sheriff's office was using
Rekognition. He \href{https://youtu.be/LwjaPKo1Qkk?t=26m42s}{spoke} at
one of the company's technical conferences, and local media began
reporting on their efforts. After the attention, other law enforcement
agencies in Oregon, Arizona and California began to reach to Washington
County to learn more about how it was using Amazon's system, emails
show.

In February of last year, before the publicity wave, Mr. Adzima told an
Amazon representative in an email that the county's lawyer was worried
the public might believe ``that we are constantly checking faces from
everything, kind of a Big Brother vibe.''

``They are concerned that A.C.L.U. might consider this the government
getting in bed with big data,'' Mr. Adzima said in an email. He did not
respond to a request for comment for this article.

Deputy Jeff Talbot, a spokesman for the Washington County Sheriff's
Office, said Amazon's facial recognition system was not being used for
mass surveillance by the office. The company has a policy to use the
technology only to identify a suspect in a criminal investigation, he
said, and has no plans to use it with footage from body cameras or
real-time surveillance systems.

``We are aware of those privacy concerns,'' he said. ``That's why we
have a policy drafted and why we've tried to educate the public about
what we do and don't do.''

Advertisement

\protect\hyperlink{after-bottom}{Continue reading the main story}

\hypertarget{site-index}{%
\subsection{Site Index}\label{site-index}}

\hypertarget{site-information-navigation}{%
\subsection{Site Information
Navigation}\label{site-information-navigation}}

\begin{itemize}
\tightlist
\item
  \href{https://help.nytimes3xbfgragh.onion/hc/en-us/articles/115014792127-Copyright-notice}{©~2020~The
  New York Times Company}
\end{itemize}

\begin{itemize}
\tightlist
\item
  \href{https://www.nytco.com/}{NYTCo}
\item
  \href{https://help.nytimes3xbfgragh.onion/hc/en-us/articles/115015385887-Contact-Us}{Contact
  Us}
\item
  \href{https://www.nytco.com/careers/}{Work with us}
\item
  \href{https://nytmediakit.com/}{Advertise}
\item
  \href{http://www.tbrandstudio.com/}{T Brand Studio}
\item
  \href{https://www.nytimes3xbfgragh.onion/privacy/cookie-policy\#how-do-i-manage-trackers}{Your
  Ad Choices}
\item
  \href{https://www.nytimes3xbfgragh.onion/privacy}{Privacy}
\item
  \href{https://help.nytimes3xbfgragh.onion/hc/en-us/articles/115014893428-Terms-of-service}{Terms
  of Service}
\item
  \href{https://help.nytimes3xbfgragh.onion/hc/en-us/articles/115014893968-Terms-of-sale}{Terms
  of Sale}
\item
  \href{https://spiderbites.nytimes3xbfgragh.onion}{Site Map}
\item
  \href{https://help.nytimes3xbfgragh.onion/hc/en-us}{Help}
\item
  \href{https://www.nytimes3xbfgragh.onion/subscription?campaignId=37WXW}{Subscriptions}
\end{itemize}
