Sections

SEARCH

\protect\hyperlink{site-content}{Skip to
content}\protect\hyperlink{site-index}{Skip to site index}

\href{https://www.nytimes3xbfgragh.onion/section/obituaries}{Obituaries}

\href{https://myaccount.nytimes3xbfgragh.onion/auth/login?response_type=cookie\&client_id=vi}{}

\href{https://www.nytimes3xbfgragh.onion/section/todayspaper}{Today's
Paper}

\href{/section/obituaries}{Obituaries}\textbar{}Ella Brennan, Grande
Dame Restaurateur of New Orleans, Dies at 92

\url{https://nyti.ms/2LTLBCu}

\begin{itemize}
\item
\item
\item
\item
\item
\end{itemize}

Advertisement

\protect\hyperlink{after-top}{Continue reading the main story}

Supported by

\protect\hyperlink{after-sponsor}{Continue reading the main story}

\hypertarget{ella-brennan-grande-dame-restaurateur-of-new-orleans-dies-at-92}{%
\section{Ella Brennan, Grande Dame Restaurateur of New Orleans, Dies at
92}\label{ella-brennan-grande-dame-restaurateur-of-new-orleans-dies-at-92}}

\includegraphics{https://static01.graylady3jvrrxbe.onion/images/2018/06/03/obituaries/03BRENNAN-OBIT1/merlin_115908125_8d343722-50ef-4edb-bc5c-9f4fcdf092cf-articleLarge.jpg?quality=75\&auto=webp\&disable=upscale}

By \href{https://www.nytimes3xbfgragh.onion/by/richard-sandomir}{Richard
Sandomir}

\begin{itemize}
\item
  June 1, 2018
\item
  \begin{itemize}
  \item
  \item
  \item
  \item
  \item
  \end{itemize}
\end{itemize}

Ella Brennan, the exacting matriarch of a New Orleans restaurant family
whose flagship, Commander's Palace, is renowned for serving a blend of
Louisiana and nouvelle cuisines with generous, dramatic flair, died on
Thursday in New Orleans. She was 92.

Her death, in her elegant 19th-century house next door to the
restaurant, in the Garden District, was confirmed by a family spokesman.

Miss Brennan, as she preferred to be called, could not cook and never
wanted to. Yet she was the grande dame of restaurant royalty in New
Orleans whose infighting and power struggles have long been the subject
of gossip and intrigue in a city as dedicated to its food as any in
America.

Restaurants mattered deeply to her --- what they served, how they were
run and how they treated customers --- so she dived deeply into
cookbooks and business books. She quizzed wine merchants and smart
diners for tips on how to make her restaurants better. She traveled
across the country and to Europe to experience fine dining and, along
the way, developed a sophisticated palate that told her what tasted
right and what needed a little more this, or a little more that.

Miss Brennan helped propel the careers of renowned chefs like Paul
Prudhomme, Emeril Lagasse and Jamie Shannon. They introduced changes to
the Commander's menu that were sometimes met with skepticism in a city
that can adhere unmercifully to a strict culinary canon. But each new
twist of the menu seemed to bring with it a new set of accolades.

``She was there to embrace and elevate, not just me, but the entire
staff,'' Mr. Lagasse, whom Miss Brennan hired when he was 23,
\href{https://www.nytimes3xbfgragh.onion/2017/03/27/dining/ella-brennan-new-orleans-restaurants.html}{told
The New York Times} last year. ``The list of people she has impacted in
the hospitality industry in America is endless.''

But she could be tough; her nickname was Hurricane Ella.

Miss Brennan extolled the importance of fresh local ingredients and
emphasized the joie de vivre of dining at Commander's. Guests receive
chefs' hats, and during weekend brunches jazz combos move from table to
table, taking requests.
\href{http://www.nola.com/dining/index.ssf/2018/05/ella_brennan_renowned_new_orle.html}{She
often joined in,} exhorting diners to do the same while walking behind
the combo, waving a handkerchief.

``I've got to tell you,'' she wrote in her autobiography, ``Miss Ella of
Commander's Palace'' (2016), written with her daughter, Ti Adelaide
Martin. ``I don't want a restaurant where a jazz band can't come
marching through.''

The dishes she ushered onto the menu have become culinary smashes,
including turtle soup au sherry, eggs Sardou and
\href{https://www.npr.org/sections/thesalt/2016/09/30/493157144/the-sweet-success-of-bananas-foster-has-an-unsavory-past}{bananas
Foster}, which she created for a customer one night, drawing on her
memories of the sautéed bananas her mother, Nellie, had made.

``Nellie stirred the eggs a little in a bowl,'' Miss Brennan wrote in
her autobiography, ``poured them into a pan sizzling with butter,
stirred them again gently and slid them out while they were still very
soft. Perfection! Then she'd put a little brown sugar and cinnamon on
bananas sliced lengthwise into quarters, turn them over in hot butter
until they were caramelized --- imagine what our kitchen smelled like!
--- and served them with the fluffy eggs.''

She underscored her belief in gracious service as the heartbeat of a
restaurant in a terse speech she delivered at a Manhattan hotel in 1993,
when Commander's won an award for outstanding service from the James
Beard Foundation.

``I accept this award for every damn captain and waiter in the
country,'' she said --- and said no more, sitting down to great
applause.

Miss Brennan won the Beard Foundation's lifetime achievement award in
2009.

Ella Brennan was born in New Orleans on Nov. 27, 1925. Her father, Owen,
was a shipyard superintendent, and her mother, Nellie (Valentine)
Brennan, was a homemaker whose inventive home cooking left an indelible
impression on Ella, the fourth of six children.

\includegraphics{https://static01.graylady3jvrrxbe.onion/images/2018/06/03/obituaries/03BRENNAN-OBIT2/03BRENNAN-OBIT2-articleLarge-v2.jpg?quality=75\&auto=webp\&disable=upscale}

``My mother had magic in her hands,'' Miss Brennan wrote in her
autobiography.

She graduated from high school in 1943, but after four months at
business school she left, bored by typing and filing classes.

Looking for something to do, she accepted an offer from her oldest
brother, Owen, to do clerical work at the Old Absinthe House, his saloon
on Bourbon Street in the French Quarter. Her mother was worried that
Ella would be mingling with carousers and habitués of burlesque houses.
But her brother said he would take care of her.

Miss Brennan discovered that she loved the French Quarter night life,
especially a tavern called Lafitte's. ``Some girls went to finishing
school,'' she told The Times-Picayune of New Orleans in 2007. ``I went
to Lafitte's.''

When Owen Brennan and his father bought a restaurant, Vieux Carré, she
went to work there. Soon she was pressing them to revamp its limited
menu.

``Heck, I had eaten better food every day of my life at home, and here
customers were paying good money for this inferior stuff,'' Miss Brennan
wrote, disdaining in one instance an entree of roasted leg of lamb with
mint jelly (from the jar). ``Awful.''

She worked with the chefs to reinvent the menu and traveled to New York
City to explore finer fare at the ``21'' Club and other restaurants.

Owen Brennan died of a heart attack in 1955 as the family was planning
to move Vieux Carré to a larger space. Miss Brennan, having absorbed her
brother's lessons, proceeded to lead the relocated restaurant, renamed
Brennan's, into a new culinary era, with a greater appreciation for
Southern cuisine.

But she was ousted in 1973, resulting from a family feud with Owen's
widow, Maud, and their sons that lasted 40 years. Miss Brennan did not
enter the restaurant again until her nephew Ralph bought it in a
sheriff's sale in 2014.

By then, though, Miss Brennan had another place to go. She and her older
sister Adelaide had bought Commander's, a run-down restaurant, in 1969,
and it became part of the family's expanding business interests.

Miss Brennan took over the restaurant's management in 1974, and with
Adelaide and their siblings Dottie, Dickie and John, she began to
transform its traditional Creole menu into something more daring.

``I was convinced,'' she wrote, ``that Louisiana cuisine, in both its
countrified Cajun and refined Creole forms, was a world-class,
indigenous cuisine --- America's best --- that was poised to break out
of home kitchens and into restaurants.''

Despite some unfavorable early reviews, Commander's succeeded, and it
became the centerpiece of an empire of 14 restaurants in the New Orleans
area, Houston and at Disneyland in California, all run by various
branches of the family. Commander's, Cafe Adelaide, Sobou and Brennan's
of Houston are operated by Miss Brennan's daughter and her niece Lally
Brennan.

In addition to her daughter, Miss Brennan is survived by her son,
\href{https://www.brennanshouston.com/alex-brennan-martin/}{Alex
Brennan-Martin}, the president of Brennan's of Houston; two
grandchildren; and her sister, Dottie Brennan, with whom she lived in
the house next to Commander's Palace. Her marriage to Paul Martin ended
in divorce.

Ruth Reichl, the former editor in chief of Gourmet magazine, said Miss
Brennan would be remembered for helping to popularize indigenous New
Orleans cuisine, with its myriad influences from Africa, France and
elsewhere

``She brought that to everyone's attention in a proud way,'' Ms. Reichl
said in a telephone interview. ``She was really aware of the New Orleans
heritage and wanted the rest of us to know about it.''

Advertisement

\protect\hyperlink{after-bottom}{Continue reading the main story}

\hypertarget{site-index}{%
\subsection{Site Index}\label{site-index}}

\hypertarget{site-information-navigation}{%
\subsection{Site Information
Navigation}\label{site-information-navigation}}

\begin{itemize}
\tightlist
\item
  \href{https://help.nytimes3xbfgragh.onion/hc/en-us/articles/115014792127-Copyright-notice}{©~2020~The
  New York Times Company}
\end{itemize}

\begin{itemize}
\tightlist
\item
  \href{https://www.nytco.com/}{NYTCo}
\item
  \href{https://help.nytimes3xbfgragh.onion/hc/en-us/articles/115015385887-Contact-Us}{Contact
  Us}
\item
  \href{https://www.nytco.com/careers/}{Work with us}
\item
  \href{https://nytmediakit.com/}{Advertise}
\item
  \href{http://www.tbrandstudio.com/}{T Brand Studio}
\item
  \href{https://www.nytimes3xbfgragh.onion/privacy/cookie-policy\#how-do-i-manage-trackers}{Your
  Ad Choices}
\item
  \href{https://www.nytimes3xbfgragh.onion/privacy}{Privacy}
\item
  \href{https://help.nytimes3xbfgragh.onion/hc/en-us/articles/115014893428-Terms-of-service}{Terms
  of Service}
\item
  \href{https://help.nytimes3xbfgragh.onion/hc/en-us/articles/115014893968-Terms-of-sale}{Terms
  of Sale}
\item
  \href{https://spiderbites.nytimes3xbfgragh.onion}{Site Map}
\item
  \href{https://help.nytimes3xbfgragh.onion/hc/en-us}{Help}
\item
  \href{https://www.nytimes3xbfgragh.onion/subscription?campaignId=37WXW}{Subscriptions}
\end{itemize}
