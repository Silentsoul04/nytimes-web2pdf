\href{/section/food}{Food}\textbar{}Still the Boss of Le Bernardin,
Maguy Le Coze Strides On

\url{https://nyti.ms/2M1Lk0L}

\begin{itemize}
\item
\item
\item
\item
\item
\item
\end{itemize}

\includegraphics{https://static01.graylady3jvrrxbe.onion/images/2018/06/06/dining/06lecoze/06lecoze-articleLarge.jpg?quality=75\&auto=webp\&disable=upscale}

Sections

\protect\hyperlink{site-content}{Skip to
content}\protect\hyperlink{site-index}{Skip to site index}

\hypertarget{still-the-boss-of-le-bernardin-maguy-le-coze-strides-on}{%
\section{Still the Boss of Le Bernardin, Maguy Le Coze Strides
On}\label{still-the-boss-of-le-bernardin-maguy-le-coze-strides-on}}

Five decades after her first restaurant in Paris, she has retreated from
the dining room but remains a fast-moving force, worrying every detail.

Maguy Le Coze at Le Bernardin, the Midtown Manhattan citadel of seafood
that she owns with the chef Eric Ripert. Mr. Ripert called her ``the
soul, the spirit and the boss'' of the restaurant.Credit...Damon
Winter/The New York Times

Supported by

\protect\hyperlink{after-sponsor}{Continue reading the main story}

By Alan Richman

\begin{itemize}
\item
  June 5, 2018
\item
  \begin{itemize}
  \item
  \item
  \item
  \item
  \item
  \item
  \end{itemize}
\end{itemize}

She grew up in a modest hotel run by her parents in a French seaport
village, sharing a room with her younger brother. Her father cooked, her
mother oversaw the dining room, and Maguy Le Coze was waiting tables by
the time she was 12, a serious child who forever remained that way.

The two children attended separate elementary schools in Port Navalo ---
Gilbert the one for boys, Maguy the one for girls. There was another
difference: ``I would do my homework,'' she said. ``My brother never did
his. I always tried to do the right thing, to learn.''

Her career path, as a woman born into an unassuming French restaurant
family, seemed predestined: tending to customers while her man cooked
for them. The chef, even behind kitchen doors, was the face of the
restaurant. The dutiful woman rarely looked beyond the front door.

For Ms. Le Coze, such an existence was understandable but
unacceptable*.* ``They were good parents,'' she said. ``They thought
this was the right way to bring us up.''

She has been working in restaurants for more than 60 years, and what
might have been a life spent in a simple family business never occurred.
She is today the co-owner of \href{https://www.le-bernardin.com/}{Le
Bernardin}, the Midtown Manhattan citadel of seafood created by her and
her brother, its first chef, and heaped with accolades: three stars from
Michelin,
\href{https://www.nytimes3xbfgragh.onion/2012/05/23/dining/reviews/le-bernardin-in-midtown-manhattan.html}{four
from The New York Times}, ratings attained and never lost.

\includegraphics{https://static01.graylady3jvrrxbe.onion/images/2018/06/06/dining/06lecoze1/06lecoze1-articleLarge.jpg?quality=75\&auto=webp\&disable=upscale}

Secure in her position and her accomplishments, she is hesitant only to
admit her age, worried about the reaction of the staff. ``When they see
I am 73, they will say the boss is old,'' Ms. Le Coze said.

They will not, of course, say anything of the sort, not to a woman so
regal, so formidable, so unbending --- and so charming when she chooses
to be. If Eric Ripert, the chef and other owner of Le Bernardin, is the
face of the restaurant and its ambassador, she is its queen. Decades ago
her role was to pamper guests; these days she is more often laboring
over ledgers in the underground offices.

Mr. Ripert calls her ``the soul, the spirit and'' --- he smiled ---
``the boss of Le Bernardin.''

Yet Ms. Le Coze's name is not widely known, except among her peers. Mr.
Ripert calls her ``the most underrated, ultrasuccessful woman in the
restaurant world.''

Amanda Cohen, the chef and owner of
\href{https://www.nytimes3xbfgragh.onion/2012/11/28/dining/reviews/restaurant-review-dirt-candy-in-the-east-village.html}{Dirt
Candy}, a vegetarian restaurant on the Lower East Side, said she finds
it fascinating that ``so many of the people who do the hard work behind
the scenes are women,'' mentioning the late
\href{https://www.nytimes3xbfgragh.onion/2018/06/01/obituaries/ella-brennan-grande-dame-restaurateur-of-new-orleans-dies-at-92.html}{Ella
Brennan} (the New Orleans doyenne who championed Paul Prudhomme and
Emeril Lagasse) and \href{http://www.barbaralazaroff.com/}{Barbara
Lazaroff} (the business partner and ex-wife of
\href{https://www.nytimes3xbfgragh.onion/2012/10/31/dining/wolfgang-puck-the-original-celebrity-chef-is-still-keeping-busy.html}{Wolfgang
Puck}). Ms. Cohen pointed out that Gilbert Le Coze, who died in 1994,
and Eric Ripert have Wikipedia pages, but Ms. Le Coze does not.

Those who cook become stars. Those who toil at other tasks in
restaurants rarely have such luck. ``Many of my friends say to me, `Look
at what you have done.'\emph{''} Ms. Le Coze said, uncharacteristically
at rest on a patio overlooking the Caribbean Sea outside her sprawling
home on the private island of Mustique. ``I have the same answer: `I
worked. That's it.'''

If fame has eluded her, the rewards of her success have been numerous: a
Légion d'honneur from France, the home on Mustique, apartments in New
York and Paris, the house where she grew up in Brittany.

She has purposefully forgone much that others prize. She never married
and never desired children, preferring romantic relationships that
lasted no more than four or five years. She recently took a two-week
road trip through the United States with her current boyfriend, and
returned enamored of \href{https://www.bigtexan.com/}{the Big Texan
Steak Ranch} in Amarillo, Tex. Unexpected, indeed, for a woman with a
polished palate, but not the most unusual aspect of the relationship.

She met this boyfriend 35 years ago in Saint-Tropez, but had no further
contact until he phoned her at Le Bernardin 10 years ago. She wasn't
there. They met shortly afterward for dinner in Paris. She didn't see
him for another two years, when they happened to be waiting at adjoining
gates at Newark Liberty International Airport. They reconnected. Who
says flying has lost its romance?

``My life has been strange,'' Ms. Le Coze allowed.

Image

Maguy and Gilbert Le Coze in 1950 with their father, who cooked at the
family hotel in Brittany where they grew up.

When she was 15 and brought boyfriends home to meet her parents, ``I
told them I have a boyfriend and that's it. At 18 I said to them, `I
will not get married, I will not have children, I will have lovers.' My
parents thought they had a very unusual daughter.''

The reason, she says, was the infidelities of the men in her life: ``My
father cheated on my mother. My brother had so many women.''

Her lifelong passion, it would seem, was none of the men in her life,
but her beloved restaurant.

She thinks it's more than that. ``It is not possible to have passion for
30 or 40 years,'' she said. ``That is something you have for four or
five years. It is the same with the restaurant as it is with a man you
meet. After passion comes love.''

\hypertarget{she-sees-everything}{%
\subsection{`She Sees Everything'}\label{she-sees-everything}}

She never aspired to be a cook, and could not have become one even if
that had been her dream. ``I am a disaster,'' she said. ``The only thing
I can do is steam vegetables.''

Ms. Le Coze was, in the early years of Le Bernardin, a constant and
striking presence in the dining room, poised and perfect, attired in
Chanel outfits, earrings, bracelets and necklaces. She daringly broke
with the prescribed dining room etiquette of the era by perching on the
arm of a chair across from her customers, extolling sea-urchin butter
poured over raw sea urchins, or sea scallops still alive in their shell,
awaiting the stove.

These days she prefers to remain in the background. She is primarily the
financial overseer of the restaurant and, as always, remains obsessed
with orderliness and quality control. She often works from her home
office on Mustique, but when she is in New York she is known to spend 18
hours a day at her desk.

Mandy Oser, the owner of \href{https://www.ardesia-ny.com/}{Ardesia Wine
Bar} in Hell's Kitchen, worked with her at Le Bernardin for nine years.
``I think about Maguy often when I think about how to run my business,''
she said. ``I absorbed everything: Look over profit-and-loss statements.
Know how much you spend on napkins. Get your hands dirty with unsexy
details.''

Ms. Le Coze said running a restaurant is two-thirds about business,
one-third about the kitchen. Then she reconsidered, conceding that the
kitchen of a culinary landmark like Le Bernardin deserves more credit.
Surprisingly, Mr. Ripert agreed with this uneven division: ``I'd say
it's 60-40, and she gets the 60. An expensive restaurant is very complex
to run. If it is only about the food, you can go to a food truck.''

Image

Ms. Le Coze and Mr. Ripert in the New York dining room. ``If she is
eating,'' he said, ``she will always choose the dish that is not the
best. Bingo, she orders it. If she goes into the bathroom, it will be
right after a client throws paper on the floor. She always arrives at
the moment when something is wrong. I don't know how she does
it.''Credit...Damon Winter/The New York Times

Having said that, Mr. Ripert happily proceeded to catalog his partner's
idiosyncrasies. He walked around the conference room of Le Bernardin,
pointing out untidy piles of books, guides and plaques that had
accumulated while she was in Mustique.

He shook his head. ``I guarantee this is not going to fly,'' he said.
``She's very particular about how she wants Le Bernardin to be run. When
she's been traveling for a week, she comes into the office, looks for
shoes under desks, plastic knives and forks in drawers. It drives her
crazy. She's very neat. Everything has to be a certain way.''

Somewhat in admiration but more in wonder, he praised her uncanny
ability to stumble upon the unseemly or inappropriate. ``She sees
everything,'' he said. ``Basically, she has a sixth sense. If we are at
a meeting, looking at a contract, she for some reason will open to the
right page, see when something is wrong. If there is a mistake, she
finds it.

``If she is eating, she will always choose the dish that is not the
best. Bingo, she orders it. If she goes into the bathroom, it will be
right after a client throws paper on the floor. She always arrives at
the moment when something is wrong. I don't know how she does it.''

Image

The two business partners in 1997, celebrating the 25th anniversary of
the opening of the first Le Bernardin.Credit...Jean-Luce Huré for The
New York Times

Ms. Le Coze is also a woman of rigorous self-discipline, dedicated to
physical fitness. She keeps dumbbells in her office on Mustique, and
works out every day, there or in a local gym. She walks daily for at
least an hour, except in New York, where she never has enough time, so
she compensates with two-hour strolls through Central Park on spring and
summer weekends. She does not suffer the slow-footed gladly.

``When we walk, I am always slowing to look at things,'' said one of her
best friends, Marianne Tesler, who runs marathons. ``She walks with
determination, says to me, `Ahh, you're too slow.'''

Mr. Ripert claims to be a faster walker, but concedes he cannot swim
with her. ``In 10 minutes she is so far ahead I give up and go on the
beach,'' he said. Trying to walk to an airport gate with her is futile:
``In an airport she walks at the speed of light.''

\hypertarget{from-the-quai-to-the-pantheon}{%
\subsection{From the Quai to the
Pantheon}\label{from-the-quai-to-the-pantheon}}

Born 18 months before her brother, Ms. Le Coze was the boss of an
inseparable childhood partnership, until he turned 14 and took over.
They left for Paris in 1964, after high school. ``When I was 21 I had
just one thing in mind, having fun,'' she said. ``I left Brittany for
the life I didn't have with my parents.''

In Paris she worked as a model at a fashion house, posing in the latest
designs for customers. She was tall enough at 5-foot-7 to become a
runway model, but lacked the self-confidence.

``I was beautiful,'' Ms. Le Coze recalled. ``but I was too shy to go to
the big names in the fashion industry, intending to do the big shows.''
Gilbert worked for a hairdresser, fetching sandwiches, moving cars and
making more money in tips than she made modeling.

Image

On the beach in Mustique, in the early 1990s.

They returned to Brittany in summers to help their parents operate the
hotel. Asked what she learned about business from them, she emphatically
replied, ``Nothing! My brother and I learned nothing from our parents.

``When we opened our first Le Bernardin in Paris in 1972, we did not
even know how to count. We had no education in how to run a restaurant.
You do not learn from cooking in the kitchen with your father, as
Gilbert did, or working in the dining room with your mother, as I did.
After two years, we were almost bankrupt.''

That tiny Le Bernardin, seating 25, was on the same quai as
\href{https://tourdargent.com/en/}{La Tour d'Argent} and named after
\href{https://www.youtube.com/watch?v=CSrQLHZ6M_c}{``Les Moines de
Saint-Bernardin,''} a song their father sang to them when they were
infants. Mr. Le Coze cooked, aided by a dishwasher. Ms. Le Coze ran the
dining room with one waiter. They never locked the wine cellar;
residents of the apartments above the restaurant would come by and help
themselves.

The place received a fine review from Minute, a weekly newspaper with a
circulation of 250,000. Believing that this guaranteed success, the Le
Cozes started having the fun they had promised themselves. Soon they got
a second review, ``a disaster,'' she said, from the influential Le
Monde.

What made it worse, she recalled, was that the critic republished his
views under different names in six or seven other newspapers and
magazines. The Gault \& Millau guide added to the onslaught, awarding
them a score of eight out of 20, a devastatingly poor rating. She
recalls one line in that review: ``Send them back to the end of the
quai.''

The Le Cozes returned from the fish market one morning and saw a sign on
the restaurant window offering the contents for sale: They had failed to
pay their taxes, and the government stepped in.

``We used the money we made in the summer working for our parents to pay
the taxes, little by little,'' Ms. Le Coze said. ``And that's when we
started to change.''

Image

At the family home in Brittany, in the mid-1990s.

Mr. Le Coze redid the menu, his uncomplicated but thoughtful cooking
aided by the growing popularity of nouvelle cuisine. A consultant taught
Ms. Le Coze how to control costs. They began weighing the fish that
arrived from the market rather than simply paying whatever was asked.

The two were invited to the apartment of the renowned chef
\href{https://www.nytimes3xbfgragh.onion/1992/02/09/magazine/food-regarding-guerard.html}{Michel
Guérard}, who subsequently invited the food critic for L'Express to join
him at Le Bernardin. A review appeared shortly afterward, praising Mr.
Le Coze's cooking as well as Ms. Le Coze's ``tasteful interior'' and
``seductive smile.''

\hypertarget{dancing-on-the-tables}{%
\subsection{Dancing on the Tables}\label{dancing-on-the-tables}}

Ms. Le Coze claims to have been entirely different during those Paris
days --- going out to nightclubs, dancing on tables and banquettes. Her
lifestyle would have stunned the staff of today's Le Bernardin, she
said. ``This is the Maguy you do not know.''

In August, when the restaurant and the rest of Paris closed, she would
go to Saint-Tropez while her brother went home to Brittany. She dressed
in cowboy boots and short shorts. ``That was my license one month a year
to go crazy,'' she said. ``But when I came back to Paris in September
and a customer would say to me, `We saw you in Saint-Tropez,' I would
say, `No, you must be wrong.' When I am back in Paris, I am running the
restaurant and I am different.''

Eventually, she and her brother moved Le Bernardin to a bigger space
near the Arc de Triomphe and earned two Michelin stars. Yet Ms. Le Coze
remained fascinated by the idea of a restaurant in New York, which she
had visited twice in the 1970s. ``To me it was a vision, a spiritual
thing, if you believe in those things,'' she said.

In the early 1980s, 10 friends each promised to invest \$100,000 in a
small restaurant not far from the Lipstick Building in Midtown. Four
eventually backed out. Two years after that*,* she and her brother
received an offer from the Equitable insurance company to open a much
grander establishment on 51st Street west of Avenue of the Americas,
then considered a risky area for a restaurant.

Once plans were finalized, Le Bernardin went up in fewer than six
months. The feat was made possible by ``overtime, overtime and more
overtime,'' said her friend Gail George, the wife of the late
\href{http://www.legacy.com/obituaries/nytimes/obituary.aspx?n=philip-george\&pid=189162854}{Philip
George}, who designed the interior.

Image

Sister and brother in Central Park in 1986, the year they opened in New
York.

The restaurant opened in January 1986, and later that year the Le Cozes
sold its Paris sibling to the chef Guy Savoy. Three months in, the Le
Bernardin in New York received a
\href{https://www.nytimes3xbfgragh.onion/1986/03/28/arts/restaurants-023486.html}{four-star
review} from The Times, a rating it maintained even after her brother's
death in 1994.

Mr. Le Coze died in an ambulance of a heart attack suffered while
working out at a health club. He was 49.

Getting Ms. Le Coze to speak about him is difficult; getting her to
express her feelings after his death is impossible. Friends say she
never totally recovered from the shock.

She recalled, more sad than angry, ``After Gilbert died, every other
restaurant tried to take people from us. They said, `She is not here for
long.''' Nobody left, and under Mr. Ripert, who took over as chef and
several years later became her business partner, the restaurant thrived.

From 1994 until Ms. Le Coze redecorated her New York apartment two years
ago, the only photographs on display there were of her brother.

``She and Gilbert were like one person,'' Ms. George said. ``Maguy was
always fussing over him, adored him like I have never seen anyone adore
a younger brother. None of Maguy's boyfriends back then lasted long. As
long as she and Gilbert had each other, they didn't need anybody else.
She provided the asset of a wife without being a wife.''

\hypertarget{life-in-black-and-white}{%
\subsection{Life in Black and White}\label{life-in-black-and-white}}

In January, this very private woman was drawn into a very public lawsuit
filed by a former server at Le Bernardin.

The plaintiff alleged that she had been sexually harassed by both the
kitchen staff and the wait staff, and that Ms. Le Coze had strongly
suggested she lose weight after the birth of her daughter. The
litigation came at a time when the restaurant industry has been troubled
by a multitude of accusations that employees have been verbally or
physically abused.

The woman
\href{https://www.reuters.com/article/us-new-york-le-bernardin/former-le-bernardin-server-ends-lawsuit-against-new-york-restaurant-idUSKCN1IJ1X5}{dropped
her lawsuit} in May; her lawyer said no settlement had been made. Ms. Le
Coze would not comment on the suit, citing the advice of her lawyer. She
said she barely knew the woman who sued and ``doesn't recall working
directly with her.''

``Why would I be ashamed of a woman being pregnant?'' she asked. ``I do
not have children, but my friends have children. It is not reasonable
that I think like that.''

She said she has never been harassed in a restaurant. ``Never once, I
swear, never.'' She attributes this to her demeanor, which she describes
as, ``Stop, don't go any further, don't try.''

At work she is frank, sensible and at times relentless, but she
professed not to be overly demanding of her staff, merely confident that
she knows what is right and wrong.

``For me, there is black and white,'' she said. ``Never gray. Gray is
not my life. I am a strong person, strong with myself and with others.
You don't come from a small village in Brittany and open one of the top
restaurants in New York if you are a weak person.''

Ms. Le Coze said that becoming the boss of Le Bernardin was not easily
accomplished, and that she has no intention of giving up the job.

``Absolutely no retirement,'' she said. ``I am still young.''

\href{https://www.facebookcorewwwi.onion/nytfood/}{\emph{Follow NYT Food
on Facebook}}\emph{,}
\href{https://instagram.com/nytfood}{\emph{Instagram}}\emph{,}
\href{https://twitter.com/nytfood}{\emph{Twitter}} \emph{and}
\href{https://www.pinterest.com/nytfood/}{\emph{Pinterest}}\emph{.}
\href{https://www.nytimes3xbfgragh.onion/newsletters/cooking}{\emph{Get
regular updates from NYT Cooking, with recipe suggestions, cooking tips
and shopping advice}}\emph{.}

Advertisement

\protect\hyperlink{after-bottom}{Continue reading the main story}

\hypertarget{site-index}{%
\subsection{Site Index}\label{site-index}}

\hypertarget{site-information-navigation}{%
\subsection{Site Information
Navigation}\label{site-information-navigation}}

\begin{itemize}
\tightlist
\item
  \href{https://help.nytimes3xbfgragh.onion/hc/en-us/articles/115014792127-Copyright-notice}{©~2020~The
  New York Times Company}
\end{itemize}

\begin{itemize}
\tightlist
\item
  \href{https://www.nytco.com/}{NYTCo}
\item
  \href{https://help.nytimes3xbfgragh.onion/hc/en-us/articles/115015385887-Contact-Us}{Contact
  Us}
\item
  \href{https://www.nytco.com/careers/}{Work with us}
\item
  \href{https://nytmediakit.com/}{Advertise}
\item
  \href{http://www.tbrandstudio.com/}{T Brand Studio}
\item
  \href{https://www.nytimes3xbfgragh.onion/privacy/cookie-policy\#how-do-i-manage-trackers}{Your
  Ad Choices}
\item
  \href{https://www.nytimes3xbfgragh.onion/privacy}{Privacy}
\item
  \href{https://help.nytimes3xbfgragh.onion/hc/en-us/articles/115014893428-Terms-of-service}{Terms
  of Service}
\item
  \href{https://help.nytimes3xbfgragh.onion/hc/en-us/articles/115014893968-Terms-of-sale}{Terms
  of Sale}
\item
  \href{https://spiderbites.nytimes3xbfgragh.onion}{Site Map}
\item
  \href{https://help.nytimes3xbfgragh.onion/hc/en-us}{Help}
\item
  \href{https://www.nytimes3xbfgragh.onion/subscription?campaignId=37WXW}{Subscriptions}
\end{itemize}
