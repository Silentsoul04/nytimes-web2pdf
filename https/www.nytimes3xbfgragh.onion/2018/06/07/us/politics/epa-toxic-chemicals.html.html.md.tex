Sections

SEARCH

\protect\hyperlink{site-content}{Skip to
content}\protect\hyperlink{site-index}{Skip to site index}

\href{https://www.nytimes3xbfgragh.onion/section/politics}{Politics}

\href{https://myaccount.nytimes3xbfgragh.onion/auth/login?response_type=cookie\&client_id=vi}{}

\href{https://www.nytimes3xbfgragh.onion/section/todayspaper}{Today's
Paper}

\href{/section/politics}{Politics}\textbar{}The Chemical Industry Scores
a Big Win at the E.P.A.

\url{https://nyti.ms/2JmSxGU}

\begin{itemize}
\item
\item
\item
\item
\item
\item
\end{itemize}

Advertisement

\protect\hyperlink{after-top}{Continue reading the main story}

Supported by

\protect\hyperlink{after-sponsor}{Continue reading the main story}

\hypertarget{the-chemical-industry-scores-a-big-win-at-the-epa}{%
\section{The Chemical Industry Scores a Big Win at the
E.P.A.}\label{the-chemical-industry-scores-a-big-win-at-the-epa}}

\includegraphics{https://static01.graylady3jvrrxbe.onion/images/2018/06/08/us/08regs-toxic-print/merlin_58032689_a2a31050-eaa4-4040-93ff-a85ed8bbfd98-articleLarge.jpg?quality=75\&auto=webp\&disable=upscale}

By \href{https://www.nytimes3xbfgragh.onion/by/eric-lipton}{Eric Lipton}

\begin{itemize}
\item
  June 7, 2018
\item
  \begin{itemize}
  \item
  \item
  \item
  \item
  \item
  \item
  \end{itemize}
\end{itemize}

WASHINGTON --- The Trump administration, after
\href{https://www.documentcloud.org/documents/4496312-How-the-EPA-Revised-the-Way-It-Will-Evaluate.html\#document/p24/a427924}{heavy
lobbying} by the chemical industry, is scaling back the way the federal
government determines health and safety risks associated with the most
dangerous chemicals on the market,
\href{https://www.documentcloud.org/documents/4496312-How-the-EPA-Revised-the-Way-It-Will-Evaluate.html}{documents
from the Environmental Protection Agency} show.

\href{https://www.epa.gov/assessing-and-managing-chemicals-under-tsca/frank-r-lautenberg-chemical-safety-21st-century-act}{Under
a law passed by Congress} during the final year of the Obama
administration, the E.P.A. was required for the first time to evaluate
hundreds of potentially toxic chemicals and determine if they should
face new restrictions, or even be removed from the market. The chemicals
include many in everyday use, such as
\href{https://www.epa.gov/sites/production/files/2016-09/documents/tetrachloroethylene.pdf}{dry-cleaning
solvents},
\href{https://www.epa.gov/sites/production/files/2014-03/documents/ffrro_factsheet_contaminant_14-dioxane_january2014_final.pdf}{paint
strippers} and substances used in health and beauty products like
shampoos and cosmetics.

But as it moves forward reviewing the first batch of
\href{https://www.nytimes3xbfgragh.onion/2017/10/21/us/epa-toxic-chemicals.html}{10
chemicals}, the E.P.A. has in most cases decided to exclude from its
calculations any potential exposure caused by the substances' presence
in the air, the ground or water, according to
\href{https://www.documentcloud.org/documents/4495888-Problem-Forumulations.html}{more
than 1,500 pages} of documents released last week by the agency.

Instead, the agency will focus on possible harm caused by direct contact
with a chemical in the workplace or elsewhere. The approach means that
the improper disposal of chemicals --- leading to the contamination of
drinking water, for instance --- will often not be a factor in deciding
whether to restrict or ban them.

The approach is a big victory for the chemical industry, which has
repeatedly pressed the E.P.A. to narrow the scope of its risk
evaluations. Nancy B. Beck, the Trump administration's appointee to
\href{https://www.nytimes3xbfgragh.onion/2017/10/21/us/trump-epa-chemicals-regulations.html}{help
oversee the E.P.A.'s toxic chemical unit}, previously worked as an
executive at the American Chemistry Council, one of the industry's main
lobbying groups.

A spokesman for the E.P.A. said that the Clean Air Act, the Clean Water
Act and other laws already provided the agency with the authority to
regulate chemicals found in the air, rivers and drinking water, so there
was no need to revisit them under the 2016 law, which updated the
\href{https://www.epa.gov/laws-regulations/summary-toxic-substances-control-act}{Toxic
Substances Control Act} of 1976.

The agency can ``better protect human health and the environment by
focusing on those pathways that are likely to represent the greatest
areas of concern to E.P.A.,'' said the spokesman, Jahan Wilcox.

Image

Nancy B. Beck, who oversees the E.P.A.'s toxic chemical unit, previously
worked at one of the industry's main lobbying groups.Credit...U.S.
Senate Committee Channel

But three former agency officials, including a former supervisor of the
toxic chemical program, said that the E.P.A.'s approach would result in
a flawed analysis of the threat presented by chemicals.

``It is ridiculous,'' said
\href{https://19january2017snapshot.epa.gov/aboutepa/wendy-cleland-hamnett-principal-deputy-assistant-administrator-office-chemical-safety-and_.html}{Wendy
Cleland-Hamnett}, who retired last year after nearly four decades at the
E.P.A., where she ran the toxic chemical unit during her last year.
``You can't determine if there is an unreasonable risk without doing a
comprehensive risk evaluation.''

Senator Tom Udall, Democrat of New Mexico, and Representative Frank
Pallone Jr., Democrat of New Jersey, who played leading roles in passing
the 2016 law, said the E.P.A. was ignoring its directive for a
comprehensive analysis of risks.

``Congress worked hard in bipartisan fashion to reform our nation's
broken chemical safety laws, but Pruitt's E.P.A. is failing to put the
new law to use as intended,'' Mr. Udall said in a statement referring to
Scott Pruitt, the E.P.A. administrator.

A spokesman for Senator John Barrasso, Republican of Wyoming, who is
chairman of the Senate committee that oversees the agency, declined to
comment.

Cumulatively, the approach being taken for the 10 chemicals means the
E.P.A.'s risk analysis will not take into account an estimated
\href{http://blogs.edf.org/health/files/2018/06/First-Ten-Emissions-TRI.pdf}{68
million pounds a year} of emissions, according to an analysis by the
Environmental Defense Fund, based on agency data.

Dr. Beck declined requests for comment. She had
\href{https://www.documentcloud.org/documents/4496312-How-the-EPA-Revised-the-Way-It-Will-Evaluate.html\#document/p33/a427929}{pushed
the E.P.A. during the Obama administration} to narrow the scope of the
risk evaluations, in a fashion similar to the approach under her watch.

Also helping oversee the risk evaluation effort is Erik Baptist, a
former senior lawyer at
\href{https://www.documentcloud.org/documents/4387727-Erik-Baptist-Financial-Disclosure.html}{the
American Petroleum Institute}, another big player in the chemical
industry.

The American Chemistry Council said in a statement last week that the
E.P.A.'s approach met ``the requirements of the law,'' adding that it
wanted the risk assessments to be ``protective and practical.''

\includegraphics{https://static01.graylady3jvrrxbe.onion/images/2018/06/07/us/07regs-toxics2/07regs-toxics2-articleLarge.jpg?quality=75\&auto=webp\&disable=upscale}

Under the approach, the E.P.A. will examine what harm can be caused, for
example, to anyone directly exposed to
\href{https://www.epa.gov/assessing-and-managing-chemicals-under-tsca/risk-evaluation-perchloroethylene}{perchloroethylene}
--- a dry-cleaning solvent and metal degreaser designated by the E.P.A.
as a
\href{https://www.epa.gov/sites/production/files/2016-09/documents/tetrachloroethylene.pdf}{likely
carcinogen} --- during manufacturing or when using it in dry cleaning,
carpet cleaning or handling certain ink-removal products.

But
\href{https://www.documentcloud.org/documents/4496312-How-the-EPA-Revised-the-Way-It-Will-Evaluate.html\#document/p4/a428062}{the
agency will not focus} on exposures that occur from traces of the
\href{https://www.ewg.org/tapwater/contaminant.php?contamcode=2987\#.Wxkdgu4vwkI}{chemical
found in drinking water} in 44 states as a result of improper disposal
over decades, the E.P.A. documents say. The
\href{https://www.documentcloud.org/documents/4496312-How-the-EPA-Revised-the-Way-It-Will-Evaluate.html\#document/p3/a428061}{decision
conflicts with a risk assessment plan} detailed by the agency a year
ago, which included drinking water. And the change came after the
American Chemistry Council
\href{https://www.documentcloud.org/documents/4496312-How-the-EPA-Revised-the-Way-It-Will-Evaluate.html\#document/p27/a427925}{argued
in February last year} that ``the E.P.A. has discretion to select the
conditions of use that it will consider.''

The agency will also not consider the hazards of perchloroethylene
discharged into streams or lakes, landfills or the air from dry-cleaning
stores or manufacturing or processing plants, the documents say.

The documents contain similar conclusions about nine of the 10 chemicals
under review. One of these is
\href{https://www.epa.gov/assessing-and-managing-chemicals-under-tsca/risk-evaluation-14-dioxane}{1,4-dioxane},
which can be found in small amounts in antifreeze, deodorants, shampoos
and cosmetics and is considered ``likely to be carcinogenic to humans.''
Another is trichloroethylene, which is used to make a refrigerant
chemical and remove grease from metal parts and is
\href{https://www.atsdr.cdc.gov/phs/phs.asp?id=171\&tid=30}{associated
with} cancers of the liver, kidneys and blood.

Other changes identified in the E.P.A. documents narrow the definitions
of certain chemicals, including asbestos. Some asbestos-like fibers will
not be included in the risk assessments, one agency staff member said,
nor will the 8.8 million pounds a year of asbestos deposited in
hazardous landfills or the 13.1 million pounds discarded in routine dump
sites.

The most likely outcome of the changes will be that the agency finds
lower levels of risks associated with many chemicals, and as a result,
imposes fewer new restrictions or prohibitions, several current and
former agency officials said.

``They don't want to open Pandora's box by looking comprehensively at
the risk, as they may prove to be significant and then they have to deal
with it,'' said \href{https://www.lw.com/people/RobertMSussman}{Robert
M. Sussman}, a former chemical industry lawyer and E.P.A. official who
now works as a consultant to \href{https://saferchemicals.org/}{Safer
Chemicals, Healthy Families}, an advocacy group.

Despite the changes, the E.P.A. is
\href{https://www.epa.gov/newsreleases/epa-announces-action-methylene-chloride}{still
expected} to ban the use of
\href{https://www.epa.gov/assessing-and-managing-chemicals-under-tsca/risk-evaluation-methylene-chloride-0}{methylene
chloride} as a paint stripper soon --- an action first proposed at the
end of the Obama administration. The chemical, one of the 10 under
review, is a popular ingredient used in dozens of products sold at home
improvement stores nationwide, and has been blamed in dozens of deaths.

A collection of more than a dozen groups --- representing environmental,
public-health and labor organizations --- are
\href{https://www.edf.org/sites/default/files/Petitioners_Opening_Brief.pdf}{suing
the E.P.A.} to challenge earlier changes in the toxic chemical
evaluation program. The case is before the United States Court of
Appeals for the Ninth Circuit in San Francisco.

Advertisement

\protect\hyperlink{after-bottom}{Continue reading the main story}

\hypertarget{site-index}{%
\subsection{Site Index}\label{site-index}}

\hypertarget{site-information-navigation}{%
\subsection{Site Information
Navigation}\label{site-information-navigation}}

\begin{itemize}
\tightlist
\item
  \href{https://help.nytimes3xbfgragh.onion/hc/en-us/articles/115014792127-Copyright-notice}{©~2020~The
  New York Times Company}
\end{itemize}

\begin{itemize}
\tightlist
\item
  \href{https://www.nytco.com/}{NYTCo}
\item
  \href{https://help.nytimes3xbfgragh.onion/hc/en-us/articles/115015385887-Contact-Us}{Contact
  Us}
\item
  \href{https://www.nytco.com/careers/}{Work with us}
\item
  \href{https://nytmediakit.com/}{Advertise}
\item
  \href{http://www.tbrandstudio.com/}{T Brand Studio}
\item
  \href{https://www.nytimes3xbfgragh.onion/privacy/cookie-policy\#how-do-i-manage-trackers}{Your
  Ad Choices}
\item
  \href{https://www.nytimes3xbfgragh.onion/privacy}{Privacy}
\item
  \href{https://help.nytimes3xbfgragh.onion/hc/en-us/articles/115014893428-Terms-of-service}{Terms
  of Service}
\item
  \href{https://help.nytimes3xbfgragh.onion/hc/en-us/articles/115014893968-Terms-of-sale}{Terms
  of Sale}
\item
  \href{https://spiderbites.nytimes3xbfgragh.onion}{Site Map}
\item
  \href{https://help.nytimes3xbfgragh.onion/hc/en-us}{Help}
\item
  \href{https://www.nytimes3xbfgragh.onion/subscription?campaignId=37WXW}{Subscriptions}
\end{itemize}
