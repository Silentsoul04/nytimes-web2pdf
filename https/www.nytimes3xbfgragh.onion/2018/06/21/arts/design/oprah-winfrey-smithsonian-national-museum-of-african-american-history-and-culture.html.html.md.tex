Sections

SEARCH

\protect\hyperlink{site-content}{Skip to
content}\protect\hyperlink{site-index}{Skip to site index}

\href{https://www.nytimes3xbfgragh.onion/section/arts/design}{Art \&
Design}

\href{https://myaccount.nytimes3xbfgragh.onion/auth/login?response_type=cookie\&client_id=vi}{}

\href{https://www.nytimes3xbfgragh.onion/section/todayspaper}{Today's
Paper}

\href{/section/arts/design}{Art \& Design}\textbar{}Oprah Earned This
Museum Show. And It's a Potent Spectacle.

\url{https://nyti.ms/2IcPkrM}

\begin{itemize}
\item
\item
\item
\item
\item
\end{itemize}

Advertisement

\protect\hyperlink{after-top}{Continue reading the main story}

Supported by

\protect\hyperlink{after-sponsor}{Continue reading the main story}

Critic's Notebook

\hypertarget{oprah-earned-this-museum-show-and-its-a-potent-spectacle}{%
\section{Oprah Earned This Museum Show. And It's a Potent
Spectacle.}\label{oprah-earned-this-museum-show-and-its-a-potent-spectacle}}

\includegraphics{https://static01.graylady3jvrrxbe.onion/images/2018/06/22/arts/22OPRAH-TOP/22OPRAH-TOP-articleLarge.jpg?quality=75\&auto=webp\&disable=upscale}

By \href{http://www.nytimes3xbfgragh.onion/by/wesley-morris}{Wesley
Morris}

\begin{itemize}
\item
  June 21, 2018
\item
  \begin{itemize}
  \item
  \item
  \item
  \item
  \item
  \end{itemize}
\end{itemize}

WASHINGTON --- Has any American spent more of her career flinging her
arms up in shock and elation than Oprah Winfrey? Maybe --- \emph{maybe
---} certain long-suffering and spoiled-rotten sports fans. But in 25
years of hosting a daytime talk show five days a week, nine months out
of 12, often to gigantic ratings, Ms. Winfrey raised her arms a lot ---
over makeovers and giveaways and celebrity surprises, like that time, in
2011, nearing the final broadcast of ``The Oprah Winfrey Show,'' when
she \href{https://www.youtube.com/watch?v=Le4eqSw6GMY}{turned around and
saw Stevie Wonder}, at a piano, rising out of an arena floor. O.K., it
was only one arm, but it went up with the force of two. Her reaction was
part ``buzzer beater,'' part ``pageant win.''

She and her thousands of hours of TV are now the subjects of a big,
fascinating exhibition at the Smithsonian's National Museum of African
American History and Culture that captures what the show was, did and
has meant. And it includes that Stevie Wonder moment, in a short montage
focused on the show's leap, in its final years, into lavish,
thrill-a-minute mega-production.

\includegraphics{https://static01.graylady3jvrrxbe.onion/images/2018/06/22/arts/22oprahlegacy3/merlin_139635567_2d98e491-48ea-46ef-bc06-daadc9283d8d-articleLarge.jpg?quality=75\&auto=webp\&disable=upscale}

The installation ---
\href{https://nmaahc.si.edu/explore/exhibitions/watchingoprah}{``Watching
Oprah: The Oprah Winfrey Show and American Culture''} --- is what can
only be called \emph{so} Smithsonian: a mouthwatering, heartwarming,
eye-opening, foot-aching conflation of biography, anthropology,
sociology, nostalgia, history and insight (about culture, race, gender,
technology, media, education, consumerism, economics, beauty, fashion
and the law) into a potent dioramic spectacle. You leave it with a
fuller understanding of Ms. Winfrey's rare determination to matter to
everybody and in awe of how acutely she still does.

That bit with Mr. Wonder lasts about 15 seconds in the montage. But it
sticks with you. For one thing, it exposes something endearingly normal
in a woman, who as a very famous TV host, could presumably have had Mr.
Wonder pop out of any floor anytime she wanted. For another, the women
in the audience (thousands of them, black and white; I didn't spot a
single man) leap and shout and wail, with \emph{their} arms in the air,
in a dozen different ways, from ``Lotto win'' to ``praise Jesus.''
They're going nuts for Mr. Wonder, obviously. But they might be more
ecstatic about the joy he's bringing Ms. Winfrey. This isn't a cult,
exactly. It's a living, screaming symbiotic social network. Affirmation
and intent became muscular cornerstones of the ``Oprah Winfrey''
enterprise. When she banged the ``like'' button, her vast constituency
banged on it, too.

The show's been off the air for seven years, and we miss it: More than a
year before ``Watching Oprah,'' Chicago's WBEZ released the podcast
``Making Oprah,'' a delicious behind-the-scenes casserole that Jenn
White served with a fan's appreciation and a critic's forensic eye. And
Ms. Winfrey hasn't disappeared at all. Since ``Oprah'' went off the air,
Winfrey has evolved into an even more instinctive screen actress, for
one thing. She's
\href{https://books.google.com/books/about/The_Wisdom_of_Sundays.html?id=0dw1DwAAQBAJ\&printsec=frontcover\&source=kp_read_button\#v=onepage\&q\&f=false}{written
books}; she's vividly alive on Instagram, her cable network and wellness
podcast; and she
\href{https://www.nytimes3xbfgragh.onion/slideshow/2018/05/19/style/the-royal-wedding/s/xxRoyal-Wedding-Arrive-slide-6NCY.html}{looked
supremely tickled} to be at Meghan Markle and Prince Harry's wedding
last month.

And yet, if America can't actually \emph{miss} Oprah Winfrey, it might
miss an \emph{idea} of her. As ``Oprah Winfrey Show'' Oprah, a totem of
humanity, respect, largess and fun. As a figure of immense, almost
assaultive generosity, who could unleash, in us, bombastic yet utterly
sincere gratitude.

Maybe the chaos extremity of current events has made us wistful for the
moral authority of ``Oprah Winfrey'': school massacres, police shootings
of unarmed black people, men chronically mistreating women, the
government's separation of children from their migrant parents. Whenever
somebody pleads for a national conversation --- about anything, really
--- what they're saying is, ``Where the hell is Oprah?''

Many a desperate ``O''-shaped Bat signal has gone up in the last
half-dozen years, and in January, the country believed she was answering
it. That's one way to interpret the thunderous response to her speech
\href{https://www.nytimes3xbfgragh.onion/2018/01/07/movies/oprah-winfrey-golden-globes-speech-transcript.html}{accepting
the Cecil B. DeMille Award} for lifetime achievement at the Golden
Globes. \emph{Oprah for President!} As oratory, the moment really was
electrifying. She endorsed the evening's gender-equity platform, in
part, by telling the story of
\href{https://www.nytimes3xbfgragh.onion/2017/12/29/obituaries/recy-taylor-alabama-rape-victim-dead.html}{Recy
Taylor}, a black woman from Alabama whom white men repeatedly raped one
night in 1944. She denounced the coarseness of our national moment by
praising the news media. Her pulpit gravitas would have brought down the
house at a political convention. But the speech wasn't enough. People
seemed desperate to work themselves into
\href{https://www.nytimes3xbfgragh.onion/2018/01/08/movies/oprah-winfrey-lifetime-achievement-golden-globes.html}{Oprah
2020 fever}.

To the extent that ``queen of daytime'' is any kind of office, it's one
Ms. Winfrey has never abused. She loves people, and she seems to
understand the intensity of people's love for her. But people also love
power, and Ms. Winfrey's display of it that night (and perhaps a New
York Post column
\href{http://ew.com/tv/2017/09/29/oprah-for-president-2020/}{she
retweeted}) sparked pandemonium for her to ride it into Washington.
President Oprah was fantasized about as an antidote to a caustic,
whimsical president: the woman with the extensive
\href{http://www.oprah.com/pressroom/about-oprahs-angel-network}{``angel
network''} taking on a master Twitter troll, one television genius
locking horns with another.

But the Smithsonian show leaves you thinking that she'd probably expect
better fantasies from us. It makes you think she might be too good for
whatever a candidate would have to do or say in this political climate
to be elected president of anything.

\textbf{BEFORE YOU EXIT} ``Watching Oprah,'' you've scrutinized a case
full of childhood photos, diary entries, high school letters and
\href{https://www.youtube.com/watch?v=fx447ShQLeE}{a signed copy of Maya
Angelou's} ``I Know Why the Caged Bird Sings.'' You've soaked up the
music, speeches, imagery and writing in a room devoted to the musicians,
actors, authors and political movements that helped a young Oprah
determine who she wanted to be. You've checked out the amusingly
arranged spot devoted to her Oscar-losing performance in ``The Color
Purple'' (she had her Oscar luncheon biscuit bronzed, instead) and the
space that enumerates her early television-news work, including a
three-minute montage of her in Baltimore and Chicago in the 1970s and
1980s that is one of the most charming pieces of editing you're going to
see. At some point, a young Ms. Winfrey, in spandex, has to put her legs
up for an aerobics-class segment and jestingly complains, ``Oh, you're
gonna love this shot.''

Image

Ms. Winfrey's high school scrapbook, from around 1971.Credit...Harpo
Inc.

Image

Ms. Winfrey was the first black contestant to win the title of Miss Fire
Prevention, in 1971, at this Nashville pageant. She represented the
local radio station where she worked and told the judges she aspired to
be a television journalist like Barbara Walters.Credit...Metropolitan
Government Archives of Nashville

You've walked through the replication of the short, declining hallway
Ms. Winfrey trod to get to the stage. It opens into the space devoted to
the show itself --- a pair of armchairs on a platform in front of a big
monitor that plays a six-minute highlight reel. You've seen the large,
almost sentient, encased Sony TV camera and the signed guest books and a
copy of one of the show's look books and some of Oprah's actual outfits:
the black turtleneck and leather pants she put on for Tina Turner, the
gown from the DeMille speech, the legendary Calvins she wore the day she
unsheathed the slim new figure that would vaguely haunt the show.

Image

A mock set of ``The Oprah Winfrey Show,'' with a pair of armchairs in
front of a monitor that plays a highlight reel. To the right is a giant
wall with the titles and airdates of Ms. Winfrey's 4,561
shows.Credit...Justin T. Gellerson for The New York Times

Image

Two microphones and the glass mug and metal straw used by Ms. Winfrey
during the final season of her show.Credit...Justin T. Gellerson for The
New York Times

You've noticed the decades of hair styles and pivots in emotional
intent, from the loaded confrontations of what the exhibition reminds us
was once called ``talk-back TV'' to the ``best self'' era --- basically,
from her inspiring the cage matches of Jerry Springer and Maury Povich
to featuring the real talk of Iyanla Vanzant, the gender decoding of
John Gray and the Richter-scale-registering impact of Oprah's Book Club.
You've close-read the two blue cards of questions --- printed and
handwritten, presumably by the host --- for Ms. Winfrey's first
conversation with Tom Cruise after his
\href{https://www.youtube.com/watch?v=qQgXEkL3NV4}{excruciating pounce
on her sofa}. (``Do you have any regrets about anything these last three
years?'')

Image

The Calvin Klein jeans worn by Ms. Winfrey on the 1988 episode in which
she revealed a slim new figure. The jeans are among some of her actual
outfits on display in the exhibition.Credit...Harpo Inc

You've stood aghast before the giant wall printed, randomly, with every
single one of the show's 4,561 titles and airdates. ``What Do You Stand
For?'' (4/24/00). ``Tipper Gore on Depression'' (6/22/99). ``Wives
Confess They Are Gay'' (10/2/06). ``Men Who Can't Be Intimate''
(7/21/88). ``Sexual Abuse Ramifications'' (4/14/88). ``Jennifer Aniston
and Beyoncé'' (11/13/2008). ``How Safe Is Your Home When It's Alone''
(12/1/06). ``Cooking With Patti LaBelle'' (7/2/99). ``Donald J. Trump''
(4/25/1988). ``Are You Normal? Take the Test!'' (12/1/2010). ``What Is a
Wigger?'' (9/9/93). ``How to Use Your Life'' (4/10/00). You've noticed
that the wall seems to reach to an absurdly illegible height. It could
double as the meanest vision test of all time.

There's a lot here. And you depart it all mystified by the absurd
contradictions that Ms. Winfrey's achievements reveal about this
country. Here's a black woman who grew up poor in the segregated South
and became the country's first black female billionaire. Her prosperity
inspired others to prosper, yet ``Watching Oprah'' is situated not far
from the museum's moral and scholastic centerpiece (``A People's
Journey''), a devastating
\href{https://www.nytimes3xbfgragh.onion/2016/09/22/arts/design/smithsonian-african-american-museum-review.html}{odyssey}
down into --- and then up out of --- the creation of the United States
from slavery, racism, revolution, innovation, hard work and good luck.
She doesn't seem to know how she made it, but like a lot of successful
Americans, she appears to have moments when she can't entirely believe
she has.

Image

An advertisement from Ms. Winfrey's early days as an evening news anchor
in Nashville in the mid-1970s.Credit...NewsChannel 5, Nashville, Tenn.

You wonder whether the show's integrationist philosophy arises from its
host's having been raised, reared and professionally trained in
Milwaukee, Mississippi, Tennessee and the broadcast environs of
Baltimore and Chicago. Just geographically, Ms. Winfrey is
intersectional. But it also explains something like the
\href{https://www.youtube.com/watch?v=WErjPmFulQ0}{trip the show took}
in 1987 to Forsyth County, Ga., after it purged itself of nearly all its
black residents. She wanted to know what about black people so scared
the white residents, and she keeps having to remind the racists in her
audience that the woman interrogating them is also black.

Ms. Winfrey contributed more than \$20 million to the sponsorship of the
museum. So there's an urge to distrust the intent of an exhibition like
this, to say that she bought it. But her museum donation doesn't seem at
all like vanity. It's ``how to use your life,'' ``what do you stand
for'' money. Across from ``A People's Journey'' sits the Oprah Winfrey
Theater. Maybe she paid for a piece of \emph{that}. Anyway, our tax
dollars are hard at work here, too. So Ms. Winfrey just paid a little
more than I did.

Nonetheless, ``Watching Oprah,'' in its uncompromised captioning, goes
out of its way to remind you about the chronic dissatisfaction, among
some black people, with the lack of attention to the crises of black
America. The show includes a 1986 letter from a black woman upset that
Ms. Winfrey didn't call on her during a broadcast because she didn't
``look like an ugly, fat, uneducated, frustrated black woman which is
typical of the majority of the women you allow to speak on your show.''
If that was ever true (suburban white women made up its biggest
demographic), it wasn't that way for long.

This might be the only show in television history to feature a ferocious
four-way argument among black women about being a Republican. You watch
a moment like that, in the exhibition's ``Talk-Back TV'' montage, and
you remember the show's deep roots as a roving dialogue, often through
national events, tragedies and disasters, with Ms. Winfrey holding the
microphone (several of which are on display). It was a show that, in
1992, devoted a handful of daring episodes to racism, including a couple
after the Los Angeles riots and one that featured a panel of American
Indians and a white audience actually \emph{hearing} the panelists'
dismay. Even when it was in the mud, ``The Oprah Winfrey Show'' was
determined to make so-called rednecks understand the problem
with``redskins.''

\textbf{ONE PROBLEM WITH} being really good at your job is that people
won't let you stop doing it. But you watch enough of these montages and
realize two things. First, ``Watching Oprah'' needs a lot more of
``Oprah'' to watch, more clips, segments, whole episodes, something.
Second, Oprah didn't do this work alone. She helped \emph{us} do it. She
was a platform. She was Facebook. Forget the presidency. She was the
facilitator in chief.

The more she empowered us to speak, the better she got at knowing how
her emotional algorithm could supply us with books and feelings and
tools for betterment. And she took real risks to better understand this
country, too.

That Forsyth County episode might have been a stunt, but it's more
audacious than Geraldo Rivera's dragging millions of Americans into a
bloody brawl with skinheads the following year. ``Watching Oprah''
doesn't privilege any one episode over any other. So it's hard, at
first, to see what \emph{exactly} it is about the show that matters. But
then you think about that massive wall of episode titles and how it's
impossible to take it all the way in. And that incomprehensible vastness
seems perfectly right, both for the enduring vitality of the show itself
and the woman at its center.

Advertisement

\protect\hyperlink{after-bottom}{Continue reading the main story}

\hypertarget{site-index}{%
\subsection{Site Index}\label{site-index}}

\hypertarget{site-information-navigation}{%
\subsection{Site Information
Navigation}\label{site-information-navigation}}

\begin{itemize}
\tightlist
\item
  \href{https://help.nytimes3xbfgragh.onion/hc/en-us/articles/115014792127-Copyright-notice}{©~2020~The
  New York Times Company}
\end{itemize}

\begin{itemize}
\tightlist
\item
  \href{https://www.nytco.com/}{NYTCo}
\item
  \href{https://help.nytimes3xbfgragh.onion/hc/en-us/articles/115015385887-Contact-Us}{Contact
  Us}
\item
  \href{https://www.nytco.com/careers/}{Work with us}
\item
  \href{https://nytmediakit.com/}{Advertise}
\item
  \href{http://www.tbrandstudio.com/}{T Brand Studio}
\item
  \href{https://www.nytimes3xbfgragh.onion/privacy/cookie-policy\#how-do-i-manage-trackers}{Your
  Ad Choices}
\item
  \href{https://www.nytimes3xbfgragh.onion/privacy}{Privacy}
\item
  \href{https://help.nytimes3xbfgragh.onion/hc/en-us/articles/115014893428-Terms-of-service}{Terms
  of Service}
\item
  \href{https://help.nytimes3xbfgragh.onion/hc/en-us/articles/115014893968-Terms-of-sale}{Terms
  of Sale}
\item
  \href{https://spiderbites.nytimes3xbfgragh.onion}{Site Map}
\item
  \href{https://help.nytimes3xbfgragh.onion/hc/en-us}{Help}
\item
  \href{https://www.nytimes3xbfgragh.onion/subscription?campaignId=37WXW}{Subscriptions}
\end{itemize}
