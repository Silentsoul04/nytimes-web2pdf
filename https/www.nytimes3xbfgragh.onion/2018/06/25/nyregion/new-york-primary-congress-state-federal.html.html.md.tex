Sections

SEARCH

\protect\hyperlink{site-content}{Skip to
content}\protect\hyperlink{site-index}{Skip to site index}

\href{https://www.nytimes3xbfgragh.onion/section/nyregion}{New York}

\href{https://myaccount.nytimes3xbfgragh.onion/auth/login?response_type=cookie\&client_id=vi}{}

\href{https://www.nytimes3xbfgragh.onion/section/todayspaper}{Today's
Paper}

\href{/section/nyregion}{New York}\textbar{}Only in New York: Where
Primary Day Comes Twice a Year

\url{https://nyti.ms/2IrCEOa}

\begin{itemize}
\item
\item
\item
\item
\item
\end{itemize}

Advertisement

\protect\hyperlink{after-top}{Continue reading the main story}

Supported by

\protect\hyperlink{after-sponsor}{Continue reading the main story}

\hypertarget{only-in-new-york-where-primary-day-comes-twice-a-year}{%
\section{Only in New York: Where Primary Day Comes Twice a
Year}\label{only-in-new-york-where-primary-day-comes-twice-a-year}}

\includegraphics{https://static01.graylady3jvrrxbe.onion/images/2018/06/26/nyregion/26voting/merlin_140305155_fc3af2f0-10cf-4ee6-951b-d3c86471467c-articleLarge.jpg?quality=75\&auto=webp\&disable=upscale}

By \href{https://www.nytimes3xbfgragh.onion/by/lisa-w-foderaro}{Lisa W.
Foderaro}

\begin{itemize}
\item
  June 25, 2018
\item
  \begin{itemize}
  \item
  \item
  \item
  \item
  \item
  \end{itemize}
\end{itemize}

As New Yorkers go to the polls for the congressional primary on Tuesday,
the ballots they cast will have a dubious distinction: Their state is
the only in the nation that is holding separate state and federal
primary elections in this tumultuous political year.

How and why did that happen?

The practice dates to 2012 when the
\href{https://cityroom.blogs.nytimes3xbfgragh.onion/2012/01/27/judge-moves-congressional-primary-date-to-june/}{federal
government won} a lawsuit compelling New York to move its federal
primary elections to the fourth Tuesday in June from September, when
they had been held simultaneously with state primaries. The federal
government asked for the change because it contended that a September
primary did not leave enough time for military and overseas voters to
get absentee ballots for the general election.

State lawmakers then could have moved the state primary election from
September to June so that they would again line up. Although Democrats
favored the switch, Republicans in Albany balked, arguing that because
the state legislative session runs through the end of June, they would
have no time in their districts to campaign.

New York's bifurcated primaries, however, have caused the state to spend
millions of dollars more; at the same time, having two separate
primaries can also suppress voter interest, election law experts and
state officials say.

``It adds to voter confusion and it reduces turnout,'' said Jerry H.
Goldfeder, an election lawyer and an adjunct professor at Fordham
University School of Law. ``Until the Legislature sees fit to enact a
rational election calendar, we are stuck with two different primaries.''

The system would also seem to favor incumbents. James Felton Keith, a
Democrat who dropped his challenge to Representative Adriano Espaillat
in the spring, said in a phone interview that special-interest groups
and advocates have done a good job at trying to rally voters to the
polls.

But some of the most powerful groups, he said, were more focused on
state-oriented issues and saved their firepower for the state primary in
September.

``If you are a state-based group, you are not inclined to work as hard
in June as you are in September,'' he said, adding that aligning the two
primaries would give candidates ``maximum impact.''

Jonathan Lewis, a Democrat who is challenging Representative Eliot L.
Engel, wooed voters Monday evening as they disembarked from a
Metro-North train at the Fleetwood train station in Westchester County.
He said that many voters seemed unaware of the June 26 primary. ``The
two separate dates causes some hesitancy in voters' minds about when to
vote,'' he said, ``and that's certainly not ideal for participation in a
democracy.''

A survey of \href{https://ballotpedia.org/Main_Page}{Ballotpedia}, the
\href{http://www.thegreenpapers.com/}{Green Papers} and the
\href{http://www.ncsl.org/}{National Conference of State Legislatures}
--- organizations that track election dates --- revealed that New York
is alone in holding separate congressional and state primaries this
year.

But at least a few other states occasionally split their primaries as
well. In recent years, for example, both California and North Carolina
have held separate primaries.

Gov. Andrew M. Cuomo of New York has for years criticized the
Legislature for not voting to consolidate the primaries. In a statement
on Monday, Richard Azzopardi, Mr. Cuomo's spokesman, said that ``the
governor has been very clear that the status quo is wasteful, expensive
and only leads to confusion and lower voter turnout. He's long sought a
unified primary date, which the Legislature hasn't been able to agree
on.''

Holding a separate primary is costly given the logistics of overseeing a
statewide election. According to the New York State Board of Elections,
the primary on Tuesday will cost about \$10 million, which covers
ballots, Election Day materials, inspector pay, poll site rentals,
moving costs for voting machines, staff and mailings.

By contrast, a general election runs about \$25 million, said John
Conklin, a spokesman for the Board of Elections.

``If you combine the federal primary with the state and local primary,
you would save some money --- there's no question. But I don't think you
can say that it would save the entire cost of the separate election,''
he said, explaining that there would still be some duplicate expenses
relating to ballots, additional machines and mailings.

Mr. Goldfeder and others say that moving the state primaries to June
would capture the focus of voters before they leave for summer
vacations. ``People are paying attention and it doesn't coincide with
the start of school,'' he said.

As for which party was put at a disadvantage by split primaries, Mr.
Goldfeder said the jury was out. ``It does hurt voter turnout, but I
don't think it favors one party over another,'' he said. ``It's like
trying to prove a negative.''

Advertisement

\protect\hyperlink{after-bottom}{Continue reading the main story}

\hypertarget{site-index}{%
\subsection{Site Index}\label{site-index}}

\hypertarget{site-information-navigation}{%
\subsection{Site Information
Navigation}\label{site-information-navigation}}

\begin{itemize}
\tightlist
\item
  \href{https://help.nytimes3xbfgragh.onion/hc/en-us/articles/115014792127-Copyright-notice}{©~2020~The
  New York Times Company}
\end{itemize}

\begin{itemize}
\tightlist
\item
  \href{https://www.nytco.com/}{NYTCo}
\item
  \href{https://help.nytimes3xbfgragh.onion/hc/en-us/articles/115015385887-Contact-Us}{Contact
  Us}
\item
  \href{https://www.nytco.com/careers/}{Work with us}
\item
  \href{https://nytmediakit.com/}{Advertise}
\item
  \href{http://www.tbrandstudio.com/}{T Brand Studio}
\item
  \href{https://www.nytimes3xbfgragh.onion/privacy/cookie-policy\#how-do-i-manage-trackers}{Your
  Ad Choices}
\item
  \href{https://www.nytimes3xbfgragh.onion/privacy}{Privacy}
\item
  \href{https://help.nytimes3xbfgragh.onion/hc/en-us/articles/115014893428-Terms-of-service}{Terms
  of Service}
\item
  \href{https://help.nytimes3xbfgragh.onion/hc/en-us/articles/115014893968-Terms-of-sale}{Terms
  of Sale}
\item
  \href{https://spiderbites.nytimes3xbfgragh.onion}{Site Map}
\item
  \href{https://help.nytimes3xbfgragh.onion/hc/en-us}{Help}
\item
  \href{https://www.nytimes3xbfgragh.onion/subscription?campaignId=37WXW}{Subscriptions}
\end{itemize}
