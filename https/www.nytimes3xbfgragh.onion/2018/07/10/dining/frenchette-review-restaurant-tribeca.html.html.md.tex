Sections

SEARCH

\protect\hyperlink{site-content}{Skip to
content}\protect\hyperlink{site-index}{Skip to site index}

\href{https://www.nytimes3xbfgragh.onion/section/food}{Food}

\href{https://myaccount.nytimes3xbfgragh.onion/auth/login?response_type=cookie\&client_id=vi}{}

\href{https://www.nytimes3xbfgragh.onion/section/todayspaper}{Today's
Paper}

\href{/section/food}{Food}\textbar{}It's Not Fake French, It's
Frenchette

\url{https://nyti.ms/2mciDT7}

\begin{itemize}
\item
\item
\item
\item
\item
\item
\end{itemize}

Advertisement

\protect\hyperlink{after-top}{Continue reading the main story}

Supported by

\protect\hyperlink{after-sponsor}{Continue reading the main story}

\href{/column/restaurant-review}{Restaurant Review}

\hypertarget{its-not-fake-french-its-frenchette}{%
\section{It's Not Fake French, It's
Frenchette}\label{its-not-fake-french-its-frenchette}}

\href{https://www.nytimes3xbfgragh.onion/slideshow/2018/07/10/dining/frenchette-nyc.html}{}

\hypertarget{an-old-formula-with-new-freedom}{%
\subsection{An Old Formula With New
Freedom}\label{an-old-formula-with-new-freedom}}

9 Photos

View Slide Show ›

\includegraphics{https://static01.graylady3jvrrxbe.onion/images/2018/07/11/dining/11rest-1/merlin_140661672_e3fa7066-cec8-4ed5-a71c-aba6893e816b-articleLarge.jpg?quality=75\&auto=webp\&disable=upscale}

Cole Wilson for The New York Times

\begin{itemize}
\tightlist
\item
  Frenchette\\
  **NYT Critic's Pick ★★★ French \$\$\$ 241 West Broadway 212-334-3883
\end{itemize}

\href{https://resy.com/cities/ny/frenchette?utm_source=nyt\&utm_medium=restoprofile\&utm_campaign=affiliates\&aff_id=c1fe784}{Reserve
a Table}

When you make a reservation at an independently reviewed restaurant
through our site, we earn an affiliate commission.

By \href{https://www.nytimes3xbfgragh.onion/by/pete-wells}{Pete Wells}

\begin{itemize}
\item
  July 10, 2018
\item
  \begin{itemize}
  \item
  \item
  \item
  \item
  \item
  \item
  \end{itemize}
\end{itemize}

Riad Nasr and Lee Hanson named their new restaurant after a David
Johansen song from 1978,
\href{https://www.youtube.com/watch?v=i7m7gAzkRrg}{``Frenchette.''} The
first line is, ``You call that love in French, but it's just
Frenchette,'' and later when he rhymes that with ``naturalette'' and
``leatherette'' you know the suffix isn't diminutive, it's dismissive.
The song is about what you do once you figure out that you're not going
to get the real thing, and Mr. Johansen's answer is simple enough.
``Let's just dance,'' he sings. Never mind love.

\href{https://www.frenchettenyc.com/}{Frenchette}, which opened in
TriBeCa in April, isn't fake French, but it isn't the real thing,
either. Since 1997, when
\href{https://www.nytimes3xbfgragh.onion/2016/11/02/dining/keith-mcnally-opening-a-restaurant-augustine.html}{Keith
McNally} hired them as the opening chefs at
\href{https://www.nytimes3xbfgragh.onion/2004/05/26/dining/restaurants-a-soho-brasserie-now-authentically-worn.html}{Balthazar},
Mr. Nasr and Mr. Hanson have cooked side by side, building a kind of
brasserie-steakhouse hybrid out of standards from both genres. Mr.
McNally kept loading them up with new restaurants, giving them joint
command of Pastis, Schiller's Liquor Bar and
\href{https://www.nytimes3xbfgragh.onion/2009/05/20/dining/reviews/20rest.html}{Minetta
Tavern}, none of which could really be called chef-driven.

Frenchette, their own place and their first without Mr. McNally, takes
some liberties with the formula, but not enough to get them recognized
as visionaries on the level of
\href{http://www.pierre-gagnaire.com/}{Pierre Gagnaire}. Frenchette
says: Never mind art, let's just cook.

The restaurant is divided into two chambers. In front and on display to
the street is a lounge with Art Deco curves, where bartenders
percussively clack shakers behind a long river of zinc. If you have a
seat, this room is the height of metropolitan civility. If you are
waiting for one, as many people are on any given night, it's
purgatorial.

The dining room is in back, behind a pair of arches and up a step so
slight that its only conceivable purpose is to raise the insurance
premiums; every time I approached it a worried server materialized to
tell me to look out. The small tables are too small and the big ones,
encircled by red-leather banquettes the size of living-room sofas, give
you enough space to spread out a map while you eat. The room is grandly
scaled and conspicuously underdesigned, a simplification of the
brasserie archetype that doesn't try to reproduce every chipped tile and
nicotine stain. It's a room for people who have outgrown illusions.

That must describe half the people in New York, judging by the number of
times I've been asked how to get a reservation at Frenchette at an hour
when an adult might reasonably be in the mood for dinner. Even
well-placed magazine editors with highly resourceful assistants can end
up eating at 6 or 10 p.m., as I did.

\emph{{[}Read about some of the}
\href{https://www.nytimes3xbfgragh.onion/2018/11/15/nyregion/best-new-nyc-restaurants.html?action=click\&module=Intentional\&pgtype=Article}{\emph{best
new restaurants in New York City}} \emph{(for now).{]}}

In the brasserie mode, the menu rambles. It goes on too long; not
everything on it is worth rambling about. But once you sort it out, it's
full of dishes worth planning a night around.

Some of them don't look wildly impressive unless you know how they're
made. For each order of brouillade, a pan of eggs has to be stirred
constantly over a small flame for a long time, until they look like
grits. (There's a reason you don't see brouillade on many menus.)
Dropped on top are a few excellent snails in parsley and garlic, a
buttery garnish for very buttery scrambled eggs.

Calf's liver was sweet and custardy, the goal in cooking it, and one
that's not often reached. Sweetbreads with creamy white insides were
fried to a pale, crinkled gold, then served with a brown, French, lovely
and anachronistic sauce made from veal jus and crayfish. Lobster was
roasted on a rotisserie, sending some of the shell's flavor into the
meat, which was then given a very luxurious bath in curry butter.

The entire roast chicken is juicy without tasting of brine, a rare thing
these days. Hidden under the drumsticks and thighs are rafts of baguette
that sat under the rotisserie, imbibing every drop that fell from the
bird. Some diners might say bread wet with drippings is too homespun for
a dish that costs \$68. They don't know what they're talking about.

The menu is full of things that take years of practice to get right.
Duck breast with a side of fries sounds ridiculously simple, but look at
the crackle on that duck skin, and listen to that echoing crunch on the
fries. If these chefs learned anything from Mr. McNally, it's that
people will overlook many lapses if you feed them great French fries. At
Frenchette, they overlook the peculiar steak knife brought out with the
duck; its sharp side is straight and its dull side is curved, and
everybody who picks one up makes a joke about slicing open a finger.
That would be one way to turn tables faster, but I never saw any
casualties.

The kitchen is at its most sure-footed with big pieces of animals,
whether they're classed as appetizers or main courses. (The guinea hen
terrine appetizer is a chunky pink brick that could serve as dinner.)
All those meats, combined with the chefs' fondness for fat-enhanced
sauces that cling to your lips, give the menu a wintry feel, something
that the recent proliferation of peas and artichokes on the plates here
doesn't quite hide.

\includegraphics{https://static01.graylady3jvrrxbe.onion/images/2018/07/11/dining/11rest-6/merlin_140661627_1124d4da-9278-4e9a-acfe-485a783c12bd-articleLarge.jpg?quality=75\&auto=webp\&disable=upscale}

So far, Mr. Hanson and Mr. Nasr have responded to the hot weather by
changing ingredients but not the menu's spirit. Green celtuce stems can
be a nice crisp side dish for summer, but Frenchette serves them in a
steaming-hot gratin, slick with bone marrow; it's perfect for January.
The blood sausage slab on top of a buckwheat galette that surfaced at
the end of June was as seasonal as earmuffs.

Reversing the natural order of the universe, the appetizers and the
smaller ``amuses'' often seem more tentative than the main courses.

Blowfish tails brushed with mustard and spice-bearing bread crumbs are
terrific when they're available, and the chefs should get some kind of
award from the French government for serving iced oysters
Burgundy-style, with peppery finger-length chipolata sausages. But
gumball-size smoked eel beignets with a crème-fraîche ranch dressing
called ``Franchette'' didn't taste much like eel; pig-foot croquettes
were a washout; and a bowl of oily whole-wheat spaghetti with bottarga
needed to be slapped with more lemon and salt.

With desserts, Frenchette tries the blunt approach that works so well
for its main courses, and comes up short. Not very short, but enough to
make you wish more attention were being paid. The mille-feuille doesn't
shatter on impact like the ones at
\href{https://www.nytimes3xbfgragh.onion/2016/06/15/dining/le-coq-rico-review.html}{Le
Coq Rico} or
\href{https://www.nytimes3xbfgragh.onion/2016/02/03/dining/benoit-review.html}{Benoit},
and the pastry crust under the long band of cooked fruit in the apple
tart needs to be crisper or more buttery.

One of the more clever moves Mr. Hanson and Mr. Nasr made is hiring
Jorge Riera to shepherd a list of natural, biodynamic and organic wines.
Not every sommelier who has become infatuated with these labels knows
where to find the good stuff, but he does. This means, though, that you
will probably not end up drinking what you thought you wanted to drink.
There are few safe harbors of familiarity; even if you just want
sparkling water, you have a choice of Vichy Catalan or Mondariz.

You'd have to be delusional to wager money on the longevity of a
restaurant in New York, but Frenchette is the closest thing to a safe
bet that I've seen in years. Mr. Nasr and Mr. Hanson seem to be in it
for the long haul; if one of them hasn't held the other's head down in a
pot of French onion soup by now, it's probably never going to happen.
More than that, they know how to flatter New Yorkers with food that
doesn't scream for attention but that smiles back when you notice it.

\href{https://www.facebookcorewwwi.onion/nytfood/}{\emph{Follow NYT Food
on Facebook}}\emph{,}
\href{https://instagram.com/nytfood}{\emph{Instagram}}\emph{,}
\href{https://twitter.com/nytfood}{\emph{Twitter}} \emph{and}
\href{https://www.pinterest.com/nytfood/}{\emph{Pinterest}}\emph{.}
\href{https://www.nytimes3xbfgragh.onion/newsletters/cooking}{\emph{Get
regular updates from NYT Cooking, with recipe suggestions, cooking tips
and shopping advice}}\emph{.}

Advertisement

\protect\hyperlink{after-bottom}{Continue reading the main story}

\hypertarget{site-index}{%
\subsection{Site Index}\label{site-index}}

\hypertarget{site-information-navigation}{%
\subsection{Site Information
Navigation}\label{site-information-navigation}}

\begin{itemize}
\tightlist
\item
  \href{https://help.nytimes3xbfgragh.onion/hc/en-us/articles/115014792127-Copyright-notice}{©~2020~The
  New York Times Company}
\end{itemize}

\begin{itemize}
\tightlist
\item
  \href{https://www.nytco.com/}{NYTCo}
\item
  \href{https://help.nytimes3xbfgragh.onion/hc/en-us/articles/115015385887-Contact-Us}{Contact
  Us}
\item
  \href{https://www.nytco.com/careers/}{Work with us}
\item
  \href{https://nytmediakit.com/}{Advertise}
\item
  \href{http://www.tbrandstudio.com/}{T Brand Studio}
\item
  \href{https://www.nytimes3xbfgragh.onion/privacy/cookie-policy\#how-do-i-manage-trackers}{Your
  Ad Choices}
\item
  \href{https://www.nytimes3xbfgragh.onion/privacy}{Privacy}
\item
  \href{https://help.nytimes3xbfgragh.onion/hc/en-us/articles/115014893428-Terms-of-service}{Terms
  of Service}
\item
  \href{https://help.nytimes3xbfgragh.onion/hc/en-us/articles/115014893968-Terms-of-sale}{Terms
  of Sale}
\item
  \href{https://spiderbites.nytimes3xbfgragh.onion}{Site Map}
\item
  \href{https://help.nytimes3xbfgragh.onion/hc/en-us}{Help}
\item
  \href{https://www.nytimes3xbfgragh.onion/subscription?campaignId=37WXW}{Subscriptions}
\end{itemize}
