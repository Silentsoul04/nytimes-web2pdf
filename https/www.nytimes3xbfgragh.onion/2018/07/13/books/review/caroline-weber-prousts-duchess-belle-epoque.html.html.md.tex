Sections

SEARCH

\protect\hyperlink{site-content}{Skip to
content}\protect\hyperlink{site-index}{Skip to site index}

\href{https://www.nytimes3xbfgragh.onion/section/books/review}{Book
Review}

\href{https://myaccount.nytimes3xbfgragh.onion/auth/login?response_type=cookie\&client_id=vi}{}

\href{https://www.nytimes3xbfgragh.onion/section/todayspaper}{Today's
Paper}

\href{/section/books/review}{Book Review}\textbar{}French High Society
During the Belle Époque

\url{https://nyti.ms/2LcVyKB}

\begin{itemize}
\item
\item
\item
\item
\item
\end{itemize}

Advertisement

\protect\hyperlink{after-top}{Continue reading the main story}

Supported by

\protect\hyperlink{after-sponsor}{Continue reading the main story}

Nonfiction

\hypertarget{french-high-society-during-the-belle-uxe9poque}{%
\section{French High Society During the Belle
Époque}\label{french-high-society-during-the-belle-uxe9poque}}

\includegraphics{https://static01.graylady3jvrrxbe.onion/images/2018/07/12/books/review/12showalter5/12showalter5-articleLarge.jpg?quality=75\&auto=webp\&disable=upscale}

Buy Book ▾

\begin{itemize}
\tightlist
\item
  \href{https://www.amazon.com/gp/search?index=books\&tag=NYTBSREV-20\&field-keywords=Proust\%27s+Duchess\%3A+How+Three+Celebrated+Women+Captured+the+Imagination+of+Fin-De-Si\%C3\%A8cle+Paris+Caroline+Weber}{Amazon}
\item
  \href{https://du-gae-books-dot-nyt-du-prd.appspot.com/buy?title=Proust\%27s+Duchess\%3A+How+Three+Celebrated+Women+Captured+the+Imagination+of+Fin-De-Si\%C3\%A8cle+Paris\&author=Caroline+Weber}{Apple
  Books}
\item
  \href{https://www.anrdoezrs.net/click-7990613-11819508?url=https\%3A\%2F\%2Fwww.barnesandnoble.com\%2Fw\%2F\%3Fean\%3D9780307961785}{Barnes
  and Noble}
\item
  \href{https://www.anrdoezrs.net/click-7990613-35140?url=https\%3A\%2F\%2Fwww.booksamillion.com\%2Fp\%2FProust\%2527s\%2BDuchess\%253A\%2BHow\%2BThree\%2BCelebrated\%2BWomen\%2BCaptured\%2Bthe\%2BImagination\%2Bof\%2BFin-De-Si\%25C3\%25A8cle\%2BParis\%2FCaroline\%2BWeber\%2F9780307961785}{Books-A-Million}
\item
  \href{https://bookshop.org/a/3546/9780307961785}{Bookshop}
\item
  \href{https://www.indiebound.org/book/9780307961785?aff=NYT}{Indiebound}
\end{itemize}

When you purchase an independently reviewed book through our site, we
earn an affiliate commission.

By Elaine Showalter

\begin{itemize}
\item
  July 13, 2018
\item
  \begin{itemize}
  \item
  \item
  \item
  \item
  \item
  \end{itemize}
\end{itemize}

\textbf{PROUST'S DUCHESS}\\
\textbf{How Three Celebrated Women Captured the Imagination of
Fin-De-Siècle Paris}\\
By Caroline Weber\\
Illustrated. 715 pp. Alfred A. Knopf. \$35.

Even if you haven't read ``À la Recherche du Temps Perdu,'' you
shouldn't be afraid to read ``Proust's Duchess,'' Caroline Weber's
beguiling group biography of three aristocratic \emph{salonnières} of
Parisian high society in the
\href{https://www.mdc.edu/wolfson/academic/artsletters/art_philosophy/humanities/belleepoque.htm}{Belle
Époque}. Together they inspired the composite figure of the exquisite,
elegantly remote Duchesse de Guermantes, the muse of Proust's dreams of
chivalric French history, romance and \emph{la vie Parisienne}.
Watching, following, and even stalking them as a young law student in
the 1890s, he idolized Geneviève Halévy Bizet Straus; Laure de Sade
(great-granddaughter of the Marquis), who became Comtesse de Chevigné;
and especially
\href{http://www.fitnyc.edu/museum/exhibitions/prousts-muse.php}{Élisabeth
Comtesse Greffulhe}, whom he imagined as a goddess ``made of different
stuff from ordinary people.''

They viewed themselves, and were praised by Proust and others, as the
exotic golden birds of French high society. ``When confined to an aviary
or a cage,'' Élisabeth wrote, ``woman cultivates her plumage and her
song for reasons of pleasure, power.'' Her kinsman, the gay aesthete
\href{https://bonjourparis.com/history/robert-de-montesquiou/}{Robert de
Montesquiou}, called her ``the Swan,'' and coined the term ``cygniform''
to describe her sinuous grace; and she adopted mythic personas from
``Swan Lake,'' ``Lohengrin'' and the story of Leda, wrapping herself in
a frothy winglike cloak of swan's plumes, gliding in white satin mules
trimmed with swan's feathers or waving a gigantic swan's-down fan.

All three women married for money and status, and lived unhappily with
neurasthenic mothers, chilly in-laws, unfaithful husbands, disappointing
children, loneliness, depression and ennui. But they were brilliant in
exploiting the language of fashion. Upper-class Parisiennes changed
their clothes seven or eight times a day, so they had many opportunities
to model their wardrobes for society photographers like Otto Wegener and
\href{https://publicdomainreview.org/collections/photographs-of-the-famous-by-felix-nadar/}{Nadar}.
The beautiful, wasp-waisted Élisabeth showed particular genius in
``\emph{le shopping poétique}'': ``I believe there is no ecstasy in the
world,'' she wrote, ``that can compare to the ecstasy of a woman who
feels she is the object of every gaze, and draws nourishment and joy
from the crowd.'' In her lavish accessories and spectacular gowns by
Worth and Fortuny, she courted admiration, celebrity and copious fan
mail.

Image

Laure, Comtesse de Chevigné, cultivated her persona as a sophisticated
intellectual and swaggering rebel. At the Bal des Bêtes in 1885, when
1,700 guests came costumed as insects, vermin, crustaceans and big-game
animals, she appeared as a white snowy owl, symbol of Minerva, goddess
of wisdom. In 1891, at the fashionable Black and White Ball, she defied
the invitation by dressing as an androgynous harlequin in yellow and
blue.

Geneviève had connections to Bohemia and the artistic world through her
first marriage to Georges Bizet. They established a salon in their
Montmartre home, attended by Turgenev, Berlioz, Jules Massenet, Charles
Gounod, Gustave Moreau and Gustave Doré, and led a summer artists'
colony in Bougival. After her second marriage, to the lawyer Georges
Straus, Geneviève began a new salon that became the center for the
pro-Dreyfusards. As the daughter of Fromental Halévy, the composer of
``The Jewess'' (1835) and ``The Wandering Jew'' (1852), she was active
in protesting the vicious anti-Semitism of the 1880s and 1890s.

But over all, the \emph{grandes dames} were uninterested in politics,
public affairs or feminism. As Caroline Kaufmann, the leader of the
militant women's rights group Solidarité des Femmes, wrote: ``They are
and want to remain the phoenix, the rare bird, the priceless object.
They don't care at all about women's emancipation, and for good reason
--- they wouldn't stand to gain anything from it.'' Although Élisabeth
longed to be a writer, and labored on a secret autobiographical novel
for 20 years, she never published it. She lived to be seen, not read.

Weber, a professor of French and comparative literature at Barnard
College, is an erudite literary historian as well as a fashion
connoisseur, and she spent years of archival research amassing the
sumptuous details, apt and amusing illustrations, lengthy endnotes, huge
bibliography and three appendixes of this engrossing story. She
describes not only the three women, but an enormous cast of the dandies,
decadents, artists, writers, musicians and financiers of the \emph{fin
de siècle}. Clearly Weber loves this period; while the book is long and
weighty, it is never dull. Still, I wish she had gone even longer
through the Dreyfus affair, which marked a tragic turn in what she calls
``a soon-to-be-extinct society.'' Geneviève Straus, for example, was
shunned by many of her noble guests, who no longer saw her as a
glamorous hostess but ``as a troublemaking Jew.''

Soon Proust was disenchanted by his goddesses. As Weber notes, his
``idealizing vision of his ladies changed over time into something
darker.'' To the Comtesse de Chevigné, he wrote, ``What one used to love
turns out to be \emph{very, very stupid}.'' She, he told a friend, was
just ``a tough old bird I mistook, long ago, for a bird of paradise.''
No longer infatuated, he mocked even the divine Élisabeth as
superficial, pretentious and shallow. And in old age, she proved his
point by remembering him as ``a displeasing little man who was forever
skulking about in doorways.''

Maybe you'll be tempted to give Proust another go when you read about
them all. In any case, Weber has succeeded much as he did in bringing
that lost time back to glorious life.

Advertisement

\protect\hyperlink{after-bottom}{Continue reading the main story}

\hypertarget{site-index}{%
\subsection{Site Index}\label{site-index}}

\hypertarget{site-information-navigation}{%
\subsection{Site Information
Navigation}\label{site-information-navigation}}

\begin{itemize}
\tightlist
\item
  \href{https://help.nytimes3xbfgragh.onion/hc/en-us/articles/115014792127-Copyright-notice}{©~2020~The
  New York Times Company}
\end{itemize}

\begin{itemize}
\tightlist
\item
  \href{https://www.nytco.com/}{NYTCo}
\item
  \href{https://help.nytimes3xbfgragh.onion/hc/en-us/articles/115015385887-Contact-Us}{Contact
  Us}
\item
  \href{https://www.nytco.com/careers/}{Work with us}
\item
  \href{https://nytmediakit.com/}{Advertise}
\item
  \href{http://www.tbrandstudio.com/}{T Brand Studio}
\item
  \href{https://www.nytimes3xbfgragh.onion/privacy/cookie-policy\#how-do-i-manage-trackers}{Your
  Ad Choices}
\item
  \href{https://www.nytimes3xbfgragh.onion/privacy}{Privacy}
\item
  \href{https://help.nytimes3xbfgragh.onion/hc/en-us/articles/115014893428-Terms-of-service}{Terms
  of Service}
\item
  \href{https://help.nytimes3xbfgragh.onion/hc/en-us/articles/115014893968-Terms-of-sale}{Terms
  of Sale}
\item
  \href{https://spiderbites.nytimes3xbfgragh.onion}{Site Map}
\item
  \href{https://help.nytimes3xbfgragh.onion/hc/en-us}{Help}
\item
  \href{https://www.nytimes3xbfgragh.onion/subscription?campaignId=37WXW}{Subscriptions}
\end{itemize}
