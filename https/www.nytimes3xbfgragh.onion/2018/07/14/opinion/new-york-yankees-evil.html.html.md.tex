Sections

SEARCH

\protect\hyperlink{site-content}{Skip to
content}\protect\hyperlink{site-index}{Skip to site index}

\href{https://myaccount.nytimes3xbfgragh.onion/auth/login?response_type=cookie\&client_id=vi}{}

\href{https://www.nytimes3xbfgragh.onion/section/todayspaper}{Today's
Paper}

\href{/section/opinion}{Opinion}\textbar{}The New York Yankees Are a
Moral Abomination

\url{https://nyti.ms/2NT4qqq}

\begin{itemize}
\item
\item
\item
\item
\item
\item
\end{itemize}

Advertisement

\protect\hyperlink{after-top}{Continue reading the main story}

Supported by

\protect\hyperlink{after-sponsor}{Continue reading the main story}

\href{/section/opinion}{Opinion}

\href{/column/sporting}{Sporting}

\hypertarget{the-new-york-yankees-are-a-moral-abomination}{%
\section{The New York Yankees Are a Moral
Abomination}\label{the-new-york-yankees-are-a-moral-abomination}}

By David Bentley Hart

Mr. Hart is~the author of ``The New Testament: A Translation.''

\begin{itemize}
\item
  July 14, 2018
\item
  \begin{itemize}
  \item
  \item
  \item
  \item
  \item
  \item
  \end{itemize}
\end{itemize}

\includegraphics{https://static01.graylady3jvrrxbe.onion/images/2018/07/15/opinion/sunday/15hart/merlin_141223467_c71ae7d7-2023-4550-87e9-56e57221d9d3-articleLarge.jpg?quality=75\&auto=webp\&disable=upscale}

Soberly considered, the New York Yankees and their fans present a moral
dilemma. Our consciences, naturally abhorring everything abominable,
tell us that such things simply ought not exist. And yet we also know
that the evil they represent is one we would not really want eradicated.
Somehow we depend on it, not because it appeals to some morbid
subliminal fascination with the horrific in us, and not even because it
teaches us about the world's deep Darwinian laws, but because it answers
to a psychological need.

By exciting in the rest of us that sweet cold loathing that only they
induce --- that strangely tender malice, at once so delicious and yet so
purifying --- the Yankees and their followers provide an emotional
cleansing. They give us occasion for the discharge of a dark, dangerous
passion, but one unburdened by guilt. The detestation that any rational
soul spontaneously feels for the Yankees is so innocent, so
uncontaminated by spite --- just instinctive revulsion before something
obscene, like the goat-headed god of the diabolists. And there are few
luxuries more gorgeously nourishing than the license to hate with an
unclouded conscience.

Yankees fans, of course, never having drunk from those healing springs,
typically mistake this hatred for envy, and so for an inverted
admiration. But nothing could be further from the truth. Yes, those of
us whose teams hail from smaller markets sometimes fall prey to a
\emph{slightly} petulant, even bilious resentment of all that boughten
glory --- the exorbitant free-agent contracts, the legions of scouts,
the colossal television revenues --- but who can blame us? And how could
we fail to be vexed by the fawning servility of a national media
incapable of telling the beautiful from the meretricious?

I mean, be reasonable: How often, as Derek Jeter's retirement approached
in 2014, were we made to endure the squealing ecstasies of television
announcers too bedazzled by the fastidious delicacy of his dainty
coupé-chassé en tournant on grounders to his right to notice his
minuscule range or flimsy arm? Why were we forced to see him awarded a
preposterous \emph{two} additional Gold Gloves in his dotage when his
defense was scarcely better than mediocre in his prime?

Who, moreover, can forget the obligatorily bibulous rhapsodies from
sports commentators in the waning days of the old Yankee Stadium in 2008
--- grown men dissolving in foaming raptures over a ``great tradition''
in its twilight or intoning solemn encomiums to the glorious ``temple of
sport'' soon be reduced to dust? \emph{Temple}, forsooth! More like the
largest brothel in the world, being torn down only because a larger,
glitzier brothel was being erected across the street. (Really, how does
a Yankees fan's pride in all those purchased championships differ from
the self-delusion of a man staggering out of a bawdy house at dawn,
complimenting himself on his magnificent powers of seduction?)

So, I confess it: There is some resentment. But it never degenerates
into emulousness or envy. No one elsewhere wants to root for a team
\emph{like} the Yankees. The notion is appalling. Could any franchise be
more devoid of romance? What has it ever represented but the brute power
of money? One can admire the St. Louis Cardinals' magnificent history,
or cherish fond memories of the great Baltimore Orioles, Cincinnati Reds
or Oakland A's teams of the past. But no morally sane soul could delight
in that graceless enormity in the Bronx, or its supremacy over smaller
markets. It is an intrinsically depraved pleasure, like a taste for
bearbaiting. And certainly none of us wants to be anything like Yankees
\emph{fans} --- especially after seeing them at close quarters.
Certainly, I have witnessed them often enough in Baltimore during
weekend series against my beloved Orioles to know the horror in full.

Not that the horror is easy to recall clearly. The trauma is too
violent. Memory cringes, whines, tries to slink away. One recollects
only a kaleidoscopic flux of gruesomely fragmentary impressions, too
outlandish to be perfectly accurate, too vivid to be entirely false:
nightmarish revenants from the dim haunts of the collective unconscious
\ldots{} monstrous, abortive shapes emerging from the abysmal murk of
evolutionary history \ldots{} things pre-hominid, even pre-mammalian
\ldots{} forms never quite resolving into discrete organisms, spilling
over and into one another, making it uncertain where one ends and
another begins. \ldots{} It really is awful: ghastly glistening flesh
\ldots{} tentacles coiling and uncoiling, stretching and contracting
\ldots{} lidless orbicular eyes eerily waving on slender stalks \ldots{}
squamous hides, barbed quills, the unguinous sheen of cutaneous toxins
\ldots{} serrated tails, craggy horns, sallow fangs, gleaming talons
\ldots{} fragrances fungal and poisonous \ldots{} sickly iridescences
undulating across pallid, gelatinous underbellies or shimmering along
slick, filmy scales. \ldots{}

And what raucous yawps of elation they emit, like sea lions crying out
in erotic transport. How languidly and grossly they intertwine with one
another --- how clumsily, lewdly, indiscriminately --- like lascivious
cephalopods merged in seething tangles of prehensile carnality. And
somehow, without having to see, one knows things about them: that the
categories ``parent,'' ``sibling'' and ``mate'' are only hazily
delineated in their minds; that they suck nourishment from cellulose,
heavy metals and cactus spines; that, should they grow hungry on the
journey home from the game, they may pull over to the side of the road
to devour their young. One simply knows. \ldots{}

Or so it seems to me now. Admittedly, my bitterness over the Orioles'
dismal play this season might be distorting my perspective a little. And
perhaps I am avoiding a truth about the Yankees and their fans that I
would rather not admit: that it is not everything grotesquely strange
about them that terrifies us, but rather everything that is all too
familiar. We may fear becoming like them; our greater fear, however, is
of discovering that we already are. And by ``we,'' I mean ``Americans.''

Major League Baseball, like America, is in decline. A faint air of doom
hangs about this most exquisite of games. The median age of its fans
rises each year; the young increasingly prefer other diversions; some
savants predict a contraction of the National and American Leagues in
the near future. Meanwhile, the only solutions the owners can contrive
are trivial measures for shortening time of play, and never with
appreciable effect. Yet the real cause of the problem is obvious. Though
there has always been an immense inequality of resources between the
richest and poorest franchises, the division has widened to catastrophic
proportions in recent decades. It is hard to persuade children to invest
their love in teams that cannot plausibly hope for a championship any
time within, oh, the first 30 years of their lives.

Yet M.L.B. would never consider the wisdom of creating a real system of
shared revenues and salary caps. The richest franchises --- among which
the Yankees enjoy archetypal pre-eminence --- are content to let the
poorest wither in a laissez-faire desert rather than make any reasonable
sacrifices for the common good. Thus the business of baseball ---
through greed, profligacy, shortsightedness and an insatiable appetite
for immediate gratification --- consumes itself by relentlessly allowing
its own \emph{communal} basis to disintegrate beneath it, and by
ignoring the needs of future generations.

The analogy is imperfect, but irresistible. America --- with its
decaying infrastructure, its third-world public transit, its shrinking
labor market, its evaporating middle class, its expanding gulf between
rich and poor, its heartless health insurance system, its mindless
indifference to a dying ecology, its predatory credit agencies, its
looming Social Security collapse, its interminable war, its
metastasizing national debt and all the social pathologies that gave it
a degenerate imbecile and child-abducting sadist as its president ---
remains the only developed economy in the world that believes it wrong
to use \emph{civic} wealth for \emph{civic} goods. Its absurdly engorged
military budget diverts hundreds of billions of dollars a year from the
public weal to those who profit from the military-industrial complex.
Its plutocratic policies and libertarian ethos are immune to all appeals
of human solidarity. It towers over the world, but promises secure
shelter only to the fortunate few.

And so, of course, the Yankees cannot help but be emblematic of
everything that characterizes us as a nation and as an idea: a thing
gargantuan and heedless, invincible and yet bizarrely fragile and
self-destructive. Still, I suppose one must be fair. M.L.B.'s decline,
America's --- the Yankees may contribute mightily to the former, but
they only epitomize the latter.

Though, truth be told, I would blame them for both if I could.

Advertisement

\protect\hyperlink{after-bottom}{Continue reading the main story}

\hypertarget{site-index}{%
\subsection{Site Index}\label{site-index}}

\hypertarget{site-information-navigation}{%
\subsection{Site Information
Navigation}\label{site-information-navigation}}

\begin{itemize}
\tightlist
\item
  \href{https://help.nytimes3xbfgragh.onion/hc/en-us/articles/115014792127-Copyright-notice}{©~2020~The
  New York Times Company}
\end{itemize}

\begin{itemize}
\tightlist
\item
  \href{https://www.nytco.com/}{NYTCo}
\item
  \href{https://help.nytimes3xbfgragh.onion/hc/en-us/articles/115015385887-Contact-Us}{Contact
  Us}
\item
  \href{https://www.nytco.com/careers/}{Work with us}
\item
  \href{https://nytmediakit.com/}{Advertise}
\item
  \href{http://www.tbrandstudio.com/}{T Brand Studio}
\item
  \href{https://www.nytimes3xbfgragh.onion/privacy/cookie-policy\#how-do-i-manage-trackers}{Your
  Ad Choices}
\item
  \href{https://www.nytimes3xbfgragh.onion/privacy}{Privacy}
\item
  \href{https://help.nytimes3xbfgragh.onion/hc/en-us/articles/115014893428-Terms-of-service}{Terms
  of Service}
\item
  \href{https://help.nytimes3xbfgragh.onion/hc/en-us/articles/115014893968-Terms-of-sale}{Terms
  of Sale}
\item
  \href{https://spiderbites.nytimes3xbfgragh.onion}{Site Map}
\item
  \href{https://help.nytimes3xbfgragh.onion/hc/en-us}{Help}
\item
  \href{https://www.nytimes3xbfgragh.onion/subscription?campaignId=37WXW}{Subscriptions}
\end{itemize}
