Sections

SEARCH

\protect\hyperlink{site-content}{Skip to
content}\protect\hyperlink{site-index}{Skip to site index}

\href{https://www.nytimes3xbfgragh.onion/section/food}{Food}

\href{https://myaccount.nytimes3xbfgragh.onion/auth/login?response_type=cookie\&client_id=vi}{}

\href{https://www.nytimes3xbfgragh.onion/section/todayspaper}{Today's
Paper}

\href{/section/food}{Food}\textbar{}Missy Robbins Embraces Simplicity,
at Misi

\url{https://nyti.ms/2Nqp5Sj}

\begin{itemize}
\item
\item
\item
\item
\item
\end{itemize}

Advertisement

\protect\hyperlink{after-top}{Continue reading the main story}

Supported by

\protect\hyperlink{after-sponsor}{Continue reading the main story}

The restaurant Preview

\hypertarget{missy-robbins-embraces-simplicity-at-misi}{%
\section{Missy Robbins Embraces Simplicity, at
Misi}\label{missy-robbins-embraces-simplicity-at-misi}}

The Lilia chef's second Italian restaurant, also in Williamsburg,
Brooklyn, includes an airy, glassed-in room where pasta is made.

\includegraphics{https://static01.graylady3jvrrxbe.onion/images/2018/08/29/dining/29missy1-sub/merlin_142571682_c13c529d-c175-4493-898f-f7e167a47e65-articleLarge.jpg?quality=75\&auto=webp\&disable=upscale}

\href{https://www.nytimes3xbfgragh.onion/by/florence-fabricant}{\includegraphics{https://static01.graylady3jvrrxbe.onion/images/2018/07/16/multimedia/author-florence-fabricant/author-florence-fabricant-thumbLarge.png}}

By
\href{https://www.nytimes3xbfgragh.onion/by/florence-fabricant}{Florence
Fabricant}

\begin{itemize}
\item
  Aug. 28, 2018
\item
  \begin{itemize}
  \item
  \item
  \item
  \item
  \item
  \end{itemize}
\end{itemize}

Missy Robbins, the chef and owner **** of Lilia in Williamsburg,
Brooklyn, now has a challenge: distinguishing Misi, her new restaurant,
from its older sibling about a mile away.

From nearly the moment it opened in 2016, Lilia was a hit with diners
and
\href{https://www.nytimes3xbfgragh.onion/2016/03/30/dining/lilia-restaurant-review.html}{critics,
who extolled} the Italian menu and her touch with pastas.

``I expected Lilia to be a neighborhood restaurant where people would
just come in for a plate of pasta, but people are really dining there,
and they come from all over,'' Ms. Robbins said. ``Misi is meant to be a
little simpler.''

The space will not have a wood-burning oven, but it will still have some
eye-catching features. Among them are a gleaming stretch of open kitchen
along one side and an airy, glassed-in room for making pasta on the
other, where three to five artisans will be on view, mixing, rolling,
extruding, twisting and filling. The two butcher-block work tables,
where the pasta will be made during the day, will be available for
private dining at night.

If all goes well, Ms. Robbins expects that on weekends, she'll be
serving about 500 portions of pasta a day. She does not plan to sell the
pasta retail.

``I've been working on all of this, in my head, while I'm running
Lilia,'' she said.

The subdued gray, white and black design by Peter Guzy, in a more
straightforward room than Lilia's quirky space, includes a total of 98
seats, 35 at a counter facing the kitchen and the bar. (Ms. Robbins says
that, from her experience at Lilia, they add warmth and make dining fun
and interactive.)

She added that she did not intend to serve anything from the menu at
Lilia. ``The menu will be very simple,'' Ms. Robbins said, ``10 varied
pastas and 10 vegetable dishes, some with proteins.''

Among the pastas she expects to serve are tortelli in brown butter with
a spinach and Swiss chard filling; bigoli with pork sugo; and chickpea
pappardelle with rosemary. The vegetable-focused dishes may include
charred radicchio with bone marrow and balsamic vinegar, and olive
oil-poached zucchini with capers, oregano and torn grilled bread. There
may be a nightly special. For dessert, there will be gelato.

Her new wine director, Eliza Christen, came from the highly rated
Meadowood resort in the Napa Valley and is assembling a collection,
mostly Mediterranean, that's more varied than the all-Italian list at
Lilia.

\textbf{Misi} 329 Kent Avenue (South Fourth Street), Williamsburg,
Brooklyn, September.

Advertisement

\protect\hyperlink{after-bottom}{Continue reading the main story}

\hypertarget{site-index}{%
\subsection{Site Index}\label{site-index}}

\hypertarget{site-information-navigation}{%
\subsection{Site Information
Navigation}\label{site-information-navigation}}

\begin{itemize}
\tightlist
\item
  \href{https://help.nytimes3xbfgragh.onion/hc/en-us/articles/115014792127-Copyright-notice}{©~2020~The
  New York Times Company}
\end{itemize}

\begin{itemize}
\tightlist
\item
  \href{https://www.nytco.com/}{NYTCo}
\item
  \href{https://help.nytimes3xbfgragh.onion/hc/en-us/articles/115015385887-Contact-Us}{Contact
  Us}
\item
  \href{https://www.nytco.com/careers/}{Work with us}
\item
  \href{https://nytmediakit.com/}{Advertise}
\item
  \href{http://www.tbrandstudio.com/}{T Brand Studio}
\item
  \href{https://www.nytimes3xbfgragh.onion/privacy/cookie-policy\#how-do-i-manage-trackers}{Your
  Ad Choices}
\item
  \href{https://www.nytimes3xbfgragh.onion/privacy}{Privacy}
\item
  \href{https://help.nytimes3xbfgragh.onion/hc/en-us/articles/115014893428-Terms-of-service}{Terms
  of Service}
\item
  \href{https://help.nytimes3xbfgragh.onion/hc/en-us/articles/115014893968-Terms-of-sale}{Terms
  of Sale}
\item
  \href{https://spiderbites.nytimes3xbfgragh.onion}{Site Map}
\item
  \href{https://help.nytimes3xbfgragh.onion/hc/en-us}{Help}
\item
  \href{https://www.nytimes3xbfgragh.onion/subscription?campaignId=37WXW}{Subscriptions}
\end{itemize}
