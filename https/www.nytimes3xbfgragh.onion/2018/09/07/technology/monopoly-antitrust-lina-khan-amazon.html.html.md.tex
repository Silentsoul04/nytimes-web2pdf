\href{/section/technology}{Technology}\textbar{}Amazon's Antitrust
Antagonist Has a Breakthrough Idea

\begin{itemize}
\item
\item
\item
\item
\item
\end{itemize}

\includegraphics{https://static01.graylady3jvrrxbe.onion/images/2018/09/09/business/09Khan1.print/09Khan1.print-articleLarge-v2.jpg?quality=75\&auto=webp\&disable=upscale}

Sections

\protect\hyperlink{site-content}{Skip to
content}\protect\hyperlink{site-index}{Skip to site index}

\hypertarget{amazons-antitrust-antagonist-has-a-breakthrough-idea}{%
\section{Amazon's Antitrust Antagonist Has a Breakthrough
Idea}\label{amazons-antitrust-antagonist-has-a-breakthrough-idea}}

With a single scholarly article, Lina Khan, 29, has reframed decades of
monopoly law.

In early 2017, when she was an unknown law student, Lina Khan published
``Amazon's Antitrust Paradox'' in the Yale Law Journal.Credit...Brandon
Thibodeaux for The New York Times

Supported by

\protect\hyperlink{after-sponsor}{Continue reading the main story}

By \href{https://www.nytimes3xbfgragh.onion/by/david-streitfeld}{David
Streitfeld}

\begin{itemize}
\item
  Sept. 7, 2018
\item
  \begin{itemize}
  \item
  \item
  \item
  \item
  \item
  \end{itemize}
\end{itemize}

The dead books are on the top floor of Southern Methodist University's
law library.

``Antitrust Dilemma.'' ``The Antitrust Impulse.'' ``Antitrust in an
Expanding Economy.'' Shelf after shelf of volumes ignored for decades.
There are a dozen fat tomes with transcripts of the congressional
hearings on monopoly power in 1949, when the world was in ruins and the
Soviets on the march. Lawmakers believed economic concentration would
make America more vulnerable.

At the end of the antitrust stacks is a table near the window. ``This is
my command post,'' said Lina Khan.

It's nothing, really. A few books are piled up haphazardly next to a
bottle with water and another with tea. Ms. Khan was in Dallas quite a
bit over the last year, refining an argument about monopoly power that
takes aim at one of the most admired, secretive and feared companies of
our era: Amazon.

The retailer overwhelmingly dominates online commerce, employs more than
half a million people and powers much of the internet itself through its
cloud computing division. On Tuesday, it briefly became the
\href{https://www.nytimes3xbfgragh.onion/2018/09/04/technology/amazon-stock-price-1-trillion-value.html?action=click\&module=Top\%20Stories\&pgtype=Homepage}{second
company to be worth a trillion dollars}.

If competitors tremble at Amazon's ambitions, consumers are mostly
delighted by its speedy delivery and low prices. They stream its
Oscar-winning movies and clamor for the company to build a second
headquarters in their hometowns. Few of Amazon's customers, it is safe
to say, spend much time thinking they need to be protected from it.

But then, until recently, no one worried about Facebook, Google or
Twitter either. Now politicians, the media, academics and regulators are
kicking around ideas that would, metaphorically or literally, cut them
down to size. Members of Congress
\href{https://www.nytimes3xbfgragh.onion/2018/09/05/technology/lawmakers-facebook-twitter-foreign-influence-hearing.html?action=click\&module=Well\&pgtype=Homepage\&section=Technology}{grilled
social media executives on Wednesday} in yet another round of hearings
on Capitol Hill. Not since the Department of Justice took on Microsoft
in the mid-1990s has Big Tech been scrutinized like this.

Amazon has more revenue than Facebook, Google and Twitter put together,
but it has largely escaped sustained examination. That is beginning to
change, and one significant reason is Ms. Khan.

\includegraphics{https://static01.graylady3jvrrxbe.onion/images/2018/09/09/business/09khan2/merlin_140063802_2cbcf897-0703-4f45-9abd-01cca30177ca-articleLarge.jpg?quality=75\&auto=webp\&disable=upscale}

In early 2017, when she was an unknown law student, Ms. Khan published
\href{https://www.yalelawjournal.org/note/amazons-antitrust-paradox}{``Amazon's
Antitrust Paradox''} in the Yale Law Journal. Her argument went against
a consensus in antitrust circles that dates back to the 1970s --- the
moment when regulation was redefined to focus on consumer welfare, which
is to say price. Since Amazon is renowned for its cut-rate deals, it
would seem safe from federal intervention.

Ms. Khan disagreed. Over 93 heavily footnoted pages, she presented the
case that the company should not get a pass on anticompetitive behavior
just because it makes customers happy. Once-robust monopoly laws have
been marginalized, Ms. Khan wrote, and consequently Amazon is amassing
structural power that lets it exert increasing control over many parts
of the economy.

Amazon has so much data on so many customers, it is so willing to forgo
profits, it is so aggressive and has so many advantages from its
shipping and warehouse infrastructure that it exerts an influence much
broader than its market share. It resembles the all-powerful railroads
of the Progressive Era, Ms. Khan wrote: ``The thousands of retailers and
independent businesses that must ride Amazon's rails to reach market are
increasingly dependent on their biggest competitor.''

The paper got 146,255 hits, a runaway best-seller in the world of legal
treatises. That popularity has rocked the antitrust establishment, and
is making an unlikely celebrity of Ms. Khan in the corridors of
Washington.

She has her own critics now: Several leading scholars have found fault
with Ms. Khan's proposals to revive and expand antitrust, and some have
tried to dismiss her paper with the mocking label ``Hipster Antitrust.''
Unwilling or perhaps unable to accept that a woman wrote a breakthrough
legal text, they keep talking about bearded dudes.

Ms. Khan was born in London to Pakistani parents who emigrated to the
United States when she was 11. She is now 29, an Amazon critic whose
Amazon account is largely inactive, newly married to a Texas doctor who
uses his Amazon Prime account all the time. Ms. Khan was supposed to
move this summer to Los Angeles, where she had a clerkship with Stephen
Reinhardt, the Ninth Circuit Court of Appeals judge and liberal icon,
but he suddenly
\href{https://www.nytimes3xbfgragh.onion/2018/04/02/obituaries/stephen-reinhardt-liberal-lion-of-federal-court-dies-at-87.html}{died
in March}. Instead, Ms. Khan is set to start a fellowship at Columbia
this fall, and is considering other projects as well. There is no
shortage of parties that want her advice on how to reckon with Big Tech.

``As consumers, as users, we love these tech companies,'' she said.
``But as citizens, as workers, and as entrepreneurs, we recognize that
their power is troubling. We need a new framework, a new vocabulary for
how to assess and address their dominance.''

At the S.M.U. library in Dallas, Ms. Khan was finding that vocabulary.
These dead books, many from an era that predated the price-based era of
monopoly law, were an influence and an inspiration. She was planning to
expand her essay into a book, she said in an interview here in June.

Then her life shifted, and she abruptly went from an outsider proposing
reform to an insider formulating policy. Rohit Chopra, a new Democratic
commissioner at the Federal Trade Commission, pulled her in as a
temporary adviser in July, at a time when urgent questions about
privacy, data, competition and antitrust were suddenly in the air. The
F.T.C. is holding a series of hearings this fall, the first of their
type since 1995, on whether a changing economy requires changing
enforcement attitudes.

\href{https://www.ftc.gov/news-events/events-calendar/2018/09/ftc-hearing-1-competition-consumer-protection-21st-century}{The
hearings will begin on Sept. 13} at Georgetown University Law Center.
Two panels will debate whether antitrust should keep its narrow focus
or, as Ms. Khan urges, expand its range.

``Ideas and assumptions that it was heretical to question are now openly
being contested,'' she said. ``We're finally beginning to examine how
antitrust laws, which were rooted in deep suspicion of concentrated
private power, now often promote it.''

Genuinely original voices are rare in Washington policy circles, and Mr.
Chopra is pleased to have Ms. Khan in his camp. ``It's rare to come
across a legal prodigy like Lina Khan,'' he said. ``Nothing about her
career is typical. You don't see many law students publish
groundbreaking legal research, or research that had such a deep impact
so quickly.''

Image

Jeff Bezos, the founder of Amazon.Credit...David Ryder/Getty Images

\hypertarget{then-rockefeller-now-bezos}{%
\subsection{Then: Rockefeller. Now:
Bezos.}\label{then-rockefeller-now-bezos}}

Ida Tarbell, the journalist whose investigation of Standard Oil helped
bring about its breakup,
\href{http://www.reformation.org/mcclure-rockefeller.html}{wrote this
about John D. Rockefeller} in 1905:

``It takes time to crush men who are pursuing legitimate trade. But one
of Mr. Rockefeller's most impressive characteristics is patience.
\ldots{} He was like a general who, besieging a city surrounded by
fortified hills, views from a balloon the whole great field, and sees
how, this point taken, that must fall; this hill reached, that fort is
commanded. And nothing was too small: the corner grocery in Browntown,
the humble refining still on Oil Creek, the shortest private pipeline.
Nothing, for little things grow.''

When Ms. Khan read that, she thought: Jeff Bezos.

Her Yale Law Journal paper argued that monopoly regulators who focus on
consumer prices are thinking too short-term. In Ms. Khan's view, a
company like Amazon --- one that sells things, competes against others
selling things, and owns the platform where the deals are done --- has
an inherent advantage that undermines fair competition.

``The long-term interests of consumers include product quality, variety
and innovation --- factors best promoted through both a robust
competitive process and open markets,'' she wrote.

The issue Ms. Khan's article really brought to the fore is this: Do we
trust Amazon, or any large company, to create our future? In think tanks
and universities, the battle has been joined.

``It's one thing to say that antitrust enforcement has gotten far too
weak,'' said Daniel Crane, a University of Michigan scholar who doesn't
agree with Ms. Khan but credits her with opening up a much-needed
debate. ``It's a bridge much further to say we should go back to the
populist goal of leveling playing fields and checking `bigness.' ''

As Mr. Crane writes in a
\href{https://repository.law.umich.edu/cgi/viewcontent.cgi?article=1264\&amp=\&context=law_econ_current\&amp=\&sei-redir=1\&referer=https\%253A\%252F\%252Fwww.google.com\%252Furl\%253Fq\%253Dhttps\%253A\%252F\%252Frepository.law.umich.edu\%252Fcgi\%252Fviewcontent.cgi\%253Farticle\%25253D1264\%252526context\%25253Dlaw_econ_current\%2526sa\%253DD\%2526source\%253Dhangouts\%2526ust\%253D1536207313850000\%2526usg\%253DAFQjCNE_HOt_Sn9-tVjtGaIY6BmPZlkpeA\#search=\%22https\%3A\%2F\%2Frepository.law.umich.edu\%2Fcgi\%2Fviewcontent.cgi\%3Farticle\%3D1264\%26context\%3Dlaw_econ_current\%22}{forthcoming
law review article}: ``Antitrust law stands at its most fluid and
negotiable moment in a generation.''

The resistance is fierce and prominent. Herbert Hovenkamp, an antitrust
expert at the University of Pennsylvania Law School, wrote that
\href{https://scholarship.law.upenn.edu/faculty_scholarship/1769/}{if
companies like Amazon are targeted} simply because their low prices hurt
competitors, we might ``quickly drive the economy back into the Stone
Age, imposing hysterical costs on everyone.''

Timothy Muris, a former chairman of the F.T.C., and Jonathan
Nuechterlein, a former F.T.C. general counsel, published a paper in June
that was a response to Ms. Khan and the antitrust reform movement.
Called
\href{https://papers.ssrn.com/sol3/papers.cfm?abstract_id=3186569}{``Antitrust
in the Internet Era,''} it was about the A.\&P. grocery chain.

A.\&P. essentially invented the modern supermarket in the 1920s. With
its low prices, wide range of products and penchant for disruption, the
chain became the leading retailer of its era. It owned 70 factories and
eliminated middlemen, which allowed it to keep costs down. Yet, Mr.
Muris and Mr. Nuechterlein wrote, ``A.\&P.'s very popularity triggered a
backlash.'' The government pursued A.\&P. on antitrust grounds during
the 1940s, egged on by competitors that could not compete. After decades
of decline, A.\&P.
\href{https://www.nytimes3xbfgragh.onion/2015/07/21/business/ap-files-for-bankruptcy-and-aims-to-sell-120-stores.html}{shut
its doors for good} in 2015.

The analogies with Amazon are explicit. Don't let the government pursue
Amazon the way it pursued A.\&P., Mr. Muris and Mr. Nuechterlein warned.

``Amazon has added hundreds of billions of dollars of value to the U.S.
economy,'' they wrote. ``It is a brilliant innovator'' whose
``breakthroughs have in turn helped launch new waves of innovation
across retail and technology sectors, to the great benefit of
consumers.''

Amazon itself could not have made the argument any better. Which isn't
surprising, because in a footnote on the first page, the authors noted:
``We approached Amazon Inc. for funding to tell the story'' of A.\&P.,
``and we gratefully acknowledge its support.'' They added at the end of
footnote 85: ``The authors have advised Amazon on a variety of antitrust
issues.''

Amazon declined to say how much its support came to in dollars. It also
declined to comment on Ms. Khan or her paper directly, but issued a
statement.

``We operate in a diverse range of businesses, from retail and
entertainment to consumer electronics and technology services, and we
have intense and well-established competition in each of these areas,''
the company said. ``Retail is our largest business today and we
represent less than 1 percent of global retail.''

Image

``As consumers, as users, we love these tech companies,'' Ms. Khan said.
``But as citizens, as workers, and as entrepreneurs, we recognize that
their power is troubling. We need a new framework, a new vocabulary for
how to assess and address their dominance.''Credit...Lexey Swall for The
New York Times

\hypertarget{were-at-the-very-beginning-of-solutions-to-this}{%
\subsection{`We're at the Very Beginning of Solutions to
This'}\label{were-at-the-very-beginning-of-solutions-to-this}}

The first time Ms. Khan held power to account involved a Starbucks in
suburban New York that was banning students from sitting down. Ms. Khan
decided to write an article about the policy; Starbucks wouldn't answer
her questions, but she managed to interview the employees. The New York
Times
\href{https://www.nytimes3xbfgragh.onion/2004/10/17/nyregion/education/a-tempest-in-a-coffee-shop.html}{picked
up on the tempest}, leaning on her reporting. Ms. Khan was 15, a
correspondent for her high school newspaper.

Her father was a management consultant; her mother an executive in
information services. Ms. Khan went to Williams College, where she wrote
a thesis on the political philosopher Hannah Arendt. She was the editor
of the student paper but worked hard at everything.

``We were routinely emailing each other on separate floors of the
library as it was closing at 2 a.m.,'' said Amanda Korman, a classmate.

Like many a wonkish youth, Ms. Khan headed to Washington after
graduating in 2010, applying for a position at the left-leaning New
America Foundation. Barry Lynn, who headed the organization's Open
Markets antimonopoly initiative, seized on her application. ``It's so
much easier to teach public policy to people who already know how to
write than teach writing to public policy experts,'' said Mr. Lynn, a
former journalist.

Ms. Khan wrote about industry consolidation and monopolistic practices
for Washington publications that specialize in policy, went to Yale Law
School, published her Amazon paper and then came back to Washington last
year, just as interest was starting to swell in her work.

In the summer of 2017, Open Markets was ejected from New America amid
messy accusations that it
\href{https://www.nytimes3xbfgragh.onion/2017/08/30/us/politics/eric-schmidt-google-new-america.html}{displeased
Google}, a prominent funder, after the company was rebuked by European
regulators for anticompetitive behavior. The think tank is now
independent.

``Polls show huge concerns about concentrated power, corporate power,
but if people are asked, `Do we have a monopoly problem?' they answer,
`I don't know,' '' said Mr. Lynn. ``They don't have the language for
it.''

Amazon's \$14 billion
\href{https://www.nytimes3xbfgragh.onion/2017/06/16/business/dealbook/amazon-whole-foods.html}{purchase
of Whole Foods} in the summer of 2017 --- a startling move into physical
retail --- was almost a watershed, but not quite. Rep. David Cicilline
of Rhode Island, the ranking Democrat on the Subcommittee on Regulatory
Reform, Commercial and Antitrust Law, called for hearings but did not
get them.

``The whole country has been struggling to understand why the economy is
not operating in the right way,'' Mr. Cicilline said. ``Wages have
remained stagnant. Workers have less and less power. All we're trying to
do is create a level playing field, and that's harder when you have
megacompanies that make it virtually impossible for small competitors.''
He added, ``We're at the very beginning of solutions to this.''

Somewhere in the midst of all this, Ms. Khan found the time to marry
Shah Ali, a doctor now doing a cardiology fellowship in Dallas, which
explains why she was camping out at the S.M.U. law library. The
honeymoon was in Hawaii. Dr. Ali took Jane Austen's ``Persuasion,''
because he hadn't reread it in a while. Ms. Khan brought a book on
corporations and American democracy.

Image

``It's rare to come across a legal prodigy like Lina Khan,'' said Rohit
Chopra, a new Democratic commissioner at the Federal Trade Commission.
``Nothing about her career is typical. You don't see many law students
publish groundbreaking legal research, or research that had such a deep
impact so quickly.''Credit...Lexey Swall for The New York Times

\hypertarget{the-new-brandeisians-lacks-a-certain-something}{%
\subsection{`The New Brandeisians' Lacks a Certain
Something}\label{the-new-brandeisians-lacks-a-certain-something}}

The battle for intellectual supremacy takes place less these days in
learned journals and more on social media, where tongues are sharp and
branding is all. This is not Ms. Khan's strong suit. She is always
polite, even on Twitter. One consequence is that she didn't give much
thought about what to call the movement to reboot antitrust. Neither did
anyone else.

That presented an opening for the reformers' critics, who have tried
with a limited degree of success to popularize the term ``Hipster
Antitrust.'' Konstantin Medvedovsky, an antitrust lawyer in New York,
came up with the label last summer
\href{https://twitter.com/kmedved/status/876869328934711296}{in a tweet}
that was responding to a tweet that was responding to a tweet by Ms.
Khan.

``Antitrust Hipsterism,'' he wrote. ``Everything old is cool again.''

Mr. Medvedovsky, who calls Ms. Khan's article ``the face of this
movement,'' said the term was designed to be ``playful rather than
pejorative.''

Admirers of Ms. Khan and her fellow reformers have sometimes called them
the New Brandeis School or the New Brandeisians, after Louis Brandeis,
the Progressive Era foe of big business. As brands go, these are
somewhat less catchy than ``Hipster Antitrust.''

The April issue of the journal
\href{https://www.competitionpolicyinternational.com/wp-content/uploads/2018/05/AC_APRIL.pdf}{Antitrust
Chronicle}, edited by Mr. Medvedovsky, features a drawing of a bearded
man on the cover right above the words ``Hipster Antitrust.'' In the
middle of an article by Philip Marsden, a professor of competition law
and economics at the College of Europe in Bruges, there's a photograph
of a bearded man taking a selfie next to the chapter heading ``Battle of
the Beards.'' It is perhaps relevant that only one of the 12 authors or
experts in the issue is female.

The Hipster issue was sponsored by Facebook, another sign that Big Tech
is striving to shape the monopoly-law debate. The company declined to
comment.

Things are moving fast, so there is a lot to write papers about.

Mr. Chopra, with Ms. Khan's assistance, pushed the argument further on
Sept. 6 with a 14-page
\href{https://www.ftc.gov/system/files/documents/public_statements/1408196/chopra_-_comment_to_hearing_1_9-6-18.pdf}{official
comment} that suggested the F.T.C. bring back a tool buried in its
toolbox: the ability to make rules.

Contemporary antitrust regulation, the commissioner wrote, is conducted
in the courts, which makes it numbingly slow and dependent on high-paid
expert witnesses. He called for the agency to use its authority to issue
rules that would ``advance clarity and certainty'' about what is, and
what is not, an unfair method of competition.

These rules would not be ``some inflexible prescription'' but standards,
guidelines, pointers or presumptions, he wrote. Since everyone affected
by a proposed rule would have the opportunity to weigh in on it, the
process would be more democratic.

There is more than an echo here of Ms. Khan's notion that the past can
help rescue the future.

``These are new technologies and new business models,'' Ms. Khan said.
``The remedy is new thinking that is informed by traditional
principles.''

\hypertarget{antitrust-foot-soldiers}{%
\subsection{Antitrust Foot Soldiers}\label{antitrust-foot-soldiers}}

Big Tech's great strength is that it is everywhere. Hardly anyone can
\href{https://www.nytimes3xbfgragh.onion/interactive/2017/05/10/technology/Ranking-Apple-Amazon-Facebook-Microsoft-Google.html}{live
without it}. But that omnipresence can be a weakness too. Just ask
Facebook. It was the only global social media network, an enviable
position --- until it wasn't. Ideas for regulating Facebook that were
once unimaginable are now on the table.

Ms. Khan was not the first to criticize Amazon, and she said the company
was not really her target anyway. ``Amazon is not the problem --- the
state of the law is the problem, and Amazon depicts that in an elegant
way,'' she said.

From Amazon's point of view, however, it is a problem indeed that Ms.
Khan concludes in the Yale paper that regulating parts of the company
like a utility ``could make sense.'' She also said it ``could make
sense'' to treat Amazon's e-commerce operation like a bridge, highway,
port, power grid or telephone network --- all of which are required to
allow access to their infrastructure on a nondiscriminatory basis.

Ms. Khan put those ideas out there, which is how Rachel Tsuna found
them.

Last fall, the Barnard College senior was casting about for a subject
for her senior thesis. ``What is really interesting to you?'' her
adviser asked. Ms. Tsuna, now 22, had worked for a chewing gum start-up
--- yes, there are such things --- that sold through Amazon, and knew
firsthand the retailer's tight grip. ``Amazon is scary!'' she exclaimed.

This impulsive declaration suggested a topic: Did the F.T.C. have the
grounds to move against Amazon? Ms. Tsuna made little progress until she
came across Ms. Khan's paper.

``I finally felt like I was pursuing something valid,'' Ms. Tsuna said.
``Lina Khan gave me the confidence I needed.'' The thesis, which is
quite fair to Amazon, got an A minus.

That's the way movements begin. Little things grow.

``This is a moment in time that invites a movement,'' said Ms. Khan.
``It's bigger than antitrust, bigger than Big Tech. It's about whether
the laws serve democratic ends.''

It was late at night in late July, and she was eating a burrata
concoction at a popular restaurant near the Washington apartment she
uses when not in Texas with her husband. After the death of Judge
Reinhardt, her options opened up. She had the Columbia fellowship. Maybe
she would also write the book. Or go back to the F.T.C. full time. Or
somehow do it all.

``Amazon is a monopoly, and I worry that it monopolizes Lina,'' said her
husband, Dr. Ali. ``I learn about what she is doing from looking at her
Twitter feed.''

``I throw myself into things,'' Ms. Khan agreed. ``My life is spread out
now.''

With some cajoling, she revealed her Amazon account. There were just
three purchases in 18 months. An altimeter for her father, who has taken
up hiking, is the only one she will agree to have mentioned, although
the other two are incredibly benign. One attribute Ms. Khan shares with
Amazon is a strong desire to control the flow of information.

Somewhat to her surprise, she is becoming a public figure. Before
beginning her stint at the F.T.C., she said the news of her working
there might be no more than a sentence or two at news sites that cover
policy intensively. Instead it was a full-fledged story. The
Information, a tech news site, declared:
\href{https://www.theinformation.com/articles/amazon-antitrust-push-slowly-gains-ground}{``Amazon
Antitrust Push Slowly Gains Ground.''} Politico just named her
\href{https://www.politico.com/interactives/2018/politico50/lina-khan/}{one
of the Politico 50,} its annual list of the people driving the ideas
driving politics.

Balancing the attention and the achievement, the expectations and the
demands, is difficult, perhaps impossible.

``I don't think of my work in grandiose terms. I feel an urgency but I'm
also wary of hubris,'' Ms. Khan said. ``Nobody has been expecting this
to succeed. I'm awed by the challenge.''

Advertisement

\protect\hyperlink{after-bottom}{Continue reading the main story}

\hypertarget{site-index}{%
\subsection{Site Index}\label{site-index}}

\hypertarget{site-information-navigation}{%
\subsection{Site Information
Navigation}\label{site-information-navigation}}

\begin{itemize}
\tightlist
\item
  \href{https://help.nytimes3xbfgragh.onion/hc/en-us/articles/115014792127-Copyright-notice}{©~2020~The
  New York Times Company}
\end{itemize}

\begin{itemize}
\tightlist
\item
  \href{https://www.nytco.com/}{NYTCo}
\item
  \href{https://help.nytimes3xbfgragh.onion/hc/en-us/articles/115015385887-Contact-Us}{Contact
  Us}
\item
  \href{https://www.nytco.com/careers/}{Work with us}
\item
  \href{https://nytmediakit.com/}{Advertise}
\item
  \href{http://www.tbrandstudio.com/}{T Brand Studio}
\item
  \href{https://www.nytimes3xbfgragh.onion/privacy/cookie-policy\#how-do-i-manage-trackers}{Your
  Ad Choices}
\item
  \href{https://www.nytimes3xbfgragh.onion/privacy}{Privacy}
\item
  \href{https://help.nytimes3xbfgragh.onion/hc/en-us/articles/115014893428-Terms-of-service}{Terms
  of Service}
\item
  \href{https://help.nytimes3xbfgragh.onion/hc/en-us/articles/115014893968-Terms-of-sale}{Terms
  of Sale}
\item
  \href{https://spiderbites.nytimes3xbfgragh.onion}{Site Map}
\item
  \href{https://help.nytimes3xbfgragh.onion/hc/en-us}{Help}
\item
  \href{https://www.nytimes3xbfgragh.onion/subscription?campaignId=37WXW}{Subscriptions}
\end{itemize}
