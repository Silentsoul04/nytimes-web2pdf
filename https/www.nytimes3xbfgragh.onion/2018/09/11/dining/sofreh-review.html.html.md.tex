Sections

SEARCH

\protect\hyperlink{site-content}{Skip to
content}\protect\hyperlink{site-index}{Skip to site index}

\href{https://www.nytimes3xbfgragh.onion/section/food}{Food}

\href{https://myaccount.nytimes3xbfgragh.onion/auth/login?response_type=cookie\&client_id=vi}{}

\href{https://www.nytimes3xbfgragh.onion/section/todayspaper}{Today's
Paper}

\href{/section/food}{Food}\textbar{}For Fans of Iran's Sophisticated
Cuisine, a Reason to Cheer

\url{https://nyti.ms/2N3tput}

\begin{itemize}
\item
\item
\item
\item
\item
\item
\end{itemize}

Advertisement

\protect\hyperlink{after-top}{Continue reading the main story}

Supported by

\protect\hyperlink{after-sponsor}{Continue reading the main story}

\href{/column/restaurant-review}{Restaurant Review}

\hypertarget{for-fans-of-irans-sophisticated-cuisine-a-reason-to-cheer}{%
\section{For Fans of Iran's Sophisticated Cuisine, a Reason to
Cheer}\label{for-fans-of-irans-sophisticated-cuisine-a-reason-to-cheer}}

\href{https://www.nytimes3xbfgragh.onion/slideshow/2018/09/11/dining/sofreh-brooklyn.html}{}

\hypertarget{nigella-seeds-and-rose-water-sorbet}{%
\subsection{Nigella Seeds and Rose-Water
Sorbet}\label{nigella-seeds-and-rose-water-sorbet}}

10 Photos

View Slide Show ›

\includegraphics{https://static01.graylady3jvrrxbe.onion/images/2018/09/12/dining/12REST-slide-Q083/12REST-slide-Q083-articleLarge.jpg?quality=75\&auto=webp\&disable=upscale}

Jeenah Moon for The New York Times

\begin{itemize}
\tightlist
\item
  Sofreh\\
  **NYT Critic's Pick ★★ Middle Eastern \$\$\$ 75 St. Marks Avenue
  646-340-0322
\end{itemize}

\href{https://www.opentable.com/single.aspx?ref=4201\&rid=1021087}{Reserve
a Table}

When you make a reservation at an independently reviewed restaurant
through our site, we earn an affiliate commission.

By \href{https://www.nytimes3xbfgragh.onion/by/pete-wells}{Pete Wells}

\begin{itemize}
\item
  Sept. 11, 2018
\item
  \begin{itemize}
  \item
  \item
  \item
  \item
  \item
  \item
  \end{itemize}
\end{itemize}

Considering the history, the influence, the depth and the sophistication
of Iranian food, it has always seemed unfair that New York City has so
few places to eat it. Unfair to the cuisine and unfair to would-be
eaters.

If you want a symbol for the state of the cuisine in New York City,
\href{https://www.nytimes3xbfgragh.onion/2013/09/04/dining/reviews/hungry-city-taste-of-persia-nyc-in-flatiron.html}{Taste
of Persia NYC} will fit the bill. Until recently I might have called it
the city's most promising Iranian kitchen, if the word kitchen weren't
such a stretch. Taste of Persia is a takeout operation squeezed into the
front window of a Chelsea pizzeria, except during the two months when
fire damage forced it to take up temporary residence in a holiday
market. It does show off some of the pleasures of the Iranian table, but
not the table itself.

So a new Brooklyn restaurant called
\href{http://www.sofrehnyc.com/}{Sofreh}, which joined the Iranian ranks
in June, would have caused a stir even if the cooking weren't quite as
good as it is. The dining room is spare and modern, with exposed
rafters, molded plywood chairs and shades of soft white. You have to
look carefully to see the breaks with garden-variety minimalism, like
the echoes of medieval arches behind the bar and the Persian calligraphy
carved into the plaster in the back, where a small deck hangs above a
garden. Minimalism surrenders entirely in the bathroom, papered in a
pulpy collage of posters from pre-revolutionary Iranian cinema.

All summer, Iranian expats and hopeful fans of Iranian cuisine have been
filing in, armed with questions: Does the saffron rice stand high in
voluminous heaps of discrete grains? Are nigella seeds embedded in the
dimpled, golden crusts of the oval loaves of barbari? Do the cold shards
of sorbet frozen around threads of noodles in Sofreh's faloodeh contain
just enough rose water to soften the bite of lime syrup? Check, check
and check.

\includegraphics{https://static01.graylady3jvrrxbe.onion/images/2018/09/12/dining/12Rest1/merlin_143332464_4a0437af-6247-4a4c-817c-d9c4e04e56bb-articleLarge.jpg?quality=75\&auto=webp\&disable=upscale}

\emph{{[}Read about some of the}
\href{https://www.nytimes3xbfgragh.onion/2018/11/15/nyregion/best-new-nyc-restaurants.html?action=click\&module=Intentional\&pgtype=Article}{\emph{best
new restaurants in New York City}} \emph{(for now).{]}}

The main dining room is on the street level of a Prospect Heights
brownstone just off Flatbush Avenue, and the food is largely in a
domestic mode.
\href{https://www.nytimes3xbfgragh.onion/2018/08/20/dining/sofreh-persian-restaurant-new-york-city.html}{Nasim
Alikhani}, who owns the restaurant with her husband, Theodore Petroulas,
is responsible for the menu.

Ms. Alikhani has never worked in a restaurant before. Rather than
following the plating fashions of more trend-conscious chefs, she models
the cooking after the things she might make when company comes over.
Platoons of kebabs dominate the menus of other Iranian restaurants
around town, but home kitchens are rarely built with vast indoor grills;
Sofreh's cooking, entrusted to two chefs named Ali Saboor and Soroosh
Golbabae, revolves around the oven and stovetop.

A loaf of barbari, its crust baked to a Roman-pizza crackle in a
revolving oven, is served with many of the appetizers, and it is always
gone so soon that I end up ordering one or two more. This flatbread is
of course ideal for tearing and swishing into the thick yogurt dips, one
mixed with golden raisins and shredded cucumber, another with minced
shallots and chives. It is indispensable with the eggplant mash enriched
with walnuts, strained yogurt and sweet fried onions. A loaf of barbari
is also spread with herbs and feta that is more creamy than briny.

Earlier in the summer Sofreh offered an appetizer of dry, underseasoned
beef meatballs with sour cherries. One of the only things on the menu
that was hard to warm up to, it has been replaced lately by beef kofte,
made light and tender by rice and stewed split peas. The kofte sit in a
gentle saffron-tomato sauce that is another natural partner for Sofreh's
bread.

Ms. Alikhani, 59, grew up in the sprawling desert city Isfahan, and
moved to the United States in her early 20s. For years she cooked
Iranian dishes from memory, and it would be decades before she traveled
back to Iran to learn more about its food in preparation for opening
Sofreh. Her version of ash, the herb and whole-wheat noodle stew, stands
out on the menu because it stays so close to tradition. Other dishes
show the acclimatizing effect of New York.

Image

Nasim Alikhani, an owner, dreamed of opening Sofreh for
years.Credit...Jeenah Moon for The New York Times

It's an article of faith for chefs in the city that any dish can be
improved by putting a poached egg on it. I'm not so sure that Sofreh's
smoked eggplant halves with their sweet, garlic-drenched tomato sauce
need one, though.

Is the watermelon feta salad familiar from restaurants that have nothing
to do with Iran? Maybe just a bit, but Sofreh's version stands out for
its intriguing sauce of nigella seeds crushed into mint oil. And while
the Shiraz salad isn't chopped to fine bits in the traditional way that
makes it something like Iran's answer to pico de gallo, its sour purple
flecks of sumac distinguish it from the herd of tomato-cucumber
assemblages.

The menu is fairly concise, and occasionally I wished that it plunged
into tradition with less restraint. For a dish called ``catch of the
day,'' Ms. Alikhani has adapted ghalieh mahi, a tamarind-soured stew of
fish and fresh herbs from southern Iran. In her version, a pan-seared
hunk of whitefish --- halibut and cod have taken turns in the role ---
is set over a long-cooked sauce. Dark with caramelized onions and fried
cilantro and fresh fenugreek leaves, the sauce is hypnotically complex.
Still, a sauce sitting under fish is not the same as one that has been
cooked with fish so the flavors can get acquainted.

A simple flattened and griddled half chicken, though, is a wonderful
foil for a topping of tart barberries and a captivating sauce of dried
Persian plums and saffron. And there's lovely simplicity to the braised
lamb shank with roasted garlic and sizzled onions; the turmeric- and
cinnamon-scented sauce might have been put on this earth to be spooned
over fluffy rice.

There is no shame in ending the meal with a goblet of thick yogurt
parfait with jam and pistachios. But there are harder-to-find sweets,
too, like the rose-water custard, grainy with rice flour, transformed
into a sort of tart by a crust of chopped nuts. Saffron and rose-water
ice cream is luxuriously rich despite a few stray ice crystals. And the
faloodeh, one of the world's oldest frozen desserts, will put to rest
all doubts about the wisdom of embedding sorbet in a nest of vermicelli,
if there were any doubts to begin with.

\emph{Follow} \emph{\href{https://twitter.com/nytfood}{NYT Food on
Twitter}} \emph{and}
\emph{\href{https://www.instagram.com/nytcooking/}{NYT Cooking on
Instagram},}
\emph{\href{https://www.facebookcorewwwi.onion/nytcooking/}{Facebook}}
\emph{and}
\emph{\href{https://www.pinterest.com/nytcooking/}{Pinterest}.}
\emph{\href{https://www.nytimes3xbfgragh.onion/newsletters/cooking}{Get
regular updates from NYT Cooking, with recipe suggestions, cooking tips
and shopping advice}.}

Advertisement

\protect\hyperlink{after-bottom}{Continue reading the main story}

\hypertarget{site-index}{%
\subsection{Site Index}\label{site-index}}

\hypertarget{site-information-navigation}{%
\subsection{Site Information
Navigation}\label{site-information-navigation}}

\begin{itemize}
\tightlist
\item
  \href{https://help.nytimes3xbfgragh.onion/hc/en-us/articles/115014792127-Copyright-notice}{©~2020~The
  New York Times Company}
\end{itemize}

\begin{itemize}
\tightlist
\item
  \href{https://www.nytco.com/}{NYTCo}
\item
  \href{https://help.nytimes3xbfgragh.onion/hc/en-us/articles/115015385887-Contact-Us}{Contact
  Us}
\item
  \href{https://www.nytco.com/careers/}{Work with us}
\item
  \href{https://nytmediakit.com/}{Advertise}
\item
  \href{http://www.tbrandstudio.com/}{T Brand Studio}
\item
  \href{https://www.nytimes3xbfgragh.onion/privacy/cookie-policy\#how-do-i-manage-trackers}{Your
  Ad Choices}
\item
  \href{https://www.nytimes3xbfgragh.onion/privacy}{Privacy}
\item
  \href{https://help.nytimes3xbfgragh.onion/hc/en-us/articles/115014893428-Terms-of-service}{Terms
  of Service}
\item
  \href{https://help.nytimes3xbfgragh.onion/hc/en-us/articles/115014893968-Terms-of-sale}{Terms
  of Sale}
\item
  \href{https://spiderbites.nytimes3xbfgragh.onion}{Site Map}
\item
  \href{https://help.nytimes3xbfgragh.onion/hc/en-us}{Help}
\item
  \href{https://www.nytimes3xbfgragh.onion/subscription?campaignId=37WXW}{Subscriptions}
\end{itemize}
