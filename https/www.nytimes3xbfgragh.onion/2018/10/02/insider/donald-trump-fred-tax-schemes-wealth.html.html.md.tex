Sections

SEARCH

\protect\hyperlink{site-content}{Skip to
content}\protect\hyperlink{site-index}{Skip to site index}

\href{https://www.nytimes3xbfgragh.onion/section/reader-center}{Times
Insider}

\href{https://myaccount.nytimes3xbfgragh.onion/auth/login?response_type=cookie\&client_id=vi}{}

\href{https://www.nytimes3xbfgragh.onion/section/todayspaper}{Today's
Paper}

\href{/section/reader-center}{Times Insider}\textbar{}How Times
Journalists Uncovered the Original Source of the President's Wealth

\url{https://nyti.ms/2P2VeQF}

\begin{itemize}
\item
\item
\item
\item
\item
\item
\end{itemize}

Advertisement

\protect\hyperlink{after-top}{Continue reading the main story}

Supported by

\protect\hyperlink{after-sponsor}{Continue reading the main story}

\hypertarget{how-times-journalists-uncovered-the-original-source-of-the-presidents-wealth}{%
\section{How Times Journalists Uncovered the Original Source of the
President's
Wealth}\label{how-times-journalists-uncovered-the-original-source-of-the-presidents-wealth}}

Three reporters spent over a year digging through more than 100,000
pages of documents and chasing down key sources familiar with President
Trump's father and his empire.

\includegraphics{https://static01.graylady3jvrrxbe.onion/images/2018/10/03/insider/03itt-inheritance/03itt-inheritance-articleLarge.jpg?quality=75\&auto=webp\&disable=upscale}

\href{https://www.nytimes3xbfgragh.onion/by/melina-delkic}{\includegraphics{https://static01.graylady3jvrrxbe.onion/images/2019/02/05/multimedia/author-melina-delkic/author-melina-delkic-thumbLarge.png}}

By \href{https://www.nytimes3xbfgragh.onion/by/melina-delkic}{Melina
Delkic}

\begin{itemize}
\item
  Oct. 2, 2018
\item
  \begin{itemize}
  \item
  \item
  \item
  \item
  \item
  \item
  \end{itemize}
\end{itemize}

\href{http://www.nytimes3xbfgragh.onion/section/insider}{\emph{Times
Insider}} \emph{delivers behind-the-scenes insights into how news,
features and opinion come together at The New York Times.}

In the three years since Donald J. Trump announced his candidacy for
president, there has been plenty of investigation into his financial
history --- especially because he broke with tradition and declined to
release his tax returns.

In 2016, David Barstow, Susanne Craig and Russ Buettner of The New York
Times
\href{https://www.nytimes3xbfgragh.onion/2016/10/02/us/politics/donald-trump-taxes.html}{obtained
his 1995 tax returns}, showing that he could have avoided paying taxes
for nearly two decades. And for their
\href{https://www.nytimes3xbfgragh.onion/interactive/2018/10/02/us/politics/donald-trump-tax-schemes-fred-trump.html}{article
on}\href{https://www.nytimes3xbfgragh.onion/interactive/2018/10/02/us/politics/donald-trump-tax-schemes-fred-trump.html}{Wednesday's
front page}, they worked together for more than a year to investigate
the wealth that the president inherited from his father.

``It's unusual to dive into what you think is an extremely well-covered
subject and to find so much completely new stuff, stuff that just is
astonishing,'' Mr. Barstow said. ``It's a great reminder that even
things that you think are well described, there are these other deeper
layers.''

Over all, the effort was sprawling and multilayered, involving more than
100,000 pages of documents, both public and confidential; interviews
with key sources and requests through the Freedom of Information Act.
Together, they showed that the president participated in dubious tax
schemes in the 1990s, including outright fraud, and that he wasn't the
self-made billionaire he has claimed to be.

The investigation started with a simple question: What was going on with
the president's finances from 1995 to 2005? The team knew he had
reported a loss in 1995 of almost \$1 billion, as
\href{https://www.nytimes3xbfgragh.onion/2016/10/02/us/politics/donald-trump-taxes.html}{they
had reported} in 2016, and then a \$150 million profit in 2005, as David
Cay Johnston, a former New York Times journalist who is now the
editor-in-chief at \href{https://www.dcreport.org/}{DCReport.org},
\href{https://www.nytimes3xbfgragh.onion/2017/03/14/us/politics/donald-trump-taxes.html}{reported}
on Rachel Maddow's MSNBC program last year, based on two pages of those
tax returns.

Mr. Barstow, Ms. Craig and Mr. Buettner dug into the fortune that Fred
C. Trump, the prolific New York City builder who died in 1999, passed on
to President Trump and his surviving siblings, Maryanne Trump Barry,
Robert Trump and Elizabeth Trump Grau.

A central finding of the story began to emerge in April 2017, when Ms.
Craig had been Google searching an arcane term the group was interested
in --- ``mortgage receivable,'' which the Trumps used to describe the
mortgages from the children to Fred --- paired with the last name
``Trump.'' She found the disclosure form that the president's sister
Maryanne, a federal judge, had filed related to her Senate confirmation
hearing. Unlike the many she filed during her years on the bench, this
one was not redacted. In that document, Ms. Craig noticed a \$1 million
contribution from an obscure family-owned company: All County Building
Supply \& Maintenance.

```What the heck?''' Ms. Craig remembered thinking. ``That was the first
inkling we had that, hey, there's something to do with this company that
we need to figure out.''

The trio began to talk to people familiar with the president's father
and his empire. Those people told them that the company was a middleman
entity created by President Trump and his siblings essentially to move
cash from Fred Trump's companies to his children. After All County
bought various items for Fred Trump's buildings, like boilers and
cleaning supplies, a secretary would bill the items to Fred Trump's
buildings with a 20 to 50 percent markup. The siblings would then pocket
the difference.

In short, the siblings received millions in untaxed gifts from their
father, skirting a 55 percent tax on gifts over a certain value that
would have cut the total significantly.

``When we came to that realization, that was a big day for us,'' Mr.
Buettner said.

Over the next several months, the reporters would obtain tens of
thousands of pages of documents, including more than 200 tax returns
from Fred Trump, his companies and various Trump partnerships and
trusts. (``We have a virtual mountain of spreadsheets,'' Mr. Barstow
said. ``We should have spreadsheets for our spreadsheets.'') The trove
included previously secret depositions, including one in which Robert
Trump, the president's brother, admitted that the family used the padded
receipts from All County to justify higher rent increases for their
tenants in rent-regulated apartments.

This year, another breakthrough came when the team matched a boiler
receipt from a personal injury lawsuit Mr. Buettner found that named All
County --- a man was injured by the boiler in a Trump building --- to a
boiler receipt they obtained via a FOIA request to New York City. They
found two identical purchase-order numbers, with the bill from All
County to Fred Trump marking up the boiler price by 20 percent.

``That's a rare moment of reporting serendipity, right?'' Mr. Buettner
said. ``Two pieces of paper from two separate places that combined to
tell you a bigger truth.''

``It was like these two puzzle pieces came together --- one from the
lawsuit and the other from the FOIA request,'' Ms. Craig said. ``We call
it the Perry Mason moment.''

Mr. Barstow then tracked down the man who sold that boiler to Fred
Trump: Leon Eastmond, the owner of A. L. Eastmond \& Sons, a Bronx
company that makes industrial boilers. Mr. Eastmond said he remembered
the lunch meeting during which Fred Trump negotiated the price of 60
boilers, and later receiving checks in the mail from All County. He said
that he had never heard of All County before those checks, and that he
had mainly interacted with Fred Trump, his secretary and Robert Trump.

If the investigation was a ``massive jigsaw puzzle,'' then Mr. Eastmond
was ``the piece that fit perfectly in the center of the puzzle,'' Mr.
Barstow said.

Throughout the investigation, they didn't hit the typical lulls
investigative reporters face --- the dead ends or the weekslong digs
without any big breakthroughs. ``There never went three weeks that went
by where we didn't move the ball that much further down,'' Ms. Craig
said. ``We just kept finding stuff and finding stuff.''

The worry, instead, was how they would decide which threads to follow.

The article emphasizes just how much remains to be uncovered in the
Trump tax return saga, Ms. Craig, Mr. Barstow and Mr. Buettner said.

``One of the big themes of this story is this one little alleyway that
we wandered down, and all that it sort of revealed to us is how little
we actually do know about the president's financial history,'' Mr.
Barstow said. ``In all of the books, all of the profiles, all of the
newspaper stories, we haven't found one mention of Donald Trump and All
County Building Supply.''

Advertisement

\protect\hyperlink{after-bottom}{Continue reading the main story}

\hypertarget{site-index}{%
\subsection{Site Index}\label{site-index}}

\hypertarget{site-information-navigation}{%
\subsection{Site Information
Navigation}\label{site-information-navigation}}

\begin{itemize}
\tightlist
\item
  \href{https://help.nytimes3xbfgragh.onion/hc/en-us/articles/115014792127-Copyright-notice}{©~2020~The
  New York Times Company}
\end{itemize}

\begin{itemize}
\tightlist
\item
  \href{https://www.nytco.com/}{NYTCo}
\item
  \href{https://help.nytimes3xbfgragh.onion/hc/en-us/articles/115015385887-Contact-Us}{Contact
  Us}
\item
  \href{https://www.nytco.com/careers/}{Work with us}
\item
  \href{https://nytmediakit.com/}{Advertise}
\item
  \href{http://www.tbrandstudio.com/}{T Brand Studio}
\item
  \href{https://www.nytimes3xbfgragh.onion/privacy/cookie-policy\#how-do-i-manage-trackers}{Your
  Ad Choices}
\item
  \href{https://www.nytimes3xbfgragh.onion/privacy}{Privacy}
\item
  \href{https://help.nytimes3xbfgragh.onion/hc/en-us/articles/115014893428-Terms-of-service}{Terms
  of Service}
\item
  \href{https://help.nytimes3xbfgragh.onion/hc/en-us/articles/115014893968-Terms-of-sale}{Terms
  of Sale}
\item
  \href{https://spiderbites.nytimes3xbfgragh.onion}{Site Map}
\item
  \href{https://help.nytimes3xbfgragh.onion/hc/en-us}{Help}
\item
  \href{https://www.nytimes3xbfgragh.onion/subscription?campaignId=37WXW}{Subscriptions}
\end{itemize}
