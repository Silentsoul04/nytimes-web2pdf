Sections

SEARCH

\protect\hyperlink{site-content}{Skip to
content}\protect\hyperlink{site-index}{Skip to site index}

\href{https://www.nytimes3xbfgragh.onion/section/technology/personaltech}{Personal
Tech}

\href{https://myaccount.nytimes3xbfgragh.onion/auth/login?response_type=cookie\&client_id=vi}{}

\href{https://www.nytimes3xbfgragh.onion/section/todayspaper}{Today's
Paper}

\href{/section/technology/personaltech}{Personal Tech}\textbar{}Wireless
Charging Is Here. So What Is It Good For?

\url{https://nyti.ms/2NZ90aq}

\begin{itemize}
\item
\item
\item
\item
\item
\item
\end{itemize}

Advertisement

\protect\hyperlink{after-top}{Continue reading the main story}

Supported by

\protect\hyperlink{after-sponsor}{Continue reading the main story}

Tech fix

\hypertarget{wireless-charging-is-here-so-what-is-it-good-for}{%
\section{Wireless Charging Is Here. So What Is It Good
For?}\label{wireless-charging-is-here-so-what-is-it-good-for}}

The technology, also known as magnetic induction, is a relatively new
feature for powering iPhones and popular Android phones. Most people
don't use it, but here are a few benefits.

\includegraphics{https://static01.graylady3jvrrxbe.onion/images/2018/10/04/business/04Techfix-illo/04Techfix-illo-articleLarge.gif?quality=75\&auto=webp\&disable=upscale}

\href{https://www.nytimes3xbfgragh.onion/by/brian-x-chen}{\includegraphics{https://static01.graylady3jvrrxbe.onion/images/2018/02/16/multimedia/author-brian-x-chen/author-brian-x-chen-thumbLarge.jpg}}

By \href{https://www.nytimes3xbfgragh.onion/by/brian-x-chen}{Brian X.
Chen}

\begin{itemize}
\item
  Oct. 3, 2018
\item
  \begin{itemize}
  \item
  \item
  \item
  \item
  \item
  \item
  \end{itemize}
\end{itemize}

Besides getting bigger, smartphones keep getting --- for lack of a
better word --- glassier. From front to back, the bodies of many of the
newest smartphones are composed of glass.

The trend is not part of a broad conspiracy to make you shatter your
phone so that you buy a new one. Instead, glass lets energy pass through
the phone so that it can be charged wirelessly. The technology relies on
magnetic induction, which involves using an electrical current to
generate a magnetic field, creating voltage that powers the phone
without your plugging a wire into it.

Many people are excited about charging without cords. A study by
\href{https://www.surveymonkey.com/mp/survey-methodology/}{SurveyMonkey}
found that wireless charging was the most anticipated feature in last
year's new iPhones. Yet in a
\href{https://technology.ihs.com/600120/half-a-billion-smartphones-and-other-devices-with-wireless-power-technology-shipped-in-2017-ihs-markit-says}{survey}
by the research firm IHS, only 29 percent said they used wireless
charging last year.

That may be because wireless charging isn't truly wireless. People
typically need accessories from companies like Samsung, Mophie and Anker
--- which generally look like mats and stands that you can set your
phone on --- to wirelessly power up. And while you don't have to plug a
cable into the phone, the accessories themselves have to be hooked up to
a power outlet.

There's a trade-off, too: Wireless charging is less efficient at
transferring energy than a wire, and is thus slower at refilling a
battery. (Mophie said that generally, when both types of chargers were
on the same wattage, wireless was about 15 percent slower.)

So what's the point?

Charlie Quong, vice president of product development for Mophie, said
placing wireless chargers in areas where people spent a lot of time ---
like their bedroom, car and office --- could enable them to top off
their phones more frequently by removing the hassle of plugging in.

The products help people ``get charged throughout their day without
having to deliberately park their phone down,'' Mr. Quong said. ``It's
really, really convenient.''

As a longtime skeptic of wireless chargers, I was eager to find a useful
application for the technology. So I scattered the chargers around the
focal points of my daily routine: on my nightstand, in my car, in my
briefcase, on my office desk and on a living room table.

In the end, I would consider keeping a wireless charger only in the
bedroom or in a briefcase. Here's what I found.

\hypertarget{in-the-bedroom}{%
\subsection{In the Bedroom}\label{in-the-bedroom}}

Two types of wireless chargers can be placed on a bedside table: pads
and stands.

The pad is a disc, and you lay your phone down on it. The stand keeps
your phone upright so that the screen faces you while it charges. I
tested a \href{http://www.mophie.com/shop/charge-stream-pad-plus}{Mophie
charging pad} and
\href{https://www.anker.com/products/variant/powerwave-7-5-stand/B2522121}{Anker's
PowerWave stand}.

\hypertarget{the-stand-is-convenient}{%
\subsubsection{The stand is convenient.}\label{the-stand-is-convenient}}

In this case, the charging stand was beneficial. It elevated my phone at
an angle to turn the phone into an alarm clock --- with a quick poke at
the screen, I could glance at the time, disable an alarm or look at a
calendar alert. At night when the lights were off, setting the phone on
the dock was easier than fumbling around for a wire to plug in. And the
slower charging didn't matter, because I was asleep.

The pad was less useful. With the phone lying flat, I had to remove it
from the charger whenever I wanted to check it.

\hypertarget{in-the-office}{%
\subsection{In the Office}\label{in-the-office}}

A wireless charging pad or stand can also be placed on an office desk.
The idea is to quickly set your phone down on the charger whenever you
return to the table.

\hypertarget{wireless-charging-is-less-productive}{%
\subsubsection{Wireless charging is less
productive.}\label{wireless-charging-is-less-productive}}

Wireless charging isn't beneficial in an office environment, assuming
you care about speed and productivity.

If you want your device to stay charged between meetings, the slower
speed is inconvenient. In my test, the Anker stand took about 10 minutes
longer to replenish 25 percent of battery than a wired charger.

The other downside in the office is that you can't easily use the phone
to write a message without removing it from the charger. With a cord,
you can do all your important tasks while staying plugged in.

\hypertarget{in-a-briefcase}{%
\subsection{In a Briefcase}\label{in-a-briefcase}}

For when you're on the go, Mophie offers
\href{http://www.mophie.com/shop/wireless/powerstation-wireless-xl}{rectangular
battery packs} with built-in inductive charging. All you do is place
your phone on top of the battery pack. Conveniently, the battery pack
does not have to be plugged into a power source, making this one of the
few truly wireless power products.

\hypertarget{this-is-sometimes-convenient}{%
\subsubsection{\texorpdfstring{\textbf{This is sometimes
convenient.}}{This is sometimes convenient.}}\label{this-is-sometimes-convenient}}

A wireless battery pack was great for date night. During dinner with my
partner, it was nice to skip carrying a cord. We set the battery down on
the table and took turns recharging our devices.

A wireless battery pack was less practical while on the move --- like
when I was sitting on public transportation --- because I could not
easily use the phone while charging it. Fortunately, Mophie's wireless
battery pack includes a port to plug in a power cable for those times.

\hypertarget{in-the-living-room}{%
\subsection{In the Living Room}\label{in-the-living-room}}

You could place a charging pad on a TV stand or end table for guests to
replenish their phones when they come over. It's not only convenient and
hospitable, but it also looks less tacky than dangling cords all over
your common areas.

\hypertarget{wires-are-better}{%
\subsubsection{Wires are better.}\label{wires-are-better}}

Unless all your friends and members of your family are tech enthusiasts,
it's unlikely that they all have phones capable of wireless charging.
It's a safer bet to leave a few power cables out; if you want the wires
to look less messy, you can organize them with
\href{https://www.amazon.com/gp/product/B07B4S4H5C/ref=oh_aui_search_detailpage?ie=UTF8\&psc=1}{cheap
magnetic clips}.

When you don't have company, the other downside of wireless charging in
the living room is that you can't use the phone while decompressing. For
me, it was more convenient to have an
\href{https://www.nytimes3xbfgragh.onion/2018/05/03/smarter-living/good-10-foot-charging-cables-tech.html}{extra-long
smartphone cable} plugged in while I sat on the couch and vegged out on
Instagram.

\hypertarget{in-the-car}{%
\subsection{In the Car}\label{in-the-car}}

For cars, accessory makers offer wireless charging mounts that can be
attached to air-conditioner vents or CD player slots. I tested
\href{http://www.mophie.com/shop/charge-stream-vent-mount}{Mophie's
mount}, which clipped to an air vent, with two adjustable arms holding
the phone in place. To power the charger, I had to plug in a wire
through my car's accessory port.

\hypertarget{this-is-impractical}{%
\subsubsection{This is impractical.}\label{this-is-impractical}}

Many cars now include
\href{https://www.nytimes3xbfgragh.onion/2018/02/01/business/car-navigation-systems-apps.html}{CarPlay
or Android Auto}, the infotainment systems offered by Apple and Google,
which mirror your phone's maps and apps on the screen of your car's
console. Though some cars can connect with CarPlay and Android Auto
wirelessly, many still require a wire to connect your phone to the
infotainment systems. The wire simultaneously charges your phone, so why
bother with a charging mount?

An inductive charging mount might be more useful for older cars that
lack Android Auto or CarPlay In my car, which has no infotainment
system, I have a cheap
\href{https://thewirecutter.com/reviews/best-smartphone-car-mount/}{mount}
that clips into the CD player slot, as well as a
\href{https://thewirecutter.com/reviews/best-usb-car-charger/}{USB power
charger}. I preferred my configuration to the wireless charger because
my mount felt sturdier and my wired charger was faster.

Mr. Quong said the mount was conceived as an all-in-one solution,
eliminating the need to buy two separate accessories for mounting and
charging a phone. But given the high price of \$70 for the Mophie mount,
I recommend going piecemeal.

Plus, a wire was still involved anyway.

Advertisement

\protect\hyperlink{after-bottom}{Continue reading the main story}

\hypertarget{site-index}{%
\subsection{Site Index}\label{site-index}}

\hypertarget{site-information-navigation}{%
\subsection{Site Information
Navigation}\label{site-information-navigation}}

\begin{itemize}
\tightlist
\item
  \href{https://help.nytimes3xbfgragh.onion/hc/en-us/articles/115014792127-Copyright-notice}{©~2020~The
  New York Times Company}
\end{itemize}

\begin{itemize}
\tightlist
\item
  \href{https://www.nytco.com/}{NYTCo}
\item
  \href{https://help.nytimes3xbfgragh.onion/hc/en-us/articles/115015385887-Contact-Us}{Contact
  Us}
\item
  \href{https://www.nytco.com/careers/}{Work with us}
\item
  \href{https://nytmediakit.com/}{Advertise}
\item
  \href{http://www.tbrandstudio.com/}{T Brand Studio}
\item
  \href{https://www.nytimes3xbfgragh.onion/privacy/cookie-policy\#how-do-i-manage-trackers}{Your
  Ad Choices}
\item
  \href{https://www.nytimes3xbfgragh.onion/privacy}{Privacy}
\item
  \href{https://help.nytimes3xbfgragh.onion/hc/en-us/articles/115014893428-Terms-of-service}{Terms
  of Service}
\item
  \href{https://help.nytimes3xbfgragh.onion/hc/en-us/articles/115014893968-Terms-of-sale}{Terms
  of Sale}
\item
  \href{https://spiderbites.nytimes3xbfgragh.onion}{Site Map}
\item
  \href{https://help.nytimes3xbfgragh.onion/hc/en-us}{Help}
\item
  \href{https://www.nytimes3xbfgragh.onion/subscription?campaignId=37WXW}{Subscriptions}
\end{itemize}
