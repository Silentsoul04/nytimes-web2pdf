Sections

SEARCH

\protect\hyperlink{site-content}{Skip to
content}\protect\hyperlink{site-index}{Skip to site index}

\href{https://www.nytimes3xbfgragh.onion/section/books/review}{Book
Review}

\href{https://myaccount.nytimes3xbfgragh.onion/auth/login?response_type=cookie\&client_id=vi}{}

\href{https://www.nytimes3xbfgragh.onion/section/todayspaper}{Today's
Paper}

\href{/section/books/review}{Book Review}\textbar{}December's Book Club
Pick: Casey Gerald's memoir, `There Will Be No Miracles Here'

\url{https://nyti.ms/2Jefg8J}

\begin{itemize}
\item
\item
\item
\item
\item
\end{itemize}

Advertisement

\protect\hyperlink{after-top}{Continue reading the main story}

Supported by

\protect\hyperlink{after-sponsor}{Continue reading the main story}

Now Read This

\hypertarget{decembers-book-club-pick-casey-geralds-memoir-there-will-be-no-miracles-here}{%
\section{December's Book Club Pick: Casey Gerald's memoir, `There Will
Be No Miracles
Here'}\label{decembers-book-club-pick-casey-geralds-memoir-there-will-be-no-miracles-here}}

\includegraphics{https://static01.graylady3jvrrxbe.onion/images/2018/11/25/books/review/25Jackson01/merlin_145740045_476c630f-7594-4ec2-a600-9da2ae2a2ef7-articleLarge.jpg?quality=75\&auto=webp\&disable=upscale}

Buy Book ▾

\begin{itemize}
\tightlist
\item
  \href{https://www.amazon.com/gp/search?index=books\&tag=NYTBSREV-20\&field-keywords=There+Will+Be+No+Miracles+Here\%3A+A+Memoir+Casey+Gerald}{Amazon}
\item
  \href{https://du-gae-books-dot-nyt-du-prd.appspot.com/buy?title=There+Will+Be+No+Miracles+Here\%3A+A+Memoir\&author=Casey+Gerald}{Apple
  Books}
\item
  \href{https://www.anrdoezrs.net/click-7990613-11819508?url=https\%3A\%2F\%2Fwww.barnesandnoble.com\%2Fw\%2F\%3Fean\%3D9780735214200}{Barnes
  and Noble}
\item
  \href{https://www.anrdoezrs.net/click-7990613-35140?url=https\%3A\%2F\%2Fwww.booksamillion.com\%2Fp\%2FThere\%2BWill\%2BBe\%2BNo\%2BMiracles\%2BHere\%253A\%2BA\%2BMemoir\%2FCasey\%2BGerald\%2F9780735214200}{Books-A-Million}
\item
  \href{https://bookshop.org/a/3546/9780735214200}{Bookshop}
\item
  \href{https://www.indiebound.org/book/9780735214200?aff=NYT}{Indiebound}
\end{itemize}

When you purchase an independently reviewed book through our site, we
earn an affiliate commission.

By Mitchell S. Jackson

\begin{itemize}
\item
  Oct. 24, 2018
\item
  \begin{itemize}
  \item
  \item
  \item
  \item
  \item
  \end{itemize}
\end{itemize}

\textbf{THERE WILL BE NO MIRACLES HERE}\\
\textbf{A Memoir}\\
By Casey Gerald\\
386 pp. Riverhead Books. \$27.

There we were, about a thousand people gathered at the 2016 TED
conference, waiting for the very last speaker of a weeklong lineup of
exceptional humans. The session's host announced that the speaker would
be Casey Gerald, who swaggered onto the stage, an athletically built,
baldheaded and clean-shaven black man dressed in all black. Gerald
stood, backdropped by velvet curtains, waited for the applause to quiet,
and then began sharing an anecdote about the time when, on New Year's
Eve 1999, he sat in a church with his grandmother and her congregation,
fearing that when the clock struck midnight, the rapture would commence.

Gerald went on to share stories from a journey that began when he was a
boy in a blighted Dallas neighborhood and spanned up to his role as the
cynosure of a room comprising no small number of the 1 percent. Near the
end of his talk, Gerald announced the disbandment of MBAs Across
America, an organization he co-founded to connect business students with
entrepreneurs around the country. He also proclaimed that he was
shirking the role of savior that had been foisted upon him, ``because
our time is too short and our odds are too long to wait for second
comings, when the truth is, that there will be no miracles here.''

Gerald's magnificent memoir, ``There Will Be No Miracles Here,'' opens
with the same anecdote that began his TED talk, though in the book, he
punctuates the retelling by announcing a kind of thesis. ``Mine, then,
is the story of a peasant boy...and, with luck, God and His miracles or
lack thereof,'' he writes. Indeed, in just over three decades, what a
phenomenal life the self-proclaimed peasant boy has lived.

He spent his early childhood in Ohio, where his father had been a
football star at Ohio State University. When Gerald was 8 years old, his
father moved the family, which includes his older sister and mother,
back to their hometown --- the Oak Cliff neighborhood of Dallas. Back
home, Gerald's father began working for his father, who is, as Gerald
puts it, involved in ``the greatest business in America: the business of
saving souls.''

The successes of Gerald's grandfather proved enough to turn Gerald's
father into his ``supplicant.'' Meanwhile, Gerald's mother, whom he
describes as a woman of curious habits, stayed at home applying makeup
for a fair amount of the day. We learn early on that she suffered from
manic depression and bipolar disorder, that his father developed a drug
habit, one that landed him in prison, and that his older sister assumed
the role of his caretaker. Gerald's mother disappeared later, leaving
him to wonder for years whether she had died.

Around this time, Gerald started to explore his sexual identity with the
help of a new thing called the internet. He did it in secret, since, as
he puts it, ``I was in the early stages of crafting a new life, or a new
story, in the image of perfection.'' He also began playing sports,
although his athletic success wasn't immediate. In a hilarious passage,
he describes a youth football game where the defense kept blocking his
end-zone attempts right around the line of scrimmage. ``Goddamn it,
son!'' his coach said. ``Listen to me. You're embarrassing yourself.
You're embarrassing your family. Get your ass low, keep your eyes open,
and run for your life!''

Image

Gerald lived an itinerant existence in high school until his sister, who
had briefly escaped to college, returned to Dallas and insisted the two
live together. They scraped by until Gerald came up with a scheme to
supplement his sister's meager income by cashing the disability checks
of their missing mother. After a year of this, the siblings discovered
the account had been shut down, a fact that ended their hustle but also
suggested their missing mother was alive. His sister located her in St.
Louis, and they drove to pick her up.

While all of this domestic chaos was going on, Gerald evolved into a
celebrated scholar athlete, one recruited by the Yale University
football team. And though he hadn't heard of Yale before that
recruitment, he decided to attend the school in the belief that his
acceptance had transmuted him into a symbol, into the great pride of his
school, town, people.

It didn't take long after he arrived at Yale for Gerald to divine the
ethos of the students and, in particular, the apparent class divide
among the black students, who were invested, he writes, ``in the
distinction between their kind and mine.'' He couldn't shake his feeling
of alienation from people he imagined would help him: ``The more time I
spent in their midst, the more I became convinced that \emph{they} were
the problem --- not any individual boy or girl or mother or father but
the ideas that they represented, of a class apart, and all the trappings
that came with it: the mixer, the galas, the networking reception, the
panels to discuss blackness in theory when actual blackness was having
one hell of a hard time right down the street --- when \emph{I} was
having a hard time.''

Fueled in part by an intent to surpass Yale's black bourgeoisie, he and
a few friends established the Yale Black Men's Union. He later joined
Wolf's Head, one of the college's oldest and most esteemed secret
societies. Meanwhile, his football cohort matured from a crew of bench
warmers into starters on some of the best teams in Yale history. Gerald
became not only a team star but a finalist for the Draddy Trophy, which
honors the nation's top scholar athlete, as well as a finalist for a
Rhodes scholarship. One of the book's most engrossing moments involves
the crisis of having his Rhodes interview scheduled on the same day as
the Yale-Harvard game.

At times Gerald moves too quickly to the next scene or idea, when he
might have benefited from a more sustained explanation of his thinking.
On the other hand, he just might have crafted a consummate 21st-century
memoir for readers whose brains have been rewired by Google, their
attention always under siege. Gerald also pushes stylistic conventions,
with short passages where he writes about himself in the third person or
directly addresses the reader. He includes metanarratives as well as
letters, emails and speeches. And ever present is the enchantment of his
voice, one that is at turns exuberant, humorous, unsentimental,
imaginative, keen. While Gerald's style is engaging, the locus of the
book is his extraordinary journey.

Though the chronology is a little unclear regarding the end of his time
at Yale and beyond, his odyssey includes tenures in Massachusetts, New
York, Washington, D.C., and back home in Texas. It leads him to Lehman
Brothers right before their 2008 collapse, then to one of Washington's
most influential think tanks, then to Harvard Business School, where he
and a few peers founded MBAs Across America. Along the way, he learns
plenty about his country, the elites who run it and the underclass
subject to their rule. He often relays his insight with indelible
aphorism. For instance, he writes that America is ``ruled on the surface
by people with \emph{authority}, ruled in fact by people with
\emph{power} --- people, often, in the shadows.''

A few years before Gerald suffered the terror of believing he'd been
left behind in the rapture, his fifth-grade teacher assigned him to
write a speech titled ``I'm the Mayor Now and This Is My New Plan.''
Gerald explains that since he was unsure whether he had to deliver the
speech from memory, he ``assumed the worst.'' He enlists some of his
sister's friends to help him brainstorm and writes the speech from his
notes. The next day, he recites it in class without botching a single
word. He's insouciant about the deed but his teacher screeches her
astonishment. ``One night, in slavish fear, I got my homework so wrong
that it was perfect,'' he writes about the experience. Gerald might have
once seen himself as a peasant boy, and maybe deep down still does. But
his life, and this memoir, serve as proof of his prodigious talents, of
the truth that, for the gifted like him, struggles that range from a
serious hardship to a little mistake can yield something miraculous.

Advertisement

\protect\hyperlink{after-bottom}{Continue reading the main story}

\hypertarget{site-index}{%
\subsection{Site Index}\label{site-index}}

\hypertarget{site-information-navigation}{%
\subsection{Site Information
Navigation}\label{site-information-navigation}}

\begin{itemize}
\tightlist
\item
  \href{https://help.nytimes3xbfgragh.onion/hc/en-us/articles/115014792127-Copyright-notice}{©~2020~The
  New York Times Company}
\end{itemize}

\begin{itemize}
\tightlist
\item
  \href{https://www.nytco.com/}{NYTCo}
\item
  \href{https://help.nytimes3xbfgragh.onion/hc/en-us/articles/115015385887-Contact-Us}{Contact
  Us}
\item
  \href{https://www.nytco.com/careers/}{Work with us}
\item
  \href{https://nytmediakit.com/}{Advertise}
\item
  \href{http://www.tbrandstudio.com/}{T Brand Studio}
\item
  \href{https://www.nytimes3xbfgragh.onion/privacy/cookie-policy\#how-do-i-manage-trackers}{Your
  Ad Choices}
\item
  \href{https://www.nytimes3xbfgragh.onion/privacy}{Privacy}
\item
  \href{https://help.nytimes3xbfgragh.onion/hc/en-us/articles/115014893428-Terms-of-service}{Terms
  of Service}
\item
  \href{https://help.nytimes3xbfgragh.onion/hc/en-us/articles/115014893968-Terms-of-sale}{Terms
  of Sale}
\item
  \href{https://spiderbites.nytimes3xbfgragh.onion}{Site Map}
\item
  \href{https://help.nytimes3xbfgragh.onion/hc/en-us}{Help}
\item
  \href{https://www.nytimes3xbfgragh.onion/subscription?campaignId=37WXW}{Subscriptions}
\end{itemize}
