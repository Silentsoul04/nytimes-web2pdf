Sections

SEARCH

\protect\hyperlink{site-content}{Skip to
content}\protect\hyperlink{site-index}{Skip to site index}

\href{https://www.nytimes3xbfgragh.onion/section/technology}{Technology}

\href{https://myaccount.nytimes3xbfgragh.onion/auth/login?response_type=cookie\&client_id=vi}{}

\href{https://www.nytimes3xbfgragh.onion/section/todayspaper}{Today's
Paper}

\href{/section/technology}{Technology}\textbar{}China's King of Internet
Fluff Wants to Conquer the World

\url{https://nyti.ms/2yFmMVZ}

\begin{itemize}
\item
\item
\item
\item
\item
\end{itemize}

Advertisement

\protect\hyperlink{after-top}{Continue reading the main story}

Supported by

\protect\hyperlink{after-sponsor}{Continue reading the main story}

\hypertarget{chinas-king-of-internet-fluff-wants-to-conquer-the-world}{%
\section{China's King of Internet Fluff Wants to Conquer the
World}\label{chinas-king-of-internet-fluff-wants-to-conquer-the-world}}

!{[}Bytedance says that more than half a billion people worldwide use
its video apps Douyin (in China) or TikTok (the rest of the world) at
least once a month.

Credit...Images: TikTok and Douyin. Photo illustration by The New York
Times{]}(\url{https://static01.graylady3jvrrxbe.onion/images/2018/10/29/business/00BYTEDANCE-COMBO/00BYTEDANCE-COMBO-articleLarge.jpg?quality=75\&auto=webp\&disable=upscale})

By \href{https://www.nytimes3xbfgragh.onion/by/raymond-zhong}{Raymond
Zhong}

\begin{itemize}
\item
  Oct. 29, 2018
\item
  \begin{itemize}
  \item
  \item
  \item
  \item
  \item
  \end{itemize}
\end{itemize}

\href{https://cn.nytimes3xbfgragh.onion/technology/20181030/bytedance-app-funding-china/}{阅读简体中文版}\href{https://cn.nytimes3xbfgragh.onion/technology/20181030/bytedance-app-funding-china/zh-hant/}{閱讀繁體中文版}

BEIJING --- A Chinese internet company that serves up homemade
break-dancing videos, dishy news bites and goofy hashtag challenges has
become one of the planet's most richly valued start-ups, with a roughly
\href{https://www.nytimes3xbfgragh.onion/2018/09/28/technology/bytedance-fundraising-toutiao-tiktok.html}{\$75
billion price tag}. And it has big plans for storming phone screens
across the rest of the globe, too.

You may not have heard of the company, Bytedance. You may never have
used any of its breezy, colorful apps. But your nearest teenager is
probably already obsessed with Musical.ly, the
\href{https://www.nytimes3xbfgragh.onion/2016/08/10/technology/china-homegrown-internet-companies-rest-of-the-world.html}{video-sharing
platform} that
\href{https://www.nytimes3xbfgragh.onion/2017/11/10/business/dealbook/musically-sold-app-video.html}{Bytedance
bought for around \$1 billion last year} and folded into its own video
service, TikTok.

``Frankly, it's meaningless stuff,'' said Dong Yaxin, 20, a college
student in Beijing who says he is active every day on Douyin, the
Chinese version of TikTok. Bytedance says that more than half a billion
people worldwide use Douyin or TikTok at least once a month.

Cute pet videos. Lip-syncing to pop ear worms. Glossy digital effects
...

A clip from @falcopunch on TikTok.

... and people who are very good at doing the robot.

A clip from @rezbkr on TikTok.

``There isn't such a strong sense of purpose on Douyin,'' Mr. Dong said.
``That's actually what's so good about it.''

But even for a purveyor of fluff, crossing the tech world's most
treacherous divide will not be carefree. There are two major internets
right now: China's and the rest of the world's. Beijing's tough rules on
content and operations have long made China difficult, even impossible,
terrain for American internet companies.

Those rules have also largely penned in homegrown titans like Tencent,
whose overseas expansion plans have been hamstrung by the unique demands
of catering to China's online population.

So far, Bytedance --- which recently secured \$3 billion in new funding
from SoftBank and other heavyweight investors --- has found a rare
measure of success in both internets by doing things a little
differently.

For one, it is making no pretense to be bridging the two digital realms.

Users of Douyin are entirely walled off from users of TikTok and vice
versa; the better to manage the material that people in China can see.
Beijing's tightening controls have made these decidedly un-fun times to
be in the business of fun.

\href{https://www.nytimes3xbfgragh.onion/2018/08/31/technology/china-videogames-myopia-tencent.html}{Video
game companies},
\href{https://www.nytimes3xbfgragh.onion/2017/06/09/world/asia/china-celebrity-news-wechat.html}{celebrity
gossip bloggers} and
\href{https://www.nytimes3xbfgragh.onion/2018/10/16/world/asia/china-yang-kaili-anthem.html}{live-streaming
stars} have all been through the wringer recently as the government
works harder to stamp out cultural content that it deems unhealthy or
unwholesome. The crackdown has not spared Bytedance --- the authorities
ordered the company's joke-sharing app offline
\href{https://www.nytimes3xbfgragh.onion/2018/04/11/technology/china-toutiao-bytedance-censor.html}{in
April this year}.

The company has also crossed borders with relative ease by focusing on
light, affirming fare, and on attracting young ---
\href{https://www.nytimes3xbfgragh.onion/2016/09/17/business/media/a-social-network-frequented-by-children-tests-the-limits-of-online-regulation.html}{very
young} --- users. But the Chinese Communist Party is not alone in having
discovered a sordid side to Bytedance's platforms.

Both before the company bought Musical.ly and since, horrified parents
and others have reported finding adolescent users
\href{https://medium.com/s/parenting-stories/porn-is-not-the-worst-thing-on-musical-ly-5df07ab842af}{showing
off suggestive dance moves} on the app, mouthing lyrics about rough sex
and worse. Police in Britain have
\href{http://www.edp24.co.uk/news/crime/police-investigation-after-two-children-from-eastgate-academy-in-king-s-lynn-allegedly-groomed-on-musical-ly-app-1-5036808}{investigated}
reports of adults propositioning children through Musical.ly.

Bytedance added new privacy settings and parental controls to TikTok in
June. But if the company, which declined to comment for this article,
cannot expand its ability to manage such issues at the same rapid clip
at which it is drawing new users, its products could become the bane of
many more parents and governments in many more countries.

Their children might not care.

Kang Sae-eun, 14, an eighth grader in Seoul, loves watching other young
South Koreans on TikTok. There's the girl who makes crazy faces, and the
excellent dancer. There's the cool girl with short hair --- real ``girl
crush'' material, she said.

They are funny and uninhibited, Sae-eun said. And best of all, they are
regular kids like her.

``It is much harder for young people like elementary school students to
become famous on the better-known platforms, like YouTube, Facebook or
Instagram, all of which I also use,'' she said.

Sae-eun said she didn't realize that TikTok was made in China, which
raises what might be the most interesting question about Bytedance: How
did a company that is further democratizing self-expression come out of
sternly undemocratic China in the first place?

Bytedance, which was founded in 2012, did not set out to dominate the
market for bite-size videos. For many years, the company's best-known
product was not Douyin but a news aggregator called Jinri Toutiao, which
uses machine learning to figure out what users like, then feeds them
more of it.

In China, few media outlets command much loyalty among readers. That
means an aggregator is a valuable and timesaving way to figure out what
to read.

After a while, though, Beijing realized that an app that gave people
exactly what they wanted ended up giving them a lot of
not-very-wholesome stuff.

Last December, after China's internet regulator accused Toutiao of
\href{https://www.nytimes3xbfgragh.onion/2018/01/02/business/china-toutiao-censorship.html}{spreading
``pornographic'' information}, Bytedance halted updates to several
sections of the app and removed or suspended hundreds of content
creators. A few months later, Toutiao was
\href{https://www.nytimes3xbfgragh.onion/2018/04/11/technology/china-toutiao-bytedance-censor.html}{temporarily
removed from app stores} for unspecified reasons. And Bytedance's
joke-sharing app, Neihan Duanzi, was
\href{https://www.nytimes3xbfgragh.onion/2018/04/12/business/china-bytedance-duanzi-censor.html}{shut
down entirely}.

In a lengthy letter of apology, the company's founder and chief
executive, Zhang Yiming, vowed to increase the number of employees
moderating content to 10,000 from 6,000.

``The product went astray, and content appeared that did not accord with
core socialist values,'' Mr. Zhang wrote.

By then, Bytedance had another rising star in its stable.

Douyin was not even Bytedance's first video app when it was released in
2016. But in the somewhat arbitrary, mildly mysterious way in which
these things happen, it became huge.

The app is engineered for swift, maximal addictiveness.

Open Douyin or TikTok and you are plunged right into a video. Swipe up
to get another, each refresh of the screen providing a dopamine jolt.
The videos fill your phone display entirely, blocking the clock at the
top and preventing you from seeing how many hours you have spent
watching puppies and comedy skits and synchronized dancing.

Satsuki Hatashita, a 20-year-old college student in western Japan, has
been hooked for months. She now knows not to use the app before taking a
shower. ``I wouldn't be able to shower for a long time, until I finally
stopped watching TikTok,'' she said.

She, too, was surprised to learn that the app was Chinese.

People like Ms. Hatashita have given Bytedance confidence in its march
overseas. The company has opened offices in Japan, Brazil, India, the
United States and beyond.

Still, Chinese staff stationed in China oversee significant aspects of
Bytedance's international apps. They even produce some culturally
specific content, such as push notifications suggesting videos to watch.
The company is hiring speakers of more than a dozen languages, including
Portuguese, Polish, Malay and Arabic, for positions in China, according
to an online posting.

An episode this year points to the importance, for Bytedance, of having
people on the ground in at least one area: government relations.

In July, the authorities in Indonesia
\href{http://www.thejakartapost.com/news/2018/07/03/what-is-tik-tok-an-app-blocked-before-we-older-people-even-heard-of-it.html}{temporarily
blocked TikTok} for hosting what they called ``pornography,
inappropriate content and blasphemy.'' The Indonesian government had
contacted TikTok's Singapore office to give a few days' warning. But it
didn't receive a response until after the app was shut down, Rudiantara,
Indonesia's minister of information, said in an interview.

Bytedance's recent hires suggest that it wants to avoid similar
incidents. Instagram's head of public policy for the Asia-Pacific
region, Helena Lersch, recently resigned to become Bytedance's director
for global public policy. Facebook's public policy leads in Indonesia
and Japan recently left to join Bytedance, too.

Before Douyin took off, China's internet didn't have a reigning social
platform dedicated to short, easy-to-make videos. In the rest of the
world's internet, where Instagram, Snapchat and others are already
popular, TikTok faces stiff competition.

For Tao Ni, a 25-year-old newspaper reporter in eastern China, Tencent's
messaging app WeChat has already become more of a tool for work than a
fun way to kill time. Weibo, a popular Twitter-like platform, can be
wearying. But not Douyin, Ms. Tao said.

It's because each video is so short, she said, that she can end up
spending hours on what amounts to channel-surfing. ``Anything longer
than 15 seconds, and I might start to feel tired.''

Advertisement

\protect\hyperlink{after-bottom}{Continue reading the main story}

\hypertarget{site-index}{%
\subsection{Site Index}\label{site-index}}

\hypertarget{site-information-navigation}{%
\subsection{Site Information
Navigation}\label{site-information-navigation}}

\begin{itemize}
\tightlist
\item
  \href{https://help.nytimes3xbfgragh.onion/hc/en-us/articles/115014792127-Copyright-notice}{©~2020~The
  New York Times Company}
\end{itemize}

\begin{itemize}
\tightlist
\item
  \href{https://www.nytco.com/}{NYTCo}
\item
  \href{https://help.nytimes3xbfgragh.onion/hc/en-us/articles/115015385887-Contact-Us}{Contact
  Us}
\item
  \href{https://www.nytco.com/careers/}{Work with us}
\item
  \href{https://nytmediakit.com/}{Advertise}
\item
  \href{http://www.tbrandstudio.com/}{T Brand Studio}
\item
  \href{https://www.nytimes3xbfgragh.onion/privacy/cookie-policy\#how-do-i-manage-trackers}{Your
  Ad Choices}
\item
  \href{https://www.nytimes3xbfgragh.onion/privacy}{Privacy}
\item
  \href{https://help.nytimes3xbfgragh.onion/hc/en-us/articles/115014893428-Terms-of-service}{Terms
  of Service}
\item
  \href{https://help.nytimes3xbfgragh.onion/hc/en-us/articles/115014893968-Terms-of-sale}{Terms
  of Sale}
\item
  \href{https://spiderbites.nytimes3xbfgragh.onion}{Site Map}
\item
  \href{https://help.nytimes3xbfgragh.onion/hc/en-us}{Help}
\item
  \href{https://www.nytimes3xbfgragh.onion/subscription?campaignId=37WXW}{Subscriptions}
\end{itemize}
