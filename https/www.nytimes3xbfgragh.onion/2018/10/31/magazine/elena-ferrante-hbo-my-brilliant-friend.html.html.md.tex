Elena Ferrante Stays Out of the Picture

\url{https://nyti.ms/2CP5oBb}

\begin{itemize}
\item
\item
\item
\item
\item
\item
\end{itemize}

\includegraphics{https://static01.graylady3jvrrxbe.onion/images/2018/11/04/magazine/04mag-ferrante-image1/04mag-ferrante-image1-articleLarge.jpg?quality=75\&auto=webp\&disable=upscale}

Sections

\protect\hyperlink{site-content}{Skip to
content}\protect\hyperlink{site-index}{Skip to site index}

Feature

\hypertarget{elena-ferrante-stays-out-of-the-picture}{%
\section{Elena Ferrante Stays Out of the
Picture}\label{elena-ferrante-stays-out-of-the-picture}}

For two years, the pseudonymous literary sensation corresponded
regularly with the director adapting ``My Brilliant Friend'' for HBO.
But the resulting show is a testament to her elusiveness.

Credit...Illustration by Lino Lago

Supported by

\protect\hyperlink{after-sponsor}{Continue reading the main story}

By Merve Emre

\begin{itemize}
\item
  Oct. 31, 2018
\item
  \begin{itemize}
  \item
  \item
  \item
  \item
  \item
  \item
  \end{itemize}
\end{itemize}

Saverio Costanzo, the 43-year-old director of the HBO limited series
``My Brilliant Friend,'' is a haunted man. For over a decade, he has
corresponded with a woman whose face he cannot see, whose voice he
cannot hear, whose existence is confirmed only by the many thousands of
words she has written dissecting his artistic choices. When he speaks of
her, his black eyes turn upward, as if seeking a trace of her in the
cracks of the ceiling or in some metaphysical plane high above the
penthouse suite of the Beverly Hilton Hotel, where Costanzo and the cast
of ``My Brilliant Friend'' have arrived for HBO's summer press tour.
``Sometimes she was so strong,'' he said, gruffly. ``I don't know. I'm
still trying to put everything together. It's very hard. It was like
working with a ghost.''

Costanzo's ghost has a name: Elena Ferrante, the pseudonymous author of
the four beloved Neapolitan novels, of which ``My Brilliant Friend'' is
the first to be adapted for television.
(\href{https://www.nytimes3xbfgragh.onion/2017/05/26/books/elena-ferrante-on-my-brilliant-friend-moving-to-the-screen.html}{It
will air on HBO on Nov. 18.}) She initially appeared to Costanzo in
2007, when he wrote to her Italian publisher, Edizioni E/O, to purchase
the film rights to Ferrante's 2006 novella ``The Lost Daughter.'' He was
drawn to Ferrante's ``very small, very accurate, very dangerous''
novellas, this one about Leda, a middle-aged English professor seized by
guilt and a sense of inadequacy over the onetime abandonment of her
husband and children. While summering on the Ionian coast, Leda steals a
little girl's doll at the beach and watches as her mother tries, and
fails, to contain the child's rippling misery. Costanzo wanted to see if
he could create a visual idiom to match Ferrante's ability to make
readers ``uncomfortable.''

It seemed unlikely that Ferrante would agree. Costanzo thinks she might
have been disappointed by the adaptations of her two previous novellas
and that she wanted nothing more to do with what she has called ``the
world of show business, with its many moving parts and conspicuous cash
flow.'' He had already abandoned the idea when he received, through her
publisher, an admiring message from Ferrante, issuing him a challenge.
She was willing to cede him the rights to ``The Lost Daughter'' for six
months, enough time for him to devise an adaptation that would please
them both. For six months, Costanzo labored; for six months, ``The Lost
Daughter'' resisted his intrusions, until finally he told her publisher
he would renounce the rights. ``I was just a kid,'' he recalls.

For nine years, Costanzo heard nothing from Ferrante. He grew up and
became one of Italian cinema's youngest and most challenging auteurs. He
directed a series of claustrophobic dramas not unlike Ferrante's
novellas, featuring characters whose lonely and inscrutable acts of
destruction --- a teenager's self-mutilation in ``The Solitude of Prime
Numbers'' (2010), a mother's slow starvation of her child in ``Hungry
Hearts'' (2014) --- poison the people around them. His female leads,
especially, inspire pity, fear and revulsion before they inspire
sympathy, and then only sparingly. Then one day in 2016, he received a
surprising phone call from Edizioni E/O informing him that he was one of
a few directors Ferrante suggested for a television adaptation of ``My
Brilliant Friend''; some weeks later, the producers called to tell him
that he had been chosen to direct.

Costanzo was reluctant. The last thing he wanted to do with his career
was adapt a novel --- and not just any novel, but a novel that had
surpassed ordinary best-seller status to emerge, instead, as an event, a
sensation, a literary pathology: ``Ferrante fever,'' as readers had
taken to calling the frenzy that greeted the publication of each
Neapolitan novel --- the midnight-release parties; the grave discussions
about the books' covers; the jostling reviews, with each critic claiming
to know her art more intimately than the critic who came before. He did
not want to deal with the expectations of Ferrante's readers, who were
inclined to project onto her punishing tale of female friendship the
faces of women they had once loved and hated in equal measure. But ``My
Brilliant Friend'' was the rarest of opportunities: a second chance for
him to create his own story with Elena Ferrante. ``She was giving me her
hands and saying: `I did it --- why don't you do it?' '' he told me,
reaching his hands into the empty space before him, as if she might
appear to fold them into hers.

Now when Costanzo talks about Ferrante, it is with a deference you
rarely see directors exercise toward writers whose work they adapt. She
has commented by email on drafts of all eight of his scripts. She has
flagged moments when his dialogue verges on the melodramatic. (``She was
just saying, `This dialogue is ridiculous, the way she talks here is
ridiculous.' '') She has protected him from serious missteps, like when
he thought to cut the loud, quarrelsome wedding banquet that ends ``My
Brilliant Friend'' from the series because he was overbudget and running
behind schedule. (``She said: `Listen, the first moment I thought about
``My Brilliant Friend,'' the first image I had was a banquet, a very
vulgar banquet of Neapolitan life. Please put the banquet back in.' '')
``She is very strong,'' Costanzo repeated, permitting himself a sheepish
little laugh. ``I like that.'' When he recalled all the times he had
risked disappointing her, he pouted, like a child who has failed, yet
again, to live up to his mother's expectations.

\textbf{The Neapolitan novels} tell the story of a writer, also named
Elena (``Lenù,'' for short), whose subject is the filth of the
neighborhood in Naples where she grew up and her long acquaintance with
Lila, the brilliant, disagreeable classmate she leaves behind and
returns to intermittently over the next 50 years. When Lila disappears
without warning at age 66, leaving no clothes, shoes, letters or
photographs to testify to her existence, Lenù decides to write about
their long and troublesome relationship: their shared love of reading,
the education they defied their parents to pursue, the writing they
collaborated on, the men they both loved, the children they raised
together, all set against the backdrop of postwar Italy's social and
political turbulence. Spurred by anger and a desire for vengeance, Lenù
sets out to counter Lila's self-erasure by preserving her life in the
form of a novel, making their history irrevocably present in the
reader's imagination.

\href{https://www.nytimes3xbfgragh.onion/2018/05/18/books/review/animal-gazer-edgardo-franzosini.html?action=click\&module=Intentional\&pgtype=Article}{\emph{{[}Want
to read Italian fiction beyond Ferrante? Here are some more picks{]}}}

Though the novels are billed as tales of female friendship,
``friendship'' always skates on the edge of absurdity --- intimacy is
inseparable from violation. Lenù is the one making art out of Lila's
life, but she suspects that Lila has driven her to do it, that the
pleasure she derives from writing and reading is spiked by the pain of
submitting to another's will. It is a fitting model for the relationship
Ferrante's readers have with her novels, which are universally
celebrated for their addictiveness. You are pulled, sometimes dragged,
along by Ferrante's prose with an intensity that seems at once utterly
singular and reassuringly dispersed. To read her novels is to feel that
you are drawing on a reservoir of shared emotion --- rage, disgust,
pity, indignation, tenderness --- to which you have somehow, secretly,
contributed.

Ferrante's women are inscrutable, their minds deep and disordered and
disinclined to sentimentality, to easy morals. As a narrator, Lenù
recalls her younger self tentatively, piecing together hypotheses that
tend to obscure rather than clarify her motivations. Her choice words in
examining her actions are ``maybe,'' ``or'' and ``who knows.'' ``Maybe,
I thought, I've given too much weight to the cultivated use of reason,
to good reading, to well-controlled language, to political
affiliation,'' she reflects on her education, the reason she has escaped
the neighborhood while Lila has stayed behind. ``Maybe, in the face of
abandonment, we are all the same.'' There is consciousness here --- the
quickening of a mind eager to reflect on its past --- but no
interiority: no private, orderly, honest ``I'' that maps the depths and
boundaries of the self.

Yet it is precisely because Ferrante's characters are so undefined that
they seem readily inhabited by others, both inside and outside the
novel. ``I realized then that she wasn't capable of thinking that she
was her self and I was my self,'' Lenù observes of Lila. It is a will to
identification she reciprocates. ``My model remained Lila,'' she writes.
``I wanted to say and do what I imagined she would say and do if she had
my tools, if she had not confined herself within the space of the
neighborhood.'' The ``I'' that Ferrante conjures is restless, unbounded,
permeable to the monstrous desires that many women feel but few dare
express. It is easy to slip on and to mistake for your own.

Though her characters' minds are indefinite and abstract, their bodies
are always present. The women in Ferrante's novels bleed and break. They
know the monotonous injuries inflicted on them by men seeking only their
own satisfaction, as well as the frank, intense sexual pleasure that
arrives when you least expect it. In an extraordinary scene toward the
end of ``My Brilliant Friend,'' when Lenù bathes Lila before her wedding
and, many years later, remembers ``the violent emotion that overwhelms
you, so that it forces you to stay, to rest your gaze on the childish
shoulders, on the breasts and stiffly cold nipples, on the narrow hips
and the tense buttocks, on the black sex, on the long legs, on the
tender knees, on the curved ankles, on the elegant feet; and to act as
if it's nothing when instead everything is there.'' Here, there is no
doubting ``I.'' There is only ``you'' --- you, the reader --- seduced
into sharing the exquisite, confounding pleasure of desiring an imagined
woman's body.

\includegraphics{https://static01.graylady3jvrrxbe.onion/images/2018/11/04/magazine/04mag-04ferrante-t_CA0/04mag-04ferrante-t_CA0-articleLarge.jpg?quality=75\&auto=webp\&disable=upscale}

Lenù's agitated gaze mirrors the desires of readers who have sought,
behind the name Elena Ferrante, the flesh-and-blood person who has
inflamed their imaginations. Since the publication of ``Troubling Love''
in 1992, Ferrante has abstained from interviews, festivals, prize
ceremonies. It is not clear when her abstention turned into anonymity,
or when that anonymity acquired its peculiar aura, but it might have
been in the mid-1990s, when she began to correspond with journalists,
answering their questions about her life, sometimes with the caveat that
her answers might be lies. By the early 2000s, there was an impressive
shortlist of people rumored to be Elena Ferrante --- men, women,
couples, collectives. In 2006, the physicists Vittorio Loreto and Andrea
Baronchelli, collaborating with the journalist Luigi Galella, used
stylometric analysis to compare her novels against a corpus of Italian
literature and concluded that she was most likely the Italian novelist
Domenico Starnone. Ten years later, the reporter Claudio Gatti used
Edizioni E/O's leaked financial statements to name someone else, a
woman. This disclosure (neither confirmed nor denied by her publisher)
was met with a public outcry that he ``spoils the fun.''

What fun, exactly? The theorist Michel Foucault once observed that
literary anonymity was nothing more than a puzzle to be solved. But
literary anonymity, as Ferrante practices it, is not a puzzle --- it is
an expressive strategy. It has a style and goals, one of which is to
multiply and muddle the distinct egos of the author: Elena as the writer
of the Neapolitan novels; Elena as their first-person narrator; Elena as
a commentator on the novels she has written. Sometimes the tension that
holds these egos in check is precisely calibrated, thrilling to behold.
``Elena Ferrante is the author of several novels,'' she wrote in an
interview with The Guardian, weaving between the first and third person.
``There is nothing mysterious about her, given how she manifests herself
--- perhaps even too much --- in her own writing, the place where her
creative life transpires in absolute fullness.'' The final Neapolitan
novel, ``The Story of the Lost Child,'' ends with Lenù writing a
``remarkably successful story'' about her and Lila called ``A
Friendship,'' a double of the Neapolitan novels, which are full of
ur-texts written by Elena Greco. Shielded by her anonymity, Ferrante has
subsumed all traces of her life into an elaborate fiction, and asked us,
her readers, to help sustain the enchantment --- to dissolve the
boundaries between the Elenas until we can no longer disentangle fiction
from reality, or identify who among us is responsible for creating this
enthralling state of affairs. We are all her collaborators.

Ferrante often describes her novels as mysterious, inviolable creatures
that have escaped her grasp and journeyed freely into the world. Their
immortal life offers a supplement to her mortal one --- and suggests
that we can revive the historical moment before authorship, before
writers owned the words they wrote, before the spines of books came
bearing names. Yet, paradoxically,
\href{https://www.nytimes3xbfgragh.onion/2016/10/03/books/elena-ferrante-anita-raja-domenico-starnone.html}{Ferrante's
self-erasure} has had the opposite effect from what she claims. It has
resurrected a powerful, almost transcendent, myth of the author as
removed from the realities of time and space, a creator whose novels
spring from her head armored and fully formed, a theorist of her own
conditions of existence. What other writer enjoys such power?

\textbf{It is dangerous} to draw too close to that power; it convinces
you that you can share in it. When I first asked Ferrante's
English-language editor if I could put some questions to her for this
article, he explained that she was not doing any interviews. Then for
reasons unknown to me, she made an exception. It was impossible not to
speculate about why. I was vain, imagining that the questions I proposed
to her editor about literary form and the politics of collaboration were
smarter, more respectful, than the questions she was used to fielding
about friendship or identity. I wanted to please, and I imagined that if
I did, our exchange would vibrate with intellectual camaraderie.

Yet over the course of a two-month correspondence, which was mediated by
her editor, my editor and her translator, Ann Goldstein, the distance
between us seemed only to expand. She answered questions I had not asked
and ignored the ones I had. She got irritated, apologized,
misinterpreted my phrasing --- willfully, I suspected. When I asked her
what living authors she enjoyed reading, she wrote: ``I would have to
give a very complex answer, talking about various stages of my life.
I'll answer you some other time.'' When, I wondered, imagining that one
day I might open my door and find a children's wagon full of moldy
novels, with no address, no note, no glimpse of a telltale figure
hurrying away.

An interview is a collaboration, too, though like all collaborations
with Ferrante, an imbalanced one. Often she answered my questions in the
same oblique style as her narrator. ``Maybe in more than a few cases I
was overly frank,'' she wrote when I asked her what instructions she
gave Costanzo. ``Maybe I intervened, with some presumptuousness, in
irrelevant details.'' She told me she thinks collaborations between
women are more difficult than collaborations between a woman and a man,
whose authority a woman can either submit to or pretend to recognize
while pursuing her own agenda. ``Certainly it's more complicated to
recognize the authority of another woman; tradition in that case is more
fragile,'' she wrote. ``It works if, in a relationship between the
person in charge and the subordinate, the first wants the other to grow
and free herself from her subordinate status, and the second gains her
autonomy without feeling obliged to diminish the other.''

As the subordinate, I could only strategize how to ask questions that
would compel her to write useful answers for me. My initial plan was to
present myself as a new mother who found in Ferrante's fiction the
emotional tumult of motherhood as I am living it. In the note that
preceded my questions, I told her that I have found myself returning to
the third book in the Neapolitan series, ``Those Who Leave and Those Who
Stay,'' many times since having my two children. No other novel I have
read captures the vicissitudes of motherhood with such precision: the
power and vulnerability of caring for others, the intimacy and distance
between mother and child. When I became a mother, it was painful to
realize that my mother had a separate life, a different self, before she
became my mother; painful too to think that my children might not
realize this about me until it was too late.

She did not acknowledge my note.

I tried again with a question, only this time my tone was less
sentimental, more acerbic. I observed that contemporary writing on
motherhood has an irritating tendency to treat children as psychological
impediments to creativity --- as if a child must steal not only time and
energy from his mother but also language and thought. But her novels are
different: They entertain the possibility that motherhood might be an
experience conducive to creativity, even when it is tiring or onerous.
For a short time, Lila transforms motherhood into an act of grace, and
though she finds her children burdensome, Lenù's greatest professional
success comes after she becomes a mother. What did she take to be the
relationship between time spent taking care of words and time spent
taking care of children?

She was more receptive, if a little scolding. ``I very much like the way
you've formulated the question,'' she wrote. ``But I want to say that
it's not right to speak of motherhood in general. The troubles of the
poor mother are different from those of the well-off mother, who can pay
another woman to help her. But whether the mother is rich or poor, if
there is a real, powerful creative urge, the care of children, however
much it absorbs and at times even consumes us, doesn't win out over the
care of words: One finds the time for both. Or at least that was my
experience: I found the time when I was a terrified mother, without any
support, and also when I was a well-off mother. So I will take the
liberty of asserting that women should in no case give up the power of
reproduction in the name of production.''

There was something different about the style of this answer. The ``I''
she wielded seemed more present, the defenseless voice of the writer
behind the author. I asked her to say more about being a terrified
mother. What, I asked, was the nature of that terror for her?

She retreated, adopting the impersonal tone of the commentator once
again. ``I'm afraid of mothers who sacrifice their lives to their
children,'' she wrote. ``I'm afraid of mothers who surrender themselves
completely and live for their children, who hide the difficulties of
motherhood and pretend even to themselves to be perfect mothers.'' It is
tempting to rewrite these statements to reclaim the immediacy of her
``I'': ``\emph{I} was afraid of sacrificing my life to my children;
\emph{I} was afraid of surrendering myself completely.'' But nothing
authorizes it. It may not even be the right interpretation; she may
really be talking about her fear of other mothers. Why do I want to make
it about her? To do so would be to traffic in fiction. But the traffic
in fiction is pleasurable. It prompts me to study her language
carefully, to appreciate anew the words she has chosen, the phrases she
repeats, how easily she moves between sentences. It prompts me to
rewrite her words to project fears I may or may not have onto the figure
of the author --- the character she and I are sustaining. It lets me
speak without speaking for myself.

Last try. For the past two months, I told her, my 2-year-old son has
developed an obsession with her children's book ``The Beach at Night.''
The book involves a self-pitying doll a little girl abandons on the
beach at sunset, preferring to play with her new pet cat. At night, the
doll is discovered by the Mean Beach Attendant, a man who pulls a thin
golden hook from his lips and forces it into the doll's mouth, ripping
from her a secret that she has guarded with great care: her name. It
struck me as an unsubtle allegory for Ferrante's anonymity, and it was
hard to shake the sense that children were not its target audience. But
my son has two copies of the book: one he keeps in his school bag, one
for his bedside table, and sometimes before he goes to bed, he stares
for a very long time at the strange, sad pictures of the doll.

``I wrote `The Beach at Night' for a 4-year-old friend of mine who, to
her great disappointment, had just had a little sister,'' she wrote. ``I
was very surprised that my little book was considered unsuitable for
young children --- my friend had liked it. I've always believed that
stories for children should have the same energy, the same authenticity,
as good books for adults. It's a mistake to think that childhood needs
syrupy fables. The traditional fairy tales weren't made with cotton
candy.''

My son has just had a little brother, I told her. He is also
disappointed, and I use ``disappointment'' to mirror how I think she is
using it: to minimize a child's sense of abandonment, making his despair
more palatable to the mother responsible for upending his world. Maybe
my son is more discerning than I have realized. Maybe he has taken the
book as it is, innocent of authors and allegories, and found in it a
trace of his experience: a story that begins with the injury of
replacement and ends with partial restitution --- the reunion of the
little girl, her doll and, begrudgingly, the new cat. In his innocence,
my son may be a better reader than I am.

She did not respond.

\textbf{Having failed to} see more than a glimpse of Ferrante in our
correspondence, I sought traces of her influence in the early episodes
of ``My Brilliant Friend.'' Over the course of two years, she and
Costanzo have exchanged regular emails; screenwriter Francesco Piccolo
described her as ``a kind of supervisor'' of his work. ``There is
nothing wrong with a man wanting to make a film from my books,'' she
wrote in The Guardian last month. But ``even if he had a strongly
defined vision of his own, I would ask him to respect my view, to adhere
to my world, to enter the cage of my story without trying to drag it
into his.'' How had Ferrante coaxed Costanzo into the cage of her story?
What were her instructions for transforming the uncertain stuff of
Lenù's consciousness into a definite series of images?

Ferrante pronounced the child Lila ``perfect'' and the child Lenù
``effective'' at setting up the narrator's ``indecipherable'' quality.
They and their teenage counterparts are amateur actresses chosen from an
open casting call for the show that drew more than 9,000 people;
Costanzo and his casting director were looking for girls with ``sad
eyes'' and ``something a little bit broken.'' But their excellent
performances do not stop the episodes from feeling constrained. Costanzo
outsources the narrator's emotional utterances to a voice-over, which he
worried Ferrante would find ``cheesy.'' Unlike the voice-over, which
addresses the audience in formal Italian, the actresses speak stiffly in
a 1950s Neapolitan dialect tailored for the series. The sets are
realistic, but strenuously so, resembling the backdrops you often see in
heritage dramas. The neighborhood is sufficiently dusty and poor but
curiously underpopulated compared with the human riot of Ferrante's
novels, whose atmosphere of violence Costanzo literalizes in overdrawn,
almost comical, fights among angry women.

But none of this bothered Ferrante, who, Costanzo tells me, mainly
insisted on one thing: that he and the girls convey the unknowability of
her characters' minds through a technique she called acquiring
``density.'' He illustrates the concept with a metaphor he has borrowed
from Ferrante. Imagine that the lines an actress reads are a river that
runs calmly along the surface of the earth. Then imagine that the
actresses are the earth, and that under the earth is another river, a
wilder one whose current leaps in the opposite direction, whose roar is
muted. Every time the actress speaks her lines, she must offer a glimpse
of the river that runs beneath: the mysterious churn of her
consciousness, the lawlessness of a person's doubts or desires. ``What
decides the success of a character,'' Ferrante wrote to me when I asked
about density, ``is often half a sentence, a noun, an adjective that
jams the psychological machine like a wrench thrown into the works and
produces an effect that is no longer that of a well-regulated device but
of flesh and blood, of genuine life, and therefore incoherent and
unpredictable.''

The show is not always successful in capturing the senselessness of
inner life. Sometimes the glances or grimaces intended to convey
disorder simply make the actresses look confused or vacant. But when
they --- Ferrante, Costanzo, the actresses --- get it right, it is
electrifying. There is a magnificent scene in the second episode of ``My
Brilliant Friend,'' just after Lenù has been beaten for skipping school,
when she and Lila gaze at each other from opposite ends of the courtyard
where they live. It is a shot familiar from Costanzo's recent films but
intensified by Ferrante's feminist sensibilities: The space between the
girls hangs heavy with pain, injustice, loneliness, but also the dawning
of a collective consciousness. You can sense the confused stirrings of
opposition, which over the course of the four novels will swell into
defiance, a desire for retribution, the mutual yearning to fight
alongside each other --- a desire we, as viewers, can share. It's an
opening of Ferrante's cage, an invitation to join her in the shadows.

Advertisement

\protect\hyperlink{after-bottom}{Continue reading the main story}

\hypertarget{site-index}{%
\subsection{Site Index}\label{site-index}}

\hypertarget{site-information-navigation}{%
\subsection{Site Information
Navigation}\label{site-information-navigation}}

\begin{itemize}
\tightlist
\item
  \href{https://help.nytimes3xbfgragh.onion/hc/en-us/articles/115014792127-Copyright-notice}{©~2020~The
  New York Times Company}
\end{itemize}

\begin{itemize}
\tightlist
\item
  \href{https://www.nytco.com/}{NYTCo}
\item
  \href{https://help.nytimes3xbfgragh.onion/hc/en-us/articles/115015385887-Contact-Us}{Contact
  Us}
\item
  \href{https://www.nytco.com/careers/}{Work with us}
\item
  \href{https://nytmediakit.com/}{Advertise}
\item
  \href{http://www.tbrandstudio.com/}{T Brand Studio}
\item
  \href{https://www.nytimes3xbfgragh.onion/privacy/cookie-policy\#how-do-i-manage-trackers}{Your
  Ad Choices}
\item
  \href{https://www.nytimes3xbfgragh.onion/privacy}{Privacy}
\item
  \href{https://help.nytimes3xbfgragh.onion/hc/en-us/articles/115014893428-Terms-of-service}{Terms
  of Service}
\item
  \href{https://help.nytimes3xbfgragh.onion/hc/en-us/articles/115014893968-Terms-of-sale}{Terms
  of Sale}
\item
  \href{https://spiderbites.nytimes3xbfgragh.onion}{Site Map}
\item
  \href{https://help.nytimes3xbfgragh.onion/hc/en-us}{Help}
\item
  \href{https://www.nytimes3xbfgragh.onion/subscription?campaignId=37WXW}{Subscriptions}
\end{itemize}
