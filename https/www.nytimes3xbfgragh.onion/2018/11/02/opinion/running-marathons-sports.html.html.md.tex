Sections

SEARCH

\protect\hyperlink{site-content}{Skip to
content}\protect\hyperlink{site-index}{Skip to site index}

\href{https://myaccount.nytimes3xbfgragh.onion/auth/login?response_type=cookie\&client_id=vi}{}

\href{https://www.nytimes3xbfgragh.onion/section/todayspaper}{Today's
Paper}

\href{/section/opinion}{Opinion}\textbar{}I Was Never an Athlete. Then I
Found Running.

\url{https://nyti.ms/2CVkFQO}

\begin{itemize}
\item
\item
\item
\item
\item
\end{itemize}

Advertisement

\protect\hyperlink{after-top}{Continue reading the main story}

\href{/section/opinion}{Opinion}

Supported by

\protect\hyperlink{after-sponsor}{Continue reading the main story}

Sporting

\hypertarget{i-was-never-an-athlete-then-i-found-running}{%
\section{I Was Never an Athlete. Then I Found
Running.}\label{i-was-never-an-athlete-then-i-found-running}}

The sport can be solitary. But you can also find incredible camaraderie
racing with a pack.

\href{https://www.nytimes3xbfgragh.onion/by/lela-moore}{\includegraphics{https://static01.graylady3jvrrxbe.onion/images/2019/02/14/multimedia/author-lela-moore/author-lela-moore-thumbLarge.png}}

By \href{https://www.nytimes3xbfgragh.onion/by/lela-moore}{Lela Moore}

Ms. Moore has run five marathons and 16 half marathons.

\begin{itemize}
\item
  Nov. 2, 2018
\item
  \begin{itemize}
  \item
  \item
  \item
  \item
  \item
  \end{itemize}
\end{itemize}

\includegraphics{https://static01.graylady3jvrrxbe.onion/images/2018/11/03/opinion/03sporting-top/03sporting-top-articleLarge-v2.jpg?quality=75\&auto=webp\&disable=upscale}

More than 50,000 runners are expected to take part in the New York City
Marathon this Sunday. Some of them will be elite runners chasing glory;
the rest will be looking for a personal record or a Boston Marathon
qualifying time, or just looking to finish. I won't be doing the
marathon this year, but I will be there to cheer the runners on, cowbell
in hand. And while I take in this remarkable spectacle, I will also be
looking forward to my next race --- and reflecting on how grateful I am
to have found a sport like this.

For much of my life, sports held little appeal. I spent my time in high
school editing newspaper columns and conjugating Latin verbs. The only
running I did was when we were forced to slog around our bright blue
track for a mile to complete the Presidential Fitness Test.

But next April, I will join about 20 percent of my 1994 classmates and
run a half-marathon in Nashville to raise money for our school's
scholarship fund. We've started a Facebook group to support one another
during our training. We share photos of the scenery we encounter on our
runs --- for one friend in Georgia, that's hay bales and the occasional
snake; for me, it's the urban greenery of Prospect Park in Brooklyn ---
and encourage each other through injuries and raccoon sightings.

I started running 10 years ago, in my 30s, after years of exercise fits
and starts --- cardio kickboxing! Tae Bo! Dance 'n' Sweat! --- and
struggles with eating disorders. I needed to find a workout I could do
consistently, anytime, without paying a membership fee. Only one
problem: I didn't know how to run. I'd done it only with a physical
education teacher holding a stopwatch and calling me ``grandma.''

I found a Couch to 5K program online and worked my way up to the
prescribed 3.1 miles. On my own, without the emphasis on speed or time,
I found to my amazement that I enjoyed running. A friend mentioned to me
that she was planning to do a half marathon with a charity group --- why
didn't I come to a practice with her and check it out? The next thing I
knew, I was training for a half marathon with her. One mile at a time.

I was never an athlete. I liked to swim, but floundered on land. I liked
gymnastics class, but I'm 5 feet 10 inches tall. I hated all team
sports, particularly those scourges of elementary-school recess,
kickball and dodgeball. Every mistake caused the failure of an entire
group of people, and they weren't shy about letting me know. I chose my
secondary school because it did not require going out for a team.

After I ran that first half marathon in San Francisco in 2008, I trained
for and ran a marathon. Then I ran the New York City Marathon, twice. I
ran a half marathon in less than two hours. I won my age group in a 5K.
And when I didn't succeed, when I was injured or tired or my legs
cramped, they told me it was O.K., rest, try again tomorrow or next week
or next month.

Image

The author running in the 2017 Boston marathon.Credit...Michele Drislane

In this respect, running in the charity group was a revelation. I had a
team of sorts --- people to exercise with, to talk to on grueling
15-milers and through hill repeats and
\href{https://www.runnersworld.com/advanced/a20816077/finding-fartlek/}{fartleks},
but I wasn't competing with them. I was competing only with myself. And
when I achieved a personal success, these people cheered me on. They
told me I could go even faster, even farther. They became my best
friends.

And now I've come full circle, back to the classmates who knew me as an
awkward teenager, to share in this sport that has become a lifestyle. We
are shedding the labels we hid behind as girls and coming into our own
on the road.

Most of us have young children, and it's a struggle to find the time to
fit our runs in. But as we've begun training, our kids are noticing. One
woman runs with her teenage daughter; they recently placed in their age
groups in their first 5K race. Little boys go out to pace their moms.
I've started taking my 1-year-old son out once a week in his jogging
stroller, which up to this point I'd complained was too heavy, too
bulky, too much. Now it's an essential part of my week.

I'll take him to watch the New York City Marathon on Sunday. I ran the
Boston Marathon for charity when I was pregnant with him, and my race
bib hangs above his changing table. I want him to see that his mom
exercises both because it is fun and because it is necessary, and I want
him to find a sport he enjoys as much as I do mine. I'll tell him how I
ran that thrilling 26.2-mile course through the five boroughs twice, and
yearn to do it again with him cheering me on. I'll tell him how running
keeps me healthy and happy, and how working with others to compete
against yourself is the best of both worlds.

\textbf{\href{https://timesevents.nytimes3xbfgragh.onion/nytrunning}{\emph{Run
Like a Woman}}\emph{:}} \emph{A special offer for New York Times
subscribers. On Monday, Nov. 5, you can join us for a live event in New
York City and hear from some of the top American runners in the marathon
--- Des Linden, Steph Rothstein, and Allie Kieffer --- and Jen A.
Miller, the editor of The Times's}
\href{https://www.nytimes3xbfgragh.onion/newsletters/running?module=inline}{\emph{Running}}
\emph{newsletter. Moderated by Lindsay Crouse.}
\href{https://timesevents.nytimes3xbfgragh.onion/nytrunning}{\emph{Read
more and get tickets here.}} \emph{(Use the promo code}
\emph{\textbf{NYT}} \emph{to get \$5 off the ticket price.)}

\emph{Follow The New York Times Opinion section on}
\href{https://www.facebookcorewwwi.onion/nytopinion}{\emph{Facebook}}\emph{,}
\href{http://twitter.com/NYTOpinion}{\emph{Twitter (@NYTopinion)}}
\emph{and}
\href{https://www.instagram.com/nytopinion/}{\emph{Instagram}}\emph{.}

Advertisement

\protect\hyperlink{after-bottom}{Continue reading the main story}

\hypertarget{site-index}{%
\subsection{Site Index}\label{site-index}}

\hypertarget{site-information-navigation}{%
\subsection{Site Information
Navigation}\label{site-information-navigation}}

\begin{itemize}
\tightlist
\item
  \href{https://help.nytimes3xbfgragh.onion/hc/en-us/articles/115014792127-Copyright-notice}{©~2020~The
  New York Times Company}
\end{itemize}

\begin{itemize}
\tightlist
\item
  \href{https://www.nytco.com/}{NYTCo}
\item
  \href{https://help.nytimes3xbfgragh.onion/hc/en-us/articles/115015385887-Contact-Us}{Contact
  Us}
\item
  \href{https://www.nytco.com/careers/}{Work with us}
\item
  \href{https://nytmediakit.com/}{Advertise}
\item
  \href{http://www.tbrandstudio.com/}{T Brand Studio}
\item
  \href{https://www.nytimes3xbfgragh.onion/privacy/cookie-policy\#how-do-i-manage-trackers}{Your
  Ad Choices}
\item
  \href{https://www.nytimes3xbfgragh.onion/privacy}{Privacy}
\item
  \href{https://help.nytimes3xbfgragh.onion/hc/en-us/articles/115014893428-Terms-of-service}{Terms
  of Service}
\item
  \href{https://help.nytimes3xbfgragh.onion/hc/en-us/articles/115014893968-Terms-of-sale}{Terms
  of Sale}
\item
  \href{https://spiderbites.nytimes3xbfgragh.onion}{Site Map}
\item
  \href{https://help.nytimes3xbfgragh.onion/hc/en-us}{Help}
\item
  \href{https://www.nytimes3xbfgragh.onion/subscription?campaignId=37WXW}{Subscriptions}
\end{itemize}
