At Giorgio Armani's Weekend Retreat, Live Swans and Gilded Arches

\url{https://nyti.ms/2zi3WVb}

\begin{itemize}
\item
\item
\item
\item
\item
\item
\end{itemize}

\includegraphics{https://static01.graylady3jvrrxbe.onion/images/2018/11/13/t-magazine/tktmag-armani-slide-4Q84/tktmag-armani-slide-4Q84-articleLarge.jpg?quality=75\&auto=webp\&disable=upscale}

Sections

\protect\hyperlink{site-content}{Skip to
content}\protect\hyperlink{site-index}{Skip to site index}

\hypertarget{at-giorgio-armanis-weekend-retreat-live-swans-and-gilded-arches}{%
\section{At Giorgio Armani's Weekend Retreat, Live Swans and Gilded
Arches}\label{at-giorgio-armanis-weekend-retreat-live-swans-and-gilded-arches}}

For decades, the fashion designer has been escaping to an estate outside
Milan that proves even minimalists (sometimes) like a little coziness.

Giorgio Armani leans on an antique rocking horse in front of a Giovanni
Battista Tiepolo painting in the dining room.Credit...Simon Upton

Supported by

\protect\hyperlink{after-sponsor}{Continue reading the main story}

By \href{https://www.nytimes3xbfgragh.onion/by/nancy-hass}{Nancy Hass}

\begin{itemize}
\item
  Nov. 19, 2018
\item
  \begin{itemize}
  \item
  \item
  \item
  \item
  \item
  \item
  \end{itemize}
\end{itemize}

GIORGIO ARMANI, NOW 84, spends a great deal of time shuttling between
his nine residences throughout Europe, the United States and the
Caribbean. Unsurprisingly for someone whose exacting, understated
aesthetic has profoundly shaped both fashion and the culture at large,
he likes things to be just so in each of them. One of the reasons he
keeps such a vast array of homes is that he doesn't like to stay in
hotels; he is easily upset by a sink mounted too high or an
unnecessarily oily
\href{https://cooking.nytimes3xbfgragh.onion/recipes/3846-sourdough-bruschetta}{bruschetta}.
Decorative flamboyance dismays him, as does poor execution.

In view of such fabled perfectionism, not to mention the sharp lines and
hushed neutrals that define both his clothing and home décor
collections, the weekend compound he has owned for nearly 30 years in
Broni --- an unremarkable industrial town about an hour and a half south
of
\href{https://www.nytimes3xbfgragh.onion/2015/01/11/travel/what-to-do-in-36-hours-in-milan.html}{Milan}
--- is nothing you would expect. Unlike Lake Garda or
\href{https://www.nytimes3xbfgragh.onion/2017/12/28/travel/lake-como-italy-younger-travelers.html}{Como},
the area is not known as a fashionable retreat for wealthy Milanese, but
from the moment you drive through the unmarked metal-plate gates
incongruously set on a two-lane highway, you are transported into a vast
Impressionist canvas stippled with fruit trees and fields of roses,
where sunshine diffuses into a blur of pastels.

The house that anchors the property is grand --- Armani is, after all,
on the
\href{https://www.forbes.com/profile/giorgio-armani/\#35f49c492c5b}{Forbes
list of the world's billionaires} --- but it is also pink, a color one
does not readily associate with him. The nearly 15,000-square-foot,
26-room chalky rose stucco villa that he visits at least a dozen times a
year --- far more than any of his other vacation homes --- stands on a
slight rise overlooking 25 groomed acres. Its color evokes the 1950s,
the era in which the house was built (by the Italian count Franco Cella
di Rivara, who made a fortune creating Marvis toothpaste), but its style
is late 18th century --- a supremely decorative period that the
designer, who made his name displacing 1980s kitsch with sophisticated
Italian minimalism, seems to crave in his off time. It is to this
romantically accoutered house that he repairs, in the back of his
chauffeured Bentley after a workweek in Milan. He grew up the son of an
accountant in nearby Piacenza, so the area has profound resonance for
him. ``When you are here, you want to escape from things that are dark
or serious,'' he says (in Italian --- he has never mastered English), as
he squeezes your arm gently for emphasis. ``I come on the weekend to see
the light.''

\href{https://www.nytimes3xbfgragh.onion/slideshow/2018/11/19/t-magazine/inside-giorgio-armanis-italian-villa.html}{}

\hypertarget{inside-giorgio-armanis-italian-villa}{%
\subsection{Inside Giorgio Armani's Italian
Villa}\label{inside-giorgio-armanis-italian-villa}}

10 Photos

View Slide Show ›

\includegraphics{https://static01.graylady3jvrrxbe.onion/images/2018/11/13/t-magazine/tktmag-armani-slide-7DHA/tktmag-armani-slide-7DHA-articleLarge.jpg?quality=75\&auto=webp\&disable=upscale}

Simon Upton

He also comes to surround himself with his collection of fauna, many of
which he has imported in the past few years from distant lands. (``Now I
am a zookeeper!'') Beyond the pink villa and the pond stocked with swans
and egrets lie large grassy enclosures where a menagerie of exotic
animals lazily grazes --- zebras, guanacos, alpacas, longhorn deer:
about 80 animals in total. A pair of South American parrots spread wide
wings of crimson, marigold and emerald as they nuzzle each other in a
giant outdoor cage. The designer's dogs, a pack of six that includes not
merely obscure purebreds with noble lineage but a football-size
coffee-brown mutt called Pepe, greet you enthusiastically; the
soundtrack is the soft call of the cuckoos hiding in the beech trees
that rim the property's distant borders.

Inside the main house as well, Armani's lingua franca --- white, gray,
taupe and black --- is markedly absent. In its place is a palette that
somehow in his hands is both subtle and vibrant: celadon, slate blue and
pale coral. The smooth plaster walls have a gently mottled finish that
suggests age and depth. ``I wanted this house to make one feel as though
you are part of history,'' he says.

Sunlight streams into the 70-foot-long main gallery through a series of
arched glass doors to the patio and a sweeping back staircase inlaid
with smooth pebbles. Most of the doorways on the main floor are crowned
with ornate 18th-century gilded arches, rescued by the home's former
owner from a neglected, centuries-old villa nearby. The repeating motif
lends a gravitas that belies the house's relatively recent vintage.
``These symbols of the past make you let go,'' says Armani. ``You can
breathe.''

\includegraphics{https://static01.graylady3jvrrxbe.onion/images/2018/11/13/t-magazine/tktmag-armani-slide-96NC/tktmag-armani-slide-96NC-articleLarge.jpg?quality=75\&auto=webp\&disable=upscale}

IT IS DIFFICULT to imagine the notoriously precise designer ---
seemingly always in his uniform of navy cashmere pullover, navy
trousers, white trainers and a late August tan --- taking the time for
deep breaths, much less relinquishing control. And yet, despite its
classical proportions, the house has a warmth, a relaxed coziness, that
makes you question everything you thought you knew about the man who
spent decades building an international empire.

While his Milan palazzo on Via Borgonuovo in the Brera neighborhood was
famously done up by the designer
\href{https://www.nytimes3xbfgragh.onion/2012/03/11/magazine/peter-marino-likes-playing-bad-cop.html}{Peter
Marino} in the 1980s in linear monochrome, there is no trace of an
outside decorator's statement here. Nor has Armani updated it much
through the decades. Absent, too, are sculptural contemporary furniture
pieces or antiques of the great-to-look-at-but-miserable-to-sit-in sort.
There is also little evidence of the spare, Deco-inspired furniture he
creates for his own line,
\href{https://www.armani.com/casa/us/}{Armani/Casa}. Instead, the living
areas are dominated by deep-cushioned ivory roll-armed sofas that you
instantly want to sink into for a nap, perhaps with a dog or two
snoozing beside you. The lush neutral carpets are lightly stained here
and there; there may be evidence of a canine tooth mark on a low
wood-edged coffee table. In a nook off the grand main entrance with its
mirror-polished marble floors (the doorway is flanked by a
seven-foot-tall pair of classical stone statues whimsically turned into
giant lamps) stands a foosball table.

Armani has never been much for hanging art on his walls ---
``distracting,'' he says --- and the \emph{objets} arranged on the low
Asian-influenced tables and painted étagères in the living areas are a
mix of the precious and the sentimental: a marble bust picked up in the
local flea market (``to me, it resembles Seneca''), a collection of
19th-century pharmacy bottles from Tunisia (``charmed away, for a price,
from a man who looked about 120 years old''), a stone lion from the
1700s purchased at the
\href{https://www.nytimes3xbfgragh.onion/2014/11/30/travel/at-a-paris-flea-market-tips-for-treasure-hunters.html}{Marché
aux Puces} outside Paris. On a side chest in the dining room is a
collection of silver serving pieces from Gran Caffè Doney, once his
favorite osteria in
\href{https://www.nytimes3xbfgragh.onion/2014/09/28/travel/things-to-do-in-36-hours-in-florence-italy.html}{Florence},
long defunct. A small table is arrayed, shrinelike, with sculptures of
graceful male bodies and a framed image of
\href{https://www.nytimes3xbfgragh.onion/1985/08/16/nyregion/sergio-galeotti.html}{Sergio
Galeotti}, Armani's partner in life and business, who died in 1985.
There is another framed image of Galeotti in the master bedroom; almost
no other photographs are on display.

While other designers may use an invitation to their weekend homes to
cement relationships with celebrities, Armani's guests at the villa are
never famous. Too tiring, he says, with a wave of his hand: ``And then
one tells the other and you have to have them all.'' Instead, he prefers
to be surrounded by family --- his sister and nephew once worked for the
company, and his nieces still do --- and his close group of key
employees and friends, many of them journalists. After a day strolling
the gravel paths, visiting the alpacas, a late afternoon swim in the
pool and a simple dinner by his chef, his guests will gather in one of
the living rooms. The lofty ceilinged space is made cozy by his
collection of folding screens --- Japanese antiques and one by the
revered 20th-century architect
\href{https://www.nytimes3xbfgragh.onion/2016/04/06/t-magazine/renzo-mongiardino-architect.html}{Renzo
Mongiardino} --- that soften the room's angles. At the far end sits a
pool table that
\href{https://www.nytimes3xbfgragh.onion/2008/09/28/movies/28newman.html}{Paul
Newman} played on in
``\href{https://www.nytimes3xbfgragh.onion/1986/10/17/movies/screen-paul-newman-in-the-color-of-money.html}{The
Color of Money},'' a gift from one of his nieces.

And then, once the lights are dimmed, Armani, finally far from the
demands of the sprawling principality he has ruled for nearly half a
century, settles into what could be an almost ordinary life. He pushes
up the sleeves of his fine-gauge sweater. He arranges himself just so on
the goose-down cushions of one of the ivory sofas. He flicks on the
enormous flat-screen that dominates one corner. It is time to binge late
into the night on
``\href{https://www.nytimes3xbfgragh.onion/2017/12/07/arts/television/the-crown-season-2-review.html?action=click\&contentCollection=Television\&module=RelatedCoverage\&region=EndOfArticle\&pgtype=article}{The
Crown}.'' Even the emperor sometimes needs a break.

Advertisement

\protect\hyperlink{after-bottom}{Continue reading the main story}

\hypertarget{site-index}{%
\subsection{Site Index}\label{site-index}}

\hypertarget{site-information-navigation}{%
\subsection{Site Information
Navigation}\label{site-information-navigation}}

\begin{itemize}
\tightlist
\item
  \href{https://help.nytimes3xbfgragh.onion/hc/en-us/articles/115014792127-Copyright-notice}{©~2020~The
  New York Times Company}
\end{itemize}

\begin{itemize}
\tightlist
\item
  \href{https://www.nytco.com/}{NYTCo}
\item
  \href{https://help.nytimes3xbfgragh.onion/hc/en-us/articles/115015385887-Contact-Us}{Contact
  Us}
\item
  \href{https://www.nytco.com/careers/}{Work with us}
\item
  \href{https://nytmediakit.com/}{Advertise}
\item
  \href{http://www.tbrandstudio.com/}{T Brand Studio}
\item
  \href{https://www.nytimes3xbfgragh.onion/privacy/cookie-policy\#how-do-i-manage-trackers}{Your
  Ad Choices}
\item
  \href{https://www.nytimes3xbfgragh.onion/privacy}{Privacy}
\item
  \href{https://help.nytimes3xbfgragh.onion/hc/en-us/articles/115014893428-Terms-of-service}{Terms
  of Service}
\item
  \href{https://help.nytimes3xbfgragh.onion/hc/en-us/articles/115014893968-Terms-of-sale}{Terms
  of Sale}
\item
  \href{https://spiderbites.nytimes3xbfgragh.onion}{Site Map}
\item
  \href{https://help.nytimes3xbfgragh.onion/hc/en-us}{Help}
\item
  \href{https://www.nytimes3xbfgragh.onion/subscription?campaignId=37WXW}{Subscriptions}
\end{itemize}
