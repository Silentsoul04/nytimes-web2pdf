Sections

SEARCH

\protect\hyperlink{site-content}{Skip to
content}\protect\hyperlink{site-index}{Skip to site index}

\href{https://www.nytimes3xbfgragh.onion/section/food}{Food}

\href{https://myaccount.nytimes3xbfgragh.onion/auth/login?response_type=cookie\&client_id=vi}{}

\href{https://www.nytimes3xbfgragh.onion/section/todayspaper}{Today's
Paper}

\href{/section/food}{Food}\textbar{}A Full-Throated Defense of
Traditional Indian Cooking

\url{https://nyti.ms/2zlmOmn}

\begin{itemize}
\item
\item
\item
\item
\item
\item
\end{itemize}

Advertisement

\protect\hyperlink{after-top}{Continue reading the main story}

Supported by

\protect\hyperlink{after-sponsor}{Continue reading the main story}

\href{/column/restaurant-review}{Restaurant Review}

\hypertarget{a-full-throated-defense-of-traditional-indian-cooking}{%
\section{A Full-Throated Defense of Traditional Indian
Cooking}\label{a-full-throated-defense-of-traditional-indian-cooking}}

\href{https://www.nytimes3xbfgragh.onion/slideshow/2018/11/20/dining/adda-indian-restaurant-nyc.html}{}

\hypertarget{indian-cuisine-for-people-who-love-spices}{%
\subsection{Indian Cuisine for People Who Love
Spices}\label{indian-cuisine-for-people-who-love-spices}}

10 Photos

View Slide Show ›

\includegraphics{https://static01.graylady3jvrrxbe.onion/images/2018/11/21/dining/21REST-slide-WLIU/21REST-slide-WLIU-articleLarge.jpg?quality=75\&auto=webp\&disable=upscale}

An Rong Xu for The New York Times

\begin{itemize}
\tightlist
\item
  Adda Indian Canteen\\
  **NYT Critic's Pick ★★ Indian \$\$ 31-31 Thomson Avenue 718-433-3888
\end{itemize}

\href{https://www.yelp.com/reservations/adda-long-island-city}{Reserve a
Table}

When you make a reservation at an independently reviewed restaurant
through our site, we earn an affiliate commission.

By \href{https://www.nytimes3xbfgragh.onion/by/pete-wells}{Pete Wells}

\begin{itemize}
\item
  Nov. 20, 2018
\item
  \begin{itemize}
  \item
  \item
  \item
  \item
  \item
  \item
  \end{itemize}
\end{itemize}

Something has been missing from New York's current Indian restaurant
scene, but I wasn't sure what it was until I ate the kaleji masala at
\href{https://www.addanyc.com/}{Adda Indian Canteen} in Long Island
City, Queens.

Modernization is the theme unifying many of the Indian restaurants that
have taken root around the city over the past few years. At the rarefied
end of the spectrum, we have
\href{https://www.nytimes3xbfgragh.onion/2016/05/25/dining/indian-accent-restaurant-review.html}{Indian
Accent} bringing the cuisine into line with the intricate techniques and
plating that can land restaurants on the itineraries of list-clutching
gastrotourists. At the populist end, we've seen
\href{https://www.nytimes3xbfgragh.onion/2018/03/13/dining/indian-restaurants-manhattan-babu-ji.html}{Indian
cooking gastro-pubbed} by
\href{https://www.nytimes3xbfgragh.onion/2015/09/09/dining/hungry-city-babu-ji-east-village.html}{Babu
Ji} and other new places that try to make non-desis feel at home with
house-party vibes, beer in a self-service fridge, and dishes like naan
pizzas and samosa burgers.

New York hasn't seen nearly as many recently opened restaurants devoted
to the pure, original stuff that is getting modernized. The snacks
pitched together in a blur by hawkers at folding tables and rolling
carts by the side of the road; the gravy-soaked stews and sauceless dry
curries patiently made from 25 or so vegetables and seasonings, all of
them chopped, ground, fried and simmered at home by those keepers of the
culinary flame known collectively as the aunties --- we've seen
interpretations of this food, squeezed from eyedroppers and prepared for
their photo shoots with edible flowers. But new places cooking the
genuine article have been scarce.

There is a problem of missing context here; twists on tradition don't
really resonate when tradition is hard to find in its untwisted form.
And for anyone who truly loves Indian cooking, there is a more pressing
problem of missing flavor. Five dots of tandoori-spice oil just won't
land on the taste buds with the impact of a full plate of charred and
steaming tandoori chicken, no matter how evenly spaced those dots are.

I hadn't given much thought to either problem, to tell the truth, until
I made it to the out-of-the-way patch of Long Island City where Adda
Indian Canteen operates. This is not the part of that neighborhood where
high-rise apartments grow overnight and Amazon will soon tiptoe in with
25,000 employees. Adda is east of all that, under a National Guard
recruiting office, surrounded by a thicket of railroad tracks and
highway overpasses. Unless you are looking for it, you are not likely to
end up in the small dining room, in which Indian tabloids have been
upcycled as wallpaper.

\includegraphics{https://static01.graylady3jvrrxbe.onion/images/2018/11/21/dining/21rest2/merlin_146602671_ee4b6742-bf89-4e95-91e2-bb980f29e850-articleLarge.jpg?quality=75\&auto=webp\&disable=upscale}

Once I'd found it, one of the first things I ate was the kaleji masala,
chicken livers in a dark gravy seasoned with fresh ginger and garam
masala. Right away I realized that this was the kind of Indian food the
city has been hungry for, or at least the kind that I'd been hungry for:
made with care but no pretense; seasoned for people who love the
interplay of spices; presented without apology in all its brown, lumpy
glory; and complex in ways that demand full attention.

Adda (ah DAH) Indian Canteen was opened in September by Roni Mazumdar,
who owns it, and Chintan Pandya, who is the chef. The two men already
run \href{https://www.rahinyc.com/}{Rahi}, in Greenwich Village, perhaps
the best of the casual, modernizing Indian restaurants, where Mr. Pandya
serves original notions like tandoori skate and sends almost everything
out of the kitchen decorated with edible flowers.

They've taken another angle at Adda, mining family recipes and the
lessons Mr. Pandya absorbed from civilian cooks around India. The menu
travels beyond the standard north Indian dishes most New Yorkers already
know, but even those dishes, when they appear, are prepared so
emphatically that they don't resemble anything else in town. Adda is a
lusty, full-throated defense of traditional cooking.

Saag paneer, the grayest of old gray mares in many Indian restaurants,
is coltishly energetic at Adda. The cheese is made in the kitchen, and
it's gorgeously soft, while the saag is a tart and peppery mix of wilted
arugula, sorrel, spinach and mustard greens.

Mr. Pandya does very well by the class of sidewalk treats called chaat.
Each delivers the messy riot you'd hope for, busy with crunchy
hieroglyphs of fried noodles and splattered with yogurt and tamarind
sauce. One chaat is anchored by a big kale fritter, another by several
smaller chips of battered, spiced lentil cakes, and a third by fried
smashed potatoes and chickpeas. But discerning chaat lovers should
probably go directly to the dahi batata puri, soft potato packed into
hollow globes of fried dough that you can pick up and eat with your
fingers, like chocolates.

Anything that passes through Adda's tandoor is worth investigating.
Seekh kebabs, made with lamb that's coarsely ground by hand, come out of
the tandoor juicier and pinker than the usual; Mumbai-style tandoori
macchi, a skewered pompano rubbed with ground mustard seeds and
cilantro, is lightly charred and smoky after roasting, but still moist;
bhatti da murgh, a double-marinated chicken thigh and drumstick, is so
thickly crusted with coriander and cumin that it crunches when you bite
it.

Image

Adda serves home-style cooking from around India, like the Lucknow
biryani at center.Credit...An Rong Xu for The New York Times

My server expressed an enthusiasm about the poussin --- plunged into the
tandoor in one piece after marinating in vinegar, yogurt, fresh red
chiles and kala namak, the sulfurous black salt of India --- that you
typically find in new converts to a cult. After tasting it, I was ready
to join.

He was enthusiastic about almost everything, as it turned out; the
service at Adda is nothing if not eager to talk up the food. (Getting
your water glass refilled is another story.) Placing a dish of yogurt
and pomegranate seeds next to the goat biryani that is steamed under a
lid of dough, he said, ``This is going to be your best friend when the
spice in the biryani starts to hit you.'' It was, in fact, one of the
fiercest biryanis I've ever met, and the yogurt tamed it so that I could
taste the tender goat meat and crisp, bittersweet fried onions.

Few things on the menu are quite as chile-drenched as the biryani, but
Mr. Pandya definitely favors India's more intense flavors. Adda is not
big on soft-spoken cream sauces; the coconut milk in the Malvani prawn
curry from the South Konkan coast, for instance, has a pronounced kick.
(It's very good if you get it on a night when the prawns are firm and
not spongy.) And while Adda does serve some vegetables, it would not
make my list of the 20 best local Indian restaurants for vegetarians.

It would, however, be the place I'd send you if you want to know how
good Delhi butter chicken can be. Or if you don't yet believe that rara
gosht, chunks of lamb stewed on the bone with spiced minced lamb, is not
redundant but luxurious.

And Adda is the first restaurant I'd tell you about if you woke up one
morning hungry for bheja fry, goat brains cooked with onions, ginger and
a considerable number of fresh green chiles. ``Like soft scrambled
eggs,'' our server said, accurately. I might point out that a typical
Indian bheja fry is drier, with a higher ratio of brains to sauce. But I
would not necessarily mean that as a complaint.

\emph{Follow} \emph{\href{https://twitter.com/nytfood}{NYT Food on
Twitter}} \emph{and}
\emph{\href{https://www.instagram.com/nytcooking/}{NYT Cooking on
Instagram},}
\emph{\href{https://www.facebookcorewwwi.onion/nytcooking/}{Facebook}}
\emph{and}
\emph{\href{https://www.pinterest.com/nytcooking/}{Pinterest}.}
\emph{\href{https://www.nytimes3xbfgragh.onion/newsletters/cooking}{Get
regular updates from NYT Cooking, with recipe suggestions, cooking tips
and shopping advice}.}

Advertisement

\protect\hyperlink{after-bottom}{Continue reading the main story}

\hypertarget{site-index}{%
\subsection{Site Index}\label{site-index}}

\hypertarget{site-information-navigation}{%
\subsection{Site Information
Navigation}\label{site-information-navigation}}

\begin{itemize}
\tightlist
\item
  \href{https://help.nytimes3xbfgragh.onion/hc/en-us/articles/115014792127-Copyright-notice}{©~2020~The
  New York Times Company}
\end{itemize}

\begin{itemize}
\tightlist
\item
  \href{https://www.nytco.com/}{NYTCo}
\item
  \href{https://help.nytimes3xbfgragh.onion/hc/en-us/articles/115015385887-Contact-Us}{Contact
  Us}
\item
  \href{https://www.nytco.com/careers/}{Work with us}
\item
  \href{https://nytmediakit.com/}{Advertise}
\item
  \href{http://www.tbrandstudio.com/}{T Brand Studio}
\item
  \href{https://www.nytimes3xbfgragh.onion/privacy/cookie-policy\#how-do-i-manage-trackers}{Your
  Ad Choices}
\item
  \href{https://www.nytimes3xbfgragh.onion/privacy}{Privacy}
\item
  \href{https://help.nytimes3xbfgragh.onion/hc/en-us/articles/115014893428-Terms-of-service}{Terms
  of Service}
\item
  \href{https://help.nytimes3xbfgragh.onion/hc/en-us/articles/115014893968-Terms-of-sale}{Terms
  of Sale}
\item
  \href{https://spiderbites.nytimes3xbfgragh.onion}{Site Map}
\item
  \href{https://help.nytimes3xbfgragh.onion/hc/en-us}{Help}
\item
  \href{https://www.nytimes3xbfgragh.onion/subscription?campaignId=37WXW}{Subscriptions}
\end{itemize}
