Sections

SEARCH

\protect\hyperlink{site-content}{Skip to
content}\protect\hyperlink{site-index}{Skip to site index}

\href{https://www.nytimes3xbfgragh.onion/section/food}{Food}

\href{https://myaccount.nytimes3xbfgragh.onion/auth/login?response_type=cookie\&client_id=vi}{}

\href{https://www.nytimes3xbfgragh.onion/section/todayspaper}{Today's
Paper}

\href{/section/food}{Food}\textbar{}Seafood From Two Tapas Masters at
Saint Julivert Fisherie

\url{https://nyti.ms/2DOOvXX}

\begin{itemize}
\item
\item
\item
\item
\item
\item
\end{itemize}

Advertisement

\protect\hyperlink{after-top}{Continue reading the main story}

Supported by

\protect\hyperlink{after-sponsor}{Continue reading the main story}

\href{/column/restaurant-review}{Restaurant Review}

\hypertarget{seafood-from-two-tapas-masters-at-saint-julivert-fisherie}{%
\section{Seafood From Two Tapas Masters at Saint Julivert
Fisherie}\label{seafood-from-two-tapas-masters-at-saint-julivert-fisherie}}

\href{https://www.nytimes3xbfgragh.onion/slideshow/2018/11/27/dining/saint-julivert-fisherie-brooklyn.html}{}

\hypertarget{a-harbor-for-fish-lovers-in-brooklyn}{%
\subsection{A Harbor for Fish Lovers in
Brooklyn}\label{a-harbor-for-fish-lovers-in-brooklyn}}

10 Photos

View Slide Show ›

\includegraphics{https://static01.graylady3jvrrxbe.onion/images/2018/11/28/dining/28REST-slide-IMOI/28REST-slide-IMOI-articleLarge.jpg?quality=75\&auto=webp\&disable=upscale}

Daniel Krieger for The New York Times

\begin{itemize}
\tightlist
\item
  Saint Julivert Fisherie\\
  ★ Seafood \$\$ 264 Clinton Street 347-987-3710
\end{itemize}

\href{https://www.opentable.com/r/saint-julivert-reservations-brooklyn?ref=4201\&rid=1028269}{Reserve
a Table}

When you make a reservation at an independently reviewed restaurant
through our site, we earn an affiliate commission.

By \href{https://www.nytimes3xbfgragh.onion/by/pete-wells}{Pete Wells}

\begin{itemize}
\item
  Nov. 27, 2018
\item
  \begin{itemize}
  \item
  \item
  \item
  \item
  \item
  \item
  \end{itemize}
\end{itemize}

Alex Raij and Eder Montero met while working in the kitchen of a
sprawling, impersonal, gloomy modern-Spanish restaurant that lasted
about two years. They went on to get married and, as joint chefs and
owners, opened a string of compact, intimate, slinky modern-Spanish
restaurants that are still in business.

Their one misfire was a small, intimate non-Spanish coffee shop in
Cobble Hill, Brooklyn. After dusting themselves off, they replaced the
cafe tables with taller and longer ones, traded the pastry cases for a
bar, installed a stripe of backlighted glass blocks that cast a
subaqueous glow on the room, and in September reopened as
\href{https://www.saintjulivertbk.com/}{Saint Julivert Fisherie}.

And what is a fisherie, you ask, having quickly consulted your French,
Spanish and English dictionaries and found no such word? Saint Julivert
is my first, but if it is anything to go by, then a fisherie is a
seafood establishment that aspires to be more than a raw bar but does
not want to be mistaken for a full-bore restaurant. Wines come mostly
from coastal regions (and are organized by the nearest body of salt
water). Small plates abound. And if you guessed that they are something
like the tapas that Ms. Raij and Mr. Montero explore at
\href{https://www.nytimes3xbfgragh.onion/2014/05/14/dining/restaurant-review-el-quinto-pino-in-chelsea.html}{El
Quinto Pino},
\href{https://www.nytimes3xbfgragh.onion/2009/04/22/dining/reviews/22rest.html}{Txikito}
and
\href{https://www.nytimes3xbfgragh.onion/2012/08/22/dining/reviews/la-vara-in-cobble-hill-brooklyn-restaurant-review.html}{La
Vara}, but without the running Spanish theme, you are not far off.

Not that the menu is entirely un-Spanish. The octopus carpaccio from
Txikito makes a special guest appearance at Saint Julivert Fisherie, the
way the Fresh Prince
\href{https://www.youtube.com/watch?v=ekveD9J4vAs}{once turned up} on an
episode of ``Blossom.'' Under marjoram leaves and pinprick grains of
Espelette pepper, the carpaccio is as good as you remember; the warm
plate halfway melts the fattier bits of octopus into delicious goo.

\includegraphics{https://static01.graylady3jvrrxbe.onion/images/2018/11/28/dining/28rest1/merlin_147073617_f5669589-bb01-4acd-a643-4e0f0af92dde-articleLarge.jpg?quality=75\&auto=webp\&disable=upscale}

There is another cameo: El Quinto Pino's very fine combination of
pink-fleshed anchovies with cold vanilla butter. The butter soothes the
anchovies the way the vermouth in a martini helps the gin settle down.
(On a related topic, the short cocktail list includes a deceptively
soft-spoken number called the Kanpai Martini. It's a variation on the
Vesper, which itself is the amended martini that Ian Fleming had James
Bond order in ``Casino Royale.'')

At times Saint Julivert reminds me of
\href{https://www.calpep.com/Ingles/index_ing.html}{Cal Pep}, in
Barcelona, Spain, whose plain, cramped counter looks like the last place
in the world where you are going to have an epiphany of the taste buds,
right up until the moment the txipirones and thumbtack-size clams knock
you off your barstool. Like Cal Pep, and unlike nearly every other New
York restaurant, Saint Julivert serves gooseneck barnacles when they're
available, boiling them with bay leaves and salt and mounding them on a
cloth napkin before they cool so that when you twist the wrinkled sheath
of skin away from the sweet, edible meat inside there is a good chance
somebody in the vicinity will get hit with a squirt of hot barnacle
juice.

Mostly, though, any Spanish-ness in Saint Julivert has more to do with
its spirit than its recipes, which come from all over: the deep-fried
Puerto Rican cylinders of cornmeal known as sorrullitos; a kanpachi
collar whose juicy meat is shockingly white under a black rub of jerk
spices; tender slices of warm beef tenderloin on a fist-size roll, known
in Portugal as a prego sandwich.

There is also the crispy tuna bake, a peculiarly satisfying hybrid that
crosses a tuna noodle casserole by way of India with fideuà, the Spanish
dish that treats pasta like the rice in paella. The noodles, short
ridged tubes, are toasted and cooked with oil-cured tuna in a tomato
sauce that is seasoned with turmeric and curry leaves. It is one of the
most fearless, not to mention filling, dishes on a menu that could use a
little more of both qualities.

Although Saint Julivert works with exceptional seafood, and treats it
with the care that fans of Ms. Raij and Mr. Montero have come to expect,
the menu plays it safe more often than it should. Like all of the
couple's restaurants, Saint Julivert toggles between innovation and
tradition. But the other places explore Spanish cuisine, so when they
toss off a straight-ahead classic like patatas bravas, there's a reason.
When Saint Julivert serves fluke ceviche or a crab and avocado salad
dressed with yuzu juice and trout roe, it can seem to be grasping at
ideas that other restaurants are already doing --- even when Saint
Julivert does them better.

Image

Alex Raij and Eder Montero opened their latest restaurant on a quiet
corner in Cobble Hill, Brooklyn.Credit...Daniel Krieger for The New York
Times

The sautéed skate splashed with Manzanilla could not be fresher or more
skillfully browned, but do the garlic chips and sliced dried chiles on
top make this version of a well-known standard one I'll remember a month
from now? Maybe not, although at the time I was very happy to have it,
not least because it was one of the few things on the menu the size of a
standard main course.

Even if you're an old hand at navigating the shallow waters of small
plates, it can be unnervingly easy to spend more than \$100 on food and
drinks at Saint Julivert and still wonder whether you've actually had
dinner. Seafood this good is expensive, but the menu doesn't stray very
far above \$20; the result is that portions can seem somewhat airy.

When there's an inexpensive opportunity to pad out a dish, Saint
Julivert doesn't always take it. I understand why I can't have more than
three excellent wild shrimp in a \$21 bowl of red pozole, but couldn't
there be more than a spoonful of hominy kernels? Saltines fried in
canola oil are served with the ceviche and with a cold and highly
appealing bowl of pickled shrimp under pink pickled onions; grilled
bread slices accompany the mackerel whipped into a hummus-like orange
spread with piri-piri oil. Still, each time I went, somebody at my table
asked for more bread.

None of Saint Julivert's issues are insoluble. The lighting could be
warmed up so the room looks less like an aquarium. A few more dishes you
won't see anywhere else, like the tuna bake, would make it harder to
resist. Another two or three large plates would make it a place you
think of when you're hungry.

The excellent \$19 prego sandwich is more snack than dinner, until you
pay another \$7 to have fried oysters tucked into the roll beside the
beef, making it into a kind of crunchy carpetbagger. On the other hand,
that's worth doing even if you're not hungry.

\emph{Follow} \emph{\href{https://twitter.com/nytfood}{NYT Food on
Twitter}} \emph{and}
\emph{\href{https://www.instagram.com/nytcooking/}{NYT Cooking on
Instagram},}
\emph{\href{https://www.facebookcorewwwi.onion/nytcooking/}{Facebook}}
\emph{and}
\emph{\href{https://www.pinterest.com/nytcooking/}{Pinterest}.}
\emph{\href{https://www.nytimes3xbfgragh.onion/newsletters/cooking}{Get
regular updates from NYT Cooking, with recipe suggestions, cooking tips
and shopping advice}.}

Advertisement

\protect\hyperlink{after-bottom}{Continue reading the main story}

\hypertarget{site-index}{%
\subsection{Site Index}\label{site-index}}

\hypertarget{site-information-navigation}{%
\subsection{Site Information
Navigation}\label{site-information-navigation}}

\begin{itemize}
\tightlist
\item
  \href{https://help.nytimes3xbfgragh.onion/hc/en-us/articles/115014792127-Copyright-notice}{©~2020~The
  New York Times Company}
\end{itemize}

\begin{itemize}
\tightlist
\item
  \href{https://www.nytco.com/}{NYTCo}
\item
  \href{https://help.nytimes3xbfgragh.onion/hc/en-us/articles/115015385887-Contact-Us}{Contact
  Us}
\item
  \href{https://www.nytco.com/careers/}{Work with us}
\item
  \href{https://nytmediakit.com/}{Advertise}
\item
  \href{http://www.tbrandstudio.com/}{T Brand Studio}
\item
  \href{https://www.nytimes3xbfgragh.onion/privacy/cookie-policy\#how-do-i-manage-trackers}{Your
  Ad Choices}
\item
  \href{https://www.nytimes3xbfgragh.onion/privacy}{Privacy}
\item
  \href{https://help.nytimes3xbfgragh.onion/hc/en-us/articles/115014893428-Terms-of-service}{Terms
  of Service}
\item
  \href{https://help.nytimes3xbfgragh.onion/hc/en-us/articles/115014893968-Terms-of-sale}{Terms
  of Sale}
\item
  \href{https://spiderbites.nytimes3xbfgragh.onion}{Site Map}
\item
  \href{https://help.nytimes3xbfgragh.onion/hc/en-us}{Help}
\item
  \href{https://www.nytimes3xbfgragh.onion/subscription?campaignId=37WXW}{Subscriptions}
\end{itemize}
