Sections

SEARCH

\protect\hyperlink{site-content}{Skip to
content}\protect\hyperlink{site-index}{Skip to site index}

\href{https://myaccount.nytimes3xbfgragh.onion/auth/login?response_type=cookie\&client_id=vi}{}

\href{https://www.nytimes3xbfgragh.onion/section/todayspaper}{Today's
Paper}

\href{/section/opinion}{Opinion}\textbar{}The Tao of Gravity

\url{https://nyti.ms/2G7vu5M}

\begin{itemize}
\item
\item
\item
\item
\item
\end{itemize}

Advertisement

\protect\hyperlink{after-top}{Continue reading the main story}

\href{/section/opinion}{Opinion}

Supported by

\protect\hyperlink{after-sponsor}{Continue reading the main story}

Sporting

\hypertarget{the-tao-of-gravity}{%
\section{The Tao of Gravity}\label{the-tao-of-gravity}}

All martial arts are a quest for beauty and transcendence. I found those
not in doing karate but rather having karate done to me.

By Gare Joyce

Mr. Joyce is a features editor for Sportsnet.

\begin{itemize}
\item
  Dec. 8, 2018
\item
  \begin{itemize}
  \item
  \item
  \item
  \item
  \item
  \end{itemize}
\end{itemize}

\includegraphics{https://static01.graylady3jvrrxbe.onion/images/2018/12/08/opinion/08sporting/08sporting-articleLarge.jpg?quality=75\&auto=webp\&disable=upscale}

Two hundred men and women knelt in rapt and reverential silence. All of
them wore crisp, white karate outfits, or gis, cinched with black belts.
Hardened karate-ka who had trained two decades and earned third- and
fourth-dan black belts took honored places in the front row. I was in
the back by the fire exit. We were gathered in a university gym in
Toronto for an international tournament. I was in my late 30s at the
time. With my still new first dan, I felt like a humble 10-handicapper
in the company of Tiger Woods.

At the front of the hall stood the shihan, a master instructor with a
seventh dan. He was demonstrating leg sweeps, techniques associated with
judo more than karate. In Japan, he had been a university judo champion.

At this point, the shihan passed over the champions up front and
summoned me from the ranks. Karate etiquette demands stoicism, but the
skepticism in the ranks was not entirely disguised.

``Kame,'' he said. Kame was my far-from-fearsome handle in the dojo.
English translation: Turtle. This wasn't intended to evoke Gamera, the
monster turtle who fought Godzilla in Japanese horror films. And all of
this predated the Teenage Mutant Ninja Turtles. No, Turtle had been my
boyhood nickname because I had a pet tortoise.

The tag proved apropos for an entirely different reason, a gift that
revealed itself in the dojo: I could fall. Yes, my best asset was a
durable back.

I know what you're thinking: Anyone can fall, it's just gravity. But
you're wrong. Breaking a fall is as complex as any offensive technique.
You're utterly exposed and in immediate danger of real physical harm.
Your opponent has weaponized the ultimate blunt force object: the floor.
In this instance with the shihan, a hardwood floor.

I spent many hours practicing break-falls, some from great heights. I
did judo in grade school. In prep school, I routinely recreated Dick Van
Dyke's \href{https://www.youtube.com/watch?v=BttXQJ_gDnc}{tumble over
the footstool} in the intro to his old TV show, a letter-perfect judo
break-fall. I could have paid my way through school working as an
adolescent stuntman.

I was 200 pounds back then and my break-falls produced thunderclaps,
like sound effects laid over fight scenes in ``Enter the Dragon.''

In hockey circles, they'll say that a guy on the wrong end of a
one-sided fight has been ``rag-dolled.'' In karate circles, they say
nothing. Instead, they kneel and study the physics of bodies in motion.

All martial arts are a quest for beauty and transcendence. I found those
not in doing karate but rather having karate done to me. I was honored
to be tossed like a bag of wet cement by the shihan.

There was no looking down for a soft spot to fall, nor for the leg that
would undercut my own. I had to properly simulate the fighting
condition, willing myself unaware of my fate so that I didn't
reflexively start falling until I was actually being felled.

On my descent, I locked eyes with the shihan and grabbed a fistful of
the sleeve of his gi. I had let him and gravity do their business and
landed in position to counter. He might have been the only one in the
room who recognized this, but no matter --- it wasn't about me, and I
wasn't brought up to compete.

The shihan let go of me. I sprang to my feet and assumed a fighting
position. Once again I was thrown to the floor. And again. And 20 times
more. Each time I broke my fall as if out of an ancient textbook, none
the worse for wear.

In the material world the martial arts are often described and even
advertised as a means of self-defense. You sincerely hope you never have
to use your martial art outside a dojo. And you definitely hope that you
never have to perform a break-fall in any situation. That said, my
ability to fall spared me injury and possibly saved my life in the
workplace.

My job as a sportswriter has often landed me in strange circumstances,
but none stranger than my trip in 1991 to Calgary, Alberta, to write
about \href{https://www.brethart.com/}{Bret Hart}, a big dog in the
World Wrestling Federation. This led to a fateful encounter with Bret's
father, Stu, who had retired as the proprietor of Stampede Wrestling,
leaders in the mayhem industry in Western Canada.

Stu Hart began his ring career in the 1940s and threw one of his last
elbow smashes in apparent anger on an early-1990s pay-per-view show,
knocking out Bret's rival Shawn Michaels. Some doubted the authenticity
of that blow: Could a septuagenarian really ice a 240-pound champion in
his prime? I, too, considered it far-fetched, but only until I wound up
in the same position with Hart as I had with the shihan. That position,
as ever, was supine.

I was interviewing Hart and minding my manners when he asked me about a
splint on the middle finger of my left hand. To my instant regret I told
him my finger had been dislocated blocking a roundhouse kick in the
dojo. This prompted what old-school wrestlers called a ``snatching,'' an
act of bodily appropriation that I was powerless to fend off while
trying to take notes.

``I could shoot an angle,'' he said. ``You'd be the wrestling
reporter.'' Before I could beg off this narrative, I was in fact a
wrestling reporter, or at least a reporter being wrestled. Hart lifted
me to shoulder height and body-slammed me onto his dining-room floor.

Chin in, arms extended, hitting the floor with open hands: check, check
and check. I took inventory: I was in one piece and breathing, but the
latter seemed only temporary as 270 pounds of wrestling history landed
on me. Reverting to his days in the ring, Hart started to choke me out.

Thankfully his wife, Helen, happened on the scene. Rather than counting
me out, she offered profuse apologies and upbraided her husband for
snatching yet another guest. I thanked her for the well-timed
intervention but told her no apology was necessary.

If you want to start to understand a fighting art, you have to be
willing to go to the mat. The warrior might be as sacred as Shihan or as
slapstick as Stu, but regardless best viewed from the ground up.

Gare Joyce, a features editor for Sportsnet, is the author of ``The
Code,'' a mystery novel that was adapted for the television series
``Private Eyes.''

\emph{Follow The New York Times Opinion section on}
\href{https://www.facebookcorewwwi.onion/nytopinion}{\emph{Facebook}}\emph{,}
\href{http://twitter.com/NYTOpinion}{\emph{Twitter (@NYTopinion)}}
\emph{and}
\href{https://www.instagram.com/nytopinion/}{\emph{Instagram}}\emph{.}

Advertisement

\protect\hyperlink{after-bottom}{Continue reading the main story}

\hypertarget{site-index}{%
\subsection{Site Index}\label{site-index}}

\hypertarget{site-information-navigation}{%
\subsection{Site Information
Navigation}\label{site-information-navigation}}

\begin{itemize}
\tightlist
\item
  \href{https://help.nytimes3xbfgragh.onion/hc/en-us/articles/115014792127-Copyright-notice}{©~2020~The
  New York Times Company}
\end{itemize}

\begin{itemize}
\tightlist
\item
  \href{https://www.nytco.com/}{NYTCo}
\item
  \href{https://help.nytimes3xbfgragh.onion/hc/en-us/articles/115015385887-Contact-Us}{Contact
  Us}
\item
  \href{https://www.nytco.com/careers/}{Work with us}
\item
  \href{https://nytmediakit.com/}{Advertise}
\item
  \href{http://www.tbrandstudio.com/}{T Brand Studio}
\item
  \href{https://www.nytimes3xbfgragh.onion/privacy/cookie-policy\#how-do-i-manage-trackers}{Your
  Ad Choices}
\item
  \href{https://www.nytimes3xbfgragh.onion/privacy}{Privacy}
\item
  \href{https://help.nytimes3xbfgragh.onion/hc/en-us/articles/115014893428-Terms-of-service}{Terms
  of Service}
\item
  \href{https://help.nytimes3xbfgragh.onion/hc/en-us/articles/115014893968-Terms-of-sale}{Terms
  of Sale}
\item
  \href{https://spiderbites.nytimes3xbfgragh.onion}{Site Map}
\item
  \href{https://help.nytimes3xbfgragh.onion/hc/en-us}{Help}
\item
  \href{https://www.nytimes3xbfgragh.onion/subscription?campaignId=37WXW}{Subscriptions}
\end{itemize}
