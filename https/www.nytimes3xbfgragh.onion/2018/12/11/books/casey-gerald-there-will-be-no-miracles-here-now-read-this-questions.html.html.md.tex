Sections

SEARCH

\protect\hyperlink{site-content}{Skip to
content}\protect\hyperlink{site-index}{Skip to site index}

\href{https://www.nytimes3xbfgragh.onion/section/books}{Books}

\href{https://myaccount.nytimes3xbfgragh.onion/auth/login?response_type=cookie\&client_id=vi}{}

\href{https://www.nytimes3xbfgragh.onion/section/todayspaper}{Today's
Paper}

\href{/section/books}{Books}\textbar{}Discussion Questions for `There
Will Be No Miracles Here'

\url{https://nyti.ms/2zWtyY2}

\begin{itemize}
\item
\item
\item
\item
\item
\end{itemize}

Advertisement

\protect\hyperlink{after-top}{Continue reading the main story}

Supported by

\protect\hyperlink{after-sponsor}{Continue reading the main story}

Now Read This

\hypertarget{discussion-questions-for-there-will-be-no-miracles-here}{%
\section{Discussion Questions for `There Will Be No Miracles
Here'}\label{discussion-questions-for-there-will-be-no-miracles-here}}

\includegraphics{https://static01.graylady3jvrrxbe.onion/images/2018/12/11/books/11CASEY-QUESTIONS-IMAGE/11CASEY-QUESTIONS-IMAGE-articleLarge.jpg?quality=75\&auto=webp\&disable=upscale}

Dec. 11, 2018

\begin{itemize}
\item
\item
\item
\item
\item
\end{itemize}

Our December pick for the PBS NewsHour-New York Times book club, ``Now
Read This,'' is
\href{https://www.nytimes3xbfgragh.onion/2018/10/24/books/review/casey-gerald-memoir-there-will-be-no-miracles-here.html?action=click\&contentCollection=books\&region=rank\&module=package\&version=highlights\&contentPlacement=2\&pgtype=collection}{Casey
Gerald's ``There Will Be No Miracles Here,''} a memoir about attaining
--- and then questioning --- the American dream. Become a member of the
book club by joining our
\href{https://www.facebookcorewwwi.onion/groups/NowReadThisBookClub}{Facebook
group}, or by signing up to our
\href{https://pbs.us1.list-manage.com/subscribe?u=8aa1c620fd96b27384151c36e\&id=2fe6581b35}{newsletter}.
Learn more about the book club
\href{https://www.pbs.org/newshour/arts/what-is-now-read-this}{here}.

Below are questions to help guide your discussions as you read the book
over the next month. You can also submit your own questions for Gerald
on our
\href{https://www.facebookcorewwwi.onion/groups/NowReadThisBookClub}{Facebook
page}, which he will answer on the NewsHour broadcast at the end of the
month. Spoiler alert on questions further down.

1. Where is the title, ``There Will Be No Miracles Here,'' drawn from,
and why?

2. Early on in the book, Gerald writes that in our society, people who
are addicted to fame and money are perceived as just ``a little lower
than the angels'' while people addicted to drugs are ``down below water
bugs.'' How does he make clear early on that this won't be a typical
rags-to-riches memoir?

3. Gerald's religion plays a big role in this book. At one point, he
argues that in today's world many people believe themselves too
brilliant or too secular to believe in God, which he attributes to a
kind of ``anti-hope.'' What does he mean by anti-hope? What do you think
of his argument?

4. Following in the footsteps of his father, Gerald tries out for the
high school football team, makes it to varsity, and then sees intimately
the injuries and sacrifice the game requires. He writes that it's no
wonder that football is America's favorite pastime, and that he learned
``how far you can make it in America if you have enough disregard for
your personal welfare.'' Do you agree with this statement?

5. Gerald describes growing up in the blighted South Oak Cliff
neighborhood of Dallas, in a childhood characterized by absent parents,
drug use and disability checks. But while many believe that people are
defined by their circumstances, he says we are actually defined by
running from them. How is this true for him? Is it true for you?

6. ``There Will Be No Miracles Here'' begins, as did our earlier book
club pick
\href{https://www.nytimes3xbfgragh.onion/2018/03/01/books/review/tara-westover-educated.html}{``Educated,''}
at the supposed end of the world at the turn of the millennium. On its
face, the two memoirs share certain similarities, following the
narrators from a difficult childhood to an Ivy League school. But sharp
differences quickly become clear. How do these two books diverge?

7. What did you make of Gerald's relationship with River? Why do you
think it didn't --- or couldn't --- work out?

8. Gerald writes: ``You never know what you need to know when you need
to know it.'' Do you agree? What can you do with instruction after the
fact?

9. When Gerald leaves his Dallas youth behind to go play football at
Yale, he writes that he quickly learned that the word ``network'' is the
defining word of today. Relatedly, he writes that the real American
dream is not about finding success through working hard, but about
knowing the right people. Do you agree or disagree with this statement?

10. When Gerald describes forming the Yale Black Men's Union, he also
writes that ``every grand purpose grows from personal pain,'' and that
meaning and ideals often follow. Have you observed this to be true in
your own life? Which ``purposes'' of your own have begun with personal
pain?

11. When people tell Gerald as an adult that he is the embodiment of the
American dream, he smiles and says ``thank you.'' But he writes that he
should have cried. Why?

12. Of all the critiques Gerald makes of the American dream in ``There
Will Be No Miracles Here,'' which do you find most important? How does
this book particularly resonate in today's political, social and
economic realities?

13. At the end of the book, what does Gerald learn, both from the death
of his friend Elijah, and his last visit with River? What do you make of
the book's closing lines?

Advertisement

\protect\hyperlink{after-bottom}{Continue reading the main story}

\hypertarget{site-index}{%
\subsection{Site Index}\label{site-index}}

\hypertarget{site-information-navigation}{%
\subsection{Site Information
Navigation}\label{site-information-navigation}}

\begin{itemize}
\tightlist
\item
  \href{https://help.nytimes3xbfgragh.onion/hc/en-us/articles/115014792127-Copyright-notice}{©~2020~The
  New York Times Company}
\end{itemize}

\begin{itemize}
\tightlist
\item
  \href{https://www.nytco.com/}{NYTCo}
\item
  \href{https://help.nytimes3xbfgragh.onion/hc/en-us/articles/115015385887-Contact-Us}{Contact
  Us}
\item
  \href{https://www.nytco.com/careers/}{Work with us}
\item
  \href{https://nytmediakit.com/}{Advertise}
\item
  \href{http://www.tbrandstudio.com/}{T Brand Studio}
\item
  \href{https://www.nytimes3xbfgragh.onion/privacy/cookie-policy\#how-do-i-manage-trackers}{Your
  Ad Choices}
\item
  \href{https://www.nytimes3xbfgragh.onion/privacy}{Privacy}
\item
  \href{https://help.nytimes3xbfgragh.onion/hc/en-us/articles/115014893428-Terms-of-service}{Terms
  of Service}
\item
  \href{https://help.nytimes3xbfgragh.onion/hc/en-us/articles/115014893968-Terms-of-sale}{Terms
  of Sale}
\item
  \href{https://spiderbites.nytimes3xbfgragh.onion}{Site Map}
\item
  \href{https://help.nytimes3xbfgragh.onion/hc/en-us}{Help}
\item
  \href{https://www.nytimes3xbfgragh.onion/subscription?campaignId=37WXW}{Subscriptions}
\end{itemize}
