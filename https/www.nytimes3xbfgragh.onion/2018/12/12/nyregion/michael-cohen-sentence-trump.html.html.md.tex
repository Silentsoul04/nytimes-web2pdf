Sections

SEARCH

\protect\hyperlink{site-content}{Skip to
content}\protect\hyperlink{site-index}{Skip to site index}

\href{https://www.nytimes3xbfgragh.onion/section/nyregion}{New York}

\href{https://myaccount.nytimes3xbfgragh.onion/auth/login?response_type=cookie\&client_id=vi}{}

\href{https://www.nytimes3xbfgragh.onion/section/todayspaper}{Today's
Paper}

\href{/section/nyregion}{New York}\textbar{}Michael Cohen Sentenced to 3
Years After Implicating Trump in Hush-Money Scandal

\url{https://nyti.ms/2zUVKdP}

\begin{itemize}
\item
\item
\item
\item
\item
\item
\end{itemize}

Advertisement

\protect\hyperlink{after-top}{Continue reading the main story}

Supported by

\protect\hyperlink{after-sponsor}{Continue reading the main story}

\hypertarget{michael-cohen-sentenced-to-3-years-after-implicating-trump-in-hush-money-scandal}{%
\section{Michael Cohen Sentenced to 3 Years After Implicating Trump in
Hush-Money
Scandal}\label{michael-cohen-sentenced-to-3-years-after-implicating-trump-in-hush-money-scandal}}

\includegraphics{https://static01.graylady3jvrrxbe.onion/images/2018/12/13/nyregion/13cohen-promo2/13cohen-promo2-videoSixteenByNine3000-v4.jpg}

By \href{https://www.nytimes3xbfgragh.onion/by/benjamin-weiser}{Benjamin
Weiser} and
\href{https://www.nytimes3xbfgragh.onion/by/william-k-rashbaum}{William
K. Rashbaum}

\begin{itemize}
\item
  Dec. 12, 2018
\item
  \begin{itemize}
  \item
  \item
  \item
  \item
  \item
  \item
  \end{itemize}
\end{itemize}

\href{https://www.nytimes3xbfgragh.onion/2020/07/21/nyregion/michael-cohen-trump-book.html}{Michael
D. Cohen}, a former lawyer for President Trump, was sentenced to three
years in prison on Wednesday after denouncing Mr. Trump and explaining
that ``I felt it was my duty to cover up his dirty deeds.''

Mr. Cohen gave an emotional apology to the court for his involvement in
a hush-money scandal that could threaten the Trump presidency --- a
scheme to buy the silence of two women who said they had affairs with
Mr. Trump to protect his chances before the 2016 election. Mr. Cohen
said his blind loyalty to Mr. Trump led him to ignore ``my own inner
voice and my moral compass.''

The sentencing in federal court in Manhattan capped a startling fall for
Mr. Cohen, 52, who had once hoped to work by Mr. Trump's side in the
White House but ended up a central figure in the inquiry into payments
to an adult-film star and a former Playboy model before the 2016
election.

Judge William H. Pauley III called Mr. Cohen's crimes a ``veritable
smorgasbord of fraudulent conduct'' and added, ``Each of the crimes
involved deception and each appears to have been motivated by personal
greed and ambition.''

He added that Mr. Cohen's particular crimes --- breaking campaign
finance laws, tax evasion and lying to Congress --- ``implicate a far
more insidious harm to our democratic institutions.''

``As a lawyer, Mr. Cohen should have known better,'' the judge said.

Mr. Cohen had pleaded guilty in two separate cases, one brought by
federal prosecutors in Manhattan, the other by the office of the special
counsel, Robert S. Mueller III, who is investigating Russian
interference in the 2016 election.

Before he was sentenced, a solemn Mr. Cohen, standing at a lectern,
sounded emotional, but resolved, as he told the judge he had been
tormented by the anguish and embarrassment he had caused his family.

``I blame myself for the conduct which has brought me here today,'' he
said, ``and it was my own weakness and a blind loyalty to this man'' ---
a reference to Mr. Trump --- ``that led me to choose a path of darkness
over light.''

Mr. Cohen then apologized to the public: ``You deserve to know the truth
and lying to you was unjust.''

Rudolph W. Giuliani, one of Mr. Trump's lawyers, called Mr. Cohen's
assertion he had acted out of loyalty to Mr. Trump ``a complete lie.''

``I feel sorry for him,'' Mr. Giuliani added. ``He's a pathetic serial
liar.''

Federal agents
\href{https://www.nytimes3xbfgragh.onion/2018/04/09/us/politics/fbi-raids-office-of-trumps-longtime-lawyer-michael-cohen.html}{raided
Mr. Cohen's office and home in April}, and he later turned on Mr. Trump,
making the remarkable admission in court that Mr. Trump had directed him
to arrange the payments.

Mr. Trump at first denied knowing anything about the payments, but then
acknowledged that he had known about them. This week,
\href{https://www.nytimes3xbfgragh.onion/2018/12/10/us/politics/trump-campaign-finance-crimes-defense.html}{he
insisted that the payments} were ``a simple private transaction'' ---
not election-related spending subject to campaign-finance laws.

He also maintained that even if the hush-money payments were campaign
transactions in violation of election law, that should be considered
only a civil offense, not a criminal one.

Since Mr. Cohen came under investigation, Mr. Trump has mocked him as a
``weak person'' who was giving information to prosecutors in an effort
to obtain leniency when he is sentenced.

In fact, Mr. Cohen did not sign a formal cooperation agreement with the
United States attorney's office in Manhattan or with Mr. Mueller. In
addition to the campaign-finance violations, Mr. Cohen pleaded guilty to
charges of tax evasion, making false statements to a bank and lying to
Congress.

He took a calculated gamble in pleading guilty to this litany of federal
crimes without first entering into a cooperation agreement with the
government. He offered to help prosecutors, but only on his terms, and
there were some subjects he declined to discuss.

His lawyers argued he should not serve time in prison. Federal
prosecutors in Manhattan said he deserved around four years.

Judge Pauley had the final say. The judge said Mr. Cohen's assistance to
the special counsel's office, though useful, did not ``wipe the slate
clean,'' and a ``significant term'' of prison was justified.

\href{https://www.nytimes3xbfgragh.onion/interactive/2018/08/21/us/mueller-trump-charges.html}{}

\includegraphics{https://static01.graylady3jvrrxbe.onion/images/2018/08/22/us/mueller-trump-charges-promo-1534968452597/mueller-trump-charges-promo-1534968452597-articleLarge-v7.jpg}

\hypertarget{roger-stone-and-everyone-charged-in-the-2016-election-investigations}{%
\subsection{Roger Stone and Everyone Charged in the 2016 Election
Investigations}\label{roger-stone-and-everyone-charged-in-the-2016-election-investigations}}

Roger J. Stone Jr., one of six Trump advisers convicted in cases
stemming from the investigation by the special counsel, was sentenced to
more than three years in prison.

In the end, the judge gave Mr. Cohen three years for the crimes he
committed in New York and two months for lying to Congress, to be served
at the same time. He was also asked to pay nearly \$2 million in fines,
forfeitures and restitution. The judge ordered Mr. Cohen to begin
serving his sentence on March 6.

Mr. Cohen's sentencing was unusual because it involved guilty pleas he
had made in cases brought by the two separate prosecutors.

In the case brought by Mr. Mueller's office, Mr. Cohen
\href{https://www.nytimes3xbfgragh.onion/2018/11/29/nyregion/michael-cohen-trump-russia-mueller.html}{pleaded
guilty to lying to Congress} about the duration of negotiations to build
a Trump Tower in Moscow, as well as about the extent of the involvement
of Mr. Trump.

Mr. Cohen revealed that Mr. Trump was more involved in discussions over
the potential deal during the election campaign than previously known.

Mr. Cohen's three-year sentence is the first substantial prison term in
a case stemming from Mr. Mueller's inquiry. The special counsel had
referred the case to the United States attorney's office in Manhattan,
where it was overseen by Robert S. Khuzami, the No. 2 official there,
who attended the hearing on Wednesday.

The investigation of Mr. Cohen by the United States attorney's office in
Manhattan burst into public view in April when the F.B.I. raided his
office, apartment and hotel room. Agents hauled off eight boxes of
documents, about 30 cellphones, iPads and computers, even the contents
of a shredder.

Four months later, on Aug. 21,
\href{https://www.nytimes3xbfgragh.onion/2018/08/21/nyregion/michael-cohen-plea-deal-trump.html}{Mr.
Cohen pleaded guilty to campaign finance violations}, tax evasion and
making false statements to a financial institution.

Mr. Cohen admitted in court that he had arranged the payments ``for the
principal purpose of influencing the election'' for president in 2016.

The payments included \$130,000 to the adult-film actress Stormy
Daniels, which the government considers an illegal donation to Mr.
Trump's campaign since it was intended to improve Mr. Trump's election
chances. (The legal limit for individual contributions is \$2,700 in a
general election.)

Mr. Cohen also admitted he had arranged for an illegal corporate
donation to be made to Mr. Trump when he orchestrated a \$150,000
payment by American Media Inc. to a former Playboy playmate, Karen
McDougal, in late summer 2016.

Prosecutors in Manhattan wrote last Friday to Judge Pauley that Mr.
Cohen, in arranging the payments,
``\href{https://www.nytimes3xbfgragh.onion/2018/12/07/nyregion/michael-cohen-sentence.html}{acted
in coordination with and at the direction}'' of Mr. Trump, whom they
referred to as Individual 1.

On Nov. 29, charged by Mr. Mueller's office with lying to Congress, Mr.
Cohen pleaded guilty again.

The two prosecuting offices each wrote to Judge Pauley, offering sharply
contrasting portrayals of Mr. Cohen. The Southern District depicted him
as deceitful and greedy and unwilling to fully cooperate with its
investigation. Mr. Mueller, on the other hand, said Mr. Cohen had ``gone
to significant lengths to assist'' the Russia investigation and
recommended that he receive some credit for his help.

Mr. Cohen's lawyer, Guy Petrillo, made an impassioned plea for leniency,
citing his client's courage in cooperating with the Russia inquiry,
which he said was ``of the utmost national significance,'' comparing it
to Watergate.

He added: ``He came forward to offer evidence against the most powerful
person in our country,'' without knowing what the result would be, how
the politics would play out, or whether ``the special counsel would even
survive.''

Jeannie Rhee, a prosecutor from Mr. Mueller's office, with which Mr.
Cohen met seven times, told the judge that Mr. Cohen had accepted
responsibility for the lies he told Congress and had provided ``credible
and reliable information about core Russia-related issues.''

But a Manhattan prosecutor, Nicholas Roos, said of Mr. Cohen's
cooperation that as much as he ``claims he's done for the republic, the
same can be true in the way in which he's undermined it.''

In the end, Judge Pauley seemed to side with the defense. He said that
``cooperation, even when it is not the product of a formal agreement,
should be encouraged'' when it advances a criminal investigation. ``Our
system of justice would be less robust without the use of cooperating
witnesses to assist law enforcement,'' the judge said.

Shortly before Mr. Cohen and his family and friends walked out of the
courtroom, roughly 20 minutes after the sentencing had ended, he briefly
addressed a cluster of reporters who were waiting in an anteroom between
the courtroom and the hallway.

``This is my last time talking to you guys,'' he said abruptly, then
said nothing more.

He and his lawyers waited briefly in the courthouse lobby while his
wife, son and daughter, left the building, passed through a huge gantlet
of television crews and photographers outside, and climbed into a
waiting black Infiniti QX60.

Cohen then walked out of the building with a hangdog expression and made
a silent beeline to the car.

Advertisement

\protect\hyperlink{after-bottom}{Continue reading the main story}

\hypertarget{site-index}{%
\subsection{Site Index}\label{site-index}}

\hypertarget{site-information-navigation}{%
\subsection{Site Information
Navigation}\label{site-information-navigation}}

\begin{itemize}
\tightlist
\item
  \href{https://help.nytimes3xbfgragh.onion/hc/en-us/articles/115014792127-Copyright-notice}{©~2020~The
  New York Times Company}
\end{itemize}

\begin{itemize}
\tightlist
\item
  \href{https://www.nytco.com/}{NYTCo}
\item
  \href{https://help.nytimes3xbfgragh.onion/hc/en-us/articles/115015385887-Contact-Us}{Contact
  Us}
\item
  \href{https://www.nytco.com/careers/}{Work with us}
\item
  \href{https://nytmediakit.com/}{Advertise}
\item
  \href{http://www.tbrandstudio.com/}{T Brand Studio}
\item
  \href{https://www.nytimes3xbfgragh.onion/privacy/cookie-policy\#how-do-i-manage-trackers}{Your
  Ad Choices}
\item
  \href{https://www.nytimes3xbfgragh.onion/privacy}{Privacy}
\item
  \href{https://help.nytimes3xbfgragh.onion/hc/en-us/articles/115014893428-Terms-of-service}{Terms
  of Service}
\item
  \href{https://help.nytimes3xbfgragh.onion/hc/en-us/articles/115014893968-Terms-of-sale}{Terms
  of Sale}
\item
  \href{https://spiderbites.nytimes3xbfgragh.onion}{Site Map}
\item
  \href{https://help.nytimes3xbfgragh.onion/hc/en-us}{Help}
\item
  \href{https://www.nytimes3xbfgragh.onion/subscription?campaignId=37WXW}{Subscriptions}
\end{itemize}
