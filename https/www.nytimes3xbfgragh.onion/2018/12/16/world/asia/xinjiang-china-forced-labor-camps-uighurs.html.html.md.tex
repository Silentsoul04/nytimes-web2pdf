Sections

SEARCH

\protect\hyperlink{site-content}{Skip to
content}\protect\hyperlink{site-index}{Skip to site index}

\href{https://www.nytimes3xbfgragh.onion/section/world/asia}{Asia
Pacific}

\href{https://myaccount.nytimes3xbfgragh.onion/auth/login?response_type=cookie\&client_id=vi}{}

\href{https://www.nytimes3xbfgragh.onion/section/todayspaper}{Today's
Paper}

\href{/section/world/asia}{Asia Pacific}\textbar{}China's Detention
Camps for Muslims Turn to Forced Labor

\url{https://nyti.ms/2QUZ3vh}

\begin{itemize}
\item
\item
\item
\item
\item
\item
\end{itemize}

Advertisement

\protect\hyperlink{after-top}{Continue reading the main story}

Supported by

\protect\hyperlink{after-sponsor}{Continue reading the main story}

\hypertarget{chinas-detention-camps-for-muslims-turn-to-forced-labor}{%
\section{China's Detention Camps for Muslims Turn to Forced
Labor}\label{chinas-detention-camps-for-muslims-turn-to-forced-labor}}

\includegraphics{https://static01.graylady3jvrrxbe.onion/images/2018/12/17/world/17xinjiang-labor1/merlin_148213692_0d79ec0d-c129-4886-9ab0-1a8f8723d82a-articleLarge.jpg?quality=75\&auto=webp\&disable=upscale}

By \href{https://www.nytimes3xbfgragh.onion/by/chris-buckley}{Chris
Buckley} and
\href{https://www.nytimes3xbfgragh.onion/by/austin-ramzy}{Austin Ramzy}

\begin{itemize}
\item
  Dec. 16, 2018
\item
  \begin{itemize}
  \item
  \item
  \item
  \item
  \item
  \item
  \end{itemize}
\end{itemize}

\href{https://cn.nytimes3xbfgragh.onion/china/20181217/xinjiang-china-forced-labor-camps-uighurs/}{阅读简体中文版}\href{https://cn.nytimes3xbfgragh.onion/china/20181217/xinjiang-china-forced-labor-camps-uighurs/zh-hant/}{閱讀繁體中文版}\href{https://www.nytimes3xbfgragh.onion/es/2018/12/18/campos-trabajo-musulmanes-china}{Leer
en español}

KASHGAR, China --- Muslim inmates from internment camps in far western
China hunched over sewing machines, in row after row. They were among
\href{https://www.nytimes3xbfgragh.onion/2018/09/08/world/asia/china-uighur-muslim-detention-camp.html}{hundreds
of thousands who had been detained} and spent month after month
renouncing their religious convictions. Now the government was showing
them on television as models of repentance, earning good pay --- and
political salvation --- as factory workers.

China's ruling Communist Party has said in a surge of upbeat propaganda
that a sprawling network of camps in the Xinjiang region is providing
job training and putting detainees on production lines for their own
good, offering an escape from poverty, backwardness and the temptations
of radical Islam.

But mounting evidence suggests a system of forced labor is emerging from
the camps, a development likely to intensify international condemnation
of China's drastic efforts to
\href{https://www.nytimes3xbfgragh.onion/2018/09/08/world/asia/china-uighur-muslim-detention-camp.html}{control
and indoctrinate} a Muslim ethnic minority population of more than 12
million in Xinjiang.

Accounts from the region, satellite images and previously unreported
official documents indicate that growing numbers of detainees are being
sent to new factories, built inside or near the camps, where inmates
have little choice but to accept jobs and follow orders.

``These people who are detained provide free or low-cost forced labor
for these factories,'' said Mehmet Volkan Kasikci, a researcher in
Turkey who has collected accounts of inmates in the factories by
interviewing relatives who have left China. ``Stories continue to come
to me,'' he said.

China has defied an international outcry against the vast internment
program in Xinjiang, which holds Muslims and forces them to renounce
religious piety and pledge loyalty to the party. The emerging labor
program underlines the government's determination to continue operating
the camps despite calls from United Nations
\href{https://www.nytimes3xbfgragh.onion/2018/11/13/world/asia/un-china-xinjiang-muslim-internments.html}{human
rights officials}, the
\href{https://www.nytimes3xbfgragh.onion/2018/09/10/world/asia/us-china-sanctions-muslim-camps.html}{United
States} and
\href{https://www.nytimes3xbfgragh.onion/2018/11/06/world/asia/china-detention-uighur-muslims.html}{other
governments} to close them.

\includegraphics{https://static01.graylady3jvrrxbe.onion/images/2018/12/17/world/17xinjiang-labor5/17xinjiang-labor5-articleLarge.jpg?quality=75\&auto=webp\&disable=upscale}

The program aims to transform scattered Uighurs, Kazakhs and other
ethnic minorities --- many of them farmers, shopkeepers and tradespeople
--- into a disciplined, Chinese-speaking industrial work force, loyal to
the Communist Party and factory bosses, according to official plans
published online.

These documents describe the camps as vocational training centers and do
not specify whether inmates are required to accept assignments to
factories or other jobs. But pervasive restrictions on the movement and
employment of Muslim minorities in Xinjiang, as well as a government
effort to persuade businesses to open factories around the camps,
suggest that they have little choice.

Independent accounts from inmates who have worked in the factories are
rare. The police block attempts to get near the camps and closely
monitor foreign journalists who travel to Xinjiang, making it all but
impossible to conduct interviews in the region. And most Uighurs who
have fled Xinjiang did so before the factory program grew in recent
months.

But Serikzhan Bilash, a founder of Atajurt Kazakh Human Rights, an
organization in Kazakhstan that helps ethnic Kazakhs who have left
neighboring Xinjiang, said he had interviewed relatives of 10 inmates
who had told their families that they were made to work in factories
after undergoing indoctrination in the camps.

They mostly made clothes, and they called their employers ``black
factories,'' because of the low wages and tough conditions, he said.

Mr. Kasikci also described several cases based on interviews with family
members: Sofiya Tolybaiqyzy, who was sent from a camp to work in a
carpet factory. Abil Amantai, 37, who was put in a camp a year ago and
told relatives he was working in a textile factory for \$95 a month.
Nural Razila, 25, who had studied oil drilling but after a year in a
camp was sent to a new textile factory nearby.

``It's not as though they have a choice of whether they get to work in a
factory, or what factory they are assigned to,'' said
\href{https://anthropology.washington.edu/people/darren-byler}{Darren
Byler}, a lecturer at the University of Washington who studies Xinjiang
and visited the region in April.

Image

Uighur men at a tea house in Kashgar, an area in southern Xinjiang that
is a focus of the expanding labor program.Credit...Bryan Denton for The
New York Times

He said it was safe to conclude that hundreds of thousands of detainees
could be compelled to work in factories if the program were put in place
at all of the region's internment camps.

The Xinjiang government did not respond to faxed questions about the
factories, nor did the State Council Information Office, the central
government agency that answers reporters' questions.

The documents detail plans for inmates, even those formally released
from the camps, to take jobs at factories that work closely with the
camps to continue to monitor and control them. The socks, suits, skirts
and other goods made by these laborers would be sold in Chinese stores
and could trickle into overseas markets.

Kashgar, an ancient, predominantly Uighur area of southern Xinjiang that
is a focus of the program, reported that in 2018 alone it aimed to send
100,000 inmates who had been through the ``vocational training centers''
to work in factories, according to a
\href{http://kashi.gov.cn/Government/PublicInfoShow.aspx?ID=2963}{plan
issued in August}.

That figure may be an ambitious political goal rather than a realistic
target. But it suggests how many Uighurs and other Muslim ethnic
minorities may be held in the camps and sent to factories. Scholars have
estimated that
\href{https://jamestown.org/program/evidence-for-chinas-political-re-education-campaign-in-xinjiang/}{as
many as one million people} have been detained. The Chinese government
has not issued or confirmed any figures.

``I don't see China yielding an inch on Xinjiang,'' said
\href{https://duihua.org/who-we-are/\#0}{John Kamm}, the founder of the
Dui Hua Foundation, a San Francisco-based group that lobbies China on
human rights issues. ``Now it seems we have entrepreneurs coming in and
taking advantage of the situation.''

The evolution of the Xinjiang camps echoes China's
``\href{https://www.nytimes3xbfgragh.onion/1981/01/03/world/hundreds-of-thousands-toil-in-chinese-labor-camps.html}{re-education
through labor}'' system, where citizens once were sent without trial to
toil for years. China abolished ``re-education through labor'' five
years ago, but Xinjiang appears to be creating a new version.

Retailers in the United States and other countries should guard against
buying goods made by workers from the Xinjiang camps, which could
\href{https://www.uscc.gov/sites/default/files/Research/Staff\%20Report_Prison\%20Labor\%20Exports\%20from\%20China_Final\%20Report\%20070914.pdf}{violate
laws banning imports} produced by prison or forced labor, Mr. Kamm said.

While the bulk of clothes and other textile goods manufactured in
Xinjiang ends up in domestic and Central Asian markets, some makes its
way to the United States and Europe.

Badger Sportswear, a company based in North Carolina, last month
received a container of polyester knitted T-shirts from Hetian Taida, a
company in Xinjiang that was shown on a
\href{https://www.youtube.com/watch?v=REGM8I2iMjU}{prime-time state
television broadcast promoting the camps}.

The program showed workers at a Hetian Taida plant, including a woman
who was described as a former camp inmate. But the small factory did not
appear to be on a camp site, and it is unclear whether it made the
T-shirts sent to North Carolina.

Ginny Gasswint, a Badger Sportswear executive, said the company had
ordered a small amount of products from Xinjiang, and used Worldwide
Responsible Accredited Production, a nonprofit certification
organization, to ensure that its suppliers meet standards.

Seth Lennon, a spokesman for Worldwide, said that Hetian Taida had only
recently enrolled in its program, and the organization had no
information on possible coerced labor in Xinjiang. ``We will certainly
look into this,'' he said.

Repeated calls over several days to Wu Hongbo, the chairman of Hetian
Taida, went unanswered.

Satellite imagery suggests that production lines are being built inside
some internment camps.

Image

A state television broadcast promoting the internment camps showed
textile workers at a company named Hetian Taida. The company shipped
T-shirts to North Carolina last month.

Images of
\href{https://medium.com/@shawnwzhang/satellite-imagery-of-xinjiang-re-education-camp-63-\%E6\%96\%B0\%E7\%96\%86\%E5\%86\%8D\%E6\%95\%99\%E8\%82\%B2\%E9\%9B\%86\%E4\%B8\%AD\%E8\%90\%A5\%E5\%8D\%AB\%E6\%98\%9F\%E5\%9B\%BE-63-5b324fa241d1}{one
camp featured in the state television broadcast}, for example, show 10
to 12 large buildings with a single-story, one-room design commonly used
for factories, said Nathan Ruser, a researcher at the Australian
Strategic Policy Institute. The buildings are surrounded by fencing and
security towers, indicating that they are heavily guarded like the rest
of the camp.

``It seems unlikely that any detainee would be able to go to any
building that they were not taken to,'' Mr. Ruser said.

Commercial registration records also show at least a few companies have
been established this year at addresses inside internment camps. They
include
\href{https://xin.baidu.com/detail/compinfo?pid=N4qLKt5szCV8cY9yeyLwcmh10gPAQIFMIw7H}{a
printing factory}, a
\href{https://xin.baidu.com/detail/compinfo?pid=lAMlHRitOZ*osCEY8tZVKeUAhr9D0tF-ewXX}{noodle
factory} and at least
\href{https://gongshang.mingluji.com/xinjiang/name/\%E8\%8E\%8E\%E8\%BD\%A6\%E5\%AE\%9C\%E5\%8F\%B0\%E5\%98\%89\%E7\%A6\%BE\%E7\%BA\%BA\%E7\%BB\%87\%E6\%9C\%89\%E9\%99\%90\%E5\%85\%AC\%E5\%8F\%B8}{two}
\href{https://web.archive.org/web/20181215064833/https://gongshang.mingluji.com/xinjiang/name/\%E6\%96\%B0\%E7\%96\%86\%E6\%81\%92\%E6\%98\%8C\%E4\%BC\%9F\%E4\%B8\%9A\%E6\%9C\%8D\%E8\%A3\%85\%E6\%9C\%89\%E9\%99\%90\%E5\%85\%AC\%E5\%8F\%B8}{clothing}
and textile manufacturers at camps in rural areas around Kashgar.
Another clothing and bedding manufacturer is registered in a camp in
Aksu in northwestern Xinjiang.

The government's effort to connect the internment camps with factories
emerged this year as the number of detainees climbed and Xinjiang faced
rising costs to build and run the camps.

Many camps were once called ``transformation through education centers''
by the government, reflecting their mission: inducing inmates to cast
aside Islamic devotion and accept Communist Party supremacy.

But
\href{https://www.nytimes3xbfgragh.onion/2018/08/13/world/asia/china-xinjiang-un.html}{since
August}, the Chinese government has defended the camps by
\href{https://www.nytimes3xbfgragh.onion/2018/10/16/world/asia/china-muslim-camps-xinjiang-uighurs.html}{arguing
that they are job training centers} that will help lift detainees and
their families out of poverty by giving them the skills to join China's
economic mainstream. Many rural Uighurs speak little Chinese, and
language training has been advertised as one of the main purposes of the
camps.

Yet the practical training in the camps often appears to be rudimentary,
said \href{https://www.awm-korntal.eu/en/person/1302.html}{Adrian Zenz},
a social scientist at the European School of Culture and Theology who
has studied
\href{https://jamestown.org/program/xinjiangs-re-education-and-securitization-campaign-evidence-from-domestic-security-budgets/}{the
campaign}.

Image

The old city of Kashgar, where officials set a goal of sending 100,000
camp inmates to work in factories this year.Credit...Bryan Denton for
The New York Times

An early hint of the factory labor program came in March when Sun
Ruizhe, the president of the China National Textile and Apparel Council,
\href{http://www.ccta.org.cn/hyzx/201803/t20180305_3683861.html}{described
it to senior industry representatives}, according to a transcript of his
speech that was posted on industry websites.

Mr. Sun said that Xinjiang planned to recruit from three main sources to
increase the textile and garment sector's work force by more than
100,000 in 2018: impoverished households, struggling relatives of
prisoners and detainees, and the camp inmates, whose training ``could be
combined with developing the textile and apparel section.''

In April, the Xinjiang government began rolling out
\href{https://www.sohu.com/a/227485314_280643}{a plan} to attract
textile and garment companies. Local governments would receive funds to
build production sites for them near the camps; companies would receive
a subsidy of \$260 to train each inmate they took on, as well other
incentives.

In remarks in October
\href{http://www.chinaxinjiang.cn/zixun/xjxw/201810/t20181016_570748.htm}{defending
the camps}, a top official in Xinjiang, Shohrat Zakir, said the
government was busy preparing ``job assignments'' for inmates formally
finishing indoctrination and training. A
\href{http://shache.xinjiang.gov.cn/webpub/bpf/mlgk/viewDetail.jsp?identifier=568884582/2018-00067}{budget
document} this year from Yarkant, a county in Kashgar, said the camps
were responsible for ``employment services.''

The inmates assigned to factories may have to stay for years.

Mr. Byler said a relative of a Uighur friend was sent to an
indoctrination camp in March and formally released this fall. But he was
then told he had to work for up to three years in a clothing factory.

A government official, Mr. Byler said, suggested to his friend's family
that if the relative worked hard, his time in the factory might be
reduced.

The Chinese state media has praised the centers as leading wayward
people toward modern civilization. It also reports that the workers are
generously paid.

``The training will turn them from `nomads' into skilled marvels,'' the
official Xinjiang Daily
\href{http://wap.xjdaily.com/xjrb/20181111/118479.html}{said last
month}. ``Education and training will make them into `modern people,'
useful to society.''

Advertisement

\protect\hyperlink{after-bottom}{Continue reading the main story}

\hypertarget{site-index}{%
\subsection{Site Index}\label{site-index}}

\hypertarget{site-information-navigation}{%
\subsection{Site Information
Navigation}\label{site-information-navigation}}

\begin{itemize}
\tightlist
\item
  \href{https://help.nytimes3xbfgragh.onion/hc/en-us/articles/115014792127-Copyright-notice}{©~2020~The
  New York Times Company}
\end{itemize}

\begin{itemize}
\tightlist
\item
  \href{https://www.nytco.com/}{NYTCo}
\item
  \href{https://help.nytimes3xbfgragh.onion/hc/en-us/articles/115015385887-Contact-Us}{Contact
  Us}
\item
  \href{https://www.nytco.com/careers/}{Work with us}
\item
  \href{https://nytmediakit.com/}{Advertise}
\item
  \href{http://www.tbrandstudio.com/}{T Brand Studio}
\item
  \href{https://www.nytimes3xbfgragh.onion/privacy/cookie-policy\#how-do-i-manage-trackers}{Your
  Ad Choices}
\item
  \href{https://www.nytimes3xbfgragh.onion/privacy}{Privacy}
\item
  \href{https://help.nytimes3xbfgragh.onion/hc/en-us/articles/115014893428-Terms-of-service}{Terms
  of Service}
\item
  \href{https://help.nytimes3xbfgragh.onion/hc/en-us/articles/115014893968-Terms-of-sale}{Terms
  of Sale}
\item
  \href{https://spiderbites.nytimes3xbfgragh.onion}{Site Map}
\item
  \href{https://help.nytimes3xbfgragh.onion/hc/en-us}{Help}
\item
  \href{https://www.nytimes3xbfgragh.onion/subscription?campaignId=37WXW}{Subscriptions}
\end{itemize}
