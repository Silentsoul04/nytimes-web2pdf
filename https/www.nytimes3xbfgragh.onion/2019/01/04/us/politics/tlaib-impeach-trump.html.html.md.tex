Sections

SEARCH

\protect\hyperlink{site-content}{Skip to
content}\protect\hyperlink{site-index}{Skip to site index}

\href{https://www.nytimes3xbfgragh.onion/section/politics}{Politics}

\href{https://myaccount.nytimes3xbfgragh.onion/auth/login?response_type=cookie\&client_id=vi}{}

\href{https://www.nytimes3xbfgragh.onion/section/todayspaper}{Today's
Paper}

\href{/section/politics}{Politics}\textbar{}Rashida Tlaib's
Expletive-Laden Cry to Impeach Trump Upends Democrats' Talking Points

\url{https://nyti.ms/2GQXVoK}

\begin{itemize}
\item
\item
\item
\item
\item
\end{itemize}

Advertisement

\protect\hyperlink{after-top}{Continue reading the main story}

Supported by

\protect\hyperlink{after-sponsor}{Continue reading the main story}

\hypertarget{rashida-tlaibs-expletive-laden-cry-to-impeach-trump-upends-democrats-talking-points}{%
\section{Rashida Tlaib's Expletive-Laden Cry to Impeach Trump Upends
Democrats' Talking
Points}\label{rashida-tlaibs-expletive-laden-cry-to-impeach-trump-upends-democrats-talking-points}}

\includegraphics{https://static01.graylady3jvrrxbe.onion/images/2019/01/05/us/politics/05dc-impeach/05dc-impeach-articleLarge.jpg?quality=75\&auto=webp\&disable=upscale}

By \href{https://www.nytimes3xbfgragh.onion/by/nicholas-fandos}{Nicholas
Fandos}

\begin{itemize}
\item
  Jan. 4, 2019
\item
  \begin{itemize}
  \item
  \item
  \item
  \item
  \item
  \end{itemize}
\end{itemize}

WASHINGTON --- Impeachment was always going to hang heavily over a
divided Washington. But it took little more than 24 hours this week for
a freshman House Democrat's exuberant, expletive-laden impeachment
promise to upend the bonhomie of a new Congress and prompt President
Trump, by his own telling, to ask the newly elected speaker if Democrats
planned to impeach him.

The episode began Thursday night, just hours after the 116th Congress
was sworn in, when a camera captured Representative Rashida Tlaib of
Michigan promising profanely to impeach Mr. Trump as she drew cheers
from liberal activists at a celebration at a bar near the Capitol. By
the time Mr. Trump discussed the matter directly in a news conference in
the Rose Garden on Friday afternoon, weeks of speculation about his
potential peril had burst into the open.

Republicans, eager to portray Democrats as out to destroy Mr. Trump's
presidency, piled on criticism of Ms. Tlaib --- some of it racially
tinged. (Ms. Tlaib, who is Palestinian-American, is one of the first
Muslims in Congress. The Christian Broadcasting Network referred to her
as a ``foul-mouthed Islamic congresswoman.'') Democratic leaders, who
view discussion of impeachment as politically dangerous and premature,
offered worried words meant to tamp down speculation about their
intentions.

\href{https://www.nytimes3xbfgragh.onion/2019/01/04/us/politics/democrats-trump-meeting-government-shutdown.html?action=click\&module=Top\%20Stories\&pgtype=Homepage}{\emph{{[}President
Trump threatened to keep the federal government partly closed for
``months or even years.''{]}}}

Perhaps out of a belief that an impeachment fight would help him
politically --- as it did President Bill Clinton in the 1990s --- or
outright fear that newly empowered Democrats actually might threaten his
presidency, Mr. Trump dived into the conversation head first.

``We even talked about that today,'' he told reporters in the Rose
Garden, referring to an exchange with Speaker Nancy Pelosi during
\href{https://www.nytimes3xbfgragh.onion/2019/01/04/us/politics/democrats-trump-meeting-government-shutdown.html}{a
meeting earlier Friday} to try to negotiate an end to the shutdown of
the government, which Mr. Trump threatened to keep closed for years if
he did not get money from Congress for a wall on the southern border.
``I said, Why don't you use this for impeachment? And Nancy said, We're
not looking to impeach you.''

Senior aides to Ms. Pelosi and other Democrats in the room disputed that
characterization. The president had indeed invoked Ms. Tlaib and other
House Democrats who want to impeach him, they said, but Ms. Pelosi tried
to shift the meeting back to its intended topic and did not offer the
president reassurances.

``In his opening comments at the meeting, President Trump brought up
impeachment,'' Ms. Pelosi's spokesman, Drew Hammill,
\href{https://twitter.com/Drew_Hammill/status/1081283127451049985}{wrote
on Twitter}. ``Speaker Pelosi made clear that today's meeting was about
re-opening government, not impeachment.''

Regardless, Mr. Trump's frank embrace of the issue was another
astonishing development: a president of the United States talking openly
about his potential impeachment at a White House news conference.

Earlier in the day, Mr. Trump
\href{https://twitter.com/realDonaldTrump/status/1081177511592108032}{had
asked on Twitter}, ``How do you impeach a president who has won perhaps
the greatest election of all time?'' He continued the theme at his news
conference by asserting that ``you can't impeach somebody who is doing a
great job.''

Ms. Pelosi and senior Democrats said they were determined not to take
the bait for now and risk generating a backlash from Mr. Trump's
supporters, who would most likely see impeachment as the overreaction of
out-of-control Democrats. But the words of Ms. Tlaib, who stood by her
comments on Friday, made evident the pressure already mounting from the
left, where public opinion polls suggest a majority of liberals want the
president removed from office.

``People love you and you win,'' Ms. Tlaib told the crowd Thursday
night. ``And when your son looks at you and says: `Momma, look, you won.
Bullies don't win.' And I said, `Baby, they don't.' Because we're going
to go in there, and we're going to impeach the motherfucker.''

She made no apologies for the remark on Friday, proclaiming that ``I
will always speak truth to power'' and fashioning her own hashtag,
\#unapologeticallyme. She told a Detroit television station that ``it's
probably exactly how my grandmother, if she was alive, would say it.''

Her outburst ran counter to all Democratic talking points. Ms. Pelosi
and her deputies have repeatedly made the case that it is too early to
consider impeachment. Even as Mr. Trump's legal perils have deepened ---
and federal prosecutors in New York appear to have gathered evidence
implicating him in a campaign finance crime ---
\href{https://www.nytimes3xbfgragh.onion/2018/11/30/us/politics/jerrold-nadler-trump-impeachment.html}{Democrats
have said they want to wait to see the findings} of an investigation by
the special counsel, Robert S. Mueller III, of the president, his
campaign and Russia's attempts to interfere in the 2016 election.

``I don't really like that kind of language,'' Representative Jerrold
Nadler, Democrat of New York and
\href{https://www.nytimes3xbfgragh.onion/2018/11/30/us/politics/jerrold-nadler-trump-impeachment.html}{the
chairman of the House Judiciary Committee} --- where any impeachment
inquiry must begin --- said on CNN. ``But more to the point, I disagree
with what she said. It is too early to talk about that intelligently. We
have to follow the facts.''

Ms. Pelosi defended Ms. Tlaib on Friday at a town hall hosted by MSNBC
at the speaker's alma mater in Washington, Trinity University. ``I'm not
in the censorship business,'' Ms. Pelosi said.

The Constitution grants the House the power to impeach executive branch
officials for ``treason, bribery or other high crimes and
misdemeanors,'' but what constitutes such crimes has traditionally been
defined by the majority party of the House. A vote to impeach takes a
simple majority in the House, but a two-thirds vote is needed in the
Senate to convict and remove an official from office.

Mr. Nadler, Ms. Pelosi and other party elders believe Mr. Trump is
threatening the country's democratic institutions. Privately, many
harbor suspicions that he obstructed justice, collaborated with the
Russians in 2016 or both. But they also argue that an impeachment that
does not have a reasonable shot of winning a conviction in the Senate
will backfire and strengthen Mr. Trump in the 2020 campaign. In the
meantime, they are planning to open multiple investigations into
accusations of wrongdoing around the president, his campaign and his
administration.

``We shouldn't be impeaching for a political reason, and we shouldn't
avoid impeachment for a political reason,'' Ms. Pelosi said. ``So we'll
just have to see how it comes.''

Republicans ignored the distinction.

``Is this the behavior that we are going to find with this new majority
party in Congress?'' Representative Kevin McCarthy of California, the
Republican leader, asked at a news conference. He repeatedly singled out
Ms. Pelosi, asking why she had not censured Ms. Tlaib.

Mr. Trump was similarly critical of Ms. Tlaib.

``I thought her comments were disgraceful,'' Mr. Trump said at the news
conference. ``I think she dishonored herself, and I think she dishonored
her family.'' He added that her comments were ``disrespectful to the
United States of America.''

Ms. Tlaib is far from alone among House Democrats. Representatives Brad
Sherman of California and Al Green of Texas formally introduced an
article of impeachment on Thursday, charging that Mr. Trump had
obstructed justice in firing James B. Comey, the F.B.I. director. Others
are expected to follow.

``I continue to believe that obstruction of justice is the clearest,
simplest and most provable high crime and misdemeanor committed by
Donald J. Trump,'' Mr. Sherman said in a statement. ``I hope that the
articles of impeachment are the subject of hearings before the Judiciary
Committee early in 2019.''

Many of Ms. Tlaib's new colleagues expressed sympathy for her
sentiments, even as they said the House should proceed differently.

``Donald Trump is going to be impeached whether it is by the ballot box
or Congress,'' said Representative Eric Swalwell, Democrat of California
and a member of the Intelligence Committee. ``It will just be a matter
of which one comes first.''

But, Mr. Swalwell added, Democrats need to avoid making ``a martyr'' out
of Mr. Trump by affording him ``a fairer investigation than he
deserves.''

Advertisement

\protect\hyperlink{after-bottom}{Continue reading the main story}

\hypertarget{site-index}{%
\subsection{Site Index}\label{site-index}}

\hypertarget{site-information-navigation}{%
\subsection{Site Information
Navigation}\label{site-information-navigation}}

\begin{itemize}
\tightlist
\item
  \href{https://help.nytimes3xbfgragh.onion/hc/en-us/articles/115014792127-Copyright-notice}{©~2020~The
  New York Times Company}
\end{itemize}

\begin{itemize}
\tightlist
\item
  \href{https://www.nytco.com/}{NYTCo}
\item
  \href{https://help.nytimes3xbfgragh.onion/hc/en-us/articles/115015385887-Contact-Us}{Contact
  Us}
\item
  \href{https://www.nytco.com/careers/}{Work with us}
\item
  \href{https://nytmediakit.com/}{Advertise}
\item
  \href{http://www.tbrandstudio.com/}{T Brand Studio}
\item
  \href{https://www.nytimes3xbfgragh.onion/privacy/cookie-policy\#how-do-i-manage-trackers}{Your
  Ad Choices}
\item
  \href{https://www.nytimes3xbfgragh.onion/privacy}{Privacy}
\item
  \href{https://help.nytimes3xbfgragh.onion/hc/en-us/articles/115014893428-Terms-of-service}{Terms
  of Service}
\item
  \href{https://help.nytimes3xbfgragh.onion/hc/en-us/articles/115014893968-Terms-of-sale}{Terms
  of Sale}
\item
  \href{https://spiderbites.nytimes3xbfgragh.onion}{Site Map}
\item
  \href{https://help.nytimes3xbfgragh.onion/hc/en-us}{Help}
\item
  \href{https://www.nytimes3xbfgragh.onion/subscription?campaignId=37WXW}{Subscriptions}
\end{itemize}
