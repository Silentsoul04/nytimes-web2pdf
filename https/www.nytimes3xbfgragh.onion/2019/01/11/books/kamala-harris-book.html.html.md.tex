Sections

SEARCH

\protect\hyperlink{site-content}{Skip to
content}\protect\hyperlink{site-index}{Skip to site index}

\href{https://www.nytimes3xbfgragh.onion/section/books}{Books}

\href{https://myaccount.nytimes3xbfgragh.onion/auth/login?response_type=cookie\&client_id=vi}{}

\href{https://www.nytimes3xbfgragh.onion/section/todayspaper}{Today's
Paper}

\href{/section/books}{Books}\textbar{}Kamala Harris Talks About Her
Personal Story and `The Truths We Hold'

\url{https://nyti.ms/2H8kzJB}

\begin{itemize}
\item
\item
\item
\item
\item
\end{itemize}

Advertisement

\protect\hyperlink{after-top}{Continue reading the main story}

Supported by

\protect\hyperlink{after-sponsor}{Continue reading the main story}

5 Things About Your Book

\hypertarget{kamala-harris-talks-about-her-personal-story-and-the-truths-we-hold}{%
\section{Kamala Harris Talks About Her Personal Story and `The Truths We
Hold'}\label{kamala-harris-talks-about-her-personal-story-and-the-truths-we-hold}}

By \href{https://www.nytimes3xbfgragh.onion/by/john-williams}{John
Williams}

\begin{itemize}
\item
  Jan. 11, 2019
\item
  \begin{itemize}
  \item
  \item
  \item
  \item
  \item
  \end{itemize}
\end{itemize}

\includegraphics{https://static01.graylady3jvrrxbe.onion/images/2019/01/14/books/14harris-cover/14harris-cover-articleLarge.jpg?quality=75\&auto=webp\&disable=upscale}

Kamala Harris's new memoir, ``The Truths We Hold: An American Journey,''
begins with a prologue set on Nov. 8, 2016, the night Harris was elected
a United States senator from California. The rest of the book addresses
the urgent political matters that have risen in the wake of that night,
but it also goes back to cover, among other things, Harris's tenure as
California's attorney general and her childhood in Oakland as the
daughter of immigrant parents: her father an economist from Jamaica and
her mother a cancer researcher from India. Though rumors of Harris
\href{https://www.nytimes3xbfgragh.onion/2019/01/12/us/politics/kamala-harris-democrats.html}{gearing
up for a presidential run in 2020} are becoming noisier by the minute
(when
\href{https://www.facebookcorewwwi.onion/colbertlateshow/videos/374316359794531/}{Stephen
Colbert asked her on Thursday} if she would run, Harris coyly said, ``I
might''), she told me the memoir is not meant to help lay the groundwork
for
\href{https://www.nytimes3xbfgragh.onion/2018/12/29/us/politics/2020-democratic-candidates-kamala-elizabeth-warren.html}{such
a campaign}. ``At the expense of sounding immodest,'' she said, the book
is ``really about the work I've done already that's had national impact,
and what I hope to come from it.'' Below, Harris talks about how she
connects personal experiences to her professional life, the breakneck
speed of the news cycle, the inspiration she takes from Bob Marley and
more.

\textbf{When did you first get the idea to write this book?}

Election night, 2016. I sat on our couch at home after my night at the
election party with a family-size bag of Doritos, which I ate by myself,
one after the other, in awe and in shock about what I was watching on
TV. It was a night that was bittersweet for my campaign, for all of us.
None of us saw it coming.

After that night, I really felt a more urgent need to tell people what
we're fighting for. When we talk about a fight, it's born out of
optimism; and it's not a fight against something, but it's a fight for
something. It was that emotion that led me to speak the words I spoke
that night, about the need to fight; and that, by extension, led to the
book.

\textbf{What's the most surprising thing you learned while writing it?}

I was raised to do things, not to talk about myself or my feelings ---
or frankly, even to look back. It was an effort to talk about my
feelings as things were happening. It was difficult. I talk about a lot
that's really personal, and that I had not talked about in public. That
was a component of it that made me feel very vulnerable. But I felt it
was important to talk about for a couple of reasons. One, I'm really
clear in my mind that there are a lot of experiences I've had, emotional
experiences and responses, that are in common with a lot of people. But
more important, I wanted to give context to the work I've done. Almost
everything I've done professionally has been motivated by some
experience I've been exposed to.

\includegraphics{https://static01.graylady3jvrrxbe.onion/images/2019/01/14/books/review/14harris-fivequestions3/merlin_148985091_325322f1-6874-41c0-983c-c2bae48c8bb3-articleLarge.jpg?quality=75\&auto=webp\&disable=upscale}

The process of writing the book required me to really explore what I was
feeling at those moments. For example, the whole chapter that we named
``Underwater'' --- I had never talked about the fact that our mother
bought our first house when I was a teenager. I'll never forget, when my
mother came back and said, ``This is going to be our home.'' The
pictures and the excitement she had, and the excitement we then had. I
connected that emotion to what it meant for all those homeowners who
either had that hope when they engaged in what ended up being a
fraudulent mortgage scheme or when they lost their homes. Knowing what
that meant, when I'm sitting across the table from executives at the
biggest banks in the country and feeling a sense of responsibility, that
this wasn't simply a financial transaction. When your mother comes home
with the picture of the first home you're ever going to have, it's not
like someone waving around a piece of paper with a stock portfolio. It's
a whole other thing.

\textbf{In what way is the book you wrote different from the book you
set out to write?}

Hopefully the book takes the reader on a journey down memory lane about
the last 12 months and how much happened. Everything is happening so
rapidly right now that a lot of people tend to forget what just happened
six months ago, when the thing that happened six months ago was
earth-shattering. There's a lot in the book that was happening in real
time; so literally as I'm writing it, it's happening. The book was due
and then the Brett Kavanaugh hearings happened, and so how do I handle
that? It was important to me to at least try to talk about that, knowing
that people will be reading about it months after it happened.

Image

Kamala Harris, center, at an event in California calling for the end of
family separations at the border, in June 2018.Credit...via Kamala
Harris

\textbf{Who is a creative person (not a writer) who has influenced you
and your work?}

Certainly my mother. She was incredibly creative, as a scientist. But
when I think about performers: Bob Marley. I first started listening to
him when I was a child. My father had an incredible jazz collection but
also a lot of Marley. I saw him in concert at the Greek Theater in
Berkeley. I was hooked.

Jamaica's history is actually not that well known in the context of the
issues we deal with in the United States. But Jamaica grappled with
vicious slavery for generations, and then colonists, with a very strong
sense of identity in terms of what it meant to be particularly a black
Jamaican. A lot of his music was about what it means to fight for the
people. He was a very spiritual person also. I'm very spiritual. I don't
talk a lot about it, but the idea that there is a higher being and that
we should be motivated by love of one another --- that also requires us
to fight.

\textbf{Persuade someone to read ``The Truths We Hold'' in 50 words or
less.}

I hope you'll walk away renewing your faith in the nobility and
importance of public service, and convinced that we are a country that
was founded on noble ideals. Imperfect though we may be, what makes us
strong, and special, is that we've always aspired to reach those ideals.

\emph{This interview has been condensed and edited.}

Advertisement

\protect\hyperlink{after-bottom}{Continue reading the main story}

\hypertarget{site-index}{%
\subsection{Site Index}\label{site-index}}

\hypertarget{site-information-navigation}{%
\subsection{Site Information
Navigation}\label{site-information-navigation}}

\begin{itemize}
\tightlist
\item
  \href{https://help.nytimes3xbfgragh.onion/hc/en-us/articles/115014792127-Copyright-notice}{©~2020~The
  New York Times Company}
\end{itemize}

\begin{itemize}
\tightlist
\item
  \href{https://www.nytco.com/}{NYTCo}
\item
  \href{https://help.nytimes3xbfgragh.onion/hc/en-us/articles/115015385887-Contact-Us}{Contact
  Us}
\item
  \href{https://www.nytco.com/careers/}{Work with us}
\item
  \href{https://nytmediakit.com/}{Advertise}
\item
  \href{http://www.tbrandstudio.com/}{T Brand Studio}
\item
  \href{https://www.nytimes3xbfgragh.onion/privacy/cookie-policy\#how-do-i-manage-trackers}{Your
  Ad Choices}
\item
  \href{https://www.nytimes3xbfgragh.onion/privacy}{Privacy}
\item
  \href{https://help.nytimes3xbfgragh.onion/hc/en-us/articles/115014893428-Terms-of-service}{Terms
  of Service}
\item
  \href{https://help.nytimes3xbfgragh.onion/hc/en-us/articles/115014893968-Terms-of-sale}{Terms
  of Sale}
\item
  \href{https://spiderbites.nytimes3xbfgragh.onion}{Site Map}
\item
  \href{https://help.nytimes3xbfgragh.onion/hc/en-us}{Help}
\item
  \href{https://www.nytimes3xbfgragh.onion/subscription?campaignId=37WXW}{Subscriptions}
\end{itemize}
