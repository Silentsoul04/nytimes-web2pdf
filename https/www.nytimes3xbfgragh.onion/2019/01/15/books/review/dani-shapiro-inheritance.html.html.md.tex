Sections

SEARCH

\protect\hyperlink{site-content}{Skip to
content}\protect\hyperlink{site-index}{Skip to site index}

\href{https://www.nytimes3xbfgragh.onion/section/books/review}{Book
Review}

\href{https://myaccount.nytimes3xbfgragh.onion/auth/login?response_type=cookie\&client_id=vi}{}

\href{https://www.nytimes3xbfgragh.onion/section/todayspaper}{Today's
Paper}

\href{/section/books/review}{Book Review}\textbar{}March's Book Club
Pick: Dani Shapiro's New Memoir Uncovers a Life-Changing Family Secret

\url{https://nyti.ms/2RPtB2b}

\begin{itemize}
\item
\item
\item
\item
\item
\end{itemize}

Advertisement

\protect\hyperlink{after-top}{Continue reading the main story}

Supported by

\protect\hyperlink{after-sponsor}{Continue reading the main story}

Nonfiction

\hypertarget{marchs-book-club-pick-dani-shapiros-new-memoir-uncovers-a-life-changing-family-secret}{%
\section{March's Book Club Pick: Dani Shapiro's New Memoir Uncovers a
Life-Changing Family
Secret}\label{marchs-book-club-pick-dani-shapiros-new-memoir-uncovers-a-life-changing-family-secret}}

\includegraphics{https://static01.graylady3jvrrxbe.onion/images/2019/01/27/books/review/27Franklin3/merlin_149030637_41834e3a-c3db-4a91-903f-e86e1917c6e2-articleLarge.jpg?quality=75\&auto=webp\&disable=upscale}

Buy Book ▾

\begin{itemize}
\tightlist
\item
  \href{https://www.amazon.com/gp/search?index=books\&tag=NYTBSREV-20\&field-keywords=Inheritance\%3A+A+Memoir+of+Genealogy\%2C+Paternity\%2C+and+Love+Dani+Shapiro}{Amazon}
\item
  \href{https://du-gae-books-dot-nyt-du-prd.appspot.com/buy?title=Inheritance\%3A+A+Memoir+of+Genealogy\%2C+Paternity\%2C+and+Love\&author=Dani+Shapiro}{Apple
  Books}
\item
  \href{https://www.anrdoezrs.net/click-7990613-11819508?url=https\%3A\%2F\%2Fwww.barnesandnoble.com\%2Fw\%2F\%3Fean\%3D9781524732714}{Barnes
  and Noble}
\item
  \href{https://www.anrdoezrs.net/click-7990613-35140?url=https\%3A\%2F\%2Fwww.booksamillion.com\%2Fp\%2FInheritance\%253A\%2BA\%2BMemoir\%2Bof\%2BGenealogy\%252C\%2BPaternity\%252C\%2Band\%2BLove\%2FDani\%2BShapiro\%2F9781524732714}{Books-A-Million}
\item
  \href{https://bookshop.org/a/3546/9781524732714}{Bookshop}
\item
  \href{https://www.indiebound.org/book/9781524732714?aff=NYT}{Indiebound}
\end{itemize}

When you purchase an independently reviewed book through our site, we
earn an affiliate commission.

By Ruth Franklin

\begin{itemize}
\item
  Jan. 15, 2019
\item
  \begin{itemize}
  \item
  \item
  \item
  \item
  \item
  \end{itemize}
\end{itemize}

\textbf{INHERITANCE}\\
\textbf{A Memoir of Genealogy, Paternity, and Love}\\
By Dani Shapiro\\
249 pp. Alfred A. Knopf. \$24.95.

``You may discover things about yourself and/or your family members that
may be upsetting,'' warns the boilerplate legal language at 23andme.com,
the website of a company that analyzes samples from DNA testing kits.
Spitting into one of those test tubes a few years ago, I felt as if I
were perched on the edge of a void: Here was a moment when all the veils
had the potential to fall. At the time, my greatest fear was that I
might be genetically predisposed to Alzheimer's disease or breast
cancer. But we've all heard stories of people who discover, quite by
accident, that their family history isn't quite what they thought.

\includegraphics{https://static01.graylady3jvrrxbe.onion/images/2019/01/27/books/review/27Franklin4/merlin_149030640_b265d14b-bcf8-4b72-b70c-29bc992000e8-articleLarge.jpg?quality=75\&auto=webp\&disable=upscale}

The chance that such a thing might happen at all seems remote --- but
that it might happen to Dani Shapiro, a novelist and memoirist who has
devoted her life to telling stories about families and their secrets, is
even more incredible. Shapiro's 2010 memoir, ``Devotion,'' is the story
of her search for a new spirituality after becoming disillusioned with
the Orthodox Judaism of her childhood; she grew up steeped in the
history of her Eastern European ancestors and taught to take pride in
the accomplishments of her grandfather, a pillar of modern Orthodoxy.
But, as she recounts in her latest memoir, ``Inheritance,'' a few years
ago, with both her parents long dead, she took a DNA test and discovered
that she was only half Jewish --- and unrelated to the woman she had
always thought was her half sister.

{[}\emph{\href{https://www.nytimes3xbfgragh.onion/2019/01/10/books/serial-memoir-writer.html?action=click\&module=Well\&pgtype=Homepage\&section=Books}{Read
our piece about serial memoirists, including Shapiro, here}}{]}

Shapiro had long known that she was conceived in Philadelphia, at a
clinic for couples with fertility problems: ``Not a pretty story,'' in
her mother's words. The clinic was run by Edmond Farris, a doctor who
had developed a new method for pinpointing when a woman ovulated. When
the time was right, Shapiro's mother had told her, her father would rush
down from New York, where he worked on the stock exchange, and provide
sperm for artificial insemination. Shapiro had heard rumors that such
clinics used to ``mix sperm'' --- that is, the semen of men with low
sperm count would be combined with donor sperm to increase the chances
of pregnancy --- but she didn't give it more thought. Now she learns
that in those days, many sperm donors were medical students. A Twitter
acquaintance who calls herself a ``genealogy geek'' needs only a family
tree on Ancestry.com showing a first cousin previously unknown to
Shapiro and a few web searches to locate the man who turns out to be
Shapiro's biological father --- a decidedly non-Jewish doctor in Oregon
who went to medical school at Penn.

All this takes place within the first third of the book, so I'm not
giving much away. At any rate, the true drama of ``Inheritance'' is not
Shapiro's discovery of her father's identity but the meaning she makes
of it. In many ways, the knowledge comes as a relief. Her parents'
relationship was fraught; her mother suffered from borderline
personality disorder, and her father was depressive. She always felt out
of place in her birth family, as if on some level she knew she didn't
belong. Relatives, friends and strangers commented that she didn't look
Jewish; once, when she was a child, a family friend (who will eventually
be Jared Kushner's grandmother) ran a hand through her platinum hair and
remarked, chillingly: ``We could have used you in the ghetto, little
blondie. You could have gotten us bread from the Nazis.'' When Shapiro
comes upon a YouTube video of her biological father --- a man with her
features and coloring, who even gesticulates the same way she does ---
the resemblance is more than astonishing; it's consoling. ``I knew in a
place beyond thought that I was seeing the truth --- the answer to the
unanswerable questions I had been exploring all my life,'' she writes.

Image

The discovery that Shapiro carries a stranger's genes has profound
implications for every aspect of her life, from the photographs of
supposed relatives that line the walls of her house to the need to
revise her medical history. (``How could I explain that my father was no
longer deceased?'' she wonders at the doctor's office.) It also leads
her to investigate the early days of artificial insemination, in which
she finds more than a tinge of eugenics. Farris is quoted in an
interview as saying that he saw ``nothing wrong in trying to bring
children of fine quality into the world''; his donors were the ``best
material that Philadelphia medical schools can offer.'' Couples who used
donor sperm were advised to have sex before and after the insemination,
to intentionally introduce an element of ambiguity. It was simply
assumed that their children would never be told. No one seems to have
worried about those children growing up with inaccurate medical
histories, much less a pervasive sense of unease in their own skin.

Shapiro's account is beautifully written and deeply moving --- it
brought me to tears more than once. I couldn't help feeling unnerved,
though, by the strength of her conviction that blood will out, which
leads her uncomfortably close to genetic determinism. ``Our lifetime of
disconnection, finally explained,'' she writes of her lack of kinship
with the woman she believed to be her half sister. Donating sperm, she
believes, is ``the passing along of an essence that was inseparable from
personhood itself''; on a visit to the California Cryobank, the nation's
largest donor sperm repository, she wonders about the ``millions of
souls'' within its vials. But by all accounts, many children of sperm
(and egg) donors grow up fulfilled and content, nurtured by the love of
the parents who raise them and uninterested in seeking out their
biological relatives --- who, when found, often turn out to be a
disappointment. And for many children of unhappy families, genetic bonds
aren't sufficient to maintain connection to parents who are abusive or
neglectful.

``Neither of my two fathers could ever be entirely mine,'' Shapiro comes
to realize. Indeed, no one's parents can ever be entirely one's own;
they have histories and secrets of which we know nothing. And among the
mysteries of adulthood is the way parents and children, once apparently
inseparable, can part like amicable lovers: still fond, but no longer
close. As the song goes, it's love --- not genes --- that will keep us
together.

Advertisement

\protect\hyperlink{after-bottom}{Continue reading the main story}

\hypertarget{site-index}{%
\subsection{Site Index}\label{site-index}}

\hypertarget{site-information-navigation}{%
\subsection{Site Information
Navigation}\label{site-information-navigation}}

\begin{itemize}
\tightlist
\item
  \href{https://help.nytimes3xbfgragh.onion/hc/en-us/articles/115014792127-Copyright-notice}{©~2020~The
  New York Times Company}
\end{itemize}

\begin{itemize}
\tightlist
\item
  \href{https://www.nytco.com/}{NYTCo}
\item
  \href{https://help.nytimes3xbfgragh.onion/hc/en-us/articles/115015385887-Contact-Us}{Contact
  Us}
\item
  \href{https://www.nytco.com/careers/}{Work with us}
\item
  \href{https://nytmediakit.com/}{Advertise}
\item
  \href{http://www.tbrandstudio.com/}{T Brand Studio}
\item
  \href{https://www.nytimes3xbfgragh.onion/privacy/cookie-policy\#how-do-i-manage-trackers}{Your
  Ad Choices}
\item
  \href{https://www.nytimes3xbfgragh.onion/privacy}{Privacy}
\item
  \href{https://help.nytimes3xbfgragh.onion/hc/en-us/articles/115014893428-Terms-of-service}{Terms
  of Service}
\item
  \href{https://help.nytimes3xbfgragh.onion/hc/en-us/articles/115014893968-Terms-of-sale}{Terms
  of Sale}
\item
  \href{https://spiderbites.nytimes3xbfgragh.onion}{Site Map}
\item
  \href{https://help.nytimes3xbfgragh.onion/hc/en-us}{Help}
\item
  \href{https://www.nytimes3xbfgragh.onion/subscription?campaignId=37WXW}{Subscriptions}
\end{itemize}
