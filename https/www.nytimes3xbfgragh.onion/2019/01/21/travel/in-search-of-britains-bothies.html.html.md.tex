Sections

SEARCH

\protect\hyperlink{site-content}{Skip to
content}\protect\hyperlink{site-index}{Skip to site index}

\href{/section/travel}{Travel}\textbar{}In Britain, Enraptured by the
Wild, Lonely and Remote

\url{https://nyti.ms/2RY4vOt}

\begin{itemize}
\item
\item
\item
\item
\item
\item
\end{itemize}

\includegraphics{https://static01.graylady3jvrrxbe.onion/images/2018/12/02/travel/surfacing-bothies-1/surfacing-bothies-1-articleLarge-v2.jpg?quality=75\&auto=webp\&disable=upscale}

Surfacing

\hypertarget{in-britain-enraptured-by-the-wild-lonely-and-remote}{%
\section{In Britain, Enraptured by the Wild, Lonely and
Remote}\label{in-britain-enraptured-by-the-wild-lonely-and-remote}}

Rustic shelters called bothies ---~more than 100 of which are scattered
throughout England, Wales and Scotland --- are an indispensable, if
little-known, element of British hill culture.

The Hutchison Memorial Hut, colloquially called the Hutchie Hut,
illuminated by moonlight in Scotland's Cairngorms National
Park.Credit...Stephen Hiltner/The New York Times

Supported by

\protect\hyperlink{after-sponsor}{Continue reading the main story}

\href{https://www.nytimes3xbfgragh.onion/by/stephen-hiltner}{\includegraphics{https://static01.graylady3jvrrxbe.onion/images/2018/06/13/multimedia/author-stephen-hiltner/author-stephen-hiltner-thumbLarge-v2.jpg}}

By \href{https://www.nytimes3xbfgragh.onion/by/stephen-hiltner}{Stephen
Hiltner}

\begin{itemize}
\item
  Jan. 21, 2019
\item
  \begin{itemize}
  \item
  \item
  \item
  \item
  \item
  \item
  \end{itemize}
\end{itemize}

By the time the tiny hut came into view, nestled high in a corrie in
Scotland's 1,748-square-mile Cairngorms National Park, I'd trekked for
nearly nine miles, three of which, regrettably, I'd had to navigate
after nightfall. The hike, through a broad valley in the Eastern
Highlands called Glen Derry, carried me past groves of Scots pines and
over a series of streams, some of which, lined with slick
steppingstones, made for precarious crossings. All the while, two rows
of smooth, eroded mountain peaks enclosed me in an amphitheater of muted
colors: hazel-hued heather, golden grasses. Though much of my walk was
solitary, the flickering glow in the hut's main window, I knew, meant
I'd have some company for the night and the warmth of a fire to greet
me.

My overnight home, the Hutchison Memorial Hut, colloquially called the
Hutchie Hut, which I visited in late October, is one of more than 100
rustic shelters scattered throughout England, Wales and Scotland that
are frequented by a motley assortment of outdoor adventurers. Left
unlocked, free to use and with most offering little more than a roof,
four walls and perhaps a small wood-burning stove, the buildings, called
bothies (rhymes with ``frothy''), are an indispensable --- if for many
years underground --- element of British hill culture.

\includegraphics{https://static01.graylady3jvrrxbe.onion/images/2019/01/05/autossell/DSC_4409-1/DSC_4409-1-superJumbo.jpg}

A vast majority of bothies are repurposed structures --- crofters'
homes, shepherds' huts, mining outbuildings --- that have been salvaged
from various states of disrepair by the
\href{https://www.mountainbothies.org.uk/}{Mountain Bothies
Association}, a charitable organization founded in 1965 whose aim is
``to maintain simple shelters in remote country for the use and benefit
of all who love wild and lonely places.'' Some, like Warnscale Head in
England's Lake District, date to the 1700s. Collectively, since they
came into recreational use in the 1930s as weekend getaways (sometimes
used clandestinely) for working-class laborers, bothies have given rise
to a unique culture that values communal respect for fellow visitors,
for the bothies themselves and for the land on which they're situated.

But bothy culture, some longtime proponents fear, is imperiled by a
generation unaccustomed to shrewdly guarded secrets. Map coordinates for
the often hard-to-find dwellings, once dispersed only among hiking
insiders, are now available openly on the internet. Popular hashtags
have helped create something of a buzz on Instagram, where bothies are
sometimes presented, misguidedly, as an alternative to Airbnb rentals.
(The\href{https://www.mountainbothies.org.uk/bothies/bothy-code/}{bothy
code} unequivocally prohibits the use of bothies for commercial
purposes, and discourages their use by large groups.) A hugely popular
and impressively researched guide,
\href{https://thescottishbothybible.com/product/the-scottish-bothy-bible/}{``The
Scottish Bothy Bible,''} published in 2017, lines shelves in stores
throughout the U.K., the first of many bothy guides to achieve a kind of
mainstream success. It, too, has increased foot traffic.

\includegraphics{https://static01.graylady3jvrrxbe.onion/images/2018/12/05/arts/00surfacing-bothies9/merlin_147770925_bc58dade-6116-4ad2-8a84-bf452a399b94-articleLarge.jpg?quality=75\&auto=webp\&disable=upscale}

Image

Warnscale Head at night. Since bothies are often built with local
stones, they're easily camouflaged in their surrounding
landscapes.Credit...Stephen Hiltner/The New York Times

Bothies, I should mention at the outset, are not for everyone. Over the
course of two weeks, while hiking some 200 miles and visiting 20 of them
(12 of which I slept overnight in), I battled sopping boots, squally
winds, dispiriting cold, blinding rain and seemingly impenetrable bogs
only to reach dwellings that, by most modern standards, are ill-suited
for human occupancy. The interiors are often dark and dank, with cold
stone floors that double as stiff sleeping platforms. With few
exceptions, toilets consist of the great outdoors, along with a small
spade and posted instructions to deposit one's waste a considerate
distance from the building. More than once I awoke to the sound of mice
skittering near my head.

But to me and many others, the discomforts are a barely distinguishable
blip --- and, dare I say, often an ascetic pleasure --- in an otherwise
rapturous experience. Bothies are a portal. In all their understated
glory, bothies allow for prolonged access to Britain's rugged,
restorative and majestical hidden corners, places that might otherwise
prove unforgiving or impractical as day-hike destinations for the casual
explorer.

Image

The path to Ruigh Aiteachain, a bothy in the Eastern
Highlands.Credit...Stephen Hiltner/The New York Times

When, cold and exhausted, I finally crossed the threshold of the Hutchie
Hut, I was greeted by three strangers: Tom and Lee, two undergraduate
students at the University of St. Andrews who were perched on the tiny
room's tiny sleeping platform; and Yakub, a fellow journalist from
Manchester who'd made a pallet for himself on the floor. Seeing that
their fuel was low, I offered up my bundle of firewood and my small bag
of coal, then unfolded my sleeping bag on the floor and gratefully
accepted a swig of Tom's whiskey. Within minutes, buoyed by tales of our
sundry mishaps on the way up the mountain, the four of us were strangers
no more.

Such encounters were common on my trip, during which I crossed paths
with a few dozen fellow adventurers: climbers, environmentalists,
families, solitary hikers, groups of friends, young and old. Some were
drawn by the promise of nighttime carousing, others by a deep connection
to the land. Lynn Munro, who I met at Coiremor and Magoo's, two adjacent
bothies in the northern Highlands, called the surrounding area her
``favorite square kilometer in the whole world.'' With two friends, Tom
and Francis, she'd ventured out for a restorative trip to a place she's
visited on and off for her entire life. I fell asleep that night to the
sound of the trio singing harmonies beside the fire.

Image

Three friends --- Francis, Tom and Lynn --- sit near a fire at Magoo's,
a bothy in the Northern Highlands. Lynn called the surrounding area her
``favorite square kilometer in the whole world.''Credit...Stephen
Hiltner/The New York Times

I also met a number of ``Munro baggers,'' climbers in search of the 282
Scottish peaks that exceed 3,000 feet, for whom certain bothies serve as
convenient base camps. After a particularly grueling hike to Ben Alder
Cottage, a remote bothy in the Central Highlands, I pushed open a
brightly painted blue door to find a couple sets of gear sitting
unaccompanied near the stove. Moments later, two chipper climbers bolted
in, both wearing shorts, despite the driving rain and near-freezing
temperatures. Thomas and Benedict, students from the University of
Edinburgh, had ventured out --- by train, then bicycle --- to make an
attempt at summiting Ben Alder, the 3,766-foot mountain that loomed
nearby.

Image

ShenavallCredit...Stephen Hiltner/The New York Times

Image

SchoolhouseCredit...Stephen Hiltner/The New York Times

Image

SuileagCredit...Stephen Hiltner/The New York Times

At Ruigh Aiteachain, far and away the most luxurious bothy I visited, I
shared a coffee and a meal with Lyndsay Bryce, who for many years has
volunteered as the building's caretaker. Like most of Britain's bothies,
Ruigh Aiteachain is part of the network maintained by the Mountain
Bothies Association. (The organization has around 3,800 members. Of the
104 bothies under the its care, 83 are in Scotland, 12 are in England
and nine are in Wales.)

\includegraphics{https://static01.graylady3jvrrxbe.onion/images/2019/01/05/autossell/DSC_2881-1/DSC_2881-1-superJumbo.jpg}

In M.B.A. parlance, Mr. Bryce is a ``maintenance organizer'' and a
liaison, in a sense, between the
\href{http://www.glenfeshie.scot/Glenfeshie/Glenfeshie_Estate_Welcome.html}{Glenfeshie
estate}, which owns Ruigh Aiteachain and allows for its communal use
(and has
\href{https://jamescarron.wordpress.com/features/the-new-ruigh-aiteachain/}{invested
significantly in its renovation}), and the M.B.A., which helps provide
for its maintenance. (Each M.B.A. bothy has at least one dedicated
maintenance organizer, and most bothies, while publicly accessible, are
privately owned.) Mr. Bryce spends several days a month at the bothy, he
said, completing minor repairs, arranging for routine maintenance and
monitoring its use.

On my trek to the Burnmouth Cottage, a far-flung bothy on the Isle of
Hoy (two separate ferries and a six-mile walk were required to reach
it), I ran into Jeff Clark, who was out for a walk with his dog, Skye.
One of only five full-time residents of Rackwick, a tiny village on the
island, Mr. Clark was the caretaker of Burnmouth from around 2000 to
2010. (Unlike most bothies, Burnmouth Cottage is owned communally, by
the local Hoy Trust; it has no affiliation with the M.B.A.)

Image

Jeff Clark, a resident of Rackwick, and his very good pup, Skye.
Burnmouth Cottage sits in the background.Credit...Stephen Hiltner/The
New York Times

``People visit the bothy from all over the world --- a lot of them to
climb the Old Man,'' he said, gesturing in the direction of the Old Man
of Hoy, a nearby
\href{https://hoyorkney.com/attractions/hoy-geography/old-man-of-hoy/}{450-foot
sea stack} that balances, unsettlingly, just off the coast.

``It's nice to know that they can come and enjoy the area,'' he said,
adding that many of the Americans who visit can't believe it's free.

\includegraphics{https://static01.graylady3jvrrxbe.onion/images/2018/11/05/autossell/bothy17/bothy17-superJumbo.jpg}

Apart from the communal revelry that accompanies the sometimes crowded
dwellings, bothy culture can convey itself in quiet ways, too. Notebooks
provided by M.B.A. volunteers offer a chance for visitors to log entries
about their stays, and often contain helpful advice. ``Top tip: Peat
burns best in palm-size bits,'' read one entry at a far-flung bothy
called Strathchailleach (formerly a hermit's home), where the only
locally available fuel comes in the form of a nearby peat deposit.
``Check for ticks,'' read an entry at the isolated Ollisdal bothy,
tucked away on the Isle of Skye's Duirinish Peninsula. (Sure enough,
when I unlaced my boots to look, I found four of the tiny critters
inching up the bottom of my leg. I brushed them off.)

Flittingford, one of the newest M.B.A. projects, was in complete
disrepair --- its roof long ago collapsed, its gables crumbling, its
ruins fully hidden by a grove of fir trees --- when it was rediscovered
during a recent round of timber harvesting. Now, visitors to the bothy,
newly renovated and opened in 2017, can quietly commune with the memory
of R. Robson and J. Proudlock, two local forestry workers who, 60 years
ago, etched their names into the building's cement window frame.

Image

Flittingford, a bothy in Northern England.Credit...Stephen Hiltner/The
New York Times

Image

Credit...Stephen Hiltner/The New York Times

There's no doubt that Britain's hills, lakes and heaths make for
perpetually awe-inspiring settings. But, particularly as one presses
ever northward into the Scottish highlands, the moorlands can also make
for a challenging and sometimes perilous landscape. The mix of rain and
gales can be blinding. Trails can consist of little more than slightly
trampled grass, easily mistaken for the paths left by wandering rivulets
or grazing sheep. While traipsing through a moor during or after a hard
rain, each step becomes something of a calculation: Which foothold is
least likely to give way, leading to a boot filled with boggy water?
Over time, with practice, one's calculus improves. But despite my hiking
experience, I never felt 100 percent certain that what looked like firm,
dry land wouldn't give way and envelop my entire leg --- which,
inevitably, it did.

Image

On the trail to Suileag, a bothy in the Northern Highlands. Suilven, one
of Britain's more distinctive mountains, is visible in the distance.
(With a peak of 2,398 feet, it falls short of Munro
status.)Credit...Stephen Hiltner/The New York Times

And therein lies what may ultimately serve as a saving grace against the
threat of overcrowding. Reaching Britain's truly spectacular bothies
requires a good deal of effort and often a little risk. Those driven by
the thrills of Instagram ``likes'' and free accommodation will, by and
large, find themselves insufficiently motivated for the trek. And those
who do value the restorative spirit of the places are likely to become
ambassadors of a continuing and evolving tradition.

\includegraphics{https://static01.graylady3jvrrxbe.onion/images/2018/12/07/travel/360-bothies3-still/360-bothies3-still-superJumbo.jpg}

Bothy culture, in other words, may prove more resilient than some would
have you believe.

All of which helps to explain an endearing entry I spotted in the
Hutchie Hut's bothy book, logged in mid-September. ``Last here 50 years
ago,'' begins the brief note from a hillwalker named GW. ``Nothing much
has changed.''

Image

``Last here 50 years ago,'' begins a brief entry in the Hutchie Hut's
bothy book. ``Nothing much has changed.''Credit...Stephen Hiltner/The
New York Times

\emph{An interactive map with pins for all 20 of the bothies visited for
this story is available}
\href{https://www.google.com/maps/d/viewer?mid=1Sh6WFFIyXw-KMxPUjV5q3xhFZTTFprTZ\&ll=56.07544038548426\%2C-1.0472777259095665\&z=6}{\emph{here}}\emph{.}

\emph{Stephen Hiltner reported this article for the}
\href{https://www.nytimes3xbfgragh.onion/series/surfacing-subculture-communities}{\emph{Surfacing}}
\emph{series.}

\emph{Follow} \href{https://twitter.com/nytimestravel}{\emph{NY Times
Travel on Twitter}}\emph{,}
\href{https://www.instagram.com/nytimestravel/}{\emph{Instagram}}
\emph{and}
\href{https://www.facebookcorewwwi.onion/nytimestravel/}{\emph{Facebook}}\emph{.}
\href{https://www.nytimes3xbfgragh.onion/newsletters/traveldispatch?module=inline}{\emph{Get
weekly updates from our Travel Dispatch newsletter, with tips on
traveling smarter, destination coverage and photos from all over the
world.}}

Advertisement

\protect\hyperlink{after-bottom}{Continue reading the main story}

\hypertarget{site-index}{%
\subsection{Site Index}\label{site-index}}

\hypertarget{site-information-navigation}{%
\subsection{Site Information
Navigation}\label{site-information-navigation}}

\begin{itemize}
\tightlist
\item
  \href{https://help.nytimes3xbfgragh.onion/hc/en-us/articles/115014792127-Copyright-notice}{©~2020~The
  New York Times Company}
\end{itemize}

\begin{itemize}
\tightlist
\item
  \href{https://www.nytco.com/}{NYTCo}
\item
  \href{https://help.nytimes3xbfgragh.onion/hc/en-us/articles/115015385887-Contact-Us}{Contact
  Us}
\item
  \href{https://www.nytco.com/careers/}{Work with us}
\item
  \href{https://nytmediakit.com/}{Advertise}
\item
  \href{http://www.tbrandstudio.com/}{T Brand Studio}
\item
  \href{https://www.nytimes3xbfgragh.onion/privacy/cookie-policy\#how-do-i-manage-trackers}{Your
  Ad Choices}
\item
  \href{https://www.nytimes3xbfgragh.onion/privacy}{Privacy}
\item
  \href{https://help.nytimes3xbfgragh.onion/hc/en-us/articles/115014893428-Terms-of-service}{Terms
  of Service}
\item
  \href{https://help.nytimes3xbfgragh.onion/hc/en-us/articles/115014893968-Terms-of-sale}{Terms
  of Sale}
\item
  \href{https://spiderbites.nytimes3xbfgragh.onion}{Site Map}
\item
  \href{https://help.nytimes3xbfgragh.onion/hc/en-us}{Help}
\item
  \href{https://www.nytimes3xbfgragh.onion/subscription?campaignId=37WXW}{Subscriptions}
\end{itemize}
