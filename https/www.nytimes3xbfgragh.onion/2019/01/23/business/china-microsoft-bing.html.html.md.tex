Sections

SEARCH

\protect\hyperlink{site-content}{Skip to
content}\protect\hyperlink{site-index}{Skip to site index}

\href{https://www.nytimes3xbfgragh.onion/section/business}{Business}

\href{https://myaccount.nytimes3xbfgragh.onion/auth/login?response_type=cookie\&client_id=vi}{}

\href{https://www.nytimes3xbfgragh.onion/section/todayspaper}{Today's
Paper}

\href{/section/business}{Business}\textbar{}China Appears to Block
Microsoft's Bing as Censorship Intensifies

\href{https://nyti.ms/2S1MTBw}{https://nyti.ms/2S1MTBw}

\begin{itemize}
\item
\item
\item
\item
\item
\end{itemize}

Advertisement

\protect\hyperlink{after-top}{Continue reading the main story}

Supported by

\protect\hyperlink{after-sponsor}{Continue reading the main story}

\hypertarget{china-appears-to-block-microsofts-bing-as-censorship-intensifies}{%
\section{China Appears to Block Microsoft's Bing as Censorship
Intensifies}\label{china-appears-to-block-microsofts-bing-as-censorship-intensifies}}

\includegraphics{https://static01.graylady3jvrrxbe.onion/images/2019/01/24/business/00bing/00bing-articleLarge.jpg?quality=75\&auto=webp\&disable=upscale}

By \href{https://www.nytimes3xbfgragh.onion/by/paul-mozur}{Paul Mozur}
and \href{https://www.nytimes3xbfgragh.onion/by/karen-weise}{Karen
Weise}

\begin{itemize}
\item
  Jan. 23, 2019
\item
  \begin{itemize}
  \item
  \item
  \item
  \item
  \item
  \end{itemize}
\end{itemize}

\href{https://cn.nytimes3xbfgragh.onion/business/20190124/china-microsoft-bing/}{阅读简体中文版}\href{https://cn.nytimes3xbfgragh.onion/business/20190124/china-microsoft-bing/zh-hant/}{閱讀繁體中文版}

SHANGHAI --- Under China's president, Xi Jinping, the last vestiges of
the global internet have slowly disappeared from an online world that
had already shut out Twitter, Google and Facebook.

Now one of the last survivors, Microsoft's Bing search engine, appears
to have joined them --- even though the American company already censors
its results in China.

The Chinese government appeared to block the search engine on Wednesday,
in what would be a startling renunciation of more than a decade of
efforts by Microsoft to engage with Beijing to make its products
available. If the block proves to be permanent, it would suggest that
Western companies can do little to persuade China
\href{https://www.nytimes3xbfgragh.onion/2018/03/23/technology/trump-china-tariffs-tech-cold-war.html}{to
give them access to what has become the world's largest internet market
by users}, especially at a time of increased trade and economic tensions
with the United States.

The Redmond, Wash., company has cooperated with local companies to
provide its Windows and cloud services to win acceptance by the Chinese
government. Its long-established research and development center has
turned out valuable products and launched the careers of a generation of
artificial-intelligence experts who have started important new companies
in China.

Beijing has carried out several waves of increasingly intense crackdowns
on internet freedoms as the Communist Party has cemented its control
over more aspects of Chinese life. That includes cracking down on
foreign internet products, including blocks on Instagram and WhatsApp in
recent years.

Lately, the Chinese authorities have
\href{https://www.nytimes3xbfgragh.onion/2019/01/10/business/china-twitter-censorship-online.html}{questioned
or detained activists for posting on Twitter}, even though the vast
majority of people in China can't access the microblogging service. (The
activists generally posted on Twitter via special software that can
circumvent China's censors.)

Chinese officials disclose few specifics about their censorship
practices, and Bing's status as of Thursday was not entirely clear. The
Cyberspace Administration of China did not immediately respond to a
request for comment.

In a statement, Microsoft said, ``We've confirmed that Bing is currently
inaccessible in China and are engaged to determine next steps.''

\href{http://greatfire.org/}{Greatfire.org}, a group that tracks what
sites are blocked in China, said the site appeared to be inaccessible in
parts of the country but reachable in others. China's blockages often
take time to spread nationwide, though in the past some services have
been blocked in some places only to be restored later.

With Bing, Microsoft tried to play by China's rules. For example, a
search for the Dalai Lama, the religious leader, would turn up state
media accounts within China that accused him of stirring up hatred and
separatism. Outside the country, it would point to sites like Wikipedia.

Other searches, like for Tiananmen Square or the Falun Gong religious
group, were similarly scrubbed, though over the years users reported
that using coded language could help turn up posts about some topics
that were generally controlled.

Blocking Bing would brick over one of the last holes in a wall of online
filters that has
\href{https://www.nytimes3xbfgragh.onion/2018/08/06/technology/china-generation-blocked-internet.html}{isolated
China's internet from the rest of the world}. Although not widely used
in China, Bing has remained an option of last resort for some in China
looking for an alternative to the dominant local search engine, Baidu.
While it continues to dominate search traffic in China, Baidu has been
at the center of complaints about poor search results and
\href{https://www.nytimes3xbfgragh.onion/2016/05/04/world/asia/china-baidu-investigation-student-cancer.html}{advertisements
for questionable medical treatments}.

Earlier this week, a former journalist, Fang Kecheng, accused Baidu of
largely returning search results that were links to its own products
instead of those from external sites. The accusation, which Mr. Fang
posted on social media with the headline ``Baidu the Search Engine is
Dead,'' went viral in China.

Baidu said in a statement that less than 10 percent of its search
results included one specific Baidu product that Mr. Fang had singled
out, and that its practices of using its own products in search results
helped speed up download times.

In an interview, Mr. Fang said the Chinese internet was developing into
a series of walled gardens, rather than the sprawling forum for ideas
that makes online life appealing to many, thanks to censorship and to
the rise of big Chinese internet companies like Tencent and Bytedance
that dominate the online experience on mobile phones. Blocking Bing
would only make it worse.

``Bing compromised in order to have a Chinese version to get into the
country,'' said Mr. Fang, a doctoral candidate at the Annenberg School
for Communication at University of Pennsylvania. ``It would be pathetic
if even this can't exist. We have one less alternative.''

China has long been a difficult market for Microsoft. For years, it
struggled to contend with widespread piracy of its Windows and Office
software.

In July 2014, four of its offices in China were stormed by officials who
questioned executives, copied contracts and records and downloaded data
from the company's servers. What was described as an antitrust inquiry
was spurred by Microsoft's decision to end support for older Windows
software to encourage users to switch to newer versions that were more
difficult to pirate, according to analysts. The withdrawal of support
for a still widely used, if dated, version of Windows
\href{https://www.nytimes3xbfgragh.onion/2016/01/06/business/international/microsoft-china-antitrust-inquiry.html}{only
underscored the country's reliance} on foreign software.

In 2017, to ensure state support for Windows, the company partnered with
a state-run firm to produce a government-approved version of its Windows
10 software. The firm, Chinese Electronics Technology Corporation, makes
electronics for the Chinese military and is a major vendor of
surveillance technology in Xinjiang, where the government has thrown
hundreds of thousands of Uighurs, a local Muslim minority, into
re-education camps.

Beijing has insisted on closer scrutiny of software used by government
agencies and companies since Edward Snowden, a former American
government consultant, revealed that United States intelligence agencies
used American technology for hacking purposes.

Despite the difficulties, foreign companies still hold out hope for
access to China, which with its size and its thriving mobile phone
culture represents a potentially vast market. LinkedIn and Airbnb still
have businesses there, though they take pains to comply with local laws.
LinkedIn, which Microsoft bought in 2016, censors content in China, and
Airbnb coordinates with local authorities to provide access about who is
staying at its listings.

But trade tensions between China and the United States are only adding
to the difficulties. Many American companies, along with Trump
administration officials, accuse China of systematically forcing
American companies to hand over their intellectual property to local
partners or to government officials in the name of national security. So
far, China has stopped short of boycotting American products, which
could escalate the trade conflict and damage the Chinese economy.

Baidu's falling reputation led to some support among China's
intelligentsia for
\href{https://www.nytimes3xbfgragh.onion/2018/08/22/technology/google-china-conventionality.html}{a
re-entry by Google} into the market with a censored search engine
\href{https://www.nytimes3xbfgragh.onion/2018/08/01/technology/china-google-censored-search-engine.html}{internally
code-named Project Dragonfly}. Its plans led to widespread criticism
\href{https://www.nytimes3xbfgragh.onion/2018/08/16/technology/google-employees-protest-search-censored-china.html}{from
Google's own employees}. Some Chinese activists also protested, arguing
a censored search would be tantamount to complicity with Chinese
censorship. Google's chief executive, Sundar Pichai, has said it is
``not close to launching a search product in China.''

Mr. Fang, the former journalist, said he had mixed feelings about a
Google re-entry, noting that it would be good for Chinese who want a
better search engine, but also that it would normalize censorship.

``I feel people in the West can say from their point of view that they
are totally against Google coming back, as they have nothing to do with
China,'' he said. ``But as someone who truly wishes that Chinese are
able to get better-quality information, I can't 100 percent stand
against it.''

Advertisement

\protect\hyperlink{after-bottom}{Continue reading the main story}

\hypertarget{site-index}{%
\subsection{Site Index}\label{site-index}}

\hypertarget{site-information-navigation}{%
\subsection{Site Information
Navigation}\label{site-information-navigation}}

\begin{itemize}
\tightlist
\item
  \href{https://help.nytimes3xbfgragh.onion/hc/en-us/articles/115014792127-Copyright-notice}{©~2020~The
  New York Times Company}
\end{itemize}

\begin{itemize}
\tightlist
\item
  \href{https://www.nytco.com/}{NYTCo}
\item
  \href{https://help.nytimes3xbfgragh.onion/hc/en-us/articles/115015385887-Contact-Us}{Contact
  Us}
\item
  \href{https://www.nytco.com/careers/}{Work with us}
\item
  \href{https://nytmediakit.com/}{Advertise}
\item
  \href{http://www.tbrandstudio.com/}{T Brand Studio}
\item
  \href{https://www.nytimes3xbfgragh.onion/privacy/cookie-policy\#how-do-i-manage-trackers}{Your
  Ad Choices}
\item
  \href{https://www.nytimes3xbfgragh.onion/privacy}{Privacy}
\item
  \href{https://help.nytimes3xbfgragh.onion/hc/en-us/articles/115014893428-Terms-of-service}{Terms
  of Service}
\item
  \href{https://help.nytimes3xbfgragh.onion/hc/en-us/articles/115014893968-Terms-of-sale}{Terms
  of Sale}
\item
  \href{https://spiderbites.nytimes3xbfgragh.onion}{Site Map}
\item
  \href{https://help.nytimes3xbfgragh.onion/hc/en-us}{Help}
\item
  \href{https://www.nytimes3xbfgragh.onion/subscription?campaignId=37WXW}{Subscriptions}
\end{itemize}
