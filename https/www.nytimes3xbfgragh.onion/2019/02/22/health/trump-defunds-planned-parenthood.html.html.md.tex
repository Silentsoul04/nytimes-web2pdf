Sections

SEARCH

\protect\hyperlink{site-content}{Skip to
content}\protect\hyperlink{site-index}{Skip to site index}

\href{https://www.nytimes3xbfgragh.onion/section/health}{Health}

\href{https://myaccount.nytimes3xbfgragh.onion/auth/login?response_type=cookie\&client_id=vi}{}

\href{https://www.nytimes3xbfgragh.onion/section/todayspaper}{Today's
Paper}

\href{/section/health}{Health}\textbar{}Trump Administration Blocks
Funds for Planned Parenthood and Others Over Abortion Referrals

\url{https://nyti.ms/2BOeX21}

\begin{itemize}
\item
\item
\item
\item
\item
\end{itemize}

Advertisement

\protect\hyperlink{after-top}{Continue reading the main story}

Supported by

\protect\hyperlink{after-sponsor}{Continue reading the main story}

\hypertarget{trump-administration-blocks-funds-for-planned-parenthood-and-others-over-abortion-referrals}{%
\section{Trump Administration Blocks Funds for Planned Parenthood and
Others Over Abortion
Referrals}\label{trump-administration-blocks-funds-for-planned-parenthood-and-others-over-abortion-referrals}}

The new rule would steer federal family planning funds under Title X to
anti-abortion and faith based groups.

\includegraphics{https://static01.graylady3jvrrxbe.onion/images/2019/02/23/science/23ABORTION/merlin_147967233_bfce11fe-553d-4094-a3b9-d2873f3dca93-articleLarge.jpg?quality=75\&auto=webp\&disable=upscale}

\href{https://www.nytimes3xbfgragh.onion/by/pam-belluck}{\includegraphics{https://static01.graylady3jvrrxbe.onion/images/2018/02/16/multimedia/author-pam-belluck/author-pam-belluck-thumbLarge-v2.png}}

By \href{https://www.nytimes3xbfgragh.onion/by/pam-belluck}{Pam Belluck}

\begin{itemize}
\item
  Feb. 22, 2019
\item
  \begin{itemize}
  \item
  \item
  \item
  \item
  \item
  \end{itemize}
\end{itemize}

The Trump
administration\href{https://www.hhs.gov/opa/sites/default/files/title-x-notice-of-final-rule.pdf}{announced
on Friday that it will bar organizations that provide abortion referrals
from receiving federal family planning money}, a step that could strip
millions of dollars from Planned Parenthood and direct it toward
religiously-based, anti-abortion groups.

The new federal rule is almost certain to be challenged in court.
Clinics will be able to talk to patients about abortion, but not where
they can get one. And clinics will no longer have to counsel women on
all reproductive options, including abortion, a change that will make
anti-abortion providers eligible for funding.

The rule, which has been expected for months, is the most recent step by
the Trump administration to shift the direction of federal health
programs in a conservative direction. The administration has expanded
the ability of employers to claim religious or moral objections to the
Affordable Care Act's requirement that they offer employees insurance
coverage for contraception. It has channeled funding for teen pregnancy
prevention programs and family planning grants into programs that
emphasize sexual abstinence over contraception.

Some of these changes are being challenged in lawsuits by groups that
support reproductive rights, but the new policies have broad support
among evangelicals, who are a big part of the president's political
base.

The rule announced on Friday is not a wholesale defunding of Planned
Parenthood, a long-held goal of conservatives. Organizations receiving
money through the federal family planning program, called Title X, will
still be able to perform abortions but will have to do so in a separate
facility from their other operations and adhere to the new requirement
that they not refer patients to it.

Organizations that receive federal funds have already been prohibited
for years from using that money to finance abortion services. The new
rule goes a step further by ordering them to keep separate books for
their abortion operations.

Many women's organizations said the new requirements will interfere with
health providers' responsibilities to fully counsel patients about
reproductive health.

``Trump's domestic gag rule harms women in more ways than one,''
Stephanie Schriock, president of EMILY'S List, said in a statement. ``It
effectively dismantles Title X, forces doctors to lie and forbids them
from referring their patients for abortion, and prevents women from
being able to access Planned Parenthood's services.''

Anti-abortion organizations heralded the regulation as a long-sought
victory. ``The finalized `protect life rule' draws a bright line between
abortion and family planning programs,'' Tony Perkins, president of the
Family Research Council, said in a statement, adding that the rule will
loosen Planned Parenthood's ``hold on tens of millions of tax dollars.''

\textbf{\emph{{[}}\href{http://on.fb.me/1paTQ1h}{\emph{Like the Science
Times page on Facebook.}}} ****** \emph{\textbar{} Sign up for the}
\textbf{\href{http://nyti.ms/1MbHaRU}{\emph{Science Times
newsletter.}}\emph{{]}}}

Title X provides \$286 million in funding for programs that provide
services like birth control, screening for breast cancer and cervical
cancer and screening and treatment for sexually transmitted diseases.
These programs serve about 4 million patients each year, many of them
poor, at more than 4,000 clinics. About 40 percent of those clinics are
operated by Planned Parenthood, which receives close to \$60 million
through the family planning program each year.

``In many parts of the country, Planned Parenthood is the only provider
who participates in the program,'' said Dr. Leana Wen, president of the
Planned Parenthood Federation of America.

She added: ``Patients expect their doctors to speak honestly with them,
to answer their questions, to help them in their time of need. Imagine
if the Trump administration prevented doctors from talking to our
patients with diabetes about insulin. It would never happen.
Reproductive health care should be no different.''

The new rule is not the first time a Republican administration has tried
to withhold family planning funds from organizations that provide
abortion counseling or services. In 1988, President Ronald Reagan
\href{https://www.nytimes3xbfgragh.onion/1988/01/30/us/reagan-bars-mention-of-abortion-at-clinics-receiving-us-money.html}{barred
clinics from not only referring patients for abortions, but also from
counseling them about abortions}.

That rule was challenged all the way to the Supreme Court, which upheld
it in 1991, during the administration of President George H.W. Bush. But
the Bush administration did not implement it and in 1993 Bill Clinton
was sworn in as president and eliminated the rule.

Conservative lawmakers hailed the new rule as a long overdue move back
toward the Reagan policy, and some were especially pleased by the
prospect that anti-abortion health providers could now be eligible for
federal family planning funds.

``Importantly, faith-based health organizations will no longer be forced
to compromise their pro-life principles to receive government funding,''
said Rep. Steve Scalise, the House Republican Whip.

But several medical organizations predicted that the new rule would
ultimately leave large numbers of patients, especially low-income and
minority women, without access to basic care.

Dr. Niva Lubin-Johnson, president of the National Medical Association,
which represents African-American physicians and their patients, said
that if Planned Parenthood loses the funding, other providers supported
by the program would have to increase their caseloads by an average of
70 percent to care for the 1.6 million people who currently receive such
services through Planned Parenthood.

``Many providers have already said they would be unable to fill this
gap,'' Dr. Lubin-Johnson said. ``This rule will have dire and
disproportionate consequences for African-American patients, who make up
22 percent of people who access health care through Title X.''

Most of the changes required by the new regulation will be phased in
beginning 60 days after it is published in the Federal Register.
Compliance with the financial separation requirement takes effect 120
days after publication and clinics have a year to comply with the
physical separation requirements.

\emph{Robert Pear contributed reporting from Washington.}

Advertisement

\protect\hyperlink{after-bottom}{Continue reading the main story}

\hypertarget{site-index}{%
\subsection{Site Index}\label{site-index}}

\hypertarget{site-information-navigation}{%
\subsection{Site Information
Navigation}\label{site-information-navigation}}

\begin{itemize}
\tightlist
\item
  \href{https://help.nytimes3xbfgragh.onion/hc/en-us/articles/115014792127-Copyright-notice}{©~2020~The
  New York Times Company}
\end{itemize}

\begin{itemize}
\tightlist
\item
  \href{https://www.nytco.com/}{NYTCo}
\item
  \href{https://help.nytimes3xbfgragh.onion/hc/en-us/articles/115015385887-Contact-Us}{Contact
  Us}
\item
  \href{https://www.nytco.com/careers/}{Work with us}
\item
  \href{https://nytmediakit.com/}{Advertise}
\item
  \href{http://www.tbrandstudio.com/}{T Brand Studio}
\item
  \href{https://www.nytimes3xbfgragh.onion/privacy/cookie-policy\#how-do-i-manage-trackers}{Your
  Ad Choices}
\item
  \href{https://www.nytimes3xbfgragh.onion/privacy}{Privacy}
\item
  \href{https://help.nytimes3xbfgragh.onion/hc/en-us/articles/115014893428-Terms-of-service}{Terms
  of Service}
\item
  \href{https://help.nytimes3xbfgragh.onion/hc/en-us/articles/115014893968-Terms-of-sale}{Terms
  of Sale}
\item
  \href{https://spiderbites.nytimes3xbfgragh.onion}{Site Map}
\item
  \href{https://help.nytimes3xbfgragh.onion/hc/en-us}{Help}
\item
  \href{https://www.nytimes3xbfgragh.onion/subscription?campaignId=37WXW}{Subscriptions}
\end{itemize}
