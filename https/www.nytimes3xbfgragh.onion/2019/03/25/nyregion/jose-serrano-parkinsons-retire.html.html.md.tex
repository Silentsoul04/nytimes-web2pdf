Sections

SEARCH

\protect\hyperlink{site-content}{Skip to
content}\protect\hyperlink{site-index}{Skip to site index}

\href{https://www.nytimes3xbfgragh.onion/section/nyregion}{New York}

\href{https://myaccount.nytimes3xbfgragh.onion/auth/login?response_type=cookie\&client_id=vi}{}

\href{https://www.nytimes3xbfgragh.onion/section/todayspaper}{Today's
Paper}

\href{/section/nyregion}{New York}\textbar{}Representative Serrano of
the Bronx to Retire, Potentially Opening Seat for Younger Progressive

\url{https://nyti.ms/2TzUeVw}

\begin{itemize}
\item
\item
\item
\item
\item
\end{itemize}

Advertisement

\protect\hyperlink{after-top}{Continue reading the main story}

Supported by

\protect\hyperlink{after-sponsor}{Continue reading the main story}

\hypertarget{representative-serrano-of-the-bronx-to-retire-potentially-opening-seat-for-younger-progressive}{%
\section{Representative Serrano of the Bronx to Retire, Potentially
Opening Seat for Younger
Progressive}\label{representative-serrano-of-the-bronx-to-retire-potentially-opening-seat-for-younger-progressive}}

\includegraphics{https://static01.graylady3jvrrxbe.onion/images/2019/03/26/nyregion/26serrano-print/merlin_137868696_8abfb5b7-823c-4147-8d89-eb4437a9ecd4-articleLarge.jpg?quality=75\&auto=webp\&disable=upscale}

By \href{https://www.nytimes3xbfgragh.onion/by/william-neuman}{William
Neuman} and
\href{https://www.nytimes3xbfgragh.onion/by/jesse-mckinley}{Jesse
McKinley}

\begin{itemize}
\item
  March 25, 2019
\item
  \begin{itemize}
  \item
  \item
  \item
  \item
  \item
  \end{itemize}
\end{itemize}

Representative José E. Serrano, who is currently the nation's
longest-tenured Hispanic congressman, said on Monday that he would not
run for re-election, citing the effects of Parkinson's disease.

Mr. Serrano, who has served in Congress since 1990 and represents the
southern Bronx, one of the poorest districts in the country, has
repeatedly won re-election with more than 90 percent of the vote. But
with the Democratic Party seeking a younger brand of progressives, there
was talk that he might have faced a serious primary challenge next year.

Two powerful members of the Bronx political establishment praised Mr.
Serrano while predicting a wide open race for his seat; Ritchie Torres,
a Bronx councilman often described as a rising star,
\href{https://www.nytimes3xbfgragh.onion/2019/02/14/nyregion/ny-congress-primaries-2020.html}{said
last month} that he was exploring a primary challenge to Mr. Serrano.

``I'm sure there will be a lot of people who are now going to want to
run for Congress,'' said Carl E. Heastie, the Bronx assemblyman who
serves as speaker of the State Assembly. ``And I will not be one of
them.''

Assemblyman Marcos A. Crespo,
\href{http://bronxdems.org/executive-board/}{the borough's party
chairman} who, like Mr. Serrano, was born in Puerto Rico, called the
congressman ``a legendary figure in Latino politics, Puerto Rican
politics, Bronx politics.''

Mr. Serrano said in a written statement that he had initially planned to
stay in Congress despite the disease, but that he recently changed his
mind. ``I've come to the realization that Parkinson's will eventually
take a toll and that I cannot predict its rate of advancement,'' the
statement said. He said that he would serve out his current term, which
runs through the end of next year.

Mr. Serrano sits on the powerful Appropriations Committee and is the
chairman of the Commerce, Justice, Science Subcommittee. He is also
known for advocating for Puerto Rico, including a
\href{https://www.nytimes3xbfgragh.onion/2017/10/09/nyregion/politicians-with-puerto-rican-roots-challenge-trump-in-push-for-aid.html}{recent
push} to have the Trump administration improve relief efforts for the
island after Hurricane Maria.

``I always tried to speak for those who are marginalized in our society,
to give them a voice and a vote here in Washington,'' he said, citing
his efforts to bring federal funding to the Bronx, increase educational
funding for minority students, clean up the polluted Bronx River and end
the use of Vieques Island as a military bombing range.

Mr. Serrano, who had previously been a state assemblyman, was first
elected to Congress in 1990 in a special election to fill the seat of
\href{https://www.nytimes3xbfgragh.onion/2017/01/26/nyregion/robert-garcia-dead-bronx-congressman.html}{Representative
Robert Garcia}, who was convicted of extorting money from Wedtech, a
Bronx military contractor. Mr. Garcia's conviction
\href{https://www.nytimes3xbfgragh.onion/1990/06/30/nyregion/garcias-extortion-convictions-are-reversed-by-appeals-panel.html?module=inline}{was
later overturned}. (Another Bronx congressman, Mario Biaggi,
\href{https://www.nytimes3xbfgragh.onion/2015/06/26/nyregion/mario-biaggi-10-term-new-york-congressman-who-went-to-prison-dies-at-97.html}{was
also convicted} in the Wedtech scandal; his granddaughter, Alessandra
Biaggi,
\href{https://www.nytimes3xbfgragh.onion/2018/09/13/nyregion/state-senate-election-results-idc-klein.html}{was
elected} last year to the New York State Senate.)

Mr. Serrano was born in Mayagüez, Puerto Rico, and moved with his family
to the South Bronx when he was seven years old. He said that, as a child
in Puerto Rico, he learned English by singing along to Frank Sinatra
records. Mr. Serrano lived in public housing in the Bronx and did not
receive a college degree.

``He is an icon, not only to the Bronx but to Latinos throughout the
country,'' said Mr. Torres. ``As a young Latino who ran for public
office when I was 24 years old, I would not be where I am today were it
not for trailblazers like José Serrano who paved the way.''

Mr. Torres, 31, declined to say definitively whether he would run for
Mr. Serrano's seat. ``I'd rather keep the focus on his legacy,'' he
said. ``He secured the federal funding that led to the cleanup of the
Bronx River, which is an enormous legacy to leave on.''

Mr. Serrano's son, José M. Serrano, a Democratic state senator, said he
would not seek his father's seat because he did not want to be away from
his wife and two children.

The younger Mr. Serrano praised his father's faith in their borough.
``When I was a child growing up in the 1970s, the Bronx was burning,''
he said. ``Now it's thriving.''

Advertisement

\protect\hyperlink{after-bottom}{Continue reading the main story}

\hypertarget{site-index}{%
\subsection{Site Index}\label{site-index}}

\hypertarget{site-information-navigation}{%
\subsection{Site Information
Navigation}\label{site-information-navigation}}

\begin{itemize}
\tightlist
\item
  \href{https://help.nytimes3xbfgragh.onion/hc/en-us/articles/115014792127-Copyright-notice}{©~2020~The
  New York Times Company}
\end{itemize}

\begin{itemize}
\tightlist
\item
  \href{https://www.nytco.com/}{NYTCo}
\item
  \href{https://help.nytimes3xbfgragh.onion/hc/en-us/articles/115015385887-Contact-Us}{Contact
  Us}
\item
  \href{https://www.nytco.com/careers/}{Work with us}
\item
  \href{https://nytmediakit.com/}{Advertise}
\item
  \href{http://www.tbrandstudio.com/}{T Brand Studio}
\item
  \href{https://www.nytimes3xbfgragh.onion/privacy/cookie-policy\#how-do-i-manage-trackers}{Your
  Ad Choices}
\item
  \href{https://www.nytimes3xbfgragh.onion/privacy}{Privacy}
\item
  \href{https://help.nytimes3xbfgragh.onion/hc/en-us/articles/115014893428-Terms-of-service}{Terms
  of Service}
\item
  \href{https://help.nytimes3xbfgragh.onion/hc/en-us/articles/115014893968-Terms-of-sale}{Terms
  of Sale}
\item
  \href{https://spiderbites.nytimes3xbfgragh.onion}{Site Map}
\item
  \href{https://help.nytimes3xbfgragh.onion/hc/en-us}{Help}
\item
  \href{https://www.nytimes3xbfgragh.onion/subscription?campaignId=37WXW}{Subscriptions}
\end{itemize}
