Sections

SEARCH

\protect\hyperlink{site-content}{Skip to
content}\protect\hyperlink{site-index}{Skip to site index}

\href{https://www.nytimes3xbfgragh.onion/section/technology}{Technology}

\href{https://myaccount.nytimes3xbfgragh.onion/auth/login?response_type=cookie\&client_id=vi}{}

\href{https://www.nytimes3xbfgragh.onion/section/todayspaper}{Today's
Paper}

\href{/section/technology}{Technology}\textbar{}Turing Award Won by 3
Pioneers in Artificial Intelligence

\url{https://nyti.ms/2UZit13}

\begin{itemize}
\item
\item
\item
\item
\item
\end{itemize}

Advertisement

\protect\hyperlink{after-top}{Continue reading the main story}

Supported by

\protect\hyperlink{after-sponsor}{Continue reading the main story}

\hypertarget{turing-award-won-by-3-pioneers-in-artificial-intelligence}{%
\section{Turing Award Won by 3 Pioneers in Artificial
Intelligence}\label{turing-award-won-by-3-pioneers-in-artificial-intelligence}}

\includegraphics{https://static01.graylady3jvrrxbe.onion/images/2019/03/27/business/27TURING-triptych/27TURING-triptych-articleLarge.jpg?quality=75\&auto=webp\&disable=upscale}

By \href{https://www.nytimes3xbfgragh.onion/by/cade-metz}{Cade Metz}

\begin{itemize}
\item
  March 27, 2019
\item
  \begin{itemize}
  \item
  \item
  \item
  \item
  \item
  \end{itemize}
\end{itemize}

SAN FRANCISCO --- In 2004, Geoffrey Hinton doubled down on his pursuit
of a technological idea called a neural network.

It was a way for machines to see the world around them, recognize sounds
and even understand natural language. But scientists had spent more than
50 years working on the concept of neural networks, and machines
couldn't really do any of that.

Backed by the Canadian government, Dr. Hinton, a computer science
professor at the University of Toronto, organized a new research
community with several academics who also tackled the concept. They
included Yann LeCun, a professor at New York University, and Yoshua
Bengio at the University of Montreal.

On Wednesday, the Association for Computing Machinery, the world's
largest society of computing professionals, announced that Drs. Hinton,
LeCun and Bengio had won this year's Turing Award for their work on
neural networks. The Turing Award, which was introduced in 1966, is
often called the Nobel Prize of computing, and it includes a \$1 million
prize, which the three scientists will share.

Over the past decade, the big idea nurtured by these researchers has
reinvented the way technology is built, accelerating the development of
\href{https://www.nytimes3xbfgragh.onion/2018/06/29/business/newspaper-shooting-facial-recognition.html}{face-recognition
services},
\href{https://www.nytimes3xbfgragh.onion/interactive/2018/08/17/technology/alexa-siri-conversation.html}{talking
digital assistants},
\href{https://www.nytimes3xbfgragh.onion/2017/09/10/business/warehouse-robots-learning.html}{warehouse
robots} and
\href{https://www.nytimes3xbfgragh.onion/2018/01/04/technology/self-driving-cars-aurora.html}{self-driving
cars}. Dr. Hinton is now at Google, and Dr. LeCun works for Facebook.
Dr. Bengio has inked deals with IBM and Microsoft.

``What we have seen is nothing short of a paradigm shift in the
science,'' said Oren Etzioni, the chief executive officer of the Allen
Institute for Artificial Intelligence in Seattle and a prominent voice
in the A.I. community. ``History turned their way, and I am in awe.''

Loosely modeled on the web of neurons in the human brain, a
\href{https://www.nytimes3xbfgragh.onion/2018/03/06/technology/google-artificial-intelligence.html}{neural
network} is a complex mathematical system that can learn discrete tasks
by analyzing vast amounts of data. By analyzing thousands of old phone
calls, for example, it can learn to recognize spoken words.

This allows many artificial intelligence technologies to progress at a
rate that was not possible in the past. Rather than coding behavior into
systems by hand --- one logical rule at a time --- computer scientists
can build technology that learns behavior largely on its own.

The London-born Dr. Hinton, 71, first embraced the idea as a graduate
student in the early 1970s, a time when most artificial intelligence
researchers turned against it. Even his own Ph.D. adviser questioned the
choice.

\includegraphics{https://static01.graylady3jvrrxbe.onion/images/2019/03/27/business/27turing2/27turing2-articleLarge.jpg?quality=75\&auto=webp\&disable=upscale}

``We met once a week,'' Dr. Hinton said in an interview. ``Sometimes it
ended in a shouting match, sometimes not.''

Neural networks had a brief revival in the late 1980s and early 1990s.
After a year of postdoctoral research with Dr. Hinton in Canada, the
Paris-born Dr. LeCun moved to AT\&T's Bell Labs in New Jersey, where he
designed a neural network that could read handwritten letters and
numbers. An AT\&T subsidiary sold the system to banks, and at one point
it read about 10 percent of all checks written in the United States.

Though a neural network could read handwriting and help with some other
tasks, it could not make much headway with big A.I. tasks, like
recognizing faces and objects in photos, identifying spoken words, and
understanding the natural way people talk.

``They worked well only when you had lots of training data, and there
were few areas that had lots of training data,'' Dr. LeCun, 58, said.

But some researchers persisted, including the Paris-born Dr. Bengio, 55,
who worked alongside Dr. LeCun at Bell Labs before taking a
professorship at the University of Montreal.

In 2004, with less than \$400,000 in funding from the Canadian Institute
for Advanced Research, Dr. Hinton created a research program dedicated
to what he called ``neural computation and adaptive perception.'' He
invited Dr. Bengio and Dr. LeCun to join him.

By the end of the decade, the idea had caught up with its potential. In
2010, Dr. Hinton and his students helped Microsoft, IBM, and Google push
the boundaries of speech recognition. Then they did much the same with
image recognition.

``He is a genius and knows how to create one impact after another,''
said Li Deng, a former speech researcher at Microsoft who brought Dr.
Hinton's ideas into the company.

Dr. Hinton's image recognition breakthrough was based on an algorithm
developed by Dr. LeCun. In late 2013, Facebook hired the N.Y.U.
professor to build a research lab around the idea. Dr. Bengio resisted
offers to join one of the big tech giants, but the research he oversaw
in Montreal helped drive the progress of
\href{https://www.nytimes3xbfgragh.onion/2018/11/18/technology/artificial-intelligence-language.html}{systems
that aim to understand natural language} and technology that can
\href{https://www.nytimes3xbfgragh.onion/interactive/2018/01/02/technology/ai-generated-photos.html}{generate
fake photos that are indistinguishable from the real thing}.

Though these systems have undeniably accelerated the progress of
artificial intelligence, they are still a very long way from true
intelligence. But Drs. Hinton, LeCun and Bengio believe that new ideas
will come.

``We need fundamental additions to this toolbox we have created to reach
machines that operate at the level of true human understanding,'' Dr.
Bengio said.

Advertisement

\protect\hyperlink{after-bottom}{Continue reading the main story}

\hypertarget{site-index}{%
\subsection{Site Index}\label{site-index}}

\hypertarget{site-information-navigation}{%
\subsection{Site Information
Navigation}\label{site-information-navigation}}

\begin{itemize}
\tightlist
\item
  \href{https://help.nytimes3xbfgragh.onion/hc/en-us/articles/115014792127-Copyright-notice}{©~2020~The
  New York Times Company}
\end{itemize}

\begin{itemize}
\tightlist
\item
  \href{https://www.nytco.com/}{NYTCo}
\item
  \href{https://help.nytimes3xbfgragh.onion/hc/en-us/articles/115015385887-Contact-Us}{Contact
  Us}
\item
  \href{https://www.nytco.com/careers/}{Work with us}
\item
  \href{https://nytmediakit.com/}{Advertise}
\item
  \href{http://www.tbrandstudio.com/}{T Brand Studio}
\item
  \href{https://www.nytimes3xbfgragh.onion/privacy/cookie-policy\#how-do-i-manage-trackers}{Your
  Ad Choices}
\item
  \href{https://www.nytimes3xbfgragh.onion/privacy}{Privacy}
\item
  \href{https://help.nytimes3xbfgragh.onion/hc/en-us/articles/115014893428-Terms-of-service}{Terms
  of Service}
\item
  \href{https://help.nytimes3xbfgragh.onion/hc/en-us/articles/115014893968-Terms-of-sale}{Terms
  of Sale}
\item
  \href{https://spiderbites.nytimes3xbfgragh.onion}{Site Map}
\item
  \href{https://help.nytimes3xbfgragh.onion/hc/en-us}{Help}
\item
  \href{https://www.nytimes3xbfgragh.onion/subscription?campaignId=37WXW}{Subscriptions}
\end{itemize}
