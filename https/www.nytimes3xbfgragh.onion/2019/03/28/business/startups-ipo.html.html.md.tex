\href{/section/business}{Business}\textbar{}In This Tech I.P.O. Wave,
Big Investors Grab More of the Gains

\url{https://nyti.ms/2YB2isK}

\begin{itemize}
\item
\item
\item
\item
\item
\item
\end{itemize}

\includegraphics{https://static01.graylady3jvrrxbe.onion/images/2019/03/29/business/29lyftoff/29lyftoff-articleLarge.jpg?quality=75\&auto=webp\&disable=upscale}

Sections

\protect\hyperlink{site-content}{Skip to
content}\protect\hyperlink{site-index}{Skip to site index}

\hypertarget{in-this-tech-ipo-wave-big-investors-grab-more-of-the-gains}{%
\section{In This Tech I.P.O. Wave, Big Investors Grab More of the
Gains}\label{in-this-tech-ipo-wave-big-investors-grab-more-of-the-gains}}

Unicorns like Lyft are finally going public, after large gains have been
captured by elite early investors.

Credit...Illustration by Andrew Sondern/The New York Times; Photographs
by Thom Baur/Reuters and Christie Hemm Klok for The New York Times

Supported by

\protect\hyperlink{after-sponsor}{Continue reading the main story}

By \href{https://www.nytimes3xbfgragh.onion/by/matt-phillips}{Matt
Phillips} and
\href{https://www.nytimes3xbfgragh.onion/by/erin-griffith}{Erin
Griffith}

\begin{itemize}
\item
  March 28, 2019
\item
  \begin{itemize}
  \item
  \item
  \item
  \item
  \item
  \item
  \end{itemize}
\end{itemize}

The latest generation of Silicon Valley start-ups is now sprinting to
the public markets, raising hopes among large and small investors eager
to invest in these high-profile, fast-growing firms.

But the class of 2019 is far different from its predecessors. These
companies, including gig economy darlings like Uber and Lyft, are
generally older and larger, powered for years by billions of dollars of
private money that has reshaped the start-up world.

The additional maturity of the companies may curb wild swings --- both
big gains and big losses --- for new investors.

But it could also mean that the companies' fastest phases of growth are
behind them. As a result, there is an increased risk that in this wave
of tech I.P.O.s, an elite group of investors, like sovereign wealth
funds and venture capitalists, will grab a larger share of the winnings
compared with new investors.

``Individual investors are going to get in too late,'' said Jason DeSena
Trennert, managing partner at Strategas Research Partners, a markets and
economic analysis firm. ``They're going to be the last investors in, and
that's the concern.''

{[}\href{https://www.nytimes3xbfgragh.onion/2019/03/29/technology/lyft-stock-price.html}{\emph{Lyft's
shares soared on Friday when it began trading on public markets.}}{]}

The change reflects a large-scale shift in the way that American
entrepreneurs raise money to build their companies. Instead of quickly
turning to the public markets and the scrutiny that comes with that, as
Amazon and Google did, they are building huge businesses over the course
of many years on the back of private money, and with fewer demands for
financial disclosures.

Uber, the giant ride-hailing company, has raised more than \$20 billion
over the last decade. Lyft, its smaller rival, which priced its I.P.O.
late Thursday at \$72 a share and is expected to start trading on the
Nasdaq Friday, raised \$4.9 billion over seven years.

Many start-ups in earlier waves --- back to the dot-com boom of the late
1990s, when Amazon was listed on the Nasdaq --- went public just a few
years after their founding. Some, like Pets.com, had tiny amounts of
revenue, and now exist only in Silicon Valley lore.

A combination of policy changes and vast new riches in the tech industry
has been changing the equation for start-ups for more than a decade.

\includegraphics{https://static01.graylady3jvrrxbe.onion/images/2019/03/28/business/28ipoprospects-2/merlin_152662398_95cd276a-bc9c-4815-8ec3-a27d370825aa-articleLarge.jpg?quality=75\&auto=webp\&disable=upscale}

Mutual funds and hedge funds --- the typical investors in a start-up's
I.P.O. --- began buying stakes in large private companies as a way to
build up larger stakes before the new businesses went public. Other big
investors joined in, including large sovereign wealth funds and the
outsize SoftBank Vision Fund, creating an even hotter market.

Venture capital investments into United States-based companies grew to
\$99.5 billion in 2018, the highest level since 2000, according to CB
Insights, a company that tracks start-ups.

Those investments have driven valuations of start-ups to unusual
heights. There are now at least 333 so-called unicorns, companies valued
at \$1 billion or more, according to CB Insights. In 2014 there were
around 80.

Lyft has a private valuation of more than \$11 billion. So does
Pinterest, another company in the process of going public. That is
roughly the market value that public investors put on the retailer
Kohl's and the online trading firm E-Trade Financial.

Lyft Ranks Near the Top of Tech I.P.O.s

Internet companies

Other tech. companies

\$30 bil.

Amount

raised

Alibaba

Facebook

\$10 bil.

Lyft

estimate

Snap

Twitter

Google

\$1 bil.

\$100 mil.

\$10 mil.

``Internet bubble''

\$1 mil.

'90

'00

'10

'19

Lyft Ranks Near the Top of Tech I.P.O.s

Amount raised

Other tech. companies

Internet companies

\$30,000,000,000

Alibaba

Facebook

\$10,000,000,000

Lyft

estimate

Infineon Tech.

Agere Systems

Snap

Twitter

Google

\$1,000,000,000

\$100,000,000

\$10,000,000

``Internet bubble''

\$1,000,000

1990

1995

2000

2005

2010

2015

2019

Lyft Ranks Near the Top of Tech I.P.O.s

Internet companies

Other technology companies

Amount raised

\$30,000,000,000

Alibaba

Facebook

\$10,000,000,000

Lyft

estimate

Infineon Technologies

Agere Systems

Snap

Twitter

Google

\$1,000,000,000

\$100,000,000

\$10,000,000

``Internet bubble''

\$1,000,000

1990

1995

2000

2005

2010

2015

2019

Notes: Graph shows only tech I.P.O.s that raised over \$1 million.
Vertical scale is adjusted to orders of magnitude, making percentage
differences comparable.

Source: Refinitiv

By Karl Russell

Uber, the largest of the private companies expected to head for the
stock market this year, has a private-market valuation of more than \$70
billion. In public markets, that's roughly the same size as corporate
giants such as Goldman Sachs and CVS Health.

Matt Murphy, a partner at Menlo Ventures, a leading venture capital
firm, said the higher valuations and larger investments correlate with
bigger opportunities created by smartphones and cloud computing.

``The magnitude of the audiences that can be reached and the
monetization per user has grown,'' he said. ``Their growth potential is
much higher than was previously anticipated.''

Some industry groups and investors who urge fewer regulations say the
emphasis on the private markets is an outgrowth of the Sarbanes-Oxley
Act, the federal law passed in 2002 that tightened accounting rules for
public companies after the accounting scandals of the early 2000s.

Image

President George W. Bush signing, in 2002, the Sarbanes-Oxley Act, which
tightened accounting standards for public companies.Credit...Stephen
Jaffe/Agence France-Presse --- Getty Images

Besides raising disclosure requirements and other changes, the law
required top executives to attest to the accuracy of corporate financial
statements. Some say those higher costs to guarantee compliance can
dissuade smaller companies from going public.

Private companies, by comparison, can operate with far less disclosure.
They are under no obligation to file quarterly earnings updates or
audited annual financial statements with the Securities and Exchange
Commission. Nor are they required to broadly distribute updates on
business developments to the public.

Others say the decline in public offerings began before Sarbanes-Oxley
passed. They attribute the change to a wave of federal deregulation,
which made it easier to raise money and sell companies privately.
Lighter antitrust enforcement set off a boom in mergers and
acquisitions, allowing smaller firms to sell to bigger companies instead
of going public.

At the same time,
\href{https://papers.ssrn.com/sol3/papers.cfm?abstract_id=3017610}{new
laws made it easier for private companies to sell securities} to
qualified investors around the country, bolstering funding from private
equity and venture capital.

Whatever the driver, the net result has been a clear downturn in the
number of public companies in the United States. The number of listed
companies has declined by 52 percent since 1997, to a bit more than
3,600 in 2016, according to \href{https://www.nber.org/papers/w24265}{a
working paper from the National Bureau of Economic Research} published
last year.

This long-term shrinkage in publicly available shares is a reason that
some analysts expect Uber, Lyft and other prominent start-ups to receive
a warm response from the institutional investors who typically buy
freshly issued shares, even if the potential upside could be smaller.

``There's an element, I think, of pent-up demand here,'' said David
Ethridge, who advises on public offerings at the consulting firm PwC.
``I think people will have a feeling of, I don't really want to miss
out.''

Investors have some reason to be skeptical of paying top dollar for
newly minted public companies, however. Over the last two years, the
value of companies that completed public offerings actually fell by an
average of 8 percent, according to a recent research report from Goldman
Sachs analysts.

Over the same period, the S\&P 500 stock index was up about 12 percent.

Advertisement

\protect\hyperlink{after-bottom}{Continue reading the main story}

\hypertarget{site-index}{%
\subsection{Site Index}\label{site-index}}

\hypertarget{site-information-navigation}{%
\subsection{Site Information
Navigation}\label{site-information-navigation}}

\begin{itemize}
\tightlist
\item
  \href{https://help.nytimes3xbfgragh.onion/hc/en-us/articles/115014792127-Copyright-notice}{©~2020~The
  New York Times Company}
\end{itemize}

\begin{itemize}
\tightlist
\item
  \href{https://www.nytco.com/}{NYTCo}
\item
  \href{https://help.nytimes3xbfgragh.onion/hc/en-us/articles/115015385887-Contact-Us}{Contact
  Us}
\item
  \href{https://www.nytco.com/careers/}{Work with us}
\item
  \href{https://nytmediakit.com/}{Advertise}
\item
  \href{http://www.tbrandstudio.com/}{T Brand Studio}
\item
  \href{https://www.nytimes3xbfgragh.onion/privacy/cookie-policy\#how-do-i-manage-trackers}{Your
  Ad Choices}
\item
  \href{https://www.nytimes3xbfgragh.onion/privacy}{Privacy}
\item
  \href{https://help.nytimes3xbfgragh.onion/hc/en-us/articles/115014893428-Terms-of-service}{Terms
  of Service}
\item
  \href{https://help.nytimes3xbfgragh.onion/hc/en-us/articles/115014893968-Terms-of-sale}{Terms
  of Sale}
\item
  \href{https://spiderbites.nytimes3xbfgragh.onion}{Site Map}
\item
  \href{https://help.nytimes3xbfgragh.onion/hc/en-us}{Help}
\item
  \href{https://www.nytimes3xbfgragh.onion/subscription?campaignId=37WXW}{Subscriptions}
\end{itemize}
