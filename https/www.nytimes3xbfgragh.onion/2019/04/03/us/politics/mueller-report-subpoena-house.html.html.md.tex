Sections

SEARCH

\protect\hyperlink{site-content}{Skip to
content}\protect\hyperlink{site-index}{Skip to site index}

\href{https://www.nytimes3xbfgragh.onion/section/politics}{Politics}

\href{https://myaccount.nytimes3xbfgragh.onion/auth/login?response_type=cookie\&client_id=vi}{}

\href{https://www.nytimes3xbfgragh.onion/section/todayspaper}{Today's
Paper}

\href{/section/politics}{Politics}\textbar{}House Democrat Demands Six
Years of Trump Tax Returns From I.R.S.

\url{https://nyti.ms/2JYCdRg}

\begin{itemize}
\item
\item
\item
\item
\item
\item
\end{itemize}

Advertisement

\protect\hyperlink{after-top}{Continue reading the main story}

Supported by

\protect\hyperlink{after-sponsor}{Continue reading the main story}

\hypertarget{house-democrat-demands-six-years-of-trump-tax-returns-from-irs}{%
\section{House Democrat Demands Six Years of Trump Tax Returns From
I.R.S.}\label{house-democrat-demands-six-years-of-trump-tax-returns-from-irs}}

\includegraphics{https://static01.graylady3jvrrxbe.onion/images/2019/04/03/us/politics/03dc-report1-sub2/merlin_149024247_98b54dbc-d482-4342-9c91-52e5c1fb1e62-articleLarge.jpg?quality=75\&auto=webp\&disable=upscale}

By \href{https://www.nytimes3xbfgragh.onion/by/nicholas-fandos}{Nicholas
Fandos}

\begin{itemize}
\item
  April 3, 2019
\item
  \begin{itemize}
  \item
  \item
  \item
  \item
  \item
  \item
  \end{itemize}
\end{itemize}

WASHINGTON --- The chairman of the House Ways and Means Committee, using
a little-known provision in the federal tax code, formally requested on
Wednesday that the I.R.S. hand over six years of President Trump's
personal and business tax returns, starting what is likely to be a
momentous fight with his administration.

Representative Richard E. Neal, Democrat of Massachusetts,
hand-delivered a two-page letter laying out the request to Charles P.
Rettig, the Internal Revenue Service commissioner, ending months of
speculation about when he would do so and almost certainly prompting a
legal challenge from the Trump administration.

Responding to questions from reporters in the Oval Office, Mr. Trump
suggested that he would fight the request because, he said, he was being
audited.

``I guess when you have a name, you are audited, but until such time as
I'm not under audit I would not be inclined to do that,'' he said.

\emph{{[}}\href{https://int.graylady3jvrrxbe.onion/data/documenthelper/740-congress-request-trump-tax-returns/8fca455b44e383714434/optimized/full.pdf\#page=1}{\emph{Read
Mr. Neal's letter to the I.R.S. commissioner}}\emph{.{]}}

The move by Mr. Neal came as other panels controlled by House Democrats
were flexing their muscles. The House Judiciary Committee on Wednesday
morning authorized its chairman to use a subpoena to try to force the
Justice Department to give Congress a full copy of the special counsel's
report and all of the underlying evidence used to reach his conclusions
on Russian interference in the 2016 election.

And the chairman of the House Oversight and Reform Committee said that
he would soon ask for a vote on a subpoena of his own to compel Mazars
USA, an accounting firm tied to the president, to produce a decade's
worth of Mr. Trump's financial records.

``They have told us that they will provide the information pretty much
when they have a subpoena,'' the chairman, Representative Elijah E.
Cummings of Maryland, told reporters. ``And we'll get them a subpoena.''

Unlike the chairmen of other committees, Mr. Neal is not relying on a
subpoena or standard congressional processes. Instead, he is invoking an
authority enshrined in the tax code granted only to the tax-writing
committees in Congress that gives the chairmen of the House Ways and
Means Committee and the Senate Finance Committee the power to request
tax information on any filer.

Mr. Neal gave the agency until April 10 to comply with the request, and
if he receives the information, he will then confidentially review it
with his committee staff.

The provision, which dates in some form to the Teapot Dome scandal of
Warren G. Harding's administration, at least on its face gives the Trump
administration little room to decline a request like Mr. Neal's. It only
says that the Treasury secretary ``shall'' furnish the information.

``President Trump is the first president in nearly a half century to
break precedent and refuse to voluntarily release his tax returns,''
said Representative Dan Kildee, Democrat of Michigan and a member of the
Ways and Means Committee. ``The president is the only person who can
sign bills into law, and the public deserves to know whether the
president's personal financial interests affect his public decision
making.''

The Treasury Department and the I.R.S. did not immediately respond to
requests for comment.

But Democrats anticipate that the Trump administration will object to
the request and force the matter into the courts, where its adjudication
could take months or longer. Though the provision --- No. 6103 in the
tax code --- is invoked frequently by the committee, there is little
precedent for using it to view the returns of a president who has not
invited the scrutiny.

Republicans have vigorously argued against the request, saying that
whatever justification Democrats produce will belie their true intent:
to fish for information that could embarrass the president politically.

\includegraphics{https://static01.graylady3jvrrxbe.onion/images/2019/04/03/us/politics/03dc-report1-sub/03dc-report1-sub-videoSixteenByNine3000.jpg}

Representative Kevin Brady of Texas, the top Republican on the Ways and
Means Committee, called the request ``an abuse of the tax-writing
committees' statutory authority.''

``Weaponizing our nation's tax code by targeting political foes sets a
dangerous precedent and weakens Americans' privacy rights,'' Mr. Brady
said in a letter to Treasury Secretary Steven Mnuchin. ``As you know, by
law all Americans have a fundamental right to the privacy of the
personal information found in their tax returns.''

Defying modern presidential norms, Mr. Trump has refused since he became
a candidate for president to release any of his tax returns. Democrats
suspect the tax information could provide clues to wrongdoing by Mr.
Trump, and they made getting the documents one of their top oversight
priorities when they reclaimed control of the House in January.

\emph{{[}}\href{https://www.nytimes3xbfgragh.onion/interactive/2018/10/02/us/politics/donald-trump-tax-schemes-fred-trump.html}{\emph{A
New York Times investigation}} \emph{showed that the president engaged
in suspect tax schemes as he reaped riches from his father.{]}}

Mr. Neal said he was making the request as part of his committee's
oversight of ``the extent to which the I.R.S. audits and enforces the
federal tax laws against a president.'' Under I.R.S. policy, the
personal tax returns of presidents and vice presidents are supposed to
be automatically audited each year. Mr. Neal said the committee was
considering legislation related to the issue.

``I take the authority to make this request very seriously, and I
approach it with the utmost care and respect,'' Mr. Neal said in a
statement. ``This request is about policy, not politics; my preparations
were made on my own track and timeline, entirely independent of other
activities in Congress and the administration.''

He added, ``I trust that in this spirit, the I.R.S. will comply with
federal law and furnish me with the requested documents in a timely
manner.''

In addition to Mr. Trump's personal returns for 2013 to 2018, Mr. Neal
requested returns for Mr. Trump's trust and seven other core Trump
business entities that control scores of other Trump operations,
including his golf club in Bedminster, N.J. He also asked the I.R.S. to
share any information it had related to the entities, including whether
they had been audited.

Liberal Democrats have complained for weeks that Mr. Neal, 70 and a
roll-up-your-sleeves legislator, was dragging his feet on making the
request. They have organized events in his district, taken out
advertisements and produced legal briefs meant to make a case that he
should act and act quickly.

Mr. Neal said throughout that he was chiefly concerned with crafting a
request, alongside the House general counsel and the Ways and Means
Committee staff, that could withstand legal challenge.

``I am certain we are within our legitimate legislative, legal and
oversight rights,'' he said on Wednesday.

In the Judiciary Committee, the chairman, Representative Jerrold Nadler
of New York, said he would not immediately issue the subpoena for the
Mueller report. But the party-line vote won by Democrats who control the
committee ratchets up pressure on Attorney General William P. Barr as he
decides how much of the nearly 400-page report to share with lawmakers.

``I will give him time to change his mind,'' Mr. Nadler said in his
opening statement. ``But if we cannot reach an accommodation, then we
will have no choice but to issue subpoenas for these materials.''

The committee also approved subpoenas for five former White House aides
who Democrats said were relevant to an investigation into possible
obstruction of justice, abuse of power and corruption within the Trump
administration.

They included Donald F. McGahn II, a former White House counsel; Stephen
K. Bannon, the president's former chief strategist; Hope Hicks, a former
White House communications director; Reince Priebus, the president's
first chief of staff; and Annie Donaldson, a deputy of Mr. McGahn.

Advertisement

\protect\hyperlink{after-bottom}{Continue reading the main story}

\hypertarget{site-index}{%
\subsection{Site Index}\label{site-index}}

\hypertarget{site-information-navigation}{%
\subsection{Site Information
Navigation}\label{site-information-navigation}}

\begin{itemize}
\tightlist
\item
  \href{https://help.nytimes3xbfgragh.onion/hc/en-us/articles/115014792127-Copyright-notice}{©~2020~The
  New York Times Company}
\end{itemize}

\begin{itemize}
\tightlist
\item
  \href{https://www.nytco.com/}{NYTCo}
\item
  \href{https://help.nytimes3xbfgragh.onion/hc/en-us/articles/115015385887-Contact-Us}{Contact
  Us}
\item
  \href{https://www.nytco.com/careers/}{Work with us}
\item
  \href{https://nytmediakit.com/}{Advertise}
\item
  \href{http://www.tbrandstudio.com/}{T Brand Studio}
\item
  \href{https://www.nytimes3xbfgragh.onion/privacy/cookie-policy\#how-do-i-manage-trackers}{Your
  Ad Choices}
\item
  \href{https://www.nytimes3xbfgragh.onion/privacy}{Privacy}
\item
  \href{https://help.nytimes3xbfgragh.onion/hc/en-us/articles/115014893428-Terms-of-service}{Terms
  of Service}
\item
  \href{https://help.nytimes3xbfgragh.onion/hc/en-us/articles/115014893968-Terms-of-sale}{Terms
  of Sale}
\item
  \href{https://spiderbites.nytimes3xbfgragh.onion}{Site Map}
\item
  \href{https://help.nytimes3xbfgragh.onion/hc/en-us}{Help}
\item
  \href{https://www.nytimes3xbfgragh.onion/subscription?campaignId=37WXW}{Subscriptions}
\end{itemize}
