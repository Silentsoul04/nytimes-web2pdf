Sections

SEARCH

\protect\hyperlink{site-content}{Skip to
content}\protect\hyperlink{site-index}{Skip to site index}

\href{https://myaccount.nytimes3xbfgragh.onion/auth/login?response_type=cookie\&client_id=vi}{}

\href{https://www.nytimes3xbfgragh.onion/section/todayspaper}{Today's
Paper}

\href{/section/opinion}{Opinion}\textbar{}Actually, the Electoral
College Was a Pro-Slavery Ploy

\url{https://nyti.ms/2WNyxDt}

\begin{itemize}
\item
\item
\item
\item
\item
\item
\end{itemize}

Advertisement

\protect\hyperlink{after-top}{Continue reading the main story}

\href{/section/opinion}{Opinion}

Supported by

\protect\hyperlink{after-sponsor}{Continue reading the main story}

\hypertarget{actually-the-electoral-college-was-a-pro-slavery-ploy}{%
\section{Actually, the Electoral College Was a Pro-Slavery
Ploy}\label{actually-the-electoral-college-was-a-pro-slavery-ploy}}

That fact alone doesn't mean it ought to be scrapped. But we should be
clear about its disreputable origins.

By Akhil Reed Amar

Mr. Amar is a professor at Yale Law School.

\begin{itemize}
\item
  April 6, 2019
\item
  \begin{itemize}
  \item
  \item
  \item
  \item
  \item
  \item
  \end{itemize}
\end{itemize}

\includegraphics{https://static01.graylady3jvrrxbe.onion/images/2019/04/06/opinion/06amal/merlin_115808603_271ccba8-b43a-429e-ae8b-ba81e9280f42-articleLarge.jpg?quality=75\&auto=webp\&disable=upscale}

Many Americans are critical of the Electoral College, an attitude that
seems to have intensified since Donald Trump defeated Hillary Clinton in
the 2016 presidential election despite losing the popular vote. These
critics often make two arguments: first, that electing the president by
direct popular vote would be preferable in a democracy; and second, that
the Electoral College has disreputable origins, having been put into the
Constitution to protect the institution of slavery.

Defenders of the Electoral College often counter that it was designed
not to help maintain slavery but for other reasons, many of them still
relevant, such as to balance the power of big states against that of
small states. (Even some critics of the Electoral College have
\href{https://www.nytimes3xbfgragh.onion/2019/04/04/opinion/the-electoral-college-slavery-myth.html}{made
this argument}.)

Both sides are misguided. There are legitimate reasons to keep the
Electoral College system, odd and creaky though it may be, but we must
accept the fact that it does have deep roots in efforts by the founders
to accommodate slavery.

The Electoral College was not mainly designed to balance big states
against small states. It certainly did not have that effect: Eight of
the first nine presidential elections were won by candidates who were
plantation owners from Virginia, then America's biggest state. Only
three candidates from small states have ever been elected president:
Zachary Taylor, Franklin Pierce and Bill Clinton.

As James Madison made clear at the Constitutional Convention in 1787 in
Philadelphia, the big political divide in America was not between big
and small states; it was between North and South and was all about
slavery. So, too, was the Electoral College at the founding, both in its
original incarnation in 1787 and in the version later created by the
12th Amendment, which was adopted in 1804.

Behind closed doors at the Constitutional Convention, when the idea of
direct presidential election was proposed by the Northerner James
Wilson, the Southerner James Madison explained why this was a political
nonstarter: Slaves couldn't vote, so the slaveholding South would
basically lose every time in a national direct vote. But if slaves could
somehow be counted in an indirect system, maybe at a discount (say,
three-fifths), well, that might sell in the South. Thus were planted the
early seeds of an Electoral College system.

Some have argued that direct election was doomed because the
Philadelphia delegates disdained democracy. Behind closed doors these
elites did indeed bad-mouth the masses (as do elites today). But look at
what the framers of the Constitution did, rather than what they said.
They put the Constitution itself to a far more democratic vote than had
been seen before. They provided for a directly elected House of
Representatives (which the earlier Articles of Confederation did not
do). They omitted all property qualifications for leading federal
positions, unlike almost every state constitution then on the books.

So why didn't they go even further, providing direct presidential
election? Because of Madison's political calculation: Direct election
would have been a dealbreaker for the South. For a while, having members
of Congress elect the president emerged as a possible alternative, but
this idea, too, would have been pro-slavery for the same reason: Thanks
to the three-fifths clause, slave states got extra votes in the House,
just as in the Electoral College system that was finally adopted.

When George Washington left the political stage in the mid-1790s,
America witnessed its first two contested presidential elections. Twice,
most Southerners backed a Southerner (Thomas Jefferson) and most
Northerners backed a Northerner (John Adams). Without the extra
electoral votes generated by its enormous slave population, the South
would have lost the election of 1800, which Jefferson won.

It is true, as some have noted, that some Northerners manipulated the
vote in that election to their advantage, but that does not erase the
ugly fact that the South had extra seats in the Electoral College
because of its slaves. When the Constitution was amended to modify the
Electoral College after 1800, all America had seen the pro-slavery tilt
of the system, but Jefferson's Southern allies steamrollered over
Northern congressmen who explicitly proposed eliminating the system's
pro-slavery bias.

As a result, every president until Abraham Lincoln was either a
Southerner or a Northerner who was willing (while president) to
accommodate the slaveholding South. The dominant political figure in
antebellum America was the pro-slavery Andrew Jackson, who in 1829
proposed eliminating electors while retaining pro-slavery apportionment
rules rooted in the three-fifths clause --- in effect creating a system
of pro-slavery electoral-vote counts without the need for electors
themselves.

Today, of course, slavery no longer skews and stains our system --- and
maybe the Electoral College system should remain intact. The best
argument in its favor is simply inertia: Any reforms might backfire,
with unforeseen and adverse consequences. The Electoral College is the
devil we know.

But we should not kid ourselves: This devil does indeed have devilish
origins.

Akhil Reed Amar is a professor at Yale Law School.

\emph{The Times is committed to publishing}
\href{https://www.nytimes3xbfgragh.onion/2019/01/31/opinion/letters/letters-to-editor-new-york-times-women.html}{\emph{a
diversity of letters}} \emph{to the editor. We'd like to hear what you
think about this or any of our articles. Here are some}
\href{https://help.nytimes3xbfgragh.onion/hc/en-us/articles/115014925288-How-to-submit-a-letter-to-the-editor}{\emph{tips}}\emph{.
And here's our email:}
\href{mailto:letters@NYTimes.com}{\emph{letters@NYTimes.com}}\emph{.}

\emph{Follow The New York Times Opinion section on}
\href{https://www.facebookcorewwwi.onion/nytopinion}{\emph{Facebook}}\emph{,}
\href{http://twitter.com/NYTOpinion}{\emph{Twitter (@NYTopinion)}}
\emph{and}
\href{https://www.instagram.com/nytopinion/}{\emph{Instagram}}\emph{.}

Advertisement

\protect\hyperlink{after-bottom}{Continue reading the main story}

\hypertarget{site-index}{%
\subsection{Site Index}\label{site-index}}

\hypertarget{site-information-navigation}{%
\subsection{Site Information
Navigation}\label{site-information-navigation}}

\begin{itemize}
\tightlist
\item
  \href{https://help.nytimes3xbfgragh.onion/hc/en-us/articles/115014792127-Copyright-notice}{©~2020~The
  New York Times Company}
\end{itemize}

\begin{itemize}
\tightlist
\item
  \href{https://www.nytco.com/}{NYTCo}
\item
  \href{https://help.nytimes3xbfgragh.onion/hc/en-us/articles/115015385887-Contact-Us}{Contact
  Us}
\item
  \href{https://www.nytco.com/careers/}{Work with us}
\item
  \href{https://nytmediakit.com/}{Advertise}
\item
  \href{http://www.tbrandstudio.com/}{T Brand Studio}
\item
  \href{https://www.nytimes3xbfgragh.onion/privacy/cookie-policy\#how-do-i-manage-trackers}{Your
  Ad Choices}
\item
  \href{https://www.nytimes3xbfgragh.onion/privacy}{Privacy}
\item
  \href{https://help.nytimes3xbfgragh.onion/hc/en-us/articles/115014893428-Terms-of-service}{Terms
  of Service}
\item
  \href{https://help.nytimes3xbfgragh.onion/hc/en-us/articles/115014893968-Terms-of-sale}{Terms
  of Sale}
\item
  \href{https://spiderbites.nytimes3xbfgragh.onion}{Site Map}
\item
  \href{https://help.nytimes3xbfgragh.onion/hc/en-us}{Help}
\item
  \href{https://www.nytimes3xbfgragh.onion/subscription?campaignId=37WXW}{Subscriptions}
\end{itemize}
