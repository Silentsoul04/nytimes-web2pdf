Sections

SEARCH

\protect\hyperlink{site-content}{Skip to
content}\protect\hyperlink{site-index}{Skip to site index}

\href{https://www.nytimes3xbfgragh.onion/section/politics}{Politics}

\href{https://myaccount.nytimes3xbfgragh.onion/auth/login?response_type=cookie\&client_id=vi}{}

\href{https://www.nytimes3xbfgragh.onion/section/todayspaper}{Today's
Paper}

\href{/section/politics}{Politics}\textbar{}Kirstjen Nielsen Resigns as
Trump's Homeland Security Secretary

\url{https://nyti.ms/2OXIImq}

\begin{itemize}
\item
\item
\item
\item
\item
\item
\end{itemize}

Advertisement

\protect\hyperlink{after-top}{Continue reading the main story}

Supported by

\protect\hyperlink{after-sponsor}{Continue reading the main story}

\hypertarget{kirstjen-nielsen-resigns-as-trumps-homeland-security-secretary}{%
\section{Kirstjen Nielsen Resigns as Trump's Homeland Security
Secretary}\label{kirstjen-nielsen-resigns-as-trumps-homeland-security-secretary}}

\includegraphics{https://static01.graylady3jvrrxbe.onion/images/2019/04/05/briefing/00dc-nielsen-promo/00dc-nielsen-promo-videoSixteenByNine3000-v5.jpg}

By \href{https://www.nytimes3xbfgragh.onion/by/zolan-kanno-youngs}{Zolan
Kanno-Youngs},
\href{https://www.nytimes3xbfgragh.onion/by/maggie-haberman}{Maggie
Haberman},
\href{https://www.nytimes3xbfgragh.onion/by/michael-d-shear}{Michael D.
Shear} and
\href{https://www.nytimes3xbfgragh.onion/by/eric-schmitt}{Eric Schmitt}

\begin{itemize}
\item
  April 7, 2019
\item
  \begin{itemize}
  \item
  \item
  \item
  \item
  \item
  \item
  \end{itemize}
\end{itemize}

WASHINGTON --- Kirstjen Nielsen, the homeland security secretary,
resigned on Sunday after meeting with President Trump, ending a
tumultuous tenure in charge of the border security agency that had made
her the target of the president's criticism.

``I have determined that it is the right time for me to step aside,''
Ms. Nielsen said in a resignation letter. ``I hope that the next
secretary will have the support of Congress and the courts in fixing the
laws which have impeded our ability to fully secure America's borders
and which have contributed to discord in our nation's discourse.''

Ms. Nielsen had requested the meeting to plan ``a way forward'' at the
border, in part thinking she could have a reasoned conversation with Mr.
Trump about the role, according to three people familiar with the
meeting. She came prepared with a list of things that needed to change
to improve the relationship with the president.

Mr. Trump in recent weeks had asked Ms. Nielsen to close the ports of
entry along the border and to stop accepting asylum seekers, which Ms.
Nielsen found ineffective and inappropriate. While the 30-minute meeting
was cordial, Mr. Trump was determined to ask for her resignation. After
the meeting, she submitted it.

\includegraphics{https://static01.graylady3jvrrxbe.onion/images/2017/01/29/podcasts/the-daily-album-art/the-daily-album-art-articleInline-v2.jpg?quality=75\&auto=webp\&disable=upscale}

\hypertarget{listen-to-the-daily-the-brief-controversial-tenure-of-kirstjen-nielsen}{%
\subsubsection{Listen to `The Daily': The Brief, Controversial Tenure of
Kirstjen
Nielsen}\label{listen-to-the-daily-the-brief-controversial-tenure-of-kirstjen-nielsen}}

As homeland security secretary, she enacted and publicly defended the
family separation policy. In President Trump's eyes, she didn't go far
enough.

transcript

Back to The Daily

bars

0:00/26:23

-26:23

transcript

\hypertarget{listen-to-the-daily-the-brief-controversial-tenure-of-kirstjen-nielsen-1}{%
\subsection{Listen to `The Daily': The Brief, Controversial Tenure of
Kirstjen
Nielsen}\label{listen-to-the-daily-the-brief-controversial-tenure-of-kirstjen-nielsen-1}}

\hypertarget{hosted-by-michael-barbaro-produced-by-rachel-quester-jessica-cheung-and-eric-krupke-and-edited-by-paige-cowett}{%
\subsubsection{Hosted by Michael Barbaro, produced by Rachel Quester,
Jessica Cheung and Eric Krupke, and edited by Paige
Cowett}\label{hosted-by-michael-barbaro-produced-by-rachel-quester-jessica-cheung-and-eric-krupke-and-edited-by-paige-cowett}}

\hypertarget{as-homeland-security-secretary-she-enacted-and-publicly-defended-the-family-separation-policy-in-president-trumps-eyes-she-didnt-go-far-enough}{%
\paragraph{As homeland security secretary, she enacted and publicly
defended the family separation policy. In President Trump's eyes, she
didn't go far
enough.}\label{as-homeland-security-secretary-she-enacted-and-publicly-defended-the-family-separation-policy-in-president-trumps-eyes-she-didnt-go-far-enough}}

\begin{itemize}
\item
  michael barbaro\\
  From The New York Times, I'm Michael Barbaro. This is ``The Daily.''

  Today: The secretary of homeland security was forced out of her job,
  even after carrying out and defending President Trump's most
  restrictive immigration policies. Why that wasn't enough.

  It's Tuesday, April 9.

  Caitlin, describe what happened over the weekend.
\item
  caitlin dickerson\\
  So on Sunday, Kirstjen Nielsen, who's the secretary of the Department
  of Homeland Security --- she's the nation's highest-ranking
  immigration officer --- she goes to the White House for an unscheduled
  meeting with President Trump.
\item
  michael barbaro\\
  Caitlin Dickerson covers immigration for The Times.
\item
  caitlin dickerson\\
  And he, in recent weeks, has been very riled up publicly over an
  increasing number of people crossing the border, especially families
  who seek asylum who have legal protections that mean they have to be
  allowed into the country. So the secretary arrives at her meeting with
  a list of ideas for how to address this problem, and she thinks she
  and the president are going to come up with a way forward. But
  instead, a few hours later, the president sends a tweet announcing
  that she will be leaving her job.
\item
  archived recording 1\\
  Breaking news here on CNN. From the White House, the secretary of
  homeland security, Kirstjen Nielsen, has resigned.
\item
  archived recording 2\\
  President Trump made the announcement via Twitter.
\item
  archived recording 3\\
  In a tweet, the president said, quote, ``Secretary of homeland
  security Kirstjen Nielsen will be leaving her position, and I would
  like to thank her for her service.''
\item
  archived recording 4\\
  So it appears that the president is swinging the revolving door yet
  again, trying to install new people in his administration to carry out
  his bidding.
\end{itemize}

michael barbaro

That sounds like a meeting that went very poorly for the secretary.

caitlin dickerson

Clearly, they were not able to come up with a solution that both could
agree on.

michael barbaro

And by tweeting her resignation, I think what we're politely suggesting
is that he fired her.

caitlin dickerson

Fired her. Yes, exactly.

michael barbaro

So what is the history between President Trump and Secretary Nielsen?

caitlin dickerson

So Secretary Nielsen takes her job at the end of 2017. She's replacing
her former boss, John Kelly, who went to be President Trump's chief of
staff. And she takes on this role as basically the person who has to
approve any new policy that's going to be introduced along the border.
And it's at this period of time when, I think, the President is settling
into his role. He's fired up about really coming up with some hard
evidence that he can point to ahead of the midterm elections, for
example, and say, look, I'm keeping my campaign promises, and I'm
lowering immigration really dramatically. So right away, she's faced
with aggressive ideas for how to prevent people from coming to the
United States, and the first one is family separation.

michael barbaro

And what do we know about how Nielsen responds to that very
controversial policy when it is first introduced to her?

caitlin dickerson

She slow-walks it. We reported in December of 2017 that a family
separation policy had reached her desk for that final signature, and it
took her four months to approve it, during which time she and President
Trump clashed quite a bit. She had questions about the legality of it.
She had questions about the practicality and how the public was going to
respond. And she had just a lot of reservations about moving forward.
But ultimately, she agreed, and the policy was introduced in a formal
way in April.

\begin{itemize}
\item
  archived recording 1\\
  Quit separating the kids! They're separating the children. Mr.
  President, don't you have kids? Don't you have kids, Mr. President?
\item
  archived recording 2\\
  I don't believe in this. This is not America. This is not our country.
  This is not what we should be doing.
\item
  archived recording (kirstjen nielsen)\\
  This administration did not create a policy of separating families at
  the border.
\item
  archived recording\\
  Shame on everybody that separates children and allows them to stay at
  the other side of the border, fearing death, and allow the secretary
  to come here and lie.
\item
  archived recording (kirstjen nielsen)\\
  Calling me a liar are fighting words. I'm not a liar. We've never had
  a policy for family separation. And let's be clear --- if an American
  were to commit a crime anywhere in the United States, they would go to
  jail, and they would be separated from their family. This is not a
  controversial idea.
\end{itemize}

caitlin dickerson

Internally, she's, if not resisting, she's certainly questioning and
slowing down a lot of these policies. And then, externally, to the
public, she's having to defend them and really look like the face of
them, and that ends up being something that she does actually over and
over again in her job.

michael barbaro

And remind us what ends up happening to the family separation policy.

caitlin dickerson

So President Trump stopped the practice himself by signing an executive
order.

\begin{itemize}
\tightlist
\item
  archived recording (donald trump)\\
  I consider it to be a very important executive order. It's about
  keeping families together.
\end{itemize}

caitlin dickerson

But it didn't matter, because a few days later, a federal judge
intervened, deemed the practice unconstitutional, and he didn't stop
there. He said that ---

\begin{itemize}
\tightlist
\item
  archived recording\\
  Families who have been separated as a result of the Trump
  administration's zero-tolerance policy be reunited within 30 days. The
  order says kids under 5 must be back with their parents within 14 days
  from now.
\end{itemize}

caitlin dickerson

So sort of a resounding disavowal of this policy.

michael barbaro

By the legal system.

caitlin dickerson

Exactly. And so even though she'd pushed back against it, Secretary
Nielsen ends up taking the blame.

michael barbaro

So what's another example something that she expresses internal qualms
about but then goes out and defends publicly?

caitlin dickerson

So one of the next policies to be introduced was an idea to limit asylum
pretty significantly so that people could only apply if they showed up
at a legal port of entry --- whereas beforehand, people could and did
very often show up anywhere along the border, present themselves to a
border agency, ask for protection, and they were ushered into this legal
process. The idea was to say, you can't do that anymore. You have to
show up at a specific office, which is, of course, a pretty tall order
for somebody who's coming from Central America with maybe a cell phone,
maybe not, not a whole lot of resources. They don't always know exactly
where these offices exist.

\begin{itemize}
\tightlist
\item
  archived recording (kirstjen nielsen)\\
  You do not need to break the law of this country by entering illegally
  to claim asylum. If you are seeking asylum, go to a port of entry.
\end{itemize}

caitlin dickerson

But as Nielsen pointed out to the president, it was very hard to
justify, because immigration law explicitly says that you can request
asylum regardless of where you enter.

michael barbaro

So again, Nielsen is saying inside the White House, hey, this might not
be legal. I have reservations about it. But it nevertheless gets
implemented. And as you've said, she's the decider. So she signs off on
it.

caitlin dickerson

Exactly. And it's important to point out that when Nielsen pushed back
against President Trump, it wasn't a reflection of her being a liberal
on immigration or having a more sympathetic view toward asylum seekers.
She just happened to be the face of an entire agency full of people who
work in immigration enforcement but who still made clear to her that
there were going to be legal and logistical roadblocks to putting these
ideas into place.

michael barbaro

And what happens to that port of entry policy that would limit the
number of locations that people can apply for asylum?

caitlin dickerson

Very quickly after, it's blocked by federal courts, and it remains that
way.

\begin{itemize}
\tightlist
\item
  archived recording\\
  A district court ruled that the ban conflicts with immigration law.
\end{itemize}

caitlin dickerson

So this was the executive branch trying to very clearly contradict laws
that were approved by the legislative branch.

\begin{itemize}
\tightlist
\item
  archived recording\\
  And now the Supreme Court has refused to step in to unblock the ban.
\end{itemize}

michael barbaro

So her legal reservations are well-founded. In other words, her advice
to the president is correct, even if it's ignored.

caitlin dickerson

That's right. Her analysis is sound, because each time she is pushed
back, she's been right. And these ideas have been blocked.

michael barbaro

Do any of these policies that we're describing --- family separation,
limiting the locations at which people can seek asylum --- do any of
them succeed in curbing the flow of migrants, even for the very short
period of time that they're in place before they are legally challenged?

caitlin dickerson

They don't.

\begin{itemize}
\tightlist
\item
  archived recording\\
  New numbers show a 400 percent --- 4-0-0, that is --- 400 percent
  increase over just last year.
\end{itemize}

caitlin dickerson

Last October, we saw a record number of families seek asylum in the
United States. This is after family separation is introduced. And every
month since then, the numbers have gotten even higher.

\begin{itemize}
\tightlist
\item
  archived recording\\
  C.B.P. officials say Border Patrol agents are on pace for
  apprehensions and encounters with more than 100,000 migrants in March.
\end{itemize}

michael barbaro

So not only are these policies being blocked in court, for the very
short bit of time that they were ever in place, they're not doing their
job of limiting the flow of migrants.

caitlin dickerson

They're not. The numbers keep going up. And some people would argue that
maybe the policies weren't left in place for long enough to show any
concrete outcome, and we can't know that for sure. But what we do know
is that these policies that Nielsen was pushing back on, saying they're
not going to work, didn't work. And the numbers have continued to rise.

michael barbaro

And so what does that do to the president's relationship to Nielsen?
She's in a sense being vindicated, but the problem she's there to solve
is only getting worse.

caitlin dickerson

She's simultaneously sort of being vindicated if you look at it from a
legal standpoint, but from a relationship standpoint, she's the face, in
President Trump's eyes, of these failures.

\begin{itemize}
\item
  archived recording 1\\
  Kirstjen Nielsen appears to be on shaky ground. That's according to a
  New York Times report. Nielsen told colleagues that she was close to
  resigning Wednesday after being berated by the president in front of
  the entire cabinet.
\item
  archived recording 2\\
  Nothing sets him off more than immigration, and I'm told that meeting
  on Wednesday in the cabinet room was very heated, was incredibly
  heated. She spoke back to him about it, tried to defend herself.
\end{itemize}

caitlin dickerson

So it's in this context of the policies that are being introduced being
blocked in the courts, the number of people crossing the border rising,
and Trump's relationship with Secretary Nielsen falling apart that he
comes up with his most aggressive idea yet.

\begin{itemize}
\tightlist
\item
  archived recording\\
  Another day, another threat from President Trump, and today, he is
  threatening to shut down the southern border.
\end{itemize}

caitlin dickerson

Where he wants to completely seal the border, 100 percent. Not let
anybody come into the U.S.

\begin{itemize}
\tightlist
\item
  archived recording (donald trump)\\
  And if they don't stop them, we're closing the border. We'll close it,
  and we'll keep it closed for a long time. I'm not playing games.
\end{itemize}

michael barbaro

And how exactly would that work?

caitlin dickerson

Well, we shouldn't assume that the president had any particular policy
in his mind when he tweeted that he wanted to seal the border. We can
assume, though, that he wasn't talking about, for example, cargo moving
back and forth or people with actual visas and permission to come into
the United States, but really that he's talking about shutting down
asylum and shutting down the ability of people who don't previously have
permission to come into the United States. And the policy that we know
has come closest to achieving that goal, one that's been kicking around
in Washington for some months now, would be to get rid of asylum as we
know it. It would no longer allow anyone to apply for asylum in the
United States. And instead, people who needed that status or wanted that
status would have to stay in their home country, similar to the way that
Syrian refugees apply to come to the United States --- apply from home,
wait many months, go through lots of vetting and background checks, and
then only if they're approved, they would be allowed to come here.

michael barbaro

And I guess, what could be a more extreme version of limiting asylum
than literally telling people, you cannot come to this country and apply
for asylum?

caitlin dickerson

I don't think there is a more extreme version, because I think this idea
means eliminating asylum. It goes away.

michael barbaro

And as best we understand it, what was Secretary Nielsen's response to
this idea?

caitlin dickerson

From what we know, it's been very similar to the way she reacted to
family separation, to that idea to significantly limit asylum to the
ports of entry, which is that this is going to be challenged by the
courts immediately. It's going to be a huge lift to get a policy like
this introduced, to work out the logistics, and all of it will be for
naught, because it'll be enjoined by the courts. And it's that
oppositional and resistant stance that she's in when she walks into the
White House for her meeting with the president.

It's unclear who threw up their hands first or second, or whether they
both did it, but what we know is that the president did not leave her
with an option, that he, at least, decided this isn't going to work.
You're out. And she was by the end of that night.

And so what that means is the homeland security secretary who oversaw
some of the most controversial and aggressive immigration policies this
country has ever seen --- even she wasn't aggressive enough for
President Trump. And so now she's gone, and he's looking for a
replacement who will go even further.

michael barbaro

We'll be right back.

So Caitlin, who is the president turning to to replace Nielsen as he
looks for somebody to go further than she was willing to go?

caitlin dickerson

He's turning to Kevin McAleenan, who's currently the head of United
States Customs and Border Protection. That's the agency that oversees
both the customs officers who you meet at the airport when you've gone
on vacation abroad and then the Border Patrol. Those are the police, the
boots on the ground along the border. And McAleenan is known as sort of
a policy wonk, a really smart and reasonable guy who's willing to work
with Democrats, and somebody who served under President Obama as well.
He's got more than 10 years of experience at the agency. People know him
as this reasonable guy. But when you dig a little bit deeper, McAleenan
was inspired to get into homeland security work right after 9/11. And at
C.B.P., especially most recently when he was overseeing this agency, it
really became known as the sort of policy engine with him at the center
of it coming up with ways to turn President Trump's ideas into actual
policies that could be carried out. If D.H.S. is this agency run out of
Washington that's overseeing all of this work, it was McAleenan's staff
members that were actually physically taking children away from their
parents, that were physically turning asylum seekers away at the border
or telling them to go elsewhere or to wait in line. So now he's moving
into this role where he's got even more power, but I think we can expect
that he'll continue to do what he was doing before, which was take
President Trump's ideas that are sort of extreme and that aren't
necessarily encumbered by the immigration laws, norms, history, and
he'll try to translate them into something that's practical and that can
be introduced on the ground.

michael barbaro

So even though he might seem moderate at first glance, he's actually
someone who the president trusts to put these more restrictive, maybe
even extreme policies into practice. And he's been involved in doing
that already. Now he's being elevated to do it at an even higher level.

caitlin dickerson

Exactly.

michael barbaro

And I guess that all makes sense. If Nielsen is seen as too reluctant to
put these policies into practice, then the president would turn to
somebody who is willing to do that.

caitlin dickerson

That's right, who is willing to try. But remember that McAleenan is
going to run up against the exact same legal framework and logistical
challenges that anybody else in that role is going to.

michael barbaro

Right. So if these policies are going to run into legal trouble in the
courts, regardless of who is leading the agency, what does it matter, in
the end, if the leader of the Department of Homeland Security is gung-ho
about the President's policies or is reluctantly saying yes, as Nielsen
was?

caitlin dickerson

I think you can look at that question a couple of different ways. On one
hand, you're right that it's not going to make a huge difference who's
sitting in the office at the head of the Homeland Security Department,
because no matter what they introduce, if it violates the immigration
law, they're going to wind up in court. But another way of looking at it
is family separation only existed as an official sort of policy for
what, 45 days under zero tolerance? Not a very long period of time
before it had to end. But still, I don't think that very many people are
going to argue that family separation was a small thing or something
that didn't affect very many people. I mean, I think it had a huge
impact. So I do think that if you have a hawk running the Homeland
Security Department, even when they're sort of encumbered by the
existing legal framework, they can make some pretty big changes, even
if, ultimately, those changes wind up in court. I think that Kevin
McAleenan has two things going for him in President Trump's eyes. The
first, like we said, is that he knows the policies and he knows the
situation on the ground like the back of his hand. And the second thing
is that he's shown a willingness to follow President Trump's lead, and
so he may be the person who's best positioned to come up with the most
legal ways to achieve President Trump's goals. And even if the policies
that he comes up with don't remain in place in the long term, they might
at least exist long enough to give President Trump something to point to
and show his supporters when they ask, where are you on these campaign
promises? Where are you on this idea of, if not sealing the border,
significantly decreasing the number of immigrants coming to the United
States?

michael barbaro

So even if these policies are ultimately blocked by the courts, it feels
like it's important to this president to have someone at the Department
of Homeland Security who is willing to try them, and that may be enough.

caitlin dickerson

I think in President Trump's heart of hearts, right, he would hope that
he would find not only somebody who is willing to try but somebody who
is going to succeed. He wants aggressive policies to be instituted in
the long term. But I think it's better in his mind than nothing to have
somebody who's willing to try. And as we know, politically, it's better
than nothing, because then the president can point to these policies
that he tried to introduce, if only it wasn't for the courts who had
blocked him, or if only it wasn't for Congress who blocked him.

michael barbaro

Right. My intention was there. Somebody else is at fault for it not
working.

caitlin dickerson

Exactly. He has at least an attempt to meet these campaign promises to
point to when voters ask, why haven't you made a significant change? So
I think that having a more aggressive leader in place who's more willing
to introduce these policies, when they get stuck in the courts, it
allows Trump to blame the courts rather than his own administration for
getting in his way. He can say, look, I tried, but these judges are
blocking me, whereas it wasn't just the judges blocking him with
Secretary Nielsen. It was her, too.

michael barbaro

If Congress is not going to change the law, and I think the assumption
is that they are not going to do that anytime soon when it comes to
basic immigration law, is the president right to suggest that the
current system is broken, and the only way to fix it is to test these
new ideas, even if they push the boundaries of the law?

caitlin dickerson

I would say yes to the first part of your question. Most people agree
that the system is broken. But when it comes to fixing the problem, what
are you trying to fix? In the president's mind, the problem is the vast
number of people coming here to seek asylum. His idea of a fix would
decrease that number. But other people see the problem as not that
people are coming here to seek asylum at all but the ways that we're
dealing with them. We don't have space to place people physically in
custody when they enter. We don't have judges to hear their cases for
years and years. So if you see that as the problem, then ---

michael barbaro

He's not really addressing those.

caitlin dickerson

He's not addressing those problems, and I think that's because a lot of
people who agree with the president are concerned that if we come up
with better systems for processing asylum seekers --- we make it more
organized and efficient and, in some cases, comfortable --- that all
that will do is encourage more and more people to come. And again, if
the numbers themselves are what you see as the problem, then making the
system better only makes it worse.

michael barbaro

Caitlin, what does all of this tell us about where President Trump plans
to go on immigration?

caitlin dickerson

I think it shows that President Trump is going to continue full steam
ahead to achieve his immigration goals and that if there are people,
even informed career officials, who want to stand in the way of that,
they're going to be pushed out, because he's very committed to the goal
of limiting the number of people who come to the border, regardless of
the many legal challenges he's already faced, those that are sure to
come, the public reaction, all these other things notwithstanding. We're
not seeing any signs of slowing down. I think this week is actually an
indicator of a ramping up of these goals.

michael barbaro

In other words, the gloves are coming off, even though, in a lot of
people's minds, they thought the gloves were already off.

caitlin dickerson

That's right. It's like another set of gloves are coming off.

michael barbaro

Caitlin, thank you very much.

caitlin dickerson

Thank you.

michael barbaro

On Monday, The Times reported that President Trump plans to push out
more officials from the Department of Homeland Security, including the
department's general counsel and the director of U.S. Citizenship and
Immigration Services, as he seeks to carry out his harder-line approach
to immigration. A few hours later, in the latest legal setback to that
approach, a federal judge blocked a Trump administration policy that
required those seeking asylum to wait in Mexico rather than in the U.S.
while their cases made their way through U.S. immigration court. The
judge found that the policy violated federal law.

Here's what else you need to know today.

\begin{itemize}
\tightlist
\item
  archived recording (mike pompeo)\\
  Today, the United States is continuing to build its maximum pressure
  campaign against the Iranian regime. I am announcing our intent to
  designate the Islamic Revolutionary Guard Corps, including its Quds
  Force, as a foreign terrorist organization in accordance with Section
  219 of the Immigration and Nationality Act.
\end{itemize}

michael barbaro

On Monday, the Trump administration said that it was designating a
powerful arm of the Iranian military as a foreign terrorist
organization, the first time that the U.S. has classified part of any
country's government as such a threat.

\begin{itemize}
\tightlist
\item
  archived recording (mike pompeo)\\
  We're doing it because the Iranian regime's use of terrorism as a tool
  of statecraft makes it fundamentally different from any other
  government.
\end{itemize}

michael barbaro

The move was debated at the highest levels of the administration, with
top officials at the Defense Department and the C.I.A. opposing the
designation, arguing it could justify Iranian attacks against the U.S.
and its allies. But the president's national security adviser, John
Bolton, and his secretary of state, Mike Pompeo, advocated for the
decision, arguing the designation would further isolate Iran by
discouraging businesses from working with its military.

\begin{itemize}
\tightlist
\item
  archived recording (mike pompeo)\\
  This historic step will deprive the world's leading state sponsor of
  terror the financial means to spread misery and death around the
  world.
\end{itemize}

michael barbaro

In response, Iran's government said it was designating the U.S. Central
Command, which oversees military operations in the Middle East, as a
terrorist organization as well.

That's it for ``The Daily.'' I'm Michael Barbaro. See you tomorrow.

The move comes just two days after Mr. Trump, who has repeatedly
expressed anger at a rise in migrants at the southwestern border,
\href{https://www.nytimes3xbfgragh.onion/2019/04/05/us/politics/ronald-vitiello-ice.html}{withdrew
his nominee} to run Immigration and Customs Enforcement because he
wanted the agency to go in a ``tougher'' direction.

Mr. Trump has ratcheted up his anti-immigration message in recent months
as he seeks to galvanize supporters before the 2020 election, shutting
down the government and then declaring a national emergency to secure
funding to build a border wall, cutting aid to Central American
countries and repeatedly denouncing what he believes is a crisis of
migrants trying to enter the country.

\includegraphics{https://static01.graylady3jvrrxbe.onion/images/2019/04/07/us/politics/07dc-nielsen2/merlin_139816341_45fa829e-2d8a-4965-850d-ca6a187cb82c-articleLarge.jpg?quality=75\&auto=webp\&disable=upscale}

He took aim again Sunday night after announcing Ms. Nielsen's departure,
\href{https://twitter.com/realDonaldTrump/status/1115057524770844672}{tweeting},
``Our Country is FULL!''

Ms. Nielsen
\href{https://twitter.com/SecNielsen/status/1115080823068332032}{said}
she planned ``to stay on as secretary through Wednesday'' in order ``to
assist with an orderly transition.'' The abruptness was unusual because
the Department of Homeland Security currently does not have a deputy
secretary, who would normally take the reins.

The president said in a tweet that Kevin McAleenan, the commissioner of
Customs and Border Protection, would take over as the acting replacement
for Ms. Nielsen, who became the sixth secretary to lead the agency in
late 2017. But by law, the under secretary for management, Claire Grady,
who is currently serving as acting deputy secretary, is next in line to
be acting secretary. The White House will have to fire her to make Mr.
McAleenan acting secretary, people familiar with the transition said.
Ms. Grady has told colleagues that she has no intention of resigning to
make way for Mr. McAleenan.

\emph{{[}}\href{https://www.nytimes3xbfgragh.onion/2019/04/07/us/politics/kevin-mcaleenan-dhs-cbp.html}{\emph{Read
about Kevin McAleenan, the new acting homeland security
secretary}}\emph{.{]}}

Among the possible replacements for Ms. Nielsen in the long term is Ken
Cuccinelli, the former Virginia attorney general who is a favorite among
conservative activists and who fits the profile that Mr. Trump wants the
next homeland secretary to have, people familiar with the discussions
said.

Ms. Nielsen had been pressured by Mr. Trump to be more aggressive in
stemming the influx of
\href{https://www.nytimes3xbfgragh.onion/2019/03/05/us/border-crossing-increase.html}{migrant
crossings} at the border, people familiar with their discussions in
recent months said.

Her entire time in the job was spent batting back suspicion from the
president, even as he told people he liked how she performed on
television and enjoyed dealing with her personally. He initially was
skeptical because of Ms. Nielsen's previous service in the George W.
Bush administration, and then because she was close to John F. Kelly,
Mr. Trump's former chief of staff.

\hypertarget{kirstjen-nielsens-resignation-letter}{%
\subsection{Kirstjen Nielsen's Resignation
Letter}\label{kirstjen-nielsens-resignation-letter}}

In her letter resigning as secretary of homeland security, Kirstjen
Nielsen wrote: ``We have taken unprecedented action to protect
Americans. We have implemented historic efforts to defend our borders,
combat illegal immigration, obstruct the inflow of drugs, and uphold our
laws and values.''

\includegraphics{https://int.graylady3jvrrxbe.onion/data/documenthelper/747-kirstjen-nielsen-resignation-letter/16535b3bb1a4a097c5ca/optimized/thumbnail.png}

The president called Ms. Nielsen at home early in the mornings to demand
that she take action to stop migrants from entering the country,
including doing things that were clearly illegal, such as blocking all
migrants from seeking asylum. She repeatedly noted the limitations
imposed on her department by federal laws, court settlements and
international obligations.

Those responses only infuriated Mr. Trump further. The president's fury
erupted in the spring of 2018 as Ms. Nielsen hesitated for weeks about
whether to sign a memo ordering the routine separation of migrant
children from their families so that the parents could be detained.

In a cabinet meeting surrounded by her peers, Mr. Trump castigated her
repeatedly, leading her to draft a resignation letter and to tell
colleagues that there was no reason for her to lead the department any
longer. By the end of the week, she had reconsidered and remained in her
position, becoming an increasingly fierce supporter of his policies,
including the family separations.

Mr. Trump and Stephen Miller, the president's top immigration adviser,
have privately but regularly complained about Ms. Nielsen. Lou Dobbs, a
Fox News host who is one of the president's favorite sounding boards,
has also encouraged Mr. Trump's negative views of her handling of the
migrant crisis.

Ms. Nielsen lost a powerful protector when Mr. Kelly, her mentor, left
his job as White House chief of staff at the beginning of the year. Mr.
Kelly was the Trump administration's first homeland security secretary
and lobbied for Ms. Nielsen to replace him.

Multiple White House officials said she had grown deeply paranoid in
recent months, after numerous stories about her job being on the line.
She also had supported the Immigration and Customs Enforcement nominee
Mr. Trump withdrew, Ronald D. Vitiello, and her support for him was
described as problematic for her with the president. Mr. Trump felt Mr.
Vitiello did not favor closing the border, as the president threatened
again to do
\href{https://twitter.com/realDonaldTrump/status/1115053022974029824}{in
a tweet} on Sunday night.

\href{https://www.nytimes3xbfgragh.onion/interactive/2018/03/16/us/politics/all-the-major-firings-and-resignations-in-trump-administration.html}{}

\includegraphics{https://static01.graylady3jvrrxbe.onion/images/2018/07/05/us/all-the-major-firings-and-resignations-in-trump-administration-promo-1530825933054/all-the-major-firings-and-resignations-in-trump-administration-promo-1530825933054-articleLarge-v2.jpg}

\hypertarget{the-turnover-at-the-top-of-the-trump-administration}{%
\subsection{The Turnover at the Top of the Trump
Administration}\label{the-turnover-at-the-top-of-the-trump-administration}}

Since President Trump's inauguration, White House staffers and cabinet
officials have left in firings and resignations, one after the other.

In early 2019, as the number of migrant families from Central American
countries surged, the president's fury at Ms. Nielsen did, too. He
repeatedly demanded that she cut off foreign aid to Central American
countries even though the funding was the responsibility of the State
Department. She repeatedly deflected his demands.

One day after Ms. Nielsen traveled to Honduras to sign a regional
compact with officials from Guatemala, Honduras and El Salvador, Mr.
Trump
\href{https://www.nytimes3xbfgragh.onion/2019/03/29/us/politics/trump-mexico-illegal-immigration.html}{cut
State Department funding} for the countries. And in recent days, the
president made public moves to undercut her authority, leaking news that
he might nominate an ``immigration czar'' to assume oversight of the
issue at the heart of Ms. Nielsen's department.

Still, Ms. Nielsen embraced the president's ``crisis'' language as
apprehensions of migrants at the border shot up to thousands per day. On
Friday, Mr. Trump
\href{https://www.nytimes3xbfgragh.onion/2019/04/05/us/politics/trump-border-wall.html}{traveled
with Ms. Nielsen and Mr. McAleenan} to Calexico, Calif., to highlight
the issue.

While the number of border crossings is not as high as in the
\href{https://www.nytimes3xbfgragh.onion/2018/06/20/us/politics/fact-check-trump-border-crossings-declining-.html?module=inline}{early
2000s}, the demographic of migrants has shifted largely from individual
Mexicans looking for jobs --- who could easily be deported --- to
Central American families, overwhelming detention facilities and
prompting mass releases of migrants into cities along the border.

Ms. Nielsen estimated last month that border officials had stopped as
many as 100,000 migrants in March.

But despite the trip and several stories about how much better her
relationship with Mr. Trump was, Ms. Nielsen never learned how to manage
him, people familiar with their discussions said. He often felt lectured
to by Ms. Nielsen, the people familiar with the discussions said.

Image

Ms. Nielsen lost a powerful protector when John F. Kelly, her mentor and
the Trump administration's first homeland security secretary, left his
job as White House chief of staff at the beginning of the
year.Credit...Al Drago for The New York Times

And his son-in-law, Jared Kushner, was not an admirer of Ms. Nielsen,
several administration officials said. That came to a head recently as
Mr. Kushner had inserted himself into immigration discussions.

While Mr. Trump often blamed Ms. Nielsen for the surge in migrant
crossings, she will be remembered for leading the department during the
Trump administration's ``zero tolerance'' policy along the southwestern
border, which initially resulted in the separation of thousands of
migrant children from their families.

An intense backlash ensued, and the Department of Homeland Security was
unprepared to deal with separating nearly 3,000 children from their
parents.

``Hampered by misstep after misstep, Kirstjen Nielsen's tenure at the
Department of Homeland Security was a disaster from the start,'' said
Representative Bennie Thompson, a Mississippi Democrat and the chairman
of the House's committee on Homeland Security. ``It is clearer now than
ever that the Trump administration's border security and immigration
policies --- that she enacted and helped craft --- have been an abysmal
failure and have helped create the humanitarian crisis at the border.''

Mr. Trump eventually moved to halt the family separations, though the
government struggled in some cases to reunite those it had already
separated.

By naming Mr. McAleenan acting secretary, Mr. Trump is installing
another veteran of previous administrations, not a loyal foot soldier of
Mr. Trump's campaign.

Married to a Salvadoran immigrant, Mr. McAleenan is a lawyer who wrote
an honors thesis at Amherst College on marriage equality and applied at
the F.B.I. after the Sept. 11 attacks.

Described by colleagues as a savvy political operator, Mr. McAleenan
worked cooperatively with Obama administration officials but later
embraced Mr. Trump's agenda, which included unshackling Border Patrol
agents from restrictions that the previous administration had imposed.

Mr. McAleenan was also one of three Department of Homeland Security
officials who had urged Ms. Nielsen to sign the memo authorizing the
routine separation of migrant families at the border.

The department, which has a budget of more than \$40 billion and more
than 240,000 employees, is an amalgam of 22 government agencies that was
created after the Sept. 11 terrorist attacks. It is responsible for
everything from protecting the nation from cyberattacks to responding to
natural disasters.

At 46, Ms. Nielsen was the youngest person to lead the sprawling
department, and an unlikely choice for the job.

In the months immediately after the Sept. 11 attacks, she helped set up
the Transportation Security Administration, now an agency within the
department. She also worked as a special assistant to President George
W. Bush on natural disaster response while serving on the White House
Homeland Security Council.

When Mr. Trump moved Mr. Kelly to the White House in July 2017, Ms.
Nielsen moved with him. As the principal deputy chief of staff, she
enforced Mr. Kelly's attempts to regulate access to Mr. Trump in the
Oval Office, including the president's schedule --- irritating White
House staff members, who complained she was uncompromising.

Mr. Kelly later backed Ms. Nielsen to succeed him at the Homeland
Security Department, though she was criticized as too inexperienced for
the job by Democrats and anti-immigration groups. Mr. Trump, however,
said she was ``ready on Day 1.''

``There will be no on-the-job training for Kirstjen,'' Mr. Trump said in
October 2017, announcing her nomination for the post.

But by the following spring, Ms. Nielsen was telling associates she was
miserable in the job.

Advertisement

\protect\hyperlink{after-bottom}{Continue reading the main story}

\hypertarget{site-index}{%
\subsection{Site Index}\label{site-index}}

\hypertarget{site-information-navigation}{%
\subsection{Site Information
Navigation}\label{site-information-navigation}}

\begin{itemize}
\tightlist
\item
  \href{https://help.nytimes3xbfgragh.onion/hc/en-us/articles/115014792127-Copyright-notice}{©~2020~The
  New York Times Company}
\end{itemize}

\begin{itemize}
\tightlist
\item
  \href{https://www.nytco.com/}{NYTCo}
\item
  \href{https://help.nytimes3xbfgragh.onion/hc/en-us/articles/115015385887-Contact-Us}{Contact
  Us}
\item
  \href{https://www.nytco.com/careers/}{Work with us}
\item
  \href{https://nytmediakit.com/}{Advertise}
\item
  \href{http://www.tbrandstudio.com/}{T Brand Studio}
\item
  \href{https://www.nytimes3xbfgragh.onion/privacy/cookie-policy\#how-do-i-manage-trackers}{Your
  Ad Choices}
\item
  \href{https://www.nytimes3xbfgragh.onion/privacy}{Privacy}
\item
  \href{https://help.nytimes3xbfgragh.onion/hc/en-us/articles/115014893428-Terms-of-service}{Terms
  of Service}
\item
  \href{https://help.nytimes3xbfgragh.onion/hc/en-us/articles/115014893968-Terms-of-sale}{Terms
  of Sale}
\item
  \href{https://spiderbites.nytimes3xbfgragh.onion}{Site Map}
\item
  \href{https://help.nytimes3xbfgragh.onion/hc/en-us}{Help}
\item
  \href{https://www.nytimes3xbfgragh.onion/subscription?campaignId=37WXW}{Subscriptions}
\end{itemize}
