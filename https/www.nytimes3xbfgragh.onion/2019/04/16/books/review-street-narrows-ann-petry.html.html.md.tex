Sections

SEARCH

\protect\hyperlink{site-content}{Skip to
content}\protect\hyperlink{site-index}{Skip to site index}

\href{https://www.nytimes3xbfgragh.onion/section/books}{Books}

\href{https://myaccount.nytimes3xbfgragh.onion/auth/login?response_type=cookie\&client_id=vi}{}

\href{https://www.nytimes3xbfgragh.onion/section/todayspaper}{Today's
Paper}

\href{/section/books}{Books}\textbar{}May's Book Club Pick: Two Novels
by Ann Petry, a Writer Who Believed in Art That Delivers a Message

\url{https://nyti.ms/2VMugjy}

\begin{itemize}
\item
\item
\item
\item
\item
\item
\end{itemize}

Advertisement

\protect\hyperlink{after-top}{Continue reading the main story}

Supported by

\protect\hyperlink{after-sponsor}{Continue reading the main story}

\href{/column/books-of-the-times}{Books of The Times}

\hypertarget{mays-book-club-pick-two-novels-by-ann-petry-a-writer-who-believed-in-art-that-delivers-a-message}{%
\section{May's Book Club Pick: Two Novels by Ann Petry, a Writer Who
Believed in Art That Delivers a
Message}\label{mays-book-club-pick-two-novels-by-ann-petry-a-writer-who-believed-in-art-that-delivers-a-message}}

By \href{https://www.nytimes3xbfgragh.onion/by/parul-sehgal}{Parul
Sehgal}

\begin{itemize}
\item
  April 16, 2019
\item
  \begin{itemize}
  \item
  \item
  \item
  \item
  \item
  \item
  \end{itemize}
\end{itemize}

\includegraphics{https://static01.graylady3jvrrxbe.onion/images/2019/04/17/arts/16bookpetry1/16bookpetry1-articleLarge.jpg?quality=75\&auto=webp\&disable=upscale}

Buy Book ▾

\begin{itemize}
\tightlist
\item
  \href{https://www.amazon.com/gp/search?index=books\&tag=NYTBSREV-20\&field-keywords=The+Street+and+The+Narrows+Ann+Petry}{Amazon}
\item
  \href{https://du-gae-books-dot-nyt-du-prd.appspot.com/buy?title=The+Street+and+The+Narrows\&author=Ann+Petry}{Apple
  Books}
\item
  \href{https://www.anrdoezrs.net/click-7990613-11819508?url=https\%3A\%2F\%2Fwww.barnesandnoble.com\%2Fw\%2F\%3Fean\%3D9781598536010}{Barnes
  and Noble}
\item
  \href{https://www.anrdoezrs.net/click-7990613-35140?url=https\%3A\%2F\%2Fwww.booksamillion.com\%2Fp\%2FThe\%2BStreet\%2Band\%2BThe\%2BNarrows\%2FAnn\%2BPetry\%2F9781598536010}{Books-A-Million}
\item
  \href{https://bookshop.org/a/3546/9781598536010}{Bookshop}
\item
  \href{https://www.indiebound.org/book/9781598536010?aff=NYT}{Indiebound}
\end{itemize}

When you purchase an independently reviewed book through our site, we
earn an affiliate commission.

Not all who hide wish to be found.

Ann Petry's first novel, ``The Street,'' was a literary event in 1946,
praised and translated around the world --- the first book by a black
woman to sell more than a million copies. It's the story of a
catastrophe in agonizingly slow motion. A mother and her young son
living in Harlem in the 1940s are ground down by poverty and the bitter
racism and constant predation in their neglected neighborhood.

``Streets like the one she lived on were no accident. They were the
North's lynch mobs,'' the mother, Lutie, thinks. ``The method the big
cities used to keep Negroes in their place.'' The book was greeted as a
female counterpart to Richard Wright's ``Native Son,'' a new classic of
social realism and one of the early (and only) glimpses into the lives
of black working-class women. Its author was feted and photographed and
made utterly miserable.

Petry experienced celebrity as a kind of spiritual theft. ``My soul was
no longer my own,'' she
\href{https://www.nytimes3xbfgragh.onion/1992/01/08/books/an-author-s-look-at-1940-s-harlem-is-being-reissued.html}{recalled
to The Times} in 1992. ``I was a black woman at a point in time when
being a writer was not usual, and I was besieged. Everyone wanted a part
of me.'' She fled Harlem for her hometown in Connecticut, where she
\href{https://www.nytimes3xbfgragh.onion/1997/04/30/arts/ann-petry-88-first-to-write-a-literary-portrait-of-harlem.html}{lived
in seclusion until her death}, at 88, in 1997. The threat of exposure
still loomed, and she destroyed her letters and other pieces of writing,
and remained ambivalent whenever her work was reissued: ``I feel as
though I were a helpless creature impaled on a dissecting table --- for
public viewing,'' she wrote in her journal.

The Library of America recently published ``The Street'' in one volume
along with Petry's 1953 masterpiece, ``The Narrows,'' and a sampling of
her critical writing, edited by Farah Jasmine Griffin, a professor of
English and African American Literature at Columbia University and the
author of ``Harlem Nocturne,'' a group biography of radical women
artists in the 1940s, including Petry. Read together, these works by
Petry reveal, with fluorescent clarity, the through line between the
life and the work --- an intimate knowledge, and horror, of
surveillance.

\emph{{[} Read Tayari Jones's}
\href{https://www.nytimes3xbfgragh.onion/2018/11/10/books/review/in-praise-of-ann-petry.html}{\emph{appreciation
of Ann Petry}}\emph{. {]}}

``In `The Street' my aim is to show how simply and easily the
environment can change the course of a person's life,'' Petry once said.
It's impossible not to connect her own childhood to that privacy she
found crucial for self-preservation. Her family was one of four black
households in small-town Connecticut and faced routine harassment. She
was ordered to leave a public beach as a child, and pelted with stones
as she walked to school, where white teachers refused to instruct her.
These experiences of feeling scrutinized, even hunted --- and her
observations of Harlem from her time writing for Adam Clayton Powell's
newspaper The People's Voice --- course through her fiction. The fact of
American racism is so large and encompassing it finds personification in
her novels as the elements themselves, the winds that assault the
characters, the fog that blinds them.

Image

Ann PetryCredit...Carl Van Vechten

``She never felt really human until she reached Harlem,'' Petry says of
Lutie. Only there, ``away from the hostility in the eyes of the white
women'' and the ``openly appraising looks of the white men,'' can she
feel free --- if only for a little while. Mrs. Hedges, who runs a
brothel in their apartment building, takes a covetous interest in her.
Worse, there is the building's sinister super; ``she could feel his eyes
traveling over --- estimating her, summing her up, wondering about
her.'' He wants her terribly, and sneaks into her bedroom when she's
out, to fondle her clothes. When rejected, he coolly decides to ``fix
her good'' --- how convenient that she has such a vulnerable young son.

Petry's characters are watched constantly but cannot themselves see. In
the first scene of ``The Street,'' Lutie struggles to read a ``for
rent'' sign in the middle of a gale; in ``The Narrows,'' a black man and
a white woman meet in the middle of a dense fog, each mistaking the race
of the other.

It is just the beginning of their troubles. The novel is often (and
naïvely) described as an interracial romance between Camilo, a white
heiress, and Link, a black war veteran. A pact of assured mutual
destruction is more accurate. They cannot recognize each other's
humanity. To Camilo, Link is inexorably his race; to Link, Camilo is a
beautiful machine, ``a racehorse or an airplane, all the essential parts
in the exact right place.'' They have been designed to destroy each
other.

Petry wrote unabashed protest art, in the mode of Steinbeck and Stephen
Crane. ``It seems to me that all truly great art is propaganda, whether
it be the Sistine Chapel or La Gioconda, `Madame Bovary' or `War and
Peace,''' she wrote in her essay ``The Novel as Social Criticism,''
which is included in this volume. ``The moment the novelist begins to
show how society affected the lives of his characters, how they were
formed and shaped by the sprawling inchoate world in which they lived,
he is writing a novel of social criticism whether he calls it that or
not.''

This Petry --- the surprisingly pugnacious practitioner-critic rising to
the defense of her own work, as social realism was falling out of
fashion --- might be my favorite of all her registers, but she does her
writing a slight disservice. Her work endures (despite her own efforts)
not merely because of the strength of its message but its artistry.
``Sometimes when a writer is regarded as `before her time,' we don't
quite understand that the same work is still right on time,'' the
novelist
\href{https://www.nytimes3xbfgragh.onion/2018/11/10/books/review/in-praise-of-ann-petry.html}{Tayari
Jones has written} in praise of Petry.

Petry will \emph{always} feel on time. Her kind of talent will always
feel startling and sui generis: The music of her sentences, and their
discipline; her unerring sense of psychology; the fullness with which
she endows each character, which must be understood as a kind of love;
the plots that commandeer whole hours and days. (I am writing this
review in a swivet of shame, in fact, in the baleful eyeline of an
unwalked dog, unwashed dishes, unanswered emails.) Her work endures not
only because it illuminates reality, but because it harnesses the power
of fiction to \emph{supplant it}.

Advertisement

\protect\hyperlink{after-bottom}{Continue reading the main story}

\hypertarget{site-index}{%
\subsection{Site Index}\label{site-index}}

\hypertarget{site-information-navigation}{%
\subsection{Site Information
Navigation}\label{site-information-navigation}}

\begin{itemize}
\tightlist
\item
  \href{https://help.nytimes3xbfgragh.onion/hc/en-us/articles/115014792127-Copyright-notice}{©~2020~The
  New York Times Company}
\end{itemize}

\begin{itemize}
\tightlist
\item
  \href{https://www.nytco.com/}{NYTCo}
\item
  \href{https://help.nytimes3xbfgragh.onion/hc/en-us/articles/115015385887-Contact-Us}{Contact
  Us}
\item
  \href{https://www.nytco.com/careers/}{Work with us}
\item
  \href{https://nytmediakit.com/}{Advertise}
\item
  \href{http://www.tbrandstudio.com/}{T Brand Studio}
\item
  \href{https://www.nytimes3xbfgragh.onion/privacy/cookie-policy\#how-do-i-manage-trackers}{Your
  Ad Choices}
\item
  \href{https://www.nytimes3xbfgragh.onion/privacy}{Privacy}
\item
  \href{https://help.nytimes3xbfgragh.onion/hc/en-us/articles/115014893428-Terms-of-service}{Terms
  of Service}
\item
  \href{https://help.nytimes3xbfgragh.onion/hc/en-us/articles/115014893968-Terms-of-sale}{Terms
  of Sale}
\item
  \href{https://spiderbites.nytimes3xbfgragh.onion}{Site Map}
\item
  \href{https://help.nytimes3xbfgragh.onion/hc/en-us}{Help}
\item
  \href{https://www.nytimes3xbfgragh.onion/subscription?campaignId=37WXW}{Subscriptions}
\end{itemize}
