Sections

SEARCH

\protect\hyperlink{site-content}{Skip to
content}\protect\hyperlink{site-index}{Skip to site index}

\href{https://www.nytimes3xbfgragh.onion/section/us}{U.S.}

\href{https://myaccount.nytimes3xbfgragh.onion/auth/login?response_type=cookie\&client_id=vi}{}

\href{https://www.nytimes3xbfgragh.onion/section/todayspaper}{Today's
Paper}

\href{/section/us}{U.S.}\textbar{}Man Charged With Killing 5 in
Annapolis Newsroom Uses Insanity Defense

\url{https://nyti.ms/2V2sHBG}

\begin{itemize}
\item
\item
\item
\item
\item
\end{itemize}

Advertisement

\protect\hyperlink{after-top}{Continue reading the main story}

Supported by

\protect\hyperlink{after-sponsor}{Continue reading the main story}

\hypertarget{man-charged-with-killing-5-in-annapolis-newsroom-uses-insanity-defense}{%
\section{Man Charged With Killing 5 in Annapolis Newsroom Uses Insanity
Defense}\label{man-charged-with-killing-5-in-annapolis-newsroom-uses-insanity-defense}}

\includegraphics{https://static01.graylady3jvrrxbe.onion/images/2019/04/29/multimedia/29xp-annapolis-01/merlin_140483241_421bc40d-e499-4892-8cae-8bbd4301b299-articleLarge.jpg?quality=75\&auto=webp\&disable=upscale}

By \href{https://www.nytimes3xbfgragh.onion/by/mihir-zaveri}{Mihir
Zaveri}

\begin{itemize}
\item
  April 29, 2019
\item
  \begin{itemize}
  \item
  \item
  \item
  \item
  \item
  \end{itemize}
\end{itemize}

Lawyers for the man accused of fatally shooting five people in the
newsroom of The Capital Gazette last year invoked an insanity defense on
Monday, saying in a court filing that he should not be held criminally
responsible for the shooting because of a mental disorder.

Jarrod W. Ramos, 39, faces five charges of first-degree murder in the
June 28 shooting at the Annapolis, Md., newsroom, considered the
deadliest attack against journalists in United States history.

Before he blasted his way into the newsroom offices with a 12-gauge
shotgun, the authorities said, Mr. Ramos sent a number of letters,
including one
\href{https://www.nytimes3xbfgragh.onion/2018/07/02/us/annapolis-shooting-woman-harassed.html}{to
The Capital Gazette's lawyer} that said he planned to go there ``with
the objective of killing every person present.''

Mr. Ramos pleaded not guilty in July.

Monday's court filing enters an additional plea of ``not criminally
responsible'' because a mental disorder either prevented him from
appreciating the ``criminality of his conduct'' or prevented him from
following the law at the time of the shooting.

Anne Colt Leitess, the state's attorney for Anne Arundel County,
declined to comment on the facts of the case against Mr. Ramos, but
described in an interview the process of finding someone ``not
criminally responsible'' for a crime.

``You don't understand what you're doing is a crime or you cannot
conform your conduct to the requirements of the law,'' she said.

It is not clear what mental disorder Mr. Ramos's lawyers are saying he
had, or how the disorder supports the new plea. William M. Davis, a
public defender who is representing Mr. Ramos, did not respond to
requests for comment Monday.

A judge is now expected to request that a state forensic psychiatrist
examine Mr. Ramos, his medical history and evidence in the case. The
psychiatrist would then make a recommendation on whether Mr. Ramos
should or should not be held criminally responsible for the shooting,
Ms. Leitess said.

This finding, which would ultimately be made by a judge or jury, is
separate from a guilty or not guilty verdict.

Image

Jarrod W. Ramos is pleading not guilty and not criminally responsible by
reason of insanity.Credit...Anne Arundel Police, via Associated Press

Mr. Ramos could be found guilty in the shooting but not criminally
responsible. If that is the case, he could be sent to a state
psychiatric facility instead of prison. Health officials could
periodically re-evaluate the level of his confinement.

A finding of not criminally responsible is Maryland's version of what is
commonly referred to as the insanity defense.

The shooting's impact continues to reverberate following
\href{https://www.nytimes3xbfgragh.onion/2018/10/11/world/americas/journalists-killed.html}{a
particularly deadly year} for journalists. Earlier this month, employees
of The Capital Gazette
\href{https://www.nytimes3xbfgragh.onion/2019/04/15/business/media/pulitzer-prizes.html}{received
a special citation} from the Pulitzer Prize Board. Dana Canedy, the
awards' administrator, cited The Capital Gazette's ``unflagging
commitment to covering news at a time of unspeakable grief.''

In December, Time magazine honored the Capital Gazette staff, among
other journalists,
\href{https://www.nytimes3xbfgragh.onion/2018/12/11/business/media/jamal-khashoggi-person-of-the-year-time.html}{as
its 2018 person of the year}.

Before the shooting, Mr. Ramos had a long-running feud with The Capital,
the daily newspaper of the Capital Gazette community newspaper chain,
over a 2011 column that detailed his harassment of a former high school
classmate.

Mr. Ramos sued the owners of The Capital in 2012, claiming the article
that described his behavior was defamatory. He had also posted tweets,
laced with profanities, that railed against newspaper employees.

The attacks online prompted members of the Anne Arundel County Police
Department to visit Mr. Ramos in 2013, at the behest of the newspaper's
editor at the time, who was concerned about the hostile messages. But
the police then did not believe Mr. Ramos to be a threat to Capital
Gazette employees.

During the shooting, the authorities said, Mr. Ramos barricaded the back
door to the newsroom to prevent people from fleeing. He then called 911
to surrender, according to court filings.

Mr. Ramos is being held in the county jail after being denied bail.

In addition to the five murder charges, Mr. Ramos faces 18 other
charges, including attempted murder, assault and felony gun charges. If
found guilty, he faces life in prison without parole.

A trial date has been set for November.

Advertisement

\protect\hyperlink{after-bottom}{Continue reading the main story}

\hypertarget{site-index}{%
\subsection{Site Index}\label{site-index}}

\hypertarget{site-information-navigation}{%
\subsection{Site Information
Navigation}\label{site-information-navigation}}

\begin{itemize}
\tightlist
\item
  \href{https://help.nytimes3xbfgragh.onion/hc/en-us/articles/115014792127-Copyright-notice}{©~2020~The
  New York Times Company}
\end{itemize}

\begin{itemize}
\tightlist
\item
  \href{https://www.nytco.com/}{NYTCo}
\item
  \href{https://help.nytimes3xbfgragh.onion/hc/en-us/articles/115015385887-Contact-Us}{Contact
  Us}
\item
  \href{https://www.nytco.com/careers/}{Work with us}
\item
  \href{https://nytmediakit.com/}{Advertise}
\item
  \href{http://www.tbrandstudio.com/}{T Brand Studio}
\item
  \href{https://www.nytimes3xbfgragh.onion/privacy/cookie-policy\#how-do-i-manage-trackers}{Your
  Ad Choices}
\item
  \href{https://www.nytimes3xbfgragh.onion/privacy}{Privacy}
\item
  \href{https://help.nytimes3xbfgragh.onion/hc/en-us/articles/115014893428-Terms-of-service}{Terms
  of Service}
\item
  \href{https://help.nytimes3xbfgragh.onion/hc/en-us/articles/115014893968-Terms-of-sale}{Terms
  of Sale}
\item
  \href{https://spiderbites.nytimes3xbfgragh.onion}{Site Map}
\item
  \href{https://help.nytimes3xbfgragh.onion/hc/en-us}{Help}
\item
  \href{https://www.nytimes3xbfgragh.onion/subscription?campaignId=37WXW}{Subscriptions}
\end{itemize}
