Sections

SEARCH

\protect\hyperlink{site-content}{Skip to
content}\protect\hyperlink{site-index}{Skip to site index}

\href{https://www.nytimes3xbfgragh.onion/section/politics}{Politics}

\href{https://myaccount.nytimes3xbfgragh.onion/auth/login?response_type=cookie\&client_id=vi}{}

\href{https://www.nytimes3xbfgragh.onion/section/todayspaper}{Today's
Paper}

\href{/section/politics}{Politics}\textbar{}Supreme Court Allows
Antitrust Lawsuit Against Apple to Proceed

\url{https://nyti.ms/30dcPeE}

\begin{itemize}
\item
\item
\item
\item
\item
\item
\end{itemize}

Advertisement

\protect\hyperlink{after-top}{Continue reading the main story}

Supported by

\protect\hyperlink{after-sponsor}{Continue reading the main story}

\hypertarget{supreme-court-allows-antitrust-lawsuit-against-apple-to-proceed}{%
\section{Supreme Court Allows Antitrust Lawsuit Against Apple to
Proceed}\label{supreme-court-allows-antitrust-lawsuit-against-apple-to-proceed}}

\includegraphics{https://static01.graylady3jvrrxbe.onion/images/2019/05/06/us/politics/00dc-scotusapple/merlin_149904843_a35446c7-3738-4bba-a276-36417e70f30b-articleLarge.jpg?quality=75\&auto=webp\&disable=upscale}

By \href{https://www.nytimes3xbfgragh.onion/by/adam-liptak}{Adam Liptak}
and \href{https://www.nytimes3xbfgragh.onion/by/jack-nicas}{Jack Nicas}

\begin{itemize}
\item
  May 13, 2019
\item
  \begin{itemize}
  \item
  \item
  \item
  \item
  \item
  \item
  \end{itemize}
\end{itemize}

WASHINGTON --- The Supreme Court on Monday allowed an enormous antitrust
class action against Apple
\href{https://www.supremecourt.gov/opinions/18pdf/17-204_bq7d.pdf}{to
move forward}, saying consumers should be allowed to try to prove that
the technology giant had used monopoly power to raise the prices of
iPhone apps.

The lawsuit is in its early stages, and it must overcome other legal
hurdles. But the case brings the most direct legal challenge in the
United States to the clout that Apple has built up through its App
Store. And it raises questions about how the company has wielded that
power, amid a wave of anti-tech sentiment that has also prompted
concerns about the dominance of other tech behemoths such as Facebook
and Amazon.

The court's 5-to-4 vote featured an unusual alignment of justices, with
President Trump's two appointees on opposite sides. Justice Brett M.
Kavanaugh, who joined the court in October, wrote the majority opinion,
which was also signed by the court's four more liberal justices. Justice
Neil M. Gorsuch, who joined the court in 2017, wrote the dissent.

The class-action lawsuit focuses on the fees that Apple takes on sales
in its App Store, which millions of people use every day to download
games, messaging apps and other programs. The company charges up to a 30
percent commission to developers who sell their products through its
store, bars them from selling their apps elsewhere and plays a role in
setting prices. App makers have long complained that the fee and other
practices are unfair.

Over the years, people and app makers have come to rely on the App Store
almost as if it were a utility. The store features more than two million
apps,
\href{https://www.supremecourt.gov/DocketPDF/17/17-204/59108/20180810114516634_17-204\%20ts.pdf}{Apple
has told the court}, and generated more than \$26 billion in payments to
developers in 2017. More important, the availability of apps transformed
the iPhone from merely a phone to the center of our digital lives.
Without the App Store, there would be no Uber or Lyft for calling cars,
Instagram for sharing pictures and videos, or Postmates for ordering
food.

Competitors have accused Apple of using the App Store to harm rivals.
Spotify, Netflix, Amazon and others have sought to avoid the Apple fee
by directing their customers to subscribe to their services directly.
But smaller app makers would struggle to make such a move.

Apple has appeared to use its power over the App Store to its advantage
in other instances, including a
\href{https://www.nytimes3xbfgragh.onion/2019/04/27/technology/apple-screen-time-trackers.html}{move
to restrict or block apps} that provide parental controls or monitor
time spent on a phone. Apple's competitors complained that the company
had targeted them after it created its own tool for those tasks. Apple
said it removed some of the apps for privacy reasons.

Apple said in a statement that it was ``confident we will prevail when
the facts are presented and that the App Store is not a monopoly by any
metric.''

How the case unfolds is likely to be closely watched as lawmakers and
regulators around the world debate how to rein in the power of tech
companies. In the United States,
\href{https://www.nytimes3xbfgragh.onion/2018/09/07/technology/monopoly-antitrust-lina-khan-amazon.html}{legal
scholars have questioned} whether antitrust arguments focused on price
are enough to deal with the tech giants and whether antitrust law should
take competitive process into account.

This year, the music streaming service Spotify filed
\href{https://www.nytimes3xbfgragh.onion/2019/03/13/business/spotify-apple-complaint.html?smid=nytcore-ios-share}{a
complaint with European regulators} that accused Apple of using its
power over the App Store to harm rivals.

Other tech powers like Google, Facebook and Amazon have faced antitrust
scrutiny for their outsize market shares in areas like internet search,
social media and online commerce, but Apple escaped it for years because
iPhones make up less than half of the American smartphone market,
according to some estimates, and far less in other countries. Instead,
Google's Android software backs most of the world's smartphones.

Still, iPhone users spend far more than owners of smartphones running
Android, meaning the App Store, which was opened in 2008, has become the
dominant marketplace for app makers to reach customers.

Successful antitrust plaintiffs are entitled to triple damages, so the
damages could be large if Apple loses the lawsuit. Still, Apple has deep
pockets --- it generated \$266 billion in sales and \$59.5 billion in
profit in its
\href{https://www.apple.com/newsroom/pdfs/Q4-FY18-Consolidated-Financial-Statements.pdf}{last
fiscal year}.

A ruling that forces Apple to reduce its share of app sales is likely to
have an even longer-term effect on the company. Apple, which makes most
of its sales from iPhones, has tried to shift its business to rely more
on revenue from sales of apps and other services. In January, it said it
had paid out \$120 billion to app makers since 2008, putting its own
take from App Store revenue at around \$50 billion.

Apple has tried to address some developer concerns about its App Store.
In 2016, it announced that it would reduce its take from subscription
accounts to 15 percent after the first year.

The App Store has also become a major talking point for Apple's lobbying
efforts. The company has called itself ``one of the biggest job creators
in the U.S.,'' citing the number of jobs generated by companies writing
software for the App Store. As of 2017, it said, the app industry had
generated more than 1.5 million American jobs in software design and
development.

Sandeep Vaheesan, legal director at the Open Markets Institute, a
Washington think tank that advocates for stronger antitrust enforcement,
said the App Store gave Apple too much power over pricing and allowed it
to censor content.

``What Apple has done since the launch of the iPhone is tell all iPhone
owners and iPhone app developers that if they want to buy and sell apps,
they have to go through the App Store,'' Mr. Vaheesan said. ``So Apple
has set up this app store as a bottleneck where everyone in the iPhone
ecosystem must transact.''

Apple shares fell more than 5 percent on Monday, with some investors
selling on news of the Supreme Court ruling, as well as the
\href{https://www.nytimes3xbfgragh.onion/2019/05/13/business/trump-trade-china.html}{renewed
trade war between China and the United States}, which poses a risk to
the company's business. Apple relies on China for much of its
manufacturing and roughly a fifth of its iPhone sales.

The legal question in the case, Apple v. Pepper, was whether the lawsuit
was barred by a 1977 decision in
\href{https://caselaw.findlaw.com/us-supreme-court/431/720.html}{Illinois
Brick Company v. Illinois}, a case that allowed only direct purchasers
of products to bring federal antitrust lawsuits. Apple argued that it
was an intermediary and so not subject to a lawsuit.

The majority rejected that argument. ``The plaintiffs' allegations boil
down to one straightforward claim: that Apple exercises monopoly power
in the retail market for the sale of apps and has unlawfully used its
monopoly power to force iPhone owners to pay Apple
higher-than-competitive prices for apps,'' Justice Kavanaugh wrote.

Apple argued that app developers set their own prices, meaning that
consumers should not be able to sue the company. Justice Kavanaugh
responded that the argument missed the economic reality of the
relationship between Apple and app developers.

``A `who sets the price' rule,'' he wrote, ``would draw an arbitrary and
unprincipled line among retailers based on retailers' financial
arrangements with their manufacturers or suppliers.''

``Under Apple's rule a consumer could sue a monopolistic retailer when
the retailer set the retail price by marking up the price it had paid
the manufacturer or supplier for the good or service,'' he wrote. ``But
a consumer could not sue a monopolistic retailer when the manufacturer
or supplier set the retail price and the retailer took a commission on
each sale.''

``In sum,'' Justice Kavanaugh wrote, ``Apple's theory would disregard
statutory text and precedent, create an unprincipled and economically
senseless distinction among monopolistic retailers and furnish
monopolistic retailers with a how-to guide for evasion of the antitrust
laws.''

Justices Ruth Bader Ginsburg, Stephen G. Breyer, Sonia Sotomayor and
Elena Kagan joined Justice Kavanaugh's majority opinion.

In dissent, Justice Gorsuch said the 1977 decision meant Apple should
prevail, and he suggested that the majority had undermined the precedent
by questioning all of its basic rationales.

``Without any invitation or reason to revisit our precedent, and with so
many grounds for caution, I would have thought the proper course today
would have been to afford Illinois Brick full effect,'' Justice Gorsuch
wrote, ``not to begin whittling it away to a bare formalism.''

Chief Justice John G Roberts Jr. and Justices Clarence Thomas and Samuel
A. Alito Jr. joined Justice Gorsuch's dissent.

Advertisement

\protect\hyperlink{after-bottom}{Continue reading the main story}

\hypertarget{site-index}{%
\subsection{Site Index}\label{site-index}}

\hypertarget{site-information-navigation}{%
\subsection{Site Information
Navigation}\label{site-information-navigation}}

\begin{itemize}
\tightlist
\item
  \href{https://help.nytimes3xbfgragh.onion/hc/en-us/articles/115014792127-Copyright-notice}{©~2020~The
  New York Times Company}
\end{itemize}

\begin{itemize}
\tightlist
\item
  \href{https://www.nytco.com/}{NYTCo}
\item
  \href{https://help.nytimes3xbfgragh.onion/hc/en-us/articles/115015385887-Contact-Us}{Contact
  Us}
\item
  \href{https://www.nytco.com/careers/}{Work with us}
\item
  \href{https://nytmediakit.com/}{Advertise}
\item
  \href{http://www.tbrandstudio.com/}{T Brand Studio}
\item
  \href{https://www.nytimes3xbfgragh.onion/privacy/cookie-policy\#how-do-i-manage-trackers}{Your
  Ad Choices}
\item
  \href{https://www.nytimes3xbfgragh.onion/privacy}{Privacy}
\item
  \href{https://help.nytimes3xbfgragh.onion/hc/en-us/articles/115014893428-Terms-of-service}{Terms
  of Service}
\item
  \href{https://help.nytimes3xbfgragh.onion/hc/en-us/articles/115014893968-Terms-of-sale}{Terms
  of Sale}
\item
  \href{https://spiderbites.nytimes3xbfgragh.onion}{Site Map}
\item
  \href{https://help.nytimes3xbfgragh.onion/hc/en-us}{Help}
\item
  \href{https://www.nytimes3xbfgragh.onion/subscription?campaignId=37WXW}{Subscriptions}
\end{itemize}
