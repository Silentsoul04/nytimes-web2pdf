Sections

SEARCH

\protect\hyperlink{site-content}{Skip to
content}\protect\hyperlink{site-index}{Skip to site index}

\href{https://myaccount.nytimes3xbfgragh.onion/auth/login?response_type=cookie\&client_id=vi}{}

\href{https://www.nytimes3xbfgragh.onion/section/todayspaper}{Today's
Paper}

At Restaurants, Thank You for Not Sharing

\url{https://nyti.ms/2Q8ZL5t}

\begin{itemize}
\item
\item
\item
\item
\item
\end{itemize}

Advertisement

\protect\hyperlink{after-top}{Continue reading the main story}

Supported by

\protect\hyperlink{after-sponsor}{Continue reading the main story}

Notes on the Culture

\hypertarget{at-restaurants-thank-you-for-not-sharing}{%
\section{At Restaurants, Thank You for Not
Sharing}\label{at-restaurants-thank-you-for-not-sharing}}

After a decade of treating every plate like a pie, individual dishes are
making a welcome comeback.

!{[}The roasted quail with flowering thyme, lentils and warm tapenade at
King in New York's West Village.

Credit...David Williams for The New York
Times{]}(\url{https://static01.graylady3jvrrxbe.onion/images/2019/06/10/t-magazine/10tmag-sharing/10tmag-sharing-articleLarge.jpg?quality=75\&auto=webp\&disable=upscale})

By \href{https://www.nytimes3xbfgragh.onion/by/kurt-soller}{Kurt Soller}

\begin{itemize}
\item
  May 16, 2019
\item
  \begin{itemize}
  \item
  \item
  \item
  \item
  \item
  \end{itemize}
\end{itemize}

In 2011, upon visiting Taavo Somer's erstwhile Brooklyn restaurant Isa,
the New York Times critic
\href{https://www.nytimes3xbfgragh.onion/by/eric-asimov}{Eric Asimov}
wrote that he ``couldn't help noticing many people eating a few small
plates, or sharing a main course.'' He wondered, ``Is this really how
people want to eat nowadays?''

Sharing food has been both a boon and a boogeyman to the restaurant
industry since the 1980s, when ``grazing'' restaurants such as the
Casual Quilted Giraffe thrived in Manhattan. Spanish-inspired tapas were
chic in the '90s, and then came Chinese and Middle Eastern restaurants
that were --- and always will be --- meant for family-style eating.
Asimov wasn't the first to bemoan sharing, but his inquiry presaged a
decade in New York dominated by a new type of cuisine, one defined less
by provenance or a chef's palate than by the way it was meant to be
consumed, with each diner taking a single bite before relinquishing the
plate. At \href{https://www.estelanyc.com/}{Estela}, which opened in
2013, there were hedgehog-like piles of pale endive crowned with
anchovies; at \href{http://wildair.nyc/}{Wildair}, opened in 2015, beef
tartare hidden beneath a blizzard of white Cheddar. Both iconic dishes,
they adhered to a symmetrically circular plating style and highlighted
ingredients (oily fish, raw meat) that exhausted taste buds after just a
few forkfuls.

\emph{{[}}\href{https://www.nytimes3xbfgragh.onion/newsletters/t-list?module=inline}{\emph{Sign
up here}} \emph{for the T List newsletter, a weekly roundup of what T
Magazine editors are noticing and coveting now.{]}}

Living in the ``sharing economy,'' we are accustomed to apportioning
cars, offices and, yes, plates of food. Lately, though, chefs and diners
seem to have grown weary of the communal experience. Two of the most
exciting restaurants to have opened in New York within the past year are
\href{https://www.bennorestaurant.com/}{Benno} and
\href{https://oxalisnyc.com/}{Oxalis}, both of which serve
casual-leaning prix fixe menus. Elsewhere --- at
\href{https://www.nytimes3xbfgragh.onion/topic/person/rocco-dispirito}{Rocco
DiSpirito}'s revamped \href{http://www.thestandardgrill.com/}{Standard
Grill} and Adam Leonti's \href{http://www.leontinyc.com/}{eponymous
restaurant} --- one finds modern renditions of Continental cuisine such
as stuffed rabbit and truffled poussin (a young chicken), which can
technically be shared but seem designed for one, their textures and
flavors varying from bite to bite. At
\href{http://kingrestaurant.nyc/}{King}, an Italian-inspired restaurant
that opened in 2016, dishes like quail with braised chicories and tiny
potatoes are pointedly, deliciously unsharable. ``Sharing is often to
the detriment, because then awkward social interactions get in the way
of having dinner out together,'' says King general manager Annie Shi.
``Everyone is staring at the plate --- no one wants to touch it.''

Food is always political, and yet this debate seems particularly aligned
with our era. With socialism back in the national discourse, what could
be a better use of collective resources than collaborating on a meal?
But that, like socialism itself, is an impure ideal, as it's impossible
to get through a shared supper without someone (or everyone) feeling
like an autocrat: There is the bully who orders for the entire table,
the allergy sufferer who regrets forcing her sensitivities upon friends,
the hungry person who snags the last lobster ravioli and is then filled
with shame. If in other realms it is prudent to share, here is an
opportunity for everyone to feel heard by doing the opposite. In that
sense, at least, ordering --- and eating --- for one's self is downright
democratic.

Advertisement

\protect\hyperlink{after-bottom}{Continue reading the main story}

\hypertarget{site-index}{%
\subsection{Site Index}\label{site-index}}

\hypertarget{site-information-navigation}{%
\subsection{Site Information
Navigation}\label{site-information-navigation}}

\begin{itemize}
\tightlist
\item
  \href{https://help.nytimes3xbfgragh.onion/hc/en-us/articles/115014792127-Copyright-notice}{©~2020~The
  New York Times Company}
\end{itemize}

\begin{itemize}
\tightlist
\item
  \href{https://www.nytco.com/}{NYTCo}
\item
  \href{https://help.nytimes3xbfgragh.onion/hc/en-us/articles/115015385887-Contact-Us}{Contact
  Us}
\item
  \href{https://www.nytco.com/careers/}{Work with us}
\item
  \href{https://nytmediakit.com/}{Advertise}
\item
  \href{http://www.tbrandstudio.com/}{T Brand Studio}
\item
  \href{https://www.nytimes3xbfgragh.onion/privacy/cookie-policy\#how-do-i-manage-trackers}{Your
  Ad Choices}
\item
  \href{https://www.nytimes3xbfgragh.onion/privacy}{Privacy}
\item
  \href{https://help.nytimes3xbfgragh.onion/hc/en-us/articles/115014893428-Terms-of-service}{Terms
  of Service}
\item
  \href{https://help.nytimes3xbfgragh.onion/hc/en-us/articles/115014893968-Terms-of-sale}{Terms
  of Sale}
\item
  \href{https://spiderbites.nytimes3xbfgragh.onion}{Site Map}
\item
  \href{https://help.nytimes3xbfgragh.onion/hc/en-us}{Help}
\item
  \href{https://www.nytimes3xbfgragh.onion/subscription?campaignId=37WXW}{Subscriptions}
\end{itemize}
