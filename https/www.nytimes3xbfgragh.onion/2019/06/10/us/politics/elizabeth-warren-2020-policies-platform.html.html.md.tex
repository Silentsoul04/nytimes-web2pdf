\href{/section/politics}{Politics}\textbar{}Elizabeth Warren Has Lots of
Plans. Together, They Would Remake the Economy.

\url{https://nyti.ms/2MDE7IB}

\begin{itemize}
\item
\item
\item
\item
\item
\item
\end{itemize}

\begin{itemize}
\item
  \href{https://www.nytimes3xbfgragh.onion/2020/07/31/us/elections/biden-vs-trump.html?action=click\&pgtype=Article\&state=default\&region=TOP_BANNER\&context=storylines_menu}{Election
  Updates}
\item
  \href{https://www.nytimes3xbfgragh.onion/article/biden-vice-president-2020.html?action=click\&pgtype=Article\&state=default\&region=TOP_BANNER\&context=storylines_menu}{Biden's
  V.P. Search}
\item
  \href{https://www.nytimes3xbfgragh.onion/interactive/2020/07/24/us/politics/trump-biden-campaign-donors.html?action=click\&pgtype=Article\&state=default\&region=TOP_BANNER\&context=storylines_menu}{Map
  of Donations}
\item
  \href{https://www.nytimes3xbfgragh.onion/interactive/2020/us/elections/delegate-count-primary-results.html?action=click\&pgtype=Article\&state=default\&region=TOP_BANNER\&context=storylines_menu}{Delegate
  Count}
\item
  \href{https://www.nytimes3xbfgragh.onion/interactive/2019/us/politics/2020-presidential-candidates.html?action=click\&pgtype=Article\&state=default\&region=TOP_BANNER\&context=storylines_menu}{The
  Candidates}
\item
  \href{https://www.nytimes3xbfgragh.onion/newsletters/politics?action=click\&pgtype=Article\&state=default\&region=TOP_BANNER\&context=storylines_menu}{Politics
  Newsletter}
\end{itemize}

\includegraphics{https://static01.graylady3jvrrxbe.onion/images/2019/06/10/us/politics/10warren-policy-top1/10warren-policy-top1-articleLarge.jpg?quality=75\&auto=webp\&disable=upscale}

Sections

\protect\hyperlink{site-content}{Skip to
content}\protect\hyperlink{site-index}{Skip to site index}

\hypertarget{elizabeth-warren-has-lots-of-plans-together-they-would-remake-the-economy}{%
\section{Elizabeth Warren Has Lots of Plans. Together, They Would Remake
the
Economy.}\label{elizabeth-warren-has-lots-of-plans-together-they-would-remake-the-economy}}

Ms. Warren could face a difficult path to winning over moderates, but by
pushing out so many proposals so early, she is forcing her Democratic
rivals to play catch-up.

Credit...Mark Felix for The New York Times

Supported by

\protect\hyperlink{after-sponsor}{Continue reading the main story}

By \href{https://www.nytimes3xbfgragh.onion/by/thomas-kaplan}{Thomas
Kaplan} and
\href{https://www.nytimes3xbfgragh.onion/by/jim-tankersley}{Jim
Tankersley}

\begin{itemize}
\item
  June 10, 2019
\item
  \begin{itemize}
  \item
  \item
  \item
  \item
  \item
  \item
  \end{itemize}
\end{itemize}

WASHINGTON --- As the 23 candidates seeking the Democratic nomination
struggle to distinguish themselves, Senator Elizabeth Warren has set
herself apart with a series of sweeping proposals that would
significantly remake the American economy, covering everything from tax
policy to student debt relief and offering a detailed portrait of what
her presidency might look like.

Many of the proposals from Ms. Warren, a former Harvard law professor
and hawk on financial regulation, could face a difficult path to winning
over moderates in a general election, and to gaining approval in
Congress if she did take the White House. But the sheer volume of her
plans, and their detail and variety, is forcing her rivals to play
catch-up and stake out their own positions.

Her proposals would tip power from executives and investors to workers
and allow the federal government to more aggressively steer the
development of industries. She has called for splintering technology
companies, like Amazon, that millions of consumers rely on in their
daily lives. She would reduce the rewards for entrepreneurs to build
billionaire fortunes and for companies to create global supply chains,
scrambling the incentives for work, investment and economic growth.

Ms. Warren would seek big tax increases on the wealthiest individuals
and corporations,
\href{https://www.nytimes3xbfgragh.onion/2019/01/24/us/politics/wealth-tax-democrats.html}{creating
a new tax} on household assets that exceed \$50 million as well as a
\href{https://www.nytimes3xbfgragh.onion/2019/04/29/us/politics/democrats-taxes-2020.html}{new
tax on corporate profits}. From those two steps alone, she says she
would raise at least \$3.8 trillion over a decade --- money that would
go toward her plans on student debt cancellation, free college, child
care, the opioid crisis and green manufacturing.

\emph{{[}}\href{https://www.nytimes3xbfgragh.onion/2019/06/08/us/politics/iowa-poll-2020-democrats.html?action=click\&module=Intentional\&pgtype=Article}{\emph{Read
more: Ms. Warren is picking up support in Iowa, according to a new
poll.}}\emph{{]}}

A review of the policy rollouts the 2020 Democrats have made since
entering the race shows that Ms. Warren has issued the largest number of
detailed plans among the major candidates --- roughly 20 in all, on
subjects as varied as Big Tech regulation, housing costs and Pentagon
contracting. Most of her rivals have released fewer than half a dozen;
former Vice President Joseph R. Biden Jr., who leads in early polling,
has issued only two.

But Ms. Warren has already drawn criticism from centrists and
conservatives who say her plans --- many calling for new regulations ---
would hurt business and the economy, stifle innovation and potentially
harm the very workers they were intended to help. ``Were they to get
done, they would cause significant problems,'' said Tony Fratto, a
former Treasury official in the George W. Bush administration who is a
partner at Hamilton Place Strategies in Washington.

By pushing out so many proposals so early, Ms. Warren has framed much of
the debate in the Democratic primary race, aiding
\href{https://www.nytimes3xbfgragh.onion/2019/06/08/us/politics/iowa-poll-2020-democrats.html}{her
own rise in the polls}.

``It's rare, at this stage of a presidential campaign, somebody
distinguishes themselves by the boldness and detail of their policies,''
said Robert B. Reich, who served as labor secretary under President Bill
Clinton. ``She is asking the biggest questions that exist, and that is:
How do you make a free market work? How do you make capitalism actually
work for the many rather than the few?''

\hypertarget{warrens-policy-proposals}{%
\subsection{Warren's Policy Proposals}\label{warrens-policy-proposals}}

Senator Elizabeth Warren has been rolling out detailed policy proposals
nearly every week since March, outpacing her major Democratic rivals.

Jan. 24

Wealth tax\href{https://elizabethwarren.com/ultra-millionaire-tax/}{»}

Feb. 19

Universal child
care\href{https://medium.com/@teamwarren/my-plan-for-universal-child-care-762535e6c20a}{»}

March 8

Breaking up big tech
companies\href{https://medium.com/@teamwarren/heres-how-we-can-break-up-big-tech-9ad9e0da324c}{»}

March 16

Housing\href{https://medium.com/@teamwarren/my-housing-plan-for-america-20038e19dc26}{»}

March 27

Agriculture\href{https://medium.com/@teamwarren/leveling-the-playing-field-for-americas-family-farmers-823d1994f067}{»}

April 2

Corporate executive
accountability\href{https://www.washingtonpost.com/opinions/elizabeth-warren-its-time-to-scare-corporate-america-straight/2019/04/02/ca464ab0-5559-11e9-8ef3-fbd41a2ce4d5_story.html?utm_term=.d08b4f4b871f}{»}

April 11

Corporate
taxation\href{https://medium.com/@teamwarren/im-proposing-a-big-new-idea-the-real-corporate-profits-tax-29dde7c960d}{»}

April 15

Public
lands\href{https://medium.com/@teamwarren/my-plan-for-public-lands-e4be1d88a01c}{»}

April 22

Student debt cancellation and free
college\href{https://medium.com/@teamwarren/im-calling-for-something-truly-transformational-universal-free-public-college-and-cancellation-of-a246cd0f910f}{»}

April 24

Maternal
mortality\href{https://twitter.com/ewarren/status/1121164657455136768}{»}

April 26

Military
housing\href{https://medium.com/@teamwarren/my-plan-to-improve-our-military-housing-b1a46ba235b8}{»}

May 2

Puerto Rico debt
relief\href{https://medium.com/@teamwarren/my-plan-to-provide-comprehensive-debt-relief-to-puerto-rico-f8b575a81b06}{»}

May 8

Opioid
crisis\href{https://medium.com/@teamwarren/my-comprehensive-plan-to-end-the-opioid-crisis-9d85deaa3ccb}{»}

May 15

The military and climate
change\href{https://medium.com/@teamwarren/our-military-can-help-lead-the-fight-in-combating-climate-change-2955003555a3}{»}

May 16

Pentagon
contracting\href{https://medium.com/@teamwarren/its-time-to-reduce-corporate-influence-at-the-pentagon-98f52ee0fcf1}{»}

May 17

Abortion\href{https://medium.com/@teamwarren/congressional-action-to-protect-choice-aaf94ed25fb5}{»}

May 31

Indicting a sitting
president\href{https://medium.com/@teamwarren/no-president-is-above-the-law-f4812e580336}{»}

June 4

``Economic
Patriotism''\href{https://medium.com/@teamwarren/a-plan-for-economic-patriotism-13b879f4cfc7}{»}

June 4

Green
manufacturing\href{https://medium.com/@teamwarren/my-green-manufacturing-plan-for-america-fc0ad53ab614}{»}

By Jason Kao/The New York Times

Ms. Warren, of Massachusetts, is hoping that her ambitious agenda will
win over Democratic primary voters, and that her emphasis on protecting
American workers will have crossover appeal in a general election. But
President Trump and his Republican allies would almost certainly use her
proposals to portray her as too extreme.

Like other Democrats running for president, she would need her party to
not only keep the House next year but also take control of the Senate, a
difficult feat, to have any chance of pushing the bulk of her agenda
through Congress. Even then, her transformative policies would surely
face fierce resistance. In the House, she would face overwhelming
opposition from Republican lawmakers as well as misgivings from the sort
of centrist Democrats who helped deliver the majority last year. And
Senate Republicans would be positioned to block many of her proposals
using the filibuster, a tactic that Ms. Warren has said should end but
that still enjoys bipartisan support.

\includegraphics{https://static01.graylady3jvrrxbe.onion/images/2019/06/10/us/politics/10warren-policy2/merlin_153658404_2c21f574-1f44-4544-a75d-19c86b33fcbd-articleLarge.jpg?quality=75\&auto=webp\&disable=upscale}

Some of the policy ideas Ms. Warren has promoted in recent months have
been hallmarks of her political career. A bankruptcy expert who helped
create the Consumer Financial Protection Bureau, she has long pushed
Democrats to embrace more structural changes to the economy, even at the
risk of putting off wealthy donors.

\hypertarget{latest-updates-2020-election}{%
\section{\texorpdfstring{\href{https://www.nytimes3xbfgragh.onion/2020/07/31/us/elections/biden-vs-trump.html?action=click\&pgtype=Article\&state=default\&region=MAIN_CONTENT_1\&context=storylines_live_updates}{Latest
Updates: 2020
Election}}{Latest Updates: 2020 Election}}\label{latest-updates-2020-election}}

Updated 2020-08-01T01:26:45.732Z

\begin{itemize}
\tightlist
\item
  \href{https://www.nytimes3xbfgragh.onion/2020/07/31/us/elections/biden-vs-trump.html?action=click\&pgtype=Article\&state=default\&region=MAIN_CONTENT_1\&context=storylines_live_updates\#link-29fdff45}{Kamala
  Harris, a top vice-presidential contender, confronts double
  standards.}
\item
  \href{https://www.nytimes3xbfgragh.onion/2020/07/31/us/elections/biden-vs-trump.html?action=click\&pgtype=Article\&state=default\&region=MAIN_CONTENT_1\&context=storylines_live_updates\#link-13ec3d9c}{Karen
  Bass and Susan Rice are rising on Biden's vice-presidential
  shortlist.}
\item
  \href{https://www.nytimes3xbfgragh.onion/2020/07/31/us/elections/biden-vs-trump.html?action=click\&pgtype=Article\&state=default\&region=MAIN_CONTENT_1\&context=storylines_live_updates\#link-49e9a016}{Trump
  says Russian bounties to kill U.S. troops `never took place.'}
\end{itemize}

\href{https://www.nytimes3xbfgragh.onion/2020/07/31/us/elections/biden-vs-trump.html?action=click\&pgtype=Article\&state=default\&region=MAIN_CONTENT_1\&context=storylines_live_updates}{See
more updates}

But Ms. Warren's advisers have also seen the policy rollouts as an
opportunity to put pressure on other campaigns and steer the party's
center of gravity leftward. They were keen to announce her student debt
cancellation proposal before any other rival's, partly to claim that
turf as their own.

They have also carefully structured the order of her policy
announcements, beginning with the ``Ultramillionaire Tax'' that provided
a built-in answer to the question, ``How will you pay for it?''

Now, other Democratic campaigns are being measured against Ms. Warren's
policy benchmarks. In a night of back-to-back CNN town-hall events in
April, every candidate faced a question that related to a proposal from
Ms. Warren.

\emph{{[}}\href{https://www.nytimes3xbfgragh.onion/newsletters/politics?smid=rd?action=click\&module=Intentional\&pgtype=Article}{\emph{Make
sense of the people, issues and ideas shaping American politics with our
newsletter.}}\emph{{]}}

Ms. Warren's agenda includes a plan to
\href{https://www.nytimes3xbfgragh.onion/2019/04/22/us/politics/elizabeth-warren-student-debt.html}{cancel
up to \$50,000 in student loan debt}, depending on a borrower's income,
and to eliminate tuition at public colleges. She has proposed a
\href{https://www.nytimes3xbfgragh.onion/2019/02/19/us/politics/elizabeth-warren-child-care.html}{universal
child care system} that would be free for low-income families and limit
other families' costs to 7 percent of their income. And last week, she
offered a
\href{https://www.nytimes3xbfgragh.onion/2019/06/04/us/politics/elizabeth-warren-economy-jobs.html}{broad
economic program} to promote American exports and spur job creation.

As part of that program, Ms. Warren called for a \$2 trillion federal
investment in climate-friendly industries and suggested other steps like
more actively managing the value of the dollar.

Her ideas resonate with a growing group of liberal economists who see
evidence that free markets need more forceful government intervention in
order to function properly and not just deliver spoils to the very
wealthy.

\includegraphics{https://static01.graylady3jvrrxbe.onion/images/2019/06/11/us/politics/xxvid_warren/xxvid_warren-videoSixteenByNine3000.jpg}

Fans of Ms. Warren's proposals say they would help the economy by
attacking income inequality: New taxes on the rich would fund
investments in workers and encourage companies to spend more on wages
and strategic investments than on executive pay.

They say that her aggressive use of antitrust regulation would unleash
more competition and dynamism in an economy increasingly dominated by
incumbent businesses, that her industrial policies would help the United
States capture global market share in emerging industries like clean
energy and that her spending on child care would encourage more
Americans, particularly women, to work.

``You're going to change norms, you're going to change the way people
act and the things they do with that money,'' said Heather Boushey, the
executive director at the Washington Center for Equitable Growth and a
former top adviser to Hillary Clinton's 2016 presidential campaign. ``I
can only imagine the kind of innovation and productivity it would
unleash in our economy.''

Critics say Ms. Warren --- with her proposals for new business
regulation and focus on encouraging companies to locate production in
the United States, rather than seeking the lowest-cost and most
efficient production hubs around the world --- will hinder American
companies as they attempt to sell into India, China and other developing
nations.

``To me that is the biggest risk in this, that you hobble American
corporations so they cannot be globally competitive,'' Mr. Fratto said.
``We want these companies to compete globally. Because sooner or later,
they have to.''

Ms. Warren would also risk hurting consumers by breaking up technology
companies, particularly Amazon, said Natasha Sarin, a University of
Pennsylvania economist who favors many of Ms. Warren's goals but has
written skeptically about her proposed wealth tax. She said taxing
wealth could discourage innovation and risk-taking by entrepreneurs who
invest time in their ideas hoping for a large payoff down the road.

``It's a really important shift in how tax policy works in the U.S.,''
she said, ``and it's not obvious to me that it's a shift for the
better.''

Ms. Warren would impose a
\href{https://www.nytimes3xbfgragh.onion/2019/02/18/upshot/warren-wealth-tax.html}{2
percent annual tax on a household's assets}, including stocks and real
estate, that exceed \$50 million. She would add another 1 percent tax on
assets above \$1 billion. Some other advanced countries, like Spain,
impose similar taxes. But the United States never has, and some experts
question whether Ms. Warren's plan is constitutional. Some economists,
including Ms. Sarin, say the tax would struggle to raise the revenues
Ms. Warren forecasts, because it is relatively easy for the ultrawealthy
to hide or shield assets from taxing authorities.

Ms. Warren's campaign has amplified the impact of her policy rollouts by
timing many of them to campaign trips. She unveiled her broad economic
program and her green manufacturing plan ahead of a visit to Michigan
and Indiana. She issued a proposal on public lands before visiting
Colorado and Utah. Her opioid plan came before a visit to West Virginia
and Ohio.

Image

Ms. Warren asked if members of the audience knew someone affected by the
opioid crisis during an event in Kermit, W.Va., last
month.Credit...Craig Hudson/Charleston Gazette-Mail, via Associated
Press

Policy makes up a large part of Ms. Warren's pitch to voters on the
campaign trail, where, from school gyms to house parties, she guides
audiences through one proposal after another. She explains her wealth
tax by likening it to the property tax paid by homeowners, only
broadened to include the ``diamonds, the stock portfolio, the Rembrandts
and the yachts'' of the super-rich.

``Right now in America, there is a real hunger,'' Ms. Warren said at a
Democratic Party event in Iowa on Sunday. ``There are people who are
ready for big, structural change in this country. They're ready for
change, and I got a plan for that.''

``I got a plan'' has become a personal trademark for Ms. Warren, drawing
cheers from crowds and inspiring T-shirts and tote bags. And her policy
announcements serve as fund-raising opportunities, too, helping to drive
news coverage and give her supporters more reasons to donate.

Though Ms. Warren has far outpaced her major rivals in issuing detailed
policy plans, some are working to make up ground.

\href{https://www.nytimes3xbfgragh.onion/2019/06/05/us/politics/booker-renters-credit.html}{In
just one day last week}, Senator Cory Booker of New Jersey offered a
housing plan that would provide a tax credit to renters, former
Representative Beto O'Rourke of Texas introduced a voting rights plan
and Senator Kirsten Gillibrand of New York issued a plan to legalize
marijuana.

Mr. Biden --- who entered the race in late April, four months after Ms.
Warren --- introduced a
\href{https://www.nytimes3xbfgragh.onion/2019/06/04/us/politics/joe-biden-climate-plan.html}{climate
plan} last week and an
\href{https://www.nytimes3xbfgragh.onion/2019/05/28/us/politics/biden-education-plan-2020.html}{education
plan} the week before. Senator Kamala Harris of California has put forth
plans on
\href{https://www.nytimes3xbfgragh.onion/2019/03/26/us/politics/kamala-harris-teacher-pay.html}{teacher
pay},
\href{https://www.nytimes3xbfgragh.onion/2019/04/22/us/politics/kamala-harris-gun-control.html}{gun
control},
\href{https://www.nytimes3xbfgragh.onion/2019/05/20/us/politics/kamala-harris-gender-pay-gap.html}{equal
pay} and
\href{https://www.nytimes3xbfgragh.onion/2019/05/28/us/politics/kamala-harris-abortion.html}{abortion}.
Senator Bernie Sanders of Vermont has issued plans about
\href{https://www.nytimes3xbfgragh.onion/2019/05/18/us/bernie-sanders-education-plan.html}{education},
rural America and banking.

Mr. Sanders, like other senators in the presidential race, is also
drawing on legislation he has proposed in Congress, most notably his
``Medicare for all'' bill, the latest version of which he
\href{https://www.nytimes3xbfgragh.onion/2019/04/10/us/politics/bernie-sanders-medicare-for-all.html}{introduced
in April}. Ms. Harris has a bill to create a
\href{https://www.nytimes3xbfgragh.onion/2019/05/22/business/democrats-taxes-middle-class.html}{big
tax credit} for low- and middle-income Americans. Ms. Warren likes to
talk about the anticorruption package that she proposed last year.

\emph{{[}}\href{https://www.nytimes3xbfgragh.onion/interactive/2019/us/politics/2020-presidential-candidates.html?action=click\&module=Intentional\&pgtype=Article}{\emph{Keep
tabs on all 23 Democrats running for president with our candidate
tracker.}}\emph{{]}}

Mayor Pete Buttigieg of South Bend, Ind., has not released any detailed
policy plans, though he has talked about
\href{https://www.nbcnews.com/politics/2020-election/inside-pete-buttigieg-s-plan-overhaul-supreme-court-n1012491}{overhauling
the Supreme Court} and plans to deliver a speech on foreign policy on
Tuesday.

For sheer variety, no candidate can match Andrew Yang, a businessman and
political newcomer. His
\href{https://www.yang2020.com/policies/}{website} lays out his ideas on
roughly 100 topics, including widely shared goals like fighting climate
change as well as less common ones like abolishing the penny and
supporting the unionization of mixed martial arts fighters. Most of his
proposals, though, are described only briefly.

Image

Ms. Warren promoted her green manufacturing plan in Michigan last
week.Credit...Brittany Greeson for The New York Times

One subject Ms. Warren has not broadly addressed in her policy plans is
health care, which is a top concern of voters. Ms. Warren supports a
Medicare for all system in which the government would provide health
insurance to everyone, but
\href{https://www.nytimes3xbfgragh.onion/2019/02/02/us/politics/medicare-for-all-2020.html}{she
has been less specific} on the role she foresees for private insurers.

And Ms. Warren and most of the other Democratic candidates have yet to
issue detailed plans on immigration --- a central issue for Mr. Trump.

Jared Bernstein, a former White House economic adviser to President
Barack Obama and top economist for Mr. Biden when he was vice president,
said he applauded Ms. Warren's ideas but wondered whether the federal
government --- particularly after several years of management by Mr.
Trump --- was up to the task of implementing them.

``It's probably important to be appropriately humble about our ability
to see the implications of big interventions,'' he said. ``But I love
the aspirations. I love thinking big.''

Isabella Grullón Paz and Astead W. Herndon contributed reporting from
New York.

\hypertarget{our-2020-election-guide}{%
\section{Our 2020 Election Guide}\label{our-2020-election-guide}}

Updated July 31, 2020

\begin{itemize}
\item
  \begin{center}\rule{0.5\linewidth}{\linethickness}\end{center}

  \hypertarget{the-latest}{%
  \subsection{The Latest}\label{the-latest}}

  \begin{itemize}
  \tightlist
  \item
    President Trump's assault on the Postal Service is intersecting with
    his attacks on mail-in voting.
    \href{https://www.nytimes3xbfgragh.onion/2020/07/31/us/politics/trump-usps-mail-delays.html?action=click\&pgtype=Article\&state=default\&region=BELOW_MAIN_CONTENT\&context=storylines_guide}{Voting
    rights groups say it is a recipe for disaster.}
  \end{itemize}
\item
  \begin{center}\rule{0.5\linewidth}{\linethickness}\end{center}

  \hypertarget{bidens-vp-search}{%
  \subsection{Biden's V.P. Search}\label{bidens-vp-search}}

  \begin{itemize}
  \tightlist
  \item
    \href{https://www.nytimes3xbfgragh.onion/article/biden-vice-president-2020.html?action=click\&pgtype=Article\&state=default\&region=BELOW_MAIN_CONTENT\&context=storylines_guide}{Here
    are 13 women} who have been under consideration to be Joe Biden's
    running mate, and why each might be chosen --- and might not be.
  \end{itemize}
\item
  \begin{center}\rule{0.5\linewidth}{\linethickness}\end{center}

  \hypertarget{keep-up-with-our-coverage}{%
  \subsection{Keep Up With Our
  Coverage}\label{keep-up-with-our-coverage}}

  \begin{itemize}
  \tightlist
  \item
    Get an
    \href{https://www.nytimes3xbfgragh.onion/newsletters/politics?action=click\&pgtype=Article\&state=default\&region=BELOW_MAIN_CONTENT\&context=storylines_guide}{email}
    recapping the day's news
  \end{itemize}

  \begin{itemize}
  \tightlist
  \item
    Download our mobile app on
    \href{https://apps.apple.com/us/app/nytimes/id284862083?ls=1\&mat_click_id=5c79ae7455014fd1bd66b5610c05b8f2-20191112-16948\&referrer=mat_click_id\%3D5c79ae7455014fd1bd66b5610c05b8f2-20191112-16948\%26link_click_id\%3D722930677036718082}{iOS}
    and
    \href{http://a.localytics.com/android?id=com.nytimes.android\&referrer=utm_source\%3Dother_nyt_mobile_web\%26utm_medium\%3DWeb\%2520page\%26utm_term\%3DGeneral\%2520Mobile\%2520Page\%26utm_campaign\%3DNYT\%2520Mobile\%2520General\%2520Page}{Android}
    and turn on Breaking News and Politics alerts
  \end{itemize}
\end{itemize}

Advertisement

\protect\hyperlink{after-bottom}{Continue reading the main story}

\hypertarget{site-index}{%
\subsection{Site Index}\label{site-index}}

\hypertarget{site-information-navigation}{%
\subsection{Site Information
Navigation}\label{site-information-navigation}}

\begin{itemize}
\tightlist
\item
  \href{https://help.nytimes3xbfgragh.onion/hc/en-us/articles/115014792127-Copyright-notice}{©~2020~The
  New York Times Company}
\end{itemize}

\begin{itemize}
\tightlist
\item
  \href{https://www.nytco.com/}{NYTCo}
\item
  \href{https://help.nytimes3xbfgragh.onion/hc/en-us/articles/115015385887-Contact-Us}{Contact
  Us}
\item
  \href{https://www.nytco.com/careers/}{Work with us}
\item
  \href{https://nytmediakit.com/}{Advertise}
\item
  \href{http://www.tbrandstudio.com/}{T Brand Studio}
\item
  \href{https://www.nytimes3xbfgragh.onion/privacy/cookie-policy\#how-do-i-manage-trackers}{Your
  Ad Choices}
\item
  \href{https://www.nytimes3xbfgragh.onion/privacy}{Privacy}
\item
  \href{https://help.nytimes3xbfgragh.onion/hc/en-us/articles/115014893428-Terms-of-service}{Terms
  of Service}
\item
  \href{https://help.nytimes3xbfgragh.onion/hc/en-us/articles/115014893968-Terms-of-sale}{Terms
  of Sale}
\item
  \href{https://spiderbites.nytimes3xbfgragh.onion}{Site Map}
\item
  \href{https://help.nytimes3xbfgragh.onion/hc/en-us}{Help}
\item
  \href{https://www.nytimes3xbfgragh.onion/subscription?campaignId=37WXW}{Subscriptions}
\end{itemize}
