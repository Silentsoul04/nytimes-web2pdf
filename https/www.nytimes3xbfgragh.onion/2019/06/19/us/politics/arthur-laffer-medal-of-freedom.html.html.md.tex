Sections

SEARCH

\protect\hyperlink{site-content}{Skip to
content}\protect\hyperlink{site-index}{Skip to site index}

\href{https://www.nytimes3xbfgragh.onion/section/politics}{Politics}

\href{https://myaccount.nytimes3xbfgragh.onion/auth/login?response_type=cookie\&client_id=vi}{}

\href{https://www.nytimes3xbfgragh.onion/section/todayspaper}{Today's
Paper}

\href{/section/politics}{Politics}\textbar{}Trump Awards Presidential
Medal of Freedom to Arthur Laffer, Tax-Cut Guru

\url{https://nyti.ms/2MXfAhG}

\begin{itemize}
\item
\item
\item
\item
\item
\end{itemize}

Advertisement

\protect\hyperlink{after-top}{Continue reading the main story}

Supported by

\protect\hyperlink{after-sponsor}{Continue reading the main story}

\hypertarget{trump-awards-presidential-medal-of-freedom-to-arthur-laffer-tax-cut-guru}{%
\section{Trump Awards Presidential Medal of Freedom to Arthur Laffer,
Tax-Cut
Guru}\label{trump-awards-presidential-medal-of-freedom-to-arthur-laffer-tax-cut-guru}}

\includegraphics{https://static01.graylady3jvrrxbe.onion/images/2019/06/19/us/politics/19dc-medal/merlin_156696726_01ceaf7e-8ae1-4f32-98ab-5d1742d63a12-articleLarge.jpg?quality=75\&auto=webp\&disable=upscale}

By \href{https://www.nytimes3xbfgragh.onion/by/jim-tankersley}{Jim
Tankersley}

\begin{itemize}
\item
  June 19, 2019
\item
  \begin{itemize}
  \item
  \item
  \item
  \item
  \item
  \end{itemize}
\end{itemize}

WASHINGTON --- President Trump bestowed the nation's highest civilian
honor on the conservative economist Arthur Laffer on Wednesday, praising
him for policies that he said brought ``greater opportunity for all
Americans.''

Mr. Laffer is the father of so-called supply-side economics, and is the
namesake of a famed theory --- the Laffer Curve --- which posits that
reducing certain tax rates can actually increase government tax revenues
by accelerating economic growth. He helped write Mr. Trump's campaign
tax plan and has advised the president on economic policy. He is also
the mentor of Larry Kudlow, the director of the White House National
Economic Council.

Mr. Laffer's relentless and sunny advocacy of tax cuts, deregulation and
free trade have influenced decades of Republican policy proposals, most
famously under President Ronald Reagan in the 1980s. Democrats have
criticized him for repeatedly promising that tax cuts would deliver
growth and revenues that did not appear, such as
\href{https://www.npr.org/2017/10/25/560040131/as-trump-proposes-tax-cuts-kansas-deals-with-aftermath-of-experiment}{damaging
state tax cuts in Kansas} that produced a large shortfall in the state
budget and prompted the Republican-controlled Legislature to ultimately
reverse them.

In a ceremony in the Oval Office, with Mr. Laffer's six children in
attendance, Mr. Trump said, ``I've heard and studied the Laffer Curve
for many years.'' He called the theory, which Mr. Laffer famously
\href{https://www.nytimes3xbfgragh.onion/2017/10/13/us/politics/arthur-laffer-napkin-tax-curve.html}{sketched
on a cocktail napkin} in the 1970s for Donald H. Rumsfeld and Dick
Cheney, then Republican policy hands, ``a very important thing you did,
Art.''

``Few people in history have revolutionized economic theory like Arthur
Laffer,'' Mr. Trump said.

``Academics called his theory insanity, totally wacky and completely off
the wall,'' he said. ``With optimism, confidence and exceptional
intellect, Art would go on to prove them all wrong on a number of
occasions.''

Mr. Trump also took the occasion to praise the performance of the United
States economy, including unexpectedly strong growth figures for the
first quarter of this year, which economists expect to slow in the
quarters to come. ``Our economy has never, ever been stronger than it is
today,'' he said.

Mr. Laffer is an author of a fawning book about Mr. Trump's economic
policies,
``\href{https://www.amazon.com/Trumponomics-Inside-America-Revive-Economy/dp/1250193710}{Trumponomics}.''
That book was written with Stephen Moore, another Trump adviser and
Laffer disciple, whom Mr. Trump said he would nominate to the Federal
Reserve board this year but who
\href{https://www.nytimes3xbfgragh.onion/2019/05/02/business/stephen-moore-fed.html}{withdrew
from contention} after Republican senators raised objections.

Mr. Laffer, Mr. Moore and Mr. Kudlow pushed Mr. Trump early in his 2016
presidential run to propose large tax cuts, for individuals and
corporations, saying they would fuel a sharp acceleration in economic
growth, to above 4 or 5 percent a year.

Mr. Trump did usher through a giant \$1.5 trillion package of cuts in
2017, which helped spur about 3 percent economic growth in 2018. But the
cuts did not generate more federal tax revenues and have instead
contributed to a
\href{https://www.nytimes3xbfgragh.onion/2019/01/11/business/trump-tax-cuts-revenue.html}{widening
budget deficit} that is on pace
\href{https://www.nytimes3xbfgragh.onion/2019/01/08/us/politics/budget-deficit-trillion.html}{to
top \$1 trillion} this year.

Mr. Laffer's award ceremony was attended by several of his longtime
allies in the decades-long push for tax cuts at all levels of
government, including Mr. Moore; a former presidential candidate, Steve
Forbes; and Mr. Kudlow, who championed to Mr. Trump the idea of awarding
Mr. Laffer the medal. When Mr. Trump praised supply-side economic
policies, Mr. Kudlow let out a low ``yee-haw.''

Accepting the award, Mr. Laffer praised a wide range of economists and
politicians who advocated tax cuts, including Mr. Cheney; the economist
Milton Friedman; former Representative Jack Kemp; Presidents Reagan and
John F. Kennedy; and Margaret Thatcher, the former prime minister of
Britain --- along with Mr. Trump.

He also praised his family.

``And all I can say is wow,'' Mr. Laffer said.

Advertisement

\protect\hyperlink{after-bottom}{Continue reading the main story}

\hypertarget{site-index}{%
\subsection{Site Index}\label{site-index}}

\hypertarget{site-information-navigation}{%
\subsection{Site Information
Navigation}\label{site-information-navigation}}

\begin{itemize}
\tightlist
\item
  \href{https://help.nytimes3xbfgragh.onion/hc/en-us/articles/115014792127-Copyright-notice}{©~2020~The
  New York Times Company}
\end{itemize}

\begin{itemize}
\tightlist
\item
  \href{https://www.nytco.com/}{NYTCo}
\item
  \href{https://help.nytimes3xbfgragh.onion/hc/en-us/articles/115015385887-Contact-Us}{Contact
  Us}
\item
  \href{https://www.nytco.com/careers/}{Work with us}
\item
  \href{https://nytmediakit.com/}{Advertise}
\item
  \href{http://www.tbrandstudio.com/}{T Brand Studio}
\item
  \href{https://www.nytimes3xbfgragh.onion/privacy/cookie-policy\#how-do-i-manage-trackers}{Your
  Ad Choices}
\item
  \href{https://www.nytimes3xbfgragh.onion/privacy}{Privacy}
\item
  \href{https://help.nytimes3xbfgragh.onion/hc/en-us/articles/115014893428-Terms-of-service}{Terms
  of Service}
\item
  \href{https://help.nytimes3xbfgragh.onion/hc/en-us/articles/115014893968-Terms-of-sale}{Terms
  of Sale}
\item
  \href{https://spiderbites.nytimes3xbfgragh.onion}{Site Map}
\item
  \href{https://help.nytimes3xbfgragh.onion/hc/en-us}{Help}
\item
  \href{https://www.nytimes3xbfgragh.onion/subscription?campaignId=37WXW}{Subscriptions}
\end{itemize}
