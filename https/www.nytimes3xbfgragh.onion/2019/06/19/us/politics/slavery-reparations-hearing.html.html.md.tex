Sections

SEARCH

\protect\hyperlink{site-content}{Skip to
content}\protect\hyperlink{site-index}{Skip to site index}

\href{https://www.nytimes3xbfgragh.onion/section/politics}{Politics}

\href{https://myaccount.nytimes3xbfgragh.onion/auth/login?response_type=cookie\&client_id=vi}{}

\href{https://www.nytimes3xbfgragh.onion/section/todayspaper}{Today's
Paper}

\href{/section/politics}{Politics}\textbar{}At Historic Hearing, House
Panel Explores Reparations

\url{https://nyti.ms/2FlxGUp}

\begin{itemize}
\item
\item
\item
\item
\item
\item
\end{itemize}

Advertisement

\protect\hyperlink{after-top}{Continue reading the main story}

Supported by

\protect\hyperlink{after-sponsor}{Continue reading the main story}

\hypertarget{at-historic-hearing-house-panel-explores-reparations}{%
\section{At Historic Hearing, House Panel Explores
Reparations}\label{at-historic-hearing-house-panel-explores-reparations}}

\includegraphics{https://static01.graylady3jvrrxbe.onion/images/2019/06/24/us/politics/19dc-reparations-sub1/19dc-reparations-sub1-videoSixteenByNine3000.jpg}

By
\href{https://www.nytimes3xbfgragh.onion/by/sheryl-gay-stolberg}{Sheryl
Gay Stolberg}

\begin{itemize}
\item
  June 19, 2019
\item
  \begin{itemize}
  \item
  \item
  \item
  \item
  \item
  \item
  \end{itemize}
\end{itemize}

WASHINGTON --- Frail but sharp at 88, the Rev. Doris Sherman woke up at
4 a.m. on Wednesday to travel here from Philadelphia for an event that,
even after the nation elected its first black president, she never
thought she would see: a meeting in the capital of the United States on
reparations for African-Americans.

Dressed all in white, the color of the suffragist movement --- it was a
coincidence, she said --- Ms. Sherman, who is black, reflected on the
unfulfilled Civil War-era promise to former slaves of ``40 acres and a
mule.'' As a schoolteacher for 30 years before entering the ministry,
she recalled so many black parents struggling to provide day care, their
children ``left back and left out.''

If the government did anything, she said, it should do something for the
children. ``We don't want that mule now,'' she said. ``We don't want
that 40 acres. We are asking for remembrance. Remember the struggle.
Remember the injustice and remember the now.''

Ms. Sherman was among hundreds of other mostly black spectators --- so
many that they filled three overflow rooms --- who descended on Capitol
Hill for Wednesday's historic hearing, the first time Congress has
considered a bill, H.R. 40, that would create a commission to develop
proposals to address the lingering effects of slavery and consider a
``national apology'' for the harm it has caused.

The sometimes raucous session before a subcommittee of the House
Judiciary Committee lasted nearly three and a half hours and dug into
the darkest corners of the nation's history, exposing the bitter
cultural and ideological divides in Washington and beyond. Republican
lawmakers and witnesses --- including Burgess Owens, the retired
football star --- were jeered when they argued that black people could
pull themselves up by their own bootstraps and that reparations might
damage their psyches.

``We've become successful like no other because of this great
opportunity to live the American dream,'' Mr. Owens, who is black, told
the panel. ``Let's not steal that from our kids by telling them they
can't do it.''

That the hearing took place at all was remarkable, a reflection of the
shifting landscape in the Democratic Party and the wrenching national
debate over racial justice in the era of President Trump. Nearly 60
House Democrats, including Speaker Nancy Pelosi, support the bill. And
at least 11 Democratic presidential candidates --- with former Vice
President Joseph R. Biden Jr. a notable exception --- have embraced
either the concept of reparations or the bill to study it.

``We have not had a conversation about reparations on this scale or
level since the Reconstruction Era,'' William A. Darity Jr., a professor
of public policy at Duke University who is writing a book on
reparations, said in a telephone interview. ``To be blunt, I am more
optimistic than I have ever been in my life about the prospect of the
enactment of a reparations program that is comprehensive and
transformative.''

\includegraphics{https://static01.graylady3jvrrxbe.onion/images/2019/06/19/us/politics/19dc-reparations-2/merlin_156677973_09881d44-3f5e-4b16-ab97-e502f27c5b53-articleLarge.jpg?quality=75\&auto=webp\&disable=upscale}

The first time the federal government considered reparations for black
people was in 1865, when 400,000 acres of coastal land were awarded to
former slaves, the result of a special order issued by the Union
general, William T. Sherman. It lasted less than a year. When President
Abraham Lincoln died, he was succeeded by Andrew Johnson, who rescinded
Sherman's order.

In the late 1800s, the idea
\href{https://www.archives.gov/publications/prologue/2010/summer/slave-pension.html}{of
pensions for former slaves} --- similar to pensions for Union soldiers
--- took hold, championed for a time by a Nebraska congressman. But the
idea fizzled in the face of strong opposition from federal agencies.

In 1989, Representative John Conyers Jr., who retired in 2017,
introduced legislation to create a commission to develop proposals for
reparations. He introduced it every year for nearly 30 years. It went
nowhere. Even President Barack Obama
\href{https://www.theatlantic.com/politics/archive/2016/12/ta-nehisi-coates-obama-transcript-ii/511133/}{opposed
reparations}, calling the idea impractical.

It is that bill, titled the
``\href{https://www.congress.gov/bill/116th-congress/house-bill/40/text}{Commission
to Study and Develop Reparation Proposals for African-Americans Act},''
and now sponsored by Representative Sheila Jackson Lee, Democrat of
Texas, that the subcommittee has before it. It would authorize \$12
million for a 13-member commission to study the effects of slavery and
make recommendations to Congress.

``I just simply ask: Why not?'' Ms. Jackson Lee said Wednesday. ``And
why not now?''

But Professor Darity's optimism may be overstated.

Even if it passes the House, the bill has little chance of getting
through the Republican-controlled Senate, where Senator Mitch McConnell
of Kentucky, the majority leader, spoke out against it on Tuesday,
telling reporters he does not favor reparations ``for something that
happened 150 years ago, for whom none of us currently living are
responsible.''

Mr. McConnell's remark prompted a sharp rebuke from the hearing's star
witness, the writer Ta-Nehisi Coates, whose 2014 article
``\href{https://www.theatlantic.com/magazine/archive/2014/06/the-case-for-reparations/361631/}{The
Case for Reparations}'' in The Atlantic rekindled the debate, arguing
that African-Americans had been exploited by nearly every American
institution. Mr. Coates, who is black, ticked off a list of
government-sponsored discriminatory policies --- including those in Mr.
McConnell's birthplace of Alabama --- such as redlining and poll taxes.

``He was alive for the redlining of Chicago and the looting of black
homeowners of some \$4 billion,'' Mr. Coates said. ``Victims of their
plunder are very much alive today. I am sure they would love a word with
the majority leader.''

``While emancipation dead-bolted the door against the bandits of
America, Jim Crow wedged the windows wide open,'' he added. ``That's the
thing about Senator McConnell's `something.' It was 150 years ago, and
it was right now.''

Image

Asmara Sium and her son Khalab Blagburn attended the hearing on
reparations, which exposed bitter cultural and ideological
divides.Credit...Michael A. McCoy for The New York Times

Advocates for reparations say their cause is misunderstood, and
emphasize that it does not necessarily mean the government would be
writing checks to black people, though Mr. Coates said he was not
opposed to the idea.

Rather, they say, the government could offer various types of assistance
--- zero-interest loans for prospective black homeowners, free college
tuition, community development plans to spur the growth of black-owned
businesses in black neighborhoods --- to address the social and economic
fallout of slavery and racially discriminatory federal policies that
have resulted in a huge wealth gap between white and black people.

``When a black woman or man is arrested, they may land in jail for how
many days because they don't have the home, the mortgage to get the bail
--- and cash bail is discriminatory,'' Julianne Malveaux, an economist,
told the subcommittee, her voice rising in anger. ``I want y'all
Congress people to deal with issues of economic structure.''

Wednesday's hearing was laden with symbolism. This year is the 400th
anniversary of the
\href{https://www.nps.gov/jame/learn/historyculture/african-americans-at-jamestown.htm}{first
documented arrival of Africans to the port of Jamestown} in what was
then the colony of Virginia. Wednesday, June 19, was
\href{https://www.nytimes3xbfgragh.onion/aponline/2019/06/14/us/politics/ap-us-ap-explains-juneteenth.html?module=inline}{Juneteenth},
the holiday that celebrates the end of slavery in the United States. And
the bill carries the designation H.R. 40, a reference to ``40 acres and
a mule.''

As passions flared, the subcommittee chairman, Representative Steve
Cohen of Tennessee, repeatedly told the spectators to simmer down. And
politics was at work: A Democratic presidential candidate, Senator Cory
Booker, who is carrying the bill in the Senate, was the first witness,
declaring himself ``brokenhearted and very angry'' at the nation's
reluctance to deal with what he called ``a cancer on the soul of our
country.''

``I believe right now we have a historic opportunity to break the
silence,'' Mr. Booker said. ``To speak to the ugly past and talk
constructively about how to move this nation forward.''

One Republican congressman, Representative Louie Gohmert of Texas,
lashed out at ``today's claim that the Republicans are the party of
racism,'' noting that southern segregationist Democrats were responsible
for the era of Jim Crow.

Another Republican, Representative Mike Johnson of Louisiana, drew
hisses when he suggested that black leaders like Frederick Douglass and
Booker T. Washington ``encouraged people to take control of and
responsibility for their own lives, because that gives every human being
a greater sense of meaning, purpose and satisfaction.''

The actor and activist Danny Glover told of his great-grandmother, Mary
Brown, a slave who was freed by Lincoln's Emancipation Proclamation of
1863. A documentary filmmaker, Katrina Browne, who is white, recounted
her painful discovery that her Rhode Island ancestors had been ``the
largest slave trading family in United States history,'' and brought
more than 12,000 Africans to the Americas in chains.

Her message to the lawmakers: ``It is good for the soul of a person, a
people and of a nation to set things right.''

Advertisement

\protect\hyperlink{after-bottom}{Continue reading the main story}

\hypertarget{site-index}{%
\subsection{Site Index}\label{site-index}}

\hypertarget{site-information-navigation}{%
\subsection{Site Information
Navigation}\label{site-information-navigation}}

\begin{itemize}
\tightlist
\item
  \href{https://help.nytimes3xbfgragh.onion/hc/en-us/articles/115014792127-Copyright-notice}{©~2020~The
  New York Times Company}
\end{itemize}

\begin{itemize}
\tightlist
\item
  \href{https://www.nytco.com/}{NYTCo}
\item
  \href{https://help.nytimes3xbfgragh.onion/hc/en-us/articles/115015385887-Contact-Us}{Contact
  Us}
\item
  \href{https://www.nytco.com/careers/}{Work with us}
\item
  \href{https://nytmediakit.com/}{Advertise}
\item
  \href{http://www.tbrandstudio.com/}{T Brand Studio}
\item
  \href{https://www.nytimes3xbfgragh.onion/privacy/cookie-policy\#how-do-i-manage-trackers}{Your
  Ad Choices}
\item
  \href{https://www.nytimes3xbfgragh.onion/privacy}{Privacy}
\item
  \href{https://help.nytimes3xbfgragh.onion/hc/en-us/articles/115014893428-Terms-of-service}{Terms
  of Service}
\item
  \href{https://help.nytimes3xbfgragh.onion/hc/en-us/articles/115014893968-Terms-of-sale}{Terms
  of Sale}
\item
  \href{https://spiderbites.nytimes3xbfgragh.onion}{Site Map}
\item
  \href{https://help.nytimes3xbfgragh.onion/hc/en-us}{Help}
\item
  \href{https://www.nytimes3xbfgragh.onion/subscription?campaignId=37WXW}{Subscriptions}
\end{itemize}
