Sections

SEARCH

\protect\hyperlink{site-content}{Skip to
content}\protect\hyperlink{site-index}{Skip to site index}

\href{https://www.nytimes3xbfgragh.onion/section/climate}{Climate}

\href{https://myaccount.nytimes3xbfgragh.onion/auth/login?response_type=cookie\&client_id=vi}{}

\href{https://www.nytimes3xbfgragh.onion/section/todayspaper}{Today's
Paper}

\href{/section/climate}{Climate}\textbar{}As Coal Fades in the U.S.,
Natural Gas Becomes the Climate Battleground

\url{https://nyti.ms/2Fybpmr}

\begin{itemize}
\item
\item
\item
\item
\item
\item
\end{itemize}

\hypertarget{climate-and-environment}{%
\subsubsection{\texorpdfstring{\href{https://www.nytimes3xbfgragh.onion/section/climate?name=styln-climate\&region=TOP_BANNER\&variant=undefined\&block=storyline_menu_recirc\&action=click\&pgtype=Article\&impression_id=e07e3bf0-e38b-11ea-befc-036b724b9a9b}{Climate
and
Environment}}{Climate and Environment}}\label{climate-and-environment}}

\begin{itemize}
\tightlist
\item
  \href{https://www.nytimes3xbfgragh.onion/2020/08/17/climate/alaska-oil-drilling-anwr.html?name=styln-climate\&region=TOP_BANNER\&variant=undefined\&block=storyline_menu_recirc\&action=click\&pgtype=Article\&impression_id=e07e3bf1-e38b-11ea-befc-036b724b9a9b}{Arctic
  Refuge}
\item
  \href{https://www.nytimes3xbfgragh.onion/interactive/2020/climate/trump-environment-rollbacks.html?name=styln-climate\&region=TOP_BANNER\&variant=undefined\&block=storyline_menu_recirc\&action=click\&pgtype=Article\&impression_id=e07e3bf2-e38b-11ea-befc-036b724b9a9b}{Trump's
  Changes}
\item
  \href{https://www.nytimes3xbfgragh.onion/interactive/2020/04/19/climate/climate-crash-course-1.html?name=styln-climate\&region=TOP_BANNER\&variant=undefined\&block=storyline_menu_recirc\&action=click\&pgtype=Article\&impression_id=e07e3bf3-e38b-11ea-befc-036b724b9a9b}{Climate
  101}
\item
  \href{https://www.nytimes3xbfgragh.onion/interactive/2018/08/30/climate/how-much-hotter-is-your-hometown.html?name=styln-climate\&region=TOP_BANNER\&variant=undefined\&block=storyline_menu_recirc\&action=click\&pgtype=Article\&impression_id=e07e3bf4-e38b-11ea-befc-036b724b9a9b}{Is
  Your Hometown Hotter?}
\end{itemize}

Advertisement

\protect\hyperlink{after-top}{Continue reading the main story}

Supported by

\protect\hyperlink{after-sponsor}{Continue reading the main story}

\hypertarget{as-coal-fades-in-the-us-natural-gas-becomes-the-climate-battleground}{%
\section{As Coal Fades in the U.S., Natural Gas Becomes the Climate
Battleground}\label{as-coal-fades-in-the-us-natural-gas-becomes-the-climate-battleground}}

\includegraphics{https://static01.graylady3jvrrxbe.onion/images/2019/06/27/science/27cli-naturalgas4/merlin_156629928_6ee8d11b-43c6-472f-aeba-bb115c9af5d0-articleLarge.jpg?quality=75\&auto=webp\&disable=upscale}

\href{https://www.nytimes3xbfgragh.onion/by/brad-plumer}{\includegraphics{https://static01.graylady3jvrrxbe.onion/images/2018/02/20/multimedia/author-brad-plumer/author-brad-plumer-thumbLarge.jpg}}

By \href{https://www.nytimes3xbfgragh.onion/by/brad-plumer}{Brad Plumer}

\begin{itemize}
\item
  June 26, 2019
\item
  \begin{itemize}
  \item
  \item
  \item
  \item
  \item
  \item
  \end{itemize}
\end{itemize}

America's coal-burning power plants are
\href{https://www.nytimes3xbfgragh.onion/interactive/2018/06/13/climate/coal-nuclear-bailout.html?module=inline}{shutting
down at a rapid pace}, forcing electric utilities to face the next big
climate question: Embrace natural gas, or shift aggressively to
renewable energy?

Some large utilities, including Xcel Energy in the Upper Midwest, are
now planning to sharply cut their coal and gas use in favor of clean and
abundant wind and solar power, which have steadily fallen in cost. But
in the Southeast and other regions, natural gas continues to dominate,
because of its reliability and low prices driven by the fracking boom.
Nationwide, energy companies plan to add
\href{https://www.eia.gov/electricity/annual/pdf/epa.pdf}{at least 150
new gas plants} and thousands of miles of pipelines in the years ahead.

\emph{Want climate news in your inbox?}
\href{https://www.nytimes3xbfgragh.onion/newsletters/climate-change}{\emph{Sign
up here
for}}\textbf{\href{https://www.nytimes3xbfgragh.onion/newsletters/climate-change}{\emph{Climate
Fwd:}}}\emph{, our email newsletter.}

A rush to build gas-fired plants, even though they emit only half as
much carbon pollution as coal, has the potential to lock in decades of
new fossil-fuel use right as scientists say emissions
\href{https://www.nytimes3xbfgragh.onion/interactive/2018/10/07/climate/ipcc-report-half-degree.html}{need
to fall drastically by midcentury} to avert the worst impacts of global
warming.

``Gas infrastructure that's built today is going to be with us for 30
years,'' said Daniel Cohan, an associate professor of civil and
environmental engineering at Rice University.

``But if you look at scenarios that take climate change seriously, that
say we need to get to net zero emissions by 2050,'' he said, ``that's
not going to be compatible with gas plants that don't capture their
carbon.''

In some states, policymakers are now pushing to leave gas behind to meet
ambitious climate goals. Last week, New York lawmakers
\href{https://www.nytimes3xbfgragh.onion/2019/06/18/nyregion/greenhouse-gases-ny.html}{passed
a sweeping energy bill} that calls for the state to switch to entirely
carbon-free electricity sources by 2040, following states like
California and New Mexico that have passed similar laws.

Since 2005, most power companies have lowered their carbon dioxide
emissions significantly, in large part by shifting from coal to gas.
Coal plants have become uncompetitive with other kinds of energy
generation in much of the country, despite the Trump administration's
efforts to save them by
\href{https://www.nytimes3xbfgragh.onion/2019/06/19/climate/epa-coal-emissions.html}{rolling
back federal pollution regulations}.

But in \href{http://www.energyandpolicy.org/utility-carbon-targets}{a
recent analysis}, David Pomerantz, the executive director of the Energy
and Policy Institute, a pro-renewables group, looked at the long-term
plans of the 22 biggest investor-owned utilities. Some in the Midwest
are planning to speed up the rate at which they cut emissions between
now and 2030. But other large utilities, like Duke Energy and American
Electric Power, expect to reduce their carbon emissions at a slower pace
over the next decade than they had over the previous decade.

``I really think gas is at the crux of it,'' Mr. Pomerantz said.
``You've got some utilities looking at gas and saying, `No thanks, we
think there's a cleaner and cheaper path.' But then you've got others
going all-in on gas.''

\href{https://www.nytimes3xbfgragh.onion/section/climate?action=click\&pgtype=Article\&state=default\&region=MAIN_CONTENT_1\&context=storylines_keepup}{}

\hypertarget{climate-and-environment-}{%
\subsubsection{Climate and Environment
›}\label{climate-and-environment-}}

\hypertarget{keep-up-on-the-latest-climate-news}{%
\paragraph{Keep Up on the Latest Climate
News}\label{keep-up-on-the-latest-climate-news}}

Updated Aug. 18, 2020

Here's what you need to know this week:

\begin{itemize}
\item
  \begin{itemize}
  \tightlist
  \item
    Five automakers
    \href{https://www.nytimes3xbfgragh.onion/2020/08/17/climate/california-automakers-pollution.html?action=click\&pgtype=Article\&state=default\&region=MAIN_CONTENT_1\&context=storylines_keepup}{sealed
    a binding agreement} with California to follow the state's stricter
    tailpipe emissions rules.
  \item
    The Trump
    administration\href{https://www.nytimes3xbfgragh.onion/2020/08/13/climate/trump-methane.html?action=click\&pgtype=Article\&state=default\&region=MAIN_CONTENT_1\&context=storylines_keepup}{eliminated
    a major methane rule}, even as leaks are worsening, in a decision
    that researchers warned ignored science.
  \item
    Climate change leaders said
    \href{https://www.nytimes3xbfgragh.onion/2020/08/12/climate/kamala-harris-environmental-justice.html?action=click\&pgtype=Article\&state=default\&region=MAIN_CONTENT_1\&context=storylines_keepup}{the
    vice-presidential choice of Kamala Harris} signaled that Democrats
    will have a focus on environmental justice.
  \end{itemize}
\end{itemize}

\hypertarget{where-natural-gas-plants-are-expanding}{%
\subsection{Where Natural Gas Plants Are
Expanding}\label{where-natural-gas-plants-are-expanding}}

Last fall, in North and South Carolina, a pair of utilities owned by
Duke Energy
\href{https://www.utilitydive.com/news/duke-15-year-plans-lean-heavy-on-gas-to-replace-coal/531924/}{filed
plans with state regulators} to continue retiring coal plants and
largely replace them with more than 9,500 megawatts of new natural gas
capacity by 2033. The utilities also plan to add a smaller amount of
solar capacity, about 3,600 megawatts, over the same time frame.

``Right now, gas is still the most cost-effective option for us,'' said
Kenneth Jennings, Duke's director of renewable strategy and policy.

\href{https://www.nytimes3xbfgragh.onion/interactive/2018/12/24/climate/how-electricity-generation-changed-in-your-state.html}{}

\includegraphics{https://static01.graylady3jvrrxbe.onion/images/2018/12/23/us/how-electricity-generation-changed-in-your-state-promo-1545597148124/how-electricity-generation-changed-in-your-state-promo-1545597148124-articleLarge.png}

\hypertarget{how-does-your-state-make-electricity}{%
\subsection{How Does Your State Make
Electricity?}\label{how-does-your-state-make-electricity}}

There's been a major shift in how America makes electricity over the
past two decades. Each state has its own story.

One challenge with using more solar power, he noted, is finding a way to
supply electricity when the sun isn't shining. Although Duke is
installing some large lithium-ion batteries to store solar energy for
less-sunny hours, the company says batteries still haven't reached the
point where they're as cheap or effective as gas power, which can run at
all hours.

Mr. Jennings also said that it can be tough to add wind power in the
Carolinas, where the terrain is less favorable than the wide-open
Midwest and lawmakers have limited the construction of new turbines on
mountain ridges and near military bases along the coast.

Opponents of Duke's plans, including environmental groups and local
renewable energy producers, have urged state regulators to push the
utility to reconsider. They
\href{https://drive.google.com/file/d/1BXbaHrhTTSAk7yq7WsMQwlnlkCBG5l35/view}{have
sharply disputed Duke's analysis}, arguing that the utility is
downplaying the potential for solar, wind and batteries.

\includegraphics{https://static01.graylady3jvrrxbe.onion/images/2019/06/18/science/18cli-naturalgas1/merlin_154785057_5d6d2308-b381-4df2-9c79-c606b4319b1c-articleLarge.jpg?quality=75\&auto=webp\&disable=upscale}

A similar fight is unfolding in Florida, where the local Sierra Club
\href{https://www.sierraclub.org/press-releases/2019/05/despite-opposition-judge-allows-teco-plans-for-fracked-gas-and-coal-move}{is
challenging a proposal} by Tampa Electric to replace two older coal
units with a large new natural gas plant. The Sierra Club's pitch to the
governor, who still has to approve the plan: Florida can't afford to
deepen its reliance on gas at a time when climate change and sea level
rise are threatening the state's coast.

For Tampa Electric, the choice is complex. The utility plans to get 7
percent of its power from solar by 2021, but says that until storage
technologies improve, gas will form the backbone of its energy mix as it
tries to meet energy needs in a fast-growing part of the state.

These disputes are popping up in states around the country. Over the
last decade, groups like the Sierra Club have tried to persuade
utilities and regulators that they could save money by retiring coal and
shifting to a cleaner mix of gas and renewables. Now they're running the
same playbook against gas, arguing that the costs of wind, solar and
batteries
\href{https://rmi.org/insight/the-economics-of-clean-energy-portfolios/}{have
declined so drastically} that it's time to stop building new gas plants,
too.

So far, results have been mixed: Regulators in
\href{https://www.utilitydive.com/news/arizona-regulators-move-to-place-gas-plant-moratorium-on-utilities/519176/}{Arizona}
and
\href{https://www.utilitydive.com/news/indiana-regulators-reject-vectren-gas-plant-over-stranded-asset-concerns/553456/}{Indiana}
have recently blocked plans for new gas plants, agreeing with opponents
that utilities hadn't fully considered alternatives and that large new
gas projects could be a risky bet at a time when clean energy technology
is improving fast.

But last year in Michigan,
\href{https://www.bridgemi.com/michigan-environment-watch/michigan-approves-1b-dte-natural-gas-plant-blow-environmentalists}{regulators
approved} DTE Energy's plan to build a new \$1 billion gas plant,
rejecting
\href{https://blog.ucsusa.org/sam-gomberg/dte-customers-could-save-340-million-with-clean-energy-compared-to-proposed-gas-plant}{analyses}
by outside groups that the utility could save ratepayers money by
scrapping the plant and making greater use of wind, solar and energy
efficiency.

Image

Vast stretches of solar panels at Babcock Ranch's solar farm in Florida
in earlier this year.Credit...Zack Wittman for The New York Times

\hypertarget{where-renewables-are-gaining}{%
\subsection{Where Renewables Are
Gaining}\label{where-renewables-are-gaining}}

At the same time, some utilities are discovering on their own that it
can make financial sense to take a more ambitious leap toward renewable
energy.

Last year in Indiana, the Northern Indiana Public Service Company, or
Nipsco, opened bidding to outside energy developers and found that
adding a mix of wind, solar and batteries
\href{https://www.nipsco.com/docs/librariesprovider11/rates-and-tariffs/irp/irp-executive-summary.pdf?sfvrsn=9}{would
be cheaper} than building a new gas plant to replace its retiring coal
units. (The company will keep its older gas plants online to fill in
gaps when wind and solar aren't available.) Doing so, the utility
estimated, would reduce its emissions 90 percent below 2005 levels by
2030.

``We were surprised by that,'' said Joe Hamrock, the chief executive of
the company that owns the Nipsco. ``Renewables in our particular
situation were far more competitive than we realized.''

Mr. Hamrock noted that his utility had advantages that others might not
have: Its territory sits near land that's ripe for wind development,
making it easier to build new turbines close by without the need for
lots of costly new transmission lines. ``The answer we got might look
very different for someone just 100 miles away,'' he said.

Indeed, things look very different nearby in
\href{https://www.pjm.com/}{the vast regional grid known as PJM} that
serves 65 million people from Ohio to New Jersey. There power plants
compete in a largely deregulated market and companies are expected to
build over 10,000 megawatts of new gas plants by 2024 to take advantage
of cheap natural gas from the nearby fracking boom in Ohio, Pennsylvania
and West Virginia.

``The shale gas revolution has, frankly, caused a delay in the growth of
renewables here,'' said Stu Bresler, senior vice president for
operations and markets at PJM Interconnection, which oversees the
system. Wind and solar make up less than 6 percent of the region's
generating capacity, well below the national average.

\hypertarget{decisions-from-the-states}{%
\subsection{Decisions From the States}\label{decisions-from-the-states}}

State legislatures are also increasingly weighing in on which energy
sources get built. To date, 29 states
\href{http://www.ncsl.org/research/energy/renewable-portfolio-standards.aspx}{have
enacted laws} that require their utilities to get a certain fraction of
their power from wind and solar.

Now, some states are going further. Over the past year, California,
Colorado, Maine, Nevada, New Mexico, New York and Washington
\href{https://www.catf.us/wp-content/uploads/2019/05/State-and-Utility-Climate-Change-Targets.pdf}{have
all passed laws} aimed at getting 100 percent of their electricity from
carbon-free sources by midcentury, which would eventually mean phasing
out conventional gas plants.

Yet even utilities that are already shifting more heavily into
renewables say that it will be challenging to get rid of gas altogether.

Last year, Xcel Energy, which serves eight states including Colorado and
Minnesota,
\href{https://www.xcelenergy.com/environment/carbon_reduction_plan}{said
it would shut down all its remaining coal plants in the years ahead} and
push to go completely carbon-free by 2050, saying that renewable energy,
helped in part by federal subsidies, had fallen so much in price that
this was now the cheapest option.

While the utility thinks it can get 80 percent of the way to its
emissions goals by 2030 with a mix of wind, solar, batteries and its
existing nuclear plants, it will still rely on natural gas to provide
the rest of its power and is building a new gas plant in Minnesota to
balance out its supply.

Ben Fowke, the chief executive of Xcel, said that getting to 100 percent
carbon-free power will likely require new technology that can supplant
natural gas as a cost-effective backup fuel. Some possibilities include
burning clean hydrogen instead of gas in power plants, developing
techniques that enable carbon produced by gas plants to be captured and
stored underground, advanced nuclear power or the invention of new
energy storage techniques.

Perfecting that technology would likely require big new investments in
research and support from policymakers, he said. ``But I'm convinced we
can get there.''

\emph{For more news on climate and the environment,}
\href{https://twitter.com/nytclimate}{\emph{follow @NYTClimate on
Twitter}}\emph{.}

Advertisement

\protect\hyperlink{after-bottom}{Continue reading the main story}

\hypertarget{site-index}{%
\subsection{Site Index}\label{site-index}}

\hypertarget{site-information-navigation}{%
\subsection{Site Information
Navigation}\label{site-information-navigation}}

\begin{itemize}
\tightlist
\item
  \href{https://help.nytimes3xbfgragh.onion/hc/en-us/articles/115014792127-Copyright-notice}{©~2020~The
  New York Times Company}
\end{itemize}

\begin{itemize}
\tightlist
\item
  \href{https://www.nytco.com/}{NYTCo}
\item
  \href{https://help.nytimes3xbfgragh.onion/hc/en-us/articles/115015385887-Contact-Us}{Contact
  Us}
\item
  \href{https://www.nytco.com/careers/}{Work with us}
\item
  \href{https://nytmediakit.com/}{Advertise}
\item
  \href{http://www.tbrandstudio.com/}{T Brand Studio}
\item
  \href{https://www.nytimes3xbfgragh.onion/privacy/cookie-policy\#how-do-i-manage-trackers}{Your
  Ad Choices}
\item
  \href{https://www.nytimes3xbfgragh.onion/privacy}{Privacy}
\item
  \href{https://help.nytimes3xbfgragh.onion/hc/en-us/articles/115014893428-Terms-of-service}{Terms
  of Service}
\item
  \href{https://help.nytimes3xbfgragh.onion/hc/en-us/articles/115014893968-Terms-of-sale}{Terms
  of Sale}
\item
  \href{https://spiderbites.nytimes3xbfgragh.onion}{Site Map}
\item
  \href{https://help.nytimes3xbfgragh.onion/hc/en-us}{Help}
\item
  \href{https://www.nytimes3xbfgragh.onion/subscription?campaignId=37WXW}{Subscriptions}
\end{itemize}
