Sections

SEARCH

\protect\hyperlink{site-content}{Skip to
content}\protect\hyperlink{site-index}{Skip to site index}

\href{https://www.nytimes3xbfgragh.onion/section/science}{Science}

\href{https://myaccount.nytimes3xbfgragh.onion/auth/login?response_type=cookie\&client_id=vi}{}

\href{https://www.nytimes3xbfgragh.onion/section/todayspaper}{Today's
Paper}

\href{/section/science}{Science}\textbar{}NASA Announces New Dragonfly
Drone Mission to Explore Titan

\url{https://nyti.ms/2LorrmE}

\begin{itemize}
\item
\item
\item
\item
\item
\end{itemize}

Advertisement

\protect\hyperlink{after-top}{Continue reading the main story}

Supported by

\protect\hyperlink{after-sponsor}{Continue reading the main story}

\hypertarget{nasa-announces-new-dragonfly-drone-mission-to-explore-titan}{%
\section{NASA Announces New Dragonfly Drone Mission to Explore
Titan}\label{nasa-announces-new-dragonfly-drone-mission-to-explore-titan}}

The quadcopter was selected to study the moon of Saturn after a ``Shark
Tank''-like competition that lasted two and a half years.

\includegraphics{https://static01.graylady3jvrrxbe.onion/images/2019/06/27/science/27NASA2/merlin_157087080_5786b661-60c0-45db-bb2a-ce16bd7d86a2-videoSixteenByNineJumbo1600.jpg}

By David W. Brown

\begin{itemize}
\item
  June 27, 2019
\item
  \begin{itemize}
  \item
  \item
  \item
  \item
  \item
  \end{itemize}
\end{itemize}

NASA announced Thursday that it is sending a drone-style quadcopter to
Titan, Saturn's largest moon.

Dragonfly, as the mission is called, will be capable of soaring across
the skies of Titan and landing intermittently to take scientific
measurements, studying the world's mysterious atmosphere and topography
while searching for hints of life on the only world other than Earth in
our solar system with standing liquid on its surface. The mission will
be developed and led from the Applied Physics Laboratory at Johns
Hopkins University in Laurel, Md.

``This revolutionary mission would have been unthinkable just a few
years ago,'' said Jim Bridenstine, the administrator of NASA, in
\href{https://twitter.com/JimBridenstine/status/1144334797101260800}{a
video statement announcing the mission}.

The spacecraft is scheduled to launch in 2026. Once at Titan in 2034,
Dragonfly will have a life span of at least two-and-a-half years, with a
battery that will be recharged with a radioactive power source between
flights. Cameras on Dragonfly will stream images during flight, offering
people on Earth a bird's-eye view of the Saturn moon.

``We will be flying initially over dunes and then into rugged terrain,''
said Elizabeth Turtle, who will lead the mission for the lab as its
principal investigator. ``We will take images with both downward-looking
cameras along the ground track underneath Dragonfly as we fly over the
surface, as well as forward-looking cameras, so we'll be able to look
out toward the horizon as well.''

\emph{\emph{\emph{{[}}\href{https://www.nytimes3xbfgragh.onion/interactive/2019/science/astronomy-space-calendar.html}{\emph{Sign
up to get reminders for space and astronomy events on your
calendar}}}.{]}}**

Titan has long intrigued planetary scientists. On Christmas Day 2004,
\href{https://www.nytimes3xbfgragh.onion/interactive/2017/09/14/science/cassini-saturn-images.html}{NASA's
Cassini spacecraft} sent a probe, Huygens, to the moon's surface. It
landed in one piece, revealing a world analogous to a primordial Earth
--- Dr. Turtle described it as, ``eerily familiar on such a different
and exotic world.'' Rather than water, Titan's seas are filled with
liquid methane.

In addition to a camera, Dragonfly will carry an assortment of
scientific instruments: spectrometers to study Titan's composition; a
suite of meteorology sensors; and even a seismometer to detect
titanquakes when it lands on the ground. Drills in the landing skids
will collect samples of the Titan surface for onboard analysis.

``Titan is an incredibly unique opportunity scientifically,'' Dr. Turtle
said in an interview in April before NASA's announcement. ``Not only is
it an ocean world --- an icy satellite with a water ocean in its
interior --- but it is the only satellite with an atmosphere. And the
atmosphere at Titan has methane in it, which leads to all sorts of rich
organic chemistry happening at even the upper reaches of the
atmosphere.''

\includegraphics{https://static01.graylady3jvrrxbe.onion/images/2019/06/27/science/27NASA4/27NASA4-articleLarge.jpg?quality=75\&auto=webp\&disable=upscale}

Part of the Dragonfly mission is to study whether the moon of Saturn
could now be, or once was, home to life.

Because of the nature of its atmosphere, Titan is a very Earthlike
place. Chemically, it is very much like our world's primordial past. The
surface pressure of Titan is one-and-a-half times the surface pressure
of Earth, and the same sorts of interactions between air, land and sea
take place. Titan thus has familiar geology. Methane on Titan plays the
role that water plays here. Its methane cycle is analogous to Earth's
water cycle. It has methane clouds, methane rain and methane lakes and
seas on the surface.

``There's going to be a tremendous change in the fabric of how we see
Titan as a world,'' said Dr. Ralph Lorenz of Applied Physics Laboratory,
the Dragonfly project scientist in an April interview. He predicted that
features of Titan will be, ``recognizable, but different in flavor from
what you see on Earth and Mars.''

That might include the things that wiggle. Complex organic molecules
fall from its atmosphere onto the surface of Titan, gather over long
periods of time and can be processed further. If cryovolcanoes erupt on
Titan's surface, as data from the Cassini spacecraft suggests, the
organic material can mix with liquid water. Sunlight, at the same time,
drives the moon's photochemistry, introducing energy to a system primed
for life.

``We have all these ingredients necessary for life as we know it, and
they're just sitting there doing chemistry experiments on the surface of
Titan. That's why we want to send a lander there,'' said Dr. Turtle.

\href{https://www.nytimes3xbfgragh.onion/interactive/2017/09/14/science/cassini-saturn-images.html}{}

\includegraphics{https://static01.graylady3jvrrxbe.onion/images/2017/09/16/insider/cassini-saturn-images-1505332179034/cassini-saturn-images-1505332179034-square640-v3.jpg}

\hypertarget{100-images-from-cassinis-mission-to-saturn}{%
\subsection{100 Images From Cassini's Mission to
Saturn}\label{100-images-from-cassinis-mission-to-saturn}}

NASA's Cassini spacecraft burned up in Saturn's atmosphere on Friday,
after 20 years in space.

The rotocopter comes after years of studying alternative concepts for
studying Titan, such as a conventional orbiter or lander, a hot-air
balloon and even a boat. Because it takes about two hours for a signal
from Earth to reach Titan, Dragonfly is designed to fly and land
autonomously; onboard hazard-detection will keep it safe

``One of the things great about Dragonfly is that we are not inventing
anything. We are just applying technology already developed for other
things to a new problem,'' says Dr. Turtle.

Dragonfly is similar in size to a Mars rover, or about the size of a
large lawn mower. Where a Mars rover is limited to inching forward over
a decade or longer, however, for the Dragonfly team, Titan's sky and the
drone's nuclear fuel source are the limit.

``We have the capacity, over the mission's lifetime, to go hundreds of
kilometers. One of the advantages we have is that we can always scout
the next site. We can fly ahead, look at it, see what kind of terrain
there is, and decide whether we want to go there or elsewhere,'' said
Dr. Turtle.

The spacecraft has been under consideration for two-and-a-half years in
NASA's class of science missions, called New Frontiers, which are
supposed to cost less than \$1 billion. The competition, held between
multiple institutions in government and academia, is not unlike a
``Shark Tank'' for deep space exploration.

Earlier winners of the New Frontiers competition include
the\href{https://www.nytimes3xbfgragh.onion/interactive/2016/03/17/science/pluto-images-charon-moons-new-horizons-flyby.html}{New
Horizons spacecraft, which visited
Pluto};\href{https://www.nytimes3xbfgragh.onion/2017/12/13/science/jupiter-great-red-spot-juno.html}{Juno,
which now orbits Jupiter};
and\href{https://www.nytimes3xbfgragh.onion/interactive/2019/03/19/science/osiris-rex-bennu-photos.html}{Osiris-rex,
which will soon collect a sample} from the asteroid Bennu and return it
to Earth.

NASA has announced other new missions recently. Last week, the agency
said twin missions
---\href{https://www.nasa.gov/press-release/nasa-selects-missions-to-study-our-sun-its-effects-on-space-weather}{Punch
and Tracers} --- would seek to further scientific understanding of the
sun. And in May, the Trump administration renewed its drive to return
astronauts to the moon in
2024,\href{https://www.nytimes3xbfgragh.onion/2019/05/22/science/trump-moon-nasa.html}{renaming
the mission Artemis} and seeking an additional \$1.6 billion in funds
for NASA as a down payment.

Advertisement

\protect\hyperlink{after-bottom}{Continue reading the main story}

\hypertarget{site-index}{%
\subsection{Site Index}\label{site-index}}

\hypertarget{site-information-navigation}{%
\subsection{Site Information
Navigation}\label{site-information-navigation}}

\begin{itemize}
\tightlist
\item
  \href{https://help.nytimes3xbfgragh.onion/hc/en-us/articles/115014792127-Copyright-notice}{©~2020~The
  New York Times Company}
\end{itemize}

\begin{itemize}
\tightlist
\item
  \href{https://www.nytco.com/}{NYTCo}
\item
  \href{https://help.nytimes3xbfgragh.onion/hc/en-us/articles/115015385887-Contact-Us}{Contact
  Us}
\item
  \href{https://www.nytco.com/careers/}{Work with us}
\item
  \href{https://nytmediakit.com/}{Advertise}
\item
  \href{http://www.tbrandstudio.com/}{T Brand Studio}
\item
  \href{https://www.nytimes3xbfgragh.onion/privacy/cookie-policy\#how-do-i-manage-trackers}{Your
  Ad Choices}
\item
  \href{https://www.nytimes3xbfgragh.onion/privacy}{Privacy}
\item
  \href{https://help.nytimes3xbfgragh.onion/hc/en-us/articles/115014893428-Terms-of-service}{Terms
  of Service}
\item
  \href{https://help.nytimes3xbfgragh.onion/hc/en-us/articles/115014893968-Terms-of-sale}{Terms
  of Sale}
\item
  \href{https://spiderbites.nytimes3xbfgragh.onion}{Site Map}
\item
  \href{https://help.nytimes3xbfgragh.onion/hc/en-us}{Help}
\item
  \href{https://www.nytimes3xbfgragh.onion/subscription?campaignId=37WXW}{Subscriptions}
\end{itemize}
