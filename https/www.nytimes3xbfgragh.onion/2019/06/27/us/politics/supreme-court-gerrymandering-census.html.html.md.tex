Sections

SEARCH

\protect\hyperlink{site-content}{Skip to
content}\protect\hyperlink{site-index}{Skip to site index}

\href{https://www.nytimes3xbfgragh.onion/section/politics}{Politics}

\href{https://myaccount.nytimes3xbfgragh.onion/auth/login?response_type=cookie\&client_id=vi}{}

\href{https://www.nytimes3xbfgragh.onion/section/todayspaper}{Today's
Paper}

\href{/section/politics}{Politics}\textbar{}Supreme Court Green-Lights
Gerrymandering and Blocks Census Citizenship Question

\url{https://nyti.ms/2LnZ87O}

\begin{itemize}
\item
\item
\item
\item
\item
\item
\end{itemize}

Advertisement

\protect\hyperlink{after-top}{Continue reading the main story}

Supported by

\protect\hyperlink{after-sponsor}{Continue reading the main story}

\hypertarget{supreme-court-green-lights-gerrymandering-and-blocks-census-citizenship-question}{%
\section{Supreme Court Green-Lights Gerrymandering and Blocks Census
Citizenship
Question}\label{supreme-court-green-lights-gerrymandering-and-blocks-census-citizenship-question}}

\includegraphics{https://static01.graylady3jvrrxbe.onion/images/2019/06/27/us/27dc-scotus-1/merlin_157077816_11710f58-5749-42da-80e6-5a9d910ec3e1-articleLarge.jpg?quality=75\&auto=webp\&disable=upscale}

By \href{https://www.nytimes3xbfgragh.onion/by/adam-liptak}{Adam Liptak}

\begin{itemize}
\item
  June 27, 2019
\item
  \begin{itemize}
  \item
  \item
  \item
  \item
  \item
  \item
  \end{itemize}
\end{itemize}

WASHINGTON --- In a pair of decisions with vast implications for the
American political landscape, the Supreme Court on Thursday delivered a
victory to Republicans by
\href{https://www.supremecourt.gov/opinions/18pdf/18-422_9ol1.pdf}{ruling
that federal courts are powerless to hear} challenges to extreme
partisan
\href{https://www.nytimes3xbfgragh.onion/2020/07/21/us/politics/trump-immigrants-census-redistricting.html}{gerrymandering}
but gave a reprieve to Democrats by
\href{https://www.supremecourt.gov/opinions/18pdf/18-966_bq7c.pdf}{delaying
the Trump administration's efforts} to add a question on citizenship to
the 2020 census.

The key parts of both decisions were decided by 5-to-4 votes.

In the gerrymandering case,
\href{https://www.nytimes3xbfgragh.onion/2019/06/27/us/politics/chief-justice-roberts.html}{Chief
Justice John G. Roberts Jr.} joined his usual conservative allies. In
the census case, he broke with them to vote with the court's four-member
liberal wing in preventing, for now, what advocates have argued would be
a deterrent to immigrants from participating in the once-a-decade count.

\emph{{[}The Supreme Court's rulings on gerrymandering and the census}
\href{https://www.nytimes3xbfgragh.onion/2019/06/27/us/supreme-court-gerrymandering-census.html}{\emph{have
profound implications}} \emph{for American politics. Here's what the
decisions mean.{]}}

The gerrymandering decision was momentous, definitively closing the door
on judicial challenges to voting maps warped by politics. The practice
of redrawing the boundaries of voting districts is almost as old as the
nation. Both parties have used it, but in recent years, Republicans have
been the primary beneficiaries.

The drafters of the Constitution, Chief Justice Roberts wrote for the
majority, understood that politics would play a role in drawing election
districts when they gave the task to state legislatures. Judges, he
said, are not entitled to second-guess lawmakers' judgments.

``We conclude that partisan gerrymandering claims present political
questions beyond the reach of the federal courts,'' the chief justice
wrote.

\emph{{[}}\href{https://www.nytimes3xbfgragh.onion/2019/06/27/us/gerrymander-explainer.html?action=click\&module=Intentional\&pgtype=Article}{\emph{Here's
what you need to know}} \emph{about gerrymandering.{]}}

In an impassioned dissent delivered from the bench, Justice Elena Kagan
said American democracy would suffer thanks to the court's ruling in the
two consolidated cases decided Thursday, Rucho v. Common Cause, No.
18-422, and Lamone v. Benisek, No. 18-726.

``The practices challenged in these cases imperil our system of
government,'' she said. ``Part of the court's role in that system is to
defend its foundations. None is more important than free and fair
elections.''

She added that she was dissenting with ``deep sadness.'' She was joined
by Justices Ruth Bader Ginsburg, Stephen G. Breyer and Sonia Sotomayor.

Chief Justice Roberts did not say the current system of drawing
districts was desirable as a matter of policy. ``Excessive partisanship
in districting leads to results that reasonably seem unjust,'' he wrote.

``The districting plans at issue here are highly partisan, by any
measure,'' he wrote. ``The question is whether the courts below
appropriately exercised judicial power when they found them
unconstitutional as well.''

The answer, he wrote, is no, as courts lack the authority and competence
to decide when politics has played too large a role in
\href{https://www.nytimes3xbfgragh.onion/2020/07/21/us/politics/trump-immigrants-census-redistricting.html}{redistricting}.
``There are no legal standards discernible in the Constitution for
making such judgments,'' the chief justice wrote, ``let alone limited
and precise standards that are clear, manageable and politically
neutral.''

Justices Clarence Thomas, Samuel A. Alito Jr., Neil M. Gorsuch and Brett
M. Kavanaugh joined the majority opinion.

\includegraphics{https://static01.graylady3jvrrxbe.onion/images/2019/06/27/us/27dc-scotus-2/merlin_150270153_2fd92b10-fc4e-45ca-b0d4-5e0fe33bb78d-articleLarge.jpg?quality=75\&auto=webp\&disable=upscale}

In her dissent, Justice Kagan said the court had abdicated one of its
most crucial responsibilities.

``The only way to understand the majority's opinion,'' she wrote, ``is
as follows: In the face of grievous harm to democratic governance and
flagrant infringements on individuals' rights --- in the face of
escalating partisan manipulation whose compatibility with this nation's
values and law no one defends --- the majority declines to provide any
remedy. For the first time in this nation's history, the majority
declares that it can do nothing about an acknowledged constitutional
violation because it has searched high and low and cannot find a
workable legal standard to apply.''

Chief Justice Roberts countered that his majority opinion was a modest
one that recognized the limits of judicial power.

``No one can accuse this court of having a crabbed view of the reach of
its competence,'' he wrote. ``But we have no commission to allocate
political power and influence in the absence of a constitutional
directive or legal standards to guide us in the exercise of such
authority.''

Thursday's second major ruling, on the census, may turn out to be less
consequential. But it was nonetheless a striking setback for the Trump
administration. Since 1950, the government has not included a question
about citizenship in the forms sent to each household, but the
administration was confident it would prevail before a court it views as
generally sympathetic to its assertions of executive power.

But court rejected the administration's stated reason for adding a
question on citizenship to the census, leaving in doubt whether the
question would appear on the forms sent to every household in the nation
next year.

Chief Justice Roberts, writing for the majority, said the
administration's explanation for adding the question ``seems to have
been contrived.'' But he left open the possibility that it could provide
an adequate answer.

Executive branch officials must ``offer genuine justifications for
important decisions, reasons that can be scrutinized by courts and the
interested public,'' the chief justice wrote. ``Accepting contrived
reasons would defeat the purpose of the enterprise. If judicial review
is to be more than an empty ritual, it must demand something better than
the explanation offered for the action taken in this case.''

The practical effect of the decision was not immediately clear. While
the question is banned for now, it is at least possible that the
administration will be able to offer adequate justifications for it. But
time is short, as the
\href{https://www.nytimes3xbfgragh.onion/2019/06/27/us/census-printing.html}{census
forms must be printed} soon.

President Trump
\href{https://twitter.com/realDonaldTrump/status/1144298731887628288}{commented
on Twitter}, writing that he had ``asked the lawyers if they can delay
the Census, no matter how long, until the United States Supreme Court is
given additional information from which it can make a final and decisive
decision on this very critical matter.''

Thursday's decision was fractured, but the key passage in the chief
justice's majority opinion was joined only by the court's four-member
liberal wing.

The census, the nation's
\href{https://www.oig.doc.gov/OIGPublications/OIG-11-030-I.pdf}{largest
peacetime mobilization}, is overseen by the Commerce Department. In
March 2018, Wilbur Ross, the secretary of commerce, announced that he
planned to add a citizenship question.

Chief Justice Roberts wrote that executive branch officials ordinarily
had broad discretion to make policy judgments. But he said the record in
the case demonstrated that Mr. Ross had not given a full and accurate
account of his decision to add the question.

In sworn testimony before Congress, the secretary said he had decided to
add the question ``solely'' in response to
\href{https://www.documentcloud.org/documents/4340651-Text-of-Dec-2017-DOJ-letter-to-Census.html}{a
Justice Department request} in December 2017 for data to help it enforce
the Voting Rights Act, or the V.R.A. Three federal trial judges
\href{https://www.brennancenter.org/sites/default/files/legal-work/2019-01-15-574-Findings\%20Of\%20Fact.pdf}{have
ruled} that
\href{https://www.brennancenter.org/sites/default/files/legal-work/Order_3\%3A6\%3A19.pdf}{the
evidence} in the record
\href{https://www.brennancenter.org/sites/default/files/legal-work/FindingsofFact_\%202019-04-05.pdf}{demonstrated}
that Mr. Ross was not being truthful.

Image

Demonstrators outside the Supreme Court on Thursday. The ruling on the
census may turn out to be less consequential.Credit...Samuel Corum for
The New York Times

\emph{{[}}\href{https://www.nytimes3xbfgragh.onion/2019/06/27/us/census-question-citizenship.html?action=click\&module=Intentional\&pgtype=Article}{\emph{Here's
what you need to know}} \emph{about the debate over adding a citizenship
question to the census.{]}}

Chief Justice Roberts wrote that the evidence in the case showed that
``the V.R.A. played an insignificant role in the decision-making
process.''

``The secretary,'' he wrote, ``was determined to reinstate a citizenship
question from the time he entered office; instructed his staff to make
it happen; waited while commerce officials explored whether another
agency would request census-based citizenship data; subsequently
contacted the attorney general himself to ask if D.O.J. would make the
request; and adopted the Voting Rights Act rationale late in the
process.''

``Altogether,'' the chief justice wrote, ``the evidence tells a story
that does not match the explanation the secretary gave for his
decision.''

The trial judge in the case had given the administration another chance
to provide an explanation, and the Supreme Court affirmed that ruling.

``In these unusual circumstances,'' Chief Justice Roberts wrote, ``the
district court was warranted in remanding to the agency, and we affirm
that disposition.''

Justices Ginsburg, Breyer, Sotomayor and Kagan joined the key part of
the chief justice's opinion.

In dissent, Justice Clarence Thomas said the majority had done something
extraordinary. ``For the first time ever,'' he wrote, ``the court
invalidates an agency action solely because it questions the sincerity
of the agency's otherwise adequate rationale.''

Justices Gorsuch and Kavanaugh joined Justice Thomas's partial dissent.

Justice Thomas said the consequences of the majority decision would be
far-reaching. ``Now that the court has opened up this avenue of
attack,'' he wrote, ``opponents of executive actions have strong
incentives to craft narratives that would derail them.''

Justice Alito filed his own partial dissent.

``To put the point bluntly,'' he wrote, ``the federal judiciary has no
authority to stick its nose into the question whether it is good policy
to include a citizenship question on the census or whether the reasons
given by Secretary Ross for that decision were his only reasons or his
real reasons.''

The case --- United States Department of Commerce v. New York, No.
18-966 --- has its roots in the text of the Constitution, which requires
an ``actual enumeration'' every 10 years, with the House to be
apportioned based on ``the whole number of persons in each state.''

But the government has long used the census to gather information beyond
raw population data. In 2020, for instance,
\href{https://www.census.gov/newsroom/press-releases/2018/2020-question.html}{the
short form} that goes to every household will include questions about
sex, age, race and Hispanic or Latino origin. Some of those questions
may discourage participation, too.

In announcing that he planned to add a citizenship question, Mr. Ross
acknowledged that it could have ``some impact on responses'' but said
the information sought was ``of greater importance than any adverse
effect that may result from people violating their legal duty to
respond.''

Documents disclosed in the case showed that Mr. Ross had discussed the
citizenship issue early in his tenure with Stephen K. Bannon, the former
White House chief strategist and an architect of the Trump
administration's tough immigration policies, and that Mr. Ross had met
at Mr. Bannon's direction with Kris Kobach, the former Kansas secretary
of state and an opponent of unlawful immigration.

After the justices
\href{https://www.nytimes3xbfgragh.onion/2019/04/23/us/politics/supreme-court-census-citizenship.html}{heard
arguments in April}, more evidence emerged from the computer files of
\href{https://www.nytimes3xbfgragh.onion/2018/08/21/obituaries/thomas-hofeller-republican-master-of-political-maps-dies-at-75.html}{Thomas
B. Hofeller, a Republican strategist}. It suggested that the Trump
administration sought to collect citizenship information so that states
could draw voting districts by counting only eligible voters rather than
all residents, as is the current practice. That would, Mr. Hofeller
wrote, ``be advantageous to Republicans and non-Hispanic whites.''

The administration has said that census forms must be printed by June,
but the groups challenging the citizenship question said the real
deadline is October, leaving time for further legal proceedings.

Advertisement

\protect\hyperlink{after-bottom}{Continue reading the main story}

\hypertarget{site-index}{%
\subsection{Site Index}\label{site-index}}

\hypertarget{site-information-navigation}{%
\subsection{Site Information
Navigation}\label{site-information-navigation}}

\begin{itemize}
\tightlist
\item
  \href{https://help.nytimes3xbfgragh.onion/hc/en-us/articles/115014792127-Copyright-notice}{©~2020~The
  New York Times Company}
\end{itemize}

\begin{itemize}
\tightlist
\item
  \href{https://www.nytco.com/}{NYTCo}
\item
  \href{https://help.nytimes3xbfgragh.onion/hc/en-us/articles/115015385887-Contact-Us}{Contact
  Us}
\item
  \href{https://www.nytco.com/careers/}{Work with us}
\item
  \href{https://nytmediakit.com/}{Advertise}
\item
  \href{http://www.tbrandstudio.com/}{T Brand Studio}
\item
  \href{https://www.nytimes3xbfgragh.onion/privacy/cookie-policy\#how-do-i-manage-trackers}{Your
  Ad Choices}
\item
  \href{https://www.nytimes3xbfgragh.onion/privacy}{Privacy}
\item
  \href{https://help.nytimes3xbfgragh.onion/hc/en-us/articles/115014893428-Terms-of-service}{Terms
  of Service}
\item
  \href{https://help.nytimes3xbfgragh.onion/hc/en-us/articles/115014893968-Terms-of-sale}{Terms
  of Sale}
\item
  \href{https://spiderbites.nytimes3xbfgragh.onion}{Site Map}
\item
  \href{https://help.nytimes3xbfgragh.onion/hc/en-us}{Help}
\item
  \href{https://www.nytimes3xbfgragh.onion/subscription?campaignId=37WXW}{Subscriptions}
\end{itemize}
