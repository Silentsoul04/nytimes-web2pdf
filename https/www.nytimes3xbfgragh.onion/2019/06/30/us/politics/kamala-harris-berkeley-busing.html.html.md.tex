Sections

SEARCH

\protect\hyperlink{site-content}{Skip to
content}\protect\hyperlink{site-index}{Skip to site index}

\href{https://www.nytimes3xbfgragh.onion/section/politics}{Politics}

\href{https://myaccount.nytimes3xbfgragh.onion/auth/login?response_type=cookie\&client_id=vi}{}

\href{https://www.nytimes3xbfgragh.onion/section/todayspaper}{Today's
Paper}

\href{/section/politics}{Politics}\textbar{}Kamala Harris and Classmates
Were Bused Across Berkeley. The Experience Changed Them.

\url{https://nyti.ms/2Ny6yIq}

\begin{itemize}
\item
\item
\item
\item
\item
\item
\end{itemize}

\begin{itemize}
\item
  \href{https://www.nytimes3xbfgragh.onion/live/2020/08/20/us/dnc-convention-election?action=click\&pgtype=Article\&state=default\&region=TOP_BANNER\&context=storylines_menu}{D.N.C.
  Updates}
\item
  \href{https://www.nytimes3xbfgragh.onion/2020/08/20/us/politics/biden-presidential-nomination-dnc.html?action=click\&pgtype=Article\&state=default\&region=TOP_BANNER\&context=storylines_menu}{Biden's
  Speech}
\item
  \href{https://www.nytimes3xbfgragh.onion/interactive/2019/us/elections/2020-presidential-election-calendar.html?action=click\&pgtype=Article\&state=default\&region=TOP_BANNER\&context=storylines_menu}{Election
  Calendar}
\item
  \href{https://www.nytimes3xbfgragh.onion/interactive/2020/08/11/us/politics/vote-by-mail-us-states.html?action=click\&pgtype=Article\&state=default\&region=TOP_BANNER\&context=storylines_menu}{Voting
  by Mail}
\item
  \href{https://www.nytimes3xbfgragh.onion/newsletters/politics?action=click\&pgtype=Article\&state=default\&region=TOP_BANNER\&context=storylines_menu}{Politics
  Newsletter}
\end{itemize}

Advertisement

\protect\hyperlink{after-top}{Continue reading the main story}

Supported by

\protect\hyperlink{after-sponsor}{Continue reading the main story}

\hypertarget{kamala-harris-and-classmates-were-bused-across-berkeley-the-experience-changed-them}{%
\section{Kamala Harris and Classmates Were Bused Across Berkeley. The
Experience Changed
Them.}\label{kamala-harris-and-classmates-were-bused-across-berkeley-the-experience-changed-them}}

\includegraphics{https://static01.graylady3jvrrxbe.onion/images/2019/06/30/us/politics/30BERKELEY1/merlin_157269288_0f999406-b851-474b-9dfa-62ee200c15f1-articleLarge.jpg?quality=75\&auto=webp\&disable=upscale}

By \href{https://www.nytimes3xbfgragh.onion/by/nellie-bowles}{Nellie
Bowles}

\begin{itemize}
\item
  June 30, 2019
\item
  \begin{itemize}
  \item
  \item
  \item
  \item
  \item
  \item
  \end{itemize}
\end{itemize}

BERKELEY, Calif. --- In 1967, the superintendent of the Berkeley,
Calif., school district had resolved to desegregate the city schools.
``We will set an example for all the cities of America,'' he wrote in a
report called ``Integration: A Plan for Berkeley,'' which he presented
to the Berkeley Board of Education. ``The children of Berkeley will grow
in a community where justice is part of their pattern of life,'' the
report stated.

Several years later, a young girl named Kamala Harris, the daughter of
an Indian mother and a Jamaican father, boarded a school bus --- part of
that school integration program that would change her, the city and the
country's conversation about racial politics.

``There was a little girl in California who was part of the second class
to integrate her public schools, and she was bused to school every day,
and that little girl was me,'' Ms. Harris, now a senator and candidate
for president, said on Thursday evening onstage at the
\href{https://www.nytimes3xbfgragh.onion/news-event/democratic-debates}{Democratic
debate}. She was directly addressing former Vice President Joseph R.
Biden Jr., and what she described as his history of opposition to
mandatory busing.

In that moment, Ms. Harris invoked a complex part of American history,
and the way cities tried to address how segregated the country's
classrooms remained more than a decade after Brown v. Board of Education
was decided. She also made plain how the conversation about integration
that took place in Washington and in cities around the nation directly
affected the life of a first grader on a school bus. She and her peers
played hand-clapping games to pass the time, one classmate remembered,
aware that their bus ride took them to a neighborhood different from
theirs, but not that it was something a superintendent had to fight for.

\emph{{[}Make sense of the people, issues and ideas}
\emph{\href{https://www.nytimes3xbfgragh.onion/newsletters/politics?smid=rd?action=click\&module=Intentional\&pgtype=Article}{shaping
American politics with our newsletter.}{]}}

Students who attended Berkeley public schools during the early years of
the integration effort recall the experience as a defining one for them.

``Racism didn't go away because we were bused,'' said Doris Alkebulan,
58, who was part of the initial group of black students in Berkeley to
be bused to a majority white school. ``What about play dates? Were you
going to be invited to the birthday party? Would you be chosen for the
team?''

She remembered that kids would say, ``Oh we can't play with you'' and
explain the reason with a racial slur.

``I didn't even know I was black until then,'' Ms. Alkebulan said.

Ms. Alkebulan said that the experience profoundly shaped her life and
that it was emotional watching Ms. Harris call upon that history. ``She
got it right,'' Ms. Alkebulan said.

If Mr. Biden represents a more moderate vision of the Democratic Party,
Ms. Harris onstage at the debate reflected on a time in Berkeley history
when moderate politics were falling apart. Though the city was a
microcosm of seismic political changes in the country as a whole,
including antiwar sentiment and free speech activism, one major catalyst
for transformation was school busing. As white families who opposed
busing left town for the suburbs in 1967, they gave way to an insurgent
new left.

``Busing was really when Berkeley split and became leftist because a lot
of people who couldn't handle that change, they left,'' said Jef
Findley, a librarian at the Berkeley Public Library specializing in city
history, who helped make
\href{https://www.youtube.com/results?q=berkeley+public+desegregation}{an
oral history of the city's busing and desegregation}. ``The moderate,
pro-business rightist town became a leftist town.''

\includegraphics{https://static01.graylady3jvrrxbe.onion/images/2019/07/01/reader-center/01BERKELEYpageone/01BERKELEYpageone-articleLarge.jpg?quality=75\&auto=webp\&disable=upscale}

Carole Porter, 55, who now works in information technology, lived around
the corner from Ms. Harris. The two girls took the bus up the hill from
the middle-class Berkeley flats where they lived to Thousand Oaks, a
school in the white, more affluent Berkeley hills. Ms. Porter recalled
that the ride took about 40 minutes.

Ms. Harris attended a Montessori school for kindergarten and joined Ms.
Porter at Thousand Oaks in first grade. During the debate, she described
herself as ``part of the second class to integrate her public schools,''
and her classmates were the second group of kindergartners to be bused
to schools outside their neighborhoods. A campaign spokeswoman confirmed
that Ms. Harris joined the class in 1970, her first grade year, which
was the third year of integration. The school had been 2.5 percent black
in 1963. In 1969, it was 40.2 percent black as a result of integration.

\hypertarget{latest-updates-2020-election}{%
\section{\texorpdfstring{\href{https://www.nytimes3xbfgragh.onion/live/2020/08/19/us/dnc-convention-election?action=click\&pgtype=Article\&state=default\&region=MAIN_CONTENT_1\&context=storylines_live_updates}{Latest
Updates: 2020
Election}}{Latest Updates: 2020 Election}}\label{latest-updates-2020-election}}

\href{https://www.nytimes3xbfgragh.onion/live/2020/08/19/us/dnc-convention-election?action=click\&pgtype=Article\&state=default\&region=MAIN_CONTENT_1\&context=storylines_live_updates\#night-3-featured-more-policy-a-focus-on-women-and-a-full-throated-rejection-of-trump-by-his-predecessor}{7h
ago}

\href{https://www.nytimes3xbfgragh.onion/live/2020/08/19/us/dnc-convention-election?action=click\&pgtype=Article\&state=default\&region=MAIN_CONTENT_1\&context=storylines_live_updates\#night-3-featured-more-policy-a-focus-on-women-and-a-full-throated-rejection-of-trump-by-his-predecessor}{Night
3 featured more policy, a focus on women and a full-throated rejection
of Trump by his predecessor.}

\href{https://www.nytimes3xbfgragh.onion/live/2020/08/19/us/dnc-convention-election?action=click\&pgtype=Article\&state=default\&region=MAIN_CONTENT_1\&context=storylines_live_updates\#trump-live-tweeted-obamas-speech-tonight-hell-appear-on-fox-news-right-before-bidens-tomorrow}{9h
ago}

\href{https://www.nytimes3xbfgragh.onion/live/2020/08/19/us/dnc-convention-election?action=click\&pgtype=Article\&state=default\&region=MAIN_CONTENT_1\&context=storylines_live_updates\#trump-live-tweeted-obamas-speech-tonight-hell-appear-on-fox-news-right-before-bidens-tomorrow}{Trump
live-tweeted Obama's speech tonight. He'll appear on Fox News right
before Biden's tomorrow.}

\href{https://www.nytimes3xbfgragh.onion/live/2020/08/19/us/dnc-convention-election?action=click\&pgtype=Article\&state=default\&region=MAIN_CONTENT_1\&context=storylines_live_updates\#advocates-for-domestic-violence-survivors-praised-biden-in-a-video}{9h
ago}

\href{https://www.nytimes3xbfgragh.onion/live/2020/08/19/us/dnc-convention-election?action=click\&pgtype=Article\&state=default\&region=MAIN_CONTENT_1\&context=storylines_live_updates\#advocates-for-domestic-violence-survivors-praised-biden-in-a-video}{Advocates
for domestic violence survivors praised Biden in a video.}

\href{https://www.nytimes3xbfgragh.onion/live/2020/08/19/us/dnc-convention-election?action=click\&pgtype=Article\&state=default\&region=MAIN_CONTENT_1\&context=storylines_live_updates}{See
more updates}

Ms. Porter, who has donated to Ms. Harris's campaign, remembered her
former classmate as a responsible student who was being raised by a
single mother.

They played games like ``Miss Mary Mack'' and cat's cradle and discussed
recess plans, Ms. Porter said. She remembered the neighborhood the two
girls shared as a lower-middle class area full of working people.

``It was very diverse --- we had a United Nations in our neighborhood,''
Ms. Porter, who is black, said. ``Busing gave us opportunities to leave
our neighborhood and see affluence.''

Most Americans at the time were in favor of integration, but
\href{https://www.nytimes3xbfgragh.onion/1973/09/09/archives/gallup-finds-few-favor-busing-for-integration.html}{few
thought busing was the best method}, according to a 1973 Gallup poll.
Given alternatives like low-income housing in middle-income areas or
changed school boundaries, only 9 percent of blacks said they preferred
busing, and just 5 percent of whites did.

Approval of busing largely broke along racial lines, though opposition
was often framed around mandatory busing, or rules set at the national
level, rather than on integration itself. Mr. Biden expressed a version
of that
\href{https://www.nytimes3xbfgragh.onion/2019/06/28/us/politics/joe-biden-busing-kamala-harris.html}{view}
at the debate last week.

``I did not oppose busing in America,'' Mr. Biden responded to Ms.
Harris. ``What I opposed is busing ordered by the Department of
Education. That's what I opposed.''

\emph{{[}}\href{https://www.nytimes3xbfgragh.onion/2019/06/28/us/politics/joe-biden-busing-kamala-harris.html}{\emph{What
we know about Joe Biden and busing.}}\emph{{]}}

Across the country in the 1970s, courts were ordering schools to
desegregate. In Boston, thousands of whites marched carrying American
and Irish flags
\href{https://www.nytimes3xbfgragh.onion/1975/10/28/archives/boston-whites-march-in-busing-protest.html}{to
protest court-ordered desegregation}. In Los Angeles voters
\href{https://www.latimes.com/local/lanow/la-me-busing-schools-los-angeles-harris-biden-20190628-story.html}{recalled
the school board president}.

Berkeley was different. The opposition to integration in Berkeley was
quieter. An
\href{https://www.berkeleyschools.net/2018/12/50th-anniversary-of-berkeleys-pioneering-busing-plan-for-school-integration/}{attempt
to recall} the school board failed.

School segregation in Berkeley had never been enforced by law in the
first place. There were
\href{https://www.berkeleyside.com/2018/09/20/redlining-the-history-of-berkeleys-segregated-neighborhoods}{neighborhoods
where black families could not buy houses}, a tactic known as redlining,
and each neighborhood had its own schools, leading to de facto
segregation.

Image

Ms. Harris confronted former Vice President Joseph R. Biden Jr. about
his history of opposing busing at the Democratic debate on
Thursday.Credit...Doug Mills/The New York Times

The city's flat western side, where Ms. Harris lived as a child, was
generally black, while the eastern hills were white.

Nor did the federal government have to step in to force integration:
Berkeley's comprehensive two-way busing program was undertaken by the
school board voluntarily, and it was the first large city to do so when
the program began in 1968.

After Ms. Harris spoke about her elementary school experience on
Thursday, conspiracy theorists quickly sprung on the story, arguing that
Berkeley's schools were never segregated because some black students
attended Ms. Harris's predominantly white school before busing began.
The district put out
\href{https://www.berkeleyschools.net/2019/06/the-history-of-integration-in-berkeley-elementary-schools-and-senator-harris/}{a
statement on Friday supporting Ms. Harris's account}.

``Our elementary schools,'' the district's statement said, ``reflected
the racial composition of our neighborhoods, which like many
neighborhoods across America reflected the history of segregation
stemming from policies which restricted the opportunities of nonwhite
residents.''

\emph{{[}Want to keep up with everything happening in California?}
\href{https://www.nytimes3xbfgragh.onion/newsletters/california-today}{\emph{Sign
up to get California Today}} \emph{delivered to your inbox.{]}}

The late 1960s and early 1970s in Berkeley was a time flush with
idealism, and advocates envisioned school integration ushering in a new,
better society.

``We will not fail here,'' the superintendent wrote in his integration
report. ``Where else would there be hope of success if there were
failure here, in Berkeley?''

Harold Williams, 61, was at a predominantly black school when white
students started being bused in, changing the demographics from 26
percent white to 47 percent white.

``It had a very positive impact on me,'' said Mr. Williams, who is black
and now lives in Charlotte, N.C. ``For Kamala, or for anyone who went
through it, I think it puts them in a position to have a lot of
empathy.''

Other students felt they were sometimes fending for themselves in a
complicated new dynamic.

``We were all just kids trying to figure it out on our own,'' said Ned
Garrett, 60, who is white and was bused to a black school in West
Berkeley. ``It stuck with me. I try to tell my kids now, `Don't make
assumptions, always give people a chance to show who they are.'''

Tucked away in the Berkeley Public Library are the city's archives of
the era. There in big folders are stacks of newspaper clippings. Those
who opposed busing said they were worried about bus safety, travel
times, neighborhood integrity and school district costs.

Image

Ms. Harris has argued that school integration is one of the reasons she
was able to become a senator.Credit...Erin Schaff for The New York Times

``To force total mixing in schools is neither constructive nor
effective; individual families should have an option as to where their
children go to school --- and this applies to all races,'' a group of
Berkeley residents wrote in an open letter to the superintendent on
Sept. 19, 1967.

``Berkeley was going too hard, and I don't think they got it right a lot
of the time,'' said Mike Davisson, 61, who is white and was in fifth
grade when he began being bused to a different elementary school. ``It
was scary and hard for the black kids as well as the white kids, but
overall, in the end, it worked out.''

Along with integrating the students, the school district embarked on an
effort to hire more black teachers.

Lynn Sherrell, 78, taught sixth grade the first year of integration.

``Among the black and white teachers there tended to be a division, and
there was lunchroom segregation,'' Ms. Sherrell, who is white, said.
``It got ugly.''

Black faculty members were meeting separately, she said, and white
teachers felt left out.

But she and her friends believed a better, less racially tense future
was ahead, Ms. Sherrell said. Sometimes now she is less sure.

``It was one of the primary elements of my life and my career,'' said
Ms. Sherrell, who later became a tax lawyer. ``We're so much wondering
now --- did it do anything? Did it do any good? Was it just a big
waste?''

Berkeley voted to phase out its original busing program in 1994, but
still has integration initiatives in place. Busing has largely been seen
\href{https://www.nytimes3xbfgragh.onion/1975/12/21/archives/busing-the-solution-that-has-failed-to-solve.html}{as
a failed effort}: Across the country today, schools are still
segregated, and
\href{https://www.nytimes3xbfgragh.onion/2019/05/10/us/threatening-the-future-the-high-stakes-of-deepening-school-segregation.html?action=click\&module=News\&pgtype=Homepage}{the
number of intensely segregated schools is growing}.

Ms. Harris has argued that school integration is one of the reasons she
was able to become a senator. During the Supreme Court confirmation
hearings for Brett M. Kavanaugh, during
which\href{https://www.nytimes3xbfgragh.onion/2018/09/06/us/politics/kavanaugh-hearings-kamala-harris-cory-booker.html}{Ms.
Harris grilled the future justice}, she also wrote about her own life.

``I wouldn't be part of Kavanaugh's confirmation hearings had Chief
Justice Warren not been on the Supreme Court to lead the unanimous
decision in Brown v. Board,''
\href{https://twitter.com/kamalaharris/status/1038952698438197250?lang=en}{Ms.
Harris posted on Twitter in September 2018}. ``Had someone else been
there, I may not have become a U.S. Senator. I know the impact one
Justice can have.''

Now, Berkeley is a wealthier and whiter town than when Ms. Harris was
growing up. In 1970, the city was
\href{http://www.bayareacensus.ca.gov/cities/Berkeley70.htm}{23 percent
black}; today it is
\href{http://www.bayareacensus.ca.gov/cities/Berkeley.htm}{only 10
percent black}. The median sale price for a two-bedroom apartment is
\$1.2 million, up from \$682,500 just five years ago, according to the
real estate listings site Trulia. Many of the historically black
neighborhoods have been gentrified.

Ms. Alkebulan, who was bused to an integrated elementary school, went on
to become a civil engineer, but it was housing prices that drove her out
of Berkeley. She moved to Sacramento.

``I can live in Sacramento a lot better than I can in Berkeley, so we
moved,'' Ms. Alkebulan said. ``And then we could afford to send our sons
to private school.''

\hypertarget{our-2020-election-guide}{%
\section{Our 2020 Election Guide}\label{our-2020-election-guide}}

Updated Aug. 20, 2020

\begin{itemize}
\item
  \begin{center}\rule{0.5\linewidth}{\linethickness}\end{center}

  \hypertarget{convention-recap}{%
  \subsection{Convention Recap}\label{convention-recap}}

  \begin{itemize}
  \tightlist
  \item
    Joe Biden accepted the Democratic nomination, urging Americans to
    have faith that they could
    \href{https://www.nytimes3xbfgragh.onion/2020/08/20/us/politics/Joe-Biden-accepts-democratic-nomination.html?action=click\&pgtype=Article\&state=default\&region=BELOW_MAIN_CONTENT\&context=storylines_guide}{``overcome
    this season of darkness.''}
  \end{itemize}
\item
  \begin{center}\rule{0.5\linewidth}{\linethickness}\end{center}

  \hypertarget{news-analysis}{%
  \subsection{News Analysis}\label{news-analysis}}

  \begin{itemize}
  \tightlist
  \item
    Looming over Mr. Biden's nomination was the ever-present shadow of
    another man who's poised to dominate the campaign:
    \href{https://www.nytimes3xbfgragh.onion/2020/08/20/us/politics/biden-dnc-speech-trump.html?action=click\&pgtype=Article\&state=default\&region=BELOW_MAIN_CONTENT\&context=storylines_guide}{Donald
    J. Trump}.
  \end{itemize}
\item
  \begin{center}\rule{0.5\linewidth}{\linethickness}\end{center}

  \hypertarget{keep-up-with-our-coverage}{%
  \subsection{Keep Up With Our
  Coverage}\label{keep-up-with-our-coverage}}

  \begin{itemize}
  \tightlist
  \item
    Get an
    \href{https://www.nytimes3xbfgragh.onion/newsletters/politics?action=click\&pgtype=Article\&state=default\&region=BELOW_MAIN_CONTENT\&context=storylines_guide}{email}
    recapping the day's news
  \end{itemize}

  \begin{itemize}
  \tightlist
  \item
    Download our mobile app on
    \href{https://apps.apple.com/us/app/nytimes/id284862083?ls=1\&mat_click_id=5c79ae7455014fd1bd66b5610c05b8f2-20191112-16948\&referrer=mat_click_id\%3D5c79ae7455014fd1bd66b5610c05b8f2-20191112-16948\%26link_click_id\%3D722930677036718082}{iOS}
    and
    \href{http://a.localytics.com/android?id=com.nytimes.android\&referrer=utm_source\%3Dother_nyt_mobile_web\%26utm_medium\%3DWeb\%2520page\%26utm_term\%3DGeneral\%2520Mobile\%2520Page\%26utm_campaign\%3DNYT\%2520Mobile\%2520General\%2520Page}{Android}
    and turn on Breaking News and Politics alerts
  \end{itemize}
\end{itemize}

Advertisement

\protect\hyperlink{after-bottom}{Continue reading the main story}

\hypertarget{site-index}{%
\subsection{Site Index}\label{site-index}}

\hypertarget{site-information-navigation}{%
\subsection{Site Information
Navigation}\label{site-information-navigation}}

\begin{itemize}
\tightlist
\item
  \href{https://help.nytimes3xbfgragh.onion/hc/en-us/articles/115014792127-Copyright-notice}{©~2020~The
  New York Times Company}
\end{itemize}

\begin{itemize}
\tightlist
\item
  \href{https://www.nytco.com/}{NYTCo}
\item
  \href{https://help.nytimes3xbfgragh.onion/hc/en-us/articles/115015385887-Contact-Us}{Contact
  Us}
\item
  \href{https://www.nytco.com/careers/}{Work with us}
\item
  \href{https://nytmediakit.com/}{Advertise}
\item
  \href{http://www.tbrandstudio.com/}{T Brand Studio}
\item
  \href{https://www.nytimes3xbfgragh.onion/privacy/cookie-policy\#how-do-i-manage-trackers}{Your
  Ad Choices}
\item
  \href{https://www.nytimes3xbfgragh.onion/privacy}{Privacy}
\item
  \href{https://help.nytimes3xbfgragh.onion/hc/en-us/articles/115014893428-Terms-of-service}{Terms
  of Service}
\item
  \href{https://help.nytimes3xbfgragh.onion/hc/en-us/articles/115014893968-Terms-of-sale}{Terms
  of Sale}
\item
  \href{https://spiderbites.nytimes3xbfgragh.onion}{Site Map}
\item
  \href{https://help.nytimes3xbfgragh.onion/hc/en-us}{Help}
\item
  \href{https://www.nytimes3xbfgragh.onion/subscription?campaignId=37WXW}{Subscriptions}
\end{itemize}
