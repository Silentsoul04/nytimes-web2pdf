Sections

SEARCH

\protect\hyperlink{site-content}{Skip to
content}\protect\hyperlink{site-index}{Skip to site index}

\href{/section/business/energy-environment}{Energy \&
Environment}\textbar{}Florida's Utilities Keep Homeowners From Making
the Most of Solar Power

\url{https://nyti.ms/2xBRq1K}

\begin{itemize}
\item
\item
\item
\item
\item
\end{itemize}

\includegraphics{https://static01.graylady3jvrrxbe.onion/images/2019/06/20/business/00solar01/00solar01-articleLarge-v3.jpg?quality=75\&auto=webp\&disable=upscale}

\hypertarget{floridas-utilities-keep-homeowners-from-making-the-most-of-solar-power}{%
\section{Florida's Utilities Keep Homeowners From Making the Most of
Solar
Power}\label{floridas-utilities-keep-homeowners-from-making-the-most-of-solar-power}}

The political clout and incentives of the state's big power companies
have discouraged installation of rooftop solar panels.

When Timothy Nathan Shields wanted to install solar panels on his home
in Largo, Fla., Duke Energy pushed back. ``Every time I turned around,
they would drag their feet,'' he said.Credit...Zack Wittman for The New
York Times

Supported by

\protect\hyperlink{after-sponsor}{Continue reading the main story}

By \href{https://www.nytimes3xbfgragh.onion/by/ivan-penn}{Ivan Penn}

\begin{itemize}
\item
  July 7, 2019
\item
  \begin{itemize}
  \item
  \item
  \item
  \item
  \item
  \end{itemize}
\end{itemize}

ST. PETERSBURG, Fla. --- Florida calls itself the Sunshine State. But
when it comes to the use of solar power, it trails 19 states, including
not-so-sunny Massachusetts, New Jersey, New York and Maryland.

Solar experts and environmentalists blame the state's utilities.

The utilities have hindered potential rivals seeking to offer
residential solar power. They have spent tens of millions of dollars on
lobbying, ad campaigns and political contributions. And when homeowners
purchase solar equipment, the utilities have delayed connecting the
systems for months.

Solar energy is widely considered an essential part of addressing
climate change by weaning the electric grid from fossil fuels.
California, a clean energy trendsetter, last year became the
\href{https://www.nytimes3xbfgragh.onion/2018/05/09/business/energy-environment/california-solar-power.html}{first
state to require} solar power for all new homes.

But many utilities across the country have fought homeowners' efforts to
install solar panels. The industry's trade organization, the Edison
Electric Institute,
\href{http://roedel.faculty.asu.edu/PVGdocs/EEI-2013-report.pdf}{has
warned} that the technology threatens the foundation of the power
companies' business.

In Florida, utilities make money on virtually all aspects of the
electricity system --- producing the power, transmitting it, selling it
and delivering it. And critics say the companies have much at stake in
preserving that control.

``I've had electric utility executives say with a straight face that we
can't have solar power in Florida because we have so many cloudy days,''
said Representative Kathy Castor, a Democrat from the Tampa area. ``I
have watched as other states have surpassed us. I think that is largely
because of the political influence of the investor-owned utilities.''

\includegraphics{https://static01.graylady3jvrrxbe.onion/images/2019/06/29/business/29solar1/merlin_154785099_2eb30592-4d6e-41d9-9975-cb882b1b67f9-articleLarge.jpg?quality=75\&auto=webp\&disable=upscale}

The state's utilities have been expanding their own production of solar
power. But Florida is one of eight states that prohibit the sale of
solar electricity directly to consumers unless the provider is a
utility. There is also a state rule, enforced by the utilities,
requiring expensive insurance policies for big solar arrays on houses.

In 2009, a measure to require a certain amount of energy to be generated
from renewable sources passed the State Senate but died in the House of
Representatives when the utilities fought it. Solar proponents have been
unable to find legislative traction for similar measures since then.

Mayor Rick Kriseman of St. Petersburg --- the site of Duke Energy's
Florida headquarters --- has argued for changing the way utilities are
regulated so they would embrace more energy efficiency, residential
solar power and energy storage. The companies essentially see the
solar-equipped homeowner as a competitor, not a customer, he said.

``If your profits are based on consumption, where's your incentive to
reduce electricity use?'' Mr. Kriseman said.

\hypertarget{a-homeowners-struggle}{%
\subsection{A Homeowner's Struggle}\label{a-homeowners-struggle}}

Art Graham, chairman of the Florida Public Service Commission, which
regulates Duke, Florida Power \& Light and other investor-owned
utilities, said simple economics was one reason the state had lagged in
adopting renewable energy sources. Because Florida has kept electricity
rates lower than those in the Northeast and California, he said, the
cost savings for homeowners in switching to solar power are more
limited.

But there are other obstacles. Timothy Nathan Shields is still stunned
by the resistance he faced from Duke, the state's second-largest
utility, when he wanted to put solar panels on his home.

\href{https://www.nytimes3xbfgragh.onion/interactive/2018/12/24/climate/how-electricity-generation-changed-in-your-state.html}{}

\includegraphics{https://static01.graylady3jvrrxbe.onion/images/2018/12/23/us/how-electricity-generation-changed-in-your-state-promo-1545597148124/how-electricity-generation-changed-in-your-state-promo-1545597148124-articleLarge.png}

\hypertarget{how-does-your-state-make-electricity}{%
\subsection{How Does Your State Make
Electricity?}\label{how-does-your-state-make-electricity}}

There's been a major shift in how America makes electricity over the
past two decades. Each state has its own story.

Mr. Shields, a 57-year-old retired nurse, wanted a system to cover the
electricity needs of his 2,000-square-foot house in Largo, north of St.
Petersburg, as well as the cost of charging his electric car. So a year
ago he bought a setup twice the size of the average rooftop system from
Sunrun, the leading residential solar company.

First, Mr. Shields said, a Duke representative told him that he would
not benefit much from solar power because ``it rains.'' Then the utility
told him that it wouldn't save him any money. After he made a commitment
to buy the system, Duke told him that it needed to be insured, citing
its size and saying it could ``harm the electric grid.''

So he bought a \$1 million insurance policy costing \$200 a year.

``It's absurd,'' said Brad Heavner, policy director for the California
Solar and Storage Association, a trade group. ``There's no way you can
justify that based on studies of the risk. I would call that an
outrageous solar requirement.'' He said he was not aware of such a rule
in other states.

Sunrun installed Mr. Shields's system in days. But Duke took two months
to turn it on, forcing him to continue to pay electric bills of as much
as \$310 a month. He will pay \$240 a month for the system for the next
six years, when it will be paid off, plus a monthly fee of \$11.57 to
Duke for a grid connection.

``Every time I turned around, they would drag their feet,'' Mr. Shields
said. ``They want you to think it's hard and horrible and difficult.''

Randy Wheeless, a Duke spokesman, said that he regretted Mr. Shields's
experience, but that the company was simply following state requirements
for larger home systems. The utility has been reducing connection times
and adding as many as 750 rooftop solar customers a month, he said.

Image

Duke told Mr. Shields that his solar power system needed to be insured,
saying it could ``harm the electric grid.'' So he bought a \$1 million
insurance policy costing \$200 a year.Credit...Zack Wittman for The New
York Times

From the state's perspective, Mr. Graham, the chief regulator, said, ``I
think we definitely could do some things differently'' --- like revising
the policy that will cost Mr. Shields as much as \$6,000 in insurance
premiums over the life of his system, potentially more than 30 years.

\hypertarget{political-dollars}{%
\subsection{Political Dollars}\label{political-dollars}}

The experience of homeowners like Mr. Shields has largely been shaped by
the utilities' political spending.

From 2014 through the end of May, Florida's four largest investor-owned
utilities together spent more than \$57 million on campaign
contributions, according to an analysis by Integrity Florida, a
nonprofit research organization, and the Energy and Policy Institute, a
watchdog group. FPL, the state's largest utility, accounted for \$31
million of that total.

The utilities also hired enough lobbyists to have one for every two
lawmakers in Tallahassee. From 2014 through 2017, the four companies
spent \$6 million on lobbying, Integrity Florida reported.

Sunrun broke through one of the barriers to rooftop solar last year when
it won approval to lease solar panels to homeowners, a step subsequently
taken by Vivint Solar and Tesla. But regulators stopped short of
allowing solar companies to own the panels and simply sell the power
directly to consumers, as they can in at least 27 states, the District
of Columbia and Puerto Rico.

``There's no solar competition happening,'' said Abigail Ross Hopper,
president of the Solar Energy Industries Association, a trade group.

Image

The solar farm for Babcock Ranch, whose developer calls it the nation's
first sustainable town.Credit...Zack Wittman for The New York Times

When it comes to the expansion of the utilities' own solar arrays,
Florida's growth rate led the nation in the first quarter, and the state
is positioned to hold that ranking for the next six years, according to
the energy consulting firm Wood Mackenzie and the Solar Energy
Industries Association.

Still, solar energy accounted for only 1 percent of electricity
generation in Florida last year, far less than the 19 percent in
California and nearly 11 percent in Vermont and Massachusetts, the
association said. The state relies largely on natural gas, and several
utilities get as much as a quarter of their power from coal.

A spokeswoman for Gov. Ron DeSantis defended the state's clean energy
efforts, saying in an email, ``Florida's renewable energy industry is
growing rapidly.''

But solar advocates, rather than the utilities, have been the primary
drivers for change at the consumer level.

An unlikely grass-roots coalition has emerged in Florida in the last
five years to promote solar power --- residential in particular --- as
environmentalists from the Southern Alliance for Clean Energy and the
Sierra Club joined with groups like the Tea Party and the Christian
Coalition.

While the groups' rationales for joining the effort varied from
environmental protection to a libertarian view of energy freedom, the
issue united them against the utilities, which backed a ballot measure
in 2016 to impose more fees on solar users and keep solar companies
other than utilities out of the state.

Image

Syd Kitson, the developer of Babcock Ranch, considers himself an
environmentalist. He convinced Florida Power \& Light that the project
could showcase the benefits of solar power.Credit...Zack Wittman for The
New York Times

Although the utilities spent more than \$20 million on the campaign, the
measure was defeated. And the next year, the grass-roots effort
persuaded lawmakers to exempt up to 80 percent of the value of solar
installations from property taxes. It seemed a great victory for
consumers --- but the utilities also benefited, because it eased their
tax burden on dozens or even hundreds of acres of solar farms.

``I would say that none of Florida's utilities are enthusiastic about
their customers' deploying solar,'' said Stephen Smith, executive
director of the Southern Alliance for Clean Energy. ``I am not surprised
at the horror stories.''

\hypertarget{a-vision-for-the-future}{%
\subsection{A Vision for the Future}\label{a-vision-for-the-future}}

FPL points to its role in a particular bet on a solar future: Babcock
Ranch, developed near Fort Myers by a company that extols it as the
nation's first sustainable town. The power company built a solar farm
that largely supplies the town's energy needs.

FPL announced four similarly sized projects in April, and Duke says it
is also building farms that size.

``FPL has been working for many years to advance solar energy while
keeping customer bills low,'' said Mark Bubriski, a company spokesman.
The utility said it plans to add enough solar capacity to power about
1.5 million homes and provide 20 percent of its total generation by
2030.

During legislative hearings in Tallahassee, Syd Kitson, the developer of
Babcock Ranch, which will include 20,000 homes when fully developed,
proposed building a town that could showcase the benefits of solar
power.

``I'm an environmentalist who is a developer,'' Mr. Kitson said. ``It is
the Sunshine State, so it made a lot of sense to us.''

But solar proponents feel the utilities need to be pushed further.

Scott McIntyre, chief executive of Solar Energy Management, a statewide
leader in commercial solar power based in St. Petersburg, said the gains
the state appeared to be making were little more than a facade.

``Florida is not going to do any type of energy policy that benefits
consumers, not for a long time,'' Mr. McIntyre said. ``They just keep
making the hurdles higher and higher.''

Advertisement

\protect\hyperlink{after-bottom}{Continue reading the main story}

\hypertarget{site-index}{%
\subsection{Site Index}\label{site-index}}

\hypertarget{site-information-navigation}{%
\subsection{Site Information
Navigation}\label{site-information-navigation}}

\begin{itemize}
\tightlist
\item
  \href{https://help.nytimes3xbfgragh.onion/hc/en-us/articles/115014792127-Copyright-notice}{©~2020~The
  New York Times Company}
\end{itemize}

\begin{itemize}
\tightlist
\item
  \href{https://www.nytco.com/}{NYTCo}
\item
  \href{https://help.nytimes3xbfgragh.onion/hc/en-us/articles/115015385887-Contact-Us}{Contact
  Us}
\item
  \href{https://www.nytco.com/careers/}{Work with us}
\item
  \href{https://nytmediakit.com/}{Advertise}
\item
  \href{http://www.tbrandstudio.com/}{T Brand Studio}
\item
  \href{https://www.nytimes3xbfgragh.onion/privacy/cookie-policy\#how-do-i-manage-trackers}{Your
  Ad Choices}
\item
  \href{https://www.nytimes3xbfgragh.onion/privacy}{Privacy}
\item
  \href{https://help.nytimes3xbfgragh.onion/hc/en-us/articles/115014893428-Terms-of-service}{Terms
  of Service}
\item
  \href{https://help.nytimes3xbfgragh.onion/hc/en-us/articles/115014893968-Terms-of-sale}{Terms
  of Sale}
\item
  \href{https://spiderbites.nytimes3xbfgragh.onion}{Site Map}
\item
  \href{https://help.nytimes3xbfgragh.onion/hc/en-us}{Help}
\item
  \href{https://www.nytimes3xbfgragh.onion/subscription?campaignId=37WXW}{Subscriptions}
\end{itemize}
