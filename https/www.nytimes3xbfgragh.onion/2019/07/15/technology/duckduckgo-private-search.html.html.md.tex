Sections

SEARCH

\protect\hyperlink{site-content}{Skip to
content}\protect\hyperlink{site-index}{Skip to site index}

\href{https://www.nytimes3xbfgragh.onion/section/business}{Business}

\href{https://myaccount.nytimes3xbfgragh.onion/auth/login?response_type=cookie\&client_id=vi}{}

\href{https://www.nytimes3xbfgragh.onion/section/todayspaper}{Today's
Paper}

\href{/section/business}{Business}\textbar{}A Feisty Google Adversary
Tests How Much People Care About Privacy

\href{https://nyti.ms/2YRONF7}{https://nyti.ms/2YRONF7}

\begin{itemize}
\item
\item
\item
\item
\item
\item
\end{itemize}

Advertisement

\protect\hyperlink{after-top}{Continue reading the main story}

Supported by

\protect\hyperlink{after-sponsor}{Continue reading the main story}

\hypertarget{a-feisty-google-adversary-tests-how-much-people-care-about-privacy}{%
\section{A Feisty Google Adversary Tests How Much People Care About
Privacy}\label{a-feisty-google-adversary-tests-how-much-people-care-about-privacy}}

\includegraphics{https://static01.graylady3jvrrxbe.onion/images/2019/07/08/business/00duckduckgo1/merlin_157633476_a2f29beb-0044-4404-8a95-549092b5fe61-articleLarge.jpg?quality=75\&auto=webp\&disable=upscale}

By
\href{https://www.nytimes3xbfgragh.onion/by/nathaniel-popper}{Nathaniel
Popper}

\begin{itemize}
\item
  July 15, 2019
\item
  \begin{itemize}
  \item
  \item
  \item
  \item
  \item
  \item
  \end{itemize}
\end{itemize}

\href{https://www.nytimes3xbfgragh.onion/es/2019/07/22/privacidad-google-duckduckgo/}{Leer
en español}

PAOLI, Pa. --- Gabriel Weinberg is taking aim at Google from a small
building 20 miles west of Philadelphia that looks like a fake castle. An
optometrist has an office downstairs.

Mr. Weinberg's company, DuckDuckGo, has become one of the feistiest
adversaries of Google. Started over a decade ago, DuckDuckGo offers a
privacy-focused alternative to Google's search engine.

The company's share of the search engine market is still tiny --- about
1 percent compared with Google's 85 percent, according to
\href{http://gs.statcounter.com/about}{StatCounter}. But it has tripled
over the past two years and is now handling around 40 million searches a
day. It has also made a profit in each of the last five years, Mr.
Weinberg said.

Mr. Weinberg, 40, is among the most outspoken critics of the internet
giants. DuckDuckGo's chief executive has repeatedly called for new
privacy-focused legislation and has warned at hearings and in
\href{https://www.nytimes3xbfgragh.onion/2019/06/19/opinion/facebook-google-privacy.html}{newspaper
opinion pieces} about the problems that big companies can cause by
tracking our every move online.

But the challenges faced by DuckDuckGo reflect just how difficult it is
to take on the giants and build an internet business that is focused on
the privacy of its users.

After a decade, the private company's modest success is an indication
that, even as regulators around the world consider tougher rules for the
data-tracking methods of big tech companies, selling consumers on
privacy-focused services is still an uphill battle.

Like other search companies, DuckDuckGo displays ads at the top of each
search page. But unlike others, it does not track the online behavior of
its users to personalize the ads.

DuckDuckGo constantly bumps into Google's business, which stretches far
beyond search. It also has to contend with the fact that most people
don't seem willing to give up much to recover their privacy, and are
easily overwhelmed when they decide to try to make a change.

``It's not as easy to switch as we'd like it to be,'' Mr. Weinberg said
while sitting in his office in jeans, red sneakers and a black
short-sleeve shirt. ``There is a lot of inertia drawing people back to
the existing system.''

These limitations are reflected in DuckDuckGo's modest offices.
DuckDuckGo, which has 65 employees, has done only two relatively small
fund-raising rounds, about \$13 million, that add up to less than what
Google makes in an hour. Mr. Weinberg parks his 2011 Honda minivan in a
parking lot you can see from his corner office, which is papered mostly
in drawings by his two sons.

\includegraphics{https://static01.graylady3jvrrxbe.onion/images/2019/07/08/business/00duckduckgo4/00duckduckgo4-articleLarge.jpg?quality=75\&auto=webp\&disable=upscale}

For people who care about privacy, DuckDuckGo is a reminder that it is
possible to offer internet access and build an online business without
logging every move made by users. It also provides a vision of what the
internet might look like if companies are forced to scale back the
surveillance economy.

``DuckDuckGo is very much a poster child for a future in which companies
stand with their users and still make money,'' said Gennie Gebhart, a
researcher at the Electronic Frontier Foundation, a nonprofit focused on
privacy and online rights. ``They counter the assumption that we've all
been socialized to accept: that it is normal to hand over all your
information. DuckDuckGo shows that doesn't have to be the way.''

Google has so far avoided any showdowns with DuckDuckGo. Last year,
Google added DuckDuckGo as one of the four default search engines
available for users of its internet browser, Chrome.

But the internet giant has not taken kindly to DuckDuckGo's suggestions
that it is selling information about its users to the highest bidder.

``The data we collect makes our product more helpful for people in a
variety of ways, such as improving our understanding of queries and
combating threats like spam and fraud,'' said Lara Levin, a Google
spokeswoman. ``We keep this data private and secure, and we provide
controls so people can make their own choices.''

Image

DuckDuckGo's office in Paoli, Pa., about 20 miles west of
Philadelphia.Credit...Michelle Gustafson for The New York Times

Mr. Weinberg started DuckDuckGo in 2008 when he was a stay-at-home dad,
after struggling to get two previous start-ups off the ground. The early
versions of DuckDuckGo (its name is a nod to the children's game) did
not have any focus on privacy. When he went in that direction, it was
attractive to only a small number of privacy advocates.

But after Edward J. Snowden, a former National Security Agency
contractor, revealed extensive online surveillance by the United States
government in 2013, privacy became a selling point. Business began to
grow.

DuckDuckGo is not the only upstart that is trying to capitalize on
privacy concerns. Proton Technologies, a Swiss start-up, is taking on
Gmail with an alternative email service. The Firefox and Brave browsers
are more focused on privacy than Chrome. And Google Maps users can
switch to OpenStreetMap.

{[}\href{https://www.nytimes3xbfgragh.onion/newsletters/privacy-project?action=click\&module=Intentional\&pgtype=Article}{\emph{If
you're online --- and, well, you are --- chances are someone is using
your information. We'll tell you what you can do about it. Sign up for
our limited-run newsletter.}}\emph{{]}}

DuckDuckGo's search site looks similar to Google, with the G replaced by
a playful cartoon head of a duck, who goes by the name Dax. Type in a
question and up pops a list of links that looks like what you'd get from
Google, with definitions from Wikipedia at the top and maps supplied
through a partnership with Apple Maps.

Mr. Weinberg said the site was designed to make it feel similar to
Google, to ease the transition for new users. While DuckDuckGo does not
keep data on its users, it can pull the geographic location from each
query and serve local results for things like restaurants and news.

``We don't think privacy is stopping getting good search results for
anybody,'' Mr. Weinberg said.

In
\href{https://www.reddit.com/r/linux/comments/7rwmgt/do_you_prefer_duckduckgo_over_google_search/}{online}
\href{https://www.quora.com/Is-DuckDuckGo-a-good-search-engine-1}{reviews}
at sites like Reddit and Quora, users praising DuckDuckGo's little
features --- like shortcuts for searching specific websites, known as
bangs --- are roughly equal in number to those who find its search
results inferior.

I made DuckDuckGo my default search engine while I was doing my
reporting for this article. I mostly didn't notice a difference, though
I did find the maps and local results to be less on target. And there
were times when DuckDuckGo couldn't figure out what I was looking for
when Google could, especially for more obscure queries.

Joseph Turow, a professor at the University of Pennsylvania who has
written extensively about privacy, said that whenever he had tried
DuckDuckGo, he had given up after a few days because of nagging doubts
about quality of the results.

Image

Credit...Michelle Gustafson for The New York Times

Image

Inside the office. DuckDuckGo also offers a free widget that blocks
trackers from Google and other companies while grading the privacy and
security of websites.Credit...Michelle Gustafson for The New York Times

``I'm almost embarrassed to say that I don't use it more than I do,''
Mr. Turow said. ``There is something in my head that tells me I'll get a
better search from Google, even when I don't know if that is
demonstrably correct or not.''

DuckDuckGo is also up against another hard reality of the online world:
If you call up DuckDuckGo on the Chrome browser, for example, Google is
still logging your search queries.

``It can feel a bit futile as an individual,'' said Liz Coll, the head
of digital change at Consumers International, a consumer advocacy group.
``It's quite hard to isolate your search engine as opposed to all the
other things you do online.''

DuckDuckGo is expanding beyond the search engine. The company's second
most popular product, a free widget that can be installed in any
browser, blocks invisible trackers from Google and other companies while
providing a privacy and security grade for every website.

The plug-in gives Google's main website a grade of D, because,
DuckDuckGo says, Google tracks you across the web and keeps permanent
logs of your activity, which either the
\href{https://www.nytimes3xbfgragh.onion/interactive/2019/04/13/us/google-location-tracking-police.html}{government}
or hackers might exploit.

Google has said it offers users many settings so they can opt out of
data collection. Most people don't, the company said, because the data
collection makes online services better and hasn't led to big problems
--- at least at Google.

Mr. Weinberg said the challenge of parsing what Google did and did not
do was an indication of why it was so important to build alternatives
that don't keep any data.

``All of these companies were saying you can't make money without
tracking your users,'' he said. ``By us existing and getting bigger and
being profitable, we serve as the existence proof that it is possible.''

Advertisement

\protect\hyperlink{after-bottom}{Continue reading the main story}

\hypertarget{site-index}{%
\subsection{Site Index}\label{site-index}}

\hypertarget{site-information-navigation}{%
\subsection{Site Information
Navigation}\label{site-information-navigation}}

\begin{itemize}
\tightlist
\item
  \href{https://help.nytimes3xbfgragh.onion/hc/en-us/articles/115014792127-Copyright-notice}{©~2020~The
  New York Times Company}
\end{itemize}

\begin{itemize}
\tightlist
\item
  \href{https://www.nytco.com/}{NYTCo}
\item
  \href{https://help.nytimes3xbfgragh.onion/hc/en-us/articles/115015385887-Contact-Us}{Contact
  Us}
\item
  \href{https://www.nytco.com/careers/}{Work with us}
\item
  \href{https://nytmediakit.com/}{Advertise}
\item
  \href{http://www.tbrandstudio.com/}{T Brand Studio}
\item
  \href{https://www.nytimes3xbfgragh.onion/privacy/cookie-policy\#how-do-i-manage-trackers}{Your
  Ad Choices}
\item
  \href{https://www.nytimes3xbfgragh.onion/privacy}{Privacy}
\item
  \href{https://help.nytimes3xbfgragh.onion/hc/en-us/articles/115014893428-Terms-of-service}{Terms
  of Service}
\item
  \href{https://help.nytimes3xbfgragh.onion/hc/en-us/articles/115014893968-Terms-of-sale}{Terms
  of Sale}
\item
  \href{https://spiderbites.nytimes3xbfgragh.onion}{Site Map}
\item
  \href{https://help.nytimes3xbfgragh.onion/hc/en-us}{Help}
\item
  \href{https://www.nytimes3xbfgragh.onion/subscription?campaignId=37WXW}{Subscriptions}
\end{itemize}
