Sections

SEARCH

\protect\hyperlink{site-content}{Skip to
content}\protect\hyperlink{site-index}{Skip to site index}

\href{https://myaccount.nytimes3xbfgragh.onion/auth/login?response_type=cookie\&client_id=vi}{}

\href{https://www.nytimes3xbfgragh.onion/section/todayspaper}{Today's
Paper}

When You Wear Sunscreen, You're Taking Part in a Safety Study

\url{https://nyti.ms/2KcgPVN}

\begin{itemize}
\item
\item
\item
\item
\item
\end{itemize}

Advertisement

\protect\hyperlink{after-top}{Continue reading the main story}

Supported by

\protect\hyperlink{after-sponsor}{Continue reading the main story}

\href{/column/studies-show}{Studies Show}

\hypertarget{when-you-wear-sunscreen-youre-taking-part-in-a-safety-study}{%
\section{When You Wear Sunscreen, You're Taking Part in a Safety
Study}\label{when-you-wear-sunscreen-youre-taking-part-in-a-safety-study}}

\includegraphics{https://static01.graylady3jvrrxbe.onion/images/2019/07/28/magazine/Studies_Show_01/Studies_Show_01-articleLarge-v2.jpg?quality=75\&auto=webp\&disable=upscale}

By Kim Tingley

\begin{itemize}
\item
  July 23, 2019
\item
  \begin{itemize}
  \item
  \item
  \item
  \item
  \item
  \end{itemize}
\end{itemize}

\href{https://www.nytimes3xbfgragh.onion/es/2019/08/08/espanol/estilos-de-vida/protector-solar.html}{Leer
en español}

The Food and Drug Administration almost never tests products itself. But
in May, the Journal of the American Medical Association published the
results of
\href{https://jamanetwork.com/journals/jama/article-abstract/2733085}{a
randomized trial}, conducted by F.D.A. researchers, to determine whether
the chemicals in four commercially available sunscreens are absorbed
through the skin into the bloodstream. Four times daily, subjects were
coated in one of the formulas in an amount determined to be the maximum
a person might use: two milligrams per square centimeter of skin over 75
percent of the body. Later, blood samples were drawn and analyzed. All
of the sunscreen chemicals were detected in concentrations that exceeded
an F.D.A. threshold past which manufacturers are required to do further
toxicology tests. ``People who use sunscreens very reasonably presume
they have been tested and are safe and effective,'' says Kanade Shinkai,
a dermatologist at the University of California, San Francisco, and an
author of
\href{https://jamanetwork.com/journals/jama/article-abstract/2733084?widget=personalizedcontent\&previousarticle=2733085}{an
editorial accompanying the JAMA study}. ``And we don't really have that
evidence.''

Legally, the U.S. regards sunscreen as a drug, meaning a substance
``intended for use in the diagnosis, cure, mitigation, treatment or
prevention of disease'' --- in this case, sunburn and skin cancer ---
and/or one that affects ``the structure or any function of the body.''
Until 1962, drugs could be sold in the U.S. without any data to support
claims of their efficacy. But that year, reports that a sedative called
thalidomide had caused severe birth defects in thousands of babies in
Western Europe led to
\href{https://www.fda.gov/consumers/consumer-updates/kefauver-harris-amendments-revolutionized-drug-development}{the
Kefauver-Harris Amendment to the Federal Food, Drug and Cosmetic Act},
which requires drug makers to satisfy the F.D.A. that their products are
safe and effective before they go on sale.

But more than 100,000 over-the-counter drug products were already on the
market, including sunscreens, each of which, under the new law, needed
review. To streamline the process, in 1972 the F.D.A. sorted them into
therapeutic categories (antacids, for example) as assigned
``monographs,'' which included lists of active ingredients. If publicly
available data demonstrated that these ingredients were generally safe
and effective, they could be used in current and future products under
conditions specified in the monograph without further review.

Almost 50 years later, a third of the monographs, including the one for
sunscreens, have not been finalized; hundreds of over-the-counter drugs
currently on sale have not yet been determined to be safe and effective.
(The agency blames the delay on an antiquated system and a lack of
funding.) Sunscreens are unique, however, in that the way we use them
has changed significantly. Decades ago, the sunscreens that Americans
were dabbing on their noses were often mineral concoctions, like zinc
oxide and titanium dioxide, that sat on the skin in a thick white cream
and physically blocked the sun's rays. But as awareness grew that
ultraviolet rays can cause skin cancer even without burning the skin,
public health experts began to advise that people wear sunscreen daily
on all exposed areas of the body. (Hats, long-sleeves and avoiding
prolonged sun exposure are also recommended.) This substantially
increased the usage of less messy, chemical sunscreens --- which contain
molecules, or ``filters,'' that can absorb a ``broad spectrum'' of
ultraviolet light --- including by pregnant women and children as young
as 6 months.

Initially, it was assumed these chemicals, like mineral sunscreens,
stayed on the surface of the skin. Then, in 1997,
\href{https://www.thelancet.com/journals/lancet/article/PIIS0140-6736(05)62032-6/fulltext}{a
study published in The Lancet} demonstrated that after subjects applied
sunscreen, the UV filter oxybenzone was present in their urine; in 2008,
\href{https://www.ncbi.nlm.nih.gov/pmc/articles/PMC2453157/}{a national
health survey by the Centers for Disease Control and Prevention} found
oxybenzone in 97 percent of urine samples;
\href{https://www.sciencedirect.com/science/article/pii/S004565351001132X}{a
2010 study of nursing mothers in a Swiss hospital} reported that 85
percent had UV filters in their breast milk. (Because UV filters are in
everything from makeup to shampoo to patio furniture, sunscreen was
probably not the only source.) This widespread use has raised
environmental concerns:
\href{https://www.nytimes3xbfgragh.onion/2019/02/07/us/sunscreen-coral-reef-key-west.html}{Hawaii
and Key West, Fla., have recently banned sunscreen ingredients},
including oxybenzone, that studies have suggested may be damaging coral
reefs. Yet despite sunscreen's extensive use over decades, there has
never been any indication that sunscreen chemicals are harmful to
humans.

\href{https://www.nytimes3xbfgragh.onion/2019/06/10/upshot/how-safe-is-sunscreen.html}{\emph{{[}Read
more about recent sunscreen research.{]}}}

\textbf{While drugs} --- like other products and health recommendations
--- can be tested for years in clinical trials of hundreds, or even
thousands, of people, this doesn't always predict how they will affect
millions of people after decades of use. In the general population,
dangerous side effects can remain invisible in the absence of large,
long-term studies. In 2002, some six million women were using
hormone-replacement drugs to relieve menopause symptoms --- one chemical
was substituted for another, innocuously, it seemed --- when a large
federal study showed that after five years, the drugs increased the risk
of breast cancer, heart attack and blood clots.

So, the lack of negative evidence alone doesn't prove that sunscreens
are safe. Animal studies have raised the possibility that some UV
filters, including oxybenzone, may disrupt the endocrine system, which
can adversely affect reproduction, development and immunity. Even if
those filters only slightly elevated the same risks in people, which no
evidence suggests they do, says Dr. Robert Califf, a former F.D.A.
Commissioner and professor at the Duke University School of Medicine and
the other author of the JAMA editorial, ``small differences in a big
population can be very important, and it would be very hard to see it.
People have trouble getting pregnant, men are sterile --- these things
happen every day, so you don't think back, Gee, it must be the
sunscreen. And it's probably not.''

\includegraphics{https://static01.graylady3jvrrxbe.onion/images/2019/07/28/magazine/28Studies_Show_02/28Studies_Show_02-articleLarge-v2.jpg?quality=75\&auto=webp\&disable=upscale}

But population-wide studies of sunscreens are especially difficult to
perform because so many variables are involved. In the U.S., 14 active
ingredients in the sunscreen monograph are available to be combined any
number of ways. People use many formulas, apply them in varying amounts
using different methods (e.g., sprays and lotions) and engage in a wide
range of activities while wearing them. This is partly why, Shinkai
says, basic research is lacking: ``We don't actually know what the
proper dose is to prevent skin cancer and whether that is different for
different agents or even different combinations of agents that are used
in sunscreen.'' (The American Academy of Dermatologists recommends one
ounce for most people.) More information could be lifesaving: For
instance, diagnoses of melanoma, the most lethal type of skin cancer,
are increasing. And though there is strong evidence that sunscreen use
prevents skin cancer, experts disagree whether the available data shows
that current sunscreen formulas and application methods protect against
melanoma specifically. There might be better practices we aren't aware
of.

There may also be more effective sunscreens unavailable to Americans. In
Europe, newer formulations that are broader-spectrum have been in use
for years, but the F.D.A. won't consider them without more data, leaving
the U.S. marketplace with 1970s-era ingredients that, except for mineral
varieties, the F.D.A. has not deemed safe and effective, either, but
that it also can't remove without eliminating a valuable form of
skin-cancer protection. ``The public needs to know, Should I continue to
use sunscreen?'' says Henry Lim, former president of the American
Academy of Dermatology and a dermatologist at Henry Ford Hospital in
Detroit. ``And that is becoming a challenge.''

\href{https://www.fda.gov/drugs/understanding-over-counter-medicines/sunscreen-how-help-protect-your-skin-sun}{The
F.D.A. is maintaining its official position that Americans should keep
using sunscreen}. But by showing that chemicals are being absorbed, the
agency has effectively forced manufacturers to provide additional data,
which they argue is unreasonably exhaustive, by November or risk having
their products pulled off U.S. shelves. (The F.D.A. will likely grant
them an extension.)

All of which raises the question of how deeply the unknown risks of a
medically beneficial drug can and should be explored. ``Science doesn't
hold still, and we keep learning things,'' Dr. Janet Woodcock, director
of the F.D.A. Center for Drug Evaluation and Research, points out. That
is, all of us are taking part in toxicology studies, whether we like it
--- or know it --- or not.

Advertisement

\protect\hyperlink{after-bottom}{Continue reading the main story}

\hypertarget{site-index}{%
\subsection{Site Index}\label{site-index}}

\hypertarget{site-information-navigation}{%
\subsection{Site Information
Navigation}\label{site-information-navigation}}

\begin{itemize}
\tightlist
\item
  \href{https://help.nytimes3xbfgragh.onion/hc/en-us/articles/115014792127-Copyright-notice}{©~2020~The
  New York Times Company}
\end{itemize}

\begin{itemize}
\tightlist
\item
  \href{https://www.nytco.com/}{NYTCo}
\item
  \href{https://help.nytimes3xbfgragh.onion/hc/en-us/articles/115015385887-Contact-Us}{Contact
  Us}
\item
  \href{https://www.nytco.com/careers/}{Work with us}
\item
  \href{https://nytmediakit.com/}{Advertise}
\item
  \href{http://www.tbrandstudio.com/}{T Brand Studio}
\item
  \href{https://www.nytimes3xbfgragh.onion/privacy/cookie-policy\#how-do-i-manage-trackers}{Your
  Ad Choices}
\item
  \href{https://www.nytimes3xbfgragh.onion/privacy}{Privacy}
\item
  \href{https://help.nytimes3xbfgragh.onion/hc/en-us/articles/115014893428-Terms-of-service}{Terms
  of Service}
\item
  \href{https://help.nytimes3xbfgragh.onion/hc/en-us/articles/115014893968-Terms-of-sale}{Terms
  of Sale}
\item
  \href{https://spiderbites.nytimes3xbfgragh.onion}{Site Map}
\item
  \href{https://help.nytimes3xbfgragh.onion/hc/en-us}{Help}
\item
  \href{https://www.nytimes3xbfgragh.onion/subscription?campaignId=37WXW}{Subscriptions}
\end{itemize}
