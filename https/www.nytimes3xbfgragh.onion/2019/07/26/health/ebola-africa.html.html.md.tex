Sections

SEARCH

\protect\hyperlink{site-content}{Skip to
content}\protect\hyperlink{site-index}{Skip to site index}

\href{https://www.nytimes3xbfgragh.onion/section/health}{Health}

\href{https://myaccount.nytimes3xbfgragh.onion/auth/login?response_type=cookie\&client_id=vi}{}

\href{https://www.nytimes3xbfgragh.onion/section/todayspaper}{Today's
Paper}

\href{/section/health}{Health}\textbar{}In Congo, a New Plan to Fight
Ebola Follows a Government Power Struggle

\url{https://nyti.ms/2SFstMD}

\begin{itemize}
\item
\item
\item
\item
\item
\item
\end{itemize}

Advertisement

\protect\hyperlink{after-top}{Continue reading the main story}

Supported by

\protect\hyperlink{after-sponsor}{Continue reading the main story}

Global health

\hypertarget{in-congo-a-new-plan-to-fight-ebola-follows-a-government-power-struggle}{%
\section{In Congo, a New Plan to Fight Ebola Follows a Government Power
Struggle}\label{in-congo-a-new-plan-to-fight-ebola-follows-a-government-power-struggle}}

After the resignation of the country's health minister, the president
will take over the response to the epidemic and distribute a new
vaccine.

\includegraphics{https://static01.graylady3jvrrxbe.onion/images/2019/07/26/science/26EBOLA1/26EBOLA1-articleLarge.jpg?quality=75\&auto=webp\&disable=upscale}

\href{https://www.nytimes3xbfgragh.onion/by/donald-g-mcneil-jr}{\includegraphics{https://static01.graylady3jvrrxbe.onion/images/2018/06/13/multimedia/author-donald-g-mcneil-jr/author-donald-g-mcneil-jr-thumbLarge-v4.png}}

By
\href{https://www.nytimes3xbfgragh.onion/by/donald-g-mcneil-jr}{Donald
G. McNeil Jr.}

\begin{itemize}
\item
  July 26, 2019
\item
  \begin{itemize}
  \item
  \item
  \item
  \item
  \item
  \item
  \end{itemize}
\end{itemize}

Faced with a lethal Ebola outbreak threatening eastern Africa, public
health officials are conceding that their battle plan is failing and
have proposed a comprehensive new strategy for containing the virus.

It envisions reframing the epidemic as a regional humanitarian crisis,
not simply a health emergency. That may include more troops or police to
quell the murders and arson that have made medical work difficult, as
well as food aid to win over skeptical locals.

The Democratic Republic of Congo also plans to deploy a second vaccine
to form a protective ``curtain'' of immunity around outbreak areas.

The outbreak, which began a year ago in Congo and
\href{https://www.nytimes3xbfgragh.onion/2019/07/17/health/ebola-outbreak.html}{was
declared a global health emergency this month}, is now the
second-biggest in history, with more than
\href{https://who.maps.arcgis.com/apps/opsdashboard/index.html\#/e70c3804f6044652bc37cce7d8fcef6c}{2,600
cases and more than 1,750 dead}. It has persisted in part because of a
fierce but hidden power struggle within Congo's government for control
of the response, according to documents obtained by The New York Times
and interviews with Ebola experts.

The country's health minister,
\href{https://twitter.com/olyilunga?lang=en}{Dr. Oly Ilunga}, resigned
on Monday after a public dispute with donors at a meeting in Geneva over
whether to roll out the second vaccine, which he opposed. The
containment effort will no longer be overseen by the health ministry but
by an expert committee reporting directly to Congo's new president,
Felix Tshisekedi.

\hypertarget{ebola-cases-by-week}{%
\subsection{Ebola Cases by Week}\label{ebola-cases-by-week}}

Reported cases in the Democratic Republic of Congo, as of July 21.

120

Confirmed cases

Probable cases

100

80

60

40

20

May

2018

July

Sept.

Nov.

Jan.

2019

March

May

July

2019

120

Confirmed cases

Probable cases

100

80

60

40

20

July

2018

Sept.

Nov.

Jan.

2019

March

May

July

2019

By The New York Times \textbar{} Source: World Health Organization

Dr. Ilunga was the target of a scathing internal government report
produced in April, just as new cases began soaring above 100 per week.
The report was written by a commission convened by Congo's new
president, some of whose members are now overseeing the response.

The report said ``arrogant'' national health officials took ``an
aggressive and ostentatious attitude'' when they visited the outbreak
area, renting deluxe hotel rooms and expensive cars and ``brandishing
large dollar bills'' while local health workers went unpaid.

A spokeswoman for Dr. Ilunga called the report ``weak.'' She said he had
resigned not because of it, but because the president had split the
authority to oversee the response between his office and an independent
commission, which she claimed was a violation of the Congolese
Constitution.

\includegraphics{https://static01.graylady3jvrrxbe.onion/images/2019/07/26/science/26EBOLA2/merlin_158308296_3c51baad-3d16-49f4-806d-39fddfe2f7d9-articleLarge.jpg?quality=75\&auto=webp\&disable=upscale}

Image

Health workers attend to an Ebola patient in a plastic isolation cube in
Beni.~Credit...Jerome Delay/Associated Press

Dr. Ilunga's departure pleased some donors and agencies supporting the
fight against Ebola. The United States is by far the biggest donor.
Tibor P. Nagy, the State Department's top official for African affairs,
told a Senate subcommittee on Wednesday that Dr. Ilunga's resignation
``may be an improvement to the situation.''

The country is seeking \$288 million to implement its new Ebola
strategy, and is likely to get it. The World Bank recently offered \$300
million. The United States increased its previous giving by \$38 million
this week, and federal aid officials have said they are committed to
containing the outbreak at its source.

\textbf{\emph{{[}}\href{http://on.fb.me/1paTQ1h}{\emph{Like the Science
Times page on Facebook.}}} ****** \emph{\textbar{} Sign up for the}
\textbf{\href{http://nyti.ms/1MbHaRU}{\emph{Science Times
newsletter.}}\emph{{]}}}

The new plan may include a campaign to win the hearts of the traumatized
population in the isolated eastern provinces by immunizing them against
other diseases, treating children for parasites, handing out food and
even creating thousands of jobs. Experts hope efforts will be made to
negotiate a truce with local militias.

The health ministry's strategic plan for the period from July to
December, written in cooperation with the W.H.O., has not been
officially released but is circulating among the donors and health
agencies, and a copy was obtained by The Times. While it envisions a
much broader response, the plan is vague on specifics --- omitting even
references to which vaccines should be used.

By contrast, the commission's report in April endorsed the second
vaccine by name and called for many specific actions, like giving hot
meals to malnourished children. Because its main authors are now leading
the response, experts expect those steps will be taken.

In May, the United Nations made a similar shift. David Gressly, the
deputy
\href{https://www.who.int/news-room/detail/23-05-2019-united-nations-strengthens-ebola-response-in-democratic-republic-of-the-congo}{head
of the U.N. peacekeeping mission} in Congo, was put in charge of
coordinating U.N. humanitarian and political efforts so that the World
Health Organization could focus on the health response.

SUDAN

CHAD

BENIN

NIGERIA

ETHIOPIA

TOGO

SOUTH

SUDAN

Ebola R.

200 miles

UGANDA

Current

outbreak

KENYA

GABON

DEM. REP.

OF CONGO

CONGO

REP.

TANZANIA

MALAWI

ANGOLA

Forested areas

ZAMBIA

Areas at risk of

Ebola outbreaks

SUDAN

CHAD

SOUTH

SUDAN

Ebola R.

UGANDA

Current

outbreak

DEM. REP.

OF CONGO

ANGOLA

Forested areas

ZAMBIA

Areas at risk of

Ebola outbreaks

200 miles

By The New York Times \textbar{} Sources: World Health Organization;
David M. Pigott et al., eLife Sciences

Even though the W.H.O. and many
donors\href{https://www.nytimes3xbfgragh.onion/2019/05/08/health/ebola-vaccine-congo.html}{endorsed
the new vaccine, made by Johnson \& Johnson, in May}, Dr. Ilunga
vigorously opposed using it. He said it would confuse the populace and
be difficult to administer, since it requires two doses given 56 days
apart.

To avoid confusion, it will be deployed differently from the current
single-dose Merck vaccine. While that one is used to ``ring-vaccinate''
everyone around each known case, the new vaccine will be used in areas
further away to encircle the hot zones with immunized people.

For example, while the Merck vaccine has been given to Ugandan health
workers on the Congo border, the new one will be deployed in Mbarara, a
regional capital 60 miles away with a big hospital that ill patients
might travel to.

Image

Congo's president, Felix Tshisekedi, third from left, during a visit to
Angola earlier this month. He will oversee the new strategy to contain
the Ebola outbreak.Credit...Ampe Rogerio/EPA, via Shutterstock

Image

Administering an Ebola vaccine at Himbi Health Center in Goma this
month.~Credit...Olivia Acland/Reuters

Close to 200,000 doses of the Merck vaccine have been distributed. The
company has plans to produce 800,000, but some experts fear shortages,
especially if the virus escapes into South Sudan, which is as
dysfunctional and war-torn as eastern Congo.

Johnson \& Johnson has offered 500,000 doses; the vaccine is easier to
store
and\href{https://www.reuters.com/article/us-health-ebola-vaccine/deployment-of-second-ebola-vaccine-would-not-be-quick-fix-experts-warn-idUSKCN1UK1KR}{has
been tested for safety on 6,000 human volunteers}, but has not been
deployed in the field.

At its core, the political struggle within Congo pitted Dr. Ilunga, 59,
who had been a minister since 2016, against Dr. Jean-Jacques Muyembe,
director-general of the country's National Institute for Biomedical
Research.

It also was a struggle between President Tshisekedi and his predecessor,
Joseph Kabila, 48. After 18 contentious years in office, Mr. Kabila
stepped down last year and is now a senator for life. In December, Mr.
Tshisekedi won a disputed election, beating Mr. Kabila's chosen
successor. But since then, he has only slowly replaced Mr. Kabila's
cabinet ministers.

Dr. Ilunga, who visited the outbreak area several times, is respected by
some Ebola experts. The head of one international agency, speaking on
condition of anonymity to avoid involvement in another country's
dispute, called him ``principled and data-driven.''

Dr. Muyembe, 77, is an internationally respected authority on Ebola who
has helped fight every outbreak since the virus was discovered in 1976,
when the country was named Zaire.

The report by the commission led by Dr. Muyembe accused Dr. Ilunga of
``weak governance, weak leadership and a hyper-centralized response''
that failed to coordinate with other ministries, including the police
and army.

Image

Dr. Jean-Jacques Muyembe,~director general of Congo's National Institute
for Biomedical Research, has opposed Dr. Ilunga.Credit...Matthieu
Alexandre/Agence France-Presse --- Getty Images

Image

An Ebola treatment center at the Goma General Hospital in
Congo.Credit...Salym Fayad/EPA, via Shutterstock

In addition to accusing Dr. Ilunga and his staff of arrogance and
ostentation, the report also cataloged serious medical failures.

People with fevers who entered screening centers to see if they had
Ebola did not get test results for three to five days, by which time
they might become infected with Ebola --- or infect others if they had
it. Private clinics and traditional healers held patients in order to
make money and little was done to protect areas where the virus had not
yet appeared or to coordinate with neighboring countries, the report
said. And while the commission's field visits had gone ``generally
well,'' its work was hampered because Dr. Ilunga had refused repeated
requests to meet with members and his office had been uncooperative with
information requests.

\hypertarget{affected-areas}{%
\subsection{Affected Areas}\label{affected-areas}}

Reported cases by health zone, as of July 21.

LAKE

ALBERT

DEM. REP.

OF CONGO

197

UGANDA

Beni

369

526

Butembo

NUMBER OF CASES

258

Katwa

100

25

250

638

LAKE

EDWARD

Mbarara

20 mileS

DEM. REP.

OF CONGO

197

Beni

369

526

Butembo

258

Katwa

UGANDA

638

LAKE

EDWARD

20 mileS

By The New York Times \textbar{} Source: World Health Organization

Jessica Ilunga, a spokeswoman for the former health minister, derided
the report and said it ``had no data'' --- such as the names of health
workers who claimed they had gone unpaid.

She denied that national officials had spent too much money and scoffed
at the idea that outbreak cities like Butembo even had luxury hotels.
National officials had needed running water and internet connections to
do their jobs, she said, while some had even slept in tents.

Image

Congolese soldiers in Beni. The new Ebola strategy~may include more
troops or police to quell the violence in epidemic hot
zones.Credit...Jerome Delay/Associated Press

Image

Workers buried Mussa Kathembo, an Islamic scholar who had prayed over
those who were sick, in Beni on July 14.Credit...Jerome Delay/Associated
Press

The ministry did cooperate with neighboring countries, she said,
pointing to efforts with Ugandan officials when the virus briefly
crossed the border.

In her view, the commission had failed to contact the ministry, even
after being asked to explain what it was investigating. Also, she said,
its members had met separately with outside agencies and had tried to
countermand ministry orders to its front-line workers.

In addition, she said, since he took office in January, Mr. Tshisekedi
had refused to see Dr. Ilunga.

The tense atmosphere was fostered by rivalry between the current
president and the former one, said Dr. Peter Piot, director-general of
the London School of Hygiene and Tropical Medicine.

But, added Dr. Piot, who is a discoverer of the Ebola virus and serves
on W.H.O. advisory committees, there also appeared to be personal
antagonism between the two doctors.

In late spring, he said, when he and Dr. Muyembe went to Dr. Ilunga's
office to discuss the second vaccine. Dr. Ilunga --- ``appearing quite
autocratic,'' he said --- agreed to see him but refused to admit Dr.
Muyembe.

``I thought, `Oh, God. I didn't realize how bad the dynamics were.'''

\emph{Denise Grady contributed reporting.}

Advertisement

\protect\hyperlink{after-bottom}{Continue reading the main story}

\hypertarget{site-index}{%
\subsection{Site Index}\label{site-index}}

\hypertarget{site-information-navigation}{%
\subsection{Site Information
Navigation}\label{site-information-navigation}}

\begin{itemize}
\tightlist
\item
  \href{https://help.nytimes3xbfgragh.onion/hc/en-us/articles/115014792127-Copyright-notice}{©~2020~The
  New York Times Company}
\end{itemize}

\begin{itemize}
\tightlist
\item
  \href{https://www.nytco.com/}{NYTCo}
\item
  \href{https://help.nytimes3xbfgragh.onion/hc/en-us/articles/115015385887-Contact-Us}{Contact
  Us}
\item
  \href{https://www.nytco.com/careers/}{Work with us}
\item
  \href{https://nytmediakit.com/}{Advertise}
\item
  \href{http://www.tbrandstudio.com/}{T Brand Studio}
\item
  \href{https://www.nytimes3xbfgragh.onion/privacy/cookie-policy\#how-do-i-manage-trackers}{Your
  Ad Choices}
\item
  \href{https://www.nytimes3xbfgragh.onion/privacy}{Privacy}
\item
  \href{https://help.nytimes3xbfgragh.onion/hc/en-us/articles/115014893428-Terms-of-service}{Terms
  of Service}
\item
  \href{https://help.nytimes3xbfgragh.onion/hc/en-us/articles/115014893968-Terms-of-sale}{Terms
  of Sale}
\item
  \href{https://spiderbites.nytimes3xbfgragh.onion}{Site Map}
\item
  \href{https://help.nytimes3xbfgragh.onion/hc/en-us}{Help}
\item
  \href{https://www.nytimes3xbfgragh.onion/subscription?campaignId=37WXW}{Subscriptions}
\end{itemize}
