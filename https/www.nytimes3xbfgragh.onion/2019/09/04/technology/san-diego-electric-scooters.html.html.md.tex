Sections

SEARCH

\protect\hyperlink{site-content}{Skip to
content}\protect\hyperlink{site-index}{Skip to site index}

\href{https://www.nytimes3xbfgragh.onion/section/technology}{Technology}

\href{https://myaccount.nytimes3xbfgragh.onion/auth/login?response_type=cookie\&client_id=vi}{}

\href{https://www.nytimes3xbfgragh.onion/section/todayspaper}{Today's
Paper}

\href{/section/technology}{Technology}\textbar{}Welcome to San Diego.
Don't Mind the Scooters.

\url{https://nyti.ms/34nkr08}

\begin{itemize}
\item
\item
\item
\item
\item
\end{itemize}

Advertisement

\protect\hyperlink{after-top}{Continue reading the main story}

Supported by

\protect\hyperlink{after-sponsor}{Continue reading the main story}

\hypertarget{welcome-to-san-diego-dont-mind-the-scooters}{%
\section{Welcome to San Diego. Don't Mind the
Scooters.}\label{welcome-to-san-diego-dont-mind-the-scooters}}

A year ago, electric rental scooters were hailed as the next big thing
in transportation. But their troubles in San Diego show how the services
have now hit growing pains.

\includegraphics{https://static01.graylady3jvrrxbe.onion/images/2019/09/04/business/04scootermess8/00scootermess8-articleLarge.jpg?quality=75\&auto=webp\&disable=upscale}

\href{https://www.nytimes3xbfgragh.onion/by/erin-griffith}{\includegraphics{https://static01.graylady3jvrrxbe.onion/images/2019/06/18/reader-center/author-erin-griffith/author-erin-griffith-thumbLarge.png}}

By \href{https://www.nytimes3xbfgragh.onion/by/erin-griffith}{Erin
Griffith}

\begin{itemize}
\item
  Published Sept. 4, 2019Updated Sept. 16, 2019
\item
  \begin{itemize}
  \item
  \item
  \item
  \item
  \item
  \end{itemize}
\end{itemize}

SAN DIEGO --- The first thing you notice in San Diego's historic Gaslamp
Quarter is not the brick sidewalks, the rows of bars and the roving
gaggles of bachelorette parties and conferencegoers, or even the actual
gas lamps.

It's the electric rental scooters. Hundreds are scattered around the
sidewalks, clustered in newly painted corrals on the street and piled up
in the gutters. In early July, one corner alone had 37. In the area
around Mission Beach, one of the city's main beaches, a single side of
one block had 70. Most sat unused.

Since scooter rental companies like Bird, Lime, Razor, Lyft and
Uber-owned Jump moved into San Diego last year, inflating the city's
scooter population to as many as 40,000 by some estimates, the vehicles
have led to injuries, deaths, lawsuits
\href{https://fox5sandiego.com/2019/06/17/dozens-of-electric-scooters-bikes-found-defaced-or-damaged-in-ocean-beach/}{and
vandals}. Regulators and local activists have pushed back against them.
One company has even started collecting the vehicles to help keep the
sidewalks clear.

``My constituents hate them pretty universally,'' said Barbara Bry, a
San Diego City Council member. She called for a moratorium on the
scooters when they arrived, saying they clogged sidewalks and were a
danger to pedestrians.

San Diego's struggle to contain the havoc provides a glimpse of how
reality has set in for scooter companies like Bird and Lime. Last year,
\href{https://www.nytimes3xbfgragh.onion/2018/06/06/technology/how-i-learned-to-stop-worrying-and-love-electric-scooters.html}{the
services were hailed} as the next big thing in personal transportation.
\href{https://www.nytimes3xbfgragh.onion/2018/06/12/technology/bird-electric-scooter-investment.html}{Investors
poured money} into the firms, valuing Bird at \$2.3 billion and Lime at
\$2.4 billion and prompting an array of followers.

\includegraphics{https://static01.graylady3jvrrxbe.onion/images/2019/08/28/business/28scootermess2/28scootermess2-articleLarge.jpg?quality=75\&auto=webp\&disable=upscale}

Image

From left, Mariah Fortune, Matt Nesbitt, Alise Robers and Wesley Collins
in Mission Beach.Credit...Tara Pixley for The New York Times

The scooter companies distribute their electric vehicles around cities
and universities --- often on sidewalks --- and rent them by the minute
via apps. At the end of a rental period, a rider leaves the scooter for
the next customer to retrieve. Scooter speeds vary by company, model and
city, as do helmet laws, although helmets generally are not required.

But now, skepticism about scooter services is rising. Some cities,
including
\href{https://www.nytimes3xbfgragh.onion/2018/08/30/technology/san-francisco-scooter-permits.html}{San
Francisco}, Paris, Atlanta and
\href{https://www.nytimes3xbfgragh.onion/2019/01/15/technology/electric-scooters-portland-oregon.html}{Portland},
Ore., have imposed stricter regulations on scooter speed limits, parking
or nighttime riding. Columbia, S.C., has temporarily banned them. New
York recently
\href{https://www.nytimes3xbfgragh.onion/2019/06/19/nyregion/scooters-nyc.html}{passed
legislation} that would allow scooters to operate in some parts of New
York City, but not in Manhattan.

Safety has become a big issue. A
\href{http://www.austintexas.gov/sites/default/files/files/Health/Web_Dockless_Electric_Scooter-Related_Injury_Study_final_version_EDSU_5.14.19.pdf}{three-month
study} published in May from the Centers for Disease Control and
Prevention and the Public Health and Transportation Departments of
Austin, Tex., found that for every 100,000 scooter rides, 20 people were
injured. Nearly half of the injuries were to the head; 15 percent of
those showed evidence of traumatic brain injury.

Bird, Lime and Skip are trying to
\href{https://www.nytimes3xbfgragh.onion/2019/07/22/technology/bird-scooters-valuation.html}{secure}
new funding, according to three people familiar with the talks, who
declined to be identified because the discussions were not finished. In
May, Lime replaced its chief executive; several other top executives
also left. And in July, Bird's chief executive called a report about the
company's losses
``\href{https://twitter.com/travisv/status/1149762439593861120}{fake}.''

Scooters are ``a fun and convenient mode of transportation that really
does put people at risk and introduces significant spatial challenges to
the civic commons,'' said Adie Tomer, a metropolitan policy fellow at
the Brookings Institution. ``Those tensions are not going anywhere
anytime soon.''

Bird declined to comment.

Many scooter companies miscalculated how long the scooters would last
--- often not long enough for rental fees to cover their costs --- and
are struggling with profitability, acknowledged Sanjay Dastoor, Skip's
chief executive. His company has designed a way to produce more durable
scooters that can be repaired more easily and last long enough to turn a
profit, he said, allowing it to ``run a safe fleet that we are proud
of.''

Lindsey Haswell, Lime's head of communications, said new industries
often faced regulatory challenges, ``but our investors are willing to
take the long view.'' She added that the issues in San Diego did not
reflect the global scooter market. Lime has provided more than three
million trips in San Diego, she said, and has ``as many supporters as we
have detractors'' there.

Hans Tung, an investor at GGV, which has backed Lime, said he was
encouraged by the company's progress and was confident it would make its
scooters safe and profitable. ``I don't see how that couldn't be
achieved,'' he said.

\emph{{[}Read more on}
\href{https://www.nytimes3xbfgragh.onion/2019/09/10/reader-center/to-cover-scooter-disruption-take-the-ride.html}{\emph{Erin
Griffith's trip to San Diego}} \emph{to witness peak scooter-share for
herself.{]}}

Bird and Lime deployed their scooters in San Diego in February 2018,
followed by other companies. The start-ups pitched themselves as
environmentally friendly, a message that jibed with San Diego's goal to
reduce greenhouse emissions.

San Diego initially took a hands-off approach. The scooters became
popular, with an average of 30,000 riders per day, according to city
officials.

``Millennials and post-millennials want to live in a thriving, bustling
city that has dynamic choices for mobility,'' said Erik Caldwell, San
Diego's deputy head of operations for smart and sustainable communities.

But as more scooters flooded San Diego last summer, local business
owners and residents began objecting. Alex Stennet, a bouncer at Coyote
Ugly Saloon in the Gaslamp District, said people tripped over the
vehicles and threw them around. He said he had witnessed at least 20
scooter accidents in front of Coyote Ugly.

Image

ScootScoop has deals with 250 local businesses~to remove scooters; it
has towed more than 12,500. Credit...Tara Pixley for The New York Times

Image

Dan Borelli, left, and John Heinkel, co-owners of
ScootScoop.Credit...Tara Pixley for The New York Times

Dan Borelli, who owns a bike rental shop called Boardwalk Electric Rides
in Pacific Beach, said the scooters frequently blocked the entrance to
his store. In July 2018, he teamed up with John Heinkel, owner of a
local towing company, to haul away scooters that they deemed to be
parked on private property. They charge Bird, Lime and others a
retrieval fee of \$50 per scooter, plus \$2 for each day of storage.

Their company, ScootScoop, has essentially turned them into scooter
bounty hunters. They said they have struck deals with 250 local
businesses and hotels and have towed more than 12,500 scooters. Some
scooter companies have paid to get them back, they said.

In March, Lime and Bird sued Mr. Borelli and Mr. Heinkel for the scooter
removals. ScootScoop countersued Bird and Lime last week.

Other cities have called ScootScoop for advice, Mr. Borelli said. Mr.
Heinkel said the scooter companies underestimated them. ``They assumed
we were two hillbillies in a pickup truck, as opposed to business
owners,'' he said.

Lime's Ms. Haswell said Mr. Borelli and Mr. Heinkel ``are opportunistic
businessmen who troll the streets stealing scooters, with no respect for
the law, trying to make a profit at San Diego's expense.''

Late last year, the scooters turned from annoyances into hazards. In
December, a man in Chula Vista, a San Diego suburb, died after he was
hit by a car while riding a Bird scooter, according to the Chula Vista
Police Department. A tourist died a few months later after crashing his
rental scooter into a tree. Another visitor died of ``blunt force torso
trauma'' after his scooter collided with another, the San Diego Police
Department said.

The department said it counted 15 ``serious injury collisions''
involving scooters in the first half of this year. Last month, three
separate scooter-related skull fractures
\href{https://fox5sandiego.com/2019/08/06/third-person-fractures-skull-from-scooter-crash-this-week/}{happened
in one week}.

Image

On one day in July, there were 150 available Bird scooters within a
two-block radius in Mission Beach.Credit...Tara Pixley for The New York
Times

Image

Scooter parking corrals were introduced in July as part of San Diego's
new rules.Credit...Tara Pixley for The New York Times

As the injuries piled up, Safe Walkways, an activist group, amassed
hundreds of members in a Facebook group to oppose the scooters and file
complaints to government agencies. In April, around 50 protesters
gathered on Mission Beach's boardwalk with signs bearing messages like
``Safety Not Scooters'' and ``BoardWALK.''

Lawsuits have also piled up. Clients of Matthew Souther, an attorney at
Neil Dymott, filed a potential class action suit in March that accused
Bird, Lime and the City of San Diego of not complying with disability
rights laws to keep sidewalks clear. He said he was working on a dozen
other injury lawsuits against scooter companies.

San Diego has started cracking down on the scooters. In July, the city
enacted rules restricting where they could be parked and driven and
issued permits for 20,000 scooters, across all companies, to operate. In
three days that month, authorities impounded 2,500 scooters that
violated parking rules. San Diego later sent notices of violations to
Bird, Lyft, Lime and Skip.

Last month, San Diego told Lime that it planned to revoke its permit to
operate in the city because of the violations, pending a hearing.

Christina Chadwick, a spokeswoman for San Diego's mayor, Kevin
Faulconer, said the scooter operators had been warned that the city
would aggressively monitor them.

To deal with critics and improve safety and costs, the scooter companies
have upgraded their fleets with sturdier scooters. Bird has said its
Bird Zero model, which makes up a majority of its fleet, lasts an
average of 10 months, compared with three months for past models. Skip
recently announced a scooter with modular parts, which makes repairs
easier.

And after a year recalling scooters with cracked baseboards and
batteries that caught fire, Lime has introduced new vehicles with bigger
wheels and baseboards, as well as interchangeable batteries and parts.

Ms. Haswell said Lime was eager to show the progress it had made. ``We
admit that we haven't always gotten it right in San Diego,'' she said.

Advertisement

\protect\hyperlink{after-bottom}{Continue reading the main story}

\hypertarget{site-index}{%
\subsection{Site Index}\label{site-index}}

\hypertarget{site-information-navigation}{%
\subsection{Site Information
Navigation}\label{site-information-navigation}}

\begin{itemize}
\tightlist
\item
  \href{https://help.nytimes3xbfgragh.onion/hc/en-us/articles/115014792127-Copyright-notice}{©~2020~The
  New York Times Company}
\end{itemize}

\begin{itemize}
\tightlist
\item
  \href{https://www.nytco.com/}{NYTCo}
\item
  \href{https://help.nytimes3xbfgragh.onion/hc/en-us/articles/115015385887-Contact-Us}{Contact
  Us}
\item
  \href{https://www.nytco.com/careers/}{Work with us}
\item
  \href{https://nytmediakit.com/}{Advertise}
\item
  \href{http://www.tbrandstudio.com/}{T Brand Studio}
\item
  \href{https://www.nytimes3xbfgragh.onion/privacy/cookie-policy\#how-do-i-manage-trackers}{Your
  Ad Choices}
\item
  \href{https://www.nytimes3xbfgragh.onion/privacy}{Privacy}
\item
  \href{https://help.nytimes3xbfgragh.onion/hc/en-us/articles/115014893428-Terms-of-service}{Terms
  of Service}
\item
  \href{https://help.nytimes3xbfgragh.onion/hc/en-us/articles/115014893968-Terms-of-sale}{Terms
  of Sale}
\item
  \href{https://spiderbites.nytimes3xbfgragh.onion}{Site Map}
\item
  \href{https://help.nytimes3xbfgragh.onion/hc/en-us}{Help}
\item
  \href{https://www.nytimes3xbfgragh.onion/subscription?campaignId=37WXW}{Subscriptions}
\end{itemize}
