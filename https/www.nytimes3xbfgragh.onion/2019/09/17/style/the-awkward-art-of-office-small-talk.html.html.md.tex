Sections

SEARCH

\protect\hyperlink{site-content}{Skip to
content}\protect\hyperlink{site-index}{Skip to site index}

\href{/section/style}{Style}\textbar{}The Awkward but Essential Art of
Office Chitchat

\url{https://nyti.ms/2O9MV7Z}

\begin{itemize}
\item
\item
\item
\item
\item
\item
\end{itemize}

\includegraphics{https://static01.graylady3jvrrxbe.onion/images/2019/09/17/fashion/17office-smalltalk-1/17office-smalltalk-1-articleLarge.jpg?quality=75\&auto=webp\&disable=upscale}

The office: AN ANALYSIS

\hypertarget{the-awkward-but-essential-art-of-office-chitchat}{%
\section{The Awkward but Essential Art of Office
Chitchat}\label{the-awkward-but-essential-art-of-office-chitchat}}

We regret to inform you that you need to make small talk with your
co-workers. Here's how to master it.

Credit...Illustration by Shannon Lin/The New York Times

Supported by

\protect\hyperlink{after-sponsor}{Continue reading the main story}

By Lindsay Mannering

\begin{itemize}
\item
  Published Sept. 17, 2019Updated Sept. 22, 2019
\item
  \begin{itemize}
  \item
  \item
  \item
  \item
  \item
  \item
  \end{itemize}
\end{itemize}

Every day around the world,
\href{http://datatopics.worldbank.org/jobs/}{an estimated} three billion
people go to work and 2.9 billion of them avoid making small talk with
their co-workers once they get there.

Their avoidance strategies vary. Some will keep their headphones in and
their eyes low. Others will pantomime receiving an urgent message that
requires an immediate, brow-furrowing, life-or-death rapid response,
which incapacitates them from doing pretty much anything else, not
excluding riding in, or communally waiting for, an elevator in their
office building; making conversation while heating up lunch lasagna in
the office microwave; walking from the entrance of their office building
to the nearest public transit stop, or to literally anywhere, unless
wait, you're also going there? Because I actually meant to pop in this
fine Persian rug wholesaler. See you tomorrow!

If these strategies sound familiar, if you've convinced yourself that
avoiding small talk with co-workers is smart self-preservation, that the
risk of saying something ``dumb'' or offensive or coming across as
socially inept is not worth the reward of connecting with somebody (yes,
even if that connection is a shared concern about it raining), then bad
news: Your false logic could be costing you a promotion. Not to scare
you or anything.

\textbf{\href{https://www.nytimes3xbfgragh.onion/interactive/2019/09/17/style/the-office.html}{\emph{{[}Read
our full package, ``The Office: An In-Depth Analysis of Workplace User
Behavior.''{]}}}}

Jamie Terran, a \href{http://jamieterran.com/}{licensed career coach} in
New York City, said that small talk between colleagues and supervisors
builds rapport, which in turn builds trust. ``Rapport is the feeling
that allows you to extend a deadline, or overlook smaller mistakes,
because it makes it easy for you to remember we're only human. Right or
wrong, building rapport through interaction with colleagues could be the
thing that gets you the promotion or keeps you in the role you're in.''

Building rapport applies when you're interviewing, too. People hire
people they want to work with, not necessarily who's perfect for the
job. Engaging in small talk with your interviewer helps make a positive
impression.

But, how? Small talk, while small and just talk, is intimidating. This
is 2019 and we're all anxious about something, including a 15-second
chat with Janet from accounting about how freaking cold the A/C is in
the conference room. The good news is that you can just go ahead and
repurpose your anxiety about making small talk with your co-workers and
worry instead about \emph{not} making small talk with your co-workers.
See? Easy switch.

Because while small talk can be torture, the absence of it can also make
us feel bad about ourselves, like we're true failures at life for not
being able to connect with a fellow member of the herd, worried deep
down that we will be kicked out of society and left to rot alone on the
plains, to pay for our own streaming services instead of sharing a
login.

Here are a few thoughts on how to avoid that feeling.

\hypertarget{remember-youre-more-likable-than-you-think}{%
\subsection{Remember: You're More Likable Than You
Think}\label{remember-youre-more-likable-than-you-think}}

A
\href{https://www.nytimes3xbfgragh.onion/2018/09/23/smarter-living/how-to-be-more-likeable.html}{2018
study published in Psychological Science} showed that people
``systematically underestimated how much their conversation partners
liked them and enjoyed their company.''

Think about it: when you have an awkward small talk interaction with a
co-worker (it's stunted, there were silences, neither of you could think
of something to say) do you normally go back to your desk and think,
``Wow, Alex is a terrible conversationalist''? No. You go back to your
desk and think, ``Wow, I'm a rotten garbage human being who should be
shunned from society.'' And Alex is thinking the same thing about him or
herself.

Point is, you're more likable than you think you are, so try not to
judge yourself so harshly. According to Ellie Hearne, founder and C.E.O.
of the leadership communications agency
\href{https://www.pencilorink.com/}{Pencil or Ink}, which, among other
services, teaches companies and executives how to have better internal
communications, ``people don't remember what you say --- they remember
how they felt when they were with you.''

\hypertarget{a-little-planning-goes-a-long-way}{%
\subsection{A Little Planning Goes a Long
Way}\label{a-little-planning-goes-a-long-way}}

If you're generally anxious in social situations, i.e. human, Ms. Terran
suggested coming up with core questions or stories from which you can
pull.

``Whether or not you share personal information about yourself is up to
you, but discussing things you truly care about is always the best
strategy,'' she said. ``Topics relating to your professional field, for
example, an article you saw or book you read, is a great place to
start.''

Did something weird or interesting happen to you recently? Workshop (in
your mind, at least) that story ahead of time to unveil at your next
office outing.

And definitely remember to ask questions. We're all ultimately pretty
narcissistic at heart.

\hypertarget{advance-the-dreaded-how-are-you-loop}{%
\subsection{Advance the Dreaded ``How Are You?''
Loop}\label{advance-the-dreaded-how-are-you-loop}}

The ping-pong of ``How are you? Good, how are you?'' can feel like a
waste of time and energy, but be the change you wish to see in the world
and break the cycle. Go to your inner Rolodex of topics (see: planning
ahead) and move the short conversation forward by replying \emph{why}
you're ``good.'' As in, ``I'm good. I just started a book/podcast/TV
show and I'm really enjoying it. Have you heard of it?'' Or mention
something office-related, where there's a shared common experience:
``I'm good. They restocked the cold brew in the kitchen and it's so
strong. Have you tried it?''

\hypertarget{dont-panic-its-almost-over}{%
\subsection{Don't Panic, It's Almost
Over}\label{dont-panic-its-almost-over}}

Small talk doesn't last long. ``If you're a generally anxious person,
you have an out --- you're at work! You're not supposed to spend too
much time chatting. After a few moments you can reference a meeting or
project you are supposed to work on,'' Ms. Terran advised. A simple
exchange of pleasantries followed by a concise but polite exit (``Have a
good day!'') is perfectly acceptable.

\hypertarget{you-occasionally-have-the-right-to-remain-silent}{%
\subsection{You (Occasionally) Have the Right to Remain
Silent}\label{you-occasionally-have-the-right-to-remain-silent}}

If you're having a bad day and don't want to talk, that might be best
for everyone involved. Enter
\href{https://www.nytimes3xbfgragh.onion/2015/12/24/fashion/headphones-now-playing-nothing.html}{headphones}.
``It's fine to take a step back from engaging. Most people know the new
workplace etiquette, à la earbuds in means `give me some space,''' Ms.
Hearne said. A simple smile or nod to acknowledge your co-worker will
still go a long way.

I'll leave you with a warning: There are very few ways to have
successful small talk in the office bathroom. It should go without
saying that attempting to chat with someone while they're in the
bathroom stall is totally off-limits.

That said, one of the more memorable (in a good way) office chitchats
I've ever had happened at the bathroom sink. A co-worker who was clearly
excellent at storing away fun facts and sharing them appropriately told
me about the ``\href{http://shakeandfold.org/}{shake and fold}'' method
of using a paper towel to decrease waste.

I have used the method, and used it as a small talk device, ever since.

\begin{center}\rule{0.5\linewidth}{\linethickness}\end{center}

Lindsay Mannering was a founding team member of
\href{https://www.bustle.com/}{Bustle.com} and writes from Brooklyn, NY.

Advertisement

\protect\hyperlink{after-bottom}{Continue reading the main story}

\hypertarget{site-index}{%
\subsection{Site Index}\label{site-index}}

\hypertarget{site-information-navigation}{%
\subsection{Site Information
Navigation}\label{site-information-navigation}}

\begin{itemize}
\tightlist
\item
  \href{https://help.nytimes3xbfgragh.onion/hc/en-us/articles/115014792127-Copyright-notice}{©~2020~The
  New York Times Company}
\end{itemize}

\begin{itemize}
\tightlist
\item
  \href{https://www.nytco.com/}{NYTCo}
\item
  \href{https://help.nytimes3xbfgragh.onion/hc/en-us/articles/115015385887-Contact-Us}{Contact
  Us}
\item
  \href{https://www.nytco.com/careers/}{Work with us}
\item
  \href{https://nytmediakit.com/}{Advertise}
\item
  \href{http://www.tbrandstudio.com/}{T Brand Studio}
\item
  \href{https://www.nytimes3xbfgragh.onion/privacy/cookie-policy\#how-do-i-manage-trackers}{Your
  Ad Choices}
\item
  \href{https://www.nytimes3xbfgragh.onion/privacy}{Privacy}
\item
  \href{https://help.nytimes3xbfgragh.onion/hc/en-us/articles/115014893428-Terms-of-service}{Terms
  of Service}
\item
  \href{https://help.nytimes3xbfgragh.onion/hc/en-us/articles/115014893968-Terms-of-sale}{Terms
  of Sale}
\item
  \href{https://spiderbites.nytimes3xbfgragh.onion}{Site Map}
\item
  \href{https://help.nytimes3xbfgragh.onion/hc/en-us}{Help}
\item
  \href{https://www.nytimes3xbfgragh.onion/subscription?campaignId=37WXW}{Subscriptions}
\end{itemize}
