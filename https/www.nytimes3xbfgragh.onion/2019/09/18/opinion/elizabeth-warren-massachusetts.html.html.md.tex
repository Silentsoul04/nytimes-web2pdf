Sections

SEARCH

\protect\hyperlink{site-content}{Skip to
content}\protect\hyperlink{site-index}{Skip to site index}

\href{https://myaccount.nytimes3xbfgragh.onion/auth/login?response_type=cookie\&client_id=vi}{}

\href{https://www.nytimes3xbfgragh.onion/section/todayspaper}{Today's
Paper}

\href{/section/opinion}{Opinion}\textbar{}Does Elizabeth Warren Have a
Critical Vulnerability?

\url{https://nyti.ms/2QeWjde}

\begin{itemize}
\item
\item
\item
\item
\item
\item
\end{itemize}

Advertisement

\protect\hyperlink{after-top}{Continue reading the main story}

\href{/section/opinion}{Opinion}

Supported by

\protect\hyperlink{after-sponsor}{Continue reading the main story}

\hypertarget{does-elizabeth-warren-have-a-critical-vulnerability}{%
\section{Does Elizabeth Warren Have a Critical
Vulnerability?}\label{does-elizabeth-warren-have-a-critical-vulnerability}}

She has struggled with white, working-class voters like those important
to winning Pennsylvania, Michigan and Wisconsin.

By Paul Starobin

Mr. Starobin is a journalist based in Orleans, Mass.

\begin{itemize}
\item
  Sept. 18, 2019
\item
  \begin{itemize}
  \item
  \item
  \item
  \item
  \item
  \item
  \end{itemize}
\end{itemize}

\includegraphics{https://static01.graylady3jvrrxbe.onion/images/2019/09/18/opinion/18starobin/merlin_154797912_4e8bb8d9-57ea-41fb-8b8e-0c56de9acda4-articleLarge.jpg?quality=75\&auto=webp\&disable=upscale}

ORLEANS, Mass. --- I have voted two times for Elizabeth Warren to
represent Massachusetts in the Senate. I would certainly vote for her
for president over Donald Trump. But as the Democratic primary unfolds
and she extends a steady rise in the polls, I keep coming back to a
political vulnerability of which many followers of Massachusetts
politics are aware but others may not be.

The problem is that she has a relatively weak standing in Massachusetts
with non-college-educated working-class voters, and especially white
workers. These voters are critical, especially in the Midwest and in
states crucial to Mr. Trump's victory like Michigan, Pennsylvania and
Wisconsin.

You might call it the Warren Paradox. Her core message as a politician
--- that America has become rigged in favor of the very wealthy, and the
rich get richer as the rest of us get shafted --- is very much aimed at
the working class. What's more, her personal narrative, of her rise from
``the ragged edge of the middle class'' in her native Oklahoma, as
\href{https://twitter.com/ewarren/status/1133840956158042115?lang=en}{she
has put it}, to professional success and acclaim in the fields of
education and government might seem to embody a character trait of grit
that appeals to blue-collar workers.

Yet while all of the major Democratic presidential candidates face
difficulty with this constituency, polls suggest that this is especially
a problem for Ms. Warren. For example, in a Fox
\href{https://www.foxnews.com/politics/fox-news-poll-8-15}{survey}, she
drew 33 percent of white, non-college respondents in a matchup against
President Trump, versus 38 percent for Joe Biden and 37 percent for
Bernie Sanders. **** For Democrats to feel fully confident about
nominating Ms. Warren as their standard-bearer, she needs to figure out
this puzzle.

In Massachusetts, the Warren Paradox can be glimpsed in towns like
Rockland, population near 18,000, a suburb about 20 miles south of
Boston, overwhelmingly white and working class. In her November 2018
Senate race against a pro-Trump Republican, Ms. Warren won 60 percent of
the vote statewide but only
\href{https://www.wbur.org/news/2018/11/06/warren-beats-geoff-diehl}{44
percent} of the vote in Rockland. By contrast, northwest of Boston, in
the upscale suburb of Lexington, where the median home value is \$1.15
million, (compared with \$340,000 in Rockland), Ms. Warren took 74
percent of the vote.

On a recent visit to Rockland, I encountered a sentiment that her
policies to address economic hardships might actually penalize those who
have played by the rules. In a conversation in the parking lot of a
McDonald's, a young mother, after depositing her two children into the
back seat of her car, said she viewed as unfair Ms. Warren's proposal to
forgive college student loans for most people carrying such debt. Now a
manager at a local restaurant, she said she had attended a technical
institute after high school and duly paid off her loans. ``Probably,''
she told me, she would vote for Mr. Trump for delivering on his promise
to create more jobs.

I also came across what certainly sounded like, although it was not
overtly expressed, reluctance to embracing her because she is a woman.
``I can't even listen to her. I just shut it off'' --- the television
--- ``when she comes on,'' a man at Uptown's Finest Barbershop told me.

In part, Ms. Warren is afflicted by an authenticity problem with these
voters. A former Harvard law professor, she is viewed by some, whatever
her declared agenda, as typical of an elite that is out of touch with
the concerns of ordinary working people. Doubts about her genuineness
are nourished by her claim of Native American ancestry --- which her
detractors in Massachusetts have long framed as a dubious attempt to
elevate her career prospects over equally qualified white job
candidates. In 2012, Scott Brown, her Republican opponent in her first
Senate race, tried to use this issue against her.

These misgivings feed a conviction that she doesn't have Rockland's back
--- a belief common to white non-college voters, often held against the
Democratic Party in general. ``She'll tax me,'' insisted a 49-year-old
high school graduate who works at a town agency. (Ms. Warren's proposed
wealth tax targets only households with assets exceeding \$50 million.)

``She wants to have open borders,'' he added, voicing another reason
that some people in Rockland think a President Warren won't protect
them. (Like a number of the Democratic presidential candidates, Ms.
Warren is in favor of
\href{https://www.nytimes3xbfgragh.onion/2019/07/31/us/border-crossing-decriminalization.html}{decriminalizing
unauthorized border crossings}.) And the sense that Ms. Warren, who has
voted in Congress for a ban on assault weapons, is soft on gun rights
also plays into the notion that she would leave Rockland unprotected.
The 49-year-old voted for Mr. Trump in 2016 and fully plans to do so
again in 2020, with the ``good job'' the president is doing on the
economy. (After speaking to me, he climbed into a car with a National
Rifle Association sticker affixed to the back windshield.)

As Ms. Warren's Senate campaigns attest, she is by no means uniformly
unpopular in Rockland or, for that matter, in neighboring communities
with a similar socio-economic profile.

``I love Elizabeth Warren,'' said a welder and member of a plumbers
union local, age 68, by phone. ``She's my bulldog. She is 100 percent
for us --- for the working man, the exploited person, the underdog.'' He
is from Weymouth, next door to Rockland. ``If she were a man, they would
love her.'' He paused. ``Or they would like her more.''

As he explained, places like Rockland, on the South Shore of
Massachusetts, need to be understood as products of ``white flight''
from Boston, following court-ordered school busing in the mid-1970s.

Should she win the Democratic nomination, it's easy to see the
difficulties she will face in gaining the allegiance of the white
working class in a matchup with Mr. Trump. White flight also defines a
number of working-class suburbs in the Midwest, as in the metropolitan
Detroit region.

But even though the Warren Paradox will be a real challenge, she still
has the opportunity to impress potentially unreceptive voters with her
``bulldog'' tenacity, as in her visit this year to a small town in West
Virginia to talk about the opioid crisis --- a state, 93 percent white,
taken by Mr. Trump in 2016 by nearly 42 points. She has also put gut
economic issues at the centerpiece of her agenda: For instance, her
``\href{https://medium.com/@teamwarren/a-plan-for-economic-patriotism-13b879f4cfc7}{Plan
for Economic Patriotism},'' an industrial-policy tack calling for such
steps as ``more **** actively managing'' the currency value of the
dollar ``to promote exports and domestic manufacturing'' and a tenfold
increase in government spending on job apprenticeship programs,
\href{https://www.youtube.com/watch?v=SUW8kbZyucI}{won praise} from the
Fox News commentator Tucker Carlson. ``It's just pure old-fashioned
economics: how to preserve good-paying American jobs,'' he told his
audience, a form of ``economic nationalism.''

And to be sure, while white working-class voters get a great deal of
attention in battleground states like Michigan, if Democrats can
increase turnout among African-American voters in 2020, that would help
counterbalance any weakness among white working-class voters.

As I was reminded in Rockland, the task of beating Mr. Trump doesn't
require passion for the president's challenger, whoever that may be. The
president, too, arouses a visceral dislike among some people there. One
man, a Vietnam veteran who works at the American Legion post in
Rockland, screwed up his face at my mention of the president. Among the
things he finds unappealing is Mr. Trump's disdainful posture toward the
news media. We chatted about the fractious state of American politics at
the Rockland Bar and Grill, as he sipped his Guinness. ``If it's down to
Trump and Warren, it's definitely Warren,'' he declared without
hesitation. Ms. Warren versus Mr. Trump would be a grind, but that, it
might be said, is the story of her life.

Paul Starobin is a journalist based in Orleans, Mass., and the author,
most recently, of
``\href{https://www.publicaffairsbooks.com/titles/paul-starobin/madness-rules-the-hour/9781610396233/}{Madness
Rules the Hour}: Charleston, 1860 and the Mania for War.''

\emph{The Times is committed to publishing}
\href{https://www.nytimes3xbfgragh.onion/2019/01/31/opinion/letters/letters-to-editor-new-york-times-women.html}{\emph{a
diversity of letters}} \emph{to the editor. We'd like to hear what you
think about this or any of our articles. Here are some}
\href{https://help.nytimes3xbfgragh.onion/hc/en-us/articles/115014925288-How-to-submit-a-letter-to-the-editor}{\emph{tips}}\emph{.
And here's our email:}
\href{mailto:letters@NYTimes.com}{\emph{letters@NYTimes.com}}\emph{.}

\emph{Follow The New York Times Opinion section on}
\href{https://www.facebookcorewwwi.onion/nytopinion}{\emph{Facebook}}\emph{,}
\href{http://twitter.com/NYTOpinion}{\emph{Twitter (@NYTopinion)}}
\emph{and}
\href{https://www.instagram.com/nytopinion/}{\emph{Instagram}}\emph{.}

Advertisement

\protect\hyperlink{after-bottom}{Continue reading the main story}

\hypertarget{site-index}{%
\subsection{Site Index}\label{site-index}}

\hypertarget{site-information-navigation}{%
\subsection{Site Information
Navigation}\label{site-information-navigation}}

\begin{itemize}
\tightlist
\item
  \href{https://help.nytimes3xbfgragh.onion/hc/en-us/articles/115014792127-Copyright-notice}{©~2020~The
  New York Times Company}
\end{itemize}

\begin{itemize}
\tightlist
\item
  \href{https://www.nytco.com/}{NYTCo}
\item
  \href{https://help.nytimes3xbfgragh.onion/hc/en-us/articles/115015385887-Contact-Us}{Contact
  Us}
\item
  \href{https://www.nytco.com/careers/}{Work with us}
\item
  \href{https://nytmediakit.com/}{Advertise}
\item
  \href{http://www.tbrandstudio.com/}{T Brand Studio}
\item
  \href{https://www.nytimes3xbfgragh.onion/privacy/cookie-policy\#how-do-i-manage-trackers}{Your
  Ad Choices}
\item
  \href{https://www.nytimes3xbfgragh.onion/privacy}{Privacy}
\item
  \href{https://help.nytimes3xbfgragh.onion/hc/en-us/articles/115014893428-Terms-of-service}{Terms
  of Service}
\item
  \href{https://help.nytimes3xbfgragh.onion/hc/en-us/articles/115014893968-Terms-of-sale}{Terms
  of Sale}
\item
  \href{https://spiderbites.nytimes3xbfgragh.onion}{Site Map}
\item
  \href{https://help.nytimes3xbfgragh.onion/hc/en-us}{Help}
\item
  \href{https://www.nytimes3xbfgragh.onion/subscription?campaignId=37WXW}{Subscriptions}
\end{itemize}
