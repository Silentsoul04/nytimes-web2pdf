Sections

SEARCH

\protect\hyperlink{site-content}{Skip to
content}\protect\hyperlink{site-index}{Skip to site index}

\href{https://www.nytimes3xbfgragh.onion/section/nyregion}{New York}

\href{https://myaccount.nytimes3xbfgragh.onion/auth/login?response_type=cookie\&client_id=vi}{}

\href{https://www.nytimes3xbfgragh.onion/section/todayspaper}{Today's
Paper}

\href{/section/nyregion}{New York}\textbar{}Trump Lawyers Argue He
Cannot Be Criminally Investigated

\url{https://nyti.ms/3554ZGy}

\begin{itemize}
\item
\item
\item
\item
\item
\item
\end{itemize}

Advertisement

\protect\hyperlink{after-top}{Continue reading the main story}

Supported by

\protect\hyperlink{after-sponsor}{Continue reading the main story}

\hypertarget{trump-lawyers-argue-he-cannot-be-criminally-investigated}{%
\section{Trump Lawyers Argue He Cannot Be Criminally
Investigated}\label{trump-lawyers-argue-he-cannot-be-criminally-investigated}}

The president's legal team is trying to block a subpoena seeking his tax
returns, claiming that any criminal investigation of Mr. Trump is
unconstitutional.

\includegraphics{https://static01.graylady3jvrrxbe.onion/images/2019/09/19/nyregion/19nytrump/merlin_161020518_25cd66c5-d6ad-4f03-8dbf-6bbf9aa1f9c6-articleLarge.jpg?quality=75\&auto=webp\&disable=upscale}

\href{https://www.nytimes3xbfgragh.onion/by/michael-gold}{\includegraphics{https://static01.graylady3jvrrxbe.onion/images/2018/06/12/multimedia/author-michael-gold/author-michael-gold-thumbLarge.png}}

By \href{https://www.nytimes3xbfgragh.onion/by/michael-gold}{Michael
Gold}

\begin{itemize}
\item
  Published Sept. 19, 2019Updated July 9, 2020
\item
  \begin{itemize}
  \item
  \item
  \item
  \item
  \item
  \item
  \end{itemize}
\end{itemize}

Lawyers for President Trump argued in a lawsuit filed on Thursday that
he could not be criminally investigated while in office, as they sought
to block a subpoena from state prosecutors in Manhattan demanding eight
years of his tax returns.

Taking a broad position that the lawyers acknowledged had not been
tested, the president's legal team argued in the complaint that the
Constitution effectively makes sitting presidents immune from all
criminal inquiries until they leave the White House.

Presidents, they asserted, have such enormous responsibility and play a
unique role in government that they cannot be subject to the burden of
investigations, especially from local prosecutors who may use the
criminal process for political gain.

Several constitutional law scholars interviewed by The New York Times
said that if the lawyers' position were accepted by the court, it would
set a sweeping new precedent.

But they also said it was far from certain that the theory, which was
not based on established case law, would succeed. While an onslaught of
investigations would most assuredly disrupt a presidency, the
Constitution does not explicitly say that presidents are shielded from
criminal inquiries.

``President Trump's position that he is unequivocally beyond the reach
of criminal investigators is doubtfully absolutist,'' said Joshua Matz,
who wrote, with Laurence H. Tribe, ``To End a Presidency: The Power of
Impeachment.''

At the least, the lawsuit, which was filed in federal court in
Manhattan, is likely to delay the latest attempt to secure Mr. Trump's
financial records.

The lawsuit was filed in response to a subpoena issued late last month
by the Manhattan district attorney's office to Mr. Trump's accounting
firm. The subpoena sought eight years of the president's personal and
corporate tax returns as the office investigates the role that Mr. Trump
and his family business played in hush-money payments made in the run-up
to the 2016 presidential election.

\href{https://www.nytimes3xbfgragh.onion/2018/08/21/nyregion/michael-cohen-plea-deal-trump.html}{Both
Mr. Trump and the company reimbursed Michael D. Cohen}, the president's
former lawyer and fixer, for money that Mr. Cohen paid to buy the
silence of Stormy Daniels, an adult film actress who said she had an
affair with Mr. Trump. The president has denied the affair.

The federal prosecutors who charged Mr. Cohen said in a court filing in
July that they had ``effectively concluded'' their investigation into
possible crimes committed by the president's company, the Trump
Organization, or its executives. Neither the company nor any of its
leaders were charged. However, the office of the Manhattan district
attorney, Cyrus R. Vance Jr., is exploring whether the reimbursements
violated any New York state laws.

The lawsuit filed on Thursday was the latest effort by the president and
his legal team to stymie
\href{https://www.nytimes3xbfgragh.onion/2019/08/13/us/politics/trump-house-lawsuits.html?module=inline}{multiple
attempts to obtain copies of his tax returns}, which Mr. Trump said
during the 2016 campaign that he would make public but has since refused
to disclose.

Mr. Trump's lawyers have sued to block attempts by congressional
Democrats and New York lawmakers to gain access to his tax returns and
financial records. They also
\href{https://www.nytimes3xbfgragh.onion/2019/08/06/us/politics/california-trump-tax-returns.html?module=inline}{challenged
a California}law requiring presidential primary candidates to release
their tax returns, and a
\href{https://www.latimes.com/california/story/2019-09-19/trump-tax-returns-federal-court-challenge-california}{federal
judge ruled in their favor} on Thursday. But their arguments in those
cases had been made on narrower grounds.

\href{https://www.nytimes3xbfgragh.onion/2017/05/29/us/politics/a-constitutional-puzzle-can-the-president-be-indicted.html?module=inline}{It
is an open question whether sitting presidents are immune} from
prosecution while in office. The Constitution does not explicitly
address the issue, and the Supreme Court has never answered the
question.

Federal prosecutors are barred from charging a sitting president with a
federal crime because the Justice Department --- in memos written during
the Nixon and Clinton administrations --- has decided that presidents
have temporary immunity while they are in office. The memos indicate
that any wrongdoing should be addressed through impeachment, not the
courts.

Those memos, however, do not bind the hands of state prosecutors.

``I think there is some force to the argument that states can't be
allowed to hobble presidents with local prosecutions, but there is
certainly no authority for the claim that they cannot at least
investigate while a president is in office,'' said Frank O. Bowman III,
a law professor at the University of Missouri and the author of ``High
Crimes and Misdemeanors: A History of Impeachment for the Age of
Trump.''

Presidents have been subject to federal criminal inquiries in the past,
including Mr. Trump, who was recently a subject of an investigation
conducted by Robert S. Mueller III, the special counsel, that examined
ties between the Trump campaign and Russia.

Mr. Mueller said in May that the Justice Department ``explicitly
permits'' the investigation of presidents, although he acknowledged that
they could not be charged with federal crimes.

In their lawsuit, Mr. Trump's lawyers took their arguments a step
further, saying that not only are criminal charges against a president
unconstitutional, so are investigations. They took particular issue with
any investigation conducted by ``a county prosecutor,'' such as the
Manhattan district attorney.

``We are in court to protect the president's rights and the
Constitution,'' said Marc L. Mukasey, a lawyer for the Trump
Organization.

Mr. Vance, a Democrat, is seeking a range of tax documents from Mr.
Trump's accounting firm, Mazars USA, including federal and state tax
returns for both the president and his company, dating to 2011.

In their lawsuit, the lawyers, who have repeatedly called Mr. Vance's
investigation politically motivated, wrote that prosecutors looking to
advance their careers were particularly susceptible to opening
investigations that could interfere with presidential duties.

``A county prosecutor in New York, for what appears to be the first time
in our nation's history, is attempting to do just that,'' the complaint
said.

The court papers described behind-the-scenes negotiations between the
two sides. Mr. Trump's lawyers said that they had been in negotiations
with the district attorney's office
\href{https://www.nytimes3xbfgragh.onion/2019/08/01/nyregion/trump-cohen-stormy-daniels-vance.html}{over
an earlier subpoena issued to the Trump Organization.} When Mr. Vance's
office asked for the company's tax returns, Mr. Trump's lawyers
resisted. The prosecutors then sent a subpoena to the accounting firm,
this time including the president's personal returns as well.

The president's legal team called the action a ``bad faith effort to
harass the president by obtaining and exposing his confidential
financial information.'' The lawyers are seeking an injunction
preventing both Mazars and Mr. Vance from taking action on the subpoena
until after Mr. Trump leaves office.

A spokesman for Mr. Vance said the district attorney would respond in
court and had no additional comment. Mazars said it would ``respect the
legal process and fully comply with its legal obligations.''

Judge Victor Marrero of Federal District Court in Manhattan will hear
arguments in the case on Wednesday. In a letter, Mr. Vance's office said
it had agreed to delay enforcement of its subpoena until then.

Mr. Vance's investigation has been focused on \$130,000 that Mr. Cohen
paid Ms. Daniels, whose legal name is Stephanie Clifford, just before
the election.
\href{https://www.nytimes3xbfgragh.onion/2018/08/21/nyregion/michael-cohen-plea-deal-trump.html?module=inline}{Mr.
Cohen pleaded guilty last year} to violating federal campaign finance
laws. He received a
\href{https://www.nytimes3xbfgragh.onion/2018/12/12/nyregion/michael-cohen-sentence-trump.html?module=inline}{three-year
prison sentence}.

Mr. Vance's office has been investigating whether the Trump Organization
falsely accounted for the reimbursements as a legal expense. It was
unclear if the scope of the subpoena --- including Mr. Trump's personal
records --- meant that the inquiry had widened.

In New York, filing a false business record can be a felony only if
prosecutors can prove that the filing was made to commit or conceal
another crime, such as tax violations or bank fraud. Mr. Trump's tax
returns, if Mr. Vance's office obtains them, would be protected by rules
governing grand jury secrecy and not become public unless they were used
as evidence in a criminal case.

Reporting was contributed by Adam Liptak, Ben Protess, William K.
Rashbaum and Benjamin Weiser.

Advertisement

\protect\hyperlink{after-bottom}{Continue reading the main story}

\hypertarget{site-index}{%
\subsection{Site Index}\label{site-index}}

\hypertarget{site-information-navigation}{%
\subsection{Site Information
Navigation}\label{site-information-navigation}}

\begin{itemize}
\tightlist
\item
  \href{https://help.nytimes3xbfgragh.onion/hc/en-us/articles/115014792127-Copyright-notice}{©~2020~The
  New York Times Company}
\end{itemize}

\begin{itemize}
\tightlist
\item
  \href{https://www.nytco.com/}{NYTCo}
\item
  \href{https://help.nytimes3xbfgragh.onion/hc/en-us/articles/115015385887-Contact-Us}{Contact
  Us}
\item
  \href{https://www.nytco.com/careers/}{Work with us}
\item
  \href{https://nytmediakit.com/}{Advertise}
\item
  \href{http://www.tbrandstudio.com/}{T Brand Studio}
\item
  \href{https://www.nytimes3xbfgragh.onion/privacy/cookie-policy\#how-do-i-manage-trackers}{Your
  Ad Choices}
\item
  \href{https://www.nytimes3xbfgragh.onion/privacy}{Privacy}
\item
  \href{https://help.nytimes3xbfgragh.onion/hc/en-us/articles/115014893428-Terms-of-service}{Terms
  of Service}
\item
  \href{https://help.nytimes3xbfgragh.onion/hc/en-us/articles/115014893968-Terms-of-sale}{Terms
  of Sale}
\item
  \href{https://spiderbites.nytimes3xbfgragh.onion}{Site Map}
\item
  \href{https://help.nytimes3xbfgragh.onion/hc/en-us}{Help}
\item
  \href{https://www.nytimes3xbfgragh.onion/subscription?campaignId=37WXW}{Subscriptions}
\end{itemize}
