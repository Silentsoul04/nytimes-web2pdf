Sections

SEARCH

\protect\hyperlink{site-content}{Skip to
content}\protect\hyperlink{site-index}{Skip to site index}

\href{https://myaccount.nytimes3xbfgragh.onion/auth/login?response_type=cookie\&client_id=vi}{}

\href{https://www.nytimes3xbfgragh.onion/section/todayspaper}{Today's
Paper}

\href{/section/opinion}{Opinion}\textbar{}Why India and the World Need
Gandhi

\url{https://nyti.ms/2mPOWuX}

\begin{itemize}
\item
\item
\item
\item
\item
\item
\end{itemize}

Advertisement

\protect\hyperlink{after-top}{Continue reading the main story}

\href{/section/opinion}{Opinion}

Supported by

\protect\hyperlink{after-sponsor}{Continue reading the main story}

\hypertarget{why-india-and-the-world-need-gandhi}{%
\section{Why India and the World Need
Gandhi}\label{why-india-and-the-world-need-gandhi}}

The great leader envisioned a world where every citizen has dignity and
prosperity.

By Narendra Modi

Mr. Modi is the prime minister of India.

\begin{itemize}
\item
  Oct. 2, 2019
\item
  \begin{itemize}
  \item
  \item
  \item
  \item
  \item
  \item
  \end{itemize}
\end{itemize}

\includegraphics{https://static01.graylady3jvrrxbe.onion/images/2019/10/01/opinion/01modi1/01modi1-articleLarge.jpg?quality=75\&auto=webp\&disable=upscale}

NEW DELHI --- Upon reaching India in 1959, the Rev. Dr. Martin Luther
King Jr. remarked, ``To other countries I may go as a tourist, but to
India I come as a pilgrim.'' He added, ``Perhaps, above all, India is
the land where the techniques of nonviolent
\href{https://kinginstitute.stanford.edu/king-papers/documents/account-lawrence-dunbar-reddick-press-conference-new-delhi-10-february-1959}{social
change} were developed that my people have used in Montgomery, Alabama,
and elsewhere throughout the American South. We have found them to be
effective and sustaining --- they work!''

The guiding light whose inspiration got Dr. King to India was Mohandas
Karamchand Gandhi, the Mahatma, the Great Soul. On Wednesday, we observe
his 150th birth anniversary. Gandhi Ji, or Bapu, continues to give
courage to millions globally.

Gandhian methods of resistance ignited a spirit of hope among several
African nations. Dr. King remarked: ``When I was visiting in Ghana, West
Africa, Prime Minister Nkrumah told me that he had read the works of
Gandhi and felt that nonviolent resistance could be extended there. We
recall that South Africa has had bus boycotts also.''

Nelson Mandela referred to Gandhi as ``the Sacred Warrior'' and wrote,
``His strategy of noncooperation, his assertion that we can be dominated
only if we cooperate with our dominators, and his
\href{http://content.time.com/time/magazine/article/0,9171,993025,00.html}{nonviolent
resistance} inspired anticolonial and antiracist movements
internationally in our century.''

For Mr. Mandela, Gandhi was Indian and South African. Gandhi would have
approved. He had the unique ability to become a bridge between some of
the greatest contradictions in human society.

In 1925, Gandhi wrote in ``Young India'': ``It is impossible for one to
be internationalist without being a nationalist. Internationalism is
possible only when nationalism becomes a fact, i.e., when peoples
belonging to different countries have organized themselves and are able
to act as one man.'' He envisioned Indian nationalism as one that was
never narrow or exclusive but one that worked for the service of
humanity.

Mahatma Gandhi also epitomized trust among all sections of society. In
1917, Ahmedabad in Gujarat witnessed a huge textile strike. When the
conflict between the mill workers and owners escalated to a point of no
return, it was Gandhi who mediated an equitable settlement.

Gandhi formed the Majoor Mahajan Sangh, an association for workers'
rights. At first sight, it may seem just another name of an organization
but it reveals how small steps created a large impact. During those
days, ``Mahajan'' was used as a title of respect for elites. Gandhi
inverted the social structure by attaching the name ``Mahajan'' to
``Majoor,'' or laborers. With that linguistic choice, Gandhi enhanced
the pride of workers.

And Gandhi combined ordinary objects with mass politics. Who else could
have used a charkha, a spinning wheel, and khadi, Indian homespun cloth,
as symbols of economic self-reliance and empowerment for a nation?

Who else could have created a mass agitation through a pinch of salt!
During colonial rule, Salt Laws, which placed a new tax on Indian salt,
had become a burden. Through the Dandi March in 1930, Gandhi challenged
the Salt Laws. His picking up a small lump of natural salt from the
Arabian Sea shore led to the historic civil disobedience movement.

There have been many mass movements in the world, many strands of the
freedom struggle even in India, but what sets apart the Gandhian
struggle and those inspired by him is the wide-scale public
participation. He never held administrative or elected office. He was
never tempted by power.

For him, independence was not absence of external rule. He saw a deep
link between political independence and personal empowerment. He
envisioned a world where every citizen has dignity and prosperity. When
the world spoke about rights, Gandhi emphasized duties. He wrote in
``Young India'': ``The true source of rights is duty. If we all
discharge our duties, rights will not be far to seek.'' He wrote in the
journal Harijan, ``Rights accrue automatically to him who duly performs
his duties.''

Gandhi gave us the doctrine of trusteeship, which emphasized the
socio-economic welfare of the poor. Inspired by that, we should think
about a spirit of ownership. We, as inheritors of the earth, are
responsible for its well-being, including that of the flora and fauna
with whom we share our planet.

In Gandhi, we have the best teacher to guide us. From uniting those who
believe in humanity to furthering sustainable development and ensuring
economic self-reliance, Gandhi offers solutions to every problem.

We in India are doing our bit. India is among the fastest when it comes
to eliminating poverty. Our sanitation efforts have drawn global
attention. India is also taking the lead in harnessing renewable
resources through efforts like the International Solar Alliance, which
has brought together several nations to leverage solar energy for a
sustainable future. We want to do even more, with the world and for the
world.

As a tribute to Gandhi, I propose what I call the Einstein Challenge. We
know Albert Einstein's famous words on Gandhi: ``Generations to come
will scarce believe that such a one as this ever in flesh and blood
walked upon this earth.''

How do we ensure the ideals of Gandhi are remembered by future
generations? I invite thinkers, entrepreneurs and tech leaders to be at
the forefront of spreading Gandhi's ideas through innovation.

Let us work shoulder to shoulder to make our world prosperous and free
from hate, violence and suffering. That is when we will fulfill Mahatma
Gandhi's dream, summed up in his favorite hymn, ``Vaishnava Jana To,''
which says that a true human is one who feels the pain of others,
removes misery and is never arrogant.

The world bows to you, beloved Bapu!

Narendra Modi is the prime minister of India.

\emph{The Times is committed to publishing}
\href{https://www.nytimes3xbfgragh.onion/2019/01/31/opinion/letters/letters-to-editor-new-york-times-women.html}{\emph{a
diversity of letters}} \emph{to the editor. We'd like to hear what you
think about this or any of our articles. Here are some}
\href{https://help.nytimes3xbfgragh.onion/hc/en-us/articles/115014925288-How-to-submit-a-letter-to-the-editor}{\emph{tips}}\emph{.
And here's our email:}
\href{mailto:letters@NYTimes.com}{\emph{letters@NYTimes.com}}\emph{.}

\emph{Follow The New York Times Opinion section on}
\href{https://www.facebookcorewwwi.onion/nytopinion}{\emph{Facebook}}\emph{,}
\href{http://twitter.com/NYTOpinion}{\emph{Twitter (@NYTopinion)}}
\emph{and}
\href{https://www.instagram.com/nytopinion/}{\emph{Instagram}}\emph{.}

Advertisement

\protect\hyperlink{after-bottom}{Continue reading the main story}

\hypertarget{site-index}{%
\subsection{Site Index}\label{site-index}}

\hypertarget{site-information-navigation}{%
\subsection{Site Information
Navigation}\label{site-information-navigation}}

\begin{itemize}
\tightlist
\item
  \href{https://help.nytimes3xbfgragh.onion/hc/en-us/articles/115014792127-Copyright-notice}{©~2020~The
  New York Times Company}
\end{itemize}

\begin{itemize}
\tightlist
\item
  \href{https://www.nytco.com/}{NYTCo}
\item
  \href{https://help.nytimes3xbfgragh.onion/hc/en-us/articles/115015385887-Contact-Us}{Contact
  Us}
\item
  \href{https://www.nytco.com/careers/}{Work with us}
\item
  \href{https://nytmediakit.com/}{Advertise}
\item
  \href{http://www.tbrandstudio.com/}{T Brand Studio}
\item
  \href{https://www.nytimes3xbfgragh.onion/privacy/cookie-policy\#how-do-i-manage-trackers}{Your
  Ad Choices}
\item
  \href{https://www.nytimes3xbfgragh.onion/privacy}{Privacy}
\item
  \href{https://help.nytimes3xbfgragh.onion/hc/en-us/articles/115014893428-Terms-of-service}{Terms
  of Service}
\item
  \href{https://help.nytimes3xbfgragh.onion/hc/en-us/articles/115014893968-Terms-of-sale}{Terms
  of Sale}
\item
  \href{https://spiderbites.nytimes3xbfgragh.onion}{Site Map}
\item
  \href{https://help.nytimes3xbfgragh.onion/hc/en-us}{Help}
\item
  \href{https://www.nytimes3xbfgragh.onion/subscription?campaignId=37WXW}{Subscriptions}
\end{itemize}
