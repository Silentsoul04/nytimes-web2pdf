Sections

SEARCH

\protect\hyperlink{site-content}{Skip to
content}\protect\hyperlink{site-index}{Skip to site index}

\href{https://www.nytimes3xbfgragh.onion/section/nyregion}{New York}

\href{https://myaccount.nytimes3xbfgragh.onion/auth/login?response_type=cookie\&client_id=vi}{}

\href{https://www.nytimes3xbfgragh.onion/section/todayspaper}{Today's
Paper}

\href{/section/nyregion}{New York}\textbar{}Trump Taxes: President
Ordered to Turn Over Returns to Manhattan D.A.

\url{https://nyti.ms/339BBwY}

\begin{itemize}
\item
\item
\item
\item
\item
\item
\end{itemize}

Advertisement

\protect\hyperlink{after-top}{Continue reading the main story}

Supported by

\protect\hyperlink{after-sponsor}{Continue reading the main story}

\hypertarget{trump-taxes-president-ordered-to-turn-over-returns-to-manhattan-da}{%
\section{Trump Taxes: President Ordered to Turn Over Returns to
Manhattan
D.A.}\label{trump-taxes-president-ordered-to-turn-over-returns-to-manhattan-da}}

A judge rejected the president's argument that he was immune from
criminal investigations.

\includegraphics{https://static01.graylady3jvrrxbe.onion/images/2019/10/07/nyregion/07NYTrump/merlin_162142308_f1ed51d7-3b60-4e69-a9bb-9cb3f5c68d2e-articleLarge.jpg?quality=75\&auto=webp\&disable=upscale}

\href{https://www.nytimes3xbfgragh.onion/by/william-k-rashbaum}{\includegraphics{https://static01.graylady3jvrrxbe.onion/images/2018/06/13/multimedia/author-william-k-rashbaum/author-william-k-rashbaum-thumbLarge.jpg}}\href{https://www.nytimes3xbfgragh.onion/by/benjamin-weiser}{\includegraphics{https://static01.graylady3jvrrxbe.onion/images/2018/07/16/multimedia/author-benjamin-weiser/author-benjamin-weiser-thumbLarge.png}}

By
\href{https://www.nytimes3xbfgragh.onion/by/william-k-rashbaum}{William
K. Rashbaum} and
\href{https://www.nytimes3xbfgragh.onion/by/benjamin-weiser}{Benjamin
Weiser}

\begin{itemize}
\item
  Published Oct. 7, 2019Updated July 15, 2020
\item
  \begin{itemize}
  \item
  \item
  \item
  \item
  \item
  \item
  \end{itemize}
\end{itemize}

A federal judge on Monday rejected President Trump's effort to shield
his
\href{https://www.nytimes3xbfgragh.onion/2020/07/15/nyregion/donald-trump-taxes-cyrus-vance.html}{tax
returns} from Manhattan state
prosecutors\href{https://www.nytimes3xbfgragh.onion/2019/09/19/nyregion/trump-tax-returns-lawsuit.html}{,}
calling the president's argument that he was immune from criminal
investigation ``repugnant to the nation's governmental structure and
constitutional values.''

The decision from Judge Victor Marrero of Federal District Court in
Manhattan was the first significant ruling in a case that could require
Mr. Trump to hand over his tax returns and ultimately test the limits of
presidential power.

The judge dismissed a lawsuit that had been filed by Mr. Trump, who was
seeking to block a subpoena for eight years of his personal and
corporate tax returns. The Manhattan district attorney demanded the
records in late August as part of an investigation into hush-money
payments made in the run-up to the 2016 presidential election.

Mr. Trump's tax returns, however, remain protected for now. His lawyers
quickly
\href{https://www.nytimes3xbfgragh.onion/2019/11/04/nyregion/trump-taxes-vance-appeal.html}{appealed
to the Second Circuit Court of Appeals} in Manhattan, which agreed to
temporarily delay enforcement of the subpoena while it considers
arguments in the case.

In a 75-page ruling that included detailed constitutional analysis and
cited Supreme Court precedents, Judge Marrero systematically dismantled
the president's arguments that investigating a sitting president was
unconstitutional. The judge said Mr. Trump's lawyers were, in essence,
arguing that the president, along with his family, associates and
companies, was above the law.

``This court finds aspects of such a doctrine repugnant to the nation's
governmental structure and constitutional values,'' wrote the judge, who
was appointed to the bench in 1999 by President Bill Clinton.

\hypertarget{judge-victor-marreros-ruling}{%
\subsection{Judge Victor Marrero's
Ruling}\label{judge-victor-marreros-ruling}}

Court ruling in Trump v. New York D.A. (PDF, 75 pages, 1.98 MB)

\includegraphics{https://int.graylady3jvrrxbe.onion/data/documenthelper/1878-07nytrump-ruling/3febb8a88a32dc2e6521/optimized/thumbnail.png}

The dispute has pitted the Manhattan district attorney, Cyrus R. Vance
Jr., against the president and his Justice Department and has raised a
host of issues that have not been tested in the courts.
\href{https://www.nytimes3xbfgragh.onion/2017/05/29/us/politics/a-constitutional-puzzle-can-the-president-be-indicted.html?module=inline}{The
Constitution does not explicitly say} whether presidents can be charged
with a crime while in office, and the Supreme Court has not ruled on the
issue.

Walter Dellinger, who served as acting United States solicitor general
in the Clinton administration, said Judge Marrero's opinion was ``an
emphatic rejection of the imperial presidency claim that a president
cannot even be investigated.''

The judge's decision came a little more than a month after
\href{https://www.nytimes3xbfgragh.onion/2019/09/16/nyregion/trump-tax-returns-cy-vance.html}{Mr.
Vance subpoenaed Mr. Trump's accounting firm}, Mazars USA, for his
personal and corporate tax returns dating to 2011.

Mr. Vance's office has been investigating whether any New York State
laws were broken when Mr. Trump and his company, the Trump Organization,
reimbursed the president's former lawyer and fixer,
\href{https://www.nytimes3xbfgragh.onion/2018/08/21/nyregion/michael-cohen-plea-deal-trump.html}{Michael
D. Cohen, for payments he made to the pornographic film actress Stormy
Daniels}, who had said she had an affair with Mr. Trump. Mr. Trump has
denied the affair.

Mr. Trump's lawyers sued last month to block the subpoena. The lawyers
acknowledged that their constitutional argument had not been tested, but
said presidents have such enormous responsibility and a unique position
in government that they cannot be burdened with investigations,
especially by local prosecutors who might be politically motivated.

``This case presents momentous questions of first impression regarding
the presidency, federalism and the separation of powers,'' a lawyer for
the president, Patrick Strawbridge, wrote to the appeals court on
Monday. He said the losing party should be given time to appeal to the
Supreme Court.

\href{https://www.nytimes3xbfgragh.onion/2019/10/02/nyregion/trump-taxes-lawsuit.html}{The
case also has drawn in Mr. Trump's own Justice Department,} which has
not taken a position on the president's argument but supported his
request to delay enforcement of the subpoena because of the
``significant constitutional issues.''

Shortly after the judge's ruling was released, Mr. Trump wrote on
Twitter that the ``radical left'' had pushed New York prosecutors to
target him.

A lawyer for the president and a spokesman for Mr. Vance both declined
to comment on Monday, as did a spokeswoman for the Justice Department.

The decision was a victory for
\href{https://www.nytimes3xbfgragh.onion/2019/09/23/nyregion/trump-tax-returns-lawsuit.html?module=inline}{Mr.
Vance, whose office had asked Judge Marrero to dismiss Mr. Trump's
suit}, accusing the president and his team of trying to drag out the
investigation until the statute of limitations runs out on any possible
crime.

Mr. Trump's lawyers have called the investigation by Mr. Vance, a
Democrat, politically motivated.

Longstanding policy from the Justice Department bars federal prosecutors
from charging a sitting president with a crime. Department lawyers have
concluded that presidents have temporary immunity while they are in
office.

But in the past, that position has not precluded investigating a
president. Mr. Trump and other presidents have been the subjects of
federal criminal investigations while in office. Local prosecutors, such
as Mr. Vance, are also not bound by the Justice Department's policy.

Mr. Trump's arguments went a step further, starting with a central claim
that the Constitution gave him sweeping immunity not just from
indictment and prosecution, but also from any investigation by federal
or state authorities.

In his opinion, Judge Marrero pointedly noted that in throwing off the
yoke of the British crown, the country's founders had dismissed the
notion of broad immunity.

``Shunning the concept of the inviolability of the person of the King of
England and the bounds of the monarch's protective screen,'' the judge
wrote, ``the founders disclaimed any notion that the Constitution
generally conferred similarly all-encompassing immunity upon the
president.''

Judge Marrero also dispatched Mr. Trump's other claims, including that
the district attorney had no authority to subpoena his tax returns or
had acted in bad faith, and that being forced to turn over the returns
would cause Mr. Trump ``irreparable harm.''

The judge rejected the conclusions of three Justice Department memos
dating back to as early as 1973 that he said have long been cited as
supporting the interpretation that a sitting president cannot be charged
with a crime.

He said the memos rely on ``suppositions, practicalities and public
policy'' as well as dire pictures of hypothetical scenarios --- and not
on an actual case.

Late Monday, the appeals court said that it would hear arguments in the
case later this month and that enforcement of the subpoena would be
delayed at least until then.

If Mr. Vance ultimately prevails in obtaining the president's tax
returns, they would not automatically become public. The documents would
be protected by rules governing the secrecy of grand jury investigations
unless the documents became evidence in a criminal case.

The president and his lawyers have fought vigorously in other venues to
shield his tax returns, which Mr. Trump said during the 2016 campaign
that he would make public but has since refused to disclose.

Mr. Trump's lawyers have sued to stop attempts by congressional
Democrats to gain access to his tax returns and financial records and to
block a New York State law that would share state tax returns with
congressional committees. They also
\href{https://www.nytimes3xbfgragh.onion/2019/08/06/us/politics/california-trump-tax-returns.html?module=inline}{successfully
challenged a California law} requiring presidential primary candidates
to release their tax returns.

Mr. Vance's office has been investigating whether the Trump Organization
falsely accounted for the reimbursements to Mr. Cohen as a legal
expense. In New York, filing a false business record can be a crime.

But it becomes a felony only if prosecutors can prove that the false
filing was made to commit or conceal another crime, such as bank fraud
or tax violations. It was unclear why the office has attempted to obtain
Mr. Trump's personal financial records as part of that inquiry.

Mr. Trump's accounting firm, Mazars, which he sued along with the
district attorney's office to bar the company from turning over his
returns, reiterated an earlier statement that it would comply with its
legal obligations.

Advertisement

\protect\hyperlink{after-bottom}{Continue reading the main story}

\hypertarget{site-index}{%
\subsection{Site Index}\label{site-index}}

\hypertarget{site-information-navigation}{%
\subsection{Site Information
Navigation}\label{site-information-navigation}}

\begin{itemize}
\tightlist
\item
  \href{https://help.nytimes3xbfgragh.onion/hc/en-us/articles/115014792127-Copyright-notice}{©~2020~The
  New York Times Company}
\end{itemize}

\begin{itemize}
\tightlist
\item
  \href{https://www.nytco.com/}{NYTCo}
\item
  \href{https://help.nytimes3xbfgragh.onion/hc/en-us/articles/115015385887-Contact-Us}{Contact
  Us}
\item
  \href{https://www.nytco.com/careers/}{Work with us}
\item
  \href{https://nytmediakit.com/}{Advertise}
\item
  \href{http://www.tbrandstudio.com/}{T Brand Studio}
\item
  \href{https://www.nytimes3xbfgragh.onion/privacy/cookie-policy\#how-do-i-manage-trackers}{Your
  Ad Choices}
\item
  \href{https://www.nytimes3xbfgragh.onion/privacy}{Privacy}
\item
  \href{https://help.nytimes3xbfgragh.onion/hc/en-us/articles/115014893428-Terms-of-service}{Terms
  of Service}
\item
  \href{https://help.nytimes3xbfgragh.onion/hc/en-us/articles/115014893968-Terms-of-sale}{Terms
  of Sale}
\item
  \href{https://spiderbites.nytimes3xbfgragh.onion}{Site Map}
\item
  \href{https://help.nytimes3xbfgragh.onion/hc/en-us}{Help}
\item
  \href{https://www.nytimes3xbfgragh.onion/subscription?campaignId=37WXW}{Subscriptions}
\end{itemize}
