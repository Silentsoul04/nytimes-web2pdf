Sections

SEARCH

\protect\hyperlink{site-content}{Skip to
content}\protect\hyperlink{site-index}{Skip to site index}

\href{https://www.nytimes3xbfgragh.onion/section/technology}{Technology}

\href{https://myaccount.nytimes3xbfgragh.onion/auth/login?response_type=cookie\&client_id=vi}{}

\href{https://www.nytimes3xbfgragh.onion/section/todayspaper}{Today's
Paper}

\href{/section/technology}{Technology}\textbar{}Silicon Valley Is Trying
Out a New Mantra: Make a Profit

\url{https://nyti.ms/2IyOWXQ}

\begin{itemize}
\item
\item
\item
\item
\item
\item
\end{itemize}

Advertisement

\protect\hyperlink{after-top}{Continue reading the main story}

Supported by

\protect\hyperlink{after-sponsor}{Continue reading the main story}

\hypertarget{silicon-valley-is-trying-out-a-new-mantra-make-a-profit}{%
\section{Silicon Valley Is Trying Out a New Mantra: Make a
Profit}\label{silicon-valley-is-trying-out-a-new-mantra-make-a-profit}}

Start-up investors are warning of a reckoning after the stumbles of some
high-profile ``unicorns.'' Now turning a profit is in.

\includegraphics{https://static01.graylady3jvrrxbe.onion/images/2019/10/08/business/08valley/08valley-articleLarge.jpg?quality=75\&auto=webp\&disable=upscale}

\href{https://www.nytimes3xbfgragh.onion/by/erin-griffith}{\includegraphics{https://static01.graylady3jvrrxbe.onion/images/2019/06/18/reader-center/author-erin-griffith/author-erin-griffith-thumbLarge.png}}

By \href{https://www.nytimes3xbfgragh.onion/by/erin-griffith}{Erin
Griffith}

\begin{itemize}
\item
  Oct. 8, 2019
\item
  \begin{itemize}
  \item
  \item
  \item
  \item
  \item
  \item
  \end{itemize}
\end{itemize}

SAN FRANCISCO --- Fred Wilson, a venture capitalist at Union Square
Ventures, recently published a blog post titled
``\href{https://avc.com/2019/09/the-great-public-market-reckoning/}{The
Great Public Market Reckoning}.'' In it, he argued that the narrative
that had driven start-up hype and valuations for the last decade was now
falling apart.

His post quickly ricocheted across Silicon Valley. Other venture
capitalists, including Bill Gurley of Benchmark and Brad Feld of Foundry
Group, soon weighed in with their own warnings about fiscal
responsibility.

At some start-ups, entrepreneurs began behaving more cautiously. Travis
VanderZanden,
\href{https://www.nytimes3xbfgragh.onion/2018/04/20/technology/electric-scooters-are-causing-havoc-this-man-is-shrugging-it-off.html}{chief
executive of the scooter start-up Bird}, declared at a tech conference
in San Francisco last week that his company was now focused on profit
and not growth. ``The challenge is to try to stay disciplined,'' he
said.

The moves all point to a new gospel that is starting to spread in
start-up land. For the last decade, young tech companies were fueled by
a wave of venture capital-funded excess, which encouraged fast growth
above all else. But now some investors and start-ups are beginning to
rethink that mantra and instead invoke turning a profit and generating
``positive unit economics'' as their new priorities.

\includegraphics{https://static01.graylady3jvrrxbe.onion/images/2019/10/08/business/08valley1/merlin_134164781_4dedf542-e0cc-4b7c-80d9-cdfbdee09895-articleLarge.jpg?quality=75\&auto=webp\&disable=upscale}

The nascent change is being driven by the stumbles of some high-profile
``unicorns'' --- the start-ups that were valued at \$1 billion and above
in the private markets --- just as they reached the stock market.

The most visible of those was the office rental start-up WeWork, which
dramatically
\href{https://www.nytimes3xbfgragh.onion/2019/09/24/business/dealbook/wework-ceo-adam-neumann.html}{ousted
its chief executive} and
\href{https://www.nytimes3xbfgragh.onion/2019/09/30/business/wework-ipo.html}{withdrew
its initial public offering} last month. At the same time, shares of
Peloton, a fitness start-up, and SmileDirectClub, an online orthodontics
company, immediately cratered after the companies went public. And Uber,
Lyft and Slack --- which also listed their stocks this year --- have
similarly dealt with falling stock prices for months.

The lackluster performances have raised questions about Silicon Valley's
start-up formula of spending lots of money to grow at the expense of
profits. (All of those companies lose money.) Public market investors,
it seemed,
\href{https://www.nytimes3xbfgragh.onion/2019/09/26/business/tech-ipo-market.html}{just
weren't having it}.

``A lot of these highly valued companies have run into the buzz saw of
Wall Street, where they're questioning or reminding us that
profitability matters,'' said Patricia Nakache, a partner at Trinity
Ventures, a Silicon Valley venture capital firm.

She added that she anticipated a ``ripple effect'' on private start-up
valuations that would start with the largest, most valuable companies
and trickle down to the smaller, younger ones.

For start-ups and investors that were used to heady times and big
spending, that means it may be time for a reset.

Aileen Lee, an investor at Cowboy Ventures, a venture capital firm in
Palo Alto, Calif., said she considered dusting off a four-year-old
``winter is coming'' email she had sent to start-ups in 2015, telling
them to prepare for a downturn. She hasn't revived the warning yet, she
said, because ``I worry about becoming the boy who cried wolf.''

Image

WeWork ousted its chief executive and withdrew its initial public
offering last month.Credit...Haruka Sakaguchi for The New York Times

Other venture capitalists are being more forward. At Eniac Ventures, a
venture firm in New York and San Francisco, the partners recently combed
through their companies and identified the ``gross margins'' --- a
measure of profitability --- for each one, said Nihal Mehta, general
partner of the firm. This was not something the firm regularly looked
at, he said, but they were inspired by Mr. Wilson's cautionary blog
post.

They ultimately decided that in future meetings with entrepreneurs, they
would push for more detailed financial models, even though the companies
are very young, Mr. Mehta said. While Eniac had looked at this when
making investments before, ``now it's more important,'' he said.

Tech start-ups have long gone through different cycles of fear and
loathing. When the 2008 recession began, Sequoia Capital, one of the
highest-profile venture firms, called a mass meeting with its start-ups
and presented a slide deck, titled
``\href{https://www.sequoiacap.com/article/rip-good-times}{R.I.P. Good
Times},'' that featured a graphic of a ``death spiral'' and a skull.

The event was intended as a way to shock the start-ups into reining in
costs to survive the downturn. Sequoia's presentation quickly became the
talk of Silicon Valley, which did not fall into as deep an economic funk
as other parts of the United States.

Yet other alarms about the state of the start-up economy fell on deaf
ears.

In 2015, as unicorn start-ups sucked in billions of dollars in funding
and soared to stratospheric valuations,
\href{https://www.nytimes3xbfgragh.onion/2017/03/18/technology/bill-gurley-uber-travis-kalanick-silicon-valley.html}{Mr.
Gurley of Benchmark} bemoaned ``the complete absence of fear'' in
Silicon Valley and said ``dead unicorns'' would soon appear. In 2016,
Jim Breyer, a venture capitalist who was an early Facebook investor,
also predicted ``blood in the water'' for the unicorns.

But the money continued to flood into tech start-ups from overseas
investors, private equity firms, corporations and SoftBank's behemoth
Vision Fund. That allowed founders to command higher valuations and
delay going public. By the end of 2018, start-ups in the United States
had raised a
\href{https://pitchbook.com/media/press-releases/us-venture-capital-investment-reached-1309-billion-in-2018-surpassing-dot-com-era}{record
\$131 billion in venture funding}, surpassing the amount collected
during the late 1990s dot-com boom, according to Pitchbook and the
National Venture Capital Association.

Mr. Gurley gave up on his warnings of excess. ``You have to adjust to
reality and play the game on the field,'' he
\href{https://www.nytimes3xbfgragh.onion/2018/08/14/technology/venture-capital-mega-round.html}{said}\href{https://www.nytimes3xbfgragh.onion/2018/08/14/technology/venture-capital-mega-round.html}{**}\href{https://www.nytimes3xbfgragh.onion/2018/08/14/technology/venture-capital-mega-round.html}{in
an
interview}\href{https://www.nytimes3xbfgragh.onion/2018/08/14/technology/venture-capital-mega-round.html}{**}\href{https://www.nytimes3xbfgragh.onion/2018/08/14/technology/venture-capital-mega-round.html}{last
year}.

(Complaining about high valuations is a longstanding pastime among
venture capitalists, of course, since most prefer to invest their money
in cheaply priced start-ups rather than expensive ones.)

Image

Uber and Lyft have been dealing with falling stock prices since they
went public this year.Credit...Sarahbeth Maney for The New York Times

This year, the warnings are being revived. In his recent blog post, Mr.
Wilson wrote that many of today's start-ups were focused on traditional
physical industries like real estate, exercise or transportation. They
should not command the high valuations that pure software companies ---
which tend to have less overhead --- have, he wrote.

In several message exchanges, Mr. Wilson said he had already seen that
as criticism of WeWork mounted over the last month, some start-up
fundings were taking place at lower valuations and with stricter terms
than the companies had hoped for.

``What I would like to see is a bit more rationality, and I'm hopeful we
are going to get it,'' he said.

By last week, his words appeared to be sinking in elsewhere.

At a start-up conference held by the tech publication TechCrunch at a
San Francisco convention center, around 10,000 founders, investors and
``innovators'' watched interviews with slightly more famous founders,
investors and ``innovators'' from a dark, cavernous room. Onstage,
entrepreneurs lamented the unforgiving stock market and challenging
investment environment.

Postmates, a food delivery start-up that confidentially filed to go
public in February, attended the confab. The company has not yet gone
public because the markets have been ``choppy when it comes to growth
companies,'' said Bastian Lehmann, Postmates' chief executive, at the
event.

Image

Bird was able to raise new funding only because it had taken steps to
shore up its losses, Mr. VanderZanden said.Credit...Coley Brown for The
New York Times

Bird, the scooter start-up, announced \$275 million in fresh funding at
the conference. But its chief executive, Mr. VanderZanden, said he had
been able to raise that money only because his unprofitable company had
taken steps this year to shore up its losses. Many scooter companies
have
\href{https://www.nytimes3xbfgragh.onion/2019/09/04/technology/san-diego-electric-scooters.html}{lost
their shine}this year because of regulatory pushback and safety issues.

The shift toward making a profit wasn't easy, Mr. VanderZanden said.
``I'm an ex-growth guy, and sometimes it's painful for me,'' he said.

But spending fast to grow fast was just no longer feasible, he added. It
is now difficult for ``a growth-at-all-costs company burning hundreds of
millions of dollars with negative unit economics'' to get funding, he
said. ``This is going to be a healthy reset for the tech industry.''

Advertisement

\protect\hyperlink{after-bottom}{Continue reading the main story}

\hypertarget{site-index}{%
\subsection{Site Index}\label{site-index}}

\hypertarget{site-information-navigation}{%
\subsection{Site Information
Navigation}\label{site-information-navigation}}

\begin{itemize}
\tightlist
\item
  \href{https://help.nytimes3xbfgragh.onion/hc/en-us/articles/115014792127-Copyright-notice}{©~2020~The
  New York Times Company}
\end{itemize}

\begin{itemize}
\tightlist
\item
  \href{https://www.nytco.com/}{NYTCo}
\item
  \href{https://help.nytimes3xbfgragh.onion/hc/en-us/articles/115015385887-Contact-Us}{Contact
  Us}
\item
  \href{https://www.nytco.com/careers/}{Work with us}
\item
  \href{https://nytmediakit.com/}{Advertise}
\item
  \href{http://www.tbrandstudio.com/}{T Brand Studio}
\item
  \href{https://www.nytimes3xbfgragh.onion/privacy/cookie-policy\#how-do-i-manage-trackers}{Your
  Ad Choices}
\item
  \href{https://www.nytimes3xbfgragh.onion/privacy}{Privacy}
\item
  \href{https://help.nytimes3xbfgragh.onion/hc/en-us/articles/115014893428-Terms-of-service}{Terms
  of Service}
\item
  \href{https://help.nytimes3xbfgragh.onion/hc/en-us/articles/115014893968-Terms-of-sale}{Terms
  of Sale}
\item
  \href{https://spiderbites.nytimes3xbfgragh.onion}{Site Map}
\item
  \href{https://help.nytimes3xbfgragh.onion/hc/en-us}{Help}
\item
  \href{https://www.nytimes3xbfgragh.onion/subscription?campaignId=37WXW}{Subscriptions}
\end{itemize}
