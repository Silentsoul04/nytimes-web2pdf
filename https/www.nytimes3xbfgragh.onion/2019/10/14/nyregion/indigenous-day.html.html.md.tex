Sections

SEARCH

\protect\hyperlink{site-content}{Skip to
content}\protect\hyperlink{site-index}{Skip to site index}

\href{https://www.nytimes3xbfgragh.onion/section/nyregion}{New York}

\href{https://myaccount.nytimes3xbfgragh.onion/auth/login?response_type=cookie\&client_id=vi}{}

\href{https://www.nytimes3xbfgragh.onion/section/todayspaper}{Today's
Paper}

\href{/section/nyregion}{New York}\textbar{}Indigenous Peoples' Day: The
Unofficial, Columbus-Free Celebration

\url{https://nyti.ms/2MeIibf}

\begin{itemize}
\item
\item
\item
\item
\item
\end{itemize}

Advertisement

\protect\hyperlink{after-top}{Continue reading the main story}

Supported by

\protect\hyperlink{after-sponsor}{Continue reading the main story}

\hypertarget{indigenous-peoples-day-the-unofficial-columbus-free-celebration}{%
\section{Indigenous Peoples' Day: The Unofficial, Columbus-Free
Celebration}\label{indigenous-peoples-day-the-unofficial-columbus-free-celebration}}

No governor. No mayor. But plenty of activism, food, dancing and fun in
New York City.

\includegraphics{https://static01.graylady3jvrrxbe.onion/images/2019/10/15/nyregion/15NY-INDIGENOUS1-print/NY-INDIGENOUS-articleLarge.jpg?quality=75\&auto=webp\&disable=upscale}

By Aaron Randle

\begin{itemize}
\item
  Oct. 14, 2019
\item
  \begin{itemize}
  \item
  \item
  \item
  \item
  \item
  \end{itemize}
\end{itemize}

Roughly five miles from the nation's largest Christopher Columbus
celebration, hundreds gathered Sunday and Monday on Randalls Island to
celebrate indigenous people at an event that has become part of a larger
conversation about how New York City should
\href{https://www.nytimes3xbfgragh.onion/2017/08/30/nyregion/ordering-review-of-statues-puts-de-blasio-in-tricky-spot.html?module=inline}{honor
controversial historical figures.}

Though other cities, including Los Angeles, Denver, Dallas, Phoenix and
Washington, have designated the second Monday in October as
\href{https://www.nytimes3xbfgragh.onion/2019/10/13/us/indigenous-peoples-day-columbus-day.html}{Indigenous
Peoples' Day}, New York has so far not followed suit. Mayor Bill de
Blasio and Gov. Andrew Cuomo marched along Fifth Avenue in Manhattan in
the 75th annual Columbus Day parade on Monday and did not make
appearances at the event honoring Native Americans.

``Politicians not showing up, that's something we're accustomed to.
We're the side New York seems to have forgotten about,'' said Cliff
Mattias, the founder of the \href{https://www.ipdnyc.org/}{Indigenous
Peoples Day New York City event}. ``But look around. There's an eclectic
mix of people here. Indigenous. Black people. Anglos, allies from around
the world. That makes it special.''

He added: ``This isn't a day about protesting Columbus, it's about
celebrating indigenous people.''

Image

Credit...Jeremy Dennis for The New York Times

Image

Credit...Jeremy Dennis for The New York Times

The two-day event, branded a ``communal gathering,'' was free, inviting
and festive. Attendees sat in a large circle (traditionally called a
``continuous hoop'') and were entertained by musicians, dancers and
speakers from around the world --- in addition to North America, there
were representatives from the Caribbean, Polynesian Islands and South
America.

\includegraphics{https://static01.graylady3jvrrxbe.onion/images/2019/10/14/nyregion/NY-INDIGENOUS-02/NY-INDIGENOUS-02-articleLarge.jpg?quality=75\&auto=webp\&disable=upscale}

Outside the circle, vendors and artisans sold incense, native clothing,
custom jewelry and more.

The longest lines were for the food --- traditional dishes like Native
American fry bread, venison, bison and maize.

Image

Credit...Jeremy Dennis for The New York Times

Activism was part of the celebration, too. Pua Case, a climate activist
from Hawaii, left a
\href{https://www.nytimes3xbfgragh.onion/2019/07/17/science/mauna-kea-protest.html?module=inline}{protest
against the building of a telescope on Mauna Kea} to attend. LaDonna
Brave Bull Allard, a Lakotan historian and principal founder of the
camps protesting the construction of the Dakota Access Pipeline, also
came.

Image

Credit...Jeremy Dennis for The New York Times

For some, supporting indigenous people also meant voicing disapproval at
honoring Columbus. After a review last year of statues in the city, Mr.
de Blasio announced that the statue of Columbus would remain in Midtown
Manhattan with the addition of ``markers'' explaining the history of the
period.

``Columbus Day is an insult and a slap to the face,'' said Serena Davis,
a black woman from Harlem. ``I came to pay homage to the true natives of
this country.''

Still, the overall tone of the event was one of positivity and
fellowship.

Anthony Vairacocha, a 25-year-old from Manhattan and member of the
Kichwa indigenous tribe, has been coming to the New York event since it
began in 2014. He has made lasting connections, he said. ``I've met
friends through this. Bonds have been formed here.''

Advertisement

\protect\hyperlink{after-bottom}{Continue reading the main story}

\hypertarget{site-index}{%
\subsection{Site Index}\label{site-index}}

\hypertarget{site-information-navigation}{%
\subsection{Site Information
Navigation}\label{site-information-navigation}}

\begin{itemize}
\tightlist
\item
  \href{https://help.nytimes3xbfgragh.onion/hc/en-us/articles/115014792127-Copyright-notice}{©~2020~The
  New York Times Company}
\end{itemize}

\begin{itemize}
\tightlist
\item
  \href{https://www.nytco.com/}{NYTCo}
\item
  \href{https://help.nytimes3xbfgragh.onion/hc/en-us/articles/115015385887-Contact-Us}{Contact
  Us}
\item
  \href{https://www.nytco.com/careers/}{Work with us}
\item
  \href{https://nytmediakit.com/}{Advertise}
\item
  \href{http://www.tbrandstudio.com/}{T Brand Studio}
\item
  \href{https://www.nytimes3xbfgragh.onion/privacy/cookie-policy\#how-do-i-manage-trackers}{Your
  Ad Choices}
\item
  \href{https://www.nytimes3xbfgragh.onion/privacy}{Privacy}
\item
  \href{https://help.nytimes3xbfgragh.onion/hc/en-us/articles/115014893428-Terms-of-service}{Terms
  of Service}
\item
  \href{https://help.nytimes3xbfgragh.onion/hc/en-us/articles/115014893968-Terms-of-sale}{Terms
  of Sale}
\item
  \href{https://spiderbites.nytimes3xbfgragh.onion}{Site Map}
\item
  \href{https://help.nytimes3xbfgragh.onion/hc/en-us}{Help}
\item
  \href{https://www.nytimes3xbfgragh.onion/subscription?campaignId=37WXW}{Subscriptions}
\end{itemize}
