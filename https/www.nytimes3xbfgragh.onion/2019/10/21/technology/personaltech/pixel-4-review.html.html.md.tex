Sections

SEARCH

\protect\hyperlink{site-content}{Skip to
content}\protect\hyperlink{site-index}{Skip to site index}

\href{https://www.nytimes3xbfgragh.onion/section/technology/personaltech}{Personal
Tech}

\href{https://myaccount.nytimes3xbfgragh.onion/auth/login?response_type=cookie\&client_id=vi}{}

\href{https://www.nytimes3xbfgragh.onion/section/todayspaper}{Today's
Paper}

\href{/section/technology/personaltech}{Personal Tech}\textbar{}Pixel 4
Review: Google Needs to Do More to Stand Out

\href{https://nyti.ms/2BwyVh0}{https://nyti.ms/2BwyVh0}

\begin{itemize}
\item
\item
\item
\item
\item
\item
\end{itemize}

Advertisement

\protect\hyperlink{after-top}{Continue reading the main story}

Supported by

\protect\hyperlink{after-sponsor}{Continue reading the main story}

Tech Fix

\hypertarget{pixel-4-review-google-needs-to-do-more-to-stand-out}{%
\section{Pixel 4 Review: Google Needs to Do More to Stand
Out}\label{pixel-4-review-google-needs-to-do-more-to-stand-out}}

The new Pixel 4 is a solid higher-end phone. But dollar for dollar, it
pales against the latest iPhones and Galaxy devices. Here's why.

\includegraphics{https://static01.graylady3jvrrxbe.onion/images/2019/10/21/business/21techfix1/merlin_162749409_76b80368-c5cf-48c2-835b-f37deda4b8e9-articleLarge.jpg?quality=75\&auto=webp\&disable=upscale}

\href{https://www.nytimes3xbfgragh.onion/by/brian-x-chen}{\includegraphics{https://static01.graylady3jvrrxbe.onion/images/2018/02/16/multimedia/author-brian-x-chen/author-brian-x-chen-thumbLarge.jpg}}

By \href{https://www.nytimes3xbfgragh.onion/by/brian-x-chen}{Brian X.
Chen}

\begin{itemize}
\item
  Oct. 21, 2019
\item
  \begin{itemize}
  \item
  \item
  \item
  \item
  \item
  \item
  \end{itemize}
\end{itemize}

Google's new Pixel 4 phones will reach stores this week.
\href{https://www.nytimes3xbfgragh.onion/2019/09/17/technology/personaltech/iphone-11-review.html?rref=collection\%2Fbyline\%2Fbrian-x.-chen\&action=click\&contentCollection=undefined\&region=stream\&module=stream_unit\&version=latest\&contentPlacement=9\&pgtype=collection}{Apple's
new iPhone 11s} debuted last month. The question is: Will you be able to
tell the difference between the smartphones?

It used to be easy to distinguish them. While the Pixel's hardware
features were never as impressive as those of other high-end devices,
Google stood out by leveraging its prowess in software and artificial
intelligence to meet --- and sometimes exceed --- its competitors in
areas like smartphone photography.

But this year, rivals have caught up with Google's camera software.
\href{https://www.nytimes3xbfgragh.onion/2019/09/17/technology/personaltech/iphone-11-review.html}{The
latest iPhones}, for instance, have added Google-esque capabilities like
a night mode for shooting better photos in the dark.

That's bad news for Google, underscoring some of the Pixel's weaknesses.

When evaluated in a vacuum,
\href{https://www.nytimes3xbfgragh.onion/2019/10/15/technology/personaltech/google-pixel-photography.html?rref=collection\%2Fbyline\%2Fbrian-x.-chen\&action=click\&contentCollection=undefined\&region=stream\&module=stream_unit\&version=latest\&contentPlacement=2\&pgtype=collection}{the
Pixel 4}, which comes in two screen sizes, is a solid all-around device.
It has a second camera lens, making Google's already excellent camera
system slightly more capable than last year's. It has incorporated an
iPhone-like face scanner and new software that mimics the swipe-gesture
controls for using an iPhone. The screen looks rich and bright. It's
also cool that Google's voice recorder can automatically transcribe
audio clips.

\includegraphics{https://static01.graylady3jvrrxbe.onion/images/2019/10/21/business/21techfix2/merlin_162756939_7b0b52c4-ab90-41d7-89cb-d0039930c471-articleLarge.jpg?quality=75\&auto=webp\&disable=upscale}

But Apple and Samsung phones now have triple-lens cameras, which are
more versatile for taking photographs. The iPhone's face scanner is also
more secure than the Pixel's.

So dollar for dollar, it's tough to recommend a Pixel 4, which costs
\$800 or \$900, when you can get a new iPhone for \$700 to \$1,100, or a
\href{https://www.nytimes3xbfgragh.onion/2019/02/27/technology/personaltech/samsung-galaxy-s10-review.html}{Samsung
Galaxy S10} for \$900.

The Pixel 4 is the best at one thing: integrating Google's software and
internet services into a mobile communications device. Unlike other
Android phones, Pixels aren't cluttered with clunky software and
confusing interfaces. But for most people, that won't be enough.

I tested a \$900 Pixel 4 XL side by side with Apple's new \$1,000 iPhone
11 Pro for about a week. Here's what I found.

\hypertarget{a-flawed-face-scanner}{%
\subsection{A flawed face scanner}\label{a-flawed-face-scanner}}

The most notable new feature on Google's Pixel is also the most flawed
part of the device.

Google decided to go all in on face-scanning as a way of unlocking the
Pixel 4. When you set the phone up, you scan a 3-D model of your face.
From there, whenever you pick up the device, it will unlock as soon as
it verifies your mug.

The face scanner is part of a new system that Google calls Motion Sense,
which is an array of sensors including infrared and depth-sensing
cameras and a miniature radar. The radar senses when someone reaches for
the phone and activates the infrared cameras so they can scan your face
in less than a second.

The problem? BBC News reported last week that the face scanner would
\href{https://www.bbc.com/news/technology-50085630}{unlock even with a
user's eyes closed}, which I confirmed in my tests. This is a major
security flaw. If you're asleep, all someone has to do to get access to
your personal data is take your phone and hold it up to your face. That
makes the face scanner, in some ways, a weaker security feature than a
fingerprint sensor.

Google said in a statement that it would release a software update in
coming months adding the option to require a person's eyes to be open
before unlocking the phone. In the meantime, the company said, people
could temporarily disable face unlock and use a PIN, pattern or password
instead.

In contrast, the iPhone requires that its owner look toward the screen
before it unlocks. Apple also claims that the likelihood of bypassing
its Face ID scanner with the incorrect face is
\href{https://support.apple.com/en-us/HT208108}{one in a million}.

I asked Google what the false-acceptance rate was for the Pixel 4's face
scanner. Google would say only that the biometric requirements exceeded
its \href{https://source.android.com/compatibility/cdd}{standards for
Android phones} in general.

For now, Google's face unlock looks like an unfinished security feature
that doesn't feel safe to use, and releasing it in this state suggests
that the search giant treats device security as an afterthought. This is
a bad look for Google in a time when many people are concerned about
their digital privacy.

\hypertarget{losing-its-edge-in-camera-software}{%
\subsection{Losing its edge in camera
software}\label{losing-its-edge-in-camera-software}}

To compare the Pixel 4 camera with the iPhone 11 Pro camera, I tested
them in a challenging nighttime environment: an outdoor Thom Yorke
concert at the Greek Theater in Berkeley, Calif.

Image

The Pixel 4, which uses computational photography to shoot photos in the
dark, produced impressive low-light photos. But in many shots at an
outdoor concert, photos looked overexposed.Credit...Brian X. Chen / The
New York Times

Both cameras produced decent photos in low light. But night shots taken
from the iPhone looked consistently better. The low-light pictures from
the Pixel 4 often looked overexposed compared with the iPhone's. This
was especially pronounced in a photo in which dozens of members of the
audience turned on their phone flashlights as a form of applause.

Image

Photos taken with the iPhone's night mode looked more natural than
similar night shots taken with the Pixel 4. Credit...Brian X. Chen / The
New York Times

To test the cameras in daytime, I took the phones to a dog park in San
Francisco. Both cameras excelled. Photos of my dogs looked crisp and
clear with pleasing colors.

But in several shots, the Pixel phone's portrait mode, which sharpens
the foreground and gently blurs the background, unintentionally created
an ugly digital mask around my corgi, Max. Google said this happened
when the camera's depth map didn't perfectly follow the outline of the
subject.

Image

The Pixel 4 camera's portrait mode sharpens a subject in the foreground
while gently blurring the background --- but it's not
perfect.Credit...Brian X. Chen / The New York Times

The iPhone 11 Pro wasn't perfect with portrait mode, either.
Occasionally it left parts of a subject blurred when they should have
been sharpened.

Still, I was more disappointed with the Pixel 4 camera in this area. The
phone is supposed to be better at portrait shots than last year's model
thanks to its second camera lens, but I didn't notice a marked
improvement.

Image

In some test shots, the Pixel 4's portrait mode left an artificial mask
around subjects.Credit...Brian X. Chen

Lastly, Google left out a feature that Apple and Samsung just introduced
on their phones: a so-called ultrawide lens. The company said it felt
that making the zoom ability of the camera better was more important. In
my tests, zoomed-in shots looked great on the Pixel 4.

But the ultrawide lens on other phone cameras is useful for taking
photos with a broader field of view in some situations, like a shot of
the Grand Canyon or a large group gathering for Thanksgiving dinner ---
an effect that the Pixel 4's software cannot replicate. The lack of this
special lens on the Pixel 4 is not a deal breaker, but it diminishes its
value.

\hypertarget{the-bottom-line}{%
\subsection{The bottom line}\label{the-bottom-line}}

The Pixel 4 has a few intriguing features, like the transcription
feature built directly into its voice recorder, which worked well in my
tests. The screen also has a higher refresh rate, which makes motion
look smoother.

But over all, these perks did not make up for the Pixel 4's weaknesses,
and I was disappointed that Google didn't do more to distinguish its
premium phone from competitors. With its rivals catching up on
sophisticated photo software, Google looks behind the curve in hardware.

Recently I revised my
\href{https://www.nytimes3xbfgragh.onion/2019/09/26/technology/personaltech/why-we-upgraded-our-reviews-approach-for-apples-iphone-11.html}{smartphone
upgrade criteria} to include advice on when people absolutely must jump
to a new phone. For those with older Pixel phones, I recommend taking a
wait-and-see approach before considering an upgrade. All the older
Pixels, including the original model from 2016, are still getting
software and security updates, so there is no rush to buy. The Pixel 4
will be a nice upgrade only if Google meaningfully strengthens the
security of the face scanner.

In the meantime, people who enjoy Google products have a good option:
the
\href{https://www.nytimes3xbfgragh.onion/2019/05/07/technology/personaltech/pixel-3a.html}{\$400
Pixel 3A}, the budget version of the phone that was released in May. It
lacks frills of higher-end phones like wireless charging and
waterproofing, but it includes Google's smart camera and a nice screen.
It's the best Android phone you can get at that price.

In other words, if Google doesn't step up its efforts in the premium
hardware market, the low end may become the only place where it stays
relevant.

Advertisement

\protect\hyperlink{after-bottom}{Continue reading the main story}

\hypertarget{site-index}{%
\subsection{Site Index}\label{site-index}}

\hypertarget{site-information-navigation}{%
\subsection{Site Information
Navigation}\label{site-information-navigation}}

\begin{itemize}
\tightlist
\item
  \href{https://help.nytimes3xbfgragh.onion/hc/en-us/articles/115014792127-Copyright-notice}{©~2020~The
  New York Times Company}
\end{itemize}

\begin{itemize}
\tightlist
\item
  \href{https://www.nytco.com/}{NYTCo}
\item
  \href{https://help.nytimes3xbfgragh.onion/hc/en-us/articles/115015385887-Contact-Us}{Contact
  Us}
\item
  \href{https://www.nytco.com/careers/}{Work with us}
\item
  \href{https://nytmediakit.com/}{Advertise}
\item
  \href{http://www.tbrandstudio.com/}{T Brand Studio}
\item
  \href{https://www.nytimes3xbfgragh.onion/privacy/cookie-policy\#how-do-i-manage-trackers}{Your
  Ad Choices}
\item
  \href{https://www.nytimes3xbfgragh.onion/privacy}{Privacy}
\item
  \href{https://help.nytimes3xbfgragh.onion/hc/en-us/articles/115014893428-Terms-of-service}{Terms
  of Service}
\item
  \href{https://help.nytimes3xbfgragh.onion/hc/en-us/articles/115014893968-Terms-of-sale}{Terms
  of Sale}
\item
  \href{https://spiderbites.nytimes3xbfgragh.onion}{Site Map}
\item
  \href{https://help.nytimes3xbfgragh.onion/hc/en-us}{Help}
\item
  \href{https://www.nytimes3xbfgragh.onion/subscription?campaignId=37WXW}{Subscriptions}
\end{itemize}
