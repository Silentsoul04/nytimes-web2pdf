Sections

SEARCH

\protect\hyperlink{site-content}{Skip to
content}\protect\hyperlink{site-index}{Skip to site index}

\href{https://www.nytimes3xbfgragh.onion/section/books}{Books}

\href{https://myaccount.nytimes3xbfgragh.onion/auth/login?response_type=cookie\&client_id=vi}{}

\href{https://www.nytimes3xbfgragh.onion/section/todayspaper}{Today's
Paper}

\href{/section/books}{Books}\textbar{}Jack Reacher Is Still Restless.
But His Creator Has Settled Down.

\url{https://nyti.ms/33Ud0fT}

\begin{itemize}
\item
\item
\item
\item
\item
\end{itemize}

Advertisement

\protect\hyperlink{after-top}{Continue reading the main story}

Supported by

\protect\hyperlink{after-sponsor}{Continue reading the main story}

\hypertarget{jack-reacher-is-still-restless-but-his-creator-has-settled-down}{%
\section{Jack Reacher Is Still Restless. But His Creator Has Settled
Down.}\label{jack-reacher-is-still-restless-but-his-creator-has-settled-down}}

\includegraphics{https://static01.graylady3jvrrxbe.onion/images/2019/10/21/books/00LeeChild3/00LeeChild3-articleLarge.jpg?quality=75\&auto=webp\&disable=upscale}

By \href{https://www.nytimes3xbfgragh.onion/by/janet-maslin}{Janet
Maslin}

\begin{itemize}
\item
  Published Oct. 23, 2019Updated Nov. 5, 2019
\item
  \begin{itemize}
  \item
  \item
  \item
  \item
  \item
  \end{itemize}
\end{itemize}

LARAMIE, Wyo. --- Two books ago, Lee Child's Jack Reacher passed through
the southeast corner of Wyoming in his efforts to track down the owner
of a pawned West Point ring. The book was ``The Midnight Line'' and it
was unusually transporting for a Reacher thriller. You could visualize
the immense flat expanses of old railroad land stretching toward the
foothills of the Rockies, the long miles of dirt road down which anyone
could disappear under a vast open sky.

Reacher moved on. He always does. But Child and Jane, his wife of 44
years, decided to stay.

Now, with a Reacher TV project in the works and the 24th novel in the
series on the way (titled ``Blue Moon,'' it will arrive on Child's 65th
birthday, Oct. 29), I am in Laramie to have coffee in a diner with
Reacher's creator. As any devotee knows, Reacher spends a lot of time in
diners. He downs amounts of coffee that would put most people on life
support. He sits with his back to the wall, eats like a trencherman and
gets acquainted with the waitress. He wants her to remember him, because
it might be handy.

But here in the real world, I'm meeting the reedy, 6-foot-4 Child ---
actual name, James Dover Grant --- and not his brawnier, inch-taller
hero. Laramie has no real diner. It's got a place that serves cappuccino
and arty beer that, Child confirms, Reacher would be ``bemused by.''
Sitting in a booth with his back to the wall, Child faces a rainbow
flag; Laramie is where Matthew Shepard was killed for being gay 21 years
ago, and this city of 32,000 continues to honor his memory. We're a
block away from two bookstores. Outside their doors, few people know
that a best-selling author has started spending three months of the year
nearby.

Child drinks a meager half cup of coffee, claiming to have had a whole
pot at home. He insists on paying the check, either out of gallantry or
for tax reasons. Then we climb into his distinctly un-Reacher-like
electric blue S.U.V. and begin a drive straight out of his novel. He
lives 40 or 50 miles from town. The intersection closest to his roost is
a 10-minute drive away. That's also where the paved road ends.

Image

The Jack Reacher in ``Blue Moon'' is older than previous incarnations.
He gets into a vicious gang war between Ukrainians and Albanians. ``I
wanted him completely out of his comfort zone,'' Child says.

The sky is as big as skies get. The high prairie is golden. Miles of
dirt road lead upward to an immaculate, rustic house with decks on three
sides. Unobstructed views stretch 20 miles into the distance. Child's
place is on 35 acres, protected by thousands of acres of forest, lakes
and ranch land. It all cost less than he got for the 900-square-foot
apartment on 22nd Street where he used to live. If you want to relocate
to the middle of nowhere, this is how it's done.

Child, an Englishman, has gone native. He's dressed in boots, jeans,
T-shirt and a leather barn coat. He owns two cowboy hats, but didn't
wear one for this interview ``for fear of making you laugh.'' He already
lived here when he wrote
``\href{https://www.nytimes3xbfgragh.onion/2017/11/08/books/review-midnight-line-jack-reacher-lee-child.html}{The
Midnight Line},'' and acknowledges that describing a familiar setting
was more satisfying than making one up. Though Wyoming's renown as a tax
haven was a factor in his move, he says, the decision had ``more to do
with an immigrant's sense that there's always somewhere else to
explore.'' (Child and his wife have numerous homes, including one above
St. Tropez and a spread in East Sussex, England, that he bought for
bragging rights after growing up poor. He still spends time at an
apartment he owns on Central Park West, but Jane has decided she's
through with New York.)

The Laramie area also happens to abut Colorado, where recreational
marijuana is legal. Child made waves when he talked about being a
regular user; his habit goes back 50 years. He finds it especially handy
for reading his work, claiming the high helps him judge his writing. And
he likes doing actors' voices: Tony Curtis, a pretty good Michael Caine.

As we approach the house for lunch, Child proudly points out a few
landmarks. The address number is printed in a clean font and hangs from
a sturdy post. He did that himself. The generator on the hillside got
them through all of last winter. To the right of the driveway sits what
Child calls his rockery. Since his hands, unlike Reacher's, aren't the
size of small animals, he's dexterous enough to treat this tiny garden
lovingly. So does Jane, who has arranged some low, heathery sprigs in a
small vase for lunchtime.

Their 39-year-old daughter, Ruth, who studied forensic linguistics and
will move to Fort Collins, Colo., later this year, has come by for the
occasion. She and her mother have made lunch. ``It was supposed to be
tuna niçoise, but we all like different things,'' Jane says; these three
are strong willed as well as close-knit. So lunch means a different kind
of tuna salad for each of them, and a highly entertaining debate about
apostrophe usage. Ruth was only about 7 when she went to a market with
her father and asked, ``Dad, shouldn't that sign say `10 items or
\emph{fewer}?'''

Jim and Jane, as they're known here, have made local friends. But
they're both voracious readers (she is a dedicated environmentalist) and
they're mostly home alone. The place is set up for that. Their work
areas are far apart, and he has lucked into the best room in the house
as his office: double height and mostly glass with a fireplace. It
should be the master bedroom. But the noisy furnace room below ruled
that out, so here are two desks face to face, one for Reacher projects
and the other for correspondence; plus high piles of books for Child's
recreational reading; the longest available leather sofa, which still
isn't long enough; and a golden trophy shaped like a pen nib, the
\href{https://www.thebookseller.com/british-book-awards/2019-author-year}{2019
British Book Award for author of the year}, on the otherwise bare
shelves.

\includegraphics{https://static01.graylady3jvrrxbe.onion/images/2019/10/21/books/00LeeChild1/merlin_161030019_dbe0990e-e75b-4e4d-9cc1-5348e504732f-articleLarge.jpg?quality=75\&auto=webp\&disable=upscale}

(Child will receive another award, the Commander of the British Empire,
in February, though his fierce objections to the British class system
make him reluctant to bow to the royal family member presenting it. What
if that turns out to be the queen? ``She \emph{is} tiny,'' Child says,
``so there'll be a certain amount of craning, which might pass for a
bow.'')

The study is also where Child is taking on a major new job: the Reacher
TV reboot, for Amazon. Earlier this decade, when Paramount made a couple
of
\href{https://www.nytimes3xbfgragh.onion/2012/12/21/movies/tom-cruise-in-jack-reacher.html}{Reacher
movies starring Tom Cruise}, Child knew better than to cede full control
of his popular character. He sold the rights to two films, with the
condition that he'd have to sign off on any others. After Reacher fans
\href{https://www.nytimes3xbfgragh.onion/2012/12/09/movies/tom-cruise-as-lee-childs-fictional-hero-in-jack-reacher.html}{complained
that Cruise was too small} to play the big bruiser, and Cruise coaxed
the first film's director, Christopher McQuarrie, to his ``Mission:
Impossible'' franchise, Paramount decided two Reacher films were enough.
And along came streaming.

Child settled on Amazon rather than Netflix or any other streaming
service in part because of the synergy it allowed. Every time he
publishes a book, Amazon delivers a lot of copies. The company is
willing to advertise the show on its book packaging, and email a link to
Child's readers. So he's begun work on what could be an eight-year
project.

The TV Reacher will be large, very large, and in his 30s or 40s. (The
lead hasn't been cast yet; an Army boot is waiting for the right Mr.
Cinderella.) The series will start where the books did. ``Killing
Floor,'' the first Reacher novel, will anchor Season 1, with other books
picked over for subplots. Child will supervise the screenplays but not
write them, and will help reinvent Reacher from the ground up. Or
Reachers, plural. The one in ``Blue Moon'' is older, hipper, richer and
hotter than previous incarnations. He doesn't wear Army surplus. He
likes black. He knows about performance art. And he gets into a
sustained sexual relationship as well as a vicious gang war between
Ukrainians and Albanians.

``I wanted him completely out of his comfort zone,'' Child says. ``He's
capable of getting comfortable pretty quick.''

From the looks of it, so is his maker.

Advertisement

\protect\hyperlink{after-bottom}{Continue reading the main story}

\hypertarget{site-index}{%
\subsection{Site Index}\label{site-index}}

\hypertarget{site-information-navigation}{%
\subsection{Site Information
Navigation}\label{site-information-navigation}}

\begin{itemize}
\tightlist
\item
  \href{https://help.nytimes3xbfgragh.onion/hc/en-us/articles/115014792127-Copyright-notice}{©~2020~The
  New York Times Company}
\end{itemize}

\begin{itemize}
\tightlist
\item
  \href{https://www.nytco.com/}{NYTCo}
\item
  \href{https://help.nytimes3xbfgragh.onion/hc/en-us/articles/115015385887-Contact-Us}{Contact
  Us}
\item
  \href{https://www.nytco.com/careers/}{Work with us}
\item
  \href{https://nytmediakit.com/}{Advertise}
\item
  \href{http://www.tbrandstudio.com/}{T Brand Studio}
\item
  \href{https://www.nytimes3xbfgragh.onion/privacy/cookie-policy\#how-do-i-manage-trackers}{Your
  Ad Choices}
\item
  \href{https://www.nytimes3xbfgragh.onion/privacy}{Privacy}
\item
  \href{https://help.nytimes3xbfgragh.onion/hc/en-us/articles/115014893428-Terms-of-service}{Terms
  of Service}
\item
  \href{https://help.nytimes3xbfgragh.onion/hc/en-us/articles/115014893968-Terms-of-sale}{Terms
  of Sale}
\item
  \href{https://spiderbites.nytimes3xbfgragh.onion}{Site Map}
\item
  \href{https://help.nytimes3xbfgragh.onion/hc/en-us}{Help}
\item
  \href{https://www.nytimes3xbfgragh.onion/subscription?campaignId=37WXW}{Subscriptions}
\end{itemize}
