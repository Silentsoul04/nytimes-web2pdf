Sections

SEARCH

\protect\hyperlink{site-content}{Skip to
content}\protect\hyperlink{site-index}{Skip to site index}

\href{https://www.nytimes3xbfgragh.onion/section/technology}{Technology}

\href{https://myaccount.nytimes3xbfgragh.onion/auth/login?response_type=cookie\&client_id=vi}{}

\href{https://www.nytimes3xbfgragh.onion/section/todayspaper}{Today's
Paper}

\href{/section/technology}{Technology}\textbar{}Microsoft Wins
Pentagon's \$10 Billion JEDI Contract, Thwarting Amazon

\url{https://nyti.ms/2BLDJzn}

\begin{itemize}
\item
\item
\item
\item
\item
\end{itemize}

Advertisement

\protect\hyperlink{after-top}{Continue reading the main story}

Supported by

\protect\hyperlink{after-sponsor}{Continue reading the main story}

\hypertarget{microsoft-wins-pentagons-10-billion-jedi-contract-thwarting-amazon}{%
\section{Microsoft Wins Pentagon's \$10 Billion JEDI Contract, Thwarting
Amazon}\label{microsoft-wins-pentagons-10-billion-jedi-contract-thwarting-amazon}}

\includegraphics{https://static01.graylady3jvrrxbe.onion/images/2019/10/25/business/25jedi/25jedi-articleLarge-v2.jpg?quality=75\&auto=webp\&disable=upscale}

By \href{https://www.nytimes3xbfgragh.onion/by/kate-conger}{Kate
Conger},
\href{https://www.nytimes3xbfgragh.onion/by/david-e-sanger}{David E.
Sanger} and
\href{https://www.nytimes3xbfgragh.onion/by/scott-shane}{Scott Shane}

\begin{itemize}
\item
  Published Oct. 25, 2019Updated Dec. 9, 2019
\item
  \begin{itemize}
  \item
  \item
  \item
  \item
  \item
  \end{itemize}
\end{itemize}

SAN FRANCISCO --- The Department of Defense on Friday
\href{https://www.defense.gov/Newsroom/Contracts/Contract/Article/1999639/}{awarded
a \$10 billion technology contract} to Microsoft over Amazon in a
contest that was closely watched after
\href{https://www.nytimes3xbfgragh.onion/2019/07/18/us/politics/trump-amazon-defense-department-contract.html}{President
Trump ramped up his criticism of Amazon's founder, Jeff Bezos, and said
he might intervene}.

\emph{{[}Update:}
\href{https://www.nytimes3xbfgragh.onion/2019/12/09/technology/amazon-pentagon-contract-trump.html}{\emph{Amazon
accuses Trump of ``improper pressure'' on JEDI contract}}\emph{.{]}}

The 10-year contract for the Joint Enterprise Defense Infrastructure,
known as JEDI, had set off a showdown among Amazon, Microsoft, IBM,
Oracle and Google for the right to transform the military's cloud
computing systems. The acrimonious process involved intense lobbying
efforts and legal challenges among the rivals.

The contract has an outsize importance because it is central to the
Pentagon's efforts to modernize its technology. Much of the military
operates on 1980s and 1990s computer systems, and the Defense Department
has spent billions of dollars trying to make them talk to one another.

The decision was a surprise because
\href{https://www.nytimes3xbfgragh.onion/2019/08/02/us/politics/amazon-pentagon-contract-trump.html}{Amazon
had been considered the front-runner}, in part because it had built
cloud services for the Central Intelligence Agency. But that was before
Mr. Trump became publicly hostile to Mr. Bezos, who also owns The
Washington Post. The president often refers to the newspaper as the
``Amazon Washington Post'' and has accused it of spreading ``fake
news.''

In public, Mr. Trump said there were other ``great companies'' that
should have a chance at the contract. But a speechwriter for former
Defense Secretary Jim Mattis says
\href{https://www.washingtonpost.com/national-security/2019/10/23/syria-disagreement-with-trump-used-pretext-mattiss-departure-pentagon-chief-new-book-says/}{in
a book} scheduled for publication next week that Mr. Trump had wanted to
foil Amazon and give the contract to another company.

The issue quickly became radioactive at the Pentagon. The new defense
secretary, Mark T. Esper, at first said he wanted to
\href{https://www.nytimes3xbfgragh.onion/2019/08/01/us/politics/amazon-pentagon-contract.html}{take
several months to review the issue} and then, a few days ago, recused
himself from the bidding. He said he could not participate because his
son worked for IBM, one of the competitors for the contract.

As recently as this month, the betting was that Microsoft would, at
most, get only part of the contract and that the Pentagon would use
multiple suppliers for its cloud services, as do many private companies.
Microsoft was considered in the lead for other government cloud
programs, including an intelligence contract; only recently has
Microsoft opened enough classified server facilities to be able to
handle data on the scale of the Pentagon contract.

``The acquisition process was conducted in accordance with applicable
laws and regulations,'' the Defense Department said in a statement on
Friday. ``All offerors were treated fairly and evaluated consistently
with the solicitation's stated evaluation criteria.''

Microsoft did not immediately have a comment. Amazon, which calls its
cloud platform Amazon Web Services, or AWS, said in a statement that it
was surprised by the decision.

``AWS is the clear leader in cloud computing, and a detailed assessment
purely on the comparative offerings clearly led to a different
conclusion,'' Drew Herdener, a spokesman for Amazon, said. ``We remain
deeply committed to continuing to innovate for the new digital
battlefield where security, efficiency, resiliency and scalability of
resources can be the difference between success and failure.''

The award to Microsoft is likely to fuel suspicions that Mr. Trump may
have weighed in privately as well as publicly against Amazon. Experts on
federal contracting said it would be highly improper for a president to
intervene in the awarding of a contract.

Price Floyd, a former head of public affairs at the Pentagon who
consulted briefly for Amazon, said he thought Mr. Trump's vocal
criticism of Amazon would give it ample grounds to protest the award to
Microsoft.

``He's the commander in chief, and he hasn't been subtle about his
hostility toward Amazon,'' Mr. Floyd said.

Microsoft's win has implications for the cloud computing industry, in
which businesses rent space on technology companies' server computers,
giving them cheap and fast access to storage and processing. Amazon has
long been the dominant player, with about 45 percent of the market,
trailed by Microsoft with around 25 percent, said Daniel Ives, an
analyst for Wedbush Securities who has closely followed the JEDI saga.

Landing the JEDI contract puts Microsoft in a prime position to earn the
roughly \$40 billion that the federal government is expected to spend on
cloud computing over the next several years, he said.

Losing the bid is also a hit to the reputation of Amazon, which decided
last year to open
\href{https://www.nytimes3xbfgragh.onion/2018/11/13/business/national-landing-amazon-va.html}{a
large outpost in Northern Virginia} that will eventually employ at least
25,000 people.

Unifying information in the cloud has obvious benefits for the Pentagon
as the military moves to greater use of remote sensors, semiautonomous
weapons and, ultimately, artificial intelligence. It is particularly
crucial now that United States Cyber Command has been elevated to the
equivalent of Central Command, which runs operations in the Middle East,
or the Northern Command, which defends the continental United States.

But some critics of the process argued that such a large contract should
not be awarded to a single company, while proponents said using only one
provider would eliminate glitches in military systems and streamline
communications.

The initial reaction on Friday from some lawmakers was positive, mostly
because the long-delayed contract had finally been issued.

Representative Jim Langevin, a Rhode Island Democrat who has immersed
himself in cyber issues, suggested the military was finally catching up
with private industry.

``Advanced general-purpose cloud is the industry norm, and it's past
time the Department of Defense had access to these capabilities,'' said
Mr. Langevin, the chairman of the Armed Services Subcommittee on
Intelligence and Emerging Threats and Capabilities. ``I look forward to
continuing to use my position in Congress to increase access to
next-generation technologies that support our war fighters.''

But Senator Mark Warner, a Democrat of Virginia,
\href{https://twitter.com/MarkWarner/status/1157341554346008576}{said on
Twitter} that it was ``important that we maintain a fair \& competitive
process'' and that ``for the President to use the power of his office to
punish critics in the media would be a complete abuse of power.''

Amazon, Microsoft, IBM, Oracle and Google began battling for the JEDI
contract more than a year ago.
\href{https://www.bloomberg.com/news/articles/2018-10-08/google-drops-out-of-pentagon-s-10-billion-cloud-competition}{Google
dropped out} last October without submitting a formal bid, saying the
military work conflicted with its corporate principles, which preclude
the use of artificial intelligence in weaponry.

The Pentagon said in April that only
\href{https://www.nytimes3xbfgragh.onion/2019/04/10/technology/amazon-microsoft-jedi-pentagon.html}{Amazon
and Microsoft met its technical requirements}for fulfilling the
contract. In
\href{https://www.nytimes3xbfgragh.onion/2019/03/20/technology/military-contract-deap-ubhi.html?module=inline}{an
unsuccessful legal challenge, Oracle} alleged that Amazon had biased the
process in its favor by hiring Defense Department employees to work on
the bidding process.

In August, the Defense Department's inspector general announced that it
had assembled a team to review the JEDI process. But while that was
underway, Mr. Trump raised his objections. The process froze, and
Pentagon officials said time was being wasted --- which would ultimately
put the United States at a military disadvantage.

``In 20 years of covering tech, I've never seen a battle for any type of
contract reach this level of nastiness,'' Mr. Ives said.

He said he saw the ferocity of the contest mainly as a response to
Amazon's enormous success as the pioneer of cloud computing, which is
now the foundation of much of the digital infrastructure of private
industry. He said Amazon's revenue from federal government contracts,
about \$200 million in 2014, had reached \$2 billion this year, much of
it from the C.I.A. and other intelligence agencies.

Advertisement

\protect\hyperlink{after-bottom}{Continue reading the main story}

\hypertarget{site-index}{%
\subsection{Site Index}\label{site-index}}

\hypertarget{site-information-navigation}{%
\subsection{Site Information
Navigation}\label{site-information-navigation}}

\begin{itemize}
\tightlist
\item
  \href{https://help.nytimes3xbfgragh.onion/hc/en-us/articles/115014792127-Copyright-notice}{©~2020~The
  New York Times Company}
\end{itemize}

\begin{itemize}
\tightlist
\item
  \href{https://www.nytco.com/}{NYTCo}
\item
  \href{https://help.nytimes3xbfgragh.onion/hc/en-us/articles/115015385887-Contact-Us}{Contact
  Us}
\item
  \href{https://www.nytco.com/careers/}{Work with us}
\item
  \href{https://nytmediakit.com/}{Advertise}
\item
  \href{http://www.tbrandstudio.com/}{T Brand Studio}
\item
  \href{https://www.nytimes3xbfgragh.onion/privacy/cookie-policy\#how-do-i-manage-trackers}{Your
  Ad Choices}
\item
  \href{https://www.nytimes3xbfgragh.onion/privacy}{Privacy}
\item
  \href{https://help.nytimes3xbfgragh.onion/hc/en-us/articles/115014893428-Terms-of-service}{Terms
  of Service}
\item
  \href{https://help.nytimes3xbfgragh.onion/hc/en-us/articles/115014893968-Terms-of-sale}{Terms
  of Sale}
\item
  \href{https://spiderbites.nytimes3xbfgragh.onion}{Site Map}
\item
  \href{https://help.nytimes3xbfgragh.onion/hc/en-us}{Help}
\item
  \href{https://www.nytimes3xbfgragh.onion/subscription?campaignId=37WXW}{Subscriptions}
\end{itemize}
