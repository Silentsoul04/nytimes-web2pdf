Sections

SEARCH

\protect\hyperlink{site-content}{Skip to
content}\protect\hyperlink{site-index}{Skip to site index}

\href{https://www.nytimes3xbfgragh.onion/section/business}{Business}

\href{https://myaccount.nytimes3xbfgragh.onion/auth/login?response_type=cookie\&client_id=vi}{}

\href{https://www.nytimes3xbfgragh.onion/section/todayspaper}{Today's
Paper}

\href{/section/business}{Business}\textbar{}In China, Working Mothers
Say They Are Fired or Sidelined

\url{https://nyti.ms/36n6kJl}

\begin{itemize}
\item
\item
\item
\item
\item
\end{itemize}

Advertisement

\protect\hyperlink{after-top}{Continue reading the main story}

Supported by

\protect\hyperlink{after-sponsor}{Continue reading the main story}

\hypertarget{in-china-working-mothers-say-they-are-fired-or-sidelined}{%
\section{In China, Working Mothers Say They Are Fired or
Sidelined}\label{in-china-working-mothers-say-they-are-fired-or-sidelined}}

Women who become pregnant say that employers discriminate against them
in defiance of labor laws, and go largely unpunished.

\includegraphics{https://static01.graylady3jvrrxbe.onion/images/2019/10/31/business/00chinamaternity-sub/00chinamaternity-sub-articleLarge.jpg?quality=75\&auto=webp\&disable=upscale}

By
\href{https://www.nytimes3xbfgragh.onion/by/alexandra-stevenson}{Alexandra
Stevenson} and
\href{https://www.nytimes3xbfgragh.onion/by/elsie-chen}{Elsie Chen}

\begin{itemize}
\item
  Nov. 1, 2019
\item
  \begin{itemize}
  \item
  \item
  \item
  \item
  \item
  \end{itemize}
\end{itemize}

\href{https://cn.nytimes3xbfgragh.onion/china/20191104/china-mothers-discrimination-working/}{阅读简体中文版}\href{https://cn.nytimes3xbfgragh.onion/china/20191104/china-mothers-discrimination-working/zh-hant/}{閱讀繁體中文版}

The women's accusations briefly lit up the Chinese internet. Three
employees of a big Chinese logistics company said their bosses had fired
them or cut their pay after they became pregnant.

In one **** case, the company, China Railway Logistics, did not send a
representative to an arbitration proceeding. The company then went to
court to challenge the award the woman received. But it backed off as
media attention intensified and because the stakes were so small: The
award was less than \$5,000.

``It isn't worth it to waste time on this little money,'' the
\href{http://www.bjnews.com.cn/news/2017/12/05/467293.html}{company's
lawyer told a local reporter for a
state-controlled}\href{http://www.bjnews.com.cn/news/2017/12/05/467293.html}{newspaper}.

China's leaders are
\href{https://www.nytimes3xbfgragh.onion/2018/08/11/world/asia/china-one-child-policy-birthrate.html}{encouraging
women to have it all}: a career and more babies. In reality, Chinese
women are filing lawsuits and pursuing arbitration against employers who
they claim are ignoring laws meant to keep mothers in the workplace.

\href{https://www.nytimes3xbfgragh.onion/newsletters/parenting?module=inline}{\emph{{[}The
topics new parents are talking about. Evidence-based guidance. Personal
stories that matter. Sign up now to get NYT Parenting in your inbox
every week.{]}}}

China has
\href{https://www.nytimes3xbfgragh.onion/2019/02/21/world/china-gender-discrimination-workplace.html}{laws
against gender and pregnancy discrimination} and a maternity policy that
ensures 98 days of paid leave, but enforcement is scant. Women who seek
a legal remedy for mistreatment often win a pittance in compensation.
They worry that a fight --- any fight --- will ruin their chances to
find other work. And they fear retribution if they speak publicly.

In the case of China Railway Logistics, a company based in Beijing that
operates 5,000 warehouses, the three women said they had were either
expecting or nursing when they lost their jobs. They turned their cases
into a rare coordinated **** action and were identified in court papers
by their first initials and last names. None of them want to talk about
it now.

But other women, like Li Xiaoping, want to speak up. She was fired by an
electronics company, she said, after disclosing that she was pregnant.
``Should a woman just go back to fulfilling her traditional role as a
wife and be shut out of society after giving birth?'' she said.

Ms. Li, 33, said that her former employer**,** Qingdao Keyrin
Electronics**,** billed her \$18,430, the equivalent of 52 months of pay
for what it said was ``great economic loss.'' She said that the company
had locked her out.

Ms. Li prevailed in arbitration but lost when the company appealed in
court. She is fighting that and another legal dispute: the company has
sued Ms. Li, her husband, a Chinese reporter and a newspaper that wrote
about her case for defamation, seeking \$1 million in damages.

Li Wei, the chief executive of Keyrin, disputed Ms. Li's story, saying
that it was ``completely inconsistent with the facts'' and that she did
not show up for work. ``In the last three years, 11 female employees
have had babies,'' Mr. Li said. ``This is the best evidence that we
treat our female employees well.''

It might look like the time has never been better for working mothers in
China. Faced with an aging population, officials
\href{https://www.nytimes3xbfgragh.onion/2015/11/14/world/asia/china-one-child-policy-loneliest-generation.html}{have
abolished} the country's one-child policy. Some local authorities have
extended maternity leaves or considered tax breaks to encourage women to
have a second child.

But experts see a paradox in China's approach. The country, the world's
second-largest economy after the United States, needs more children
because of its shrinking population and says it wants women to work, but
it offers few incentives for working mothers. Female participation in
the labor force has fallen since the 1990s, and
\href{http://www3.weforum.org/docs/WEF_GGGR_2018.pdf}{the pay gap
between women and men has widened}.

China, like the United States, does not subsidize maternity leave.
Companies tend to equate leave with lost revenue: They get nothing when
mothers take time off. Legal complaints filed by mothers show that local
authorities did not investigate reports of bias or firings, although the
country itself seems to be rethinking some workplace standards.

This year, the Supreme People's Court created a legal category called
employment equality that allowed women to report gender bias. A small
number of pregnant women are using that to make claims.

\includegraphics{https://static01.graylady3jvrrxbe.onion/images/2019/11/02/business/00chinamaternity-2/merlin_154986729_4e722df0-1937-4696-919d-7e0c4e894bdc-articleLarge.jpg?quality=75\&auto=webp\&disable=upscale}

``There are laws but there is no enforcement,'' said Yaqiu Wang, who has
researched pregnancy bias for Human Rights Watch. ``If you go to labor
arbitration, and you look at the composition of the committee: One is a
representative from the company, one is a representative from the labor
union, the third is someone from the government.''

The discrimination often begins with the job application.

It is illegal in China for employers to ask a woman about her marital
status or family plans, but many companies still do. Some
\href{https://www.nytimes3xbfgragh.onion/2019/07/16/world/asia/china-women-discrimination.html}{even
force new employee}s to sign agreements not to get pregnant. Some women,
like Liu Yang, are reconsidering the fairness of earlier demotions.

Ms. Liu, 42, worked for eight years selling sanitary napkins and
maternity products for Beishute, a company in Jinan. She was a deputy
manager when she had her first child, in 2014, and she was demoted when
she returned, she said. Her monthly pay was cut from \$1,120 to \$980.

Ms. Liu did not view it as discrimination at the time.

``I just internalized it,'' she said. ``I realize now that the company
treated me unfairly, but I didn't know much about the law.''

She said that other pregnant women were demoted, and that one was fired.
When she got pregnant again, Beishute cut her pay to \$630, she said.
She did not receive maternity pay. She was fired within a week of
returning, with the company citing unsatisfactory work.

``I felt helpless and mistreated,'' Ms. Liu said. She and her husband
support two children and two grandparents.

She pursued arbitration, seeking \$105,000 in unpaid wages, lost
overtime and damages. She was awarded \$25,000. She is appealing in
court. Her husband, Wang Guanghui, is as determined as Ms. Liu to make
the company do what they think is right.

Image

Huang Sha, a lawyer who represented women in a case against China
Railway Logistics, said that Chinese companies had little incentive to
change.Credit...Yuyang Liu for The New York Times

``Society has laws and rules, and they cannot bully us like that,'' Mr.
Wang said.

Beishute would not respond to questions about the case. Emails to the
company went unanswered. A representative of the legal department said
in a brief phone interview that the company had not fired Ms. Liu. He
declined to elaborate and gave only a last name, Peng.

In the 2017 case involving China Railway Logistics, a customer service
manager said that her pay had been cut after she had a baby. Identified
as Y.W. Wu in court filings, she sought \$5,600 in compensation. She won
by default after no one from the company attended arbitration. China
Railway Logistics appealed in court, citing financial losses and
restructuring as the reason her pay had been cut. The company, which
employs 50,000 people, did not respond to requests for comment.

The three women won about \$17,000 in total compensation. Huang Sha,
their lawyer, said that such small payouts gave companies little
incentive to obey the law.

``Women's rights should not just be propaganda,'' he said. ``The
government wants to protect the company, but it also wants to encourage
women to have more kids. There is a contradiction here.''

Image

Fan Huiling got pregnant at 41. When she asked to take a few days off on
her doctor's recommendation, she said, her boss told her not to
return.Credit...Yuyang Liu for The New York Times

Fan Huiling, a school security guard, was excited when she discovered,
at 41, that she was pregnant for a second time. But when she asked her
employer, Zhuhai Yingli Property Management, for a few days off based on
a doctor's recommendation, she was told not to return. Her belongings
were left in a pile outside the school.

She panicked. ``Without this job,'' she said, ``I don't have any life
security.''

Ms. Fan sought advice online and visited one government office after
another looking for help. She eventually filed for arbitration. The next
day her doctor told her she had had a miscarriage. She sued, citing the
new employment equality law category.

In a brief phone interview, a Zhuhai Yingli manager disputed Ms. Fan's
accusations**,** saying that the company had other pregnant employees.
``We don't want to comment on this anymore,'' the manager said,
declining to give her full name.

In October, a court ruled in Ms. Fan's favor, requiring the company to
pay her \$2,000 --- the equivalent of four months' pay --- and issue a
formal apology.

For all of the women seeking help for pregnancy bias, there are some who
stay silent at work. They share their stories on the internet, and keep
looking for employers who follow the law.

When Li Ronghua returned to Suning Consumer Finance, an arm of the
online retailer, after giving birth, she said she was criticized after
leaving her desk to pump milk twice a day. In a meeting, executives
cited her as an example of a bad employee for leaving without
permission. The head of human resources later told her that she should
quit, Ms. Li said. **** The company did not respond to emails and phone
calls.

Ms. Li, 36, has since found a job at a small financial leasing company.
But in job interviews, she said, she was routinely asked if she planned
to have a second child.

``The weak cannot win against the strong,'' she said. ``I just accept
it, gradually.'' ****

Advertisement

\protect\hyperlink{after-bottom}{Continue reading the main story}

\hypertarget{site-index}{%
\subsection{Site Index}\label{site-index}}

\hypertarget{site-information-navigation}{%
\subsection{Site Information
Navigation}\label{site-information-navigation}}

\begin{itemize}
\tightlist
\item
  \href{https://help.nytimes3xbfgragh.onion/hc/en-us/articles/115014792127-Copyright-notice}{©~2020~The
  New York Times Company}
\end{itemize}

\begin{itemize}
\tightlist
\item
  \href{https://www.nytco.com/}{NYTCo}
\item
  \href{https://help.nytimes3xbfgragh.onion/hc/en-us/articles/115015385887-Contact-Us}{Contact
  Us}
\item
  \href{https://www.nytco.com/careers/}{Work with us}
\item
  \href{https://nytmediakit.com/}{Advertise}
\item
  \href{http://www.tbrandstudio.com/}{T Brand Studio}
\item
  \href{https://www.nytimes3xbfgragh.onion/privacy/cookie-policy\#how-do-i-manage-trackers}{Your
  Ad Choices}
\item
  \href{https://www.nytimes3xbfgragh.onion/privacy}{Privacy}
\item
  \href{https://help.nytimes3xbfgragh.onion/hc/en-us/articles/115014893428-Terms-of-service}{Terms
  of Service}
\item
  \href{https://help.nytimes3xbfgragh.onion/hc/en-us/articles/115014893968-Terms-of-sale}{Terms
  of Sale}
\item
  \href{https://spiderbites.nytimes3xbfgragh.onion}{Site Map}
\item
  \href{https://help.nytimes3xbfgragh.onion/hc/en-us}{Help}
\item
  \href{https://www.nytimes3xbfgragh.onion/subscription?campaignId=37WXW}{Subscriptions}
\end{itemize}
