Sections

SEARCH

\protect\hyperlink{site-content}{Skip to
content}\protect\hyperlink{site-index}{Skip to site index}

\href{https://www.nytimes3xbfgragh.onion/section/technology}{Technology}

\href{https://myaccount.nytimes3xbfgragh.onion/auth/login?response_type=cookie\&client_id=vi}{}

\href{https://www.nytimes3xbfgragh.onion/section/todayspaper}{Today's
Paper}

\href{/section/technology}{Technology}\textbar{}TikTok Said to Be Under
National Security Review

\url{https://nyti.ms/2qb9Gi4}

\begin{itemize}
\item
\item
\item
\item
\item
\item
\end{itemize}

Advertisement

\protect\hyperlink{after-top}{Continue reading the main story}

Supported by

\protect\hyperlink{after-sponsor}{Continue reading the main story}

\hypertarget{tiktok-said-to-be-under-national-security-review}{%
\section{TikTok Said to Be Under National Security
Review}\label{tiktok-said-to-be-under-national-security-review}}

The review comes after lawmakers raised concerns about TikTok's growing
influence in the United States.

\includegraphics{https://static01.graylady3jvrrxbe.onion/images/2019/11/01/business/01tiktok1/01tiktok1-articleLarge.jpg?quality=75\&auto=webp\&disable=upscale}

\href{https://www.nytimes3xbfgragh.onion/by/jack-nicas}{\includegraphics{https://static01.graylady3jvrrxbe.onion/images/2018/11/26/multimedia/author-jack-nicas/author-jack-nicas-thumbLarge.png}}\href{https://www.nytimes3xbfgragh.onion/by/mike-isaac}{\includegraphics{https://static01.graylady3jvrrxbe.onion/images/2018/02/16/multimedia/author-mike-isaac/author-mike-isaac-thumbLarge.jpg}}\href{https://www.nytimes3xbfgragh.onion/by/ana-swanson}{\includegraphics{https://static01.graylady3jvrrxbe.onion/images/2018/12/10/multimedia/author-ana-swanson/author-ana-swanson-thumbLarge.png}}

By \href{https://www.nytimes3xbfgragh.onion/by/jack-nicas}{Jack Nicas},
\href{https://www.nytimes3xbfgragh.onion/by/mike-isaac}{Mike Isaac} and
\href{https://www.nytimes3xbfgragh.onion/by/ana-swanson}{Ana Swanson}

\begin{itemize}
\item
  Nov. 1, 2019
\item
  \begin{itemize}
  \item
  \item
  \item
  \item
  \item
  \item
  \end{itemize}
\end{itemize}

\href{https://cn.nytimes3xbfgragh.onion/technology/20191104/tiktok-national-security-review/}{阅读简体中文版}\href{https://cn.nytimes3xbfgragh.onion/technology/20191104/tiktok-national-security-review/zh-hant/}{閱讀繁體中文版}

The United States government has opened a national security review of a
Chinese company's acquisition of the American company that became
TikTok, the hugely popular short-form video app, according to people
briefed on the inquiry.

The Committee on Foreign Investment in the United States, a federal
panel that reviews foreign acquisitions of American firms on
national-security grounds, is now reviewing the two-year-old deal after
\href{https://www.rubio.senate.gov/public/_cache/files/9ba023e4-2f4b-404a-a8c0-e87ea784f440/FCEFFE1F54F3899795B4E5F1F1804630.20191009-letter-to-secretary-mnuchin-re-tiktok.pdf}{lawmakers
raised concerns} about TikTok's growing influence in the United States,
said the people, who spoke on the condition of anonymity because the
investigation was confidential. One of the people said that the American
government had evidence of the app sending data to China.

The move is the latest in a back and forth between the United States and
China, which are enmeshed in a global competition for technological
dominance that has begun to cleave the high-tech world in two and start
what some analysts refer to as a new Cold War.

\href{https://www.nytimes3xbfgragh.onion/2018/10/29/technology/bytedance-app-funding-china.html}{ByteDance},
a seven-year-old company based in Beijing,
\href{https://www.nytimes3xbfgragh.onion/2017/11/10/business/dealbook/musically-sold-app-video.html}{acquired
Musical.ly in November 2017 for \$800 million to \$1 billion}. At the
time, Musical.ly, an app popular with teenagers to make homemade karaoke
videos, had about 60 million users in the United States and Europe.
ByteDance said it would keep Musical.ly separate from its family of
Chinese apps. Less than a year later, ByteDance
\href{https://www.theverge.com/2018/8/2/17644260/musically-rebrand-tiktok-bytedance-douyin}{merged
Musical.ly with its similar service}, called TikTok, and the result has
since become one of world's fastest-growing apps and
\href{https://www.nytimes3xbfgragh.onion/interactive/2019/10/10/arts/TIK-TOK.html}{a
global cultural phenomenon}.

Over the past 12 months, TikTok's app has been downloaded more than 750
million times, more than Facebook, Instagram, YouTube and Snapchat,
according to the research firm Sensor Tower.

``While we cannot comment on ongoing regulatory processes, TikTok has
made clear that we have no higher priority than earning the trust of
users and regulators in the U.S.,'' a ByteDance spokesman said in an
email. ``Part of that effort includes working with Congress, and we are
committed to doing so.'' TikTok does not send any user data to China, he
added.

Reuters
\href{https://www.reuters.com/article/us-tiktok-cfius-exclusive/exclusive-u-s-opens-national-security-investigation-into-tiktok-sources-idUSKBN1XB4IL}{earlier
reported the review} by the federal panel, known as CFIUS, of the
Musical.ly acquisition.

China blocks many foreign companies from openly existing online in the
country, but Chinese companies that have developed cutting-edge
technologies are growing more popular around the world. Many lawmakers
and Trump administration officials see the trend as a threat to American
national security and the economy, and they have set up numerous
barriers to stop Chinese firms from acquiring American data and
technology.

The Trump administration
\href{https://www.nytimes3xbfgragh.onion/2018/03/12/technology/trump-broadcom-qualcomm-merger.html}{prevented
the Singapore-based Broadcom} from purchasing Qualcomm, an American
chip-maker, and quashed deals like
\href{https://www.nytimes3xbfgragh.onion/2018/01/02/business/moneygram-ant-financial-china-cfius.html}{Ant
Financial's bid for Moneygram}.

President Trump has also
\href{https://www.nytimes3xbfgragh.onion/2019/05/16/technology/huawei-ban-president-trump.html}{placed
Huawei}
\href{https://www.nytimes3xbfgragh.onion/2019/10/07/us/politics/us-to-blacklist-28-chinese-entities-over-abuses-in-xinjiang.html}{and
other Chinese tech firms} on a blacklist that blocks them from
purchasing American products over national security and human rights
concerns. He has also
\href{https://www.nytimes3xbfgragh.onion/2019/10/08/us/politics/trump-trade-war-imf.html}{imposed
tariffs on more than \$360 billion of Chinese products} in a trade war
that was at least partly in response to Chinese theft of American
intellectual property.

U.S. government officials have been particularly alarmed by the
implications of China's 2017 national intelligence law, which contains
sweeping language that requires companies to comply with intelligence
gathering operations, if asked. Chinese officials have pushed back
against these assertions, saying that companies should comply with local
laws while abroad.

On TikTok, users create and share short, inventive videos and bizarre
memes, an endless scroll of clips that has been called
``\href{https://www.newyorker.com/magazine/2019/09/30/how-tiktok-holds-our-attention}{the
last sunny corner on the internet}.'' The vast majority of its footage
comes from Western users, in large part because TikTok isn't available
in mainland China; ByteDance instead offers a highly similar service
there called Douyin.

But TikTok's Chinese connections and growing popularity in the United
States have drawn new concern in Washington after news reports
highlighted that there were
\href{https://www.washingtonpost.com/technology/2019/09/15/tiktoks-beijing-roots-fuel-censorship-suspicion-it-builds-huge-us-audience/}{few
signs of the Hong Kong protests on the app} and that TikTok moderators
were
\href{https://www.theguardian.com/technology/2019/sep/25/revealed-how-tiktok-censors-videos-that-do-not-please-beijing}{instructed
to censor videos that featured a number of political themes}.

ByteDance has said that the Chinese government does not order it to
censor content on TikTok. The spokesman said that the app's content
policies are led by a team in the United States and are not influenced
by any government.

``To be clear: we do not remove videos based on the presence of Hong
Kong protest content,'' a ByteDance spokesman said.

A former content moderator for TikTok said in an interview that managers
in the United States had instructed moderators to hide videos that
included any political messages or themes, not just those related to
China. The moderator spoke on the condition of anonymity because he
didn't want to speak publicly about a former employer while seeking
another job in the tech industry.

The moderator said that the policy was to allow such political posts to
remain on users' profile pages but to prevent them from being shared
more widely in TikTok's main video feed. The person said that while
moderators were told to censor racy videos, such as those featuring
scantily clad women, in Muslim countries, he never received specific
instructions to censor content related to China.

The ByteDance spokesman said the company had recently changed the policy
restricting political videos and called the previous rules a blunt
approach intended to keep the app fun.

Last month, Republican Senator Marco Rubio of Florida
\href{https://www.rubio.senate.gov/public/_cache/files/9ba023e4-2f4b-404a-a8c0-e87ea784f440/FCEFFE1F54F3899795B4E5F1F1804630.20191009-letter-to-secretary-mnuchin-re-tiktok.pdf}{sent
a letter} to Treasury Secretary Steven Mnuchin, who leads CFIUS, urging
him to open a national security review of ByteDance's purchase of
Muscial.ly.

``There continues to be ample and growing evidence that TikTok's
platform for Western markets, including those in the U.S., is censoring
content that is not in line with the Chinese Government and Communist
Party directives,'' Mr. Rubio wrote.

Chuck Schumer, Democrat of New York and the Senate minority leader, said
the security review was ``validation of our concern that apps like
TikTok --- that store massive amounts of personal data accessible to
foreign governments --- may pose serious risks to millions of
Americans.''

Derek Scissors, a resident scholar at the American Enterprise Institute,
a conservative think tank, said that new guidelines passed by Congress
last year about personal data clearly indicate that CFIUS should review
the purchase. He said the companies would likely have to take steps to
convince the administration that American data would be secure.

``It's a large transaction, it's a tech transaction, and it does involve
American users transferring data to a Chinese firm,'' Mr. Scissors said.
``I don't see how they get away with not taking some sort of mitigation
here.''

The ByteDance spokesman said the company hired a consulting firm in
Colorado called Special Counsel to analyze TikTok's app to understand
where it sent user data.

Data about TikTok users, including their videos, names, dates of birth
and other information, was stored exclusively on computer servers in
Virginia and Singapore, said Douglas Brush, who led the analysis for
Special Counsel. He added that in the analysis from July to October,
which included interviews with TikTok employees and a review of the
app's underlying computer code, his team found no way TikTok could send
data to China during those months.

ByteDance has tried to build its relationships in Washington amid the
growing scrutiny. TikTok has joined NetChoice, a trade association that
has been aggressive in pushing back on critics of tech companies. One of
Bytedance's own staff members registered to lobby for the company this
summer. The company also hired the powerful corporate law firm Covington
\& Burling --- whose clients include Facebook, among others --- to
advocate on its behalf.

In total, ByteDance spent \$120,000 on its federal lobbying operation
last quarter, according to a public disclosure posted last week.

TikTok
\href{https://newsroom.tiktok.com/en-us/our-commitment-to-our-users-and-the-tik-tok-experience}{announced}
last month that it was working with the law firm K\&L Gates on its
moderation policies, and it
\href{https://www.fosi.org/about/press/fosi-welcomes-tiktok/}{joined a
nonprofit} organization focused on children's online safety. In
February, ByteDance agreed to pay a \$5.7 million fine to the Federal
Trade Commission to settle accusations that Musical.ly illegally
collected information on users under 13.

``We've just gotten started,'' said Bart Gordon, the K\&L Gates partner
working with TikTok, on Friday.

\emph{David McCabe contributed reporting.}

Advertisement

\protect\hyperlink{after-bottom}{Continue reading the main story}

\hypertarget{site-index}{%
\subsection{Site Index}\label{site-index}}

\hypertarget{site-information-navigation}{%
\subsection{Site Information
Navigation}\label{site-information-navigation}}

\begin{itemize}
\tightlist
\item
  \href{https://help.nytimes3xbfgragh.onion/hc/en-us/articles/115014792127-Copyright-notice}{©~2020~The
  New York Times Company}
\end{itemize}

\begin{itemize}
\tightlist
\item
  \href{https://www.nytco.com/}{NYTCo}
\item
  \href{https://help.nytimes3xbfgragh.onion/hc/en-us/articles/115015385887-Contact-Us}{Contact
  Us}
\item
  \href{https://www.nytco.com/careers/}{Work with us}
\item
  \href{https://nytmediakit.com/}{Advertise}
\item
  \href{http://www.tbrandstudio.com/}{T Brand Studio}
\item
  \href{https://www.nytimes3xbfgragh.onion/privacy/cookie-policy\#how-do-i-manage-trackers}{Your
  Ad Choices}
\item
  \href{https://www.nytimes3xbfgragh.onion/privacy}{Privacy}
\item
  \href{https://help.nytimes3xbfgragh.onion/hc/en-us/articles/115014893428-Terms-of-service}{Terms
  of Service}
\item
  \href{https://help.nytimes3xbfgragh.onion/hc/en-us/articles/115014893968-Terms-of-sale}{Terms
  of Sale}
\item
  \href{https://spiderbites.nytimes3xbfgragh.onion}{Site Map}
\item
  \href{https://help.nytimes3xbfgragh.onion/hc/en-us}{Help}
\item
  \href{https://www.nytimes3xbfgragh.onion/subscription?campaignId=37WXW}{Subscriptions}
\end{itemize}
