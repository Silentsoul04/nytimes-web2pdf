\href{/section/business}{Business}\textbar{}Adam Neumann and the Art of
Failing Up

\url{https://nyti.ms/2NxjDym}

\begin{itemize}
\item
\item
\item
\item
\item
\end{itemize}

\includegraphics{https://static01.graylady3jvrrxbe.onion/images/2019/10/31/business/00NEUMANN-illo/merlin_163626369_e2ac3573-e318-45d0-a2a2-6a9ba428303c-articleLarge.jpg?quality=75\&auto=webp\&disable=upscale}

Sections

\protect\hyperlink{site-content}{Skip to
content}\protect\hyperlink{site-index}{Skip to site index}

\hypertarget{adam-neumann-and-the-art-of-failing-up}{%
\section{Adam Neumann and the Art of Failing
Up}\label{adam-neumann-and-the-art-of-failing-up}}

WeWork's chief risk-taker found a kindred spirit with an open checkbook:
SoftBank's Masayoshi Son. Now he's walking away from the wreckage with
more than \$1 billion.

Credit...Illustration by Nigel Buchanan

Supported by

\protect\hyperlink{after-sponsor}{Continue reading the main story}

By \href{https://www.nytimes3xbfgragh.onion/by/amy-chozick}{Amy Chozick}

\begin{itemize}
\item
  Published Nov. 2, 2019Updated May 18, 2020
\item
  \begin{itemize}
  \item
  \item
  \item
  \item
  \item
  \end{itemize}
\end{itemize}

Adam Neumann stood on the 57th floor of the Woolworth Building, the
neo-Gothic skyscraper that was once the tallest in the world. It was
late on a Friday night in 2013, and the
\href{https://www.nytimes3xbfgragh.onion/2020/05/18/business/wework-rent-coronavirus.html}{WeWork}founder
and chief executive had just made a move to add the top 30 floors to his
rapidly expanding real estate dealings.

Mr. Neumann and three employees had already enjoyed a few drinks when he
decided to bring them to tour his latest coup. In the gutted-out space,
they tossed beer bottles into empty elevator shafts, listening to them
clink on the way down. Then, Mr. Neumann told them all to follow him out
to the ledge. No guardrails. No enclosures. Just four inebriated
start-up executives teetering on the edge of death.

``I was up there with him on the top of the world, and he said,
`Everything is going to be amazing,''' recalled Harrison Weber, WeWork's
editorial director at the time.

Then, Mr. Neumann picked up an old beer bottle --- a remnant,
apparently, from some previous bender. He asked the employees to drink
the rank liquid. Everyone took a swig, except Mr. Weber. ``It felt like
a loyalty thing,'' he said. ``In that moment, I felt what a deeply
persuasive person he is.''

The last 80 days have seen an implosion unlike any other in the history
of start-ups. WeWork filed for an initial public offering with a
prospectus that was quickly ridiculed for its incoherence; investors
learned of several red-flag financial arrangements by Mr. Neumann; the
company's valuation plummeted; Mr. Neumann was forced to resign; and the
I.P.O. was withdrawn. Once estimated to be worth \$47 billion, WeWork
was reduced to \$7 billion, after a rescue by the Japanese giant
SoftBank.

But WeWork's astonishing downfall came with an even more astonishing
exit package for Mr. Neumann: The 40-year-old could receive
\href{https://www.nytimes3xbfgragh.onion/2019/10/22/business/dealbook/wework-softbank.html?module=inline}{more
than \$1 billion} after selling his shares to SoftBank and collecting a
\$185 million consulting fee. As the scope of the disaster comes into
focus, the question on everyone's mind --- from his co-working customers
to Wall Streeters to
\href{https://www.nytimes3xbfgragh.onion/2019/11/08/business/wework-employees-letter.html}{soon-to-be-laid-off
WeWork employees} --- is how Mr. Neumann managed to fail up so
spectacularly.

The answer has a lot to do with what Mr. Weber glimpsed atop the
Woolworth Building --- an inexplicably persuasive charisma and a taste
for risk. But Mr. Neumann, who grew up in Israel, also had an uncanny
ability to read people, from potential investors to reporters, gain
their loyalty and then sell them on his vision of a ``capitalist
kibbutz'' on a global scale. He benefited from a frenetic, nonstop
energy, and silly as it may sound, there's no question that Mr.
Neumann's good hair and looks helped his cause. At 6 feet 5, he had a
physical presence that could dominate a room. (Through a spokeswoman, he
declined to comment.)

Crucially, Mr. Neumann was selling to an eager audience at the right
time: WeWork's rebranding of the office as an expansion of one's
personality made sense to a generation of the intermittently employed.
If you were inclined to believe his vision of a world where work and
play bled into one, you might have grouped WeWork with other start-ups
--- like Uber and Lyft --- that were unprofitable at the moment but
would surely figure out the economics in time.

Mr. Neumann would talk eloquently about creating the first ``physical
social network,'' a place where members could talk about jobs, family,
love. ``It was like, wait, you mean life. What you're talking about is
just regular life,'' Mr. Weber said. But as Mr. Neumann framed things,
it sounded revolutionary. As more people bought into his vision,
WeWork's value kept soaring. It may have never reached the stratosphere,
though, if Mr. Neumann had not found the perfect benefactor: SoftBank's
chief executive, Masayoshi Son.

Like Mr. Neumann, Mr. Son --- known as Masa --- quotes Yoda (``feel the
force''), trusts his instincts and tries to think centuries into the
future. At \$100 billion, SoftBank's Vision Fund is the world's largest
technology investment fund, flush with cash from Saudi Arabia and Abu
Dhabi. Some of its gigantic bets, including one on Uber, have
\href{https://www.nytimes3xbfgragh.onion/2019/09/26/business/softbank-wework-masayoshi-son.html}{performed
poorly}, but Mr. Son has rejected the idea that he was putting too much
money into an already overvalued start-up cycle. ``Those who are calling
the current environment a `bubble' and `dangerous' are those who do not
understand technology,'' Mr. Son told Japan's Nikkei news service in
July.

Famously, in 2017, Mr. Neumann spent just
\href{http://nymag.com/intelligencer/2019/06/wework-adam-neumann.html}{12
minutes} walking Mr. Son around WeWork's headquarters, prompting an
investment of \$4.4 billion. Afterward, an elated Mr. Neumann zoomed
uptown in the back seat of his chauffeured white Maybach, blaring rap,
with an iPad open to a rendering of the hasty digital spit-swear he'd
just made with Mr. Son.

\includegraphics{https://static01.graylady3jvrrxbe.onion/images/2017/01/29/podcasts/the-daily-album-art/the-daily-album-art-articleInline-v2.jpg?quality=75\&auto=webp\&disable=upscale}

\hypertarget{listen-to-the-daily-the-spectacular-rise-and-fall-of-wework}{%
\subsubsection{Listen to `The Daily': The Spectacular Rise and Fall of
WeWork}\label{listen-to-the-daily-the-spectacular-rise-and-fall-of-wework}}

He sold a utopian vision that ultimately left a company in shambles.
Then he walked away with more than \$1 billion.

transcript

Back to The Daily

bars

0:00/25:54

-25:54

transcript

\hypertarget{listen-to-the-daily-the-spectacular-rise-and-fall-of-wework-1}{%
\subsection{Listen to `The Daily': The Spectacular Rise and Fall of
WeWork}\label{listen-to-the-daily-the-spectacular-rise-and-fall-of-wework-1}}

\hypertarget{hosted-by-michael-barbaro-produced-by-adizah-eghan-and-austin-mitchell-and-edited-by-mj-davis-lin-and-lisa-tobin}{%
\subsubsection{Hosted by Michael Barbaro, produced by Adizah Eghan and
Austin Mitchell, and edited by M.J. Davis Lin and Lisa
Tobin}\label{hosted-by-michael-barbaro-produced-by-adizah-eghan-and-austin-mitchell-and-edited-by-mj-davis-lin-and-lisa-tobin}}

\hypertarget{he-sold-a-utopian-vision-that-ultimately-left-a-company-in-shambles-then-he-walked-away-with-more-than-1-billion}{%
\paragraph{He sold a utopian vision that ultimately left a company in
shambles. Then he walked away with more than \$1
billion.}\label{he-sold-a-utopian-vision-that-ultimately-left-a-company-in-shambles-then-he-walked-away-with-more-than-1-billion}}

\begin{itemize}
\item
  michael barbaro\\
  From The New York Times, I'm Michael Barbaro. This is ``The Daily.''

  Today: It was the most valuable start-up in the United States, with
  plans to revolutionize how and where people around the world worked.
  Amy Chozick on the spectacular rise and fall of WeWork and the story
  of the man behind it all.

  It's Monday, November 18.

  Amy, I wonder if you could read this letter that WeWork employees sent
  to their bosses.
\item
  amy chozick\\
  Sure. Here's what they wrote. ``To the We Company Management Team.
  WeWork's company values encourage us to be entrepreneurial, inspired,
  authentic, tenacious, grateful and together. Today, we are embracing
  these qualities wholeheartedly as we band together to ensure the
  well-being of our peers. Thousands of us will be laid off in the
  coming weeks, but we want our time here to have meant something. We
  don't want to be defined by the scandals, the corruption and the greed
  exhibited by the company's leadership. We want to leave behind a
  legacy that represents the true character and intentions of WeWork
  employees. In the immediate term, we want those being laid off to be
  provided fair and reasonable separation terms commensurate with their
  contributions, including severance pay, continuation of company-paid
  health insurance and compensation for lost equity.''
\item
  michael barbaro\\
  It's a pretty grim letter.
\item
  amy chozick\\
  It is.
\item
  michael barbaro\\
  So how did WeWork get to this point? What's the story here?
\item
  archived recording\\
  Co-founder and C.E.O. of WeWork, Adam Neumann.
\end{itemize}

amy chozick

So the story of WeWork really starts with a man named Adam Neumann.

\begin{itemize}
\tightlist
\item
  archived recording (adam neumann)\\
  Growing up in Israel, watching American television and movies, I
  believed that the American dream is get a degree, get a great job,
  have lots of fun, make lots of money.
\end{itemize}

amy chozick

He was born in Israel. He moved to New York.

\begin{itemize}
\tightlist
\item
  archived recording (adam neumann)\\
  I majored in entrepreneurship and marketing.
\end{itemize}

amy chozick

And he started all of these sort of fly-by-night entrepreneurial ideas.
One of them was ---

\begin{itemize}
\tightlist
\item
  archived recording (adam neumann)\\
  This genius idea was going to be a women's high-heel shoe with a
  collapsible heel.
\end{itemize}

amy chozick

--- women's high heels with collapsible heels. Another was ---

michael barbaro

For storage.

amy chozick

No, because it's uncomfortable, Michael, to walk around in heels all
day. So you collapse them when you're going down subway stairs.

\begin{itemize}
\tightlist
\item
  archived recording (adam neumann)\\
  And this is definitely not my passion. And I came up with my second
  great idea --- Krawlers, with a K. Krawlers was baby pants with
  kneepads on them to protect the babies' knees for the crawling age.
\end{itemize}

amy chozick

Knee pads for crawling babies.

michael barbaro

Hmm.

amy chozick

The slogan was ---

\begin{itemize}
\tightlist
\item
  archived recording (adam neumann)\\
  Just because they don't tell you doesn't mean they don't hurt.
\end{itemize}

michael barbaro

{[}LAUGHS{]}

\begin{itemize}
\tightlist
\item
  archived recording (adam neumann)\\
  Of course, the business was a tremendous failure.
\end{itemize}

amy chozick

So he settles on co-working.

\begin{itemize}
\item
  archived recording (adam neumann)\\
  And in 2010, started WeWork. {[}APPLAUSE{]}
\item
  archived recording\\
  WeWork is a leader in the business of renting out spaces to
  entrepreneurs. The company ---
\end{itemize}

amy chozick

And it was a very specific time to be in the co-working business. There
were a lot of people who had a lot of start-up ideas, Silicon Valley was
booming, and you needed a place to work outside your parents' basement,
right? So ---

\begin{itemize}
\tightlist
\item
  archived recording\\
  It lets people rent out a desk or a private office equipped with
  amenities like internet, coffee and spacious common areas. WeWork also
  ---
\end{itemize}

amy chozick

--- they were sleek. They had community space, sofas all over the place
for team-building. There was cold brew, kombucha, taco Tuesdays. There
was beer and wine on tap. Instead of going out for a drink, you'd stay
at the office. It really came to symbolize the kind of start-up
entrepreneurial hustle of millennials.

michael barbaro

In other words, this kind of space was perfectly timed for a generation
of people who didn't see themselves as office workers, but as individual
businesses and entrepreneurs who needed a place to do that.

amy chozick

Exactly.

\begin{itemize}
\tightlist
\item
  archived recording (adam neumann)\\
  WeWork is the office space of tomorrow. The future is about light,
  innovation, creativity. It's going from me to we. We give you space
  that will inspire you, uplift you and help you innovate the products
  of tomorrow.
\end{itemize}

amy chozick

So in Adam's mind, this was a community company. His mission was to
elevate the world's consciousness. He would talk about this. So he grew
up partly on a kibbutz in Israel, and he would talk about it as a
capitalist kibbutz. It was this mix of, like, capitalistic hustle with
this yearning for communal meaning. Adam Neumann would say that he
wanted WeWork to create a world where people don't just make a living,
they make a life.

michael barbaro

And they make a life in these WeWork offices.

amy chozick

Exactly. The thing Adam Neumann is really best at is communicating his
vision, convincing people that he can change the way we work and live.
He fully adopted the persona of the iconoclast start-up founder, and he
looked the part. You know, at 6 foot 5, with flowing brown hair, you
know, people looked up to him. When he walked into a meeting ---

michael barbaro

Literally.

amy chozick

--- people paid, literally, paid attention to him. But the most
important person who bought into his vision is a Japanese executive, the
head of a company called SoftBank ---

\begin{itemize}
\tightlist
\item
  archived recording (masayoshi son)\\
  The thing is, you know, I have a vision.
\end{itemize}

amy chozick

--- called Masayoshi Son. Everyone calls him Masa.

\begin{itemize}
\tightlist
\item
  archived recording (masayoshi son)\\
  We go and change the world together.
\end{itemize}

amy chozick

Masa's company oversees something called the Vision Fund.

\begin{itemize}
\tightlist
\item
  archived recording (masayoshi son)\\
  --- thinking about what is the future? What is the --- how we can
  change the life of people for the better humanity.
\end{itemize}

amy chozick

This is the largest tech investment fund in the world. They have \$100
billion dollars to play with, largely from the Saudis who want to
diversify away from oil and invest in tech. And Masa is this very
interesting character who's always trusted his gut.

\begin{itemize}
\item
  archived recording\\
  One of the investments you made is considered by many people to be the
  most successful investment in the history of mankind. You invested
  roughly \$20 million in Alibaba. And at the time it went public, it
  was worth roughly \$90 billion. What is it that made you feel this was
  worth putting in \$20 million.
\item
  archived recording (masayoshi son)\\
  Well, he had no business plan.

  But his eyes was very strong --- strong eyes, strong, shining eyes. I
  could tell.
\end{itemize}

amy chozick

And so he meets Adam Neumann. Adam Neumann gives Masa a tour of his
offices. This includes the right soundtrack in the background and all
the sleek office space and virtual-reality renderings of WeWork office
space in the future. So you could wear these glasses and feel like
you're standing in Tokyo or Shanghai, right down to looking out the
street and seeing the scene that you would see from the offices. And
this really blows Masa's mind. He loves this vision. He loves Adam's
energy.

\begin{itemize}
\tightlist
\item
  archived recording\\
  You know, both WeWork and SoftBank Vision Fund are shoot-the-moon
  operations, so ---
\end{itemize}

amy chozick

So after a 12-minute tour of the WeWork offices and Adam Newman's vision
---

\begin{itemize}
\tightlist
\item
  archived recording\\
  --- basically both willing to make humongous bets, and they're sort of
  enabling one another.
\end{itemize}

amy chozick

--- Masa Son invests over \$4 billion in WeWork.

\begin{itemize}
\item
  archived recording (speaker 1)\\
  And by the way, I work in a WeWork, so WeWork's good.
\item
  archived recording (speaker 2)\\
  {[}LAUGHS{]} I hear they have good drinks on tap, David. Look ---
\end{itemize}

amy chozick

This is an enormous investment. And he doesn't say, Adam, I need you to
be a very careful steward of this extremely important investment. Be
careful with my money. Instead he says, I need you to go crazier. I need
you to do more. I need you to explore your wildest visions.

michael barbaro

He says that?

amy chozick

He says, go as far as you can with this. And there's more money where
this came from.

michael barbaro

So just how wild does Adam Neumann go with this \$4 billion investment?

\begin{itemize}
\tightlist
\item
  archived recording\\
  So we're at WeWork in Austin, Texas, right now.
\end{itemize}

amy chozick

He started opening WeWorks all over the country.

\begin{itemize}
\tightlist
\item
  archived recording\\
  I'm in my new office at the Crystal City WeWork that just opened up
  today.
\end{itemize}

amy chozick

Every major American city had a WeWork.

\begin{itemize}
\tightlist
\item
  archived recording\\
  --- my new office space, which is in the WeWork in Long Beach,
  California.
\end{itemize}

amy chozick

And then he expanded massively abroad.

michael barbaro

I feel like I remember this moment, because I woke up one day and had a
long walk around Manhattan, and there was a WeWork at every single turn.

amy chozick

The Lord \& Taylor building. Absolutely, they became ubiquitous. WeWork
now is 45 million square feet of real estate. It's the largest private
landlord in New York, Washington, and London.

michael barbaro

Wow.

amy chozick

And then, giant companies started also moving their employees into
WeWork offices, thinking it's a draw for employees to work in these
spaces.

michael barbaro

Like which companies?

amy chozick

Verizon, Salesforce, IBM all moved a lot of employees into WeWork office
spaces.

michael barbaro

Hmm.

amy chozick

And this is the point when Adam really leans into Masa's advice, which
is to go wild. Pursue your craziest dreams. He opens a WeLive apartment
building and wants to expand that. He says these are places that will
have community and cut down on the suicide rate because people never
feel alone. He was talking about WeGrow, and he and his wife opened a
school in downtown Manhattan. There was WeBank, there was WeSail. There
was WeSleep. There was talk of an airline.

michael barbaro

WeFly?

amy chozick

WeFly, presumably. There was talk of WeMars, even putting office space
on the red planet.

michael barbaro

That really is wild.

amy chozick

It was wild. So Adam Neumann becomes fantastically rich. He also
indulges his eccentricities. I mean, he was known to walk around the
office barefoot. But now, he's installed a private plunge pool in his
office --- a cold plunge, an infrared sauna in his office. He has a
white Maybach. He's, like, blaring hip-hop as this chauffeured white
Maybach takes him all over Manhattan. He also convinced the company to
buy a \$60-million private plane, which he and other executives hotbox.

michael barbaro

I'm sorry?

amy chozick

That's, you know, getting high in a confined space with the marijuana
smoke filling the cabin. Tequila --- Adam Neumann loved his Don Julio
tequila. He would even get people who didn't seem the tequila type, like
Jared Kushner, to take tequila shots at 9:00 in the morning while they
were scoping out some real estate in Philly.

michael barbaro

Wow.

amy chozick

But employees said that there was just free-flowing booze all the time,
a lot of marijuana, a lot of these summer camp kind of weekend retreats.
People would get drunk, and they'd dance around a fire singing to
Journey ---

{[}music{]}

amy chozick

--- and other sort of ``Animal House'' antics, but always infused with
this, like, larger purpose. Adam would get on stage with, like, Deepak
Chopra and address employees. And it was this culture of WeWork that was
very specific to Adam's vision for the company.

michael barbaro

So Amy, as Adam Neumann becomes this larger-than-life character, as
WeWork is opening offices all over the United States, all over the
world, and as this term ``We'' starts to get applied to all areas of
life, are people starting to question whether this is getting just a
little too big and whether it's all kind of adding up?

amy chozick

By and large, people looked at what Adam had accomplished, WeWork on
every corner, as you said, taking over iconic buildings, and they
believed in what he was selling. And then on top of that, you throw in
Masa, this Japanese tycoon, who had made a fortune investing early on in
these founders --- Alibaba, Yahoo, Uber --- just based on gut instincts.
And if he believed in Adam, why shouldn't everyone else?

michael barbaro

Mm-hmm.

amy chozick

So WeWork does what successful start-ups do. They prepare to go public,
meaning they can sell their shares to the public. There's a valuation on
the company of \$47 billion. That makes it the most valuable start-up in
the country. And part of going public, they have to disclose everything
in paperwork. And that's when things start to unravel.

{[}music{]}

michael barbaro

We'll be right back.

\begin{itemize}
\tightlist
\item
  archived recording\\
  WeWork just unveiling its I.P.O. filing.
\end{itemize}

michael barbaro

So why did things start to unravel when WeWork files all this paperwork
to go public?

amy chozick

So this is the first time that journalists, investors, the public can
really look under the hood at what's going on at WeWork ---

\begin{itemize}
\tightlist
\item
  archived recording\\
  Do we have any numbers in terms of how much money they're making or
  losing right now?
\end{itemize}

amy chozick

--- and it's not pretty.

\begin{itemize}
\tightlist
\item
  archived recording\\
  Net losses were just under a billion dollars, \$904 million.
\end{itemize}

amy chozick

The company had lost \$900 million in the first half of 2019 alone.

michael barbaro

Wow.

amy chozick

They had rapidly expanded into markets that weren't necessarily friendly
to WeWork. And there had also been some kind of questionable financial
dealings between Adam and the company that he started.

michael barbaro

Like what?

amy chozick

Well, he had trademarked the word ``We'' and sold it back to the company
for \$5.9 million dollars. He ended up returning that money, but that
certainly raised eyebrows. He had bought a lot of the buildings that
WeWork was now leasing, so the company had paid him millions of dollars
for space in the offices that he owned.

michael barbaro

So this is all very unconventional and potentially even self-dealing.

amy chozick

Right. But then there's the text of the I.P.O., the legal document that
they filed with the S.E.C. The word ``community'' is listed more than
150 times. And there was all of this kind of propping up what we now
know is a very unprofitable business with this lingo of self-help and
creating community. And people just started to see it as selling air,
you know, couching this unprofitable business in language that made
people feel good. So this paperwork filed with the S.E.C. really sets
off an implosion unlike any other in start-up history.

\begin{itemize}
\item
  archived recording 1\\
  All the bad press and the bad moves caused the company's expected
  valuation to drop from \$47 billion down to just \$15 billion, so ---
\item
  archived recording 2\\
  It could come in as low as \$10 billion, according to ---
\item
  archived recording 3\\
  It may value the office-sharing company below \$8 billion dollars, and
  that is a fraction of the \$47 billion ---
\end{itemize}

amy chozick

The value of WeWork plummets. So this is when discussion starts to
spread of, well, should there even be an I.P.O.? Is this company ready
to go public? And people start to turn on Adam pretty quickly at this
point.

\begin{itemize}
\item
  archived recording 1\\
  WeWork C.E.O. Adam Neumann is facing new pressure from some of his top
  investors following the company's decision to postpone the I.P.O.
\item
  archived recording 2\\
  Some board members and large investors in the company are privately
  discussing whether they could replace C.E.O. Adam Neumann and how they
  would do it.
\end{itemize}

amy chozick

They think he should be removed. He should not be the C.E.O. of WeWork.
Not only should he not be the C.E.O., but he should have no involvement
in WeWork anymore.

\begin{itemize}
\tightlist
\item
  archived recording\\
  Yeah, this is now official. WeWork has officially come out saying that
  Adam Neumann is stepping down as C.E.O.
\end{itemize}

amy chozick

Eventually, Adam is forced to step down.

\begin{itemize}
\tightlist
\item
  archived recording\\
  People are supposed to say I hate to be an ``I told you so.'' But I
  love to be an ``I told you so.'' This house of cards was coming down,
  this business based on beanbags, distressed furniture, and Nespresso
  machines. And yet, somehow this guy's been treated as some guru genius
  because he's such an effective self-promoter. P.T. Barnum is supposed
  to have said, ``There's a sucker born every minute.'' That's what's
  going on here. And Wall Street, they were suckered, too, along with
  SoftBank buying in. Where'd they get this \$47 billion for \$12
  million in cash? You look at their ---
\end{itemize}

{[}music{]}

amy chozick

So they decide that even getting rid of Adam does not save this effort
to go public, and they have to pull the IPO. I mean, this is hugely
embarrassing, this company that had been valued at \$47 billion, the
most valuable start-up in the country, is suddenly valued at a tiny
fraction of that. And this Japanese investor, Masa Son, who had been
such a visionary, it's extremely embarrassing for him. Ultimately, he
has to apologize to investors, specifically for putting so much faith in
Adam Neumann.

michael barbaro

So what becomes of Adam Neumann at this point? He has been pushed out of
his company. In a way, he has been exposed as someone who is a better
salesman than operator of a company. And it becomes pretty clear that he
has built something that is not all that revolutionary but is a little
bit of a financial house of cards.

amy chozick

Well, here's what's so interesting about what happened to Adam Neumann.
He walked away from the company he founded with over a billion dollars.

michael barbaro

How is that possible?

amy chozick

SoftBank bought his shares in WeWork. And they also agreed to pay him
\$46 million in consulting fees for four years.

michael barbaro

Wow.

amy chozick

Nice work if you can get it.

michael barbaro

Why?

amy chozick

SoftBank would say that this was the cost to get Adam out and start to
clean up the mess, which includes laying off potentially thousands of
employees. You know, in this letter the employees call it ``graft.''
Here he was walking away with over a billion dollars, I mean,
generational wealth as they faced layoffs, losing their health care,
getting nothing.

They write, ``We are not the Adam Neumanns of this world --- we are a
diverse workforce with rents to pay, households to support and children
to raise. We are not asking for this level of graft. We are asking to be
treated with humanity and dignity so we can continue living life while
searching to make a living elsewhere.''

michael barbaro

Amy, how do we explain what's happened here, the pretty sad saga of
WeWork and this situation that has ended with all these workers waiting
to be laid off?

amy chozick

So the story of WeWork and Adam Neumann is essentially a story of his
exploitation of two phenomenon. The first is this kind of late-stage
tech capitalism, when investors really want to believe in these start-up
founders who have businesses that are essentially sort of traditional
businesses with tech layered on, and Uber is a perfect example of that
--- calling a taxi with your phone --- and in turn, give these companies
enormous valuations when the business underpinnings are not necessarily
there. So the irony, of course, was that Adam Neumann tapped into this
tech vernacular about changing the world, about revolutionizing office
space, but his company had really nothing to do with tech. It was a
commercial real estate company.

michael barbaro

But people believed that.

amy chozick

But it had a tech valuation because investors believed it. The second
thing he quite brilliantly exploited was this yearning of millennials
having this capitalist ambition and hustle, but also this yearning for
community, this ``we'' generation, as he called it, that it's not about
you or me, it's about we. And in the end, it essentially was about him.

\begin{itemize}
\tightlist
\item
  archived recording (adam neumann)\\
  It is us who will blaze the path forward, paved not with algorithms,
  not with software, but with values, with friendship, with common
  goals, and most importantly, with humanity. {[}APPLAUSE{]} Thank you.
\end{itemize}

amy chozick

And I think that's been particularly hard for WeWork employees to
stomach. It's not just that the business turned out to be not as
profitable as they thought it was, and their CEO walked away with a
giant golden parachute. It was that they believed that it was bigger
than that. They believed it was a community. And now they're saying that
it wasn't.

\begin{itemize}
\tightlist
\item
  archived recording (adam neumann)\\
  And the we revolution is going to be led by the we generation, and it
  will restore in each one of us the sense of dignity and community
  without which braveness cannot be achieved. The we generation knows
  that you must treat other people the way you want to be treated.
\end{itemize}

{[}music{]}

michael barbaro

Thank you, Amy.

amy chozick

Thank you, Michael.

michael barbaro

On Sunday night, The Times reported that WeWork is preparing to lay off
at least 4,000 employees. Those layoffs are expected to be announced as
early as this week.

We'll be right back.

Here's what else you need to know today. Over the weekend, impeachment
investigators released the closed-door testimony of a National Security
Council official, Tim Morrison, who further tied President Trump to the
quid pro quo with Ukraine. Morrison testified that Gordon Sondland, the
U.S. ambassador to the European Union, acted at Trump's behest when he
repeatedly communicated to Ukrainian officials that opening
investigations helpful to Trump would result in the release of \$400
million in U.S. military assistance to Ukraine withheld by Trump.
Morrison also testified that National Security Adviser John Bolton had
tried and failed to convince Trump to release the military assistance
during a meeting over the summer. And ---

\begin{itemize}
\tightlist
\item
  archived recording (john bel edwards)\\
  Thank you! What a great night it is for Louisiana. {[}APPLAUSE{]} And
  to God be the glory!
\end{itemize}

michael barbaro

--- Governor John Bel Edwards of Louisiana, the only Democratic governor
in the Deep South, narrowly won re-election on Saturday, defeating a
Republican supported by President Trump. It was the second time in two
weeks that a Republican candidate for governor backed by Trump has been
defeated. After the defeat of Kentucky's Republican governor, Trump had
told Louisiana voters that he needed them to deliver him a victory.

\begin{itemize}
\tightlist
\item
  archived recording (donald trump)\\
  So Trump took a loss, so you've got to give me a big win please, O.K.?
\end{itemize}

michael barbaro

On Saturday, during his victory speech, Louisiana's governor gently
teased Trump about the outcome.

\begin{itemize}
\tightlist
\item
  archived recording (john bel edwards)\\
  And as for the president, God bless his heart. {[}CHEERING{]}
\end{itemize}

michael barbaro

That's it for ``The Daily.'' I'm Michael Barbaro. See you tomorrow.

\includegraphics{https://static01.graylady3jvrrxbe.onion/images/2019/11/03/business/00NEUMANN-02/merlin_132882803_2ade236b-2b3e-42e9-9a77-c30d059e87cb-articleLarge.jpg?quality=75\&auto=webp\&disable=upscale}

\hypertarget{crazier-faster-bigger-and-more}{%
\subsection{`Crazier, faster, bigger and
more'}\label{crazier-faster-bigger-and-more}}

To WeWork insiders who know Mr. Neumann --- most of whom spoke on the
condition of anonymity because of nondisclosure agreements signed with
the company --- the SoftBank deal changed things precipitously. They
talk about WeWork as existing pre- and post-Masa. The investment
transformed the start-up from a mere unicorn into something with nearly
unlimited ambition.

Even by the standards of brash start-up founders, Mr. Neumann's
eccentricities became the stuff of legend. He could be earthy, walking
around in bare feet at the office, and he organized debauched ``summer
camp'' events for WeWork employees, which attendees described as a sort
of
Coachella-meets-``\href{https://www.youtube.com/watch?v=hBLS_OM6Puk}{Wild
Wild Country}''-meets-nerdy-fraternity-party.

Mr. Neumann would convince employees to take shots of pricey Don Julio
tequila, work 20-hour days, attend 2 a.m. meetings. He'd convince them
to smoke marijuana at work, dance to Journey around a fire in the woods
on weekend excursions, smoke more pot, drink more tequila. Even people
who don't really seem the tequila type would go along with his act ---
including a pre-White House Jared Kushner, who
\href{https://www.wsj.com/articles/wework-a-20-billion-startup-fueled-by-silicon-valley-pixie-dust-1508424483}{imbibed}
while scoping out a property in Philadelphia.

Mr. Neumann had a talent for imbuing ``Animal House'' antics with a
larger meaning. In his view, WeWork didn't simply sublease office space
to workers; it supplied them with kombucha, cold-brew coffee and an
ecstatic sense of community. ``They're coming to us for energy, for
culture,'' Mr. Neumann would say.

He envisioned customers residing in WeLive apartment buildings that
would drive down suicide rates because ``no one ever feels alone.'' He
imagined a WeGrow school and
\href{http://nymag.com/intelligencer/2019/06/wework-adam-neumann.html}{an
effort to shelter the world's orphans}. (``We want to solve this problem
and give them a new family: the WeWork family.'') There was talk of a
WeBank, WeSail, WeSleep, an airline.

Even if the business lines were widely derided, their grandiosity helped
Mr. Neumann cast WeWork as a tech start-up, so many of which are known
to have nearly messianic mission statements.

For those who were intoxicated by this pitch, Mr. Neumann added a hefty
dose of self-help spirituality that he picked up from his wife, Rebekah
Paltrow Neumann, a cousin to the Goop founder Gwyneth Paltrow, and a
certified Jivamukti yogi. ``My intention was never to find a way to make
the most money,'' Ms. Neumann
\href{https://www.youtube.com/watch?v=nAQR-NVPX6o}{said last year}. ``My
intention when I met him was just, `How do we expand this good vibration
to the planet?'''

Image

A WeLive common space at 110 Wall Street.Credit...Cole Wilson for The
New York Times

One person who indisputably vibrated to the Neumanns' frequency was Mr.
Son. He and Mr. Neumann became acquainted in 2016 in India, during a
gathering of start-up luminaries with Prime Minister Narendra Modi.

On the surface, Mr. Son --- who is one of Japan's richest men but is
often described as modest --- and Mr. Neumann, who has said he wants to
become the world's first trillionaire and ``president of the world,''
couldn't have been more different. In college, Mr. Son invented an
electronic translator that he sold to Sharp Corp and later founded
SoftBank as a software distributor. In his early days, Mr. Neumann
struggled to get dubious ventures off the ground, including women's
shoes with collapsible high heels and Krawlers, kneepads for babies,
with the slogan ``Just because they don't tell you, doesn't mean they
don't hurt.''

But both men had been outsiders. Mr. Neumann's parents divorced when he
was 7 and he bounced from city to city --- including a stint on a
kibbutz, or communal settlement --- before following his sister, the
Israeli model Adi Neumann, to New York. Mr. Son, the descendant of
Korean grandparents, was born in a small town in Japan and felt the
sting of discrimination before he moved to California as a teenager.

Like Mr. Neumann, Mr. Son was known to follow his gut and ignore the
naysayers. In 2000, he made a \$20 million early investment in the
Chinese e-commerce venture Alibaba, now worth more than \$100 billion,
because he'd noticed a ``sparkle'' in the chief executive's eyes.

If a founder asked the Vision Fund for \$40 million, Mr. Son might ask,
``What would you do with \$400 million?''

``Masa has his own style and others might choke, but Adam would be like,
`\$400 million? How about \$4 billion, and I can do \emph{this} for
you,''' said a senior executive with direct knowledge of the men's
interactions.

The two entrepreneurs became close, with Mr. Neumann joining Mr. Son for
sushi in Tokyo and dinner at his nine-acre Bay Area home, the most
expensive ever sold in California. Mr. Son would stop by Mr. Neumann's
Corte Madera, Calif., mansion, which featured
a\href{https://www.wsj.com/articles/wework-founder-adam-neumanns-trove-of-pricey-properties-11570730254}{guitar-shaped
living room}.

Mr. Son's decision to put billions into WeWork may have thrilled early
investors and made the Vision Fund's partners feel like they had a piece
of a world-changing start-up, but the deal severed Mr. Neumann from any
sense of reality. ``You've got a guy who meets Adam for 10 minutes and
cuts him a check for \$4.4 billion, and it's just insane,'' the former
executive said. ``And he's not told, `I need you to be the most careful
steward of this capital.' It's like, `I need you to go crazier, faster,
bigger and more.'''

Image

Masayoshi Son, the chairman and chief executive of SoftBank, in
August.Credit...Kiyoshi Ota/Bloomberg

\hypertarget{every-single-word-that-comes-out-of-your-mouth-is-fake}{%
\subsection{`Every single word that comes out of your mouth is
fake'}\label{every-single-word-that-comes-out-of-your-mouth-is-fake}}

Enabled by Mr. Son's instructions to be ``crazier,'' Mr. Neumann dove
into expanding WeWork around the world and pursuing his loftier goals.
But his uncurbed ambitions --- and the company's growing losses ---
started to wear on employees and investors, said nearly a dozen people
who know Mr. Neumann.

Last year, WeWork bought access to a Gulfstream G650 for \$60 million,
about the sum that the company was losing every two weeks. He installed
an infrared sauna and a cold plunge pool in his Manhattan office. In a
glaring conflict of interest, he made millions
\href{https://www.wsj.com/articles/weworks-ceo-makes-millions-as-landlord-to-wework-11547640000}{leasing
buildings he partly owned} back to WeWork. Indulging his penchant for
mysticism, Mr. Neumann changed the company's name to the We Company. Its
I.P.O. filing, which included at least 150 references to the word
``community,'' noted that Mr. Neumann had acted to trademark ``We'' and
extract a \$5.9 million payment from the company for the use of the
pronoun. He later returned the fee.

Some employees found his behavior noxious. In a
\href{https://www.nytimes3xbfgragh.onion/2019/10/31/business/wework-neumann-discrimination-complaint.html}{federal
complaint} filed Thursday, Medina Bardhi, a former chief of staff to Mr.
Neumann, accused him of retaliating against her for becoming pregnant
and derided her maternity leave as a ``vacation'' and ``retirement.''
(She also said she had to stop traveling with Mr. Neumann while pregnant
because he liked to hotbox the company jet. A WeWork spokeswoman said
the company would ``vigorously defend itself'' and had ``zero tolerance
for discrimination.'')

Over the years, many on Wall Street and in the business press have
scoffed openly at WeWork's business model; last year,
\href{https://www.vanityfair.com/news/2018/04/wework-bond-sale-financial-analysis}{Vanity
Fair questioned} its ``\$20 billion house of cards.'' And some came to
wonder if Mr. Neumann had been wise to share,
\href{https://www.youtube.com/watch?v=wX5tREnC-FE}{during a 2017 speech}
at Baruch College, a story from his first date with Rebekah. ``She
looked me straight in the eye and she said, `You, my friend, are full
of'' crap, Mr. Neumann recalled. ```She then said, `Every single word
that comes out of your mouth is fake.'''

No one disputes that Mr. Neumann had an uncommon vision. In nine years,
WeWork grew from a single office to encompass more than 45 million
square feet of real estate, with roughly 527,000 tenants --- or
``memberships'' --- in some 110 cities. WeWork became the single-largest
private occupier of office space in London, New York and Washington. Its
sleek quarters became synonymous with entrepreneurship in the gig
economy and
\href{https://www.nytimes3xbfgragh.onion/2019/01/26/business/against-hustle-culture-rise-and-grind-tgim.html}{millennial
hustle}, complete with ``Thank God It's Monday!'' T-shirts. Large
corporations including IBM, Microsoft and Salesforce moved employees
into WeWork spaces.

When luminaries like the JPMorgan Chase chief executive Jamie Dimon, the
Microsoft chief executive Satya Nadella and the private-equity tycoon
Henry R. Kravis came by WeWork offices, Mr. Neumann knew how to wow
them. He might instruct his team to ``activate the space,'' with an
enormous touch screen displaying WeWork's premium locations, like
London's postmodern No. 1 Poultry building and an old opium factory in
Shanghai with pastel terrazzo tiles. And Mr. Neumann made sure potential
investors stopped by the desk of WeWork's ``head of visualization,'' who
handled virtual reality renderings. Suddenly, visitors could see
themselves inside a future WeWork space in Paris or Tokyo. As Dave Fano,
the company's chief growth officer,
\href{https://www.forbes.com/sites/stevenbertoni/2017/10/02/the-way-we-work/\#e17b9fe1b181}{told
Forbes in 2017}, ``Landlords just sell aluminum. We make iPhones.''

For a long time, Mr. Son protected Mr. Neumann. In mid-September, as the
I.P.O. effort was collapsing and Mr. Neumann was facing outside and
board pressure to resign, Mr. Son invited him to sit at his table at the
Langham Hotel in Pasadena, Calif., for a Vision Fund social gathering.
John Legend would be playing. According to a person briefed on the
conversations, SoftBank executives told Mr. Son that the optics of Mr.
Neumann attending the event, much less joining him at the main table,
would be awful.

Ultimately, Mr. Neumann did not appear --- and pointedly, Mr. Son gave
other attendees a stern reminder about the importance of profitability
and strengthening corporate governance before a company attempts to go
public.

Mr. Neumann
\href{https://www.nytimes3xbfgragh.onion/2019/09/24/business/dealbook/wework-ceo-adam-neumann.html}{resigned
as chief executive on Sept. 24}. In an October conference call with
Vision Fund investors, according to a person familiar with the
discussion, Mr. Son apologized for putting so much faith in the founder.
On Oct. 22, SoftBank
\href{https://group.softbank/en/corp/set/data/news/press/sb/2019/20191023_01/pdf/20191023_01.pdf}{bailed
out} WeWork, taking roughly an 80 percent stake in the company. The next
day, the company's new executive chairman --- SoftBank's chief operating
officer,
\href{https://www.nytimes3xbfgragh.onion/reuters/2019/10/23/business/23reuters-wework-chairman-workers.html}{Marcelo
Claure} --- held a town-hall meeting with employees.

``Are there going to be layoffs? Yes. How many? I don't know,'' he said,
\href{https://www.vox.com/2019/10/24/20929236/wework-layoffs-adam-neumann-marcelo-claure}{according
to a transcript}. ``I'd like to put the Adam story behind us, the
payout,'' he added.

Whether Mr. Neumann was hailed as a visionary or denigrated as a
huckster, he had always maintained a powerful hold on voting power
within WeWork. At one point, his shares had been worth 20 votes each. To
get control of his stake, SoftBank decided, it had to pay up.

``Adam is not going to have any role in the company, he's not going to
be in the board of directors, but I do plan to use some of his
knowledge,'' Mr. Claure told employees.

Image

A WeWork office on Fifth Avenue.Credit...An Rong Xu for The New York
Times

Some on Wall Street think WeWork still has a ways to fall. On Oct. 30,
the hedge fund manager Bill Ackman said publicly that the company's
value ``has a pretty high probability of being a zero.'' Other analysts
say WeWork does have a workable business model, but that model will
likely hew closer to the core idea of leasing office space, largely to
long-term corporate clients --- rather than, say, schools and orphanages
--- providing them with a cheery atmosphere and free Wi-Fi.

``Is there a viable business? The answer is yes, but not the way WeWork
pursued it,'' said Nori Gerardo Lietz, a senior lecturer at Harvard
Business School who has studied the company extensively. She added that
WeWork would have to move past its ``profligate excesses.''

Improbably enough, there are those who still have faith in Mr. Neumann
as a businessman. ``I actually think he's probably one of the greatest
entrepreneurs I've ever met,'' Marc Benioff, the co-chief executive of
Salesforce,
\href{https://www.businessinsider.com/marc-benioff-wework-adam-neumann-is-a-great-entrepreneur-2019-10}{told
Business Insider} on Oct. 15. ``He's an incredible evangelist. He's an
incredible visionary. He's hired a lot of amazing people. He's built an
amazing brand, right?'' Mr. Benioff did allow: ``Unfortunately, there
were some things, obviously, in the company that you probably would have
preferred to change if he could do it all over again.''

Mr. Neumann has been uncharacteristically quiet in recent weeks, dealing
with legal and public relations matters as he shuttles among homes in
Greenwich Village, Montauk and Westchester County. The man who promised
that WeWork would ``create a world where people make a life and not just
a living'' is in between jobs.

Advertisement

\protect\hyperlink{after-bottom}{Continue reading the main story}

\hypertarget{site-index}{%
\subsection{Site Index}\label{site-index}}

\hypertarget{site-information-navigation}{%
\subsection{Site Information
Navigation}\label{site-information-navigation}}

\begin{itemize}
\tightlist
\item
  \href{https://help.nytimes3xbfgragh.onion/hc/en-us/articles/115014792127-Copyright-notice}{©~2020~The
  New York Times Company}
\end{itemize}

\begin{itemize}
\tightlist
\item
  \href{https://www.nytco.com/}{NYTCo}
\item
  \href{https://help.nytimes3xbfgragh.onion/hc/en-us/articles/115015385887-Contact-Us}{Contact
  Us}
\item
  \href{https://www.nytco.com/careers/}{Work with us}
\item
  \href{https://nytmediakit.com/}{Advertise}
\item
  \href{http://www.tbrandstudio.com/}{T Brand Studio}
\item
  \href{https://www.nytimes3xbfgragh.onion/privacy/cookie-policy\#how-do-i-manage-trackers}{Your
  Ad Choices}
\item
  \href{https://www.nytimes3xbfgragh.onion/privacy}{Privacy}
\item
  \href{https://help.nytimes3xbfgragh.onion/hc/en-us/articles/115014893428-Terms-of-service}{Terms
  of Service}
\item
  \href{https://help.nytimes3xbfgragh.onion/hc/en-us/articles/115014893968-Terms-of-sale}{Terms
  of Sale}
\item
  \href{https://spiderbites.nytimes3xbfgragh.onion}{Site Map}
\item
  \href{https://help.nytimes3xbfgragh.onion/hc/en-us}{Help}
\item
  \href{https://www.nytimes3xbfgragh.onion/subscription?campaignId=37WXW}{Subscriptions}
\end{itemize}
