Sections

SEARCH

\protect\hyperlink{site-content}{Skip to
content}\protect\hyperlink{site-index}{Skip to site index}

\href{https://www.nytimes3xbfgragh.onion/section/business}{Business}

\href{https://myaccount.nytimes3xbfgragh.onion/auth/login?response_type=cookie\&client_id=vi}{}

\href{https://www.nytimes3xbfgragh.onion/section/todayspaper}{Today's
Paper}

\href{/section/business}{Business}\textbar{}I Got Access to My Secret
Consumer Score. Now You Can Get Yours, Too.

\href{https://nyti.ms/34tKTo9}{https://nyti.ms/34tKTo9}

\begin{itemize}
\item
\item
\item
\item
\item
\item
\end{itemize}

Advertisement

\protect\hyperlink{after-top}{Continue reading the main story}

Supported by

\protect\hyperlink{after-sponsor}{Continue reading the main story}

\hypertarget{i-got-access-to-my-secret-consumer-score-now-you-can-get-yours-too}{%
\section{I Got Access to My Secret Consumer Score. Now You Can Get
Yours,
Too.}\label{i-got-access-to-my-secret-consumer-score-now-you-can-get-yours-too}}

Little-known companies are amassing your data --- like food orders and
Airbnb messages --- and selling the analysis to clients. Here's how to
get a copy of what they have on you.

\includegraphics{https://static01.graylady3jvrrxbe.onion/images/2019/11/05/business/05SecretScores-illo/SecretScores-illo-articleLarge.jpg?quality=75\&auto=webp\&disable=upscale}

By \href{https://www.nytimes3xbfgragh.onion/by/kashmir-hill}{Kashmir
Hill}

\begin{itemize}
\item
  Published Nov. 4, 2019Updated Nov. 5, 2019
\item
  \begin{itemize}
  \item
  \item
  \item
  \item
  \item
  \item
  \end{itemize}
\end{itemize}

As consumers, we all have ``secret scores'': hidden ratings that
determine how long each of us waits on hold when calling a business,
whether we can return items at a store, and what type of service we
receive. A low score sends you to the back of the queue; high scores get
you elite treatment.

Every so often, journalists lament these systems' inaccessibility.
They're ``largely invisible to the public,'' The New York Times
\href{https://www.nytimes3xbfgragh.onion/2012/08/19/business/electronic-scores-rank-consumers-by-potential-value.html}{wrote}
in 2012. ``Most people have no inkling they even exist,'' The Wall
Street Journal
\href{https://www.wsj.com/articles/on-hold-for-45-minutes-it-might-be-your-secret-customer-score-1541084656}{said}
in 2018. Most recently, in April, The Journal's Christopher Mims looked
at a company called Sift, whose proprietary scoring system tracks 16,000
factors for companies like Airbnb and OkCupid. ``Sift judges whether or
not you can be trusted,'' he
\href{https://www.wsj.com/articles/the-secret-trust-scores-companies-use-to-judge-us-all-11554523206}{wrote},
``yet there's no file with your name that it can produce upon request.''

As of this summer, though, Sift \emph{does} have a file on you, which it
can produce upon request. I got mine, and I found it shocking: More than
400 pages long, it contained all the messages I'd ever sent to hosts on
Airbnb; years of Yelp delivery orders; a log of every time I'd opened
the Coinbase app on my iPhone. Many entries included detailed
information about the device I used to do these things, including my IP
address at the time.

Sift knew, for example, that I'd used my iPhone to order chicken tikka
masala, vegetable samosas and garlic naan on a Saturday night in April
three years ago. It knew I used my Apple laptop to sign into Coinbase in
January 2017 to change my password. Sift knew about a nightmare
Thanksgiving I had in California's wine country, as captured in my
messages to the Airbnb host of a rental called ``Cloud 9.''

``The heater in the room with the big couch has been running since we
got here and we're not sure how to turn it off,'' I wrote on Wednesday
afternoon.

``The air in the main house is really musty, like maybe there's a mildew
or mold issue,'' I wrote on Thursday, then added apologetically, ``Sorry
to be bothering you on Thanksgiving!''

``The bathroom flooded during the rainstorm. The carpet outside the
bathroom is very wet,'' I wrote on Friday. ``Ants are coming in from the
interior wall of the house.''

This may sound somewhat comical, but the companies gathering and paying
for this data find it extremely valuable for rooting out fraud and
increasing the revenue they can collect from big spenders. Sift has this
data because the company has been hired by Airbnb, Yelp, and Coinbase to
identify stolen credit cards and help spot identity thieves and abusive
behavior. Still, the fact that obscure companies are accumulating
information about years of our online and offline behavior is
unsettling, and at a minimum it creates the potential for abuse or
discrimination --- particularly when those companies decide we don't
stack up.

\hypertarget{how-to-get-your-data}{%
\subsection{How to get your data}\label{how-to-get-your-data}}

There are many companies in the business of scoring consumers. The
challenge is to identify them. Once you do, the instructions on getting
your data will probably be buried in their privacy policies. Ctrl-F
``request'' is a good way to find it. Most of these companies will also
require you to send a photo of your driver's license to verify your
identity. Here are five that say they'll share the data they have on
you.

\begin{itemize}
\item
  Sift, which determines consumer trustworthiness, asks you to email
  \href{mailto:privacy@sift.com}{\nolinkurl{privacy@sift.com}}. (An
  earlier version of this article contained a link to an online form;
  the company disabled the page after receiving thousands of
  submissions.)
\item
  Zeta Global, which identifies people with a lot of money to spend,
  lets you request your data via
  \href{https://privacyportal-cdn.onetrust.com/dsarwebform/bc2d3301-11a5-4de5-b15e-ce796187a352/d0720d0f-d427-4a7d-a773-5d6793229f15.html}{an
  online form}.
\item
  Retail Equation, which helps companies such as Best Buy and Sephora
  decide whether to accept or reject a product return, will send you a
  report if you email
  \href{mailto:returnactivityreport@theretailequation.com}{\nolinkurl{returnactivityreport@theretailequation.com}}.
\item
  Riskified, which develops fraud scores, will tell you what data it has
  gathered on your possible crookedness if you contact
  \href{mailto:privacy@riskified.com}{\nolinkurl{privacy@riskified.com}}.
\item
  Kustomer, a database company that
  \href{https://www.kustomer.com/platform/}{provides} what it calls
  ``unprecedented insight into a customer's past experiences and current
  sentiment,'' tells people to email
  \href{mailto:privacy@kustomer.com}{\nolinkurl{privacy@kustomer.com}}.
\end{itemize}

Just because the companies say they'll provide your data doesn't mean
they actually will.

Kustomer, for example, gave me the runaround. When I first contacted the
company from my personal email address, a representative wrote back that
I would have the report by the end of the week. After a couple of weeks
passed, I emailed again and was told the company was ``instituting a new
process'' and had ``hit a few snags.'' I never got the report. When I
contacted a company spokeswoman, I was told that I would need to get my
data instead from the companies that used Kustomer to analyze me.

\hypertarget{thanks-california}{%
\subsection{Thanks, California}\label{thanks-california}}

Most of the companies only recently started honoring these requests in
response to the
\href{https://www.nytimes3xbfgragh.onion/2018/06/28/technology/california-online-privacy-law.html}{California
Consumer Privacy Act.} Set to go into effect in 2020, the law will grant
Californians the right to see what data a company holds on them. It
follows a 2018 European privacy law,
\href{https://www.nytimes3xbfgragh.onion/2018/05/24/technology/europe-gdpr-privacy.html}{called
General Data Protection Regulation}, that lets Europeans gain access to
and delete their online data. Some companies have decided to honor the
laws' transparency requirements even for those of us who are not lucky
enough to live in Europe or the Golden State.

``We expect these are the first of many laws,'' said Jason Tan, the
chief executive of Sift. The company, founded in 2011, started making
files available to ``all end users'' this June, even where not legally
required to do so --- such as in New York, where I live. ``We're trying
to be more privacy conscious. We want to be good citizens and stewards
of the internet. That includes transparency.''

I was inspired to chase down my data files by
\href{https://www.representconsumers.org/wp-content/uploads/2019/06/2019.06.24-FTC-Letter-Surveillance-Scores.pdf}{a
June report} from the Consumer Education Foundation, which wants the
Federal Trade Commission to investigate secret surveillance scores
``generated by a shadowy group of privacy-busting firms that operate in
the dark recesses of the American marketplace.'' The report named 11
firms that rate shoppers, potential renters and prospective employees. I
pursued data from the firms most likely to have information on me.

One of the co-authors of the report was Laura Antonini, the policy
director at the Consumer Education Foundation. At my suggestion, she
sought out her own data. She got a voluminous report from Sift, and like
me, had several companies come up empty-handed despite their claims to
have information on hundreds of millions of people. Retail Equation, the
company that helps decide whether customers should be allowed to make a
return, had nothing on me and one entry for Ms. Antonini: a return of
three items worth \$78 to Victoria's Secret in 2009.

``I don't really care that these data analytics companies know I made a
return to Victoria's Secret in 2009, or that I had chicken kebabs
delivered to my apartment, but how is this information being used
against me when you generate scores for your clients?'' Ms. Antonini
said. ``That is what consumers deserve to know. The lack of the
information I received back is the most alarming part of this.''

In other words, most of these companies are just showing you the data
they used to make decisions about you, not how they analyzed that data
or what their decision was.

\hypertarget{its-incredible-what-machines-can-do-when-they-can-look-under-every-stone}{%
\subsection{`It's incredible what machines can do when they can look
under every
stone'}\label{its-incredible-what-machines-can-do-when-they-can-look-under-every-stone}}

My Sift file didn't come with a credit-score-type number at the top, but
many of the entries included a percentage rating as to whether the
behavior was ``abuse'' or ``not abuse,'' ``normal'' or ``fraud'' or
``account takeover'' versus ``not account takeover.''

When I told Mr. Tan that I was alarmed to see my Airbnb messages and
Yelp orders in the hands of a company I'd never heard of before, he
responded by saying that Sift doesn't sell or share any of the data it
has with third parties.

``We are in the business of predicting risks for particular events at
particular times, for particular fraud,'' he said. Sift is looking at
all my online activity to make sure it's me, and not someone trying to
impersonate or hack me.

``Behind the scenes, we're trying to create connections between
fraudulent accounts,'' Mr. Tan said. To score risk, the more data Sift
has, the better. It's able to use what it knows across the accounts of
all its clients, so if a certain device has been used to make an order
on Yelp with a stolen credit card, Sift can flag that device when it
shows up on Airbnb.

``We're not looking at the data. It's just machines and algorithms doing
this work,'' said Mr. Tan. ``But it's incredible what machines can do
when they can look under every stone.''

I asked Mr. Tan how many people had requested their data from Sift since
the company introduced the option to get it.

``Honestly, we haven't seen much of a response,'' he said.

A spokeswoman from Zeta Global, which created a portal for data requests
in August, told me that 10 people have requested their data so far.
``There was only data on two people,'' she said. (I was one of them; the
company had a record of all the comments I had made on a blog a decade
ago.)

This may be because most people have no idea Sift, Zeta and the other
secret scorers exist. But now you do, and you know how you can get your
files.

If you submit a request to any of these companies and get back something
weird, please share your experience with me at
\href{mailto:kashmir.hill@NYTimes.com}{\nolinkurl{kashmir.hill@NYTimes.com}}.

Advertisement

\protect\hyperlink{after-bottom}{Continue reading the main story}

\hypertarget{site-index}{%
\subsection{Site Index}\label{site-index}}

\hypertarget{site-information-navigation}{%
\subsection{Site Information
Navigation}\label{site-information-navigation}}

\begin{itemize}
\tightlist
\item
  \href{https://help.nytimes3xbfgragh.onion/hc/en-us/articles/115014792127-Copyright-notice}{©~2020~The
  New York Times Company}
\end{itemize}

\begin{itemize}
\tightlist
\item
  \href{https://www.nytco.com/}{NYTCo}
\item
  \href{https://help.nytimes3xbfgragh.onion/hc/en-us/articles/115015385887-Contact-Us}{Contact
  Us}
\item
  \href{https://www.nytco.com/careers/}{Work with us}
\item
  \href{https://nytmediakit.com/}{Advertise}
\item
  \href{http://www.tbrandstudio.com/}{T Brand Studio}
\item
  \href{https://www.nytimes3xbfgragh.onion/privacy/cookie-policy\#how-do-i-manage-trackers}{Your
  Ad Choices}
\item
  \href{https://www.nytimes3xbfgragh.onion/privacy}{Privacy}
\item
  \href{https://help.nytimes3xbfgragh.onion/hc/en-us/articles/115014893428-Terms-of-service}{Terms
  of Service}
\item
  \href{https://help.nytimes3xbfgragh.onion/hc/en-us/articles/115014893968-Terms-of-sale}{Terms
  of Sale}
\item
  \href{https://spiderbites.nytimes3xbfgragh.onion}{Site Map}
\item
  \href{https://help.nytimes3xbfgragh.onion/hc/en-us}{Help}
\item
  \href{https://www.nytimes3xbfgragh.onion/subscription?campaignId=37WXW}{Subscriptions}
\end{itemize}
