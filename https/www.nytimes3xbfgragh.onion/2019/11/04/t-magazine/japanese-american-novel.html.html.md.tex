Sections

SEARCH

\protect\hyperlink{site-content}{Skip to
content}\protect\hyperlink{site-index}{Skip to site index}

\href{https://myaccount.nytimes3xbfgragh.onion/auth/login?response_type=cookie\&client_id=vi}{}

\href{https://www.nytimes3xbfgragh.onion/section/todayspaper}{Today's
Paper}

The Story of the Great Japanese-American Novel

\url{https://nyti.ms/36pbqom}

\begin{itemize}
\item
\item
\item
\item
\item
\item
\end{itemize}

Advertisement

\protect\hyperlink{after-top}{Continue reading the main story}

Supported by

\protect\hyperlink{after-sponsor}{Continue reading the main story}

Social Studies

\hypertarget{the-story-of-the-great-japanese-american-novel}{%
\section{The Story of the Great Japanese-American
Novel}\label{the-story-of-the-great-japanese-american-novel}}

John Okada's ``No-No Boy'' captures the injustice of incarcerating
Japanese-Americans during World War II --- and serves as a warning today
for our own fractured society.

\includegraphics{https://static01.graylady3jvrrxbe.onion/images/2019/11/17/t-magazine/17tmag-nonoboy-slide-6XRL/17tmag-nonoboy-slide-6XRL-articleLarge.jpg?quality=75\&auto=webp\&disable=upscale}

By
\href{https://www.nytimes3xbfgragh.onion/by/thessaly-la-force}{Thessaly
La Force}

\begin{itemize}
\item
  Nov. 4, 2019
\item
  \begin{itemize}
  \item
  \item
  \item
  \item
  \item
  \item
  \end{itemize}
\end{itemize}

IT WAS THE LATE '60s, and a University of California, Berkeley,
undergraduate named Shawn Wong wanted to write the next great American
novel. He was born in
\href{https://www.nytimes3xbfgragh.onion/interactive/2015/12/02/travel/what-to-do-in-36-hours-in-oakland-california.html}{Oakland},
Calif., in 1949 to parents who emigrated from Tianjin, China, both of
whom died by the time he was 15. Wong had fallen in love with
literature, enchanted by
\href{https://www.nytimes3xbfgragh.onion/topic/person/james-joyce}{James
Joyce},
\href{https://www.nytimes3xbfgragh.onion/topic/person/samuel-beckett}{Samuel
Beckett} and
\href{https://www.nytimes3xbfgragh.onion/topic/person/william-carlos-williams}{William
Carlos Williams}, and he, too, wanted to embark on a life as a writer.

At the time, Asian-Americans made up only 6 percent of U.C. Berkeley's
student population, and when Wong asked --- professors, other English
majors --- what Asian-American literature he should read, what
Asian-American writers he should know, no one could answer him. ``I felt
like the only Asian-American writer in the whole world,'' he told me.
After meeting three other writers, Frank Chin, Lawson Fusao Inada and
Jeffery Paul Chan, the four began to search used bookstores in the Bay
Area for works by Asian-Americans. They bought it all: Charlie Chan and
Fu Manchu books, books with the word ``Chinaman'' in the title,
restaurant guides, oil-stained cookbooks, books by people with
Asian-sounding names who turned out not to be Asian. It didn't matter.
They were determined to find who came before.

\emph{{[}}\href{https://www.nytimes3xbfgragh.onion/newsletters/t-list?module=inline}{\emph{Sign
up here}} \emph{for the T List newsletter, a weekly roundup of what T
Magazine editors are noticing and coveting now.{]}}

Most of the books turned out to be useless --- racist pulp and
propaganda churned out at the height of Yellow Peril hysteria in the
first half of the century or basking, decades later, in nostalgia for
it. Then there was a book called
``\href{https://uwapress.uw.edu/book/9780295994048/no-no-boy/}{No-No
Boy},'' published in 1957 by an author named John Okada. Chan bought it
for 50 cents. They put it aside, saving it to read later. There wasn't
any rush. After all, a decade had already passed. No one else they knew
had bothered to read the book with the barbed wire on the cover.

``NO-NO BOY'' was published by Charles E. Tuttle the year after Allen
Ginsberg's ``Howl'' and
\href{https://www.nytimes3xbfgragh.onion/2019/09/05/t-magazine/james-baldwin-giovannis-room.html}{James
Baldwin's ``Giovanni's Room.''} Like those books, it is a kind of
generational reckoning with American bigotry. Unlike them, it gained
little notice upon its release. It tells the story of Ichiro Yamada, a
second-generation Japanese-American, or Nisei (a Japanese term for the
generation born in the United States; those who immigrated from Japan
are considered first-generation Americans, or Issei), who has just
returned home to Seattle at the end of World War II. He is one of
several hundred Japanese-Americans who refused to be drafted into
service while incarcerated by the American government and were
consequently sent to federal prison. The book's title refers to the act
of answering no to two questions in a mandatory survey issued by the
government in February 1943, midway through the war, to all persons over
the age of 17 in the camps: The first asked if men and women would be
willing to serve in the armed forces if qualified. The second asked if
they were willing to swear their allegiance to the United States and, in
essence, renounce Japanese citizenship. It was a confusing, poorly
conceived set of questions: The Issei could not become American citizens
because of the discriminatory naturalization laws of the time, and would
effectively be rendered stateless if they answered yes to the second.
Meanwhile, most Nisei, who were already American citizens, objected to
the suggestion that they had ever been loyal to the Japanese emperor. As
a result of answering no or failing to respond to these questions, about
12,000 people were branded as disloyal and segregated in the harshest
camp at
\href{https://www.nytimes3xbfgragh.onion/video/us/100000005024375/japanese-internment-tule-lake-360-video.html}{Tule
Lake}, in California.

Image

John Okada's ``No-Boy Boy'' (1957).Credit...Joshua Scott

Image

Yoshiko Uchida's ``Journey to Topaz'' (1971).Credit...Joshua Scott

The Japanese-American concentration camps --- more commonly called
internment camps, though many Japanese-Americans today reject such
euphemistic language --- were effectively established in February 1942,
when President Roosevelt
\href{https://www.nytimes3xbfgragh.onion/2015/11/27/us/from-pearl-harbor-to-an-apology-an-internment-timeline.html}{signed
Executive Order 9066}. The act authorized the military to set up zones
``from which any or all persons may be excluded''; 110,000 people of
Japanese ancestry (two-thirds of whom were United States citizens)
living across the West Coast were stripped of their possessions,
forcibly removed from their homes and imprisoned. But halfway through
the war, as the fear of Japanese invasion subsided, the government began
to question if the camps remained a good idea. They were costly, and
morale was low. People in the camps felt abandoned by the American
government, and small factions were becoming violent. Concerned that
abruptly disbanding the camps would pose a risk, U.S. officials created
the survey as a way to appease lingering paranoia throughout the country
while allowing loyal Issei and Nisei to leave. According to Okada's
\href{https://uwapress.uw.edu/book/9780295743516/john-okada/}{biographer},
Frank Abe, the prevailing mentality at the time was a combination of
resignation (\emph{shikata ga nai}, Japanese for ``it can't be helped'')
and American patriotic fervor, epitomized by the Hawaiian pidgin
expression ``go for broke,'' or ``spilling one's blood to prove one's
loyalty.'' Therefore, the vast majority answered yes; saying no was not
an option.

OKADA, A NISEI, was born in 1923 in Seattle and was forced to abandon
pursuit of his pre-law degree at the University of Washington to be
incarcerated at Minidoka, in Hunt, Idaho, in 1942. He was only there for
three weeks before leaving for Scottsbluff Junior College in Nebraska,
through a dispensation that allowed Nisei students to apply to schools
outside the West Coast exclusion zone. After a year, he left to serve in
the war in a secret mission, where more linguistically gifted Nisei
studied Japanese so they could work as military translators. (To the
surprise of the American government, most Nisei did not speak Japanese
well enough to understand the enemy.) Okada spent his years in the war
flying over the Pacific, intercepting messages from the Japanese forces
while in the belly of a B-24. Upon his return to America, he completed
his education, married and took up work as a librarian. Because his
wartime service remained classified, Okada turned to the experiences of
his friend from home, Hajime ``Jim'' Akutsu, who had been a draft
resister, to write his book.

``No-No Boy'' explores the deep sense of alienation Japanese-Americans
experienced after the war: the displacement; the confusing, unsatisfying
homecoming; the unspoken humiliation of having been ``evacuated''; the
anger and frustration of being treated as an enemy. Okada circles again
and again around the idea of resistance: the importance of saying no
because you still had a brother in Japan, or because your mother was
sent to a different camp from your father and leaving her alone just
wasn't possible --- but most of all because saying yes just wasn't
right. On Ichiro's first day back from prison, he slouches along
Seattle's Jackson Street, past the pool halls and movie houses and
taverns, the gambling and prostitution dens run by the Chinese. His
mother, who is Issei, is a Japanese nationalist. She believes the false
reports from Brazil claiming Japan has won the war and is proud that her
son refused to enlist. That evening, she brings Ichiro to visit their
friends, the Kumasakas, who have lost their son Bobbie to the war.
Ichiro realizes his mother is trying to show him off to the grieving
parents. Furious, he walks home alone and contemplates his decision not
to be a soldier like Bobbie Kumasaka.

Ichiro feels like a traitor. He is ostracized within his own community.
Having resisted offers no glory: He cannot reclaim what his family has
lost. The idea of returning to school feels unfathomable. He cannot find
it in himself to feel absolved, to see his refusal as an act of protest,
because he cannot shake the sense that he has committed some kind of
betrayal --- though what that is and against whom is unclear: ``Why is
it then that I am unable to convince myself that I am no different from
any other American? Why is it that, in my freedom, I feel more
imprisoned in the wrongness of myself and the thing I did than when I
was in prison?''

Though Okada hardly mentions the camps in the novel, their presence is
everywhere as Ichiro watches his community try to return to life as it
once was. As with most great injustices, there is no recourse for those
wronged. No one will be held accountable for Ichiro's shame but himself.

\includegraphics{https://static01.graylady3jvrrxbe.onion/images/2019/11/17/t-magazine/17tmag-nonoboy-slide-F0VT/17tmag-nonoboy-slide-F0VT-articleLarge.jpg?quality=75\&auto=webp\&disable=upscale}

INCARCERATION LASTED FOUR years. Though Japanese-Americans were as
shocked as everyone else about the
\href{https://www.nytimes3xbfgragh.onion/2016/12/07/world/pearl-harbor-anniversary.html}{attack
on Pearl Harbor}, most proceeded with life as usual: They pounded rice
for mochi, they made Christmas roasts, they exchanged presents. But
already there were signs of growing intolerance. Within hours of the
attack, all Issei within the United States and its territories,
including Hawaii, had been marked as ``alien enemies'' by President
Roosevelt. More than 1,200 Japanese community leaders (mostly men) were
quickly arrested out of fear of espionage and imprisoned with little or
no evidence. In the months after the exclusion order was issued, posters
went up across the West Coast, giving people of Japanese ancestry one
week to dismantle their lives --- they were allowed to bring only what
they could carry. There was no indication of where, exactly, they were
going to live, or for how long. Carpetbaggers arrived, offering to buy
what was left, automobiles or appliances. Properties and leases were
signed over to non-Japanese-Americans --- friends, neighbors, business
partners --- with the vague promise that they would be returned once the
ordeal was over.

The initial 16 temporary detention centers were barely inhabitable.
Many, set up in vacated racetracks and fairgrounds, reeked of manure.
There was no privacy. Partitions between the converted stalls were thin
and incomplete --- the more intimate moments of life, sex and fights,
could be heard by all. Once the 10 more permanent camps were built,
families packed up again, loaded into trains and were carried off into
the night. The sick and elderly were hoisted by planks through the
windows.
``\href{https://www.scholastic.com/teachers/books/journey-to-topaz-by-yoshiko-uchida/}{Journey
to Topaz},'' a young-adult novel published by
\href{https://www.nytimes3xbfgragh.onion/1992/06/24/obituaries/yoshiko-uchida-70-a-children-s-author.html}{Yoshiko
Uchida} in 1971, is about an 11-year-old girl from Berkeley named Yuki
who, after her father is arrested and detained at a separate camp, is
sent to live in Utah with her older brother and mother. The camp is
described as barren, a cardboard city in the middle of nowhere. Dust
storms pelt stones against her calves. An elderly man named Mr. Kurihara
is shot by a guard for wandering too close to the perimeter of the camp.
He was searching for arrowheads.

Yuki's family never resists what is happening to them. Her older brother
eventually enlists. Uchida is not interested in telling a story of
outrage. Instead, she uses Yuki's innocence as a foil --- Yuki is the
one to peek out of the windows of the train when it is forbidden. She
struggles to accept what is happening to her; she finds her mother's
explanations weak.

She reminds me of a photograph by
\href{https://www.nytimes3xbfgragh.onion/2018/11/28/lens/dorothea-lange-migrant-mother.html}{Dorothea
Lange}, taken in 1942, of a group of schoolchildren in San Francisco,
holding their hands over their hearts, pledging allegiance to the flag.
Two girls stand in the front row: Hideno Nakamoto and Yoko Itashiki.
Like Yuki, both were later incarcerated in Topaz. Itashiki would
ultimately be separated from her mother, who died in the camps. In the
picture, their winter coats are buttoned up. Nakamoto looks skeptical.
Itashiki, who is still missing a tooth or two, has raised her eyebrows
with an innocence only a child can possess. Yet there is an urgency to
the picture: \emph{This} is who you call the enemy?

WEREN'T THE CAMPS a mistake we already learned from, a lesson in
ignorance that would never again be ignored? Last November, a
\href{https://widerimage.reuters.com/story/honduran-migrant-flees-tear-gas-with-her-children}{photograph
by Kim Kyung-Hoon} for Reuters drew wide notice; it showed a Honduran
mother named Maria Lidia Meza Castro running away from the border in
Tijuana, Mexico, with her 5-year-old twins as tear gas is thrown at
them. And so we ask the same question --- \emph{Are these people really
our enemy?} --- again as a new crisis continues: one where families are
being separated, and the young, the weak and the elderly are suffering.

There are few works of literature from incarceration, and only a handful
are read widely, or given the weight they deserve. But together they
tell a powerful story in relief. They tell us to find dignity in the
face of injustice, to see tragedy for what it is. They question the
ideas of equality and freedom that this country purportedly believes in.
They also represent a point of no return --- an event that forever
scarred generations of Americans. They are a document not just of what
happened but of something worse: the aftermath and all its emptiness.
Yuki lost her childhood; Ichiro lost his future. No one really cares, at
least not the people who had the power to take those things away. And
now there is a new set of camps, a new schism --- and we don't yet know
how devastating they will be.

Image

Hideno Nakamoto (left) and Yoko Itashiki (center), photographed by
Dorothea Lange in San Francisco in 1942, before they were sent to the
camps.Credit...Dorothea Lange's ``Pledge of Allegiance\textbar{}One
Nation Indivisible\textbar{}One Nation Indivisible, San
Francisco\textbar{}Pledge of Allegiance at Raphael Weill Elementary
School a Few Weeks Prior to Evacuation, San Francisco'' (April 20,
1942), gift of Paul S. Taylor, courtesy of the Oakland Museum of
California

It's incorrect to say that ``No-No Boy'' is a forgotten masterwork (this
year, \href{https://uwapress.uw.edu/book/9780295994048/no-no-boy/}{the
novel} was reissued by Penguin Classics, though a dispute about
\href{https://www.nytimes3xbfgragh.onion/2019/06/06/books/no-no-boy-penguin.html}{the
book's copyright} has led Penguin to stop selling it in the U.S.), but
it isn't often acknowledged for articulating what had never been said
before. The novel was a turning point in the consciousness of
Japanese-Americans, and of Asian-Americans more generally --- it marked
the moment when identity shifted away from the homeland, away from
Japan, because Japan was a country that Nisei, like Okada, never really
quite knew. It was a novel that struggled to understand the entitlement
that came so easily to other Americans --- to explain why so few
Japanese-Americans protested what had been done to them, that explored
the shame of an immigrant who doesn't feel he has a place in the world.
Most of all, the novel conveys the weight of Ichiro's resignation. Other
people try to help him --- he's offered a well-paying engineering job,
but he doesn't feel he deserves it; he is told by a lover that he needs
to stop being so hard on himself but is unable to believe her. He cannot
shake the feeling that it is all --- the camps, the war, the isolation
he now feels --- his fault. Something about him is simply not American
enough.

BY 1945, JAPAN was vanquished. Six months later, the camps had all been
shut down. Though the Roosevelt administration had entreated Americans
to treat the returning Japanese-Americans as their fellow citizens (``if
you find one or two Japanese-American families settled in your
neighborhood ... try to regard them as individuals,''
\href{https://www2.gwu.edu/~erpapers/documents/articles/challengetoamerican.cfm}{wrote}
the first lady in Collier's in 1943), they were barred from resettling
on the West Coast until January 1945.

Most returned to nothing at all. It wasn't just their property they had
lost --- it was their dream of America itself, and their right to belong
to it. For many Issei, particularly the elderly, only the simplest
pleasures offered a respite from their suffering. The writer
\href{https://www.latimes.com/local/obituaries/la-xpm-2011-feb-13-la-me-hisaye-yamamoto-20110213-story.html}{Hisaye
Yamamoto}, who was incarcerated in Poston, Ariz., and published her only
book of stories, ``Seventeen Syllables,'' in 1988, written across 40
years, understood this well. Her short story ``Las Vegas Charley'' is
about an old man called Charley, a widower who works as a dishwasher
after the camps and spends all his money gambling. His mother, who is
still alive in Japan, sends letters begging him to come see her. But he
hasn't saved any money.

A penniless protagonist with no hope --- this is, on one level, a
classic American story. It's a scene told in better-known works by
writers like Theodore Dreiser and Grace Paley, writers who depict
characters reluctant to adapt to the necessary changes of a modern
society, who fail to recognize the changing tide. But though something
horrible has been \emph{done} to Charley, it is Charley's job to find a
place in the world anyway, despite all that was taken from him.

If other writers managed to capture the aftermath, then the short
stories of Toshio Mori are a time capsule of what life was like for the
Issei and Nisei before the start of the war. Mori, who was born in
Oakland in 1910, worked most of his life at a plant nursery in Northern
California. His dream, however, was to be a writer. During the 1930s and
early '40s, he had amassed a collection of stories titled ``Yokohama,
California.'' With the exception of two stories Mori wrote during and
after his own incarceration, the collection harks back to a time before
everything changed. Mori's stories reveal small acts of courage and
compassion among the Japanese-American community of San Francisco's East
Bay as well as a winking sense of self-awareness, as in ``Akira Yano,''
a story about a young engineering student who wishes to be a writer.
Yano struggles to be published, but his destiny never matches his
ambitions.

Image

Hisaye Yamamoto's ``Seventeen Syllables'' (1988).Credit...Joshua Scott

Image

Toshio Mori's ``Yokohama, California'' (1949).Credit...Joshua Scott

In real life, Mori, faced a fate not unlike his own character.
``Yokohama'' had originally been slated to be published by the Caxton
Printers in Idaho in 1942. But with the arrival of the war, Mori's
publisher likely decided that readers would be unsympathetic to such
human portraits of Japanese-Americans. It wasn't until 1949, seven years
after the initial publication date, and four years after Mori left the
camps, that his first story collection was finally published. The book
went mostly ignored; Mori died in 1980.

AFTER TOO LONG, silence is complicit. We still don't talk about the
camps the way we should: We still call it ``internment,'' we still find
it inconvenient to learn that people died without the help they needed,
that families were separated, that a young man was expected to serve a
country that chose to imprison him. We still use the term
``evacuation,'' even though the guns pointed into the camps, and it's
more than an evacuation when you're being forced to sell the car, sell
the farm, sell the house and pack up forever. The message was clear, and
it's a message many Americans are all too familiar with to this day:
\emph{Go back to where you came from.}

When Wong, Chin, Inada and Chan --- who all became writers themselves,
and edited the seminal 1974 anthology of Asian-American literature,
``\href{https://www.barnesandnoble.com/w/aiiieeeee-frank-chin/1131510723}{Aiiieeeee!}''
--- finally read the copy of ``No-No Boy,'' they were stunned. This was
the great American novel they had been searching for. Their
correspondence with Tuttle, the book's publisher, revealed that Okada
had been working on a second novel, this one about the experience of the
Issei. A few of them arranged to visit Okada's widow, Dorothy, who had
moved into a small apartment in Pasadena, Calif. Okada had died just
nine months earlier from a heart attack at the age of 47. The men asked
her if she had a draft of her husband's second book. She said she had
written to U.C.L.A., asking if they were interested in her husband's
work about the Nisei. When no response arrived, she burned all his
papers. Wong recalled: ``We couldn't believe it. One, we couldn't
believe we put off reading the book. And two, in a fit of grief, she had
burned everything. I remember sitting there just looking at her like
either I wanted to strangle her or tell her how sorry I was.''

Okada's biographer, Abe, told me that there was a time when he had
searched and then waited, hoping that other great works from the Nisei
would emerge. Others have surfaced, but nothing to rival ``No-No Boy.''
Abe suspects it is the only complete novel to exist from its time. The
work of these writers sits in the shadow of other great American
literature --- a truth underscored by the fact that the country was too
intolerant, too apathetic to what happened to them. But it's also true
that Japanese-Americans of Okada's generation were not interested in
reading about themselves. The wounds were too deep, the loss too
overwhelming. It's telling that Okada's widow burned his papers, that
Okada, Yamamoto and Mori all wrote on the side, removed from a larger
literary community. Had Okada lived just a decade longer, he might have
begun to see the beginnings of his own revival, as Asian-American
writers began to become more political, more vocal. Dorothy told Abe
that Okada had spoken of creating a multigenerational trilogy, and I
wonder if the third, about the Sansei --- a generation born during the
postwar baby boom --- would have captured something different about the
Asian-American experience. In Yamamoto's 2001 preface to the four
additional stories that she added to ``Seventeen Syllables,'' she wrote:
``I came to realize that our internment was a trifle compared to the two
hundred years or so of enslavement and prejudice that others in this
country were heir to.'' It is exactly what a Nisei writer would say. But
Yamamoto wrote, I suspect, for another reason. Surely she dreamed of
literary success, as all writers do. But she also wrote because she
understood that if she didn't, no one else would. We are always bound to
repeat ourselves if we don't try to remember.

Advertisement

\protect\hyperlink{after-bottom}{Continue reading the main story}

\hypertarget{site-index}{%
\subsection{Site Index}\label{site-index}}

\hypertarget{site-information-navigation}{%
\subsection{Site Information
Navigation}\label{site-information-navigation}}

\begin{itemize}
\tightlist
\item
  \href{https://help.nytimes3xbfgragh.onion/hc/en-us/articles/115014792127-Copyright-notice}{©~2020~The
  New York Times Company}
\end{itemize}

\begin{itemize}
\tightlist
\item
  \href{https://www.nytco.com/}{NYTCo}
\item
  \href{https://help.nytimes3xbfgragh.onion/hc/en-us/articles/115015385887-Contact-Us}{Contact
  Us}
\item
  \href{https://www.nytco.com/careers/}{Work with us}
\item
  \href{https://nytmediakit.com/}{Advertise}
\item
  \href{http://www.tbrandstudio.com/}{T Brand Studio}
\item
  \href{https://www.nytimes3xbfgragh.onion/privacy/cookie-policy\#how-do-i-manage-trackers}{Your
  Ad Choices}
\item
  \href{https://www.nytimes3xbfgragh.onion/privacy}{Privacy}
\item
  \href{https://help.nytimes3xbfgragh.onion/hc/en-us/articles/115014893428-Terms-of-service}{Terms
  of Service}
\item
  \href{https://help.nytimes3xbfgragh.onion/hc/en-us/articles/115014893968-Terms-of-sale}{Terms
  of Sale}
\item
  \href{https://spiderbites.nytimes3xbfgragh.onion}{Site Map}
\item
  \href{https://help.nytimes3xbfgragh.onion/hc/en-us}{Help}
\item
  \href{https://www.nytimes3xbfgragh.onion/subscription?campaignId=37WXW}{Subscriptions}
\end{itemize}
