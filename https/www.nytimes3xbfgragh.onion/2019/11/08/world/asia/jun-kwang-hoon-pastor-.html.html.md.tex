Sections

SEARCH

\protect\hyperlink{site-content}{Skip to
content}\protect\hyperlink{site-index}{Skip to site index}

\href{https://www.nytimes3xbfgragh.onion/section/world/asia}{Asia
Pacific}

\href{https://myaccount.nytimes3xbfgragh.onion/auth/login?response_type=cookie\&client_id=vi}{}

\href{https://www.nytimes3xbfgragh.onion/section/todayspaper}{Today's
Paper}

\href{/section/world/asia}{Asia Pacific}\textbar{}The Populist Pastor
Leading a Conservative Revival in South Korea

\url{https://nyti.ms/2rqGtjM}

\begin{itemize}
\item
\item
\item
\item
\item
\end{itemize}

Advertisement

\protect\hyperlink{after-top}{Continue reading the main story}

Supported by

\protect\hyperlink{after-sponsor}{Continue reading the main story}

The Saturday profile

\hypertarget{the-populist-pastor-leading-a-conservative-revival-in-south-korea}{%
\section{The Populist Pastor Leading a Conservative Revival in South
Korea}\label{the-populist-pastor-leading-a-conservative-revival-in-south-korea}}

Invoking God, patriotism and family values, the Rev. Jun Kwang-hoon is
creating a backlash against the ``Communizing'' government of President
Moon Jae-in.

\includegraphics{https://static01.graylady3jvrrxbe.onion/images/2019/12/08/world/08skorea-profile-1sub/merlin_164197812_04c8e51c-bdbc-4a28-b777-0f2d61961272-articleLarge.jpg?quality=75\&auto=webp\&disable=upscale}

\href{https://www.nytimes3xbfgragh.onion/by/choe-sang-hun}{\includegraphics{https://static01.graylady3jvrrxbe.onion/images/2018/07/18/multimedia/author-choe-sang-hun/author-choe-sang-hun-thumbLarge.png}}

By \href{https://www.nytimes3xbfgragh.onion/by/choe-sang-hun}{Choe
Sang-Hun}

\begin{itemize}
\item
  Published Nov. 8, 2019Updated Nov. 11, 2019
\item
  \begin{itemize}
  \item
  \item
  \item
  \item
  \item
  \end{itemize}
\end{itemize}

SEOUL, South Korea​ --- Supporters credit him with ``Moses' leadership​
and Solomon's wisdom​.'' Detractors invoke labels like ``narcissistic
demagogue'' and ``fake prophet.''

Whatever else the Rev. Jun Kwang-hoon may be called, there is no denying
that the 63-year-old Presbyterian pastor has become a force to be
reckoned with in South Korea, spearheading a conservative pushback
against President Moon Jae-in. Once dismissed as a crank, Mr. Jun has
attracted huge crowds to
\href{https://www.nytimes3xbfgragh.onion/2019/10/12/world/asia/south-korea-protests.html?searchResultPosition=4}{his
rallies} in central Seoul in recent weeks, forcing Mr. Moon's justice
minister,
\href{https://www.nytimes3xbfgragh.onion/2019/10/14/world/asia/south-korea-cho-kuk-resigns.html?searchResultPosition=3}{Cho
Kuk}​, to step down. He is also demanding Mr. Moon's resignation,
​calling it ``an order from the Lord.''

``We cannot let a madman drive a car,'' Mr. Jun said about Mr. Moon
during an interview, a reference to a comment by a German pastor,
Dietrich Bonhoeffer, \href{https://www.azquotes.com/quote/917469}{on
Hitler}. He adds, without evidence: ``Moon Jae-in is the main North
Korean spy.''

Mr. Jun rouses his crowds, mostly older Christians, by constantly
repeating such incendiary but easy-to-remember tag lines. Progressive
leaders like Mr. Moon are ``Communizing'' South Korea, he says. Those
``followers of North Korea'' are ``driving the country to ruin'' by
prying South Korea away from the United States and taking it closer to
North Korea and China​, he warns​.

Mr. Jun's rise shares many aspects with the surge of Western right-wing
populism: an appeal to patriotism and nativism; a penchant for
ideological and anti-immigrant slurs; a frequent invocation of God and
tradition; and the use of alternative news sources on social media to
spread resentment and stoke fear that the country is in danger of
``collapsing'' or being ``wiped off the face of the Earth.''

Not surprisingly, Mr. Jun is a great admirer of President Trump. When
Mr. Trump visited Seoul in 2017, Mr. Jun's church members took to the
streets, holding placards that read, ``God be with President Trump'' or
``We pray for President Trump.'' He says American evangelical Christians
were ``cheated by Obama'' and elected Mr. Trump to prevent the United
States from being Islamized through immigration.

Mr. Jun says South Korea is ``a child of the United States'' because the
Americans liberated Korea from Japanese colonial rule at the end of
World War II and defended it from Communist invaders during the 1950-53
Korean War. His rallies feature as many
\href{https://www.youtube.com/watch?v=rJ1zzu9xLMk}{American flags}as
South Korean. Speaker after speaker calls anyone they suspect of
undermining the alliance with Washington ``evil'' or ``Satan,'' while
the crowds​ respond with ``Amen!'' or ``Hallelujah!''

​The chances of Mr. Moon resigning are all but nil, and analysts treat
Mr. Jun as a quixotic firebrand whose flame will eventually peter out.
But in past months, the pastor has brewed a political firestorm by
exploiting two powerful sentiments: a fear of North Korea that is
widespread among older South Koreans and
\href{https://www.nytimes3xbfgragh.onion/2019/10/21/world/asia/south-korea-cho-kuk-gold-spoon-elite.html?searchResultPosition=2}{growing
discontent} over an ailing domestic economy.

\includegraphics{https://static01.graylady3jvrrxbe.onion/images/2019/12/08/world/08skorea-profile-3/merlin_164197980_2dfcce77-9e25-47c6-b33e-718e283ca009-articleLarge.jpg?quality=75\&auto=webp\&disable=upscale}

Mr. Moon's office initially dismissed Mr. Jun's
\href{https://www.youtube.com/watch?v=pjyYe_vhLmY}{expletive-ridden
diatribes} against the president as ``not worth commenting upon.'' But
last month, Mr. Moon's Democratic Party asked the police to investigate
Mr. Jun on charges of inciting sedition after he
\href{https://www.youtube.com/watch?v=ntrDZvMqomw}{exhorted followers to
join a band of ``martyrs''} who would invade Mr. Moon's presidential
Blue House to topple him.

``His rallies could be off-putting to non-Christians because they look
like church revival meetings, and some of his remarks, like his claim
that Moon is a North Korean spy, sound over-the-top and
propagandistic,'' said Hwang Gui-hag, editor in chief of the Seoul-based
Law Times, which specializes in church law and news. ``But the thing is,
his strategy works, making him a force that cannot be ignored.''

Mr. Jun​ was born in Yecheon, in central South Korea, the eldest son in
a deeply religious family that was converted ​by American missionaries
​who ​reached​ their village ​by river more than a century ago.

He reached a defining moment in his life when, falling behind in his
school classes, he was sent to live with a relative who was a pastor. By
day he attended a vocational high school in electronics. At night, the
Princeton-educated pastor taught Mr. Jun English and had him read
widely, including the biography of South Korea's autocratic founding
president, Syngman Rhee, another Princeton-educated Christian, who
relied on humanitarian aid from American churches and favored fellow
Christians in his government.

Mr. Jun said he was strongly influenced by the pastor, who was dedicated
to the rights of the urban poor. Schooled in the idea that the church
could serve as an instrument for social and political change, he
enrolled in a seminary after high school.

``Throughout history, the church has always been a political
organization,'' Mr. Jun said.

South Korea's churches have a history of political activism. Progressive
pastors and priests campaigned against the military dictators who ruled
the country in past decades. But conservative pastors equate religious
faith with anti-Communist patriotism. Many of the mega churches in
Seoul, with congregations of tens of thousands, were founded by
evangelical Christians who fled Communist persecution ​in North Korea
​​before the Korean War​.

Image

The Rev. Jun Kwang-hoon this month.Credit...Jean Chung for The New York
Times

Mr. Jun said he began organizing his ``patriots' rallies'' in 2005,
after his high school son came home one day to say that President George
W. Bush ​should be killed. The episode convinced him that unionized
progressive teachers were poisoning children with anti-American and
pro-North Korean ideology.

Mr. Jun's
\href{https://www.youtube.com/watch?v=FuhLN8-Uqjk\&list=UUQM0Zn_5zHh8zdGrfBbAekg\&index=12}{Sarang
Jeil Church} in Seoul claims a congregation of 5,000, and while his
profile is rising, local news outlets have tended to write about him
only to ridicule his ideas.

He once said South Korea should boost its birthrates, one of the world's
lowest, by punishing families that produce fewer than five children. He
also said that South Korea should Christianize itself by incarcerating
all Buddhist monks on an island. ​He once remarked that he was so
trusted in his church that female members ​would ``take their panties
off​''​ before him if he ​told them to.

Mr. Jun dismisses those remarks, saying either that he did not mean them
literally or that the news media had quoted him out of context. ​

But ​Mr. Jun ​still ​calls abortion ``murder'' and says homosexuality
makes the world ``dirty.'' His sermons brim with
\href{https://www.youtube.com/watch?v=5ObAfx2ZuJ4\&list=UUQM0Zn_5zHh8zdGrfBbAekg\&index=14}{Islamophobic
messages,}and he routinely calls Arab immigrants potential
``terrorists.''

In January, Mr. Jun ​won an important political perch when he ​was
elected head of the Christian Council of Korea, ​an umbrella group for
conservative churches.

When he held news conferences with his new title in June to demand Mr.
Moon's resignation, rival pastors called him
``\href{https://www.youtube.com/watch?v=JyNeAT7LCQg}{a son of vipers.''}
The National Council of Churches in Korea denounced him for leading his
followers into ``mass hysteria'' through lies and fake statistics.

Still, most South Koreans did not take Mr. Jun seriously until Mr.
Moon's appointment of Mr. Cho as justice minister in August. Following
news reports of a slew of ethical lapses in Mr. Cho's family, Mr. Jun
pounced. As public ire soared, his weekend rallies ballooned into some
of the largest anti-government protests South Korea has seen in years.

Mr. Jun has strong supporters among right-wing YouTube channels, which
live-stream his rallies and promote viral narratives that spread
resentment and polarize​ the society​. Mr. Jun runs his
\href{http://youtube.com/channel/UC1qldNOqaqIY2PwhbVwIhWA}{own YouTube
channel} and helps fund like-minded YouTubers. He is also sponsoring the
Christian Liberty Party, which hopes to become the first faith-based
political party to win a seat in Parliament in April.

Each night, hundreds of Mr. Jun's followers camp out near Mr. Moon's
office to demand his resignation. When Mr. Jun appears in the morning,
some rush to him offering cash donations and seeking his blessing. Some
come from afar, like Kim Seok-nam, 69, ​who ​flew from Sacramento,
Calif., ​to join the sit-in for a few days. ``He is a latter-day
prophet​,'' Ms. Kim said after Mr. Jun blessed her with a prayer,
putting his hand on her head, while she knelt and wept​. She said she
has donated \$4,000 to Mr. Jun's cause.

``It's the work of the Holy Spirit,'' Mr. Jun said, when asked about his
success. ``It's not me, but people's anger against Moon that brought
them out to my rallies.''

Advertisement

\protect\hyperlink{after-bottom}{Continue reading the main story}

\hypertarget{site-index}{%
\subsection{Site Index}\label{site-index}}

\hypertarget{site-information-navigation}{%
\subsection{Site Information
Navigation}\label{site-information-navigation}}

\begin{itemize}
\tightlist
\item
  \href{https://help.nytimes3xbfgragh.onion/hc/en-us/articles/115014792127-Copyright-notice}{©~2020~The
  New York Times Company}
\end{itemize}

\begin{itemize}
\tightlist
\item
  \href{https://www.nytco.com/}{NYTCo}
\item
  \href{https://help.nytimes3xbfgragh.onion/hc/en-us/articles/115015385887-Contact-Us}{Contact
  Us}
\item
  \href{https://www.nytco.com/careers/}{Work with us}
\item
  \href{https://nytmediakit.com/}{Advertise}
\item
  \href{http://www.tbrandstudio.com/}{T Brand Studio}
\item
  \href{https://www.nytimes3xbfgragh.onion/privacy/cookie-policy\#how-do-i-manage-trackers}{Your
  Ad Choices}
\item
  \href{https://www.nytimes3xbfgragh.onion/privacy}{Privacy}
\item
  \href{https://help.nytimes3xbfgragh.onion/hc/en-us/articles/115014893428-Terms-of-service}{Terms
  of Service}
\item
  \href{https://help.nytimes3xbfgragh.onion/hc/en-us/articles/115014893968-Terms-of-sale}{Terms
  of Sale}
\item
  \href{https://spiderbites.nytimes3xbfgragh.onion}{Site Map}
\item
  \href{https://help.nytimes3xbfgragh.onion/hc/en-us}{Help}
\item
  \href{https://www.nytimes3xbfgragh.onion/subscription?campaignId=37WXW}{Subscriptions}
\end{itemize}
