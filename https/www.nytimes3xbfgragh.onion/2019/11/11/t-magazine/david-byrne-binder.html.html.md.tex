Sections

SEARCH

\protect\hyperlink{site-content}{Skip to
content}\protect\hyperlink{site-index}{Skip to site index}

\href{https://myaccount.nytimes3xbfgragh.onion/auth/login?response_type=cookie\&client_id=vi}{}

\href{https://www.nytimes3xbfgragh.onion/section/todayspaper}{Today's
Paper}

David Byrne and David Binder on Breaking Into the Mainstream

\url{https://nyti.ms/2X1BtO4}

\begin{itemize}
\item
\item
\item
\item
\item
\item
\end{itemize}

Advertisement

\protect\hyperlink{after-top}{Continue reading the main story}

Supported by

\protect\hyperlink{after-sponsor}{Continue reading the main story}

Admiration Society

\hypertarget{david-byrne-and-david-binder-on-breaking-into-the-mainstream}{%
\section{David Byrne and David Binder on Breaking Into the
Mainstream}\label{david-byrne-and-david-binder-on-breaking-into-the-mainstream}}

We interview two creative people in different fields in one wide-ranging
conversation. This time: the producer and the musician.

\includegraphics{https://static01.graylady3jvrrxbe.onion/images/2019/11/17/t-magazine/17tmag-byrne/17tmag-byrne-articleLarge.jpg?quality=75\&auto=webp\&disable=upscale}

By Boris Kachka

\begin{itemize}
\item
  Nov. 11, 2019
\item
  \begin{itemize}
  \item
  \item
  \item
  \item
  \item
  \item
  \end{itemize}
\end{itemize}

One measure of artistic success is the ability to move the mainstream
--- to bring once-unimaginable ideas to the widest possible audience. By
that metric, few people have been as successful as the musician
\href{https://www.nytimes3xbfgragh.onion/topic/person/david-byrne}{David
Byrne} and the producer
\href{https://www.bam.org/about/leadership/david-binder}{David Binder}.
Byrne, of course, headlined the '80s rock band
\href{https://www.nytimes3xbfgragh.onion/topic/organization/the-talking-heads}{Talking
Heads}, which dissolved the barriers between disco and rock, conceptual
art and dance pop. Even before going solo in 1989, Byrne's side projects
tentacled into disparate worlds: He collaborated with similarly
polymathic artists such as Brian Eno, Robert Wilson and
\href{https://www.nytimes3xbfgragh.onion/topic/person/twyla-tharp}{Twyla
Tharp}; he starred in the writer-director Jonathan Demme's 1984
documentary ``Stop Making Sense''; and two years later, he directed the
Robert Altman-on-acid jukebox art film ``True Stories.'' In the '90s and
beyond, Byrne used his fame to advance and collaborate with a wide range
of artists --- the world-music acts on his label,
\href{https://luakabop.com/}{Luaka Bop}; the D.J. Fatboy Slim on
``\href{https://www.nytimes3xbfgragh.onion/2013/04/07/theater/david-byrnes-here-lies-love-about-imelda-marcos.html}{Here
Lies Love},'' their 2013 musical about
\href{https://www.nytimes3xbfgragh.onion/topic/person/imelda-r-marcos}{Imelda
Marcos}; the singer
\href{https://www.nytimes3xbfgragh.onion/2018/02/16/t-magazine/alex-da-corte-st-vincent.html}{St.
Vincent} on their joint 2012 album ``Love This Giant.'' His latest act
of radical pop is ``American Utopia,'' an album and touring rock show
(which debuted in 2018) that wears its theme lightly but seriously
reimagines space: Mobile musicians hop across a bare stage, dancing to
their own music. The work opened on Broadway last month and runs through
Feb. 16.

It was inevitable that Byrne, now 67, would cross paths with the
Brooklyn Academy of Music, one of America's most powerful arts
institutions. He wrote lyrics for the composer
\href{https://www.nytimes3xbfgragh.onion/topic/person/philip-glass}{Philip
Glass}'s ``The Photographer: Far From the Truth,'' a mixed-media work
about the 19th-century English photographer Eadweard Muybridge that
premiered in the United States at BAM's inaugural Next Wave Festival in
1983. More recently, in 2012, he designed the word-art-inspired bike
racks in front of its Peter Jay Sharp Building.

\emph{{[}}\href{https://www.nytimes3xbfgragh.onion/newsletters/t-list?module=inline}{\emph{Sign
up here}} \emph{for the T List newsletter, a weekly roundup of what T
Magazine editors are noticing and coveting now.{]}}

Last year, David Binder, 52, was
\href{https://www.nytimes3xbfgragh.onion/2018/02/07/theater/broadway-producer-named-bams-new-artistic-director.html}{hired}
to succeed
\href{https://www.nytimes3xbfgragh.onion/2018/09/13/theater/next-wave-festival-joseph-melillo-brooklyn-academy-of-music.html}{Joseph
Melillo} as the artistic director of BAM. Like Byrne, Binder has always
been adept at bringing the cutting edge to the masses. Early in his
career, he helped turn John Cameron Mitchell's crazy rock-opera idea
into ``Hedwig and the Angry Inch,'' eventually bringing it to Broadway
in 2014. In 1998, he placed a bet on an Argentine troupe to make ``De La
Guarda,'' creating not only an Off Broadway money machine but a
contemporary template for immersive theater. These hits powered his dual
career as a festival guru and Broadway macher: Binder was a producer of
the 2007 High Line Festival (which raised funds to help develop New
York's actual High Line) as well as the guest artistic director of last
year's London International Festival of Theater. Meanwhile, he played
the Broadway celebrity game with savvy, casting
\href{https://www.nytimes3xbfgragh.onion/topic/person/sean-combs}{Sean
Combs} to draw crowds to 2004's revival of ``A Raisin in the Sun,''
James Franco to star in 2014's ``Of Mice and Men'' revival and Adam
Driver and Keri Russell to headline this year's revival of
``\href{https://www.nytimes3xbfgragh.onion/2019/03/07/theater/adam-driver-keri-russell-burn-this-star-wars.html}{Burn
This}.'' One of Binder's first moves at BAM was to reimagine the Next
Wave Festival, which this year replaces Byrne's generation of mainstays
(\href{https://www.nytimes3xbfgragh.onion/topic/person/laurie-anderson}{Laurie
Anderson},
\href{https://www.nytimes3xbfgragh.onion/2019/06/04/arts/music/atlas-la-philharmonic-meredith-monk.html}{Meredith
Monk}) with a slate of artists never before seen on a BAM stage. Such an
evolution delighted Byrne when the two men met recently, sharing ideas
and anecdotes over breakfast at Morandi in New York's West Village.

\includegraphics{https://static01.graylady3jvrrxbe.onion/images/2019/11/12/t-magazine/12tmag-americanutopia-02/12tmag-americanutopia-02-articleLarge.jpg?quality=75\&auto=webp\&disable=upscale}

\textbf{David Binder:} I saw
``\href{https://americanutopiabroadway.com/}{American Utopia}'' last
year at the Kings Theater in Brooklyn, and I just thought it was one of
the great rock concerts. There isn't a kind of existing language to
define what it is --- it's unclassifiable. You've talked about really
using the empty space ...

\textbf{David Byrne:} Yeah, I imagined that it might be possible to have
all the musicians untethered. No mic stands, no platforms, no drums, no
monitor wedges, no water bottles even, which is a little tough on me,
but that's fine. I realized that you could take contemporary dance and
mix it with pop, put it in front of an audience that has never seen a
contemporary dance performance in their life, and they love it.

\textbf{Binder:} That's always the fun of it. That's what I think is so
fantastic about it going to Broadway, because it allows you to knock
down all these conceptions: What is commercial? What's a nonprofit work,
what's an uptown work, what's a downtown work? That is always the thing
about you --- for 30 years. I have always loved doing that, too. But I
was thinking about the title: ``American Utopia.'' You've said it's not
irony; it's an imagining of what could be.

\textbf{Byrne:} That's a pretty good summary.

\textbf{Binder:} There's a kind of optimism in your work, which is not
something you traditionally associate with downtown artists of a certain
era. You say very optimistic things about the future but in tension with
very negative things. I think the audience is willing to go with you
because they ultimately walk away with not just darkness but some sense
of possibility.

\textbf{T:} But aren't the things you've both made --- pop music,
Broadway shows, art films --- often aimed at different audiences?

\textbf{Byrne:} I don't let the marketing drive the creative stuff, but
I do think at some point, ``Oh, this is going to have a limited appeal,
so don't get your hopes up.'' We don't always measure success by ticket
sales, but going on Broadway, the dream for me --- and I don't think it
will happen --- is to eventually get an audience that's never heard of
me, never heard of Talking Heads, but has heard this is a good show.

Image

Keri Russell as Anna and Adam Driver as Pale in the play ``Burn This,''
produced by David Binder.Credit...Sara Krulwich/The New York Times

\textbf{Binder:} That's always my goal --- to get the people who
wouldn't go see a story of something like ``Hedwig,'' wouldn't go see a
character who had a botched sex-change operation and a punk rock score.
So, David, I talked to the BAM archivists and I didn't know this, but
you were a contributor to the very first Next Wave Festival, for ``The
Photographer.'' What was it like for you to work in that moment, before
BAM had the kind of fame it would have three or four years later?

\textbf{Byrne:} I knew Phil Glass and {[}the director{]} JoAnne
Akalaitis from
\href{https://www.nytimes3xbfgragh.onion/2007/08/31/theater/31mabo.html}{Mabou
Mines}, the theater group {[}founded in 1970{]}. When I moved to New
York in the mid-70s {[}having grown up mostly in Baltimore{]}, I started
going to see downtown theater, and it kind of blew my mind. Mabou Mines
and the Wooster Group and the experimental stage director Bob Wilson ---
I'd never seen anything like it. And I thought, ``Well, this is as
exciting for me as when I first heard rock 'n' roll on a transistor
radio as a kid.'' A door opens and you go, ``Look at all this, it's
possible!'' I was just thrilled.

\textbf{Binder:} I grew up in Los Angeles, and mostly I was exposed to
touring shows: chandelier-dropping, barricade-busting big Broadway
musicals via London. When I moved to New York in 1990, someone took me
to BAM --- a really good friend. She taught a class, and she brought
them to the entire season of Next Wave: They had season tickets. And
very early on I saw John Adams's 1991 opera ``The Death of Klinghoffer''
{[}with a book by Alice Goodman{]}. I was actually working in
\href{https://www.nytimes3xbfgragh.onion/2006/01/29/magazine/william-ivey-long-keeps-his-clothes-on.html}{William
Ivey Long}'s costume shop on
{[}\href{https://www.nytimes3xbfgragh.onion/2017/10/16/t-magazine/lin-manuel-miranda-stephen-sondheim.html}{Stephen
Sondheim} and John Weidman's 1990 musical{]} ``Assassins,'' and that was
beyond my wildest dreams. But going to BAM completely reshaped my
aesthetic.

\textbf{Byrne:} Did you do some shows before ``Hedwig''?

\textbf{Binder:} Right around the same time, I did ``De La Guarda,'' and
David, you were at the very first performance!

\textbf{Byrne:} Could have been! An Argentine musician friend of mine,
his wife was in the company. So I went with my daughter. They had these
bungee cords, they would come down and pick up a member of the audience
and lift them up. And my daughter was traumatized. She was really little
and she said, ``My dad has been abducted by a man with a hairy butt!''
{[}\emph{All laugh}{]}

\textbf{Binder:} It was the most expensive Off Broadway show ever, and
the numbers made no sense. Everyone said it was for young people and
young people won't come. It opened in June of '98, and it lost money
every single week until November, and then it clicked. This was
pre-social media --- now there'd be a way to talk about it. That didn't
exist. But I remember that, on the very first day, I thought, ``Maybe
we're not so nuts because David Byrne is here. He knows what's going
on.''

\textbf{T:} Immersive theater can be a hard sell for critics, and some
artists, too.

\textbf{Byrne:} Maybe when I was younger, I might have been more like,
``Just let me watch.'' But as I get older, I'm ready to jump in. I went
to see
\href{https://www.nytimes3xbfgragh.onion/interactive/2019/04/10/t-magazine/jackie-sibblies-drury-play.html}{Jackie
Sibblies Drury}'s Off Broadway play ``Fairview'' last summer, and when
it comes to the point {[}\emph{spoiler alert}{]} where the audience and
the actors switch places, nobody moved. So I thought, ``I guess I gotta
do it. I gotta be the first one up there.''

\textbf{T:} You also dabbled in immersive theater in
``\href{https://www.nytimes3xbfgragh.onion/2016/10/04/arts/next-from-david-byrne-neuroscience-in-an-art-gallery.html}{The
Institute Presents: Neurosociety},'' a 2016 exhibit in Silicon Valley
that reframed scientific experiments as a theatrical experience. That
kind of thing seems increasingly popular.

\textbf{Byrne:} I think it's more common as a kind of a marketing thing.
And it becomes an Instagram kind of thing. Some business writers in the
'90s published a book called ``The Experience Economy.'' And they said
it's really successful if they can get shoppers to pay to go shop, which
is kind of what Disneyland does.

Image

Members of the Argentine theater company De La Guarda, whom David Binder
brought to New York in 1998, are suspended over the audience in the show
``Villa Villa.''Credit...Sara Krulwich/The New York Times

\textbf{Binder:} At the Glastonbury Festival, there's a group called
Block9. Since 2016, they've built this nightclub area that's recreated a
warehouse in the meatpacking district circa 1982, but like exactly ---
and to scale. They're taking it back away from the marketing --- making
extraordinary work. But David, you've done so many things. Do you think
there was a time when there were more artists who worked across genres?
What enabled people to do that?

\textbf{Byrne:} I don't know. The audience was sometimes really small,
so maybe the risk wasn't so high. And the cost of living was cheaper. I
remember the early CBGB days when suddenly all these fine artists
decided they were going to be in bands, which brought a different
aesthetic to music-making. And then not too many years later, it went
totally the other way. Everybody who could was suddenly painting.

\textbf{Binder:} It's interesting also that now {[}in New York{]}, you
have {[}multipurpose performance venues like{]} the
\href{http://www.armoryonpark.org/}{Park Avenue Armory} and
\href{https://theshed.org/}{the Shed}, but back then, BAM was the only
place where large-scale work by that whole community of artists could be
presented.

\textbf{T:} But that was a relatively small community of young artists.
How do you nurture that now that BAM is a juggernaut?

\textbf{Binder:} I always want the work to have the widest possible
audience. Doing something like ``Hedwig'' on Broadway, the only thing
commercial about it was that it succeeded. I feel like those kinds of
boundaries and categories are all falling down, and things that are at
the intersection of multiple genres are, of course, BAM's bread and
butter. But the idea of the Next Wave Festival this season is to go back
to {[}the longtime BAM president{]} Harvey Lichtenstein and Joe
Melillo's original intention. So every single artist in the Next Wave
Festival is brand-new to BAM.

\textbf{Byrne:} It's about time. It's exciting. There's a good chance
everything won't be for everyone. But there are places that have
cultivated an audience who will go to see something they've never heard
of. And that can take a little while, but then once you build up that
trust, they'll see almost whatever you put on.

\textbf{Binder:} We will eventually reintroduce some of the more legacy
BAM artists into Next Wave, but we're going to try new folks for a
couple of years. The things we're looking for could probably not be done
anywhere else in the city. Like this show, ``What if They Went to
Moscow?'' by Christiane Jatahy. Half the audience goes into BAM Rose
Cinemas and half goes into the Fisher theater, and in the Fisher, they
see a play inspired by Chekhov's ``Three Sisters,'' which is filmed with
multiple cameras and screened live in the cinema. And then the audiences
switch places. I'm also interested in site-specific work. New York
organizations tend not to want to get out of the buildings. So we're
going to try that this fall, too.

Image

Byrne, center, in ``American Utopia.''Credit...Sara Krulwich/The New
York Times

\textbf{Byrne:} So, how do you find the shows? Do you spend months going
to Edinburgh Fringe and various festivals?

\textbf{Binder:} I think that Joe Melillo is superhuman. He'd be like,
``Oh it's Tuesday, I'm going to Paris. I'll be back on Thursday.'' I'm
not built that way. I tend to go in bigger groupings. So, this
Wednesday, I'm going to Zurich, Hamburg, the Netherlands, Edinburgh,
Amsterdam and London in nine days. And then that will be it for about a
month. But I always feel like the New York audience does feel
idiosyncratic. Just because the show is working in those cities, I don't
necessarily think it would here. David, how were the audiences different
with ``American Utopia?''

\textbf{Byrne:} It's pretty consistent, but occasionally we'd get
surprised. In Santiago, Chile, we did a festival, and I don't think this
happened anywhere else: A large portion of the audience copied the
choreography. So if we did some gesture, they would all start doing it
back to us --- hundreds of people. I'm going to guess it's because
Santiago has this tradition of mass movements. A number of years ago,
they had these huge street protests. They were like performances. There
was one where the entire group did the dance from
\href{https://www.nytimes3xbfgragh.onion/topic/person/michael-jackson}{Michael
Jackson}'s ``Thriller.''

\textbf{Binder:} That's how I learned to produce. Seriously! Doing
street demonstrations and actions in Act Up. One time, Broadway
Cares/Equity Fights AIDS hired me to produce this action about
Guantánamo Bay {[}where the United States quarantined Haitian refugees
with H.I.V. in 1993{]}. We staged this giant thing in Rockefeller
Center, with a Statue of Liberty wrapped up in chains. Jonathan Demme
was there, and I remember having this conversation with him. He was
like, ``I don't know if I should get arrested.'' He's trying to work it
out with me and I'm, like, 25. And he decided he would, so he did.

\textbf{T:} That's collaboration! ``American Utopia'' is also, in a
sense, about the utopian world of collaboration. That seems to define
both of your careers. Does it also help keep the work fresh?

\textbf{Byrne:} Oh yeah. It pushes me out of my comfort zone, the things
I do by habit. I know that there are some artists who feel that
collaboration betrays their vision as an artist. There are some who feel
like, ``My personal vision is sacred. I don't want to dilute it.'' I
find that to be riskier --- you have this danger of falling into the
trap of only being inside yourself.

\emph{This interview has been edited and condensed.}

Advertisement

\protect\hyperlink{after-bottom}{Continue reading the main story}

\hypertarget{site-index}{%
\subsection{Site Index}\label{site-index}}

\hypertarget{site-information-navigation}{%
\subsection{Site Information
Navigation}\label{site-information-navigation}}

\begin{itemize}
\tightlist
\item
  \href{https://help.nytimes3xbfgragh.onion/hc/en-us/articles/115014792127-Copyright-notice}{©~2020~The
  New York Times Company}
\end{itemize}

\begin{itemize}
\tightlist
\item
  \href{https://www.nytco.com/}{NYTCo}
\item
  \href{https://help.nytimes3xbfgragh.onion/hc/en-us/articles/115015385887-Contact-Us}{Contact
  Us}
\item
  \href{https://www.nytco.com/careers/}{Work with us}
\item
  \href{https://nytmediakit.com/}{Advertise}
\item
  \href{http://www.tbrandstudio.com/}{T Brand Studio}
\item
  \href{https://www.nytimes3xbfgragh.onion/privacy/cookie-policy\#how-do-i-manage-trackers}{Your
  Ad Choices}
\item
  \href{https://www.nytimes3xbfgragh.onion/privacy}{Privacy}
\item
  \href{https://help.nytimes3xbfgragh.onion/hc/en-us/articles/115014893428-Terms-of-service}{Terms
  of Service}
\item
  \href{https://help.nytimes3xbfgragh.onion/hc/en-us/articles/115014893968-Terms-of-sale}{Terms
  of Sale}
\item
  \href{https://spiderbites.nytimes3xbfgragh.onion}{Site Map}
\item
  \href{https://help.nytimes3xbfgragh.onion/hc/en-us}{Help}
\item
  \href{https://www.nytimes3xbfgragh.onion/subscription?campaignId=37WXW}{Subscriptions}
\end{itemize}
