Sections

SEARCH

\protect\hyperlink{site-content}{Skip to
content}\protect\hyperlink{site-index}{Skip to site index}

\href{https://www.nytimes3xbfgragh.onion/section/world/asia}{Asia
Pacific}

\href{https://myaccount.nytimes3xbfgragh.onion/auth/login?response_type=cookie\&client_id=vi}{}

\href{https://www.nytimes3xbfgragh.onion/section/todayspaper}{Today's
Paper}

\href{/section/world/asia}{Asia Pacific}\textbar{}Behind Hong Kong's
Protesters, an Army of Volunteer Pastors, Doctors and Artists

\url{https://nyti.ms/34KxVT4}

\begin{itemize}
\item
\item
\item
\item
\item
\item
\end{itemize}

Advertisement

\protect\hyperlink{after-top}{Continue reading the main story}

Supported by

\protect\hyperlink{after-sponsor}{Continue reading the main story}

\hypertarget{behind-hong-kongs-protesters-an-army-of-volunteer-pastors-doctors-and-artists}{%
\section{Behind Hong Kong's Protesters, an Army of Volunteer Pastors,
Doctors and
Artists}\label{behind-hong-kongs-protesters-an-army-of-volunteer-pastors-doctors-and-artists}}

\includegraphics{https://static01.graylady3jvrrxbe.onion/images/2019/11/11/world/00hk-supportnetwork-8/merlin_162727938_c8dfa5c8-9644-49e0-b736-237b2d13d30b-articleLarge.jpg?quality=75\&auto=webp\&disable=upscale}

By \href{https://www.nytimes3xbfgragh.onion/by/andrew-jacobs}{Andrew
Jacobs}

\begin{itemize}
\item
  Published Nov. 11, 2019Updated Nov. 25, 2019
\item
  \begin{itemize}
  \item
  \item
  \item
  \item
  \item
  \item
  \end{itemize}
\end{itemize}

\href{https://cn.nytimes3xbfgragh.onion/china/20191112/hong-kong-protests-volunteer/}{阅读简体中文版}\href{https://cn.nytimes3xbfgragh.onion/china/20191112/hong-kong-protests-volunteer/zh-hant/}{閱讀繁體中文版}

HONG KONG --- The pastor pulled on his respirator and ran directly into
the fog of tear gas in central Hong Kong. He was trailed by a homemaker,
a retired accountant and a middle-school teacher.

Undaunted by the pandemonium of
\href{https://www.nytimes3xbfgragh.onion/2019/10/27/world/asia/hong-kong-protests-test-china.html}{gasping
protesters}, they pointed people to safety and poured saline into the
eyes of those overcome by the fumes.

With their yellow vests and portable loudspeakers, Pastor Ka-Kit Ao and
his volunteers are an unmistakable presence at the antigovernment
protests that
\href{https://www.nytimes3xbfgragh.onion/2019/10/31/world/asia/hong-kong-protests.html}{have
upended} this semiautonomous
\href{https://www.nytimes3xbfgragh.onion/2019/11/06/world/asia/hong-kong-protests-china-national-security.html}{Chinese
territory}. They form human cordons between protesters and
\href{https://www.nytimes3xbfgragh.onion/2019/10/03/world/asia/hong-kong-protests-police.html}{advancing
police}. They beg baton-swinging officers to go easy. And they solicit
the names of those being hustled away in handcuffs so pro bono lawyers
can follow up with assistance.

``I sometimes wonder whether we are doing anything of value, but we
can't just sit at home,'' Pastor Ao, 34, said one recent afternoon
before heading into the maelstrom with members of his group, Protect the
Children.

Now entering their sixth month, Hong Kong's protests have been notable
for\href{https://www.nytimes3xbfgragh.onion/interactive/2019/world/asia/hong-kong-protests-arc.html}{their
longevity}, and for the huge throngs willing to defy the authorities
with their
\href{https://www.nytimes3xbfgragh.onion/2019/11/25/world/asia/hong-kong-election-protests.html}{demands
for democracy} and police accountability. Thousands of protesters,
including office workers,
\href{https://www.nytimes3xbfgragh.onion/2019/11/11/world/asia/hong-kong-protests-shooting.html}{descended
Tuesday on Central}, the main business and shopping district, forcing
businesses to close and paralyzing traffic and the city's fabled tram
service.

\includegraphics{https://static01.graylady3jvrrxbe.onion/images/2019/11/11/world/00hk-support-network-top/merlin_157939323_54569567-0314-4347-b143-6eaa0e9620b8-articleLarge.jpg?quality=75\&auto=webp\&disable=upscale}

Behind the scenes, this largely leaderless movement has been sustained
by a vast network of ordinary people who hand out bottled water and red
bean soup at marches, drive home stranded protesters late at night and
donate the gas masks that fortify demonstrators during their pitched
battles with police. Hong Kong professionals have been especially vital.

Graphic artists create the eye-catching
\href{https://www.nytimes3xbfgragh.onion/2019/10/11/world/asia/hong-kong-protest-art.html}{protest
posters} across the city. Psychologists provide free counseling to the
emotionally distressed. And emergency room doctors, working in
clandestine clinics, set shattered bones.

One measure of community spirit can be heard many nights at 10 p.m.,
when residents in densely packed neighborhoods open their windows and
shout protest slogans to the heavens. Another is expressed through the
crowdfunding campaigns that have raised millions of dollars for medical
treatment, legal defense funds and other expenses.

``Without this public support, the movement would have lost steam a lot
sooner,'' said Victoria Hui, a political scientist at the University of
Notre Dame and the author of a book about
\href{https://www.nytimes3xbfgragh.onion/2019/04/23/world/asia/hong-kong-umbrella-movement.html}{the
Umbrella Movement}, the 2014 pro-democracy protests that fizzled after
10 weeks. ``It encourages young people to keep going, giving them the
sense they are not alone and that what they are doing is righteous.''

Although actions like setting the man on fire risk eroding support, the
protest movement so far has enjoyed broad backing among Hong Kong's
seven million people. A
\href{https://www.independent.co.uk/voices/hong-kong-protests-police-violence-public-opinion-polling-support-a9158061.html}{recent
survey} by the Chinese University of Hong Kong found that nearly 60
percent of respondents approved of the protesters'
\href{https://www.nytimes3xbfgragh.onion/2019/10/27/world/asia/hong-kong-protests.html}{violent
tactics}, agreeing that they were justified in the face of an
increasingly aggressive police response and a government unwilling to
compromise.

Image

Posters and protest-themed art on a wall in the Ma On Shan
neighborhood.Credit...Lam Yik Fei for The New York Times

This public support presents a thorny challenge to the authorities, who
have been hoping to quell the protests by driving a wedge between
\href{https://www.nytimes3xbfgragh.onion/2019/09/27/world/asia/hong-kong-protests-identity.html}{the
increasingly radical agitators and those sympathetic} to their cause.

``The more the government suppresses this movement and tries to scare
people, the more people will step out and stand up,'' said Pastor Roy
Chan, a founder of Protect the Children, which has nearly 200 members.

The encrypted messaging app Telegram serves as the town hall for the
support network, with dozens of channels that match volunteers to those
in need. Most prolific are the channels offering rides to protesters
affected by the subway shutdowns that the authorities impose to dampen
protest turnout. The rides also help protesters avoid the police sweeps
that target public buses.

Like many drivers, Patrick Chan, 38, a garment factory manager, said
fear of arrest kept him away from the protests, most of which the police
have deemed illegal. Guilt and shame, though, are powerful motivators.

Mr. Chan spends hours in his beat-up BMW sedan ferrying weary,
sweat-drenched protesters to housing complexes across the city.

\href{https://www.nytimes3xbfgragh.onion/interactive/2019/11/02/world/asia/hong-kong-protest-photos.html}{}

\includegraphics{https://static01.graylady3jvrrxbe.onion/images/2019/11/02/world/02hkg-promo/merlin_163714761_3d8ff863-86b7-488d-ae58-74beaebb059c-articleLarge.jpg}

\hypertarget{photographing-hong-kongs-urban-battleground}{%
\subsection{Photographing Hong Kong's Urban
Battleground}\label{photographing-hong-kongs-urban-battleground}}

Follow a Times photojournalist through a day of antigovernment protests
filled with tear gas, arrests and Molotov cocktails.

``These young people are trying to right the wrongs that we have long
been avoiding,'' he said, referring to Beijing's two-decade effort to
chip away at the vaunted liberties that differentiate this former
British colony from mainland China. ``They are paying with their
futures, risking the possibility of being locked up for years. We owe
them.''

The sense of public service has also mobilized dozens of doctors, nurses
and medics. Much of their work takes place in secret. That is because
all but the most grievously injured protesters avoid Hong Kong's
hospitals following the arrest in June of several people who had sought
care for broken bones and blunt trauma. These days, the injured are
sometimes treated at clandestine clinics that provide X-rays and
rudimentary surgery.

Dr. Tim Wong works the protests after his regular hospital shift. An
emergency room doctor, he decided to act after the police made a number
of arrests at his hospital, which he declined to name for fear that it
might endanger his employment.

``Since then, no one has come to our emergency room for treatment,
unless they are escorted by the police,'' he said. ``It's outrageous.
Hospitals should be sanctuaries.''

One recent evening, he hovered near the front lines of a skirmish as
Molotov cocktails, bricks and tear gas canisters arced overhead. Many of
those needing medical treatment were bystanders caught up in the mayhem.

Image

Medical workers staged an anti-government protest in the lobby of Queen
Elizabeth Hospital in September.Credit...Lam Yik Fei for The New York
Times

Just then, Pastor Ao and another member of his group rushed by carrying
a man injured by a tear-gas canister. All three of them were weeping.
``I can't believe this is happening to our city,'' the pastor wailed as
they dragged the man to a first aid clinic inside a Methodist Church
that has become a beacon for protesters.

Earlier that afternoon, Pastor Ao and scores of volunteers had gathered
at a subway station to plot the day's movements. After dividing up into
teams of seven, he reminded everyone to refrain from chanting slogans
and urged them to be polite to law enforcement authorities.

``They might call us cockroaches but we should refer to them as police
officers,'' he said. Then everyone bowed their heads in prayer. ``May we
have God's protection and the patience, love and wisdom to deal with the
police,'' Pastor Ao said.

Volunteers say the police rarely return the favor, treating them as
antagonists. In September, the police were widely criticized after
\href{https://www.nytimes3xbfgragh.onion/2019/09/24/world/asia/hong-kong-yellow-object.html}{a
video emerged} that appeared to show a knot of officers kicking a
Protect the Children member as he lay on the ground. The man, wearing
the group's trademark yellow vest, was later arrested.

At a news conference, a senior police official dismissed allegations of
abuse, suggesting that the video had been doctored and that what many
saw as a person was actually ``a yellow object.'' In the weeks that
followed, the group's ranks swelled with new recruits, Pastor Ao said.

Image

A volunteer driver is stopped at a roadblock set up by protesters in
September.Credit...Laurel Chor for The New York Times

Many of the group's volunteers are retirees like Ah Lin He. A fiery,
reed-thin woman, Ms. He, 68, was born in the Chinese city of Guangzhou
and swam to Hong Kong in 1972 to escape the chaos of the Cultural
Revolution. She doggy-paddled for 10 hours with five other people. Only
three of them made it to shore.

``I've seen the repression and madness that can be unleashed by the
Communists in China,'' she said as the group trudged to a protest that
\href{https://www.nytimes3xbfgragh.onion/2019/10/07/world/asia/hong-kong-protesters-masks-violence.html}{had
turned violent}.

Walking beside her was Joe Pao, a 29-year-old pastor, who joined the
group after a brief stint as a protester. ``I realized I could do
something more useful than throwing bricks,'' Pastor Pao said.

He acknowledged that his role as a putative peacemaker was rarely
gratifying. Most of his work involves urging the police to exercise
restraint. ``When they catch people, we tell them to please respect the
powers they have and not abuse them,'' he said. ``The impact is
definitely small.''

The majority of protest supporters operate more independently. Nam Kwan,
a cultural foundation administrator, has fed, housed and comforted
scores of youths whose parents, enraged by their participation in the
protests, tossed them out of their homes.

She traces her transformation from silent sympathizer to frenetic den
mother to June 12, when the
police\href{https://www.nytimes3xbfgragh.onion/2019/06/12/world/asia/hong-kong-protests.html}{escalated
their tactics} by firing rubber bullets and beanbag rounds at unarmed
protesters.

``When I heard the first gunshot, a bell rang inside me and I
automatically found my place,'' she said. ``Nowadays my phone is on 24
hours a day because I'm afraid I might miss urgent messages or calls for
help.''

In addition to buying protective gear for protesters, she coordinates
financial support and car pools from wealthy friends eager to help but
reluctant to do so publicly. Oftentimes, she finds herself on the
street, dispensing hugs or patiently listening to the worries of young
protesters.

``Every time these kids go to the front lines, they fear for their
lives,'' she said. ``But what they fear more is abandonment, that one
day we will all turn our backs and leave them alone.''

Advertisement

\protect\hyperlink{after-bottom}{Continue reading the main story}

\hypertarget{site-index}{%
\subsection{Site Index}\label{site-index}}

\hypertarget{site-information-navigation}{%
\subsection{Site Information
Navigation}\label{site-information-navigation}}

\begin{itemize}
\tightlist
\item
  \href{https://help.nytimes3xbfgragh.onion/hc/en-us/articles/115014792127-Copyright-notice}{©~2020~The
  New York Times Company}
\end{itemize}

\begin{itemize}
\tightlist
\item
  \href{https://www.nytco.com/}{NYTCo}
\item
  \href{https://help.nytimes3xbfgragh.onion/hc/en-us/articles/115015385887-Contact-Us}{Contact
  Us}
\item
  \href{https://www.nytco.com/careers/}{Work with us}
\item
  \href{https://nytmediakit.com/}{Advertise}
\item
  \href{http://www.tbrandstudio.com/}{T Brand Studio}
\item
  \href{https://www.nytimes3xbfgragh.onion/privacy/cookie-policy\#how-do-i-manage-trackers}{Your
  Ad Choices}
\item
  \href{https://www.nytimes3xbfgragh.onion/privacy}{Privacy}
\item
  \href{https://help.nytimes3xbfgragh.onion/hc/en-us/articles/115014893428-Terms-of-service}{Terms
  of Service}
\item
  \href{https://help.nytimes3xbfgragh.onion/hc/en-us/articles/115014893968-Terms-of-sale}{Terms
  of Sale}
\item
  \href{https://spiderbites.nytimes3xbfgragh.onion}{Site Map}
\item
  \href{https://help.nytimes3xbfgragh.onion/hc/en-us}{Help}
\item
  \href{https://www.nytimes3xbfgragh.onion/subscription?campaignId=37WXW}{Subscriptions}
\end{itemize}
