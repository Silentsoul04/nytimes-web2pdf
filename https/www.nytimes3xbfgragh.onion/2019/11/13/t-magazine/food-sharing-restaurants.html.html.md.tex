Sections

SEARCH

\protect\hyperlink{site-content}{Skip to
content}\protect\hyperlink{site-index}{Skip to site index}

\href{https://myaccount.nytimes3xbfgragh.onion/auth/login?response_type=cookie\&client_id=vi}{}

\href{https://www.nytimes3xbfgragh.onion/section/todayspaper}{Today's
Paper}

I'll Have What She's Having

\url{https://nyti.ms/34RuwBY}

\begin{itemize}
\item
\item
\item
\item
\item
\end{itemize}

Advertisement

\protect\hyperlink{after-top}{Continue reading the main story}

Supported by

\protect\hyperlink{after-sponsor}{Continue reading the main story}

Food Matters

\hypertarget{ill-have-what-shes-having}{%
\section{I'll Have What She's Having}\label{ill-have-what-shes-having}}

A new crop of restaurants is embracing family-style, communal eating,
creating a necessary spirit of communication and collaboration for our
fractious times.

\includegraphics{https://static01.graylady3jvrrxbe.onion/images/2019/11/14/t-magazine/14tmag-foodsharing-slide-M171/14tmag-foodsharing-slide-M171-articleLarge.jpg?quality=75\&auto=webp\&disable=upscale}

By Priya Krishna

\begin{itemize}
\item
  Nov. 13, 2019
\item
  \begin{itemize}
  \item
  \item
  \item
  \item
  \item
  \end{itemize}
\end{itemize}

AT NEW YORK'S \href{https://www.nichenichenyc.com/}{Niche Niche}, an
American restaurant that opened in SoHo in March, 40 dinner guests
arrive all at once, at 6 p.m., and are escorted from the entryway to
their preassigned seats in a dining room that looks more like a living
room, with stuffed chairs and velvet pillows. Fifteen minutes later, the
hostess bangs on a golden swan statuette with a wooden spoon; dinner is
ready. There are no options: Out of the kitchen come massive platters of
dishes like trout tartare, dry-aged strip steak and, for dessert,
chocolate cake with whipped cream. The assembled strangers eat the same
meal at the same pace, passing plates, making conversation.

This feels less like a restaurant than a dinner party --- and that's the
point. Across the United States, restaurants like Niche Niche are
subverting the traditional power dynamic between chefs and diners,
refocusing the meal not on what's eaten but \emph{how}. At Brooklyn's
\href{https://www.gertie.nyc/}{Gertie}, an elevated version of a classic
diner that opened in Williamsburg in February, there are only one or two
servers; during daytime table-service, patrons must visit the open
kitchen when they want to order a Waldorf salad or cauliflower melt,
navigating their way among the interconnected tables, which they choose
themselves. At \href{https://celesteunionsquare.com/}{Celeste}, a
Peruvian restaurant in Somerville, Mass., each of the eight small tables
is crowded with colorful, oddly sized dishes of ceviche, from which
diners serve each other.

While some of these elements --- group seating, shared entrees, preset
menus --- seem familiar, what's novel is these restaurants' underlying
ethos: The goal is to bring people of all backgrounds together in this
splintered time, to make eating out a collective enterprise.
\href{https://www.nytimes3xbfgragh.onion/2019/05/16/t-magazine/the-end-of-sharing-food.html}{Communal
dining, of course, isn't new}. But these restaurants are both more
urbane and more ambitious than their forebears --- places where the
food, wine and design are considered so carefully that the casual,
family-style service and ambience feel, at first, like paradoxes. The
format seems theatrical, or at least experiential: In these
establishments, customers aren't only paying for the food but for the
upending of dining-out conventions. Restaurants are, to some extent,
valued for their predictability and consistency --- in this new model,
the element of surprise is part of the attraction.

\emph{{[}}\href{https://www.nytimes3xbfgragh.onion/newsletters/t-list?module=inline}{\emph{Sign
up here}} \emph{for the T List newsletter, a weekly roundup of what T
Magazine editors are noticing and coveting now.{]}}

\includegraphics{https://static01.graylady3jvrrxbe.onion/images/2019/11/14/t-magazine/14tmag-foodsharing-slide-JD73/14tmag-foodsharing-slide-JD73-articleLarge.jpg?quality=75\&auto=webp\&disable=upscale}

THE RESTAURANT AS we know it was born in Paris. In the 17th century,
elaborate feasting was reserved for the royal court: Louis XIV wielded
food as a means of power, throwing lavish parties with mountains of
strawberries, cherries and melons to provoke envy among his subjects,
who ate at home or else at rough inns, where bread and meat were slammed
unceremoniously on the bar.

The first modern restaurant was opened in 1766 by a businessman named
Mathurin Roze de Chantoiseau and was inspired in part by 17th- and
18th-century salons, intellectual gatherings in Paris run primarily by
women with an emphasis on ``reaching across class,'' says Faith Beasley,
a French literature and culture professor at Dartmouth College. Places
like Chantoiseau's were quickly embraced as social spaces where people
could meet, share ideas and celebrate:
\href{https://www.nytimes3xbfgragh.onion/2019/05/23/t-magazine/laila-gohar-dinner-party.html}{ad
hoc suppers}, open to all. In their early days, the food --- sausage,
turbot and peaches --- was presented en masse to the table in a style
known as \emph{service à la française}, which tempered the spectacle and
abundance of the court feasts with better-tasting, higher-quality
dishes. But by the early 19th century, after the Napoleonic era, the
banquet had become passé; restaurants abandoned it in a favor of the
newly chic \emph{service à la russe} --- said to have been brought to
France in 1810 by the Russian ambassador Alexander Kurakin. It was
essentially a tasting menu, where each course was plated separately and
served in a particular order, and it became the predominant approach at
fine-dining establishments throughout the West.

These latest restaurants, then, update the old French system --- and
represent an evolution of the family-style meals that were popularized
by chefs, notably
\href{https://www.nytimes3xbfgragh.onion/2017/11/10/t-magazine/asian-american-cuisine.html}{Momofuku's
David Chang}, more than a decade ago. At Niche Niche, where visitors pay
\$80 for a five-course meal and wine, the owner, Ariel Arce, hopes to
create something ``bigger than ourselves,'' she says, ``which is
important in such an isolating space like New York.'' Her restaurant's
aesthetic --- plush upholstery, dim lighting, overlapping rugs --- is
meant to feel comforting. At the Israeli chef
\href{https://www.nytimes3xbfgragh.onion/2018/01/08/dining/eyal-shani-miznon-restaurant-nyc.html}{Eyal
Shani}'s HaSalon, a Tel Aviv spot with a
\href{https://www.hasalonnyc.com/}{second location} in New York's Hell's
Kitchen that started dinner service in April, the tavern-like space
features mismatched chairs at about a dozen large tables, upon which
patrons often dance to Israeli pop music after the meal concludes.

Celeste, which opened in 2018, takes this concept even further.
Customers seem visibly nervous when they first walk into the crowded,
cramped space, hook their coats on a tangle of metal wiring that doubles
as an art installation and, as a party of four, are seated at a table
that in theory should only fit two, where the lomo saltado beef barely
fits alongside the chaufa, a kind of Peruvian fried rice. If the
co-owners Maria Rondeau and JuanMa Calderon see a group that hasn't
ordered enough food, they'll bring over samples of different dishes from
other tables, just as they did when they previously ran Celeste as a
series of dinner parties out of their Cambridge apartment. Throughout
the night, the volume of the salsa music constantly increases,
encouraging people to lean in. ``Not so it's obnoxious,'' Rondeau says,
``but it's another way of creating intimacy.''

It's also a way of forcing people to communicate. When the earliest
restaurants emerged, they soon became part of France's new identity ---
a symbol of democracy after a period of political upheaval. And perhaps
that's why this dinner-party format is resonating right now, at a moment
when divisions define every social interaction, when we all struggle to
determine how we might enact change. ``It's that feeling of being
politically powerless,'' Beasley says. ``People want influence on the
public sphere.''

Image

At Niche Niche, the dining experience feels less like a restaurant and
more like a dinner party.Credit...© Noah Fecks

THIS COLLABORATIVE MODEL isn't just for the diner's benefit, though. As
labor costs at restaurants continue to grow --- as of August, according
to the Bureau of Labor Statistics, average hourly earnings were up 4
percent in the past year --- it also makes economic sense. Seth Bregman,
who, along with his wife Jenni, opened the midcentury-inspired
\href{https://www.bardooakland.com/}{Bardo Lounge} in Oakland, Calif.,
in 2018, says that without the stress of having to precisely time
orders, his kitchen, which serves New American food like country-fried
quail, is more relaxed. At Gertie, the do-it-yourself system requires
half the front-of-house staff as other restaurants this size. And at
\href{https://www.tailornashville.com/}{Tailor} in Nashville, diners pay
ahead for their prix fixe Indian feast, allowing the owner, Vivek Surti,
to budget for ingredient costs.

When people arrive at Tailor --- a minimalist, half-hidden spot
connected to another restaurant --- they're greeted the way Surti's
grandmother did when she hosted parties: with snacks, such as potato
ghatiya or sorghum popcorn, offered in little bowls. Surti presents each
course standing in the center of the dining room, retelling a story from
his Tennessee childhood, like the time his parents adapted the Southern
fish-fry tradition as their own, coating catfish in turmeric, cumin and
garlic. Often, he sees elderly aunties chatting with college students at
the close-crowded tables; it's important to him that people of all ages,
no matter their knowledge of Indian food, feel comfortable eating here.
Serving a communal menu allows the chef to introduce guests to dishes
they may never have chosen themselves, such as kadhi, a silky
turmeric-yogurt soup, or dal bhat, a Gujarati staple of lentils and
rice. ``This is how we explain what Indian home food is,'' Surti says,
noting that by the meal's end, there's a level of trust between host and
guest.

Of course, the magic doesn't always take hold. On a recent visit to
Niche Niche, the room was animated, but as one woman signed her check at
the front bar, she lamented to her dining companion that she didn't make
any new friends. Many evenings at Celeste, something goes wrong: The
air-conditioning breaks down; the coat rack becomes overstuffed; the
poorly functioning exhaust system leaves people's shirts smelling of
chaufa. But those surprises and accidents inspire an attitude ``of
embracing whatever comes our way,'' Rondeau says. ``We're getting
through it and resolving it together.''

Advertisement

\protect\hyperlink{after-bottom}{Continue reading the main story}

\hypertarget{site-index}{%
\subsection{Site Index}\label{site-index}}

\hypertarget{site-information-navigation}{%
\subsection{Site Information
Navigation}\label{site-information-navigation}}

\begin{itemize}
\tightlist
\item
  \href{https://help.nytimes3xbfgragh.onion/hc/en-us/articles/115014792127-Copyright-notice}{©~2020~The
  New York Times Company}
\end{itemize}

\begin{itemize}
\tightlist
\item
  \href{https://www.nytco.com/}{NYTCo}
\item
  \href{https://help.nytimes3xbfgragh.onion/hc/en-us/articles/115015385887-Contact-Us}{Contact
  Us}
\item
  \href{https://www.nytco.com/careers/}{Work with us}
\item
  \href{https://nytmediakit.com/}{Advertise}
\item
  \href{http://www.tbrandstudio.com/}{T Brand Studio}
\item
  \href{https://www.nytimes3xbfgragh.onion/privacy/cookie-policy\#how-do-i-manage-trackers}{Your
  Ad Choices}
\item
  \href{https://www.nytimes3xbfgragh.onion/privacy}{Privacy}
\item
  \href{https://help.nytimes3xbfgragh.onion/hc/en-us/articles/115014893428-Terms-of-service}{Terms
  of Service}
\item
  \href{https://help.nytimes3xbfgragh.onion/hc/en-us/articles/115014893968-Terms-of-sale}{Terms
  of Sale}
\item
  \href{https://spiderbites.nytimes3xbfgragh.onion}{Site Map}
\item
  \href{https://help.nytimes3xbfgragh.onion/hc/en-us}{Help}
\item
  \href{https://www.nytimes3xbfgragh.onion/subscription?campaignId=37WXW}{Subscriptions}
\end{itemize}
