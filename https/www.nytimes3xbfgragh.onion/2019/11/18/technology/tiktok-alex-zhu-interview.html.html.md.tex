\href{/section/technology}{Technology}\textbar{}TikTok's Chief Is on a
Mission to Prove It's Not a Menace

\url{https://nyti.ms/2NX6xvC}

\begin{itemize}
\item
\item
\item
\item
\item
\item
\end{itemize}

\includegraphics{https://static01.graylady3jvrrxbe.onion/images/2019/11/17/business/17tiktok-01/17tiktok-01-articleLarge.jpg?quality=75\&auto=webp\&disable=upscale}

Sections

\protect\hyperlink{site-content}{Skip to
content}\protect\hyperlink{site-index}{Skip to site index}

\hypertarget{tiktoks-chief-is-on-a-mission-to-prove-its-not-a-menace}{%
\section{TikTok's Chief Is on a Mission to Prove It's Not a
Menace}\label{tiktoks-chief-is-on-a-mission-to-prove-its-not-a-menace}}

Alex Zhu, the head of the Chinese-owned viral video app, is trying to
assuage Washington's fears. ``I am quite optimistic,'' he said.

Alex Zhu, the head of TikTok, in New York.Credit...Alexander Harris for
The New York Times

Supported by

\protect\hyperlink{after-sponsor}{Continue reading the main story}

\href{https://www.nytimes3xbfgragh.onion/by/raymond-zhong}{\includegraphics{https://static01.graylady3jvrrxbe.onion/images/2018/10/15/multimedia/author-raymond-zhong/author-raymond-zhong-thumbLarge.png}}

By \href{https://www.nytimes3xbfgragh.onion/by/raymond-zhong}{Raymond
Zhong}

\begin{itemize}
\item
  Nov. 18, 2019
\item
  \begin{itemize}
  \item
  \item
  \item
  \item
  \item
  \item
  \end{itemize}
\end{itemize}

\href{https://cn.nytimes3xbfgragh.onion/technology/20191118/tiktok-alex-zhu-interview/}{阅读简体中文版}\href{https://cn.nytimes3xbfgragh.onion/technology/20191118/tiktok-alex-zhu-interview/zh-hant/}{閱讀繁體中文版}

Like almost everybody who runs a big tech company these days, Alex Zhu,
the head of
\href{https://www.nytimes3xbfgragh.onion/2019/03/10/style/what-is-tik-tok.html}{the
of-the-moment video app TikTok}, is worried about an image problem.

To him --- and to millions of TikTok's users --- the app is a haven for
creativity, earnest self-expression and
\href{https://www.nytimes3xbfgragh.onion/2019/10/19/style/high-school-tiktok-clubs.html}{silly
dance videos}. In almost no time, TikTok has emerged as the refreshing
weirdo upstart of the American social media landscape, reconfiguring the
culture in its
\href{https://www.nytimes3xbfgragh.onion/interactive/2019/10/10/arts/TIK-TOK.html}{joyful,
strange wake}.

But to some people in the United States government, TikTok is a menace.
And one big reason is the nationality of its owner, a seven-year-old
Chinese social media company called ByteDance. The fear is that TikTok
is exposing America's youth to Communist Party indoctrination and
\href{https://www.blackburn.senate.gov/blackburn-tiktok-you-are-chinas-best-detective}{smuggling
their data to Beijing's servers}.

The desire to fix this perception gap is what brought Mr. Zhu last week
to a WeWork in Manhattan, where a handful of his colleagues are based.
Mr. Zhu, a trim 40-year-old who speaks fluent if lightly accented
English, helped found Musical.ly, a Shanghai-based lip-syncing app that
ByteDance acquired in 2017 and folded into TikTok.

In an interview --- his first since taking the reins at TikTok this year
--- Mr. Zhu denied, in unambiguous terms, several key accusations.

No, TikTok does not censor videos that displease China, he said. And no,
it does not share user data with China, or even with its Beijing-based
parent company. All data on TikTok users worldwide is stored in
Virginia, he said, with a backup server in Singapore.

But China is a murky place for companies. Even if TikTok's policies are
clear on paper, what if Chinese authorities decided they didn't like
them and pressured ByteDance? What if China's top leader, Xi Jinping,
personally asked Mr. Zhu to take down a video or hand over user data?

``I would turn him down,'' Mr. Zhu said, after barely a moment's
thought.

Washington at this moment is suspicious of Chinese tech companies to a
degree that can
\href{https://www.nytimes3xbfgragh.onion/2019/07/20/us/politics/china-red-scare-washington.html}{feel
like paranoia}. The Trump administration's biggest target has been
Huawei, the giant supplier of smartphones and telecommunications
equipment. But it has also tried kneecapping Chinese producers of
\href{https://www.nytimes3xbfgragh.onion/2018/10/29/us/politics/fujian-jinhua-china-sales.html}{microchips},
\href{https://www.nytimes3xbfgragh.onion/2019/10/07/us/politics/us-to-blacklist-28-chinese-entities-over-abuses-in-xinjiang.html}{surveillance
gear} and
\href{https://www.nytimes3xbfgragh.onion/2019/06/21/us/politics/us-china-trade-blacklist.html}{supercomputers}.

That a lip-syncing app now finds itself in the same position shows the
extent to which any Chinese advancement is seen in Washington as harmful
to American interests. Over the past year, TikTok's app has been
downloaded more than 750 million times --- more than Facebook,
Instagram, YouTube and Snapchat, according to the research firm Sensor
Tower.

The weapon being
\href{https://www.nytimes3xbfgragh.onion/2019/11/01/technology/tiktok-national-security-review.html}{wielded
against TikTok} is the Committee on Foreign Investment in the United
States. The secretive federal panel, known as CFIUS, is looking into
\href{https://www.nytimes3xbfgragh.onion/2017/11/10/business/dealbook/musically-sold-app-video.html}{ByteDance's
purchase of Musical.ly}.

Earlier this year, the committee forced a different Chinese company to
relinquish
\href{https://www.nytimes3xbfgragh.onion/2019/03/28/us/politics/grindr-china-national-security.html}{control
over the dating app Grindr}, which it had bought in 2016. The concern
was also that Beijing might gain access to personal information.

Mr. Zhu said TikTok user data was segregated from the rest of ByteDance,
and was not even used to help improve ByteDance's artificial
intelligence and other technologies.

``The data of TikTok is only being used by TikTok for TikTok users,'' he
said.

It is unclear how such assurances will be received in Washington.

``If Instagram or Facebook wanted to be sold to a Chinese firm in some
way, I would 100 percent see the same issues at hand,'' said Clark
Fonda, a former congressional chief of staff and an author of a 2018 law
that expanded CFIUS's powers. ``It's about the underlying distrust of
the Chinese government and what, theoretically, they could do with this
data.''

In this tense time, Mr. Zhu is an unlikely peacemaker. With his long
salt-and-pepper hair and light mustache and goatee, he looks more like a
poet than a tech founder. He seems to relish a little artsy oddness. On
\href{https://www.linkedin.com/in/keepsilence/}{his LinkedIn profile},
he describes himself as a ``designtrepreneur'' and gives his work
location as ``Mars.''

``In the past, my personal focus was always design and user
experience,'' Mr. Zhu said. He spent a lot of time thinking about the
colors of buttons.

Now as TikTok's boss, he reports to ByteDance's 36-year-old founder,
Zhang Yiming. Mr. Zhu said dealing with TikTok's sudden crisis had been
``very interesting,'' if nothing else.

``I am quite optimistic,'' he said.

Mr. Zhu grew up in the landlocked Chinese province of Anhui. After
studying civil engineering at Zhejiang University in eastern China, he
worked in the United States at SAP, the German software company.

As he tells it,
\href{https://www.nytimes3xbfgragh.onion/2016/08/10/technology/china-homegrown-internet-companies-rest-of-the-world.html}{the
idea for Musical.ly} came to him as an epiphany. On a train once from
San Francisco to Mountain View, Calif., he noticed the teenagers around
him listening to music, taking selfies and passing their phones around.
Why not combine all that into a single app?

Musical.ly debuted in 2014. It quickly attracted tens of millions of
monthly users, and Mr. Zhu moved to Shanghai.

\includegraphics{https://static01.graylady3jvrrxbe.onion/images/2019/11/17/business/17tiktok-02/17tiktok-02-articleLarge.jpg?quality=75\&auto=webp\&disable=upscale}

He went to great lengths to learn about the young --- sometimes
\href{https://www.nytimes3xbfgragh.onion/2016/09/17/business/media/a-social-network-frequented-by-children-tests-the-limits-of-online-regulation.html}{very
young} --- Americans flocking to his platform. He registered fake
Musical.ly accounts so he could comment on videos and understand their
creators, \href{https://www.youtube.com/watch?v=wTyg2E44pBA}{he said} in
2016.

Around the same time, ByteDance was storming phone screens in China with
\href{https://www.nytimes3xbfgragh.onion/2018/01/02/business/china-toutiao-censorship.html}{a
news aggregator called Jinri Toutiao}. In 2016, the company released a
video app for China named Douyin; TikTok followed soon after. The
platforms are similar but separate --- TikTok is unavailable in mainland
China and vice versa.

In late 2017, Musical.ly agreed to be taken over by ByteDance. Last
year, the Musical.ly app was merged into TikTok.

Mr. Zhu stayed to help with the transition. He then took a few months
off last year to rest, go clubbing in Shanghai and listen to jazz. He
rejoined TikTok early this year, not long after ByteDance raised funding
at a
\href{https://www.nytimes3xbfgragh.onion/2018/09/28/technology/bytedance-fundraising-toutiao-tiktok.html}{valuation
of around \$75 billion}, making it one of the planet's most richly
valued start-ups.

TikTok surely owes some of its success to the
\href{https://www.nytimes3xbfgragh.onion/2018/12/03/technology/tiktok-a-chinese-video-app-brings-fun-back-to-social-media.html}{sunny,
fun-for-its-own-sake vibe} it has cultivated. But that has led to
suspicions that TikTok suppresses material, such as clips of the Hong
Kong protests, that could be a buzzkill. The company says it previously
penalized content that
\href{https://www.theguardian.com/technology/2019/sep/25/revealed-how-tiktok-censors-videos-that-do-not-please-beijing}{``promoted
conflict.''}

Now ``we don't take any action on any politically sensitive content as
long as it goes along with our
\href{https://support.tiktok.com/en/privacy-safety/community-policy-en}{community
guidelines},'' said Vanessa Pappas, general manager for TikTok in the
United States. Those cover things like hate speech, harassment and
misleading information.

Mr. Zhu said TikTok, which makes money by selling ads, was still drawing
up its content policies.

``Today, we are lucky,'' he said, ``because users perceive TikTok as a
platform for memes, for lip-syncing, for dancing, for fashion, for
animals --- but not so much for political discussion.''

He acknowledged this could change. ``For political content that still
aligns with this creative and joyful experience, I don't see why we
should control it,'' he said.

The deeper concern is that ByteDance's vast business in China could give
Beijing leverage over the company, and over TikTok. In its brief
existence, ByteDance has had
\href{https://www.nytimes3xbfgragh.onion/2018/04/12/business/china-bytedance-duanzi-censor.html}{plenty
of run-ins} with Chinese authorities. This month, regulators
\href{https://mp.weixin.qq.com/s/cNOmVWg4lagP5kTAlvoFHw}{hauled up
company executives} after finding search results from ByteDance's search
engine that supposedly defamed a revolutionary hero.

There are other steps ByteDance could take to try to convince Washington
of TikTok's independence, such as reorganizing TikTok as a separate
company with a new board of directors.

Mr. Zhu said the company wouldn't rule out such possibilities. But there
had been no discussion about selling off TikTok's American business, he
said.

Harry Clark, a CFIUS specialist at the law firm Orrick, said that was
probably what the committee would end up demanding. CFIUS might have
entertained other options had the companies applied for a review before
doing the deal, Mr. Clark said. Now, he said, Washington's concerns
about China and data protection are deepening.

``Three years ago, I doubt any CFIUS expert would have said it's crucial
that you go to CFIUS here,'' Mr. Clark said. ``Now, most would.''

Wang Yiwei contributed research.

Advertisement

\protect\hyperlink{after-bottom}{Continue reading the main story}

\hypertarget{site-index}{%
\subsection{Site Index}\label{site-index}}

\hypertarget{site-information-navigation}{%
\subsection{Site Information
Navigation}\label{site-information-navigation}}

\begin{itemize}
\tightlist
\item
  \href{https://help.nytimes3xbfgragh.onion/hc/en-us/articles/115014792127-Copyright-notice}{©~2020~The
  New York Times Company}
\end{itemize}

\begin{itemize}
\tightlist
\item
  \href{https://www.nytco.com/}{NYTCo}
\item
  \href{https://help.nytimes3xbfgragh.onion/hc/en-us/articles/115015385887-Contact-Us}{Contact
  Us}
\item
  \href{https://www.nytco.com/careers/}{Work with us}
\item
  \href{https://nytmediakit.com/}{Advertise}
\item
  \href{http://www.tbrandstudio.com/}{T Brand Studio}
\item
  \href{https://www.nytimes3xbfgragh.onion/privacy/cookie-policy\#how-do-i-manage-trackers}{Your
  Ad Choices}
\item
  \href{https://www.nytimes3xbfgragh.onion/privacy}{Privacy}
\item
  \href{https://help.nytimes3xbfgragh.onion/hc/en-us/articles/115014893428-Terms-of-service}{Terms
  of Service}
\item
  \href{https://help.nytimes3xbfgragh.onion/hc/en-us/articles/115014893968-Terms-of-sale}{Terms
  of Sale}
\item
  \href{https://spiderbites.nytimes3xbfgragh.onion}{Site Map}
\item
  \href{https://help.nytimes3xbfgragh.onion/hc/en-us}{Help}
\item
  \href{https://www.nytimes3xbfgragh.onion/subscription?campaignId=37WXW}{Subscriptions}
\end{itemize}
