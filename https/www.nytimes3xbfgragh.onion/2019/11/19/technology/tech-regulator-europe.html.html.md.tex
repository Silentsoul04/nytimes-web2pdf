Sections

SEARCH

\protect\hyperlink{site-content}{Skip to
content}\protect\hyperlink{site-index}{Skip to site index}

\href{https://www.nytimes3xbfgragh.onion/section/technology}{Technology}

\href{https://myaccount.nytimes3xbfgragh.onion/auth/login?response_type=cookie\&client_id=vi}{}

\href{https://www.nytimes3xbfgragh.onion/section/todayspaper}{Today's
Paper}

\href{/section/technology}{Technology}\textbar{}Big Tech's Toughest
Opponent Says She's Just Getting Started

\url{https://nyti.ms/2QA8ZtF}

\begin{itemize}
\item
\item
\item
\item
\item
\item
\end{itemize}

Advertisement

\protect\hyperlink{after-top}{Continue reading the main story}

Supported by

\protect\hyperlink{after-sponsor}{Continue reading the main story}

\hypertarget{big-techs-toughest-opponent-says-shes-just-getting-started}{%
\section{Big Tech's Toughest Opponent Says She's Just Getting
Started}\label{big-techs-toughest-opponent-says-shes-just-getting-started}}

Margrethe Vestager won praise for her oversight of the tech industry.
Now, with more authority from the European Union, she envisions a more
aggressive agenda.

\includegraphics{https://static01.graylady3jvrrxbe.onion/images/2019/11/19/business/19VESTAGER/merlin_164270316_ac41e3ac-53c1-4d85-91dd-e9ff0ef72f35-articleLarge.jpg?quality=75\&auto=webp\&disable=upscale}

By \href{https://www.nytimes3xbfgragh.onion/by/adam-satariano}{Adam
Satariano} and
\href{https://www.nytimes3xbfgragh.onion/by/matina-stevis-gridneff}{Matina
Stevis-Gridneff}

\begin{itemize}
\item
  Nov. 19, 2019
\item
  \begin{itemize}
  \item
  \item
  \item
  \item
  \item
  \item
  \end{itemize}
\end{itemize}

BRUSSELS --- Margrethe Vestager spent the past five years developing a
well-earned reputation as the world's top tech industry watchdog. From
her perch overseeing Europe's competition rules, she fined Google more
than \$9 billion for breaking antitrust laws, and forced Apple to pay
about \$14.5 billion for dodging taxes.

Now she says that work, which made her a
\href{https://www.nytimes3xbfgragh.onion/2018/05/05/world/europe/margrethe-vestager-silicon-valley-data-privacy.html}{hero
among tech critics}, did not go far enough. The biggest tech companies
continue to test the limits of antitrust laws, behave unethically and
push back against government intervention, she said.

But she said the public's growing skepticism about technology has given
her an opportunity for a tougher approach.

``In the last five years,'' Ms. Vestager said in an extended interview,
``some of the darker sides of digital technologies have become
visible.''

So Ms. Vestager, a 51-year-old former Danish lawmaker, is doubling down.
She has signed on for a rare second five-year term as the head of the
European Commission's antitrust division, and assumed
\href{https://www.nytimes3xbfgragh.onion/2019/09/10/world/europe/margrethe-vestager-european-union-tech-regulation.html}{expanded
responsibility over digital policy} across the 28-nation bloc.

With the new power, she has outlined an
\href{http://www.europarl.europa.eu/RegData/etudes/BRIE/2019/640171/EPRS_BRI(2019)640171_EN.pdf}{agenda}
that squarely targets the tech giants. She's weighing whether to remove
some protections that shield large internet platforms from liability for
content posted by users. She is also working on policies to make
companies pay more taxes in Europe and investigating how the companies
use data to box out competitors.

Ms. Vestager has pledged to create the world's first regulations around
artificial intelligence and called for giving collective bargaining
rights to so-called gig economy workers like Uber drivers. The push
comes on top of an investigation into
\href{https://www.nytimes3xbfgragh.onion/2019/07/17/technology/amazon-eu.html}{Amazon's
use of data}to gain an edge on competitors that had already started, and
her look into accusations of unfair business practices by Facebook and
Apple.

``She has these accomplishments, but she didn't get as much as she
wanted,'' said David Balto, a former lawyer in the Justice Department's
antitrust division whose clients now include large tech companies. ``Now
she can be more aggressive.''

But Ms. Vestager's agenda amounts to a wish list. Her success will
depend on support and collaboration from other European officials who
are already grappling with challenges like Britain's exit from the
European Union, the rise of populism and fraying diplomatic relations
with the United States.

It will require standing up to relentless resistance from the tech
companies, too.

``One of the important things is, of course, to prioritize because
otherwise you will be in the process of back and forth for a very, very
long time,'' Ms. Vestager said.

In person, Ms. Vestager's manner defies her tough enforcer reputation.
She is unfailingly polite, meeting guests by offering tea and
apologizing for a lingering cold. (She assured everyone that she had
just washed her hands.)

She is a challenging interview subject, prone to filibuster and rarely
veering from oft-repeated talking points. A skilled politician, she
projects modesty while not exactly turning away from the spotlight. A
sign in the hallway outside her office says, in Danish, ``Vestager
Street.''

She is also fast to shrug off criticism, including by numerous tech
executives and President Trump, that she has been unfair to American
tech companies.

Tim Cook, Apple's chief executive, called the penalty against his
company in 2016 for skirting Irish taxes
\href{https://www.independent.ie/business/irish/no-one-did-anything-wrong-here-and-ireland-is-being-picked-on-it-is-total-political-crap-apple-chief-tim-cook-35012145.html}{``total
political crap.''} Google is appealing her three decisions against the
company.

``She hates the United States,'' President Trump said in a
\href{https://www.nytimes3xbfgragh.onion/2019/06/26/business/economy/trump-china-tariffs-g20.html}{television
interview in June}, ``perhaps worse than any person I've ever met.''

Ms. Vestager feigns to hardly remember the president's comment. ``Since
I know the very good relationship I have with the United States, then he
must only meet people who really like the States if I am the one who
likes you the least,'' she said.

If anything, American authorities are coming around to share her tech
skepticism. Federal, state and congressional investigators are
\href{https://www.nytimes3xbfgragh.onion/2019/07/23/technology/justice-department-tech-antitrust.html}{scrutinizing
the tech industry} over unfair business practices. Ms. Vestager said she
saw opportunities to collaborate, but was waiting to see how the
inquiries unfolded.

``Obviously it's very interesting to see what will come of it,'' she
said.

As the United States begins to investigate Amazon, Apple, Facebook and
Google, some American officials are trying to learn lessons from
Europe's efforts. The investigations of Google and others
\href{https://www.nytimes3xbfgragh.onion/2019/11/11/business/europe-technology-antitrust-regulation.html}{took
years to complete}, giving the companies extra time to solidify their
dominance. And once the inquiries were completed, critics said, the
penalties focused on large fines that the companies could easily afford,
rather than enforcing structural changes that would restore competition.

Luther Lowe, the head of public policy at Yelp, the reviews website that
has been a frequent critic of Google's behavior, praised Ms. Vestager's
efforts. But he said companies like Yelp ``have to date still not seen a
shred of practical relief, despite having prevailed in concept.''

Ms. Vestager needs to use all powers at her disposal, he said, ``or be
granted new ones.''

Ms. Vestager said some of the criticism was valid. She is taking steps
to speed up investigations and is applying a rarely used rule known as
\href{https://www.nytimes3xbfgragh.onion/2019/10/16/business/-big-tech-europe-antitrust.html}{``interim
measures,''} that acts as a cease-and-desist order for companies to stop
acting a certain way while an investigation can be conducted.

She will play a leading role in the European Union's debate over a new
Digital Services Act, which could bring sweeping reforms to how the
internet operates, including forcing online platforms to remove illegal
content or risk fines and other penalties. Facebook, she said, must be
quicker to stop the spread of false and misleading information, violent
material and hate speech.

``You have to take it down because it spreads like a virus,'' she said.
``But if it's not fast enough, of course, eventually we will have to
regulate this.''

And she remains focused on whether the largest technology companies
squeezed out businesses that rely on them to reach customers. Amazon is
under investigation for mistreating third-party sellers that offer
products similar to what it sells. Apple is being questioned over
accusations that it uses the App Store to harm rivals such as Spotify.

``Some of these platforms, they have the role both as player and
referee, and how can that be fair?'' she asked. ``You would never accept
a football match where the one team was also being the referee.''

In Europe, a broader debate is underway about a lack of homegrown tech
giants. President Emmanuel Macron of France, for instance, has called
for more government support of European companies. Ursula von der Leyen,
the new head of the European Commission, who appointed Ms. Vestager, has
called for Europe to achieve ``technological sovereignty.''

The companies facing Ms. Vestager's scrutiny are warning about taking
regulation too far.

Christian Borggreen, vice president of the Computer and Communications
Industry Association in Brussels, a trade group representing Apple,
Google and other companies, warned that new laws could put Europe at a
disadvantage.

``We hope future E.U. legislation will be evidence-based and never
become an excuse for protectionism,'' he said.

Ms. Vestager has said that European companies must compete on their
merits.

``One of the main reasons that U.S. tech companies are popular in Europe
is that their products are good,'' she said. Her job, she added, has
been to step in when companies ``cut corners.''

Ms. Vestager said Europe had a different view of technology than the
wide-open policies of the United States and government control of China.
Europe, she said, must forge its own approach.

``Market forces are more than welcome, but we do not leave it to market
forces to have the final say,'' she said. ``Markets are not perfect.''

Advertisement

\protect\hyperlink{after-bottom}{Continue reading the main story}

\hypertarget{site-index}{%
\subsection{Site Index}\label{site-index}}

\hypertarget{site-information-navigation}{%
\subsection{Site Information
Navigation}\label{site-information-navigation}}

\begin{itemize}
\tightlist
\item
  \href{https://help.nytimes3xbfgragh.onion/hc/en-us/articles/115014792127-Copyright-notice}{©~2020~The
  New York Times Company}
\end{itemize}

\begin{itemize}
\tightlist
\item
  \href{https://www.nytco.com/}{NYTCo}
\item
  \href{https://help.nytimes3xbfgragh.onion/hc/en-us/articles/115015385887-Contact-Us}{Contact
  Us}
\item
  \href{https://www.nytco.com/careers/}{Work with us}
\item
  \href{https://nytmediakit.com/}{Advertise}
\item
  \href{http://www.tbrandstudio.com/}{T Brand Studio}
\item
  \href{https://www.nytimes3xbfgragh.onion/privacy/cookie-policy\#how-do-i-manage-trackers}{Your
  Ad Choices}
\item
  \href{https://www.nytimes3xbfgragh.onion/privacy}{Privacy}
\item
  \href{https://help.nytimes3xbfgragh.onion/hc/en-us/articles/115014893428-Terms-of-service}{Terms
  of Service}
\item
  \href{https://help.nytimes3xbfgragh.onion/hc/en-us/articles/115014893968-Terms-of-sale}{Terms
  of Sale}
\item
  \href{https://spiderbites.nytimes3xbfgragh.onion}{Site Map}
\item
  \href{https://help.nytimes3xbfgragh.onion/hc/en-us}{Help}
\item
  \href{https://www.nytimes3xbfgragh.onion/subscription?campaignId=37WXW}{Subscriptions}
\end{itemize}
