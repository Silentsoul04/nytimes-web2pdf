Sections

SEARCH

\protect\hyperlink{site-content}{Skip to
content}\protect\hyperlink{site-index}{Skip to site index}

\href{https://www.nytimes3xbfgragh.onion/section/technology/personaltech}{Personal
Tech}

\href{https://myaccount.nytimes3xbfgragh.onion/auth/login?response_type=cookie\&client_id=vi}{}

\href{https://www.nytimes3xbfgragh.onion/section/todayspaper}{Today's
Paper}

\href{/section/technology/personaltech}{Personal Tech}\textbar{}Want the
Greenest Device? You May Already Own It

\href{https://nyti.ms/2s2ovo2}{https://nyti.ms/2s2ovo2}

\begin{itemize}
\item
\item
\item
\item
\item
\end{itemize}

Advertisement

\protect\hyperlink{after-top}{Continue reading the main story}

Supported by

\protect\hyperlink{after-sponsor}{Continue reading the main story}

Tech We're Using

\hypertarget{want-the-greenest-device-you-may-already-own-it}{%
\section{Want the Greenest Device? You May Already Own
It}\label{want-the-greenest-device-you-may-already-own-it}}

One way to help the planet is not to buy new tech, especially stuff the
planet never needed, says Kendra Pierre-Louis, who reports on the
environment.

\includegraphics{https://static01.graylady3jvrrxbe.onion/images/2019/11/20/business/20techusing/20techusing-articleLarge.jpg?quality=75\&auto=webp\&disable=upscale}

\href{https://www.nytimes3xbfgragh.onion/by/kendra-pierre-louis}{\includegraphics{https://static01.graylady3jvrrxbe.onion/images/2018/07/16/multimedia/author-kendra-pierre-louis/author-kendra-pierre-louis-thumbLarge.png}}

Featuring
\href{https://www.nytimes3xbfgragh.onion/by/kendra-pierre-louis}{Kendra
Pierre-Louis}

\begin{itemize}
\item
  Nov. 20, 2019
\item
  \begin{itemize}
  \item
  \item
  \item
  \item
  \item
  \end{itemize}
\end{itemize}

\emph{How do New York Times journalists use technology in their jobs and
in their personal lives? Kendra Pierre-Louis, who covers the environment
and climate, discussed the tech she's using.}

\textbf{Kendra, what tech tools are most important for doing your work?}

I use the standard suite of office software, from Google Docs to Office,
depending on my specific needs. (Docs is better when it comes to edit
trace, but I prefer Excel for certain things.)

I also use Tabula a lot to scrape data out of PDFs, and the
\href{http://www.hemingwayapp.com/}{Hemingway app} to make sure that
what I wrote is at least somewhat understandable. I appreciate the app's
color coding of things as diverse as complex sentences and passive
voice.

For recording interviews, I use an external recorder because I don't
trust phone recording apps --- I want to see that the device is actually
recording. If it's a phone call (or a Skype call), I'll use it with a
pickup microphone. I've had my setup for about five years, which is a
pretty good run.

\textbf{You recently wrote about}
\textbf{\href{https://www.nytimes3xbfgragh.onion/interactive/2019/climate/sustainable-clothing.html}{buying
clothes that last}. Is there a method to buying tech that lasts?}

Tech has a tremendous footprint. One estimate by the
\href{https://eta.lbl.gov/publications/united-states-data-center-energy}{Lawrence
Berkeley Lab} said it took 70 billion kilowatt-hours in 2014, or nearly
2 percent of the total electricity generation in the United States that
year, just to run the internet.

And then, of course, there are the materials used to create tech. The
lithium-ion batteries that are in so many things, like my smartwatch, my
cellphone and your earbuds, typically contain cobalt, which was
potentially mined in the Democratic Republic of Congo using forced child
labor and in conditions that hurt both people's health and the
environment.

Many companies will say their phones are recyclable, but even when they
are recycled (and that process can be incredibly environmentally
polluting as well), the metal is generally too low a grade to go into a
new phone. All of which, yes, points to a need for tech that lasts.

I don't know if there's a method as detailed as the one I laid out in my
article on clothes. I can say I tend to hold on to electronics for years
--- I once had a laptop that lasted nine years. Toward the end, people
teased me about it because it was, physically speaking, a brick. I got
my workout carrying that thing around.

Part of the reason it lasted so long was that I bought a machine that
was faster, had a larger hard drive and could expand its memory more
than I needed. So as software and the internet evolved to require more
memory and higher processor speeds, the computer could handle it.

Also, I really took care of it. Every couple of years, I had tech repair
shops clean its insides to remove the dust that built up inside. You can
also do this yourself, but I liked having a professional do it for me.

And part of it is that the machine itself was well constructed, which I
learned the hard way when I dumped an entire bowl of basil chicken on
the keyboard. In most current laptops, that would immediately fry the
computer, but the way that computer was constructed I only fried the
keyboard. Replacing the keyboard only cost me \$40 and the computer
lasted another three years.

The secret is buying tech that really fits your uses, looking at reviews
like the ones on iFixit about how easy it is to repair and
\href{https://www.nytimes3xbfgragh.onion/2016/04/21/technology/personaltech/choosing-to-skipthe-upgrade-and-care-for-the-gadget-youve-got.html}{taking
care of things once you have them}. Put a case on your phone. For my
current phone, which is too big for me to use in one hand because most
phones are designed for men's hands, which tend to be larger, I got a
PopSocket to make it less likely that I will drop it.

\includegraphics{https://static01.graylady3jvrrxbe.onion/images/2019/11/20/business/20techusing2/20techusing2-articleLarge.jpg?quality=75\&auto=webp\&disable=upscale}

\textbf{What does your eco-conscious tech setup look like at home?}

I think the greenest things I do are the things that I don't buy,
honestly.

\textbf{What tech trends or products are the least sustainable?}

The products that are the least sustainable are the ones that don't,
objectively, need to exist.

I try to tread lightly here because I'm aware that some products that
leave me scratching my head can be for people with physical disabilities
or mobility issues. Maybe that's the reason behind, say,
motion-detecting garbage cans, which I acknowledge are cool.

But even if that really is a mobility need, does it have to be connected
to the internet so it can reorder its own garbage bags?

When I think about the tremendous resources devoted to just that, it
shocks me. Amazon has since canceled the Dash buttons, but each button
had a full circuit board inside it, and the company has said it sold
millions. Think of millions of circuit boards --- the oil converted into
plastic, the ore mined and the electricity used to turn that into metal
for the circuits just so you won't forget to order laundry detergent.

Does the world need a smart water bottle, or can you just set a reminder
on your phone to drink water?

Image

Ms. Pierre-Louis uses a sleep tracker app in her quest to sleep
better.Credit...Gabby Jones for The New York Times

\textbf{You've started tracking your sleep. What have you done with the
data? Has}
\textbf{\href{https://www.nytimes3xbfgragh.onion/2019/07/17/technology/personaltech/sleep-tracking-devices-apps.html}{sleep
tracking been helpful}?}

It's less that I have sleep problems and more that I have a bad habit of
neglecting sleep and then wondering why I'm so exhausted. It's a common
malady of modern life.

At the same time, I know how important sleep is. Matthew Walker, a
neuroscientist at the University of California, Berkeley, has this
expression, ``The shorter you sleep, the shorter your life,'' so for the
past year and a half I've been trying to get better about allocating
enough sleep.

To that end, I have the dumbest smartwatch on the planet. It doesn't
have a GPS tracker, for example, and it has a standard watch face so
almost nobody suspects it's a smartwatch. I thought a really long time
about whether I really needed it, but I've had it for just under two
years and just replaced the watchband, so I don't feel bad about it. I
can send it back to the manufacturer to swap out the battery when it
dies.

I don't care at all about the sleep breakdowns like deep sleep versus
R.E.M. sleep, which most research says are flawed and
\href{https://www.statnews.com/2019/07/24/fitbit-accuracy-dark-skin/}{may
be especially flawed for people with dark skin}. I just care about total
sleep. I try to use the data to make my sleep routine, and make sure
that my weekly averages are decent.

Advertisement

\protect\hyperlink{after-bottom}{Continue reading the main story}

\hypertarget{site-index}{%
\subsection{Site Index}\label{site-index}}

\hypertarget{site-information-navigation}{%
\subsection{Site Information
Navigation}\label{site-information-navigation}}

\begin{itemize}
\tightlist
\item
  \href{https://help.nytimes3xbfgragh.onion/hc/en-us/articles/115014792127-Copyright-notice}{©~2020~The
  New York Times Company}
\end{itemize}

\begin{itemize}
\tightlist
\item
  \href{https://www.nytco.com/}{NYTCo}
\item
  \href{https://help.nytimes3xbfgragh.onion/hc/en-us/articles/115015385887-Contact-Us}{Contact
  Us}
\item
  \href{https://www.nytco.com/careers/}{Work with us}
\item
  \href{https://nytmediakit.com/}{Advertise}
\item
  \href{http://www.tbrandstudio.com/}{T Brand Studio}
\item
  \href{https://www.nytimes3xbfgragh.onion/privacy/cookie-policy\#how-do-i-manage-trackers}{Your
  Ad Choices}
\item
  \href{https://www.nytimes3xbfgragh.onion/privacy}{Privacy}
\item
  \href{https://help.nytimes3xbfgragh.onion/hc/en-us/articles/115014893428-Terms-of-service}{Terms
  of Service}
\item
  \href{https://help.nytimes3xbfgragh.onion/hc/en-us/articles/115014893968-Terms-of-sale}{Terms
  of Sale}
\item
  \href{https://spiderbites.nytimes3xbfgragh.onion}{Site Map}
\item
  \href{https://help.nytimes3xbfgragh.onion/hc/en-us}{Help}
\item
  \href{https://www.nytimes3xbfgragh.onion/subscription?campaignId=37WXW}{Subscriptions}
\end{itemize}
