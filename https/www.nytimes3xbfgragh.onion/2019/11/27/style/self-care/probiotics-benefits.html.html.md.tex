Sections

SEARCH

\protect\hyperlink{site-content}{Skip to
content}\protect\hyperlink{site-index}{Skip to site index}

\href{https://www.nytimes3xbfgragh.onion/section/style/self-care/}{Self-Care}

\href{https://myaccount.nytimes3xbfgragh.onion/auth/login?response_type=cookie\&client_id=vi}{}

\href{https://www.nytimes3xbfgragh.onion/section/todayspaper}{Today's
Paper}

\href{/section/style/self-care/}{Self-Care}\textbar{}What Are the
Benefits of Probiotics?

\url{https://nyti.ms/2QVrPf9}

\begin{itemize}
\item
\item
\item
\item
\item
\end{itemize}

Advertisement

\protect\hyperlink{after-top}{Continue reading the main story}

Supported by

\protect\hyperlink{after-sponsor}{Continue reading the main story}

SCAM OR NOT

\hypertarget{what-are-the-benefits-of-probiotics}{%
\section{What Are the Benefits of
Probiotics?}\label{what-are-the-benefits-of-probiotics}}

Studies suggest that certain probiotics can help in certain contexts.
But you will need to do your research. We can help.

\includegraphics{https://static01.graylady3jvrrxbe.onion/images/2019/12/01/fashion/27scam-probiotics-1/27scam-probiotics-1-articleLarge.jpg?quality=75\&auto=webp\&disable=upscale}

By Melinda Wenner Moyer

\begin{itemize}
\item
  Nov. 27, 2019
\item
  \begin{itemize}
  \item
  \item
  \item
  \item
  \item
  \end{itemize}
\end{itemize}

Walk into a health food store, or even a drugstore, and you're likely to
find an entire aisle, maybe two, dedicated to probiotics. Probiotics are
live micro-organisms, usually bacteria, that provide health benefits
when consumed at appropriate doses.

According to some surveys, approximately
\href{https://nccih.nih.gov/research/statistics/NHIS/2012/natural-products/biotics}{four
million} Americans take probiotics, which are available as pills,
powders, foods and drinks. Probiotics are a huge industry --- at least a
\href{https://www.zionmarketresearch.com/report/probiotics-market}{\$40
billion dollar} one, according to Zion Market Research --- and popular
brands \href{https://smile.amazon.com/s?k=probiotics}{sell} for 35 cents
to \$1 a dose, with a shelf life of several months. ****

Proponents argue that they improve the composition of the gut
microbiome, which is involved in many aspects of health including
\href{https://www.ncbi.nlm.nih.gov/pubmed/27383981}{immunity},
\href{https://www.ncbi.nlm.nih.gov/pubmed/22674330}{metabolism} and
\href{https://www.ncbi.nlm.nih.gov/pubmed/27814521}{mood}.

Framed this way, probiotics may seem like a no-brainer. But before you
reach for your wallet, keep in mind that while many scientists and
doctors believe that probiotics have promise, they also say that a lot
of products on the market don't live up to the hype.

``The current evidence does not convince me to recommend probiotics for
any of my healthy patients,'' said Dr. Pieter Cohen, an assistant
professor at Harvard Medical School and an internist at Cambridge Health
Alliance.

In a \href{https://www.ncbi.nlm.nih.gov/pubmed/29581563}{review} of the
scientific literature on probiotics published in January 2019,
researchers concluded that ``the benefits and feasibility of probiotic
consumption in healthy adults remain uncertain.'' Recent research has
raised questions about how well probiotics are tested for safety, too.

\begin{center}\rule{0.5\linewidth}{\linethickness}\end{center}

\hypertarget{are-there-benefits-to-probiotics}{%
\subsection{Are there benefits to
probiotics?}\label{are-there-benefits-to-probiotics}}

Some clinical studies suggest that certain probiotics can help in
certain contexts. According to the
\href{https://gi.org/topics/probiotics-for-the-treatment-of-adult-gastrointestinal-disorders/}{American
College of Gastroenterology}, the probiotic Bifidobacterium infantis
35624 can help treat irritable bowel syndrome (I.B.S.), while
Saccharomyces boulardii, a yeast, and Lactobacillus rhamnosus GG can
each reduce the risk of diarrhea in adults taking antibiotics.

Specific probiotics have also been shown to help treat pouchitis,
ulcerative colitis,
\href{https://www.cochrane.org/CD012473/BEHAV_probiotics-prevent-infantile-colic}{colic}
and infectious diarrhea, and to reduce the risk of developing
Clostridium difficile infections after taking antibiotics.

To understand the various contexts in which probiotics could be useful,
check out
\href{http://www.usprobioticguide.com/PBCIntroduction.html?utm_source=intro_pg\&utm_medium=civ\&utm_campaign=USA_CHART}{this
guide} recommended by Gregor Reid, a microbiologist and immunologist at
Western University in Ontario, Canada, and the former president of the
industry-funded **** International Scientific Association for Probiotics
and Prebiotics (ISAPP), although some products on the list are backed by
more research than others.

And no matter what the science says, probiotics sold as dietary
supplements are not disease-treating drugs. If companies want to market
their probiotics as medical treatments, they have to pursue
a\href{https://www.fda.gov/regulatory-information/search-fda-guidance-documents/early-clinical-trials-live-biotherapeutic-products-chemistry-manufacturing-and-control-information}{form}
of Food and Drug Administration approval to market them as live
biotherapeutic products. Thus far, no probiotics
\href{https://www.fda.gov/news-events/press-announcements/statement-fda-commissioner-scott-gottlieb-md-advancing-science-and-regulation-live-microbiome-based}{have
been given} this designation.

\begin{center}\rule{0.5\linewidth}{\linethickness}\end{center}

\hypertarget{are-probiotics-a-scam}{%
\subsection{Are probiotics a scam?}\label{are-probiotics-a-scam}}

With probiotics, details matter. It's silly to walk into a drugstore,
grab a probiotic off the shelf and think it's going to do you any good.

Be wary, too, of recommendations made by people who haven't consulted
the scientific literature. ``Don't blindly trust a pharmacist, doctor,
health-food shop attendant or dietary book, as sadly, most are ill
informed,'' Dr. Reid said.

But if a doctor you trust recommends a particular strain that has been
shown in clinical studies to help your condition, or if you have
identified a strain that is backed up by solid clinical research and
really want to give it a try, then sure, go ahead --- after
double-checking with your doctor first. Don't expect miracles, however.

\begin{center}\rule{0.5\linewidth}{\linethickness}\end{center}

\hypertarget{so-what-are-the-issues-with-probiotics}{%
\subsection{So what are the issues with
probiotics?}\label{so-what-are-the-issues-with-probiotics}}

There is still the question of whether some probiotics stick around long
enough in the body to do anything.

In a 2018
\href{https://linkinghub.elsevier.com/retrieve/pii/S0092-8674(18)31102-4}{study}
published in Cell, a team of scientists gave
\href{https://www.supherb.co.il/en/it-works-naturally/probiotics-series/bio-25/}{Supherb's
Bio-25}, a mixture of 11 bacterial strains commercially available in
Israel, to 10 healthy people for four weeks and found that the
probiotics passed right through four of them.

This suggests that in some people, these strains ``will not have an
effect,'' said Eran Segal, a computational biologist at the Weizmann
Institute of Science in Israel and a co-author of the study.

Studies aside, though, the other major problem with probiotics is that
they may not contain what they say they do. In the United States, the
F.D.A. regulates most probiotics as dietary supplements, which means
that their manufacturing and quality-control standards are far less
stringent than standards for prescription and over-the-counter drugs.

In a 2016
\href{https://www.ncbi.nlm.nih.gov/pmc/articles/PMC4916961/}{study},
researchers at the University of California-Davis and other institutions
used DNA analysis to compare the bacterial strains listed on the labels
of 16 commercially available probiotics with what the products actually
had in them.

The researchers found that only one of the 16 products contained the
strains listed on the label; some had entirely different bacterial
species. Good quality control matters: In 2014, a premature baby
\href{https://www.ncbi.nlm.nih.gov/pmc/articles/PMC4584706/}{died} from
what was believed to be mold contamination in a probiotic supplement.

Moreover, some probiotics don't even say what strains they purportedly
contain or how many colony-forming units are in each dose (a measure of
how many viable bacteria each dose contains), because the F.D.A. doesn't
require it.

As a result, consumers ``need to do a lot of homework,'' Dr. Reid said,
in order to make informed decisions. They will need to hunt for brands
that include this information on the label, and cross-reference the
labels with their doctors' recommendations or published research.
Ideally, they should take the dose shown to work in clinical studies.

\begin{center}\rule{0.5\linewidth}{\linethickness}\end{center}

\hypertarget{is-this-a-scam}{%
\subsection{Is This A Scam?}\label{is-this-a-scam}}

\hypertarget{is-}{%
\subsubsection{Is ...}\label{is-}}

Celery Juice

,

Kombucha

,

Activated Charcoal

,

CBD

,

Turmeric

,

Fish Oil

,

Chlorophyll

,

Intermittent Fasting

,

The Keto Diet

,

Probiotics

,

Collagen

,

Coffee

,

\hypertarget{a-scam}{%
\subsubsection{A Scam?}\label{a-scam}}

Facts about wellness.

Will these trends change your life --- or

take your money?

\begin{center}\rule{0.5\linewidth}{\linethickness}\end{center}

\hypertarget{what-about-taking-probiotics-in-food}{%
\subsection{What about taking probiotics in
food?}\label{what-about-taking-probiotics-in-food}}

\href{http://www.usprobioticguide.com/PBCFunctionalFoods.html?utm_source=funcfood_ind\&utm_medium=civ\&utm_campaign=USA_CHART}{The
guide recommended} by Dr. Reid mentions a handful of foods that have
been shown to provide benefits: some Activia, Goodbelly, DanActive and
Yakult drinks and a few infant formulas.

Keep in mind that some yogurts and fermented foods don't contain live
micro-organisms, and that even when they do, these bacteria may not do
anything useful.

``Unfortunately, misuse of the term `probiotic' has also become a major
issue, with many products exploiting the term without meeting the
requisite criteria,'' according to a 2014 ISAPP
\href{https://www.nature.com/articles/nrgastro.2014.66}{consensus}
statement published in the journal Gastroenterology \& Hepatology.

\begin{center}\rule{0.5\linewidth}{\linethickness}\end{center}

\hypertarget{what-about-taking-probiotics-after-a-course-of-antibiotics}{%
\subsection{What about taking probiotics after a course of
antibiotics?}\label{what-about-taking-probiotics-after-a-course-of-antibiotics}}

Some doctors recommend taking probiotics after a course of antibiotics
as a way of jump-starting microbial regrowth in the gut, but a
\href{https://linkinghub.elsevier.com/retrieve/pii/S0092-8674(18)31108-5}{study}
in Cell questioned the wisdom of that practice.

Researchers found that the same probiotic mixture, Bio-25, colonized the
guts of a small group of healthy people who had recently taken
antibiotics, but that those who took Bio-25 then regrew their normal gut
bacteria more slowly than people who didn't take Bio-25.

The Cell ** studies were small, though, and their findings apply only to
the Bio-25 product; no one knows whether other strains have similar
effects.

And this brings up another important point: Every probiotic strain is
different, and one strain's effects have no bearing on others. Put
another way, probiotics ``cannot be generalized,'' Dr. Reid said.

\begin{center}\rule{0.5\linewidth}{\linethickness}\end{center}

\hypertarget{why-arent-doctors-convinced-of-the-benefits-of-probiotics}{%
\subsection{Why aren't doctors convinced of the benefits of
probiotics?}\label{why-arent-doctors-convinced-of-the-benefits-of-probiotics}}

It takes time, not to mention lots of money, to amass the evidence
needed to blanket-recommend a product for a large swath of the
population. (After all, doctors
\href{https://annals.org/aim/fullarticle/1789253/enough-enough-stop-wasting-money-vitamin-mineral-supplements}{still
argue} over whether healthy people should take multivitamins.)

Thus far, many clinical studies on probiotics have been small, poorly
designed or difficult to interpret. Many have focused on short-term
outcomes, rather than looking at long-term effects (in part because
long-term trials are so expensive).

Moreover, a 2018
\href{https://www.ncbi.nlm.nih.gov/pubmed/30014150}{systematic review}
published in the Annals of Internal Medicine found that probiotic trials
often do not report adequate safety data, particularly when it comes to
potential side effects, which raises questions about whether probiotics
are as safe as they have been made out to be.

Among other things, Dr. Cohen wrote in a 2018
\href{https://jamanetwork.com/journals/jamainternalmedicine/article-abstract/2702973}{editorial},
since probiotics are live organisms, they could potentially spread
dangerous antibiotic resistance genes among people's gut bacteria ---
although, thankfully, there is
\href{https://www.ncbi.nlm.nih.gov/pubmed/19997864}{no evidence} that
this has happened in humans.

Still, if one thing is clear about probiotics, it is that we need more
high-quality research to determine how safe and effective the products
on store shelves are.

Advertisement

\protect\hyperlink{after-bottom}{Continue reading the main story}

\hypertarget{site-index}{%
\subsection{Site Index}\label{site-index}}

\hypertarget{site-information-navigation}{%
\subsection{Site Information
Navigation}\label{site-information-navigation}}

\begin{itemize}
\tightlist
\item
  \href{https://help.nytimes3xbfgragh.onion/hc/en-us/articles/115014792127-Copyright-notice}{©~2020~The
  New York Times Company}
\end{itemize}

\begin{itemize}
\tightlist
\item
  \href{https://www.nytco.com/}{NYTCo}
\item
  \href{https://help.nytimes3xbfgragh.onion/hc/en-us/articles/115015385887-Contact-Us}{Contact
  Us}
\item
  \href{https://www.nytco.com/careers/}{Work with us}
\item
  \href{https://nytmediakit.com/}{Advertise}
\item
  \href{http://www.tbrandstudio.com/}{T Brand Studio}
\item
  \href{https://www.nytimes3xbfgragh.onion/privacy/cookie-policy\#how-do-i-manage-trackers}{Your
  Ad Choices}
\item
  \href{https://www.nytimes3xbfgragh.onion/privacy}{Privacy}
\item
  \href{https://help.nytimes3xbfgragh.onion/hc/en-us/articles/115014893428-Terms-of-service}{Terms
  of Service}
\item
  \href{https://help.nytimes3xbfgragh.onion/hc/en-us/articles/115014893968-Terms-of-sale}{Terms
  of Sale}
\item
  \href{https://spiderbites.nytimes3xbfgragh.onion}{Site Map}
\item
  \href{https://help.nytimes3xbfgragh.onion/hc/en-us}{Help}
\item
  \href{https://www.nytimes3xbfgragh.onion/subscription?campaignId=37WXW}{Subscriptions}
\end{itemize}
