Sections

SEARCH

\protect\hyperlink{site-content}{Skip to
content}\protect\hyperlink{site-index}{Skip to site index}

\href{https://www.nytimes3xbfgragh.onion/section/health}{Health}

\href{https://myaccount.nytimes3xbfgragh.onion/auth/login?response_type=cookie\&client_id=vi}{}

\href{https://www.nytimes3xbfgragh.onion/section/todayspaper}{Today's
Paper}

\href{/section/health}{Health}\textbar{}New Strawberry-Flavored H.I.V.
Drugs for Babies Are Offered at \$1 a Day

\url{https://nyti.ms/2L5cdBX}

\begin{itemize}
\item
\item
\item
\item
\item
\end{itemize}

Advertisement

\protect\hyperlink{after-top}{Continue reading the main story}

Supported by

\protect\hyperlink{after-sponsor}{Continue reading the main story}

Global health

\hypertarget{new-strawberry-flavored-hiv-drugs-for-babies-are-offered-at-1-a-day}{%
\section{New Strawberry-Flavored H.I.V. Drugs for Babies Are Offered at
\$1 a
Day}\label{new-strawberry-flavored-hiv-drugs-for-babies-are-offered-at-1-a-day}}

Thousands of infants are doomed to early deaths each year, in part
because pediatric medicines come in hard pills or bitter syrups that
need refrigeration.

\includegraphics{https://static01.graylady3jvrrxbe.onion/images/2019/11/29/science/29AIDS2/merlin_165079518_d1513391-9f85-4c2a-aca0-49b4c13bb51c-articleLarge.jpg?quality=75\&auto=webp\&disable=upscale}

\href{https://www.nytimes3xbfgragh.onion/by/donald-g-mcneil-jr}{\includegraphics{https://static01.graylady3jvrrxbe.onion/images/2018/06/13/multimedia/author-donald-g-mcneil-jr/author-donald-g-mcneil-jr-thumbLarge-v4.png}}

By
\href{https://www.nytimes3xbfgragh.onion/by/donald-g-mcneil-jr}{Donald
G. McNeil Jr.}

\begin{itemize}
\item
  Nov. 29, 2019
\item
  \begin{itemize}
  \item
  \item
  \item
  \item
  \item
  \end{itemize}
\end{itemize}

About 80,000 babies and toddlers die of AIDS each year, mostly in
Africa, in part because their medicines come in hard pills or bitter
syrups that are very difficult for small children to swallow or keep
down.

But on Friday, the Indian generic drug manufacturer Cipla announced a
new, more palatable pediatric formulation. The new drug, called
Quadrimune, comes in strawberry-flavored granules the size of grains of
sugar that can be mixed with milk or sprinkled on baby cereal. Experts
said it could save the lives of thousands of children each year.

``This is excellent news for all children living with H.I.V.,'' said
\href{https://twitter.com/Winnie_Byanyima?ref_src=twsrc\%5Egoogle\%7Ctwcamp\%5Eserp\%7Ctwgr\%5Eauthor}{Winnie
Byanyima}, the new executive director of UNAIDS, the United Nations
agency in charge of the fight against the disease. ``We have been
eagerly waiting for child-friendly medicines that are easy to use and
good to taste.''

Cipla
\href{https://www.nytimes3xbfgragh.onion/2001/02/07/world/indian-company-offers-to-supply-aids-drugs-at-low-cost-in-africa.html}{revolutionized
the provision of AIDS drugs for adults almost two decades ago}, pricing
them at \$1 a day. The new pediatric formulation will likewise be priced
at \$1 a day. The announcement by Cipla and the Drugs for Neglected
Diseases Initiative, an offshoot of Doctors Without Borders that
supported the development of the drug, was timed to coincide with World
AIDS Day, which is Sunday.

Despite big advances in the prevention of mother-child transmission of
H.I.V., \href{https://www.unaids.org/en/resources/fact-sheet}{about
160,000 children are still born infected each year}, according to
UNAIDS, mostly in the poorest towns and villages of Africa. Almost half
of them die before the age of 2, usually because they have
\href{https://www.aidsmap.com/news/may-2019/why-has-uptake-lopinavirritonavir-oral-pellets-children-been-slow}{no
access to drugs} or cannot tolerate them.

Quadrimune is still under review by the Food and Drug Administration,
and F.D.A. approval almost inevitably leads to rapid certification by
the World Health Organization. The company hopes to get a decision by
May.

Trials in healthy adults showed that the new formulation gets the drugs
into the blood; the four drugs in it were approved in the 1990s and are
used in many combinations.

A
\href{https://www.dndi.org/2019/media-centre/news-views-stories/news/study-for-a-child-friendly-hiv-treatment-begins-in-uganda/}{clinical
trial in H.I.V.-infected infants}, run by
\href{https://epicentre.msf.org/en/acceuil}{Epicentre}, the research arm
of Doctors Without Borders, is now underway in Uganda to prove to
African health ministries that children accept the new formulation. Most
of the research costs have been paid by
\href{https://unitaid.org/\#en}{UNITAID}, a Geneva-based organization
set up by​ France, Norway, Brazil and some other countries ​which
imposed special taxes on airline flights that are dedicated to
​bettering​ global health.

Currently, the most common pediatric drug combination includes a syrup
that is 40 percent alcohol, has a
\href{https://www.accessdata.fda.gov/drugsatfda_docs/label/2012/020945s034lbl.pdf}{bitter
metallic taste} that lingers for hours and must be transported in cold
trucks and then kept in a refrigerator --- something that many poor
rural families do not own.

``Some families try to bury it in wet sand or dirt to keep it cool,''
said
\href{https://www.dndi.org/about-dndi/our-people/leadership/bernard-pecoul/}{Dr.
Bernard Pécoul}, executive director of the neglected diseases
initiative. ``And the children
\href{https://www.youtube.com/watch?v=Lbo2zL9JfG8}{are vomiting it} on a
regular basis.''

\includegraphics{https://static01.graylady3jvrrxbe.onion/images/2019/11/29/science/29AIDS1/merlin_165079491_26b17857-b1fe-40b4-84d3-0ee050357c80-articleLarge.jpg?quality=75\&auto=webp\&disable=upscale}

Moreover, each drug must be
\href{https://www.youtube.com/watch?v=1pJNJaxZnNs\&feature=emb_title}{squirted
into a child's mouth with a separate syringe}, so a mother must have up
to four syringes on hand and clean them for each subsequent use.
Children generally have to take the medicines twice a day for the first
four years of life. When liquid versions are unavailable, some pills
cannot be crushed and mixed in juice; they must be swallowed whole.

In contrast, Quadrimune contains four H.I.V. drugs: ritonavir,
lopinavir, abacavir and lamivudine. The granules are coated first in a
polymer that doesn't melt until it reaches the stomach, and then with
sweet, fruity flavoring.

\href{https://www.fic.nih.gov/News/Examples/Pages/ImprovingHIVAIDS.aspx}{Dr.
Kogie Naidoo}, who heads treatment research at
\href{https://www.caprisa.org/}{Caprisa}, an AIDS treatment and research
group based in Durban, South Africa, who was not involved in
Quadrimune's development, said the new formulation could solve many
problems she and her colleagues encounter while treating children.

Cipla, founded in 1935, was
\href{https://www.amfar.org/articles/around-the-world/treatasia/older/an-interview-with-cipla-s-yusuf-hamied\%E2\%80\%94indian-drug-maker-leads-the-charge-for-low-cost-aids-drugs/}{the
first generic drug company to offer H.I.V. drugs in Africa}. In 2001,
its chairman,
\href{https://www.hbs.edu/creating-emerging-markets/interviews/Pages/profile.aspx?profile=yhamied}{Yusuf
K. Hamied},
\href{https://www.nytimes3xbfgragh.onion/2001/02/07/world/indian-company-offers-to-supply-aids-drugs-at-low-cost-in-africa.html}{upended
the global pharmaceutical industry by offering to supply a three-drug
cocktail}to Doctors Without Borders for \$1 a day.

At the time, multinational drug companies were charging up to \$15,000
for their regimens and refusing to lower prices except in secret
negotiations with a few countries and were working to block generic
competitors from the market. An estimated 25 million Africans were then
infected and thousands were dying every day. (The industry was also
suing South Africa's president, Nelson Mandela, over a law he had signed
authorizing the government to cancel drug patents and award them to
generic makers.)

In 2001, Dr. Hamied said he was losing money at the \$350 a year price;
his break-even point was \$600, he said, and he offered it to other
buyers for that.

But he said he acceded to requests from AIDS activists for the \$1 a day
price to deliver a shock to his Western competitors and because such a
nice round figure was likely to make headlines (a gambit he is clearly
repeating now).

In the decades since, increased generic competition has driven the price
of triple therapy in poor countries to below \$100 a year.

``Over the past 20 years, Cipla has pioneered fixed-dose combinations
for children and I do believe our Quadrimune could be a winner,'' Dr.
Hamied said in an interview this week.

Because all four drugs in the formulation are older and no longer
patented, Cipla might eventually offer it in wealthy countries too, he
said. But that market is quite small because most pregnant women in the
West are tested for H.I.V. and immediately put on antiretroviral drugs,
which reduces to near zero the chances that they will infect their
babies in the womb, during birth or through breastfeeding.

The \$1 a day price is for Quadrimune doses appropriate for children of
between 20 and 30 pounds, he noted, so the cost for newborns would be
even lower.

Paradoxically, treating infants with H.I.V. has actually become harder
in recent years than it was two decades ago.

In the early 2000s, Cipla produced Triomune Baby and Triomune Junior,
two pediatric formulations of the world's first adult three-in-one pill,
introduced in 2001.

But they contained nevirapine, a drug that in those days was often given
to pregnant women to prevent mother-child transmission. As a result,
many babies were born with nevirapine-resistant forms of the virus, and
the efficacy of pediatric Triomune fell by about half, Dr. Pécoul said.

\textbf{\emph{{[}}\href{http://on.fb.me/1paTQ1h}{\emph{Like the Science
Times page on Facebook.}}} ****** \emph{\textbar{} Sign up for the}
\textbf{\href{http://nyti.ms/1MbHaRU}{\emph{Science Times
newsletter.}}\emph{{]}}}

Some nevirapine substitutes that work in adults do not work well in
children, and the combination that does work has the bitter taste.

Nowadays, many pediatric H.I.V. specialists are frustrated that they
cannot prescribe some of the newest drugs, such as tenofovir and
dolutegravir, because there is little or no data on how safe they are in
small children. The major drugmakers have little incentive to test their
products in children because there are so few customers who can pay high
prices.

If a child's virus develops resistance to any regimen, a new one must be
tried, so more research is needed, said Dr. Naidoo, the AIDS researcher
in Durban.

But by any measure, she said, Cipla's new, gentler formulation for
children is a major advance: ``This is indeed great news for treating
pediatric H.I.V.''

Advertisement

\protect\hyperlink{after-bottom}{Continue reading the main story}

\hypertarget{site-index}{%
\subsection{Site Index}\label{site-index}}

\hypertarget{site-information-navigation}{%
\subsection{Site Information
Navigation}\label{site-information-navigation}}

\begin{itemize}
\tightlist
\item
  \href{https://help.nytimes3xbfgragh.onion/hc/en-us/articles/115014792127-Copyright-notice}{©~2020~The
  New York Times Company}
\end{itemize}

\begin{itemize}
\tightlist
\item
  \href{https://www.nytco.com/}{NYTCo}
\item
  \href{https://help.nytimes3xbfgragh.onion/hc/en-us/articles/115015385887-Contact-Us}{Contact
  Us}
\item
  \href{https://www.nytco.com/careers/}{Work with us}
\item
  \href{https://nytmediakit.com/}{Advertise}
\item
  \href{http://www.tbrandstudio.com/}{T Brand Studio}
\item
  \href{https://www.nytimes3xbfgragh.onion/privacy/cookie-policy\#how-do-i-manage-trackers}{Your
  Ad Choices}
\item
  \href{https://www.nytimes3xbfgragh.onion/privacy}{Privacy}
\item
  \href{https://help.nytimes3xbfgragh.onion/hc/en-us/articles/115014893428-Terms-of-service}{Terms
  of Service}
\item
  \href{https://help.nytimes3xbfgragh.onion/hc/en-us/articles/115014893968-Terms-of-sale}{Terms
  of Sale}
\item
  \href{https://spiderbites.nytimes3xbfgragh.onion}{Site Map}
\item
  \href{https://help.nytimes3xbfgragh.onion/hc/en-us}{Help}
\item
  \href{https://www.nytimes3xbfgragh.onion/subscription?campaignId=37WXW}{Subscriptions}
\end{itemize}
