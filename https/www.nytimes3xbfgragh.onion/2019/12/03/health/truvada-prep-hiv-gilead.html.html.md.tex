Sections

SEARCH

\protect\hyperlink{site-content}{Skip to
content}\protect\hyperlink{site-index}{Skip to site index}

\href{https://www.nytimes3xbfgragh.onion/section/health}{Health}

\href{https://myaccount.nytimes3xbfgragh.onion/auth/login?response_type=cookie\&client_id=vi}{}

\href{https://www.nytimes3xbfgragh.onion/section/todayspaper}{Today's
Paper}

\href{/section/health}{Health}\textbar{}200,000 Uninsured Americans to
Get Free H.I.V.-Prevention Drugs

\url{https://nyti.ms/2PfQx7A}

\begin{itemize}
\item
\item
\item
\item
\item
\item
\end{itemize}

Advertisement

\protect\hyperlink{after-top}{Continue reading the main story}

Supported by

\protect\hyperlink{after-sponsor}{Continue reading the main story}

Global health

\hypertarget{200000-uninsured-americans-to-get-free-hiv-prevention-drugs}{%
\section{200,000 Uninsured Americans to Get Free H.I.V.-Prevention
Drugs}\label{200000-uninsured-americans-to-get-free-hiv-prevention-drugs}}

A new government program will provide donated drugs through major
drugstore chains.

\includegraphics{https://static01.graylady3jvrrxbe.onion/images/2019/12/03/science/03PREP1/03PREP1-articleLarge.jpg?quality=75\&auto=webp\&disable=upscale}

\href{https://www.nytimes3xbfgragh.onion/by/donald-g-mcneil-jr}{\includegraphics{https://static01.graylady3jvrrxbe.onion/images/2018/06/13/multimedia/author-donald-g-mcneil-jr/author-donald-g-mcneil-jr-thumbLarge-v4.png}}

By
\href{https://www.nytimes3xbfgragh.onion/by/donald-g-mcneil-jr}{Donald
G. McNeil Jr.}

\begin{itemize}
\item
  Dec. 3, 2019
\item
  \begin{itemize}
  \item
  \item
  \item
  \item
  \item
  \item
  \end{itemize}
\end{itemize}

With donated drugs and services provided by major pharmacy chains,
200,000 uninsured Americans will gain access to H.I.V.-preventive
medicines at no cost, the Trump administration announced on Tuesday.

The announcement, by Alex M. Azar II, the health and human services
secretary, essentially explained how the government plans to distribute
the drugs for pre-exposure prophylaxis, or PrEP, that were
\href{https://www.nytimes3xbfgragh.onion/2019/05/09/health/gilead-truvada-hiv-aids.html}{promised
in May by the drugmaker Gilead Sciences}.

PrEP describes a strategy of preventing infection with H.I.V. by taking
a single pill a day, either Truvada or Descovy. Both are made by Gilead.
The strategy is 99 percent effective at preventing infection, studies
have shown, and is a mainstay of the administration's campaign to end
the H.I.V. epidemic.

Some American cities with high H.I.V. rates, such as San Francisco,
already have programs that pay the costs of PrEP for the uninsured.
Gilead itself offers the drug at no cost to those who cannot afford it,
or picks up insurance co-pays for patients who qualify.

But the new program --- called Ready, Set, PrEP --- marks the first time
the federal government is supplying PrEP to patients not enrolled in
Medicaid, the Veterans Health Administration or any other federal health
program.

Under the new program, any patient who lacks health insurance, has had a
recent negative H.I.V. test and has a prescription for PrEP ---
presumably obtained from a doctor --- can call 855-447-8410 or sign onto
a new government website,
\href{https://www.getyourprep.com/}{getyourprep.com}, to apply for free
H.I.V.-prevention drugs.

They can also apply in person, Mr. Azar said, through a participating
health care provider, such as a community clinic.

Until March 30, the government will
\href{https://news.bloomberglaw.com/health-law-and-business/gilead-wins-6-million-deal-to-distribute-hiv-drug-it-donated}{pay
Gilead \$200 per bottle} --- each bottle of Truvada contains 30 pills
--- to cover the cost of moving donated drugs from factories through the
supply chain to patients, Mr. Azar said.

\includegraphics{https://static01.graylady3jvrrxbe.onion/images/2019/12/03/science/03PREP2/merlin_138012693_50ad4676-0de0-4e9d-b490-4c88612ca071-articleLarge.jpg?quality=75\&auto=webp\&disable=upscale}

After that, he said, the Walgreens, Rite Aid and CVS chains will donate
dispensing services and offer counseling to patients, and the government
will seek cheaper ways to get the drugs from Gilead to those chains.

About
\href{https://www.cdc.gov/nchhstp/newsroom/2019/ending-HIV-transmission-press-release.html}{1.2
million Americans could benefit from PrEP} because they are at high risk
of getting H.I.V. from unprotected sex or needle-sharing, according to
Centers for Disease Control and Prevention. Only an estimated 270,000
people are now taking the drugs.

Truvada and Descovy are now
\href{https://www.nytimes3xbfgragh.onion/2019/11/08/health/hiv-prevention-truvada-patents.html}{mired
in billion-dollar patent lawsuits pitting H.H.S. against Gilead}. The
federal government and the company both claim ownership of patents
covering the use of the drugs to prevent H.I.V. infection.

But Mr. Azar said the new program was ``not related'' to the lawsuits.
``That matter will be resolved in the court system,'' he said.

Dr. Rochelle P. Walensky, leader of a team at Massachusetts General
Hospital that has
\href{https://www.nytimes3xbfgragh.onion/2019/03/12/health/trump-hiv-aids-costs.html}{analyzed
the costs of the H.I.V. plans of the Obama and Trump administrations},
said she was ``having a hard time understanding why the government is
paying \$200 a month per bottle to dispense these drugs.''

Truvada does not need refrigeration or special handling, and it is
distributed free in France and Norway. The drug costs patients only \$96
a year in Australia, \$384 in Germany and \$720 in Ireland.

The company has promised to donate enough of the drugs to cover as many
as 200,000 people for 11 years in the United States.

Mr. Azar described the \$200 per bottle payments as a stopgap measure
that will enable rolling out the drugs as soon as possible. The amount
is what Gilead claimed it pays the pharmacy supply chain to distribute
its drugs, he said.

After March 30, Mr. Azar said, with a combination of donated pharmacy
costs and competitive bidding for distribution contracts, ``I think
we'll be able to do better.''

Gilead makes no money from the distribution arrangement, said Ryan
McKeel, a company spokesman. It will reimburse vendors but not charge
the government for the time of any Gilead employees involved.

Any tax deductions the company takes for its donation, he added, will be
based on the cost of making the pills, not on their market value or
distribution costs.

James Krellenstein, a founder of the advocacy group Prep4All
Collaboration, said the government's plan ``is poised to repeat the
errors of Gilead's own medication\\
assistance program.''

While uninsured patients may get free drugs under the program, he said,
they get no help paying for the medical exam and laboratory tests needed
to get and keep renewing the prescription, which
\href{https://breakthepatent.org/wp-content/uploads/2018/07/White_Paper_Final_Edits-PS-Version.pdf}{can
cost up to \$1,000 a year}.

Instead of focusing on paying Gilead up to \$6 million for high
pharmaceutical supply chain costs, the government could pay for lab
tests for 6,000 patients, Mr. Krellenstein said.

\textbf{\emph{{[}}\href{http://on.fb.me/1paTQ1h}{\emph{Like the Science
Times page on Facebook.}}} ****** \emph{\textbar{} Sign up for the}
\textbf{\href{http://nyti.ms/1MbHaRU}{\emph{Science Times
newsletter.}}\emph{{]}}}

Mr. Azar said that some of those costs are already covered by public
clinics, and that his department is
\href{https://www.nytimes3xbfgragh.onion/2019/03/12/health/trump-hiv-aids-costs.html}{seeking
\$291 million this year from Congress} to defray other fees in about 50
high-risk areas, beginning with Baltimore; Baton Rouge, La.; DeKalb
County, in Georgia; and Cherokee Nation reservations.

To increase awareness of PrEP, Mr. Azar said, Walgreens and Health Mart,
a coalition of independent pharmacies, will publicize the program.

In some cities, including New York, posters and billboards encouraging
gay men and other people at risk to use PrEP are ubiquitous. But in more
conservative parts of the country --- including the rural South, which
is now one of the epidemic's hottest zones --- awareness campiagns are
uncommon.

Advertisement

\protect\hyperlink{after-bottom}{Continue reading the main story}

\hypertarget{site-index}{%
\subsection{Site Index}\label{site-index}}

\hypertarget{site-information-navigation}{%
\subsection{Site Information
Navigation}\label{site-information-navigation}}

\begin{itemize}
\tightlist
\item
  \href{https://help.nytimes3xbfgragh.onion/hc/en-us/articles/115014792127-Copyright-notice}{©~2020~The
  New York Times Company}
\end{itemize}

\begin{itemize}
\tightlist
\item
  \href{https://www.nytco.com/}{NYTCo}
\item
  \href{https://help.nytimes3xbfgragh.onion/hc/en-us/articles/115015385887-Contact-Us}{Contact
  Us}
\item
  \href{https://www.nytco.com/careers/}{Work with us}
\item
  \href{https://nytmediakit.com/}{Advertise}
\item
  \href{http://www.tbrandstudio.com/}{T Brand Studio}
\item
  \href{https://www.nytimes3xbfgragh.onion/privacy/cookie-policy\#how-do-i-manage-trackers}{Your
  Ad Choices}
\item
  \href{https://www.nytimes3xbfgragh.onion/privacy}{Privacy}
\item
  \href{https://help.nytimes3xbfgragh.onion/hc/en-us/articles/115014893428-Terms-of-service}{Terms
  of Service}
\item
  \href{https://help.nytimes3xbfgragh.onion/hc/en-us/articles/115014893968-Terms-of-sale}{Terms
  of Sale}
\item
  \href{https://spiderbites.nytimes3xbfgragh.onion}{Site Map}
\item
  \href{https://help.nytimes3xbfgragh.onion/hc/en-us}{Help}
\item
  \href{https://www.nytimes3xbfgragh.onion/subscription?campaignId=37WXW}{Subscriptions}
\end{itemize}
