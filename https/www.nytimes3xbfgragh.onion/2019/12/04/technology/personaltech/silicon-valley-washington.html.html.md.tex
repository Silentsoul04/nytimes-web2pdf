Sections

SEARCH

\protect\hyperlink{site-content}{Skip to
content}\protect\hyperlink{site-index}{Skip to site index}

\href{https://www.nytimes3xbfgragh.onion/section/technology/personaltech}{Personal
Tech}

\href{https://myaccount.nytimes3xbfgragh.onion/auth/login?response_type=cookie\&client_id=vi}{}

\href{https://www.nytimes3xbfgragh.onion/section/todayspaper}{Today's
Paper}

\href{/section/technology/personaltech}{Personal Tech}\textbar{}Silicon
Valley Learns Washington's Language (and Vice Versa)

\url{https://nyti.ms/2OQkcow}

\begin{itemize}
\item
\item
\item
\item
\item
\end{itemize}

Advertisement

\protect\hyperlink{after-top}{Continue reading the main story}

Supported by

\protect\hyperlink{after-sponsor}{Continue reading the main story}

Tech We're Using

\hypertarget{silicon-valley-learns-washingtons-language-and-vice-versa}{%
\section{Silicon Valley Learns Washington's Language (and Vice
Versa)}\label{silicon-valley-learns-washingtons-language-and-vice-versa}}

Big Tech's presence in the capital is unmistakable, and its interests
intersect with more and more issues, says David McCabe, a tech policy
reporter.

\includegraphics{https://static01.graylady3jvrrxbe.onion/images/2019/12/04/business/04techusing/merlin_165174558_81dadebe-7f3a-4287-a92c-a11bc7774a46-articleLarge.jpg?quality=75\&auto=webp\&disable=upscale}

Featuring David McCabe

\begin{itemize}
\item
  Dec. 4, 2019
\item
  \begin{itemize}
  \item
  \item
  \item
  \item
  \item
  \end{itemize}
\end{itemize}

\emph{How do New York Times journalists use technology in their jobs and
in their personal lives? David McCabe, who covers tech policy in
Washington, discussed the tech he's using.}

\textbf{What tech is most important for doing your job?}

One element of covering policy is understanding what forces are pushing
certain ideas forward in Washington. I often turn to government and
private disclosure databases to help answer that question.

There's a lot of information a few clicks away: Congress
\href{https://soprweb.senate.gov/index.cfm?event=selectfields}{maintains}
\href{http://disclosures.house.gov/ld/ldsearch.aspx}{databases} of whom
companies are paying to lobby on their behalf. The Justice Department
\href{https://efile.fara.gov/ords/f?p=1381:1:244311337896:::::}{tracks
which operatives} are registered to work for foreign firms. The service
Guidestar collects the forms that influential nonprofits file with the
I.R.S. describing their finances, and many prominent foundations list
their grants online.

These records don't tell the full story, but they can be a source of
valuable context when you're chasing a tip or trying to understand the
landscape around a new issue.

I try to speak with sources where they are most comfortable. That can be
on the phone, by text or encrypted message, or in person. Twitter's and
LinkedIn's direct message features can be great ways to reach out to
people.

For on-the-record interviews, I've recently started using the automated
transcription app Otter at the recommendation of other reporters. If I'm
doing an interview by phone, I use an Olympus earpiece that plugs into
my recorder to capture the conversation.

\textbf{How is Silicon Valley having an impact on Washington?}

Washington has become intertwined with the Valley in lots of different
ways.

Every major tech company has ramped up its presence here. Small armies
of lobbyists work Capitol Hill and a vast swath of the administration to
fight attempts to regulate the industry or to shape the rules when they
become inevitable.

The revolving door between tech and Washington may have lost some of its
luster --- a job at Google or Facebook isn't as shiny as it used to be
--- but it is still functioning just fine. For example, Google just
hired the chief of staff to Senator Rob Portman, a Republican from Ohio,
to lead its Washington office.

\includegraphics{https://static01.graylady3jvrrxbe.onion/images/2019/12/04/business/04techusing2/merlin_165174549_c0260810-e1d3-4254-9096-b66c6faf5acd-articleLarge.jpg?quality=75\&auto=webp\&disable=upscale}

\textbf{How do Washington folks use tech differently, compared with
other parts of the country?}

My unscientific conclusion is that the Washington metro area has the
highest per-capita concentration of Twitter users whose profile photo is
a shot of them appearing on cable news. Some of those people are my
friends, and I would like to take this opportunity to urge them to
reconsider.

\textbf{How has the tech fluency among lawmakers changed, or not, over
time?}

Many lawmakers have become more fluent in tech --- or at least more
conversant --- since the infamous
\href{https://www.nytimes3xbfgragh.onion/2018/04/10/us/politics/zuckerberg-facebook-senate-hearing.html}{Mark
Zuckerberg hearings in 2018}. There's an understanding that before they
became giants, Capitol Hill didn't pay enough attention to the platforms
that now account for a large share of the economy. In both the House and
the Senate, lawmakers are often asking sharper questions than they used
to and doing a better job of diving into the specifics.

There's a related issue: For years, ``tech policy'' largely meant
communications policy. It was about regulating the infrastructure that
allows us to share information with one another. That has much wider
implications than it used to.

If you oversee banking, you have to grok
\href{https://www.nytimes3xbfgragh.onion/2019/10/11/technology/facebook-libra-partners.html}{Facebook's
attempt to build a cryptocurrency}.
\href{https://www.nytimes3xbfgragh.onion/2019/11/11/business/google-ascension-health-data.html}{Google's
use of patient records} is a health policy question as much as it's a
tech policy one. I'm interested in how lawmakers and their staff members
adapt to this moment, when major tech companies are trying to play in
more and more parts of modern life even as they face scrutiny for the
business models that made them successful.

\textbf{And what is the tech industry like in Washington?}

All eyes are on Amazon and, more specifically, on its
\href{https://www.nytimes3xbfgragh.onion/2018/11/13/business/national-landing-amazon-va.html}{second
headquarters} in Northern Virginia. Its arrival has prompted a lot of
questions about whether it will trigger even more gentrification in an
\href{https://dcist.com/story/19/03/19/d-c-has-had-the-most-gentrifying-neighborhoods-in-the-country-study-finds/}{already
gentrifying} city.

Amazon has been aggressive in building relationships with local
educational institutions. In October, its chief executive, Jeff Bezos,
visited a class at Dunbar High School in Washington, and the company is
counting on local universities to help build the work force for its new
offices.

The efforts have been met with resistance that is emblematic of the
wider reckoning tech faces around its impact on society. Jay Carney, the
former White House press secretary who leads Amazon's policy apparatus,
visited Trinity Washington University this fall. While he was there, an
undocumented student raised concerns about Amazon's work with United
States Immigration and Customs Enforcement, and he was met by
anti-Amazon activists outside during his visit.

\textbf{Outside of work, what tech are you personally obsessed with?}

I'm a heavy Spotify user and find that
\href{https://www.nytimes3xbfgragh.onion/2015/06/04/technology/personaltech/spotify-sees-a-future-where-music-genres-dont-really-matter.html}{music
streaming} has broadened my musical horizons. I listen to more artists
than I did before and pay more attention to new releases.

I am also one of those people who fell hard for AirPods. They look
silly, but they work very well. Contrary to my expectations, I haven't
lost one of the earbuds yet. That said: Buying the new, pricey AirPods
Pro seems like tempting the earbud-losing fates. I'm staying away and
saving my money.

Frankly, I spend more of my personal time trying to disconnect. I delete
Twitter from my phone on vacation and on weekends when I'm being
especially disciplined. I have been far less successful in cutting down
on my Instagram use, but hope springs eternal.

Advertisement

\protect\hyperlink{after-bottom}{Continue reading the main story}

\hypertarget{site-index}{%
\subsection{Site Index}\label{site-index}}

\hypertarget{site-information-navigation}{%
\subsection{Site Information
Navigation}\label{site-information-navigation}}

\begin{itemize}
\tightlist
\item
  \href{https://help.nytimes3xbfgragh.onion/hc/en-us/articles/115014792127-Copyright-notice}{©~2020~The
  New York Times Company}
\end{itemize}

\begin{itemize}
\tightlist
\item
  \href{https://www.nytco.com/}{NYTCo}
\item
  \href{https://help.nytimes3xbfgragh.onion/hc/en-us/articles/115015385887-Contact-Us}{Contact
  Us}
\item
  \href{https://www.nytco.com/careers/}{Work with us}
\item
  \href{https://nytmediakit.com/}{Advertise}
\item
  \href{http://www.tbrandstudio.com/}{T Brand Studio}
\item
  \href{https://www.nytimes3xbfgragh.onion/privacy/cookie-policy\#how-do-i-manage-trackers}{Your
  Ad Choices}
\item
  \href{https://www.nytimes3xbfgragh.onion/privacy}{Privacy}
\item
  \href{https://help.nytimes3xbfgragh.onion/hc/en-us/articles/115014893428-Terms-of-service}{Terms
  of Service}
\item
  \href{https://help.nytimes3xbfgragh.onion/hc/en-us/articles/115014893968-Terms-of-sale}{Terms
  of Sale}
\item
  \href{https://spiderbites.nytimes3xbfgragh.onion}{Site Map}
\item
  \href{https://help.nytimes3xbfgragh.onion/hc/en-us}{Help}
\item
  \href{https://www.nytimes3xbfgragh.onion/subscription?campaignId=37WXW}{Subscriptions}
\end{itemize}
