Sections

SEARCH

\protect\hyperlink{site-content}{Skip to
content}\protect\hyperlink{site-index}{Skip to site index}

\href{https://www.nytimes3xbfgragh.onion/section/science}{Science}

\href{https://myaccount.nytimes3xbfgragh.onion/auth/login?response_type=cookie\&client_id=vi}{}

\href{https://www.nytimes3xbfgragh.onion/section/todayspaper}{Today's
Paper}

\href{/section/science}{Science}\textbar{}How an Icy Moon of Saturn Got
Its Stripes

\url{https://nyti.ms/2RBYuXg}

\begin{itemize}
\item
\item
\item
\item
\item
\item
\end{itemize}

Advertisement

\protect\hyperlink{after-top}{Continue reading the main story}

Supported by

\protect\hyperlink{after-sponsor}{Continue reading the main story}

Trilobites

\hypertarget{how-an-icy-moon-of-saturn-got-its-stripes}{%
\section{How an Icy Moon of Saturn Got Its
Stripes}\label{how-an-icy-moon-of-saturn-got-its-stripes}}

Scientists have developed an explanation for one of the most striking
features of Enceladus, an ocean world that has the right ingredients for
life.

\includegraphics{https://static01.graylady3jvrrxbe.onion/images/2019/12/17/science/17OBS-ICYMOON1/09TB-ICYMOON1-articleLarge.jpg?quality=75\&auto=webp\&disable=upscale}

By Nadia Drake

\begin{itemize}
\item
  Dec. 9, 2019
\item
  \begin{itemize}
  \item
  \item
  \item
  \item
  \item
  \item
  \end{itemize}
\end{itemize}

Of the strange and unexplained terrains in our solar system, the south
pole of Saturn's moon Enceladus is among the most perplexing.

Enceladus is an ocean world, with a vast and briny sea tucked beneath
its icy crust; this makes it one of the most tantalizing places in the
solar system to look for life beyond Earth. But unlike other frozen
moons, Enceladus constantly erupts. The tiny world blasts salty water
into space through cracks in its crystalline shell. These fissures,
raked across the moon's southern pole, are roughly parallel and evenly
spaced. And ever since scientists first took a good look at this alien
moon, they've had a tough time explaining those ``tiger stripes.''

``What is going on?''
said\href{https://dtm.carnegiescience.edu/people/postdoctoral/doug-hemingway}{Doug
Hemingway} of the Carnegie Institution for Science. ``In a way, it's an
obvious question --- it's been in the back of everyone's mind for a long
time.''

Now, Dr. Hemingway and his colleagues think they know how the moon got
its stripes --- and, curiously, why the stripes are found only at the
Enceladian south pole. They
\href{https://www.nature.com/articles/s41550-019-0958-x}{described their
hypothesis Monday in Nature Astronomy}. Learning more about how
extraterrestrial oceans on worlds such as Enceladus evolve and interact
with planetary surfaces is important for understanding how life might
exist beyond Earth and how we might find it, Dr. Hemingway said.

In 2005, NASA's
\href{https://www.nytimes3xbfgragh.onion/interactive/2017/09/14/science/cassini-saturn-images.html}{Cassini
spacecraft} first swooped in and stared at 313-mile-wide Enceladus. The
spacecraft saw
\href{https://www.nytimes3xbfgragh.onion/2015/10/29/science/space/in-icy-breath-of-saturns-moon-enceladus-cassini-hunts-for-life.html}{a
stunning array of geysers} erupting from the moon's south pole ---
eruptions that vent the moon's ocean into space and sculpt Saturn's E
ring. Later, when Cassini flew through the jets, it tasted an alien soup
containing all the ingredients necessary for life as we know it.

\includegraphics{https://static01.graylady3jvrrxbe.onion/images/2019/12/09/science/09TB-ICYMOON2/09TB-ICYMOON2-articleLarge.jpg?quality=75\&auto=webp\&disable=upscale}

The moon's southern hemisphere is riven by four prominent fractures.
Approximately parallel and spaced about 20 miles apart, the massive
cracks are, like other features of
Enceladus,\href{https://planetarynames.wr.usgs.gov/Page/Categories}{named
after locations in ``One Thousand and One Nights}.''

Previous ideas about the origins of the cracks --- Alexandria, Baghdad,
Cairo, Damascus and a smaller crack unpoetically referred to as ``E''
--- included massive impacts, hot spots, strike-slip faulting and a
migrating icy shell. Dr. Hemingway and his colleagues modeled the
evolution of the moon's icy shell, accounting for its thickness,
elasticity, strength and temperature, and uncovered a simpler, more
comprehensive explanation.

Some time after it formed, they think, Enceladus slowly began to cool.
Some of its inner ocean froze, expanded --- as frozen water does --- and
strained the moon's icy crust, which was thinner at the poles.

Eventually, the swelling sea fractured the southern crust.

\href{https://www.nytimes3xbfgragh.onion/interactive/2017/09/14/science/cassini-saturn-images.html}{}

\includegraphics{https://static01.graylady3jvrrxbe.onion/images/2017/09/16/insider/cassini-saturn-images-1505332179034/cassini-saturn-images-1505332179034-square640-v3.jpg}

\hypertarget{100-images-from-cassinis-mission-to-saturn}{%
\subsection{100 Images From Cassini's Mission to
Saturn}\label{100-images-from-cassinis-mission-to-saturn}}

NASA's Cassini spacecraft burned up in Saturn's atmosphere on Friday,
after 20 years in space.

The first fissure to form was 80-mile-long Baghdad, the largest and most
prominent of the tiger stripes. As water began erupting through Baghdad,
some of it snowed back to the moon's surface, piling up near the
fracture's margins. The weight of that accumulating material strained
the ice shell, and new cracks --- Cairo and Damascus --- opened up on
either side of Baghdad, each roughly parallel and about 20 miles away.

Then Alexandria and ``E'' opened up.

``Why doesn't this cascading sequence just keep going?'' Dr. Hemingway
said. ``It's not clear. Maybe as you open up more and more of these
fractures, the eruption rate per ridge kind of drops, or the overall
background ice shell thickness just gets too big.''

The process may have taken between 100,000 and one million years. The
tiger stripes' even spacing is simply a result of the ice's elasticity
and its thickness, which is thinner at the poles and bulkier at the
equator.

But why did the stripes only rupture the southern pole?

``It's kind of a coin toss whether that first fracture happens at the
north pole or the south pole,'' Dr. Hemingway said. But as soon as the
crust breaks open, he added, the swelling ocean's pressure is relieved
``and the other pole will just stay quiet for the rest of time.''

Although this new model answers many questions about the strange moon
orbiting Saturn, several remain.
\href{http://rhodenresearch.weebly.com/}{Alyssa Rhoden} of the Southwest
Research Institute called the hypothesis plausible, but she wonders how
the model links up with other, less dramatic fractures and features on
the moon:

``Now that we have this idea, how do we fit it into the broader picture
of how Enceladus evolved over time?''

Image

Credit...JPL/NASA

Advertisement

\protect\hyperlink{after-bottom}{Continue reading the main story}

\hypertarget{site-index}{%
\subsection{Site Index}\label{site-index}}

\hypertarget{site-information-navigation}{%
\subsection{Site Information
Navigation}\label{site-information-navigation}}

\begin{itemize}
\tightlist
\item
  \href{https://help.nytimes3xbfgragh.onion/hc/en-us/articles/115014792127-Copyright-notice}{©~2020~The
  New York Times Company}
\end{itemize}

\begin{itemize}
\tightlist
\item
  \href{https://www.nytco.com/}{NYTCo}
\item
  \href{https://help.nytimes3xbfgragh.onion/hc/en-us/articles/115015385887-Contact-Us}{Contact
  Us}
\item
  \href{https://www.nytco.com/careers/}{Work with us}
\item
  \href{https://nytmediakit.com/}{Advertise}
\item
  \href{http://www.tbrandstudio.com/}{T Brand Studio}
\item
  \href{https://www.nytimes3xbfgragh.onion/privacy/cookie-policy\#how-do-i-manage-trackers}{Your
  Ad Choices}
\item
  \href{https://www.nytimes3xbfgragh.onion/privacy}{Privacy}
\item
  \href{https://help.nytimes3xbfgragh.onion/hc/en-us/articles/115014893428-Terms-of-service}{Terms
  of Service}
\item
  \href{https://help.nytimes3xbfgragh.onion/hc/en-us/articles/115014893968-Terms-of-sale}{Terms
  of Sale}
\item
  \href{https://spiderbites.nytimes3xbfgragh.onion}{Site Map}
\item
  \href{https://help.nytimes3xbfgragh.onion/hc/en-us}{Help}
\item
  \href{https://www.nytimes3xbfgragh.onion/subscription?campaignId=37WXW}{Subscriptions}
\end{itemize}
