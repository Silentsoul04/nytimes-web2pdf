Sections

SEARCH

\protect\hyperlink{site-content}{Skip to
content}\protect\hyperlink{site-index}{Skip to site index}

\href{/section/business}{Business}\textbar{}Fashion Nova's Secret:
Underpaid Workers in Los Angeles Factories

\url{https://nyti.ms/2PMWZTC}

\begin{itemize}
\item
\item
\item
\item
\item
\item
\end{itemize}

\includegraphics{https://static01.graylady3jvrrxbe.onion/images/2019/12/16/fashion/16fashionnova-cortes-top/16fashionnova-cortes-top-articleLarge-v2.jpg?quality=75\&auto=webp\&disable=upscale}

\hypertarget{fashion-novas-secret-underpaid-workers-in-los-angeles-factories}{%
\section{Fashion Nova's Secret: Underpaid Workers in Los Angeles
Factories}\label{fashion-novas-secret-underpaid-workers-in-los-angeles-factories}}

The online retailer makes fast fashion for the Instagram elite. The way
many of its garments are made is much less glamorous.

Mercedes Cortes sewing Fashion Nova clothing in a garment factory in
downtown Los Angeles.Credit...Jessica Pons for The New York Times

Supported by

\protect\hyperlink{after-sponsor}{Continue reading the main story}

\href{https://www.nytimes3xbfgragh.onion/by/natalie-kitroeff}{\includegraphics{https://static01.graylady3jvrrxbe.onion/images/2019/03/01/multimedia/author-natalie-kitroeff/author-natalie-kitroeff-thumbLarge.png}}

By \href{https://www.nytimes3xbfgragh.onion/by/natalie-kitroeff}{Natalie
Kitroeff}

\begin{itemize}
\item
  Dec. 16, 2019
\item
  \begin{itemize}
  \item
  \item
  \item
  \item
  \item
  \item
  \end{itemize}
\end{itemize}

\href{https://www.nytimes3xbfgragh.onion/es/2019/12/17/espanol/negocios/nova-trabajadores.html}{Leer
en español}

LOS ANGELES --- Fashion Nova has perfected fast fashion for the
Instagram era.

The mostly online retailer leans on a vast network of celebrities,
influencers, and random selfie takers who post about the brand
relentlessly on social media. It is built to satisfy a very online
clientele, mass-producing cheap clothes that look expensive.

``They need to buy a lot of different styles and probably only wear them
a couple times so their Instagram feeds can stay fresh,'' Richard
Saghian, Fashion Nova's founder, said in an interview last year.

To enable that habit, he gives them a constant stream of new options
that are priced to sell.

The days of \$200 jeans are over, if you ask Mr. Saghian. Fashion Nova's
skintight denim goes for \$24.99. And, he said, the company can get its
clothes made ``in less than two weeks,'' often by manufacturers in Los
Angeles, a short drive from the company's headquarters.

That model hints at an ugly secret behind the brand's runaway success:
The federal Labor Department has found that many Fashion Nova garments
are stitched together by a work force in the United States that is paid
illegally low wages.

\includegraphics{https://static01.graylady3jvrrxbe.onion/images/2019/12/16/fashion/16fashionnova-cardi-3/merlin_164697915_7a692c59-d9a3-4105-91f9-640df0b6ca0f-articleLarge.jpg?quality=75\&auto=webp\&disable=upscale}

Los Angeles is filled with factories that pay workers off the books and
as little as possible, battling overseas competitors that can pay even
less. Many of the people behind the sewing machines are undocumented,
and unlikely to challenge their bosses.

``It has all the advantages of a sweatshop system,'' said David Weil,
who led the United States Labor Department's wage and hour division from
2014 to 2017.

Every year, the department investigates allegations of wage violations
at sewing contractors in Los Angeles, showing up unannounced to review
payroll data, interview employees and question the owners.

In investigations conducted from 2016 through this year, the department
discovered Fashion Nova clothing being made in dozens of factories that
owed \$3.8 million in back wages to hundreds of workers, according to
internal federal documents that summarized the findings and were
reviewed by The New York Times.

Those factories, which are hired by middlemen to produce garments for
fashion brands, paid their sewers as little as \$2.77 an hour, according
to a person familiar with the investigation.

The Labor Department declined to comment on the details of the
investigations. In a statement, a spokeswoman said the department
``continues to ensure employers receive compliance assistance with the
overtime and minimum wage requirements, and the Wage and Hour Division
is committed to enforcing the law.''

After repeated violations were found at factories making Fashion Nova
clothes, federal officials met with company representatives. ``We have
already had a highly productive and positive meeting with the Department
of Labor in which we discussed our ongoing commitment to ensuring that
all workers involved with the Fashion Nova brand are appropriately
compensated for the work they do,'' Erica Meierhans, Fashion Nova's
general counsel, said in a statement to The Times. ``Any suggestion that
Fashion Nova is responsible for underpaying anyone working on our brand
is categorically false.''

In 2018, Mr. Saghian said about 80 percent of the brand's clothes were
made in the United States. Fashion Nova's supply chain has shifted since
then, and now the brand says it makes less than half of its clothes in
Los Angeles. It would not specify the overall percentage made in the
United States.

The company does not deal directly with factories. Instead, it places
bulk orders with companies that design the clothes and then ship fabric
to separately owned sewing contractors, where workers stitch the clothes
together and stick Fashion Nova's label on them.

The brand's clingy dresses and animal-print jumpsuits are often made by
people like Mercedes Cortes, working in ramshackle buildings that smell
like bathrooms.

Ms. Cortes, 56, sewed Fashion Nova clothes for several months at Coco
Love, a dusty factory close to Fashion Nova's offices in Vernon, Calif.
``There were cockroaches. There were rats,'' she said. ``The conditions
weren't good.''

Image

Ms. Cortes would notice \$12 price tags when sewing Fashion Nova
clothes, which she said was ``very expensive for what they pay
us.''~Credit...Jessica Pons for The New York Times

She worked every day of the week, but her pay varied depending on how
quickly her fingers could move. Ms. Cortes was paid for each piece of a
shirt she sewed together --- about 4 cents to sew on each sleeve, 5
cents for each of the side seams, 8 cents for the seam on the neckline.
On average, she earned \$270 in a week, the equivalent of \$4.66 an
hour, she said.

In 2016, Ms. Cortes left Coco Love and later reached a settlement with
the company for \$5,000 in back wages. She continued to work in
factories sewing Fashion Nova clothes, noticing the \$12 price tags on
the tops she had stitched together for cents. ``The clothes are very
expensive for what they pay us,'' Ms. Cortes said.

``Consumers can say, `Well, of course that's what it's like in
Bangladesh or Vietnam,' but they are developing countries,'' Mr. Weil
said. ``People just don't want to believe it's true in their own
backyard.''

For all their seediness, these factories are still producing clothes for
major American retailers. Under federal law, brands cannot be penalized
for wage theft in factories if they can credibly claim that they did not
know their clothes were made by workers paid illegally low wages. The
Labor Department has collected millions in back wages and penalties from
Los Angeles garment businesses in recent years, but has not fined a
retailer.

This year, Fashion Nova's labels were the ones found the most frequently
by federal investigators looking into garment factories that pay
egregiously low wages, according to a person familiar with the
investigations.

In September, three officials from the department met with Fashion
Nova's lawyers to tell them that, over four years, the brand's clothes
had been found in 50 investigations of factories paying less than the
federal minimum wage or failing to pay overtime.

Image

Scrap fabric from a sewing machine used by Ms. Cortes.Credit...Jessica
Pons for The New York Times

The company's lawyers told the officials that they had taken immediate
action and had already updated the brand's agreement with vendors. Now,
if Fashion Nova learns that a factory has been charged with violating
laws ``governing the wages and hours of its employees, child labor,
forced labor or unsafe working conditions,'' the brand will put the
middleman who hired that factory on a six-month ``probation,'' it said
in a statement.

The working relationship would continue, unless workers file another
complaint against the same factory or another one that the contractor
hired during those six months. At that point, the brand will suspend the
contractor until it passes a third-party audit.

While Fashion Nova has taken steps to address the Labor Department's
findings, Ms. Meierhans, the brand's general counsel, noted that it
works with hundreds of manufacturers and ``is not responsible for how
these vendors handle their payrolls.''

\hypertarget{everyone-wants-to-have-more-followers}{%
\subsection{`Everyone wants to have more
followers'}\label{everyone-wants-to-have-more-followers}}

Mr. Saghian opened the first Fashion Nova store in 2006, in a Los
Angeles mall. Seven years and four storefronts later, he realized that
he was losing customers to online outlets selling the same clothes.

A web developer talked him out of starting a website; it would get no
traffic, because no one knew what Fashion Nova was. Mr. Saghian had a
better shot on Instagram, where ``there were some really basic boutiques
that had 300,000 followers,'' he said in the interview.

In 2013, Mr. Saghian opened an Instagram account and began posting
photos of his clothing on mannequins and customers. He noticed that some
of his stores' regular visitors were influencers he had seen on
Instagram, where they had hundreds of thousands of followers.

``I had rappers' girlfriends, female rappers, models,'' he said.

Image

Fashion Nova makes inexpensive clothes that look expensive.Credit...Rich
Fury/Getty Images for Fashion Nova

Image

Instagram influencers help drive sales for the company.Credit...Rich
Fury/Getty Images for Fashion Nova

Mr. Saghian started giving them free clothing, and they posted photos of
themselves draped in Fashion Nova garb. In turn, he reposted their
photos and tagged their handles.

``Everyone wants to be famous. Everyone wants to have more followers,''
Mr. Saghian said. ``By tagging them, the influencer would grow their
following.''

Gradually, the strategy brought Fashion Nova from the outskirts of the
internet into the mainstream. The brand earned mentions on hip-hop
tracks. In 2017, its sales grew by about 600 percent.

Cardi B, the Grammy-winning rap star, unveiled her first collection with
the brand in an Instagram video in November last year.

``I wanted to do something that is like, `Wow, what is that? Is that
Chanel? Is that YSL? Is that Gucci?' No,'' she said, adding an
expletive, ``it's Fashion Nova.''

All 82 styles in Cardi B's collection sold out hours after they became
available. She posted another video the same night, promising a full
restock ``in two or three weeks.'' (Cardi B's line is made in Los
Angeles, but the government has not found any of the clothes in
factories where workers have alleged they were paid less than the
minimum, Fashion Nova said.)

There were more searches for Fashion Nova last year than for Versace or
Gucci, according to Google's year in search data. It has 17 million
followers on Instagram, and at any given moment there are enough people
browsing clothes on its website to fill a basketball arena, Mr. Saghian
said.

To keep them interested, Fashion Nova produces more than a thousand new
styles every week, thanks in part to an army of local suppliers that can
respond instantly to the brand's requests.

``If there was a design concept that came to mind Sunday night, on a
Monday afternoon I would have a sample,'' he said.

Image

A Los Angeles factory where a supplier of clothing to Fashion Nova
outsources manufacturing.~~Credit...Jessica Pons for The New York Times

\hypertarget{the-best-possible-price}{%
\subsection{`The best possible price'}\label{the-best-possible-price}}

Many of the people vying for Mr. Saghian's business occupy glass-walled
storefronts jammed into the six frenetic blocks of the garment district
in downtown Los Angeles.

These are the companies that design clothing samples and sell them in
bulk to Fashion Nova and other retailers. Those businesses outsource the
job of making clothes to nearby factories that work as subcontractors.

In November, The Times visited seven companies that got Fashion Nova
clothes made in factories that underpaid workers, according to the Labor
Department investigations. Some spoke freely about their work with the
brand. Others refused to comment or talked on the condition of
anonymity, fearing that they might lose the company as a client if they
went on the record.

The five owners and employees who agreed to be interviewed said Fashion
Nova would always push to pay the lowest price possible for each
garment, and would demand a quick turnaround.

``They give me the best possible price they can give it to me, for that
will allow them to still break a profit,'' Mr. Saghian said.

The companies can negotiate with Fashion Nova, but their power is
limited. A dwindling number of retailers are still doing business in Los
Angeles, and a couple of big orders from Fashion Nova can keep a small
garment shop afloat for another year. So they look for subcontractors
who can sew clothes as quickly and cheaply as possible.

Image

A garment worker recalled her factory's receiving orders from Fashion
Nova for up to 5,000 pieces of clothing at a time.Credit...Jessica Pons
for The New York Times

Amante Clothing, which occupies a stuffy storefront filled with racks of
colorful samples, regularly works with Fashion Nova. The brand paid
Amante \$7.15 per top for a bulk order last year, according to a Labor
Department investigation conducted last December. Amante then went to a
sewing contractor called Karis Apparel, which made the tops.

Amante paid Karis \$2.20 to sew each garment, the Labor Department
found. Fashion Nova sold the top for \$17.99.

``We don't own the sewing contractor, so whatever the sewing contractor
does, that's his problem,'' said a designer at Amante, who declined to
be named for fear of losing her job. ``We don't know what they do to
give us the lowest price. We assume they're paying their employees the
minimum.''

Karis, the factory that worked with Amante, went out of business in
April. Another manufacturer ensnared in the investigations moved
production to Mexico this year.

But many more factories have evaded punishment.

\hypertarget{same-owners-different-names}{%
\subsection{Same owners, different
names}\label{same-owners-different-names}}

When Teresa Garcia started working at Sugar Sky, it was called Xela
Fashion. It was 2014, and Xela Fashion, state records show, was owned by
Demetria Sajche, a woman whom Ms. Garcia was told to call Angelina.

Several months later --- Ms. Garcia does not remember how many --- the
name on her checks had changed, though she worked in the same grungy
factory in the heart of downtown, a few blocks from a SoulCycle.

Image

Teresa Garcia next to her sewing machine at home. She sewed at a factory
that specialized in Fashion Nova clothes.Credit...Jessica Pons for The
New York Times

Now her employer was called Nena Fashion, a company that was founded by
Leslie Sajche, a relative of Ms. Garcia's boss, according to business
records filed with California's secretary of state. About a year after
that, the name changed again, to GYA Fashion.

In 2017, the factory moved to an industrial stretch of Olympic Boulevard
in East Los Angeles and began using a \emph{new} new name: Sugar Sky.
About a year later, Ms. Sajche stopped running the day-to-day operations
and handed the job over to Eric Alfredo Ajitaz Puac, whom workers knew
as her boyfriend.

Ms. Garcia said that she believed the point of all the name changes was
to avoid being shut down by federal or state officials. Several workers,
including Ms. Garcia, have filed claims against Xela, Nena, Gya and
Sugar Sky for back wages with California's labor commissioner, the state
agency that handles such disputes.

In her claim, which is active, Ms. Garcia included checks showing she
earned as little as \$225 for 65 hours of work in a week, the equivalent
of \$3.46 an hour. She remembers the factory's receiving orders from
Fashion Nova for up to 5,000 pieces of clothing at a time.

``They needed it so fast, they couldn't wait,'' Ms. Garcia said of the
brand. ``We would need to turn it around within a week.''

Weeks of trying to reach Mr. Puac and Ms. Sajche were unsuccessful. A
trip to Sugar Sky's last known location just before Thanksgiving found a
furniture store. Neighbors said the garment factory had packed up and
moved out two months earlier.

Fernando Axjup, who was listed as an owner of one iteration of the
factory, agreed to an interview. He was recently fired from the company
and had filed his own claim for back wages.

Image

``There was a lot of exploitation,'' Fernando Axjup said of a factory
where he worked.Credit...Jessica Pons for The New York Times

``They keep changing their names so they don't have to pay people,'' Mr.
Axjup said. ``There was a lot of exploitation.'' As a manager, he had
access to payroll data and said Ms. Garcia rarely earned the minimum
wage.

Mr. Axjup suggested that perhaps he had been fired for standing up for
workers like Ms. Garcia. Ms. Garcia said she doubted that, given that
Mr. Axjup was the one ordering her to hurry up.

He said he could never figure out why Fashion Nova did not visit the
factory floor to check on how its clothes were being made for such low
prices.

``Supposedly, the brand should supervise the people who give them work,
to find out whether they are being paid well,'' Mr. Axjup said. ``But
they never do. They never came to see.''

Kitty Bennett contributed research.

Advertisement

\protect\hyperlink{after-bottom}{Continue reading the main story}

\hypertarget{site-index}{%
\subsection{Site Index}\label{site-index}}

\hypertarget{site-information-navigation}{%
\subsection{Site Information
Navigation}\label{site-information-navigation}}

\begin{itemize}
\tightlist
\item
  \href{https://help.nytimes3xbfgragh.onion/hc/en-us/articles/115014792127-Copyright-notice}{©~2020~The
  New York Times Company}
\end{itemize}

\begin{itemize}
\tightlist
\item
  \href{https://www.nytco.com/}{NYTCo}
\item
  \href{https://help.nytimes3xbfgragh.onion/hc/en-us/articles/115015385887-Contact-Us}{Contact
  Us}
\item
  \href{https://www.nytco.com/careers/}{Work with us}
\item
  \href{https://nytmediakit.com/}{Advertise}
\item
  \href{http://www.tbrandstudio.com/}{T Brand Studio}
\item
  \href{https://www.nytimes3xbfgragh.onion/privacy/cookie-policy\#how-do-i-manage-trackers}{Your
  Ad Choices}
\item
  \href{https://www.nytimes3xbfgragh.onion/privacy}{Privacy}
\item
  \href{https://help.nytimes3xbfgragh.onion/hc/en-us/articles/115014893428-Terms-of-service}{Terms
  of Service}
\item
  \href{https://help.nytimes3xbfgragh.onion/hc/en-us/articles/115014893968-Terms-of-sale}{Terms
  of Sale}
\item
  \href{https://spiderbites.nytimes3xbfgragh.onion}{Site Map}
\item
  \href{https://help.nytimes3xbfgragh.onion/hc/en-us}{Help}
\item
  \href{https://www.nytimes3xbfgragh.onion/subscription?campaignId=37WXW}{Subscriptions}
\end{itemize}
