Sections

SEARCH

\protect\hyperlink{site-content}{Skip to
content}\protect\hyperlink{site-index}{Skip to site index}

\href{https://www.nytimes3xbfgragh.onion/section/business}{Business}

\href{https://myaccount.nytimes3xbfgragh.onion/auth/login?response_type=cookie\&client_id=vi}{}

\href{https://www.nytimes3xbfgragh.onion/section/todayspaper}{Today's
Paper}

\href{/section/business}{Business}\textbar{}Trump's Trade Deals Raise,
Rather Than Remove, Economic Barriers

\href{https://nyti.ms/2YYXX3f}{https://nyti.ms/2YYXX3f}

\begin{itemize}
\item
\item
\item
\item
\item
\end{itemize}

Advertisement

\protect\hyperlink{after-top}{Continue reading the main story}

Supported by

\protect\hyperlink{after-sponsor}{Continue reading the main story}

\hypertarget{trumps-trade-deals-raise-rather-than-remove-economic-barriers}{%
\section{Trump's Trade Deals Raise, Rather Than Remove, Economic
Barriers}\label{trumps-trade-deals-raise-rather-than-remove-economic-barriers}}

The president's China and North American trade pacts reverse a trend of
opening markets that was decades in the making.

\includegraphics{https://static01.graylady3jvrrxbe.onion/images/2019/12/17/business/17dc-chinatrade-01/merlin_146134512_2daed757-0759-4079-8e90-f2a70fbbf02e-articleLarge.jpg?quality=75\&auto=webp\&disable=upscale}

\href{https://www.nytimes3xbfgragh.onion/by/ana-swanson}{\includegraphics{https://static01.graylady3jvrrxbe.onion/images/2018/12/10/multimedia/author-ana-swanson/author-ana-swanson-thumbLarge.png}}

By \href{https://www.nytimes3xbfgragh.onion/by/ana-swanson}{Ana Swanson}

\begin{itemize}
\item
  Dec. 17, 2019
\item
  \begin{itemize}
  \item
  \item
  \item
  \item
  \item
  \end{itemize}
\end{itemize}

WASHINGTON --- For years, America's trade agreements have tried to break
down economic barriers between nations by removing tariffs and other
impediments to cross-border commerce. President Trump's trade deals have
turned that approach on its head.

Mr. Trump's new trade deal with China promises to lower some of the
walls Beijing has erected for foreign companies --- including opening
its financial markets, streamlining imports of American agriculture and
offering greater protection for intellectual property.

But it leaves in place tariffs on the bulk of Chinese imports --- more
than \$360 billion worth of goods. And it requires voluminous Chinese
purchases of American products
---\href{https://www.nytimes3xbfgragh.onion/2019/12/13/business/economy/china-trade-deal.html}{\$200
billion of additional sales} over the next two years, according to the
Trump administration --- a significant shift that experts say moves
trade policy away from promoting free markets and back toward an earlier
era of managed trade.

Mr. Trump's newly revised North American trade deal similarly contains
provisions that open up markets for dairy, digital services and other
industries. But its most transformative changes are to tighten the
\href{https://www.nytimes3xbfgragh.onion/2019/12/11/business/nafta-usmca-auto-jobs.html}{rules
for North American automotive manufacturing} to try to spur more
production within the continent, a move
\href{https://www.nytimes3xbfgragh.onion/2019/12/01/us/politics/trump-trade-deal-usmca.html}{some
Republican lawmakers say will weigh on trade}.

The agreements are the product of Mr. Trump's transactional trade
approach, one that aims to wield America's economic power to force other
nations to buy more American products. His ``America First'' philosophy
looks upon global supply chains and the free trade deals they were built
on with suspicion, and seeks to force sprawling multinational companies
to move operations to the United States, in an effort to bolster
American growth and lower the trade deficit.

His administration also sees little use for the type of multilateral
organizations that have tried to lift economic growth around the world
by promoting free trade. Last week, the administration effectively
\href{https://www.nytimes3xbfgragh.onion/2019/12/08/business/trump-trade-war-wto.html}{crippled
the World Trade Organization's ability} to resolve trade disputes after
a sustained campaign against a critical part of the body.

Mr. Trump promoted his approach in a round table with governors at the
White House on Monday, saying that past trade rules set by
``globalists'' had allowed factories and wealth to flow out of the
United States.

``I would watch as they close plants, everybody gets fired. They move to
Mexico or some other place, including China,'' the president said. ``And
some people are happy. But no, not me.''

He praised his China deal for increasing sales of American products and
said
\href{https://www.nytimes3xbfgragh.onion/2019/12/10/us/politics/usmca-trade-deal.html}{his
revised North American trade deal} had built strong barriers to keep
companies from leaving the United States.

``It's very hard to move,'' the president said. ``Economically, it makes
it really prohibitive to get out. And it was very important to me.''

\includegraphics{https://static01.graylady3jvrrxbe.onion/images/2019/12/17/business/17dc-chinatrade-02/merlin_155141265_a4dae2b8-881f-4b69-ba70-26ae07329aa8-articleLarge.jpg?quality=75\&auto=webp\&disable=upscale}

Doug Irwin, a trade historian at Dartmouth College, said the pacts were
a substantial departure from those enacted under Mr. Trump's recent
predecessors --- both Republicans and Democrats --- who worked to lower
global tariffs and build an international system that enshrined freer
trade. ``Most trade agreements that we've seen in history are agreements
to liberalize markets, to get government out of trade in some sense,''
he said.

But Mr. Trump and his advisers display little ideological commitment to
free trade, which has animated the Republican Party for decades. They
argue that political paeans to free trade have largely been cover for
multinational companies --- and their lobbyists --- to outsource
production, with devastating results for American workers.

In an interview with the Fox Business Network on Tuesday, Robert
Lighthizer, Mr. Trump's top trade negotiator, acknowledged that the
agreements were not likely to please those who prioritized free markets.

``I understand the people that believe in just protecting investors and
pure market efficiency,'' Mr. Lighthizer said. ``They're not going to be
happy because we are making it more expensive to operate in some other
areas and less expensive in the United States.''

``The president's objective is to help manufacturing workers in this
country. It's to help farmers in this country,'' Mr. Lighthizer added.
``Global efficiency is a nice objective, but he always says he got
elected president of the United States, not president of the world.''

Mr. Trump's aggressive approach to reworking the global trading system
has been praised by some parts of industry as an attempt to fix a
situation they say has been disastrous for American workers.

``Trump and team have what appears to be a strong deal,'' Daniel
DiMicco, a former steel industry executive who leads the Coalition for a
Prosperous America, said of the China trade pact. ``The cost of
maintaining the status quo is infinitely greater.''

Yet many economists and trade experts fear the approach could backfire
on the United States, by degrading the international trading system and
raising the cost of manufacturing, resulting in lower productivity and
economic growth.

In an analysis published Tuesday, Mary E. Lovely and Jeffrey J. Schott,
two economists at the Peterson Institute for International Economics,
\href{https://www.piie.com/blogs/trade-and-investment-policy-watch/usmca-new-modestly-improved-still-costly}{projected
that the provisions in the United States-Mexico-Canada Agreement} would
hurt American industry, by driving up the cost of making cars and
weighing on growth.

Analysts at Fitch Ratings said Tuesday that the China deal had raised
their estimates for global growth, but done less to lower trade barriers
than anticipated. The trade truce leaves the effective American tariff
rate on Chinese products at 16 percent, below the 25 percent level that
Mr. Trump had threatened to raise it to, but up from roughly 3 percent
before the trade war, they said.

The North American and China pacts, which together cover countries
responsible for more than half of America's trade, are the first
translation of Mr. Trump's trade ideals into policy.

But they also bear the imprint of Mr. Lighthizer, who has a long history
of favoring a managed trade approach. As a trade negotiator for the
Reagan administration in the early 1980s, Mr. Lighthizer made a mark
\href{https://www.nytimes3xbfgragh.onion/1984/09/30/business/the-steel-trade-negotiations-the-experts-who-will-forge-the-new-quotas.html}{negotiating
agreements with Japan} to limit the amount of products it exported to
the United States. The World Trade Organization later banned agreements
that seek to restrain a country's exports.

Image

Robert Lighthizer, center, the United States trade representative, with
Vice Premier Liu He of China, right, and Treasury Secretary Steven
Mnuchin. Mr.~Lighthizer has a long history of favoring a managed trade
approach.Credit...Pool photo by Ng Han Guan

Mr. Lighthizer left government in 1985, but the Reagan administration
continued with a managed trade approach, pushing Japan and South Korea
to agree to import a certain amount of products. The Clinton
administration also considered the tactic, but faced criticism that it
would encourage state interventionism as the United States was pushing
Japan to adopt a freer market, said Mr. Irwin, the historian.

That history has direct parallels to China, where American officials
have been urging the government for decades to reduce its role in the
economy. Trump administration officials, including Mr. Lighthizer, have
also criticized Beijing for using preferential policies, subsidies and
central planning to
\href{https://www.nytimes3xbfgragh.onion/2019/05/12/business/china-trump-trade-subsidies.html}{give
its businesses an advantage} over American ones.

But the trade deal announced Friday appears to
\href{https://www.nytimes3xbfgragh.onion/2019/12/13/business/economy/china-trade-deal.html}{make
little progress on those issues}. Instead, its largest feature appears
to be purchases that are likely to be beneficial for American businesses
but may wind up further
\href{https://www.nytimes3xbfgragh.onion/2019/04/01/us/politics/us-china-trade-trump.html}{strengthening
the hand of the Chinese state}.

Some of the purchases, which
\href{https://www.nytimes3xbfgragh.onion/2019/12/15/business/economy/us-china-trade-deal.html}{Mr.
Lighthizer has projected} will roughly double American exports to China
by 2021, are expected to happen naturally, as China lowers trade
barriers to American goods. But others, including in agriculture, energy
and aviation, would most likely be done by fiat, through China's
state-controlled entities.

Critics say this approach could end up giving the Chinese state even
greater discretion over certain markets. Some agricultural producers
have expressed concern that the trade deal's firm targets could undercut
their ability to negotiate with Chinese customers.

Nicholas R. Lardy, a China expert at the Peterson Institute for
International Economics, said the purchasing agreement ``could go
against longer-term U.S. goals'' to encourage China to adopt a
market-oriented system, ``but we have to see what the exact language
is.''

``If it's an ironclad commitment, then I think it's a move in the wrong
direction,'' he said.

Mr. Lighthizer and some supporters say the targets are an effective way
to deal with a country like China that does not play by market rules.

Clyde Prestowitz, the president of the Economic Strategy Institute and a
former Reagan official, said purchasing commitments are ``anathema to
dyed-in-the-wool free traders and contrary to mathematical free trade
models.'' However, he said, they offer ``a buffer between truly open,
competitive free markets and markets that are wholly or partially
government managed.''

When it comes to China, he said, ``to imagine that foreign players can
just move in and compete as they do in the U.S. or the E.U. is to be
dreaming.''

Advertisement

\protect\hyperlink{after-bottom}{Continue reading the main story}

\hypertarget{site-index}{%
\subsection{Site Index}\label{site-index}}

\hypertarget{site-information-navigation}{%
\subsection{Site Information
Navigation}\label{site-information-navigation}}

\begin{itemize}
\tightlist
\item
  \href{https://help.nytimes3xbfgragh.onion/hc/en-us/articles/115014792127-Copyright-notice}{©~2020~The
  New York Times Company}
\end{itemize}

\begin{itemize}
\tightlist
\item
  \href{https://www.nytco.com/}{NYTCo}
\item
  \href{https://help.nytimes3xbfgragh.onion/hc/en-us/articles/115015385887-Contact-Us}{Contact
  Us}
\item
  \href{https://www.nytco.com/careers/}{Work with us}
\item
  \href{https://nytmediakit.com/}{Advertise}
\item
  \href{http://www.tbrandstudio.com/}{T Brand Studio}
\item
  \href{https://www.nytimes3xbfgragh.onion/privacy/cookie-policy\#how-do-i-manage-trackers}{Your
  Ad Choices}
\item
  \href{https://www.nytimes3xbfgragh.onion/privacy}{Privacy}
\item
  \href{https://help.nytimes3xbfgragh.onion/hc/en-us/articles/115014893428-Terms-of-service}{Terms
  of Service}
\item
  \href{https://help.nytimes3xbfgragh.onion/hc/en-us/articles/115014893968-Terms-of-sale}{Terms
  of Sale}
\item
  \href{https://spiderbites.nytimes3xbfgragh.onion}{Site Map}
\item
  \href{https://help.nytimes3xbfgragh.onion/hc/en-us}{Help}
\item
  \href{https://www.nytimes3xbfgragh.onion/subscription?campaignId=37WXW}{Subscriptions}
\end{itemize}
