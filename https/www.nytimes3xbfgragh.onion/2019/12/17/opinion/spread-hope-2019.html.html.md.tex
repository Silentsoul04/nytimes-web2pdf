Sections

SEARCH

\protect\hyperlink{site-content}{Skip to
content}\protect\hyperlink{site-index}{Skip to site index}

\href{https://myaccount.nytimes3xbfgragh.onion/auth/login?response_type=cookie\&client_id=vi}{}

\href{https://www.nytimes3xbfgragh.onion/section/todayspaper}{Today's
Paper}

\href{/section/opinion}{Opinion}\textbar{}Five Who Spread Hope in 2019

\url{https://nyti.ms/2Eu9qP8}

\begin{itemize}
\item
\item
\item
\item
\item
\item
\end{itemize}

Advertisement

\protect\hyperlink{after-top}{Continue reading the main story}

\href{/section/opinion}{Opinion}

Supported by

\protect\hyperlink{after-sponsor}{Continue reading the main story}

Fixes

\hypertarget{five-who-spread-hope-in-2019}{%
\section{Five Who Spread Hope in
2019}\label{five-who-spread-hope-in-2019}}

In a year of many dispiriting headlines, Fixes still found the better
angels of human nature at work.

\includegraphics{https://static01.graylady3jvrrxbe.onion/images/2019/02/13/opinion/tina-rosenberg/tina-rosenberg-thumbLarge-v2.png}

By Tina Rosenberg

Ms. Rosenberg is a co-founder of the
\href{http://solutionsjournalism.org}{Solutions Journalism Networ}k,
which supports rigorous reporting about responses to social problems.

\begin{itemize}
\item
  Dec. 17, 2019
\item
  \begin{itemize}
  \item
  \item
  \item
  \item
  \item
  \item
  \end{itemize}
\end{itemize}

O.K., so Time magazine has Greta Thunberg. But many other individuals
also changed the world for the better in 2019. Here, for a second year,
is a list of five whose contributions Fixes wrote about.

\includegraphics{https://static01.graylady3jvrrxbe.onion/images/2019/12/17/opinion/17Fixes1/merlin_148870704_bf85554a-5701-4e7b-b19f-fcc2f9db69bd-articleLarge.jpg?quality=75\&auto=webp\&disable=upscale}

\textbf{Scott O'Neill fights tropical disease.}

\emph{``People who understand dengue and live in transmission areas are
horrified and scared.''}

There's a new weapon in the fight against mosquito-borne diseases.

Before 1970, only nine countries had experienced severe epidemics of
dengue fever. Now, the disease is endemic in 100 countries, infects 400
million people a year and is intensifying rapidly.

Like Zika and chikungunya, dengue is spread by the bite of the Aedes
aegypti mosquito, and no workable vaccine or cure has been found.

The normal public health response to mosquitoes is attack: spray
pesticide, eliminate breeding grounds and help people ward off their
bites. But these strategies have failed to control dengue. The world is
desperate for something new.

Scott O'Neill leads a team that is doing just the opposite --- adding
\href{https://www.nytimes3xbfgragh.onion/2019/01/08/opinion/mosquito-fighting-tropical-disease.html}{millions
of mosquitoes} to areas affected by disease.

Professor O'Neill directs the \href{http://www.worldmosquito.org/}{World
Mosquito Program}, which is based at Monash University in Melbourne,
Australia. The mosquitoes the program releases are infected with
Wolbachia ** bacteria, which block their ability to transmit disease.
Wolbachia occurs naturally in most insect species and is harmless to
vertebrates and humans. When enough Wolbachia-infected mosquitoes are
released, they take over the whole population.

The World Mosquito Program tests Wolbachia in 12 countries in Asia,
Latin America and the Western Pacific. While it started releasing
mosquitoes in 2011, large-scale trials are fairly new --- and the
evidence released this year is promising. The Wolbachia initiative has
\href{https://gatesopenresearch.org/articles/3-1547}{nearly eliminated
local transmission of dengue} in the parts of Australia where it has
been tried. In Yogyakarta, Indonesia, Wolbachia zones had
\href{https://www.nature.com/articles/d41586-019-03660-8}{76 percent
fewer cases} of dengue than other areas. Wolbachia has also led to
reductions in disease in Brazil and Vietnam.

Image

Kimberly McGrath coordinates foster care services at the Citrus Health
Network in Hialeah, Fla.Credit...Maria Alejandra Cardona for The New
York Times

\textbf{Kimberly McGrath heals trafficked children}

\emph{``Now we know they really are just extremely traumatized youth.''}

What happens to a child who is exploited commercially for sex?

Kimberly McGrath is changing the answer to that question. Historically,
trafficked children have been arrested for solicitation and sent to
juvenile court.

Today, all children sold for sex are, by definition, trafficked. Yet
some are still arrested. Most are sent to group homes. ``The core
understanding was that these were defiant, rebellious youth who would
rebel in a family,'' Dr. McGrath said.

In 2013, Florida officials asked Dr. McGrath, who coordinates foster
care services at the Citrus Health Network in South Florida, to come up
with a different response. She started from the premise that these
children were not defiant criminals. A vast majority had been abused,
which made them more susceptible to the manipulations of traffickers.
And they had never gotten help to recover from that abuse.

Dr. McGrath and her colleagues looked at what had worked for other
traumatized children and adapted it to trafficked children. They
educated not just therapists and social workers, but also
\href{https://www.nytimes3xbfgragh.onion/2019/06/19/opinion/foster-child-trafficking.html}{foster
parents}.

It has been difficult to recruit foster families, but Dr. McGrath's
program has done it --- finding courageous and dedicated families who
receive special training and help from psychologists and social workers.
This therapeutic foster care costs less than group homes, and the
children do better in every way. ``When foster parents are equipped and
prepared to deal with their special needs, children thrive in
family-based environments,'' she said. ``They really are just
traumatized kids.''

Image

Dr. Dixon Chibanda turned benches into destinations for
therapy.Credit...Markus Schreiber/Associated Press

\textbf{Dr. Dixon Chibanda transforms global mental health care}

``\emph{I started to realize that psychiatry in an institution is not
the way to go. We have to take it to the community.''}

Depression occurs everywhere. By some measures, it's the world's most
debilitating disease.

But treatment is not everywhere. Even in New York City, less than 40
percent of people with depression get treatment. In poor countries, it's
closer to zero percent.

So what can be done in places with no public mental health care and only
a tiny number of mental health professionals?

As with medical care, the answer is training nonprofessionals.

Every health clinic in Harare, Zimbabwe, has a
``\href{https://www.friendshipbenchzimbabwe.org/}{friendship bench}'' in
its yard. It's an ordinary wooden bench. Seated on it is a community
health worker with a few weeks' training in problem-solving therapy.

Patients go to the bench, talk to the health worker about their problems
and come up with possible solutions. They go home and try them, and
return.

The friendship bench was invented in 2006 by a psychiatrist, Dixon
Chibanda, after a patient committed suicide. He had asked her to come
see him at Harare Central Hospital, but she lived in another city and
didn't have bus fare.

Dr. Chibanda decided to bring treatment for depression to Harare's
health clinics. At first he wanted to train nurses and put offices
inside the buildings, but the nurses had not enough time and clinics had
not enough space. But what seemed like a setback is what has allowed the
program to spread.

Now, there's a bench in the yard of every government-run health clinic
in Harare, and the practice is spreading throughout Zimbabwe and to
other African countries. In a different form, the strategy has also
reached\href{https://www.nytimes3xbfgragh.onion/2019/07/22/opinion/depressed-heres-a-bench-talk-to-me.html}{New
York}.

Research shows that friendship benches
\href{https://jamanetwork.com/journals/jama/fullarticle/2594719}{are
effective at treating depression}. And what makes them even more
valuable is that they are cheap and piggyback on government services.
They provide a treatment that works --- and that could reach anyone.

Image

Dr. Rebekah Gee, Louisiana's health secretary.Credit...Tom Williams/CQ
Roll Call, via Associated Press

\textbf{Dr. Rebekah Gee makes medicines affordable}

\emph{``Why couldn't we change health care in this country?''}

Louisiana is doing two things no other state is doing about hepatitis C,
which kills more Americans than all other infectious diseases combined.

One is that the state is suddenly treating more people.

Hep C is curable --- but the drugs are astronomically expensive. Even
the cheapest generic version in the United States costs \$24,000 for a
course of treatment. (In India, the same drug
\href{https://www.generichepatitiscdrugs.com/}{is \$550}.) Because of
the price, state Medicaid programs ration the drugs. In 2018, Louisiana
treated 1,200 people.

Contrast that with the period between mid-July of this year and late
November, in which Louisiana
\href{http://www.ldh.la.gov/index.cfm/newsroom/detail/5357}{treated
2,290 people}.

Louisiana could do that because of the second innovation: The drugs were
made a lot less expensive. In July, the state began buying hep C
medicines in a new way. Just as you pay
\href{https://www.nytimes3xbfgragh.onion/2019/03/05/opinion/can-netflix-show-americans-how-to-cut-the-cost-of-drugs.html}{Netflix
a flat fee} for all you want to watch, Louisiana now pays Asegua
Therapeutics \$58 million per year for all the hep C medicine the state
can use. That still means huge profits for Asegua, since the cost of
making each new pill is negligible. If Louisiana meets its goal of
\href{https://www.nola.com/news/healthcare_hospitals/article_ddf84210-a8d4-11e9-ae1b-b7037c312d9a.html}{treating
10,000} people in its first year of operation, from July 2019 to July
2020, that will cut the price per person to about \$6,000.

Dr. Rebekah Gee, Louisiana's secretary of health, adopted the scheme
from Australia, where it has allowed Australia to treat
\href{https://www.nejm.org/doi/full/10.1056/NEJMp1813728}{seven times as
many patients} for the same money. Louisiana is the first state in
America to do the same. The State of Washington is about to start as
well. Other states are likely to follow.

Everyone talks about bringing down drug prices. But the power of the
pharmaceutical industry has staved off reforms --- except this one.

Image

Credit...Illustration by Jeffrey Henson Scales; photographs by Marcin
Jastrzebski and Digiphoto/iStock, via Getty Images

\textbf{Phil Keisling deepens democracy}

\emph{``For millions of citizens, especially those with uncertain work
schedules, family obligations and other daily demands, the traditional
polling place has now become the most powerful voter suppression tool of
all.''}

There's a lot of attention, and rightly so, paid to Republican efforts
to suppress voting. But there's also a movement in both parties to
\emph{expand} voting. It abandons the traditional polling booth in favor
of
\href{https://www.nytimes3xbfgragh.onion/2019/06/11/opinion/the-end-of-the-polling-booth.html}{voting
at home}. It's one of the most effective ways to increase turnout ---
possibly the
\href{https://www.voteathome.org/project/proven-track-record/}{best
way}.

Increasingly, other states are following the path first set by Oregon,
which mails every voter a ballot. Voters fill it out at their leisure
and mail it in or drop it off at a ballot center.

In next year's elections, all voters in Oregon, Colorado, Washington,
Utah and Hawaii will vote at home. California will soon follow. Large
parts of North Dakota and Nebraska vote at home. In last year's
midterms, 69 percent of all votes in the West were cast by voters who
received ballots in the mail.

Phil Keisling was Oregon's secretary of state, in charge of elections,
when Oregon began home voting in 1998. Now he leads the
\href{http://www.voteathome.org}{Vote at Home Institute}.

The institute asserts that it saves taxpayers money (some election
officials disagree). It argues that because the approach uses paper
ballots, it's secure against hacking. As for whom it helps, proponents
in both parties claim it has bipartisan benefits.

Home voting probably doesn't affect turnout in big elections. But it
does in local elections, races at the end of the ballot, ballot
propositions and judicial elections. Turnout for these elections can be
in the single digits. For those races, a ballot on the kitchen table
turns many more people into voters. So the answer to whom it helps is:
democracy.

Tina Rosenberg won a Pulitzer Prize for her book
``\href{http://www.randomhouse.com/catalog/display.pperl?isbn=9780679744993}{The
Haunted Land: Facing Europe's Ghosts After Communism}.'' She is a former
editorial writer for The Times and the author, most recently, of
``\href{http://books.wwnorton.com/books/Join-the-Club}{Join the Club:
How Peer Pressure Can Transform the World}'' and the World War II spy
story
e-book\href{https://www.goodreads.com/book/show/16124470-d-for-deception}{``D
for Deception.''}

\emph{To receive email alerts for Fixes columns, sign up}
\href{http://eepurl.com/ABIxL}{\emph{here.}}

\emph{The Times is committed to publishing}
\href{https://www.nytimes3xbfgragh.onion/2019/01/31/opinion/letters/letters-to-editor-new-york-times-women.html}{\emph{a
diversity of letters}} \emph{to the editor. We'd like to hear what you
think about this or any of our articles. Here are some}
\href{https://help.nytimes3xbfgragh.onion/hc/en-us/articles/115014925288-How-to-submit-a-letter-to-the-editor}{\emph{tips}}\emph{.
And here's our email:}
\href{mailto:letters@NYTimes.com}{\emph{letters@NYTimes.com}}\emph{.}

\emph{Follow The New York Times Opinion section on}
\href{https://www.facebookcorewwwi.onion/nytopinion}{\emph{Facebook}}\emph{,}
\href{http://twitter.com/NYTOpinion}{\emph{Twitter (@NYTopinion)}}
\emph{and}
\href{https://www.instagram.com/nytopinion/}{\emph{Instagram}}\emph{.}

Advertisement

\protect\hyperlink{after-bottom}{Continue reading the main story}

\hypertarget{site-index}{%
\subsection{Site Index}\label{site-index}}

\hypertarget{site-information-navigation}{%
\subsection{Site Information
Navigation}\label{site-information-navigation}}

\begin{itemize}
\tightlist
\item
  \href{https://help.nytimes3xbfgragh.onion/hc/en-us/articles/115014792127-Copyright-notice}{©~2020~The
  New York Times Company}
\end{itemize}

\begin{itemize}
\tightlist
\item
  \href{https://www.nytco.com/}{NYTCo}
\item
  \href{https://help.nytimes3xbfgragh.onion/hc/en-us/articles/115015385887-Contact-Us}{Contact
  Us}
\item
  \href{https://www.nytco.com/careers/}{Work with us}
\item
  \href{https://nytmediakit.com/}{Advertise}
\item
  \href{http://www.tbrandstudio.com/}{T Brand Studio}
\item
  \href{https://www.nytimes3xbfgragh.onion/privacy/cookie-policy\#how-do-i-manage-trackers}{Your
  Ad Choices}
\item
  \href{https://www.nytimes3xbfgragh.onion/privacy}{Privacy}
\item
  \href{https://help.nytimes3xbfgragh.onion/hc/en-us/articles/115014893428-Terms-of-service}{Terms
  of Service}
\item
  \href{https://help.nytimes3xbfgragh.onion/hc/en-us/articles/115014893968-Terms-of-sale}{Terms
  of Sale}
\item
  \href{https://spiderbites.nytimes3xbfgragh.onion}{Site Map}
\item
  \href{https://help.nytimes3xbfgragh.onion/hc/en-us}{Help}
\item
  \href{https://www.nytimes3xbfgragh.onion/subscription?campaignId=37WXW}{Subscriptions}
\end{itemize}
