Sections

SEARCH

\protect\hyperlink{site-content}{Skip to
content}\protect\hyperlink{site-index}{Skip to site index}

\href{https://www.nytimes3xbfgragh.onion/section/style}{Style}

\href{https://myaccount.nytimes3xbfgragh.onion/auth/login?response_type=cookie\&client_id=vi}{}

\href{https://www.nytimes3xbfgragh.onion/section/todayspaper}{Today's
Paper}

\href{/section/style}{Style}\textbar{}What Do Gen Z Shoppers Want? A
Cute, Cheap Outfit That Looks Great on Instagram

\href{https://nyti.ms/38LhsB0}{https://nyti.ms/38LhsB0}

\begin{itemize}
\item
\item
\item
\item
\item
\item
\end{itemize}

Advertisement

\protect\hyperlink{after-top}{Continue reading the main story}

Supported by

\protect\hyperlink{after-sponsor}{Continue reading the main story}

\hypertarget{what-do-gen-z-shoppers-want-a-cute-cheap-outfit-that-looks-great-on-instagram}{%
\section{What Do Gen Z Shoppers Want? A Cute, Cheap Outfit That Looks
Great on
Instagram}\label{what-do-gen-z-shoppers-want-a-cute-cheap-outfit-that-looks-great-on-instagram}}

Three young women shoppers in three countries talk about what they buy,
where they buy it and why.

\includegraphics{https://static01.graylady3jvrrxbe.onion/images/2019/12/17/fashion/17SaturdayNightGirls-top/merlin_162314289_be6d76eb-87d8-40b2-bebb-aff92b5d168d-articleLarge.jpg?quality=75\&auto=webp\&disable=upscale}

By
\href{https://www.nytimes3xbfgragh.onion/by/elizabeth-paton}{Elizabeth
Paton},
\href{https://www.nytimes3xbfgragh.onion/by/taylor-lorenz}{Taylor
Lorenz} and
\href{https://www.nytimes3xbfgragh.onion/by/isabella-kwai}{Isabella
Kwai}

\begin{itemize}
\item
  Published Dec. 17, 2019Updated Dec. 26, 2019
\item
  \begin{itemize}
  \item
  \item
  \item
  \item
  \item
  \item
  \end{itemize}
\end{itemize}

For every
\href{https://www.nytimes3xbfgragh.onion/2019/09/18/climate/greta-thunberg.html}{Greta
Thunberg} and school-skipping climate change protester, there is another
member of
\href{https://www.nytimes3xbfgragh.onion/2019/03/28/us/gen-z-in-their-words.html}{Generation
Z} buying inexpensive clothes on a smartphone.

Their purchasing choices --- fueled by
\href{https://www.nytimes3xbfgragh.onion/2019/07/16/technology/vidcon-social-media-influencers.html}{influencer
culture} and catered to by a new wave of ultra-fast-fashion retailers
such as
\href{https://www.nytimes3xbfgragh.onion/2019/12/16/business/fashion-nova-underpaid-workers.html}{Fashion
Nova}, Pretty Little Thing and Missguided (responsible for a
\href{https://www.theguardian.com/fashion/2019/jun/22/one-pound-bikini-missguided-fast-fashion-leaves-high-street-behind}{£1
bikini that sold out in Britain}) --- are as much about how an outfit
will look on social media as in the real world.

Three Gen Z shoppers in America, Australia and Britain invited us into
their homes to talk about what they buy, and why. All of them work after
school or save money to pay for their own purchases.

\includegraphics{https://static01.graylady3jvrrxbe.onion/images/2019/12/17/fashion/17SaturdayNightGirls-mia-1/17SaturdayNightGirls-mia-1-articleLarge.jpg?quality=75\&auto=webp\&disable=upscale}

\hypertarget{i-browse-every-single-day}{%
\subsection{`I Browse Every Single
Day'}\label{i-browse-every-single-day}}

\href{https://www.nytimes3xbfgragh.onion/by/elizabeth-paton}{\includegraphics{https://static01.graylady3jvrrxbe.onion/images/2019/12/05/reader-center/author-elizabeth-paton/author-elizabeth-paton-thumbLarge.png}}

Interview by
\href{https://www.nytimes3xbfgragh.onion/by/elizabeth-paton}{Elizabeth
Paton}

Mia Grantham is a 16-year-old British high school student studying for
her A-levels. She lives with her father and her younger sister, Annie,
in Wilmslow, England, a town outside Manchester. Her bedroom is small
but immaculately kept, with a bulb-lit dressing table and a pillow
shaped like a speech bubble reading ``You've Got This'' on her bed.

Mia's interest in clothes ramped up about 18 months ago, when she
started getting an allowance and attracting followers on her social
media accounts. She has more than 1,500 followers on Instagram, gets
around 500 views per story on Snapchat and spends three hours per day on
her iPhone XR (about five hours on weekends and during vacation).

Her favorite going-out look is a red dress. She owns 14 of them.

\textbf{How often do you shop?}

I browse every single day --- at least once --- on the Pretty Little
Thing phone app. It's my favorite, and I don't look anywhere else,
except if I see something on an Instagram influencer I like. My current
favorite is Molly-Mae Hague, a star from the 2019 series of
``\href{https://www.nytimes3xbfgragh.onion/2018/07/26/opinion/sunday/love-island-sex-britain.html}{Love
Island.''} She recently created an exclusive clothing range for PLT,
which makes me like the brand even more. Normally I look at shopping
apps at the end of the day before bed for about 10 to 15 minutes. But if
there is an event coming up that I want a new outfit for, then I could
browse for more than an hour. I don't really go to bricks-and-mortar
stores. If I do, I go to Primark. Sometimes
\href{https://www.nytimes3xbfgragh.onion/2018/03/27/business/hm-clothes-stock-sales.html}{H\&M}.
Maybe once a month, probably less.

\textbf{What kind of an event needs a new look?}

It could literally just be a meal. Or a house party, or a friend's
birthday. It could also be school, where we have a dress code but not a
uniform.

\textbf{Why is Pretty Little Thing your favorite fashion brand?}

I pay £8.99 as part of a yearly subscription, which gives me unlimited
next-day delivery on anything I buy. I know all the delivery people
really well now --- they always know when I have plans on a Friday or
Saturday night. I don't buy from places like Boohoo.com or Missguided as
I'd have to pay for delivery, which would be a waste of money. I buy
something at least once a week, and my basket value can be anywhere from
£5.99 up. Once it was £230. Last week I bought 11 items and sent back
three. Seventy percent of the time I send some ordered items back.

\textbf{How many pieces of clothing do you think you've bought in 2019?}

Eighty? One hundred? Those are pieces I've kept.

Image

Pretty Little Thing is Mia's favorite shopping app; she says she looks
at it every single day.Credit...Rosie Matheson for The New York Times

\textbf{What is your favorite piece that you've bought, and how many
times have you worn it?}

The ones I probably wear the most are gray leggings that cost £2.50. For
going out, I bought a silky red dress with a cutout for a house party.
It cost £12.50 from the PLT Shape collection, which is for people like
me who have an hourglass figure. I've worn it out three times, which is
a lot for me. Normally I just wear a dress once.

\textbf{Why only once?}

Because I'll normally be in photos when I'm wearing it that are then
posted on social media. I wouldn't really want someone seeing me in a
dress more than once. People might think I didn't have style if I wore
the same thing over and over. Style is about changing for whatever the
situation you are in and for different events.

\textbf{When do clothes become old for you?}

Well, things like leggings that you just wear in private around the
house you can keep for years. Dresses, when you've worn them: twice.

\textbf{Is price important?}

Of course. If I'm only going to wear something once or twice, I'm going
to want to buy the cheapest possible.

\textbf{What else do you look for?}

Social media is a big consideration. I'm on Snapchat and Instagram, and
occasionally Facebook. I take selfies for social media every single time
I go out, first in my bedroom and post them online, and then always with
friends or my boyfriend, Will, when I'm at the party. More people will
see an outfit online than they probably will in real life. I'm on
Snapchat the most because of its messenger function, then Instagram,
where I have both a public and a private account and spend an hour per
day.

For IRL, if I see an item I like, normally I'll
\href{https://www.nytimes3xbfgragh.onion/2019/07/09/style/internet-girl-depop.html}{search
for it on Depop} before I buy it so I can see what a real person rather
than a model looks like in it. People buy and sell fashion so quickly, I
can usually find even the newest things on there. Most of my friends do
that too.

Image

Mia snaps a selfie while dressing: ``I take selfies for social media
every single time I go out,'' she says, ``first in my bedroom and post
them online, and then always with friends or my
boyfriend.''Credit...Rosie Matheson for The New York Times

\textbf{What constitutes a more special purchase for you?}

An Oh Polly! dress. I buy them for about £20 from Depop, though new they
cost about £40 to £60. Those dresses I keep --- I have three of them.
Teenagers don't mind buying secondhand clothes like some older people
do: You can get good looks at a cheaper price, or directly swap one
dress for another online. I tend to sell lots of the clothes I don't
want in big batches on Depop. It gives me the money to buy new things. I
also sometimes take big bags to consignment stores in town, where they
give you a bit of money for your clothes depending on how much you bring
in.

\textbf{Do you ever think about where those clothes go once you've given
or thrown them away?}

No.

\textbf{Do you ever look at where your clothes are made?}

Yes. I've noticed quite a few are made in England, which shocked me. I
thought they'd all be made in countries like China, India and
Bangladesh. Also, we have been learning a bit in Sociology about how our
clothes are made and the working conditions for people who make them. In
some countries I know they don't get very good wages. It's part of
globalization. I wouldn't talk about it with my friends casually, but we
do talk about it in the classroom.

\textbf{What do you think of sustainable fashion?}

It came on my radar three months ago, I'd say. I am hearing more and
more about it because a lot of brands are now bringing out sustainable
fashion capsule collections, where clothes are made out of recycled
materials, for example. A lot look the same as the normal collection but
cost a few pounds more. But if I'm honest, I do think: Why would I pay
more, when I can get the same for less?

Image

Andrea Vargas, an 18-year-old from Farmingdale, New York, gets ready to
go out in her room at home. ``If I have a shirt in one of my previous
pictures I try not to take a picture again in it,'' she says. ``I don't
like to repeat.''Credit...Krista Schlueter for The New York Times

\hypertarget{i-dont-like-to-repeat}{%
\subsection{`I Don't Like to Repeat'}\label{i-dont-like-to-repeat}}

By \href{https://www.nytimes3xbfgragh.onion/by/taylor-lorenz}{Taylor
Lorenz}

Andrea Vargas, an 18-year-old freshman at Hofstra University, loves
hunting for sales. She looks for them on websites like
\href{https://www.prettylittlething.us/}{Pretty Little Thing} and
\href{https://us.boohoo.com/}{Boohoo}, as well as physical stores like
H\&M and Forever 21, where she can flip through the racks and,
occasionally, find gems.

``I go shopping when the season sales are on,'' she said one Saturday
night at her family's home in Farmingdale, N.Y. She commutes to school
and spends most weekend nights out with friends: getting dinner, maybe
going to a party or a concert. Her plan for this particular evening was
to go to P.F. Chang's with three girlfriends.

Her room is small, with wood floors and inspirational quotes in photo
frames on her pale yellow walls. A Billie Eilish poster hangs opposite
her bed. A guitar she made out of an old skateboard sits in a corner.

Scanning the clothes in her room, she began talking about how she got
them. ``The back-to-school sales, the fall sales, the summer sales,''
she said. ``I love sales.''

Her absolute favorite piece of clothing is a red plush jacket from
Forever 21. She wears it relentlessly when the weather is right. ``It's
just so cute,'' Ms. Vargas said. ``I feel like it dresses up an
outfit.''

Ms. Vargas pays for her clothes herself, using money she earns by
working at Target. The red jacket cost her around \$40, and she said it
was worth every penny. But, she said, ``I feel like there's no point in
spending \$40 on a T-shirt. I personally feel like if the quality of the
shirt doesn't match the price, it doesn't make sense for me to buy it.
If a jean jacket costs \$60 and I can find it for \$20, I'm going to buy
it for \$20. Especially since I'm in college, I need to buy all these
books.''

Image

The red plush jacket is Andrea's favorite piece of clothing, and is from
Forever 21.~``It's just so cute,'' Andrea says. ``I feel like it dresses
up an outfit.''Credit...Krista Schlueter for The New York Times

Ms. Vargas guessed she had purchased between 100 and 200 items this
year, including shoes and jewelry, and that her wardrobe comprises 500
or 600 total pieces. ``I would say the majority of it is shirts,'' she
said. ``They have to be graphic tees. I like a little quote on my shirt
here and there. I have yet to buy new jeans. I like a lot of ripped
jeans. I rarely buy shoes.''

She doesn't generally check where her clothing is made, and she doesn't
feel guilty about how much of it she has. After she's done wearing
something, it can have a second life. ``My mom is from El Salvador and
my dad is from Nicaragua,'' she said. ``They're not wealthy countries,
so I like to give back to people who don't have a lot. It's hot there,
so I can't send long sleeves, but I try to send shorts that don't fit
me, things that are still presentable and wearable.''

She thinks the right amount of money to spend on clothes is \$10 to \$15
on tops, and \$20-\$40 on bottoms. For dresses, which are usually for a
special occasion, she'll go over \$40. She estimates she wears each
piece 15 times before ultimately donating it or selling it on Depop ---
but she also doesn't want to be seen wearing the same thing every day on
Instagram.

``If I have a shirt in one of my previous pictures I try not to take a
picture again in it,'' she said. ``I don't like to repeat.''

Ms. Vargas had invited her friends over to get ready. Alana Wilson, 18,
said that Instagram plays a big role in her shopping life, too. The
moon-and-stars earrings that sparkled beneath her hair were purchased
off an Instagram ad. Almost all of her clothes are from Fashion Nova.

``If it's cute, it's from Fashion Nova,'' Ms. Wilson said. ``Any time I
have money I'll do a whole spree on Fashion Nova. I like it because a
lot of IG models have it.''

Image

Sofia Barbetta, left, and Alyah Mais,~ Andrea's friends, also say they
dress for Instagram and do most of their shopping
on-line.Credit...Krista Schlueter for The New York Times

Another friend, Sofia Barbetta, also 18, agreed. ``I feel like I find
most clothes I want to buy in Instagram ads,'' she said. ``I don't even
follow that many fashion pages, but I see an ad and I'm like, `That's
really cute.'''

She unlocked her phone to show some outfits she'd posted on
\href{https://www.nytimes3xbfgragh.onion/2019/08/30/style/vsco-girls.html}{VSCO},
a photo-sharing app. ``I went through a camo pants phase,'' she said of
one look. ``This outfit, I got inspiration from Twitter.'' Ms. Barbetta
said she'd gotten very into Twitter lately. She started a Post Malone
stan account several years ago, but lately it had become a place to post
personal things.

An hour after Ms. Vargas began getting ready with her friends, she
zipped herself into her outfit for the night: a pair of black platform
military-style boots from Target, black and white houndstooth pants, and
a black off-the-shoulder top from H\&M.

``I got this outfit yesterday,'' she said. ``I was like, `This is the
outfit I'm going to wear.'''

But first, her hair. Ms. Vargas propped her iPhone up in front of her
and sat cross-legged in front of her mirror. She pulled Miss Jessie's
Jelly Soft Curls styler through her waves. ``I wanted to get one of
those vlogging cameras,'' she said, ``one of the Nikon ones.'' For now,
she uses her iPhone.

Hours later she used it to Instagram a photo of her and her friends
posing outside a restaurant in 50 degree weather. They had decided not
to go to P.F. Chang's after all, and were at Taste of Asia instead. None
of them were wearing coats.

``Trust me we were freezing,'' she declared in the caption. But they
were all smiling.

Image

University student Nicole Lambert, 20, wears her favorite black jeans
and leather jacket by Bardot as she prepares for an evening with friends
in Sydney, Australia. ``When I'm dressing to go out, I'm dressing to be
seen, which is weird to say because we're not influencers,'' she
says.Credit...Lisa Maree Williams for The New York Times

\hypertarget{im-dressing-to-be-seen}{%
\subsection{`I'm Dressing to Be Seen'}\label{im-dressing-to-be-seen}}

\href{https://www.nytimes3xbfgragh.onion/by/isabella-kwai}{\includegraphics{https://static01.graylady3jvrrxbe.onion/images/2019/09/17/reader-center/author-isabella-kwai/author-isabella-kwai-thumbLarge.png}}

As told to
\href{https://www.nytimes3xbfgragh.onion/by/isabella-kwai}{Isabella
Kwai}

Nicole Lambert, 20, lives in Sydney, Australia, with her parents and is
studying for an undergraduate degree in public relations and advertising
at the University of New South Wales. She tutors students on weekdays
and works a retail job on weekends.

When she has time off, she and her friends like to dress up and hit the
festival circuit. On a recent evening, after spending the previous day
dancing to EDM, she and her friend Helena Marshall got ready in her
bedroom for a more relaxing dinner.

\textbf{We're not influencers --- but \ldots{}}

When I'm dressing to go out, I'm dressing to be seen, which is weird to
say because we're not influencers. It sounds shallow, but I think in the
back of your head you're like: I probably should avoid wearing the same
outfit twice.

At the end of the day, I prioritize the look versus the practicality.
And that's so unbelievable.

\textbf{Working to be cute}

My friend yesterday at this festival had a really cute Tiger Mist top
with hearts all over it, but it had off-the-shoulder sleeves. I felt so
bad for her the whole day, because she couldn't put her arms up. But she
got cute photos, so it was fine.

I know when you put something up on Instagram and it does well, you're
like, ``Well, that was a good choice on my behalf.'' I love it when
people message, ``Where did you get that from?'' You know you've found
something people can't easily find.

\textbf{Staying relatable}

I think about what I'm going to post for a decent amount of time. It's a
very curated version of your life. You want to look good in your photo,
but have a funny caption so people know you're down to earth and
relatable.

That's why we have private Instagrams, because it gets tiring. That's
where we feel fully free to post whatever. The tragedies of your life.
The real me.

Image

``You want to look good in your photo, but have a funny caption so
people know you're down to earth and relatable,'' Nicole
says.Credit...Lisa Maree Williams for The New York Times

\textbf{Keeping it private}

On my main Instagram, people wouldn't know I'm funny. Because I just
overthink what I post: Will people get it? Are people actually going to
laugh at that?

Sometimes I'll get a weird feeling where I need to get off social media.
I know some people delete their Instagram, like just the app. But that's
admitting to yourself that you have a problem.

\textbf{Leaving shops empty-handed}

I look for clothes at least once a week usually --- either for an
occasion, or just as something to do either online or in store. I shop
60 percent online, 40 percent in person. But 75 percent of the time,
I'll go to the shops, have a look around, and not find one thing because
I think everything is the same.

I'm not afraid to put on something weird. I'm really big into animal
print at the moment. Almost to the point where I'll wear too much of it.
I love my snake pants --- and flares. Flares should never go out.

\textbf{Princess Polly and Tiger Mist}

For basics, 100 percent of my wardrobe is from Kookai. They're always
rotating really nice, classic things. I get a lot of stuff off Revolve,
because there are so many different brands. You've got things there that
you're not going to see five people wearing once you're out. From other
online brands like Princess Polly, Tiger Mist. Sometimes it's
overwhelming how much stuff there is online. I could go on for hours.

Often, on Instagram, I'll scroll through the Explore page, and people
just tagging outfits. It's so helpful because you just click onto the
account, find the item. That's how I find the little niche things.

\textbf{Where were these dresses made?}

If I feel so amazing in something, I'm probably not going to look too
hard into the price. But I don't like investing a lot of money for
something you might not wear too much. I like Pretty Little Thing for
crazy things for cheap, because they just do interesting little tops or
little dresses, clubbing clothes. Do I look at the labels of clothes?
Not really. In the back of my head, I **** assume that I know where the
clothes are made: in China.

In terms of how much I would spend: average price of a dress, probably
about \$180 Australian dollars. Jeans, about \$150. A good going out
top, \$50. I do like a nice pair of heels, so I've spent like \$200 for
a pair. But then again I've got ones for \$50. In my wardrobe now, I'd
say I have roughly 200 pieces.

Image

Some of Nicole's clothing and accessories; ``I'm not afraid to put on
something weird,'' she says. ``I'm really big into animal print at the
moment.''Credit...Lisa Maree Williams for The New York Times

\textbf{Cycling the wardrobe overseas}

I do a big spring clean every year and send boxes of clothes over to my
family in the Philippines. ** One of my cousins has a market stall. So I
assumed that maybe my stuff would end up there if they didn't want to
keep it for themselves.

I would say 30 percent of my wardrobe would get pulled out. Maybe 80
bits of clothes. It makes a good dent.

When I pull it all out and you see a big pile of clothes on your floor,
you feel a bit sick. I'm glad that I can send it somewhere and it's
helping at least my family.

\textbf{Supporting sustainability --- or not}

I want to support sustainable brands. But if it doesn't work for me and
what I'm doing in my lifestyle, I'm going to go with something else
instead.

Timing is important. For what I wore to the Listen Out festival
yesterday, **** I ordered on Tuesday morning, it came on Wednesday
morning: literally in 24 hours. That means so much to me. I'm the least
decisive person and the least patient person. When miniature bags were
in, I was obsessed with this one from London. You could get your
initials on it. But it said it could take 30 days and I was like, never
mind. I got a cute one from Mango.

You're pushing it after seven business days. If it's a big order I don't
mind waiting for a week. But if it's one thing, it's like: Why?

\begin{center}\rule{0.5\linewidth}{\linethickness}\end{center}

\emph{Interviews have been edited for style and clarity.}

Advertisement

\protect\hyperlink{after-bottom}{Continue reading the main story}

\hypertarget{site-index}{%
\subsection{Site Index}\label{site-index}}

\hypertarget{site-information-navigation}{%
\subsection{Site Information
Navigation}\label{site-information-navigation}}

\begin{itemize}
\tightlist
\item
  \href{https://help.nytimes3xbfgragh.onion/hc/en-us/articles/115014792127-Copyright-notice}{©~2020~The
  New York Times Company}
\end{itemize}

\begin{itemize}
\tightlist
\item
  \href{https://www.nytco.com/}{NYTCo}
\item
  \href{https://help.nytimes3xbfgragh.onion/hc/en-us/articles/115015385887-Contact-Us}{Contact
  Us}
\item
  \href{https://www.nytco.com/careers/}{Work with us}
\item
  \href{https://nytmediakit.com/}{Advertise}
\item
  \href{http://www.tbrandstudio.com/}{T Brand Studio}
\item
  \href{https://www.nytimes3xbfgragh.onion/privacy/cookie-policy\#how-do-i-manage-trackers}{Your
  Ad Choices}
\item
  \href{https://www.nytimes3xbfgragh.onion/privacy}{Privacy}
\item
  \href{https://help.nytimes3xbfgragh.onion/hc/en-us/articles/115014893428-Terms-of-service}{Terms
  of Service}
\item
  \href{https://help.nytimes3xbfgragh.onion/hc/en-us/articles/115014893968-Terms-of-sale}{Terms
  of Sale}
\item
  \href{https://spiderbites.nytimes3xbfgragh.onion}{Site Map}
\item
  \href{https://help.nytimes3xbfgragh.onion/hc/en-us}{Help}
\item
  \href{https://www.nytimes3xbfgragh.onion/subscription?campaignId=37WXW}{Subscriptions}
\end{itemize}
