Sections

SEARCH

\protect\hyperlink{site-content}{Skip to
content}\protect\hyperlink{site-index}{Skip to site index}

\href{/section/technology}{Technology}\textbar{}A Surveillance Net
Blankets China's Cities, Giving Police Vast Powers

\url{https://nyti.ms/34zMxUO}

\begin{itemize}
\item
\item
\item
\item
\item
\item
\end{itemize}

\includegraphics{https://static01.graylady3jvrrxbe.onion/images/2019/12/16/business/00china-surveillance-1/merlin_166033467_f8fd8eba-4600-4ad6-8df6-e905439ec3fb-articleLarge.jpg?quality=75\&auto=webp\&disable=upscale}

\hypertarget{a-surveillance-net-blankets-chinas-cities-giving-police-vast-powers}{%
\section{A Surveillance Net Blankets China's Cities, Giving Police Vast
Powers}\label{a-surveillance-net-blankets-chinas-cities-giving-police-vast-powers}}

The authorities can scan your phones, track your face and find out when
you leave your home. One of the world's biggest spying networks is aimed
at regular people, and nobody can stop it.

A surveillance camera on the ceiling of a packed subway car in
Zhengzhou.Credit...Gilles Sabrié for The New York Times

Supported by

\protect\hyperlink{after-sponsor}{Continue reading the main story}

By \href{https://www.nytimes3xbfgragh.onion/by/paul-mozur}{Paul Mozur}
and Aaron Krolik

\begin{itemize}
\item
  Dec. 17, 2019
\item
  \begin{itemize}
  \item
  \item
  \item
  \item
  \item
  \item
  \end{itemize}
\end{itemize}

\href{https://cn.nytimes3xbfgragh.onion/technology/20191219/china-surveillance/}{阅读简体中文版}\href{https://cn.nytimes3xbfgragh.onion/technology/20191219/china-surveillance/zh-hant}{閱讀繁體中文版}

ZHENGZHOU, China --- China is ramping up its ability to spy on its
nearly 1.4 billion people to new and disturbing levels, giving the world
a blueprint for how to build a digital totalitarian state.

Chinese authorities are knitting together old and state-of-the-art
technologies --- phone scanners,
\href{https://www.nytimes3xbfgragh.onion/2018/07/08/business/china-surveillance-technology.html}{facial-recognition
cameras}, face and fingerprint databases and
\href{https://www.nytimes3xbfgragh.onion/2019/07/02/technology/china-xinjiang-app.html}{many
others} --- into sweeping tools for authoritarian control, according to
police and private databases examined by The New York Times.

Once combined and fully operational, the tools can help police grab the
identities of people as they walk down the street, find out who they are
meeting with and identify who does and doesn't belong to the Communist
Party.

The United States and other countries
\href{https://www.nytimes3xbfgragh.onion/2015/07/25/world/europe/france-clears-final-hurdle-to-expand-spying-power.html}{use
some of the same techniques} to track terrorists or drug lords. Chinese
cities want to use them to track everybody.

The rollout has come at the expense of personal privacy. The Times found
that the authorities parked the personal data of millions of people on
servers unprotected by even basic security measures. It also found that
private contractors and middlemen have wide access to personal data
collected by the Chinese government.

This build-out has only just begun, but it is sweeping through Chinese
cities. The surveillance networks are controlled by local police, as if
county sheriffs in the United States ran their own personal versions of
the National Security Agency.

By themselves, none of China's new techniques are beyond the
capabilities of the United States or other countries. But together, they
could propel China's spying to a new level, helping its cameras and
software become smarter and more sophisticated.

\includegraphics{https://static01.graylady3jvrrxbe.onion/images/2018/09/01/business/ECUADOR_17a/ECUADOR_17a-videoSixteenByNine3000-v2.jpg}

This surveillance push is empowering China's police, who have taken a
greater role in China under Xi Jinping, its top leader. It gives them a
potent way to track criminals as well as
\href{https://www.nytimes3xbfgragh.onion/2018/02/28/world/asia/china-censorship-xi-jinping.html}{online
malcontents}, sympathizers
\href{https://www.nytimes3xbfgragh.onion/2019/06/13/world/asia/hong-kong-xi-jinping-china.html}{of
the protest movement in Hong Kong}, critics of police themselves and
other undesirables. It often targets vulnerable groups like
\href{https://www.nytimes3xbfgragh.onion/2017/11/30/world/asia/china-beijing-migrants.html}{migrant
workers} --- those who stream in from the countryside to fill China's
factories --- and ethnic minority groups
\href{https://www.nytimes3xbfgragh.onion/interactive/2019/11/16/world/asia/china-xinjiang-documents.html}{like
the largely Muslim Uighurs} on China's western frontier.

``Each person's data forms a trail,'' said Agnes Ouyang, a technology
worker in the southern city of Shenzhen whose attempts to raise
awareness about privacy drew scrutiny from the authorities. ``It can be
used by the government and it can be used by bosses at the big companies
to track us. Our lives are worth about as much as dirt.''

\includegraphics{https://static01.graylady3jvrrxbe.onion/images/2019/12/16/business/00china-surveillance-2/merlin_166033536_21c7f036-5590-4adb-bf9c-3d27ad6259b1-articleLarge.jpg?quality=75\&auto=webp\&disable=upscale}

\hypertarget{people-pass-and-leave-a-shadow}{%
\subsection{`People Pass and Leave a
Shadow'}\label{people-pass-and-leave-a-shadow}}

The police arrived one day in April to a dingy apartment complex in
Zhengzhou, an industrial city in central China. Over three days they
installed four cameras and two small white boxes at the gates of the
complex, which hosts cheap hotels and fly-by-night businesses.

Once activated, the system began to sniff for personal data. The boxes
--- phone scanners called IMSI catchers and
\href{https://www.nytimes3xbfgragh.onion/2016/02/12/nyregion/new-york-police-dept-cellphone-tracking-stingrays.html}{widely
used} in the West --- collected identification codes from mobile phones.
The cameras recorded faces.

On the back end, the system attempted to tie the data together, an
examination of its underlying database showed. If a face and a phone
appeared at the same place and time, the system grew more confident they
belonged to the same person.

Over four days in April, the boxes identified more than 67,000 phones.
The cameras captured more than 23,000 images, from which about 8,700
unique faces were derived. Combining the disparate data sets, the system
matched about 3,000 phones with faces, with varying degrees of
confidence.

Mobile device information and surveillance images captured in Zhengzhou
in April 2019

2k

1.5k

Unique

mobile

devices

1k

500

Surveillance

images

Apr. 6, midnight

Noon

Apr. 7

Noon

Apr. 8

Noon

Apr. 9

Mobile device information and surveillance

images captured in Zhengzhou in April 2019

2k

Unique

mobile devices

1.5k

1k

500

Surveillance

images

Apr. 6, midnight

Apr. 7

Apr. 8

Apr. 9

Source: Data obtained by The New York Times

This single system is part of a citywide surveillance network
encompassing license plates, phone numbers, faces and social media
information, according to a Zhengzhou Public Security Bureau database.

Other Chinese cities are copying Zhengzhou. Since 2017, government
procurement documents and official reports show that police in the
Chinese provinces of Guizhou, Zhejiang and Henan have bought similar
systems. Police in Zigong, a midsize city in Sichuan Province, bought
156 sets of the technology, the documents show.

In Wuhan, police said in a procurement document that they wanted systems
that could ``comprehensively collect the identity of all internet users
in public spaces, their internet behavior, their location, their
movement, and identifying information about their phones.''

``People pass and leave a shadow,'' reads one brochure promoting a
similar surveillance system to Chinese police departments. ``The phone
passes and leaves a number. The system connects the two.''

Even for China's police, who enjoy broad powers to question and detain
people, this level of control is unprecedented. Tracking people so
closely once required cooperation from uncooperative institutions in
Beijing. The state-run phone companies, for example, are often reluctant
to share sensitive or lucrative data with local authorities, said people
with knowledge of the system.

Image

Phone identification scanners --- called IMSI catchers --- collected
identification codes from mobile phones at the entrance of the
complex.Credit...Paul Mozur

Now local police are buying their own trackers. Improved technology
helps them share it up the chain of command, to the central Ministry of
Public Security in Beijing, the people said.

The surveillance networks fulfill a longtime goal of ensuring social
stability, dating to the
\href{https://www.nytimes3xbfgragh.onion/2019/05/30/world/asia/tiananmen-square-protest-photos.html}{1989
Tiananmen Square uprising} but
\href{https://www.nytimes3xbfgragh.onion/2013/07/21/world/asia/death-in-china-stirs-anger-over-urban-rule-enforcers.html}{given
added urgency} by
\href{https://www.nytimes3xbfgragh.onion/2017/12/04/world/middleeast/ali-abdullah-saleh-strongmen.html}{the
Arab Spring in 2011 and 2012}. In recent years, Chinese police made use
of
\href{https://www.nytimes3xbfgragh.onion/2015/12/28/world/asia/china-passes-antiterrorism-law-that-critics-fear-may-overreach.html}{fears
of unrest} to win more power and resources.

It is not clear how well police are using their new capabilities, or
just how effective they might be. But the potential is there.

In Zhengzhou, police can use software to create lists of people. They
can create virtual alarms for when a person approaches a particular
location. They can get updates on people every hour or every day. They
can monitor whom those people have met with, especially if both people
are on a blacklist for some kind of infraction, from committing a crime
to skipping a debt payment.

These networks could help China hone technologies like facial
recognition. Cameras and software often have trouble
\href{https://www.nytimes3xbfgragh.onion/2019/07/13/technology/databases-faces-facial-recognition-technology.html}{recognizing
faces shot at an angle}, for example. Combined with phone and identity
data, matches become easier to make, and the technology behind
identifying faces gets better.

Police are not hiding their surveillance push. Even the perception of
overwhelming surveillance can deter criminals and
\href{https://www.nytimes3xbfgragh.onion/2020/07/30/world/asia/coronavirus-china-qurantine.html}{dissidents}
alike.

At the complex in Zhengzhou, residents were unfazed when told that the
cameras and boxes were part of a sophisticated surveillance system.

The building manager, Liang Jianzheng, said it meant he no longer had to
help the police fight crime.

``I used to have to bust my butt helping the police,'' Mr. Liang said.
``Now they have their own cameras, and they don't bother me."

In November, after The Times asked surveillance companies about the
system, a construction crew appeared and took down the cameras and
boxes, Mr. Liang said. They didn't say why.

Image

Surveillance cameras at the Huating Apartments residential
complex.Credit...Gilles Sabrié for The New York Times

\hypertarget{the-wire-and-plywood-revolt}{%
\subsection{The Wire and Plywood
Revolt}\label{the-wire-and-plywood-revolt}}

Some residents of the Shijiachi residential complex weren't pleased when
building management, at the behest of the police, last year replaced
their old key card locks with a state-of-the-art surveillance system.
Residents would now need to scan their faces to enter their buildings.

``Old people said they were always at home, so it wasn't necessary,''
said Tang Liying, the secretary of the Chinese Communist Party for the
district in eastern China. ``Young people had concerns about privacy,
and didn't think it was necessary. We did some work to persuade them,
and in the end most people agreed.''

Those worried about privacy had a point.

Data from the Shijiachi complex was parked on an unprotected server.
Details included 482 residents' identification numbers, names, ages,
marital and family status, and records of their membership in the
Communist Party. For those who used the facial-recognition cameras to
enter and exit, it also stored a detailed account of their comings and
goings.

Nearby networks were similarly unprotected. They held data from 31
residences in the area, with details on 8,570 people. A car-tracking
system near Shijiachi showed records for 3,456 cars and personal
information about their owners. Across China, unprotected databases hold
information on students and teachers in schools, on online activity in
internet cafes and on hotel stays and travel records.

Online data leakage is a major problem in China. Local media reports
describe how people with access to the data
\href{https://www.thepaper.cn/newsDetail_forward_5095971}{sell private
details} to fraudsters, suspicious spouses and anyone else, sometimes
for just a \href{https://wxn.qq.com/cmsid/NEW2016121200672607}{few
dollars per person}. Leaks have become severe enough that police created
their own company that handles data directly, skirting third-party
systems.

A wide number of people and companies have access to the data underlying
China's mandatory identification card system through legitimate means.
Companies with police connections use faces from ID cards to train
facial-recognition systems. The card system also tracks fingerprints,
faces, ethnicity and age.

A technology contractor called Shenfenbao, for example, had access to
real-time records of every person staying in some 1,200 hotels in the
southern city of Xiamen. In a demonstration, Lin Jiahong, a Shenfenbao
salesman, searched one common name --- a Chinese equivalent of ``John
Smith'' --- and came up with three guests, their hotels, room numbers,
time of check-in, registered address, ethnicity and age.

``Through data on our platform, we can dig out all records of a
particular person, and make a comprehensive analysis of the route of
activities of this person,'' said Mr. Lin, who added that his company
also offered algorithms to flag women who check into multiple hotels in
one night for suspicion of prostitution.

Signs of a backlash are brewing. In Shanghai, residents pushed back
against a police plan to install facial-recognition cameras in a
building complex. In Zhejiang Province, a professor filed a lawsuit
against a zoo after it required mandatory facial-recognition scans for
its members to get access.

In the Shijiachi residential complex, where the facial recognition
replaced key card locks, the rebellion has been powered by wire and
plywood.

On a brisk day in November, the doors of a number of buildings had been
propped open with crude doorstops, making facial scans unnecessary.

Terry Jin, a two-year resident of Shijiachi, said technology should not
cross some lines.

``I think that facial recognition outside each building is fine,'' Mr.
Jin said. ``If they put it outside my door, that wouldn't be O.K.''

Image

A facial-recognition scanning system at the entrance of the subway
station near Huating Apartments.Credit...Gilles Sabrié for The New York
Times

\hypertarget{what-happens-when-you-say-no}{%
\subsection{What Happens When You Say
No}\label{what-happens-when-you-say-no}}

Agnes Ouyang was heading to work in Shenzhen last year when two police
officers told her she had jaywalked and would need to show them her
identity card. When she refused, she said, they grabbed her roughly and
used a phone to snap a photo of her face.

Within moments, their facial-recognition system had identified her, and
they issued her a ticket for about \$3.

``It was all too ridiculous,'' Ms. Ouyang said. ``Law-enforcement
officers of low moral stock have high-tech weapons.''

High-tech surveillance is reshaping Chinese life in ways small and
profound. The Communist Party has long ruled supreme, and the country
lacks a strong court system or other checks against government
overreach. But outside the realm of politics, Chinese life could be
freewheeling and chaotic thanks to lax enforcement or indifferent
officials.

Those days may be coming to an end. In the realms of consumer safety and
the environment, that could make life better. But it has given police
new powers to control the people.

``The whole bureaucratic system is broken,'' said Borge Bakken, a
professor at Australian National University who studies China's police.
``Under Xi Jinping, we're seeing the flowering of a police state.''

Chinese police now boast that facial-recognition systems regularly catch
crooks. At a tourist island in the picturesque port city of Xiamen,
authorities say they use facial recognition to catch unlicensed tour
guides. Shanghai police have begun using helmets with a camera embedded
in the front. Databases and procurement documents also show they search
out the mentally ill, people with a history of drug use or government
gadflies.

Some new claims are outlandish, such as software that claims to read
emotion and criminal intent from a face. But the surveillance net that
\href{https://www.nytimes3xbfgragh.onion/interactive/2019/04/04/world/asia/xinjiang-china-surveillance-prison.html}{police
have rolled out in Xinjiang}, a region of northwestern China that is
home to many predominantly Muslim ethnic groups, shows the vast
potential for the rest of the country.

Police have blanketed the region in cameras, phone trackers and
sensor-studded checkpoints. In Urumqi, the regional capital, police
sealed off 3,640 residential complexes with checkpoints and installed
18,464 sets of facial-recognition cameras in them, according to data
unveiled at a police presentation in August given by Li Yabin, a top
police official in Xinjiang. In the southern Xinjiang city of Kashgar,
The New York Times tallied a dense network of 37 phone trackers
installed permanently in a single, square-kilometer neighborhood.

Ms. Ouyang, the woman ticketed for jaywalking, knew the dangers, but
took her complaints public anyway. She posted an account of her run-in
with police on WeChat, the Chinese social media outlet, at 11 p.m. By
the time she went to work the next morning, it had been seen tens of
thousands of times. Then it vanished.

\includegraphics{https://static01.graylady3jvrrxbe.onion/images/2019/12/16/business/00china-surveillance-still/00china-surveillance-still-videoSixteenByNineJumbo1600.jpg}

After she saw police treat another woman the same way, Ms. Ouyang wrote
a second post. It came down in just two hours.

Then the police called and demanded a meeting.

``I said, `how did you find me?''' Ms. Ouyang said. ``He said, `it's
easy for the police to find a person.'''

Fearful, she asked a friend to accompany her and chose to meet the
police at a Starbucks instead of the police station. Two officers bought
them coffee and gave her a phone number to call if she had future
complaints. But mostly, they said, she needed to keep quiet. Her post
had been seen by higher-up officials and embarrassed the city's police,
they said.

Ms. Ouyang said the experience was one sign of an authoritarian turn
within China, and that some of her friends quietly talk about leaving.
She has no plans to leave, she says, but she worries about her future in
a country where everything is watched and controlled.

``You're uncomfortable with it,'' she said. ``But if you don't do it,
then there's no possibility of living a life. There's no way out.''

Paul Mozur reported from Zhengzhou, China, in addition to Xiamen,
Shaoxing, Shenzhen, Shanghai, Beijing, Kashgar and Urumqi. Aaron Krolik
reported from New York. Keith Collins contributed reporting from New
York.

Advertisement

\protect\hyperlink{after-bottom}{Continue reading the main story}

\hypertarget{site-index}{%
\subsection{Site Index}\label{site-index}}

\hypertarget{site-information-navigation}{%
\subsection{Site Information
Navigation}\label{site-information-navigation}}

\begin{itemize}
\tightlist
\item
  \href{https://help.nytimes3xbfgragh.onion/hc/en-us/articles/115014792127-Copyright-notice}{©~2020~The
  New York Times Company}
\end{itemize}

\begin{itemize}
\tightlist
\item
  \href{https://www.nytco.com/}{NYTCo}
\item
  \href{https://help.nytimes3xbfgragh.onion/hc/en-us/articles/115015385887-Contact-Us}{Contact
  Us}
\item
  \href{https://www.nytco.com/careers/}{Work with us}
\item
  \href{https://nytmediakit.com/}{Advertise}
\item
  \href{http://www.tbrandstudio.com/}{T Brand Studio}
\item
  \href{https://www.nytimes3xbfgragh.onion/privacy/cookie-policy\#how-do-i-manage-trackers}{Your
  Ad Choices}
\item
  \href{https://www.nytimes3xbfgragh.onion/privacy}{Privacy}
\item
  \href{https://help.nytimes3xbfgragh.onion/hc/en-us/articles/115014893428-Terms-of-service}{Terms
  of Service}
\item
  \href{https://help.nytimes3xbfgragh.onion/hc/en-us/articles/115014893968-Terms-of-sale}{Terms
  of Sale}
\item
  \href{https://spiderbites.nytimes3xbfgragh.onion}{Site Map}
\item
  \href{https://help.nytimes3xbfgragh.onion/hc/en-us}{Help}
\item
  \href{https://www.nytimes3xbfgragh.onion/subscription?campaignId=37WXW}{Subscriptions}
\end{itemize}
