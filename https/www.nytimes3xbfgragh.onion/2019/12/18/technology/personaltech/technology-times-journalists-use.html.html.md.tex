Sections

SEARCH

\protect\hyperlink{site-content}{Skip to
content}\protect\hyperlink{site-index}{Skip to site index}

\href{https://www.nytimes3xbfgragh.onion/section/technology/personaltech}{Personal
Tech}

\href{https://myaccount.nytimes3xbfgragh.onion/auth/login?response_type=cookie\&client_id=vi}{}

\href{https://www.nytimes3xbfgragh.onion/section/todayspaper}{Today's
Paper}

\href{/section/technology/personaltech}{Personal Tech}\textbar{}What We
Learned About the Technology That Times Journalists Use

\href{https://nyti.ms/36SNrgV}{https://nyti.ms/36SNrgV}

\begin{itemize}
\item
\item
\item
\item
\item
\end{itemize}

Advertisement

\protect\hyperlink{after-top}{Continue reading the main story}

Supported by

\protect\hyperlink{after-sponsor}{Continue reading the main story}

tech we're using

\hypertarget{what-we-learned-about-the-technology-that-times-journalists-use}{%
\section{What We Learned About the Technology That Times Journalists
Use}\label{what-we-learned-about-the-technology-that-times-journalists-use}}

After three years and more than 130 columns, the smartphone was tops.
There were also some deliberate Luddites among us.

\includegraphics{https://static01.graylady3jvrrxbe.onion/images/2019/12/18/business/18techusing1/merlin_139111119_8a989ef0-5d07-43c4-8bd7-870a780b69dd-articleLarge.jpg?quality=75\&auto=webp\&disable=upscale}

\href{https://www.nytimes3xbfgragh.onion/by/brian-x-chen}{\includegraphics{https://static01.graylady3jvrrxbe.onion/images/2018/02/16/multimedia/author-brian-x-chen/author-brian-x-chen-thumbLarge.jpg}}

By \href{https://www.nytimes3xbfgragh.onion/by/brian-x-chen}{Brian X.
Chen}

\begin{itemize}
\item
  Dec. 18, 2019
\item
  \begin{itemize}
  \item
  \item
  \item
  \item
  \item
  \end{itemize}
\end{itemize}

Several years ago, some colleagues and I were chatting about what was
missing from tech journalism. Plenty of news media outlets had written
breathlessly about hot new gadgets and apps. But what were people really
doing with that tech?

That question spawned
\href{https://www.nytimes3xbfgragh.onion/column/tech-we-are-using}{Tech
We're Using}, a weekly feature that documented how New York Times
journalists used tech to cover a wide variety of topics, including
politics, sports, wars, natural disasters, food and art.

With the decade coming to a close, we decided to also wrap up the column
after interviews with more than 130 Times reporters, editors and
photographers. Here were our biggest takeaways.

\hypertarget{indispensable-tech-encryption-apps-and-batteries}{%
\subsection{Indispensable Tech: Encryption Apps and
Batteries}\label{indispensable-tech-encryption-apps-and-batteries}}

Unsurprisingly, the smartphone was the most vital work tool among
journalists. Many reporters relied on smartphones for recording
interviews and turned to A.I.-powered apps like Trint and Rev to
\href{https://www.nytimes3xbfgragh.onion/2019/08/21/technology/personaltech/how-to-turn-an-iphone-into-a-work-only-tool.html}{automatically
transcribe interviews} into notes.

Most Times reporters now also rely on some form of encrypted
communication, particularly
\href{https://www.nytimes3xbfgragh.onion/2017/11/15/technology/personaltech/messaging-apps-katie-benner.html}{messaging
apps} like Signal and WhatsApp or the emailing service ProtonMail, to
keep their sources and conversations confidential.

That is a remarkable shift. Encryption technologies became popular only
a few years ago, after the former government security contractor
\href{https://www.nytimes3xbfgragh.onion/2019/09/13/books/review-permanent-record-edward-snowden-memoir.html}{Edward
Snowden} revealed the extent of what the United States government was
doing to surveil its own citizens.

Another indispensable tool underlined a type of tech that has not
improved much: batteries. Many reporters, especially national
correspondents who live out of a suitcase, desperately needed phones
with
\href{https://www.nytimes3xbfgragh.onion/2018/03/28/technology/personaltech/tech-tools-shootings-hurricanes.html}{longer-lasting
batteries}, so battery packs were a staple in their arsenal of tools.

\includegraphics{https://static01.graylady3jvrrxbe.onion/images/2019/12/18/business/18techusing3/merlin_162445068_0b21a155-04ba-46d9-b5ee-de00fc14d662-articleLarge.jpg?quality=75\&auto=webp\&disable=upscale}

\hypertarget{the-most-tech-enthused-journalists-were-not-tech-reporters}{%
\subsection{The Most Tech-Enthused Journalists Were Not Tech
Reporters}\label{the-most-tech-enthused-journalists-were-not-tech-reporters}}

Some of the reporters who were savviest about using tech worked in
politics and investigations. That is because they needed apps to pore
through
\href{https://www.nytimes3xbfgragh.onion/2018/07/18/technology/personaltech/tech-reporter-does-not-use-tech.html}{documents
that were thousands of pages long}, or software tools to
\href{https://www.nytimes3xbfgragh.onion/2019/05/22/technology/personaltech/maximize-online-privacy.html}{inspect
what was going on under the hood} of a website's code. Others needed
special apps to hide their identities with burner phone numbers.

Many photographers were also early adopters of new tech. One key
example: drones.
\href{https://www.nytimes3xbfgragh.onion/2018/05/02/technology/personaltech/visual-journalism-drones.html}{Those
were constantly getting smaller}, and their cameras were improving,
which created possibilities for new types of photography, like overhead
shots of houses damaged in a fire.

In contrast, many tech reporters tried to
\href{https://www.nytimes3xbfgragh.onion/2018/07/18/technology/personaltech/tech-reporter-does-not-use-tech.html}{minimize
the amount of tech they used}. That could be, in part, a symptom of
knowing too much about the companies they covered and the wide swaths of
data those companies collected.

\hypertarget{tech-has-changed-everything}{%
\subsection{Tech Has Changed
Everything}\label{tech-has-changed-everything}}

Many editors and reporters also talked about how tech had transformed
the industries they cover.

In the world of dining, digital photography and platforms like Instagram
have become the main method that
\href{https://www.nytimes3xbfgragh.onion/2017/09/20/technology/personaltech/what-technology-pete-wells-uses.html}{restaurants
use to communicate} with patrons.
\href{https://www.nytimes3xbfgragh.onion/2017/11/22/technology/personaltech/rocket-launches-space-iphone.html}{Rocket
launches} are now live-streamed online, which let our space reporter
watch from his phone instead of heading to the space station. And in the
entertainment world, video streaming has opened doors to
\href{https://www.nytimes3xbfgragh.onion/2019/06/26/technology/personaltech/for-a-tv-editor-500-new-shows-a-year-require-a-fresh-script.html}{a
wealth of new content} --- so much that reporting on movies and TV shows
has become an art of curation.

What's ahead? If tech has invaded everything, the answer is: even more
transformation.

Advertisement

\protect\hyperlink{after-bottom}{Continue reading the main story}

\hypertarget{site-index}{%
\subsection{Site Index}\label{site-index}}

\hypertarget{site-information-navigation}{%
\subsection{Site Information
Navigation}\label{site-information-navigation}}

\begin{itemize}
\tightlist
\item
  \href{https://help.nytimes3xbfgragh.onion/hc/en-us/articles/115014792127-Copyright-notice}{©~2020~The
  New York Times Company}
\end{itemize}

\begin{itemize}
\tightlist
\item
  \href{https://www.nytco.com/}{NYTCo}
\item
  \href{https://help.nytimes3xbfgragh.onion/hc/en-us/articles/115015385887-Contact-Us}{Contact
  Us}
\item
  \href{https://www.nytco.com/careers/}{Work with us}
\item
  \href{https://nytmediakit.com/}{Advertise}
\item
  \href{http://www.tbrandstudio.com/}{T Brand Studio}
\item
  \href{https://www.nytimes3xbfgragh.onion/privacy/cookie-policy\#how-do-i-manage-trackers}{Your
  Ad Choices}
\item
  \href{https://www.nytimes3xbfgragh.onion/privacy}{Privacy}
\item
  \href{https://help.nytimes3xbfgragh.onion/hc/en-us/articles/115014893428-Terms-of-service}{Terms
  of Service}
\item
  \href{https://help.nytimes3xbfgragh.onion/hc/en-us/articles/115014893968-Terms-of-sale}{Terms
  of Sale}
\item
  \href{https://spiderbites.nytimes3xbfgragh.onion}{Site Map}
\item
  \href{https://help.nytimes3xbfgragh.onion/hc/en-us}{Help}
\item
  \href{https://www.nytimes3xbfgragh.onion/subscription?campaignId=37WXW}{Subscriptions}
\end{itemize}
