Sections

SEARCH

\protect\hyperlink{site-content}{Skip to
content}\protect\hyperlink{site-index}{Skip to site index}

\href{https://www.nytimes3xbfgragh.onion/section/technology}{Technology}

\href{https://myaccount.nytimes3xbfgragh.onion/auth/login?response_type=cookie\&client_id=vi}{}

\href{https://www.nytimes3xbfgragh.onion/section/todayspaper}{Today's
Paper}

\href{/section/technology}{Technology}\textbar{}Prime Power: How Amazon
Squeezes the Businesses Behind Its Store

\href{https://nyti.ms/35IHnHy}{https://nyti.ms/35IHnHy}

\begin{itemize}
\item
\item
\item
\item
\item
\item
\end{itemize}

Advertisement

\protect\hyperlink{after-top}{Continue reading the main story}

Supported by

\protect\hyperlink{after-sponsor}{Continue reading the main story}

\hypertarget{amazon-everywhere}{%
\subsubsection{Amazon Everywhere}\label{amazon-everywhere}}

\hypertarget{prime-power-how-amazon-squeezes-the-businesses-behind-its-store}{%
\section{Prime Power: How Amazon Squeezes the Businesses Behind Its
Store}\label{prime-power-how-amazon-squeezes-the-businesses-behind-its-store}}

\includegraphics{https://static01.graylady3jvrrxbe.onion/images/2019/12/20/business/05amazontax/05amazontax-articleLarge.gif?quality=75\&auto=webp\&disable=upscale}

Twenty years ago, Amazon opened its storefront to anyone who wanted to
sell something. Then it began demanding more out of them.

\href{https://www.nytimes3xbfgragh.onion/by/karen-weise}{\includegraphics{https://static01.graylady3jvrrxbe.onion/images/2019/04/11/multimedia/author-karen-weise/author-karen-weise-thumbLarge.png}}

By \href{https://www.nytimes3xbfgragh.onion/by/karen-weise}{Karen Weise}

\begin{center}\rule{0.5\linewidth}{\linethickness}\end{center}

SEATTLE --- For tens of millions of Americans, it is so routine that
they don't think twice.

They want something --- a whisk, diapers, that dog toy --- and they turn
to Amazon. They type the product's name into Amazon's website or app,
scan the first few options and click buy. In a day or two, the purchase
appears on their doorstep.

Amazon has transformed the small miracle of each delivery into an
expectation of modern life. No car, no shopping list --- no planning ---
required.

But to make it all work, Amazon runs a machine that squeezes ever more
money out of the hundreds of thousands of companies, from tiny start-ups
to giant brands, that put the everything into Amazon's Everything Store.

In more than 60 interviews, current and former Amazon employees,
sellers, suppliers and consultants detailed how Amazon dictates the
rules for those businesses, sometimes changing those rules with little
warning. Many spoke on the condition of anonymity, for fear of
retaliation by Amazon.

Amazon punishes the businesses if their items are available for even a
penny less elsewhere. It pushes them to use the company's warehouses.
And it compels them to buy ads on the site to make sure people see their
products.

All of that leaves the suppliers more dependent on Amazon, by far the
nation's top online retailer, and scrambling to deal with its whims. For
many, Amazon eats into their profits, making it harder to develop new
products. Some worry if they can even survive.

``Every year it's been a ratchet tighter,'' said Bernie Thompson, a top
seller of computer accessories who Amazon has highlighted in its
marketing to other merchants. ``Now you are one event away from not
functioning.''

Tumi, the luxury bag maker, sold its products at wholesale prices to
Amazon for years. But executives said Amazon sometimes misjudged
consumer demand, keeping too few bags in stock, and regularly demanded
more in marketing and other fees. Last year, Tumi decided to sell its
bags to another company, which then listed the items on Amazon. The
arrangement gave Tumi more control over inventory and better sales data.

A few months later, Amazon gave Tumi an ultimatum: Stop selling through
the middleman, or do not sell to the retailer's 150 million customers at
all.

``Some guy we had never talked to gave us a call and was like, `We have
changed the rules,' '' said Charlie Cole, who runs Tumi's online
business. He pushed back, but wasn't successful.

``It was like talking to a brick wall,'' he said. ``They want to be able
to control everything.''

Companies struggling to navigate Amazon's growing chaos fill Facebook
groups, private message boards and industry conferences. One session at
a leading retail meeting next year is called ``The Big Question: Is
Selling on Amazon Worth the Hassle?'' More than 12,000 people
\href{https://www.change.org/p/jeff-bezos-amazon-com-should-only-suspend-the-asin-not-the-seller}{signed
a petition} on Change.org asking Amazon to alter an arcane rule on
counterfeit products that they said could ``destroy'' an entire
business.

Many sellers and brands on Amazon are desperate to depend less on the
tech giant. But when they look for sales elsewhere online, they come up
short. Last year, Americans bought more books, T-shirts and other
products on Amazon than eBay, Walmart and its next seven largest online
competitors combined, according to eMarketer, a research company.

``The secret of Amazon is we're happy to help you be very successful,''
said David Glick, a former Amazon vice president who left the company
last year. ``You just have to kiss the ring.''

Amazon says that its operation is so massive, the rules are necessary to
give customers a quality experience. The company said the health of
sellers was a top priority, and that it had invested billions of dollars
to support them. It said that about 200,000 sellers surpassed \$100,000
in sales in 2018, roughly a 40 percent increase from the year before.

``If sellers weren't succeeding,'' said Jeff Wilke, the chief executive
of Amazon's consumer business, ``they wouldn't be here.''

Jack Evans, a spokesman for the company, said that Amazon only succeeded
when sellers succeed, ``and claims to the contrary are wrong.''
Merchants can choose the products they sell, how they are priced and how
they fulfill the orders, he said.

The policy change that affected Tumi, Mr. Evans said, was to make sure
that Amazon had the best prices and availability for popular products.
He said that Tumi's prices were high when it sold through the middleman.

Amazon has faced harsh criticism in the past for displacing Main Street
brick-and-mortar retailers. Now, the diverging fortunes of Amazon and
many of the companies selling products on its own site are at the heart
of the antitrust scrutiny Amazon faces in Washington and Europe.
Investigators at the Federal Trade Commission and the House Judiciary
Committee are examining whether Amazon abuses its position as the
central online connection between people making products and those
buying them.

Amazon collects 27 cents of each dollar customers spend buying things
its merchants sell, a 42 percent jump from five years ago, according to
Instinet, a financial research firm. That does not include what
companies pay to place ads on Amazon, a business that Wall Street
considers as valuable as Nike.

The pennies add up. Last year, the profit from retail was so high that
it surprised even some senior leaders close to the business, according
to two of the people involved.

Thanks to the retail success, the company's profit exceeded its own Wall
Street projections by more than \$3 billion.

\hypertarget{investments-vs-contributors}{%
\subsection{Investments vs.
Contributors}\label{investments-vs-contributors}}

Image

Jeff Bezos in 1999, the year he opened Amazon's store to outside
sellers.Credit...Chris Carroll/Corbis via Getty Images

Jeff Bezos, Amazon's founder and chief executive, lumps the many parts
of the company into two buckets, according to the two people close to
the business. One bucket is investments, or bets on the future like
Alexa, its virtual assistant. The other is contributors, or the
profitable businesses that provide money for Amazon's investments.

To him, the retail operation is a contributor that can be squeezed for
cash.

Billions of dollars generated from selling products online go into
investments like Alexa, which has 10,000 employees working on it, and
the company's expensive Hollywood productions. And still, Amazon's
consumer businesses, including Alexa and other pricey projects, produced
\$5 billion in operating profit last year.

The financial success stems from a big strategy shift that was
underappreciated when Mr. Bezos made it two decades ago.

From the day the company started shipping orders in 1995, Amazon offered
customers products the same way as traditional retailers like Target,
buying them at wholesale and reselling them at a higher price. Four
years later, Mr. Bezos and his team decided that Amazon would also
\href{https://www.nytimes3xbfgragh.onion/1999/03/30/business/the-next-trick-for-amazoncom-auctions.html}{let
companies list} items on the site for a cut of the sale, more like eBay
and Alibaba. The change allowed Amazon to offer a wider variety of
products.

``We want to try and build a place where people can come to find and
discover anything that they might want to buy online,''
\href{https://charlierose.com/videos/28965}{Mr. Bezos said} that year.

The decision eventually turned Amazon into the one-stop shop it's known
as today. Shoppers could find not only well-known brands like Tide
detergent, but also obscure Christmas ornaments.

Initially, the move empowered sellers and gave them access to millions
of customers. They could ship their products however they wanted. And
they could set their own price.

Bit by bit, the sellers lost control.

\hypertarget{lured-into-shipping}{%
\subsection{Lured Into Shipping}\label{lured-into-shipping}}

Image

An Amazon fulfillment center in Baltimore, part of the company's
multibillion-dollar investment in warehouses.Credit...Gabriella Demczuk
for The New York Times

When Amazon opened its doors to sellers, the fulfillment industry ---
for storing, packing and shipping online orders --- was in its infancy.
Many top sellers on Amazon ran their own warehouses.

Seeing a competitive advantage in offering faster delivery times, Amazon
opened cavernous warehouses near major cities. Inside, workers navigated
endless rows to pick products from bins and pack them into boxes.

The expansion left Amazon with extra space to fill, and the company
turned to sellers. It pitched them on the idea of paying Amazon to store
and ship their products, even those sold on other sites.

James Thomson, a Canadian with a doctorate in marketing, managed a team
responsible for signing up sellers, leading them on tours of Amazon's
facilities near Reno, Nev., Phoenix and elsewhere. ``Look how vast this
is,'' he recalled telling sellers. ``Look at how we can easily absorb
your 10,000 orders a month.''

``You do have a bigger warehouse than mine,'' Mr. Thomson remembered
them saying, ``but I have good rates.''

Several years later, Amazon's focus changed, and so did its pitch.

In early 2011, only
\href{https://www.geekwire.com/2013/amazon-prime-10m-members-counting/}{a
few million} people were Prime members, paying \$79 a year for unlimited
two-day shipping. But Amazon knew those members spent far more on the
site. Executives wanted more people to sign up for Prime, and they
wanted to sell those customers even more stuff.

That year, Amazon began
\href{https://www.vox.com/recode/2019/5/3/18511544/amazon-prime-oral-history-jeff-bezos-one-day-shipping}{adding
more perks} to Prime. Most notable was unlimited video streaming of TV
shows like ``Mister Rogers' Neighborhood'' and movies like ``The Girl
With the Dragon Tattoo.''

As more people became members, products eligible for Prime shipping
became more popular. Amazon reminded sellers that if they used the
company's warehouses, their items would be Prime eligible, too.

``That is what we were selling,'' Mr. Thomson said.

Image

James Thomson, who ran a team that signed up sellers to use Amazon's
warehouses.Credit...Grant Hindsley for The New York Times

It worked. The number of sellers using Amazon's warehouses increased by
\href{https://www.sec.gov/Archives/edgar/data/1018724/000119312514137753/d702518dex991.htm}{65
percent in 2013}, according to a letter sent to investors. The company
has since spent billions of dollars to continue building out its
fulfillment network.

Mr. Bezos noted how intertwined sellers, warehouses and Prime had become
in a note to investors in 2015. ``At this point, I can't really think
about them separately,'' he wrote.

Amazon has since flipped back and forth over whether outside sellers
must use Amazon's warehouses to sell Prime products. But for most types
of goods, like pet supplies, cameras and baby gear, more than 85 percent
of the top-selling items ship out of Amazon's warehouses, according to
Jungle Scout, which provides data to Amazon sellers.

Amazon handles packing and shipping for the most popular products sold
on its site, even for products sold by outside sellers.

The 1,000 top-selling

products in each category

Orders sold and fulfilled

by outside sellers

Movies/TV

Baby

CDs/vinyl

Books

Kitchen/dining

Pet supplies

Camera/photo

Beauty/personal care

Computers/accessories

Orders sold

and fulfilled

by Amazon

Orders fulfilled

by Amazon

for outside

sellers

Clothing, shoes/jewelry

Video games

Home/kitchen

Toys/games

Musical instruments

Cell phones/accessories

Tools/home improvement

Industrial/scientific

Patio, lawn/garden

Grocery/gourmet food

Sports/outdoors

Automotive

Arts, crafts/sewing

Health/household

Appliances

Electronics

Office products

Software

20\%

40

60

80

Percentage of total

sales within each group

The 1,000 top-selling products in each category

Movies/TV

Baby

CDs/vinyl

Books

Kitchen/dining

Pet supplies

Camera/photo

Beauty/personal care

Computers/accessories

Orders fulfilled

by Amazon

for outside

sellers

Orders

sold and

fulfilled

by outside

sellers

Orders sold

and fulfilled

by Amazon

Clothing, shoes/jewelry

Video games

Home/kitchen

Toys/games

Musical instruments

Cell phones/accessories

Tools/home improvement

Industrial/scientific

Patio, lawn/garden

Grocery/gourmet food

Sports/outdoors

Automotive

Arts, crafts/sewing

Health/household

Appliances

Electronics

Office products

Software

20\%

40

60

80

Percentage of total sales within each group

Source: JungleScout

By Karl Russell

Amazon has surpassed DHL to become
\href{https://www.joc.com/international-logistics/logistics-providers/amazon-debuts-atop-global-3pl-rankings_20190420.html}{the
largest provider} of fulfillment and other logistics services in the
world, according to The Journal of Commerce, a trade publication.

Many sellers say that the company charges fair rates to fulfill Amazon
orders. But they say Amazon is charging them higher prices for other
services. For example, because the warehouses operate near capacity, the
company charges several times more than competitors to store items
before they ship out.

The costs can be several times higher for sellers who use Amazon to ship
orders made on other websites. Amazon charges \$13.80 for one-day
shipping on a T-shirt bought on a site other than Amazon, versus \$3.68
when bought on Amazon.

In addition, Amazon had let sellers pay \$1 to ship an order in a plain
brown box without the company's smile logo. But in 2016, the company
said it would use only Amazon boxes. Sellers were told they could take
their product back from Amazon's warehouses if they wanted. ``Return or
disposal fees will apply,'' it
\href{https://sellercentral.amazon.com/forums/t/termination-of-brand-neutral-packaging-service/165894}{wrote}
to sellers.

Amazon says that its logistics services are optional and a great value.
Sellers who choose to use it ``enjoy high-quality fulfillment services
that customers want,'' the company told Congress's investigators this
year.

The company says it offers lower costs on Amazon orders because it makes
other money from them, including commissions and advertising, that it
does not get for sales made on other websites.

Shoppers on other sites turn away when products are not available in two
days or less, said Karl Siebrecht, co-founder of Flexe, a start-up that
connects retailers with a network of fulfillment centers.

``It's new browser,'' he said. ``Amazon.com. Click. Buy. Done.''

\hypertarget{price-control}{%
\subsection{Price Control}\label{price-control}}

Image

Brandon Fishman, chief executive of VitaCup.Credit...John Francis Peters
for The New York Times

This summer, Brandon Fishman, the founder of VitaCup, a start-up that
infuses coffee with vitamins and nutrients, saw a promising opportunity.

Zulily, an e-commerce site that offers low prices in exchange for slower
shipping, wanted to list VitaCup's products 30 percent off for a short
time. It was a chance for Mr. Fishman, whose 35-employee company gets
the majority of its sales through Amazon and its own site, to reach new
customers.

But Amazon's software noticed the lower price and removed the bright
``Buy Now'' and ``Add to Cart'' buttons from its site. When those
buttons are gone, shoppers get a bland text link that says, ``Available
from these sellers'' and they must make more clicks to purchase an item.
Those extra clicks are often the difference between success and failure
for a seller.

Mr. Fishman's Amazon sales tumbled, and he emailed Zulily to quickly
take down the listing.

``I have told them about my rage many times,'' Mr. Fishman said of
Amazon. ``It has not changed them.''

Amazon has pushed to keep prices low since the day it opened. That has
become trickier as more sales came from outside sellers. According to
\href{https://www.ftc.gov/tips-advice/competition-guidance/guide-antitrust-laws/dealings-competitors/price-fixing}{antitrust
law}, each seller of goods should determine what to charge on its own.
To avoid problems, an in-house lawyer is typically present when internal
Amazon teams discuss pricing, according to two former employees.

In 2017, Amazon began reducing prices to match competitors; if the new
price was lower than the one requested by the sellers, Amazon paid the
difference. The company also alerted companies if their products were
cheaper elsewhere.

Still concerned about news reports that prices on Amazon weren't always
the lowest, the company tried another approach, the one that hit
VitaCup: removing the Buy Now and Add to Cart buttons when its software
detected lower prices. When those buttons disappear, sales tumble as
much as 75 percent, sellers say.

Executives at Amazon intended this as a tool to lower prices. The
company has told Congress that the buttons amount to an endorsement,
saying it only displays them on ``offers that it is confident will
present a great experience for its customers.''

But many brands raise their prices elsewhere to avoid losing the
buttons. Or they decide to list their product only on Amazon. That is
what happened to a health care supply company that worked with Jason
Boyce, who advises online sellers.

``My client cut off Walmart --- Walmart! --- because it was hurting
their Amazon business,'' Mr. Boyce said. ``If that's not monopoly power,
I don't know what is.''

Amazon said in statement that sellers ``have full control of their own
prices both on and off Amazon,'' and that the company helps them
maximize sales by advising them how to earn the Buy Now and Add to Cart
buttons.

\hypertarget{ads-by-necessity}{%
\subsection{Ads by Necessity}\label{ads-by-necessity}}

Image

VitaCup products at the company's headquarters.Credit...John Francis
Peters for The New York Times

The Zulily experience frustrated Mr. Fishman. But he boiled over after
another move by Amazon.

One morning in June, Mr. Fishman opened his Amazon app and typed
``VitaCup'' into the search bar at the top of the screen. On the results
page was an ad for Amazon's own line of coffee.

He had been paying Amazon almost \$200,000 a month for ads. Mr. Fishman
posted a screenshot on LinkedIn and raged.

``I have a major problem with this!!!'' he wrote.

For years, the question of whether Amazon should push ads on its site
generated fierce debate among senior managers and executives inside the
company, according to eight current and former Amazon employees. In
memos and fiery meetings, they disagreed on what was best for a company
that preached obsession with serving customers.

One camp believed that ads would erode customer trust, because shoppers
expected Amazon to show them popular products with strong reviews and a
good price.

The other camp saw ads as a cash machine Amazon could tap to drive down
prices and fund new innovations for customers. The financial potential
was obvious. When people shop online, they more often turn to Amazon
than Google to start their search, according to multiple studies. And
every brand wants to get in front of them.

Workers eventually got word that Mr. Bezos had settled the debate,
according to two senior employees. Mr. Bezos said that Amazon had two
options: Sell ads, and use the cash for investments. Or shun ads, and
get beaten by competitors.

Ads soon appeared at critical locations, in particular on the page that
pops up after a customer types a product into Amazon's search bar. Some
ads were rectangular blocks across the top of the page, and the top
several products listed in the search results were ads disguised as a
regular listing, aside from the word ``Sponsored'' in light gray.
Combined, they have at times filled almost the entire first screen.

Mr. Wilke said the internal hesitation to ads was overcome by the
results.

``It turned out they worked,'' he said. ``And by worked, I mean the ads
help customers find what they're looking for. And the reason we know
that is cause they buy more stuff.''

But it added another cost for companies. Ranking high is essential to
driving sales on the site. Competitors raced to place ads to ensure a
prominent spot.

Out of antitrust concerns, company lawyers prohibit employees and
advertising companies it works with from bragging that Amazon is where
most people search for products online, according to two people who were
warned about this.

Quartile, among the largest of a new breed of companies that help brands
navigate Amazon advertising, tested the importance of the ads last year.
It stopped running ads for 750 popular products. Immediately, sales
shrank by 24 percent.

The effect then cascaded. That's because the fewer recent sales a
product has, including sales driven by ads, the lower it ranks on the
site. At the end of 10 weeks, sales of the products without ads had
tumbled 55 percent.

``It's increasingly pay-to-play,'' said Melissa Burdick, a 10-year
Amazon veteran who now advises major consumer brands.

Amazon said its ads were optional and the majority of sellers built
their businesses without them.

John Denny, who ran e-commerce for the drink company Bai, said brands
used to believe that if they had a great product, it would show up in
the search results, and sales would follow.

``Those days are over,'' Mr. Denny said. ``There are no lightning
strikes on Amazon any more.''

\hypertarget{resigned-partners}{%
\subsection{Resigned Partners}\label{resigned-partners}}

Image

Bernie Thompson, founder of Plugable Technologies.Credit...Grant
Hindsley for The New York Times

A decade ago, Mr. Thompson, a former Microsoft software developer,
recognized a big market for computer accessories like computer docking
stations and cables. He started Plugable and betted big that depending
on Amazon would turn his idea into a business.

It worked. In 2016, Mr. Bezos
\href{https://www.sec.gov/Archives/edgar/data/1018724/000119312516530910/d168744dex991.htm}{highlighted
Mr. Thompson} when talking about the success of sellers in his annual
letter to investors. Amazon posted a video about Plugable
\href{https://services.amazon.com/stories.html}{on its website} to
attract new sellers.

``He has a history of good performance metrics, and an absence of things
like safety and authenticity complaints,'' Chris McCabe, a former Amazon
fraud investigator, said in an interview.

But in the last couple of years, as rules shifted and his profit shrank,
Mr. Thompson began warning people that working with Amazon had become
increasingly difficult.

He took his concerns to Amazon this summer, giving a 20-slide
presentation to a senior executive at the company's Seattle
headquarters. On slide No. 6, Mr. Thompson laid out his nightmare:
Amazon cutting off sales of his best seller, a laptop docking station
that is frequently one of the 100 most popular electronics products on
the site.

His plea to the executive was simple. ``No surprises,'' he said.

He got surprised.

One Sunday in July, he got an email saying that Amazon had removed the
docking stations. Amazon said it was because of complaints that
Plugable's products had not matched the condition described on the site.

Other docking stations, including one made by Amazon, filled the void
online.

Image

In July, Amazon removed Plugable Technologies' best-selling item, a
laptop dock, from its marketplace. Amazon said it was because of
complaints that Plugable's products had not matched the condition
described on the site.Credit...Grant Hindsley for The New York Times

Mr. Thompson scrambled, contacting two high-level managers he knew and
his account manager, who Amazon charges him \$5,000 a month to have.
None of them could fix it.

He and other staff members dug through customer feedback and returns.
They found only outstanding reviews, said Gary Zeller, one of Mr.
Thompson's deputies.

``There was nothing borderline about it,'' Mr. Zeller said.

After four days and at least \$100,000 in lost sales, the listing went
back up. Mr. Thompson said he still did not understand what ignited the
problem.

Amazon declined to comment on Plugable. Mr. Wilke said that the
company's future depended on policing the site without harming
well-meaning merchants.

``We have a strong incentive to be as accurate as possible in
identifying bad actors, make very few mistakes when we're wrong, on
giving second chances to people who make an honest mistake,'' he said.

Mr. Thompson is now looking for new ways to make money. But Amazon
accounts for roughly 90 percent of electronics sales online, according
to market research. His business at Walmart and eBay, the next largest
online retailers, are less than 5 percent of his revenue.

In September, Plugable hired two people to sell directly to
corporations.

``We really built the company on Amazon,'' Mr. Thompson said. ``We have
no regrets about doing that. But today our focus has to be getting
diversification off Amazon.''

He said he understood what he was up against.

``We are dealing with a partner,'' he said, ``who can and will disrupt
us for unpredictable reasons at any time.''

Advertisement

\protect\hyperlink{after-bottom}{Continue reading the main story}

\hypertarget{site-index}{%
\subsection{Site Index}\label{site-index}}

\hypertarget{site-information-navigation}{%
\subsection{Site Information
Navigation}\label{site-information-navigation}}

\begin{itemize}
\tightlist
\item
  \href{https://help.nytimes3xbfgragh.onion/hc/en-us/articles/115014792127-Copyright-notice}{©~2020~The
  New York Times Company}
\end{itemize}

\begin{itemize}
\tightlist
\item
  \href{https://www.nytco.com/}{NYTCo}
\item
  \href{https://help.nytimes3xbfgragh.onion/hc/en-us/articles/115015385887-Contact-Us}{Contact
  Us}
\item
  \href{https://www.nytco.com/careers/}{Work with us}
\item
  \href{https://nytmediakit.com/}{Advertise}
\item
  \href{http://www.tbrandstudio.com/}{T Brand Studio}
\item
  \href{https://www.nytimes3xbfgragh.onion/privacy/cookie-policy\#how-do-i-manage-trackers}{Your
  Ad Choices}
\item
  \href{https://www.nytimes3xbfgragh.onion/privacy}{Privacy}
\item
  \href{https://help.nytimes3xbfgragh.onion/hc/en-us/articles/115014893428-Terms-of-service}{Terms
  of Service}
\item
  \href{https://help.nytimes3xbfgragh.onion/hc/en-us/articles/115014893968-Terms-of-sale}{Terms
  of Sale}
\item
  \href{https://spiderbites.nytimes3xbfgragh.onion}{Site Map}
\item
  \href{https://help.nytimes3xbfgragh.onion/hc/en-us}{Help}
\item
  \href{https://www.nytimes3xbfgragh.onion/subscription?campaignId=37WXW}{Subscriptions}
\end{itemize}
