Sections

SEARCH

\protect\hyperlink{site-content}{Skip to
content}\protect\hyperlink{site-index}{Skip to site index}

\href{/section/climate}{Climate}\textbar{}Science Under Attack: How
Trump Is Sidelining Researchers and Their Work

\url{https://nyti.ms/2MyL8Yw}

\begin{itemize}
\item
\item
\item
\item
\item
\end{itemize}

\hypertarget{climate-and-environment}{%
\subsubsection{\texorpdfstring{\href{https://www.nytimes3xbfgragh.onion/section/climate?name=styln-climate\&region=TOP_BANNER\&variant=undefined\&block=storyline_menu_recirc\&action=click\&pgtype=Article\&impression_id=c6ec2a20-e3a0-11ea-8244-451f6a328e47}{Climate
and
Environment}}{Climate and Environment}}\label{climate-and-environment}}

\begin{itemize}
\tightlist
\item
  \href{https://www.nytimes3xbfgragh.onion/2020/08/17/climate/alaska-oil-drilling-anwr.html?name=styln-climate\&region=TOP_BANNER\&variant=undefined\&block=storyline_menu_recirc\&action=click\&pgtype=Article\&impression_id=c6ec5130-e3a0-11ea-8244-451f6a328e47}{Arctic
  Refuge}
\item
  \href{https://www.nytimes3xbfgragh.onion/interactive/2020/climate/trump-environment-rollbacks.html?name=styln-climate\&region=TOP_BANNER\&variant=undefined\&block=storyline_menu_recirc\&action=click\&pgtype=Article\&impression_id=c6ec5131-e3a0-11ea-8244-451f6a328e47}{Trump's
  Changes}
\item
  \href{https://www.nytimes3xbfgragh.onion/interactive/2020/04/19/climate/climate-crash-course-1.html?name=styln-climate\&region=TOP_BANNER\&variant=undefined\&block=storyline_menu_recirc\&action=click\&pgtype=Article\&impression_id=c6ec5132-e3a0-11ea-8244-451f6a328e47}{Climate
  101}
\item
  \href{https://www.nytimes3xbfgragh.onion/interactive/2018/08/30/climate/how-much-hotter-is-your-hometown.html?name=styln-climate\&region=TOP_BANNER\&variant=undefined\&block=storyline_menu_recirc\&action=click\&pgtype=Article\&impression_id=c6ec5133-e3a0-11ea-8244-451f6a328e47}{Is
  Your Hometown Hotter?}
\end{itemize}

\includegraphics{https://static01.graylady3jvrrxbe.onion/images/2019/12/29/science/29CLI-SCIENCE-epa1/merlin_166206960_8d393c85-cb38-41de-84e3-053b2c70c014-articleLarge.jpg?quality=75\&auto=webp\&disable=upscale}

\hypertarget{science-under-attack-how-trump-is-sidelining-researchers-and-their-work}{%
\section{Science Under Attack: How Trump Is Sidelining Researchers and
Their
Work}\label{science-under-attack-how-trump-is-sidelining-researchers-and-their-work}}

The Environmental Protection Agency in Washington.Credit...Victor J.
Blue for The New York Times

Supported by

\protect\hyperlink{after-sponsor}{Continue reading the main story}

\href{https://www.nytimes3xbfgragh.onion/by/brad-plumer}{\includegraphics{https://static01.graylady3jvrrxbe.onion/images/2018/02/20/multimedia/author-brad-plumer/author-brad-plumer-thumbLarge.jpg}}\href{https://www.nytimes3xbfgragh.onion/by/coral-davenport}{\includegraphics{https://static01.graylady3jvrrxbe.onion/images/2018/10/03/multimedia/author-coral-davenport/author-coral-davenport-thumbLarge-v2.png}}

By \href{https://www.nytimes3xbfgragh.onion/by/brad-plumer}{Brad Plumer}
and \href{https://www.nytimes3xbfgragh.onion/by/coral-davenport}{Coral
Davenport}

\begin{itemize}
\item
  Dec. 28, 2019
\item
  \begin{itemize}
  \item
  \item
  \item
  \item
  \item
  \end{itemize}
\end{itemize}

WASHINGTON --- In just three years, the Trump administration has
diminished the role of science in federal policymaking while halting or
disrupting research projects nationwide, marking a transformation of the
federal government whose effects, experts say, could reverberate for
years.

Political appointees have shut down government studies, reduced the
influence of scientists over regulatory decisions and in some cases
pressured researchers not to speak publicly. The administration has
particularly challenged scientific findings related to the environment
and public health opposed by industries such as oil drilling and coal
mining. It has also impeded research around human-caused climate change,
which President Trump has dismissed despite a global scientific
consensus.

But the erosion of science reaches well beyond the environment and
climate: In San Francisco, a study of the effects of chemicals on
pregnant women has stalled after federal funding abruptly ended. In
Washington, D.C., a scientific committee that provided expertise in
defending against invasive insects
\href{https://www.doi.gov/invasivespecies/isac-resources}{has been
disbanded}. In Kansas City, Mo., the hasty relocation of two
agricultural agencies that fund crop science and study the economics of
farming has led to an exodus of employees and delayed hundreds of
millions of dollars in research.

``The disregard for expertise in the federal government is worse than
it's ever been,'' said Michael Gerrard, director of the Sabin Center for
Climate Change Law at Columbia University, which
\href{https://climate.law.columbia.edu/Silencing-Science-Tracker}{has
tracked more than 200 reports of Trump administration efforts} to
restrict or misuse science since 2017. ``It's pervasive.''

Hundreds of scientists, many of whom say they are dismayed at seeing
their work undone, are departing.

Among them is Matthew Davis, a biologist whose research on the health
risks of mercury to children underpinned the
\href{https://green.blogs.nytimes3xbfgragh.onion/2011/12/21/e-p-a-announces-mercury-limits/}{first
rules cutting mercury emissions from coal power plants}. But last year,
with a new baby of his own, he was asked to help support a rollback of
those same rules. ``I am now part of defending this darker, dirtier
future,'' he said.

This year, after a decade at the Environmental Protection Agency, Mr.
Davis left.

``Regulations come and go, but the thinning out of scientific capacity
in the government will take a long time to get back,'' said Joel
Clement, a former top climate-policy expert at the Interior Department
\href{https://www.washingtonpost.com/opinions/im-a-scientist-the-trump-administration-reassigned-me-for-speaking-up-about-climate-change/2017/07/19/389b8dce-6b12-11e7-9c15-177740635e83_story.html}{who
quit in 2017} after being reassigned to a job collecting oil and gas
royalties. He is now at the Union of Concerned Scientists, an advocacy
group.

Mr. Trump has consistently said that government regulations have stifled
businesses and thwarted some of the administration's core goals, such as
increasing fossil-fuel production. Many of the starkest confrontations
with federal scientists have involved issues like environmental
oversight and energy extraction --- areas where industry groups have
argued that regulators have gone too far in the past.

``Businesses are finally being freed of Washington's overreach, and the
American economy is flourishing as a result,'' a
\href{https://www.whitehouse.gov/briefings-statements/president-donald-j-trump-following-promise-cut-burdensome-red-tape-unleash-american-economy/}{White
House statement said last year}. Asked about the role of science in
policymaking, officials from the White House declined to comment on the
record.

The administration's efforts to cut certain research projects also
reflect a longstanding conservative position that some scientific work
can be performed cost-effectively by the private sector, and taxpayers
shouldn't be asked to foot the bill. ``Eliminating wasteful spending,
some of which has nothing to do with studying the science at all, is
smart management, not an attack on science,'' two analysts at the
conservative Heritage Foundation
\href{https://www.heritage.org/energy-economics/commentary/climate-budget-cuts-are-smart-management-not-attack-science-0}{wrote
in 2017} of the administration's proposals to eliminate various climate
change and clean energy programs.

\includegraphics{https://static01.graylady3jvrrxbe.onion/images/2019/12/29/climate/29CLI-SCIENCE-dorian/00CLI-SCIENCE-dorian-articleLarge.jpg?quality=75\&auto=webp\&disable=upscale}

Industry groups have expressed support for some of the moves, including
\href{https://www.nytimes3xbfgragh.onion/2018/03/26/climate/epa-scientific-transparency-honest-act.html}{a
contentious E.P.A. proposal} to put new constraints on the use of
scientific studies in the name of transparency. The American Chemistry
Council, a chemical trade group, praised the proposal by saying, ``The
goal of providing more transparency in government and using the best
available science in the regulatory process should be ideals we all
embrace.''

In some cases, the administration's efforts to roll back government
science have been thwarted. Each year, Mr. Trump has proposed sweeping
budget cuts at a variety of federal agencies like the National
Institutes of Health, the Department of Energy and the National Science
Foundation. But Congress has the final say over budget levels and
lawmakers from both sides of the aisle have rejected the cuts.

For instance, in supporting funding for the Department of Energy's
national laboratories, Senator Lamar Alexander, Republican of Tennessee,
recently said, ``it allows us to take advantage of the United States'
secret weapon, our extraordinary capacity for basic research.''

As a result, many science programs continue to thrive, including space
exploration at NASA and medical research at the National Institutes of
Health, where the budget has increased more than 12 percent since Mr.
Trump took office and where researchers
\href{https://www.nytimes3xbfgragh.onion/2018/01/06/us/politics/congress-medical-research-health-care.html}{continue
to make advances} in areas like molecular biology and genetics.

Nevertheless, in other areas, the administration has managed to chip
away at federal science.

At the E.P.A., for instance,
\href{https://fas.org/sgp/crs/misc/IF11153.pdf}{staffing has fallen} to
its lowest levels in at least a decade. More than two-thirds of
respondents to
\href{https://www.ucsusa.org/sites/default/files/attach/2018/08/science-under-trump-report.pdf}{a
survey of federal scientists} across 16 agencies said that hiring
freezes and departures made it harder to conduct scientific work. And in
June, the White House
\href{https://www.whitehouse.gov/presidential-actions/executive-order-evaluating-improving-utility-federal-advisory-committees/}{ordered
agencies to cut by one-third} the number of federal advisory boards that
provide technical advice.

The White House said it aimed to eliminate committees that were no
longer necessary. Panels cut so far
\href{https://eos.org/articles/white-house-order-shutters-some-key-advisory-committees}{had
focused on issues} including invasive species and electric grid
innovation.

\href{https://www.nytimes3xbfgragh.onion/section/climate?action=click\&pgtype=Article\&state=default\&region=MAIN_CONTENT_1\&context=storylines_keepup}{}

\hypertarget{climate-and-environment-}{%
\subsubsection{Climate and Environment
›}\label{climate-and-environment-}}

\hypertarget{keep-up-on-the-latest-climate-news}{%
\paragraph{Keep Up on the Latest Climate
News}\label{keep-up-on-the-latest-climate-news}}

Updated Aug. 18, 2020

Here's what you need to know this week:

\begin{itemize}
\item
  \begin{itemize}
  \tightlist
  \item
    Five automakers
    \href{https://www.nytimes3xbfgragh.onion/2020/08/17/climate/california-automakers-pollution.html?action=click\&pgtype=Article\&state=default\&region=MAIN_CONTENT_1\&context=storylines_keepup}{sealed
    a binding agreement} with California to follow the state's stricter
    tailpipe emissions rules.
  \item
    The Trump
    administration\href{https://www.nytimes3xbfgragh.onion/2020/08/13/climate/trump-methane.html?action=click\&pgtype=Article\&state=default\&region=MAIN_CONTENT_1\&context=storylines_keepup}{eliminated
    a major methane rule}, even as leaks are worsening, in a decision
    that researchers warned ignored science.
  \item
    Climate change leaders said
    \href{https://www.nytimes3xbfgragh.onion/2020/08/12/climate/kamala-harris-environmental-justice.html?action=click\&pgtype=Article\&state=default\&region=MAIN_CONTENT_1\&context=storylines_keepup}{the
    vice-presidential choice of Kamala Harris} signaled that Democrats
    will have a focus on environmental justice.
  \end{itemize}
\end{itemize}

At a time when the United States is pulling back from world leadership
in other areas like human rights or diplomatic accords, experts warn
that the retreat from science is no less significant. Many of the
achievements of the past century that helped make the United States an
envied global power, including gains in life expectancy, lowered air
pollution and increased farm productivity are the result of the kinds of
government research now under pressure.

``When we decapitate the government's ability to use science in a
professional way, that increases the risk that we start making bad
decisions, that we start missing new public health risks,'' said Wendy
E. Wagner, a professor of law at the University of Texas at Austin who
studies the use of science by policymakers.

Skirmishes over the use of science in making policy occur in all
administrations: Industries routinely push back against health studies
that could justify stricter pollution rules, for example. And scientists
often gripe about inadequate budgets for their work. But many experts
say that current efforts to challenge research findings go well beyond
what has been done previously.

In \href{https://science.sciencemag.org/content/362/6415/636.summary}{an
article published} in the journal Science last year, Ms. Wagner wrote
that some of the Trump administration's moves, like a policy to restrict
certain academics from the E.P.A.'s Science Advisory Board or the
proposal to limit the types of research that can be considered by
environmental regulators, ``mark a sharp departure with the past.''
Rather than isolated battles between political officials and career
experts, she said, these moves are an attempt to legally constrain how
federal agencies use science in the first place.

Some clashes with scientists have sparked public backlash, as when Trump
officials
\href{https://www.nytimes3xbfgragh.onion/2019/09/11/us/politics/trump-alabama-noaa.html}{pressured
the nation's weather forecasting agency} to support the president's
erroneous assertion this year that Hurricane Dorian threatened Alabama.

But others have garnered little notice despite their significance.

Image

The Interior Department headquarters in Washington.Credit...Victor J.
Blue for The New York Times

This year, for instance, the National Park Service's principal climate
change scientist, Patrick Gonzalez, received a ``cease and desist''
letter from supervisors after testifying to Congress about the risks
that global warming posed to national parks.

``I saw it as attempted intimidation,'' said Dr. Gonzalez, who added
that he was speaking in his capacity as an associate adjunct professor
at the University California, Berkeley, a position he also holds. ``It's
interference with science and hinders our work.''

\hypertarget{curtailing-scientific-programs}{%
\subsection{Curtailing Scientific
Programs}\label{curtailing-scientific-programs}}

Even though Congress hasn't gone along with Mr. Trump's proposals for
\href{https://www.whitehouse.gov/sites/whitehouse.gov/files/omb/budget/fy2018/msar.pdf}{budget
cuts at scientific agencies}, the administration has still found ways to
advance its goals.

One strategy: eliminate individual research projects not explicitly
protected by Congress.

For example, just months after Mr. Trump's election, the Commerce
Department
\href{https://www.washingtonpost.com/news/energy-environment/wp/2017/08/20/the-trump-administration-just-disbanded-a-federal-advisory-committee-on-climate-change/}{disbanded}
a 15-person scientific committee that had explored how to make National
Climate Assessments, the congressionally mandated studies of the risks
of climate change, more useful to local officials. It also
\href{http://www.osec.doc.gov/opog/dmp/revocations/doo35_7.html}{closed
its Office of the Chief Economist}, which for decades had conducted
wide-ranging research on topics like the economic effects of natural
disasters. Similarly, the Interior Department has withdrawn funding for
its Landscape Conservation Cooperatives, 22 regional research centers
\href{https://www.nap.edu/read/21829/chapter/8}{that tackled issues}
like habitat loss and wildfire management. While California and Alaska
used state money to keep their centers open, 16 of 22 remain in limbo.

A Commerce Department official said the climate committee it
discontinued had not produced a report, and highlighted other efforts to
promote science, such as a major upgrade of the nation's weather models.

An Interior Department official said the agency's decisions ``are solely
based on the facts and grounded in the law,'' and that the agency would
continue to pursue other partnerships to advance conservation science.

Research that potentially posed an obstacle to Mr. Trump's promise to
expand fossil-fuel production was halted, too. In 2017, Interior
officials
\href{https://www.nytimes3xbfgragh.onion/2017/08/21/climate/coal-mining-health-study-is-halted-by-interior-department.html}{canceled
a \$1 million study} by the National Academies of Sciences, Engineering,
and Medicine on the health risks of ``mountaintop removal'' coal mining
in places like West Virginia.

Mountaintop removal is as dramatic as it sounds --- a hillside is
blasted with explosives and the remains are excavated --- but the health
consequences still aren't fully understood. The process can kick up coal
dust and send heavy metals into waterways, and a number of studies
\href{https://www.ncbi.nlm.nih.gov/pubmed/28738262}{have suggested
links} to health problems like kidney disease and birth defects.

``The industry was pushing back on these studies,'' said Joseph
Pizarchik, an Obama-era mining regulator who commissioned the
now-defunct study. ``We didn't know what the answer would be,'' he said,
``but we needed to know: Was the government permitting coal mining that
was poisoning people, or not?''

Image

Credit...Brendan Smialowski/Agence France-Presse --- Getty Images

While coal mining has declined in recent years,
\href{https://skytruth.org/2019/11/new-data-available-on-the-footprint-of-surface-mining-in-central-appalachia/}{satellite
data shows} that at least 60 square miles in Appalachia have been newly
mined since 2016. ``The study is still as important today as it was five
years ago,'' Mr. Pizarchik said.

\hypertarget{the-cost-of-lost-research}{%
\subsection{The Cost of Lost Research}\label{the-cost-of-lost-research}}

The cuts can add up to significant research setbacks.

For years, the E.P.A. and the National Institute of Environmental Health
Sciences had jointly funded
\href{https://www.niehs.nih.gov/research/supported/centers/prevention/grantees/index.cfm}{13
children's health centers nationwide} that studied, among other things,
the effects of pollution on children's development. This year, the
E.P.A. ended its funding.

At the University of California, San Francisco, one such center
\href{https://prhe.ucsf.edu/childrens-center}{has been studying} how
industrial chemicals~such as flame retardants in furniture could affect
placenta and fetal development. Key aspects of the research have now
stopped.

``The longer we go without funding, the harder it is to start that
research back up,'' said Tracey Woodruff, who directs the center.

In a statement, the E.P.A. said it anticipated future opportunities to
fund children's health research.

At the Department of Agriculture, Secretary of Agriculture Sonny Perdue
\href{https://www.usda.gov/media/press-releases/2019/06/13/secretary-perdue-announces-kansas-city-region-location-ers-and-nifa}{announced}
in June he would relocate two key research agencies to Kansas City from
Washington: The National Institute of Food and Agriculture, a scientific
agency that funds university research on topics like how to breed cattle
and corn that can better tolerate drought conditions, and the Economic
Research Service, whose economists produce studies for policymakers on
farming trends, trade and rural America.

Nearly 600 employees had less than four months to decide whether to
uproot and move. Most couldn't or wouldn't, and two-thirds of those
facing transfer left their jobs.

Image

The Department of Agriculture.Credit...Victor J. Blue for The New York
Times

In August, Mick Mulvaney, the acting White House chief of staff,
appeared to celebrate the departures.

``It's nearly impossible to fire a federal worker,'' he said
\href{https://www.washingtonpost.com/science/2019/08/05/usda-science-agencies-relocation-may-have-violated-law-inspector-general-report-says/}{in
videotaped remarks} at a Republican Party gala in South Carolina. ``But
by simply saying to people, `You know what, we're going to take you
outside the bubble, outside the Beltway, outside this liberal haven of
Washington, D.C., and move you out in the real part of the country,' and
they quit. What a wonderful way to sort of streamline government and do
what we haven't been able to do for a long time.''

The White House declined to comment on Mr. Mulvaney's speech.

The exodus has led to upheaval.

At the Economic Research Service, dozens of planned studies into topics
like dairy industry consolidation and pesticide use have been delayed or
disrupted. ``You can name any topic in agriculture and we've lost an
expert,'' said Laura Dodson, an economist and acting vice president of
the union representing agency employees.

The National Institute of Food and Agriculture manages \$1.7 billion in
grants that fund research on issues like food safety or techniques that
help farmers improve their productivity. The staff loss, employees say,
has held up hundreds of millions of dollars in funding, such as planned
research into pests and diseases afflicting grapes, sweet potatoes and
fruit trees.

Former employees say they remain skeptical that the agencies could be
repaired quickly. ``It will take 5 to 10 years to rebuild,'' said Sonny
Ramaswamy, who until 2018 directed the National Institute of Food and
Agriculture.

Mr. Perdue said the moves would save money and put the offices closer to
farmers. ``We did not undertake these relocations lightly,'' he said in
a statement. A Department of Agriculture official added that both
agencies were pushing to continue their work, but acknowledged that some
grants could be delayed by months.

\hypertarget{questioning-the-science-itself}{%
\subsection{Questioning the Science
Itself}\label{questioning-the-science-itself}}

In addition to shutting down some programs, there have been notable
instances where the administration has challenged established scientific
research. Early on, as it started
\href{https://www.nytimes3xbfgragh.onion/interactive/2019/climate/trump-environment-rollbacks.html}{rolling
back regulations} on industry, administration officials began
questioning research findings underpinning those regulations.

In 2017, aides to Scott Pruitt, the E.P.A. administrator at the time,
\href{https://www.nytimes3xbfgragh.onion/2017/08/11/us/politics/scott-pruitt-epa.html}{told
the agency's economists} to redo an analysis of wetlands protections
that had been used to help defend an Obama-era clean-water rule. Instead
of concluding that the protections would provide more than \$500 million
in economic benefits, they were told to list the benefits as
unquantifiable, according to Elizabeth Southerland, who retired in 2017
from a 30-year career at the E.P.A., finishing as a senior official in
its water office.

``It's not unusual for a new administration to come in and change policy
direction,'' Dr. Southerland said. ``But typically you would look for
new studies and carefully redo the analysis. Instead they were sending a
message that all the economists, scientists, career staff in the agency
were irrelevant.''

Image

Sonny Perdue, the secretary of Agriculture.Credit...T.J. Kirkpatrick for
The New York Times

Internal documents show that political officials at the E.P.A. have
overruled the agency's career experts on several occasions, including in
a move
\href{https://www.nytimes3xbfgragh.onion/2019/05/08/climate/epa-asbestos-rule-scientists.html}{to
regulate asbestos more lightly}, in a decision
\href{https://www.nytimes3xbfgragh.onion/2018/08/24/business/epa-pesticides-studies-epidemiology.html}{not
to ban the pesticide chlorpyrifos} and in a determination that
\href{https://www.nytimes3xbfgragh.onion/2019/05/24/climate/epa-pruitt-wisconsin-foxconn.html}{parts
of Wisconsin were in compliance with smog standards}. The Interior
Department
\href{https://www.nytimes3xbfgragh.onion/2019/09/28/climate/bernhardt-shasta-dam.html}{sidelined
its own legal and environmental analyses} in advancing a proposal to
raise the Shasta Dam in California.

Michael Abboud, an E.P.A. spokesman, disputed Dr. Southerland's account
in an emailed response, saying ``It is not true.''

The E.P.A. is now finalizing a narrower version of the Obama-era water
rule, which in its earlier form had prompted outrage from thousands of
farmers and ranchers across the country who saw it as overly
restrictive.

``E.P.A. under President Trump has worked to put forward the strongest
regulations to protect human health and the environment,'' Mr. Abboud
said, noting that several Obama administration rules had been held up in
court and needed revision. ``As required by law E.P.A. has always and
will continue to use the best available science when developing rules,
regardless of the claims of a few federal employees.''

Past administrations have, to varying degrees, disregarded scientific
findings that conflicted with their priorities. In 2011, President
Obama's top health official
\href{https://www.nytimes3xbfgragh.onion/2011/12/09/us/obama-backs-aides-stance-on-morning-after-pill.html}{overruled
experts at the Food and Drug Administration} who had concluded that
over-the-counter emergency contraceptives were safe for minors.

But in the Trump administration, the scope is wider. Many top government
positions, including at the E.P.A. and the Interior Department,
\href{https://www.nytimes3xbfgragh.onion/2019/10/18/climate/trump-cabinet-lobbyists.html}{are
now occupied by former lobbyists} connected to the industries that those
agencies oversee.

Scientists and health experts have singled out two moves they find
particularly concerning. Since 2017, the E.P.A.
\href{https://www.nytimes3xbfgragh.onion/2017/10/31/climate/pruitt-epa-science-advisory-boards.html}{has
moved to restrict} certain academics from sitting on its Science
Advisory Board,~which provides scrutiny of agency science, and has
instead increased the number of appointees connected with industry.

And, in a potentially far-reaching move, the E.P.A.
\href{https://www.nytimes3xbfgragh.onion/2018/03/26/climate/epa-scientific-transparency-honest-act.html}{has
proposed a rule to limit regulators from using scientific research}
unless the underlying raw data can be made public. Industry groups like
the Chamber of Commerce
\href{https://www.globalenergyinstitute.org/proposed-rulemaking-strengthening-transparency-regulatory-science}{have
argued} that some agency rules are based on science that can't be fully
scrutinized by outsiders. But dozens of scientific organizations
\href{https://mcmprodaaas.s3.amazonaws.com/s3fs-public/EPA\%20Transparency\%20Rule\%20FINAL.pdf?oNbdIjRo8Ick2LxdMeWaqWuYu4NM3unc}{have
warned} that the proposal in its current form could prevent the E.P.A.
from considering a vast array of research~on issues like the dangers of
air pollution if, for instance, they are based on confidential health
data.

Image

The Commerce Department.Credit...Victor J. Blue for The New York Times

``The problem is that rather than allowing agency scientists to use
their judgment and weigh the best available evidence, this could put
political constraints on how science enters the decision-making process
in the first place,'' said Ms. Wagner, the University of Texas law
professor.

The E.P.A. says its proposed rule is intended to make the science that
underpins potentially costly regulations more transparent. ``By
requiring transparency,'' said Mr. Abboud, the agency spokesman,
``scientists will be required to publish hypothesis and experimental
data for other scientists to review and discuss, requiring the science
to withstand skepticism and peer review.''

\hypertarget{an-exodus-of-expertise}{%
\subsection{An Exodus of Expertise}\label{an-exodus-of-expertise}}

``In the past, when we had an administration that was not very
pro-environment, we could still just lay low and do our work,'' said
Betsy Smith, a climate scientist with more than 20 years of experience
at the E.P.A. who in 2017 saw her long-running study of the effects of
climate change on major ports get canceled.

``Now we feel like the E.P.A. is being run by the fossil fuel
industry,'' she said. ``It feels like a wholesale attack.''

After her project was killed, Dr. Smith resigned.

The loss of experienced scientists can erase years or decades of
``institutional memory,'' said Robert J. Kavlock, a toxicologist who
retired in October 2017 after working at the E.P.A. for 40 years, most
recently as acting assistant administrator for the agency's Office of
Research and Development.

His former office, which researches topics like air pollution and
chemical testing, has lost 250 scientists and technical staff members
since Mr. Trump came to office, while hiring 124. Those who have
remained in the office of roughly 1,500 people continue to do their
work, Dr. Kavlock said, but are not going out of their way to promote
findings on lightning-rod topics like climate change.

``You can see that they're trying not to ruffle any feathers,'' Dr.
Kavlock said.

The same can't be said of Patrick Gonzalez, the National Park Service's
principal climate change scientist, whose work involves helping national
parks protect against damages from rising temperatures.

In February, Dr. Gonzalez
\href{https://naturalresources.house.gov/download/patrick-gonzalez-testimony}{testified
before Congress} about the risks of global warming, saying he was
speaking in his capacity as an associate adjunct professor at the
University of California, Berkeley. He is also using his Berkeley
affiliation to participate as a co-author on a coming report by the
Intergovernmental Panel on Climate Change, a United Nations body that
synthesizes climate science for world leaders.

But in March, shortly after testifying, Dr. Gonzalez's supervisor at the
National Park Service sent the cease-and-desist letter warning him that
his Berkeley affiliation was not separate from his government work and
that his actions were violating agency policy. Dr. Gonzalez said he
viewed the letter as an attempt to deter him from speaking out.

The Interior Department, asked to comment, said the letter did not
indicate an intent to sanction Dr. Gonzalez and that he was free to
speak as a private citizen.

Dr. Gonzalez, with the support of Berkeley, continues to warn about the
dangers of climate change and work with the United Nations climate
change panel using his vacation time, and he spoke again to Congress in
June. ``I'd like to provide a positive example for other scientists,''
he said.

Still, he noted that not everyone may be in a position to be similarly
outspoken. ``How many others are not speaking up?'' Dr. Gonzalez said.

Image

Credit...Ting Shen/Reuters

\emph{For more climate news sign up for}
\href{https://www.nytimes3xbfgragh.onion/newsletters/climate-change}{\emph{the
Climate Fwd: newsletter}} \emph{or follow}
\href{https://twitter.com/nytclimate}{\emph{@NYTClimate on
Twitter}}\emph{.}

Advertisement

\protect\hyperlink{after-bottom}{Continue reading the main story}

\hypertarget{site-index}{%
\subsection{Site Index}\label{site-index}}

\hypertarget{site-information-navigation}{%
\subsection{Site Information
Navigation}\label{site-information-navigation}}

\begin{itemize}
\tightlist
\item
  \href{https://help.nytimes3xbfgragh.onion/hc/en-us/articles/115014792127-Copyright-notice}{©~2020~The
  New York Times Company}
\end{itemize}

\begin{itemize}
\tightlist
\item
  \href{https://www.nytco.com/}{NYTCo}
\item
  \href{https://help.nytimes3xbfgragh.onion/hc/en-us/articles/115015385887-Contact-Us}{Contact
  Us}
\item
  \href{https://www.nytco.com/careers/}{Work with us}
\item
  \href{https://nytmediakit.com/}{Advertise}
\item
  \href{http://www.tbrandstudio.com/}{T Brand Studio}
\item
  \href{https://www.nytimes3xbfgragh.onion/privacy/cookie-policy\#how-do-i-manage-trackers}{Your
  Ad Choices}
\item
  \href{https://www.nytimes3xbfgragh.onion/privacy}{Privacy}
\item
  \href{https://help.nytimes3xbfgragh.onion/hc/en-us/articles/115014893428-Terms-of-service}{Terms
  of Service}
\item
  \href{https://help.nytimes3xbfgragh.onion/hc/en-us/articles/115014893968-Terms-of-sale}{Terms
  of Sale}
\item
  \href{https://spiderbites.nytimes3xbfgragh.onion}{Site Map}
\item
  \href{https://help.nytimes3xbfgragh.onion/hc/en-us}{Help}
\item
  \href{https://www.nytimes3xbfgragh.onion/subscription?campaignId=37WXW}{Subscriptions}
\end{itemize}
