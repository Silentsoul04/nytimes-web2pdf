Sections

SEARCH

\protect\hyperlink{site-content}{Skip to
content}\protect\hyperlink{site-index}{Skip to site index}

\href{https://www.nytimes3xbfgragh.onion/section/us}{U.S.}

\href{https://myaccount.nytimes3xbfgragh.onion/auth/login?response_type=cookie\&client_id=vi}{}

\href{https://www.nytimes3xbfgragh.onion/section/todayspaper}{Today's
Paper}

\href{/section/us}{U.S.}\textbar{}A Court Said Au Pairs Deserve Minimum
Wage. Some Families Are Protesting.

\url{https://nyti.ms/37Kf8ZG}

\begin{itemize}
\item
\item
\item
\item
\item
\item
\end{itemize}

Advertisement

\protect\hyperlink{after-top}{Continue reading the main story}

Supported by

\protect\hyperlink{after-sponsor}{Continue reading the main story}

\hypertarget{a-court-said-au-pairs-deserve-minimum-wage-some-families-are-protesting}{%
\section{A Court Said Au Pairs Deserve Minimum Wage. Some Families Are
Protesting.}\label{a-court-said-au-pairs-deserve-minimum-wage-some-families-are-protesting}}

A court found that au pairs were entitled to rights of domestic workers
in Massachusetts. Families said they did not know how they would pay.
Some au pairs were elated.

\includegraphics{https://static01.graylady3jvrrxbe.onion/images/2020/01/07/us/07AUPAIR-statehouse/merlin_151822158_4b1e499f-01f1-43bd-a3c8-9fda9fde517f-articleLarge.jpg?quality=75\&auto=webp\&disable=upscale}

\href{https://www.nytimes3xbfgragh.onion/by/kate-taylor}{\includegraphics{https://static01.graylady3jvrrxbe.onion/images/2018/02/20/multimedia/author-kate-taylor/author-kate-taylor-thumbLarge.jpg}}

By \href{https://www.nytimes3xbfgragh.onion/by/kate-taylor}{Kate Taylor}

\begin{itemize}
\item
  Jan. 8, 2020
\item
  \begin{itemize}
  \item
  \item
  \item
  \item
  \item
  \item
  \end{itemize}
\end{itemize}

CAMBRIDGE, Mass. --- When Stephanie Mayberg, a physician assistant,
learned that a court ruling meant her child care costs were about to
increase by 250 percent, she was stunned. The recent federal court
decision, that au pairs were entitled to the rights of domestic workers
in Massachusetts, including being paid a minimum wage, left Ms. Mayberg,
of Southborough, wondering how she and her husband could afford to keep
their au pair from Colombia for a second year.

But Claudia Villamizar was elated when she heard of the ruling. Ms.
Villamizar, who had once been an au pair in Massachusetts from Colombia,
recalled being miserable when a family required her to work 65 hours a
week --- far more than the 45 hours allowed under the federal au pair
program. She said the family had kicked her out of the house when she
complained.

``Being an au pair was the worst experience I ever had in my 38 years of
my life,'' Ms. Villamizar said.

Of the legal finding that au pairs --- young people from other countries
who come to the United States to live with families and care for their
children --- were entitled to a minimum wage and protected by
Massachusetts's
\href{https://malegislature.gov/Laws/SessionLaws/Acts/2014/Chapter148}{Domestic
Workers Bill of Rights}, passed in 2014, she added, ``I'm a big
supporter because au pairs are unprotected.''

The ruling in December by the United States Court of Appeals for the
First Circuit affirmed a lower-court decision dismissing a lawsuit by an
au pair agency against the Massachusetts attorney general. The lawsuit
sought to prevent the attorney general from applying the Domestic
Workers Bill of Rights to au pairs.

Under a federal program administered by the State Department, au pairs
are paid a stipend of roughly \$195 per week, in addition to receiving
room and board. Host families also must pay up to \$500 a year toward an
au pair's academic work.

In Massachusetts, the decision has thrown families who host au pairs
into chaos as they sort through their new responsibilities as employers,
and cope with significantly increased child care costs.

They rallied at the Massachusetts State House on Wednesday to urge
lawmakers to pass bills that would mitigate the effects of the decision.

The First Circuit decision appears likely to have an impact beyond
Massachusetts, in other states that, without explicitly exempting au
pairs, offer protections to domestic workers greater than those of the
federal au pair regulations. The ruling is binding in the First Circuit,
which, in addition to Massachusetts, includes Maine, New Hampshire,
Rhode Island and Puerto Rico.

The minimum wage in Massachusetts, \$12.75 per hour, means that families
who employ au pairs will now have to pay them roughly \$528 a week for
45 hours of work, when factoring in overtime and a \$77 deduction for
room and board. The lawsuit, which was brought in 2016, had been working
its way through the courts for several years, but it appeared that many
au pair agencies had not warned host families about the pending case or
the possibility that the domestic workers rules might apply.

Among the proposals that parents' groups are urging lawmakers to
consider, one bill would allow families to deduct up to 40 percent of
the weekly wages that they paid au pairs to cover lodging and food.
Parents also were hoping legislators would agree to delay until July
when the domestic workers rules would apply.

Several parents said the ruling fundamentally misconstrued the nature of
the au pair program, which was designed as a cultural exchange, allowing
young people to pursue their education while experiencing everyday life
in an American family. Au pair means ``on par'' in French, and the State
Department's online information about the program says au pairs should
be included ``whenever possible in family meals, outings, holidays and
other events.'' Several parents said they considered current or former
au pairs to be members of their family.

And
\href{https://www.state.gov/wp-content/uploads/2019/10/Brief-in-response-to-court\%E2\%80\%99s-request-in-Capron-v.-Massachusetts.pdf}{in
a brief filed with the court}, the federal government argued against
applying the Massachusetts law to au pairs.

``State and local regulations have the potential to severely undermine
the au pair program, particularly if increased costs or record-keeping
burdens discourage participation by host families,'' the brief said,
describing the program as ``a valuable tool of U.S. foreign policy.''

But some former au pairs disagreed with the parents and the view of the
federal government. Thaty Oliveira, 35, who is from Brazil, was an au
pair in Massachusetts in 2003. While she said she had a great experience
with the family, and worked at most 30 hours a week, she said that was
not the norm among her fellow au pairs, many of whom spent too many
hours doing child care to get a rich exposure to American culture. Even
in her case, she said, she considers the child care she performed to
have been real work, deserving of a minimum wage.

``We're not asking for a lot,'' she said. ``It's really just minimum
rights.''

Monique Tu Nguyen, the executive director of the Matahari Women Workers'
Center in Boston, which organizes women in low-wage jobs, such as house
cleaners, nannies and adult caregivers, said the shock and frustration
that families were experiencing were largely because au pair agencies
did not adequately inform them about the looming issue.

She said her organization would push for legislation that would resolve
concerns families had about being sued for back pay.

``We want to be given a chance for everyone to see that there's a third
solution beyond what everyone is presenting at the moment,'' Ms. Nguyen
said.

Massachusetts is the fifth-most popular destination for au pairs, with
1,530 new au pairs arriving in 2018,
\href{https://j1visa.state.gov/wp-content/uploads/2019/03/Au-Pair-Flyer-2018-web.pdf}{according
to data from the State Department}, which administers the program.

Advertisement

\protect\hyperlink{after-bottom}{Continue reading the main story}

\hypertarget{site-index}{%
\subsection{Site Index}\label{site-index}}

\hypertarget{site-information-navigation}{%
\subsection{Site Information
Navigation}\label{site-information-navigation}}

\begin{itemize}
\tightlist
\item
  \href{https://help.nytimes3xbfgragh.onion/hc/en-us/articles/115014792127-Copyright-notice}{©~2020~The
  New York Times Company}
\end{itemize}

\begin{itemize}
\tightlist
\item
  \href{https://www.nytco.com/}{NYTCo}
\item
  \href{https://help.nytimes3xbfgragh.onion/hc/en-us/articles/115015385887-Contact-Us}{Contact
  Us}
\item
  \href{https://www.nytco.com/careers/}{Work with us}
\item
  \href{https://nytmediakit.com/}{Advertise}
\item
  \href{http://www.tbrandstudio.com/}{T Brand Studio}
\item
  \href{https://www.nytimes3xbfgragh.onion/privacy/cookie-policy\#how-do-i-manage-trackers}{Your
  Ad Choices}
\item
  \href{https://www.nytimes3xbfgragh.onion/privacy}{Privacy}
\item
  \href{https://help.nytimes3xbfgragh.onion/hc/en-us/articles/115014893428-Terms-of-service}{Terms
  of Service}
\item
  \href{https://help.nytimes3xbfgragh.onion/hc/en-us/articles/115014893968-Terms-of-sale}{Terms
  of Sale}
\item
  \href{https://spiderbites.nytimes3xbfgragh.onion}{Site Map}
\item
  \href{https://help.nytimes3xbfgragh.onion/hc/en-us}{Help}
\item
  \href{https://www.nytimes3xbfgragh.onion/subscription?campaignId=37WXW}{Subscriptions}
\end{itemize}
