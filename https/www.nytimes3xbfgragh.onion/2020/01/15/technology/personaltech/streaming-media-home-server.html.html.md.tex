Sections

SEARCH

\protect\hyperlink{site-content}{Skip to
content}\protect\hyperlink{site-index}{Skip to site index}

\href{https://www.nytimes3xbfgragh.onion/section/technology/personaltech}{Personal
Tech}

\href{https://myaccount.nytimes3xbfgragh.onion/auth/login?response_type=cookie\&client_id=vi}{}

\href{https://www.nytimes3xbfgragh.onion/section/todayspaper}{Today's
Paper}

\href{/section/technology/personaltech}{Personal Tech}\textbar{}We're
Living in a Subscriptions World. Here's How to Navigate It.

\url{https://nyti.ms/2TrzO4T}

\begin{itemize}
\item
\item
\item
\item
\item
\item
\end{itemize}

Advertisement

\protect\hyperlink{after-top}{Continue reading the main story}

Supported by

\protect\hyperlink{after-sponsor}{Continue reading the main story}

tech fix

\hypertarget{were-living-in-a-subscriptions-world-heres-how-to-navigate-it}{%
\section{We're Living in a Subscriptions World. Here's How to Navigate
It.}\label{were-living-in-a-subscriptions-world-heres-how-to-navigate-it}}

Subscription services like Netflix and Google Drive are convenient, but
we can lose control of our content and data. Here's how to take it back.

\includegraphics{https://static01.graylady3jvrrxbe.onion/images/2020/01/15/business/15Techfix-illo/15Techfix-illo-articleLarge.gif?quality=75\&auto=webp\&disable=upscale}

\href{https://www.nytimes3xbfgragh.onion/by/brian-x-chen}{\includegraphics{https://static01.graylady3jvrrxbe.onion/images/2018/02/16/multimedia/author-brian-x-chen/author-brian-x-chen-thumbLarge.jpg}}

By \href{https://www.nytimes3xbfgragh.onion/by/brian-x-chen}{Brian X.
Chen}

\begin{itemize}
\item
  Jan. 15, 2020
\item
  \begin{itemize}
  \item
  \item
  \item
  \item
  \item
  \item
  \end{itemize}
\end{itemize}

Nowadays we don't really buy things. We just subscribe to online
services.

And how can we resist? The
\href{https://www.nytimes3xbfgragh.onion/2019/11/18/business/media/streaming-hollywood-revolution.html}{streaming
revolution} has brought us vast amounts of video and music entertainment
at the click of a button. In an era of cloud storage, where we store our
data on remote computer servers, tech companies like Google and Apple
take care of the headache of managing our information so that we no
longer lose important files or progress on our work.

For many of us, giving up control and ownership to these services is the
point. But for others, there is a downside to losing some flexibility
and freedom. While Spotify may not have all the music we want to listen
to, if we cancel our subscription, we lose access to its large catalog
of music. With cloud storage services, putting our documents and other
files online is simple, but
\href{https://www.nytimes3xbfgragh.onion/2017/01/25/technology/personaltech/dumping-a-bad-app-tips-for-a-painless-breakup.html}{pulling
them out can be a pain}.

This can make some people feel trapped. We could always resort to the
obvious old-school methods, like buying discs of music and carrying
around thumb drives of our files and documents, but who wants to do
that?

Fortunately, there are some approaches to taking control of our media
while enjoying the benefits of subscription services. Those steps range
from the obvious, like creating local copies of your data, to more
advanced methods, like making a personal cloud using an
internet-connected storage device that acts like a miniature server.

All it takes is some forethought and technological know-how. Here's what
you need to know.

\hypertarget{maintain-your-own-backups}{%
\subsection{Maintain Your Own Backups}\label{maintain-your-own-backups}}

Cloud storage services like Google Drive and Apple's iCloud --- which
let you store small amounts of data online free and which charge a few
dollars a month to hoard larger amounts --- offer major benefits.
Namely, we can get access to our data from any device with an internet
connection, and because our files are copied onto a company's servers,
we can't lose them.

But beware of becoming over-reliant on the cloud. What if one day you
decide to cancel your subscription? For anything that is stored
exclusively online, you would then have to download each piece of data
to your own drive, which can be frustrating and time-consuming.

That's why, as a rule of thumb, people should continue creating local
copies of their data for their computers and smartphones and store only
important files on the cloud.

Here are the tools you will need:

\begin{itemize}
\item
  \textbf{An external hard drive.} Portable hard drives can store vast
  amounts of data, and they are generally cheap. Seagate's Backup Plus
  Slim 2, a
  \href{https://thewirecutter.com/reviews/best-portable-hard-drive/}{Wirecutter
  recommendation}, costs about \$60 and holds two terabytes of data,
  which is probably enough to store backups of your computer, tablet and
  smartphone.
\item
  \textbf{A software program for creating computer backups.} Mac
  computers include Apple's Time Machine backup tool. Microsoft's
  Windows 10 includes a free tool called File History. Both apps can be
  set up to automatically back up your computer data.
\item
  \textbf{An app for backing up your smartphone data.} Apple users can
  \href{https://support.apple.com/en-us/HT203977\#computer}{back up
  their iPhones} to their computers via the Finder or iTunes apps.
  Android users with Windows computers can access their data via ``My
  Computer,'' and on a Mac, Android users can use the app
  \href{https://www.android.com/filetransfer/}{Android File Transfer}.
\end{itemize}

From there, the steps vary slightly depending on which device and apps
you use, but the processes are generally the same. To back up your
computer data, you plug your external hard drive into your computer and
run the backup program. To back up your smartphone data to your
computer, you plug the smartphone into the computer and run your backup
app. (If you need more steps,
\href{https://thewirecutter.com/reviews/how-to-back-up-your-computer/\#create-local-windows-backups-with-file-history}{Wirecutter
published a comprehensive guide on creating data backups}.)

This way, if we become dissatisfied with a cloud service, we can cancel
the subscription and have the ease and flexibility to take our files
elsewhere.

\hypertarget{create-a-personal-entertainment-cloud}{%
\subsection{Create a Personal Entertainment
Cloud}\label{create-a-personal-entertainment-cloud}}

Streaming services like Netflix, Apple TV Plus and Hulu offer a buffet
of TV shows and movies to binge on. Similarly, Spotify and Apple Music
give you instant access to millions of songs. But streaming services
don't have access to everything out there, like obscure art house films
or live performances by music artists.

So here's how you can take control of the content you stream to your
devices. There's a clever approach that involves creating your own media
cloud, which acts like an online locker for your own content.

Michael Calore, an editor for Wired and a part-time D.J., said that when
Spotify lacks his favorite music, he extracts the songs from a disc and
uploads them to Google Play Music, Google's online music service. Then
he plays the music on the Google Play Music app from his smartphone.

``It's basically like my own private streaming music service,'' he said.
In general, people can apply this approach to any songs they can't get
on streaming services.

For movies, I'll share my setup, which is not for the faint of heart.

As a film studies student, I owned a collection of hundreds of DVDs,
many of them obscure indie titles that are nowhere to be found on any
streaming service. So I converted the titles into digital video formats,
which I stored on a
\href{https://thewirecutter.com/reviews/best-network-attached-storage/}{network-attached
storage device}, essentially a miniature server.

From there, I installed the Plex video-streaming app on my Apple TV, and
on my smartphone, I installed Infuse 6, another video-streaming app. I
set up both apps to pull movies from my mini server. This way, I can
still enjoy the ability to stream my special collection of art house
movies via my own equipment.

Of course, for many of a certain (younger) age, physical discs are
unheard-of, and newer obscure titles will more likely be released on a
streaming service. Still, for those wanting to tailor the content they
stream, physical media is worth exploring.

So here's what you will need to create personal clouds for your movies
and music:

\begin{itemize}
\item
  \textbf{Tech to extract content from discs.} First, you will need an
  optical drive, which is still included with some desktop computers, to
  read discs.

  Second, you will need apps to ``rip'' the content and turn the movies
  into digital files. For videos, special computer programs like
  Handbrake can extract movies from discs and convert them into video
  files. For audio, programs like iTunes and Windows Media Player can
  rip digital music files from CDs.
\item
  \textbf{Tech to create a video server.} Basically, you need an
  internet-connected device with some storage for movies, which
  essentially acts as a miniature server. There are plenty of options,
  like the \$150 \href{https://www.nvidia.com/en-us/shield/shop/}{Nvidia
  Shield TV}, or the
  \href{https://www.synology.com/en-us/products/DS218+}{Synology
  DiskStation DS218+}, which costs about \$300.
\end{itemize}

\begin{itemize}
\tightlist
\item
  \textbf{Tech to play media over the internet.} For music, Google Play
  Music lets you upload your own songs to a cloud library and stream
  them through the app. For movies, streaming apps like
  \href{https://www.plex.tv/}{Plex} or
  \href{https://apps.apple.com/us/app/infuse-6/id1136220934}{Infuse 6}
  let you play movies from a TV app or smartphone.
\end{itemize}

If that all sounds complicated, that's because setting up your content
to be easily accessible over the internet is no easy feat. But these
options exist for people who want more freedom.

Mr. Calore said that despite having a nice setup for streaming media via
a personal cloud, he still consumed the vast majority of music and
movies from paid streaming services.

``We've lost the excitement and the specialness of a physical idea,'' he
said. ``But what we've gained in exchange is abundance at a scale that
we could never have imagined. That is very much worth the trade-off.''

Advertisement

\protect\hyperlink{after-bottom}{Continue reading the main story}

\hypertarget{site-index}{%
\subsection{Site Index}\label{site-index}}

\hypertarget{site-information-navigation}{%
\subsection{Site Information
Navigation}\label{site-information-navigation}}

\begin{itemize}
\tightlist
\item
  \href{https://help.nytimes3xbfgragh.onion/hc/en-us/articles/115014792127-Copyright-notice}{©~2020~The
  New York Times Company}
\end{itemize}

\begin{itemize}
\tightlist
\item
  \href{https://www.nytco.com/}{NYTCo}
\item
  \href{https://help.nytimes3xbfgragh.onion/hc/en-us/articles/115015385887-Contact-Us}{Contact
  Us}
\item
  \href{https://www.nytco.com/careers/}{Work with us}
\item
  \href{https://nytmediakit.com/}{Advertise}
\item
  \href{http://www.tbrandstudio.com/}{T Brand Studio}
\item
  \href{https://www.nytimes3xbfgragh.onion/privacy/cookie-policy\#how-do-i-manage-trackers}{Your
  Ad Choices}
\item
  \href{https://www.nytimes3xbfgragh.onion/privacy}{Privacy}
\item
  \href{https://help.nytimes3xbfgragh.onion/hc/en-us/articles/115014893428-Terms-of-service}{Terms
  of Service}
\item
  \href{https://help.nytimes3xbfgragh.onion/hc/en-us/articles/115014893968-Terms-of-sale}{Terms
  of Sale}
\item
  \href{https://spiderbites.nytimes3xbfgragh.onion}{Site Map}
\item
  \href{https://help.nytimes3xbfgragh.onion/hc/en-us}{Help}
\item
  \href{https://www.nytimes3xbfgragh.onion/subscription?campaignId=37WXW}{Subscriptions}
\end{itemize}
