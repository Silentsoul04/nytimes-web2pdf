Sections

SEARCH

\protect\hyperlink{site-content}{Skip to
content}\protect\hyperlink{site-index}{Skip to site index}

\href{https://www.nytimes3xbfgragh.onion/section/climate}{Climate}

\href{https://myaccount.nytimes3xbfgragh.onion/auth/login?response_type=cookie\&client_id=vi}{}

\href{https://www.nytimes3xbfgragh.onion/section/todayspaper}{Today's
Paper}

\href{/section/climate}{Climate}\textbar{}Trump Removes Pollution
Controls on Streams and Wetlands

\url{https://nyti.ms/2Rm0nHg}

\begin{itemize}
\item
\item
\item
\item
\item
\item
\end{itemize}

\hypertarget{climate-and-environment}{%
\subsubsection{\texorpdfstring{\href{https://www.nytimes3xbfgragh.onion/section/climate?name=styln-climate\&region=TOP_BANNER\&variant=undefined\&block=storyline_menu_recirc\&action=click\&pgtype=Article\&impression_id=07e69850-e390-11ea-ae4b-973c618b2df0}{Climate
and
Environment}}{Climate and Environment}}\label{climate-and-environment}}

\begin{itemize}
\tightlist
\item
  \href{https://www.nytimes3xbfgragh.onion/2020/08/17/climate/alaska-oil-drilling-anwr.html?name=styln-climate\&region=TOP_BANNER\&variant=undefined\&block=storyline_menu_recirc\&action=click\&pgtype=Article\&impression_id=07e69851-e390-11ea-ae4b-973c618b2df0}{Arctic
  Refuge}
\item
  \href{https://www.nytimes3xbfgragh.onion/interactive/2020/climate/trump-environment-rollbacks.html?name=styln-climate\&region=TOP_BANNER\&variant=undefined\&block=storyline_menu_recirc\&action=click\&pgtype=Article\&impression_id=07e69852-e390-11ea-ae4b-973c618b2df0}{Trump's
  Changes}
\item
  \href{https://www.nytimes3xbfgragh.onion/interactive/2020/04/19/climate/climate-crash-course-1.html?name=styln-climate\&region=TOP_BANNER\&variant=undefined\&block=storyline_menu_recirc\&action=click\&pgtype=Article\&impression_id=07e69853-e390-11ea-ae4b-973c618b2df0}{Climate
  101}
\item
  \href{https://www.nytimes3xbfgragh.onion/interactive/2018/08/30/climate/how-much-hotter-is-your-hometown.html?name=styln-climate\&region=TOP_BANNER\&variant=undefined\&block=storyline_menu_recirc\&action=click\&pgtype=Article\&impression_id=07e69854-e390-11ea-ae4b-973c618b2df0}{Is
  Your Hometown Hotter?}
\end{itemize}

Advertisement

\protect\hyperlink{after-top}{Continue reading the main story}

Supported by

\protect\hyperlink{after-sponsor}{Continue reading the main story}

\hypertarget{trump-removes-pollution-controls-on-streams-and-wetlands}{%
\section{Trump Removes Pollution Controls on Streams and
Wetlands}\label{trump-removes-pollution-controls-on-streams-and-wetlands}}

\includegraphics{https://static01.graylady3jvrrxbe.onion/images/2020/01/21/climate/00CLI-WOTUS-HFO1/00CLI-WOTUS-HFO1-articleLarge.jpg?quality=75\&auto=webp\&disable=upscale}

\href{https://www.nytimes3xbfgragh.onion/by/coral-davenport}{\includegraphics{https://static01.graylady3jvrrxbe.onion/images/2018/10/03/multimedia/author-coral-davenport/author-coral-davenport-thumbLarge-v2.png}}

By \href{https://www.nytimes3xbfgragh.onion/by/coral-davenport}{Coral
Davenport}

\begin{itemize}
\item
  Published Jan. 22, 2020Updated July 6, 2020
\item
  \begin{itemize}
  \item
  \item
  \item
  \item
  \item
  \item
  \end{itemize}
\end{itemize}

WASHINGTON --- The
\href{https://www.nytimes3xbfgragh.onion/2020/07/06/us/politics/trump-lobbyists-swamp-campaign.html}{Trump}
administration on Thursday finalized a rule to strip away environmental
protections for streams, wetlands and groundwater, handing a victory to
farmers, fossil fuel producers and real estate developers who said
Obama-era rules had shackled them with onerous and unnecessary burdens.

From Day 1 of his administration, President Trump vowed to repeal
President Barack Obama's ``Waters of the United States'' regulation,
which had frustrated rural landowners. His new rule, which will be
implemented in about 60 days, is the latest step in the Trump
administration's push to
\href{https://www.nytimes3xbfgragh.onion/interactive/2019/climate/trump-environment-rollbacks.html}{repeal
or weaken nearly 100 environmental rules and laws}, loosening or
eliminating rules on climate change, clean air, chemical pollution, coal
mining, oil drilling and endangered species protections.

Although Mr. Trump frequently speaks of his desire for the United States
to have ``crystal-clean water,'' he has called his predecessor's
signature clean-water regulation ``horrible,'' ``destructive'' and ``one
of the worst examples of federal'' overreach.

``I terminated one of the most ridiculous regulations of all: the last
administration's disastrous Waters of the United States rule,'' he told
the American Farm Bureau Federation's annual convention in Texas on
Sunday, to rousing applause.

``That was a rule that basically took your property away from you,''
added Mr. Trump, whose real estate holdings include more than a dozen
golf courses. (Golf course developers were among the key opponents of
the Obama rule and key backers of the new one.)

His administration had completed the first step of its demise in
September with the
\href{https://www.nytimes3xbfgragh.onion/2019/09/12/climate/trump-administration-rolls-back-clean-water-protections.html}{rule's
repeal}.

Mr. Trump's replacement, called the
``\href{https://www.epa.gov/nwpr}{Navigable Waters Protection Rule},''
finishes the process. It not only rolls back key portions of the 2015
rule that had guaranteed protections under the 1972 Clean Water Act to
certain wetlands and streams that run intermittently or run temporarily
underground, but also relieves landowners of the need to seek permits
that the Environmental Protection Agency had considered on a
case-by-case basis before the Obama rule.

\href{https://www.nytimes3xbfgragh.onion/section/climate?action=click\&pgtype=Article\&state=default\&region=MAIN_CONTENT_1\&context=storylines_keepup}{}

\hypertarget{climate-and-environment-}{%
\subsubsection{Climate and Environment
›}\label{climate-and-environment-}}

\hypertarget{keep-up-on-the-latest-climate-news}{%
\paragraph{Keep Up on the Latest Climate
News}\label{keep-up-on-the-latest-climate-news}}

Updated Aug. 18, 2020

Here's what you need to know this week:

\begin{itemize}
\item
  \begin{itemize}
  \tightlist
  \item
    Five automakers
    \href{https://www.nytimes3xbfgragh.onion/2020/08/17/climate/california-automakers-pollution.html?action=click\&pgtype=Article\&state=default\&region=MAIN_CONTENT_1\&context=storylines_keepup}{sealed
    a binding agreement} with California to follow the state's stricter
    tailpipe emissions rules.
  \item
    The Trump
    administration\href{https://www.nytimes3xbfgragh.onion/2020/08/13/climate/trump-methane.html?action=click\&pgtype=Article\&state=default\&region=MAIN_CONTENT_1\&context=storylines_keepup}{eliminated
    a major methane rule}, even as leaks are worsening, in a decision
    that researchers warned ignored science.
  \item
    Climate change leaders said
    \href{https://www.nytimes3xbfgragh.onion/2020/08/12/climate/kamala-harris-environmental-justice.html?action=click\&pgtype=Article\&state=default\&region=MAIN_CONTENT_1\&context=storylines_keepup}{the
    vice-presidential choice of Kamala Harris} signaled that Democrats
    will have a focus on environmental justice.
  \end{itemize}
\end{itemize}

It also gives President Trump a major policy achievement to bring to his
political base while his impeachment trial continues.

``Farmers coalesced against the E.P.A. being able to come onto their
land, and he's delivering,'' said Jessica Flanagain, a Republican
strategist in Lincoln, Neb. ``This is bigger news for agricultural
producers than whatever is happening with the sideshow in D.C.,'' she
added.

Speaking on Thursday at a conference of the National Association of Home
Builders in Las Vegas, Andrew Wheeler, the administrator of the
Environmental Protection Agency, framed the new rule as the rightful
return of power from the federal government to landowners and states.

``It respects the limited powers that the executive branch has been
given under the Constitution and the Clean Water Act to protect
navigable waters,'' he said.

The new water rule **** for the first time in decades allow landowners
and property developers to dump pollutants such as pesticides and
fertilizers directly into hundreds of thousands of waterways, and to
destroy or fill in wetlands for construction projects.

``This will be the biggest loss of clean water protection the country
has ever seen,'' said Blan Holman, a lawyer specializing in federal
water policy at the Southern Environmental Law Center. ``This puts
drinking water for millions of Americans at risk of contamination from
unregulated pollution. This is not just undoing the Obama rule. This is
stripping away protections that were put in place in the '70s and '80s
that Americans have relied on for their health.''

Mr. Holman also said that the new rule exemplifies how the Trump
administration has
\href{https://www.nytimes3xbfgragh.onion/2019/12/28/climate/trump-administration-war-on-science.html}{dismissed
or marginalized scientific evidence}. Last month, a government advisory
board of scientists, many of whom were handpicked by the Trump
administration,
\href{https://www.nytimes3xbfgragh.onion/2019/12/31/climate/epa-science-panel-trump.html}{wrote
that the proposed water rule ``neglects established science.''}

But farmers and fossil fuel groups supported the change.

``This is a big win for farmers, and this is the president delivering
what he promised,'' said Donald Parrish, senior director of regulatory
affairs for the American Farm Bureau Federation, which had lobbied for
years to weaken the Obama administration's water rules.

Karen Harbert, chief executive officer of the American Gas Association,
said the new rule ``would restore the proper balance between federal and
state regulation of our nation's waters and protect our rivers, streams
and lakes without stifling construction of important infrastructure.''

\includegraphics{https://static01.graylady3jvrrxbe.onion/images/2020/01/21/climate/00CLI-WOTUS-HFO2/merlin_167398881_aeeeb5e7-38b7-44a1-8ca7-2ce45e2ea52c-articleLarge.jpg?quality=75\&auto=webp\&disable=upscale}

The Obama rule protected about 60 percent of the nation's waterways,
including large bodies of water such as the Chesapeake Bay, Mississippi
River and Puget Sound, and smaller headwaters, wetlands, seasonal
streams and streams that run temporarily underground. It limited the
discharge of pollutants such as fertilizers, pesticides and industrial
chemicals into those waters.

The new rule, written by the E.P.A. and the Army Corps of Engineers,
will retain federal protections of large bodies of water, as well as
larger rivers and streams that flow into them and wetlands that lie
adjacent to them. But it removes protections for many other waters,
including wetlands that are not adjacent to large bodies of water, some
seasonal streams that flow for only a portion of the year, ``ephemeral''
streams that only flow after rainstorms, and groundwater.

Legal experts say that Mr. Trump's replacement rule would go further
than simply repealing and replacing the 2015 Obama rule --- it would
also eliminate protections to smaller headwaters that have been
implemented for decades under the 1972 Clean Water Act.

``This is rolling back federal jurisdiction of the Clean Water Act
further than it's ever been before,'' said Patrick Parenteau, a
professor of environmental law at Vermont Law School. ``Waters that have
been protected for almost 50 years will no longer be protected under the
Clean Water Act.''

That could open millions of acres of pristine wetlands to pollution or
destruction, and allow chemicals and other pollutants to be discharged
into smaller headland waters that eventually drain into larger water
bodies, experts in water management said. Wetlands play key roles in
filtering surface water and protecting against floods, while also
providing wildlife habitat.

Ean Thomas Tafoya, a Colorado-based clean water activist with the group
GreenLatinos, said the new rule could harm the quality of the water in
the Colorado River, which supplies water to 17 western states.

``We are a headwater state,'' he said. ``This rollback will affect
almost every single stream that flows into the Colorado River.''

Mr. Tafoya said about 90 percent of the streams that supply the Colorado
River run only after rainfall or snowmelt. Under the new Trump water
rule, many of those streams will not qualify for federal pollution
protection. But Mr. Tafoya said pollutants such as chemical pesticides
that end up in those dry stream beds could nonetheless be swept into
larger bodies of water when the streams begin running after the spring
thaw of mountain snow.

``The toxics or poisons that lie dormant will still be there when the
streams are reactivated,'' he said. ``They will still get into the
larger bodies of water.''

Government scientists, even those appointed by the Trump administration,
say those concerns are justified. The E.P.A.'s Scientific Advisory
Board, a panel of 41 scientists responsible for evaluating the
scientific integrity of the agency's regulations, concluded that the new
Trump water rule ignores science by ``failing to acknowledge watershed
systems.'' They found ``no scientific justification'' for excluding
certain bodies of water from protection under the new regulations,
concluding that pollutants from those smaller and seasonal bodies of
water can still have a significant impact on the health of larger water
systems.

Those scientific findings, although they are not reflected in the
administration's policy, could still play a role in the fate of the new
rule. Several state attorneys general are expected to join with
environmental groups to sue to overturn the Trump water rule, and those
groups are likely to cite those findings as evidence that the rule is
not legally sound.

``The legal standing all has to do with whether you have a rational
basis for what you're doing,'' said Mr. Parenteau. ``And when you have
experts saying you're not adhering to the science, that's not rational,
it's arbitrary.''

\emph{For more climate news sign up for}
\href{https://www.nytimes3xbfgragh.onion/newsletters/climate-change}{\emph{the
Climate Fwd: newsletter}} \emph{or follow}
\href{https://twitter.com/nytclimate}{\emph{@NYTClimate on
Twitter}}\emph{.}

Advertisement

\protect\hyperlink{after-bottom}{Continue reading the main story}

\hypertarget{site-index}{%
\subsection{Site Index}\label{site-index}}

\hypertarget{site-information-navigation}{%
\subsection{Site Information
Navigation}\label{site-information-navigation}}

\begin{itemize}
\tightlist
\item
  \href{https://help.nytimes3xbfgragh.onion/hc/en-us/articles/115014792127-Copyright-notice}{©~2020~The
  New York Times Company}
\end{itemize}

\begin{itemize}
\tightlist
\item
  \href{https://www.nytco.com/}{NYTCo}
\item
  \href{https://help.nytimes3xbfgragh.onion/hc/en-us/articles/115015385887-Contact-Us}{Contact
  Us}
\item
  \href{https://www.nytco.com/careers/}{Work with us}
\item
  \href{https://nytmediakit.com/}{Advertise}
\item
  \href{http://www.tbrandstudio.com/}{T Brand Studio}
\item
  \href{https://www.nytimes3xbfgragh.onion/privacy/cookie-policy\#how-do-i-manage-trackers}{Your
  Ad Choices}
\item
  \href{https://www.nytimes3xbfgragh.onion/privacy}{Privacy}
\item
  \href{https://help.nytimes3xbfgragh.onion/hc/en-us/articles/115014893428-Terms-of-service}{Terms
  of Service}
\item
  \href{https://help.nytimes3xbfgragh.onion/hc/en-us/articles/115014893968-Terms-of-sale}{Terms
  of Sale}
\item
  \href{https://spiderbites.nytimes3xbfgragh.onion}{Site Map}
\item
  \href{https://help.nytimes3xbfgragh.onion/hc/en-us}{Help}
\item
  \href{https://www.nytimes3xbfgragh.onion/subscription?campaignId=37WXW}{Subscriptions}
\end{itemize}
