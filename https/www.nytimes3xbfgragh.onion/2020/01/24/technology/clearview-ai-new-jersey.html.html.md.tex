Sections

SEARCH

\protect\hyperlink{site-content}{Skip to
content}\protect\hyperlink{site-index}{Skip to site index}

\href{https://www.nytimes3xbfgragh.onion/section/technology}{Technology}

\href{https://myaccount.nytimes3xbfgragh.onion/auth/login?response_type=cookie\&client_id=vi}{}

\href{https://www.nytimes3xbfgragh.onion/section/todayspaper}{Today's
Paper}

\href{/section/technology}{Technology}\textbar{}New Jersey Bars Police
From Using Clearview Facial Recognition App

\url{https://nyti.ms/2Rp54jv}

\begin{itemize}
\item
\item
\item
\item
\item
\end{itemize}

Advertisement

\protect\hyperlink{after-top}{Continue reading the main story}

Supported by

\protect\hyperlink{after-sponsor}{Continue reading the main story}

\hypertarget{new-jersey-bars-police-from-using-clearview-facial-recognition-app}{%
\section{New Jersey Bars Police From Using Clearview Facial Recognition
App}\label{new-jersey-bars-police-from-using-clearview-facial-recognition-app}}

Reporting about the powerful tool with a database of three billion
photos ``troubled'' the state's attorney general, who asked for an
inquiry into its use.

\includegraphics{https://static01.graylady3jvrrxbe.onion/images/2020/01/24/business/24clearview01/merlin_142020900_8aa6701d-68fc-4017-b0d3-0da919392a67-articleLarge.jpg?quality=75\&auto=webp\&disable=upscale}

By \href{https://www.nytimes3xbfgragh.onion/by/kashmir-hill}{Kashmir
Hill}

\begin{itemize}
\item
  Jan. 24, 2020
\item
  \begin{itemize}
  \item
  \item
  \item
  \item
  \item
  \end{itemize}
\end{itemize}

New Jersey police officers are now barred from using a facial
recognition app made by a start-up that has licensed its groundbreaking
technology to hundreds of law enforcement agencies around the country.

Gurbir S. Grewal, New Jersey's attorney general, told state prosecutors
in all 21 counties on Friday that police officers should stop using the
Clearview AI app.

\href{https://www.nytimes3xbfgragh.onion/2020/01/18/technology/clearview-privacy-facial-recognition.html}{The
New York Times reported} last week that Clearview had amassed a database
of more than three billion photos across the web --- including sites
like Facebook, YouTube, Twitter and Venmo. The vast database powers an
app that can match people to their online photos and link back to the
sites the images came from.

``Until this week, I had not heard of Clearview AI,'' Mr. Grewal said in
an interview. ``I was troubled. The reporting raised questions about
data privacy, about cybersecurity, about law enforcement security, about
the integrity of our investigations.''

His order to prosecutors was
\href{https://www.nj.com/news/2020/01/new-jersey-cops-told-to-halt-all-use-of-controversial-facial-recognition-technology.html}{reported
earlier by NJ.com}.

In a promotional video posted to its website this week, Clearview
included images of Mr. Grewal because the company said its app had
played a role last year in Operation Open Door, a New Jersey police
sting that led to the arrest of 19 people accused of being child
predators.

``I was surprised they used my image and the office to promote the
product online,'' said Mr. Grewal, who confirmed that Clearview's app
had been used to identify one of the people in the sting. ``I was
troubled they were sharing information about ongoing criminal
prosecutions.''

Mr. Grewal's office sent Clearview
\href{https://int.graylady3jvrrxbe.onion/data/documenthelper/6726-new-jersey-ag-clearview-cease-/07b7d49c8810740df2cb/optimized/full.pdf\#page=1}{a
cease-and-desist letter} that asked the company to stop using the office
and its investigations to promote its products.

``We've received the attorney general's letter and are complying,'' said
Tor Ekeland, Clearview's lawyer. ``The video has been removed.''

The video also included a claim that the New York Police Department had
used Clearview's app to identify a man who was accused of planting rice
cookers made to resemble bombs around the city. As reported by
\href{https://www.buzzfeednews.com/article/ryanmac/clearview-ai-nypd-facial-recognition}{BuzzFeed},
the Police Department said the app had played no role in the case.

``There is no institutional relationship between the N.Y.P.D. and
Clearview,'' said Devora Kaye, a spokeswoman for the department. ``The
N.Y.P.D. did not rely on Clearview technology to identify the suspect in
the Aug. 16 rice cooker incident. The N.Y.P.D. identified the suspect
using the department's facial recognition practice, where a still image
from a surveillance video was compared to a pool of lawfully possessed
arrest photos.''

Some officers in the Police Department are said to be using the
Clearview app without official authorization,
\href{https://nypost.com/2020/01/23/rogue-nypd-cops-are-using-sketchy-facial-recognition-app-clearview/}{The
New York Post reported} on Thursday.

In addition to placing a moratorium on the Clearview app, the New Jersey
attorney general's office has asked the state's Division of Criminal
Justice to look into how state law enforcement agencies have used the
app. Mr. Grewal wants to know which ones are using ``this product or
products like it,'' and what information those companies are tracking
about police investigations and searches.

An earlier episode in which police officers received calls from the
company after uploading a photo of a Times reporter to the app indicated
that Clearview has the ability to monitor whom law enforcement is
searching for.

Mr. Grewal said that his office would not have to preapprove use of a
tool like Clearview AI by the police, but that maybe it should. His
office reviews, for example, new forms of less-than-lethal ammunition to
make sure that it's a ``safe tool to have out there.''

``I'm not categorically opposed to using any of these types of tools or
technologies that make it easier for us to solve crimes, and to catch
child predators or other dangerous criminals,'' Mr. Grewal said. ``But
we need to have a full understanding of what is happening here and
ensure there are appropriate safeguards.''

This week, Clearview
\href{https://www.nytimes3xbfgragh.onion/2020/01/22/technology/clearview-ai-twitter-letter.html}{also
received} questions from United States senators, as well as a letter
from Twitter demanding that the start-up stop scraping photos from its
site.

Advertisement

\protect\hyperlink{after-bottom}{Continue reading the main story}

\hypertarget{site-index}{%
\subsection{Site Index}\label{site-index}}

\hypertarget{site-information-navigation}{%
\subsection{Site Information
Navigation}\label{site-information-navigation}}

\begin{itemize}
\tightlist
\item
  \href{https://help.nytimes3xbfgragh.onion/hc/en-us/articles/115014792127-Copyright-notice}{©~2020~The
  New York Times Company}
\end{itemize}

\begin{itemize}
\tightlist
\item
  \href{https://www.nytco.com/}{NYTCo}
\item
  \href{https://help.nytimes3xbfgragh.onion/hc/en-us/articles/115015385887-Contact-Us}{Contact
  Us}
\item
  \href{https://www.nytco.com/careers/}{Work with us}
\item
  \href{https://nytmediakit.com/}{Advertise}
\item
  \href{http://www.tbrandstudio.com/}{T Brand Studio}
\item
  \href{https://www.nytimes3xbfgragh.onion/privacy/cookie-policy\#how-do-i-manage-trackers}{Your
  Ad Choices}
\item
  \href{https://www.nytimes3xbfgragh.onion/privacy}{Privacy}
\item
  \href{https://help.nytimes3xbfgragh.onion/hc/en-us/articles/115014893428-Terms-of-service}{Terms
  of Service}
\item
  \href{https://help.nytimes3xbfgragh.onion/hc/en-us/articles/115014893968-Terms-of-sale}{Terms
  of Sale}
\item
  \href{https://spiderbites.nytimes3xbfgragh.onion}{Site Map}
\item
  \href{https://help.nytimes3xbfgragh.onion/hc/en-us}{Help}
\item
  \href{https://www.nytimes3xbfgragh.onion/subscription?campaignId=37WXW}{Subscriptions}
\end{itemize}
