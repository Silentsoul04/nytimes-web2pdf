Sections

SEARCH

\protect\hyperlink{site-content}{Skip to
content}\protect\hyperlink{site-index}{Skip to site index}

\href{https://www.nytimes3xbfgragh.onion/section/politics}{Politics}

\href{https://myaccount.nytimes3xbfgragh.onion/auth/login?response_type=cookie\&client_id=vi}{}

\href{https://www.nytimes3xbfgragh.onion/section/todayspaper}{Today's
Paper}

\href{/section/politics}{Politics}\textbar{}Lamar Alexander, Set to
Leave Office, Is G.O.P. Wild Card on Witnesses

\url{https://nyti.ms/3aKkt5l}

\begin{itemize}
\item
\item
\item
\item
\item
\end{itemize}

Advertisement

\protect\hyperlink{after-top}{Continue reading the main story}

Supported by

\protect\hyperlink{after-sponsor}{Continue reading the main story}

\hypertarget{lamar-alexander-set-to-leave-office-is-gop-wild-card-on-witnesses}{%
\section{Lamar Alexander, Set to Leave Office, Is G.O.P. Wild Card on
Witnesses}\label{lamar-alexander-set-to-leave-office-is-gop-wild-card-on-witnesses}}

The Tennessee Republican was a protege of Senator Howard H. Baker Jr.,
another Republican from Tennessee, who turned against Richard Nixon. Now
Mr. Alexander may hold the fate of President Trump in his hands.

\includegraphics{https://static01.graylady3jvrrxbe.onion/images/2020/01/25/us/politics/25dc-alexander/merlin_167573544_8a42d23b-f920-453e-9981-fd1615338f2b-articleLarge.jpg?quality=75\&auto=webp\&disable=upscale}

\href{https://www.nytimes3xbfgragh.onion/by/sheryl-gay-stolberg}{\includegraphics{https://static01.graylady3jvrrxbe.onion/images/2018/11/26/multimedia/author-sheryl-gay-stolberg/author-sheryl-gay-stolberg-thumbLarge.png}}

By
\href{https://www.nytimes3xbfgragh.onion/by/sheryl-gay-stolberg}{Sheryl
Gay Stolberg}

\begin{itemize}
\item
  Jan. 25, 2020
\item
  \begin{itemize}
  \item
  \item
  \item
  \item
  \item
  \end{itemize}
\end{itemize}

WASHINGTON --- The ghost of Howard H. Baker Jr., the Republican senator
from Tennessee who turned against Richard M. Nixon during Watergate, is
hovering over Senator Lamar Alexander.

Mr. Alexander, a third-term Republican from Tennessee who
\href{https://www.nytimes3xbfgragh.onion/2018/12/17/us/politics/lamar-alexander-re-election.html}{is
retiring at the end of this year}, has said that no one outside his
family has had more influence on him than Mr. Baker, the former Senate
majority leader who is remembered for the penetrating question he posed
as Nixon stared down impeachment: ``What did the president know, and
when did he know it?''

Now Mr. Alexander may hold in his hands the fate of another Republican
president who is facing removal from office. He is one of four
Republican moderates who have expressed openness to bringing witnesses
into President Trump's impeachment trial. Of the four, he stands out
because he is not running for re-election and arguably has nothing to
lose.

Yet as the Senate heads toward a vote on the matter, Mr. Alexander ---
who has broken with Mr. Trump over trade, the border wall and health
care --- does not appear ready for a Howard Baker moment. He has said he
will make a decision about witnesses after Mr. Trump's team presents its
defense and senators have an opportunity to ask questions, but he does
not sound eager to defect.

``As the House managers have said many times, they've presented us with
a mountain of overwhelming evidence,'' he told reporters in the Capitol
on Friday. ``So we have a lot to consider already.''

Mr. Alexander's caution suggests what Republicans in Tennessee and
around the country already know: that the Howard Baker wing of their
party, the one populated by moderate-leaning conservatives willing to
reach across the political aisle, is virtually extinct. Bob Corker,
another Tennessee Republican, learned as much when he spoke out against
Mr. Trump and then
\href{https://www.nytimes3xbfgragh.onion/2018/02/13/us/politics/corker-retirement-trump.html}{felt
compelled to retire in 2018} from the Senate. So did Jeff Flake,
\href{https://www.nytimes3xbfgragh.onion/2017/10/24/us/politics/jeff-flake-arizona.html}{the
former Republican senator from Arizona}, who watched some of Mr. Trump's
trial from the Senate gallery this week.

``As a Republican, it pains me when I see Republicans, House
Republicans, try to maintain that the president did no wrong, that this
is somehow normal. It's not,'' Mr. Flake told reporters, though he said
he was not sure he would vote to convict Mr. Trump.

That kind of talk is absent among Republicans in the Senate these days,
even from members like Mr. Alexander, who in 2016 made clear that
``Trump was not his first choice for president,'' as his hometown
newspaper,
\href{https://www.tennessean.com/story/news/politics/2016/05/04/delegation-members-begin-lining-up-behind-donald-trump/83921758/}{The
Nashville Tennessean reported}. But if Mr. Alexander has issues with the
president, he tends to raise them quietly, people who know him say.

There is little question that Mr. Alexander will vote to acquit Mr.
Trump. He has called the
\href{https://twitter.com/SenAlexander/status/1217894177133998081}{House
impeachment inquiry ``a circus,''}and said Democrats made a ``mistake''
in charging Mr. Trump with high crimes and misdemeanors for pressuring
Ukraine to investigate his political rivals. But he was among four
Senate Republicans --- along with Susan Collins of Maine, Lisa Murkowski
of Alaska and Mitt Romney of Utah --- who pressed Senator Mitch
McConnell, the majority leader, to allow a vote on whether to subpoena
witnesses and seek new documents.

The White House has regarded Mr. Alexander --- who does not have a close
relationship with Mr. Trump --- as a wild card in the proceeding.

Democrats, who control 47 votes in the Senate, would need four
Republicans to join them to expand the scope of the trial, but so far
only two --- Ms. Collins and Mr. Romney --- seem to be leaning into the
idea.

And Mr. McConnell, Republican of Kentucky, who is close with Mr.
Alexander, is determined to hold Republicans together to block it. The
two men met in Washington in 1969, when Mr. Alexander was a young aide
in Nixon's White House and Mr. McConnell a legislative assistant on
Capitol Hill. It was
\href{https://www.nytimes3xbfgragh.onion/2017/10/19/us/politics/mcconnell-alexander-health-care.html}{Mr.
Baker who introduced them.}

``I seek his counsel on a weekly basis on a whole variety of issues,''
Mr. McConnell said in a brief statement. ``He's my closest friend in the
Senate.''

Mr. McConnell has sometimes used Mr. Alexander as a conduit to
Democrats, particularly to Harry Reid, the former senator from Nevada,
when he was minority leader. Mr. Reid and Mr. McConnell did not get
along, so Mr. Alexander --- who had been in Republican leadership but
stepped away to focus more on legislation --- served as an ``honest
broker'' between the two, said Jim Manley, a former aide to Mr. Reid.

But Mr. Manley said Mr. Alexander ``still toed the party line.''

When Mr. McConnell put forth a resolution setting up a speedy timetable
for the impeachment trial, some Republicans balked and Democrats
objected. But Mr. Alexander issued a statement praising the rules.

People close to Mr. Alexander say they have no idea whether he will vote
to allow witnesses --- and that he may not know yet himself. Should he
do so, he would be a ``pariah'' in the state, said one conservative
activist in Tennessee, who requested anonymity to speak candidly about a
sitting senator.

\includegraphics{https://static01.graylady3jvrrxbe.onion/images/2020/01/25/us/politics/25dc-alexander-3/merlin_167681655_51abda2d-6f49-4e41-9564-1408d4d90f4f-articleLarge.jpg?quality=75\&auto=webp\&disable=upscale}

His seeming reluctance to speak out against Mr. Trump has disappointed
some of Mr. Alexander's admirers. Richard L. Clinton, a professor
emeritus of political science at Oregon State University who was in the
same fraternity as Mr. Alexander at Vanderbilt more than 60 years ago,
posted an open letter this week to the senator on the
\href{https://www.commondreams.org/views/2020/01/21/where-your-courage-and-decency-open-letter-sen-lamar-alexander-childhood-friend}{web
site of the progressive newsletter Common Dreams}.

Under the headline ``Where is Your Courage and Decency?'' Mr. Clinton
wrote that he remembered Mr. Alexander as ``an exceptionally
intelligent, hard-working, and trustworthy young man,'' and was thus
``perplexed'' by his silence. He urged the senator to renounce Mr. Trump
and ``employ his considerable abilities and unique position to begin
making our country whole again.''

But annoying others is not Mr. Alexander's style; he appears to see
himself as more of a bridge builder than a rabble-rouser, which suggests
he is unlikely to vote for witnesses in the impeachment trial.

``Lamar is not looking for a one-time event to have what I call the
shocking headline,'' said Tom Griscom, a close friend of Mr. Alexander
and former press secretary to Mr. Baker. ``You've got a template of who
he is over a career --- that doesn't change. He's not looking to write a
post-note at the back end of it.''

On policy matters, though, Mr. Alexander has not been afraid to part
ways with Mr. Trump. While he has voted with the president 90 percent of
the time, according to the
\href{https://projects.fivethirtyeight.com/congress-trump-score/lamar-alexander/}{website
FiveThirtyEight}, his departures are significant. He
\href{https://www.congress.gov/bill/116th-congress/senate-joint-resolution/54}{voted
to overturn Mr. Trump's plan} to use military funds to build a border
wall, fought the president over tariffs and sought to block him from
withdrawing troops from Syria.

At 79, Mr. Alexander is an icon in Tennessee politics --- twice elected
governor; president of the University of Tennessee; education secretary
to President George Bush; an unsuccessful presidential candidate in 1996
and 2000 and a senator for the past 17 years. Pale and bespectacled, he
is regarded as a serious legislator (he oversees the Senate health
committee) and an ``institutionalist'' --- a guardian of the chamber and
its traditions.

Image

Mr. Alexander in Bedford, N.H., in 1996 outside a polling place on the
day of the primary race.Credit...Elise Amendola/Associated Press

''I've always loved working with him; I'm a big fan of his,'' said
Senator Amy Klobuchar, Democrat of Minnesota, who is running for her
party's nomination for president. ``I just think that he's someone who
really tries to get things done.''

John Geer, a political scientist at Vanderbilt University and
co-director of the Vanderbilt Poll, said his surveys show a ``strong
majority'' of Tennesseans believe Mr. Trump did ``something wrong,'' and
while Mr. Alexander is under pressure from conservatives, ``the Baker
wing'' of the Republican Party would stick with him if he voted for
witnesses.

``He's not voting for impeachment; he's made that very clear,''
Professor Geer said. ``He's voting to learn more, which is frankly
something pretty easy to defend.''

Mr. Alexander got his start in politics working for Mr. Baker in the
1960s. In 1973, when Mr. Baker was the influential ranking minority
member of the Senate Watergate Committee, he asked Mr. Alexander, a
lawyer, to be his chief counsel. But Mr. Alexander turned down the job;
he wanted to seek public office in Tennessee. He has modeled himself
after Mr. Baker, adopting the late senator's habit of giving careful
thought to every decision.

Often forgotten about Mr. Baker is that his famous question was actually
uttered in an effort to protect Nixon; only after months and months of
hearings did he turn against the president. Victoria Bassetti, a former
Senate aide and fellow at the Brennan Center for Justice,
\href{https://www.brennancenter.org/our-work/analysis-opinion/curious-history-what-did-president-know-and-when-did-he-know-it}{who
has written about the episode, said} Mr. Alexander's situation is
different.

``What happened with Howard Baker was the result of the slow, steady
accumulation of wisdom and insight and just the scales dropping from his
eyes over the course of months and months of close careful attention to
what was going on,'' she said. ``And that's not happening in the Senate
today.''

What is happening instead is that many Republicans reflexively defend
Mr. Trump, and those who are unwilling to increasingly feel crowded out
of their party, vulnerable to primary challenges from the president's
loyal base. People close to Mr. Alexander deny that he is leaving the
Senate for that reason. He simply wants to ``go out at the top of his
game,'' as one friend put it.

But the politics of his state have shifted under Mr. Alexander's feet.
In 2014, he faced a tough primary challenge ---
\href{https://www.nytimes3xbfgragh.onion/2014/06/26/us/politics/tennessee-tea-party-lamar-alexander-senate-republican-primary.html}{his
first serious competition in years}--- from a little-known state
representative and conservative Tea Party candidate, Joe Carr. Although
Mr. Alexander won the race handily, many in Tennessee say he would have
almost certainly faced another primary fight this year.

For now, Mr. Alexander is eager to get back to accomplishing his highest
legislative priority: a
\href{https://www.help.senate.gov/chair/newsroom/press/bipartisan-house-and-senate-committee-leaders-announce-agreement-on-legislation-to-lower-health-care-costs-}{bipartisan
package of bills} aimed at lowering the cost of medical care, which
\href{https://www.nytimes3xbfgragh.onion/2019/06/26/us/politics/health-costs-prescription-drugs.html}{has
already passed his committee.} But no matter what he does on
impeachment, like Mr. Baker, he will almost certainly be remembered for
it.

``The reality is that this is Lamar's last year in the Senate,'' said
Bill Haslam, a former governor of Tennessee. ``He would rather be
working on legislation that he thinks can make a difference for the
country. This is not how he would choose to spend the first month of his
last year.''

Emily Cochrane contributed reporting.

Advertisement

\protect\hyperlink{after-bottom}{Continue reading the main story}

\hypertarget{site-index}{%
\subsection{Site Index}\label{site-index}}

\hypertarget{site-information-navigation}{%
\subsection{Site Information
Navigation}\label{site-information-navigation}}

\begin{itemize}
\tightlist
\item
  \href{https://help.nytimes3xbfgragh.onion/hc/en-us/articles/115014792127-Copyright-notice}{©~2020~The
  New York Times Company}
\end{itemize}

\begin{itemize}
\tightlist
\item
  \href{https://www.nytco.com/}{NYTCo}
\item
  \href{https://help.nytimes3xbfgragh.onion/hc/en-us/articles/115015385887-Contact-Us}{Contact
  Us}
\item
  \href{https://www.nytco.com/careers/}{Work with us}
\item
  \href{https://nytmediakit.com/}{Advertise}
\item
  \href{http://www.tbrandstudio.com/}{T Brand Studio}
\item
  \href{https://www.nytimes3xbfgragh.onion/privacy/cookie-policy\#how-do-i-manage-trackers}{Your
  Ad Choices}
\item
  \href{https://www.nytimes3xbfgragh.onion/privacy}{Privacy}
\item
  \href{https://help.nytimes3xbfgragh.onion/hc/en-us/articles/115014893428-Terms-of-service}{Terms
  of Service}
\item
  \href{https://help.nytimes3xbfgragh.onion/hc/en-us/articles/115014893968-Terms-of-sale}{Terms
  of Sale}
\item
  \href{https://spiderbites.nytimes3xbfgragh.onion}{Site Map}
\item
  \href{https://help.nytimes3xbfgragh.onion/hc/en-us}{Help}
\item
  \href{https://www.nytimes3xbfgragh.onion/subscription?campaignId=37WXW}{Subscriptions}
\end{itemize}
