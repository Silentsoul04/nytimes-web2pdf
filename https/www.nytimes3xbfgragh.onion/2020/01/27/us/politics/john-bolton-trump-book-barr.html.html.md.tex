Sections

SEARCH

\protect\hyperlink{site-content}{Skip to
content}\protect\hyperlink{site-index}{Skip to site index}

\href{https://www.nytimes3xbfgragh.onion/section/politics}{Politics}

\href{https://myaccount.nytimes3xbfgragh.onion/auth/login?response_type=cookie\&client_id=vi}{}

\href{https://www.nytimes3xbfgragh.onion/section/todayspaper}{Today's
Paper}

\href{/section/politics}{Politics}\textbar{}Bolton Was Concerned That
Trump Did Favors for Autocratic Leaders, Book Says

\url{https://nyti.ms/2RQw4ay}

\begin{itemize}
\item
\item
\item
\item
\item
\item
\end{itemize}

Advertisement

\protect\hyperlink{after-top}{Continue reading the main story}

Supported by

\protect\hyperlink{after-sponsor}{Continue reading the main story}

\hypertarget{bolton-was-concerned-that-trump-did-favors-for-autocratic-leaders-book-says}{%
\section{Bolton Was Concerned That Trump Did Favors for Autocratic
Leaders, Book
Says}\label{bolton-was-concerned-that-trump-did-favors-for-autocratic-leaders-book-says}}

The former national security adviser shared his unease with the attorney
general, who cited his own worries about the president's conversations
with the leaders of Turkey and China.

\includegraphics{https://static01.graylady3jvrrxbe.onion/images/2020/01/27/us/politics/27dc-bolton/27dc-bolton-articleLarge.jpg?quality=75\&auto=webp\&disable=upscale}

\href{https://www.nytimes3xbfgragh.onion/by/michael-s-schmidt}{\includegraphics{https://static01.graylady3jvrrxbe.onion/images/2018/06/12/multimedia/author-michael-s-schmidt/author-michael-s-schmidt-thumbLarge.png}}\href{https://www.nytimes3xbfgragh.onion/by/maggie-haberman}{\includegraphics{https://static01.graylady3jvrrxbe.onion/images/2018/07/12/multimedia/author-maggie-haberman/author-maggie-haberman-thumbLarge.png}}

By
\href{https://www.nytimes3xbfgragh.onion/by/michael-s-schmidt}{Michael
S. Schmidt} and
\href{https://www.nytimes3xbfgragh.onion/by/maggie-haberman}{Maggie
Haberman}

\begin{itemize}
\item
  Published Jan. 27, 2020Updated June 17, 2020
\item
  \begin{itemize}
  \item
  \item
  \item
  \item
  \item
  \item
  \end{itemize}
\end{itemize}

WASHINGTON ---
\href{https://www.nytimes3xbfgragh.onion/2020/06/17/us/politics/bolton-book-trump-impeached.html}{John
R. Bolton}, the former national security adviser, privately told
Attorney General William P. Barr last year that he had concerns that
President Trump was effectively granting personal favors to the
autocratic leaders of Turkey and China, according to an unpublished
manuscript by Mr. Bolton.

Mr. Barr responded by pointing to a pair of Justice Department
investigations of companies in those countries and said he was worried
that Mr. Trump had created the appearance that he had undue influence
over what would typically be independent inquiries, according to the
manuscript. Backing up his point, Mr. Barr mentioned conversations Mr.
Trump had with the leaders, President Recep Tayyip Erdogan of Turkey and
President Xi Jinping of China.

Mr. Bolton's account underscores the fact that the unease about Mr.
Trump's seeming embrace of authoritarian leaders, long expressed by
experts and his opponents, also existed among some of the senior cabinet
officers entrusted by the president to carry out his foreign policy and
national security agendas.

Mr. Bolton recounted his discussion with Mr. Barr in a
\href{https://www.nytimes3xbfgragh.onion/2020/01/26/us/politics/trump-bolton-book-ukraine.html}{draft
of an unpublished book manuscript} that he submitted nearly a month ago
to the White House for review. People familiar with the manuscript
described its contents on the condition of anonymity.

The book also contains an account of Mr. Trump telling Mr. Bolton in
August that he wanted to continue freezing \$391 million in security
assistance to Ukraine until officials there helped with investigations
of political rivals,
\href{https://www.nytimes3xbfgragh.onion/2020/01/26/us/politics/trump-bolton-book-ukraine.html}{The
New York Times reported} on Sunday. The matter is at the heart of the
articles of impeachment against the president.

Early Tuesday, the Justice Department's spokeswoman, Kerri Kupec,
\href{https://twitter.com/KerriKupecDOJ/status/1222024338469199873?s=20}{posted
a statement on Twitter} disputing aspects of Mr. Bolton's account.

``There was no discussion of `personal favors' or `undue influence' on
investigations, nor did Attorney General Barr state that the President's
conversations with foreign leaders was improper,'' the statement said.
``If this is truly what Mr. Bolton has written, then it seems he is
attributing to Attorney General Barr his own current views --- views
with which Attorney General Barr does not agree.''

A spokesman for the National Security Council declined to comment on Mr.
Barr's conversations with Mr. Bolton. In a statement on Monday, Mr.
Bolton, his publisher and his literary agency said they had not shared
the manuscript with The Times.

``There was absolutely no coordination with The New York Times or anyone
else regarding the appearance of information about his book, `The Room
Where It Happened,' at online booksellers,'' Mr. Bolton, Simon \&
Schuster and Javelin said in a joint statement. ``Any assertion to the
contrary is unfounded speculation.''

Dean Baquet, the executive editor of The Times, responded that ``The
Times does not discuss its sources, but I should point out that no one
has questioned the accuracy of our report.''

Mr. Bolton wrote in the manuscript that Mr. Barr singled out Mr. Trump's
conversations with Mr. Xi about the Chinese telecommunications firm ZTE,
which agreed in 2017
\href{https://www.justice.gov/opa/pr/zte-corporation-agrees-plead-guilty-and-pay-over-4304-million-violating-us-sanctions-sending}{to
plead guilty} and pay heavy fines for violating American sanctions on
doing business with North Korea, Iran and other countries. A year later,
Mr. Trump
\href{https://www.nytimes3xbfgragh.onion/2018/06/07/business/us-china-zte-deal.html}{lifted
the sanctions} over objections from his own advisers and Republican
lawmakers.

Mr. Barr also cited remarks Mr. Trump made to Mr. Erdogan in 2018 about
the investigation of Halkbank, Turkey's second-largest state-owned bank.
The Justice Department was scrutinizing Halkbank on fraud and
money-laundering charges for helping Iran evade sanctions imposed by the
Treasury Department.

Mr. Erdogan had been making personal appeals to Mr. Trump to use his
authority to halt any additional enforcement against the bank. In 2018,
Mr. Erdogan told reporters in Turkey that Mr. Trump had promised to
instruct cabinet members to follow through on the matter. The bank had
hired a top Republican fund-raiser to lobby the administration on the
issue.

For months, it looked as though the unusual lobbying effort might
succeed; but in October, the Justice Department
\href{https://www.nytimes3xbfgragh.onion/2019/10/15/us/politics/halkbank-turkey-iran-indictment.html}{indicted
the bank} for aiding Iran. The charges were seen in part as an attempt
by the administration to show that it was taking a tough line on Turkey
amid an outcry over
\href{https://www.nytimes3xbfgragh.onion/2019/10/07/us/politics/turkey-syria-trump.html}{Mr.
Trump's endorsement} of its incursions in Syria.

\includegraphics{https://static01.graylady3jvrrxbe.onion/images/2020/01/29/us/politics/27dc-bolton2-print/merlin_161839077_8563f946-d8ea-41c5-893e-913ad2498312-articleLarge.jpg?quality=75\&auto=webp\&disable=upscale}

\includegraphics{https://static01.graylady3jvrrxbe.onion/images/2017/01/29/podcasts/the-daily-album-art/the-daily-album-art-articleInline-v2.jpg?quality=75\&auto=webp\&disable=upscale}

\hypertarget{listen-to-the-daily-how-john-bolton-could-change-the-impeachment}{%
\subsubsection{Listen to `The Daily': How John Bolton Could Change the
Impeachment}\label{listen-to-the-daily-how-john-bolton-could-change-the-impeachment}}

Details in a coming book by the former national security adviser are
threatening to derail Republican senators' plans for a speedy acquittal.

transcript

Back to The Daily

bars

0:00/23:24

-23:24

transcript

\hypertarget{listen-to-the-daily-how-john-bolton-could-change-the-impeachment-1}{%
\subsection{Listen to `The Daily': How John Bolton Could Change the
Impeachment}\label{listen-to-the-daily-how-john-bolton-could-change-the-impeachment-1}}

\hypertarget{hosted-by-michael-barbaro-produced-by-rachel-quester-and-luke-vander-ploeg-and-edited-by-paige-cowett-and-lisa-chow}{%
\subsubsection{Hosted by Michael Barbaro, produced by Rachel Quester and
Luke Vander Ploeg, and edited by Paige Cowett and Lisa
Chow}\label{hosted-by-michael-barbaro-produced-by-rachel-quester-and-luke-vander-ploeg-and-edited-by-paige-cowett-and-lisa-chow}}

\hypertarget{details-in-a-coming-book-by-the-former-national-security-adviser-are-threatening-to-derail-republican-senators-plans-for-a-speedy-acquittal}{%
\paragraph{Details in a coming book by the former national security
adviser are threatening to derail Republican senators' plans for a
speedy
acquittal.}\label{details-in-a-coming-book-by-the-former-national-security-adviser-are-threatening-to-derail-republican-senators-plans-for-a-speedy-acquittal}}

\begin{itemize}
\item
  michael barbaro\\
  From The New York Times, I'm Michael Barbaro. This is ``The Daily.''
\item
  {[}music{]}\\
  Today: A Times investigation reveals a firsthand account from John
  Bolton directly linking President Trump to the quid pro quo at the
  center of the impeachment. Maggie Haberman and Mike Schmidt on what
  that could mean for the final phase of the Senate trial.

  It's Tuesday, January 28.

  Mike, Maggie, remind us when the discussion of John Bolton as a
  possible witness in the impeachment process starts.
\item
  michael schmidt\\
  So John Bolton left the White House in early September. Trump said he
  was fired. Bolton said he resigned. A week later, we learn about the
  whistleblower's complaint, and at that point, questions start to
  percolate. Why did Bolton resign, and what does he know? And in the
  coming weeks, as the House impeachment investigators summon White
  House officials to answer questions, we start to get different slivers
  ---
\item
  archived recording\\
  Bolton's former aide, that's Fiona Hill, testified yesterday before
  House impeachment investigators.
\end{itemize}

michael schmidt

--- of Bolton's concerns ---

\begin{itemize}
\tightlist
\item
  archived recording\\
  Bolton reportedly called Giuliani, President Trump's personal attorney
  this --- ``a hand grenade.''
\end{itemize}

michael schmidt

--- and preoccupations with what was going on inside the White House.

\begin{itemize}
\tightlist
\item
  archived recording\\
  This is something that Fiona Hill said when she was talking about John
  Bolton, the former national security adviser.
\end{itemize}

michael schmidt

We don't have a full picture, but we're hearing things, like Bolton
saying ---

\begin{itemize}
\tightlist
\item
  archived recording\\
  Basically he said, you go and tell Eisenberg that I am not part of any
  drug deal that Sondland, the U.S. ambassador to the European Union,
  and Mick Mulvaney are cooking up.
\end{itemize}

michael schmidt

--- I didn't want to participate in this drug deal that these
administration officials were doing.

\begin{itemize}
\tightlist
\item
  archived recording\\
  The term ``drug deal'' here refers to the Ukraine probe that they were
  trying to initiate.
\end{itemize}

michael barbaro

Right, and that was his way of referring to this pressure campaign
against Ukraine to start investigations into Democratic rivals.

michael schmidt

Correct. So we're learning these different things, but we're not hearing
from Bolton.

\begin{itemize}
\tightlist
\item
  archived recording\\
  Let me read it one more time. ``Ambassador Taylor recalls that Mr.
  Morrison told Ambassador Taylor that I told Mr. Morrison that I
  conveyed this message to Mr. Yermak on September 1, 2019, in
  connection with Vice President ---''
\end{itemize}

michael schmidt

And one of the problems with the impeachment investigation ---

\begin{itemize}
\tightlist
\item
  archived recording\\
  We've got six people having four conversations in one sentence, and
  you just told me this is where you got your clear understanding.
\end{itemize}

michael schmidt

--- was that the House was only really talking to people who were sort
of outside the president's inner ring.

\begin{itemize}
\item
  archived recording\\
  Ambassador, you weren't on the call, were you? You didn't listen in on
  President Trump's call and President Zelensky's call?
\item
  archived recording (william b. taylor jr.)\\
  I did not.
\item
  archived recording\\
  You've never talked with Chief of Staff Mulvaney?
\item
  archived recording (william b. taylor jr.)\\
  I never did.
\item
  archived recording\\
  You never met the president?
\item
  archived recording (william b. taylor jr.)\\
  That's correct.
\item
  archived recording\\
  This is what I can't believe, and you're their star witness. You're
  their first witness.
\end{itemize}

michael schmidt

A few of the witnesses dealt with the president directly, but many of
them were simply relaying what was going on inside the White House.

michael barbaro

So Bolton becomes an even more tantalizing figure. He's in the center of
juicy tidbits coming out of the inquiry. He may be on poor terms with
the president. It all seems to make him a perfect witness.

michael schmidt

He's someone who Republicans trust. He has a long history in the
Republican Party, and we were hearing then that he had these concerns.
So what was it that he saw? What was it that he could add? He was in the
room with the president. What did the president tell him?

michael barbaro

So Maggie, what efforts are made by House impeachment investigators to
get Bolton to testify, to get him to just spill the beans?

maggie haberman

They asked him, back in I believe it was October, to come testify
voluntarily. He said no, and he had been ordered by the White House not
to take part, but the House decided not to submit a subpoena to try to
force him to testify, because ---

michael barbaro

Why not?

maggie haberman

Because they were concerned that it was going to be a protracted legal
battle. They were very consumed with trying to wrap this all up quickly.
In hindsight, a lot of Democrats say, at least privately, they think
that that was a mistake. They think that they should have actually tried
to get him to come there.

michael barbaro

Right. So he does not end up testifying before the House.

maggie haberman

No. Bolton never spoke before the House and didn't indicate that he
really wanted to at the time. He just said that he would not cooperate
with this request for testimony. So they sent over the articles of
impeachment without having a witness like Bolton, somebody who had a
direct conversation with the president, where the withheld military aid
for Ukraine was tied to the president's desires for investigations.
There were just people who were speculating on motives or had heard
things secondhand, but there was no one with a firsthand interaction
with the president.

michael barbaro

O.K. So that brings us to November.

maggie haberman

Right, and so we get to November, and John Bolton is starting to make
noises, like he has something to say and he's willing to share it. And
we learn on November 10, that one place he might be planning to share it
is in a book that he's planning to write about his time in the White
House.

michael schmidt

So it was a pretty odd situation. You had House investigators that
wanted Bolton to talk. Bolton sort of signaling that he has something to
say. And then the news that he's writing a book that you presume is
going to have some Ukraine details in it. So who's going to get to that
information?

michael barbaro

Right. So in this situation, what do you two do as reporters to try to
figure out what he knows and maybe what he's put in this book?

maggie haberman

So the House inquiry is over, but there's all this secrecy around this
book. We knew it was coming. We had heard Simon \& Schuster would be
putting it out. They wouldn't even confirm that. We were scratching
around with people who might know. And as we were trying to do this,
Bolton then says, on January 6 --- after not complying with the House
efforts to get him to testify --- he says that he would be willing to
testify in the Senate, if there is a subpoena. It seemed like he was
trying to do a dance, where he was trying not to make Senate Republicans
angry at him, when he's worked with them for years. And he needs them to
back him, as he's embarking on this post-White House life and trying to
sell a book. But also trying to look like he was doing the right thing
and not just making it about the book. And it was really hard to
decipher what his motives were.

michael schmidt

As reporters, there is nothing that galvanizes us like a high-profile
public figure in a major story saying, I have important information, but
I'm not going to tell. And he's essentially out there doing that ---
putting the bait for reporters to try and get to the bottom of what's in
the book.

maggie haberman

So we did what we normally do when we are handed some kind of bait,
which is we continued to try to figure out what was there and what was
in it. And whether it would include some damaging information about the
president. Or whether it would include some exculpatory information and
would be something that the president's folks could even point to and
say it would help him. We just, we didn't know, but we kept scratching.

michael barbaro

Right. If the one great unanswered question was what Bolton knew, then
the most obvious thing is to find out what is inside the book.

maggie haberman

That's right, and we kept scratching and looking, and then we found out
what was in the book, and it was quite damaging to the president.

{[}music{]}

michael barbaro

We'll be right back.

\begin{itemize}
\item
  archived recording 1\\
  Brand new reaction this morning from a bombshell New York Times report
  on John Bolton's upcoming book.
\item
  archived recording 2\\
  A trial that seemed to be on a steady and speedy path to certain
  acquittal has been hit by a seismic shock.
\item
  archived recording 3\\
  Startling new report could upend the impeachment trial. According to
  The New York Times, former ---
\end{itemize}

michael barbaro

So what did you learn was actually in this book?

maggie haberman

The biggest thing that is in there is that Bolton writes about a
conversation that he claims to have had with President Trump in August
of 2019, where he pushed the issue of this withheld military aid with
the president. And the president suggested he didn't want to end the aid
freeze until Ukraine turned over materials that he wanted in connection
with investigations into Democrats, who he thought had harmed him in
2016.

michael barbaro

So Bolton is having a conversation --- he recounts in this book --- with
President Trump in which Bolton says, hey, Mr. President, I want to talk
about this financial freeze on military aid to Ukraine, presumably in
the context of Bolton wanting to end it.

maggie haberman

Bolton pushed this conversation with the president, because he, along
with the Secretary of State and the Secretary of Defense, had been
trying to get the president for weeks to end the freeze and turn the aid
over, arguing that it was necessary for Ukraine to defend itself against
Russia. And so Bolton raised this issue to test where the president was,
and the president met him back by saying he didn't want to end this
freeze until materials that he wanted were turned over in relation to
investigations into Democrats he thought had damaged him.

michael barbaro

Wow. Democrats, including Joe Biden.

maggie haberman

Democrats, including Joe Biden and his son Hunter Biden.

michael barbaro

So Bolton is confirming in this account that President Trump articulated
--- correct me if I'm wrong --- a clear quid pro quo that explained why
he was holding up the security aid to Ukraine. That it was in return for
Ukraine investigating his Democratic rivals, like Joe Biden, and that of
course is the central claim of the articles of impeachment. And what
Bolton seems to be saying here is that that central claim is accurate,
and there's now a firsthand account of it from the mouth of the
president himself.

michael schmidt

But you say confirms. This is the first time that we're hearing anyone
say this. This is the first time that someone who was in the room, who
spoke directly to the president, says, yeah, the president didn't want
to release the money until he got the fruits of the investigations. This
is new ground, and it's significant, because the president's lawyers
have time and time again argued on the Senate floor ---

\begin{itemize}
\tightlist
\item
  archived recording (michael purpura)\\
  --- that there was no connection between security assistance and
  investigations.
\end{itemize}

michael schmidt

--- that the aid and the investigations were not linked.

\begin{itemize}
\tightlist
\item
  archived recording (michael purpura)\\
  --- the pause on security assistance was distinct and unrelated to
  investigations.
\end{itemize}

michael barbaro

So this directly contradicts the way the president's own lawyers talk
about the impeachment.

maggie haberman

Correct, or at least it undercuts their main argument, which is that
there was not a connection between what the president wanted and
releasing this aid.

michael barbaro

Am I right, Mike and Maggie, that this is as close to a smoking gun as
it gets in a case like this?

michael schmidt

Maybe, but throughout the Trump presidency, we've learned similarly
explosive disclosures, and the president has been able to weather them
politically. So yeah, in a normal time, would the news of the
president's most recent national security adviser directly implicating
him in a question that is at the center of an impeachment hearing be a
smoking gun? Sure, but Trump has shown an ability to endure things like
this that gives me reticence to say, yeah, that's a smoking gun. Because
when you say smoking gun, built into that is an assumption that the end
would be near.

maggie haberman

I'm with Mike on that. I think that we are a ways away from knowing what
this means. And as our colleague Peter Baker wrote today, it could end
up being like when the ``Access Hollywood'' tape came out in 2016 in the
campaign. And the big prediction was that this was going to be the end
of Donald Trump, this was going to be the end of his campaign, and it
obviously did not go that way. So we just don't know yet.

michael barbaro

Well, what has been the reaction to this reporting, especially in the
Senate, where the trial is well underway and where the question of
calling witnesses is very much still alive?

michael schmidt

So we're coming into the home stretch of the trial. And the question of
whether Bolton will testify has still not been resolved.

michael barbaro

Right.

michael schmidt

The story comes out, and there's increased pressure on Senate
Republicans, those moderate ones who may be willing to go along with the
Democrats. How much does this story move them?

michael barbaro

And Maggie, what's the answer?

maggie haberman

So far, we are seeing the same moderates who have said they want
witnesses before still say they want to hear from Bolton. So that's Mitt
Romney.

\begin{itemize}
\item
  archived recording\\
  Four of you need to say yes. Do you think there are four votes?
\item
  archived recording (mitt romney)\\
  I think it's increasingly likely that the other Republicans will join
  those of us who think we should hear from John Bolton. And whether
  there are other witnesses and documents, well, that's another matter.
  But I think John Bolton's relevance to our decision has become
  increasingly clear.
\end{itemize}

maggie haberman

Susan Collins of Maine is another person who has said this is another
factor that points to why there should be witnesses. But two other
possible votes for witnesses, one is Lisa Murkowski of Alaska. Andother
is Lamar Alexander, who the White House is watching very closely to see
what he'll do. They have been more circumspect about whether they think
our story changes anything. And so far, Senate Majority Leader Mitch
McConnell is trying to tell everybody to stay cool and just see how this
plays out.

michael barbaro

I mean, there will be people hearing this --- hearing that the national
security adviser to the president observed him saying something that
directly implicates him in this impeachment case --- and will ask, why
would any deliberative body sworn in as jurors not want to hear from
that person? What would be the justification for not hearing from Bolton
in this moment?

maggie haberman

So one thing that has come up from the senators in the last several
hours is they're saying, if this was so important to hear from Bolton,
why didn't the House subpoena him?

\begin{itemize}
\tightlist
\item
  archived recording (john barrasso)\\
  There's nothing new here that John Bolton didn't know before the House
  managers rested their case and stopped calling witnesses, and they
  never chose to call John Bolton.
\end{itemize}

maggie haberman

And you're going to hear that, I think increasingly, if you don't see a
move toward witnesses. That's going to be an argument that senators are
going to point to.

michael barbaro

You missed your chance.

maggie haberman

Why are we doing your work for you? That part was in the House.

michael schmidt

And at another level, the reason why these senators are not going along
with calling Bolton is that Trump doesn't want that to happen. And they
have been in lockstep with Trump for most of this.

michael barbaro

So the justification is keep the president happy.

maggie haberman

I think the justification is you have a lot of senators who are facing
elections in their own states, and the base likes Trump. And in some
states, like Lamar Alexander's state, they want him to be loyal to
Trump. And so those are the concerns that they're measuring, --- is do
they let themselves be looked back on in history as turning away from
evidence, which some people will say they did. Or do they say, voters
don't really want me casting that vote, and they decide to stick with
the president's desires.

michael barbaro

So how has all of this actually landed inside the White House?

maggie haberman

Well, the White House as a whole wasn't happy about hearing about this.
But for at least some of them, it wasn't a surprise, because the White
House had been given a draft of this manuscript about 3.5 weeks ago from
John Bolton for a standard review process to look at classified
information and whether there is any in the book.

michael schmidt

So that means at least some folks in the White House have had a sense of
what Bolton would testify to in the impeachment investigation.

michael barbaro

If he testified.

maggie haberman

Correct.

michael barbaro

Wait. So does that mean that the president's lawyers, including those
who are currently defending him in the Senate trial, that they knew what
John Bolton had written and knew what John Bolton had experienced, and
then continued to make a case to the public that is quite contradictory
to what Bolton is saying happened in this book?

michael schmidt

We don't know the extent to which the manuscript, or the details about
it, were circulated. But what we do know is that in the past several
weeks, there has been a concerted effort by the president to stop Bolton
from testifying. He's made public statements about this.

\begin{itemize}
\tightlist
\item
  archived recording (donald trump)\\
  The problem with John is that it's a national security problem. You
  can't have somebody who's at national security. And if you think about
  it, John, he knows some of my thoughts. He knows what I think about
  leaders. What happens if he reveals what I think about a certain
  leader and it's not very positive, and then I have to deal on behalf
  of the country? It's going to be very hard. It's going to make the job
  very hard. He knows other things, and I don't know if we left on the
  best of terms. I would say, probably not. And so you don't like people
  testifying when they didn't leave on good terms. And that was due to
  me, not due to him. And so we'll see what happens.
\end{itemize}

michael schmidt

And he has said it privately to aides.

michael barbaro

So my final question is, if John Bolton has something to say --- and it
feels like he does --- and the world wants to hear it, senators want to
hear it, House impeachment managers want to hear it, we all want to hear
it. And it feels like he has an obligation to the Democratic process to
say it, why doesn't he just find a way to say it? Go on Fox. Go on CNN.
Have a news conference. Why hasn't he taken any of those opportunities?

maggie haberman

It's a great question, and there's nothing preventing him from doing so.
If he wanted to issue some kind of a statement or say something publicly
that didn't violate executive privilege with the president, he could do
that. He has yet to do any of that, and it's not really clear why.

michael schmidt

But you know what, if John Bolton went on television right now and said
everything he would testify to, unless he was subpoenaed to appear at
that trial or if the comments from him were put into evidence, then it
couldn't be considered by the lawmakers. This is a trial, where evidence
is brought forward. And if there is not enough votes to bring that
evidence in, then it doesn't matter whether he stands out on the highest
point in town and says everything he knows. It only matters whether it's
entered into the record in the Senate.

{[}music{]}

michael barbaro

Maggie and Mike, thank you.

maggie haberman

Thank you.

michael schmidt

Thanks for having us.

michael barbaro

On Monday, a lawyer for the White House, Patrick Philbin, tried to tamp
down talk of calling Bolton as a witness, saying that calling such a
witness would be an effort to, quote, ``redo'' the House impeachment
inquiry and would set a dangerous precedent for future impeachment
trials. The record that the House Democrats collected during that
process, Philbin said, shows that the president did nothing wrong. In a
tweet, the president denied Bolton's account of their conversation about
Ukraine, writing, quote, ``If John Bolton said this, it was only to sell
a book.''

We'll be right back.

{[}music{]}

michael barbaro

Here's what else you need to know today. On Monday, as it sought to
contain the coronavirus, the Chinese government broadened its quarantine
to more than 50 million people and said that it would spend at least \$9
billion to stop the outbreak. The U.S. government said it was organizing
an evacuation of American citizens out of the epicenter of the illness
in Wuhan. And the pharmaceutical company Johnson \& Johnson said it
would begin developing a potential vaccine against the virus, joining
several government agencies seeking to do the same. As of Monday night,
the coronavirus had infected nearly 3,000 people and killed more than
80. That's it for ``The Daily.'' I'm Michael Barbaro. See you tomorrow.

Mr. Bolton's statements in the book align with other comments he has
made since leaving the White House in September. In November, he said in
a private speech that none of Mr. Trump's advisers shared the
president's views on Turkey and that he believed Mr. Trump adopted a
more permissive approach to the country because of his financial ties
there,
\href{https://www.nbcnews.com/politics/donald-trump/private-speech-bolton-suggests-some-trump-s-foreign-policy-decisions-n1080651}{NBC
News reported}. Mr. Trump's company has a property in Turkey.

Mr. Trump has repeatedly praised dictators throughout his presidency.
Last year, he said, ``Where's my favorite dictator?'' as he waited to
meet with President Abdel Fattah el-Sisi of Egypt,
\href{https://www.wsj.com/articles/trump-awaiting-egyptian-counterpart-at-summit-called-out-for-my-favorite-dictator-11568403645}{The
Wall Street Journal reported}.

Mr. Trump's soft spot for authoritarians dates at least to his
presidential campaign, when
\href{https://www.nytimes3xbfgragh.onion/2016/07/06/us/politics/donald-trump-saddam-hussein.html}{he
praised} Saddam Hussein for being ``good'' at killing terrorists and
suggested that the world would be better off were Col. Muammar
el-Qaddafi, the deposed Libyan dictator who was killed in a violent
uprising in 2011, ``in charge right now.'' Mr. Trump then suggested the
ouster of both men was ultimately worse for the Middle East because the
Islamic State had filled the void.

Mr. Trump declared himself ``a big fan'' of Mr. Erdogan as they sat side
by side in the Oval Office last fall after Mr. Trump cleared the way for
Turkish forces to invade Syria, though he warned Mr. Erdogan behind the
scenes against the offensive.

Of Mr. Xi, Mr. Trump has been similarly effusive. When the Chinese
Communist Party eliminated term limits, allowing Mr. Xi to keep his
tenure open-ended, Mr. Trump
\href{https://www.cnn.com/2018/03/03/politics/trump-maralago-remarks/index.html}{extolled
the outcome}.

Mr. Xi had personally asked Mr. Trump to intervene to save ZTE, which
was on the brink of collapse because of tough American penalties for
sanctions violations.

\href{https://www.nytimes3xbfgragh.onion/2018/06/07/business/us-china-zte-deal.html}{Lifting
the sanctions on ZTE}, a Chinese telecommunications giant that also
serves as a geopolitical pawn for its government, most likely helped Mr.
Trump negotiate with Mr. Xi in the trade war between the two countries.
But Republican lawmakers and others objected to helping a Chinese
company that broke the law and has been accused of posing a national
security threat.

Mr. Bolton's reputation for muscular foreign policy was always an odd
fit with Mr. Trump, who often threatens excessive force but rarely
reacts with it. Mr. Bolton was pleased when Mr. Trump withdrew from the
nuclear deal between Iran and six world powers, including the United
States, that the Obama administration had entered into. Other Trump
advisers had urged him against it.

But Mr. Trump's lack of action after Iranian aggression against the
United States rankled Mr. Bolton.

Mr. Bolton's book has already netted significant sales. Shortly after
the disclosure of its contents on Sunday night, Amazon listed the book
for purchase. By Monday evening, it was No. 17 on Amazon's best-seller
list.

Eric Lipton contributed reporting.

Advertisement

\protect\hyperlink{after-bottom}{Continue reading the main story}

\hypertarget{site-index}{%
\subsection{Site Index}\label{site-index}}

\hypertarget{site-information-navigation}{%
\subsection{Site Information
Navigation}\label{site-information-navigation}}

\begin{itemize}
\tightlist
\item
  \href{https://help.nytimes3xbfgragh.onion/hc/en-us/articles/115014792127-Copyright-notice}{©~2020~The
  New York Times Company}
\end{itemize}

\begin{itemize}
\tightlist
\item
  \href{https://www.nytco.com/}{NYTCo}
\item
  \href{https://help.nytimes3xbfgragh.onion/hc/en-us/articles/115015385887-Contact-Us}{Contact
  Us}
\item
  \href{https://www.nytco.com/careers/}{Work with us}
\item
  \href{https://nytmediakit.com/}{Advertise}
\item
  \href{http://www.tbrandstudio.com/}{T Brand Studio}
\item
  \href{https://www.nytimes3xbfgragh.onion/privacy/cookie-policy\#how-do-i-manage-trackers}{Your
  Ad Choices}
\item
  \href{https://www.nytimes3xbfgragh.onion/privacy}{Privacy}
\item
  \href{https://help.nytimes3xbfgragh.onion/hc/en-us/articles/115014893428-Terms-of-service}{Terms
  of Service}
\item
  \href{https://help.nytimes3xbfgragh.onion/hc/en-us/articles/115014893968-Terms-of-sale}{Terms
  of Sale}
\item
  \href{https://spiderbites.nytimes3xbfgragh.onion}{Site Map}
\item
  \href{https://help.nytimes3xbfgragh.onion/hc/en-us}{Help}
\item
  \href{https://www.nytimes3xbfgragh.onion/subscription?campaignId=37WXW}{Subscriptions}
\end{itemize}
