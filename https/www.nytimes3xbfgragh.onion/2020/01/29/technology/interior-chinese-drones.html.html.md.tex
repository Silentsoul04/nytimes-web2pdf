Sections

SEARCH

\protect\hyperlink{site-content}{Skip to
content}\protect\hyperlink{site-index}{Skip to site index}

\href{https://www.nytimes3xbfgragh.onion/section/technology}{Technology}

\href{https://myaccount.nytimes3xbfgragh.onion/auth/login?response_type=cookie\&client_id=vi}{}

\href{https://www.nytimes3xbfgragh.onion/section/todayspaper}{Today's
Paper}

\href{/section/technology}{Technology}\textbar{}Interior Dept. Grounds
Its Drones Over Chinese Spying Fears

\url{https://nyti.ms/38TvKPj}

\begin{itemize}
\item
\item
\item
\item
\item
\end{itemize}

Advertisement

\protect\hyperlink{after-top}{Continue reading the main story}

Supported by

\protect\hyperlink{after-sponsor}{Continue reading the main story}

\hypertarget{interior-dept-grounds-its-drones-over-chinese-spying-fears}{%
\section{Interior Dept. Grounds Its Drones Over Chinese Spying
Fears}\label{interior-dept-grounds-its-drones-over-chinese-spying-fears}}

The order formalizes a decision last year to ground the federal agency's
drones pending an internal security investigation.

\includegraphics{https://static01.graylady3jvrrxbe.onion/images/2020/01/29/us/politics/29dc-drones/29dc-drones-articleLarge.jpg?quality=75\&auto=webp\&disable=upscale}

By \href{https://www.nytimes3xbfgragh.onion/by/lisa-friedman}{Lisa
Friedman} and David McCabe

\begin{itemize}
\item
  Jan. 29, 2020
\item
  \begin{itemize}
  \item
  \item
  \item
  \item
  \item
  \end{itemize}
\end{itemize}

WASHINGTON --- The Interior Department announced on Wednesday that it
was grounding its entire fleet of drones out of concerns that Chinese
parts in them might be used for spying, making exceptions only for
emergency missions like fighting wildfires and search-and-rescue
operations.

The move, an extension of an order made last year, reflects concerns
that drones made in China could expose sensitive data. Unease is growing
in Washington about potential security vulnerabilities presented by
Chinese technology, though Chinese companies have denied that their
products pose a security threat.

Last year, pending an internal security investigation, the agency
\href{https://www.nytimes3xbfgragh.onion/2019/10/30/us/politics/interior-department-chinese-made-drones.html}{temporarily
grounded its drones}, which are used for surveying critical
infrastructure like dams, collecting information about endangered
species, conducting search-and-rescue operations and tracking wildfires.

The new order, signed by David Bernhardt, the secretary of the interior,
says the current fleet of 810 drones will remain grounded ``while we
ensure that cybersecurity, technology and domestic production concerns
are adequately addressed.'' The order does not explicitly mention China,
but a senior administration official said it was ``without question''
aimed at drones made or assembled in China.

DJI, a privately held Chinese company whose drones are used by the
Interior Department, said in a statement that it was ``disappointed''
with the new order and accused the Trump administration of political
motives.

``This decision makes clear that the U.S. government's concerns about
DJI drones, which make up a small portion of the D.O.I. fleet, have
little to do with security and are instead part of a politically
motivated agenda to reduce market competition and support domestically
produced drone technology, regardless of its merits,'' Michael
Oldenburg, a DJI spokesman, said.

The grounding order does not apply to private drones used for personal
or commercial purposes.

A senior administration official said the Department of Defense and
intelligence agencies would assist the Interior Department with a review
of its drones that would include a ``significant tear down'' to the nuts
and bolts of the machines.

In the meantime, the agency will allow drones to be used for emergency
situations like disaster monitoring, and will make exceptions for
training flights.

According to the Department of Interior, since the agency grounded its
fleet in October, it has conducted 12 drone flights, all related to
firefighting missions or flood monitoring. The new order does not define
what will be included in emergency operations, but one official said it
could include monitoring for cracks in dams, which would be considered a
safety issue.

Other drone missions, like conducting geological surveys, studying
habitats and monitoring the breeding grounds of the sage grouse, an
imperiled ground-nesting bird that is found across millions of acres of
oil- and gas-rich sagebrush lands, will be conducted by airplane or
helicopter, the official said.

The move is the latest by the federal government to target Chinese
technology firms.

In recent years, regulators have
\href{https://www.nytimes3xbfgragh.onion/2019/11/22/technology/huawei-funds-cut-fcc.html}{cracked
down} on Chinese wireless network equipment and
\href{https://www.nytimes3xbfgragh.onion/2019/03/28/us/politics/grindr-china-national-security.html}{expressed
concerns} about the national security implications of Chinese companies'
operation of consumer mobile applications in the United States. They
have also
\href{https://www.nytimes3xbfgragh.onion/2019/06/21/us/politics/us-china-trade-blacklist.html}{made
it difficult} for American companies to supply certain Chinese firms.

DJI's drones --- which are popular with both hobbyists and public safety
officials --- have been one persistent point of contention. The company
is seen as the market leader, with analysts at times estimating its
market share at 70 percent or higher.

In 2017, federal officials
\href{https://www.nytimes3xbfgragh.onion/2017/11/29/technology/dji-china-data-drones.html}{said
they were worried} that DJI drones were sending data back to China,
which the firm strenuously denied, and the Army
\href{https://www.defenseone.com/technology/2017/08/us-army-just-ordered-soldiers-stop-using-drones-chinas-dji/139999/?oref=defense_one_breaking_nl}{ordered
its employees} not to use the company's products. The fear is that the
Chinese government is seeing what the United States government is seeing
through DJI drone flights.

The company has taken steps since then to reassure American officials.
Last year, it
\href{https://www.nytimes3xbfgragh.onion/2019/06/24/technology/dji-china-drones-security-us.html}{announced}
that it was moving a small part of its production to Cerritos, Calif. It
also rolled out a version of its drone specifically for government use.

Mr. Oldenburg said the company's technology designed for United States
government agencies had been ``independently tested and validated'' by
security consultants and federal officials, ``which proves today's
decision has nothing to do with security.''

Advertisement

\protect\hyperlink{after-bottom}{Continue reading the main story}

\hypertarget{site-index}{%
\subsection{Site Index}\label{site-index}}

\hypertarget{site-information-navigation}{%
\subsection{Site Information
Navigation}\label{site-information-navigation}}

\begin{itemize}
\tightlist
\item
  \href{https://help.nytimes3xbfgragh.onion/hc/en-us/articles/115014792127-Copyright-notice}{©~2020~The
  New York Times Company}
\end{itemize}

\begin{itemize}
\tightlist
\item
  \href{https://www.nytco.com/}{NYTCo}
\item
  \href{https://help.nytimes3xbfgragh.onion/hc/en-us/articles/115015385887-Contact-Us}{Contact
  Us}
\item
  \href{https://www.nytco.com/careers/}{Work with us}
\item
  \href{https://nytmediakit.com/}{Advertise}
\item
  \href{http://www.tbrandstudio.com/}{T Brand Studio}
\item
  \href{https://www.nytimes3xbfgragh.onion/privacy/cookie-policy\#how-do-i-manage-trackers}{Your
  Ad Choices}
\item
  \href{https://www.nytimes3xbfgragh.onion/privacy}{Privacy}
\item
  \href{https://help.nytimes3xbfgragh.onion/hc/en-us/articles/115014893428-Terms-of-service}{Terms
  of Service}
\item
  \href{https://help.nytimes3xbfgragh.onion/hc/en-us/articles/115014893968-Terms-of-sale}{Terms
  of Sale}
\item
  \href{https://spiderbites.nytimes3xbfgragh.onion}{Site Map}
\item
  \href{https://help.nytimes3xbfgragh.onion/hc/en-us}{Help}
\item
  \href{https://www.nytimes3xbfgragh.onion/subscription?campaignId=37WXW}{Subscriptions}
\end{itemize}
