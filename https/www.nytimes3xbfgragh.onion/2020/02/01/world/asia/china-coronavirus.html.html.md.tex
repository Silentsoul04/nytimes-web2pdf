Sections

SEARCH

\protect\hyperlink{site-content}{Skip to
content}\protect\hyperlink{site-index}{Skip to site index}

\href{https://www.nytimes3xbfgragh.onion/section/world/asia}{Asia
Pacific}

\href{https://myaccount.nytimes3xbfgragh.onion/auth/login?response_type=cookie\&client_id=vi}{}

\href{https://www.nytimes3xbfgragh.onion/section/todayspaper}{Today's
Paper}

\href{/section/world/asia}{Asia Pacific}\textbar{}As New Coronavirus
Spread, China's Old Habits Delayed Fight

\url{https://nyti.ms/37Q9QMy}

\begin{itemize}
\item
\item
\item
\item
\item
\item
\end{itemize}

\hypertarget{the-coronavirus-outbreak}{%
\subsubsection{\texorpdfstring{\href{https://www.nytimes3xbfgragh.onion/news-event/coronavirus?name=styln-coronavirus-national\&region=TOP_BANNER\&variant=undefined\&block=storyline_menu_recirc\&action=click\&pgtype=Article\&impression_id=690d9df0-e38f-11ea-8aa7-fb86b0867258}{The
Coronavirus
Outbreak}}{The Coronavirus Outbreak}}\label{the-coronavirus-outbreak}}

\begin{itemize}
\tightlist
\item
  live\href{https://www.nytimes3xbfgragh.onion/2020/08/20/world/coronavirus-covid.html?name=styln-coronavirus-national\&region=TOP_BANNER\&variant=undefined\&block=storyline_menu_recirc\&action=click\&pgtype=Article\&impression_id=690d9df1-e38f-11ea-8aa7-fb86b0867258}{Latest
  Updates}
\item
  \href{https://www.nytimes3xbfgragh.onion/interactive/2020/us/coronavirus-us-cases.html?name=styln-coronavirus-national\&region=TOP_BANNER\&variant=undefined\&block=storyline_menu_recirc\&action=click\&pgtype=Article\&impression_id=690d9df2-e38f-11ea-8aa7-fb86b0867258}{Maps
  and Cases}
\item
  \href{https://www.nytimes3xbfgragh.onion/interactive/2020/science/coronavirus-vaccine-tracker.html?name=styln-coronavirus-national\&region=TOP_BANNER\&variant=undefined\&block=storyline_menu_recirc\&action=click\&pgtype=Article\&impression_id=690d9df3-e38f-11ea-8aa7-fb86b0867258}{Vaccine
  Tracker}
\item
  \href{https://www.nytimes3xbfgragh.onion/2020/08/19/us/colleges-closing-covid.html?name=styln-coronavirus-national\&region=TOP_BANNER\&variant=undefined\&block=storyline_menu_recirc\&action=click\&pgtype=Article\&impression_id=690d9df4-e38f-11ea-8aa7-fb86b0867258}{Colleges
  Closing}
\item
  \href{https://www.nytimes3xbfgragh.onion/live/2020/08/20/business/stock-market-today-coronavirus?name=styln-coronavirus-national\&region=TOP_BANNER\&variant=undefined\&block=storyline_menu_recirc\&action=click\&pgtype=Article\&impression_id=690d9df5-e38f-11ea-8aa7-fb86b0867258}{Economy}
\end{itemize}

Advertisement

\protect\hyperlink{after-top}{Continue reading the main story}

Supported by

\protect\hyperlink{after-sponsor}{Continue reading the main story}

\hypertarget{as-new-coronavirus-spread-chinas-old-habits-delayed-fight}{%
\section{As New Coronavirus Spread, China's Old Habits Delayed
Fight}\label{as-new-coronavirus-spread-chinas-old-habits-delayed-fight}}

At critical turning points, Chinese authorities put secrecy and order
ahead of openly confronting the growing crisis and risking public alarm
or political embarrassment.

\includegraphics{https://static01.graylady3jvrrxbe.onion/images/2020/03/01/world/01virus-reconstruct-top-sub/01virus-reconstruct-top-sub-articleLarge.jpg?quality=75\&auto=webp\&disable=upscale}

\href{https://www.nytimes3xbfgragh.onion/by/chris-buckley}{\includegraphics{https://static01.graylady3jvrrxbe.onion/images/2018/10/08/multimedia/author-chris-buckley/author-chris-buckley-thumbLarge.png}}\href{https://www.nytimes3xbfgragh.onion/by/steven-lee-myers}{\includegraphics{https://static01.graylady3jvrrxbe.onion/images/2018/10/15/multimedia/author-steven-lee-myers/author-steven-lee-myers-thumbLarge.png}}

By \href{https://www.nytimes3xbfgragh.onion/by/chris-buckley}{Chris
Buckley} and
\href{https://www.nytimes3xbfgragh.onion/by/steven-lee-myers}{Steven Lee
Myers}

\begin{itemize}
\item
  Published Feb. 1, 2020Updated Feb. 7, 2020
\item
  \begin{itemize}
  \item
  \item
  \item
  \item
  \item
  \item
  \end{itemize}
\end{itemize}

\href{https://cn.nytimes3xbfgragh.onion/china/20200203/china-coronavirus/}{阅读简体中文版}\href{https://cn.nytimes3xbfgragh.onion/china/20200203/china-coronavirus/zh-hant/}{閱讀繁體中文版}

WUHAN, China --- A
\href{https://www.nytimes3xbfgragh.onion/2020/02/01/world/asia/coronavirus-china.html}{mysterious
illness} had stricken seven patients at a hospital, and a doctor
\href{https://www.nytimes3xbfgragh.onion/2020/01/27/world/asia/27china-coronavirus-health.html}{tried
to warn} his medical school classmates. ``Quarantined in the emergency
department,'' the doctor, Li Wenliang, wrote in an online chat group on
Dec. 30, referring to patients.

``So frightening,'' one recipient replied, before asking about the
epidemic that
\href{https://www.nytimes3xbfgragh.onion/2003/04/27/world/the-sars-epidemic-the-path-from-china-s-provinces-a-crafty-germ-breaks-out.html}{began
in China in 2002} and ultimately killed nearly 800 people. ``Is SARS
coming again?''

In the middle of the night, officials from the health authority in the
central city of Wuhan summoned Dr. Li, demanding to know why he had
\href{https://www.nytimes3xbfgragh.onion/2020/01/27/technology/china-coronavirus-censorship-social-media.html}{shared
the information}. Three days later, the police compelled him to sign a
statement that his warning constituted ``illegal behavior.''

The illness was not SARS, but something similar: a coronavirus that is
now on a relentless march outward from Wuhan, throughout the country and
across the globe, killing at least 304 people in China and infecting
more than 14,380 worldwide.

\href{https://www.nytimes3xbfgragh.onion/interactive/2020/world/coronavirus-maps.html}{}

\includegraphics{https://static01.graylady3jvrrxbe.onion/images/2020/03/03/world/coronavirus-map-promo/coronavirus-map-promo-articleLarge-v723.png}

\hypertarget{coronavirus-map-tracking-the-global-outbreak}{%
\subsection{Coronavirus Map: Tracking the Global
Outbreak}\label{coronavirus-map-tracking-the-global-outbreak}}

The virus has infected more than 22,659,500 people and has been detected
in nearly every country.

The government's initial handling of the epidemic allowed the virus to
gain a tenacious hold. At critical moments, officials chose to put
secrecy and order ahead of openly confronting the growing crisis to
avoid public alarm and political embarrassment.

A reconstruction of the crucial seven weeks between the appearance of
the
\href{https://www.thelancet.com/journals/lancet/article/PIIS0140-6736(20)30183-5/fulltext}{first}symptoms
in early December and the government's decision to lock down the city,
based on two dozen interviews with Wuhan residents, doctors and
officials, on government statements and on Chinese media reports, points
to decisions that delayed a concerted public health offensive.

In those weeks, the authorities silenced doctors and others for raising
red flags. They played down the dangers to the public, leaving the
city's 11 million residents unaware they should protect themselves. They
closed a food market where the virus was believed to have started, but
didn't broadly curb the wildlife trade.

\includegraphics{https://static01.graylady3jvrrxbe.onion/images/2020/02/02/world/02virus-reconstruct/02virus-reconstruct-articleLarge-v2.jpg?quality=75\&auto=webp\&disable=upscale}

Their reluctance to go public, in part, played to political motivations
as local officials prepared for their annual congresses in January. Even
as cases climbed, officials declared repeatedly that there had likely
been no more infections.

By not moving aggressively to warn the public and medical professionals,
public health experts say, the Chinese government lost one of its best
chances to keep the disease from becoming an epidemic.

``This was an issue of inaction,'' said Yanzhong Huang, a senior fellow
for global health at the Council on Foreign Relations who studies China.
``There was no action in Wuhan from the local health department to alert
people to the threat.''

The first case, the details of which are limited and the specific date
unknown, was in early December. By the time the authorities galvanized
into action on Jan. 20, the disease had grown into a formidable threat.

Image

Dr. Li Wenliang

It is now a global health emergency. It has triggered travel
restrictions around the world, shaken
\href{https://www.nytimes3xbfgragh.onion/2020/01/29/business/china-coronavirus-economy.html}{financial
markets} and created perhaps the greatest challenge yet for China's
leader, Xi Jinping. The crisis could upend Mr. Xi's agenda for months or
longer, even undermining his vision of a political system that offers
security and growth in return for submission to iron-fisted
authoritarianism.

On the last day of 2019, after Dr. Li's message was shared outside the
group, the authorities focused on controlling the narrative. The police
announced that they were investigating eight people for spreading rumors
about the outbreak.

\hypertarget{latest-updates-the-coronavirus-outbreak}{%
\section{\texorpdfstring{\href{https://www.nytimes3xbfgragh.onion/2020/08/20/world/coronavirus-covid.html?action=click\&pgtype=Article\&state=default\&region=MAIN_CONTENT_1\&context=storylines_live_updates}{Latest
Updates: The Coronavirus
Outbreak}}{Latest Updates: The Coronavirus Outbreak}}\label{latest-updates-the-coronavirus-outbreak}}

Updated 2020-08-21T09:18:03.769Z

\begin{itemize}
\tightlist
\item
  \href{https://www.nytimes3xbfgragh.onion/2020/08/20/world/coronavirus-covid.html?action=click\&pgtype=Article\&state=default\&region=MAIN_CONTENT_1\&context=storylines_live_updates\#link-68774d88}{Shutdowns,
  warnings and scoldings follow alarming incidents on college campuses.}
\item
  \href{https://www.nytimes3xbfgragh.onion/2020/08/20/world/coronavirus-covid.html?action=click\&pgtype=Article\&state=default\&region=MAIN_CONTENT_1\&context=storylines_live_updates\#link-26b58724}{Biden
  knocks Trump's pandemic response, and outlines a national strategy.}
\item
  \href{https://www.nytimes3xbfgragh.onion/2020/08/20/world/coronavirus-covid.html?action=click\&pgtype=Article\&state=default\&region=MAIN_CONTENT_1\&context=storylines_live_updates\#link-4e542da3}{U.S.
  health agencies announce moves to confront the flu season and
  plummeting child vaccination rates.}
\end{itemize}

\href{https://www.nytimes3xbfgragh.onion/2020/08/20/world/coronavirus-covid.html?action=click\&pgtype=Article\&state=default\&region=MAIN_CONTENT_1\&context=storylines_live_updates}{See
more updates}

More live coverage:
\href{https://www.nytimes3xbfgragh.onion/live/2020/08/20/business/stock-market-today-coronavirus?action=click\&pgtype=Article\&state=default\&region=MAIN_CONTENT_1\&context=storylines_live_updates}{Markets}

That same day, Wuhan's health commission, its hand forced by those
``rumors,'' announced that 27 people were suffering from pneumonia of an
unknown cause. Its statement said there was no need to be alarmed.

``The disease is preventable and controllable,''
the\href{http://wjw.wuhan.gov.cn/front/web/showDetail/2019123108989}{statement}
said.

Dr. Li, an ophthalmologist, went back to work after being reprimanded.
On Jan. 10, he treated a woman for glaucoma. He did not know she had
already been infected with the coronavirus, probably by her daughter.
They both became sick. So would he.

Image

Huanan Seafood Wholesale Market in Wuhan on Jan. 11. It was shut on Jan.
1 for an environmental and hygienic cleanup related to the
outbreak.Credit...Noel Celis/Agence France-Presse --- Getty Images

\hypertarget{hazmat-suits-and-disinfectants}{%
\subsection{Hazmat Suits and
Disinfectants}\label{hazmat-suits-and-disinfectants}}

Hu Xiaohu, who sold processed pork in the Huanan Seafood Wholesale
Market, sensed by late December that something was amiss. Workers were
coming down with nagging fevers. No one knew why but, Mr. Hu said,
several **** were in hospital quarantine.

The market occupies much of a block in a newer part of the city, sitting
incongruously near apartment buildings and shops catering to the growing
middle class. It is a warren of stalls selling meats, poultry and fish,
as well as more exotic fare, including live reptiles and wild game that
some in China prize as delicacies. According to a report by the city's
center for disease control, sanitation was dismal, with poor ventilation
and garbage piled on wet floors.

In hospitals, doctors and nurses were puzzled to see a cluster of
patients with symptoms of a viral pneumonia that did not respond to the
usual treatments. They soon noticed that many patients had one thing in
common: They worked in Huanan market.

On Jan. 1, police officers showed up at the market, along with public
health officials, and shut it down. Local officials issued a notice that
the market was undergoing an environmental and hygienic cleanup related
to the pneumonia outbreak. That morning, workers in hazmat suits moved
in, washing out stalls and spraying disinfectants.

It was, for the public, the first visible government response to contain
the disease. The day before, on Dec. 31, national authorities had
alerted the World Health Organization's office in Beijing of an
outbreak.

\href{https://www.nytimes3xbfgragh.onion/interactive/2020/world/asia/china-coronavirus-contain.html}{}

\includegraphics{https://static01.graylady3jvrrxbe.onion/images/2020/01/30/us/china-coronavirus-contain-promo-1580431440996/china-coronavirus-contain-promo-1580431440996-articleLarge-v9.png}

\hypertarget{how-bad-will-the-coronavirus-outbreak-get-here-are-6-key-factors}{%
\subsection{How Bad Will the Coronavirus Outbreak Get? Here Are 6 Key
Factors}\label{how-bad-will-the-coronavirus-outbreak-get-here-are-6-key-factors}}

Here's what early research says about how the pathogen behaves and the
factors that will determine how far it will spread.

City officials struck optimistic notes in their announcements. They
suggested they had stopped the virus at its source. The cluster of
illnesses was limited. There was no evidence the virus spread between
humans.

``Projecting optimism and confidence, if you don't have the data, is a
very dangerous strategy,'' said Alexandra Phelan, a faculty research
instructor in the department of microbiology and immunology at
Georgetown University.

``It undermines the legitimacy of the government in messaging,'' she
added. ``And public health is dependent on public trust.''

Nine days after the market closed, a man who shopped there regularly
became the first fatality of the disease, according to a report by the
Wuhan Health Commission, the agency that oversees public health and
sanitation. The 61-year-old, identified by his last name, Zeng, already
had chronic liver disease and a tumor in his abdomen, and had checked
into Wuhan Puren Hospital with a raging fever and difficulty breathing.

The authorities disclosed the man's death two days after it happened.
They did not mention a crucial detail in understanding the course of the
epidemic. Mr. Zeng's wife
\href{https://www.thelancet.com/journals/lancet/article/PIIS0140-6736(20)30183-5/fulltext}{had
developed symptoms} five days after he did.

She had never visited the market.

Image

The intensive care unit at Zhongnan Hospital of Wuhan University in
Wuhan, China, on Jan. 24.Credit...Xiong Qi/Xinhua, via Associated Press

\hypertarget{the-race-to-identify-a-killer}{%
\subsection{The Race to Identify a
Killer}\label{the-race-to-identify-a-killer}}

About 20 miles from the market, scientists at the Wuhan Institute of
Virology were studying samples from the patients checking into the
city's hospitals. One of the scientists, Zheng-Li Shi, was part of the
team that tracked down the origins of the SARS virus, which emerged in
the southern province of Guangdong in 2002.

As the public remained largely in the dark about the virus, she and her
colleagues quickly pieced together that the new outbreak was related to
SARS. The genetic composition suggested a common initial host: bats. The
SARS epidemic began when a coronavirus jumped from bats to Asian palm
civets, a catlike creature that is legally raised and consumed. It was
likely that this new coronavirus had followed a similar path ---
possibly somewhere in or on the way to the Huanan market or another
market like it.

Around the same time, Dr. Li and other medical professionals in Wuhan
started trying to provide warnings to colleagues and others when the
government did not. Lu Xiaohong, the head of gastroenterology at City
Hospital No. 5, told
\href{https://mp.weixin.qq.com/s/IzzCnz4Yr2jEIYZePiu_ow}{China Youth
Daily} that she had heard by Dec. 25 that the disease was spreading
among medical workers --- a full three weeks before the authorities
would acknowledge the fact. She did not go public with her concerns, but
privately warned a school near another market.

By the first week of January, the emergency ward in Hospital No. 5 was
filling; the cases included members of the same family, making it clear
that the disease was spreading through human contact, which the
government had said was not likely.

No one realized, the doctor said, that it was as serious as it would
become until it was too late to stop it.

``I realized that we had underestimated the enemy,'' she said.

At the Institute of Virology, Dr. Shi and her colleagues isolated the
genetic sequence and the viral strain during the first week of January.
They used samples from seven of the first patients, six of them vendors
at the market.

On Jan. 7, the institute's scientists gave the new coronavirus its
identity and began referring to it by the technical shorthand 2019-nCoV.
Four days later, the team shared the virus's genetic makeup in a public
database for scientists everywhere to use.

That allowed scientists around the world to study the virus and swiftly
share their findings. As the scientific community moved quickly to
devise a test for exposure, political leaders remained reluctant to act.

Image

Wuhan on Jan. 27. The city went ahead with a giant potluck dinner in
mid-January.Credit...Hector Retamal/Agence France-Presse --- Getty
Images

\hypertarget{politics-is-always-no-1}{%
\subsection{`Politics Is Always No. 1'}\label{politics-is-always-no-1}}

As the virus spread in early January, the mayor of Wuhan, Zhou Xianwang,
was touting futuristic health care plans for the city.

It was China's political season, when officials gather for annual
meetings of People's Congresses --- the Communist Party-run legislatures
that discuss and praise policies. It is not a time for bad news.

When Mr. Zhou \href{http://www.whtv.com.cn/p/19064.html}{delivered his
annual report} to the city's People's Congress on Jan. 7 against a
backdrop of bright red national flags, he promised the city top-class
medical schools, a \href{https://www.hbwhexpo.com/}{World Health Expo},
and a futuristic industry park for medical companies. Not once did he or
any other city or provincial leader publicly mention the viral outbreak.

\href{https://www.nytimes3xbfgragh.onion/news-event/coronavirus?action=click\&pgtype=Article\&state=default\&region=MAIN_CONTENT_3\&context=storylines_faq}{}

\hypertarget{the-coronavirus-outbreak-}{%
\subsubsection{The Coronavirus Outbreak
›}\label{the-coronavirus-outbreak-}}

\hypertarget{frequently-asked-questions}{%
\paragraph{Frequently Asked
Questions}\label{frequently-asked-questions}}

Updated August 17, 2020

\begin{itemize}
\item ~
  \hypertarget{why-does-standing-six-feet-away-from-others-help}{%
  \paragraph{Why does standing six feet away from others
  help?}\label{why-does-standing-six-feet-away-from-others-help}}

  \begin{itemize}
  \tightlist
  \item
    The coronavirus spreads primarily through droplets from your mouth
    and nose, especially when you cough or sneeze. The C.D.C., one of
    the organizations using that measure,
    \href{https://www.nytimes3xbfgragh.onion/2020/04/14/health/coronavirus-six-feet.html?action=click\&pgtype=Article\&state=default\&region=MAIN_CONTENT_3\&context=storylines_faq}{bases
    its recommendation of six feet} on the idea that most large droplets
    that people expel when they cough or sneeze will fall to the ground
    within six feet. But six feet has never been a magic number that
    guarantees complete protection. Sneezes, for instance, can launch
    droplets a lot farther than six feet,
    \href{https://jamanetwork.com/journals/jama/fullarticle/2763852}{according
    to a recent study}. It's a rule of thumb: You should be safest
    standing six feet apart outside, especially when it's windy. But
    keep a mask on at all times, even when you think you're far enough
    apart.
  \end{itemize}
\item ~
  \hypertarget{i-have-antibodies-am-i-now-immune}{%
  \paragraph{I have antibodies. Am I now
  immune?}\label{i-have-antibodies-am-i-now-immune}}

  \begin{itemize}
  \tightlist
  \item
    As of right
    now,\href{https://www.nytimes3xbfgragh.onion/2020/07/22/health/covid-antibodies-herd-immunity.html?action=click\&pgtype=Article\&state=default\&region=MAIN_CONTENT_3\&context=storylines_faq}{that
    seems likely, for at least several months.} There have been
    frightening accounts of people suffering what seems to be a second
    bout of Covid-19. But experts say these patients may have a
    drawn-out course of infection, with the virus taking a slow toll
    weeks to months after initial exposure. People infected with the
    coronavirus typically
    \href{https://www.nature.com/articles/s41586-020-2456-9}{produce}
    immune molecules called antibodies, which are
    \href{https://www.nytimes3xbfgragh.onion/2020/05/07/health/coronavirus-antibody-prevalence.html?action=click\&pgtype=Article\&state=default\&region=MAIN_CONTENT_3\&context=storylines_faq}{protective
    proteins made in response to an
    infection}\href{https://www.nytimes3xbfgragh.onion/2020/05/07/health/coronavirus-antibody-prevalence.html?action=click\&pgtype=Article\&state=default\&region=MAIN_CONTENT_3\&context=storylines_faq}{.
    These antibodies may} last in the body
    \href{https://www.nature.com/articles/s41591-020-0965-6}{only two to
    three months}, which may seem worrisome, but that's perfectly normal
    after an acute infection subsides, said Dr. Michael Mina, an
    immunologist at Harvard University. It may be possible to get the
    coronavirus again, but it's highly unlikely that it would be
    possible in a short window of time from initial infection or make
    people sicker the second time.
  \end{itemize}
\item ~
  \hypertarget{im-a-small-business-owner-can-i-get-relief}{%
  \paragraph{I'm a small-business owner. Can I get
  relief?}\label{im-a-small-business-owner-can-i-get-relief}}

  \begin{itemize}
  \tightlist
  \item
    The
    \href{https://www.nytimes3xbfgragh.onion/article/small-business-loans-stimulus-grants-freelancers-coronavirus.html?action=click\&pgtype=Article\&state=default\&region=MAIN_CONTENT_3\&context=storylines_faq}{stimulus
    bills enacted in March} offer help for the millions of American
    small businesses. Those eligible for aid are businesses and
    nonprofit organizations with fewer than 500 workers, including sole
    proprietorships, independent contractors and freelancers. Some
    larger companies in some industries are also eligible. The help
    being offered, which is being managed by the Small Business
    Administration, includes the Paycheck Protection Program and the
    Economic Injury Disaster Loan program. But lots of folks have
    \href{https://www.nytimes3xbfgragh.onion/interactive/2020/05/07/business/small-business-loans-coronavirus.html?action=click\&pgtype=Article\&state=default\&region=MAIN_CONTENT_3\&context=storylines_faq}{not
    yet seen payouts.} Even those who have received help are confused:
    The rules are draconian, and some are stuck sitting on
    \href{https://www.nytimes3xbfgragh.onion/2020/05/02/business/economy/loans-coronavirus-small-business.html?action=click\&pgtype=Article\&state=default\&region=MAIN_CONTENT_3\&context=storylines_faq}{money
    they don't know how to use.} Many small-business owners are getting
    less than they expected or
    \href{https://www.nytimes3xbfgragh.onion/2020/06/10/business/Small-business-loans-ppp.html?action=click\&pgtype=Article\&state=default\&region=MAIN_CONTENT_3\&context=storylines_faq}{not
    hearing anything at all.}
  \end{itemize}
\item ~
  \hypertarget{what-are-my-rights-if-i-am-worried-about-going-back-to-work}{%
  \paragraph{What are my rights if I am worried about going back to
  work?}\label{what-are-my-rights-if-i-am-worried-about-going-back-to-work}}

  \begin{itemize}
  \tightlist
  \item
    Employers have to provide
    \href{https://www.osha.gov/SLTC/covid-19/standards.html}{a safe
    workplace} with policies that protect everyone equally.
    \href{https://www.nytimes3xbfgragh.onion/article/coronavirus-money-unemployment.html?action=click\&pgtype=Article\&state=default\&region=MAIN_CONTENT_3\&context=storylines_faq}{And
    if one of your co-workers tests positive for the coronavirus, the
    C.D.C.} has said that
    \href{https://www.cdc.gov/coronavirus/2019-ncov/community/guidance-business-response.html}{employers
    should tell their employees} -\/- without giving you the sick
    employee's name -\/- that they may have been exposed to the virus.
  \end{itemize}
\item ~
  \hypertarget{what-is-school-going-to-look-like-in-september}{%
  \paragraph{What is school going to look like in
  September?}\label{what-is-school-going-to-look-like-in-september}}

  \begin{itemize}
  \tightlist
  \item
    It is unlikely that many schools will return to a normal schedule
    this fall, requiring the grind of
    \href{https://www.nytimes3xbfgragh.onion/2020/06/05/us/coronavirus-education-lost-learning.html?action=click\&pgtype=Article\&state=default\&region=MAIN_CONTENT_3\&context=storylines_faq}{online
    learning},
    \href{https://www.nytimes3xbfgragh.onion/2020/05/29/us/coronavirus-child-care-centers.html?action=click\&pgtype=Article\&state=default\&region=MAIN_CONTENT_3\&context=storylines_faq}{makeshift
    child care} and
    \href{https://www.nytimes3xbfgragh.onion/2020/06/03/business/economy/coronavirus-working-women.html?action=click\&pgtype=Article\&state=default\&region=MAIN_CONTENT_3\&context=storylines_faq}{stunted
    workdays} to continue. California's two largest public school
    districts --- Los Angeles and San Diego --- said on July 13, that
    \href{https://www.nytimes3xbfgragh.onion/2020/07/13/us/lausd-san-diego-school-reopening.html?action=click\&pgtype=Article\&state=default\&region=MAIN_CONTENT_3\&context=storylines_faq}{instruction
    will be remote-only in the fall}, citing concerns that surging
    coronavirus infections in their areas pose too dire a risk for
    students and teachers. Together, the two districts enroll some
    825,000 students. They are the largest in the country so far to
    abandon plans for even a partial physical return to classrooms when
    they reopen in August. For other districts, the solution won't be an
    all-or-nothing approach.
    \href{https://bioethics.jhu.edu/research-and-outreach/projects/eschool-initiative/school-policy-tracker/}{Many
    systems}, including the nation's largest, New York City, are
    devising
    \href{https://www.nytimes3xbfgragh.onion/2020/06/26/us/coronavirus-schools-reopen-fall.html?action=click\&pgtype=Article\&state=default\&region=MAIN_CONTENT_3\&context=storylines_faq}{hybrid
    plans} that involve spending some days in classrooms and other days
    online. There's no national policy on this yet, so check with your
    municipal school system regularly to see what is happening in your
    community.
  \end{itemize}
\end{itemize}

``Stressing politics is always No. 1,'' the governor of Hubei, Wang
Xiaodong, told officials on Jan. 17, citing Mr. Xi's precepts of
top-down obedience. ``Political issues are at any time the most
fundamental major issues.''

Shortly after, Wuhan went ahead with a massive annual potluck banquet
for
\href{https://news.sina.com.cn/s/2020-01-21/doc-iihnzhha3843904.shtml}{40,000
families from a city precinct}, which critics later
\href{https://www.dwnews.com/\%E4\%B8\%AD\%E5\%9B\%BD/60165812/\%E6\%AD\%A6\%E6\%B1\%89\%E8\%82\%BA\%E7\%82\%8E\%E7\%96\%AB\%E6\%83\%85\%E6\%AD\%A6\%E6\%B1\%89\%E7\%A4\%BE\%E5\%8C\%BA\%E8\%BF\%98\%E5\%9C\%A8\%E4\%B8\%BE\%E5\%8A\%9E\%E4\%B8\%87\%E5\%AE\%B6\%E5\%AE\%B4\%E5\%B8\%82\%E9\%95\%BF\%E5\%9B\%9E\%E5\%BA\%94}{cited
as evidence} that local leaders took the virus far too lightly.

As the congress was taking place, the health commission's daily updates
on the outbreak
\href{http://wjw.wuhan.gov.cn/front/web/showDetail/2020011109036}{said
again} and again that there were no new cases of infection, no firm
evidence of human transmission and no infection of medical workers.

``We knew this was not the case!'' said a complaint later filed with the
National Health Commission on a government website. The anonymous author
said he was a doctor in Wuhan and described a surge in unusual chest
illnesses beginning Jan. 12.

Officials told doctors at a top city hospital ``don't use the words
viral pneumonia on the image reports,'' according to the complaint,
which has since been removed. People were complacent, ``thinking that if
the official reports had nothing, then we were exaggerating,'' the
doctor explained.

Even those stricken felt lulled into complacency.

When Dong Guanghe developed a fever on Jan. 8 in Wuhan, his family was
not alarmed, his daughter said. He was treated in the hospital and sent
home. Then, 10 days later, Mr. Dong's wife fell ill with similar
symptoms.

``The news said nothing about the severity of the epidemic,'' said the
daughter, Dong Mingjing. ``I thought that my dad had a common cold.''

The government's efforts to minimize public disclosure persuaded more
than just untrained citizens.

``If there are no new cases in the next few days, the outbreak is
over,'' Guan Yi, a respected professor of infectious diseases at the
University of Hong Kong, said
\href{https://www.nytimes3xbfgragh.onion/2020/01/15/world/asia/coronavirus-japan-china.html}{on
Jan. 15}.

The World Health Organization's statements during this period echoed the
reassuring words of Chinese officials.

It had spread. Thailand reported the first confirmed case outside China
on Jan. 13.

Image

Health officials in Hangzhou, China, taking train passengers'
temperatures after they arrived from Wuhan on Jan. 23.Credit...China
Daily, via Reuters

\hypertarget{a-city-besieged}{%
\subsection{A City Besieged}\label{a-city-besieged}}

The first deaths and the spread of the disease abroad appeared to grab
the attention of the top authorities in Beijing. The national government
dispatched Zhong Nanshan, a renowned and now-semiretired epidemiologist
who was instrumental in the fight against SARS, to Wuhan to assess the
situation.

He arrived on Jan. 18, just as the tone of local officials was shifting
markedly. A health conference in Hubei Province that day called on
medical workers to make the disease a priority. An internal document
from Wuhan Union Hospital warned its employees that the coronavirus
could be spread through saliva.

On Jan. 20, more than a month after the first symptoms spread, the
current of anxiety that had been steadily gaining strength exploded into
public. Dr. Zhong announced in an interview on state television that
there was no doubt that the coronavirus spread with human contact.
Worse, one patient had infected at least 14 medical personnel.

Mr. Xi, fresh from a state visit to Myanmar, made his first public
statement about the outbreak, issuing a brief set of instructions.

It was only with the order from Mr. Xi that the bureaucracy leapt into
action. At that point the death toll was three; in the next 11 days, it
would rise above 200.

In Wuhan, the city banned tour groups from visiting. Residents began
pulling on masks.

Guan Yi, the Hong Kong expert who had earlier voiced optimism that the
outbreak could level off, was now alarmed. He dropped by one of the
city's other food markets and was shocked by the complacency, he said.
**** He told city officials that the epidemic was ``already beyond
control'' and would leave. ``I hurriedly booked a departure,'' Dr. Guan
told Caixin, a Chinese news organization.

Two days later, the city announced that it was shutting itself down, a
move that could only have been approved by Beijing.

In Wuhan, many residents said they did not grasp the gravity of the
epidemic until the lockdown. The mass alarm that officials feared at the
start became a reality, heightened by the previous paucity of
information.

Crowds of people crushed the airport and train stations to get out
before the deadline fell on the morning of Jan. 23. Hospitals were
packed with people desperate to know if they, too, were infected.

``We didn't wear masks at work. That would have frightened off
customers,'' Yu Haiyan, a waitress from rural Hubei, said of the days
before the shutdown. ``When they closed off Wuhan, only then did I
think, `Oh, this is really serious, this is not some average virus.'''

Wuhan's mayor, Zhou Xianwang, later took responsibility for the delay in
reporting the scale of the epidemic, but said he was hampered by the
national law on infectious diseases.
That\href{http://en.pkulaw.cn/display.aspx?cgid=221cc4a42d25144cbdfb\&lib=law}{law}
allows provincial governments to declare an epidemic only after
receiving central government approval. ``After I receive information, I
can only release it when I'm authorized,'' he said.

Image

Dr. Li in Wuhan Central Hospital on Friday.

The official reflex for suppressing discomforting information now
appears to be cracking, as officials at various levels seek to shift
blame for the government's response.

With the crisis worsening, Dr. Li's efforts are no longer viewed as
reckless. \href{https://mp.weixin.qq.com/s/ETgXN6HInzlC8cxzhDdU9g}{A
commentary} on the social media account of the Supreme People's Court
criticized the police for investigating people for circulating rumors.

``It might have been a better way to prevent and control the new
coronavirus today if the public had believed the `rumor' then and
started to wear masks and carry out sanitary measures and avoid the wild
animal market,'' the commentary said.

Dr. Li is 34 and has a child. He and his wife are expecting a second in
the summer. He is now recovering from the virus in the hospital where he
worked. In an interview via text messages, he said he felt aggrieved by
the police actions.

``If the officials had disclosed information about the epidemic
earlier,'' he said, ``I think it would have been a lot better. There
should be more openness and transparency.''

Image

Building a temporary field hospital in Wuhan on Thursday.Credit...Arek
Rataj/Associated Press

This article is based on reporting and research by Elsie Chen, Sheri
Fink, Claire Fu, Javier Hernandez, Zoe Mou, Amy Qin, Knvul Sheikh, Amber
Wang, Yiwei Wang, Sui-Lee Wee, Li Yuan, Albee Zhang and Raymond Zhong.

Advertisement

\protect\hyperlink{after-bottom}{Continue reading the main story}

\hypertarget{site-index}{%
\subsection{Site Index}\label{site-index}}

\hypertarget{site-information-navigation}{%
\subsection{Site Information
Navigation}\label{site-information-navigation}}

\begin{itemize}
\tightlist
\item
  \href{https://help.nytimes3xbfgragh.onion/hc/en-us/articles/115014792127-Copyright-notice}{©~2020~The
  New York Times Company}
\end{itemize}

\begin{itemize}
\tightlist
\item
  \href{https://www.nytco.com/}{NYTCo}
\item
  \href{https://help.nytimes3xbfgragh.onion/hc/en-us/articles/115015385887-Contact-Us}{Contact
  Us}
\item
  \href{https://www.nytco.com/careers/}{Work with us}
\item
  \href{https://nytmediakit.com/}{Advertise}
\item
  \href{http://www.tbrandstudio.com/}{T Brand Studio}
\item
  \href{https://www.nytimes3xbfgragh.onion/privacy/cookie-policy\#how-do-i-manage-trackers}{Your
  Ad Choices}
\item
  \href{https://www.nytimes3xbfgragh.onion/privacy}{Privacy}
\item
  \href{https://help.nytimes3xbfgragh.onion/hc/en-us/articles/115014893428-Terms-of-service}{Terms
  of Service}
\item
  \href{https://help.nytimes3xbfgragh.onion/hc/en-us/articles/115014893968-Terms-of-sale}{Terms
  of Sale}
\item
  \href{https://spiderbites.nytimes3xbfgragh.onion}{Site Map}
\item
  \href{https://help.nytimes3xbfgragh.onion/hc/en-us}{Help}
\item
  \href{https://www.nytimes3xbfgragh.onion/subscription?campaignId=37WXW}{Subscriptions}
\end{itemize}
