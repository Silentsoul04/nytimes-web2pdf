The Money Behind Trump's Money

\url{https://nyti.ms/2UoqX48}

\begin{itemize}
\item
\item
\item
\item
\item
\item
\end{itemize}

\includegraphics{https://static01.graylady3jvrrxbe.onion/images/2020/02/03/magazine/9mag-deutschebank-top/9mag-deutschebank-top-articleLarge-v2.jpg?quality=75\&auto=webp\&disable=upscale}

Sections

\protect\hyperlink{site-content}{Skip to
content}\protect\hyperlink{site-index}{Skip to site index}

Feature

\hypertarget{the-money-behind-trumps-money}{%
\section{The Money Behind Trump's
Money}\label{the-money-behind-trumps-money}}

The inside story of the president and Deutsche Bank, his lender of last
resort.

Credit...Illustration by Paul Sahre

Supported by

\protect\hyperlink{after-sponsor}{Continue reading the main story}

By \href{https://www.nytimes3xbfgragh.onion/by/david-enrich}{David
Enrich}

\begin{itemize}
\item
  Published Feb. 4, 2020Updated June 2, 2020
\item
  \begin{itemize}
  \item
  \item
  \item
  \item
  \item
  \item
  \end{itemize}
\end{itemize}

One Day in early 2017, Mike Offit went to the Yale Club in Manhattan for
a lunch hosted by a group called Business Executives for National
Security. Offit, who has a craggy face and shoulder-­length hair, had
spent much of his career in banking, but that had ended nearly two
decades earlier. Since then, he had puttered around the outskirts of
finance, dabbled in journalism and even published a novel about a pair
of murders at a fictional German-­owned Wall Street bank that bore a
striking resemblance to the one that he worked for until 1998:
­\href{https://www.nytimes3xbfgragh.onion/2020/06/02/business/jeffrey-epstein-deutsche-bank.html}{Deutsche
Bank}.

These days, Offit had time on his hands, which is how he found himself
at the Yale Club that afternoon. Slanting winter sunlight illuminated
the white­columned walls of the club's dining room. Offit was chatting
with an American military officer about weaponry when his iPhone buzzed.
He saw an email from the White House Executive Office of the President.
\emph{How strange}, Offit thought.

The message contained a PDF file: a scanned printout of an email he had
sent Donald Trump several months earlier, in the waning days of the
presidential campaign. Offit had known Trump for decades. At ­Deutsche
Bank, he had lined up huge loans to finance Trump's construction and
renovation of landmark Manhattan skyscrapers, at a time when the
default-­prone real estate developer and casino magnate was no longer
able to get loans from most mainstream financial institutions. The two
men stayed in touch afterward. Offit's 2014 book, ``Nothing Personal,''
even featured a blurb from Trump: ``Michael Offit offers a colorful
insight into how the big money is made --- and/or taken --- on Wall
Street.''

\emph{\href{https://www.harpercollins.com/9780062878816/dark-towers/}{{[}This
article is adapted from David Enrich's forthcoming book, ``Dark Towers:
Deutsche Bank, Donald Trump, and an Epic Trail of Destruction.''{]}}}

In October 2016, Offit tried to return the favor. Democrats were
pillorying Trump's shaky --- not to mention murky --- personal finances,
including his companies' chronic bankruptcies. Offit thought he might
dispense a little advice to his erstwhile client. On a Friday evening,
he emailed Trump a lengthy message, explaining that the defense Trump
was offering at the time --- that he was simply using the bankruptcy law
in an advantageous way --- wasn't resonating with voters. ``I believe
there is a much better answer, that may help defuse this issue, and am
just arrogant enough to suggest it,'' Offit wrote.

He advised Trump to claim that his companies had been forced to declare
bankruptcy, the victims of greedy hedge funds so obsessed with wringing
every last dollar out of him that they refused to let him renegotiate
his crushing debts. Was this true? Not really. But it sounded good, and
the line of attack meshed with Trump's populist rhetoric on the campaign
trail.

Offit got no response. He wasn't even sure that Trump had read the
email. But here, months later, was Trump's unmistakable black Sharpie
scrawl across the top of the message he had sent: ``Mike --- Such a cool
letter. Best wishes, Donald.''

``Look at this!'' Offit exclaimed to the officer. ``I just got a note
from the president!''

``What do you mean, you got a note from the president?''

Offit handed him the iPhone so he could see for himself. The man's eyes
widened. ``Wow,'' he said. ``That's more of a response than \emph{we}
can get out of him.''

\includegraphics{https://static01.graylady3jvrrxbe.onion/images/2020/02/09/magazine/09mag-Deutsche3/09mag-Deutsche3-articleLarge-v2.jpg?quality=75\&auto=webp\&disable=upscale}

The officer asked what Offit's connection was to the president. Offit
replied, ``I loaned him half a billion dollars.''

\textbf{The roughly \$425} million that Offit helped arrange for Trump
back in 1998 was the start of a very long, very complicated relationship
between ­Deutsche Bank and the future president. Over the course of two
decades, the bank lent him more than \$2 billion --- so much that by the
time he was elected, ­Deutsche Bank was by far his biggest creditor.
Against all odds, Trump paid back most of what he owed the bank. But the
relationship cemented ­Deutsche Bank's reputation as a reckless
institution willing to do business with clients nobody else would touch.
And it has made the company a magnet for prosecutors, regulators and
lawmakers hoping to penetrate the president's opaque financial affairs.

Last April, congressional Democrats subpoenaed ­Deutsche Bank for its
records on Trump, his family members and his businesses. The Trump
family
\href{https://www.nytimes3xbfgragh.onion/2019/12/03/business/trump-deutsche-bank-subpoena.html}{sued
to block the bank from complying}; after two federal courts ruled
against the Trumps, the Supreme Court has agreed to hear the case, with
oral arguments expected in the spring. State prosecutors, meanwhile, are
investigating the bank's ties with Trump, too. The F.B.I. has been
conducting its own wide­ranging
\href{https://www.nytimes3xbfgragh.onion/2019/06/19/business/deutsche-bank-money-laundering-trump.html}{investigation
of ­Deutsche Bank}, and people connected to the bank told me they have
been interviewed by special agents about aspects of the Trump
relationship.

If they ever become public, the bank's Trump records could serve as a
Rosetta Stone to decode the president's finances. Executives told me
that the bank has, or at one point had, portions of Trump's personal
federal income tax returns going back to around 2011. (Deutsche Bank
lawyers told a federal court last year that the bank does not have those
returns; it is unclear what happened to them. The Trump Organization did
not respond to multiple requests for comment.) The bank has documents
detailing the finances and operations of his businesses. And it has
records about internal deliberations over whether and how to do business
with Trump --- a paper trail that most likely reflects some bank
employees'
\href{https://www.nytimes3xbfgragh.onion/2019/05/19/business/deutsche-bank-trump-kushner.html}{concerns
about potentially suspicious transactions} that they detected in the
family's accounts.

One reason all these files could be so illuminating is that the bank's
relationship with Trump extended well beyond making simple loans.
­Deutsche Bank managed tens of millions of dollars of Trump's personal
assets. The bank also furnished him with other services that have not
previously been reported: providing sophisticated financial instruments
that shielded him from risks and outside scrutiny, and making
introductions to wealthy Russians who were interested in investing in
Western real estate. If Trump cheated on his taxes, ­Deutsche Bank would
probably know. If his net worth is measured in millions, not billions,
­Deutsche Bank would probably know. If he secretly got money from the
Kremlin, ­Deutsche Bank would probably know.

Until the 1990s, ­Deutsche Bank was a provincial German company with a
limited presence outside Europe. Today it is a \$1.5 trillion colossus,
one of the world's largest banks, with offices in 59 countries --- and,
thanks to its well­documented pattern of violating laws, an
international symbol of greed, recklessness and hubris. Its rap sheet
includes manipulating international currency markets; playing a central
role in rigging a crucial benchmark interest rate known as Libor;
whisking billions of dollars in and out of Iran, Syria, Myanmar and
other countries in violation of sanctions; laundering billions of
dollars on behalf of Russian oligarchs, among many others; and
misleading customers, investors and American, German and British
regulators.

­Deutsche Bank's envelope-­pushing helped it become the global power
player it is today, but it also left the company dangerously frail. Its
books remain stuffed with trillions of dollars of risky derivatives ---
the sort of instruments that many other banks have disposed of since the
2008 financial crisis but that persist as a kind of unexploded ordnance
in ­Deutsche Bank's accounts, threatening to inflict severe damage on
the bank and the broader financial system if something were to cause
them to detonate. Its financial cushions to absorb future shocks are
threadbare. Its core businesses are not performing well; the bank lost
\$5.8 billion last year. Because of Deutsche Bank's size and its
connections with hundreds of other major banks around the world, serious
problems could spread, virus­like, to other financial institutions. The
International Monetary Fund a few years ago branded ­Deutsche Bank ``the
most important net contributor to systemic risks'' in the global banking
system.

­Deutsche Bank's relationship with Trump, rather than being an odd
outlier, is a kind of object lesson in how the bank lost its way. The
company was hungry for growth, especially in the United States, and it
was happy to cozy up to clients that better-­established players viewed
as too damaged or dangerous. Along the way, it missed one opportunity
after another to extricate itself from the Trump relationship or at
least slow its expansion. With hindsight, the procession of miscues and
bad decisions appears almost comical.

I have spent the past two years interviewing dozens of ­Deutsche Bank
executives about the Trump relationship, among other subjects. Quite a
few look back at the relationship with a mixture of anger and regret.
They blame a small group of bad bankers for blundering into a trap that
would further damage ­Deutsche Bank's name and guarantee years of
political and prosecutorial scrutiny. But that isn't quite right; in
fact, the Trump relationship was repeatedly blessed by executives up and
down the bank's organizational ladder. The cumulative effect of those
decisions is that a German company --- one that most Americans have
probably never heard of --- played a large role in positioning a
strapped businessman to become president of the United States.

Image

40 Wall Street, New York: \$125 million loan (1998).Credit...Raymond
Boyd/Getty Images

\textbf{Founded in 1870, ­}Deutsche Bank spent most of its first 12
decades helping mighty German companies like Siemens expand
internationally, as well as bankrolling infrastructure projects ---
primarily railroads --- in Europe, Asia and North America. (Before and
during World War II, the bank was a leading financier of the Nazis,
helping to pay for projects including the construction of Auschwitz.) As
the Iron Curtain fell, ­Deutsche Bank executives saw an opportunity to
surf the tide of globalization and become a truly global institution.

The modern multinational corporation --- operating in dozens of
countries; buying and selling products and raw materials in a slew of
different currencies; seeking to protect itself from volatile prices,
fluctuating foreign-­exchange rates and sundry unforeseeable risks ---
needed banks that would provide much more than run-of-the-mill loans.
Now financial institutions were expected to underwrite stock and bond
offerings, enable seamless transactions across international borders and
in multiple currencies and create financial instruments known as
derivatives that customers could use to insulate themselves from big
swings in interest rates, dairy prices, the weather and the like. Top
Wall Street banks were beginning to offer these services to German
companies and even the German government. To defend its turf, ­Deutsche
Bank needed to learn to compete with the Americans.

In 1989, the bank took its first small step into this new financial
world by acquiring a venerable British investment bank, Morgan Grenfell.
Then, in 1995, ­Deutsche Bank hired a small crew of traders from Merrill
Lynch --- at the time one of Wall Street's leading firms --- led by a
charismatic and impulsive salesman named Edson Mitchell. The American
and British governments were rolling back regulations on the banking
industry, and the giants of Wall Street were getting bigger, fast.
­Deutsche Bank needed to act quickly to have any chance of catching up.
Over the next couple of years, Mitchell's team spent billions of dollars
hiring thousands of bankers and traders from just about every major
investment bank --- perhaps the greatest mass migration in Wall Street's
history. One recruit was Mike Offit; another was Justin Kennedy, a son
of the Supreme Court justice Anthony Kennedy.

Offit and Kennedy were traders at Goldman Sachs together, creating and
selling mortgage bonds and working for Steven Mnuchin, now the Treasury
secretary. By the time ­Deutsche Bank came knocking in the late 1990s,
both men were ready for a change. (Plus, Offit claims, Mnuchin stiffed
him when it came to compensation.) The German bank lured them, as it had
others, by promising them the freedom, budget and entrepreneurial
culture to help establish the bank as a formidable player on Wall Street
--- a euphemistic way of saying that the bank had a higher tolerance for
risk than many rivals. Come to ­Deutsche Bank, and you could do bigger
trades. Offer larger loans. Make bolder acquisitions. Earn more money.

Offit and Kennedy were hired to build, essentially from scratch, a
machine to package giant loans --- for skyscrapers, hotels and other
commercial real estate projects --- into salable securities. At the
time, selling and trading these mortgage-­backed bonds was one of the
hottest businesses on Wall Street. Success would not only be
remunerative; it would help burnish ­Deutsche Bank's credentials as a
serious challenger to the Wall Street elite.

Among the first steps was finding recipients for the loans --- and that
is how Donald Trump became a customer of ­Deutsche Bank. Two decades
after striking out on his own in real estate, Trump was then at a low
point of his career. He had recently defaulted on his debts to a number
of large Wall Street banks --- the consequence of many hundreds of
millions of dollars of unwise, overleveraged investments in the casino
and hotel industries --- and was eager to find a financier willing to
look past these sins. In 1998, Offit arranged to lend hundreds of
millions of dollars to finance Trump's gut renovations of an Art Deco
tower at 40 Wall Street and his construction of a skyscraper next to the
United Nations.

Shortly thereafter, Offit was let go --- he says he fell victim to
internal politics --- and Kennedy soon climbed toward the top of the
commercial mortgage bonds business. His specific role was finding
customers to buy parts of that debt --- a key component of Deutsche
Bank's ability to make the loans in the first place. He continued
cultivating the relationship with Trump, helping to line up hundreds of
millions of dollars in borrowings.

Kennedy became friendly with Trump, sometimes joining him in his luxury
box at the U.S. Open tennis tournament or saying hello to him at
Manhattan nightclubs, where Trump would park himself at a table in the
corner, holding court. For ­Deutsche Bank, the relationship offered both
money and prestige. Trump owed interest on the loans and also had to pay
millions of dollars in fees each time the bank arranged a new one.
Trump's recent history notwithstanding, the bank believed that the risks
were minimal --- the loans were earmarked for projects that appeared
financially sound --- and its executives were eager to build their brand
recognition in the United States. One way to do that was to publicly
align itself with a flashy businessman who, for all his recent
screw-ups, was nothing if not recognizable. Trump soon became a regular
at ­Deutsche Bank's annual pro-am golf tournament.

\textbf{Trump, however, had} a nasty tendency to stiff his business
partners and associates. There was an extensive list of Trump
contractors, vendors and lawyers who had to settle for a fraction of
what they were owed after Trump threatened to pay nothing at all. Banks,
from Wall Street powerhouses like JPMorgan to foreign lenders like
Britain's NatWest, had fared little better. Time after time, Trump's
casino companies defaulted on their debts, filed for bankruptcy and
ultimately agreed to pay back their creditors pennies on the dollar.

Image

Trump World Tower, New York: \$300 million loan guarantee
(1998).Credit...Carlo Allegri/Reuters

Merrill Lynch learned this lesson the hard way when it helped Trump sell
\$675 million of bonds to pay for work on his Taj Mahal casino on the
Atlantic City boardwalk in 1988. The gaudy casino --- the world's
largest --- opened its doors two years later, bedecked with
bright-­colored onion domes and laden with debt. Within months, it
defaulted on the bonds. The unpleasant task of mopping up this mess fell
to a Merrill executive named Seth Waugh. Trump threatened to tie his
lenders up in years of bankruptcy-­court litigation if they didn't agree
to let him largely off the hook for what he owed. Nobody doubted Trump's
litigiousness, and Waugh and his Merrill colleagues ultimately accepted
what amounted to deep losses on the bonds.

\emph{\href{https://www.nytimes3xbfgragh.onion/2004/09/12/magazine/trumpologies.html}{{[}Read
more about Trump's rise from the ashes of his Atlantic City mess.{]}}}

In 2000, Waugh joined ­Deutsche Bank. Perma-­tanned and with long,
floppy hair, Waugh developed a reputation among some ­Deutsche Bank
colleagues for being a bit of a lightweight. They derided him for
spending more time on the golf course than he did in the office. (Today
Waugh is the chief executive of the Professional Golfers' Association of
America.) But he enjoyed the confidence of one of ­Deutsche Bank's
highest-­ranking executives, Josef Ackermann, who helped recruit him
from Merrill Lynch. In 2001, Waugh learned that ­Deutsche Bank was
planning to lend Trump about \$500 million to use as he wished ---
basically an unrestricted cash infusion to stabilize his flagging
finances. Having witnessed up close the carnage that Trump could inflict
on imprudent financial institutions, Waugh was in no hurry to repeat the
experience.

The loan that was being offered now wouldn't have required Trump to put
up any hard assets as collateral; he was requesting to borrow \$500
million against the \$1 billion of ``good will'' that Trump claimed was
associated with his name. That made the transaction even riskier: If
Trump stopped repaying, ­Deutsche Bank would have no easy way to get its
money back. Waugh voiced strong objections to the loan, and the deal
died.

Waugh would soon be named the head of Deutsche Bank's American
businesses, and he had the power to put a stop to the bank's broader
Trump relationship. He didn't. And in 2003, yet another division of
­Deutsche Bank, one that focused on helping companies raise money by
selling stocks and bonds to investors, agreed to work with Trump. The
point man on this venture was Richard Byrne --- another Merrill veteran
who had also been involved in the Taj Mahal debacle. (Byrne helped sell
the ill-­fated Taj bonds to investors.) Now Trump hired Byrne's group at
­Deutsche Bank to issue bonds for his Trump Hotels \& Casino Resorts
company.

It would have been reasonable to expect that Waugh would have warned
Byrne about the recently rejected \$500 million loan, but that never
happened. So Byrne pressed on, organizing a ``road show'' for Trump to
meet with and try to win over big institutional investors. When that
didn't drum up the desired money, Byrne's salesmen worked the phones,
casting a wider net for more clients and managing to sell more than
\$400 million of junk bonds (albeit at a high interest rate that
reflected investors' fears that Trump might default). Trump repaid the
­Deutsche Bank team
\href{https://www.nytimes3xbfgragh.onion/2019/03/18/business/trump-deutsche-bank.html}{with
a weekend trip to his Mar-­a-­Lago resort in Palm Beach, Fla.}

The following year, with his casinos on the rocks, Trump's company
stopped paying interest on the bonds and filed for bankruptcy
protection. (``I don't think it's a failure,'' Trump said. ``It's a
success.'') ­Deutsche Bank's clients, the ones who had recently bought
the junk bonds, suffered painful losses. In the future, Trump would
effectively be off-­limits for Byrne's division.

But that \emph{still} didn't render Trump persona non grata for the
entire bank. This was a product, at least in part, of ­Deutsche Bank's
internal dysfunction. Many big companies are compartmentalized, but
­Deutsche Bank took it to an extreme. It wasn't just that rival
divisions of the bank didn't communicate well; they often battled
against one another to win clients and amass power for their leaders
within the organization --- a poisonous atmosphere that Trump, perhaps
inadvertently, managed to exploit.

Trump soon went back to the commercial real estate group, asking for a
loan to build a 92-­story skyscraper in Chicago. He reportedly seduced
the ­Deutsche bankers with flights on the same 727 that had recently
taken Byrne's salesmen to Florida. He invited Kennedy and his colleagues
to Trump Tower and lavished them with praise. He explained that his
daughter Ivanka would be in charge of the Chicago development --- that's
how important this project was to the Trump Organization.

Just as Waugh hadn't warned Byrne about the rejected Trump loan, now
Byrne didn't warn Kennedy and the other commercial real estate bankers
about his own recent Trump experience. ``We just looked the other way,''
a former executive explains. ``That was the ­Deutsche Bank culture.''

Image

Trump Hotels \& Casino Resorts, Atlantic City. Bond offering of more
than \$400 million (2003).Credit...Michael Ein/The Press of Atlantic
City, via Associated Press

\textbf{In 2002, Joe Ackermann} was named chief executive of ­Deutsche
Bank. A reserve officer in the Swiss military and a number cruncher with
a nearly photographic memory, Ackermann also had a notoriously short
fuse. His staff was terrified of disappointing him.

Shortly after taking charge, he made a bold and fateful public promise:
The bank would become roughly six times as profitable within a couple of
years. Senior executives from that era say that in order to achieve
Ackermann's objective, they quickly shifted the priorities inside the
bank. The overriding mission became to make as much money as possible,
as quickly as possible, with scant attention to the long-term
consequences. A saying took hold among the bank's leaders: ``The current
quarter is the most important quarter we're ever going to have.''
Shareholders now became the bank's most important constituency.

On the surface, Ackermann's strategy seemed successful. The bank met his
lofty profitability targets on schedule, in part by increasingly using
its own money --- as opposed to customers' --- to shoot for the moon
with speculative market bets, a strategy known as proprietary trading.
The bank's shares climbed higher. The company was showered with industry
awards.

Years later, it would become clear how ­Deutsche Bank pulled this off:
by taking dangerous shortcuts and, at times, breaking the law. In its
zeal to goose its profitability metrics, the bank neglected to invest
enough in internal technology systems or top-tier compliance and
risk-­management staffs. When employees cooked up schemes to launder
money, violate sanctions, manipulate markets, avoid taxes or mislead
customers, there were few internal checks and balances to detect or stop
such behavior. In the rare instances when employees voiced concerns,
they were often ignored and were sometimes punished for getting in the
way of the boss's all-­important profitability directive.

Trump was among the beneficiaries of these shortcuts. Before making the
Chicago loan, ­Deutsche Bank conducted an informal audit of his
­finances. Trump had declared to the bank that he was worth more than
\$3 billion, but
­\href{https://assets.documentcloud.org/documents/2430267/trumps-lawsuit-on-net-worth.pdf}{Deutsche
Bank concluded} that the real number was closer to \$788 million --- a
quarter of what Trump said it was. For most banks, this would have been
disqualifying, but ­Deutsche Bank was undeterred. Executives were so
eager for growth and big deals that they managed to look past the
obvious red flags. In February 2005, ­Deutsche Bank agreed to lend him
\$640 million for the Chicago project.

Around this time, and out of public view, ­Deutsche Bank provided a
series of other services to Trump that haven't been previously reported.
It created numerous ``special purpose vehicles'' to make it easier for
him quietly and inexpensively to acquire international properties for
himself and his companies. Trump at the time was trying to diversify his
portfolio of assets outside the United States, striking deals to buy
properties outright and simply slapping his name on other people's
buildings. Thanks to the magic of financial engineering, the vehicles
­Deutsche Bank created enabled Trump to do real estate deals in places
like Eastern Europe and South America without putting any of his own
money on the line. Not only was he taking out loans to finance the
acquisitions, but he was also using other people's money to cover the
small ``equity'' portion of the purchases. ­Deutsche Bank would sell
investors the rights to whatever revenue the projects generated in the
future in exchange for the investors' --- or the bank's --- covering the
money Trump owed upfront for the acquisitions. As a result, ­Deutsche
Bank and investors bore the risk, over many years, that the Trump
projects would not bring in as much money as expected.

This sort of structure was hardly unheard-­of for real estate
developers. ``It's a well-­seasoned financing technique,'' Mark Ritter,
a ­Deutsche Bank executive who was familiar with the transactions at the
time, told me. It was an easy way for the bank to generate a steady
stream of fees from deep-­pocketed customers. But it increased the
bank's already heavy exposure to Trump --- and helped the mogul strike
under-­the-­radar deals in far-flung locales, including some that were
popular destinations for people looking to hide assets.

­Deutsche Bank also helped Trump find buyers for condos in his
properties, according to people familiar with the arrangements. When he
partnered in 2006 with a Los Angeles developer to build a Trump-­branded
resort in Hawaii, ­Deutsche Bank organized get-­togethers in London and
elsewhere to connect Trump and his partners with wealthy clients ---
including some from Russia --- who used anonymous shell companies to buy
units in the Waikiki hotel complex.

Some senior executives discussed the potential pitfalls of the Trump
relationship. It wasn't only the not-­insignificant risk that Trump
would default on loans. The bankers also talked about Trump's
well-­documented ties to the organized-­crime world --- he had done
business with people in or affiliated with the mob in Atlantic City and
New York --- and the possibility that his real estate projects were
laundromats for illicit funds from countries like Russia, where
oligarchs were trying to get money out of the country. ``Everyone in the
real estate business was involved in `flight capital,' '' a top
executive told me. But the loans looked profitable, the relationship
felt valuable and the concerns went unheeded.

When the 2008 financial crisis arrived, Trump still owed \$334 million
on ­Deutsche Bank's loan for his Chicago skyscraper. With the economy
sinking, nobody was buying the luxury apartments in the building. As the
loan's due date approached that November, Trump filed a lawsuit, trying
to void the loan because of the financial crisis, which he accused
­Deutsche Bank of having helped ignite. He sought damages of \$3
billion. ­Deutsche Bank filed its own suit, seeking to recoup the \$40
million portion of the loan that Trump had personally guaranteed back in
2005. Shortly after the suit was filed, Trump bumped into Justin
Kennedy. ``Nothing personal,'' Trump said. Kennedy replied that there
were no hard feelings: Business was business. (Kennedy left the bank at
the end of 2009.) But the lawsuit did what the defaults somehow had not:
It prompted Deutsche Bank to wash its hands of Donald Trump. In the
future, he wouldn't even be permitted to enter the bank's golf
tournament.

\textbf{Relative to many} of its ruined rivals, ­Deutsche Bank enjoyed a
``good'' financial crisis. It lost billions, but it had also placed
market bets that anticipated the housing crisis, and the resulting
profits largely made up for other losses. It was one of the few major
international banks that seemed strong enough to avoid a direct
government bailout, and many investors were pacified by the knowledge
that if things got really dicey, the German government would come to the
rescue.

For a brief spell, ­Deutsche Bank was the toast of the financial world.
Within a few years, though, it had become one of the industry's leading
problem cases. The crisis --- and the waves of new regulations that came
afterward --- fundamentally changed Wall Street. ­Deutsche Bank's
business model had hinged on making enormous wagers with borrowed money;
in essence, the one-time icon of German sobriety had become a giant
casino. Now this proprietary trading was outlawed in the United States.
Regulators around the world --- deeply suspicious of the sort of
adrenaline-­crazed, rule-­bending tactics that had come to define
­Deutsche Bank --- also pushed banks to lessen their reliance on
borrowed funds.

Cocky from their performance during the crisis, ­Deutsche Bank
executives initially refused to do so. Nor did they use this moment of
relative strength to rid the bank of the mountains of unwanted assets
--- in particular, trillions of dollars' worth of derivatives that had
the potential to saddle the bank with enormous losses --- that were
polluting its balance sheet.

One of the few concessions the bank did make to the new era was to start
looking for safer ways to earn money. One was private banking: providing
personalized services to the richest of the American rich. To
differentiate its then-­sleepy private-­banking division from the
competition, ­Deutsche Bank planned to work with customers who were
untouchable for rival banks and to do deals that were too risky or too
complicated for others to stomach --- a variation on the strategy that
Offit and Kennedy's squad had deployed a decade earlier when trying to
get the commercial real estate business off the ground. And no one was
more central to that project than a woman named Rosemary Vrablic.

Vrablic grew up in the Bronx and then the New York City suburb of
Scarsdale and started out as a bank teller before eventually landing a
job analyzing proposed loans. In 1989, a headhunter recruited her for a
job in Citicorp's private-­banking arm. Citi was widening its suite of
offerings to such clients, including by making loans to finance their
big real estate projects. She quickly took advantage of this new lending
service to become one of New York's leading bankers to the superrich,
first at Citi and then at Bank of America. Vrablic specialized in
dealing with difficult men. ``They're successful, and they've earned
their money by being tough,'' she explained in an interview years later.
``So I don't have a problem with tough.'' One of her youngest clients
was Jared Kushner, who was taking over his family's real estate company.

In 2006, Vrablic met Tom Bowers, who had joined ­Deutsche Bank the year
before with a mandate to help the German bank make a name for itself
among ultrawealthy Americans. Vrablic, who was 46 at the time, soon
agreed to join ­Deutsche Bank, with a guaranteed salary and bonus of at
least \$3 million a year. She started bringing in tens of millions of
dollars in annual revenue for the bank. Bowers told me she was by far
the top producer in the bank's New York offices.

The litigation between Trump and ­Deutsche Bank over the loan for his
Chicago skyscraper was settled in 2010, with the bank agreeing to give
Trump two years to make good on his obligations, including the \$40
million that he had personally guaranteed. And if he wanted to keep
expanding his empire, he would need to identify a new source of credit.
The trouble, as ever, was that serious banks wouldn't get anywhere near
him. Even ­Deutsche Bank, it seemed, was now off-­limits after the
Chicago fight.

In 2011, Jared Kushner invited Vrablic to visit his father-­in-­law at
Trump Tower. Trump explained his situation to her and then popped the
big question: Would Vrablic's division consider lending him \$40 or \$50
million, with Chicago's Trump International Hotel \& Tower as
collateral? That would allow him to repay what he still personally owed
­to Deutsche Bank. The whole idea --- that one arm of ­the bank would
lend money for the purpose of paying off defaulted debts owed to another
part of the company --- seemed crazy, but Vrablic was excited by the
prospect of landing a major deal with a major new client. She took the
proposal to Bowers, who agreed that it was worth considering.

Image

Trump International Hotel \& Tower, Chicago: \$640 million loan
(2005).Credit...John Walton/PA Images/Getty Images

A small team sifted through Trump's personal and corporate financial
records and tax returns. The first thing the bankers noticed was that
Trump was, once again, vastly overstating his fortune, assigning
absurdly high valuations to his real estate assets. In
\href{https://www.nytimes3xbfgragh.onion/2019/02/27/business/donald-trump-buffalo-bills-deutsche-bank.html}{one
especially egregious case}, he claimed that a property he bought in
Westchester County for about \$7 million was now worth \$291 million.
The bank ended up reducing the assets' values by as much as 70 percent.

The funny thing was that Trump's underlying finances weren't all that
bad. He had limited debt, at least compared with his fellow real estate
magnates, and cash was pouring in, apparently from TV royalties and
licensing deals he had struck to put his name on properties he didn't
own. Making the transaction more palatable for ­Deutsche Bank, Trump was
willing to personally guarantee the loan, meaning that ­Deutsche Bank in
theory could seize his assets if he didn't pay it back. (The fact that a
similar, albeit smaller, personal guarantee hadn't prevented him from
defaulting on the original Chicago loan did not seem to bother the
bank.) Vrablic and Bowers tentatively agreed to lend him \$48 million.

When top executives in ­Deutsche Bank's investment-­banking unit heard
that another division was about to rekindle the Trump relationship, they
were furious. The head of that unit, Anshu Jain, rang the alarm at a
meeting of the bank's top executives: How could ­Deutsche Bank do
business with Trump after he had so publicly burned the bank? What
precedent would that set for other would-be deadbeats? If Trump
defaulted yet again, how would ­Deutsche Bank explain that to investors
and regulators?

Bowers and Vrablic argued that the loan was sound. They grumbled that
the investment bankers were just jealous that the private bankers had
figured out how to structure a loan with Trump in a way that seemed
virtually risk-free for ­Deutsche Bank. The squabble made it to the top
of the bank --- at which point Ackermann, the chief executive, said he
didn't object to the loan. (Ackermann told me that he doesn't recall
being consulted.)

The bank's lawyers reviewed the question of whether the company could
work with Trump and declared that ``this client is cleared.'' Stuart
Clarke, ­Deutsche Bank's chief operating officer in the Americas,
emailed Bowers a similar message: ``There is no objection from the bank
to proceed with the client.'' Attached to that email was a PDF file
containing Trump's personal and corporate financials, to make clear that
everyone who had vetted the proposed transaction was fully aware of the
new client's heavy baggage.

\textbf{The spigot was} open again. When Trump decided to buy the
rundown Doral Resort \& Spa in Miami for \$150 million in 2012, one of
his first phone calls was to Rich Byrne, who years earlier helped
Trump's casino company sell junk bonds. Trump's subsequent default on
those bonds ended his relationship with ­Deutsche Bank's securities
unit, which Byrne now ran, but the two men had stayed in touch. Byrne
agreed to take a look at the numbers, not bothering to tell Trump that
­Deutsche Bank was unlikely to actually help him with financing.

But the private-­banking division was in the mix. Trump invited Vrablic
to Florida to see the property. The day after she got back to New York,
she walked into Bowers's office. ``Trump wants to go buy Doral,'' she
explained, and he wanted ­Deutsche Bank to lend him the money to do it.

For the second time in a matter of weeks, Bowers dispatched a team to
study a possible Trump loan. They concluded that, in fact, Trump was
getting the resort at a reasonable price. The bank wired \$125 million
to the Trump Organization. Afterward, Trump called Byrne's office.
``Rich,'' he bellowed, ``I'm just calling to thank you! I know you
must've approved it, but Rosemary and her team gave me the money.''

Thinking on his feet and happy to take credit, Byrne pretended that he
knew all about the loan. He congratulated Trump and then, as an aside,
asked about the interest rate that Vrablic's squad was charging on the
loan. Trump said it was well under 3 percent. Byrne couldn't believe
that ­Deutsche Bank, after everything it had been through with Trump,
was now extending him a nine­figure loan at such a low interest rate.

Image

Trump National Doral Miami, Florida: \$125 million loan
(2012).Credit...Johnny Louis/Film Magic/Getty Images

When Trump bid roughly \$1 billion for the Buffalo Bills football team
in 2014, Vrablic agreed to back him. (Trump's bid was rejected, so no
loan was required.) A few months later, ­Deutsche Bank lent Trump \$170
million to finance the transformation of the Old Post Office building in
Washington into a Trump International luxury hotel. Vrablic's division
also lent millions of dollars to other members of the Trump and Kushner
families.

Then, in the spring of 2016, Trump came looking for one more loan. It
was ostensibly to pay for work on his golf course in Turnberry,
Scotland. But Trump was running for president, and it was hard to avoid
the suspicion that the loan he was requesting might have something to do
with the fact that he was burning through gobs of his own cash on the
campaign trail.

Bowers by then had left the bank. (He died in November 2019 in what the
Los Angeles County coroner said was a suicide.) Vrablic and her managers
said yes to the Scotland loan, but the deal had to be reviewed up the
food chain. This would once have been part of Seth Waugh's job, but he
was no longer in charge of ­Deutsche Bank's American businesses. His
successor was a veteran investment banker named Jacques Brand.

Somehow Brand hadn't realized until now that Trump was one of his
company's most important clients. He was alarmed. He was stunned that
the people in his wealth-­management division thought it would be a good
idea to lend tens of millions more to Trump --- especially right now, in
the middle of this violent brawl of a presidential campaign. ``Why are
we doing business with him?'' Brand fumed to his colleagues. ``The
answer is no.''

Brand didn't have the power to block the loan unilaterally, but he
referred it to a committee charged with safeguarding the bank's
reputation. After hearing a quick summary of the proposed loan, the
committee unanimously voted to reject it. ``It was an affront to all of
our senses,'' recalls an executive involved in the deliberations.

Vrablic and her colleagues appealed to another committee, in Frankfurt,
which also rejected the proposal.
\href{https://www.nytimes3xbfgragh.onion/2019/02/02/business/trump-deutsche-bank.html}{The
loan was dead.} Nine months later, Trump was elected president. Vrablic
received V.I.P. tickets to the inauguration, as well as hard-to-get
accommodations in Trump's newly opened Washington hotel that Deutsche
Bank had financed.

\textbf{In some respects,} the association with a polarizing president
has been bad for ­Deutsche Bank's business. Executives told me it has
become a bit harder to win assignments from public institutions, like
government-­employee pension plans. And ­Deutsche Bank has become an
enticing target for ambitious prosecutors and politicians around the
world --- including Democrats who, after taking control of the House of
Representatives last year, swiftly issued subpoenas for the bank's
records on Trump, his family and his companies.

But remarkably, in the final accounting, the Trump relationship may have
been an overall positive for both the loyal bank and its prized client.
Trump got the money he needed to keep buying and building, which in turn
allowed him to maintain the reputation as a respectable businessman that
he eventually rode to the White House. ­For its part, Deutsche Bank has
pocketed tens of millions of dollars in fees and interest payments. The
bank won't comment on whether Trump is up-to-date with his loan
payments, and Vrablic, who remains a bank employee, has declined to
publicly discuss her relationship with Trump. But Trump is not known to
have defaulted on any of his recent loans or otherwise burdened the bank
with large losses; in fact, the overall relationship appears to be
profitable.

Most likely, this reflects the bank's luck, not skill, but some senior
executives told me they view the outcome as a vindication of their work
for Trump. Their attitude is reminiscent of the complacency that took
hold after ­Deutsche Bank skated through the financial crisis, thanks to
a combination of good fortune, well-­timed bets and the knowledge of a
German government safety net. Some optimistic ­Deutsche Bank officials
even believe that because the bank has refused to publicly divulge much
about its famous client, it might emerge with an enhanced reputation for
probity and discretion --- and for getting big things done for desperate
customers.

In any case, ­Deutsche Bank, along with the rest of its industry, is
benefiting from having Trump in the White House. His administration has
been systematically
\href{https://www.nytimes3xbfgragh.onion/2019/10/10/business/economy/federal-reserve-bank-regulations.html}{rolling
back regulations} erected after the financial crisis that were designed
to hem in Wall Street and the world's largest banks. The prohibitions on
proprietary trading are loosening. The requirements for how much capital
banks need to have are easing. The Consumer Financial Protection Bureau,
created to shield borrowers from avaricious lenders, has been neutered.
The presumption that the government needs to actively police what banks
are doing has been replaced by the empirically dubious assumption that
the private sector can mostly look out for itself. This hasn't solved
­Deutsche Bank's fundamental problems --- its business model is broken,
its balance sheet is littered with unwanted assets, its years of
misconduct are under government investigation --- but it has afforded
the bank and the entire industry some much-­welcomed breathing room.

One week after the new president sent his thank-you email to Mike Offit
in February 2017, Trump delivered his first address to a joint session
of Congress. It was a subdued, disciplined performance in which he
hewed, more or less, to the teleprompters and sounded, more or less,
like a mainstream politician. After stepping down from the lectern,
Trump shook hands with the dignitaries in the audience. In the front row
were the Supreme Court justices.

Trump moved down the line until he got to Anthony Kennedy. As Trump
pumped the justice's hand, Kennedy congratulated him on a successful
speech. ``Very nice, thank you,'' Trump replied. ``Say hello to your
boy,'' he added, patting the justice's arm. ``Special guy.''

Advertisement

\protect\hyperlink{after-bottom}{Continue reading the main story}

\hypertarget{site-index}{%
\subsection{Site Index}\label{site-index}}

\hypertarget{site-information-navigation}{%
\subsection{Site Information
Navigation}\label{site-information-navigation}}

\begin{itemize}
\tightlist
\item
  \href{https://help.nytimes3xbfgragh.onion/hc/en-us/articles/115014792127-Copyright-notice}{©~2020~The
  New York Times Company}
\end{itemize}

\begin{itemize}
\tightlist
\item
  \href{https://www.nytco.com/}{NYTCo}
\item
  \href{https://help.nytimes3xbfgragh.onion/hc/en-us/articles/115015385887-Contact-Us}{Contact
  Us}
\item
  \href{https://www.nytco.com/careers/}{Work with us}
\item
  \href{https://nytmediakit.com/}{Advertise}
\item
  \href{http://www.tbrandstudio.com/}{T Brand Studio}
\item
  \href{https://www.nytimes3xbfgragh.onion/privacy/cookie-policy\#how-do-i-manage-trackers}{Your
  Ad Choices}
\item
  \href{https://www.nytimes3xbfgragh.onion/privacy}{Privacy}
\item
  \href{https://help.nytimes3xbfgragh.onion/hc/en-us/articles/115014893428-Terms-of-service}{Terms
  of Service}
\item
  \href{https://help.nytimes3xbfgragh.onion/hc/en-us/articles/115014893968-Terms-of-sale}{Terms
  of Sale}
\item
  \href{https://spiderbites.nytimes3xbfgragh.onion}{Site Map}
\item
  \href{https://help.nytimes3xbfgragh.onion/hc/en-us}{Help}
\item
  \href{https://www.nytimes3xbfgragh.onion/subscription?campaignId=37WXW}{Subscriptions}
\end{itemize}
