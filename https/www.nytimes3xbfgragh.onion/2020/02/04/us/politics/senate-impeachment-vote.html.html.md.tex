Sections

SEARCH

\protect\hyperlink{site-content}{Skip to
content}\protect\hyperlink{site-index}{Skip to site index}

\href{https://www.nytimes3xbfgragh.onion/section/politics}{Politics}

\href{https://myaccount.nytimes3xbfgragh.onion/auth/login?response_type=cookie\&client_id=vi}{}

\href{https://www.nytimes3xbfgragh.onion/section/todayspaper}{Today's
Paper}

\href{/section/politics}{Politics}\textbar{}Susan Collins Announces She
Will Vote to Acquit Trump, Calling Conduct `Wrong'

\url{https://nyti.ms/2v6yL08}

\begin{itemize}
\item
\item
\item
\item
\item
\item
\end{itemize}

Advertisement

\protect\hyperlink{after-top}{Continue reading the main story}

Supported by

\protect\hyperlink{after-sponsor}{Continue reading the main story}

\hypertarget{susan-collins-announces-she-will-vote-to-acquit-trump-calling-conduct-wrong}{%
\section{Susan Collins Announces She Will Vote to Acquit Trump, Calling
Conduct
`Wrong'}\label{susan-collins-announces-she-will-vote-to-acquit-trump-calling-conduct-wrong}}

The Maine senator became the latest Republican to condemn President
Trump's conduct but argue it did not warrant his removal.

\includegraphics{https://static01.graylady3jvrrxbe.onion/images/2020/02/04/us/politics/04dc-impeach-sub1/merlin_168386130_b700857a-ca77-4e86-b952-f0b17ea9798e-articleLarge.jpg?quality=75\&auto=webp\&disable=upscale}

\href{https://www.nytimes3xbfgragh.onion/by/patricia-mazzei}{\includegraphics{https://static01.graylady3jvrrxbe.onion/images/2018/11/28/multimedia/author-patricia-mazzei/author-patricia-mazzei-thumbLarge.png}}\href{https://www.nytimes3xbfgragh.onion/by/catie-edmondson}{\includegraphics{https://static01.graylady3jvrrxbe.onion/images/2019/11/20/us/politics/catie-edmonson-twitter-chatblog/catie-edmonson-twitter-chatblog-thumbLarge.png}}

By \href{https://www.nytimes3xbfgragh.onion/by/patricia-mazzei}{Patricia
Mazzei} and
\href{https://www.nytimes3xbfgragh.onion/by/catie-edmondson}{Catie
Edmondson}

\begin{itemize}
\item
  Feb. 4, 2020
\item
  \begin{itemize}
  \item
  \item
  \item
  \item
  \item
  \item
  \end{itemize}
\end{itemize}

WASHINGTON --- Senator
\href{https://www.nytimes3xbfgragh.onion/2020/07/07/us/politics/susan-collins-maine-sara-gideon.html}{Susan
Collins}, Republican of Maine, said on Tuesday that she would vote to
acquit
\href{https://www.nytimes3xbfgragh.onion/2020/02/04/us/politics/state-of-the-union-address.html}{President
Trump} of abuse of power and obstruction of Congress, calling his
conduct ``wrong'' but saying she could not support removing him from
office.

``I do not believe that the House has met its burden of showing that the
president's conduct, however flawed, warrants the extreme step of
immediate removal from office,'' Ms. Collins said in a speech from the
Senate floor.

Ms. Collins, a moderate who is facing a steep re-election challenge this
year, was one of only two Republicans to break with the party last week
and vote in favor of considering new witnesses in Mr. Trump's
impeachment trial. But in announcing her vote on the verdict on Tuesday,
she joined a growing chorus in her party who has argued that removing
the president for his misconduct little more than nine months before he
faces re-election was excessive.

``It is my judgment that except when extraordinary circumstances require
a different result, we should entrust to the people the most fundamental
decision of a democracy --- namely who should lead their country,'' Ms.
Collins said.

She said that Mr. Trump's call to President Volodymyr Zelensky of
Ukraine, in which he asked for an investigation of former Vice President
Joseph R. Biden Jr., was ``improper and demonstrated very poor
judgment,'' but said there had been ``conflicting evidence in the
record'' about Mr. Trump's motivations.

In an interview with CBS, Ms. Collins said she believed that Mr. Trump
had ``learned from this case'' and that he will be ``much more cautious
in the future.'' But the president has never acknowledged any
wrongdoing, instead insisting his call with Mr. Zelensky was
``perfect.''

Ms. Collins's speech came as senators took turns on the Senate floor on
Tuesday announcing how they would vote when they render a verdict on
Wednesday in the third presidential impeachment trial in the nation's
history.

The bitterly divided Senate is all but certain to acquit Mr. Trump on
\href{https://www.nytimes3xbfgragh.onion/interactive/2020/01/22/us/politics/impeachment-articles-arguments.html}{both
charges}. It would take a two-thirds vote, or 67 senators, to convict
and remove him, a threshold that neither side expects to materialize.

Mr. Trump is accused of pressuring Ukraine to interfere in the 2020
presidential campaign on his behalf, by withholding military aid and a
White House meeting to lean on the country to investigate his political
rivals. The impeachment trial was not formally meeting on Tuesday,
before Mr. Trump's
\href{https://www.nytimes3xbfgragh.onion/2020/02/04/us/politics/what-time-is-state-of-the-union.html}{State
of the Union} address to a joint session of Congress. But their comments
on Tuesday were the last opportunity for senators to explain their
positions before voting on the verdict, and they appeared to be aimed at
their constituents, their core supporters, and in some cases, the
president himself.

Here's a roundup of what other key senators said.

\hypertarget{mcconnell-blasts-trumps-impeachment-as-dangerous}{%
\subsection{McConnell blasts Trump's impeachment as
`dangerous.'}\label{mcconnell-blasts-trumps-impeachment-as-dangerous}}

Taking a victory lap for what he called the ``sober and stable Senate,''
Senator Mitch McConnell, Republican of Kentucky and the majority leader,
issued a stern rebuke of the House Democrats' case and strategy, casting
it as a politically motivated attack that amounted to the ``most rushed,
least fair and least thorough presidential impeachment inquiry in
American history.''

``Washington Democrats think President Donald Trump committed a high
crime or misdemeanor the moment he defeated Hillary Clinton,'' Mr.
McConnell said, referring to the president's victory in 2016. ``That is
the original sin of this presidency: that he won and they lost.''

In his most detailed remarks on the impeachment managers' case to date,
Mr. McConnell undercut an argument that the White House defense team had
presented: that impeachment requires the violation of a criminal statue.
But while Mr. McConnell said he did not subscribe to that legal theory,
he condemned House Democrats all the same for sailing into ``new and
dangerous waters.''

\hypertarget{schumer-says-the-senate-trial-fails-the-laugh-test}{%
\subsection{Schumer says the Senate trial `fails the laugh
test.'}\label{schumer-says-the-senate-trial-fails-the-laugh-test}}

Senator Chuck Schumer, Democrat of New York, spoke only briefly, to
rebut what he called the majority leader's talking points.

Defending the House managers' case as ``compelling,'' Mr. Schumer
denounced Senate Republicans for blocking his motion to consider hearing
from additional witnesses --- including
\href{https://www.nytimes3xbfgragh.onion/2020/01/31/us/politics/trump-bolton-ukraine.html}{John
R. Bolton}, the president's former national security adviser who had
offered to testify --- and receive more evidence. The trial they
created, he said, ``fails the laugh test.''

``The Republicans refused to get the evidence because they were afraid
of what it would show,'' Mr. Schumer said, ``and that's all that needs
to be said.''

\hypertarget{ernst-argues-the-voters-should-decide-trumps-fate}{%
\subsection{Ernst argues the voters should decide Trump's
fate}\label{ernst-argues-the-voters-should-decide-trumps-fate}}

Senator Joni Ernst, Republican of Iowa, who is up for re-election in
November, adopted an argument similar to one outlined by Mr. Trump's
defense team on their final day of arguments: that to remove the
president months before the election would be to subvert the will of
voters.

``The House managers' arguments have argued that the American people
cannot be trusted to render their own judgment on this president,'' Ms.
Ernst said. ``I reject this premise, the complete distrust of the
American people, with everything in my heart.''

Ms. Ernst did not directly address the president's conduct, but jabbed
at House Democrats for trying to impose their own conventional wisdom
onto how Mr. Trump conducted diplomacy. Foreign policy, Ms. Ernst said,
is an ``art, not a science'' and ``trying to insert a formula into every
presidential interaction with a foreign leader,'' she continued, was to
veer onto a ``path towards ineffectiveness.''

\hypertarget{kaine-laments-a-toxic-president-who-has-infected-the-senate}{%
\subsection{Kaine laments `a toxic president' who has infected the
Senate}\label{kaine-laments-a-toxic-president-who-has-infected-the-senate}}

Senator Tim Kaine, Democrat of Virginia, pledged to vote to remove Mr.
Trump, and lamented the Senate's ``capitulation,'' a turn of events, he
said, that surprised him.

``We have allowed a toxic president to infect the Senate and warp its
behavior,'' Mr. Kaine said. ``And now the Senate's refusal to allow a
fair trial threatens to spread a broader public anxiety about whether
`impartial justice' is a hollow fiction.''

Mr. Kaine, known in the Senate for his optimism, sounded a dark warning,
insisting that an acquittal would only embolden Mr. Trump to engage in
worse conduct. ``I will not be part of this continual degradation of
public trust,'' he said.

Advertisement

\protect\hyperlink{after-bottom}{Continue reading the main story}

\hypertarget{site-index}{%
\subsection{Site Index}\label{site-index}}

\hypertarget{site-information-navigation}{%
\subsection{Site Information
Navigation}\label{site-information-navigation}}

\begin{itemize}
\tightlist
\item
  \href{https://help.nytimes3xbfgragh.onion/hc/en-us/articles/115014792127-Copyright-notice}{©~2020~The
  New York Times Company}
\end{itemize}

\begin{itemize}
\tightlist
\item
  \href{https://www.nytco.com/}{NYTCo}
\item
  \href{https://help.nytimes3xbfgragh.onion/hc/en-us/articles/115015385887-Contact-Us}{Contact
  Us}
\item
  \href{https://www.nytco.com/careers/}{Work with us}
\item
  \href{https://nytmediakit.com/}{Advertise}
\item
  \href{http://www.tbrandstudio.com/}{T Brand Studio}
\item
  \href{https://www.nytimes3xbfgragh.onion/privacy/cookie-policy\#how-do-i-manage-trackers}{Your
  Ad Choices}
\item
  \href{https://www.nytimes3xbfgragh.onion/privacy}{Privacy}
\item
  \href{https://help.nytimes3xbfgragh.onion/hc/en-us/articles/115014893428-Terms-of-service}{Terms
  of Service}
\item
  \href{https://help.nytimes3xbfgragh.onion/hc/en-us/articles/115014893968-Terms-of-sale}{Terms
  of Sale}
\item
  \href{https://spiderbites.nytimes3xbfgragh.onion}{Site Map}
\item
  \href{https://help.nytimes3xbfgragh.onion/hc/en-us}{Help}
\item
  \href{https://www.nytimes3xbfgragh.onion/subscription?campaignId=37WXW}{Subscriptions}
\end{itemize}
