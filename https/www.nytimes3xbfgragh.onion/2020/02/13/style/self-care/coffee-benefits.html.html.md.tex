Sections

SEARCH

\protect\hyperlink{site-content}{Skip to
content}\protect\hyperlink{site-index}{Skip to site index}

\href{https://www.nytimes3xbfgragh.onion/section/style/self-care/}{Self-Care}

\href{https://myaccount.nytimes3xbfgragh.onion/auth/login?response_type=cookie\&client_id=vi}{}

\href{https://www.nytimes3xbfgragh.onion/section/todayspaper}{Today's
Paper}

\href{/section/style/self-care/}{Self-Care}\textbar{}Is Coffee Good for
You?

\href{https://nyti.ms/31PQftM}{https://nyti.ms/31PQftM}

\begin{itemize}
\item
\item
\item
\item
\item
\end{itemize}

Advertisement

\protect\hyperlink{after-top}{Continue reading the main story}

Supported by

\protect\hyperlink{after-sponsor}{Continue reading the main story}

Scam or not

\hypertarget{is-coffee-good-for-you}{%
\section{Is Coffee Good for You?}\label{is-coffee-good-for-you}}

Yes! But it depends on the kind of coffee and the quantity.

\includegraphics{https://static01.graylady3jvrrxbe.onion/images/2020/02/14/fashion/14scams-coffee2/14scams-coffee2-articleLarge.jpg?quality=75\&auto=webp\&disable=upscale}

By Dawn MacKeen

\begin{itemize}
\item
  Published Feb. 13, 2020Updated Feb. 22, 2020
\item
  \begin{itemize}
  \item
  \item
  \item
  \item
  \item
  \end{itemize}
\end{itemize}

We've come a long way from the cans of Folgers that filled our
grandparents' cupboards, with our oat milk lattes, cold brews and
Frappuccinos. Some of us are still very utilitarian about the drink
while others perform elaborate rituals. The fourth most popular beverage
in the country, coffee is steeped into our culture. Just the right
amount can improve our mood; too much may make us feel anxious and
jittery.

\begin{center}\rule{0.5\linewidth}{\linethickness}\end{center}

\hypertarget{is-coffee-good-for-me}{%
\subsection{Is coffee good for me?}\label{is-coffee-good-for-me}}

Yes.

In moderation, coffee seems to be good for most people --- that's 3 to 5
cups daily, or up to 400 milligrams of caffeine.

``The evidence is pretty consistent that coffee is associated with a
lower risk of mortality,'' said Erikka Loftfield, a research fellow at
the National Cancer Institute who has studied the beverage.

For years, coffee was believed to be a possible carcinogen, but the 2015
\href{https://health.gov/dietaryguidelines/2015-scientific-report/PDFs/Scientific-Report-of-the-2015-Dietary-Guidelines-Advisory-Committee.pdf}{Dietary
Guidelines} **** helped to change perception. **** For the first time,
moderate coffee drinking was included as part of a healthy diet. When
researchers controlled for lifestyle factors, like how **** many heavy
coffee drinkers also smoked, the data tipped in coffee's favor.

A large 2017 review on coffee consumption and human health **** in the
\href{https://www.bmj.com/content/359/bmj.j5024.long}{British Medical
Journal} also found **** that most of the time, coffee was associated
with a benefit, rather than a harm. In examining more than 200 reviews
of previous studies, the authors observed that moderate coffee drinkers
had less cardiovascular disease, and premature death from all causes,
including heart attacks and stroke, than those skipping the beverage.

In addition, experts say some of **** the strongest protective effects
may be with Type 2 diabetes, Parkinson's disease, and liver conditions
such as cirrhosis, liver cancer and chronic liver disease. For example,
having about five cups of coffee a day, instead of none, is correlated
with a 30 percent decreased risk of Type 2 diabetes, according to a
meta-analysis of 30 studies.

The potential benefit from coffee might be from the polyphenols, which
are plant compounds that have antioxidant properties, according to Dr.
Giuseppe Grosso, an assistant professor in human nutrition at University
of Catania in Italy and the lead author of an ****
\href{https://www.annualreviews.org/doi/10.1146/annurev-nutr-071816-064941}{umbrella
review} **** in the Annual Review of Nutrition.

However, coffee isn't for everyone. There are concerns about
overconsumption. This is especially true for expecting mothers because
the safety of caffeine during pregnancy is unclear. While the research
into coffee's impact on health is ongoing, most of the work in this
field is observational.

``We don't know for sure if coffee is the cause of the health
benefits,'' said Jonathan Fallowfield, a professor at the University of
Edinburgh, and co-author of the British Medical Journal review.
**``**These findings could be due to other factors or behaviors present
in coffee drinkers.''

\begin{center}\rule{0.5\linewidth}{\linethickness}\end{center}

\hypertarget{does-the-way-coffee-is-prepared-matter}{%
\subsection{Does the way coffee is prepared
matter?}\label{does-the-way-coffee-is-prepared-matter}}

Yes. Do you prefer a dark or light roast? Coarse grinding or fine?
Arabica or robusta?

``All of these different aspects affect the taste, but also affect the
compounds within the coffees,'' said Neal Freedman, a senior
investigator with the National Cancer Institute. ``But it's not clear at
all how these different levels of compounds may be related to health.''

Roasting, for example, reduces the amount of chlorogenic acids, but
other antioxidant compounds are formed. Espresso has the highest
concentration of many compounds because it has less water than drip
coffee.

A
\href{https://jamanetwork.com/journals/jamainternalmedicine/fullarticle/2686145}{study}
in JAMA Internal Medicine examined the coffee habits of nearly 500,000
people in the U.K. and found that it didn't matter if they drank one cup
or chain-drank eight --- regular or decaf --- or whether they were fast
metabolizers of coffee or slow. They were linked to a lower risk of
death from all causes, except with instant coffee, the evidence was
weaker.

The way you prepare your cup of joe may influence your cholesterol
levels, too. ``The one coffee we know not suitable to be drinking is the
boiled coffee,'' said Marilyn C. Cornelis, an assistant professor in
preventive medicine at the Northwestern University Feinberg School of
Medicine and co-author of the JAMA Internal Medicine study.

Examples of this include the plunge-happy French press, Scandinavian
coffee, or Greek and Turkish coffee --- the kind commonly consumed in
the Middle East. (When poured, the unfiltered grounds settle on the tiny
cup's bottom like sludge. To peek into the future, elders in the region
have a tradition of reading the sediment of an overturned cup, like a
crystal ball.)

However, the oil in
\href{https://www.sciencedirect.com/science/article/abs/pii/S0021915097000993}{boiled
coffee} has cafestol and kahweol, compounds called diterpenes. They are
shown to raise LDL, the bad cholesterol, and slightly lower HDL, what's
known as the good kind.

``If you filter the coffee, then it's no issue at all,'' said Rob van
Dam, a professor at Saw Swee Hock School of Public Health at National
University of Singapore. ``For people with cholesterol issues, it's
better to switch to other types of coffee.'' He's been studying coffee
for two decades. (And, yes, he's had a lot of coffee in that time.)

However, other researchers say not to throw out the boiled coffee just
yet. The clinical significance of such small increases in cholesterol
may be questionable, given that it's not associated with an increase in
cardiovascular deaths.

Many consumers have also swapped loose grounds for coffee pods. While
there are environmental concerns with single use pods, researchers
believe them to hold the same benefits as, say, drip coffee. The latter
applies to cold brew, too, but more research is needed.

\begin{center}\rule{0.5\linewidth}{\linethickness}\end{center}

\hypertarget{is-this-a-scam}{%
\subsection{Is This A Scam?}\label{is-this-a-scam}}

\hypertarget{is-}{%
\subsubsection{Is ...}\label{is-}}

Celery Juice

,

Kombucha

,

Activated Charcoal

,

CBD

,

Turmeric

,

Fish Oil

,

Chlorophyll

,

Intermittent Fasting

,

The Keto Diet

,

Probiotics

,

Collagen

,

Coffee

,

\hypertarget{a-scam}{%
\subsubsection{A Scam?}\label{a-scam}}

Facts about wellness.

Will these trends change your life --- or

take your money?

\begin{center}\rule{0.5\linewidth}{\linethickness}\end{center}

\hypertarget{do-all-kinds-of-coffee-have-the-same-amount-of-caffeine}{%
\subsection{Do all kinds of coffee have the same amount of
caffeine?}\label{do-all-kinds-of-coffee-have-the-same-amount-of-caffeine}}

No. Espresso has the highest concentration of caffeine, packing about 70
milligrams into a one-ounce shot, but is consumed in smaller quantities.
By comparison, a typical 12-ounce serving of drip coffee has 200
milligrams of caffeine, more than instant's 140. And, yes, brewed decaf
has caffeine, too --- 8 milligrams --- which can add up.

When buying coffee, you never really know what you're going to get. At
one Florida coffee house, over a six-day period, the same 16-ounce
\href{https://academic.oup.com/jat/article/27/7/520/784168}{breakfast
blend} fluctuated from 259 milligrams all the way up to 564 --- which
goes beyond federal recommendations.

But for some of us, knowing how much caffeine is in our coffee can be
especially important. You've probably noticed it before. How a friend
can pound quadruple espresso shots at 10 p.m. and sleep afterward, while
you can't have any past noon, or you'll be watching ``Seinfeld'' reruns
until dawn. Some of us have a polymorphism, a genetic variant that slows
our metabolism for caffeine. It's these individuals that Dr. Grosso
recommends limit their refills. ``They take a coffee, and then they have
the second and the third, and they still have the caffeine of the
first,'' he said.

You can even
\href{https://well.blogs.nytimes3xbfgragh.onion/2016/07/12/for-coffee-drinkers-the-buzz-may-be-in-your-genes/}{find
out whether you are a fast or slow metabolizer} through a variety of
direct-to-consumer testing services, including 23andMe.

\begin{center}\rule{0.5\linewidth}{\linethickness}\end{center}

\hypertarget{is-coffee-addictive}{%
\subsection{Is coffee addictive?}\label{is-coffee-addictive}}

Evidence suggests there can be a reliance on the drink, and tolerance
builds over time. Withdrawal symptoms include a headache, fatigue,
irritability, difficulty concentrating, and depressed mood.

Indeed, caffeine is a psychoactive drug, and coffee is its biggest
dietary source. About a half-hour after sipping a cup of joe, the
caffeine kicks in, and is quickly absorbed. Blood vessels constrict.
Blood pressure increases. A moderate amount of caffeine can wake you up,
boost your mood, energy, alertness, concentration and even athletic
performance. On average, it takes four to six hours to metabolize half
the caffeine.

For those knocking back more than 400 milligrams of caffeine a day,
there's not enough evidence to assess the safety, according to the
Dietary Guidelines.
\href{https://www.liebertpub.com/doi/10.1089/caff.2019.0020}{Higher
doses} can lead to caffeine intoxication, with its shakiness,
nervousness, and irregular heartbeat. Caffeine is also
\href{https://www.sciencedirect.com/science/article/abs/pii/S1087079216000150?via\%3Dihub}{linked}
with delaying the time it takes for you fall asleep, how long you stay
there, and the reported quality of that shut eye.

``I think that caffeine is so common and so ingrained in our culture,
and daily habits, that we often don't think about it as a potential
source of problems,'' said Mary M. Sweeney, an assistant professor of
psychiatry and behavioral sciences at the Johns Hopkins University
School of Medicine.

Cutting down coffee may help with gastroesophageal reflux, too. A
\href{https://cghjournal.org/retrieve/pii/S1542356519313801}{new study}
found that women drinking caffeinated beverages --- coffee, tea, or soda
--- were associated with a small but increased risk of symptoms, like
heartburn. The study's authors predicted fewer symptoms when
substituting two servings of the drinks with water.

Current available research hasn't determined what amount of caffeine can
be safely consumed during pregnancy, according to the American College
of Obstetricians and Gynecologists. Caffeine does cross the placenta so
some doctors may recommend pregnant women stay below 200 milligrams of
coffee daily.

Extremely high doses of caffeine can be
\href{https://www.ncbi.nlm.nih.gov/pmc/articles/PMC6011436/}{fatal.} But
researchers say that's more likely to occur accidentally with caffeine
powder or pills. ``You don't see a lot of people going into the
emergency room because they accidentally drank too much coffee,'' said
Dr. van Dam.

\begin{center}\rule{0.5\linewidth}{\linethickness}\end{center}

\hypertarget{what-is-a-coffee-bean}{%
\subsection{What is a coffee bean?}\label{what-is-a-coffee-bean}}

Inside the red fruit of \emph{coffea} lie two coffee beans. Green in
color, the duo spoon together, the rich brown hue to appear only after
roasting. In fact, they aren't beans at all. ``It's like a cherry, you
pick off the tree,'' said Patrick Brown, a professor of plant sciences
at University of California, Davis. Unlike the cherry, though, the seed
is the prize, and the flesh is discarded.

In addition to caffeine, coffee is a dark brew of a thousand chemical
compounds that could have potential therapeutic effects on the body. One
key component, chlorogenic acid, is a polyphenol found in many fruits
and vegetables. Coffee is also a good dietary source of vitamin B3,
magnesium and potassium.

``People often see coffee just as a vehicle for caffeine, but, of
course, it's a very complex plant beverage,'' said Dr. van Dam.

With \emph{coffea's} estimated 124 species, most of flavors remain
untapped; and perhaps will be forever, with an estimated 60 percent
under threat of extinction, largely from climate change, disease, pests
and deforestation. What fills our mugs at cafes, the office, and on road
trips are from two species: arabica and canephora, known as robusta.
Arabica fills specialty cafes, and costs more than robusta, which fuels
instant coffees and some espressos.

For all of the pomp swirling around arabica, the fact remains it is an
extremely homogeneous little seed. Almost all of the world's arabica
coffee progeny traces itself back a few plants from Ethiopia, coffee's
birthplace, or Yemen.

\begin{center}\rule{0.5\linewidth}{\linethickness}\end{center}

\hypertarget{does-adding-milk-or-sugar-cancel-out-benefits}{%
\subsection{Does adding milk or sugar cancel out
benefits?}\label{does-adding-milk-or-sugar-cancel-out-benefits}}

Doctors don't know. One 2015
\href{https://www.ncbi.nlm.nih.gov/pmc/articles/PMC5875735/}{study}
found that those adding sugar, cream or milk had the same associated
benefit as those who preferred it black. But the coffee industry has
exploded since the '90s when the older adults in the study filled out
their dietary history. ``It was only about a tablespoon of cream or
milk, and a teaspoon of sugar,'' said the study's lead author, Dr.
Loftfield, with the National Cancer Institute. ``This is very different,
potentially, than some of these coffee beverages you see on the market
today.''

Sweet coffee and tea are the fourth largest source of sugar in the diets
of adults, according to the October survey from the U.S.D.A. That
includes dessert-like beverages, like Dunkin' Donuts' 860-calorie creamy
frozen coconut caramel coffee drink, with 17 grams of saturated fat, and
129 grams of total sugars. Experts say some of these drinks bear little
relation to the 2-calorie cup of black
\href{https://fdc.nal.usda.gov/fdc-app.html\#_blank}{coffee} of the
past, worrying health officials.

``When you talk about a drink that has that load of unhealthy fats and
that much sugar, can't possibly be a healthy beverage on balance,'' Dr.
Jim Krieger, a clinical professor of medicine and health services at the
University of Washington. ``That amount of sugar alone is astronomical
compared to the current recommendations of U.S. Dietary Guidelines of 50
grams of sugar a day.''

The concern is heightened, experts say, especially because an estimated
43 percent of teens are now drinking coffee --- nearly doubling since
2003 --- according to the research firm Kantar, driven partly by sweet
drinks.

``People should worry a lot about what they put in the coffee and what
the food and beverage industry puts in it,'' said Laura Schmidt, a
professor at the University of California San Francisco School of
Medicine. ``And sweetened coffee is one of the things that the beverage
industry is pushing on the public now that consumers have turned away
from soda for health reasons.''

\begin{center}\rule{0.5\linewidth}{\linethickness}\end{center}

\hypertarget{should-i-start-pounding-down-more-coffee}{%
\subsection{Should I start pounding down more
coffee?}\label{should-i-start-pounding-down-more-coffee}}

Depends on your goals in life.

If you are enjoying the drink in moderation, doctors say continue
onward, and savor those sips. And for those patients with a sensitivity
to the beverage, Dr. Sophie Balzora, a gastroenterologist, weighs the
benefits and risks very carefully. The clinical associate professor of
medicine at N.Y.U. School of Medicine understands its cultural
significance, and knows to tread lightly. As she put it: ``Robbing
people of their coffee seems cruel.''

Advertisement

\protect\hyperlink{after-bottom}{Continue reading the main story}

\hypertarget{site-index}{%
\subsection{Site Index}\label{site-index}}

\hypertarget{site-information-navigation}{%
\subsection{Site Information
Navigation}\label{site-information-navigation}}

\begin{itemize}
\tightlist
\item
  \href{https://help.nytimes3xbfgragh.onion/hc/en-us/articles/115014792127-Copyright-notice}{©~2020~The
  New York Times Company}
\end{itemize}

\begin{itemize}
\tightlist
\item
  \href{https://www.nytco.com/}{NYTCo}
\item
  \href{https://help.nytimes3xbfgragh.onion/hc/en-us/articles/115015385887-Contact-Us}{Contact
  Us}
\item
  \href{https://www.nytco.com/careers/}{Work with us}
\item
  \href{https://nytmediakit.com/}{Advertise}
\item
  \href{http://www.tbrandstudio.com/}{T Brand Studio}
\item
  \href{https://www.nytimes3xbfgragh.onion/privacy/cookie-policy\#how-do-i-manage-trackers}{Your
  Ad Choices}
\item
  \href{https://www.nytimes3xbfgragh.onion/privacy}{Privacy}
\item
  \href{https://help.nytimes3xbfgragh.onion/hc/en-us/articles/115014893428-Terms-of-service}{Terms
  of Service}
\item
  \href{https://help.nytimes3xbfgragh.onion/hc/en-us/articles/115014893968-Terms-of-sale}{Terms
  of Sale}
\item
  \href{https://spiderbites.nytimes3xbfgragh.onion}{Site Map}
\item
  \href{https://help.nytimes3xbfgragh.onion/hc/en-us}{Help}
\item
  \href{https://www.nytimes3xbfgragh.onion/subscription?campaignId=37WXW}{Subscriptions}
\end{itemize}
