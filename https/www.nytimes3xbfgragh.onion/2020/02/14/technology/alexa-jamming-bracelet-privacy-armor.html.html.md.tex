\href{/section/technology}{Technology}\textbar{}Activate This `Bracelet
of Silence,' and Alexa Can't Eavesdrop

\url{https://nyti.ms/38tQOfs}

\begin{itemize}
\item
\item
\item
\item
\item
\end{itemize}

\includegraphics{https://static01.graylady3jvrrxbe.onion/images/2020/02/16/business/14PRIVACYARMOR-01/merlin_168919842_48493103-4cef-4bc7-89e4-2136a08d542f-articleLarge.jpg?quality=75\&auto=webp\&disable=upscale}

Sections

\protect\hyperlink{site-content}{Skip to
content}\protect\hyperlink{site-index}{Skip to site index}

\hypertarget{activate-this-bracelet-of-silence-and-alexa-cant-eavesdrop}{%
\section{Activate This `Bracelet of Silence,' and Alexa Can't
Eavesdrop}\label{activate-this-bracelet-of-silence-and-alexa-cant-eavesdrop}}

Microphones and cameras lurk everywhere. You may want to slip on some
privacy armor.

The bracelet will jam an Echo or any other microphone in the vicinity
from listening in on the wearer's conversations.Credit...Petra Ford for
The New York Times

Supported by

\protect\hyperlink{after-sponsor}{Continue reading the main story}

By \href{https://www.nytimes3xbfgragh.onion/by/kashmir-hill}{Kashmir
Hill}

\begin{itemize}
\item
  Feb. 14, 2020
\item
  \begin{itemize}
  \item
  \item
  \item
  \item
  \item
  \end{itemize}
\end{itemize}

Last year, Ben Zhao decided to buy an Alexa-enabled Echo speaker for his
Chicago home. Mr. Zhao just wanted a digital assistant to play music,
but his wife, Heather Zheng, was not enthused. ``She freaked out,'' he
said.

Ms. Zheng characterized her reaction differently. First she objected to
having the device in their house, she said. Then, when Mr. Zhao put the
Echo in a work space they shared, she made her position perfectly
clear:``I said, `I don't want that in the office. Please unplug it. I
know the microphone is constantly on.'''

Mr. Zhao and Ms. Zheng are computer science professors at the University
of Chicago, and they decided to channel their disagreement into
something productive. With the help of an assistant professor, Pedro
Lopes, they designed a piece of digital armor: a
``\href{http://sandlab.cs.uchicago.edu/jammer/}{bracelet of silence}''
that will jam the Echo or any other microphones in the vicinity from
listening in on the wearer's conversations.

The bracelet is like an anti-smartwatch, both in its cyberpunk aesthetic
and in its purpose of defeating technology. A large, somewhat ungainly
white cuff with spiky transducers, the bracelet has 24 speakers that
emit ultrasonic signals when the wearer turns it on. The sound is
imperceptible to most ears, with the possible exception of
\href{https://www.scientificamerican.com/article/bring-science-home-high-frequency-hearing/}{young
people} and dogs, but nearby microphones will detect the high-frequency
sound instead of other noises.

``It's so easy to record these days,'' Mr. Lopes said. ``This is a
useful defense. When you have something private to say, you can activate
it in real time. When they play back the recording, the sound is going
to be gone.''

During a phone interview, Mr. Lopes turned on the bracelet, resulting in
static-like white noise for the listener on the other end.

\hypertarget{polite-surveillance-society}{%
\subsection{Polite Surveillance
Society}\label{polite-surveillance-society}}

As American homes are steadily outfitted with recording equipment, the
surveillance state has taken on an air of domesticity. Google and Amazon
have sold millions of Nest and Ring security cameras, while an estimated
\href{https://www.nationalpublicmedia.com/insights/reports/smart-audio-report/}{one
in five American adults} now owns a smart speaker. Knocking on someone's
door or chatting in someone's kitchen now involves the distinct
possibility of being recorded.

It all presents new questions of etiquette about whether and how to warn
guests that their faces and words could end up on a tech company's
servers, or even
\href{https://www.nytimes3xbfgragh.onion/2019/08/21/technology/personaltech/alexa-siri-google-assistant-listen.html}{in
the hands of strangers}.

By design, smart speakers have microphones that are always on, listening
for so-called wake words like ``Alexa,'' ``Hey, Siri,'' or ``O.K.,
Google.'' Only after hearing that cue are they supposed to start
recording. But contractors hired by device makers to review recordings
for quality reasons
\href{https://www.vrt.be/vrtnws/en/2019/07/10/google-employees-are-eavesdropping-even-in-flemish-living-rooms/}{report}
hearing clips that were most likely
\href{https://www.bloomberg.com/news/articles/2019-04-10/is-anyone-listening-to-you-on-alexa-a-global-team-reviews-audio}{captured
unintentionally}, including
\href{https://www.theguardian.com/technology/2019/jul/26/apple-contractors-regularly-hear-confidential-details-on-siri-recordings}{drug
deals and sex}.

Two Northeastern University researchers, David Choffnes and Daniel
Dubois, recently played 120 hours of television for an audience of smart
speakers to see what activates the devices. They found that
\href{https://moniotrlab.ccis.neu.edu/smart-speakers-study/}{the
machines woke up dozens of times and started recording} after hearing
phrases similar to their wake words.

``People fear that these devices are constantly listening and recording
you. They're not,'' Mr. Choffnes said. ``But they do wake up and record
you at times when they shouldn't.''

Rick Osterloh, Google's head of hardware, recently said homeowners
should disclose the presence of smart speakers to their guests. ``I
would, and do, when someone enters into my home, and it's probably
something that the products themselves should try to indicate,'' he
\href{https://www.bbc.com/news/technology-50048144}{told the BBC} last
year.

Welcome mats might one day be swapped out for warning mats. Or perhaps
the tech companies will engineer their products to introduce themselves
when they hear a new voice or see a new face. Of course, that could also
lead to uncomfortable situations, like having the Alexa in your bedside
Echo Dot suddenly introduce herself to your one-night stand.

\hypertarget{no-longer-shunned-as-loonies}{%
\subsection{`No Longer Shunned as
Loonies'}\label{no-longer-shunned-as-loonies}}

The ``bracelet of silence'' is not the first device invented by
researchers to stuff up digital assistants' ears. In 2018, two designers
created \href{http://bjoernkarmann.dk/project_alias}{Project Alias}, an
appendage that can be placed over a smart speaker to deafen it. But Ms.
Zheng argues that a jammer should be portable to protect people as they
move through different environments, given that you don't always know
where a microphone is lurking.

At this point, the bracelet is just a prototype. The researchers say
that they could manufacture it for as little as \$20, and that a handful
of investors have asked them about commercializing it.

``With the Internet of Things, the battle is lost,'' Mr. Zhao said,
referring to a lack of control over data captured by smart devices,
whether it gets into the hands of tech companies or
\href{https://www.nytimes3xbfgragh.onion/2019/12/15/us/Hacked-ring-home-security-cameras.html}{hackers}.

``The future is to have all these devices around you, but you will have
to assume they are potentially compromised,'' he added. ``Your circle of
trust will have to be much smaller, sometimes down to your actual
body.''

Other precursors to the bracelet include a
``\href{http://www.coop-himmelblau.at/architecture/projects/chbl-jammer-coat}{jammer
coat}'' designed by an Austrian architecture firm in 2014 to block radio
waves that could collect information from a person's phone or credit
cards. In 2012, the artist Adam Harvey created silver-plated
\href{https://ahprojects.com/stealth-wear/}{stealth wear} garments that
masked people's heat signature to protect them from the eyes of drones,
as well as a line of makeup and hairstyles, called
\href{https://cvdazzle.com/}{CV Dazzle}, to thwart facial recognition
cameras.

\includegraphics{https://static01.graylady3jvrrxbe.onion/images/2020/02/14/business/14PRIVACYARMOR-reflect-01/14PRIVACYARMOR-reflect-01-articleLarge.jpg?quality=75\&auto=webp\&disable=upscale}

In 2016, Scott Urban, an eyewear maker in Chicago, developed a line of
reflective frames that turned back visible and infrared light. When a
surveillance camera films a person wearing the \$164 frames, the
reflected light blurs out the face. Mr. Urban called them
\href{https://www.reflectacles.com/}{Reflectacles}.

He is now working full time on privacy protection eyewear, including a
new version with lenses that absorb infrared light to deter
iris-scanning and facial recognition cameras. His customers include
privacy enthusiasts, political activists and card counters whose faces
have been placed on casinos' watch lists.

``People into their privacy are no longer shunned as loonies,'' Mr.
Urban said. ``It's become a concern for people of all ages, political
perspectives and walks of life.''

He added: ``New technologies are continually eroding our privacy and
anonymity. People are looking for an opt-out, which is what I'm trying
to provide.''

Woodrow Hartzog, a law and computer science professor at Northeastern
University, doesn't think privacy armor is the solution to our modern
woes.

``It creates an arms race, and consumers will lose in that race,'' he
said. ``Any of these things is a half-measure or a stopgap. There will
always be a way around it.''

Rather than building individual defenses, Mr. Hartzog believes, we need
policymakers to pass laws that more effectively guard our privacy and
give us control over our data.

``Until then, we're playing cat and mouse,'' he said. ``And that always
ends poorly for the mouse.''

Advertisement

\protect\hyperlink{after-bottom}{Continue reading the main story}

\hypertarget{site-index}{%
\subsection{Site Index}\label{site-index}}

\hypertarget{site-information-navigation}{%
\subsection{Site Information
Navigation}\label{site-information-navigation}}

\begin{itemize}
\tightlist
\item
  \href{https://help.nytimes3xbfgragh.onion/hc/en-us/articles/115014792127-Copyright-notice}{©~2020~The
  New York Times Company}
\end{itemize}

\begin{itemize}
\tightlist
\item
  \href{https://www.nytco.com/}{NYTCo}
\item
  \href{https://help.nytimes3xbfgragh.onion/hc/en-us/articles/115015385887-Contact-Us}{Contact
  Us}
\item
  \href{https://www.nytco.com/careers/}{Work with us}
\item
  \href{https://nytmediakit.com/}{Advertise}
\item
  \href{http://www.tbrandstudio.com/}{T Brand Studio}
\item
  \href{https://www.nytimes3xbfgragh.onion/privacy/cookie-policy\#how-do-i-manage-trackers}{Your
  Ad Choices}
\item
  \href{https://www.nytimes3xbfgragh.onion/privacy}{Privacy}
\item
  \href{https://help.nytimes3xbfgragh.onion/hc/en-us/articles/115014893428-Terms-of-service}{Terms
  of Service}
\item
  \href{https://help.nytimes3xbfgragh.onion/hc/en-us/articles/115014893968-Terms-of-sale}{Terms
  of Sale}
\item
  \href{https://spiderbites.nytimes3xbfgragh.onion}{Site Map}
\item
  \href{https://help.nytimes3xbfgragh.onion/hc/en-us}{Help}
\item
  \href{https://www.nytimes3xbfgragh.onion/subscription?campaignId=37WXW}{Subscriptions}
\end{itemize}
