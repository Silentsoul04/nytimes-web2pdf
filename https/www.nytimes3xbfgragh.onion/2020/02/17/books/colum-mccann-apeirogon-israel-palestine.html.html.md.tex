Sections

SEARCH

\protect\hyperlink{site-content}{Skip to
content}\protect\hyperlink{site-index}{Skip to site index}

\href{https://www.nytimes3xbfgragh.onion/section/books}{Books}

\href{https://myaccount.nytimes3xbfgragh.onion/auth/login?response_type=cookie\&client_id=vi}{}

\href{https://www.nytimes3xbfgragh.onion/section/todayspaper}{Today's
Paper}

\href{/section/books}{Books}\textbar{}How Colum McCann Shaped Loss Into
a Book

\url{https://nyti.ms/2Sy7g8G}

\begin{itemize}
\item
\item
\item
\item
\item
\end{itemize}

Advertisement

\protect\hyperlink{after-top}{Continue reading the main story}

Supported by

\protect\hyperlink{after-sponsor}{Continue reading the main story}

\hypertarget{how-colum-mccann-shaped-loss-into-a-book}{%
\section{How Colum McCann Shaped Loss Into a
Book}\label{how-colum-mccann-shaped-loss-into-a-book}}

``Apeirogon,'' the latest novel from the National Book Award winner,
delves into the Israeli-Palestinian conflict through the eyes of two
grieving fathers.

\includegraphics{https://static01.graylady3jvrrxbe.onion/images/2020/02/17/books/17McCann1-sbu/17McCann1-sbu-articleLarge-v2.jpg?quality=75\&auto=webp\&disable=upscale}

\href{https://nytimes3xbfgragh.onion/by/joumana-khatib}{\includegraphics{https://static01.graylady3jvrrxbe.onion/images/2018/09/13/multimedia/author-joumana-khatib/author-joumana-khatib-thumbLarge.png}}

By \href{https://nytimes3xbfgragh.onion/by/joumana-khatib}{Joumana
Khatib}

\begin{itemize}
\item
  Feb. 17, 2020
\item
  \begin{itemize}
  \item
  \item
  \item
  \item
  \item
  \end{itemize}
\end{itemize}

``I'm a bit of a magpie,'' Colum McCann said, sheepishly gesturing
around his office at Hunter College.

He has taught creative writing there for 13 years, and in that time has
appointed his space with plenty of mementos: old family photographs, a
self-portrait his daughter drew in neon crayon, a framed poster of what
he said were among his friend
\href{https://www.nytimes3xbfgragh.onion/2009/07/20/books/20mccourt.html}{Frank
McCourt}'s last words to him. (``So, Frank, what will you confess?!!''
McCann asked. ``Pride springing from virtue. But not from virginity!'')
But the keepsake he was looking for during an interview earlier this
month, which he can't believe he's lost, is a napkin.

McCann remembers what it said: ``Harness the power of your grief.'' It
was given to him by two fathers whose lives radically altered his own,
and who became the central characters in his latest novel,
``Apeirogon,'' out next week from Random House.

\includegraphics{https://static01.graylady3jvrrxbe.onion/images/2020/02/17/books/17McCann4/merlin_168935991_ddc372e6-a465-4ebc-a85b-b4718e44cb72-articleLarge.jpg?quality=75\&auto=webp\&disable=upscale}

The book is a fictionalized account of the men, one Israeli, one
Palestinian, whose daughters were killed near Jerusalem 10 years apart.
Rami Elhanan, who is Israeli, lost his teenager Smadar to a suicide
bombing that occurred while she was shopping with friends in 1997. Abir,
the 10-year-old daughter of Bassam Aramin, who is Palestinian, was
\href{https://www.nytimes3xbfgragh.onion/2020/01/09/movies/afterward-review.html}{fatally
shot by an Israeli soldier} in 2007. An apeirogon is a shape with an
infinite but countable number of sides, and over the course of the
book's 1,001 chapters --- a nod to ``One Thousand and One Nights'' ---
McCann delves into the two men's lives, sometimes writing in their
voices.

Aramin and Elhanan met in 2005 through
\href{https://cfpeace.org/}{Combatants for Peace}, an organization that
aims to end the occupation of Palestine, and they have become close
friends. They are now involved in the
\href{https://www.theparentscircle.org/en/pcff-home-page-en/}{Parents
Circle - Families Forum}, an organization for Israelis and Palestinians
who have lost family members in the conflict.

McCann met the men in 2015, when he first visited the region with his
storytelling nonprofit, \href{https://narrative4.com/}{Narrative 4}. He
had been interested in Israel and Palestine's history since talking with
George Mitchell, the former U.S. senator who helped broker peace
negotiations in Ireland and the Middle East, and who appeared in
McCann's continent- and century-spanning 2013 novel
``\href{https://www.nytimes3xbfgragh.onion/2013/06/23/books/review/transatlantic-by-colum-mccann.html}{TransAtlantic}.''

``If anything dilates my nostrils, it's the sense of some sort of
difficulty,'' McCann said. ``I knew I was completely ignorant of what
was going on there, and I thought I'd like to try and explore.''

\emph{{[} This book was one of our most anticipated titles of February.}
\href{https://www.nytimes3xbfgragh.onion/2020/01/29/books/new-february-books.html}{\emph{See
the full list}}\emph{. {]}}

Image

Colum McCann's ``Apeirogon'' is out on Feb. 25.Credit...Sonny
Figueroa/The New York Times

He wound up spending a week in the region. ``I was sort of priding
myself on the fact of being able to deal with everything --- I've seen
Northern Ireland, I've done this before,'' said McCann, who grew up in
Dublin during the Troubles. He listened to Elhanan and Aramin speak in
Beit Jala, a Palestinian town, and sobbed as they told their stories.

Lisa Consiglio, a co-founder of Narrative 4 who has known McCann for
years, was also there. Hearing them, she said, was a ``shock to our
system that had already been shocked very much.'' She had never seen
McCann so visibly moved. ``Subconsciously, I think I knew he had a book
at that moment,'' she said.

McCann left knowing he wanted to write about the conflict. He returned
several times over the next four years to conduct research and
interviews for ``Apeirogon.''

The book has generated pre-publication interest: Steven Spielberg's
company, Amblin Partners,
\href{https://www.hollywoodreporter.com/news/amblin-nabs-israeli-palestinian-friendship-novel-apeirogon-1278792}{bought
the movie rights}, and an Arabic version of the novel is in the works.

Image

Bassam Aramin, left, who is Palestinian, and Rami Elhanan, who is
Israeli, are the central figures in ``Apeirogon.''Credit...Rina
Castelnuovo

``Apeirogon,'' like other books by McCann, interweaves real people with
imagined conversations, scenes and other details of their lives.
McCann's National Book Award-winning novel,
``\href{https://www.nytimes3xbfgragh.onion/2009/08/02/books/review/Mahler-t.html}{Let
the Great World Spin},'' for example, uses the high-wire artist Philippe
Petit as a launching point, while another of his novels,
``\href{https://www.nytimes3xbfgragh.onion/2003/01/19/books/a-story-with-legs.html}{Dancer},''
imagines the life of the ballet dancer
\href{https://www.nytimes3xbfgragh.onion/1993/01/07/arts/rudolf-nureyev-charismatic-dancer-who-gave-fire-to-ballet-s-image-dies-at-54.html}{Rudolf
Nureyev}.

But for McCann, the moral stakes of ``Apeirogon'' felt higher. ``Nureyev
can look after himself, and the memory of Nureyev can look after
itself,'' the writer said. ``With Rami and Bassam, it's so much closer
to the edge, closer to the bone. If you get it wrong \ldots{} they can't
necessarily recover in the same way that a really, truly, public figure
could. Oh, I was terrified.''

In a moment when the publishing industry is grappling with which stories
are told and by whom, it is a striking choice. ``We're in this territory
of the real is the imagined and the imagined is real,'' McCann said of
this project. ``I think people wouldn't have trusted it as much if it
wasn't real.''

Elhanan's and Aramin's stories are well documented; their grief has
entered public consciousness. They've been the subject of films, and
have traveled the world talking about their loss and their conviction
that ``we can't build a state on the ruins of our kids,'' as Aramin said
in an interview. Both have found salvation in the work, a way, Elhanan
said, also in an interview, to ``give meaning to my life and meaning to
my loss.''

Neither man hesitated when McCann approached them about his idea. They
trusted him as an artist, they said, recognizing that the book, even as
a work of fiction, would amplify their message.

``In our first meeting, Colum was crying like a little kid,'' Elhanan,
70, said. ``His ability to show emotion shows that he's a really fine
man with heart --- it's the basic of all basics, the ability to show
emotion and to show compassion.''

Image

``If anything dilates my nostrils, it's the sense of some sort of
difficulty,'' McCann said.Credit...Jillian Freyer for The New York Times

McCann's background also helped, Aramin, 56, said, because many
Palestinians often feel a sense of solidarity with the Irish. ``We have
the same history of conflict.''

``Apeirogon'' can make for excruciating reading. McCann devotes sections
to ``the mushroom effect,'' or how the heads of suicide bombers are
nearly always separated from their torsos, and the damage that rubber
bullets cause. Readers are in the ambulance with Aramin and his
critically injured daughter, waiting two hours and 18 minutes for
traffic to clear to reach another hospital. When Elhanan remembers the
sound of the rollers in the morgue, readers can imagine it, too.

McCann's fictionalization of such wrenching stories is likely to raise
questions among readers, particularly in a story that unfolds in a
charged political conflict where virtually every fact is up for debate.
(One example: Israeli authorities initially suggested that Aramin's
daughter was killed by Palestinians throwing stones;
\href{https://www.theguardian.com/world/2011/sep/26/israel-pay-family-compensation-palestinian-girl}{an
Israeli judge later ruled in civil court} that she had been shot dead by
an Israeli soldier.)

``It's very difficult to talk about your pain or to read about your
pain,'' Aramin said. Both he and Elhanan have been reading it slowly.

Yet neither father asked McCann to change anything.

``I told him from the beginning, `It's your book, not mine,''' Elhanan
said. ``It's accurate, it's emotional, but I don't consider it to be my
story.''

He knows that it may invite criticism, he added, ``but the worst already
happened. I have nothing to fear.''

\emph{Follow New York Times Books on}
\href{https://www.facebookcorewwwi.onion/nytbooks/}{\emph{Facebook}}\emph{,}
\href{https://twitter.com/nytimesbooks}{\emph{Twitter}} \emph{and}
\href{https://www.instagram.com/nytbooks/}{\emph{Instagram}}\emph{, sign
up for}
\href{https://www.nytimes3xbfgragh.onion/newsletters/books-review}{\emph{our
newsletter}} \emph{or}
\href{https://www.nytimes3xbfgragh.onion/interactive/2017/books/books-calendar.html}{\emph{our
literary calendar}}\emph{. And listen to us on the}
\href{https://www.nytimes3xbfgragh.onion/column/book-review-podcast}{\emph{Book
Review podcast}}\emph{.}

Advertisement

\protect\hyperlink{after-bottom}{Continue reading the main story}

\hypertarget{site-index}{%
\subsection{Site Index}\label{site-index}}

\hypertarget{site-information-navigation}{%
\subsection{Site Information
Navigation}\label{site-information-navigation}}

\begin{itemize}
\tightlist
\item
  \href{https://help.nytimes3xbfgragh.onion/hc/en-us/articles/115014792127-Copyright-notice}{©~2020~The
  New York Times Company}
\end{itemize}

\begin{itemize}
\tightlist
\item
  \href{https://www.nytco.com/}{NYTCo}
\item
  \href{https://help.nytimes3xbfgragh.onion/hc/en-us/articles/115015385887-Contact-Us}{Contact
  Us}
\item
  \href{https://www.nytco.com/careers/}{Work with us}
\item
  \href{https://nytmediakit.com/}{Advertise}
\item
  \href{http://www.tbrandstudio.com/}{T Brand Studio}
\item
  \href{https://www.nytimes3xbfgragh.onion/privacy/cookie-policy\#how-do-i-manage-trackers}{Your
  Ad Choices}
\item
  \href{https://www.nytimes3xbfgragh.onion/privacy}{Privacy}
\item
  \href{https://help.nytimes3xbfgragh.onion/hc/en-us/articles/115014893428-Terms-of-service}{Terms
  of Service}
\item
  \href{https://help.nytimes3xbfgragh.onion/hc/en-us/articles/115014893968-Terms-of-sale}{Terms
  of Sale}
\item
  \href{https://spiderbites.nytimes3xbfgragh.onion}{Site Map}
\item
  \href{https://help.nytimes3xbfgragh.onion/hc/en-us}{Help}
\item
  \href{https://www.nytimes3xbfgragh.onion/subscription?campaignId=37WXW}{Subscriptions}
\end{itemize}
