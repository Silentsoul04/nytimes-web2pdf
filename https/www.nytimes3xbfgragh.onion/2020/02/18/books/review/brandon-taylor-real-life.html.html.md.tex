Sections

SEARCH

\protect\hyperlink{site-content}{Skip to
content}\protect\hyperlink{site-index}{Skip to site index}

\href{https://www.nytimes3xbfgragh.onion/section/books/review}{Book
Review}

\href{https://myaccount.nytimes3xbfgragh.onion/auth/login?response_type=cookie\&client_id=vi}{}

\href{https://www.nytimes3xbfgragh.onion/section/todayspaper}{Today's
Paper}

\href{/section/books/review}{Book Review}\textbar{}Jeremy O. Harris:
Brandon Taylor `Subjugates Us With the Deft Hand of a Dom'

\url{https://nyti.ms/2V021QV}

\begin{itemize}
\item
\item
\item
\item
\item
\item
\end{itemize}

Advertisement

\protect\hyperlink{after-top}{Continue reading the main story}

Supported by

\protect\hyperlink{after-sponsor}{Continue reading the main story}

Fiction

\hypertarget{jeremy-o-harris-brandon-taylor-subjugates-us-with-the-deft-hand-of-a-dom}{%
\section{Jeremy O. Harris: Brandon Taylor `Subjugates Us With the Deft
Hand of a
Dom'}\label{jeremy-o-harris-brandon-taylor-subjugates-us-with-the-deft-hand-of-a-dom}}

\includegraphics{https://static01.graylady3jvrrxbe.onion/images/2020/03/29/books/review/29Harris/merlin_167718528_0cc93b28-ec32-42d0-83bb-7a21d73278bc-articleLarge.jpg?quality=75\&auto=webp\&disable=upscale}

Buy Book ▾

\begin{itemize}
\tightlist
\item
  \href{https://www.amazon.com/gp/search?index=books\&tag=NYTBSREV-20\&field-keywords=Real+Life+Brandon+Taylor}{Amazon}
\item
  \href{https://du-gae-books-dot-nyt-du-prd.appspot.com/buy?title=Real+Life\&author=Brandon+Taylor}{Apple
  Books}
\item
  \href{https://www.anrdoezrs.net/click-7990613-11819508?url=https\%3A\%2F\%2Fwww.barnesandnoble.com\%2Fw\%2F\%3Fean\%3D9780525538882}{Barnes
  and Noble}
\item
  \href{https://www.anrdoezrs.net/click-7990613-35140?url=https\%3A\%2F\%2Fwww.booksamillion.com\%2Fp\%2FReal\%2BLife\%2FBrandon\%2BTaylor\%2F9780525538882}{Books-A-Million}
\item
  \href{https://bookshop.org/a/3546/9780525538882}{Bookshop}
\item
  \href{https://www.indiebound.org/book/9780525538882?aff=NYT}{Indiebound}
\end{itemize}

When you purchase an independently reviewed book through our site, we
earn an affiliate commission.

By Jeremy O. Harris

\begin{itemize}
\item
  Feb. 18, 2020
\item
  \begin{itemize}
  \item
  \item
  \item
  \item
  \item
  \item
  \end{itemize}
\end{itemize}

\textbf{REAL LIFE}\\
By Brandon Taylor

Wallace's father died several weeks ago, but more pressingly so did the
collection of nematodes he has been diligently studying all summer in an
unnamed university in an unnamed Midwestern town. Like I was quite
recently, and like the novelist
\href{https://www.nytimes3xbfgragh.onion/2020/02/10/books/brandon-taylor-real-life.html}{Brandon
Taylor was once himself}, Wallace is a black gay grad student from the
South who is mining hope for some better or different life in the
haunted halls of a white academic space. A space that demands his full
attention, lest he affirm the sense that he was never meant to be there
to begin with. In Taylor's stunning debut, ``Real Life,'' quiet
diligence toward one's goals mutates into a spiral that leaves the mind
and body bruised as if survivors of a psychic war zone.

We meet Wallace on a Friday night in his college town, as he's trudging,
begrudgingly, toward an evening hang with friends by the lake. It's
neither the death of his father nor that of his nematodes that inspires
the drag of his feet, but the overwhelming dread that accompanies
socializing with his fellow biochemists, a complicated composite of
friends and foes. Sharp spikes of worry and repression are the heartbeat
of this narrative that follows Wallace down the rabbit hole of one of
the last weekends of their summer holiday in this quiet town. So goes
``Real Life,'' a novel that probes --- painstakingly, with the same
microscopic precision its protagonist uses in the lab --- the ways that
an anxious queer black brain is mutated by the legacies of growing up in
a society (in Wallace's case, rural Alabama) where the body that houses
it is not welcome.

It is a curious novel to describe, for much of the plot involves
excavating the profound from the mundane. As in the modernist novels of
Woolf and Tolstoy cited in passing throughout, the true action of
Taylor's novel exists beneath the surface, buried in subterranean
spaces. The rhythms and rituals of Ph.D. exam prep, vegan dinners and
idle academic gossip become the landscape upon which the real work of
``Real Life'' is built. Taylor proves himself to be a keen observer of
the psychology of not just trauma, but its repercussions: how private
suffering can ricochet from one person to injure those caught in his
path. The aftershocks of these childhood horrors (``He almost jumps at
the fright of it, the wholeness of the memory. His body remembers. His
traitorous body'') set Wallace up for further humiliation within new
American contexts that are still not designed for our protagonist. For
example, the jarring moment after a female colleague accuses him of
misogyny in terms laced with racism and homophobia, as sharp as a
stiletto: ``The most unfair part of it, Wallace thinks, is that when you
tell white people that something is racist, they hold it up to the light
and try to discern if you are telling the truth.'' He abandons the
confrontation before it can escalate further, the hallway's motion
sensors failing to detect his presence. His body nothing more than an
apparition, imperceptible in this space.

\emph{{[} Read an excerpt from}
\href{https://www.nytimes3xbfgragh.onion/2020/02/18/books/review/real-life-by-brandon-taylor-an-excerpt.html}{\emph{``Real
Life.''}} \emph{{]}}

Over the course of a single weekend, beginning that Friday night and
ending on the Monday morning of orientation, ``Real Life'' moves us
through the stark internal realities of being young, gifted and black in
a town where your peers are abundantly yt, whyte and WHITE. I was struck
by the whiteness of Wallace's surroundings, a fact of many spaces of
American higher learning, and one rarely articulated in literature by
writers of any race. Yet Taylor allows whiteness to move consistently on
the periphery of Wallace's gaze, in tandem with the muted hum of desire:
its volume rising and falling as his field of vision encounters white
hands adjusting tight shorts revealing flashes of untanned, untouched
flesh. Our curiously quiet and cautious protagonist holds a well of
desire that could fill the many baths he takes and then some with the
viscous liquid of his desire: a desire to escape his past, a desire to
find refuge in some future, a desire for a man. A desire that he has
constantly sublimated and denied in favor of silent yearning, until that
Friday night. Spirals always begin and end with longing, and this is
true throughout ``Real Life.''

The novel's at times stunted and awkward dialogue (``Trying. What good
is trying?'' ``You have to try. You always have to try.'' ``What if
we're there, but I don't know we're there?'' ``You'd know. You just
would'') can clash with its often tight, beatific prose. Yet much like
the tropes of queer literary lust that populate the final half of the
novel (the Sulking Straight Top Who Sodomizes Without Fear of Feces),
even this halting dialogue never feels wholly out of step with Wallace's
psyche, which itself functions in discordant, sometimes off-putting,
thrillingly contradictory ways. Add to all this Taylor's deeply rooted
understandings of the rarefied worlds of both provincial grad school
life and biochemistry in particular, which should inspire envy in every
writer striving for specificity. There is a delicacy in the details of
working in a lab full of microbes and pipettes that dances across the
pages like the feet of a Cunningham dancer: pure, precise poetry.

An accumulation of aggressions --- micro, macro, mental, physical ---
are piled atop both Wallace and the reader as we move through the novel,
as Taylor subjugates us with the deft hand of a dom to the airless
vertigo that rests at the heart of the spiral. Wallace's memories enter
the prose like whispered secrets, drowned out like the violence of his
childhood home that, ``when the fan in his room was going, he couldn't
hear.'' Because for Wallace that's what memories are: secrets. The
simple truth of ``Real Life'' is that Wallace, like myself and many
others who've wandered dark, white halls in search of a future, has made
himself invisible by shedding the skin of his past, and adopting a new
skin unadorned with the blemishes of history. But the only way to once
again be visible, to the world or even to himself, is to adopt new
bruises in place of the old, for to get where he is going, he must
remember where he's been.

Advertisement

\protect\hyperlink{after-bottom}{Continue reading the main story}

\hypertarget{site-index}{%
\subsection{Site Index}\label{site-index}}

\hypertarget{site-information-navigation}{%
\subsection{Site Information
Navigation}\label{site-information-navigation}}

\begin{itemize}
\tightlist
\item
  \href{https://help.nytimes3xbfgragh.onion/hc/en-us/articles/115014792127-Copyright-notice}{©~2020~The
  New York Times Company}
\end{itemize}

\begin{itemize}
\tightlist
\item
  \href{https://www.nytco.com/}{NYTCo}
\item
  \href{https://help.nytimes3xbfgragh.onion/hc/en-us/articles/115015385887-Contact-Us}{Contact
  Us}
\item
  \href{https://www.nytco.com/careers/}{Work with us}
\item
  \href{https://nytmediakit.com/}{Advertise}
\item
  \href{http://www.tbrandstudio.com/}{T Brand Studio}
\item
  \href{https://www.nytimes3xbfgragh.onion/privacy/cookie-policy\#how-do-i-manage-trackers}{Your
  Ad Choices}
\item
  \href{https://www.nytimes3xbfgragh.onion/privacy}{Privacy}
\item
  \href{https://help.nytimes3xbfgragh.onion/hc/en-us/articles/115014893428-Terms-of-service}{Terms
  of Service}
\item
  \href{https://help.nytimes3xbfgragh.onion/hc/en-us/articles/115014893968-Terms-of-sale}{Terms
  of Sale}
\item
  \href{https://spiderbites.nytimes3xbfgragh.onion}{Site Map}
\item
  \href{https://help.nytimes3xbfgragh.onion/hc/en-us}{Help}
\item
  \href{https://www.nytimes3xbfgragh.onion/subscription?campaignId=37WXW}{Subscriptions}
\end{itemize}
