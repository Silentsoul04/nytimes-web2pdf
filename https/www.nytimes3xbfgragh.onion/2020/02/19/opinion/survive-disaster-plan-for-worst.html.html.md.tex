Sections

SEARCH

\protect\hyperlink{site-content}{Skip to
content}\protect\hyperlink{site-index}{Skip to site index}

\href{https://myaccount.nytimes3xbfgragh.onion/auth/login?response_type=cookie\&client_id=vi}{}

\href{https://www.nytimes3xbfgragh.onion/section/todayspaper}{Today's
Paper}

\href{/section/opinion}{Opinion}\textbar{}To Survive Disaster, Plan for
the Worst

\url{https://nyti.ms/39OGCOG}

\begin{itemize}
\item
\item
\item
\item
\item
\item
\end{itemize}

Advertisement

\protect\hyperlink{after-top}{Continue reading the main story}

\href{/section/opinion}{Opinion}

Supported by

\protect\hyperlink{after-sponsor}{Continue reading the main story}

Fixes

\hypertarget{to-survive-disaster-plan-for-the-worst}{%
\section{To Survive Disaster, Plan for the
Worst}\label{to-survive-disaster-plan-for-the-worst}}

Getting ready for a disaster is still a tiny part of the world's
response to the likelihood of one. But some governments and officials
are starting to plan well in advance.

\includegraphics{https://static01.graylady3jvrrxbe.onion/images/2019/02/13/opinion/tina-rosenberg/tina-rosenberg-thumbLarge-v2.png}

By Tina Rosenberg

Ms. Rosenberg is a co-founder of the
\href{http://solutionsjournalism.org}{Solutions Journalism Networ}k,
which supports rigorous reporting about responses to social problems.

\begin{itemize}
\item
  Feb. 19, 2020
\item
  \begin{itemize}
  \item
  \item
  \item
  \item
  \item
  \item
  \end{itemize}
\end{itemize}

\includegraphics{https://static01.graylady3jvrrxbe.onion/images/2020/02/18/opinion/00Fixes-Rosenberg/merlin_158055966_da1755f6-0435-45f9-b4e4-9245eebb9e94-articleLarge.jpg?quality=75\&auto=webp\&disable=upscale}

Disaster relief works like this: There is a flood, a drought, an
earthquake, a famine, an exodus of refugees. Reporters swarm in,
broadcasting images of suffering. Humanitarian workers on the ground
analyze who needs what relief and draw up plans. The government asks for
help. The United Nations coordinates international pledges. Relief comes
in --- money, bags of grain, medical supplies.

But by that point, weeks or months have gone by.

Rarely is there preplanning, pre-fundraising, or pre-agreement on a
plan. ``This is medieval,'' said Stefan Dercon, a professor of economic
policy at Oxford and a former chief economist of Britain's bilateral aid
agency, the Department for International Development. He and Daniel
Clarke, head of the London-based
\href{https://www.disasterprotection.org/}{Center for Disaster
Protection}, wrote the book ``Dull Disasters? How Planning Ahead Will
Make a Difference.''

``It is as if financial instruments such as insurance do not exist,''
they wrote. ``This is begging-bowl financing at its worst.''

But here's what can happen instead --- what, in fact, did happen in the
Kurigram district of northwest Bangladesh in July. With colossal rains
predicted, the United Nations World Food Program and the Bangladesh
government
\href{https://www.wfp.org/news/wfp-provides-innovative-assistance-flood-affected-people-north-western-bangladesh}{identified
about 5,000 particularly vulnerable families.} Three days before the
flood hit, they used mobile phone banking to send each family the
equivalent of \$53. With that money, the families secured their houses
and belongings --- for example, buying materials to lift their furniture
off the ground. And they could pay the costs of taking their livestock
and fleeing.

Instead of getting relief after they were wiped out by the flood, the
residents were able to avoid much of the loss --- for \$10 per person.

The accomplishment in Bangladesh is one of a handful of examples
worldwide of anticipating disaster.

But it doesn't have to be the rare exception. If disasters take us by
surprise, it's because we weren't looking. With satellite data and
mathematical modeling, we can now know about a flood or drought days or
even weeks in advance. ``We've improved so much in getting the precise
likelihood of this particular area being flooded, and the number of
people affected,'' Dr. Dercon said of the Bangladesh case. ``We probably
couldn't have done this 10 years ago.''

We can't predict the first case of a new outbreak of Ebola, but we can
know where that deadly disease recurs and use that first case to predict
later ones. Using satellite data, scientists can anticipate cholera
outbreaks \href{https://www.bbc.com/news/health-45259922}{days, even
weeks, in advance}. When violent conflict breaks out or terrible drought
sets in, we can plot the mass movement of refugees.

An early response can prevent suffering. With famine expected in Somalia
in 2017, for example, U.N. and other aid agencies sent
\href{https://reliefweb.int/report/somalia/how-aid-cash-not-goods-averted-famine-somalia}{600,000
families vouchers} by text message redeemable in local markets. (All
hail mobile phone banking!) The vouchers fed families and the local
economy, and famine was averted.

Anticipating disasters can also help when they continue. In Kenya, rural
herders can buy subsidized insurance that pays them automatically by
mobile phone when satellites determine that
the\href{https://www.nytimes3xbfgragh.onion/2016/11/08/opinion/up-in-the-sky-help-to-keep-africans-from-starving.html}{available
forage in their area is too scarce} to support livestock. A payout in
time to buy food for a cow is vastly preferable to a payout after the
cow dies. It's health insurance, not life insurance.

We can even prepare for unexpected disasters. Mexico can't predict a
specific earthquake. But it knows that the country is an enormous
earthquake zone. So in the late 1990s, the government established the
Fund for Natural Disasters, or Fonden. It allows Mexico to make action
plans and money available in advance of any quake, as well as to start
relief and reconstruction immediately when one occurs. In 2006, Mexico
issued the world's first government catastrophe bond, a form of
insurance that pays out when an earthquake strikes.

Caribbean countries can buy policies from the Caribbean Catastrophe Risk
Insurance Facility to insure against cyclones, earthquakes and very
heavy rains. In Africa, the African Risk Capacity pays countries when
rain is scarce.

Mark Lowcock, under-secretary-general of the United Nations for
humanitarian affairs --- the U.N.'s emergency relief coordinator --- is
trying to push the relief system toward
\href{https://reliefweb.int/report/world/mark-lowcock-under-secretary-general-humanitarian-affairs-and-emergency-relief}{anticipating
disasters}. He said that one sign of progress is the growth of the
U.N.'s \href{https://cerf.un.org/}{Central Emergency Response Fund},
which he administers. Its funding has been
\href{https://cerf.un.org/our-donors/contributions}{roughly steady since
2006} at about \$500 million a year, but in 2019 donors contributed
\$828 million.

Other groups are also trying anticipatory action. Red Cross and Red
Crescent societies in several countries have piloted
\href{https://www.forecast-based-financing.org/}{forecast-based
financing}, providing aid in advance of floods, mostly, but also heat
and cold waves.

But getting ready for a disaster is still a tiny part of the world's
response to the likelihood of one. ``We've got quite a lot of very
persuasive examples,'' Mr. Lowcock said. ``It's nice to have 25,000
people in Bangladesh, but most situations have caseloads in the
millions. There is much more scope to do this than is happening at the
moment.''

In general, we don't plan enough for disasters largely because we've
assumed our political processes demand those visible victims. Political
systems are notoriously bad at long-term thinking. It's hard to raise
money in the abstract. Politicians know you don't get credit for
prevention.

``We've learned in other areas of development to be more sophisticated
in our use of financial instruments,'' Dr. Dercon said. ``Somehow in the
humanitarian space we just never embraced this. The humanitarian sector
has lots of really good people, but it's built around the emotion of
helping people in need.''

How we react to disasters matters more than ever. This year, 168 million
people will need disaster assistance, a record high. Mr. Lowcock said
that in two years, that could rise to 200 million people. He said the
main reasons for the increase are droughts and floods related to climate
change, large outbreaks of infectious disease (often related to climate
as well) and protracted violent conflicts --- which are increasingly
killing children, he said.

New research shows that how we respond --- or not --- to a disaster can
follow people for the rest of their lives. A 2017
\href{https://www.worldbank.org/en/publication/wdr2017}{World Bank
report found} that prosperous countries are the ones lucky enough to
avoid crises that set them back. Just as a fall can permanently damage
the health of an older person, conflict, drought or epidemic can
permanently make a country poorer.

Drought, particularly, is crippling, because it creates lasting
malnutrition, leading to permanent cognitive and physical damage.
\href{http://documents.worldbank.org/curated/en/796341557483493173/The-Chronology-of-a-Disaster-A-Review-and-Assessment-of-the-Value-of-Acting-Early-on-Household-Welfare}{Researchers
found} that without relief, a drought causes a 4 percent drop in the
income of affected people --- for the long term. And a speedy response
is crucial. ``A response time that is one month quicker has a benefit of
0.8 percent of income per capita in the long run,'' the study says.

There is wide agreement on the value of having money set aside, a plan
for what to do and agreement on what triggers action. But what's needed
to get that done isn't always present. Mr. Lowcock gave the example of
drought in Somalia: It's not enough to predict hunger and famine. ``We
need to know at the village level who are the most vulnerable,'' he
said. ``We're trying to build models to answer that question for us.''

U.N. agencies are getting creative with the challenge of knowing when to
help. Rebeca Moreno Jiménez, the U.N. refugee agency's first data
scientist, recently traveled to Ethiopia to interview Somali refugees,
hoping to identify something measurable that can signal relief is
needed. What she found was the price of goats. Refugees told her that
before people flee, they sell their goats, which are too fragile to make
the trip. So the crash in goat prices that would accompany a mass
sell-off means people are getting ready to move. We know they will end
up in Ethiopia several days later.

This work is still experimental. ``We are bothering our colleagues in
the Food and Agriculture Organization,'' she said. ``Every month, we're
asking for goat prices in Somalia.'' Her program has now established an
interagency dashboard where everyone can track the prices F.A.O. posts.

``Cynical people think politicians are prompted to action only when they
see the starving kid in the street,'' Mr. Lowcock said. ``But when I
talk to politicians and confront them with the fact that we can
anticipate problems better than that, they get it. That's why my fund is
bigger this year than last and we are able to fund more experiments. No
one wants to see a starving kid on the street and think, `If we'd done
something earlier, we might have been able to stop that.'''

Tina Rosenberg won a Pulitzer Prize for her book
``\href{http://www.randomhouse.com/catalog/display.pperl?isbn=9780679744993}{The
Haunted Land:} Facing Europe's Ghosts After Communism.'' She is a former
editorial writer for The Times and the author, most recently, of
``\href{http://books.wwnorton.com/books/Join-the-Club}{Join the Club:}
How Peer Pressure Can Transform the World'' and the World War II spy
story e-book
\href{https://www.goodreads.com/book/show/16124470-d-for-deception}{``D
for Deception.''}

\emph{To receive email alerts for Fixes columns, sign up}
\href{http://eepurl.com/ABIxL}{\emph{here.}}

\emph{The Times is committed to publishing}
\href{https://www.nytimes3xbfgragh.onion/2019/01/31/opinion/letters/letters-to-editor-new-york-times-women.html}{\emph{a
diversity of letters}} \emph{to the editor. We'd like to hear what you
think about this or any of our articles. Here are some}
\href{https://help.nytimes3xbfgragh.onion/hc/en-us/articles/115014925288-How-to-submit-a-letter-to-the-editor}{\emph{tips}}\emph{.
And here's our email:}
\href{mailto:letters@NYTimes.com}{\emph{letters@NYTimes.com}}\emph{.}

\emph{Follow The New York Times Opinion section on}
\href{https://www.facebookcorewwwi.onion/nytopinion}{\emph{Facebook}}\emph{,}
\href{http://twitter.com/NYTOpinion}{\emph{Twitter (@NYTopinion)}}
\emph{and}
\href{https://www.instagram.com/nytopinion/}{\emph{Instagram}}\emph{.}

Advertisement

\protect\hyperlink{after-bottom}{Continue reading the main story}

\hypertarget{site-index}{%
\subsection{Site Index}\label{site-index}}

\hypertarget{site-information-navigation}{%
\subsection{Site Information
Navigation}\label{site-information-navigation}}

\begin{itemize}
\tightlist
\item
  \href{https://help.nytimes3xbfgragh.onion/hc/en-us/articles/115014792127-Copyright-notice}{©~2020~The
  New York Times Company}
\end{itemize}

\begin{itemize}
\tightlist
\item
  \href{https://www.nytco.com/}{NYTCo}
\item
  \href{https://help.nytimes3xbfgragh.onion/hc/en-us/articles/115015385887-Contact-Us}{Contact
  Us}
\item
  \href{https://www.nytco.com/careers/}{Work with us}
\item
  \href{https://nytmediakit.com/}{Advertise}
\item
  \href{http://www.tbrandstudio.com/}{T Brand Studio}
\item
  \href{https://www.nytimes3xbfgragh.onion/privacy/cookie-policy\#how-do-i-manage-trackers}{Your
  Ad Choices}
\item
  \href{https://www.nytimes3xbfgragh.onion/privacy}{Privacy}
\item
  \href{https://help.nytimes3xbfgragh.onion/hc/en-us/articles/115014893428-Terms-of-service}{Terms
  of Service}
\item
  \href{https://help.nytimes3xbfgragh.onion/hc/en-us/articles/115014893968-Terms-of-sale}{Terms
  of Sale}
\item
  \href{https://spiderbites.nytimes3xbfgragh.onion}{Site Map}
\item
  \href{https://help.nytimes3xbfgragh.onion/hc/en-us}{Help}
\item
  \href{https://www.nytimes3xbfgragh.onion/subscription?campaignId=37WXW}{Subscriptions}
\end{itemize}
