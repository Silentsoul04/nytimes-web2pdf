Sections

SEARCH

\protect\hyperlink{site-content}{Skip to
content}\protect\hyperlink{site-index}{Skip to site index}

\href{https://www.nytimes3xbfgragh.onion/section/politics}{Politics}

\href{https://myaccount.nytimes3xbfgragh.onion/auth/login?response_type=cookie\&client_id=vi}{}

\href{https://www.nytimes3xbfgragh.onion/section/todayspaper}{Today's
Paper}

\href{/section/politics}{Politics}\textbar{}Trump Renews Attack on
Justice System, Again Disregarding Barr's Pleas

\url{https://nyti.ms/39RLl2d}

\begin{itemize}
\item
\item
\item
\item
\item
\item
\end{itemize}

Advertisement

\protect\hyperlink{after-top}{Continue reading the main story}

Supported by

\protect\hyperlink{after-sponsor}{Continue reading the main story}

\hypertarget{trump-renews-attack-on-justice-system-again-disregarding-barrs-pleas}{%
\section{Trump Renews Attack on Justice System, Again Disregarding
Barr's
Pleas}\label{trump-renews-attack-on-justice-system-again-disregarding-barrs-pleas}}

In lashing out at the prosecution of Roger Stone and ``dirty cops,'' the
president confronted the attorney general with a choice about how to
respond.

\includegraphics{https://static01.graylady3jvrrxbe.onion/images/2020/02/20/us/politics/20dc-justice-sub/20dc-trump-videoSixteenByNine3000.jpg}

\href{https://www.nytimes3xbfgragh.onion/by/peter-baker}{\includegraphics{https://static01.graylady3jvrrxbe.onion/images/2018/06/13/multimedia/peter-baker/peter-baker-thumbLarge-v2.png}}

By \href{https://www.nytimes3xbfgragh.onion/by/peter-baker}{Peter Baker}

\begin{itemize}
\item
  Feb. 20, 2020
\item
  \begin{itemize}
  \item
  \item
  \item
  \item
  \item
  \item
  \end{itemize}
\end{itemize}

WASHINGTON --- President Trump once again berated the ``dirty cops'' of
the law enforcement establishment on Thursday, accusing the Justice
Department of going after his friends but not his enemies in an outburst
that flouted Attorney General William P. Barr's pleas to stop publicly
intervening in prosecutions where he had a personal interest.

Speaking out hours after
\href{https://www.nytimes3xbfgragh.onion/2020/02/20/us/roger-stone-40-months-sentencing-verdict.html?action=click\&module=Top\%20Stories\&pgtype=Homepage}{his
friend Roger J. Stone Jr. was sentenced} to more than three years in
prison for lying to protect the president, Mr. Trump belittled the case
and hinted broadly that he would use his clemency power to spare Mr.
Stone if a judge did not agree to a retrial sought by defense lawyers.

In essentially dangling a pardon or a commutation for a friend, Mr.
Trump confronted Mr. Barr with a choice about how to respond after he
\href{https://www.nytimes3xbfgragh.onion/2020/02/13/us/politics/william-barr-trump.html}{declared
last week} that the president's attacks on the criminal justice system
were making his job ``impossible.''

``A lot of bad things are happening, and we're cleaning it out,'' Mr.
Trump said of law enforcement at a Las Vegas event for former convicts
re-entering society. ``We're cleaning the swamp. We're draining the
swamp. I just never knew how deep the swamp was.''

He added: ``We had a lot of dirty cops. F.B.I. is phenomenal. I love the
people in the F.B.I. But the people at the top were dirty cops.''

Mr. Barr last week reversed the Justice Department's original request,
in line with federal guidelines,
\href{https://www.nytimes3xbfgragh.onion/2020/02/10/us/roger-stone-prison-sentence.html}{for
a term of seven to nine years} for Mr. Stone, leading
\href{https://www.nytimes3xbfgragh.onion/2020/02/11/us/politics/roger-stone-sentencing.html}{four
career prosecutors to quit the case} and one to resign from the
department altogether.

But even as he agreed with Mr. Trump that the sentencing recommendation
was excessive, he
\href{https://abcnews.go.com/Politics/barr-blasts-trumps-tweets-stone-case-impossible-job/story?id=68963276\&cid=social_twitter_wnt}{went
on ABC News} to publicly ask the president to stop commenting because
``I cannot do my job here at the department with a constant background
commentary that undercuts me.''

\href{https://www.nytimes3xbfgragh.onion/2020/02/10/us/roger-stone-prison-sentence.html}{Mr.
Trump later agreed} that he was making Mr. Barr's job harder but
indicated he would not stay quiet.

While Mr. Trump again erupted on Thursday, some in Mr. Barr's camp took
solace in the fact that the president did not directly attack the judge
in the case,
\href{https://www.nytimes3xbfgragh.onion/2020/02/20/us/politics/amy-berman-jackson-roger-stone.html}{Amy
Berman Jackson}, the prosecutors or the sentence itself and that he said
he admired the F.B.I. rank and file, even though he continued to target
its current and former leadership.

Mr. Barr has come to recognize that he may never be able to keep Mr.
Trump quiet altogether and so in parsing the president's latest
comments, the attorney general and his team chose to see them as
progress and an opportunity to work out their differences without
further public exchange.

Mr. Barr, who has suggested to associates that he may have to resign if
the president keeps tweeting about individual prosecutions, stayed
silent after Mr. Trump's remarks, and the department declined to
comment.

Over the last week, some of Mr. Trump's closest Republican allies in
Congress, including Senator Lindsey Graham of South Carolina, and some
of the president's favorite Fox News commentators have been trying to
smooth over the rift by
\href{https://twitter.com/LindseyGrahamSC/status/1228072754685136896}{vouching
for Mr. Barr} and urging the president to let the legal process play
out, according to people familiar with the matter.

After the judge sentenced Mr. Stone on Thursday, Mr. Graham
\href{https://twitter.com/LindseyGrahamSC/status/1230555882536128512?s=20}{noted
on Twitter} that the term was ``on the lower end'' and emphasized that a
president could always grant clemency. He urged that there be no undue
interference in legal cases --- without quite saying whom he was
addressing.

``It is important to give both the government and the accused a fair
trial free from influence,''
\href{https://twitter.com/LindseyGrahamSC/status/1230555883660288000}{Mr.
Graham wrote}. ``I believe this has been accomplished in Mr. Stone's
case. Like all Americans, Mr. Stone can appeal the fairness of his trial
and the verdict rendered. Under our system of justice President Trump
has all the legal authority in the world to review this case, in terms
of commuting the sentence or pardoning Mr. Stone for the underlying
offense.''

Mr. Trump's critics maintained that extending clemency to Mr. Stone
would amount to self-dealing for a president who stood to benefit from
the convicted adviser's lies about connections to WikiLeaks, which
disseminated Democratic emails stolen by Russian agents to damage
Hillary Clinton during the 2016 campaign.

``Roger Stone was found guilty of lying to Congress and threatening a
witness,''
\href{https://twitter.com/RepAdamSchiff/status/1230550196590796800}{said
Representative Adam B. Schiff}, Democrat of California and the chairman
of the House Intelligence Committee, the panel that Mr. Stone was
convicted of obstructing. ``He did it to cover up for Trump. His
sentence is justified. It should go without saying, but to pardon Stone
when his crimes were committed to protect Trump would be a breathtaking
act of corruption.''

In his speech on Thursday, Mr. Trump dismissed the significance of Mr.
Stone's crimes. He complained that the Justice Department prosecuted his
friend for lying and obstructing a congressional inquiry, but did not
charge his enemies like Mrs. Clinton, the former F.B.I. director James
B. Comey, Mr. Comey's onetime deputy Andrew G. McCabe or the former
F.B.I. officials Lisa Page and Peter Strzok.

``What happened to him is unbelievable,'' Mr. Trump said of Mr. Stone.
``They say he lied. But other people lied, too. Just to mention, Comey
lied. McCabe lied. Lisa Page lied. Her lover, Strzok, Peter Strzok,
lied. You don't know who these people are? Just trust me, they all
lied.''

He went on to revive the email case involving Mrs. Clinton. ``Hillary
Clinton leaked more classified documents than any human being, I
believe, in the history of the United States,'' he said. But, he added,
``nothing happened to her.''

In repeating his attacks on his favorite targets, Mr. Trump distorted or
misstated the facts. Mrs. Clinton was not accused of leaking classified
documents, much less the most in history. She was investigated for using
a private computer server that was not as secure as a government server
to send emails, some of which were later found to include classified
information.

Mr. Comey, then the F.B.I. director,
\href{https://www.nytimes3xbfgragh.onion/2016/07/06/us/politics/hillary-clinton-fbi-email-comey.html}{faulted
her for carelessness} but said ``no reasonable prosecutor would bring
such a case.'' His surprise announcement that he was briefly
\href{https://www.nytimes3xbfgragh.onion/2016/10/29/us/politics/fbi-hillary-clinton-email.html}{reopening
the investigation} days before the 2016 election has been blamed by many
Democrats for contributing to Mrs. Clinton's defeat.

By contrast, Mr. Trump mocked the case against Mr. Stone, including the
witness intimidation charge. ``It's not like the tampering that I see on
television when you watch a movie,'' the president said. ``That's called
tampering --- with guns to people's heads and lots of other things.''

He repeated his assertion that the jury forewoman in Mr. Stone's case
was ``totally tainted'' and an ``anti-Trump activist,'' which he said
should compel the judge to order a new trial. The Justice Department
opposes a new trial, a position approved by Mr. Barr.

Mr. Trump left the strong impression, however, that he would use his
clemency power if the judge did not go along with the defense motion,
saying that he would ``love to see Roger exonerated.''

``I'm going to watch the process. I'm going to watch it very closely,''
Mr. Trump added. ``And at some point, I'll make a determination. But
Roger Stone and everybody has to be treated fairly, and this has not
been a fair process. OK?''

Charlie Savage and Katie Benner contributed reporting.

Advertisement

\protect\hyperlink{after-bottom}{Continue reading the main story}

\hypertarget{site-index}{%
\subsection{Site Index}\label{site-index}}

\hypertarget{site-information-navigation}{%
\subsection{Site Information
Navigation}\label{site-information-navigation}}

\begin{itemize}
\tightlist
\item
  \href{https://help.nytimes3xbfgragh.onion/hc/en-us/articles/115014792127-Copyright-notice}{©~2020~The
  New York Times Company}
\end{itemize}

\begin{itemize}
\tightlist
\item
  \href{https://www.nytco.com/}{NYTCo}
\item
  \href{https://help.nytimes3xbfgragh.onion/hc/en-us/articles/115015385887-Contact-Us}{Contact
  Us}
\item
  \href{https://www.nytco.com/careers/}{Work with us}
\item
  \href{https://nytmediakit.com/}{Advertise}
\item
  \href{http://www.tbrandstudio.com/}{T Brand Studio}
\item
  \href{https://www.nytimes3xbfgragh.onion/privacy/cookie-policy\#how-do-i-manage-trackers}{Your
  Ad Choices}
\item
  \href{https://www.nytimes3xbfgragh.onion/privacy}{Privacy}
\item
  \href{https://help.nytimes3xbfgragh.onion/hc/en-us/articles/115014893428-Terms-of-service}{Terms
  of Service}
\item
  \href{https://help.nytimes3xbfgragh.onion/hc/en-us/articles/115014893968-Terms-of-sale}{Terms
  of Sale}
\item
  \href{https://spiderbites.nytimes3xbfgragh.onion}{Site Map}
\item
  \href{https://help.nytimes3xbfgragh.onion/hc/en-us}{Help}
\item
  \href{https://www.nytimes3xbfgragh.onion/subscription?campaignId=37WXW}{Subscriptions}
\end{itemize}
