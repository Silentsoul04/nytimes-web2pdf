Sections

SEARCH

\protect\hyperlink{site-content}{Skip to
content}\protect\hyperlink{site-index}{Skip to site index}

\href{https://www.nytimes3xbfgragh.onion/section/opinion/sunday}{Sunday
Review}

\href{https://myaccount.nytimes3xbfgragh.onion/auth/login?response_type=cookie\&client_id=vi}{}

\href{https://www.nytimes3xbfgragh.onion/section/todayspaper}{Today's
Paper}

\href{/section/opinion/sunday}{Sunday Review}\textbar{}Even When I'm
Psychotic, I'm Still Me

\url{https://nyti.ms/2T57AuW}

\begin{itemize}
\item
\item
\item
\item
\item
\end{itemize}

Advertisement

\protect\hyperlink{after-top}{Continue reading the main story}

\href{/section/opinion}{Opinion}

Supported by

\protect\hyperlink{after-sponsor}{Continue reading the main story}

disability

\hypertarget{even-when-im-psychotic-im-still-me}{%
\section{Even When I'm Psychotic, I'm Still
Me}\label{even-when-im-psychotic-im-still-me}}

When my bipolar disorder caused a break with reality, most everyone in
my life disappeared.

By Susanne Antonetta

Ms. Antonetta is an author and editor.

\begin{itemize}
\item
  Feb. 21, 2020
\item
  \begin{itemize}
  \item
  \item
  \item
  \item
  \item
  \end{itemize}
\end{itemize}

\includegraphics{https://static01.graylady3jvrrxbe.onion/images/2020/02/23/opinion/23disability-psychosis/23disability-psychosis-mediumSquareAt3X.jpg}

Last September, I believed my brain was on fire. Not in some
metaphorical way. It was, as far as I was concerned, on fire.

I am bipolar and I was hallucinating. My hallucinations can be sensory,
like the brain burn, but many are auditory --- I know hallucinations are
coming when I hear birds speak. I can tell you what the birds say, but
what matters is how intensely personal it is, being shouted at by a
fierce small crowd: \emph{persist persist persist} from one, \emph{six
degrees yes yes yes} from another.

I couldn't sleep in all the chatter. Then I heard whispering everywhere,
semi trucks coming to a halt right under my bedroom window. A small part
of me sensed all this was not really happening, but most of me thought
it was.

There's another hallucinatory change that's harder to describe, one that
comes every time, mild episode or intense. The world feels malleable,
like felt, or soft paper. Walls rock and steady themselves. What's
around me becomes alive, air itself humming and moving. As with the
birds, these changes feel intensely personal --- everything around me
shifts as I watch.

During the six months leading up to this brain-fire time, I'd been
having milder hallucinations, on and off. I took a medication that
controlled my psychotic symptoms until my cholesterol skyrocketed and
kept going up. The drugs used to treat people like me --- atypical
antipsychotics like Zyprexa and the one I take, Seroquel --- have
metabolic side effects. These include soaring cholesterol and
triglycerides, as well as diabetes. There may be no way out of these
side effects except dropping the medication, going, as I did, from one
that works to one that doesn't.

Doctors, and the occasional friend, kept telling me something meant to
be cheering: ``This is just a disease, the same as a broken bone or a
bout of pneumonia.'' As though my antipsychotic could just as easily be
penicillin. I've heard this statement in one form or another for several
decades, since my diagnosis at age 29. I don't accept this mechanistic
view of the brain, which suggests that if you pump in drugs (at levels
often determined by drug company-funded research), the cogs will start
working smoothly again. This model dismisses patients' individual
experience of medications, which vary wildly. It overwrites their
personhood, their lived experience, the degree to which they wish to be
medicated. Often at the heart of the medical model is the view that
patients are the last to understand their own needs and psyches.

There's another difference between psychosis and physical ailments: In
the case of psychosis, no one is likely to stop by with a casserole. No
one honestly thinks of you as having a broken bone, and no one acts as
if you do. As my hallucinations went on, I had to go back on Seroquel
and take medical leave from teaching. For several months I remained
shaky, medication-fogged, sometimes crying for no reason, sometimes
hearing birds talk to me again.

During this time in my life many people in my life disappeared. They
stayed entirely out of touch or asked my husband if they should call or
visit, and though he said yes, they didn't. Mostly they were just gone.
My husband, Bruce, supported me, dealt with my doctors, spent as much
time with me as he could, all while still working. He needed help, too.

Even doctors showed no interest, at least not in my symptoms. They heard
the word hallucination and they prescribed. I seemed to transform into
some scary beast gnashing her teeth in her pen.

I don't understand what so-called normalcy is, if it really exists, and
I don't care to try. What I wanted was to function again. This choice
came with consequences. By the time these hallucinations ended in early
October, I was taking so much medication that I stumbled as I walked,
had hand tremors, and my word retrieval skills faltered at words like
``window.'' All these side effects are typical of the drugs I was
taking.

As this hallucinatory time wore on, ebbed, then mostly ended, I made a
decision that, for me, was radical. I told people who wondered what had
become of me just what had --- that I was bipolar, had psychotic
episodes and had been hallucinating. I asked people who vanished why
they did. In response I heard comments like ``I thought you would want
privacy,'' as if psychosis equaled parading around naked. As if it
necessitated shame.

Sometimes, from those who vanished, I got a version of ``I was waiting
for you to come back to yourself.'' I had a hard time explaining that
I've always been myself, maybe at this time even more myself than when I
move through my day without too much auditory interruption. Lacking many
reference points with the commonly held world, hallucinations are
necessarily intensely you.

I don't want to believe that my brain is on fire, and I take medication
so it doesn't happen. It's frightening. I don't want trucks rumbling at
me as I sleep. I want to live in a world that's quieter, that isn't
always demanding my attention. None of that refutes the fact that I am
still me in psychosis, and I know in some way my daily perceptions
always hold those moments, those birds that speak. I can't divide myself
into the ableist Sick Me and Healthy Me, or Sick Me and Real Me. Someone
else can take up those metaphors. I don't want them.

I also believe that hallucinations hold truth, though a hard truth to
stay with for very long. I live the world as alive with a crackling
energy, open and porous, full of meaning. I don't just have delusions
when I'm in that state; I re-experience. And the re-experiencing is not
totally false. Air does move around us and hold us, walls are temporary,
birds give voice. We just normally tune these realities out.

Dorothea Buck was a German woman sterilized by the Nazis because she was
given a diagnosis of schizophrenia. She became a lifelong psychiatric
activist who argued for the need to accept psychosis as meaningful.
Rather than a sick state cut out of the brain by medication, she wrote,
it's a personal experience meant to be understood, even and perhaps
especially when treated.

My hallucinations have ended and I'm gradually returning to a lower
level of medication. I'm still unpacking for myself what this experience
of mine meant, how it still moves in some way inside me. It's elusive. I
do know I want to peel back this isolation and forced secrecy. I must if
I'm going to live honestly in the world. It's almost as disorienting as
delusion, going from that crowded and noisy place of psychosis to this
silence.

Susanne Antonetta is the author of the forthcoming ``The Terrible
Unlikelihood of Our Being Here'' and ``The Devil's Castle,'' and the
editor in chief of the Bellingham Review.

\emph{Disability is a series of essays, art and opinion by and about
people living with disabilities.}

\emph{\textbf{Now in print:}}
\emph{``}\href{https://www.aboutusbook.com/}{\emph{About Us: Essays From
the Disability Series of The New York Times}}\emph{,'' edited by Peter
Catapano and Rosemarie Garland-Thomson, published by Liveright.}

\emph{The Times is committed to publishing}
\href{https://www.nytimes3xbfgragh.onion/2019/01/31/opinion/letters/letters-to-editor-new-york-times-women.html}{\emph{a
diversity of letters}} \emph{to the editor. We'd like to hear what you
think about this or any of our articles. Here are some}
\href{https://help.nytimes3xbfgragh.onion/hc/en-us/articles/115014925288-How-to-submit-a-letter-to-the-editor}{\emph{tips}}\emph{.
And here's our email:}
\href{mailto:letters@NYTimes.com}{\emph{letters@NYTimes.com}}\emph{.}

\emph{Follow The New York Times Opinion section on}
\href{https://www.facebookcorewwwi.onion/nytopinion}{\emph{Facebook}}\emph{,}
\href{http://twitter.com/NYTOpinion}{\emph{Twitter (@NYTopinion)}}
\emph{and}
\href{https://www.instagram.com/nytopinion/}{\emph{Instagram}}\emph{.}

Advertisement

\protect\hyperlink{after-bottom}{Continue reading the main story}

\hypertarget{site-index}{%
\subsection{Site Index}\label{site-index}}

\hypertarget{site-information-navigation}{%
\subsection{Site Information
Navigation}\label{site-information-navigation}}

\begin{itemize}
\tightlist
\item
  \href{https://help.nytimes3xbfgragh.onion/hc/en-us/articles/115014792127-Copyright-notice}{©~2020~The
  New York Times Company}
\end{itemize}

\begin{itemize}
\tightlist
\item
  \href{https://www.nytco.com/}{NYTCo}
\item
  \href{https://help.nytimes3xbfgragh.onion/hc/en-us/articles/115015385887-Contact-Us}{Contact
  Us}
\item
  \href{https://www.nytco.com/careers/}{Work with us}
\item
  \href{https://nytmediakit.com/}{Advertise}
\item
  \href{http://www.tbrandstudio.com/}{T Brand Studio}
\item
  \href{https://www.nytimes3xbfgragh.onion/privacy/cookie-policy\#how-do-i-manage-trackers}{Your
  Ad Choices}
\item
  \href{https://www.nytimes3xbfgragh.onion/privacy}{Privacy}
\item
  \href{https://help.nytimes3xbfgragh.onion/hc/en-us/articles/115014893428-Terms-of-service}{Terms
  of Service}
\item
  \href{https://help.nytimes3xbfgragh.onion/hc/en-us/articles/115014893968-Terms-of-sale}{Terms
  of Sale}
\item
  \href{https://spiderbites.nytimes3xbfgragh.onion}{Site Map}
\item
  \href{https://help.nytimes3xbfgragh.onion/hc/en-us}{Help}
\item
  \href{https://www.nytimes3xbfgragh.onion/subscription?campaignId=37WXW}{Subscriptions}
\end{itemize}
