Sections

SEARCH

\protect\hyperlink{site-content}{Skip to
content}\protect\hyperlink{site-index}{Skip to site index}

\href{https://myaccount.nytimes3xbfgragh.onion/auth/login?response_type=cookie\&client_id=vi}{}

\href{https://www.nytimes3xbfgragh.onion/section/todayspaper}{Today's
Paper}

An Artist Who Doesn't Want to Feed Western Fantasies About Africa

\url{https://nyti.ms/39TXi7u}

\begin{itemize}
\item
\item
\item
\item
\item
\end{itemize}

Advertisement

\protect\hyperlink{after-top}{Continue reading the main story}

Supported by

\protect\hyperlink{after-sponsor}{Continue reading the main story}

On the Verge

\hypertarget{an-artist-who-doesnt-want-to-feed-western-fantasies-about-africa}{%
\section{An Artist Who Doesn't Want to Feed Western Fantasies About
Africa}\label{an-artist-who-doesnt-want-to-feed-western-fantasies-about-africa}}

Cinga Samson's surreal canvases engage obliquely with his identity, but
stand alone as testaments to his finely honed craft.

\includegraphics{https://static01.graylady3jvrrxbe.onion/images/2020/02/20/t-magazine/18tmag-samson-slide-E4TK/18tmag-samson-slide-E4TK-articleLarge.jpg?quality=75\&auto=webp\&disable=upscale}

By Meara Sharma

\begin{itemize}
\item
  Feb. 21, 2020
\item
  \begin{itemize}
  \item
  \item
  \item
  \item
  \item
  \end{itemize}
\end{itemize}

In the South African artist
\href{https://www.perrotin.com/artists/cinga_samson/488}{Cinga Samson}'s
``Ivory'' series of paintings --- five lush, ethereal, figurative
canvases made in 2018 --- a young black man in jeans and an ornate
gold-colored **** jacket stands in the middle of the 4-by-3 foot
compositions, reveling amid tropical ferns, twisting vines **** and
bird-of-paradise plants against a moody backdrop of rocks and sea. In
each painting, the setting is surreal but the figure's stance is coolly
elegant; his eyes, pupil-less white orbs, suggest an inner reverie.
Seemingly unconcerned with the viewer, he projects a strong presence, at
once inviting and enigmatic, joyful and antagonistic. This is a world
that belongs, unequivocally, to him.

That sense of quiet confidence suffuses many of Samson's paintings, 28
of which will be on view at the Perrotin gallery in New York this month
in the artist's first United States solo show. ``I want to create works
that aren't begging you to look at them,'' he says. ``Works that don't
need affirmation.'' From his bright studio in a former industrial
building in Cape Town's cosmopolitan Woodstock neighborhood, Samson, 33,
is refining an aesthetic sensibility that extols his identity as a young
African, but also deliberately withholds something from the viewer.
``The paintings hold their ground,'' he says. ``If ever you get too
close, you start to approach danger.''

\includegraphics{https://static01.graylady3jvrrxbe.onion/images/2020/02/20/t-magazine/18tmag-samson-slide-9UBX/18tmag-samson-slide-9UBX-articleLarge.jpg?quality=75\&auto=webp\&disable=upscale}

This restraint stems in part from Samson's frustration with
expectations, particularly in the West, that as a South African artist,
his work must engage with issues of race, poverty and corruption in an
explicit, easily digestible way. He is ** of course interested in
reckoning with what's important to him --- spirituality, nature,
sensuality, what it means to be a man --- but on his own terms, in his
own language and beyond a straightforwardly activist message. This
perspective crystallized after Samson turned 30, he says, when he found
himself confronting the fact of his limited time on earth. Rather than
fight against structural problems he couldn't fix, he decided instead to
celebrate life and his own potential. ``I want to push into the future,
not be marginalized under the politics of this moment,'' he says. ``I
don't want to be an artist that feeds Western fantasies about what
Africa is.''

Image

A detail from one of Samson's recent large-scale
paintings.Credit...Stephanie Veldman

Image

Samson works from photographs, though the figures in his paintings take
on a life of their own.Credit...Stephanie Veldman

Some of Samson's newest pieces, large-format paintings of black men and
women gathered as if for a ceremony by a rocky shoreline, play with
notions of the supernatural while also subverting stereotypical
portrayals of African spirituality. The works emerged from a family
story about Samson's aunt, who is said to have come back from the dead
after drowning, as well as a Xhosa belief that long-departed forebears
sometimes appear by rivers. But while the figures in the paintings are
intended to evoke people ``from another realm,'' Samson says, rather
than resorting to clichéd depictions of ancestors (wizened elders draped
in skins, for example), he renders them as youthful visitors who wear
jeans and play with inflatable pool toys. And while the scenes conjure
an ancient, dreamlike ritual, **** their controlled palette of bright
blue, brown and gray feels almost futuristic. In the background of one
of the works, a stark cityscape calls to mind contemporary Cape Town and
its ongoing water shortages.

Image

Samson's studio in the Woodstock neighborhood of Cape
Town.Credit...Stephanie Veldman

Born in 1986 and raised in townships and villages in the Mthatha area of
the Eastern Cape, a region of rolling hills and wide vistas, Samson
identified a connection with art early on. ``I realized I could draw
\href{https://www.nytimes3xbfgragh.onion/topic/person/snoop-dogg}{Snoop
Dogg},
\href{https://www.nytimes3xbfgragh.onion/topic/person/tupac-shakur}{Tupac}
and
\href{https://www.nytimes3xbfgragh.onion/topic/person/nelson-mandela}{Mandela}
better than my friends,'' he recalls. In his early 20s, he joined
Isibane, a collective of artists in Cape Town from whom he gleaned
support, resources and what he describes as ``a manual for being an
artist.'' Working for nearly four years at the group's studio, under the
guidance of older painters, he forged a vision of his own, as well as an
unwavering devotion to the task of making art itself. ``You can't be
drinking, buying good clothes, looking good,'' he remembers his mentors
counseling. ``If you don't have art materials, what kind of an artist
are you?'' Some 15 years later, that ethic continues to define Samson's
practice, which he approaches with an almost ascetic intensity. Though
he keeps a house ``as a formality,'' he says, he rarely leaves the
studio; his purpose, as he sees it, is to hone his ``gift,'' stretching
his capabilities as far as he can. Indeed, having shown regularly in
Cape Town galleries over the past decade, as well as in Berlin, Oaxaca,
the Minneapolis Institute of Art and New York's Armory Show, Samson is
proving himself to be a **** vital emerging figure in contemporary
painting.

Image

Samson outside of his studio.Credit...Stephanie Veldman

Though he's wary of shouldering the burden of politics, Samson's
commitment to excellence lends his work a powerful charge. ``In all my
art books, African artists are in the last chapter, if at all,'' he
says, and while he has been deeply influenced by a range of Western
greats ---
\href{https://www.nytimes3xbfgragh.onion/topic/person/alberto-giacometti}{Alberto
Giacometti},
\href{https://www.nytimes3xbfgragh.onion/topic/person/egon-schiele}{Egon
Schiele},
\href{https://www.nytimes3xbfgragh.onion/topic/person/francis-bacon}{Francis
Bacon} and
\href{https://www.nytimes3xbfgragh.onion/topic/person/louise-bourgeois}{Louise
Bourgeois} among them --- he emphasizes the need for African artists to
have their own masters to reference. Part of what animates his practice
is a belief that sheer virtuosity is an antidote to pity and
marginalization. ``I want you to look at my work and think, `That's
exquisite, that's incredible, I am blown away --- and I know it is
completely African,' without feeling that it is lacking,'' he says,
adding, with a smile: ``It's that
\href{https://www.nytimes3xbfgragh.onion/topic/person/serena-williams}{Serena
Williams} excellence, which, if you deny it, you know you're just ****
evil.''

``Amadoda Akafani, Afana Ngeentshebe Zodwa (men are different, though
they look alike)'' is on view from Feb. 22 to April 11 at Perrotin
gallery, 130 Orchard Street, New York,
\href{https://www.perrotin.com/artists/cinga_samson/488\#news}{perrotin.com}.

Advertisement

\protect\hyperlink{after-bottom}{Continue reading the main story}

\hypertarget{site-index}{%
\subsection{Site Index}\label{site-index}}

\hypertarget{site-information-navigation}{%
\subsection{Site Information
Navigation}\label{site-information-navigation}}

\begin{itemize}
\tightlist
\item
  \href{https://help.nytimes3xbfgragh.onion/hc/en-us/articles/115014792127-Copyright-notice}{©~2020~The
  New York Times Company}
\end{itemize}

\begin{itemize}
\tightlist
\item
  \href{https://www.nytco.com/}{NYTCo}
\item
  \href{https://help.nytimes3xbfgragh.onion/hc/en-us/articles/115015385887-Contact-Us}{Contact
  Us}
\item
  \href{https://www.nytco.com/careers/}{Work with us}
\item
  \href{https://nytmediakit.com/}{Advertise}
\item
  \href{http://www.tbrandstudio.com/}{T Brand Studio}
\item
  \href{https://www.nytimes3xbfgragh.onion/privacy/cookie-policy\#how-do-i-manage-trackers}{Your
  Ad Choices}
\item
  \href{https://www.nytimes3xbfgragh.onion/privacy}{Privacy}
\item
  \href{https://help.nytimes3xbfgragh.onion/hc/en-us/articles/115014893428-Terms-of-service}{Terms
  of Service}
\item
  \href{https://help.nytimes3xbfgragh.onion/hc/en-us/articles/115014893968-Terms-of-sale}{Terms
  of Sale}
\item
  \href{https://spiderbites.nytimes3xbfgragh.onion}{Site Map}
\item
  \href{https://help.nytimes3xbfgragh.onion/hc/en-us}{Help}
\item
  \href{https://www.nytimes3xbfgragh.onion/subscription?campaignId=37WXW}{Subscriptions}
\end{itemize}
