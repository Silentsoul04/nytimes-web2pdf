Sections

SEARCH

\protect\hyperlink{site-content}{Skip to
content}\protect\hyperlink{site-index}{Skip to site index}

\href{https://www.nytimes3xbfgragh.onion/section/health}{Health}

\href{https://myaccount.nytimes3xbfgragh.onion/auth/login?response_type=cookie\&client_id=vi}{}

\href{https://www.nytimes3xbfgragh.onion/section/todayspaper}{Today's
Paper}

\href{/section/health}{Health}\textbar{}Coronavirus Diagnosis in
California Highlights Testing Flaws

\url{https://nyti.ms/384Z58y}

\begin{itemize}
\item
\item
\item
\item
\item
\item
\end{itemize}

\hypertarget{the-coronavirus-outbreak}{%
\subsubsection{\texorpdfstring{\href{https://www.nytimes3xbfgragh.onion/news-event/coronavirus?name=styln-coronavirus-national\&region=TOP_BANNER\&variant=undefined\&block=storyline_menu_recirc\&action=click\&pgtype=Article\&impression_id=d1641300-e3af-11ea-a964-3f8cb8589728}{The
Coronavirus
Outbreak}}{The Coronavirus Outbreak}}\label{the-coronavirus-outbreak}}

\begin{itemize}
\tightlist
\item
  live\href{https://www.nytimes3xbfgragh.onion/2020/08/21/world/covid-19-coronavirus.html?name=styln-coronavirus-national\&region=TOP_BANNER\&variant=undefined\&block=storyline_menu_recirc\&action=click\&pgtype=Article\&impression_id=d1643a10-e3af-11ea-a964-3f8cb8589728}{Latest
  Updates}
\item
  \href{https://www.nytimes3xbfgragh.onion/interactive/2020/us/coronavirus-us-cases.html?name=styln-coronavirus-national\&region=TOP_BANNER\&variant=undefined\&block=storyline_menu_recirc\&action=click\&pgtype=Article\&impression_id=d1643a11-e3af-11ea-a964-3f8cb8589728}{Maps
  and Cases}
\item
  \href{https://www.nytimes3xbfgragh.onion/interactive/2020/science/coronavirus-vaccine-tracker.html?name=styln-coronavirus-national\&region=TOP_BANNER\&variant=undefined\&block=storyline_menu_recirc\&action=click\&pgtype=Article\&impression_id=d1643a12-e3af-11ea-a964-3f8cb8589728}{Vaccine
  Tracker}
\item
  \href{https://www.nytimes3xbfgragh.onion/2020/08/19/us/colleges-closing-covid.html?name=styln-coronavirus-national\&region=TOP_BANNER\&variant=undefined\&block=storyline_menu_recirc\&action=click\&pgtype=Article\&impression_id=d1643a13-e3af-11ea-a964-3f8cb8589728}{Colleges
  Closing}
\item
  \href{https://www.nytimes3xbfgragh.onion/live/2020/08/21/business/stock-market-today-coronavirus?name=styln-coronavirus-national\&region=TOP_BANNER\&variant=undefined\&block=storyline_menu_recirc\&action=click\&pgtype=Article\&impression_id=d1643a14-e3af-11ea-a964-3f8cb8589728}{Economy}
\end{itemize}

Advertisement

\protect\hyperlink{after-top}{Continue reading the main story}

Supported by

\protect\hyperlink{after-sponsor}{Continue reading the main story}

\hypertarget{coronavirus-diagnosis-in-california-highlights-testing-flaws}{%
\section{Coronavirus Diagnosis in California Highlights Testing
Flaws}\label{coronavirus-diagnosis-in-california-highlights-testing-flaws}}

Doctors suspected infection with the virus, but the patient did not fit
the federal criteria and was not tested for days.

\includegraphics{https://static01.graylady3jvrrxbe.onion/images/2020/02/28/science/27VIRUS-TESTING-print/merlin_169662750_9ae4d1c6-beed-4231-a031-d06288b5ece3-articleLarge.jpg?quality=75\&auto=webp\&disable=upscale}

\href{https://www.nytimes3xbfgragh.onion/by/roni-caryn-rabin}{\includegraphics{https://static01.graylady3jvrrxbe.onion/images/2018/02/20/multimedia/author-roni-caryn-rabin/author-roni-caryn-rabin-thumbLarge-v3.png}}\href{https://www.nytimes3xbfgragh.onion/by/sheri-fink}{\includegraphics{https://static01.graylady3jvrrxbe.onion/images/2018/08/24/multimedia/author-sheri-fink/author-sheri-fink-thumbLarge.png}}\href{https://www.nytimes3xbfgragh.onion/by/knvul-sheikh}{\includegraphics{https://static01.graylady3jvrrxbe.onion/images/2020/01/03/reader-center/author-knvul-sheikh/author-knvul-sheikh-thumbLarge.png}}

By \href{https://www.nytimes3xbfgragh.onion/by/roni-caryn-rabin}{Roni
Caryn Rabin},
\href{https://www.nytimes3xbfgragh.onion/by/sheri-fink}{Sheri Fink} and
\href{https://www.nytimes3xbfgragh.onion/by/knvul-sheikh}{Knvul Sheikh}

\begin{itemize}
\item
  Published Feb. 27, 2020Updated Feb. 28, 2020
\item
  \begin{itemize}
  \item
  \item
  \item
  \item
  \item
  \item
  \end{itemize}
\end{itemize}

Already in deep distress, the patient was rushed last week to a hospital
in Northern California, severely ill and unable to breathe on her own.

Doctors at the University of California, Davis Medical Center, near
Sacramento, provided the woman with critical care but also considered an
unlikely diagnosis: infection with the coronavirus.

Hospital administrators said they immediately requested diagnostic
testing from the Centers for Disease Control and Prevention, but the
procedure was not carried out because the case did not qualify under
strict federal criteria: She had not traveled to China and had not been
in contact with anyone known to be infected.

The announcement on Wednesday that the woman was indeed infected left
health officials in California searching for people she may have exposed
to the virus and testing the medical workers who have treated her. The
case has raised difficult questions about whom to test and whether the
nation is prepared to keep the virus under control.

The California woman's case may also offer the first indication that the
virus has spread beyond Americans who traveled outside the country, or
had contact with someone who had.

Even before the announcement on Wednesday, frustration had been mounting
among health providers and medical experts that the agency was testing
too few Americans, which may slow preparations for an outbreak and may
obscure the scope of infections.

``I think the diagnostic issue is the single most important thing that
keeps me up at night right now,'' said Lauren Sauer, director of
operations at the Johns Hopkins Office of Critical Event Preparedness
and Response in Baltimore.

\includegraphics{https://static01.graylady3jvrrxbe.onion/images/2020/02/27/science/27VIRUS-TESTING/merlin_169635183_33273393-de47-40b8-b992-e1082c63a77f-articleLarge.jpg?quality=75\&auto=webp\&disable=upscale}

C.D.C. officials said on Thursday that they had been unaware that
doctors in California made an urgent appeal for diagnostic testing of
the woman. But by the end of the day, the agency had revised and
broadened its testing criteria, adding to the number of Americans who
qualify.

In California, health officials are tracing close contacts of the woman,
who lives in Solano County but has not otherwise been identified. Health
care workers who have treated her are being monitored for the infection,
and some employees at the medical center have been told to stay home.

\hypertarget{latest-updates-the-coronavirus-outbreak}{%
\section{\texorpdfstring{\href{https://www.nytimes3xbfgragh.onion/2020/08/21/world/covid-19-coronavirus.html?action=click\&pgtype=Article\&state=default\&region=MAIN_CONTENT_1\&context=storylines_live_updates}{Latest
Updates: The Coronavirus
Outbreak}}{Latest Updates: The Coronavirus Outbreak}}\label{latest-updates-the-coronavirus-outbreak}}

Updated 2020-08-21T13:00:19.184Z

\begin{itemize}
\tightlist
\item
  \href{https://www.nytimes3xbfgragh.onion/2020/08/21/world/covid-19-coronavirus.html?action=click\&pgtype=Article\&state=default\&region=MAIN_CONTENT_1\&context=storylines_live_updates\#link-6a60a19d}{`Be
  adults': Universities in the U.S. are warning students about
  gatherings as they return to campus.}
\item
  \href{https://www.nytimes3xbfgragh.onion/2020/08/21/world/covid-19-coronavirus.html?action=click\&pgtype=Article\&state=default\&region=MAIN_CONTENT_1\&context=storylines_live_updates\#link-324af071}{As
  he accepts the Democratic nomination, Biden knocks Trump's pandemic
  response.}
\item
  \href{https://www.nytimes3xbfgragh.onion/2020/08/21/world/covid-19-coronavirus.html?action=click\&pgtype=Article\&state=default\&region=MAIN_CONTENT_1\&context=storylines_live_updates\#link-191d44be}{South
  Korea threatens to detain people who obstruct virus-control efforts.}
\end{itemize}

\href{https://www.nytimes3xbfgragh.onion/2020/08/21/world/covid-19-coronavirus.html?action=click\&pgtype=Article\&state=default\&region=MAIN_CONTENT_1\&context=storylines_live_updates}{See
more updates}

More live coverage:
\href{https://www.nytimes3xbfgragh.onion/live/2020/08/21/business/stock-market-today-coronavirus?action=click\&pgtype=Article\&state=default\&region=MAIN_CONTENT_1\&context=storylines_live_updates}{Markets}

Officials are bracing for a larger outbreak in Northern California.
``There's almost assuredly going to be a significant number of people
testing positive,'' said Dr. Peter L. Beilenson, the director of
Sacramento County's Department of Health Services.

The case has heightened concerns about the nation's ability to test
large numbers of people. Only the C.D.C. performs the tests that confirm
a novel coronavirus diagnosis, a process that often takes days.

The C.D.C. had distributed diagnostic testing kits to state health
departments, but they turned out to be flawed. Replacement kits have not
yet been distributed.

Ms. Sauer said Johns Hopkins had treated several patients who did not
fit the testing criteria, and for whom it requested coronavirus testing.
In all but one case, the hospital was able to persuade the C.D.C. to run
a test, or eventually identify another cause for the patient's illness.

The C.D.C. is ``pretty much the only place we can access testing,'' she
added, and the agency has been unwilling to grapple with cases that do
not fit its criteria.

``The idea that we would have to really fight to get that test done,
when C.D.C. is saying they have capacity, is alarming.'' Ms. Sauer said.
``It is a challenge when the most important piece of information ---
does this person have this disease, yes or no --- is not accessible, and
there's no timeline for improved accessibility.''

Image

California Governor Gavin Newsom delivered a coronavirus briefing at the
California Department of Public Health in Sacramento on
Thursday.Credit...Max Whittaker for The New York Times

After the diagnosis in California, the C.D.C. has pledged to greatly
expand the state's ability to test patients for the coronavirus,
officials said.

``Testing protocols have been a point of frustration for many of us,''
Gov. Gavin Newsom said at a news conference on Thursday. California had
just 200 testing kits left, he added.

The governor said Dr. Robert R. Redfield, the director of the C.D.C.,
promised that state physicians would have a much greater ability to test
patients who were showing symptoms, a change the governor said ``can't
happen soon enough.''

\href{https://www.nytimes3xbfgragh.onion/news-event/coronavirus?action=click\&pgtype=Article\&state=default\&region=MAIN_CONTENT_3\&context=storylines_faq}{}

\hypertarget{the-coronavirus-outbreak-}{%
\subsubsection{The Coronavirus Outbreak
›}\label{the-coronavirus-outbreak-}}

\hypertarget{frequently-asked-questions}{%
\paragraph{Frequently Asked
Questions}\label{frequently-asked-questions}}

Updated August 17, 2020

\begin{itemize}
\item ~
  \hypertarget{why-does-standing-six-feet-away-from-others-help}{%
  \paragraph{Why does standing six feet away from others
  help?}\label{why-does-standing-six-feet-away-from-others-help}}

  \begin{itemize}
  \tightlist
  \item
    The coronavirus spreads primarily through droplets from your mouth
    and nose, especially when you cough or sneeze. The C.D.C., one of
    the organizations using that measure,
    \href{https://www.nytimes3xbfgragh.onion/2020/04/14/health/coronavirus-six-feet.html?action=click\&pgtype=Article\&state=default\&region=MAIN_CONTENT_3\&context=storylines_faq}{bases
    its recommendation of six feet} on the idea that most large droplets
    that people expel when they cough or sneeze will fall to the ground
    within six feet. But six feet has never been a magic number that
    guarantees complete protection. Sneezes, for instance, can launch
    droplets a lot farther than six feet,
    \href{https://jamanetwork.com/journals/jama/fullarticle/2763852}{according
    to a recent study}. It's a rule of thumb: You should be safest
    standing six feet apart outside, especially when it's windy. But
    keep a mask on at all times, even when you think you're far enough
    apart.
  \end{itemize}
\item ~
  \hypertarget{i-have-antibodies-am-i-now-immune}{%
  \paragraph{I have antibodies. Am I now
  immune?}\label{i-have-antibodies-am-i-now-immune}}

  \begin{itemize}
  \tightlist
  \item
    As of right
    now,\href{https://www.nytimes3xbfgragh.onion/2020/07/22/health/covid-antibodies-herd-immunity.html?action=click\&pgtype=Article\&state=default\&region=MAIN_CONTENT_3\&context=storylines_faq}{that
    seems likely, for at least several months.} There have been
    frightening accounts of people suffering what seems to be a second
    bout of Covid-19. But experts say these patients may have a
    drawn-out course of infection, with the virus taking a slow toll
    weeks to months after initial exposure. People infected with the
    coronavirus typically
    \href{https://www.nature.com/articles/s41586-020-2456-9}{produce}
    immune molecules called antibodies, which are
    \href{https://www.nytimes3xbfgragh.onion/2020/05/07/health/coronavirus-antibody-prevalence.html?action=click\&pgtype=Article\&state=default\&region=MAIN_CONTENT_3\&context=storylines_faq}{protective
    proteins made in response to an
    infection}\href{https://www.nytimes3xbfgragh.onion/2020/05/07/health/coronavirus-antibody-prevalence.html?action=click\&pgtype=Article\&state=default\&region=MAIN_CONTENT_3\&context=storylines_faq}{.
    These antibodies may} last in the body
    \href{https://www.nature.com/articles/s41591-020-0965-6}{only two to
    three months}, which may seem worrisome, but that's perfectly normal
    after an acute infection subsides, said Dr. Michael Mina, an
    immunologist at Harvard University. It may be possible to get the
    coronavirus again, but it's highly unlikely that it would be
    possible in a short window of time from initial infection or make
    people sicker the second time.
  \end{itemize}
\item ~
  \hypertarget{im-a-small-business-owner-can-i-get-relief}{%
  \paragraph{I'm a small-business owner. Can I get
  relief?}\label{im-a-small-business-owner-can-i-get-relief}}

  \begin{itemize}
  \tightlist
  \item
    The
    \href{https://www.nytimes3xbfgragh.onion/article/small-business-loans-stimulus-grants-freelancers-coronavirus.html?action=click\&pgtype=Article\&state=default\&region=MAIN_CONTENT_3\&context=storylines_faq}{stimulus
    bills enacted in March} offer help for the millions of American
    small businesses. Those eligible for aid are businesses and
    nonprofit organizations with fewer than 500 workers, including sole
    proprietorships, independent contractors and freelancers. Some
    larger companies in some industries are also eligible. The help
    being offered, which is being managed by the Small Business
    Administration, includes the Paycheck Protection Program and the
    Economic Injury Disaster Loan program. But lots of folks have
    \href{https://www.nytimes3xbfgragh.onion/interactive/2020/05/07/business/small-business-loans-coronavirus.html?action=click\&pgtype=Article\&state=default\&region=MAIN_CONTENT_3\&context=storylines_faq}{not
    yet seen payouts.} Even those who have received help are confused:
    The rules are draconian, and some are stuck sitting on
    \href{https://www.nytimes3xbfgragh.onion/2020/05/02/business/economy/loans-coronavirus-small-business.html?action=click\&pgtype=Article\&state=default\&region=MAIN_CONTENT_3\&context=storylines_faq}{money
    they don't know how to use.} Many small-business owners are getting
    less than they expected or
    \href{https://www.nytimes3xbfgragh.onion/2020/06/10/business/Small-business-loans-ppp.html?action=click\&pgtype=Article\&state=default\&region=MAIN_CONTENT_3\&context=storylines_faq}{not
    hearing anything at all.}
  \end{itemize}
\item ~
  \hypertarget{what-are-my-rights-if-i-am-worried-about-going-back-to-work}{%
  \paragraph{What are my rights if I am worried about going back to
  work?}\label{what-are-my-rights-if-i-am-worried-about-going-back-to-work}}

  \begin{itemize}
  \tightlist
  \item
    Employers have to provide
    \href{https://www.osha.gov/SLTC/covid-19/standards.html}{a safe
    workplace} with policies that protect everyone equally.
    \href{https://www.nytimes3xbfgragh.onion/article/coronavirus-money-unemployment.html?action=click\&pgtype=Article\&state=default\&region=MAIN_CONTENT_3\&context=storylines_faq}{And
    if one of your co-workers tests positive for the coronavirus, the
    C.D.C.} has said that
    \href{https://www.cdc.gov/coronavirus/2019-ncov/community/guidance-business-response.html}{employers
    should tell their employees} -\/- without giving you the sick
    employee's name -\/- that they may have been exposed to the virus.
  \end{itemize}
\item ~
  \hypertarget{what-is-school-going-to-look-like-in-september}{%
  \paragraph{What is school going to look like in
  September?}\label{what-is-school-going-to-look-like-in-september}}

  \begin{itemize}
  \tightlist
  \item
    It is unlikely that many schools will return to a normal schedule
    this fall, requiring the grind of
    \href{https://www.nytimes3xbfgragh.onion/2020/06/05/us/coronavirus-education-lost-learning.html?action=click\&pgtype=Article\&state=default\&region=MAIN_CONTENT_3\&context=storylines_faq}{online
    learning},
    \href{https://www.nytimes3xbfgragh.onion/2020/05/29/us/coronavirus-child-care-centers.html?action=click\&pgtype=Article\&state=default\&region=MAIN_CONTENT_3\&context=storylines_faq}{makeshift
    child care} and
    \href{https://www.nytimes3xbfgragh.onion/2020/06/03/business/economy/coronavirus-working-women.html?action=click\&pgtype=Article\&state=default\&region=MAIN_CONTENT_3\&context=storylines_faq}{stunted
    workdays} to continue. California's two largest public school
    districts --- Los Angeles and San Diego --- said on July 13, that
    \href{https://www.nytimes3xbfgragh.onion/2020/07/13/us/lausd-san-diego-school-reopening.html?action=click\&pgtype=Article\&state=default\&region=MAIN_CONTENT_3\&context=storylines_faq}{instruction
    will be remote-only in the fall}, citing concerns that surging
    coronavirus infections in their areas pose too dire a risk for
    students and teachers. Together, the two districts enroll some
    825,000 students. They are the largest in the country so far to
    abandon plans for even a partial physical return to classrooms when
    they reopen in August. For other districts, the solution won't be an
    all-or-nothing approach.
    \href{https://bioethics.jhu.edu/research-and-outreach/projects/eschool-initiative/school-policy-tracker/}{Many
    systems}, including the nation's largest, New York City, are
    devising
    \href{https://www.nytimes3xbfgragh.onion/2020/06/26/us/coronavirus-schools-reopen-fall.html?action=click\&pgtype=Article\&state=default\&region=MAIN_CONTENT_3\&context=storylines_faq}{hybrid
    plans} that involve spending some days in classrooms and other days
    online. There's no national policy on this yet, so check with your
    municipal school system regularly to see what is happening in your
    community.
  \end{itemize}
\end{itemize}

The C.D.C. has committed to sending a team to California to help track
people and make sure they are contacted by health officials about their
possible exposure, Mr. Newsom said.

``They are being interviewed --- points of contact, family members and
others,'' he said.

Experts said they were perplexed by the C.D.C.'s inability to fix the
test's flaws.

``The obvious observation is that many countries are capable of testing
rather widely,'' said Dr. William Schaffner, an infectious disease
specialist at Vanderbilt University Medical Center in Nashville. ``Why
can't we?''

\textbf{\emph{{[}}\href{http://on.fb.me/1paTQ1h}{\emph{Like the Science
Times page on Facebook.}}} ****** \emph{\textbar{} Sign up for the}
\textbf{\href{http://nyti.ms/1MbHaRU}{\emph{Science Times
newsletter.}}\emph{{]}}}

The C.D.C. operates two laboratories that test for the coronavirus and
can handle approximately 400 specimens per day. Agency officials say
there is no testing backlog, but it is unclear whether the labs will be
able to keep up with demand if the need --- and eligibility ---
increases testing substantially.

Under the new federal criteria, people with respiratory symptoms who
traveled to Iran, Italy, Japan and South Korea should be tested --- not
just those who traveled in China. So should severely ill patients with
acute lower respiratory symptoms who are hospitalized and in whom other
diagnoses have been ruled out.

A criticism of the new criteria, however, is that doctors will have to
wait until someone is extremely ill to test for the virus if that person
did not travel to the affected regions or have contact with a known
case.

``If we could identify these people earlier who don't specifically meet
one of the two criteria, or some sort of broader travel criteria, we
could get them tested,'' Ms. Sauer said. ``You have to wait until
someone's really sick to push that test now, even with this new
criteria.

Image

A tally of coronavirus cases on a white board in the California health
department's Medical Health Coordinating Center.Credit...Max Whittaker
for The New York Times

Kenneth E. Raske, president of the Greater New York Hospital
Association, said he planned to appeal to Vice President Mike Pence ---
whom President Trump named to lead federal preparations --- ``to order
the C.D.C. to develop a rapid point of care test'' that hospitals could
use to screen patients.

In the meantime, Mr. Raske said, the Wadsworth Center, New York's public
health reference laboratory, should be certified to do these tests.

If more community-acquired infections turn up, and the disease cannot be
contained, the strategy will have to become one of mitigation, said Dr.
Neil Fishman, associate chief medical officer for the University of
Pennsylvania Health System.

``That's a little difficult to do when you don't have a readily
available test, and when the turnaround time for the test can be days
instead of hours,'' he said.

Thomas Fuller contributed reporting from Solano County, Calif., Nicholas
Bogel-Burroughs from New York and Michael D. Shear from Washington, D.C.

Advertisement

\protect\hyperlink{after-bottom}{Continue reading the main story}

\hypertarget{site-index}{%
\subsection{Site Index}\label{site-index}}

\hypertarget{site-information-navigation}{%
\subsection{Site Information
Navigation}\label{site-information-navigation}}

\begin{itemize}
\tightlist
\item
  \href{https://help.nytimes3xbfgragh.onion/hc/en-us/articles/115014792127-Copyright-notice}{©~2020~The
  New York Times Company}
\end{itemize}

\begin{itemize}
\tightlist
\item
  \href{https://www.nytco.com/}{NYTCo}
\item
  \href{https://help.nytimes3xbfgragh.onion/hc/en-us/articles/115015385887-Contact-Us}{Contact
  Us}
\item
  \href{https://www.nytco.com/careers/}{Work with us}
\item
  \href{https://nytmediakit.com/}{Advertise}
\item
  \href{http://www.tbrandstudio.com/}{T Brand Studio}
\item
  \href{https://www.nytimes3xbfgragh.onion/privacy/cookie-policy\#how-do-i-manage-trackers}{Your
  Ad Choices}
\item
  \href{https://www.nytimes3xbfgragh.onion/privacy}{Privacy}
\item
  \href{https://help.nytimes3xbfgragh.onion/hc/en-us/articles/115014893428-Terms-of-service}{Terms
  of Service}
\item
  \href{https://help.nytimes3xbfgragh.onion/hc/en-us/articles/115014893968-Terms-of-sale}{Terms
  of Sale}
\item
  \href{https://spiderbites.nytimes3xbfgragh.onion}{Site Map}
\item
  \href{https://help.nytimes3xbfgragh.onion/hc/en-us}{Help}
\item
  \href{https://www.nytimes3xbfgragh.onion/subscription?campaignId=37WXW}{Subscriptions}
\end{itemize}
