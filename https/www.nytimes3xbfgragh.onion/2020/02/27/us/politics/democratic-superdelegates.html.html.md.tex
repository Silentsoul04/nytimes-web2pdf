Sections

SEARCH

\protect\hyperlink{site-content}{Skip to
content}\protect\hyperlink{site-index}{Skip to site index}

\href{https://www.nytimes3xbfgragh.onion/section/politics}{Politics}

\href{https://myaccount.nytimes3xbfgragh.onion/auth/login?response_type=cookie\&client_id=vi}{}

\href{https://www.nytimes3xbfgragh.onion/section/todayspaper}{Today's
Paper}

\href{/section/politics}{Politics}\textbar{}Democratic Leaders Willing
to Risk Party Damage to Stop Bernie Sanders

\url{https://nyti.ms/2PsxaJi}

\begin{itemize}
\item
\item
\item
\item
\item
\item
\end{itemize}

\begin{itemize}
\item
  \href{https://www.nytimes3xbfgragh.onion/live/2020/08/20/us/dnc-convention-election?action=click\&pgtype=Article\&state=default\&region=TOP_BANNER\&context=storylines_menu}{D.N.C.
  Updates}
\item
  \href{https://www.nytimes3xbfgragh.onion/2020/08/20/us/politics/biden-presidential-nomination-dnc.html?action=click\&pgtype=Article\&state=default\&region=TOP_BANNER\&context=storylines_menu}{Biden's
  Speech}
\item
  \href{https://www.nytimes3xbfgragh.onion/interactive/2019/us/elections/2020-presidential-election-calendar.html?action=click\&pgtype=Article\&state=default\&region=TOP_BANNER\&context=storylines_menu}{Election
  Calendar}
\item
  \href{https://www.nytimes3xbfgragh.onion/interactive/2020/08/11/us/politics/vote-by-mail-us-states.html?action=click\&pgtype=Article\&state=default\&region=TOP_BANNER\&context=storylines_menu}{Voting
  by Mail}
\item
  \href{https://www.nytimes3xbfgragh.onion/newsletters/politics?action=click\&pgtype=Article\&state=default\&region=TOP_BANNER\&context=storylines_menu}{Politics
  Newsletter}
\end{itemize}

Advertisement

\protect\hyperlink{after-top}{Continue reading the main story}

Supported by

\protect\hyperlink{after-sponsor}{Continue reading the main story}

\hypertarget{democratic-leaders-willing-to-risk-party-damage-to-stop-bernie-sanders}{%
\section{Democratic Leaders Willing to Risk Party Damage to Stop Bernie
Sanders}\label{democratic-leaders-willing-to-risk-party-damage-to-stop-bernie-sanders}}

Interviews with dozens of Democratic Party officials, including 93
superdelegates, found overwhelming opposition to handing Mr. Sanders the
nomination if he fell short of a majority of delegates.

\includegraphics{https://static01.graylady3jvrrxbe.onion/images/2020/02/27/us/politics/27superdelegates/merlin_169596693_99e098b9-9c2e-4a5d-bf71-22ebac99ad9c-articleLarge.jpg?quality=75\&auto=webp\&disable=upscale}

\href{https://www.nytimes3xbfgragh.onion/by/lisa-lerer}{\includegraphics{https://static01.graylady3jvrrxbe.onion/images/2018/09/11/us/politics/author-lisa-lerer/lisa-lerer-headshot-thumbLarge.png}}\href{https://www.nytimes3xbfgragh.onion/by/reid-j-epstein}{\includegraphics{https://static01.graylady3jvrrxbe.onion/images/2019/06/25/reader-center/author-reid-epstein/9e877853d8234217b58e5762253aa771-thumbLarge.png}}

By \href{https://www.nytimes3xbfgragh.onion/by/lisa-lerer}{Lisa Lerer}
and \href{https://www.nytimes3xbfgragh.onion/by/reid-j-epstein}{Reid J.
Epstein}

\begin{itemize}
\item
  Published Feb. 27, 2020Updated March 2, 2020
\item
  \begin{itemize}
  \item
  \item
  \item
  \item
  \item
  \item
  \end{itemize}
\end{itemize}

WASHINGTON --- House Speaker Nancy Pelosi and Senator Chuck Schumer, the
minority leader, hear constant warnings from allies about congressional
losses in November if the party nominates Bernie Sanders for president.
Democratic House members share their Sanders fears on text-messaging
chains. Bill Clinton, in calls with old friends, vents about the party
getting wiped out in the general election.

And officials in the national and state parties are increasingly anxious
about splintered
\href{https://www.nytimes3xbfgragh.onion/2020/02/27/us/politics/super-tuesday.html}{primaries
on Super Tuesday} and beyond, where the liberal Mr. Sanders, of Vermont,
edges out moderate candidates who collectively win more votes.

Dozens of interviews with Democratic establishment leaders this week
show that they are not just worried about Mr. Sanders's candidacy, but
are also willing to risk intraparty damage to stop his nomination at the
national convention in July if they get the chance. Since
\href{https://www.nytimes3xbfgragh.onion/2020/02/22/us/politics/bernie-sanders-nevada-caucus.html}{Mr.
Sanders's victory in Nevada's caucuses on Saturday}, The Times has
interviewed 93 party officials --- all of them superdelegates, who could
have a say on the nominee at the convention --- and found overwhelming
opposition to handing the Vermont senator the nomination if he arrived
with the most delegates but
\href{https://www.nytimes3xbfgragh.onion/2020/02/22/us/politics/democratic-primary-dnc-superdelegates.html}{fell
short of a majority}.

Such a situation may result in a brokered convention, a messy political
battle the likes of which Democrats have not seen since 1952,
\href{https://www.nytimes3xbfgragh.onion/2020/02/27/us/brokered-democratic-convention.html?action=click\&module=RelatedLinks\&pgtype=Article}{when
the nominee was Adlai Stevenson}.

``We're way, way, way past the day where party leaders can determine an
outcome here, but I think there's a vibrant conversation about whether
there is anything that can be done,'' said Jim Himes, a Connecticut
congressman and superdelegate, who believes the nominee should have a
majority of delegates.

From California to the Carolinas, and North Dakota to Ohio, the party
leaders say they worry that Mr. Sanders,
\href{https://www.nytimes3xbfgragh.onion/2020/02/24/us/politics/bernie-sanders-democratic-voters.html}{a
democratic socialist with passionate but limited support so far}, will
lose to President Trump, and drag down moderate House and Senate
candidates in swing states with
\href{https://www.nytimes3xbfgragh.onion/2020/02/25/upshot/medicare-for-all-basics-bernie-sanders.html}{his
left-wing agenda of ``Medicare for all''} and free four-year public
college.

Mr. Sanders and his advisers insist that the opposite is true --- that
his ideas will generate huge excitement among young and working-class
voters, and lead to record turnout. Such hopes
\href{https://www.nytimes3xbfgragh.onion/2020/02/24/us/politics/bernie-sanders-democratic-voters.html}{have
yet to be borne out} in nominating contests so far.

Jay Jacobs, the New York State Democratic Party chairman and a
superdelegate, echoing many others interviewed, said that superdelegates
should choose a nominee they believed had the best chance of defeating
Mr. Trump if no candidate wins a majority of delegates during the
primaries. Mr. Sanders argued that he should become the nominee at the
convention with a plurality of delegates, to reflect the will of voters,
and that denying him the nomination would enrage his supporters and
split the party for years to come.

``Bernie wants to redefine the rules and just say he just needs a
plurality,'' Mr. Jacobs said. ``I don't think we buy that. I don't think
the mainstream of the Democratic Party buys that. If he doesn't have a
majority, it stands to reason that he may not become the nominee.''

This article is based on interviews with the 93 superdelegates, out of
771 total, as well as party strategists and aides to senior Democrats
about the thinking of party leaders. A vast majority of those
superdelegates --- whose ranks include federal elected officials, former
presidents and vice presidents and D.N.C. members --- predicted that no
candidate would clinch the nomination during the primaries, and that
there would be a brokered convention fight in July to choose a nominee.

In a reflection of the establishment's wariness about Mr. Sanders, only
nine of the 93 superdelegates interviewed said that Mr. Sanders should
become the nominee purely on the basis of arriving at the convention
with a plurality, if he was short of a majority.

\includegraphics{https://static01.graylady3jvrrxbe.onion/images/2020/02/27/us/politics/27superdelegates-mondale/merlin_169473342_cd3b76ba-fae5-45f7-9537-b845383e6bcc-articleLarge.jpg?quality=75\&auto=webp\&disable=upscale}

``I've had 60 years experience with Democratic delegates --- I don't
think they will do anything like that,'' said former Vice President
Walter Mondale, who is a superdelegate. ``They will each do what they
want to do, and somehow they will work it out. God knows how.''

\hypertarget{latest-updates-2020-election}{%
\section{\texorpdfstring{\href{https://www.nytimes3xbfgragh.onion/live/2020/08/19/us/dnc-convention-election?action=click\&pgtype=Article\&state=default\&region=MAIN_CONTENT_1\&context=storylines_live_updates}{Latest
Updates: 2020
Election}}{Latest Updates: 2020 Election}}\label{latest-updates-2020-election}}

\href{https://www.nytimes3xbfgragh.onion/live/2020/08/19/us/dnc-convention-election?action=click\&pgtype=Article\&state=default\&region=MAIN_CONTENT_1\&context=storylines_live_updates\#night-3-featured-more-policy-a-focus-on-women-and-a-full-throated-rejection-of-trump-by-his-predecessor}{7h
ago}

\href{https://www.nytimes3xbfgragh.onion/live/2020/08/19/us/dnc-convention-election?action=click\&pgtype=Article\&state=default\&region=MAIN_CONTENT_1\&context=storylines_live_updates\#night-3-featured-more-policy-a-focus-on-women-and-a-full-throated-rejection-of-trump-by-his-predecessor}{Night
3 featured more policy, a focus on women and a full-throated rejection
of Trump by his predecessor.}

\href{https://www.nytimes3xbfgragh.onion/live/2020/08/19/us/dnc-convention-election?action=click\&pgtype=Article\&state=default\&region=MAIN_CONTENT_1\&context=storylines_live_updates\#trump-live-tweeted-obamas-speech-tonight-hell-appear-on-fox-news-right-before-bidens-tomorrow}{9h
ago}

\href{https://www.nytimes3xbfgragh.onion/live/2020/08/19/us/dnc-convention-election?action=click\&pgtype=Article\&state=default\&region=MAIN_CONTENT_1\&context=storylines_live_updates\#trump-live-tweeted-obamas-speech-tonight-hell-appear-on-fox-news-right-before-bidens-tomorrow}{Trump
live-tweeted Obama's speech tonight. He'll appear on Fox News right
before Biden's tomorrow.}

\href{https://www.nytimes3xbfgragh.onion/live/2020/08/19/us/dnc-convention-election?action=click\&pgtype=Article\&state=default\&region=MAIN_CONTENT_1\&context=storylines_live_updates\#advocates-for-domestic-violence-survivors-praised-biden-in-a-video}{9h
ago}

\href{https://www.nytimes3xbfgragh.onion/live/2020/08/19/us/dnc-convention-election?action=click\&pgtype=Article\&state=default\&region=MAIN_CONTENT_1\&context=storylines_live_updates\#advocates-for-domestic-violence-survivors-praised-biden-in-a-video}{Advocates
for domestic violence survivors praised Biden in a video.}

\href{https://www.nytimes3xbfgragh.onion/live/2020/08/19/us/dnc-convention-election?action=click\&pgtype=Article\&state=default\&region=MAIN_CONTENT_1\&context=storylines_live_updates}{See
more updates}

As for his own vote, Mr. Mondale,
\href{https://www.nytimes3xbfgragh.onion/1984/07/20/us/transcript-of-mondale-address-accepting-party-nomination.html}{the
1984 Democratic presidential nominee}, said, ``I vote for the person I
think should be president.''

\href{https://www.nytimes3xbfgragh.onion/interactive/2020/us/elections/delegate-count-primary-results.html}{}

\includegraphics{https://static01.graylady3jvrrxbe.onion/images/2020/01/31/us/delegate-count-primary-results-promo-1580516316371/delegate-count-primary-results-promo-1580516316371-articleLarge-v80.png}

\hypertarget{democratic-delegate-count-and-primary-election-results-2020}{%
\subsection{Democratic Delegate Count and Primary Election Results
2020}\label{democratic-delegate-count-and-primary-election-results-2020}}

See how many delegates are available in each state.

\hypertarget{ive-had-60-years-experience-with-democratic-delegates-i-dont-think-they-will-do-anything-like-that}{%
\subsection{`I've had 60 years experience with Democratic delegates. I
don't think they will do anything like
that.'}\label{ive-had-60-years-experience-with-democratic-delegates-i-dont-think-they-will-do-anything-like-that}}

--- Former Vice President Walter Mondale

While there is no widespread public effort underway to undercut Mr.
Sanders, arresting his rise has emerged as the dominant topic in many
Democratic circles. Some are trying to act well before the convention:
Since Mr. Sanders won Nevada's caucuses on Saturday, four donors have
approached former Representative Steve Israel of New York to ask if he
can suggest someone to run a super PAC aimed at blocking Mr. Sanders. He
declined their offer.

``People are worried,'' said former Senator Chris Dodd of Connecticut, a
former Democratic National Committee chairman who in October endorsed
former Vice President Joseph R. Biden Jr. ``How you can spend four or
five months hoping you don't have to put a bumper sticker from that guy
on your car.''

That anxiety has led even superdelegates to suggest ideas that sound
ripped from
\href{https://www.nytimes3xbfgragh.onion/2019/12/29/us/politics/west-wing-politics.html}{the
pages of a political drama}.

In recent weeks, Democrats have placed a steady stream of calls to
Senator Sherrod Brown of Ohio, who
\href{https://www.nytimes3xbfgragh.onion/2019/03/07/us/politics/sherrod-brown-2020-president.html}{opted
against running for president nearly a year ago}, suggesting that he can
emerge as a white knight nominee at a brokered convention --- in part on
the theory that he may carry his home state in a general election.

``If you could get to a convention and pick Sherrod Brown, that would be
wonderful, but that's more like a novel,'' Representative Steve Cohen of
Tennessee said. ``Donald Trump's presidency is like a horror story, so
if you can have a horror story you might as well have a novel.''

Image

Representative Steve Cohen of Tennessee expressed enthusiasm about the
Democratic convention picking Senator Sherrod Brown as the party's
presidential nominee.Credit...Tom Brenner for The New York Times

Others are urging former President Barack Obama to get involved to
broker a truce --- either among the four moderate candidates or between
the Sanders and establishment wings, according to three people familiar
with those conversations.

William Owen, a D.N.C. member from Tennessee, suggested that if Mr.
Obama was unwilling,
\href{https://www.nytimes3xbfgragh.onion/2016/11/06/us/politics/michelle-obama-2016-presidential-election-first-lady.html}{his
wife, Michelle}, could be nominated as vice president, giving the party
a figure they could rally behind.

``She's the only person I can think of who can unify the party and help
us win,'' he said. ``This election is about saving the American
experiment as a republic. It's also about saving the world. This is not
an ordinary election.''

People close to Mr. Obama
\href{https://www.nytimes3xbfgragh.onion/2019/11/21/us/politics/obama-2020-democrats.html}{say
he has no intention of getting involved in the primary contest}, seeing
his role as less of a kingmaker than as a unifying figure to
\href{https://www.nytimes3xbfgragh.onion/2019/11/15/us/politics/barack-obama-2020-dems.html}{help
heal party divisions once Democrats settle on a nominee.} He also
believed that the Democratic Party shouldn't engage in smoke-filled-room
politics, arguing that those kinds of deals would have prevented him
from capturing the nomination when he ran against Hillary Clinton in
2008.

\hypertarget{this-election-is-about-saving-the-american-experiment-as-a-republic}{%
\subsection{`This election is about saving the American experiment as a
republic.'}\label{this-election-is-about-saving-the-american-experiment-as-a-republic}}

--- William Owen, a D.N.C. member from Tennessee

Officials at the Democratic National Committee maintain that it is
highly improbable to head to the convention without an assured nominee.
Historically, superdelegates had always supported the candidate who won
the most pledged delegates, which accrue from primary and caucus wins.
While those delegates are proportioned based on the results of those
elections, they are not legally bound --- meaning that they are
technically free to change their votes as the race progresses.

In recent days, both Mr. Biden and Senator Elizabeth Warren of
Massachusetts said that Mr. Sanders should not become the nominee if he
arrived at the convention short of a delegate majority. ``Bernie had a
big hand in writing these rules,'' Ms. Warren said during a CNN forum on
Wednesday night. ``I don't see how he thinks he gets to change them now
that he thinks there's an advantage for him.''

Slightly
\href{https://www.nytimes3xbfgragh.onion/interactive/2020/us/elections/delegate-count-primary-results.html}{less
than 3 percent of delegates have been allocated in the race so far}, and
Mr. Sanders, of course,
\href{https://www.nytimes3xbfgragh.onion/2020/02/12/upshot/bernie-sanders-2020-path.html}{can
win a majority}, making him the nominee. But while Mr. Sanders has
demonstrated momentum in the race, winning the most votes in each of the
first three contests, he has yet to show that he can expand his
coalition enough to set his campaign on a path to capturing the majority
of delegates. As a result, some within Mr. Sanders's own campaign
foresee a possible brokered convention.

The argument of Mr. Sanders and his allies --- that a plurality of
delegates should be sufficient to clinch the nomination --- is a
different standard than the one laid out in party rules that his team
helped draft two years ago. It's also
\href{https://www.nytimes3xbfgragh.onion/politics/first-draft/2016/05/01/bernie-sanders-says-superdelegates-should-follow-voters-will-in-landslide-states/}{a
reversal of their stance in 2016}, when Mr. Sanders encouraged
superdelegates to support him over Mrs. Clinton, who secured the
majority of pledged delegates.

``The will of the people should prevail,''
\href{https://www.nytimes3xbfgragh.onion/2020/02/22/us/politics/democratic-primary-dnc-superdelegates.html}{he
said when asked during last week's debate} if the candidate with the
most pledged delegates should be the Democratic nominee. ``The person
who has the most votes should become the nominee.''

Supporters of Mr. Sanders said that blocking him from the nomination if
he had the most delegates would repel progressives, and would deliver a
second term to Mr. Trump.

Image

Representative Pramila Jayapal of Washington state disagreed with the
notion of superdelegates weighing in on the nomination if Mr. Sanders
had only a plurality of delegates.Credit...Anna Moneymaker/The New York
Times

``If Bernie gets a plurality and nobody else is even close and the
superdelegates weigh in and say, `We know better than the voters,' I
think that will be a big problem,'' said Representative Pramila Jayapal
of Washington state, a Sanders supporter who is co-chairwoman of the
Congressional Progressive Caucus.

\hypertarget{we-dont-have-to-freak-out}{%
\subsection{`We don't have to freak
out.'}\label{we-dont-have-to-freak-out}}

--- Jane Kleeb, the Nebraska Democratic chairwoman

Not only would a messy convention fight risk alienating a sizable part
of the Democratic base that supports Mr. Sanders, it would also give
Republicans ammunition to use in the general election.

``We don't have to freak out,'' said
\href{https://www.nytimes3xbfgragh.onion/2014/05/18/magazine/jane-kleeb-vs-the-keystone-pipeline.html}{Jane
Kleeb, the Nebraska Democratic chairwoman}, who helped write Democrats'
presidential nominating rules and supported Mr. Sanders in 2016. ``We
shouldn't be second-guessing voters. If that's what our party leaders
are going to do, you'll see rebellion not just in the presidential race,
but in down-ballot races as well.''

Others in the party view Mr. Sanders as such an existential threat that
they see stopping him from winning the nomination as less risky than a
public convention fight. Many feared that putting Mr. Sanders on the top
of the ticket could cost Democrats the political gains of the Trump era,
a period when the party won control of the House, took governor's
mansions in deep red states and flipped statehouses across the country.

``Bernie seems to have declared war on the Democratic Party --- and it's
caused panic in the House ranks,'' said Representative Josh Gottheimer
of New Jersey, a supporter of former Mayor Michael Bloomberg of New
York. Private polling of Mr. Gottheimer's northern New Jersey district,
for example, shows a double-digit gap in the approval ratings of Mr.
Trump and Mr. Sanders.

Image

Representative Josh Gottheimer of New Jersey said that Mr. Sanders had
declared ``war'' on the Democratic Party.Credit...Bryan Anselm for The
New York Times

\href{https://www.nytimes3xbfgragh.onion/2019/07/11/us/politics/veronica-escobar-border-debate.html}{Representative
Veronica Escobar of Texas} said that if Mr. Sanders arrived at the
convention with 40 percent of the delegates, it wouldn't be enough to
convince her to vote for him on the second ballot.

``If 60 percent is not with Bernie Sanders, I think that says something,
I really do,'' she said.

Results in the Super Tuesday contests should give Democrats a strong
indication of where the nominating contest is headed.

\hypertarget{if-60-percent-is-not-with-bernie-sanders-i-think-that-says-something-i-really-do}{%
\subsection{`If 60 percent is not with Bernie Sanders, I think that says
something, I really
do.'}\label{if-60-percent-is-not-with-bernie-sanders-i-think-that-says-something-i-really-do}}

--- Representative Veronica Escobar of Texas

Should Mr. Sanders win big in the 16 states and territories holding
primaries and caucuses on Super Tuesday next week, he could be on a path
to the 1,991 pledged delegates needed to capture the nomination on the
first ballot at the party's convention. But if the Super Tuesday vote is
sharply divided among Mr. Sanders and two or more other rivals, the
Vermont senator could find himself with more delegates than the
competition but not enough to win the nomination outright.

Under the current rules, the convention would then go to a second
ballot. On that vote, all 3,979 pledged delegates and 771 superdelegates
would be free to vote for any candidate they chose.

That would give Democratic delegates a huge amount of power to determine
the nominee, setting off a fierce round of jockeying by the candidates
to win over 2,375.5 delegates and superdelegates. (Superdelegates from
Democrats Abroad count as half a vote each.)

``It is a mini primary process in the making,'' said Leah Daughtry, who
ran the party's 2008 and 2016 conventions. She's been
warnin\href{https://www.nytimes3xbfgragh.onion/2019/04/16/us/politics/bernie-sanders-democratic-party.html}{g
Democratic donors about the prospect of a contested convention for
nearly a year.} ``If you don't have a political operation that will get
you through a second ballot then what are you going to do in a
general?''

The campaigns are already strategizing about how they will handle a
protracted convention battle. Superdelegates, too, are brushing up on
the rules: Ms. Pelosi invited House Democrats to a meeting at D.N.C.
headquarters on Thursday to review the details of the convention
process.

``Whatever the atmosphere is, and I would hope that everyone would say,
no matter who the nominee is for president, we wholeheartedly embrace
that person,'' she said, in a private caucus meeting on Wednesday
morning, according to an aide in the room.

According to a person familiar with the private conversations, Mr.
Schumer told people he had so far stayed out of the primary because many
members of his caucus were running. He argued that there was one school
of thought that you needed to win the base and one that you needed to
bring new voters in, and said that he did not yet know which candidate
would be able to accomplish those goals.

A number of superdelegates dream of a savior candidate who is not now in
the race, perhaps Mr. Brown, or maybe someone who already dropped out
the race, like Senator Kamala Harris of California.

Representative Don Beyer of Virginia cast an even wider net, suggesting
senators from Virginia and Delaware, along with Ms. Pelosi, as possible
nominees.

``At some point you could imagine saying, `Let's go get Mark Warner,
Chris Coons, Nancy Pelosi,''' he said, while preparing to introduce the
former Mayor Pete Buttigieg of South Bend, Ind., at a campaign event
near his home on Sunday. ``Somebody that could win and we could all get
behind and celebrate.''

Stay up to date on primaries and caucuses. Subscribe to ``On Politics,''
and we'll send you a link to the live results.

\href{https://www.nytimes3xbfgragh.onion/newsletters/politics}{Sign up
for our politics newsletter}

\hypertarget{our-2020-election-guide}{%
\section{Our 2020 Election Guide}\label{our-2020-election-guide}}

Updated Aug. 20, 2020

\begin{itemize}
\item
  \begin{center}\rule{0.5\linewidth}{\linethickness}\end{center}

  \hypertarget{convention-recap}{%
  \subsection{Convention Recap}\label{convention-recap}}

  \begin{itemize}
  \tightlist
  \item
    Joe Biden accepted the Democratic nomination, urging Americans to
    have faith that they could
    \href{https://www.nytimes3xbfgragh.onion/2020/08/20/us/politics/Joe-Biden-accepts-democratic-nomination.html?action=click\&pgtype=Article\&state=default\&region=BELOW_MAIN_CONTENT\&context=storylines_guide}{``overcome
    this season of darkness.''}
  \end{itemize}
\item
  \begin{center}\rule{0.5\linewidth}{\linethickness}\end{center}

  \hypertarget{news-analysis}{%
  \subsection{News Analysis}\label{news-analysis}}

  \begin{itemize}
  \tightlist
  \item
    Looming over Mr. Biden's nomination was the ever-present shadow of
    another man who's poised to dominate the campaign:
    \href{https://www.nytimes3xbfgragh.onion/2020/08/20/us/politics/biden-dnc-speech-trump.html?action=click\&pgtype=Article\&state=default\&region=BELOW_MAIN_CONTENT\&context=storylines_guide}{Donald
    J. Trump}.
  \end{itemize}
\item
  \begin{center}\rule{0.5\linewidth}{\linethickness}\end{center}

  \hypertarget{keep-up-with-our-coverage}{%
  \subsection{Keep Up With Our
  Coverage}\label{keep-up-with-our-coverage}}

  \begin{itemize}
  \tightlist
  \item
    Get an
    \href{https://www.nytimes3xbfgragh.onion/newsletters/politics?action=click\&pgtype=Article\&state=default\&region=BELOW_MAIN_CONTENT\&context=storylines_guide}{email}
    recapping the day's news
  \end{itemize}

  \begin{itemize}
  \tightlist
  \item
    Download our mobile app on
    \href{https://apps.apple.com/us/app/nytimes/id284862083?ls=1\&mat_click_id=5c79ae7455014fd1bd66b5610c05b8f2-20191112-16948\&referrer=mat_click_id\%3D5c79ae7455014fd1bd66b5610c05b8f2-20191112-16948\%26link_click_id\%3D722930677036718082}{iOS}
    and
    \href{http://a.localytics.com/android?id=com.nytimes.android\&referrer=utm_source\%3Dother_nyt_mobile_web\%26utm_medium\%3DWeb\%2520page\%26utm_term\%3DGeneral\%2520Mobile\%2520Page\%26utm_campaign\%3DNYT\%2520Mobile\%2520General\%2520Page}{Android}
    and turn on Breaking News and Politics alerts
  \end{itemize}
\end{itemize}

Advertisement

\protect\hyperlink{after-bottom}{Continue reading the main story}

\hypertarget{site-index}{%
\subsection{Site Index}\label{site-index}}

\hypertarget{site-information-navigation}{%
\subsection{Site Information
Navigation}\label{site-information-navigation}}

\begin{itemize}
\tightlist
\item
  \href{https://help.nytimes3xbfgragh.onion/hc/en-us/articles/115014792127-Copyright-notice}{©~2020~The
  New York Times Company}
\end{itemize}

\begin{itemize}
\tightlist
\item
  \href{https://www.nytco.com/}{NYTCo}
\item
  \href{https://help.nytimes3xbfgragh.onion/hc/en-us/articles/115015385887-Contact-Us}{Contact
  Us}
\item
  \href{https://www.nytco.com/careers/}{Work with us}
\item
  \href{https://nytmediakit.com/}{Advertise}
\item
  \href{http://www.tbrandstudio.com/}{T Brand Studio}
\item
  \href{https://www.nytimes3xbfgragh.onion/privacy/cookie-policy\#how-do-i-manage-trackers}{Your
  Ad Choices}
\item
  \href{https://www.nytimes3xbfgragh.onion/privacy}{Privacy}
\item
  \href{https://help.nytimes3xbfgragh.onion/hc/en-us/articles/115014893428-Terms-of-service}{Terms
  of Service}
\item
  \href{https://help.nytimes3xbfgragh.onion/hc/en-us/articles/115014893968-Terms-of-sale}{Terms
  of Sale}
\item
  \href{https://spiderbites.nytimes3xbfgragh.onion}{Site Map}
\item
  \href{https://help.nytimes3xbfgragh.onion/hc/en-us}{Help}
\item
  \href{https://www.nytimes3xbfgragh.onion/subscription?campaignId=37WXW}{Subscriptions}
\end{itemize}
