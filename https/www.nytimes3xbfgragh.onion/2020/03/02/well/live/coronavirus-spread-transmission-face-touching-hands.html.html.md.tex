Sections

SEARCH

\protect\hyperlink{site-content}{Skip to
content}\protect\hyperlink{site-index}{Skip to site index}

\href{https://www.nytimes3xbfgragh.onion/section/well/live}{Live}

\href{https://myaccount.nytimes3xbfgragh.onion/auth/login?response_type=cookie\&client_id=vi}{}

\href{https://www.nytimes3xbfgragh.onion/section/todayspaper}{Today's
Paper}

\href{/section/well/live}{Live}\textbar{}Stop Touching Your Face!

\url{https://nyti.ms/2Tcty0i}

\begin{itemize}
\item
\item
\item
\item
\item
\item
\end{itemize}

\hypertarget{the-coronavirus-outbreak}{%
\subsubsection{\texorpdfstring{\href{https://www.nytimes3xbfgragh.onion/news-event/coronavirus?name=styln-coronavirus-national\&region=TOP_BANNER\&variant=undefined\&block=storyline_menu_recirc\&action=click\&pgtype=Article\&impression_id=34b59ff0-e392-11ea-b455-d1693a15a47a}{The
Coronavirus
Outbreak}}{The Coronavirus Outbreak}}\label{the-coronavirus-outbreak}}

\begin{itemize}
\tightlist
\item
  live\href{https://www.nytimes3xbfgragh.onion/2020/08/21/world/covid-19-coronavirus.html?name=styln-coronavirus-national\&region=TOP_BANNER\&variant=undefined\&block=storyline_menu_recirc\&action=click\&pgtype=Article\&impression_id=34b5c700-e392-11ea-b455-d1693a15a47a}{Latest
  Updates}
\item
  \href{https://www.nytimes3xbfgragh.onion/interactive/2020/us/coronavirus-us-cases.html?name=styln-coronavirus-national\&region=TOP_BANNER\&variant=undefined\&block=storyline_menu_recirc\&action=click\&pgtype=Article\&impression_id=34b5c701-e392-11ea-b455-d1693a15a47a}{Maps
  and Cases}
\item
  \href{https://www.nytimes3xbfgragh.onion/interactive/2020/science/coronavirus-vaccine-tracker.html?name=styln-coronavirus-national\&region=TOP_BANNER\&variant=undefined\&block=storyline_menu_recirc\&action=click\&pgtype=Article\&impression_id=34b5c702-e392-11ea-b455-d1693a15a47a}{Vaccine
  Tracker}
\item
  \href{https://www.nytimes3xbfgragh.onion/2020/08/19/us/colleges-closing-covid.html?name=styln-coronavirus-national\&region=TOP_BANNER\&variant=undefined\&block=storyline_menu_recirc\&action=click\&pgtype=Article\&impression_id=34b5c703-e392-11ea-b455-d1693a15a47a}{Colleges
  Closing}
\item
  \href{https://www.nytimes3xbfgragh.onion/live/2020/08/20/business/stock-market-today-coronavirus?name=styln-coronavirus-national\&region=TOP_BANNER\&variant=undefined\&block=storyline_menu_recirc\&action=click\&pgtype=Article\&impression_id=34b5c704-e392-11ea-b455-d1693a15a47a}{Economy}
\end{itemize}

Advertisement

\protect\hyperlink{after-top}{Continue reading the main story}

Supported by

\protect\hyperlink{after-sponsor}{Continue reading the main story}

\hypertarget{stop-touching-your-face}{%
\section{Stop Touching Your Face!}\label{stop-touching-your-face}}

It's a quirk of human nature that we touch our eyes, noses and mouths
all day long. It's also a major way we pick up infections like
coronavirus.

\includegraphics{https://static01.graylady3jvrrxbe.onion/images/2020/03/10/well/well-facetouch2/well-facetouch2-articleLarge.jpg?quality=75\&auto=webp\&disable=upscale}

\href{https://www.nytimes3xbfgragh.onion/by/tara-parker-pope}{\includegraphics{https://static01.graylady3jvrrxbe.onion/images/2018/07/10/us/Parker-Pope-Tara/Parker-Pope-Tara-thumbLarge.png}}

By \href{https://www.nytimes3xbfgragh.onion/by/tara-parker-pope}{Tara
Parker-Pope}

\begin{itemize}
\item
  Published March 2, 2020Updated March 6, 2020
\item
  \begin{itemize}
  \item
  \item
  \item
  \item
  \item
  \item
  \end{itemize}
\end{itemize}

Want to improve your chance of staying healthy? Stop touching your face!

One of the more difficult challenges in public health has been to teach
people to wash their hands frequently and to stop touching the facial
mucous membranes --- the eyes, nose and mouth, all entry portals for the
new coronavirus and many other germs.

``Scratching the nose, rubbing your eyes, leaning on your chin and your
fingers go next to your mouth --- there's multiple ways we do it,'' said
Dr. Nancy C. Elder, a professor of family medicine at Oregon Health and
Science University in Portland
\href{https://www.ncbi.nlm.nih.gov/pubmed/24808112}{who has studied face
touching} among doctors and clinic staff members. ``Everybody touches
their face, and it's a difficult habit to break.''

As communities prepare for the spread of coronavirus around the globe,
the primary advice from health officials is for people to wash their
hands. But a number of health researchers say the public health message
also should include a more forceful warning about face touching.

``The C.D.C. and W.H.O. still say something like `avoid' touching your
eyes, nose and mouth,'' said Dr. William P. Sawyer, a family physician
in Sharonville, Ohio, and creator of
\href{https://www.henrythehand.com/}{HenrytheHand.com}, which promotes
hand and face hygiene. ``The advice should be `absolutely do not touch
them!' If you never touch your facial mucous membranes, you're less
likely to be sick again from any viral respiratory infection.''

To understand why hand hygiene and face touching can make a meaningful
difference during a pandemic, consider how a virus can spread. An
infected person rides in an elevator, touching buttons both outside and
inside the elevator or maybe sneezing during the ride. When that person
leaves, microscopic droplets containing the virus stay behind. The next
people who press the same buttons or touch a surface pick up the virus
on their hands, then scratch their noses or rub their eyes.

``Eyes, nose, mouth --- all those mucous membranes are the portal into
the body for a virus like Covid-19 or SARS,'' said Mary-Louise McLaws,
professor of epidemiology, health care infection and infectious diseases
control at the University of New South Wales in Sydney, Australia.

\hypertarget{latest-updates-the-coronavirus-outbreak}{%
\section{\texorpdfstring{\href{https://www.nytimes3xbfgragh.onion/2020/08/21/world/covid-19-coronavirus.html?action=click\&pgtype=Article\&state=default\&region=MAIN_CONTENT_1\&context=storylines_live_updates}{Latest
Updates: The Coronavirus
Outbreak}}{Latest Updates: The Coronavirus Outbreak}}\label{latest-updates-the-coronavirus-outbreak}}

Updated 2020-08-21T09:36:59.270Z

\begin{itemize}
\tightlist
\item
  \href{https://www.nytimes3xbfgragh.onion/2020/08/21/world/covid-19-coronavirus.html?action=click\&pgtype=Article\&state=default\&region=MAIN_CONTENT_1\&context=storylines_live_updates\#link-4690b6aa}{Shutdowns,
  warnings and scoldings follow gatherings on college campuses.}
\item
  \href{https://www.nytimes3xbfgragh.onion/2020/08/21/world/covid-19-coronavirus.html?action=click\&pgtype=Article\&state=default\&region=MAIN_CONTENT_1\&context=storylines_live_updates\#link-324af071}{As
  he accepts the Democratic nomination, Biden knocks Trump's pandemic
  response.}
\item
  \href{https://www.nytimes3xbfgragh.onion/2020/08/21/world/covid-19-coronavirus.html?action=click\&pgtype=Article\&state=default\&region=MAIN_CONTENT_1\&context=storylines_live_updates\#link-35890b73}{Hundreds
  of doctors in Kenya go on strike over their pay and protective gear.}
\end{itemize}

\href{https://www.nytimes3xbfgragh.onion/2020/08/21/world/covid-19-coronavirus.html?action=click\&pgtype=Article\&state=default\&region=MAIN_CONTENT_1\&context=storylines_live_updates}{See
more updates}

More live coverage:
\href{https://www.nytimes3xbfgragh.onion/live/2020/08/20/business/stock-market-today-coronavirus?action=click\&pgtype=Article\&state=default\&region=MAIN_CONTENT_1\&context=storylines_live_updates}{Markets}

**``**I was in a conference yesterday watching people, and in just about
two minutes I counted a dozen times that I saw someone touching mucous
membranes,'' Dr. McLaws said. ``It is a very common practice. We rub our
eyes, scratch our nose, touch our mouth --- the general community needs
to be aware of how often they are touching their face.''

Dr. McLaws was the senior author of a
\href{https://www.ncbi.nlm.nih.gov/pubmed/25637115}{2015 study} on face
touching that documented the alarming number of times we do it. While
medical students attended a lecture, the researchers filmed them and
counted the number of times they touched any part of their faces. Over
the course of an hour, students touched their faces, on average, 23
times. Nearly half of the touches were to the eyes, nose or mouth ---
what infectious disease researchers call ``the T-zone.''

Other studies of
\href{https://www.jabfm.org/content/27/3/339/tab-article-info}{primary
care doctors}, people doing
\href{https://www.tandfonline.com/doi/full/10.1080/15459620802003896}{office
work}, and students riding
\href{https://search.proquest.com/openview/e2c136dc5c05115cdf9e55cb5133bc46/1?cbl=2040555\&pq-origsite=gscholar}{a
simulated rail car} have all found similar rates of touching the T-zone.

``I was really surprised,'' Dr. McLaws said. ``By touching your mucous
membranes, you're giving a virus 11 opportunities every hour if you've
touched something infectious.''

The risk of picking up a virus by hand-to-face contact depends on a
number of factors, including the type of virus, whether the surface was
nonporous, how long ago the virus was left behind, how much time the
infected person spent in the area and the temperature and humidity
levels.

The
\href{https://www.who.int/news-room/q-a-detail/q-a-coronaviruses}{World
Health Organization} notes that while we don't know how long the new
coronavirus survives on surfaces, it seems to behave like other
coronaviruses --- which is unsettling news. A recent study from the
\href{https://www.journalofhospitalinfection.com/article/S0195-6701(20)30046-3/fulltext}{Journal
of Hospital Infection} found that similar coronaviruses have been shown
to survive on surfaces for as long as nine days under ideal conditions.
That's far longer than the flu virus, which typically can survive under
ideal conditions only up to
\href{https://www.nhs.uk/common-health-questions/infections/how-long-do-bacteria-and-viruses-live-outside-the-body/}{24
hours on hard surfaces}.
\href{https://www.gov.uk/government/publications/covid-19-decontamination-in-non-healthcare-settings/covid-19-decontamination-in-non-healthcare-settings}{Public
Health England} says that, based on studies of other coronaviruses like
SARS and MERS, ``the risk of picking up a live virus from a contaminated
surface'' under real-life conditions ``is likely to be reduced
significantly after 72 hours.''

In general, a virus will survive the longest on nonporous surfaces made
of metal and plastics --- including door knobs, counters and railings. A
virus will die sooner on fabrics or tissues. Once on your hand, a virus
begins to lose potency, but it will probably live long enough for you to
touch your face. Although more study is needed of coronavirus,
\href{https://www.ncbi.nlm.nih.gov/pubmed/1658033}{in one study of
rhinovirus,} which causes the common cold, a small dose of virus was
placed on a participant's finger. An hour later, about 40 percent of the
virus was still viable. After three hours, 16 percent could still be
detected.

\href{https://www.nytimes3xbfgragh.onion/news-event/coronavirus?action=click\&pgtype=Article\&state=default\&region=MAIN_CONTENT_3\&context=storylines_faq}{}

\hypertarget{the-coronavirus-outbreak-}{%
\subsubsection{The Coronavirus Outbreak
›}\label{the-coronavirus-outbreak-}}

\hypertarget{frequently-asked-questions}{%
\paragraph{Frequently Asked
Questions}\label{frequently-asked-questions}}

Updated August 17, 2020

\begin{itemize}
\item ~
  \hypertarget{why-does-standing-six-feet-away-from-others-help}{%
  \paragraph{Why does standing six feet away from others
  help?}\label{why-does-standing-six-feet-away-from-others-help}}

  \begin{itemize}
  \tightlist
  \item
    The coronavirus spreads primarily through droplets from your mouth
    and nose, especially when you cough or sneeze. The C.D.C., one of
    the organizations using that measure,
    \href{https://www.nytimes3xbfgragh.onion/2020/04/14/health/coronavirus-six-feet.html?action=click\&pgtype=Article\&state=default\&region=MAIN_CONTENT_3\&context=storylines_faq}{bases
    its recommendation of six feet} on the idea that most large droplets
    that people expel when they cough or sneeze will fall to the ground
    within six feet. But six feet has never been a magic number that
    guarantees complete protection. Sneezes, for instance, can launch
    droplets a lot farther than six feet,
    \href{https://jamanetwork.com/journals/jama/fullarticle/2763852}{according
    to a recent study}. It's a rule of thumb: You should be safest
    standing six feet apart outside, especially when it's windy. But
    keep a mask on at all times, even when you think you're far enough
    apart.
  \end{itemize}
\item ~
  \hypertarget{i-have-antibodies-am-i-now-immune}{%
  \paragraph{I have antibodies. Am I now
  immune?}\label{i-have-antibodies-am-i-now-immune}}

  \begin{itemize}
  \tightlist
  \item
    As of right
    now,\href{https://www.nytimes3xbfgragh.onion/2020/07/22/health/covid-antibodies-herd-immunity.html?action=click\&pgtype=Article\&state=default\&region=MAIN_CONTENT_3\&context=storylines_faq}{that
    seems likely, for at least several months.} There have been
    frightening accounts of people suffering what seems to be a second
    bout of Covid-19. But experts say these patients may have a
    drawn-out course of infection, with the virus taking a slow toll
    weeks to months after initial exposure. People infected with the
    coronavirus typically
    \href{https://www.nature.com/articles/s41586-020-2456-9}{produce}
    immune molecules called antibodies, which are
    \href{https://www.nytimes3xbfgragh.onion/2020/05/07/health/coronavirus-antibody-prevalence.html?action=click\&pgtype=Article\&state=default\&region=MAIN_CONTENT_3\&context=storylines_faq}{protective
    proteins made in response to an
    infection}\href{https://www.nytimes3xbfgragh.onion/2020/05/07/health/coronavirus-antibody-prevalence.html?action=click\&pgtype=Article\&state=default\&region=MAIN_CONTENT_3\&context=storylines_faq}{.
    These antibodies may} last in the body
    \href{https://www.nature.com/articles/s41591-020-0965-6}{only two to
    three months}, which may seem worrisome, but that's perfectly normal
    after an acute infection subsides, said Dr. Michael Mina, an
    immunologist at Harvard University. It may be possible to get the
    coronavirus again, but it's highly unlikely that it would be
    possible in a short window of time from initial infection or make
    people sicker the second time.
  \end{itemize}
\item ~
  \hypertarget{im-a-small-business-owner-can-i-get-relief}{%
  \paragraph{I'm a small-business owner. Can I get
  relief?}\label{im-a-small-business-owner-can-i-get-relief}}

  \begin{itemize}
  \tightlist
  \item
    The
    \href{https://www.nytimes3xbfgragh.onion/article/small-business-loans-stimulus-grants-freelancers-coronavirus.html?action=click\&pgtype=Article\&state=default\&region=MAIN_CONTENT_3\&context=storylines_faq}{stimulus
    bills enacted in March} offer help for the millions of American
    small businesses. Those eligible for aid are businesses and
    nonprofit organizations with fewer than 500 workers, including sole
    proprietorships, independent contractors and freelancers. Some
    larger companies in some industries are also eligible. The help
    being offered, which is being managed by the Small Business
    Administration, includes the Paycheck Protection Program and the
    Economic Injury Disaster Loan program. But lots of folks have
    \href{https://www.nytimes3xbfgragh.onion/interactive/2020/05/07/business/small-business-loans-coronavirus.html?action=click\&pgtype=Article\&state=default\&region=MAIN_CONTENT_3\&context=storylines_faq}{not
    yet seen payouts.} Even those who have received help are confused:
    The rules are draconian, and some are stuck sitting on
    \href{https://www.nytimes3xbfgragh.onion/2020/05/02/business/economy/loans-coronavirus-small-business.html?action=click\&pgtype=Article\&state=default\&region=MAIN_CONTENT_3\&context=storylines_faq}{money
    they don't know how to use.} Many small-business owners are getting
    less than they expected or
    \href{https://www.nytimes3xbfgragh.onion/2020/06/10/business/Small-business-loans-ppp.html?action=click\&pgtype=Article\&state=default\&region=MAIN_CONTENT_3\&context=storylines_faq}{not
    hearing anything at all.}
  \end{itemize}
\item ~
  \hypertarget{what-are-my-rights-if-i-am-worried-about-going-back-to-work}{%
  \paragraph{What are my rights if I am worried about going back to
  work?}\label{what-are-my-rights-if-i-am-worried-about-going-back-to-work}}

  \begin{itemize}
  \tightlist
  \item
    Employers have to provide
    \href{https://www.osha.gov/SLTC/covid-19/standards.html}{a safe
    workplace} with policies that protect everyone equally.
    \href{https://www.nytimes3xbfgragh.onion/article/coronavirus-money-unemployment.html?action=click\&pgtype=Article\&state=default\&region=MAIN_CONTENT_3\&context=storylines_faq}{And
    if one of your co-workers tests positive for the coronavirus, the
    C.D.C.} has said that
    \href{https://www.cdc.gov/coronavirus/2019-ncov/community/guidance-business-response.html}{employers
    should tell their employees} -\/- without giving you the sick
    employee's name -\/- that they may have been exposed to the virus.
  \end{itemize}
\item ~
  \hypertarget{what-is-school-going-to-look-like-in-september}{%
  \paragraph{What is school going to look like in
  September?}\label{what-is-school-going-to-look-like-in-september}}

  \begin{itemize}
  \tightlist
  \item
    It is unlikely that many schools will return to a normal schedule
    this fall, requiring the grind of
    \href{https://www.nytimes3xbfgragh.onion/2020/06/05/us/coronavirus-education-lost-learning.html?action=click\&pgtype=Article\&state=default\&region=MAIN_CONTENT_3\&context=storylines_faq}{online
    learning},
    \href{https://www.nytimes3xbfgragh.onion/2020/05/29/us/coronavirus-child-care-centers.html?action=click\&pgtype=Article\&state=default\&region=MAIN_CONTENT_3\&context=storylines_faq}{makeshift
    child care} and
    \href{https://www.nytimes3xbfgragh.onion/2020/06/03/business/economy/coronavirus-working-women.html?action=click\&pgtype=Article\&state=default\&region=MAIN_CONTENT_3\&context=storylines_faq}{stunted
    workdays} to continue. California's two largest public school
    districts --- Los Angeles and San Diego --- said on July 13, that
    \href{https://www.nytimes3xbfgragh.onion/2020/07/13/us/lausd-san-diego-school-reopening.html?action=click\&pgtype=Article\&state=default\&region=MAIN_CONTENT_3\&context=storylines_faq}{instruction
    will be remote-only in the fall}, citing concerns that surging
    coronavirus infections in their areas pose too dire a risk for
    students and teachers. Together, the two districts enroll some
    825,000 students. They are the largest in the country so far to
    abandon plans for even a partial physical return to classrooms when
    they reopen in August. For other districts, the solution won't be an
    all-or-nothing approach.
    \href{https://bioethics.jhu.edu/research-and-outreach/projects/eschool-initiative/school-policy-tracker/}{Many
    systems}, including the nation's largest, New York City, are
    devising
    \href{https://www.nytimes3xbfgragh.onion/2020/06/26/us/coronavirus-schools-reopen-fall.html?action=click\&pgtype=Article\&state=default\&region=MAIN_CONTENT_3\&context=storylines_faq}{hybrid
    plans} that involve spending some days in classrooms and other days
    online. There's no national policy on this yet, so check with your
    municipal school system regularly to see what is happening in your
    community.
  \end{itemize}
\end{itemize}

We also know from the 2003 epidemic of SARS, a more deadly coronavirus
than the one currently spreading, that the virus was often transmitted
from surface contact. In one Hong Kong hotel, an
\href{https://www.who.int/csr/don/2003_07_04/en/}{infected doctor} who
checked into his room on the ninth floor before going to the hospital
for treatment left a trail of virus that infected at least seven people
who also had rooms on the ninth floor, who then went on to spread the
disease elsewhere. The doctor, who died from the infection, was later
identified as a ``super spreader''
\href{https://www.cnbc.com/2020/02/14/hong-kong-hotel-hosted-super-spreader-in-the-2003-sars-outbreak.html}{linked
to about 4,000}cases of SARS that occurred during the epidemic.

The good news is that frequent hand washing can make a meaningful
difference in lowering your risk. During the SARS epidemic, hand-washing
reduced the risk of transmission by 30 to 50 percent. But after washing
your hands, you must still be mindful about face touching, Dr. Sawyer
said.

``Your hands are only clean until the next surface you touch,'' he said.
``When you reach for the door knob or hand railing, you've
recontaminated your hand with something. If you touch your mucous
membranes, then you could inoculate yourself inadvertently with that
organism. If there is one behavior change that could prevent infection,
it's do not touch your T-zone.''

But it's not easy to stop face touching. In fact, many people say that
the more they think about it, the more their eyes twitch and their nose
itches.
\href{https://mashable.com/article/coronavirus-prevention-dont-touch-your-face/}{A
number of memes} have emerged on social media from people who say that
ever since the warnings about coronavirus, they can't stop touching
their own faces.

Only \href{https://www.ncbi.nlm.nih.gov/pubmed/6427652}{humans and a few
primates} (gorillas, orangutans and chimpanzees) are known to touch
their faces with little or no awareness of the habit. (Most animals
touch their faces only to groom or swat away a pest.)
\href{https://www.ncbi.nlm.nih.gov/pubmed/24530432}{German researchers}
analyzed the brain's electrical activity before and after spontaneous
face touching, and their findings suggested that we touch our faces as a
way to relieve stress and manage our emotions.

To break the face-touching habit, try using a tissue if you need to
scratch your nose or rub your eyes. Wearing makeup may reduce face
touching, since it may make you more mindful of not smudging it.
\href{https://search.proquest.com/openview/e2c136dc5c05115cdf9e55cb5133bc46/1?cbl=2040555\&pq-origsite=gscholar}{One
study} found that women touched their faces far less when they wore
makeup. Another solution: Try to identify triggers for face touching,
like dry skin or itchy eyes, and use moisturizers or eye drops to treat
those conditions so you are less likely to rub or scratch your face.

It also may help to wear glasses to create a barrier to touching your
eyes. Gloves or mittens can also make you more mindful of not touching
your face (and can make it more difficult to put your finger in your
nose or your eye). Although gloves, too, can become contaminated,
viruses don't live as long on fabric or leather.

Given that face touching is a long-ingrained habit, it makes sense to
remain vigilant about frequent hand washing and wipe down your desk,
phones and community surfaces. Carry hand sanitizer and use it often.
The more mindful you are about regular hand washing, the more mindful
you will be about your hands and what they are touching.

Advertisement

\protect\hyperlink{after-bottom}{Continue reading the main story}

\hypertarget{site-index}{%
\subsection{Site Index}\label{site-index}}

\hypertarget{site-information-navigation}{%
\subsection{Site Information
Navigation}\label{site-information-navigation}}

\begin{itemize}
\tightlist
\item
  \href{https://help.nytimes3xbfgragh.onion/hc/en-us/articles/115014792127-Copyright-notice}{©~2020~The
  New York Times Company}
\end{itemize}

\begin{itemize}
\tightlist
\item
  \href{https://www.nytco.com/}{NYTCo}
\item
  \href{https://help.nytimes3xbfgragh.onion/hc/en-us/articles/115015385887-Contact-Us}{Contact
  Us}
\item
  \href{https://www.nytco.com/careers/}{Work with us}
\item
  \href{https://nytmediakit.com/}{Advertise}
\item
  \href{http://www.tbrandstudio.com/}{T Brand Studio}
\item
  \href{https://www.nytimes3xbfgragh.onion/privacy/cookie-policy\#how-do-i-manage-trackers}{Your
  Ad Choices}
\item
  \href{https://www.nytimes3xbfgragh.onion/privacy}{Privacy}
\item
  \href{https://help.nytimes3xbfgragh.onion/hc/en-us/articles/115014893428-Terms-of-service}{Terms
  of Service}
\item
  \href{https://help.nytimes3xbfgragh.onion/hc/en-us/articles/115014893968-Terms-of-sale}{Terms
  of Sale}
\item
  \href{https://spiderbites.nytimes3xbfgragh.onion}{Site Map}
\item
  \href{https://help.nytimes3xbfgragh.onion/hc/en-us}{Help}
\item
  \href{https://www.nytimes3xbfgragh.onion/subscription?campaignId=37WXW}{Subscriptions}
\end{itemize}
