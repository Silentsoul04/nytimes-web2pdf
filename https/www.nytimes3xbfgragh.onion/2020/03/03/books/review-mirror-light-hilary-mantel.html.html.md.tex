Sections

SEARCH

\protect\hyperlink{site-content}{Skip to
content}\protect\hyperlink{site-index}{Skip to site index}

\href{https://www.nytimes3xbfgragh.onion/section/books}{Books}

\href{https://myaccount.nytimes3xbfgragh.onion/auth/login?response_type=cookie\&client_id=vi}{}

\href{https://www.nytimes3xbfgragh.onion/section/todayspaper}{Today's
Paper}

\href{/section/books}{Books}\textbar{}Hilary Mantel's Triumphant New
Novel Brings Thomas Cromwell Across the Finish Line

\url{https://nyti.ms/32KmtHs}

\begin{itemize}
\item
\item
\item
\item
\item
\item
\end{itemize}

Advertisement

\protect\hyperlink{after-top}{Continue reading the main story}

Supported by

\protect\hyperlink{after-sponsor}{Continue reading the main story}

\href{/column/books-of-the-times}{Books of The Times}

\hypertarget{hilary-mantels-triumphant-new-novel-brings-thomas-cromwell-across-the-finish-line}{%
\section{Hilary Mantel's Triumphant New Novel Brings Thomas Cromwell
Across the Finish
Line}\label{hilary-mantels-triumphant-new-novel-brings-thomas-cromwell-across-the-finish-line}}

By \href{https://www.nytimes3xbfgragh.onion/by/parul-sehgal}{Parul
Sehgal}

\begin{itemize}
\item
  March 3, 2020
\item
  \begin{itemize}
  \item
  \item
  \item
  \item
  \item
  \item
  \end{itemize}
\end{itemize}

\includegraphics{https://static01.graylady3jvrrxbe.onion/images/2020/03/04/books/03bookmantel1/03bookmantel1-articleLarge.jpg?quality=75\&auto=webp\&disable=upscale}

Buy Book ▾

\begin{itemize}
\tightlist
\item
  \href{https://www.amazon.com/gp/search?index=books\&tag=NYTBSREV-20\&field-keywords=The+Mirror+and+the+Light+Hilary+Mantel}{Amazon}
\item
  \href{https://du-gae-books-dot-nyt-du-prd.appspot.com/buy?title=The+Mirror+and+the+Light\&author=Hilary+Mantel}{Apple
  Books}
\item
  \href{https://www.anrdoezrs.net/click-7990613-11819508?url=https\%3A\%2F\%2Fwww.barnesandnoble.com\%2Fw\%2F\%3Fean\%3D9780805096606}{Barnes
  and Noble}
\item
  \href{https://www.anrdoezrs.net/click-7990613-35140?url=https\%3A\%2F\%2Fwww.booksamillion.com\%2Fp\%2FThe\%2BMirror\%2Band\%2Bthe\%2BLight\%2FHilary\%2BMantel\%2F9780805096606}{Books-A-Million}
\item
  \href{https://bookshop.org/a/3546/9780805096606}{Bookshop}
\item
  \href{https://www.indiebound.org/book/9780805096606?aff=NYT}{Indiebound}
\end{itemize}

When you purchase an independently reviewed book through our site, we
earn an affiliate commission.

How suitably some writers come named. Muriel Spark, of the scorching
short fiction. Judy Blume, of stories of young girls coming of age. Ann
Patchett, in whose work families desperately try to repair their
tattered ties.

Then there is Hilary Mantel, the author of several books, including an
acclaimed suite of novels set in Tudor England, in whose own name can be
discerned her themes --- of cloaking and secrecy, the weight of
responsibility --- and, as it happens, the particular pleasure of
submitting to her lavish and gory imagination.

When a hawk makes a kill, it drapes its wings over its prey, concealing
it from other predators. This gesture is called ``mantling,'' and it's a
fine description of reading Mantel's work. The world is blotted out as
you are enveloped in the sweep of a story rich with conquest, conspiracy
and mazy human psychology.

``The Mirror and the Light'' is the triumphant capstone to Mantel's
trilogy on Thomas Cromwell, the son of a blacksmith who rose to become
the consigliere of Henry VIII and architect of the English Reformation.
It's a story that could be the stuff of venerable and fusty historical
fiction, but Mantel clears away the cobwebs. Missing are the archaic
dialogue and laborious scene-setting, the dense clutter of details
signaling diligent research. ``Wolf Hall,''
\href{https://www.nytimes3xbfgragh.onion/2009/10/05/books/05maslin.html}{the
first book in the series}, begins mid-scene, in a galloping present
tense. Cromwell is a teenager and lying in his own vomit, enduring one
of his father's brutal beatings. His eyes leak blood; he is frightened
that he has gone blind. It is the last time we will encounter him with
his vision obscured.

He flees home, and from that moment on, he will distinguish himself with
an ability to perceive a path for himself, to seize opportunity in the
midst of chaos. He searches for the nearest war and enlists as a
mercenary (``at least, as a soldier of King Louis, he was paid to
receive blows''). He serves in the counting houses of Italy and the
banks of Antwerp, rises to become the right hand of the doomed Cardinal
Wolsey and later the king himself.

{[} \emph{Read}
\href{https://www.nytimes3xbfgragh.onion/2020/02/24/books/hilary-mantel-mirror-and-the-light-thomas-cromwell.html}{\emph{The
Times's profile}} \emph{of Hilary Mantel.} {]}

Mantel is often grouped with writers of historical fiction, like
Penelope Fitzgerald. The more apt, and useful, comparison might be with
Robert Caro, the biographer of Robert Moses and Lyndon Johnson, the
great anatomizer of political power. They share an interest in what
power reveals and conceals in human character. Intimate knowledge, of
the self and others, is among Cromwell's most reliable weapons. ``Tell
me what you want, I'll get it'' is his mantra. But his subtlety, his
crooked genius, lies in never needing to be told; he divines what you
desire, before you know it yourself. He can render himself
indispensable, and perhaps start suggesting to you what you
\emph{should} want.

\includegraphics{https://static01.graylady3jvrrxbe.onion/images/2020/03/04/books/04Book-Mantel/04Book-Mantel-articleLarge.jpg?quality=75\&auto=webp\&disable=upscale}

The curtain here rises on Cromwell in 1536. He is 50 years old, rich
beyond all his imagining and very much alone. A sickness carried off his
wife and two daughters years ago. He is the king's chief confidant and
fixer, although his primary duty is now ``to get the king new wives and
dispose of the old.'' He admits: ``I am running out of ladies.''

Henry is desperate for a male heir. He divorced his first wife,
Katherine of Aragon --- forcing a split between Rome and the Church of
England. His second wife, Anne Boleyn, mother of Elizabeth and the
charismatic antagonist of the first two novels, has been beheaded for
adultery. Cromwell was instrumental in bringing Anne to trial, which we
saw in the second volume,
\href{https://www.nytimes3xbfgragh.onion/2012/05/27/books/review/bring-up-the-bodies-by-hilary-mantel.html}{``Bring
Up the Bodies.''} He rounded up her alleged lovers (including her
brother) and squeezed out confessions. Whether they were true or trumped
up is never known. A blacksmith makes his own tools, Mantel writes.
Rumor and insinuation are Cromwell's favorites; he throws what mud he
can and sees what sticks.

When ``The Mirror and the Light'' begins, Cromwell has never been more
beloved to the king, who is now free to marry the docile Jane Seymour.
It is the beginning of Cromwell's undoing. A man who has worked in the
shadows is now too visible, envied and feared. Tax rebellions spring up
in the north. There's a movement to advance Henry's first child, Mary,
as his heir. Seymour dies, after producing a son, and another wife must
be sought. Cromwell chooses one for the king, and chooses badly. Anne of
Cleves, selected to broker an alliance with Germany, cannot hide her
distaste for Henry. Worry and guilt begin to gnaw at Cromwell. ``Dead
queens blink at him, from behind their broken mirrors.'' Wolves gather
outside his door.

At more than 700 pages, this is the longest book in the series, the most
mournful --- and the slackest. It lacks the formal play (and humor) of
``Wolf Hall'' and the ruthless compression of ``Bring Up the Bodies,''
which tracked the events of just one year, culminating in Boleyn's
beheading. ``I like books in which things happen,'' Mantel said in her
2015 Paris Review interview. ``I don't have pages and pages in which I
say, `Cromwell thought.' I tell you what he says, I tell you what he
does, and you read between the lines.''

This new novel is a different creature --- \emph{Cromwell} is a
different creature, less tentative and more ruminative. For pages and
pages, he dwells on the past, on his childhood --- that beating on the
cobblestones. The startling**,** bony style of the first two books has
been abandoned. The prose is plush, the sentences longer and more
adorned, tricked out with little tassels and extended metaphors. Even as
certain pages proved a slog, certain scenes repetitive, even as I
entertained heretical thoughts about pruning certain sections, or
striking them entirely, these choices follow a certain logic.

This is not a younger man's book, not a book of striving. It is a novel
of late middle age, a novel of preserving what one has seized --- of
fighting off young, hungry men who remind you of yourself, who will use
your own methods against you. Above all, it is a novel of living with
the dead. Mantel names the deceased in her dramatis personae at the
beginning of her books; how that list has grown.

``A man's power is in the half-light,'' Cromwell thinks to himself. ``It
is the absence of facts that frightens people: the gap you open, into
which they pour their fears, fantasies, desires.'' This is the art of
the tactician but also the novelist. For more than a decade, Mantel has
immersed her readers in the life of Cromwell, writing very close to the
historical record and correcting the record where she has deemed
necessary. It is too facile to regard the man as a Machiavellian monster
of self-interest, she has argued. In his road-building projects that
employed the poor lie the foundations of the welfare state. The son of a
blacksmith broke with the orthodoxy of the time to insist that poverty
was the product of circumstance, not character. But the magnetism of
these books derives not from its mountain of facts but from its elisions
--- all those gaps in our knowledge and understanding. How do we square
Cromwell's cruelty with his intense, almost disabling desire to protect
the vulnerable, all the lost, ``roaring boys'' he gave quarter and
instruction? Has any character seen women, \emph{liked} women, enjoyed
their conversation and stratagems quite so much as he? Has anyone used
them with such cold efficiency?

When we praise characters, we often say they ``feel real.'' But the
enduring characters are always elusive. What \emph{does} Ibsen's Hedda
Gabler want? Or Elena Ferrante's Lila? They cannot be so easily
reconciled, nor do they stay neatly quarantined in their narratives.
They trouble us; they haunt. Their questions become our own. No more so
than with Cromwell, one of literature's great ambiguous characters. Till
the end, he is full of surprises. When the wolves finally catch up to
him and haul him off to the Tower of London on charges of treason, the
plot and arrest are handled so smoothly, he can't help but marvel: ``You
would think he had done it himself.''

Advertisement

\protect\hyperlink{after-bottom}{Continue reading the main story}

\hypertarget{site-index}{%
\subsection{Site Index}\label{site-index}}

\hypertarget{site-information-navigation}{%
\subsection{Site Information
Navigation}\label{site-information-navigation}}

\begin{itemize}
\tightlist
\item
  \href{https://help.nytimes3xbfgragh.onion/hc/en-us/articles/115014792127-Copyright-notice}{©~2020~The
  New York Times Company}
\end{itemize}

\begin{itemize}
\tightlist
\item
  \href{https://www.nytco.com/}{NYTCo}
\item
  \href{https://help.nytimes3xbfgragh.onion/hc/en-us/articles/115015385887-Contact-Us}{Contact
  Us}
\item
  \href{https://www.nytco.com/careers/}{Work with us}
\item
  \href{https://nytmediakit.com/}{Advertise}
\item
  \href{http://www.tbrandstudio.com/}{T Brand Studio}
\item
  \href{https://www.nytimes3xbfgragh.onion/privacy/cookie-policy\#how-do-i-manage-trackers}{Your
  Ad Choices}
\item
  \href{https://www.nytimes3xbfgragh.onion/privacy}{Privacy}
\item
  \href{https://help.nytimes3xbfgragh.onion/hc/en-us/articles/115014893428-Terms-of-service}{Terms
  of Service}
\item
  \href{https://help.nytimes3xbfgragh.onion/hc/en-us/articles/115014893968-Terms-of-sale}{Terms
  of Sale}
\item
  \href{https://spiderbites.nytimes3xbfgragh.onion}{Site Map}
\item
  \href{https://help.nytimes3xbfgragh.onion/hc/en-us}{Help}
\item
  \href{https://www.nytimes3xbfgragh.onion/subscription?campaignId=37WXW}{Subscriptions}
\end{itemize}
