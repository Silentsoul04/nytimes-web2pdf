Sections

SEARCH

\protect\hyperlink{site-content}{Skip to
content}\protect\hyperlink{site-index}{Skip to site index}

\href{https://www.nytimes3xbfgragh.onion/section/technology}{Technology}

\href{https://myaccount.nytimes3xbfgragh.onion/auth/login?response_type=cookie\&client_id=vi}{}

\href{https://www.nytimes3xbfgragh.onion/section/todayspaper}{Today's
Paper}

\href{/section/technology}{Technology}\textbar{}Before Clearview Became
a Police Tool, It Was a Secret Plaything of the Rich

\url{https://nyti.ms/2TCudHq}

\begin{itemize}
\item
\item
\item
\item
\item
\item
\end{itemize}

Advertisement

\protect\hyperlink{after-top}{Continue reading the main story}

Supported by

\protect\hyperlink{after-sponsor}{Continue reading the main story}

\hypertarget{before-clearview-became-a-police-tool-it-was-a-secret-plaything-of-the-rich}{%
\section{Before Clearview Became a Police Tool, It Was a Secret
Plaything of the
Rich}\label{before-clearview-became-a-police-tool-it-was-a-secret-plaything-of-the-rich}}

Investors and clients of the facial recognition start-up freely used the
app on dates and at parties --- and to spy on the public.

\includegraphics{https://static01.graylady3jvrrxbe.onion/images/2020/03/09/business/05CLEARVIEW-BACKERS1-print/merlin_157126041_1f1fab82-a42e-41ff-911b-056c2cd477ec-articleLarge.jpg?quality=75\&auto=webp\&disable=upscale}

By \href{https://www.nytimes3xbfgragh.onion/by/kashmir-hill}{Kashmir
Hill}

\begin{itemize}
\item
  Published March 5, 2020Updated March 6, 2020
\item
  \begin{itemize}
  \item
  \item
  \item
  \item
  \item
  \item
  \end{itemize}
\end{itemize}

One Tuesday night in October 2018, John Catsimatidis, the billionaire
owner of the Gristedes grocery store chain, was having dinner at
Cipriani, an upscale Italian restaurant in Manhattan's SoHo
neighborhood, when his daughter, Andrea, walked in. She was on a date
with a man Mr. Catsimatidis didn't recognize. After the couple sat down
at another table, Mr. Catsimatidis asked a waiter to go over and take a
photo.

Mr. Catsimatidis then uploaded the picture to a facial recognition app,
Clearview AI, on his phone. The start-up behind the app has a database
of billions of photos, scraped from sites such as Facebook, Twitter and
LinkedIn. Within seconds, Mr. Catsimatidis was viewing a collection of
photos of the mystery man, along with the web addresses where they
appeared: His daughter's date was a venture capitalist from San
Francisco.

``I wanted to make sure he wasn't a charlatan,'' said Mr. Catsimatidis,
who then texted the man's bio to his daughter.

Ms. Catsimatidis said she and her date had no idea how her father had
identified him so quickly. ``I expect my dad to be able to do crazy
things. He's very technologically savvy,'' Ms. Catsimatidis said. ``My
date was very surprised.''

Clearview was unknown to the general public until this January, when The
New York Times reported that the secretive start-up had developed a
\href{https://www.nytimes3xbfgragh.onion/2020/01/18/technology/clearview-privacy-facial-recognition.html}{breakthrough
facial recognition system} that was in use by hundreds of law
enforcement agencies. The company quickly faced a backlash on multiple
fronts. Facebook, Google and other tech giants sent cease-and-desist
letters. Lawsuits were filed in Illinois and Virginia, and the attorney
general of New Jersey issued a moratorium against the app in that state.

In response to the criticism, Clearview published a
\href{https://blog.clearview.ai/code_of_conduct.pdf}{``code of
conduct,''} emphasizing in a
\href{https://blog.clearview.ai/post/2020-01-27-code-of-conduct/}{blog
post} that its technology was ``available only for law enforcement
agencies and select security professionals to use as an investigative
tool.''

The post added: ``We recognize that powerful tools always have the
potential to be abused, regardless of who is using them, and we take the
threat very seriously. Accordingly, the Clearview app has built-in
safeguards to ensure these trained professionals only use it for its
intended purpose: to help identify the perpetrators and victims of
crimes.''

The Times, however, has identified multiple individuals with active
access to Clearview's technology who are not law enforcement officials.
And for more than a year before the company became the subject of public
scrutiny, the app had been freely used in the wild by the company's
investors, clients and friends.

Those with Clearview logins used facial recognition at parties, on dates
and at business gatherings, giving demonstrations of its power for fun
or using it to identify people whose names they didn't know or couldn't
recall.

``As part of the ordinary course of due diligence, we provided trial
accounts to potential and current investors, and other strategic
partners, so they could test the technology,'' said Hoan Ton-That, the
company's co-founder.

Mr. Catsimatidis first heard about Clearview from his friend Richard
Schwartz, another founder of the company, who served as an aide to
Rudolph W. Giuliani when Mr. Giuliani was mayor of New York. Last
summer, Mr. Catsimatidis ran a trial project with Clearview at an East
Side Gristedes market. The company used the system to identify known
``shoplifters or people who had held up other stores,'' Mr. Catsimatidis
said.

``People were
\href{https://nypost.com/2016/08/12/ice-cream-bandits-are-wreaking-havoc-on-nyc-supermarkets/}{stealing
our Häagen-Dazs}. It was a big problem,'' he said. He described
Clearview as a ``good system'' that helped security personnel identify
problem shoppers.

BuzzFeed News
\href{https://www.buzzfeednews.com/article/carolinehaskins1/clearview-facial-recognition-insight-camera-glasses}{has
reported} that two other entities, a labor union and a real estate firm,
also ran trials with a surveillance system developed by Clearview to
flag individuals they deemed risky. The publication also
\href{https://www.buzzfeednews.com/article/ryanmac/clearview-ai-fbi-ice-global-law-enforcement}{reported}
that Clearview's software has been used by Best Buy, Macy's, Kohl's, the
National Basketball Association and numerous other organizations.

When Clearview first developed its facial recognition service in 2017,
Mr. Ton-That and Mr. Schwartz were uncertain about who might pay for it,
and they courted a range of clients including real estate firms, banks
and retailers. At the same time, Clearview was seeking outside
investment. Many of the individuals the company approached got personal
logins to the app.

\includegraphics{https://static01.graylady3jvrrxbe.onion/images/2020/03/09/business/05CLEARVIEW-BACKERS2-print/05CLEARVIEW-BACKERS-02-articleLarge.jpg?quality=75\&auto=webp\&disable=upscale}

Clearview received a seed investment round of about \$1 million in July
2018. Its backers included the billionaire investor Peter Thiel, the
venture capitalist David Scalzo and Hal Lambert, an investor in Texas
who runs an exchange-traded fund with the ticker symbol
``\href{https://www.pointbridgecapital.com/etf/}{MAGA},'' which tracks
companies that align with Republican politics.

``I have the app,'' Mr. Lambert said in an interview. ``I've used it to
talk about what we're doing in the space. I show it to friends of mine,
potential investors.

``They thought it was amazing,'' he added. ``They say, `How do I get
that?' And I say, `You can't.'''

Mr. Scalzo, the founder of the investment firm Kirenaga Partners, said
in an interview that his school-aged daughters enjoyed playing with the
app.

``They like to use it on themselves and their friends to see who they
look like in the world,'' he said. ``It's kind of fun for people.''

A spokesman for Mr. Thiel did not respond to a request for comment.

When Clearview was seeking its Series A round of funding, which was
completed in 2019, the start-up contacted a number of venture capital
firms, including Sequoia Capital and Khosla Ventures. Access to the app
was offered as a perk, according to people familiar with the company's
fund-raising attempts.

Doug Leone, a billionaire partner at Sequoia, was given a login,
according to three people with knowledge of Clearview's operations. But
his account was revoked when Sequoia declined to invest. A spokeswoman
for Sequoia declined to comment.

In September, Ashton Kutcher, the actor turned venture capitalist,
\href{https://www.youtube.com/watch?v=nNhYqLbsAGk\&feature=youtu.be\&t=872}{described
an app much like Clearview} during a YouTube series called ``Hot Ones,''
in which guests are interviewed while eating spicy chicken wings.

``I have an app in my phone in my pocket right now. It's like a beta
app,'' Mr. Kutcher said. ``It's a facial recognition app. I can hold it
up to anybody's face here and, like, find exactly who you are, what
internet accounts you're on, what they look like. It's terrifying.''

Mr. Kutcher did not respond to a request for comment.

Mr. Ton-That contends that Clearview is doing nothing wrong --- that his
app simply replicates what other search engines do. Instead of allowing
internet users to search for people's public images by name, as one can
do on Google, he said, Clearview allows them to do the search by
uploading a face.

For now, it's a power that Clearview controls and can give out as it
pleases.

In October, Clearview asked Nicholas Cassimatis, an expert on artificial
intelligence, to help conduct an internal accuracy test. He did the work
for free, he said, because he knew Mr. Ton-That socially. The test
consisted of submitting the faces of 834 federal and state legislators.
Clearview's algorithms accurately identified every one of the
politicians.

After the test was complete, Mr. Cassimatis was allowed to keep
Clearview's app on his phone. He said he had since run dozens of
searches.

``I tested it in surprising places: smoky bars, dark places. And it
worked every time,'' Mr. Cassimatis said. ``It's road testing. I do it
as a hobby. I ask people for permission. It's like a parlor trick.
People like it.''

Erin Griffith contributed reporting. Susan Beachy contributed research.

Advertisement

\protect\hyperlink{after-bottom}{Continue reading the main story}

\hypertarget{site-index}{%
\subsection{Site Index}\label{site-index}}

\hypertarget{site-information-navigation}{%
\subsection{Site Information
Navigation}\label{site-information-navigation}}

\begin{itemize}
\tightlist
\item
  \href{https://help.nytimes3xbfgragh.onion/hc/en-us/articles/115014792127-Copyright-notice}{©~2020~The
  New York Times Company}
\end{itemize}

\begin{itemize}
\tightlist
\item
  \href{https://www.nytco.com/}{NYTCo}
\item
  \href{https://help.nytimes3xbfgragh.onion/hc/en-us/articles/115015385887-Contact-Us}{Contact
  Us}
\item
  \href{https://www.nytco.com/careers/}{Work with us}
\item
  \href{https://nytmediakit.com/}{Advertise}
\item
  \href{http://www.tbrandstudio.com/}{T Brand Studio}
\item
  \href{https://www.nytimes3xbfgragh.onion/privacy/cookie-policy\#how-do-i-manage-trackers}{Your
  Ad Choices}
\item
  \href{https://www.nytimes3xbfgragh.onion/privacy}{Privacy}
\item
  \href{https://help.nytimes3xbfgragh.onion/hc/en-us/articles/115014893428-Terms-of-service}{Terms
  of Service}
\item
  \href{https://help.nytimes3xbfgragh.onion/hc/en-us/articles/115014893968-Terms-of-sale}{Terms
  of Sale}
\item
  \href{https://spiderbites.nytimes3xbfgragh.onion}{Site Map}
\item
  \href{https://help.nytimes3xbfgragh.onion/hc/en-us}{Help}
\item
  \href{https://www.nytimes3xbfgragh.onion/subscription?campaignId=37WXW}{Subscriptions}
\end{itemize}
