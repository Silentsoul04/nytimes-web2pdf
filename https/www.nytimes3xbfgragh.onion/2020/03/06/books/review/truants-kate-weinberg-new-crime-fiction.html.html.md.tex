Sections

SEARCH

\protect\hyperlink{site-content}{Skip to
content}\protect\hyperlink{site-index}{Skip to site index}

\href{https://www.nytimes3xbfgragh.onion/section/books/review}{Book
Review}

\href{https://myaccount.nytimes3xbfgragh.onion/auth/login?response_type=cookie\&client_id=vi}{}

\href{https://www.nytimes3xbfgragh.onion/section/todayspaper}{Today's
Paper}

\href{/section/books/review}{Book Review}\textbar{}Murders Most Foul

\url{https://nyti.ms/2wDxQVn}

\begin{itemize}
\item
\item
\item
\item
\item
\end{itemize}

Advertisement

\protect\hyperlink{after-top}{Continue reading the main story}

Supported by

\protect\hyperlink{after-sponsor}{Continue reading the main story}

\href{/column/crime}{Crime}

\hypertarget{murders-most-foul}{%
\section{Murders Most Foul}\label{murders-most-foul}}

\includegraphics{https://static01.graylady3jvrrxbe.onion/images/2020/03/08/books/review/08Crime/08Crime-articleLarge.jpg?quality=75\&auto=webp\&disable=upscale}

By Marilyn Stasio

\begin{itemize}
\item
  March 6, 2020
\item
  \begin{itemize}
  \item
  \item
  \item
  \item
  \item
  \end{itemize}
\end{itemize}

Coming-of-age novels aren't my particular passion --- unless there's a
murder in the story. There's a doozy of a murder in Kate Weinberg's
hypnotic debut mystery, \textbf{THE TRUANTS (Putnam, 311 pp., \$26),}
one that might have met with the approval of the Queen of Crime, Agatha
Christie. Dame Agatha happens to figure tangentially in this uncommonly
clever whodunit, which makes plentiful references to her books, plot
twists, settings and even
\href{https://www.nytimes3xbfgragh.onion/2019/06/11/books/agatha-christie-vanished-11-days-1926.html}{the
11 days in 1926 when she inexplicably disappeared}--- all while coming
across as madly original.

The story opens on the campus of an undistinguished college in Norfolk,
where the narrator, Jessica Walker, is a student with a singular
interest in Christie's mysteries. Jess is especially keen on a Christie
course, ``Murdered by the Campus,'' being taught by the charismatic Dr.
Lorna Clay, a most unconventional thinker.

Outside the classroom, Jess makes friends with the ditsy, but darling
Georgie (``She was like a slot machine flashing all its lights in
constant jackpot''), Alec, Georgie's boyfriend, and her own designated
steady, Nick. Although it's Lorna who becomes Jess's obsession, Weinberg
has created complex, unpredictable players, each with a fully drawn
history, and all of them, in one way or another, deeply untrustworthy.
(``How would you feel if I told you I'd killed someone?'' is a typical
line of table talk.) On one delicious note, they even lie about their
own childhood traumas. But as flashy as the characters are, they all
exist to serve the plot, which keeps taking surprising, even startling
twists and turns, ending up --- in a switch that would please Christie
--- on a remote, almost deserted island.

♦

**** Charles Lenox, the gentleman sleuth in the smashing Victorian
mysteries of Charles Finch, discovers he is London's most eligible
bachelor in \textbf{THE LAST PASSENGER (Minotaur, 292 pp., \$27.99),} a
series prequel set in 1855. Despite having confounded his privileged
peers by setting himself up as a private detective, Lenox is the son of
a baronet and very rich, which makes him a prize indeed. Luckily, a
murder at Paddington Station saves him from being suffocated in the
social whirl.

The victim of this ``butchery of a murder'' is discovered on a train
from Manchester, his corpse stripped of all identification, including
the labels on his clothing. Exercising his keen skills at observation
and deduction, Lenox determines that the man is an American. Then again,
a newsboy instantly comes to the same conclusion: ``Nice teeth. Didn't
smell. \ldots{} You could tell he was American.'' That's the charm of
Finch's style and Charles's own manner of sleuthing --- a brilliant
display of brainy deduction, followed by a self-deprecating witticism or
a philosophical bit of wisdom. Despite the classy writing and beguiling
details of Charles's life story, the novel is almost crushed by an
overstuffed plot burdened with subplots and over-researched background
particulars, making this the busiest of his books. (Nonetheless, I was
fascinated by a totally gratuitous sidebar about the origin of the
mythical Lady Green Sleeves.)

\textbf{♦}

Is there nowhere safe from civilization and its discontents? P. T.
Marsh, the small-town cop in John McMahon's procedural mystery,
\textbf{THE EVIL MEN DO (Putnam, 330 pp., \$27),} would like to think
he's found it at his backwater home of Mason Falls, GA. ``In moments of
peace, I find that this area of Georgia is like heaven on earth,'' he
says. Nice thought; but reality intrudes when someone swaps out Ennis
Fultz's oxygen supply for a tank of nitrogen, killing that real-estate
mogul and clearing the way for some even more rapacious land-grabbers to
move in.

McMahon packs his narrative with layers of plot, subplots and red
herrings to distract from a rather obvious case of environmental raping
and pillaging. But he tells his story with flair. The murdered man was
such a smooth talker ``he could sell two milk machines to a farmer with
one cow.'' McMahon also has the wit to name a biker bar ``Motor Mouth.''
And --- in the pièce de résistance (for me, anyway) --- he describes
someone as looking like ``the type of guy who knew how to change his own
oil.''

♦

Inspector Ian Rutledge, the Scotland Yard detective in Charles Todd's
moody mysteries, survived the battle of the Somme but returned home from
World War I suffering from shell shock. ``God knew, they were all
haunted by something,'' we're told in \textbf{A DIVIDED LOYALTY
(Morrow/HarperCollins, 327 pp., \$27.99),} a thoughtful look at how
murder can unnerve a small village where people are struggling to return
to their ``normal'' lives. One such village is Avebury, where Rutledge
is sent to investigate the stabbing death of an unknown woman whose body
is found inside a prehistoric stone ring.

Todd's astute character studies of individual townspeople, from the
upright rector to the local gossip, offer a fascinating cross section of
postwar life in a community where, as one woman says, ``we lost more men
than we got back.'' While delivering a satisfying puzzle-mystery, the
story also tasks us to think about the women who lost their lives during
the war, too. ``Had they died in the influenza epidemic? Been killed in
the early Zeppelin raids?'' Rutledge wonders as he surveys unclaimed
women's valises at Victoria Station. Death and destruction, he realizes,
``had crept into everyone's life.''

Advertisement

\protect\hyperlink{after-bottom}{Continue reading the main story}

\hypertarget{site-index}{%
\subsection{Site Index}\label{site-index}}

\hypertarget{site-information-navigation}{%
\subsection{Site Information
Navigation}\label{site-information-navigation}}

\begin{itemize}
\tightlist
\item
  \href{https://help.nytimes3xbfgragh.onion/hc/en-us/articles/115014792127-Copyright-notice}{©~2020~The
  New York Times Company}
\end{itemize}

\begin{itemize}
\tightlist
\item
  \href{https://www.nytco.com/}{NYTCo}
\item
  \href{https://help.nytimes3xbfgragh.onion/hc/en-us/articles/115015385887-Contact-Us}{Contact
  Us}
\item
  \href{https://www.nytco.com/careers/}{Work with us}
\item
  \href{https://nytmediakit.com/}{Advertise}
\item
  \href{http://www.tbrandstudio.com/}{T Brand Studio}
\item
  \href{https://www.nytimes3xbfgragh.onion/privacy/cookie-policy\#how-do-i-manage-trackers}{Your
  Ad Choices}
\item
  \href{https://www.nytimes3xbfgragh.onion/privacy}{Privacy}
\item
  \href{https://help.nytimes3xbfgragh.onion/hc/en-us/articles/115014893428-Terms-of-service}{Terms
  of Service}
\item
  \href{https://help.nytimes3xbfgragh.onion/hc/en-us/articles/115014893968-Terms-of-sale}{Terms
  of Sale}
\item
  \href{https://spiderbites.nytimes3xbfgragh.onion}{Site Map}
\item
  \href{https://help.nytimes3xbfgragh.onion/hc/en-us}{Help}
\item
  \href{https://www.nytimes3xbfgragh.onion/subscription?campaignId=37WXW}{Subscriptions}
\end{itemize}
