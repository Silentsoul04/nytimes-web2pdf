Sections

SEARCH

\protect\hyperlink{site-content}{Skip to
content}\protect\hyperlink{site-index}{Skip to site index}

\href{https://www.nytimes3xbfgragh.onion/section/food}{Food}

\href{https://myaccount.nytimes3xbfgragh.onion/auth/login?response_type=cookie\&client_id=vi}{}

\href{https://www.nytimes3xbfgragh.onion/section/todayspaper}{Today's
Paper}

\href{/section/food}{Food}\textbar{}Stocking Your Pantry, the Smart Way

\url{https://nyti.ms/39BLreQ}

\begin{itemize}
\item
\item
\item
\item
\item
\item
\end{itemize}

Advertisement

\protect\hyperlink{after-top}{Continue reading the main story}

Supported by

\protect\hyperlink{after-sponsor}{Continue reading the main story}

\hypertarget{stocking-your-pantry-the-smart-way}{%
\section{Stocking Your Pantry, the Smart
Way}\label{stocking-your-pantry-the-smart-way}}

Here's what you really need to keep on hand.

\includegraphics{https://static01.graylady3jvrrxbe.onion/images/2020/03/14/dining/23pantry1/23pantry1-articleLarge.jpg?quality=75\&auto=webp\&disable=upscale}

\href{https://www.nytimes3xbfgragh.onion/by/melissa-clark}{\includegraphics{https://static01.graylady3jvrrxbe.onion/images/2018/06/21/multimedia/author-melissa-clark/author-melissa-clark-thumbLarge.png}}

By \href{https://www.nytimes3xbfgragh.onion/by/melissa-clark}{Melissa
Clark}

\begin{itemize}
\item
  Published March 6, 2020Updated March 22, 2020
\item
  \begin{itemize}
  \item
  \item
  \item
  \item
  \item
  \item
  \end{itemize}
\end{itemize}

\href{https://www.nytimes3xbfgragh.onion/es/2020/03/12/espanol/despensa-alacena-cuarentena.html}{Leer
en español}

Having a
\href{https://cooking.nytimes3xbfgragh.onion/guides/56-how-to-stock-a-modern-pantry}{well-stocked
pantry} is always a good idea, whether you're looking to throw together
pasta with anchovies when there's nothing in the fridge, or you want to
avoid going grocery shopping when the weather is vile.

It's also true that a well-stocked pantry can provide a sense of safety
and control when the news is frightful and the future uncertain. Not
only can shelves filled with bags of rice, cans of beans and bins of
garlic give us a sense of order, they're also the beginning of
\href{https://cooking.nytimes3xbfgragh.onion/68861692-nyt-cooking/13993006-pantry-recipes}{many
excellent meals}.

The question is, what should you buy? Here are some of my personal
strategies for buying wisely and eating well.

\textbf{Pasta:} It's nice to have both long, thin pasta (like spaghetti
or linguine) and short, textured pasta (like fusilli or orecchiette) on
hand. Having both keeps meals from getting repetitive. And if you're a
boxed mac and cheese fan, stocking a couple of those is never a bad
idea.

\textbf{Rice:} Brown, white, long-grain, short-grain, I always have them
all around, plus Arborio for rice pudding and risotto. (To make a
perfect pot of rice, check out our
\href{https://cooking.nytimes3xbfgragh.onion/guides/49-how-to-make-rice}{How
to Make Rice} guide.)

\textbf{Stocks and Broths:} Whether meat- or vegetable-based, boxed
broths are essential for
\href{https://cooking.nytimes3xbfgragh.onion/guides/40-how-to-make-soup}{soups},
stews, and make the base of pan sauces. And they won't take up valuable
freezer space.

\textbf{Beans:}
\href{https://cooking.nytimes3xbfgragh.onion/guides/21-how-to-cook-beans}{Home-cooked
dried beans} taste better than canned beans; canned beans are more
convenient and faster than the dried kind. It's always good practice to
have both.

\textbf{Cured Meats:} Cured meats keep for weeks in the fridge, and a
chunk of smoked ham or salami in your bean pot adds so much flavor. But
don't forget the bacon.

\emph{{[}For more suggestions, check out
our}\href{https://cooking.nytimes3xbfgragh.onion/guides/56-how-to-stock-a-modern-pantry}{\emph{How
to Stock a Modern Pantry}} \emph{guide.{]}}

\textbf{Eggs:} They're a welcome addition to so many dishes --- and, of
course, stand on their own. Not to mention, they keep for a long time in
the fridge.

\textbf{Frozen Fruits and Veggies:} Use the fruit in smoothies, and the
vegetables in soups, stir-fries and stews. I'm partial to frozen
spinach, artichokes, kale, corn, peas and lima beans. As for fruit, I
like blueberries, mixed berries, peach, mango and cherries. (I also like
to throw ripe bananas into the freezer, to extend their lives a little.)

\textbf{Canned Fish:} Anchovies, sardines, tuna and salmon are the
building blocks of many of my favorite meals.
\href{https://www.instagram.com/p/B9S6nDElI2B/}{Anchovy toast} is a
staple at my house.

\textbf{Tahini and Peanut Butter:} For hummus, for sauces, for salad
dressing, for cookies, for sandwiches, for snacking. To mix things up,
try using peanut butter in your
\href{https://cooking.nytimes3xbfgragh.onion/recipes/1012747-peanut-noodles-with-shrimp}{salad
dressings}, and tahini on your jam sandwiches. Almond and cashew butters
are also great pantry additions, if you prefer.

\textbf{Parmesan and Other Hard Cheeses:} These kinds of cheeses last
for months in the fridge if you buy blocks rather than pre-grated. Or
store the pre-grated kind in the freezer. You can also buy some cheese
sealed in wax, which increases how long you can store them. Cream cheese
is also important, for bagels and otherwise.

\textbf{Olive Oil, Vinegar, Lemons:} I can't cook without them.

\textbf{Nuts and Dried Fruit:} For snacking and baking.

\textbf{Flour, Sugar, Yeast}: For baking projects.

\textbf{Butter:} It will keep for weeks in the fridge and longer in the
freezer.

\textbf{Garlic, Onions and Shallots:} No pantry is complete without
them.

\textbf{Root Vegetables:} Potatoes, sweet potatoes, beets, carrots,
parsnips, turnips, radishes. Radishes and turnips are excellent in
salads and last for weeks in the produce drawer. If they get soft, soak
them in a bowl of ice water until they firm up. Consider ginger for
stir-fries and making into tea.

\textbf{Celery:} Use the leaves as an herb and the stalks in salads. Or
stuff them with peanut butter or tahini for snacks.

\textbf{Scallions:} If you leave the roots on and put scallions in a
container of water on the counter, you can cut off the green tops, and
they will grow back three or four times. My mom taught me this trick.

\textbf{Spices:} If can't remember when the last time you bought new
spices was, consider buying some the next time you're at the store:
Fresh spices have so much more flavor than old ones.

Advertisement

\protect\hyperlink{after-bottom}{Continue reading the main story}

\hypertarget{site-index}{%
\subsection{Site Index}\label{site-index}}

\hypertarget{site-information-navigation}{%
\subsection{Site Information
Navigation}\label{site-information-navigation}}

\begin{itemize}
\tightlist
\item
  \href{https://help.nytimes3xbfgragh.onion/hc/en-us/articles/115014792127-Copyright-notice}{©~2020~The
  New York Times Company}
\end{itemize}

\begin{itemize}
\tightlist
\item
  \href{https://www.nytco.com/}{NYTCo}
\item
  \href{https://help.nytimes3xbfgragh.onion/hc/en-us/articles/115015385887-Contact-Us}{Contact
  Us}
\item
  \href{https://www.nytco.com/careers/}{Work with us}
\item
  \href{https://nytmediakit.com/}{Advertise}
\item
  \href{http://www.tbrandstudio.com/}{T Brand Studio}
\item
  \href{https://www.nytimes3xbfgragh.onion/privacy/cookie-policy\#how-do-i-manage-trackers}{Your
  Ad Choices}
\item
  \href{https://www.nytimes3xbfgragh.onion/privacy}{Privacy}
\item
  \href{https://help.nytimes3xbfgragh.onion/hc/en-us/articles/115014893428-Terms-of-service}{Terms
  of Service}
\item
  \href{https://help.nytimes3xbfgragh.onion/hc/en-us/articles/115014893968-Terms-of-sale}{Terms
  of Sale}
\item
  \href{https://spiderbites.nytimes3xbfgragh.onion}{Site Map}
\item
  \href{https://help.nytimes3xbfgragh.onion/hc/en-us}{Help}
\item
  \href{https://www.nytimes3xbfgragh.onion/subscription?campaignId=37WXW}{Subscriptions}
\end{itemize}
