Sections

SEARCH

\protect\hyperlink{site-content}{Skip to
content}\protect\hyperlink{site-index}{Skip to site index}

\href{https://www.nytimes3xbfgragh.onion/section/world/asia}{Asia
Pacific}

\href{https://myaccount.nytimes3xbfgragh.onion/auth/login?response_type=cookie\&client_id=vi}{}

\href{https://www.nytimes3xbfgragh.onion/section/todayspaper}{Today's
Paper}

\href{/section/world/asia}{Asia Pacific}\textbar{}Failures on the
Diamond Princess Shadow Another Cruise Ship Outbreak

\url{https://nyti.ms/2v3DZdc}

\begin{itemize}
\item
\item
\item
\item
\item
\end{itemize}

\hypertarget{the-coronavirus-outbreak}{%
\subsubsection{\texorpdfstring{\href{https://www.nytimes3xbfgragh.onion/news-event/coronavirus?name=styln-coronavirus-national\&region=TOP_BANNER\&variant=undefined\&block=storyline_menu_recirc\&action=click\&pgtype=Article\&impression_id=71016310-e395-11ea-9c33-7348c14ae8f8}{The
Coronavirus
Outbreak}}{The Coronavirus Outbreak}}\label{the-coronavirus-outbreak}}

\begin{itemize}
\tightlist
\item
  live\href{https://www.nytimes3xbfgragh.onion/2020/08/21/world/covid-19-coronavirus.html?name=styln-coronavirus-national\&region=TOP_BANNER\&variant=undefined\&block=storyline_menu_recirc\&action=click\&pgtype=Article\&impression_id=71018a20-e395-11ea-9c33-7348c14ae8f8}{Latest
  Updates}
\item
  \href{https://www.nytimes3xbfgragh.onion/interactive/2020/us/coronavirus-us-cases.html?name=styln-coronavirus-national\&region=TOP_BANNER\&variant=undefined\&block=storyline_menu_recirc\&action=click\&pgtype=Article\&impression_id=71018a21-e395-11ea-9c33-7348c14ae8f8}{Maps
  and Cases}
\item
  \href{https://www.nytimes3xbfgragh.onion/interactive/2020/science/coronavirus-vaccine-tracker.html?name=styln-coronavirus-national\&region=TOP_BANNER\&variant=undefined\&block=storyline_menu_recirc\&action=click\&pgtype=Article\&impression_id=71018a22-e395-11ea-9c33-7348c14ae8f8}{Vaccine
  Tracker}
\item
  \href{https://www.nytimes3xbfgragh.onion/2020/08/19/us/colleges-closing-covid.html?name=styln-coronavirus-national\&region=TOP_BANNER\&variant=undefined\&block=storyline_menu_recirc\&action=click\&pgtype=Article\&impression_id=71018a23-e395-11ea-9c33-7348c14ae8f8}{Colleges
  Closing}
\item
  \href{https://www.nytimes3xbfgragh.onion/live/2020/08/20/business/stock-market-today-coronavirus?name=styln-coronavirus-national\&region=TOP_BANNER\&variant=undefined\&block=storyline_menu_recirc\&action=click\&pgtype=Article\&impression_id=71018a24-e395-11ea-9c33-7348c14ae8f8}{Economy}
\end{itemize}

Advertisement

\protect\hyperlink{after-top}{Continue reading the main story}

Supported by

\protect\hyperlink{after-sponsor}{Continue reading the main story}

\hypertarget{failures-on-the-diamond-princess-shadow-another-cruise-ship-outbreak}{%
\section{Failures on the Diamond Princess Shadow Another Cruise Ship
Outbreak}\label{failures-on-the-diamond-princess-shadow-another-cruise-ship-outbreak}}

An alert about a coronavirus infection sat in an unmonitored inbox. A
cruise doctor saw ``no point'' in disinfecting the ship. And Japan
disregarded medical guidelines to contain an outbreak.

\includegraphics{https://static01.graylady3jvrrxbe.onion/images/2020/03/06/world/00japan-ship-1/merlin_168482076_4a72df56-fcd1-435a-ab7d-5da3fe1a2c06-articleLarge.jpg?quality=75\&auto=webp\&disable=upscale}

By \href{https://www.nytimes3xbfgragh.onion/by/matt-apuzzo}{Matt
Apuzzo}, \href{https://www.nytimes3xbfgragh.onion/by/motoko-rich}{Motoko
Rich} and
\href{http://www.nytimes3xbfgragh.onion/by/david-yaffe-bellany}{David
Yaffe-Bellany}

\begin{itemize}
\item
  Published March 8, 2020Updated March 10, 2020
\item
  \begin{itemize}
  \item
  \item
  \item
  \item
  \item
  \end{itemize}
\end{itemize}

\href{https://www.nytimes3xbfgragh.onion/es/2020/03/10/espanol/mundo/coronavirus-crucero.html}{Leer
en español}

The email to Princess Cruises was peppered with typos and awkward
grammar, but the warning was unmistakable. An 80-year-old passenger had
\href{https://www.nytimes3xbfgragh.onion/article/what-is-coronavirus.html}{tested
positive for the new coronavirus} after getting off the Diamond Princess
cruise ship in Hong Kong.

``Would kindly inform the ship related parties and do the necessary
disinfection,'' Princess's port representative wrote on Feb. 1, relaying
a warning from Hong Kong health officials. ``Many thanks!''

Nothing happened. Princess says it believes the alert sat unread in
unmonitored inboxes. Grant Tarling, the company's top doctor and the
person in charge of responding
\href{https://www.nytimes3xbfgragh.onion/2020/03/07/world/asia/china-coronavirus-cost.html}{to
outbreaks}, said he hadn't learned about the infection until the
following day --- after being alerted to a post on social media.

The fumbled alert was just the beginning of
\href{https://www.nytimes3xbfgragh.onion/2020/02/17/world/asia/japan-cruise-ship-coronavirus.html}{a
broader breakdown} by both the company and
\href{https://www.nytimes3xbfgragh.onion/2020/02/20/world/asia/japan-coronavirus-clusters.html}{the
Japanese authorities} who quarantined the ship in Yokohama. Hobbled by
confusion and mistakes, they played down the risk of infection, ignored
best medical practice for evacuating passengers, and activated only
low-level protocols for dealing with outbreaks. Ultimately,
\href{https://www.nytimes3xbfgragh.onion/2020/02/22/world/asia/coronavirus-japan-cruise-ship.html?searchResultPosition=3}{eight
people died and more than 700 were infected,} including some government
officials.

Now, those failures have taken on fresh urgency as Princess and Dr.
Tarling deal with yet another coronavirus cluster, on
\href{https://www.nytimes3xbfgragh.onion/2020/03/06/us/california-coronavirus-cruise-ship.html}{a
cruise ship} that has been turned away from port in San Francisco. A
passenger who recently got off the cruise died of the virus last week,
and 21 people have since tested positive.

Thousands of passengers have been told to stay in their rooms as the
cruise company and the American authorities have scrambled to figure out
a plan. The ship is expected to dock in Oakland, Calif., on Monday and
passengers will be quarantined onshore.

In conversations onboard, passengers have been asking, ``Will we become
another Diamond Princess?'' said Bill Pearce, a 54-year-old from
Lafayette, Calif.

``Whatever's the quickest way to get home, I'm all for it. I'd rather
not be on the ship,'' he said. ``Not being able to walk more than six
feet in any direction --- it's like being in a jail cell.''

The crisis on the Diamond Princess exposes the vulnerabilities in the
patchwork of international agreements, national laws and corporate
policies governing the health and safety of the \$150 billion cruise
industry that carries 30 million passengers a year.

After the infection was confirmed, company officials incorrectly assumed
that the immediate risk was minimal because the sick passenger had
disembarked. Health authorities recommended action immediately. ``We
advise thorough environmental cleansing and disinfection of the
cruise,'' Albert Lam, an epidemiologist for the Hong Kong government,
wrote to the company on Feb. 2.

The company says it stepped up cleaning the next day. But it initiated
only the lowest-level protocols for outbreaks. ``There's no point in
going and start cleaning the ship when we really didn't know what, if
any, risk there was onboard,'' Dr. Tarling said in an interview.

When the Japanese authorities boarded the vessel, they, too, ignored
medical guidelines, by leaving potentially exposed passengers onboard
during testing rather than taking them ashore as recommended.

The Japanese government and company still disagree on who was --- or
should have been --- in charge. So the responsibilities for quarantining
nearly 2,700 passengers fell mostly to about 1,000 low-paid ship workers
who were given inadequate safety gear and guidance.

Dr. Tarling, the chief medical officer for Princess's parent company,
Carnival Corporation, oversaw the response from California. He was
unaware of the working conditions onboard. He said crew members had
followed the protocols for a person ``in quarantine or isolation in a
hospital.''

\hypertarget{latest-updates-the-coronavirus-outbreak}{%
\section{\texorpdfstring{\href{https://www.nytimes3xbfgragh.onion/2020/08/21/world/covid-19-coronavirus.html?action=click\&pgtype=Article\&state=default\&region=MAIN_CONTENT_1\&context=storylines_live_updates}{Latest
Updates: The Coronavirus
Outbreak}}{Latest Updates: The Coronavirus Outbreak}}\label{latest-updates-the-coronavirus-outbreak}}

Updated 2020-08-21T09:57:24.778Z

\begin{itemize}
\tightlist
\item
  \href{https://www.nytimes3xbfgragh.onion/2020/08/21/world/covid-19-coronavirus.html?action=click\&pgtype=Article\&state=default\&region=MAIN_CONTENT_1\&context=storylines_live_updates\#link-4690b6aa}{Shutdowns,
  warnings and scoldings follow gatherings on college campuses.}
\item
  \href{https://www.nytimes3xbfgragh.onion/2020/08/21/world/covid-19-coronavirus.html?action=click\&pgtype=Article\&state=default\&region=MAIN_CONTENT_1\&context=storylines_live_updates\#link-324af071}{As
  he accepts the Democratic nomination, Biden knocks Trump's pandemic
  response.}
\item
  \href{https://www.nytimes3xbfgragh.onion/2020/08/21/world/covid-19-coronavirus.html?action=click\&pgtype=Article\&state=default\&region=MAIN_CONTENT_1\&context=storylines_live_updates\#link-35890b73}{Hundreds
  of doctors in Kenya go on strike over their pay and protective gear.}
\end{itemize}

\href{https://www.nytimes3xbfgragh.onion/2020/08/21/world/covid-19-coronavirus.html?action=click\&pgtype=Article\&state=default\&region=MAIN_CONTENT_1\&context=storylines_live_updates}{See
more updates}

More live coverage:
\href{https://www.nytimes3xbfgragh.onion/live/2020/08/20/business/stock-market-today-coronavirus?action=click\&pgtype=Article\&state=default\&region=MAIN_CONTENT_1\&context=storylines_live_updates}{Markets}

In reality, a crew member often wore the same pair of gloves to deliver
food to dozens of cabins at a time, door-to-door and face-to-face with
passengers, a potential source of infection. They also collected dirty
dishes and used linens without full protective gear.

An outbreak of a new virus aboard a crowded ship unquestionably
presented high-stakes difficulties. Princess says it did the best job
possible in the face of unprecedented challenges.

But in a series of interviews, company officials offered contradictory
and changing accounts about their response. In the end, nearly 48 hours
elapsed between the alert on Feb. 1 and the captain's announcement to
the ship on Feb. 3 that a passenger had been infected, giving the virus
time to spread.

Princess officials could not point to the social media post, or the
platform, that they say tipped them off. They said it took Dr. Tarling
until the night of Feb. 2 to confirm that a former passenger had tested
positive.

Company emails show that he knew by that morning. In an email to a Hong
Kong doctor, he listed the patient's name, his hospital wing, his
traveling companions and the date of the diagnosis.

The subject line of his email began: ``Confirmed Coronavirus Case.''

\includegraphics{https://static01.graylady3jvrrxbe.onion/images/2020/03/06/world/00japan-ship-3/merlin_168509718_7a0c8616-3370-49f0-8baa-4e1591afcb91-articleLarge.jpg?quality=75\&auto=webp\&disable=upscale}

\hypertarget{a-late-response}{%
\subsection{A Late Response}\label{a-late-response}}

Containment procedures for an outbreak unfold on a continuum.

Basic precautions might include encouraging handwashing, eliminating
self-service at buffets and increasing cleaning. More serious matters
might prompt crews to discourage handshaking, disinfect the ship or
cancel social events. Outbreaks of highly infectious diseases like Ebola
have their own stringent protocols.

The response aboard the Diamond Princess reflected concern, but not a
major one. The buffets remained open as usual. Onboard celebrations,
opera performances and goodbye parties continued.

``We immediately increased our already robust sanitation protocols,''
Gennaro Arma, the ship's captain, said in response to questions
submitted through Princess. He said the crew had increased the number of
hand sanitizers, rotated the buffet utensils more frequently and stepped
up cleaning.

Passengers aboard the Diamond Princess say they noticed few ship-wide
changes after the announcement of the infected passenger. But some crew
members began giving out their own more cautious advice, making some
passengers suspicious that the risks were higher than the company was
letting on. At the end of a trivia game, for example, a worker told
passengers not to hand back their pencils.

``He said, `Hold onto your pencils and you can consider yourself a
winner,''' said Carol Montgomery, 67, a retired administrative assistant
from San Clemente, Calif.

The crew member also advised them to avoid handrails. ``It was like he
was trying to tell us something,'' Ms. Montgomery said.

Image

A photo posted on Feb. 5 on a Twitter account with the handle @daxa\_tw
showed officials aboard the ship.Credit...Agence France-Presse --- Getty
Images

\hypertarget{underestimating-contagion}{%
\subsection{Underestimating Contagion}\label{underestimating-contagion}}

Underpinning the company's approach was an optimistic, but ultimately
inaccurate, belief that perhaps danger had been averted.

The 80-year-old infected passenger had reported no symptoms to the
medical staff while onboard. And he had disembarked more than a week
earlier, along with his daughter and their two traveling companions.

``They were off the ship,'' Dr. Tarling said. ``There's nothing to
believe that we have to put face masks on every single guest.''

Dr. Tarling also did not order the crew to begin what is known as
contact-tracing, the painstaking task of questioning everybody and
identifying who had been in contact with the infected passenger.
\href{https://www.who.int/publications-detail/operational-considerations-for-managing-covid-19-cases-outbreak-on-board-ships}{Infectious
disease experts say} that the process should begin immediately and that
anyone who was in close contact should be isolated.

Dr. Tarling said Japanese health officials planned to do the contact
tracing in a matter of hours, when the ship arrived in Yokohama on Feb.
3. Until then, he said, he considered only the elderly man and his
traveling party to be ``close contacts.''

That is a narrow interpretation. The
\href{https://www.who.int/publications-detail/operational-considerations-for-managing-covid-19-cases-outbreak-on-board-ships}{World
Health Organization's definition} covers dining partners, anyone who had
face-to-face contact with the patient --- and certainly passengers who
had shared a tour bus with him days earlier.

No restrictions were placed on passengers until around 11 p.m., when
Japanese medical teams boarded the ship and ordered everyone to their
cabins.

Two days later, when the first lab results came back, Japan reported
that 10 people had tested positive.

``We were as surprised as the Japanese were that actually there were
more positive samples on the ship,'' Dr. Tarling said.

Image

Officials in Yokohama on Feb. 4 escorting a person, under the blue
sheet, who was onboard the ship and tested positive for the
coronavirus.Credit...Kyodo, via Reuters

\hypertarget{not-recommended}{%
\subsection{`Not Recommended'}\label{not-recommended}}

Halfway around the world, in the Greek port city of Volos, an
epidemiologist named Christos Hadjichristodoulou has been studying
cruise ship outbreaks for nearly two decades. He served as a science
adviser in the lead-up to the 2004 Summer Olympics in Athens, which used
a fleet of ships to ease the hotel crunch.

By happenstance,
\href{https://www.gac.com/491364/siteassets/about-gac/coronavirus/eu-interim-advice_2019-ncov_maritime_4_2_2020_f.pdf}{on
the very day} that the Diamond Princess pulled into Yokohama, Dr.
Hadjichristodoulou and a team of European experts released new
recommendations for cruise ships.

The guidelines did not have the force of law. But experts said they were
among the first protocols --- if not the first --- written specifically
for the industry about the new coronavirus.

Those guidelines said that close contacts of a confirmed case
\href{https://www.healthygateways.eu/Portals/0/plcdocs/Flow_chart_Ships_3_2_2020.pdf}{should
be evacuated and quarantined on shore}, a step that is now supposed to
happen with the ship off California's coast after several days of
waiting. On the Diamond Princess, that would have meant removing many,
if not all, of the 273 people selected for the first round of testing.

\href{https://www.nytimes3xbfgragh.onion/news-event/coronavirus?action=click\&pgtype=Article\&state=default\&region=MAIN_CONTENT_3\&context=storylines_faq}{}

\hypertarget{the-coronavirus-outbreak-}{%
\subsubsection{The Coronavirus Outbreak
›}\label{the-coronavirus-outbreak-}}

\hypertarget{frequently-asked-questions}{%
\paragraph{Frequently Asked
Questions}\label{frequently-asked-questions}}

Updated August 17, 2020

\begin{itemize}
\item ~
  \hypertarget{why-does-standing-six-feet-away-from-others-help}{%
  \paragraph{Why does standing six feet away from others
  help?}\label{why-does-standing-six-feet-away-from-others-help}}

  \begin{itemize}
  \tightlist
  \item
    The coronavirus spreads primarily through droplets from your mouth
    and nose, especially when you cough or sneeze. The C.D.C., one of
    the organizations using that measure,
    \href{https://www.nytimes3xbfgragh.onion/2020/04/14/health/coronavirus-six-feet.html?action=click\&pgtype=Article\&state=default\&region=MAIN_CONTENT_3\&context=storylines_faq}{bases
    its recommendation of six feet} on the idea that most large droplets
    that people expel when they cough or sneeze will fall to the ground
    within six feet. But six feet has never been a magic number that
    guarantees complete protection. Sneezes, for instance, can launch
    droplets a lot farther than six feet,
    \href{https://jamanetwork.com/journals/jama/fullarticle/2763852}{according
    to a recent study}. It's a rule of thumb: You should be safest
    standing six feet apart outside, especially when it's windy. But
    keep a mask on at all times, even when you think you're far enough
    apart.
  \end{itemize}
\item ~
  \hypertarget{i-have-antibodies-am-i-now-immune}{%
  \paragraph{I have antibodies. Am I now
  immune?}\label{i-have-antibodies-am-i-now-immune}}

  \begin{itemize}
  \tightlist
  \item
    As of right
    now,\href{https://www.nytimes3xbfgragh.onion/2020/07/22/health/covid-antibodies-herd-immunity.html?action=click\&pgtype=Article\&state=default\&region=MAIN_CONTENT_3\&context=storylines_faq}{that
    seems likely, for at least several months.} There have been
    frightening accounts of people suffering what seems to be a second
    bout of Covid-19. But experts say these patients may have a
    drawn-out course of infection, with the virus taking a slow toll
    weeks to months after initial exposure. People infected with the
    coronavirus typically
    \href{https://www.nature.com/articles/s41586-020-2456-9}{produce}
    immune molecules called antibodies, which are
    \href{https://www.nytimes3xbfgragh.onion/2020/05/07/health/coronavirus-antibody-prevalence.html?action=click\&pgtype=Article\&state=default\&region=MAIN_CONTENT_3\&context=storylines_faq}{protective
    proteins made in response to an
    infection}\href{https://www.nytimes3xbfgragh.onion/2020/05/07/health/coronavirus-antibody-prevalence.html?action=click\&pgtype=Article\&state=default\&region=MAIN_CONTENT_3\&context=storylines_faq}{.
    These antibodies may} last in the body
    \href{https://www.nature.com/articles/s41591-020-0965-6}{only two to
    three months}, which may seem worrisome, but that's perfectly normal
    after an acute infection subsides, said Dr. Michael Mina, an
    immunologist at Harvard University. It may be possible to get the
    coronavirus again, but it's highly unlikely that it would be
    possible in a short window of time from initial infection or make
    people sicker the second time.
  \end{itemize}
\item ~
  \hypertarget{im-a-small-business-owner-can-i-get-relief}{%
  \paragraph{I'm a small-business owner. Can I get
  relief?}\label{im-a-small-business-owner-can-i-get-relief}}

  \begin{itemize}
  \tightlist
  \item
    The
    \href{https://www.nytimes3xbfgragh.onion/article/small-business-loans-stimulus-grants-freelancers-coronavirus.html?action=click\&pgtype=Article\&state=default\&region=MAIN_CONTENT_3\&context=storylines_faq}{stimulus
    bills enacted in March} offer help for the millions of American
    small businesses. Those eligible for aid are businesses and
    nonprofit organizations with fewer than 500 workers, including sole
    proprietorships, independent contractors and freelancers. Some
    larger companies in some industries are also eligible. The help
    being offered, which is being managed by the Small Business
    Administration, includes the Paycheck Protection Program and the
    Economic Injury Disaster Loan program. But lots of folks have
    \href{https://www.nytimes3xbfgragh.onion/interactive/2020/05/07/business/small-business-loans-coronavirus.html?action=click\&pgtype=Article\&state=default\&region=MAIN_CONTENT_3\&context=storylines_faq}{not
    yet seen payouts.} Even those who have received help are confused:
    The rules are draconian, and some are stuck sitting on
    \href{https://www.nytimes3xbfgragh.onion/2020/05/02/business/economy/loans-coronavirus-small-business.html?action=click\&pgtype=Article\&state=default\&region=MAIN_CONTENT_3\&context=storylines_faq}{money
    they don't know how to use.} Many small-business owners are getting
    less than they expected or
    \href{https://www.nytimes3xbfgragh.onion/2020/06/10/business/Small-business-loans-ppp.html?action=click\&pgtype=Article\&state=default\&region=MAIN_CONTENT_3\&context=storylines_faq}{not
    hearing anything at all.}
  \end{itemize}
\item ~
  \hypertarget{what-are-my-rights-if-i-am-worried-about-going-back-to-work}{%
  \paragraph{What are my rights if I am worried about going back to
  work?}\label{what-are-my-rights-if-i-am-worried-about-going-back-to-work}}

  \begin{itemize}
  \tightlist
  \item
    Employers have to provide
    \href{https://www.osha.gov/SLTC/covid-19/standards.html}{a safe
    workplace} with policies that protect everyone equally.
    \href{https://www.nytimes3xbfgragh.onion/article/coronavirus-money-unemployment.html?action=click\&pgtype=Article\&state=default\&region=MAIN_CONTENT_3\&context=storylines_faq}{And
    if one of your co-workers tests positive for the coronavirus, the
    C.D.C.} has said that
    \href{https://www.cdc.gov/coronavirus/2019-ncov/community/guidance-business-response.html}{employers
    should tell their employees} -\/- without giving you the sick
    employee's name -\/- that they may have been exposed to the virus.
  \end{itemize}
\item ~
  \hypertarget{what-is-school-going-to-look-like-in-september}{%
  \paragraph{What is school going to look like in
  September?}\label{what-is-school-going-to-look-like-in-september}}

  \begin{itemize}
  \tightlist
  \item
    It is unlikely that many schools will return to a normal schedule
    this fall, requiring the grind of
    \href{https://www.nytimes3xbfgragh.onion/2020/06/05/us/coronavirus-education-lost-learning.html?action=click\&pgtype=Article\&state=default\&region=MAIN_CONTENT_3\&context=storylines_faq}{online
    learning},
    \href{https://www.nytimes3xbfgragh.onion/2020/05/29/us/coronavirus-child-care-centers.html?action=click\&pgtype=Article\&state=default\&region=MAIN_CONTENT_3\&context=storylines_faq}{makeshift
    child care} and
    \href{https://www.nytimes3xbfgragh.onion/2020/06/03/business/economy/coronavirus-working-women.html?action=click\&pgtype=Article\&state=default\&region=MAIN_CONTENT_3\&context=storylines_faq}{stunted
    workdays} to continue. California's two largest public school
    districts --- Los Angeles and San Diego --- said on July 13, that
    \href{https://www.nytimes3xbfgragh.onion/2020/07/13/us/lausd-san-diego-school-reopening.html?action=click\&pgtype=Article\&state=default\&region=MAIN_CONTENT_3\&context=storylines_faq}{instruction
    will be remote-only in the fall}, citing concerns that surging
    coronavirus infections in their areas pose too dire a risk for
    students and teachers. Together, the two districts enroll some
    825,000 students. They are the largest in the country so far to
    abandon plans for even a partial physical return to classrooms when
    they reopen in August. For other districts, the solution won't be an
    all-or-nothing approach.
    \href{https://bioethics.jhu.edu/research-and-outreach/projects/eschool-initiative/school-policy-tracker/}{Many
    systems}, including the nation's largest, New York City, are
    devising
    \href{https://www.nytimes3xbfgragh.onion/2020/06/26/us/coronavirus-schools-reopen-fall.html?action=click\&pgtype=Article\&state=default\&region=MAIN_CONTENT_3\&context=storylines_faq}{hybrid
    plans} that involve spending some days in classrooms and other days
    online. There's no national policy on this yet, so check with your
    municipal school system regularly to see what is happening in your
    community.
  \end{itemize}
\end{itemize}

Instead, the Japanese government asked them to stay in their rooms while
awaiting test results. In practice, passengers still moved about and ate
at buffets.

When the first batch of positive results were reported on the morning of
Feb. 5, the Japanese authorities ordered a ship-wide quarantine.
Confirmed cases would be evacuated to hospitals, but everyone else would
remain aboard, isolated in their cabins.

``The approach they followed is not recommended for many reasons,'' Dr.
Hadjichristodoulou said. He would not criticize decisions made under
pressure, but said it was clear the virus would spread. ``We expected
this,'' he said.

Japanese health officials say that some local authorities feared
allowing potentially infected passengers ashore. And the country could
not immediately quarantine a large number of people. ``It's easy to say
that they should be moved to an onshore facility,'' said Dr. Yasuyuki
Sahara, a senior assistant minister in Japan's health ministry. ``But in
reality it is not so easy.''

Dr. Hadjichristodoulou, who also
\href{http://apps.who.int/whocc/Detail.aspx?cc_ref=GRE-27\&cc_code=gre}{leads
a team} of World Health Organization advisers, said he had contacted the
agency's offices in Lyon, France, and offered to organize a group of
experts to board the ship to advise. He said the offer was declined.

A W.H.O. spokesman said he would not discuss the group's internal
discussions. Dr. Sahara said the ministry had no record of the offer.

One W.H.O. consultant from the Manila office provided advice in the
health ministry's Tokyo office but never boarded the ship.

Image

Diamond Princess crew members in late February. Ship crews drill for
many eventualities, but one expert said, ``There was no drilling for
something like this.''Credit...Kazuhiro Nogi/Agence France-Presse ---
Getty Images

\hypertarget{no-drilling-for-something-like-this}{%
\subsection{`No Drilling for Something Like
This'}\label{no-drilling-for-something-like-this}}

Quarantining thousands of people is a huge operation requiring far more
gear and supplies than any cruise ship carries. Confining people in
shared rooms has its own challenges, as does feeding them and keeping
them entertained so they won't be tempted to leave their cabins.

Japan's foreign minister, Toshimitsu Motegi, said in a parliamentary
session that the responsibility should not have fallen solely on Japan.

``Japan is not the only state that is obliged to conduct measures to
prevent the expansion of infection,'' Mr. Motegi said. He suggested
international law was unclear as to whether both the country where the
ship was officially licensed --- Britain --- and the cruise operator
should share the burden.

Princess said it had followed Japan's lead from the moment its health
officials boarded the ship. ``When we have other outbreaks like
norovirus, we send our teams to the ship,'' Dr. Tarling said. ``Here,
we're sort of taking direction and seeing how we can best make it
work.''

That responsibility
\href{https://www.nytimes3xbfgragh.onion/2020/02/10/business/coronavirus-japan-cruise-ship.html}{fell
to the crew}. Cruise jobs are notorious for long hours and low pay. A
supervisory kitchen worker for Princess, for example, made \$1,949 a
month and was expected to work up to 13 hours a day, seven days a week,
for six months straight, according to a 2017 contract.

Those crew members drill for many eventualities, said Iain Hay, whose
company, Anchor Hygiene, conducts training for cruise companies.
``But,'' he said, ``there was no drilling for something like this.''

On the Diamond Princess, crew members delivered three meals a day to
close to 1,500 staterooms. Early in the quarantine, they served food on
china. While crew members wore masks and gloves, they risked spreading
--- or contracting --- the virus whenever they opened state room doors
and passed in trays of food.

``I'd give them a coffee cup and they'd give me back an apple,'' said
Melanie Haering, 58, whose husband, John, was hospitalized with the
virus. ``It was an exchange like that --- even though your hand is
gloved, your hand has still been soiled from the cabin next door.''

Even after switching to paper and plastic, crew members still delivered
meals into rooms rather than leaving them on the floor outside, as Dr.
Tarling believed happened. Passengers who were evacuated to military
bases in California and Texas said workers there left food outside.

Experts say the crew was not equipped to carry out the quarantine, so
lapses were inevitable. ``Look at how infectious-disease teams operate
anywhere in the world. That is their job,'' said Dr. Kate Bunyan, a
former medical director for Carnival U.K. ``They are not waiters in
their day jobs.''

But if anyone from Princess objected to the quarantine, nobody is saying
so. Dr. Tarling said it was the best option available.

The concern now is focused on the other Princess cruise thousands of
miles away. On Wednesday, a passenger in California died from the
coronavirus after completing a 10-day cruise to Mexico aboard that ship,
the Grand Princess.

For days, the ship has idled off the coast as Princess and American
officials have tried to figure out what to do. After shifting plans and
significant uncertainty, they have decided to dock in Oakland, where
sick passengers will be taken to hospitals in the state and others will
be sent to quarantine facilities around the country.

Once again, Dr. Tarling is helping manage the crisis. Asked whether he
wished he had done anything differently to contain the outbreak on the
Diamond Princess, he could not point to a single decision that he would
change.

``I believe our initial response was actually pretty good,'' he said.

Image

The Diamond Princess on Feb. 6.~Credit...Kim Kyung Hoon/Reuters

Tiffany May contributed reporting from Hong Kong. Makiko Inoue
contributed research from Tokyo, and Susan Beachy from New York.

Advertisement

\protect\hyperlink{after-bottom}{Continue reading the main story}

\hypertarget{site-index}{%
\subsection{Site Index}\label{site-index}}

\hypertarget{site-information-navigation}{%
\subsection{Site Information
Navigation}\label{site-information-navigation}}

\begin{itemize}
\tightlist
\item
  \href{https://help.nytimes3xbfgragh.onion/hc/en-us/articles/115014792127-Copyright-notice}{©~2020~The
  New York Times Company}
\end{itemize}

\begin{itemize}
\tightlist
\item
  \href{https://www.nytco.com/}{NYTCo}
\item
  \href{https://help.nytimes3xbfgragh.onion/hc/en-us/articles/115015385887-Contact-Us}{Contact
  Us}
\item
  \href{https://www.nytco.com/careers/}{Work with us}
\item
  \href{https://nytmediakit.com/}{Advertise}
\item
  \href{http://www.tbrandstudio.com/}{T Brand Studio}
\item
  \href{https://www.nytimes3xbfgragh.onion/privacy/cookie-policy\#how-do-i-manage-trackers}{Your
  Ad Choices}
\item
  \href{https://www.nytimes3xbfgragh.onion/privacy}{Privacy}
\item
  \href{https://help.nytimes3xbfgragh.onion/hc/en-us/articles/115014893428-Terms-of-service}{Terms
  of Service}
\item
  \href{https://help.nytimes3xbfgragh.onion/hc/en-us/articles/115014893968-Terms-of-sale}{Terms
  of Sale}
\item
  \href{https://spiderbites.nytimes3xbfgragh.onion}{Site Map}
\item
  \href{https://help.nytimes3xbfgragh.onion/hc/en-us}{Help}
\item
  \href{https://www.nytimes3xbfgragh.onion/subscription?campaignId=37WXW}{Subscriptions}
\end{itemize}
