Sections

SEARCH

\protect\hyperlink{site-content}{Skip to
content}\protect\hyperlink{site-index}{Skip to site index}

\href{https://www.nytimes3xbfgragh.onion/section/parenting}{Parenting}

\href{https://myaccount.nytimes3xbfgragh.onion/auth/login?response_type=cookie\&client_id=vi}{}

\href{https://www.nytimes3xbfgragh.onion/section/todayspaper}{Today's
Paper}

\href{/section/parenting}{Parenting}\textbar{}11 Questions Parents May
Have About Coronavirus

\url{https://nyti.ms/39Au1iE}

\begin{itemize}
\item
\item
\item
\item
\item
\end{itemize}

\hypertarget{the-coronavirus-outbreak}{%
\subsubsection{\texorpdfstring{\href{https://www.nytimes3xbfgragh.onion/news-event/coronavirus?name=styln-coronavirus-national\&region=TOP_BANNER\&block=storyline_menu_recirc\&action=click\&pgtype=Article\&impression_id=50c75390-efb9-11ea-9f2d-2b6e424d6802\&variant=undefined}{The
Coronavirus
Outbreak}}{The Coronavirus Outbreak}}\label{the-coronavirus-outbreak}}

\begin{itemize}
\tightlist
\item
  live\href{https://www.nytimes3xbfgragh.onion/2020/09/05/world/coronavirus-covid.html?name=styln-coronavirus-national\&region=TOP_BANNER\&block=storyline_menu_recirc\&action=click\&pgtype=Article\&impression_id=50c75391-efb9-11ea-9f2d-2b6e424d6802\&variant=undefined}{Latest
  Updates}
\item
  \href{https://www.nytimes3xbfgragh.onion/interactive/2020/us/coronavirus-us-cases.html?name=styln-coronavirus-national\&region=TOP_BANNER\&block=storyline_menu_recirc\&action=click\&pgtype=Article\&impression_id=50c75392-efb9-11ea-9f2d-2b6e424d6802\&variant=undefined}{Maps
  and Cases}
\item
  \href{https://www.nytimes3xbfgragh.onion/interactive/2020/science/coronavirus-vaccine-tracker.html?name=styln-coronavirus-national\&region=TOP_BANNER\&block=storyline_menu_recirc\&action=click\&pgtype=Article\&impression_id=50c75393-efb9-11ea-9f2d-2b6e424d6802\&variant=undefined}{Vaccine
  Tracker}
\item
  \href{https://www.nytimes3xbfgragh.onion/2020/09/02/your-money/eviction-moratorium-covid.html?name=styln-coronavirus-national\&region=TOP_BANNER\&block=storyline_menu_recirc\&action=click\&pgtype=Article\&impression_id=50c75394-efb9-11ea-9f2d-2b6e424d6802\&variant=undefined}{Eviction
  Moratorium}
\item
  \href{https://www.nytimes3xbfgragh.onion/interactive/2020/09/02/magazine/food-insecurity-hunger-us.html?name=styln-coronavirus-national\&region=TOP_BANNER\&block=storyline_menu_recirc\&action=click\&pgtype=Article\&impression_id=50c75395-efb9-11ea-9f2d-2b6e424d6802\&variant=undefined}{American
  Hunger}
\end{itemize}

Advertisement

\protect\hyperlink{after-top}{Continue reading the main story}

Supported by

\protect\hyperlink{after-sponsor}{Continue reading the main story}

\hypertarget{11-questions-parents-may-have-about-coronavirus}{%
\section{11 Questions Parents May Have About
Coronavirus}\label{11-questions-parents-may-have-about-coronavirus}}

Can I take my kid to the playground or on the subway? Should my child be
tested? We asked experts.

\href{https://www.nytimes3xbfgragh.onion/by/jessica-grose}{\includegraphics{https://static01.graylady3jvrrxbe.onion/images/2019/05/06/multimedia/00-headshot-test-Jess/00-headshot-test-Jess-thumbLarge-v5.png}}

By \href{https://www.nytimes3xbfgragh.onion/by/jessica-grose}{Jessica
Grose}

\begin{itemize}
\item
  Published March 9, 2020Updated March 26, 2020
\item
  \begin{itemize}
  \item
  \item
  \item
  \item
  \item
  \end{itemize}
\end{itemize}

\includegraphics{https://static01.graylady3jvrrxbe.onion/images/2020/03/10/multimedia/09-Parenting-takekidtoplayground/09-Parenting-takekidtoplayground-articleLarge.jpg?quality=75\&auto=webp\&disable=upscale}

As coronavirus continues to spread across the globe, we're working to
answer
\href{https://www.nytimes3xbfgragh.onion/spotlight/parenting-kids-coronavirus}{the
questions on many parents' minds}. This is a fast-moving situation, so
some information may be outdated. For the latest updates,
\href{https://www.nytimes3xbfgragh.onion/news-event/coronavirus}{read
The New York Times's live coronavirus coverage here}.

\emph{This article was last updated on March 26.}

Schools across the country have closed in response to the new
coronavirus and many parents have questions about how to go about their
daily lives while managing their children, whose personal boundaries and
hygiene levels are not always ideal.

Because the situation is evolving rapidly and the virus is new, the
advice may continue to change as we learn more. ``We're not seeing much
in the way of serious illness among children,'' said Dr. Peter J. Hotez,
M.D., Ph.D., the dean of the National School of Tropical Medicine at
Baylor College of Medicine.

On March 18, the Centers for Disease Control and Prevention released
\href{https://www.cdc.gov/mmwr/volumes/69/wr/mm6912e2.htm?s_cid=mm6912e2_w\&referringSource=articleShare}{new
preliminary data} on the outcomes of the first 4,226 Americans infected
with the new coronavirus, finding no fatalities or admissions to
intensive care units among those under 19.

Researchers from China published
\href{https://pediatrics.aappublications.org/content/pediatrics/early/2020/03/16/peds.2020-0702.full.pdf}{a
study} online in the journal Pediatrics in March, which examined more
than 2,000 children under 18 who either had or were suspected of having
Covid-19, the disease caused by the new coronavirus. The researchers
found that while the majority of children in the study had either no
symptoms, mild symptoms or moderate symptoms,
\href{https://www.nytimes3xbfgragh.onion/2020/03/17/health/coronavirus-childen.html}{nearly
6 percent became more seriously ill}--- particularly those under 5.
However, it's unclear if the children with more serious symptoms were
sick with Covid-19 or with another respiratory virus, said Dr. Sean
O'Leary, M.D., an executive member of the American Academy of
Pediatrics' Committee on Infectious Diseases.

Still, this study ``confirms what we have been suspecting, that it's
almost certainly less severe in children, but it's not zero,'' he said,
agreeing that it was prudent for many schools to close. ``We're in the
midst of something that no one alive has really experienced before,'' he
said.

With that in mind, here are some answers to common questions.

\href{https://www.nytimes3xbfgragh.onion/newsletters/parenting?module=inline}{\emph{{[}The
topics parents are talking about. Evidence-based guidance. Sign up now
to get NYT Parenting in your inbox every week.{]}}}

\hypertarget{can-i-still-take-my-child-to-public-places}{%
\subsection{Can I still take my child to public
places?}\label{can-i-still-take-my-child-to-public-places}}

The situation is changing by the hour, so your best bet is to regularly
check your state and local public health department websites for
recommendations, Dr. O'Leary said.

\hypertarget{latest-updates-the-coronavirus-outbreak}{%
\section{\texorpdfstring{\href{https://www.nytimes3xbfgragh.onion/2020/09/04/world/covid-19-coronavirus.html?action=click\&pgtype=Article\&state=default\&region=MAIN_CONTENT_1\&context=storylines_live_updates}{Latest
Updates: The Coronavirus
Outbreak}}{Latest Updates: The Coronavirus Outbreak}}\label{latest-updates-the-coronavirus-outbreak}}

Updated 2020-09-05T12:05:40.998Z

\begin{itemize}
\tightlist
\item
  \href{https://www.nytimes3xbfgragh.onion/2020/09/04/world/covid-19-coronavirus.html?action=click\&pgtype=Article\&state=default\&region=MAIN_CONTENT_1\&context=storylines_live_updates\#link-1654f6ad}{Research
  connects vaping to a higher chance of catching the virus --- and
  suffering its worst effects.}
\item
  \href{https://www.nytimes3xbfgragh.onion/2020/09/04/world/covid-19-coronavirus.html?action=click\&pgtype=Article\&state=default\&region=MAIN_CONTENT_1\&context=storylines_live_updates\#link-52e4198a}{Another
  college football game won't be played as planned.}
\item
  \href{https://www.nytimes3xbfgragh.onion/2020/09/04/world/covid-19-coronavirus.html?action=click\&pgtype=Article\&state=default\&region=MAIN_CONTENT_1\&context=storylines_live_updates\#link-181cef0}{Pharmaceutical
  companies plan a joint pledge on safety standards as they move
  vaccines to the marketplace.}
\end{itemize}

\href{https://www.nytimes3xbfgragh.onion/2020/09/04/world/covid-19-coronavirus.html?action=click\&pgtype=Article\&state=default\&region=MAIN_CONTENT_1\&context=storylines_live_updates}{See
more updates}

More live coverage:
\href{https://www.nytimes3xbfgragh.onion/live/2020/09/04/business/stock-market-today-coronavirus?action=click\&pgtype=Article\&state=default\&region=MAIN_CONTENT_1\&context=storylines_live_updates}{Markets}

But as of now, the general advice is to practice
\href{https://www.cdc.gov/nonpharmaceutical-interventions/community/index.html}{social
distancing}, Dr. Hotez said, which means sticking close to home and
avoiding large groups of people. On March 16, the Trump administration
announced new guidelines to help stop the spread of the new coronavirus,
which included closing schools that were still open and avoiding bars,
food courts, restaurants and groups of more than 10 people.

You can't be sure that popular public spaces like playgrounds are
risk-free --- the virus is estimated to survive on metal, glass and
plastic surfaces for anywhere
\href{https://www.nytimes3xbfgragh.onion/2020/03/02/health/coronavirus-how-it-spreads.html}{from
2 hours to nine days}. New York City, for example, does not regularly
clean outdoor furniture and play equipment, said Meghan Lalor, director
of media relations at the New York City Department of Parks and
Recreation. ``We have not yet committed to changing our standard
operations due to coronavirus, but we will continue to monitor the
situation as it develops,'' she said. On March 17, all New York City
``recreation centers and nature centers are closed to the public until
further notice,'' Lalor said, though city parks and playgrounds remain
open.

At this point, some communities are closing their playgrounds;
considering school is out and we don't want children congregating all
together, Dr. O'Leary said playgrounds are ``probably not the safest
place right now.'' For city dwellers, he recommended going to big,
wide-open parks when available, where kids can stay practice physical
distancing and not touch equipment. Remember there are other options for
solo outdoor play, like riding on a scooter or a bike. There are also
options for indoor movement --- for example, there are kids' yoga videos
all over YouTube that you and your family can enjoy together.

As always, encourage hand washing when children come in from outside and
before and after meals. Kids
\href{https://www.cdc.gov/handwashing/when-how-handwashing.html}{should
sing ``Happy Birthday'' twice} to know how long to wash their hands, and
then make sure they are drying them thoroughly. There's some evidence
that paper towels are
\href{https://www.nytimes3xbfgragh.onion/2019/04/17/smarter-living/using-public-bathrooms.html}{more
hygienic than hand dryers in public bathrooms}. Hand washing is also
more effective than hand sanitizer, though hand sanitizer can be used
when hand washing is not an option.

\hypertarget{can-we-see-other-families-at-all}{%
\subsection{Can we see other families at
all?}\label{can-we-see-other-families-at-all}}

``We're really encouraging zero play dates at this point,'' said Dr.
O'Leary. Theoretically, he said, it might be feasible for your child to
play outside with one other child if they can keep six feet apart, but
the reality is that keeping younger kids from touching their friends is
nearly impossible. If you have an older child who you know can
absolutely stay six feet apart from his pals in a wide open space, it's
fine for them to play, but monitor them closely. Always make sure they
are washing hands vigorously when they come inside.

\hypertarget{the-store-is-out-of-hand-sanitizer-should-i-make-my-own}{%
\subsection{The store is out of hand sanitizer. Should I make my
own?}\label{the-store-is-out-of-hand-sanitizer-should-i-make-my-own}}

Recipes for homemade hand sanitizer are circulating online, but none of
the experts I spoke to recommended making your own, even if stores have
run out. Many popular brands of hand sanitizer, like Purell or Highmark,
have established concentrations of alcohol, generally
\href{https://www.cdc.gov/handwashing/show-me-the-science-hand-sanitizer.html}{between
60 and 95 percent}, said Dr. Rebecca Pellett Madan, M.D., a pediatric
infectious disease specialist at N.Y.U. Langone's Hassenfeld Children's
Hospital, which helps ensure their effectiveness. Additionally, she
said, ``we have experience using it in hospitals, and we know how
effective it is.'' The same evidence base for homemade recipes doesn't
exist yet.

If you are using store-bought hand sanitizer, make sure that it's at
least 60 percent alcohol and that it fully dries before you or your
child touch anything --- otherwise it won't work as well. Also keep in
mind that hand sanitizers are not as effective when used on ``visibly
dirty or greasy'' hands, according to the C.D.C.

\hypertarget{my-child-has-mild-cold-or-flu-symptoms-should-i-take-him-to-the-hospital}{%
\subsection{My child has mild cold or flu symptoms. Should I take him to
the
hospital?}\label{my-child-has-mild-cold-or-flu-symptoms-should-i-take-him-to-the-hospital}}

No.
\href{https://www.cdc.gov/coronavirus/2019-ncov/downloads/COVID19-symptoms.pdf}{Coronavirus
symptoms can include} fever, dry cough or shortness of breath. If your
child has other symptoms, like mild fever, runny nose or sore throat,
you should call your pediatrician first before going anywhere. ``We want
people who are not critically ill to stay out of the hospital,'' Dr.
Madan said.

If your child develops more severe symptoms, such as trouble breathing,
an inability to eat or drink or a change in behavior, you should visit a
doctor, Dr. Madan said.

\hypertarget{if-my-child-is-very-sick-will-she-be-able-to-get-tested}{%
\subsection{If my child is very sick, will she be able to get
tested?}\label{if-my-child-is-very-sick-will-she-be-able-to-get-tested}}

Unless your child has a history of direct contact with someone who has
tested positive for the virus, a history of travel to affected areas or
is sick enough to be hospitalized, it is unlikely she will be tested.

``Availability of testing depends on where you are,'' Dr. O'Leary said.
``Even in the best case scenario, you can't test everyone because there
aren't enough test kits at this point.'' Older and higher-risk patients
are being prioritized for testing because they tend to develop the most
severe symptoms after infection.

If your child does get tested, it's unclear how quickly her results will
come back --- and the time frame will most likely depend on where you
are, which lab is testing her and how long she's been sick. ``It's all
over the map,'' Dr. O'Leary said. Anecdotally, he has heard about
results taking anywhere from a few hours to seven days, depending on the
state and the level of demand.

\href{https://www.nytimes3xbfgragh.onion/news-event/coronavirus?action=click\&pgtype=Article\&state=default\&region=MAIN_CONTENT_3\&context=storylines_faq}{}

\hypertarget{the-coronavirus-outbreak-}{%
\subsubsection{The Coronavirus Outbreak
›}\label{the-coronavirus-outbreak-}}

\hypertarget{frequently-asked-questions}{%
\paragraph{Frequently Asked
Questions}\label{frequently-asked-questions}}

Updated September 4, 2020

\begin{itemize}
\item ~
  \hypertarget{what-are-the-symptoms-of-coronavirus}{%
  \paragraph{What are the symptoms of
  coronavirus?}\label{what-are-the-symptoms-of-coronavirus}}

  \begin{itemize}
  \tightlist
  \item
    In the beginning, the coronavirus
    \href{https://www.nytimes3xbfgragh.onion/article/coronavirus-facts-history.html?action=click\&pgtype=Article\&state=default\&region=MAIN_CONTENT_3\&context=storylines_faq\#link-6817bab5}{seemed
    like it was primarily a respiratory illness}~--- many patients had
    fever and chills, were weak and tired, and coughed a lot, though
    some people don't show many symptoms at all. Those who seemed
    sickest had pneumonia or acute respiratory distress syndrome and
    received supplemental oxygen. By now, doctors have identified many
    more symptoms and syndromes. In April,
    \href{https://www.nytimes3xbfgragh.onion/2020/04/27/health/coronavirus-symptoms-cdc.html?action=click\&pgtype=Article\&state=default\&region=MAIN_CONTENT_3\&context=storylines_faq}{the
    C.D.C. added to the list of early signs}~sore throat, fever, chills
    and muscle aches. Gastrointestinal upset, such as diarrhea and
    nausea, has also been observed. Another telltale sign of infection
    may be a sudden, profound diminution of one's
    \href{https://www.nytimes3xbfgragh.onion/2020/03/22/health/coronavirus-symptoms-smell-taste.html?action=click\&pgtype=Article\&state=default\&region=MAIN_CONTENT_3\&context=storylines_faq}{sense
    of smell and taste.}~Teenagers and young adults in some cases have
    developed painful red and purple lesions on their fingers and toes
    --- nicknamed ``Covid toe'' --- but few other serious symptoms.
  \end{itemize}
\item ~
  \hypertarget{why-is-it-safer-to-spend-time-together-outside}{%
  \paragraph{Why is it safer to spend time together
  outside?}\label{why-is-it-safer-to-spend-time-together-outside}}

  \begin{itemize}
  \tightlist
  \item
    \href{https://www.nytimes3xbfgragh.onion/2020/05/15/us/coronavirus-what-to-do-outside.html?action=click\&pgtype=Article\&state=default\&region=MAIN_CONTENT_3\&context=storylines_faq}{Outdoor
    gatherings}~lower risk because wind disperses viral droplets, and
    sunlight can kill some of the virus. Open spaces prevent the virus
    from building up in concentrated amounts and being inhaled, which
    can happen when infected people exhale in a confined space for long
    stretches of time, said Dr. Julian W. Tang, a virologist at the
    University of Leicester.
  \end{itemize}
\item ~
  \hypertarget{why-does-standing-six-feet-away-from-others-help}{%
  \paragraph{Why does standing six feet away from others
  help?}\label{why-does-standing-six-feet-away-from-others-help}}

  \begin{itemize}
  \tightlist
  \item
    The coronavirus spreads primarily through droplets from your mouth
    and nose, especially when you cough or sneeze. The C.D.C., one of
    the organizations using that measure,
    \href{https://www.nytimes3xbfgragh.onion/2020/04/14/health/coronavirus-six-feet.html?action=click\&pgtype=Article\&state=default\&region=MAIN_CONTENT_3\&context=storylines_faq}{bases
    its recommendation of six feet}~on the idea that most large droplets
    that people expel when they cough or sneeze will fall to the ground
    within six feet. But six feet has never been a magic number that
    guarantees complete protection. Sneezes, for instance, can launch
    droplets a lot farther than six feet,
    \href{https://jamanetwork.com/journals/jama/fullarticle/2763852}{according
    to a recent study}. It's a rule of thumb: You should be safest
    standing six feet apart outside, especially when it's windy. But
    keep a mask on at all times, even when you think you're far enough
    apart.
  \end{itemize}
\item ~
  \hypertarget{i-have-antibodies-am-i-now-immune}{%
  \paragraph{I have antibodies. Am I now
  immune?}\label{i-have-antibodies-am-i-now-immune}}

  \begin{itemize}
  \tightlist
  \item
    As of right
    now,\href{https://www.nytimes3xbfgragh.onion/2020/07/22/health/covid-antibodies-herd-immunity.html?action=click\&pgtype=Article\&state=default\&region=MAIN_CONTENT_3\&context=storylines_faq}{~that
    seems likely, for at least several months.}~There have been
    frightening accounts of people suffering what seems to be a second
    bout of Covid-19. But experts say these patients may have a
    drawn-out course of infection, with the virus taking a slow toll
    weeks to months after initial exposure.~People infected with the
    coronavirus typically
    \href{https://www.nature.com/articles/s41586-020-2456-9}{produce}~immune
    molecules called antibodies, which are
    \href{https://www.nytimes3xbfgragh.onion/2020/05/07/health/coronavirus-antibody-prevalence.html?action=click\&pgtype=Article\&state=default\&region=MAIN_CONTENT_3\&context=storylines_faq}{protective
    proteins made in response to an
    infection}\href{https://www.nytimes3xbfgragh.onion/2020/05/07/health/coronavirus-antibody-prevalence.html?action=click\&pgtype=Article\&state=default\&region=MAIN_CONTENT_3\&context=storylines_faq}{.
    These antibodies may}~last in the body
    \href{https://www.nature.com/articles/s41591-020-0965-6}{only two to
    three months}, which may seem worrisome, but that's~perfectly normal
    after an acute infection subsides, said Dr. Michael Mina, an
    immunologist at Harvard University. It may be possible to get the
    coronavirus again, but it's highly unlikely that it would be
    possible in a short window of time from initial infection or make
    people sicker the second time.
  \end{itemize}
\item ~
  \hypertarget{what-are-my-rights-if-i-am-worried-about-going-back-to-work}{%
  \paragraph{What are my rights if I am worried about going back to
  work?}\label{what-are-my-rights-if-i-am-worried-about-going-back-to-work}}

  \begin{itemize}
  \tightlist
  \item
    Employers have to provide
    \href{https://www.osha.gov/SLTC/covid-19/standards.html}{a safe
    workplace}~with policies that protect everyone equally.
    \href{https://www.nytimes3xbfgragh.onion/article/coronavirus-money-unemployment.html?action=click\&pgtype=Article\&state=default\&region=MAIN_CONTENT_3\&context=storylines_faq}{And
    if one of your co-workers tests positive for the coronavirus, the
    C.D.C.}~has said that
    \href{https://www.cdc.gov/coronavirus/2019-ncov/community/guidance-business-response.html}{employers
    should tell their employees}~-\/- without giving you the sick
    employee's name -\/- that they may have been exposed to the virus.
  \end{itemize}
\end{itemize}

A greater number of labs will be able to provide testing in the coming
days, according to Dr. O'Leary. But because there may also be increased
demand, it's unclear whether that will speed up testing time overall.

\hypertarget{should-i-take-my-child-to-her-scheduled-well-visit}{%
\subsection{Should I take my child to her scheduled well
visit?}\label{should-i-take-my-child-to-her-scheduled-well-visit}}

If you have a newborn, toddler or young child who is still receiving
immunizations, it is important to take her to her well visit, as long as
you can do it safely,
\href{https://services.aap.org/en/pages/covid-19-clinical-guidance-q-a/}{according
to new guidance released from}the A.A.P. on March 18. ``We don't want
those kids to miss their vaccines,'' Dr. O'Leary said. ``Then we'll have
unprotected infants,'' who will be susceptible to other diseases. Your
child's first newborn visit to a pediatrician is particularly important,
Dr. O'Leary said, as your doctor will want to check her weight, test for
jaundice and help troubleshoot any breastfeeding issues.

If your child is older and has received all of her immunizations, the
A.A.P. recommends that you consider postponing your well visit for the
near term. The organization also has tips for doctors to help keep their
patients, parents and staff safe during their visits, including keeping
well visits in the morning and restricting visits for kids who are sick
to the afternoon. It also urges doctors to see kids who are sick in
different facilities, rooms or floors, and to increase the use of
telemedicine.

\hypertarget{is-it-safe-to-take-my-child-on-public-transportation}{%
\subsection{Is it safe to take my child on public
transportation?}\label{is-it-safe-to-take-my-child-on-public-transportation}}

It is not recommended. New York City has
\href{https://www1.nyc.gov/assets/doh/downloads/pdf/imm/stop-the-spread-poster.pdf}{urged
commuters of all ages to avoid getting on packed subway cars} and to
walk or bike to work, if possible. New York City Mayor Bill de Blasio
said he had no plans to shut down the subway.

\hypertarget{what-if-my-child-has-a-compromised-immune-system}{%
\subsection{What if my child has a compromised immune
system?}\label{what-if-my-child-has-a-compromised-immune-system}}

Because there isn't much information yet about how children react to
this virus, it's tough to say if there are any additional measures folks
should take beyond the isolation already recommended for the general
population, Dr. Hotez said. But Dr. Madan said that parents of children
who have compromised immune systems should be taking it ``day by day.''
If your child has asthma,
\href{https://parenting.nytimes3xbfgragh.onion/parent-life/coronavirus-asthma-kids}{available
evidence suggests they are not at increased risk for the virus}, but
that may change as we learn more.

\hypertarget{should-my-family-be-taking-any-extra-hygienic-measures-beyond-hand-washing}{%
\subsection{Should my family be taking any extra hygienic measures
beyond hand
washing?}\label{should-my-family-be-taking-any-extra-hygienic-measures-beyond-hand-washing}}

You can wash bedsheets and towels more often. Jolie Kerr, a cleaning
expert and frequent New York Times contributor, said that you could also
wash stuffed animals more often
(\href{https://parenting.nytimes3xbfgragh.onion/childrens-health/clean-stuffed-animals}{here's
how}) and clean hard toys with antibacterial wipes regularly ---
particularly after outdoor use.

\hypertarget{should-grandma-still-come-visit}{%
\subsection{Should Grandma still come
visit?}\label{should-grandma-still-come-visit}}

Older adults, especially those who have compromised immune systems, seem
to be the most vulnerable to the new coronavirus. Some areas are calling
for extreme measures: For example, California
\href{https://www.nytimes3xbfgragh.onion/2020/03/15/us/coronavirus-newsom-california-seniors-restaurants-bars.html}{called
for people over 65 to stay in their homes}, while New York state
\href{https://www.governor.ny.gov/news/video-audio-photos-rush-transcript-governor-cuomo-signs-new-york-state-pause-executive-order}{is
asking those over 70 to stay indoors}.

If possible, grandparents should not visit their grandchildren right
now. ``It appears that children and young adults are an important vector
for coronavirus, because they may be infectious even if they don't have
symptoms,'' Dr. Cynthia R. Ambler, M.D., a pediatrician at Northwestern
Medicine in Chicago told Hallie Levine
\href{https://www.nytimes3xbfgragh.onion/2020/03/20/parenting/grandparents-visit-safety-coronavirus.html}{in
a piece for the Times addressing the safety of grandparents}. If
grandparents are the primary caretakers of children, kids and
grandparents alike should
\href{https://www.nytimes3xbfgragh.onion/2020/03/13/world/how-to-wash-your-hands-coronavirus.html}{wash
hands} even more carefully and regularly disinfect frequently touched
surfaces.

\hypertarget{are-newborns-more-susceptible-to-coronavirus}{%
\subsection{Are newborns more susceptible to
coronavirus?}\label{are-newborns-more-susceptible-to-coronavirus}}

There is little data specifically on newborns at this time, Dr. O'Leary
said. Though
\href{https://jamanetwork.com/journals/jamapediatrics/fullarticle/2763787}{a
small study} published on March 25 in JAMA Pediatrics --- which assessed
the outcomes of nearly three dozen babies who were born to mothers with
Covid-19 in Wuhan, China --- provided some initial comfort. As Apoorva
Mandavilli noted in
\href{https://www.nytimes3xbfgragh.onion/2020/03/03/health/coronavirus-pregnant-women-babies.html}{her
piece for The Times}, ``Of 33 newborns born at the Wuhan Children's
Hospital, only three had any signs of the virus, and even their symptoms
were mild, the researchers reported.'' The study authors did not know
for certain what the source of the infection was in the infants who
tested positive.

In general. Dr. O'Leary recommended taking the usual hygienic
precautions you would with infants, who are particularly vulnerable to
other respiratory viruses, like respiratory syncytial virus (R.S.V.) and
the influenza virus: Wash your hands before handling an infant, and
avoid taking your infant to crowded places.

\hypertarget{what-questions-do-parents-have-about-coronavirus}{%
\subsection{What Questions Do Parents Have About
Coronavirus?}\label{what-questions-do-parents-have-about-coronavirus}}

\begin{center}\rule{0.5\linewidth}{\linethickness}\end{center}

Advertisement

\protect\hyperlink{after-bottom}{Continue reading the main story}

\hypertarget{site-index}{%
\subsection{Site Index}\label{site-index}}

\hypertarget{site-information-navigation}{%
\subsection{Site Information
Navigation}\label{site-information-navigation}}

\begin{itemize}
\tightlist
\item
  \href{https://help.nytimes3xbfgragh.onion/hc/en-us/articles/115014792127-Copyright-notice}{©~2020~The
  New York Times Company}
\end{itemize}

\begin{itemize}
\tightlist
\item
  \href{https://www.nytco.com/}{NYTCo}
\item
  \href{https://help.nytimes3xbfgragh.onion/hc/en-us/articles/115015385887-Contact-Us}{Contact
  Us}
\item
  \href{https://www.nytco.com/careers/}{Work with us}
\item
  \href{https://nytmediakit.com/}{Advertise}
\item
  \href{http://www.tbrandstudio.com/}{T Brand Studio}
\item
  \href{https://www.nytimes3xbfgragh.onion/privacy/cookie-policy\#how-do-i-manage-trackers}{Your
  Ad Choices}
\item
  \href{https://www.nytimes3xbfgragh.onion/privacy}{Privacy}
\item
  \href{https://help.nytimes3xbfgragh.onion/hc/en-us/articles/115014893428-Terms-of-service}{Terms
  of Service}
\item
  \href{https://help.nytimes3xbfgragh.onion/hc/en-us/articles/115014893968-Terms-of-sale}{Terms
  of Sale}
\item
  \href{https://spiderbites.nytimes3xbfgragh.onion}{Site Map}
\item
  \href{https://help.nytimes3xbfgragh.onion/hc/en-us}{Help}
\item
  \href{https://www.nytimes3xbfgragh.onion/subscription?campaignId=37WXW}{Subscriptions}
\end{itemize}
