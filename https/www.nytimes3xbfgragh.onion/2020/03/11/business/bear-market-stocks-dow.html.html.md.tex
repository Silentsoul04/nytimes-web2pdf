Sections

SEARCH

\protect\hyperlink{site-content}{Skip to
content}\protect\hyperlink{site-index}{Skip to site index}

\href{https://www.nytimes3xbfgragh.onion/section/business}{Business}

\href{https://myaccount.nytimes3xbfgragh.onion/auth/login?response_type=cookie\&client_id=vi}{}

\href{https://www.nytimes3xbfgragh.onion/section/todayspaper}{Today's
Paper}

\href{/section/business}{Business}\textbar{}Dow Skids Into Bear Market,
Heralding an Uncertain Future

\url{https://nyti.ms/38HtXfy}

\begin{itemize}
\item
\item
\item
\item
\item
\end{itemize}

\href{https://www.nytimes3xbfgragh.onion/news-event/coronavirus?action=click\&pgtype=Article\&state=default\&region=TOP_BANNER\&context=storylines_menu}{The
Coronavirus Outbreak}

\begin{itemize}
\tightlist
\item
  live\href{https://www.nytimes3xbfgragh.onion/2020/08/03/world/coronavirus-covid-19.html?action=click\&pgtype=Article\&state=default\&region=TOP_BANNER\&context=storylines_menu}{Latest
  Updates}
\item
  \href{https://www.nytimes3xbfgragh.onion/interactive/2020/us/coronavirus-us-cases.html?action=click\&pgtype=Article\&state=default\&region=TOP_BANNER\&context=storylines_menu}{Maps
  and Cases}
\item
  \href{https://www.nytimes3xbfgragh.onion/interactive/2020/science/coronavirus-vaccine-tracker.html?action=click\&pgtype=Article\&state=default\&region=TOP_BANNER\&context=storylines_menu}{Vaccine
  Tracker}
\item
  \href{https://www.nytimes3xbfgragh.onion/2020/08/02/us/covid-college-reopening.html?action=click\&pgtype=Article\&state=default\&region=TOP_BANNER\&context=storylines_menu}{College
  Reopening}
\item
  \href{https://www.nytimes3xbfgragh.onion/live/2020/08/03/business/stock-market-today-coronavirus?action=click\&pgtype=Article\&state=default\&region=TOP_BANNER\&context=storylines_menu}{Economy}
\end{itemize}

Advertisement

\protect\hyperlink{after-top}{Continue reading the main story}

Supported by

\protect\hyperlink{after-sponsor}{Continue reading the main story}

\hypertarget{dow-skids-into-bear-market-heralding-an-uncertain-future}{%
\section{Dow Skids Into Bear Market, Heralding an Uncertain
Future}\label{dow-skids-into-bear-market-heralding-an-uncertain-future}}

The Dow Jones industrial average closed down more than 20 percent from
its high last month, ending a bull market that ran for more than a
decade.

\includegraphics{https://static01.graylady3jvrrxbe.onion/images/2020/03/11/business/00bullmarket/merlin_170349099_056b3da4-4eb6-4918-aff3-9f1bf9026bac-articleLarge.jpg?quality=75\&auto=webp\&disable=upscale}

By \href{https://www.nytimes3xbfgragh.onion/by/matt-phillips}{Matt
Phillips}

\begin{itemize}
\item
  Published March 11, 2020Updated April 8, 2020
\item
  \begin{itemize}
  \item
  \item
  \item
  \item
  \item
  \end{itemize}
\end{itemize}

The 11-year bull market, which grew in tandem with one of the longest
economic expansions in United States history, weathered a European debt
crisis and survived President Trump's trade war with China, is dead ---
a casualty of the
\href{https://www.nytimes3xbfgragh.onion/2020/03/11/world/coronavirus-news.html}{global
coronavirus pandemic}.

On Wednesday, the
\href{https://www.nytimes3xbfgragh.onion/2020/04/08/business/media/dow-jones-wall-street-journal-william-lewis.html}{Dow
Jones} industrial average fell 5.9 percent, completing a decline of 20.3
percent from a high reached on Feb. 12 and signaling a bear market. That
threshold --- a fall of 20 percent from a high --- suggests a
fundamental change in investors' view of the economy and could be a
precursor to a recession.

\hypertarget{dow-industrials}{%
\subsection{Dow Industrials}\label{dow-industrials}}

30,000

Feb. 12

peak

Daily closes

28,000

Correction

26,000

24,000

Bear market

Wednesday:

down 20.2\%

from the peak

22,000

FEB.

12

18

24

MARCH 2

9

30,000

Feb. 12

peak

Daily closes

28,000

Correction

26,000

24,000

Bear market

Wednesday:

down 20.2\%

from the peak

22,000

FEB. 12

18

24

MARCH 2

9

Source: Refinitiv

By The New York Times

The S\&P 500 closed down 4.9 percent on Wednesday, 19 percent below its
recent high, as the market entered a realm where prices for stocks tend
to fall rather than rise, investors seek the shelter of safer assets and
fears of falling profits could prompt businesses to reduce investment
and cut payroll.

The catalyst of this sudden fall was an outbreak that has spread swiftly
across the globe, disrupting supply chains from Asia to Europe to the
United States and forcing quarantines and travel restrictions that have
ground entire countries to a halt. The pandemic, which has infected tens
of thousands of people in more than 100 countries, has sent the market
careening downward over the past few weeks --- despite occasional
reversals on bits of promising news.

\hypertarget{latest-updates-economy}{%
\section{\texorpdfstring{\href{https://www.nytimes3xbfgragh.onion/live/2020/08/03/business/stock-market-today-coronavirus?action=click\&pgtype=Article\&state=default\&region=MAIN_CONTENT_1\&context=storylines_live_updates}{Latest
Updates:
Economy}}{Latest Updates: Economy}}\label{latest-updates-economy}}

\href{https://www.nytimes3xbfgragh.onion/live/2020/08/03/business/stock-market-today-coronavirus?action=click\&pgtype=Article\&state=default\&region=MAIN_CONTENT_1\&context=storylines_live_updates\#the-chicago-fed-president-says-its-up-to-congress-to-save-the-economy}{11h
ago}

\href{https://www.nytimes3xbfgragh.onion/live/2020/08/03/business/stock-market-today-coronavirus?action=click\&pgtype=Article\&state=default\&region=MAIN_CONTENT_1\&context=storylines_live_updates\#the-chicago-fed-president-says-its-up-to-congress-to-save-the-economy}{The
Chicago Fed president says it's up to Congress to save the economy.}

\href{https://www.nytimes3xbfgragh.onion/live/2020/08/03/business/stock-market-today-coronavirus?action=click\&pgtype=Article\&state=default\&region=MAIN_CONTENT_1\&context=storylines_live_updates\#faa-says-boeing-has-effectively-mitigated-defects-in-the-737-max}{11h
ago}

\href{https://www.nytimes3xbfgragh.onion/live/2020/08/03/business/stock-market-today-coronavirus?action=click\&pgtype=Article\&state=default\&region=MAIN_CONTENT_1\&context=storylines_live_updates\#faa-says-boeing-has-effectively-mitigated-defects-in-the-737-max}{F.A.A.
says Boeing has `effectively mitigated' defects in the 737 Max.}

\href{https://www.nytimes3xbfgragh.onion/live/2020/08/03/business/stock-market-today-coronavirus?action=click\&pgtype=Article\&state=default\&region=MAIN_CONTENT_1\&context=storylines_live_updates\#small-businesses-got-emergency-loans-but-not-what-they-expected}{14h
ago}

\href{https://www.nytimes3xbfgragh.onion/live/2020/08/03/business/stock-market-today-coronavirus?action=click\&pgtype=Article\&state=default\&region=MAIN_CONTENT_1\&context=storylines_live_updates\#small-businesses-got-emergency-loans-but-not-what-they-expected}{Small
businesses got emergency loans, but not what they expected.}

\href{https://www.nytimes3xbfgragh.onion/live/2020/08/03/business/stock-market-today-coronavirus?action=click\&pgtype=Article\&state=default\&region=MAIN_CONTENT_1\&context=storylines_live_updates}{See
more updates}

More live coverage:
\href{https://www.nytimes3xbfgragh.onion/2020/08/03/world/coronavirus-covid-19.html?action=click\&pgtype=Article\&state=default\&region=MAIN_CONTENT_1\&context=storylines_live_updates}{Global}

Shares nose-dived again on Wednesday after the World Health Organization
declared that the outbreak had reached ``pandemic'' status --- which the
organization
\href{https://www.who.int/csr/disease/swineflu/frequently_asked_questions/pandemic/en/}{defined
in 2010} as ``the worldwide spread of a new disease.''

While the terminology might seem somewhat antiquated, the bulls and
bears are a metaphorical way of describing the changing sentiment among
investors. ``It seems to refer to the psychology of the market,'' said
Robert Shiller, a Yale economics professor.

The drop in the 30-stock
\href{https://www.nytimes3xbfgragh.onion/2020/04/08/business/media/dow-jones-wall-street-journal-william-lewis.html}{Dow}
index was led by an 18 percent collapse in Boeing, after the aerospace
and defense giant announced new efforts to preserve cash in the face of
a global slide in air travel and the fallout from its troubled 737 Max
plane. The high dollar value of Boeing shares makes it one of the most
influential stocks in the Dow, which is a price-weighted index.

Falling share prices have incinerated \$5 trillion in stock market
wealth in less than a month. Few other periods in history compare to the
speed of this bull market's demise from its recent record, reached just
weeks before.

The descent into bear territory is a significant moment for markets,
which often operate as something of an experiment in mass psychology.
Although investors are unlikely to immediately change their buying and
selling strategies after Wednesday, the 20 percent marker --- arbitrary,
but widely agreed upon --- carries symbolic value. Bear markets can also
herald economic recessions --- including the 1929 stock market crash
that preceded the Great Depression, or the last bear market that began
in 2007 as the United States economy plunged into a financial crisis.

A bull market is one of the best indicators of investors' confidence in
the underlying strength of the economy, and their belief that corporate
profits and consumer spending are robust and will continue to grow. For
many, it also signals America's continued pre-eminence in the global
economy, and when a bull market veers off course, investors tend to
panic about the future.

The latest bull market grew out of the ashes of the 2008 financial
crisis, with the S\&P 500 beginning its run in March 2009, and rising
over 300 percent during this period. It lasted about twice as long as
the average bull run. (The gains were 400 percent, if you included
dividend payments.) Since the bull market for the Dow began, it is up
more than 250 percent.

The performance of the S\&P and the Dow together provides a capsule
history of the American economy since it emerged from the recession that
set in after a foreclosure crisis, a panic on Wall Street and a rash of
bank failures beginning in 2007.

In the past decade or so, the United States experienced an economic
expansion with few parallels in history, as the unemployment rate fell
and low interest rates fueled consumer spending, including for major
purchases like houses and cars. Consumer spending has long been the
engine of the American economy, a driving force that has helped lift
markets through rough patches, geopolitical crises and a trade war with
China.

Still, the bull run happened even as income inequality in America has
grown, and the market's riches have not been distributed evenly. Charles
Geisst, a retired Manhattan College professor, and historian of Wall
Street, described the 11-year run as the ``most distrusted bull market''
of recent memory. ``It think it was reflective of everything that we've
seen socially and politically in the last 12 years,'' Mr. Geisst said.
``The average guy did not believe it, didn't believe in the
prosperity.''

And despite the gains, the long climb for the stock market was a steady
simmer that never quite reached a boil. The 13 percent annual pace of
gains was far less than during similar periods of the 1980s and 1990s
that generated mass enthusiasm for ownership of stocks.

In the end, the bull market was felled by a virus. That's because an
outbreak of disease such as the coronavirus presents a more complicated
problem that cannot easily be fixed by monetary or fiscal policy, such
as interest-rate and tax cuts. The outbreak's twin disruptions ---
snapping global supply chains and suppressing consumer demand --- are
leading to enormous challenges for governments scrambling to contain a
disease while also steadying the economy.

Advertisement

\protect\hyperlink{after-bottom}{Continue reading the main story}

\hypertarget{site-index}{%
\subsection{Site Index}\label{site-index}}

\hypertarget{site-information-navigation}{%
\subsection{Site Information
Navigation}\label{site-information-navigation}}

\begin{itemize}
\tightlist
\item
  \href{https://help.nytimes3xbfgragh.onion/hc/en-us/articles/115014792127-Copyright-notice}{©~2020~The
  New York Times Company}
\end{itemize}

\begin{itemize}
\tightlist
\item
  \href{https://www.nytco.com/}{NYTCo}
\item
  \href{https://help.nytimes3xbfgragh.onion/hc/en-us/articles/115015385887-Contact-Us}{Contact
  Us}
\item
  \href{https://www.nytco.com/careers/}{Work with us}
\item
  \href{https://nytmediakit.com/}{Advertise}
\item
  \href{http://www.tbrandstudio.com/}{T Brand Studio}
\item
  \href{https://www.nytimes3xbfgragh.onion/privacy/cookie-policy\#how-do-i-manage-trackers}{Your
  Ad Choices}
\item
  \href{https://www.nytimes3xbfgragh.onion/privacy}{Privacy}
\item
  \href{https://help.nytimes3xbfgragh.onion/hc/en-us/articles/115014893428-Terms-of-service}{Terms
  of Service}
\item
  \href{https://help.nytimes3xbfgragh.onion/hc/en-us/articles/115014893968-Terms-of-sale}{Terms
  of Sale}
\item
  \href{https://spiderbites.nytimes3xbfgragh.onion}{Site Map}
\item
  \href{https://help.nytimes3xbfgragh.onion/hc/en-us}{Help}
\item
  \href{https://www.nytimes3xbfgragh.onion/subscription?campaignId=37WXW}{Subscriptions}
\end{itemize}
