Sections

SEARCH

\protect\hyperlink{site-content}{Skip to
content}\protect\hyperlink{site-index}{Skip to site index}

\href{https://www.nytimes3xbfgragh.onion/section/world}{World}

\href{https://myaccount.nytimes3xbfgragh.onion/auth/login?response_type=cookie\&client_id=vi}{}

\href{https://www.nytimes3xbfgragh.onion/section/todayspaper}{Today's
Paper}

\href{/section/world}{World}\textbar{}Pelosi and Trump Reach Deal on a
Relief Package

\url{https://nyti.ms/2TMzs8w}

\begin{itemize}
\item
\item
\item
\item
\item
\item
\end{itemize}

\hypertarget{the-coronavirus-outbreak}{%
\subsubsection{\texorpdfstring{\href{https://www.nytimes3xbfgragh.onion/news-event/coronavirus?name=styln-coronavirus-national\&region=TOP_BANNER\&variant=undefined\&block=storyline_menu_recirc\&action=click\&pgtype=Article\&impression_id=9fcf9940-e39b-11ea-ad54-c52f02c320cd}{The
Coronavirus
Outbreak}}{The Coronavirus Outbreak}}\label{the-coronavirus-outbreak}}

\begin{itemize}
\tightlist
\item
  live\href{https://www.nytimes3xbfgragh.onion/2020/08/21/world/covid-19-coronavirus.html?name=styln-coronavirus-national\&region=TOP_BANNER\&variant=undefined\&block=storyline_menu_recirc\&action=click\&pgtype=Article\&impression_id=9fcf9941-e39b-11ea-ad54-c52f02c320cd}{Latest
  Updates}
\item
  \href{https://www.nytimes3xbfgragh.onion/interactive/2020/us/coronavirus-us-cases.html?name=styln-coronavirus-national\&region=TOP_BANNER\&variant=undefined\&block=storyline_menu_recirc\&action=click\&pgtype=Article\&impression_id=9fcf9942-e39b-11ea-ad54-c52f02c320cd}{Maps
  and Cases}
\item
  \href{https://www.nytimes3xbfgragh.onion/interactive/2020/science/coronavirus-vaccine-tracker.html?name=styln-coronavirus-national\&region=TOP_BANNER\&variant=undefined\&block=storyline_menu_recirc\&action=click\&pgtype=Article\&impression_id=9fcf9943-e39b-11ea-ad54-c52f02c320cd}{Vaccine
  Tracker}
\item
  \href{https://www.nytimes3xbfgragh.onion/2020/08/19/us/colleges-closing-covid.html?name=styln-coronavirus-national\&region=TOP_BANNER\&variant=undefined\&block=storyline_menu_recirc\&action=click\&pgtype=Article\&impression_id=9fcfc050-e39b-11ea-ad54-c52f02c320cd}{Colleges
  Closing}
\item
  \href{https://www.nytimes3xbfgragh.onion/live/2020/08/20/business/stock-market-today-coronavirus?name=styln-coronavirus-national\&region=TOP_BANNER\&variant=undefined\&block=storyline_menu_recirc\&action=click\&pgtype=Article\&impression_id=9fcfc051-e39b-11ea-ad54-c52f02c320cd}{Economy}
\end{itemize}

Advertisement

\protect\hyperlink{after-top}{Continue reading the main story}

Supported by

\protect\hyperlink{after-sponsor}{Continue reading the main story}

\hypertarget{pelosi-and-trump-reach-deal-on-a-relief-package}{%
\section{Pelosi and Trump Reach Deal on a Relief
Package}\label{pelosi-and-trump-reach-deal-on-a-relief-package}}

Stocks rose as President Trump declared a national emergency and said he
was making \$50 billion available. The House was set to vote on the
relief plan early Saturday.

\begin{itemize}
\item
  Published March 13, 2020Updated March 23, 2020
\item
  \begin{itemize}
  \item
  \item
  \item
  \item
  \item
  \item
  \end{itemize}
\end{itemize}

\emph{Read the latest developments in the}
\href{https://www.nytimes3xbfgragh.onion/2020/03/23/world/coronavirus-news.html}{\emph{coronavirus
outbreak here}}\emph{.} \emph{This briefing is no longer updating.}

\hypertarget{heres-what-you-need-to-know}{%
\subsubsection{Here's what you need to
know:}\label{heres-what-you-need-to-know}}

\begin{itemize}
\tightlist
\item
  \protect\hyperlink{link-47cd8806}{Nancy Pelosi and the White House
  reached a deal on a sweeping relief package.}
\item
  \protect\hyperlink{link-37509802}{Trump declares a national emergency,
  ``two very big words.''}
\item
  \protect\hyperlink{link-3359d40e}{Wall Street rebounds during Trump's
  address.}
\item
  \protect\hyperlink{link-1a92c290}{After missteps, the Trump
  administration refocuses on testing.}
\item
  \protect\hyperlink{link-50441278}{Trump promised a testing website by
  Google. He got key details wrong.}
\item
  \protect\hyperlink{link-62cf8d19}{How bad could the epidemic be in the
  United States?}
\item
  \protect\hyperlink{link-45290ade}{Gyms, apartment buildings and
  offices: How to manage the outbreak.}
\end{itemize}

\hypertarget{nancy-pelosi-and-the-white-house-reached-a-deal-on-a-sweeping-relief-package}{%
\subsection{Nancy Pelosi and the White House reached a deal on a
sweeping relief
package.}\label{nancy-pelosi-and-the-white-house-reached-a-deal-on-a-sweeping-relief-package}}

\includegraphics{https://static01.graylady3jvrrxbe.onion/images/2020/03/13/world/13virus-briefing-pelosi/13virus-briefing-pelosi-articleLarge.jpg?quality=75\&auto=webp\&disable=upscale}

Congress and the White House
\href{https://www.nytimes3xbfgragh.onion/2020/03/13/us/politics/trump-coronavirus-relief-congress.html}{struck
a deal on Friday} on a sweeping relief package to assist people affected
by the outbreak of the coronavirus, moving to confront a growing
pandemic that has upended lives and wreaked havoc on financial markets.

The House was set to vote on the plan on Friday evening, after a
roller-coaster day of negotiations that threatened to veer off track as
President Trump criticized the plan during a White House Rose Garden
news conference in which he declared a national emergency.

Instead, by dusk, Speaker Nancy Pelosi wrote to House Democrats saying,
``We are proud to have reached an agreement with the administration to
resolve outstanding challenges.''

Not long after, even as congressional and administration aides
negotiated the final legislative language of the compromise, Steven
Mnuchin, the Treasury secretary, confirmed it would move forward.

``We have an agreement that reflects what the president talked about in
his speech the other night,'' Mr. Mnuchin said on Fox News. He said Mr.
Trump was focused on making sure ``hard-working Americans don't lose
their compensation because they have to be home.''

The measure includes two weeks of paid sick leave and up to three months
of paid family and medical leave, enhanced unemployment benefits, free
virus testing including for those who lack insurance, additional food
aid and federal funds for Medicaid.

The deal is a product of an intense round of talks that unfolded between
Ms. Pelosi and Mr. Mnuchin as financial markets swung wildly amid
uncertainty about the spiraling crisis.

\href{https://www.nytimes3xbfgragh.onion/interactive/2020/us/coronavirus-us-cases.html}{Confirmed
coronavirus cases in the United States} have climbed to more than 2,100,
even with sparse testing, and the death toll has risen to at least 48.
West Virginia was the only state yet to report a known case of the virus
by Friday evening. The United States is facing the prospect that those
numbers could grow exponentially, as they did in China, Italy, South
Korea and other countries.

\emph{{[}Read:}
\href{http://www.nytimes3xbfgragh.onion/2020/03/19/world/europe/italy-coronavirus-frontlines-quarantine.html}{\emph{My
life on Italy's coronavirus frontlines, and in quarantine}}\emph{.{]}}

\hypertarget{trump-declares-a-national-emergency-two-very-big-words}{%
\subsection{Trump declares a national emergency, ``two very big
words.''}\label{trump-declares-a-national-emergency-two-very-big-words}}

\includegraphics{https://static01.graylady3jvrrxbe.onion/images/2020/03/13/us/13virus-briefing02/merlin_170481927_30f62570-82f7-49bc-85e4-067c34de0da2-videoSixteenByNine3000.jpg}

Mr. Trump on Friday afternoon officially declared a national emergency
that he said would give states and territories access to up to \$50
billion in federal funds to combat the spreading coronavirus epidemic.

In a live address in the White House Rose Garden, he also gave broad new
authority to the health secretary, Alex Azar, who he said would now be
able to waive regulations, giving doctors and hospitals more flexibility
to respond to the virus, including making it easier to treat people
remotely.

``I am officially declaring a national emergency, two very big words,''
Mr. Trump said, adding, ``I'm urging every state to set up emergency
operations centers effective immediately.''

Mr. Trump said he was waiving interest on student loans, and that with
oil prices low, the government would buy large quantities of crude oil
for the nation's strategic reserve.

His comments marked the first time he has addressed the coronavirus as a
problem within the country's borders, not just something that needed to
be kept out with travel restrictions. But the optics of the address
offered a contrast to the social distancing that many experts recommend:
Mr. Trump and the top advisers he invited to speak crowded together
around the lectern and shared a microphone.

Mr. Trump, who has been accused of downplaying the crisis, said that
millions of virus testing kits would become available, but added that he
did not think so many would be needed.

``We don't want everybody taking this test,'' he said. ``It's totally
unnecessary.''

``This will pass, this will pass through, and we will be even stronger
for it,'' the president said.

Asked if he would be tested for the coronavirus because of his contact
at his Florida estate, Mar-a-Lago, with an infected Brazilian official,
he said, ``most likely, yeah,'' countering earlier White House
statements that he would not be tested.

``I think I will do it anyway,'' he said. ``Fairly soon.''

\hypertarget{wall-street-rebounds-during-trumps-address}{%
\subsection{Wall Street rebounds during Trump's
address.}\label{wall-street-rebounds-during-trumps-address}}

Image

The floor of the New York Stock Exchange on Friday.Credit...Lucas
Jackson/Reuters

Stocks rallied on Friday, rebounding from their worst day in more than
30 years after Mr. Trump said leaders of private U.S. companies had
agreed to help with efforts to test for the coronavirus and declared a
national emergency that would free billions in funding for states and
territories.

The S\&P 500 rose more than 9 percent, with most of the gains coming as
government officials and business executives spoke at a news conference
at the White House. The chief executives of Walmart, Target and
Walgreens all said they are making facilities available for testing.

Financial markets have been nothing if not inconsistent for the past
three weeks, plunging and then rising, and then plunging again. Each day
brought new measures to contain the outbreak and new worries that the
economy, workers and businesses would take a hit as a result of them.

\hypertarget{after-missteps-the-trump-administration-refocuses-on-testing}{%
\subsection{After missteps, the Trump administration refocuses on
testing.}\label{after-missteps-the-trump-administration-refocuses-on-testing}}

Image

Testing people for the coronavirus at a drive-through clinic in Denver,
Colo., on Thursday.Credit...Chet Strange for The New York Times

The Trump administration moved on Friday to drastically speed up
coronavirus testing, rushing to catch up with surging demand for tests.

The government gave the Swiss health care giant Roche emergency
permission to sell its three-and-a-half-hour test to U.S. labs, and said
it was awarding over a million dollars to two companies to accelerate
development of one-hour tests.

\hypertarget{latest-updates-the-coronavirus-outbreak}{%
\section{\texorpdfstring{\href{https://www.nytimes3xbfgragh.onion/2020/08/21/world/covid-19-coronavirus.html?action=click\&pgtype=Article\&state=default\&region=MAIN_CONTENT_1\&context=storylines_live_updates}{Latest
Updates: The Coronavirus
Outbreak}}{Latest Updates: The Coronavirus Outbreak}}\label{latest-updates-the-coronavirus-outbreak}}

Updated 2020-08-21T10:45:05.612Z

\begin{itemize}
\tightlist
\item
  \href{https://www.nytimes3xbfgragh.onion/2020/08/21/world/covid-19-coronavirus.html?action=click\&pgtype=Article\&state=default\&region=MAIN_CONTENT_1\&context=storylines_live_updates\#link-4690b6aa}{Shutdowns,
  warnings and scoldings follow gatherings on college campuses.}
\item
  \href{https://www.nytimes3xbfgragh.onion/2020/08/21/world/covid-19-coronavirus.html?action=click\&pgtype=Article\&state=default\&region=MAIN_CONTENT_1\&context=storylines_live_updates\#link-324af071}{As
  he accepts the Democratic nomination, Biden knocks Trump's pandemic
  response.}
\item
  \href{https://www.nytimes3xbfgragh.onion/2020/08/21/world/covid-19-coronavirus.html?action=click\&pgtype=Article\&state=default\&region=MAIN_CONTENT_1\&context=storylines_live_updates\#link-35890b73}{Hundreds
  of doctors in Kenya go on strike over their pay and protective gear.}
\end{itemize}

\href{https://www.nytimes3xbfgragh.onion/2020/08/21/world/covid-19-coronavirus.html?action=click\&pgtype=Article\&state=default\&region=MAIN_CONTENT_1\&context=storylines_live_updates}{See
more updates}

More live coverage:
\href{https://www.nytimes3xbfgragh.onion/live/2020/08/20/business/stock-market-today-coronavirus?action=click\&pgtype=Article\&state=default\&region=MAIN_CONTENT_1\&context=storylines_live_updates}{Markets}

Testing has lagged in the country, infuriating the public, local leaders
and members of Congress.
\href{https://www.nytimes3xbfgragh.onion/2020/03/12/us/coronavirus-testing-challenges.html}{Sick
people across the country say they are being denied tests.}
Administration officials have promised repeatedly that enormous numbers
of tests would soon be available, only to have the reality fall far
short.

``I don't take responsibility at all,'' President Trump said in response
to a reporter's question on Friday, ``because we were given a set of
circumstances and we were given rules, regulations and specifications
from a different time.''

While South Korea is testing 10,000 people a day, overall U.S. state and
federal testing has yet to log even 15,000, according to the Centers for
Disease Control and Prevention.

Speaker Nancy Pelosi highlighted the urgency on Friday, while discussing
an emergency spending package she said the House would pass later in the
day, saying, ``The three most important parts of this bill are testing,
testing, testing.''

On Friday, the federal government said that it would allow New York
State's public health department to authorize local labs to perform
coronavirus tests.

By next week, New York could be conducting 6,000 tests a day, the
governor said. On Friday, the state opened a ``drive through'' testing
facility in New Rochelle, a city north of New York City
\href{https://www.nytimes3xbfgragh.onion/2020/03/12/nyregion/coronavirus-new-rochelle-containment.html}{that
has been at the center of the state's epidemic}.

In the United States, an avalanche of cancellations and closings
gathered speed.
\href{https://www.nytimes3xbfgragh.onion/2020/03/12/theater/coronavirus-broadway-shutdown.html}{Broadway
has gone dark} and
\href{https://www.nytimes3xbfgragh.onion/2020/03/12/business/disneyland-coronavirus.html}{Disney
parks are shutting their doors} for the first time since the Sept. 11,
2001, attacks. In Paris, the Louvre and the Eiffel Tower have closed
until further notice.

``My guess is there are thousands and thousands of cases walking around
the state of New York,'' Gov. Andrew M. Cuomo said on Friday,
underscoring that the official count of 421 cases, even as it inches up,
is not representative of total spread of the virus in the state.

Nearly every sport has been affected; the
\href{https://www.nytimes3xbfgragh.onion/2020/03/13/sports/the-boston-marathon-has-been-postponed-until-sept-14.html}{Boston
Marathon} and the Masters golf tournament were called off on Friday.

\hypertarget{trump-promised-a-testing-website-by-google-he-got-key-details-wrong}{%
\subsection{Trump promised a testing website by Google. He got key
details
wrong.}\label{trump-promised-a-testing-website-by-google-he-got-key-details-wrong}}

Image

President Donald Trump at a press conference in the Rose Garden of the
White House on Friday.Credit...Erin Schaff/The New York Times

Verily, a life sciences unit of Google's parent company Alphabet, is
working on a way to direct individuals with a high risk of coronavirus
infection to testing sites. But the program will not be as sweeping as
President Trump suggested in his public remarks.

``I want to thank Google,'' Mr. Trump said from the Rose Garden.
``Google is helping to develop a website, it's going to be very quickly
done, unlike websites of the past, to determine whether a test is
warranted and to facilitate testing at a nearby convenient location.''

Google's shares surged, to a gain of more than 9 percent, as Mr. Trump
spoke.

Late Friday, Carolyn Wang, a spokeswoman for Verily, said that the aim
was to make a website that helps triage people for virus screening
available by Monday, but that it would be limited to testing sites in
the Bay Area. If the pilot goes well, Verily aims to deploy the project
nationwide, but there is no timetable for a national rollout.

The website was originally intended only for health care workers, Ms.
Wang said, but Mr. Trump's statement prompted the company to plan to
make it available to the public.

\hypertarget{how-bad-could-the-epidemic-be-in-the-united-states}{%
\subsection{How bad could the epidemic be in the United
States?}\label{how-bad-could-the-epidemic-be-in-the-united-states}}

Image

The C.D.C. scenarios have not been publicly disclosed. Without an
understanding of how experts view the threat, it remains unclear how far
Americans will go in adopting socially disruptive steps that could help
avert deaths.Credit...Erin Schaff/The New York Times

Trying to answer that question, epidemiologists arrived at some
terrifying estimates of what would happen if --- and this was a very big
``if'' --- nothing was done to stop the spread: more than a million
Americans could die, and hospitals would be swamped with many times more
patients than beds.

Last month, officials at the Centers for Disease Control and Prevention
and experts from around the world conferred quietly about the range of
possibilities, with what was known then about the virus. The agency
presented a range of possibilities based on variables that remain hard
to pin down, like how contagious the virus is, and how severe the
resulting illnesses are.

A key, unknown factor is how quickly the virus would spread, which could
determine whether a medical system
\href{https://www.aha.org/statistics/fast-facts-us-hospitals}{with fewer
than a million hospital beds} could handle the wave of patients.

The estimates were 160 million to 214 million people in the United
States infected, or nearly half to two-thirds of the population; 2.4
million to 21 million people needing hospitalization; and 200,000 to 1.7
million dead. The results were not publicized, but have been reviewed by
The New York Times.

Those figures represented the worst-case scenario, with no organized
intervention, and clearly government at all levels is intervening, as
are private organizations and individuals. But they give an idea of the
potential seriousness of the disease.

``There is a lot of room for improvement if we act appropriately,'' said
Lauren Gardner, an associate professor at the Johns Hopkins Whiting
School of Engineering who models epidemics.

\hypertarget{gyms-apartment-buildings-and-offices-how-to-manage-the-outbreak}{%
\subsection{Gyms, apartment buildings and offices: How to manage the
outbreak.}\label{gyms-apartment-buildings-and-offices-how-to-manage-the-outbreak}}

Today, we look at how the places you interact with daily are ensuring
they stay safe while still being able to function, including how gyms
should be disinfecting their equipment, new guidance for building
managers, and how needed changes may affect workers.

\hypertarget{low-income-neighborhoods-fear-that-their-communities-are-becoming-sacrifice-zones}{%
\subsection{Low-income neighborhoods fear that their communities are
becoming sacrifice
zones.}\label{low-income-neighborhoods-fear-that-their-communities-are-becoming-sacrifice-zones}}

Image

A former Econo Lodge motel that is being used by King County, Wash., to
isolate people who might have been exposed to the
coronavirus.Credit...Andrew Burton for The New York Times

Local officials in Washington State recently settled on two locations to
house people exposed to the coronavirus who may have no safe place to
isolate themselves: a former Econo Lodge Motel in the city of Kent,
south of Seattle, and another location in an unincorporated part of the
county, both of which serve mostly lower-income neighborhoods.

The plan
\href{https://www.nytimes3xbfgragh.onion/2020/03/13/us/coronavirus-seattle-isolation-housing.html}{quickly
ran into trouble}. Kent's mayor, Dana Ralph, said that neither of the
two communities had experienced any cases of the virus before the new
facilities opened, and that residents of her town have wondered if their
city was chosen to protect wealthier neighborhoods around Seattle.

Residents and political leaders circulated petitions, complaining that
their communities were becoming sacrifice zones for the coronavirus and
warning that businesses could suffer and neighbors could become
infected.

Then on Friday some of their fears were confirmed: One of the first
residents at the converted motel, a homeless person who was awaiting the
results of a coronavirus test, ignored the instructions of a security
guard and wandered away, eventually taking a northbound bus, King County
officials said in a statement.

The bus was taken out of service for cleaning and sanitation; the
community was still reeling.

A County Council member, Girmay Zahilay, said that even in a health
crisis, the county has an obligation to all of its residents.

``During this unprecedented crisis, we have to make sure that the
consequences of our decisions don't fall on the most vulnerable and
marginalized communities,'' he said. ``That's how we all get through
this together.''

\hypertarget{a-positive-test-raises-concerns-about-the-virus-in-prisons}{%
\subsection{A positive test raises concerns about the virus in
prisons.}\label{a-positive-test-raises-concerns-about-the-virus-in-prisons}}

An employee who works in a Washington State prison tested positive for
the coronavirus, a spokeswoman for the state prison system said on
Friday, in what appeared to be the first reported case of the virus in a
person tied to a prison.

The employee last worked at the Monroe Correctional Complex, about 45
minutes from Seattle, on Sunday, and the test came back positive on
Thursday. A day earlier, Brad Burkhart, the sheriff in Hancock County,
Indiana, said a staff member at the local jail had tested positive for
the virus and was quarantining at home, as are two other staff members
who had contact with the employee.

\href{https://www.nytimes3xbfgragh.onion/news-event/coronavirus?action=click\&pgtype=Article\&state=default\&region=MAIN_CONTENT_3\&context=storylines_faq}{}

\hypertarget{the-coronavirus-outbreak-}{%
\subsubsection{The Coronavirus Outbreak
›}\label{the-coronavirus-outbreak-}}

\hypertarget{frequently-asked-questions}{%
\paragraph{Frequently Asked
Questions}\label{frequently-asked-questions}}

Updated August 17, 2020

\begin{itemize}
\item ~
  \hypertarget{why-does-standing-six-feet-away-from-others-help}{%
  \paragraph{Why does standing six feet away from others
  help?}\label{why-does-standing-six-feet-away-from-others-help}}

  \begin{itemize}
  \tightlist
  \item
    The coronavirus spreads primarily through droplets from your mouth
    and nose, especially when you cough or sneeze. The C.D.C., one of
    the organizations using that measure,
    \href{https://www.nytimes3xbfgragh.onion/2020/04/14/health/coronavirus-six-feet.html?action=click\&pgtype=Article\&state=default\&region=MAIN_CONTENT_3\&context=storylines_faq}{bases
    its recommendation of six feet} on the idea that most large droplets
    that people expel when they cough or sneeze will fall to the ground
    within six feet. But six feet has never been a magic number that
    guarantees complete protection. Sneezes, for instance, can launch
    droplets a lot farther than six feet,
    \href{https://jamanetwork.com/journals/jama/fullarticle/2763852}{according
    to a recent study}. It's a rule of thumb: You should be safest
    standing six feet apart outside, especially when it's windy. But
    keep a mask on at all times, even when you think you're far enough
    apart.
  \end{itemize}
\item ~
  \hypertarget{i-have-antibodies-am-i-now-immune}{%
  \paragraph{I have antibodies. Am I now
  immune?}\label{i-have-antibodies-am-i-now-immune}}

  \begin{itemize}
  \tightlist
  \item
    As of right
    now,\href{https://www.nytimes3xbfgragh.onion/2020/07/22/health/covid-antibodies-herd-immunity.html?action=click\&pgtype=Article\&state=default\&region=MAIN_CONTENT_3\&context=storylines_faq}{that
    seems likely, for at least several months.} There have been
    frightening accounts of people suffering what seems to be a second
    bout of Covid-19. But experts say these patients may have a
    drawn-out course of infection, with the virus taking a slow toll
    weeks to months after initial exposure. People infected with the
    coronavirus typically
    \href{https://www.nature.com/articles/s41586-020-2456-9}{produce}
    immune molecules called antibodies, which are
    \href{https://www.nytimes3xbfgragh.onion/2020/05/07/health/coronavirus-antibody-prevalence.html?action=click\&pgtype=Article\&state=default\&region=MAIN_CONTENT_3\&context=storylines_faq}{protective
    proteins made in response to an
    infection}\href{https://www.nytimes3xbfgragh.onion/2020/05/07/health/coronavirus-antibody-prevalence.html?action=click\&pgtype=Article\&state=default\&region=MAIN_CONTENT_3\&context=storylines_faq}{.
    These antibodies may} last in the body
    \href{https://www.nature.com/articles/s41591-020-0965-6}{only two to
    three months}, which may seem worrisome, but that's perfectly normal
    after an acute infection subsides, said Dr. Michael Mina, an
    immunologist at Harvard University. It may be possible to get the
    coronavirus again, but it's highly unlikely that it would be
    possible in a short window of time from initial infection or make
    people sicker the second time.
  \end{itemize}
\item ~
  \hypertarget{im-a-small-business-owner-can-i-get-relief}{%
  \paragraph{I'm a small-business owner. Can I get
  relief?}\label{im-a-small-business-owner-can-i-get-relief}}

  \begin{itemize}
  \tightlist
  \item
    The
    \href{https://www.nytimes3xbfgragh.onion/article/small-business-loans-stimulus-grants-freelancers-coronavirus.html?action=click\&pgtype=Article\&state=default\&region=MAIN_CONTENT_3\&context=storylines_faq}{stimulus
    bills enacted in March} offer help for the millions of American
    small businesses. Those eligible for aid are businesses and
    nonprofit organizations with fewer than 500 workers, including sole
    proprietorships, independent contractors and freelancers. Some
    larger companies in some industries are also eligible. The help
    being offered, which is being managed by the Small Business
    Administration, includes the Paycheck Protection Program and the
    Economic Injury Disaster Loan program. But lots of folks have
    \href{https://www.nytimes3xbfgragh.onion/interactive/2020/05/07/business/small-business-loans-coronavirus.html?action=click\&pgtype=Article\&state=default\&region=MAIN_CONTENT_3\&context=storylines_faq}{not
    yet seen payouts.} Even those who have received help are confused:
    The rules are draconian, and some are stuck sitting on
    \href{https://www.nytimes3xbfgragh.onion/2020/05/02/business/economy/loans-coronavirus-small-business.html?action=click\&pgtype=Article\&state=default\&region=MAIN_CONTENT_3\&context=storylines_faq}{money
    they don't know how to use.} Many small-business owners are getting
    less than they expected or
    \href{https://www.nytimes3xbfgragh.onion/2020/06/10/business/Small-business-loans-ppp.html?action=click\&pgtype=Article\&state=default\&region=MAIN_CONTENT_3\&context=storylines_faq}{not
    hearing anything at all.}
  \end{itemize}
\item ~
  \hypertarget{what-are-my-rights-if-i-am-worried-about-going-back-to-work}{%
  \paragraph{What are my rights if I am worried about going back to
  work?}\label{what-are-my-rights-if-i-am-worried-about-going-back-to-work}}

  \begin{itemize}
  \tightlist
  \item
    Employers have to provide
    \href{https://www.osha.gov/SLTC/covid-19/standards.html}{a safe
    workplace} with policies that protect everyone equally.
    \href{https://www.nytimes3xbfgragh.onion/article/coronavirus-money-unemployment.html?action=click\&pgtype=Article\&state=default\&region=MAIN_CONTENT_3\&context=storylines_faq}{And
    if one of your co-workers tests positive for the coronavirus, the
    C.D.C.} has said that
    \href{https://www.cdc.gov/coronavirus/2019-ncov/community/guidance-business-response.html}{employers
    should tell their employees} -\/- without giving you the sick
    employee's name -\/- that they may have been exposed to the virus.
  \end{itemize}
\item ~
  \hypertarget{what-is-school-going-to-look-like-in-september}{%
  \paragraph{What is school going to look like in
  September?}\label{what-is-school-going-to-look-like-in-september}}

  \begin{itemize}
  \tightlist
  \item
    It is unlikely that many schools will return to a normal schedule
    this fall, requiring the grind of
    \href{https://www.nytimes3xbfgragh.onion/2020/06/05/us/coronavirus-education-lost-learning.html?action=click\&pgtype=Article\&state=default\&region=MAIN_CONTENT_3\&context=storylines_faq}{online
    learning},
    \href{https://www.nytimes3xbfgragh.onion/2020/05/29/us/coronavirus-child-care-centers.html?action=click\&pgtype=Article\&state=default\&region=MAIN_CONTENT_3\&context=storylines_faq}{makeshift
    child care} and
    \href{https://www.nytimes3xbfgragh.onion/2020/06/03/business/economy/coronavirus-working-women.html?action=click\&pgtype=Article\&state=default\&region=MAIN_CONTENT_3\&context=storylines_faq}{stunted
    workdays} to continue. California's two largest public school
    districts --- Los Angeles and San Diego --- said on July 13, that
    \href{https://www.nytimes3xbfgragh.onion/2020/07/13/us/lausd-san-diego-school-reopening.html?action=click\&pgtype=Article\&state=default\&region=MAIN_CONTENT_3\&context=storylines_faq}{instruction
    will be remote-only in the fall}, citing concerns that surging
    coronavirus infections in their areas pose too dire a risk for
    students and teachers. Together, the two districts enroll some
    825,000 students. They are the largest in the country so far to
    abandon plans for even a partial physical return to classrooms when
    they reopen in August. For other districts, the solution won't be an
    all-or-nothing approach.
    \href{https://bioethics.jhu.edu/research-and-outreach/projects/eschool-initiative/school-policy-tracker/}{Many
    systems}, including the nation's largest, New York City, are
    devising
    \href{https://www.nytimes3xbfgragh.onion/2020/06/26/us/coronavirus-schools-reopen-fall.html?action=click\&pgtype=Article\&state=default\&region=MAIN_CONTENT_3\&context=storylines_faq}{hybrid
    plans} that involve spending some days in classrooms and other days
    online. There's no national policy on this yet, so check with your
    municipal school system regularly to see what is happening in your
    community.
  \end{itemize}
\end{itemize}

The revelation comes as the Bureau of Prisons, which runs the federal
prisons that hold more than 175,000 people, suspended all visits to
prisoners for 30 days, including most by lawyers. The bureau said the
densely packed nature of prisons ``creates a risk of infection and
transmission for inmates and staff.''

Many
\href{https://www.texastribune.org/2020/03/13/texas-prisons-cancel-visitation-coronavirus/}{state}\href{https://www.doc.wa.gov/information/alerts.htm\#prison-alerts}{prison}\href{https://www.charlotteobserver.com/news/local/article241173191.html}{systems}
and local jails, where
the\href{https://www.bjs.gov/content/pub/pdf/p17.pdf}{vast majority} of
imprisoned people are held, also suspended visits this week. A jail in
Santa Clara County, Calif., placed inmates in quarantine after a visitor
later tested positive for the virus,
\href{https://www.mercurynews.com/2020/03/13/coronavirus-inmates-quarantined-at-santa-clara-county-jail-after-defense-attorney-tests-positive/}{The
Mercury News reported}.

Advocates have sounded alarms over whether U.S. correctional facilities
are adequately prepared to stop an outbreak within their walls. Much of
the advice given by the C.D.C. --- such as staying away from sick people
and disinfecting surfaces --- can
be\href{https://www.usatoday.com/story/news/investigations/2020/03/07/prison-policies-inmates-best-coronavirus-practices/4978412002/}{nearly
impossible to follow} in prison.

\hypertarget{los-angeles-schools-will-be-closed-starting-monday-but-new-york-is-resisting-calls-to-follow-suit}{%
\subsection{Los Angeles schools will be closed starting Monday, but New
York is resisting calls to follow
suit.}\label{los-angeles-schools-will-be-closed-starting-monday-but-new-york-is-resisting-calls-to-follow-suit}}

Image

Students leaving Palms Middle School in Los Angeles on Thursday. Los
Angeles and San Diego have announced they will close schools, beginning
Monday.Credit...Jenna Schoenefeld for The New York Times

Mayor Bill de Blasio of New York said on Friday
\href{https://www.nytimes3xbfgragh.onion/2020/03/13/nyregion/coronavirus-nyc-schools.html}{that
public schools would remain open}, even as three more schools closed for
the day after coronavirus cases were reported there.

``There are three pillars to protecting this city and the long-term
health and safety of our people: Our schools, our mass transit and our
health care system,'' Mr. de Blasio said. ``Those three are interrelated
deeply. You take one out of the equation and it affects the others. My
goal is to keep all three of those going.''

New York is the country's largest school system, with 1.1 million
students, about 114,000 of them homeless. Closings could have a severe
effect on parents who will need to find child care, and on the many
students who depend on schools for food and shelter.

The Los Angeles and San Diego school districts said they were canceling
school beginning on Monday, affecting more than 750,000 students in
Southern California. The decisions were the latest in a slew of similar
moves across the country affecting more than six million students.

All public schools, and many if not all private schools, in Illinois,
Virginia, Ohio, Michigan, Maryland, Washington State, Oregon, Kentucky,
New Mexico and Puerto Rico were told to close beginning next week. The
Houston Independent School District, the largest school district in
Texas, also said it was closing for two weeks.

\hypertarget{american-officials-are-considering-escalating-containment-measures}{%
\subsection{American officials are considering escalating containment
measures.}\label{american-officials-are-considering-escalating-containment-measures}}

\includegraphics{https://static01.graylady3jvrrxbe.onion/images/2020/03/14/autossell/13vid-coronavirus/13vid-coronavirus-videoSixteenByNine3000.jpg}

In Washington State, where more than 30 people have died from the virus,
more than anywhere else in the country, public health officials have
escalated through most of a 13-step strategy checklist for controlling
infectious outbreaks and now have only a few remaining options: closing
workplaces, restricting people to their homes and cordoning off targeted
areas to help control the spread of infection, measures that have
already been put in place in other parts of the world.

The possibility of more draconian measures if the outbreak continues to
escalate has become a rising subject of conversation among public health
officials across the country, forcing them to confront difficult
questions about how much pain to endure --- in their local economies and
in civil liberties --- to save more lives.

Political leaders are considering their options, alarmed over research
that suggests 400 people in the Seattle area could die in the coming
weeks if the trajectory of the outbreak cannot be altered. The research
shows that if policymakers could reduce the transmission rate by 75
percent --- primarily through what is known as ``social distancing'' ---
then the number of deaths could be reduced to only about 30 in that
period.

One official said that none of the options were off the table, but that
officials were mindful of the tremendous burden that such restrictions
could impose on families and businesses.

\hypertarget{world-leaders-are-affected-as-the-virus-causes-political-and-social-disruption-worldwide}{%
\subsection{World leaders are affected as the virus causes political and
social disruption
worldwide.}\label{world-leaders-are-affected-as-the-virus-causes-political-and-social-disruption-worldwide}}

Image

Prime Minister Justin Trudeau of Canada and his wife, Sophie Grégoire
Trudeau, in Ottawa last year.Credit...Justin Tang/The Canadian Press,
via Associated Press

Even as governments around the world struggle to contain the coronavirus
pandemic, many officials are themselves falling victim to the pathogen,
undermining global efforts to address the crisis.

Prime Minister Justin Trudeau of Canada put himself in isolation
\href{https://www.nytimes3xbfgragh.onion/2020/03/12/world/canada/trudeau-wife-coronavirus.html}{after
his wife tested positive}, and senior officials from Britain to Iran to
Australia were confirmed to be infected. President Jair Bolsonaro of
Brazil said on Friday that he tested negative, after one of his aides
tested positive.

There is, though, rising concern that President Trump and members of his
staff and cabinet might have been exposed in meetings with different
officials from overseas, including with Mr. Bolsonaro's aide.

In a closely connected political world where officials crisscross the
globe as they take part in frequent meetings with heads of state and
other policymakers, the cases vividly illustrate how no one is immune
from a virus that does not distinguish between the powerful and everyone
else.

The risk is that entire cabinets or senior government officials could be
sidelined at once, potentially undermining an already complex response.

Senator Ted Cruz, Republican of Texas, said on Friday he would extend
his self-quarantine to March 17 after learning he had come into contact
with a second individual who has tested positive, though he still had no
symptoms. Two other Republican senators, Rick Scott of Florida and
Lindsey Graham of South Carolina, have also preemptively isolated
themselves after exposure to others.

\hypertarget{spain-declares-a-state-of-emergency-and-orders-its-first-mandatory-lockdowns}{%
\subsection{Spain declares a state of emergency and orders its first
mandatory
lockdowns.}\label{spain-declares-a-state-of-emergency-and-orders-its-first-mandatory-lockdowns}}

Image

Checking IDs on Thursday at the entrance to Igualada, one of the four
towns in northeastern Spain that have been placed in
lockdown.Credit...Samuel Aranda for The New York Times

Prime Minister Pedro Sánchez of Spain said on Friday that the country
would be
\href{https://www.nytimes3xbfgragh.onion/2020/03/13/world/europe/spain-coronavirus-emergency.html}{placed
under a state of emergency for 15 days}, which could allow his
government to restrict the movement of citizens, ration goods and impose
other measures to fight the spread of the new coronavirus.

The decision came as Spain's caseload rose to about 4,200 cases, the
most in Europe after Italy's more than 17,600. The Spanish death toll
reached 120 on Friday.

\emph{{[}Read:}
\href{http://www.nytimes3xbfgragh.onion/2020/03/16/world/europe/italy-coronavirus-funerals.html}{\emph{Italy's
coronavirus victims face death alone, with funerals
postponed}}\emph{.{]}}

European caseloads overall jumped to more than 35,000 on Friday, nearly
double the number three days earlier. France's case count rose by 800,
reaching more than 3,600. Germany's rose by a similar number to more
than 3,100.

Catalonia, a region of 7.5 million people in northeastern Spain,
announced a broad lockdown late Friday, saying it would start to
``restrict entrances and departures.'' It did not offer details.

In a brief televised address, Mr. Sánchez said that the state of
emergency would come into force on Saturday, and that it was designed to
use ``all the resources of the state to protect better the citizens,
especially those who are most vulnerable to the virus.''

\hypertarget{louisiana-will-postpone-its-april-4-primary-becoming-the-first-state-to-do-so}{%
\subsection{Louisiana will postpone its April 4 primary, becoming the
first state to do
so.}\label{louisiana-will-postpone-its-april-4-primary-becoming-the-first-state-to-do-so}}

Louisiana will postpone its April 4 primary election for two months,
becoming the first state to adjust its elections in response to the
coronavirus outbreak. The presidential primary will now be held on June
20, and municipal elections on July 25.

Concern for public health, and particularly the health of poll workers
--- whom are mostly senior citizens --- led officials to decide on
postponing the primary. The state's decision to postpone elections after
Hurricane Katrina in 2005 was a precedent, the secretary of state, R.
Kyle Ardoin, said at a news conference.

\emph{Read the latest developments in the}
\href{https://www.nytimes3xbfgragh.onion/2020/03/14/world/coronavirus-news.html}{\emph{coronavirus
outbreak here}}\emph{.} \emph{This briefing is no longer updating.}

Reporting was contributed by Ian Austen, Ernesto Londoño, Melissa Eddy,
Aurelien Breeden, Constant Méheut, Elisabetta Povoledo, Ivan
Nechepurenko, Davey Alba, Raphael Minder, Karen Zraick, Andy Newman,
Steven Erlanger, Marc Santora, Megan Specia, Matina Stevis-Gridneff,
Steven Lee Myers, Andrew Higgins, Damien Cave, Farah Stockman, Hannah
Beech, Heather Murphy, Gillian Wong, Jorge Arangure, Bhadra Sharma,
Emily Cochrane, Jeanna Smialek, Jim Tankersley, Nicholas
Bogel-Burroughs, Nick Corasaniti, Mike Baker, Miriam Jordan, Jason
Horowitz, Peter Baker, Maggie Haberman, Annie Karni, Katie Benner, Sarah
Mervosh, Patricia Mazzei, Neil Vigdor, Rick Gladstone, Linda Qui and Dai
Wakabayashi.

Advertisement

\protect\hyperlink{after-bottom}{Continue reading the main story}

\hypertarget{site-index}{%
\subsection{Site Index}\label{site-index}}

\hypertarget{site-information-navigation}{%
\subsection{Site Information
Navigation}\label{site-information-navigation}}

\begin{itemize}
\tightlist
\item
  \href{https://help.nytimes3xbfgragh.onion/hc/en-us/articles/115014792127-Copyright-notice}{©~2020~The
  New York Times Company}
\end{itemize}

\begin{itemize}
\tightlist
\item
  \href{https://www.nytco.com/}{NYTCo}
\item
  \href{https://help.nytimes3xbfgragh.onion/hc/en-us/articles/115015385887-Contact-Us}{Contact
  Us}
\item
  \href{https://www.nytco.com/careers/}{Work with us}
\item
  \href{https://nytmediakit.com/}{Advertise}
\item
  \href{http://www.tbrandstudio.com/}{T Brand Studio}
\item
  \href{https://www.nytimes3xbfgragh.onion/privacy/cookie-policy\#how-do-i-manage-trackers}{Your
  Ad Choices}
\item
  \href{https://www.nytimes3xbfgragh.onion/privacy}{Privacy}
\item
  \href{https://help.nytimes3xbfgragh.onion/hc/en-us/articles/115014893428-Terms-of-service}{Terms
  of Service}
\item
  \href{https://help.nytimes3xbfgragh.onion/hc/en-us/articles/115014893968-Terms-of-sale}{Terms
  of Sale}
\item
  \href{https://spiderbites.nytimes3xbfgragh.onion}{Site Map}
\item
  \href{https://help.nytimes3xbfgragh.onion/hc/en-us}{Help}
\item
  \href{https://www.nytimes3xbfgragh.onion/subscription?campaignId=37WXW}{Subscriptions}
\end{itemize}
