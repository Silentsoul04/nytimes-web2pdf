Sections

SEARCH

\protect\hyperlink{site-content}{Skip to
content}\protect\hyperlink{site-index}{Skip to site index}

\href{https://www.nytimes3xbfgragh.onion/section/nyregion}{New York}

\href{https://myaccount.nytimes3xbfgragh.onion/auth/login?response_type=cookie\&client_id=vi}{}

\href{https://www.nytimes3xbfgragh.onion/section/todayspaper}{Today's
Paper}

\href{/section/nyregion}{New York}\textbar{}New York City Schools,
Restaurants and Bars Are Shut Down Over Coronavirus

\url{https://nyti.ms/3aUd1UL}

\begin{itemize}
\item
\item
\item
\item
\item
\end{itemize}

\hypertarget{schools-reopening}{%
\subsubsection{\texorpdfstring{\href{https://www.nytimes3xbfgragh.onion/spotlight/schools-reopening?name=styln-coronavirus-schools-reopening\&region=TOP_BANNER\&variant=undefined\&block=storyline_menu_recirc\&action=click\&pgtype=Article\&impression_id=bed37ae0-e108-11ea-8d35-a17ec3519a05}{Schools
Reopening}}{Schools Reopening}}\label{schools-reopening}}

\begin{itemize}
\tightlist
\item
  \href{https://www.nytimes3xbfgragh.onion/2020/08/17/us/k-12-schools-reopening.html?name=styln-coronavirus-schools-reopening\&region=TOP_BANNER\&variant=undefined\&block=storyline_menu_recirc\&action=click\&pgtype=Article\&impression_id=bed37ae1-e108-11ea-8d35-a17ec3519a05}{State
  of Play for K-12}
\item
  \href{https://www.nytimes3xbfgragh.onion/2020/08/15/us/covid-college-tuition.html?name=styln-coronavirus-schools-reopening\&region=TOP_BANNER\&variant=undefined\&block=storyline_menu_recirc\&action=click\&pgtype=Article\&impression_id=bed37ae2-e108-11ea-8d35-a17ec3519a05}{College
  Costs}
\item
  \href{https://www.nytimes3xbfgragh.onion/2020/08/14/us/covid-schools-learning-pods.html?name=styln-coronavirus-schools-reopening\&region=TOP_BANNER\&variant=undefined\&block=storyline_menu_recirc\&action=click\&pgtype=Article\&impression_id=bed37ae3-e108-11ea-8d35-a17ec3519a05}{Priced
  Out of Learning Pods}
\item
  \href{https://www.nytimes3xbfgragh.onion/2020/08/14/nyregion/school-reopening-nyc.html?name=styln-coronavirus-schools-reopening\&region=TOP_BANNER\&variant=undefined\&block=storyline_menu_recirc\&action=click\&pgtype=Article\&impression_id=bed37ae4-e108-11ea-8d35-a17ec3519a05}{N.Y.C.
  Schools Not Ready}
\item
  \href{https://www.nytimes3xbfgragh.onion/2020/08/05/parenting/parents-distance-learning.html?name=styln-coronavirus-schools-reopening\&region=TOP_BANNER\&variant=undefined\&block=storyline_menu_recirc\&action=click\&pgtype=Article\&impression_id=bed37ae5-e108-11ea-8d35-a17ec3519a05}{Prepare
  for Distance Learning}
\end{itemize}

Advertisement

\protect\hyperlink{after-top}{Continue reading the main story}

Supported by

\protect\hyperlink{after-sponsor}{Continue reading the main story}

\hypertarget{new-york-city-schools-restaurants-and-bars-are-shut-down-over-coronavirus}{%
\section{New York City Schools, Restaurants and Bars Are Shut Down Over
Coronavirus}\label{new-york-city-schools-restaurants-and-bars-are-shut-down-over-coronavirus}}

Mayor Bill de Blasio said that it was a wrenching decision to close
places that are the ``heart and soul of the city.''

\href{https://www.nytimes3xbfgragh.onion/by/luis-ferre-sadurni}{\includegraphics{https://static01.graylady3jvrrxbe.onion/images/2018/06/22/multimedia/author-luis-ferre-sadurni/author-luis-ferre-sadurni-thumbLarge.png}}

By \href{https://www.nytimes3xbfgragh.onion/by/luis-ferre-sadurni}{Luis
Ferré-Sadurní}

\begin{itemize}
\item
  Published March 15, 2020Updated March 16, 2020
\item
  \begin{itemize}
  \item
  \item
  \item
  \item
  \item
  \end{itemize}
\end{itemize}

\includegraphics{https://static01.graylady3jvrrxbe.onion/images/2020/03/16/nyregion/16nyvirus-les-2/merlin_170496525_1f74274e-a4bc-4625-a7f0-90f7a24e3e59-articleLarge.jpg?quality=75\&auto=webp\&disable=upscale}

Facing mounting pressure, New York City officials announced on Sunday a
sweeping shutdown of tens of thousands of bars and restaurants, and the
closure of the city's public school system --- the largest in the nation
--- in an effort to suppress the spread of the coronavirus.

From California to Washington, D.C., governors and mayors are grappling
with how far government should go in constricting people's daily lives
to keep them home.

A patchwork of recent measures --- mandatory curfews in Puerto Rico and
Hoboken, N.J.; the closing of restaurant and bar dining rooms in Ohio
and Illinois; and the closure of public schools in several states,
including Minnesota, South Carolina and Rhode Island --- was a sign that
the restrictive interventions could soon become the norm nationwide.

New York provided another stark example on Sunday: Shortly before 10
p.m., Mayor Bill de Blasio announced that the city will close its bars
and restaurants, except for delivery and pickup services, leaving
waiters, bartenders and baristas uncertain about their next paycheck.

The mayor also ordered the closings of nightclubs, movie theaters, small
theater houses and concert venues. The closings go into effect on
Tuesday morning, for an indefinite period.

The order came just a few hours after officials had announced the
suspension of public schools in New York City as of Monday --- a move
that will, at least temporarily, upend the routines of 1.1 million
students and 75,000 teachers, as the city transitions to remote
learning.

``Our lives are all changing in ways that were unimaginable just a week
ago,'' Mr. de Blasio said in a statement Sunday night. ``This is not a
decision I make lightly. These places are part of the heart and soul of
our city.''

The moves dovetailed with
\href{https://www.cdc.gov/coronavirus/2019-ncov/community/large-events/mass-gatherings-ready-for-covid-19.html}{new
guidelines} issued on Sunday by the Centers for Disease Control and
Prevention; they recommended that local governments and individuals
cancel large gatherings of more than 50 people for the next eight weeks.
The recommendations apply to ``planned or spontaneous'' events,
including conferences, festivals, parades, concerts, sporting events and
weddings.

Earlier on Sunday, Dr. Anthony Fauci, the director of the National
Institute of Allergy and Infectious Diseases, warned that people were
``going to have to hunker down significantly more than we as a country
are doing.''

He suggested a 14-day national shutdown could be warranted down the road
and urged young people to practice social distancing out of fear they
could spread the disease to older people.

Image

Devoción, a popular coffee roaster in New York, often attracts customers
who linger or work for hours, but its cafe in Williamsburg, Brooklyn,
was nearly empty on Sunday.Credit...Sarah Blesener for The New York
Times

But while many government and health officials urged people to observe
such distancing, some Americans had ignored those pleas, including in
New York City.

Earlier this week, New York State, where the number of confirmed
positive results rose to 729 on Sunday, implemented a ban on large
gatherings and established rules to reduce capacity at places like
restaurants with occupancies of fewer than 500 people.

But the measures, put in place on Thursday, did not seem to be having
their desired effect: A handful of elected officials called for a total
shutdown of bars and restaurants following reports of large crowds over
the weekend.

Image

With restaurants remaining open in New York this weekend, employees were
more assiduous in cleaning surfaces.Credit...Gabriela Bhaskar for The
New York Times

``I am alarmed at the cavalier attitude of most New Yorkers who still
don't seem to understand what's about to hit us and what we need to slow
it,'' Councilman Mark Levine, who is chairman of the Council Health
Committee, said on Sunday morning.

Mr. Levine, along with a handful of City Council members, including
Corey Johnson, the Council speaker, as well as the city comptroller,
Scott M. Stringer, expressed outrage that people ignored officials'
pleas to stay home and instead converged at clubs and bars.

The officials, using the hashtag \#shutdownNYC on Twitter, described the
disregard of social distancing as reckless behavior.

By Sunday evening, Mr. de Blasio succumbed to the escalating pressure,
after both he and Gov. Andrew M. Cuomo had resisted calls for a broader
shutdown.

By noon on Sunday, as support for a larger shutdown crescendoed on
social media, Mr. Johnson, the Council speaker, had joined the campaign,
calling for the closure of schools, restaurants and bars.

Mr. Johnson, a Democrat, said that grocery stores, bodegas, pharmacies
and banks should remain open. He said all levels of government should
intervene to ease the losses of business owners, provide financial
assistance to affected workers and help parents with child care.

Then, on Sunday afternoon, Mr. Cuomo changed course: He called on
businesses to shutter voluntarily, as has happened in Boston, Cleveland
and other parts of the nation. The governor of California on Sunday also
asked bars, nightclubs and wineries to close.

\href{https://www.nytimes3xbfgragh.onion/spotlight/schools-reopening?action=click\&pgtype=Article\&state=default\&region=MAIN_CONTENT_3\&context=storylines_keepup}{}

\hypertarget{schools-reopening-}{%
\subsubsection{Schools Reopening ›}\label{schools-reopening-}}

\hypertarget{back-to-school}{%
\paragraph{Back to School}\label{back-to-school}}

Updated Aug. 17, 2020

The latest on how schools are navigating an uncertain season.

\begin{itemize}
\item
  \begin{itemize}
  \tightlist
  \item
    Universities across the country are facing
    \href{https://www.nytimes3xbfgragh.onion/2020/08/15/us/covid-college-tuition.html?action=click\&pgtype=Article\&state=default\&region=MAIN_CONTENT_3\&context=storylines_keepup}{a
    rising demand for tuition rebates} as students ask if college is
    becoming ``glorified Skype.''
  \item
    In Los Angeles, the nation's second-largest school district has
    \href{https://www.nytimes3xbfgragh.onion/2020/08/16/us/los-angeles-schools-virus-testing.html?action=click\&pgtype=Article\&state=default\&region=MAIN_CONTENT_3\&context=storylines_keepup}{perhaps
    the most ambitious plan in the country} to test for the coronavirus.
  \item
    Families
    \href{https://www.nytimes3xbfgragh.onion/2020/08/14/us/covid-schools-learning-pods.html?action=click\&pgtype=Article\&state=default\&region=MAIN_CONTENT_3\&context=storylines_keepup}{priced
    out of ``learning pods'' are seeking alternatives}.
  \item
    How are campus newspapers covering back to school?
    \href{https://www.nytimes3xbfgragh.onion/2020/08/17/us/student-newspaper-schools-reopening.html?action=click\&pgtype=Article\&state=default\&region=MAIN_CONTENT_3\&context=storylines_keepup}{We
    want to hear from student journalists}.
  \end{itemize}
\end{itemize}

``I'm asking them voluntarily to shut down their bar, their restaurant,
their gymnasium,'' Mr. Cuomo said. ``Let's see what they do. If nobody
does it, then we can take more actions.''

Image

Officials in Hoboken, N.J., have placed harsh restrictions on public
life; the entrance to Pier C Park was locked and closed on
Saturday.Credit...Bryan Anselm for The New York Times

In the end, however, as the number of confirmed cases of the new
coronavirus rose, New York officials acknowledged the urgency and
necessity of a broader ban to keep people from gathering.

Mr. de Blasio said, ``Our city is facing an unprecedented threat, and we
must respond with a wartime mentality.''

His remarks were a sign of the fast-paced fluidity of the situation and
officials' morphing response to it: On Saturday, the mayor had said he
was not ready to support broader restrictions and business closures.

``History shows us that in crisis relatively few people have a perfect,
absolutely tried and true plan,'' Mr. de Blasio said on Saturday. ``I am
not ready today at this hour to say, let's have a city with no bars, no
restaurants, no rec centers, no libraries. I'm not there.''

For days, Mr. de Blasio also faced calls to close public schools, a move
he had been reluctant to make, arguing that it could lead to classes
being canceled for the entire year.

Both the governor and mayor had raised concerns about what canceling
classes would mean for parents who cannot afford child care and children
who depend on school for their meals.

The mayor said the city would prepare teachers for remote learning this
week, as well as open sites for at-need students to pick up food and
``learning centers'' for the children of essential city workers like
health care employees.

``It has never been attempted by the City of New York at this scale, to
say the least,'' Mr. de Blasio said. ``It is a system that will improve
with each week.''

In Albany, state legislators were still planning on returning to the
State Capitol on Monday, even after two members of the Assembly who
represent parts of Brooklyn --- Helene Weinstein and Charles Barron ---
tested positive for the virus.

Some lawmakers raised concerns about that prospect, considering the
often close quarters in the legislative chambers as well as conference
rooms where members meet, but Mr. Cuomo was adamant that lawmakers
should return to the capital, likening it to service in war times.

``Should the military not show up? Should the police officers not show
up?'' the governor said, adding that ``If we can ask nurses to put on a
hazmat suit and take blood, we can ask elected officials to come and sit
at a desk and vote on a piece of legislation.''

Mr. Cuomo, who announced on Saturday that the statehouse would be closed
to visitors, said he needed the Legislature to be present to authorize
the laws and the measures the state may need to fight the outbreak. The
state's budget is also due April 1.

``We need soldiers to fight the war,'' he said. ``Government must
function because government is doing all of this. Government goes home,
none of this happens.''

Jesse McKinley contributed reporting from Albany. Matthew Haag and
Jeffrey C. Mays contributed reporting from New York City.

Advertisement

\protect\hyperlink{after-bottom}{Continue reading the main story}

\hypertarget{site-index}{%
\subsection{Site Index}\label{site-index}}

\hypertarget{site-information-navigation}{%
\subsection{Site Information
Navigation}\label{site-information-navigation}}

\begin{itemize}
\tightlist
\item
  \href{https://help.nytimes3xbfgragh.onion/hc/en-us/articles/115014792127-Copyright-notice}{©~2020~The
  New York Times Company}
\end{itemize}

\begin{itemize}
\tightlist
\item
  \href{https://www.nytco.com/}{NYTCo}
\item
  \href{https://help.nytimes3xbfgragh.onion/hc/en-us/articles/115015385887-Contact-Us}{Contact
  Us}
\item
  \href{https://www.nytco.com/careers/}{Work with us}
\item
  \href{https://nytmediakit.com/}{Advertise}
\item
  \href{http://www.tbrandstudio.com/}{T Brand Studio}
\item
  \href{https://www.nytimes3xbfgragh.onion/privacy/cookie-policy\#how-do-i-manage-trackers}{Your
  Ad Choices}
\item
  \href{https://www.nytimes3xbfgragh.onion/privacy}{Privacy}
\item
  \href{https://help.nytimes3xbfgragh.onion/hc/en-us/articles/115014893428-Terms-of-service}{Terms
  of Service}
\item
  \href{https://help.nytimes3xbfgragh.onion/hc/en-us/articles/115014893968-Terms-of-sale}{Terms
  of Sale}
\item
  \href{https://spiderbites.nytimes3xbfgragh.onion}{Site Map}
\item
  \href{https://help.nytimes3xbfgragh.onion/hc/en-us}{Help}
\item
  \href{https://www.nytimes3xbfgragh.onion/subscription?campaignId=37WXW}{Subscriptions}
\end{itemize}
