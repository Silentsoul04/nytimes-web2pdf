Sections

SEARCH

\protect\hyperlink{site-content}{Skip to
content}\protect\hyperlink{site-index}{Skip to site index}

\href{https://www.nytimes3xbfgragh.onion/section/arts/design}{Art \&
Design}

\href{https://myaccount.nytimes3xbfgragh.onion/auth/login?response_type=cookie\&client_id=vi}{}

\href{https://www.nytimes3xbfgragh.onion/section/todayspaper}{Today's
Paper}

\href{/section/arts/design}{Art \& Design}\textbar{}Art Galleries
Respond to Virus Outbreak With Online Viewing Rooms

\url{https://nyti.ms/38RPFha}

\begin{itemize}
\item
\item
\item
\item
\item
\item
\end{itemize}

Advertisement

\protect\hyperlink{after-top}{Continue reading the main story}

Supported by

\protect\hyperlink{after-sponsor}{Continue reading the main story}

\hypertarget{art-galleries-respond-to-virus-outbreak-with-online-viewing-rooms}{%
\section{Art Galleries Respond to Virus Outbreak With Online Viewing
Rooms}\label{art-galleries-respond-to-virus-outbreak-with-online-viewing-rooms}}

After canceling its fair, Art Basel Hong Kong will present more than
2,000 works online with an estimated value of \$270 million. That's just
the beginning as the art world goes virtual.

\includegraphics{https://static01.graylady3jvrrxbe.onion/images/2020/03/17/arts/16virus-galleries1/merlin_170480166_9d080f2f-ca2e-4252-841d-bdfe9bc133de-articleLarge.jpg?quality=75\&auto=webp\&disable=upscale}

\href{https://www.nytimes3xbfgragh.onion/by/robin-pogrebin}{\includegraphics{https://static01.graylady3jvrrxbe.onion/images/2019/09/15/arts/pogrebin-robin/pogrebin-robin-thumbLarge.png}}

By \href{https://www.nytimes3xbfgragh.onion/by/robin-pogrebin}{Robin
Pogrebin}

\begin{itemize}
\item
  March 16, 2020
\item
  \begin{itemize}
  \item
  \item
  \item
  \item
  \item
  \item
  \end{itemize}
\end{itemize}

In 2017, having realized how much business the gallery did through
online previews before art fairs, the dealer David Zwirner decided to
develop
\href{https://www.davidzwirner.com/viewing-room?gclid=Cj0KCQjw3qzzBRDnARIsAECmryr-zZqbRxjtGPm2UN9nVmCH4u1xV5HjQdAj1rFZfWBYiDBgwthqw6QaAh0bEALw_wcB}{virtual
viewing rooms}.

Now, as art fairs are canceled, museums close and auction houses
consider whether to call off their spring sales in response to the
coronavirus, Mr. Zwirner seems prescient.

This week Art Basel will, for the first time, offer
\href{https://www.artbasel.com/viewing-rooms}{online viewing rooms} to
replace the Hong Kong fair that was canceled this month because of the
pandemic. More than 230 dealers who planned to bring work to Asia will
instead offer some 2,000 pieces through the virtual fair with an
estimated value of \$270 million, including 70 items over \$1 million.
And galleries throughout the United States are considering web-based
works and curated online exhibitions.

The future has ``arrived so much sooner,'' Mr. Zwirner said. ``If
galleries are closed, how can we sell art? The online platform is
something we have envisioned as an important part of what we do.''

``In a funny way, the art world is late to the party if you think about
other retail experiences,'' he added.

Many in the art world say an online viewing room cannot replace the
firsthand experience of encountering a painting or a sculpture in
person. But collectors have grown comfortable buying based on PDF images
of artists they know from galleries they trust. Both galleries and
auction houses have even made some significant sales based on images
posted on Instagram. And when visiting a work of art becomes impossible,
a digital substitute is better than not seeing the art at all.

Some point to the added value that online viewing rooms can provide,
namely historical context through accompanying scholarly essays; the
ability to reach collectors who can't easily travel to galleries or art
fairs; and leaving much less of a carbon footprint by eliminating
shipping and flights to fairs.

Online art fairs could also foster a potential democratization by
removing the intimidation factor of walking into a gallery or auction
house and, perhaps most notably, by posting prices in an art market that
is typically opaque.

``You do need to eventually see things physically,'' said the artist
Lisa Yuskavage. ``However, dissemination is now digital and there is an
upside to it. People don't have to know you're looking. You don't have
to buy art to look at the viewing rooms.''

Marc Spiegler, the global director of Art Basel, said the quick pivot to
online viewing rooms --- which will be available to V.I.P.s on Wednesday
and to the public on Friday --- was possible only because of the
decision sometime ago to develop online viewing rooms to supplement the
fair experience. ``The infrastructure was in place,'' Mr. Spiegler said.

Such virtual buying experiences may become increasingly necessary for
the art market, given current restrictions on congregating. The Tefaf
Maastricht fair
\href{https://news.artnet.com/market/tefaf-shuts-maastrich-coronavirus-concerns-1800198}{closed
early} last week after an exhibitor tested positive for the coronavirus.
Art Cologne, the world's oldest art fair, has been
\href{https://www.theartnewspaper.com/news/art-cologne-postponed-from-april-to-november}{postponed}
from April to November. Whether Frieze New York and Tefaf New York
Spring will take place in May, as planned, or Art Basel Switzerland in
June, has yet to be determined.

After the Metropolitan Museum of Art
\href{https://www.nytimes3xbfgragh.onion/2020/03/12/arts/design/met-museum-opera-carnegie-hall-close-coronavirus.html}{announced}
its decision on Thursday to temporarily close, the rest of the art world
fell like dominoes, with one major museum after another following suit,
as well as just about every gallery --- though some are shifting to
appointment-only visitation.

\includegraphics{https://static01.graylady3jvrrxbe.onion/images/2020/03/16/arts/16VIRUS-GALLERY-sub/16VIRUS-GALLERY-sub-articleLarge.jpg?quality=75\&auto=webp\&disable=upscale}

Galleries are adjusting to this new reality. Not all of those in the
Hong Kong fair have signed on for the online version (Mr. Spiegler said
about 95 percent are participating). Some galleries are encouraging
potential visitors ``to visit and explore our exhibitions online,'' as
Van Doren Waxter said in a recent email announcing its temporary
closure, ``and our Richard Diebenkorn exhibition is accessible
\href{https://vandorenwaxter.us7.list-manage.com/track/click?u=0a81893ab01d673c4dd8101cd\&id=d24833bbd3\&e=651860936b}{here}.''

Jack Shainman gallery in Manhattan said in its announcement that
``digital walk-throughs'' of shows by the artists Becky Suss and Vibha
Galhotra ``are available upon request.''

Although Acquavella will have a viewing room in the online fair, this
high-end gallery has been slow to get on the digital train. ``We've
definitely thought about it, but we have not taken the necessary steps
to do it properly,'' said Nick Acquavella, a partner. ``We don't want to
close our minds to something new that could be potentially beneficial,
but we also don't have to be the tip of the spear.''

Artists might be expected to be less than enthusiastic about having
their work purchased the way one would a sweater or shoes (though you
can't click and buy yet; those interested have to contact the gallery
through email). But several said they are intrigued by this new
frontier.

``They feel personal, they feel intimate,'' said the artist Jeff Koons
of online viewing rooms. ``I love looking at images. I can be just as
happy to look at an image of a Manet painting online. It's really about
the stimulation that a work has for you.

``Of course it's great to see the original, but sometimes the lighting
may not be as nice,'' he added. ``There are always pros and cons to
everything, but the positive aspect of having these platforms is that
it's good for the dialogue of art.''

Pace, which first launched online viewing rooms privately last year,
began offering them to the public on Monday, starting with one on the
artist \href{https://www.pacegallery.com/viewing-rooms/sam-gilliam/}{Sam
Gilliam}. The gallery will continue with a series of thematic online
presentations --- including others on ceramics and photographic artists
--- during its temporary closure.

Mr. Zwirner has presented 50 viewing rooms in the last three years, and
says that online sales increased 400 percent in just the last year. The
gallery's viewing room for Art Basel Hong Kong --- its largest to date,
with a total value of more than \$16 million --- will debut a new work
by Mr. Koons, along with art by Noah Davis, Marlene Dumas, Kerry James
Marshall and Alice Neel.

And while online sales usually have a lower price point, last June,
Zwirner's\href{https://www.davidzwirner.com/viewing-room/basel-online}{Basel
Online} viewing room presented more than 20 works with a total value
over \$5.6 million, including new pieces from artists like Ms. Yuskavage
and Jordan Wolfson and historical works by artists like Donald Judd and
Dan Flavin. A pumpkin sculpture by Yayoi Kusama sold online for \$1.8
million.

The Zwirner viewing rooms already include videos of artists working in
their studios; its website also offers links to podcasts with artists.

These bells and whistles do not come cheap and are therefore less
available to small and midsize galleries that already struggle to pay
rent and afford art fairs. But they too are getting in on the act. The
\href{http://www.jttnyc.com/6953}{JTT gallery} on the Lower East Side,
for example, had planned to debut the paintings of
\href{https://en.wikipedia.org/wiki/Arca_(musician)}{Arca}, a musician
and performance artist, but instead is considering an online performance
from her that people can view from the comfort of their homes.

Scott Ogden of Shrine Gallery, which shares space with Sargent's
Daughters gallery on the Lower East Side, said the coronavirus crisis
had accelerated his exploration of an --- albeit technologically modest
--- online store. ``For us, it's going to be
\href{https://www.squarespace.com/}{Squarespace} --- the simple
do-it-yourself solution,'' he said, referring to the popular website
builder. ``I think we're all going to have to figure it out rather
quickly.''

Advertisement

\protect\hyperlink{after-bottom}{Continue reading the main story}

\hypertarget{site-index}{%
\subsection{Site Index}\label{site-index}}

\hypertarget{site-information-navigation}{%
\subsection{Site Information
Navigation}\label{site-information-navigation}}

\begin{itemize}
\tightlist
\item
  \href{https://help.nytimes3xbfgragh.onion/hc/en-us/articles/115014792127-Copyright-notice}{©~2020~The
  New York Times Company}
\end{itemize}

\begin{itemize}
\tightlist
\item
  \href{https://www.nytco.com/}{NYTCo}
\item
  \href{https://help.nytimes3xbfgragh.onion/hc/en-us/articles/115015385887-Contact-Us}{Contact
  Us}
\item
  \href{https://www.nytco.com/careers/}{Work with us}
\item
  \href{https://nytmediakit.com/}{Advertise}
\item
  \href{http://www.tbrandstudio.com/}{T Brand Studio}
\item
  \href{https://www.nytimes3xbfgragh.onion/privacy/cookie-policy\#how-do-i-manage-trackers}{Your
  Ad Choices}
\item
  \href{https://www.nytimes3xbfgragh.onion/privacy}{Privacy}
\item
  \href{https://help.nytimes3xbfgragh.onion/hc/en-us/articles/115014893428-Terms-of-service}{Terms
  of Service}
\item
  \href{https://help.nytimes3xbfgragh.onion/hc/en-us/articles/115014893968-Terms-of-sale}{Terms
  of Sale}
\item
  \href{https://spiderbites.nytimes3xbfgragh.onion}{Site Map}
\item
  \href{https://help.nytimes3xbfgragh.onion/hc/en-us}{Help}
\item
  \href{https://www.nytimes3xbfgragh.onion/subscription?campaignId=37WXW}{Subscriptions}
\end{itemize}
