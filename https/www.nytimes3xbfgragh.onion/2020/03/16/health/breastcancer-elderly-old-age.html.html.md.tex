Sections

SEARCH

\protect\hyperlink{site-content}{Skip to
content}\protect\hyperlink{site-index}{Skip to site index}

\href{https://www.nytimes3xbfgragh.onion/section/health}{Health}

\href{https://myaccount.nytimes3xbfgragh.onion/auth/login?response_type=cookie\&client_id=vi}{}

\href{https://www.nytimes3xbfgragh.onion/section/todayspaper}{Today's
Paper}

\href{/section/health}{Health}\textbar{}Aging Out of the Mammogram

\url{https://nyti.ms/3cZCUUR}

\begin{itemize}
\item
\item
\item
\item
\item
\item
\end{itemize}

Advertisement

\protect\hyperlink{after-top}{Continue reading the main story}

Supported by

\protect\hyperlink{after-sponsor}{Continue reading the main story}

the new old age

\hypertarget{aging-out-of-the-mammogram}{%
\section{Aging Out of the Mammogram}\label{aging-out-of-the-mammogram}}

For many women, regular breast cancer screening is standard medical
protocol. But for older women, should it be?

\includegraphics{https://static01.graylady3jvrrxbe.onion/images/2020/03/17/science/17SCI-SPAN/17SPANMAMMOGRAM-articleLarge.jpg?quality=75\&auto=webp\&disable=upscale}

By \href{https://www.nytimes3xbfgragh.onion/by/paula-span}{Paula Span}

\begin{itemize}
\item
  March 16, 2020
\item
  \begin{itemize}
  \item
  \item
  \item
  \item
  \item
  \item
  \end{itemize}
\end{itemize}

When Janet Halloran last saw her primary care physician, the doctor
asked whether she had undergone her annual mammogram. Yes, she replied,
she had.

At 76, Ms. Halloran, a real estate broker in Cambridge, Mass., is past
the age that most medical guidelines recommend breast cancer screening
for someone with no history of the disease. Even for younger women, the
guidelines call for a mammogram every other year, not annually.

So Ms. Halloran could consider stopping mammograms, or at least having
them less often. But her doctor has never discussed that prospect. ``She
says, `These are the things you need to do,''' Ms. Halloran said.
Besides, she added, it's an easy test: ``Go once a year, hold your
breath and you're done for another year. It's just routine.''

But for older women, should it be?

``There's been a lot of uncertainty,'' said Dr. Xabier Garcia-Albéniz,
an oncologist and epidemiologist at RTI Health Solutions and lead author
of a new observational study that tries to answer that question. ``This
is an area with a complete lack of randomized clinical trials.''

Breast cancer studies, like medical research in general,
\href{https://www.nytimes3xbfgragh.onion/2018/04/13/health/elderly-clinical-trials.html}{have
often excluded} older subjects. So the data on whether mammography
improves survival is very limited in women ages 70 to 74, and
nonexistent for those 75 and older.

That's why the independent U.S. Preventive Services Task Force has
concluded that while having mammograms every other year improves
survival for women ages 50 to 74, there's
\href{https://www.uspreventiveservicestaskforce.org/Page/Document/RecommendationStatementFinal/breast-cancer-screening}{``insufficient''
evidence} to assess their use for those over 75.

The American Geriatrics Society includes screening for breast and other
cancers on its
\href{https://www.choosingwisely.org/societies/american-geriatrics-society/}{Choosing
Wisely} list of tests that should be questioned. It urges doctors not to
recommend it ``without considering life expectancy and the risks of
testing, overdiagnosis and overtreatment.''

Yet more than half of women over 75 have had a screening mammogram (a
test for individuals with no history or symptoms of breast cancer)
within the past two years, the Centers for Disease Control and
Prevention
\href{https://www.cdc.gov/nchs/data/hus/2018/033.pdf}{reported} in 2018.

``Whether this investment in breast cancer screening alters survival is
a critical question,'' said Dr. John Hsu, a health services researcher
at Harvard Medical School and senior author of the
\href{http://annals.org/aim/article/doi/10.7326/M18-1199}{new study},
published in the Annals of Internal Medicine.

The research team used Medicare claims from 2000 to 2008 to follow more
than one million women, ages 70 to 84, who had undergone a mammogram.

They had never had breast cancer and had a ``high probability,'' based
on their medical histories, of living at least 10 more years. ``That's
the population who will reap the benefit of screening,'' Dr.
Garcia-Albéniz said, because it takes 10 years for mammography to show
reduced mortality.

The researchers divided the subjects into two groups: one that stopped
screening, and another that continued having mammograms at least every
15 months. They found that mammograms provided a survival benefit, if a
modest one, for women ages 70 to 74. In line with previous research, the
study found that annually screening 1,000 women in that age group would
result, after 10 years, in one less death from breast cancer.

But among the women who were 75 to 84, annual mammograms did not reduce
deaths, although they did, predictably, detect more cancer than in the
group that discontinued screening.

``You're diagnosing more cancer, but that's not translating to a
mortality benefit,'' Dr. Garcia-Albéniz said.

Why not? ``The cancers themselves might be different at different
ages,'' Dr. Hsu said. ``They might grow faster or slower, or be more
likely to spread.''

Treatments may also be less effective at older ages, said Dr. Otis
Brawley, an oncologist and epidemiologist at the Johns Hopkins
University School of Medicine, who wrote
\href{http://annals.org/aim/article/doi/10.7326/M20-0429}{an editorial}
accompanying the study.

But older people typically are also subject to what researchers call
``competing mortality.'' Many of the cancers detected by mammography ---
tiny tumors that earlier technology wouldn't have spotted --- are
unlikely to cause any harm if left untreated. But most older people have
other diseases that will progress.

``It's very difficult to tell someone in her 70s or 80s that we're going
to modify your treatment, or not treat you, because of the likelihood
that something else will kill you before this cancer will,'' Dr. Brawley
said.

That reluctance to discuss life expectancy and the limitations of
screening also means that many women don't recognize that, in addition
to being inconvenient, expensive and a cause of discomfort or anxiety,
mammograms can actually do harm. The tests often prompt unneeded
surgery, radiation or drug regimens for cancers that would never have
caused symptoms or shortened lives.

Still, because life expectancy varies widely, some very healthy older
women may live long enough to benefit and may indeed want screening.
Mammograms could lead to treating an aggressive cancer earlier, and with
less extensive surgery, for instance.

``I would be very happy if doctors started using our paper to inform the
discussion they have with their patients,'' Dr. Garcia-Albéniz said.

Yet women remain so committed to regular mammograms that experts doubt
they could recruit enough people for a large randomized trial in which
half the subjects forgo the tests.

Dr. Mara Schonberg, an internist at Beth Israel Deaconess Hospital in
Boston, has worked for years to help women make decisions about breast
cancer screening, and has found it tough going.

``These women were told for 40 years to get screened,'' Dr. Schonberg
said. ``They get reassurance from a negative mammogram. And it's very
hard to understand that finding breast cancer early may not help you
live longer or better.''

To help explain, Dr. Schonberg developed a decision aid:
\href{https://eprognosis.ucsf.edu/decision_aids.php}{a brochure},
written at a sixth-grade reading level, that uses research findings to
explain the pros and cons.

A
\href{https://jamanetwork.com/journals/jamainternalmedicine/fullarticle/1792914}{pilot
study} showed that, after reading it, women from 75 to 89 were more
knowledgeable about mammography, more apt to discuss it with their
doctors and less enthusiastic about continuing it.

But they did continue. More than 60 percent, including those with lower
life expectancies, had another mammogram within 15 months. A larger
study with 546 participants, being readied for publication, will report
similar results, Dr. Schonberg said.

Perhaps, as Dr. Brawley said, ``the most important thing we can do is
get people to understand what the questions are, and to understand that
nobody has the exact answers.''

But Dr. Schonberg's grandmother, who followed her doctor's
recommendation and had a mammogram at age 78, came to a more definitive
conclusion.

Ann Schonberg was a Detroit homemaker and a lifelong smoker with mild
emphysema. When her mammogram found a small Stage 1 cancer, she
underwent a lumpectomy and began endocrine therapy, taking drugs that
might lower the risk of recurrence. ``She didn't like how she felt, so
she stopped everything after three years,'' Dr. Schonberg said.

When Ann Schonberg reached her mid 80s, a mammogram picked up another
small cancer, prompting another lumpectomy. At the same time, although
she had stopped smoking at 80, her emphysema worsened steadily. That is
what caused her death at 88, not breast cancer.

``All the doctors' appointments, the surgeries, the worry --- for her,
it was all for naught,'' Dr. Schonberg said. Shortly before Ann
Schonberg's death, she told her granddaughter, ``I wish I'd never had
that mammogram.''

\textbf{\emph{{[}}\href{http://on.fb.me/1paTQ1h}{\emph{Like the Science
Times page on Facebook.}}} ****** \emph{\textbar{} Sign up for the}
\textbf{\href{http://nyti.ms/1MbHaRU}{\emph{Science Times
newsletter.}}\emph{{]}}}

Advertisement

\protect\hyperlink{after-bottom}{Continue reading the main story}

\hypertarget{site-index}{%
\subsection{Site Index}\label{site-index}}

\hypertarget{site-information-navigation}{%
\subsection{Site Information
Navigation}\label{site-information-navigation}}

\begin{itemize}
\tightlist
\item
  \href{https://help.nytimes3xbfgragh.onion/hc/en-us/articles/115014792127-Copyright-notice}{©~2020~The
  New York Times Company}
\end{itemize}

\begin{itemize}
\tightlist
\item
  \href{https://www.nytco.com/}{NYTCo}
\item
  \href{https://help.nytimes3xbfgragh.onion/hc/en-us/articles/115015385887-Contact-Us}{Contact
  Us}
\item
  \href{https://www.nytco.com/careers/}{Work with us}
\item
  \href{https://nytmediakit.com/}{Advertise}
\item
  \href{http://www.tbrandstudio.com/}{T Brand Studio}
\item
  \href{https://www.nytimes3xbfgragh.onion/privacy/cookie-policy\#how-do-i-manage-trackers}{Your
  Ad Choices}
\item
  \href{https://www.nytimes3xbfgragh.onion/privacy}{Privacy}
\item
  \href{https://help.nytimes3xbfgragh.onion/hc/en-us/articles/115014893428-Terms-of-service}{Terms
  of Service}
\item
  \href{https://help.nytimes3xbfgragh.onion/hc/en-us/articles/115014893968-Terms-of-sale}{Terms
  of Sale}
\item
  \href{https://spiderbites.nytimes3xbfgragh.onion}{Site Map}
\item
  \href{https://help.nytimes3xbfgragh.onion/hc/en-us}{Help}
\item
  \href{https://www.nytimes3xbfgragh.onion/subscription?campaignId=37WXW}{Subscriptions}
\end{itemize}
