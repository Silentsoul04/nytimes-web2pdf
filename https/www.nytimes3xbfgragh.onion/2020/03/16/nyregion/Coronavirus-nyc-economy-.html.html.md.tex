Sections

SEARCH

\protect\hyperlink{site-content}{Skip to
content}\protect\hyperlink{site-index}{Skip to site index}

\href{https://www.nytimes3xbfgragh.onion/section/nyregion}{New York}

\href{https://myaccount.nytimes3xbfgragh.onion/auth/login?response_type=cookie\&client_id=vi}{}

\href{https://www.nytimes3xbfgragh.onion/section/todayspaper}{Today's
Paper}

\href{/section/nyregion}{New York}\textbar{}N.Y.C.'s Economy Could be
Ravaged by Coronavirus Outbreak

\url{https://nyti.ms/2Qj8UcK}

\begin{itemize}
\item
\item
\item
\item
\item
\end{itemize}

\hypertarget{the-coronavirus-outbreak}{%
\subsubsection{\texorpdfstring{\href{https://www.nytimes3xbfgragh.onion/news-event/coronavirus?name=styln-coronavirus-national\&region=TOP_BANNER\&variant=undefined\&block=storyline_menu_recirc\&action=click\&pgtype=Article\&impression_id=5fba8b30-e0fd-11ea-90ad-4f7da0581532}{The
Coronavirus
Outbreak}}{The Coronavirus Outbreak}}\label{the-coronavirus-outbreak}}

\begin{itemize}
\tightlist
\item
  live\href{https://www.nytimes3xbfgragh.onion/2020/08/17/world/coronavirus-covid.html?name=styln-coronavirus-national\&region=TOP_BANNER\&variant=undefined\&block=storyline_menu_recirc\&action=click\&pgtype=Article\&impression_id=5fba8b31-e0fd-11ea-90ad-4f7da0581532}{Latest
  Updates}
\item
  \href{https://www.nytimes3xbfgragh.onion/interactive/2020/us/coronavirus-us-cases.html?name=styln-coronavirus-national\&region=TOP_BANNER\&variant=undefined\&block=storyline_menu_recirc\&action=click\&pgtype=Article\&impression_id=5fbab240-e0fd-11ea-90ad-4f7da0581532}{Maps
  and Cases}
\item
  \href{https://www.nytimes3xbfgragh.onion/interactive/2020/science/coronavirus-vaccine-tracker.html?name=styln-coronavirus-national\&region=TOP_BANNER\&variant=undefined\&block=storyline_menu_recirc\&action=click\&pgtype=Article\&impression_id=5fbab241-e0fd-11ea-90ad-4f7da0581532}{Vaccine
  Tracker}
\item
  \href{https://www.nytimes3xbfgragh.onion/2020/08/17/us/k-12-schools-reopening.html?name=styln-coronavirus-national\&region=TOP_BANNER\&variant=undefined\&block=storyline_menu_recirc\&action=click\&pgtype=Article\&impression_id=5fbab242-e0fd-11ea-90ad-4f7da0581532}{State
  of Play for K-12}
\item
  \href{https://www.nytimes3xbfgragh.onion/live/2020/08/17/business/stock-market-today-coronavirus?name=styln-coronavirus-national\&region=TOP_BANNER\&variant=undefined\&block=storyline_menu_recirc\&action=click\&pgtype=Article\&impression_id=5fbab243-e0fd-11ea-90ad-4f7da0581532}{Markets
  \& Economy}
\end{itemize}

Advertisement

\protect\hyperlink{after-top}{Continue reading the main story}

Supported by

\protect\hyperlink{after-sponsor}{Continue reading the main story}

\hypertarget{nycs-economy-could-be-ravaged-by-coronavirus-outbreak}{%
\section{N.Y.C.'s Economy Could be Ravaged by Coronavirus
Outbreak}\label{nycs-economy-could-be-ravaged-by-coronavirus-outbreak}}

With the closing of museums, Broadway theaters and restaurants, tourism
is plunging, leaving New York facing mass layoffs and business failures.

\includegraphics{https://static01.graylady3jvrrxbe.onion/images/2020/03/14/nyregion/00nyvirus-econ1/merlin_170492343_cd53de7d-890d-4e9d-aae0-aa030710101c-articleLarge.jpg?quality=75\&auto=webp\&disable=upscale}

By \href{https://www.nytimes3xbfgragh.onion/by/patrick-mcgeehan}{Patrick
McGeehan}

\begin{itemize}
\item
  Published March 16, 2020Updated March 17, 2020
\item
  \begin{itemize}
  \item
  \item
  \item
  \item
  \item
  \end{itemize}
\end{itemize}

The sudden and prolonged shutdown of New York City's museums and its
iconic Broadway theaters.
\href{https://www.nytimes3xbfgragh.onion/2020/03/16/nyregion/nyc-closing-bars-restaurants-coronavirus.html}{Restaurants
and bars also closed} except for take out and delivery. Hotels
struggling to stay open in the face of a wave of canceled reservations.
Movie theaters shuttered. The evaporation of nearly all business and
leisure travel to the city.

The coronavirus pandemic is all but bringing life to a halt in the
nation's largest city, slamming the brakes on what had been a robust
economy and leaving
\href{https://www.nytimes3xbfgragh.onion/news-event/coronavirus-new-york}{New
York}, which has more than 25,000 restaurants and 120,000 hotel rooms,
confronting a dire threat that experts say will surely lead to sweeping
layoffs and business failures.

The fallout could be more widespread than the economic damage from
either of the past two crises the city had faced --- the Sept. 11
terrorist attacks in 2001 or the 2008 financial crisis --- said Ronnie
Lowenstein, director of the city's Independent Budget Office. ``Economic
activity slows and time stops,'' she said. ``Everything gets affected.''

Both of those events caused significant harm to the city's financial
health, Ms. Lowenstein said, but in both cases the city was rescued by
huge infusions of federal aid.

But the coronavirus is spreading from coast to coast, leaving New York's
city and state officials pleading for help in a large crowd of their
counterparts, she said.

``If it was one city under siege like 9/11, that's one thing,'' Ms.
Lowenstein said. But the coronavirus pandemic is ``going to be
disastrous for a lot of economies across the country,'' she added.

New York is also particularly exposed because of its heavy dependence on
domestic and foreign visitors who come to the city to watch shows, visit
museums and take advantage of its vibrant night life.

\includegraphics{https://static01.graylady3jvrrxbe.onion/images/2020/03/14/nyregion/00nyvirus-econ5/merlin_170441724_39f2eef6-ed8d-4ca5-863d-f51ffc4b7bb7-articleLarge.jpg?quality=75\&auto=webp\&disable=upscale}

The city is likely to lose as many as 500,000 jobs in businesses that
cater to tourists and people moving about the city, said James Parrott,
director of economic and fiscal policies at the Center for New York City
Affairs at The New School. In one month, their lost wages amount to \$1
billion, he added.

Mr. Parrott said that those industries included hotels, restaurants,
museums, movie and television production and ride-hailing apps like Uber
and Lyft. Some of those businesses rely so heavily on tourists and
business travelers that they probably will close and lay off most or all
of their workers, as all 19 restaurants run by Danny Meyer's company,
Union Square Hospitality Group,
\href{https://www.nytimes3xbfgragh.onion/2020/03/13/dining/restaurant-closings-coronavirus.html}{did
on Friday.}

The owners of Chelsea Piers, a large sports and recreation complex on
the West Side of Manhattan, decided on Thursday morning to close the
location, as well as two smaller centers, in Brooklyn and Stamford,
Conn., for the rest of March, said David Tewksbury, one of the owners.

\hypertarget{latest-updates-the-coronavirus-outbreak}{%
\section{\texorpdfstring{\href{https://www.nytimes3xbfgragh.onion/2020/08/17/world/coronavirus-covid.html?action=click\&pgtype=Article\&state=default\&region=MAIN_CONTENT_1\&context=storylines_live_updates}{Latest
Updates: The Coronavirus
Outbreak}}{Latest Updates: The Coronavirus Outbreak}}\label{latest-updates-the-coronavirus-outbreak}}

Updated 2020-08-18T01:58:10.452Z

\begin{itemize}
\tightlist
\item
  \href{https://www.nytimes3xbfgragh.onion/2020/08/17/world/coronavirus-covid.html?action=click\&pgtype=Article\&state=default\&region=MAIN_CONTENT_1\&context=storylines_live_updates\#link-6fdbc8ef}{U.S.
  college campuses grapple with coronavirus fears, outbreaks and
  protests.}
\item
  \href{https://www.nytimes3xbfgragh.onion/2020/08/17/world/coronavirus-covid.html?action=click\&pgtype=Article\&state=default\&region=MAIN_CONTENT_1\&context=storylines_live_updates\#link-12d68713}{For
  primary and secondary students and staff, it's been a difficult
  back-to-school season.}
\item
  \href{https://www.nytimes3xbfgragh.onion/2020/08/17/world/coronavirus-covid.html?action=click\&pgtype=Article\&state=default\&region=MAIN_CONTENT_1\&context=storylines_live_updates\#link-6aa8318c}{Data
  reporting problems are causing confusion in Iowa and elsewhere.}
\end{itemize}

\href{https://www.nytimes3xbfgragh.onion/2020/08/17/world/coronavirus-covid.html?action=click\&pgtype=Article\&state=default\&region=MAIN_CONTENT_1\&context=storylines_live_updates}{See
more updates}

More live coverage:
\href{https://www.nytimes3xbfgragh.onion/live/2020/08/17/business/stock-market-today-coronavirus?action=click\&pgtype=Article\&state=default\&region=MAIN_CONTENT_1\&context=storylines_live_updates}{Markets}

He said that their 1,500 employees were furloughed and given two weeks'
pay but that he expected them to apply for unemployment benefits.

Mr. Tewksbury said he was unaware if any of those employees or any of
the more than 25,000 members of the three locations had tested positive
for the coronavirus.

But the decision to close, he added, was based on a concern for the
health and safety of the workers and of members, and a ``social
responsibility'' to be ``a leader as opposed to a follower.''

Image

Chelsea Piers, a large sports and recreation complex on the West Side of
Manhattan, has closed and furloughed its workers.~Credit...Brittainy
Newman/The New York Times

Much like it was before the coronavirus swept across the country, the
city's economy was relatively healthy before the Sept. 11 attacks.
``But,'' Mr. Parrott said, ``after 9/11, people woke up and realized
that we were in a recession.''

In the wake of that shock, the New York Stock Exchange and other
financial markets shut down for a few days and air travel was disrupted
for weeks.

But the city began to recover quickly, buoyed by
\href{https://www.nytimes3xbfgragh.onion/2002/12/30/nyregion/after-9-11-parcels-of-money-and-dismay.html}{support
from Washington}, which provided billions of dollars in disaster relief.
Theaters reopened, ballgames resumed and elected officials urged New
Yorkers to go out, spend money and get back to work.

The 2008 financial crisis was centered on Wall Street, where some major
investment banks failed and others survived only through a
\href{https://www.nytimes3xbfgragh.onion/2008/10/03/business/worldbusiness/03iht-bailout.4.16679355.html}{massive
federal bailout}.

Now, with no certainty about when life might return to normal, the full
effect on the city's economy is impossible to project, Ms. Lowenstein,
director of the city's Independent Budget Office, said. But Mr. Parrott
said the city would almost surely fall into a recession that would end
its longest period of expansion and job growth on record --- more than
10 years.

This week, the state labor department said that New York City's
unemployment rate fell to an all-time low of 3.5 percent in January,
compared with highs of about 8.5 percent in the aftermath of Sept. 11,
and 10 percent during the financial crisis.

The ranks of the city's unemployed peaked between 340,000 and 400,000 in
those downturns, about two and a half times the current official count.

Image

Many museums, including the Brooklyn Museum, have closed and it is
unclear when many might reopen.Credit...Victor J. Blue for The New York
Times

The prospects for the city's economy has changed so abruptly as so many
venues have shut down that its tourism agency, NYC \& Company, has
shelved its forecast for the year, said Fred Dixon, its chief executive.

As recently as two weeks ago, the city projected a slight increase in
visitors in 2020. But that was before travel from most of Europe was
banned. For now, Mr. Dixon said, the flow of international travelers has
virtually dried up and it is not clear when it may resume.

Without tourists or business travelers, hotels in New York are
struggling to continue operating and grappling with how many workers
they can retain, said Fred Grapstein, chairman of the Hotel Association
of New York City.

``What you're seeing, unfortunately, is the layoff of all personnel ---
both union and management,'' Mr. Grapstein said.

\href{https://www.nytimes3xbfgragh.onion/news-event/coronavirus?action=click\&pgtype=Article\&state=default\&region=MAIN_CONTENT_3\&context=storylines_faq}{}

\hypertarget{the-coronavirus-outbreak-}{%
\subsubsection{The Coronavirus Outbreak
›}\label{the-coronavirus-outbreak-}}

\hypertarget{frequently-asked-questions}{%
\paragraph{Frequently Asked
Questions}\label{frequently-asked-questions}}

Updated August 17, 2020

\begin{itemize}
\item ~
  \hypertarget{why-does-standing-six-feet-away-from-others-help}{%
  \paragraph{Why does standing six feet away from others
  help?}\label{why-does-standing-six-feet-away-from-others-help}}

  \begin{itemize}
  \tightlist
  \item
    The coronavirus spreads primarily through droplets from your mouth
    and nose, especially when you cough or sneeze. The C.D.C., one of
    the organizations using that measure,
    \href{https://www.nytimes3xbfgragh.onion/2020/04/14/health/coronavirus-six-feet.html?action=click\&pgtype=Article\&state=default\&region=MAIN_CONTENT_3\&context=storylines_faq}{bases
    its recommendation of six feet} on the idea that most large droplets
    that people expel when they cough or sneeze will fall to the ground
    within six feet. But six feet has never been a magic number that
    guarantees complete protection. Sneezes, for instance, can launch
    droplets a lot farther than six feet,
    \href{https://jamanetwork.com/journals/jama/fullarticle/2763852}{according
    to a recent study}. It's a rule of thumb: You should be safest
    standing six feet apart outside, especially when it's windy. But
    keep a mask on at all times, even when you think you're far enough
    apart.
  \end{itemize}
\item ~
  \hypertarget{i-have-antibodies-am-i-now-immune}{%
  \paragraph{I have antibodies. Am I now
  immune?}\label{i-have-antibodies-am-i-now-immune}}

  \begin{itemize}
  \tightlist
  \item
    As of right
    now,\href{https://www.nytimes3xbfgragh.onion/2020/07/22/health/covid-antibodies-herd-immunity.html?action=click\&pgtype=Article\&state=default\&region=MAIN_CONTENT_3\&context=storylines_faq}{that
    seems likely, for at least several months.} There have been
    frightening accounts of people suffering what seems to be a second
    bout of Covid-19. But experts say these patients may have a
    drawn-out course of infection, with the virus taking a slow toll
    weeks to months after initial exposure. People infected with the
    coronavirus typically
    \href{https://www.nature.com/articles/s41586-020-2456-9}{produce}
    immune molecules called antibodies, which are
    \href{https://www.nytimes3xbfgragh.onion/2020/05/07/health/coronavirus-antibody-prevalence.html?action=click\&pgtype=Article\&state=default\&region=MAIN_CONTENT_3\&context=storylines_faq}{protective
    proteins made in response to an
    infection}\href{https://www.nytimes3xbfgragh.onion/2020/05/07/health/coronavirus-antibody-prevalence.html?action=click\&pgtype=Article\&state=default\&region=MAIN_CONTENT_3\&context=storylines_faq}{.
    These antibodies may} last in the body
    \href{https://www.nature.com/articles/s41591-020-0965-6}{only two to
    three months}, which may seem worrisome, but that's perfectly normal
    after an acute infection subsides, said Dr. Michael Mina, an
    immunologist at Harvard University. It may be possible to get the
    coronavirus again, but it's highly unlikely that it would be
    possible in a short window of time from initial infection or make
    people sicker the second time.
  \end{itemize}
\item ~
  \hypertarget{im-a-small-business-owner-can-i-get-relief}{%
  \paragraph{I'm a small-business owner. Can I get
  relief?}\label{im-a-small-business-owner-can-i-get-relief}}

  \begin{itemize}
  \tightlist
  \item
    The
    \href{https://www.nytimes3xbfgragh.onion/article/small-business-loans-stimulus-grants-freelancers-coronavirus.html?action=click\&pgtype=Article\&state=default\&region=MAIN_CONTENT_3\&context=storylines_faq}{stimulus
    bills enacted in March} offer help for the millions of American
    small businesses. Those eligible for aid are businesses and
    nonprofit organizations with fewer than 500 workers, including sole
    proprietorships, independent contractors and freelancers. Some
    larger companies in some industries are also eligible. The help
    being offered, which is being managed by the Small Business
    Administration, includes the Paycheck Protection Program and the
    Economic Injury Disaster Loan program. But lots of folks have
    \href{https://www.nytimes3xbfgragh.onion/interactive/2020/05/07/business/small-business-loans-coronavirus.html?action=click\&pgtype=Article\&state=default\&region=MAIN_CONTENT_3\&context=storylines_faq}{not
    yet seen payouts.} Even those who have received help are confused:
    The rules are draconian, and some are stuck sitting on
    \href{https://www.nytimes3xbfgragh.onion/2020/05/02/business/economy/loans-coronavirus-small-business.html?action=click\&pgtype=Article\&state=default\&region=MAIN_CONTENT_3\&context=storylines_faq}{money
    they don't know how to use.} Many small-business owners are getting
    less than they expected or
    \href{https://www.nytimes3xbfgragh.onion/2020/06/10/business/Small-business-loans-ppp.html?action=click\&pgtype=Article\&state=default\&region=MAIN_CONTENT_3\&context=storylines_faq}{not
    hearing anything at all.}
  \end{itemize}
\item ~
  \hypertarget{what-are-my-rights-if-i-am-worried-about-going-back-to-work}{%
  \paragraph{What are my rights if I am worried about going back to
  work?}\label{what-are-my-rights-if-i-am-worried-about-going-back-to-work}}

  \begin{itemize}
  \tightlist
  \item
    Employers have to provide
    \href{https://www.osha.gov/SLTC/covid-19/standards.html}{a safe
    workplace} with policies that protect everyone equally.
    \href{https://www.nytimes3xbfgragh.onion/article/coronavirus-money-unemployment.html?action=click\&pgtype=Article\&state=default\&region=MAIN_CONTENT_3\&context=storylines_faq}{And
    if one of your co-workers tests positive for the coronavirus, the
    C.D.C.} has said that
    \href{https://www.cdc.gov/coronavirus/2019-ncov/community/guidance-business-response.html}{employers
    should tell their employees} -\/- without giving you the sick
    employee's name -\/- that they may have been exposed to the virus.
  \end{itemize}
\item ~
  \hypertarget{what-is-school-going-to-look-like-in-september}{%
  \paragraph{What is school going to look like in
  September?}\label{what-is-school-going-to-look-like-in-september}}

  \begin{itemize}
  \tightlist
  \item
    It is unlikely that many schools will return to a normal schedule
    this fall, requiring the grind of
    \href{https://www.nytimes3xbfgragh.onion/2020/06/05/us/coronavirus-education-lost-learning.html?action=click\&pgtype=Article\&state=default\&region=MAIN_CONTENT_3\&context=storylines_faq}{online
    learning},
    \href{https://www.nytimes3xbfgragh.onion/2020/05/29/us/coronavirus-child-care-centers.html?action=click\&pgtype=Article\&state=default\&region=MAIN_CONTENT_3\&context=storylines_faq}{makeshift
    child care} and
    \href{https://www.nytimes3xbfgragh.onion/2020/06/03/business/economy/coronavirus-working-women.html?action=click\&pgtype=Article\&state=default\&region=MAIN_CONTENT_3\&context=storylines_faq}{stunted
    workdays} to continue. California's two largest public school
    districts --- Los Angeles and San Diego --- said on July 13, that
    \href{https://www.nytimes3xbfgragh.onion/2020/07/13/us/lausd-san-diego-school-reopening.html?action=click\&pgtype=Article\&state=default\&region=MAIN_CONTENT_3\&context=storylines_faq}{instruction
    will be remote-only in the fall}, citing concerns that surging
    coronavirus infections in their areas pose too dire a risk for
    students and teachers. Together, the two districts enroll some
    825,000 students. They are the largest in the country so far to
    abandon plans for even a partial physical return to classrooms when
    they reopen in August. For other districts, the solution won't be an
    all-or-nothing approach.
    \href{https://bioethics.jhu.edu/research-and-outreach/projects/eschool-initiative/school-policy-tracker/}{Many
    systems}, including the nation's largest, New York City, are
    devising
    \href{https://www.nytimes3xbfgragh.onion/2020/06/26/us/coronavirus-schools-reopen-fall.html?action=click\&pgtype=Article\&state=default\&region=MAIN_CONTENT_3\&context=storylines_faq}{hybrid
    plans} that involve spending some days in classrooms and other days
    online. There's no national policy on this yet, so check with your
    municipal school system regularly to see what is happening in your
    community.
  \end{itemize}
\end{itemize}

Hotel owners, he added, were negotiating with the Hotel Trades Council,
the powerful union that represents 40,000 workers, including room
cleaners, desk clerks, waiters and bartenders. ``The layoffs that we're
talking about are truly temporary, and we hope that business will pick
up again soon,'' Mr. Grapstein said.

Peter Ward, the union's president, said, ``The union is working
diligently with management to mitigate the consequences of this
unprecedented crisis on our members and their families.''

Mr. Grapstein said that ``there has been a lot of discussion about
hotels making sure they are financially viable.'' Many of them routinely
lose money in the slower months of January and February but start
turning profits as tourism picks up in the spring, he said.

``February was kind of bandaged together but things have fallen apart,''
Mr. Grapstein continued. Some hotels saw their occupancy rates fall from
70 percent to 30 percent, he said, and even lower as the news about the
spread of the coronavirus became bleaker and spurred a cascade
cancellations.

The sudden closing last week of Broadway theaters --- a keystone of New
York's tourism industry --- as well as museums and other cultural
institutions, dealt a blow to restaurants, too.

Before the order to close restaurants, some, like Rosa Mexicano, which
has restaurants on the East Side of Manhattan and on the West Side near
Lincoln Center, had been trying to stay open as long as they could ---
though with half as many seats.

Rosa Mexicano removed tables and chairs from its dining rooms to comply
with Gov. Andrew M. Cuomo's order that smaller public spaces reduce
their capacity by 50 percent, said Chris Westcott, the chief executive
of the company that operates them.

Even without the new rule, diners had already been staying away.

``Quite frankly, there wasn't enough business to fill a lot of those
tables,'' Mr. Westcott said.

Image

``They all want to work more, but we're not in a position to keep people
whole,'' Chris Westcott, the chief executive of the company that
operates Rosa Mexicano, said of his work force.Credit...Brittainy
Newman/The New York Times

He said that on Thursday night, business was down about 60 percent at
the Lincoln Center location and about 40 percent on the East Side, a
more residential neighborhood.

Still, the company has not yet laid off any workers, opting instead to
cut their hours and to try to continue providing customers with an
atmosphere honed over 35 years: the same menu, guacamole prepared
tableside.

``Everybody still has a job,'' Mr. Westcott said. ``I think they all
want to work more, but we're not in a position to keep people whole.''

Mr. Westcott, who worked for a different company in 2001, said the
current situation made it more difficult to plan than in the days after
Sept. 11. ``With 9/11, we hit bottom almost immediately,'' he said.
``Now, we're still waiting to hit bottom.''

The city's Chinatowns, in Manhattan, Brooklyn and Queens, have been
reeling for weeks with many customers staying away after the coronavirus
started to break out in China.

Jing Fong, a large and popular dim sum restaurant in Manhattan closed
--- temporarily its owners said --- and laid off its entire staff of 180
employees.

Ming Lam, whose family owns Jing Fong, said business had dropped by 80
percent over the past six weeks.

Qing Chen, 59, a waiter at the restaurant, said he had applied for
unemployment but was worried about when he could start collecting
benefits and, more important, when he could go back to work. ``We live
in a very confusing and worrying time,'' he said.

Advertisement

\protect\hyperlink{after-bottom}{Continue reading the main story}

\hypertarget{site-index}{%
\subsection{Site Index}\label{site-index}}

\hypertarget{site-information-navigation}{%
\subsection{Site Information
Navigation}\label{site-information-navigation}}

\begin{itemize}
\tightlist
\item
  \href{https://help.nytimes3xbfgragh.onion/hc/en-us/articles/115014792127-Copyright-notice}{©~2020~The
  New York Times Company}
\end{itemize}

\begin{itemize}
\tightlist
\item
  \href{https://www.nytco.com/}{NYTCo}
\item
  \href{https://help.nytimes3xbfgragh.onion/hc/en-us/articles/115015385887-Contact-Us}{Contact
  Us}
\item
  \href{https://www.nytco.com/careers/}{Work with us}
\item
  \href{https://nytmediakit.com/}{Advertise}
\item
  \href{http://www.tbrandstudio.com/}{T Brand Studio}
\item
  \href{https://www.nytimes3xbfgragh.onion/privacy/cookie-policy\#how-do-i-manage-trackers}{Your
  Ad Choices}
\item
  \href{https://www.nytimes3xbfgragh.onion/privacy}{Privacy}
\item
  \href{https://help.nytimes3xbfgragh.onion/hc/en-us/articles/115014893428-Terms-of-service}{Terms
  of Service}
\item
  \href{https://help.nytimes3xbfgragh.onion/hc/en-us/articles/115014893968-Terms-of-sale}{Terms
  of Sale}
\item
  \href{https://spiderbites.nytimes3xbfgragh.onion}{Site Map}
\item
  \href{https://help.nytimes3xbfgragh.onion/hc/en-us}{Help}
\item
  \href{https://www.nytimes3xbfgragh.onion/subscription?campaignId=37WXW}{Subscriptions}
\end{itemize}
