Sections

SEARCH

\protect\hyperlink{site-content}{Skip to
content}\protect\hyperlink{site-index}{Skip to site index}

\href{/section/us}{U.S.}\textbar{}To Battle Virus, 7 California Counties
Order Everyone to Stay Home

\url{https://nyti.ms/2vpvxFh}

\begin{itemize}
\item
\item
\item
\item
\item
\end{itemize}

\hypertarget{the-coronavirus-outbreak}{%
\subsubsection{\texorpdfstring{\href{https://www.nytimes3xbfgragh.onion/news-event/coronavirus?name=styln-coronavirus-national\&region=TOP_BANNER\&variant=undefined\&block=storyline_menu_recirc\&action=click\&pgtype=Article\&impression_id=0907ca80-e3b6-11ea-b8d7-679efbc46ecc}{The
Coronavirus
Outbreak}}{The Coronavirus Outbreak}}\label{the-coronavirus-outbreak}}

\begin{itemize}
\tightlist
\item
  live\href{https://www.nytimes3xbfgragh.onion/2020/08/21/world/covid-19-coronavirus.html?name=styln-coronavirus-national\&region=TOP_BANNER\&variant=undefined\&block=storyline_menu_recirc\&action=click\&pgtype=Article\&impression_id=0907f190-e3b6-11ea-b8d7-679efbc46ecc}{Latest
  Updates}
\item
  \href{https://www.nytimes3xbfgragh.onion/interactive/2020/us/coronavirus-us-cases.html?name=styln-coronavirus-national\&region=TOP_BANNER\&variant=undefined\&block=storyline_menu_recirc\&action=click\&pgtype=Article\&impression_id=0907f191-e3b6-11ea-b8d7-679efbc46ecc}{Maps
  and Cases}
\item
  \href{https://www.nytimes3xbfgragh.onion/interactive/2020/science/coronavirus-vaccine-tracker.html?name=styln-coronavirus-national\&region=TOP_BANNER\&variant=undefined\&block=storyline_menu_recirc\&action=click\&pgtype=Article\&impression_id=0907f192-e3b6-11ea-b8d7-679efbc46ecc}{Vaccine
  Tracker}
\item
  \href{https://www.nytimes3xbfgragh.onion/2020/08/19/us/colleges-closing-covid.html?name=styln-coronavirus-national\&region=TOP_BANNER\&variant=undefined\&block=storyline_menu_recirc\&action=click\&pgtype=Article\&impression_id=0907f193-e3b6-11ea-b8d7-679efbc46ecc}{Colleges
  Closing}
\item
  \href{https://www.nytimes3xbfgragh.onion/live/2020/08/21/business/stock-market-today-coronavirus?name=styln-coronavirus-national\&region=TOP_BANNER\&variant=undefined\&block=storyline_menu_recirc\&action=click\&pgtype=Article\&impression_id=0907f194-e3b6-11ea-b8d7-679efbc46ecc}{Economy}
\end{itemize}

\includegraphics{https://static01.graylady3jvrrxbe.onion/images/2020/03/15/us/00virus-california04/merlin_170438664_cb68275a-79b3-4ed4-9f24-ce49d71288c5-articleLarge.jpg?quality=75\&auto=webp\&disable=upscale}

\hypertarget{to-battle-virus-7-california-counties-order-everyone-to-stay-home}{%
\section{To Battle Virus, 7 California Counties Order Everyone to Stay
Home}\label{to-battle-virus-7-california-counties-order-everyone-to-stay-home}}

On Monday, seven counties around Silicon Valley announced residents
should not go out for three weeks except to meet ``essential needs.''

A drive-through testing site at Kaiser Permanente in San
Francisco.Credit...Jim Wilson/The New York Times

Supported by

\protect\hyperlink{after-sponsor}{Continue reading the main story}

By \href{https://www.nytimes3xbfgragh.onion/by/tim-arango}{Tim Arango},
\href{https://www.nytimes3xbfgragh.onion/by/thomas-fuller}{Thomas
Fuller}, \href{https://www.nytimes3xbfgragh.onion/by/john-eligon}{John
Eligon} and
\href{https://www.nytimes3xbfgragh.onion/by/conor-dougherty}{Conor
Dougherty}

\begin{itemize}
\item
  Published March 16, 2020Updated March 23, 2020
\item
  \begin{itemize}
  \item
  \item
  \item
  \item
  \item
  \end{itemize}
\end{itemize}

SACRAMENTO --- Across California, as the coronavirus marches through
communities, life as everyone understands it in the Golden State is
changing dramatically, hour by hour, minute by minute.

The state has begun enacting extreme measures to halt the coronavirus
outbreak. On Monday, seven counties around Silicon Valley, one of the
hardest-hit areas in the nation, announced a shelter-at-home order that
begins Tuesday, which Mayor Sam Liccardo of San Jose said was the
strongest directive yet in the United States. Residents, including those
living in San Francisco, were told not to go out for three weeks except
to meet ``essential needs.''

A day earlier, Gov. Gavin Newsom had told all residents older than 65 to
stay in their homes. He called for the closure of bars, nightclubs and
wineries, and restrictions on restaurants. He banned visits to hospitals
and nursing homes unless patients were on the edge of death. He
announced plans to buy hotels to house some of the state's
\href{https://www.nytimes3xbfgragh.onion/2020/03/10/us/coronavirus-homeless.html}{150,000
homeless people}.

For days, the virus has consumed the state's leaders. In the capital,
Mr. Newsom huddled around a conference table with his advisers,
scrambling to sort out how 6 million public school children would do if
they were not in class.

In the heart of Silicon Valley, Mr. Liccardo raced down the 101 freeway
dialing technology executives, begging them to contribute to a fund to
prevent people from falling into homelessness.

And in the fields of the Central Valley, the nation's breadbasket,
Miguel Arias, Fresno's City Council president, inspected the tight
living quarters of farmworkers --- mattresses stacked up in a garage ---
and recoiled at the thought of what would happen if, or more likely
when, the virus spreads there.

California, America's most populous state, with an economy bigger than
the United Kingdom's, has been remarkably resilient since the Great
Recession, powered by technology, agriculture and Hollywood. No one
knows how far the mounting toll from the virus will climb, but
California is already one of the hardest-hit states, and stands as one
of the places with the most to lose.

The shelter-in-place order announced on Monday, which goes into effect
on Tuesday, is expected to disrupt life for millions of residents in
Alameda, Contra Costa, Marin, San Francisco, San Mateo, Santa Clara and
Santa Cruz Counties. The city of Berkeley also issued the same order.

``This is not the moment for half-measures, and history won't forgive us
for waiting an hour more,'' Mr. Liccardo said.

Residents are being ordered to stay home except for essential reasons,
which include buying food; people can also leave for outdoor activities
including ``walking, hiking or running'' and caring for a pet.

Mayor Joe Goethals of San Mateo put it this way: ``I'm asking people to
go home with their families and to stay there until they are told
otherwise.''

Mayor London Breed of San Francisco said ``necessary'' government
offices and ``essential stores'' would be allowed to remain open. The
order, she said in a Twitter post, would be effective at midnight.

\hypertarget{latest-updates-the-coronavirus-outbreak}{%
\section{\texorpdfstring{\href{https://www.nytimes3xbfgragh.onion/2020/08/21/world/covid-19-coronavirus.html?action=click\&pgtype=Article\&state=default\&region=MAIN_CONTENT_1\&context=storylines_live_updates}{Latest
Updates: The Coronavirus
Outbreak}}{Latest Updates: The Coronavirus Outbreak}}\label{latest-updates-the-coronavirus-outbreak}}

Updated 2020-08-21T13:48:59.464Z

\begin{itemize}
\tightlist
\item
  \href{https://www.nytimes3xbfgragh.onion/2020/08/21/world/covid-19-coronavirus.html?action=click\&pgtype=Article\&state=default\&region=MAIN_CONTENT_1\&context=storylines_live_updates\#link-6a60a19d}{`Be
  adults': Universities in the U.S. are warning students about
  gatherings as they return to campus.}
\item
  \href{https://www.nytimes3xbfgragh.onion/2020/08/21/world/covid-19-coronavirus.html?action=click\&pgtype=Article\&state=default\&region=MAIN_CONTENT_1\&context=storylines_live_updates\#link-324af071}{As
  he accepts the Democratic nomination, Biden knocks Trump's pandemic
  response.}
\item
  \href{https://www.nytimes3xbfgragh.onion/2020/08/21/world/covid-19-coronavirus.html?action=click\&pgtype=Article\&state=default\&region=MAIN_CONTENT_1\&context=storylines_live_updates\#link-191d44be}{South
  Korea threatens to detain people who obstruct virus-control efforts.}
\end{itemize}

\href{https://www.nytimes3xbfgragh.onion/2020/08/21/world/covid-19-coronavirus.html?action=click\&pgtype=Article\&state=default\&region=MAIN_CONTENT_1\&context=storylines_live_updates}{See
more updates}

More live coverage:
\href{https://www.nytimes3xbfgragh.onion/live/2020/08/21/business/stock-market-today-coronavirus?action=click\&pgtype=Article\&state=default\&region=MAIN_CONTENT_1\&context=storylines_live_updates}{Markets}

In a state very familiar with disasters, from wildfires to earthquakes,
the leaders of California found themselves in recent days confronting
something altogether different, with no playbook to lean on.

\emph{{[}Want to get our daily newsletter California Today by email?}
\href{http://www.nytimes3xbfgragh.onion/newsletters/california-today}{\emph{Here's
the sign-up}}\emph{.{]}}

The state that has pioneered the technology that allows people to
connect remotely is quickly realizing how much human contact is
important not only for the economy but also for the well-being of its
residents.

In the time it takes for an email to drop in an inbox, or a news alert
to flash across an iPhone screen, school districts, one after another,
were closing. Movie productions were shutting down, premieres canceled.
At the ports, terminals were shuttered, as fewer ships, loaded with
consumer goods and parts for American factories, set sail from China.
And exports, poultry and oranges, were piling up on the docks.

Even Disneyland was closing its gates, and the ski lifts stopped running
at resorts.

With the sense of crisis growing by the minute, the governor's office
was confronting a dizzying number of problems.

\includegraphics{https://static01.graylady3jvrrxbe.onion/images/2020/03/15/us/00virus-california03/merlin_170483304_bdbc41f0-ce22-4f00-b640-c8f993473cf5-articleLarge.jpg?quality=75\&auto=webp\&disable=upscale}

Image

Nursing students instructed first graders at Ethel Phillips Elementary
School in Sacramento how to properly wash their hands to reduce the
spread of the virus.~Credit...Max Whittaker for The New York Times

One moment it was how to prevent renters from being evicted and
homeowners from being foreclosed on.

Another, how to expand the state's capacity to test for the virus.

And the school closings, designed to enforce social distancing and halt
the spread of the virus, raised a number of difficult questions.

In shirt sleeves, and as aides came in and out of the room with news
updates, the governor considered how to feed low-income children who
rely on free or reduced-price lunches. What about child care for parents
who cannot work from home, especially health care workers who are needed
at hospitals to treat sick patients?

``This is real,'' Mr. Newsom said while managing the many crises. ``This
is raw.''

Mr. Newsom has said he expected the economic damage from the crisis to
be worse than the aftermath of 9/11 but not as bad as the financial
crisis of 2008, but those assessments are changing by the day. Yet with
a \$21 billion budget surplus, plus a rainy-day fund of close to \$16
billion, Mr. Newsom said he was confident the state could manage the
economic fallout from the crisis. ``We are well positioned from a cash
perspective to get through this,'' he said. ``More perhaps than any
other state.''

But at the ground level, the pain is coming fast. Drivers have been laid
off and forced to sell their trucks. Those who are still working are
putting off oil changes and maintenance to stay afloat.

``We're picking and choosing which bills to pay,'' said Gio Marz, 30, a
truck driver who hauls containers from the ports of Los Angeles and Long
Beach to warehouses in Southern California.

Restaurants are closing, and real estate agents say buyers are pulling
their offers because sagging stock portfolios have left them spooked and
shriveled the amount of cash they have for down payments.

Even the most optimistic economists are forecasting a recession. ``This
is the first time in 10 years that I've thought, `OK, this is the thing
that could finally tip us into recession,''' said Chris Thornberg,
founding partner of Beacon Economics, a consulting firm.

Fallout from the coronavirus has been swift across the state.

At one of Mayor Liccardo's favorite Chinese restaurants, Hunan Taste, in
San Jose, only two of the 15 tables had customers at a time when the
restaurant would normally be overflowing with city hall workers, lawyers
and sheriff's deputies. The owner had already laid off two kitchen
workers. At an empty Mexican restaurant Mr. Liccardo visited, the head
of a local Latino business organization told him business owners were
weeks away from shutting down. A nearby fitness studio was empty, and
classes were canceled.

``This is grim,'' Mr. Liccardo said. ``This looks worse than 2008.''

Later, in a nearly empty City Hall, he met with three aides, all sitting
several feet apart in accordance with the county's social distancing
guidelines, and discussed converting school gymnasiums and the hulking
city convention center into isolation centers for the sick.

The mayor estimates that 8 percent of city revenues have vanished,
blowing up plans for a balanced budget and raising the likelihood of
cuts to city programs.

Image

Mayor Sam Liccardo of San Jose, Calif., spoke with an epidemiologist at
Stanford University.Credit...Jason Henry for The New York Times

Image

Hunan Taste, a Chinese restaurant in San Jose, was nearly empty at
lunchtime on Friday when the restaurant would normally be overflowing
with City Hall workers, lawyers and sheriff's deputies. Credit...Jason
Henry for The New York Times

Miles away, in the farming community of Mendota in the Central Valley,
Maria Martinez's home, a single-story ranch with brick and stucco siding
and solar panels on the roof, would seem an ideal place to hunker down
and keep a safe distance from others.

But peek inside, and several single mattresses sit in the garage. Like
many of her neighbors, Ms. Martinez, a 57-year-old immigrant from El
Salvador, rents out beds for \$300 a month to the farmworkers who cycle
through town. She crams them in wherever she can find space ---
sometimes there are 20 renters living in 1,100 square feet.

Mr. Arias, the Fresno official who was visiting her, worried the cramped
living environment, pervasive around town, would fuel a spread of the
coronavirus if it reaches the community.

\href{https://www.nytimes3xbfgragh.onion/news-event/coronavirus?action=click\&pgtype=Article\&state=default\&region=MAIN_CONTENT_3\&context=storylines_faq}{}

\hypertarget{the-coronavirus-outbreak-}{%
\subsubsection{The Coronavirus Outbreak
›}\label{the-coronavirus-outbreak-}}

\hypertarget{frequently-asked-questions}{%
\paragraph{Frequently Asked
Questions}\label{frequently-asked-questions}}

Updated August 17, 2020

\begin{itemize}
\item ~
  \hypertarget{why-does-standing-six-feet-away-from-others-help}{%
  \paragraph{Why does standing six feet away from others
  help?}\label{why-does-standing-six-feet-away-from-others-help}}

  \begin{itemize}
  \tightlist
  \item
    The coronavirus spreads primarily through droplets from your mouth
    and nose, especially when you cough or sneeze. The C.D.C., one of
    the organizations using that measure,
    \href{https://www.nytimes3xbfgragh.onion/2020/04/14/health/coronavirus-six-feet.html?action=click\&pgtype=Article\&state=default\&region=MAIN_CONTENT_3\&context=storylines_faq}{bases
    its recommendation of six feet} on the idea that most large droplets
    that people expel when they cough or sneeze will fall to the ground
    within six feet. But six feet has never been a magic number that
    guarantees complete protection. Sneezes, for instance, can launch
    droplets a lot farther than six feet,
    \href{https://jamanetwork.com/journals/jama/fullarticle/2763852}{according
    to a recent study}. It's a rule of thumb: You should be safest
    standing six feet apart outside, especially when it's windy. But
    keep a mask on at all times, even when you think you're far enough
    apart.
  \end{itemize}
\item ~
  \hypertarget{i-have-antibodies-am-i-now-immune}{%
  \paragraph{I have antibodies. Am I now
  immune?}\label{i-have-antibodies-am-i-now-immune}}

  \begin{itemize}
  \tightlist
  \item
    As of right
    now,\href{https://www.nytimes3xbfgragh.onion/2020/07/22/health/covid-antibodies-herd-immunity.html?action=click\&pgtype=Article\&state=default\&region=MAIN_CONTENT_3\&context=storylines_faq}{that
    seems likely, for at least several months.} There have been
    frightening accounts of people suffering what seems to be a second
    bout of Covid-19. But experts say these patients may have a
    drawn-out course of infection, with the virus taking a slow toll
    weeks to months after initial exposure. People infected with the
    coronavirus typically
    \href{https://www.nature.com/articles/s41586-020-2456-9}{produce}
    immune molecules called antibodies, which are
    \href{https://www.nytimes3xbfgragh.onion/2020/05/07/health/coronavirus-antibody-prevalence.html?action=click\&pgtype=Article\&state=default\&region=MAIN_CONTENT_3\&context=storylines_faq}{protective
    proteins made in response to an
    infection}\href{https://www.nytimes3xbfgragh.onion/2020/05/07/health/coronavirus-antibody-prevalence.html?action=click\&pgtype=Article\&state=default\&region=MAIN_CONTENT_3\&context=storylines_faq}{.
    These antibodies may} last in the body
    \href{https://www.nature.com/articles/s41591-020-0965-6}{only two to
    three months}, which may seem worrisome, but that's perfectly normal
    after an acute infection subsides, said Dr. Michael Mina, an
    immunologist at Harvard University. It may be possible to get the
    coronavirus again, but it's highly unlikely that it would be
    possible in a short window of time from initial infection or make
    people sicker the second time.
  \end{itemize}
\item ~
  \hypertarget{im-a-small-business-owner-can-i-get-relief}{%
  \paragraph{I'm a small-business owner. Can I get
  relief?}\label{im-a-small-business-owner-can-i-get-relief}}

  \begin{itemize}
  \tightlist
  \item
    The
    \href{https://www.nytimes3xbfgragh.onion/article/small-business-loans-stimulus-grants-freelancers-coronavirus.html?action=click\&pgtype=Article\&state=default\&region=MAIN_CONTENT_3\&context=storylines_faq}{stimulus
    bills enacted in March} offer help for the millions of American
    small businesses. Those eligible for aid are businesses and
    nonprofit organizations with fewer than 500 workers, including sole
    proprietorships, independent contractors and freelancers. Some
    larger companies in some industries are also eligible. The help
    being offered, which is being managed by the Small Business
    Administration, includes the Paycheck Protection Program and the
    Economic Injury Disaster Loan program. But lots of folks have
    \href{https://www.nytimes3xbfgragh.onion/interactive/2020/05/07/business/small-business-loans-coronavirus.html?action=click\&pgtype=Article\&state=default\&region=MAIN_CONTENT_3\&context=storylines_faq}{not
    yet seen payouts.} Even those who have received help are confused:
    The rules are draconian, and some are stuck sitting on
    \href{https://www.nytimes3xbfgragh.onion/2020/05/02/business/economy/loans-coronavirus-small-business.html?action=click\&pgtype=Article\&state=default\&region=MAIN_CONTENT_3\&context=storylines_faq}{money
    they don't know how to use.} Many small-business owners are getting
    less than they expected or
    \href{https://www.nytimes3xbfgragh.onion/2020/06/10/business/Small-business-loans-ppp.html?action=click\&pgtype=Article\&state=default\&region=MAIN_CONTENT_3\&context=storylines_faq}{not
    hearing anything at all.}
  \end{itemize}
\item ~
  \hypertarget{what-are-my-rights-if-i-am-worried-about-going-back-to-work}{%
  \paragraph{What are my rights if I am worried about going back to
  work?}\label{what-are-my-rights-if-i-am-worried-about-going-back-to-work}}

  \begin{itemize}
  \tightlist
  \item
    Employers have to provide
    \href{https://www.osha.gov/SLTC/covid-19/standards.html}{a safe
    workplace} with policies that protect everyone equally.
    \href{https://www.nytimes3xbfgragh.onion/article/coronavirus-money-unemployment.html?action=click\&pgtype=Article\&state=default\&region=MAIN_CONTENT_3\&context=storylines_faq}{And
    if one of your co-workers tests positive for the coronavirus, the
    C.D.C.} has said that
    \href{https://www.cdc.gov/coronavirus/2019-ncov/community/guidance-business-response.html}{employers
    should tell their employees} -\/- without giving you the sick
    employee's name -\/- that they may have been exposed to the virus.
  \end{itemize}
\item ~
  \hypertarget{what-is-school-going-to-look-like-in-september}{%
  \paragraph{What is school going to look like in
  September?}\label{what-is-school-going-to-look-like-in-september}}

  \begin{itemize}
  \tightlist
  \item
    It is unlikely that many schools will return to a normal schedule
    this fall, requiring the grind of
    \href{https://www.nytimes3xbfgragh.onion/2020/06/05/us/coronavirus-education-lost-learning.html?action=click\&pgtype=Article\&state=default\&region=MAIN_CONTENT_3\&context=storylines_faq}{online
    learning},
    \href{https://www.nytimes3xbfgragh.onion/2020/05/29/us/coronavirus-child-care-centers.html?action=click\&pgtype=Article\&state=default\&region=MAIN_CONTENT_3\&context=storylines_faq}{makeshift
    child care} and
    \href{https://www.nytimes3xbfgragh.onion/2020/06/03/business/economy/coronavirus-working-women.html?action=click\&pgtype=Article\&state=default\&region=MAIN_CONTENT_3\&context=storylines_faq}{stunted
    workdays} to continue. California's two largest public school
    districts --- Los Angeles and San Diego --- said on July 13, that
    \href{https://www.nytimes3xbfgragh.onion/2020/07/13/us/lausd-san-diego-school-reopening.html?action=click\&pgtype=Article\&state=default\&region=MAIN_CONTENT_3\&context=storylines_faq}{instruction
    will be remote-only in the fall}, citing concerns that surging
    coronavirus infections in their areas pose too dire a risk for
    students and teachers. Together, the two districts enroll some
    825,000 students. They are the largest in the country so far to
    abandon plans for even a partial physical return to classrooms when
    they reopen in August. For other districts, the solution won't be an
    all-or-nothing approach.
    \href{https://bioethics.jhu.edu/research-and-outreach/projects/eschool-initiative/school-policy-tracker/}{Many
    systems}, including the nation's largest, New York City, are
    devising
    \href{https://www.nytimes3xbfgragh.onion/2020/06/26/us/coronavirus-schools-reopen-fall.html?action=click\&pgtype=Article\&state=default\&region=MAIN_CONTENT_3\&context=storylines_faq}{hybrid
    plans} that involve spending some days in classrooms and other days
    online. There's no national policy on this yet, so check with your
    municipal school system regularly to see what is happening in your
    community.
  \end{itemize}
\end{itemize}

``They're living on top of each other,'' he said. ``There's no way that
they can honor the six-foot social distance. That's the most dangerous
part.''

At La Nayarit, a grocery store and wire transfer operation just outside
Mendota's main commercial district, the owners use Clorox wipes on all
their counters and door handles at least once every two hours. One of
the owners, Baudelia Fuentes, 73, microwaved dollar bills on a recent
afternoon, hoping that would disinfect them.

The most anxiety-provoking question around town is, what happens if the
virus spreads among farmworkers?

``Oh, we're done,'' said Rolando Castro, the mayor of Mendota.

Image

``We don't care about coronavirus here,'' said~Joseph Riofro, the owner
of Westside Pool Hall in Mendota, Calif.Credit...Max Whittaker for The
New York Times

Image

Saturday grocery shopping in Mendota.Credit...Max Whittaker for The New
York Times

It would threaten not only farm operations, but also the economy of a
town with five dollar stores and where 42 percent of families live in
poverty. There will not be money to spend in the local pool hall, or at
the taco truck across the street. Or at the auto mechanic shop that Mr.
Castro owns.

As a sense of crisis convulsed California and the nation, tourists in
Southern California enjoyed their last few hours at Disneyland, which
was closed as of Saturday. The park has closed only twice before at
times of national crisis: after the assassination of President John F.
Kennedy in 1963 and in the wake of the Sept. 11, 2001, terrorist
attacks.

The weather was overcast, and crowds were noticeably smaller than usual.
The line for the popular Space Mountain ride was just over an hour, and
the wait for the Matterhorn ride was about 30 minutes. On a normal
spring weekday, waits for those rides can be twice as long.

``If I'm going to get corona, I might as well get it at Disneyland
instead of work,'' said Sami Nielsen while strolling down the park's
main street. She traveled to California from Arizona with friends to
celebrate her 27th birthday.

Image

Crowds at Disneyland in Anaheim, Calif., on Friday.Credit...Kendrick
Brinson for The New York Times

After a series of meetings late Friday, Mr. Newsom signed an executive
order requiring the state to continue to send money to local districts
to pay for distance learning and school meals and to help supervise
children while they are out of school.

``I am deeply concerned about the capacity of these communities that are
shutting down the schools already to meet the needs of their children
and parents,'' he said.

The terrifying backdrop to all of this is the steadily growing number of
confirmed infections --- over the weekend the number of infections in
California reached 380 --- and the certainty they will rise sharply as
testing becomes more widespread.

``Tests are going to substantially increase,'' the governor said.
``Positive rates will substantially increase. The anxiety and concern
about anticipating the public's reaction knowing that will heighten
anxiety is something we are all trying to manage.''

Tim Arango reported from Sacramento, Thomas Fuller from San Jose, John
Eligon from Mendota, and Conor Dougherty from Oakland. **** Louis Keene
contributed from Anaheim, and Joe Purtell from Elk Grove.

Advertisement

\protect\hyperlink{after-bottom}{Continue reading the main story}

\hypertarget{site-index}{%
\subsection{Site Index}\label{site-index}}

\hypertarget{site-information-navigation}{%
\subsection{Site Information
Navigation}\label{site-information-navigation}}

\begin{itemize}
\tightlist
\item
  \href{https://help.nytimes3xbfgragh.onion/hc/en-us/articles/115014792127-Copyright-notice}{©~2020~The
  New York Times Company}
\end{itemize}

\begin{itemize}
\tightlist
\item
  \href{https://www.nytco.com/}{NYTCo}
\item
  \href{https://help.nytimes3xbfgragh.onion/hc/en-us/articles/115015385887-Contact-Us}{Contact
  Us}
\item
  \href{https://www.nytco.com/careers/}{Work with us}
\item
  \href{https://nytmediakit.com/}{Advertise}
\item
  \href{http://www.tbrandstudio.com/}{T Brand Studio}
\item
  \href{https://www.nytimes3xbfgragh.onion/privacy/cookie-policy\#how-do-i-manage-trackers}{Your
  Ad Choices}
\item
  \href{https://www.nytimes3xbfgragh.onion/privacy}{Privacy}
\item
  \href{https://help.nytimes3xbfgragh.onion/hc/en-us/articles/115014893428-Terms-of-service}{Terms
  of Service}
\item
  \href{https://help.nytimes3xbfgragh.onion/hc/en-us/articles/115014893968-Terms-of-sale}{Terms
  of Sale}
\item
  \href{https://spiderbites.nytimes3xbfgragh.onion}{Site Map}
\item
  \href{https://help.nytimes3xbfgragh.onion/hc/en-us}{Help}
\item
  \href{https://www.nytimes3xbfgragh.onion/subscription?campaignId=37WXW}{Subscriptions}
\end{itemize}
