Sections

SEARCH

\protect\hyperlink{site-content}{Skip to
content}\protect\hyperlink{site-index}{Skip to site index}

\href{https://www.nytimes3xbfgragh.onion/section/food}{Food}

\href{https://myaccount.nytimes3xbfgragh.onion/auth/login?response_type=cookie\&client_id=vi}{}

\href{https://www.nytimes3xbfgragh.onion/section/todayspaper}{Today's
Paper}

\href{/section/food}{Food}\textbar{}Food, a Basic Pleasure, Is Suddenly
Fraught

\url{https://nyti.ms/3b5mgBz}

\begin{itemize}
\item
\item
\item
\item
\item
\item
\end{itemize}

\hypertarget{the-coronavirus-outbreak}{%
\subsubsection{\texorpdfstring{\href{https://www.nytimes3xbfgragh.onion/news-event/coronavirus?name=styln-coronavirus-national\&region=TOP_BANNER\&variant=undefined\&block=storyline_menu_recirc\&action=click\&pgtype=Article\&impression_id=58e38ea0-e392-11ea-982c-75d1fa15f353}{The
Coronavirus
Outbreak}}{The Coronavirus Outbreak}}\label{the-coronavirus-outbreak}}

\begin{itemize}
\tightlist
\item
  live\href{https://www.nytimes3xbfgragh.onion/2020/08/21/world/covid-19-coronavirus.html?name=styln-coronavirus-national\&region=TOP_BANNER\&variant=undefined\&block=storyline_menu_recirc\&action=click\&pgtype=Article\&impression_id=58e38ea1-e392-11ea-982c-75d1fa15f353}{Latest
  Updates}
\item
  \href{https://www.nytimes3xbfgragh.onion/interactive/2020/us/coronavirus-us-cases.html?name=styln-coronavirus-national\&region=TOP_BANNER\&variant=undefined\&block=storyline_menu_recirc\&action=click\&pgtype=Article\&impression_id=58e3b5b0-e392-11ea-982c-75d1fa15f353}{Maps
  and Cases}
\item
  \href{https://www.nytimes3xbfgragh.onion/interactive/2020/science/coronavirus-vaccine-tracker.html?name=styln-coronavirus-national\&region=TOP_BANNER\&variant=undefined\&block=storyline_menu_recirc\&action=click\&pgtype=Article\&impression_id=58e3b5b1-e392-11ea-982c-75d1fa15f353}{Vaccine
  Tracker}
\item
  \href{https://www.nytimes3xbfgragh.onion/2020/08/19/us/colleges-closing-covid.html?name=styln-coronavirus-national\&region=TOP_BANNER\&variant=undefined\&block=storyline_menu_recirc\&action=click\&pgtype=Article\&impression_id=58e3b5b2-e392-11ea-982c-75d1fa15f353}{Colleges
  Closing}
\item
  \href{https://www.nytimes3xbfgragh.onion/live/2020/08/20/business/stock-market-today-coronavirus?name=styln-coronavirus-national\&region=TOP_BANNER\&variant=undefined\&block=storyline_menu_recirc\&action=click\&pgtype=Article\&impression_id=58e3b5b3-e392-11ea-982c-75d1fa15f353}{Economy}
\end{itemize}

Advertisement

\protect\hyperlink{after-top}{Continue reading the main story}

Supported by

\protect\hyperlink{after-sponsor}{Continue reading the main story}

\hypertarget{food-a-basic-pleasure-is-suddenly-fraught}{%
\section{Food, a Basic Pleasure, Is Suddenly
Fraught}\label{food-a-basic-pleasure-is-suddenly-fraught}}

The coronavirus has forced Americans to rethink how to feed themselves,
from cooking to shopping to ordering in.

\includegraphics{https://static01.graylady3jvrrxbe.onion/images/2020/03/17/dining/17Virus-food1/merlin_170544474_a5becd7c-82f7-450c-8f7f-10fb3755309f-articleLarge.jpg?quality=75\&auto=webp\&disable=upscale}

\href{https://www.nytimes3xbfgragh.onion/by/kim-severson}{\includegraphics{https://static01.graylady3jvrrxbe.onion/images/2018/06/13/multimedia/author-kim-severson/author-kim-severson-thumbLarge.jpg}}\href{https://www.nytimes3xbfgragh.onion/by/julia-moskin}{\includegraphics{https://static01.graylady3jvrrxbe.onion/images/2018/09/25/multimedia/author-julia-moskin/author-julia-moskin-thumbLarge.png}}

By \href{https://www.nytimes3xbfgragh.onion/by/kim-severson}{Kim
Severson} and
\href{https://www.nytimes3xbfgragh.onion/by/julia-moskin}{Julia Moskin}

\begin{itemize}
\item
  Published March 17, 2020Updated March 19, 2020
\item
  \begin{itemize}
  \item
  \item
  \item
  \item
  \item
  \item
  \end{itemize}
\end{itemize}

The white-tablecloth restaurants and the dive bars are closed. The ample
buffets that feed America's tech work force and Las Vegas gamblers have
been shut down, along with millions of school cafeterias. On Monday,
McDonald's joined other fast-food companies and closed its restaurants
except for delivery and drive-through.

Almost overnight, Americans have had to rethink one of the most
elemental parts of their daily lives: food.

It's painful but not impossible to forgo some pleasures, routines and
duties for now: playing a soccer game, going to a movie, visiting a sick
friend. But eating is a cultural ritual that is also a necessity.

Mealtimes arrive three times a day. How do you feed yourself and your
family when restaurants are shuttered and store shelves emptying? And
how do you take care of a community in crisis when caregiving meals like
Sabbath suppers, pancake breakfasts, casseroles for new parents and
funeral food are off limits?

\includegraphics{https://static01.graylady3jvrrxbe.onion/images/2020/03/17/dining/17Virus-food2/17Virus-food2-articleLarge.jpg?quality=75\&auto=webp\&disable=upscale}

Catherine Powell, 58, a lawyer in Crested Butte, Colo., is like millions
of other people who suddenly have to figure out how, and what, to cook
in the age of the coronavirus.

She sat in her car outside a grocery store on Sunday, struggling to pull
up recipes her husband had sent to her phone. ``I'm a recipe follower,''
she said, not a practiced cook who can whip up a meal from a random
assortment of pantry ingredients.

In the end, it didn't matter that she couldn't download the recipes. The
store was out of pasta and canned vegetables. There was no chicken, but
plenty of pork, which she doesn't eat.

``I don't know if I'm going to be able to become a different kind of
cook,'' she said. ``But I'm going to have to try.''

All at once, it seems, many Americans are scrambling to learn how to
cook. A week ago, Google searches for cooking videos hit
Thanksgiving-level highs, and traffic to cooking websites is
skyrocketing.

Cooking teachers and chefs who don't have kitchens to run are offering
tips on social media for cooking from pantry ingredients and making
dishes that can be stretched into several meals. Families and friends
are teaching one another on FaceTime and Zoom to bake sourdough and cook
beans.

``If I'm having to think about this, and I am a culinary professional
who can basically make dinner out of a jar of mustard, then what does
that mean for the normal person?'' said
\href{https://www.nytimes3xbfgragh.onion/2018/08/07/dining/is-it-southern-food-or-soul-food.html}{Virginia
Willis,} 53, a cooking teacher and author in Atlanta.

\hypertarget{latest-updates-the-coronavirus-outbreak}{%
\section{\texorpdfstring{\href{https://www.nytimes3xbfgragh.onion/2020/08/21/world/covid-19-coronavirus.html?action=click\&pgtype=Article\&state=default\&region=MAIN_CONTENT_1\&context=storylines_live_updates}{Latest
Updates: The Coronavirus
Outbreak}}{Latest Updates: The Coronavirus Outbreak}}\label{latest-updates-the-coronavirus-outbreak}}

Updated 2020-08-21T09:36:59.270Z

\begin{itemize}
\tightlist
\item
  \href{https://www.nytimes3xbfgragh.onion/2020/08/21/world/covid-19-coronavirus.html?action=click\&pgtype=Article\&state=default\&region=MAIN_CONTENT_1\&context=storylines_live_updates\#link-4690b6aa}{Shutdowns,
  warnings and scoldings follow gatherings on college campuses.}
\item
  \href{https://www.nytimes3xbfgragh.onion/2020/08/21/world/covid-19-coronavirus.html?action=click\&pgtype=Article\&state=default\&region=MAIN_CONTENT_1\&context=storylines_live_updates\#link-324af071}{As
  he accepts the Democratic nomination, Biden knocks Trump's pandemic
  response.}
\item
  \href{https://www.nytimes3xbfgragh.onion/2020/08/21/world/covid-19-coronavirus.html?action=click\&pgtype=Article\&state=default\&region=MAIN_CONTENT_1\&context=storylines_live_updates\#link-35890b73}{Hundreds
  of doctors in Kenya go on strike over their pay and protective gear.}
\end{itemize}

\href{https://www.nytimes3xbfgragh.onion/2020/08/21/world/covid-19-coronavirus.html?action=click\&pgtype=Article\&state=default\&region=MAIN_CONTENT_1\&context=storylines_live_updates}{See
more updates}

More live coverage:
\href{https://www.nytimes3xbfgragh.onion/live/2020/08/20/business/stock-market-today-coronavirus?action=click\&pgtype=Article\&state=default\&region=MAIN_CONTENT_1\&context=storylines_live_updates}{Markets}

At the same time, a world where everyone is forced to eat at home has
turbocharged the nation's already competitive online delivery services.
\href{https://www.barrons.com/articles/grubhub-stock-nyc-food-delivery-market-coronavirus-51584381819}{GrubHub}
has been adding restaurants to its platforms at a daily rate that's
three times its previous peak, Matt Maloney, the founder and chief
executive, said Tuesday.

As countries like Italy and Spain were shutting down restaurants
completely, Mr. Maloney said his company --- which includes Seamless and
Yelp delivery services, and has the largest network of restaurant
clients in the country --- had lobbied state and local officials to
allow restaurants to keep their kitchens open for takeout and delivery,
even as dining rooms are closed. (The company also announced that
payments from the restaurants --- anywhere from 15 to 30 percent of each
order --- will be deferred but not reduced, which angered many
restaurateurs.)

In the Bay Area,
\href{https://www.nytimes3xbfgragh.onion/reuters/2020/03/16/us/16reuters-health-coronavirus-usa-sfarea-factbox.html}{where
six counties}went into lockdown on Tuesday, restaurant deliveries have
been deemed an essential service that can continue --- for now.

Image

The chef Alice Waters is trying to change her restaurant, Chez Panisse,
so that it can sell produce from farmers and offer complete
meals.Credit...T.J. Kirkpatrick for The New York Times

The California restaurateur
\href{https://www.nytimes3xbfgragh.onion/2017/08/22/dining/alice-waters-chef-author-book.html}{Alice
Waters} was sequestered in her Berkeley home on Monday while
\href{https://www.chezpanisse.com/reservations/}{Chez Panisse}, the
restaurant she opened in 1971, pivoted to selling boxes of produce from
farmers and preparing to offer complete meals of hearth-cooked beans,
greens and cornbread for online ordering.

\href{https://www.kchysmith.com/}{K.C. Hysmith}, 30, a doctoral
candidate at the University of North Carolina studying food history,
said her family was trying to order from local restaurants as much as
possible.

During World War II, Americans tended
\href{https://www.nytimes3xbfgragh.onion/1943/09/19/archives/first-year-is-hardest-despite-mistakes-victory-gardens-have-done.html}{victory
gardens}, needing both fresh food and a sense of participating in a
national cause. In the last few days, Ms. Hysmith said, a similar urge
has set in: Diners are using apps to order from favorite local
restaurants in an effort to save them, knowing that places without
financial safety nets may never reopen.

``We ordered Chinese takeout last night, knowing they could shutter at
any moment,'' she said. ``It's almost as if our victory gardens are how
we are using our digital expertise to feed ourselves and help others.''

But delivery is expensive. It's not readily available in many parts of
the country, and even those averse to cooking can't rely on it forever.

Dan Nguyen, 34, a data scientist with Airbnb in San Francisco, hasn't
cooked since he began working in the tech industry five years ago. He
would eat two meals a day, and occasionally dinner, at work. Anything
else was delivered to his apartment in the Mission district.

``When you get three meals a day, you just stop cooking altogether,''
Mr. Nguyen said. He plans to transition back into cooking at least some
of his own food, while also relying on deliveries --- for as long as
they remain available.

The return to the kitchen has underscored the differences among
America's cooks. City dwellers with small kitchens who are used to
regular shopping at a weekly farmers' market or who cook only when the
urge to make an interesting new recipe strikes have vastly different
skills than Americans used to cooking from a large pantry that is kept
stocked with monthly runs to a big-box store.

\href{https://www.hiokinai.com/}{Hi`ilei Hobart}, 38, an assistant
professor of anthropology at the University of Texas, pointed out that
many Americans, especially in rural communities, already know how to
stock and cook every day from a pantry of shelf-stable ingredients.

``They have knowledge that a lot more people had in the early 20th
century,'' she said. ``That's not to say modern urban dwellers are dumb,
but they have a different knowledge set and skill set that is
appropriate for their everyday lives.''

In Weatherford, Okla., a college town of about 15,000, people are used
to driving a half-hour or more to a grocery store and stocking up for a
week or two. So far, the shelves haven't been stripped. And there are
other sources of food.

``A lot of people here slaughter a beef, and a lot of people have a ton
of meat in their freezer,'' said Garrett King, 34, who was home from his
job at Southwest Oklahoma State University, helping tend his two
preschool-age children. ``There is no one in rural Oklahoma that doesn't
have more than one freezer.''

Still, concern about provisions is setting in, particularly among the
many there who farm for a living and know that any disruption in
government certifications or inspections for grain or cattle could mean
empty store shelves later on.

Even in communities where preserving and storing food is already a
habit, life under social distancing is proving a challenge.

Image

Christy Spackman with the food she has stored, about a month's
worth.Credit...via Christy Spackman

\href{http://christyspackman.com/}{Christy Spackman}, 42, of Tempe,
Ariz., grew up in Utah, and remains deeply involved with the Church of
Jesus Christ of Latter-day Saints, which recommends that members stock a
year's worth of food. She and her friends have joked about the boxes of
bulk ingredients that Mormon brides receive, wondering if large cans of
powdered milk, dried apples and pinto beans can be used as home décor.

\href{https://www.nytimes3xbfgragh.onion/news-event/coronavirus?action=click\&pgtype=Article\&state=default\&region=MAIN_CONTENT_3\&context=storylines_faq}{}

\hypertarget{the-coronavirus-outbreak-}{%
\subsubsection{The Coronavirus Outbreak
›}\label{the-coronavirus-outbreak-}}

\hypertarget{frequently-asked-questions}{%
\paragraph{Frequently Asked
Questions}\label{frequently-asked-questions}}

Updated August 17, 2020

\begin{itemize}
\item ~
  \hypertarget{why-does-standing-six-feet-away-from-others-help}{%
  \paragraph{Why does standing six feet away from others
  help?}\label{why-does-standing-six-feet-away-from-others-help}}

  \begin{itemize}
  \tightlist
  \item
    The coronavirus spreads primarily through droplets from your mouth
    and nose, especially when you cough or sneeze. The C.D.C., one of
    the organizations using that measure,
    \href{https://www.nytimes3xbfgragh.onion/2020/04/14/health/coronavirus-six-feet.html?action=click\&pgtype=Article\&state=default\&region=MAIN_CONTENT_3\&context=storylines_faq}{bases
    its recommendation of six feet} on the idea that most large droplets
    that people expel when they cough or sneeze will fall to the ground
    within six feet. But six feet has never been a magic number that
    guarantees complete protection. Sneezes, for instance, can launch
    droplets a lot farther than six feet,
    \href{https://jamanetwork.com/journals/jama/fullarticle/2763852}{according
    to a recent study}. It's a rule of thumb: You should be safest
    standing six feet apart outside, especially when it's windy. But
    keep a mask on at all times, even when you think you're far enough
    apart.
  \end{itemize}
\item ~
  \hypertarget{i-have-antibodies-am-i-now-immune}{%
  \paragraph{I have antibodies. Am I now
  immune?}\label{i-have-antibodies-am-i-now-immune}}

  \begin{itemize}
  \tightlist
  \item
    As of right
    now,\href{https://www.nytimes3xbfgragh.onion/2020/07/22/health/covid-antibodies-herd-immunity.html?action=click\&pgtype=Article\&state=default\&region=MAIN_CONTENT_3\&context=storylines_faq}{that
    seems likely, for at least several months.} There have been
    frightening accounts of people suffering what seems to be a second
    bout of Covid-19. But experts say these patients may have a
    drawn-out course of infection, with the virus taking a slow toll
    weeks to months after initial exposure. People infected with the
    coronavirus typically
    \href{https://www.nature.com/articles/s41586-020-2456-9}{produce}
    immune molecules called antibodies, which are
    \href{https://www.nytimes3xbfgragh.onion/2020/05/07/health/coronavirus-antibody-prevalence.html?action=click\&pgtype=Article\&state=default\&region=MAIN_CONTENT_3\&context=storylines_faq}{protective
    proteins made in response to an
    infection}\href{https://www.nytimes3xbfgragh.onion/2020/05/07/health/coronavirus-antibody-prevalence.html?action=click\&pgtype=Article\&state=default\&region=MAIN_CONTENT_3\&context=storylines_faq}{.
    These antibodies may} last in the body
    \href{https://www.nature.com/articles/s41591-020-0965-6}{only two to
    three months}, which may seem worrisome, but that's perfectly normal
    after an acute infection subsides, said Dr. Michael Mina, an
    immunologist at Harvard University. It may be possible to get the
    coronavirus again, but it's highly unlikely that it would be
    possible in a short window of time from initial infection or make
    people sicker the second time.
  \end{itemize}
\item ~
  \hypertarget{im-a-small-business-owner-can-i-get-relief}{%
  \paragraph{I'm a small-business owner. Can I get
  relief?}\label{im-a-small-business-owner-can-i-get-relief}}

  \begin{itemize}
  \tightlist
  \item
    The
    \href{https://www.nytimes3xbfgragh.onion/article/small-business-loans-stimulus-grants-freelancers-coronavirus.html?action=click\&pgtype=Article\&state=default\&region=MAIN_CONTENT_3\&context=storylines_faq}{stimulus
    bills enacted in March} offer help for the millions of American
    small businesses. Those eligible for aid are businesses and
    nonprofit organizations with fewer than 500 workers, including sole
    proprietorships, independent contractors and freelancers. Some
    larger companies in some industries are also eligible. The help
    being offered, which is being managed by the Small Business
    Administration, includes the Paycheck Protection Program and the
    Economic Injury Disaster Loan program. But lots of folks have
    \href{https://www.nytimes3xbfgragh.onion/interactive/2020/05/07/business/small-business-loans-coronavirus.html?action=click\&pgtype=Article\&state=default\&region=MAIN_CONTENT_3\&context=storylines_faq}{not
    yet seen payouts.} Even those who have received help are confused:
    The rules are draconian, and some are stuck sitting on
    \href{https://www.nytimes3xbfgragh.onion/2020/05/02/business/economy/loans-coronavirus-small-business.html?action=click\&pgtype=Article\&state=default\&region=MAIN_CONTENT_3\&context=storylines_faq}{money
    they don't know how to use.} Many small-business owners are getting
    less than they expected or
    \href{https://www.nytimes3xbfgragh.onion/2020/06/10/business/Small-business-loans-ppp.html?action=click\&pgtype=Article\&state=default\&region=MAIN_CONTENT_3\&context=storylines_faq}{not
    hearing anything at all.}
  \end{itemize}
\item ~
  \hypertarget{what-are-my-rights-if-i-am-worried-about-going-back-to-work}{%
  \paragraph{What are my rights if I am worried about going back to
  work?}\label{what-are-my-rights-if-i-am-worried-about-going-back-to-work}}

  \begin{itemize}
  \tightlist
  \item
    Employers have to provide
    \href{https://www.osha.gov/SLTC/covid-19/standards.html}{a safe
    workplace} with policies that protect everyone equally.
    \href{https://www.nytimes3xbfgragh.onion/article/coronavirus-money-unemployment.html?action=click\&pgtype=Article\&state=default\&region=MAIN_CONTENT_3\&context=storylines_faq}{And
    if one of your co-workers tests positive for the coronavirus, the
    C.D.C.} has said that
    \href{https://www.cdc.gov/coronavirus/2019-ncov/community/guidance-business-response.html}{employers
    should tell their employees} -\/- without giving you the sick
    employee's name -\/- that they may have been exposed to the virus.
  \end{itemize}
\item ~
  \hypertarget{what-is-school-going-to-look-like-in-september}{%
  \paragraph{What is school going to look like in
  September?}\label{what-is-school-going-to-look-like-in-september}}

  \begin{itemize}
  \tightlist
  \item
    It is unlikely that many schools will return to a normal schedule
    this fall, requiring the grind of
    \href{https://www.nytimes3xbfgragh.onion/2020/06/05/us/coronavirus-education-lost-learning.html?action=click\&pgtype=Article\&state=default\&region=MAIN_CONTENT_3\&context=storylines_faq}{online
    learning},
    \href{https://www.nytimes3xbfgragh.onion/2020/05/29/us/coronavirus-child-care-centers.html?action=click\&pgtype=Article\&state=default\&region=MAIN_CONTENT_3\&context=storylines_faq}{makeshift
    child care} and
    \href{https://www.nytimes3xbfgragh.onion/2020/06/03/business/economy/coronavirus-working-women.html?action=click\&pgtype=Article\&state=default\&region=MAIN_CONTENT_3\&context=storylines_faq}{stunted
    workdays} to continue. California's two largest public school
    districts --- Los Angeles and San Diego --- said on July 13, that
    \href{https://www.nytimes3xbfgragh.onion/2020/07/13/us/lausd-san-diego-school-reopening.html?action=click\&pgtype=Article\&state=default\&region=MAIN_CONTENT_3\&context=storylines_faq}{instruction
    will be remote-only in the fall}, citing concerns that surging
    coronavirus infections in their areas pose too dire a risk for
    students and teachers. Together, the two districts enroll some
    825,000 students. They are the largest in the country so far to
    abandon plans for even a partial physical return to classrooms when
    they reopen in August. For other districts, the solution won't be an
    all-or-nothing approach.
    \href{https://bioethics.jhu.edu/research-and-outreach/projects/eschool-initiative/school-policy-tracker/}{Many
    systems}, including the nation's largest, New York City, are
    devising
    \href{https://www.nytimes3xbfgragh.onion/2020/06/26/us/coronavirus-schools-reopen-fall.html?action=click\&pgtype=Article\&state=default\&region=MAIN_CONTENT_3\&context=storylines_faq}{hybrid
    plans} that involve spending some days in classrooms and other days
    online. There's no national policy on this yet, so check with your
    municipal school system regularly to see what is happening in your
    community.
  \end{itemize}
\end{itemize}

Suddenly, the jokes don't seem so funny anymore. The reality of what not
sharing food can mean to communities connected by food rituals is
starting to sink in.

Her grandfather's funeral will be held in Idaho on Wednesday. The
community of Mormon women who regularly provide food for funerals said
it will be the last one they will service until the threat from the
coronavirus is over.

``It's shocking in a way to realize this thing that is so central to how
we communicate is dangerous,'' said Dr. Spackman, a professor at the
School for the Future of Innovation in Society at Arizona State
University. ``You say to yourself, even though I trust this sisterhood
of believers, how do I trust that they are not a carrier?''

Cooking may provide a sense of control, but today even the first step
--- shopping for ingredients --- can seem risky.

Rather than visit stores, many who can afford it are using shopping
services like Postmates and
\href{https://www.nytimes3xbfgragh.onion/2019/02/06/technology/instacart-doordash-tipping-deliveries.html}{Instacart}.
Grocery delivery services and restaurants offering takeout and delivery
have told consumers that extra sanitation practices have been put in
place. (Federal health officials say the virus is spread mainly through
respiratory droplets and to a lesser degree by touching surfaces that
might hold it. There is no evidence, according to the
\href{https://www.cdc.gov/coronavirus/2019-ncov/faq.html}{Centers for
Disease Control}, that the virus has been transmitted through food.)

Image

Grocery orders awaited delivery Tuesday at a Whole Foods Market in
Washington, D.C,~Credit...Jim Lo Scalzo/EPA, via Shutterstock

Safety aside, a trip to the grocery store can be a somber exercise. At
the Whole Foods Market in Glendale, Calif., all of the chicken --- from
breasts to wings to thighs --- had been snatched up before noon on
Monday. Workers hustled carts through half-empty aisles, trying to meet
a surge in delivery orders.

Valerie Aguilar, 19, a student at California State University-Northridge
who works as a part-time shopper at Whole Foods, looked at a new order
on her phone and shook her head. ``There's no more Cheerios,'' she said
grimly. ``No more beans, no more spaghetti.''

When selection is limited and many people can't go shopping, even the
most adept cook can be challenged. And when you get the groceries home,
the joy of preparing food for yourself and others can be fraught,
especially if family members have special dietary requirements or eating
schedules.

Tina Gupta Patel, 51, lives in Nashville with her husband and two
children, one in high school and the other in college. Both are home
now, and everyone seems to eat on a different timetable.

``It has been really tough to corral everyone and say, `Let's sit down
and eat family dinner the way we used to,' '' she said. So she makes
dishes like pot roast that can be eaten as is or shredded into tortillas
throughout the day, and pushes the fruit and vegetables, which so far
are in abundant supply at the local supermarket.

Unlike some cooks who are embracing the time to cook at home, preparing
food is not cathartic or pleasurable for her at the moment. ``It takes
energy for me to want to cook, and I don't have a lot of energy right
now.''

Even those eager to turn to the kitchen as balm and social connection
are stymied.

``It's a super-strange time,'' said Ms. Powell, the lawyer in Colorado.
``In our community we often take meals to other people or drop off
dinner for our friends. We have some friends who have babies right now.
But I don't want to freak people out with a COVID Stroganoff.''

Priya Krishna, Tejal Rao and Rachel Wharton contributed reporting.

\emph{Follow} \href{https://twitter.com/nytfood}{\emph{NYT Food on
Twitter}} \emph{and}
\href{https://www.instagram.com/nytcooking/}{\emph{NYT Cooking on
Instagram}}\emph{,}
\href{https://www.facebookcorewwwi.onion/nytcooking/}{\emph{Facebook}}\emph{,}
\href{https://www.youtube.com/nytcooking}{\emph{YouTube}} \emph{and}
\href{https://www.pinterest.com/nytcooking/}{\emph{Pinterest}}\emph{.}
\href{https://www.nytimes3xbfgragh.onion/newsletters/cooking}{\emph{Get
regular updates from NYT Cooking, with recipe suggestions, cooking tips
and shopping advice}}\emph{.}

Advertisement

\protect\hyperlink{after-bottom}{Continue reading the main story}

\hypertarget{site-index}{%
\subsection{Site Index}\label{site-index}}

\hypertarget{site-information-navigation}{%
\subsection{Site Information
Navigation}\label{site-information-navigation}}

\begin{itemize}
\tightlist
\item
  \href{https://help.nytimes3xbfgragh.onion/hc/en-us/articles/115014792127-Copyright-notice}{©~2020~The
  New York Times Company}
\end{itemize}

\begin{itemize}
\tightlist
\item
  \href{https://www.nytco.com/}{NYTCo}
\item
  \href{https://help.nytimes3xbfgragh.onion/hc/en-us/articles/115015385887-Contact-Us}{Contact
  Us}
\item
  \href{https://www.nytco.com/careers/}{Work with us}
\item
  \href{https://nytmediakit.com/}{Advertise}
\item
  \href{http://www.tbrandstudio.com/}{T Brand Studio}
\item
  \href{https://www.nytimes3xbfgragh.onion/privacy/cookie-policy\#how-do-i-manage-trackers}{Your
  Ad Choices}
\item
  \href{https://www.nytimes3xbfgragh.onion/privacy}{Privacy}
\item
  \href{https://help.nytimes3xbfgragh.onion/hc/en-us/articles/115014893428-Terms-of-service}{Terms
  of Service}
\item
  \href{https://help.nytimes3xbfgragh.onion/hc/en-us/articles/115014893968-Terms-of-sale}{Terms
  of Sale}
\item
  \href{https://spiderbites.nytimes3xbfgragh.onion}{Site Map}
\item
  \href{https://help.nytimes3xbfgragh.onion/hc/en-us}{Help}
\item
  \href{https://www.nytimes3xbfgragh.onion/subscription?campaignId=37WXW}{Subscriptions}
\end{itemize}
