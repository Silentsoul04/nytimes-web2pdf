Sections

SEARCH

\protect\hyperlink{site-content}{Skip to
content}\protect\hyperlink{site-index}{Skip to site index}

\href{https://www.nytimes3xbfgragh.onion/section/well/live}{Live}

\href{https://myaccount.nytimes3xbfgragh.onion/auth/login?response_type=cookie\&client_id=vi}{}

\href{https://www.nytimes3xbfgragh.onion/section/todayspaper}{Today's
Paper}

\href{/section/well/live}{Live}\textbar{}Deciding How Much Distance You
Should Keep

\url{https://nyti.ms/33yZrUb}

\begin{itemize}
\item
\item
\item
\item
\item
\item
\end{itemize}

\hypertarget{the-coronavirus-outbreak}{%
\subsubsection{\texorpdfstring{\href{https://www.nytimes3xbfgragh.onion/news-event/coronavirus?name=styln-coronavirus-national\&region=TOP_BANNER\&block=storyline_menu_recirc\&action=click\&pgtype=Article\&impression_id=4b72e2b0-efb9-11ea-b221-c1ac8458c493\&variant=undefined}{The
Coronavirus
Outbreak}}{The Coronavirus Outbreak}}\label{the-coronavirus-outbreak}}

\begin{itemize}
\tightlist
\item
  live\href{https://www.nytimes3xbfgragh.onion/2020/09/05/world/coronavirus-covid.html?name=styln-coronavirus-national\&region=TOP_BANNER\&block=storyline_menu_recirc\&action=click\&pgtype=Article\&impression_id=4b7309c0-efb9-11ea-b221-c1ac8458c493\&variant=undefined}{Latest
  Updates}
\item
  \href{https://www.nytimes3xbfgragh.onion/interactive/2020/us/coronavirus-us-cases.html?name=styln-coronavirus-national\&region=TOP_BANNER\&block=storyline_menu_recirc\&action=click\&pgtype=Article\&impression_id=4b7309c1-efb9-11ea-b221-c1ac8458c493\&variant=undefined}{Maps
  and Cases}
\item
  \href{https://www.nytimes3xbfgragh.onion/interactive/2020/science/coronavirus-vaccine-tracker.html?name=styln-coronavirus-national\&region=TOP_BANNER\&block=storyline_menu_recirc\&action=click\&pgtype=Article\&impression_id=4b7309c2-efb9-11ea-b221-c1ac8458c493\&variant=undefined}{Vaccine
  Tracker}
\item
  \href{https://www.nytimes3xbfgragh.onion/2020/09/02/your-money/eviction-moratorium-covid.html?name=styln-coronavirus-national\&region=TOP_BANNER\&block=storyline_menu_recirc\&action=click\&pgtype=Article\&impression_id=4b7309c3-efb9-11ea-b221-c1ac8458c493\&variant=undefined}{Eviction
  Moratorium}
\item
  \href{https://www.nytimes3xbfgragh.onion/interactive/2020/09/02/magazine/food-insecurity-hunger-us.html?name=styln-coronavirus-national\&region=TOP_BANNER\&block=storyline_menu_recirc\&action=click\&pgtype=Article\&impression_id=4b7309c4-efb9-11ea-b221-c1ac8458c493\&variant=undefined}{American
  Hunger}
\end{itemize}

Advertisement

\protect\hyperlink{after-top}{Continue reading the main story}

Supported by

\protect\hyperlink{after-sponsor}{Continue reading the main story}

\hypertarget{deciding-how-much-distance-you-should-keep}{%
\section{Deciding How Much Distance You Should
Keep}\label{deciding-how-much-distance-you-should-keep}}

The concepts of social distancing and self-quarantine are being
interpreted in a variety of ways, not always correctly. Here's a guide
to help you make the right decisions.

\includegraphics{https://static01.graylady3jvrrxbe.onion/images/2020/03/24/well/well-playground1/merlin_170591526_0f5c5542-5af6-4fe6-8bb6-fdcfdac5de14-articleLarge.jpg?quality=75\&auto=webp\&disable=upscale}

\href{https://www.nytimes3xbfgragh.onion/by/tara-parker-pope}{\includegraphics{https://static01.graylady3jvrrxbe.onion/images/2018/07/10/us/Parker-Pope-Tara/Parker-Pope-Tara-thumbLarge.png}}

By \href{https://www.nytimes3xbfgragh.onion/by/tara-parker-pope}{Tara
Parker-Pope}

\begin{itemize}
\item
  March 19, 2020
\item
  \begin{itemize}
  \item
  \item
  \item
  \item
  \item
  \item
  \end{itemize}
\end{itemize}

When Dr. Asaf Bitton looked out from his window in Boston recently, he
was shocked by the scene. Although schools, offices and businesses
already had shut down to slow the spread of coronavirus, the park was
packed.

``I saw people from my window outside playing in the park together, and
I thought, `This is crazy,''' said Dr. Bitton,
\href{https://www.ariadnelabs.org/about-us/people/management/asaf-bitton/}{executive
director of Ariadne Labs} at Brigham and Women's Hospital and the
Harvard T.H. Chan School of Public Health. ``Why did we close the
schools if we're going to shift social contact from the schools to the
playground?''

With a sense of urgency, he sent his wife an email to share with friends
and post on Facebook. It quickly went viral and was later
\href{https://www.ariadnelabs.org/resources/articles/news/social-distancing-this-is-not-a-snow-day/}{published
on Medium}, under the headline: ``Social Distancing: This Is Not a Snow
Day.'' To slow the coronavirus, wrote Dr. Bitton, we must act quickly
and start ``making daily choices to stay away from each other as much as
possible.''

But how to make those choices has not always been clear. Concepts like
``social distancing,'' ``self-quarantine'' and ``isolation'' come from
the lexicon of the infectious disease community (and scary movies). Here
in real life, parents, workers and even government leaders are
struggling to make sense of it all. Here's a guide to help you make good
decisions, based on advice from infectious disease and public health
experts.

\hypertarget{social-distancing}{%
\subsection{\texorpdfstring{\emph{Social
Distancing}}{Social Distancing}}\label{social-distancing}}

Social distancing is ultimately about creating physical distance between
people who don't live together. At the community level, it means closing
schools and workplaces and canceling events like concerts and Broadway
shows. For individuals, it means keeping six feet of distance between
you and others while in public and avoiding physical contact with people
who do not share your home.

But one aspect of social distancing --- the admonition to avoid
gatherings of 10 people or more --- has created a lot of confusion. It
has given the impression that while public indoor events are bad, it's
OK to host up to nine people at your home or outside. That is not
correct. Right now \emph{everyone} should limit close contact, indoors
and outdoors, to family members only. This means no dinner parties, no
play dates, no birthday parties with a few friends.

\emph{\textbf{Who should do this?}} Everyone.

\hypertarget{shelter-in-place}{%
\subsection{\texorpdfstring{\emph{Shelter in
Place}}{Shelter in Place}}\label{shelter-in-place}}

In a nutshell, this means stay home. Don't leave the house unless you
absolutely have to. Don't socialize with people outside your family.
Don't go to a friend's house for dinner or invite a trusted friend over.

\hypertarget{latest-updates-the-coronavirus-outbreak}{%
\section{\texorpdfstring{\href{https://www.nytimes3xbfgragh.onion/2020/09/04/world/covid-19-coronavirus.html?action=click\&pgtype=Article\&state=default\&region=MAIN_CONTENT_1\&context=storylines_live_updates}{Latest
Updates: The Coronavirus
Outbreak}}{Latest Updates: The Coronavirus Outbreak}}\label{latest-updates-the-coronavirus-outbreak}}

Updated 2020-09-05T12:05:40.998Z

\begin{itemize}
\tightlist
\item
  \href{https://www.nytimes3xbfgragh.onion/2020/09/04/world/covid-19-coronavirus.html?action=click\&pgtype=Article\&state=default\&region=MAIN_CONTENT_1\&context=storylines_live_updates\#link-1654f6ad}{Research
  connects vaping to a higher chance of catching the virus --- and
  suffering its worst effects.}
\item
  \href{https://www.nytimes3xbfgragh.onion/2020/09/04/world/covid-19-coronavirus.html?action=click\&pgtype=Article\&state=default\&region=MAIN_CONTENT_1\&context=storylines_live_updates\#link-52e4198a}{Another
  college football game won't be played as planned.}
\item
  \href{https://www.nytimes3xbfgragh.onion/2020/09/04/world/covid-19-coronavirus.html?action=click\&pgtype=Article\&state=default\&region=MAIN_CONTENT_1\&context=storylines_live_updates\#link-181cef0}{Pharmaceutical
  companies plan a joint pledge on safety standards as they move
  vaccines to the marketplace.}
\end{itemize}

\href{https://www.nytimes3xbfgragh.onion/2020/09/04/world/covid-19-coronavirus.html?action=click\&pgtype=Article\&state=default\&region=MAIN_CONTENT_1\&context=storylines_live_updates}{See
more updates}

More live coverage:
\href{https://www.nytimes3xbfgragh.onion/live/2020/09/04/business/stock-market-today-coronavirus?action=click\&pgtype=Article\&state=default\&region=MAIN_CONTENT_1\&context=storylines_live_updates}{Markets}

During a shelter-in-place order, you are typically allowed to go outside
for essentials --- to pick up groceries or prescriptions --- but you
should limit those trips to no more than once a week if possible. People
with essential jobs --- public safety, medical, sanitation or grocery
worker --- can still go to work. And you can visit someone if you are
their caregiver.

There is a bright spot. In most cases, a shelter-in-place order allows
you to walk the dog or exercise outside (for brief periods) as long as
you keep a six-foot distance from others.

``Right now I'm recommending to my family and to people who are asking
that outdoor activities that are solitary or done in parallel with
someone who is far away is fine,'' said
\href{https://www.med.upenn.edu/apps/faculty/index.php/g275/p6153664}{Carolyn
C. Cannuscio}, associate professor of family medicine and community
health at the University of Pennsylvania Perelman School of Medicine.
``We're trying to avoid face to face contact, especially in close up and
confined spaces.''

\emph{\textbf{Who should do this?}} Everyone who lives in an area with a
mandatory shelter-in-place order, including some communities in Northern
California and possibly, soon, New York City. But many infectious
disease experts say that everyone else should also voluntarily shelter
in place to prevent the virus from spreading. ``People should really be
keeping to themselves,'' said Dr. Kryssie Woods, hospital epidemiologist
and \href{https://www.mountsinai.org/profiles/krystina-l-woods}{director
of infection prevention} at Mount Sinai West.

\hypertarget{self-monitoring}{%
\subsection{Self-Monitoring}\label{self-monitoring}}

This means regularly checking your temperature and watching for signs of
coronavirus infection, including fever, shortness of breath and
coughing. A person who is self-monitoring should already be staying home
and limiting interactions with others.

\emph{\textbf{Who should do this?}} **** Self-monitoring is for people
who learn they might have been exposed to the virus but had only distant
contact with the infected person. This might be someone in your orbit
--- for example, a colleague, a speaker at a conference or the parent of
your child's classmate --- but not a person with whom you had close
physical contact. Consult with your doctor to see if self-monitoring is
recommended for your specific situation.

\hypertarget{self-quarantine}{%
\subsection{Self-Quarantine}\label{self-quarantine}}

This term is used to separate and restrict the movement of someone who
is well but who recently had close contact with a person who later was
diagnosed with the virus. A person in self-quarantine should follow all
the rules of sheltering in place, except they should avoid going to
stores or interacting with the public even on a limited basis for a
14-day period. (A friend should bring you groceries.)

\href{https://www.nytimes3xbfgragh.onion/news-event/coronavirus?action=click\&pgtype=Article\&state=default\&region=MAIN_CONTENT_3\&context=storylines_faq}{}

\hypertarget{the-coronavirus-outbreak-}{%
\subsubsection{The Coronavirus Outbreak
›}\label{the-coronavirus-outbreak-}}

\hypertarget{frequently-asked-questions}{%
\paragraph{Frequently Asked
Questions}\label{frequently-asked-questions}}

Updated September 4, 2020

\begin{itemize}
\item ~
  \hypertarget{what-are-the-symptoms-of-coronavirus}{%
  \paragraph{What are the symptoms of
  coronavirus?}\label{what-are-the-symptoms-of-coronavirus}}

  \begin{itemize}
  \tightlist
  \item
    In the beginning, the coronavirus
    \href{https://www.nytimes3xbfgragh.onion/article/coronavirus-facts-history.html?action=click\&pgtype=Article\&state=default\&region=MAIN_CONTENT_3\&context=storylines_faq\#link-6817bab5}{seemed
    like it was primarily a respiratory illness}~--- many patients had
    fever and chills, were weak and tired, and coughed a lot, though
    some people don't show many symptoms at all. Those who seemed
    sickest had pneumonia or acute respiratory distress syndrome and
    received supplemental oxygen. By now, doctors have identified many
    more symptoms and syndromes. In April,
    \href{https://www.nytimes3xbfgragh.onion/2020/04/27/health/coronavirus-symptoms-cdc.html?action=click\&pgtype=Article\&state=default\&region=MAIN_CONTENT_3\&context=storylines_faq}{the
    C.D.C. added to the list of early signs}~sore throat, fever, chills
    and muscle aches. Gastrointestinal upset, such as diarrhea and
    nausea, has also been observed. Another telltale sign of infection
    may be a sudden, profound diminution of one's
    \href{https://www.nytimes3xbfgragh.onion/2020/03/22/health/coronavirus-symptoms-smell-taste.html?action=click\&pgtype=Article\&state=default\&region=MAIN_CONTENT_3\&context=storylines_faq}{sense
    of smell and taste.}~Teenagers and young adults in some cases have
    developed painful red and purple lesions on their fingers and toes
    --- nicknamed ``Covid toe'' --- but few other serious symptoms.
  \end{itemize}
\item ~
  \hypertarget{why-is-it-safer-to-spend-time-together-outside}{%
  \paragraph{Why is it safer to spend time together
  outside?}\label{why-is-it-safer-to-spend-time-together-outside}}

  \begin{itemize}
  \tightlist
  \item
    \href{https://www.nytimes3xbfgragh.onion/2020/05/15/us/coronavirus-what-to-do-outside.html?action=click\&pgtype=Article\&state=default\&region=MAIN_CONTENT_3\&context=storylines_faq}{Outdoor
    gatherings}~lower risk because wind disperses viral droplets, and
    sunlight can kill some of the virus. Open spaces prevent the virus
    from building up in concentrated amounts and being inhaled, which
    can happen when infected people exhale in a confined space for long
    stretches of time, said Dr. Julian W. Tang, a virologist at the
    University of Leicester.
  \end{itemize}
\item ~
  \hypertarget{why-does-standing-six-feet-away-from-others-help}{%
  \paragraph{Why does standing six feet away from others
  help?}\label{why-does-standing-six-feet-away-from-others-help}}

  \begin{itemize}
  \tightlist
  \item
    The coronavirus spreads primarily through droplets from your mouth
    and nose, especially when you cough or sneeze. The C.D.C., one of
    the organizations using that measure,
    \href{https://www.nytimes3xbfgragh.onion/2020/04/14/health/coronavirus-six-feet.html?action=click\&pgtype=Article\&state=default\&region=MAIN_CONTENT_3\&context=storylines_faq}{bases
    its recommendation of six feet}~on the idea that most large droplets
    that people expel when they cough or sneeze will fall to the ground
    within six feet. But six feet has never been a magic number that
    guarantees complete protection. Sneezes, for instance, can launch
    droplets a lot farther than six feet,
    \href{https://jamanetwork.com/journals/jama/fullarticle/2763852}{according
    to a recent study}. It's a rule of thumb: You should be safest
    standing six feet apart outside, especially when it's windy. But
    keep a mask on at all times, even when you think you're far enough
    apart.
  \end{itemize}
\item ~
  \hypertarget{i-have-antibodies-am-i-now-immune}{%
  \paragraph{I have antibodies. Am I now
  immune?}\label{i-have-antibodies-am-i-now-immune}}

  \begin{itemize}
  \tightlist
  \item
    As of right
    now,\href{https://www.nytimes3xbfgragh.onion/2020/07/22/health/covid-antibodies-herd-immunity.html?action=click\&pgtype=Article\&state=default\&region=MAIN_CONTENT_3\&context=storylines_faq}{~that
    seems likely, for at least several months.}~There have been
    frightening accounts of people suffering what seems to be a second
    bout of Covid-19. But experts say these patients may have a
    drawn-out course of infection, with the virus taking a slow toll
    weeks to months after initial exposure.~People infected with the
    coronavirus typically
    \href{https://www.nature.com/articles/s41586-020-2456-9}{produce}~immune
    molecules called antibodies, which are
    \href{https://www.nytimes3xbfgragh.onion/2020/05/07/health/coronavirus-antibody-prevalence.html?action=click\&pgtype=Article\&state=default\&region=MAIN_CONTENT_3\&context=storylines_faq}{protective
    proteins made in response to an
    infection}\href{https://www.nytimes3xbfgragh.onion/2020/05/07/health/coronavirus-antibody-prevalence.html?action=click\&pgtype=Article\&state=default\&region=MAIN_CONTENT_3\&context=storylines_faq}{.
    These antibodies may}~last in the body
    \href{https://www.nature.com/articles/s41591-020-0965-6}{only two to
    three months}, which may seem worrisome, but that's~perfectly normal
    after an acute infection subsides, said Dr. Michael Mina, an
    immunologist at Harvard University. It may be possible to get the
    coronavirus again, but it's highly unlikely that it would be
    possible in a short window of time from initial infection or make
    people sicker the second time.
  \end{itemize}
\item ~
  \hypertarget{what-are-my-rights-if-i-am-worried-about-going-back-to-work}{%
  \paragraph{What are my rights if I am worried about going back to
  work?}\label{what-are-my-rights-if-i-am-worried-about-going-back-to-work}}

  \begin{itemize}
  \tightlist
  \item
    Employers have to provide
    \href{https://www.osha.gov/SLTC/covid-19/standards.html}{a safe
    workplace}~with policies that protect everyone equally.
    \href{https://www.nytimes3xbfgragh.onion/article/coronavirus-money-unemployment.html?action=click\&pgtype=Article\&state=default\&region=MAIN_CONTENT_3\&context=storylines_faq}{And
    if one of your co-workers tests positive for the coronavirus, the
    C.D.C.}~has said that
    \href{https://www.cdc.gov/coronavirus/2019-ncov/community/guidance-business-response.html}{employers
    should tell their employees}~-\/- without giving you the sick
    employee's name -\/- that they may have been exposed to the virus.
  \end{itemize}
\end{itemize}

Quarantine means staying home and away from other people, including
those in your household, as much as possible, for a 14-day quarantine
period. A person in self-quarantine should sleep in a separate space
from family members.

\emph{\textbf{Who should do this?}} **** Anyone who does not have
symptoms, but who had close contact with someone who later became ill.

\hypertarget{self-isolation}{%
\subsection{Self-Isolation}\label{self-isolation}}

Isolation is used to separate a person who has a diagnosed case or
someone who has distinct symptoms including a cough, fever and shortness
of breath, but hasn't yet been tested or received test results. A person
in isolation should be confined to a separate room with no or minimal
contact with the rest of the household (including pets) and use a
separate bathroom if possible. Most of the time, a sick person will feel
a bit miserable, but he or she can pick up food trays left at the door
and sanitize a shared bathroom after using it.

\emph{\textbf{Who should do this?}} Anyone with a confirmed case of
Covid-19, in consultation with their doctor, a person waiting for test
results or a person with obvious symptoms who is still waiting to be
tested. Everyone else in the household should self-quarantine.

\hypertarget{official-or-mandatory-quarantine}{%
\subsection{Official or Mandatory
Quarantine}\label{official-or-mandatory-quarantine}}

A government-imposed lockdown on a community, as has happened in Italy,
in which movements are severely restricted. People can still go out for
essentials and to get fresh air, but they can do so only under strictly
controlled conditions or on a specific schedule imposed by public safety
officials.

\emph{\textbf{Who should do this?}} Everyone who lives in an area under
quarantine. ``We haven't seen this in the U.S.,'' Dr. Bitton said. ``I
don't know if it's coming.''

\begin{center}\rule{0.5\linewidth}{\linethickness}\end{center}

\textbf{\href{https://www.nytimes3xbfgragh.onion/newsletters/well}{\emph{Sign
up for the Well newsletter}}} \emph{\textbf{to get more stories like
this delivered to your inbox.}}

Advertisement

\protect\hyperlink{after-bottom}{Continue reading the main story}

\hypertarget{site-index}{%
\subsection{Site Index}\label{site-index}}

\hypertarget{site-information-navigation}{%
\subsection{Site Information
Navigation}\label{site-information-navigation}}

\begin{itemize}
\tightlist
\item
  \href{https://help.nytimes3xbfgragh.onion/hc/en-us/articles/115014792127-Copyright-notice}{©~2020~The
  New York Times Company}
\end{itemize}

\begin{itemize}
\tightlist
\item
  \href{https://www.nytco.com/}{NYTCo}
\item
  \href{https://help.nytimes3xbfgragh.onion/hc/en-us/articles/115015385887-Contact-Us}{Contact
  Us}
\item
  \href{https://www.nytco.com/careers/}{Work with us}
\item
  \href{https://nytmediakit.com/}{Advertise}
\item
  \href{http://www.tbrandstudio.com/}{T Brand Studio}
\item
  \href{https://www.nytimes3xbfgragh.onion/privacy/cookie-policy\#how-do-i-manage-trackers}{Your
  Ad Choices}
\item
  \href{https://www.nytimes3xbfgragh.onion/privacy}{Privacy}
\item
  \href{https://help.nytimes3xbfgragh.onion/hc/en-us/articles/115014893428-Terms-of-service}{Terms
  of Service}
\item
  \href{https://help.nytimes3xbfgragh.onion/hc/en-us/articles/115014893968-Terms-of-sale}{Terms
  of Sale}
\item
  \href{https://spiderbites.nytimes3xbfgragh.onion}{Site Map}
\item
  \href{https://help.nytimes3xbfgragh.onion/hc/en-us}{Help}
\item
  \href{https://www.nytimes3xbfgragh.onion/subscription?campaignId=37WXW}{Subscriptions}
\end{itemize}
