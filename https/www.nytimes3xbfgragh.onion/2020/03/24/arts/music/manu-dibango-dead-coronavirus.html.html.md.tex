Sections

SEARCH

\protect\hyperlink{site-content}{Skip to
content}\protect\hyperlink{site-index}{Skip to site index}

\href{https://www.nytimes3xbfgragh.onion/section/arts/music}{Music}

\href{https://myaccount.nytimes3xbfgragh.onion/auth/login?response_type=cookie\&client_id=vi}{}

\href{https://www.nytimes3xbfgragh.onion/section/todayspaper}{Today's
Paper}

\href{/section/arts/music}{Music}\textbar{}Manu Dibango, Soulful
Ambassador of African Music, Dies at 86

\url{https://nyti.ms/2WTP1gE}

\begin{itemize}
\item
\item
\item
\item
\item
\end{itemize}

\href{https://www.nytimes3xbfgragh.onion/news-event/coronavirus?action=click\&pgtype=Article\&state=default\&region=TOP_BANNER\&context=storylines_menu}{The
Coronavirus Outbreak}

\begin{itemize}
\tightlist
\item
  live\href{https://www.nytimes3xbfgragh.onion/2020/08/04/world/coronavirus-covid-19.html?action=click\&pgtype=Article\&state=default\&region=TOP_BANNER\&context=storylines_menu}{Latest
  Updates}
\item
  \href{https://www.nytimes3xbfgragh.onion/interactive/2020/us/coronavirus-us-cases.html?action=click\&pgtype=Article\&state=default\&region=TOP_BANNER\&context=storylines_menu}{Maps
  and Cases}
\item
  \href{https://www.nytimes3xbfgragh.onion/interactive/2020/science/coronavirus-vaccine-tracker.html?action=click\&pgtype=Article\&state=default\&region=TOP_BANNER\&context=storylines_menu}{Vaccine
  Tracker}
\item
  \href{https://www.nytimes3xbfgragh.onion/2020/08/02/us/covid-college-reopening.html?action=click\&pgtype=Article\&state=default\&region=TOP_BANNER\&context=storylines_menu}{College
  Reopening}
\item
  \href{https://www.nytimes3xbfgragh.onion/live/2020/08/03/business/stock-market-today-coronavirus?action=click\&pgtype=Article\&state=default\&region=TOP_BANNER\&context=storylines_menu}{Economy}
\end{itemize}

Advertisement

\protect\hyperlink{after-top}{Continue reading the main story}

Supported by

\protect\hyperlink{after-sponsor}{Continue reading the main story}

those we've lost

\hypertarget{manu-dibango-soulful-ambassador-of-african-music-dies-at-86}{%
\section{Manu Dibango, Soulful Ambassador of African Music, Dies at
86}\label{manu-dibango-soulful-ambassador-of-african-music-dies-at-86}}

His 1972 hit ``Soul Makossa'' arrived at the dawn of the disco era and
made its way to dance floors across the United States, Europe and
Africa.

\includegraphics{https://static01.graylady3jvrrxbe.onion/images/2020/03/25/obituaries/24Dibango1/merlin_158927346_b4ed5e74-77a3-4e48-b08e-5605a2637070-articleLarge.jpg?quality=75\&auto=webp\&disable=upscale}

\href{https://www.nytimes3xbfgragh.onion/by/jon-pareles}{\includegraphics{https://static01.graylady3jvrrxbe.onion/images/2018/06/14/multimedia/author-jon-pareles/author-jon-pareles-thumbLarge.png}}

By \href{https://www.nytimes3xbfgragh.onion/by/jon-pareles}{Jon Pareles}

\begin{itemize}
\item
  Published March 24, 2020Updated April 16, 2020
\item
  \begin{itemize}
  \item
  \item
  \item
  \item
  \item
  \end{itemize}
\end{itemize}

\emph{This obituary is part of a series about people who have died in
the coronavirus pandemic. Read about others}
\href{https://www.nytimes3xbfgragh.onion/series/people-who-have-died-of-the-coronavirus}{\emph{here}}\emph{.}

Manu Dibango, a saxophonist from Cameroon whose 1972 single
\href{https://www.youtube.com/watch?v=EF92yOsv3Y8}{``Soul Makossa''}
made modern African music a clear presence on Western pop charts, died
on Tuesday in a hospital in France. He was 86.

His
\href{https://www.facebookcorewwwi.onion/ManuDibangoOfficiel}{Facebook
page} said the cause was Covid-19 but did not say where in France he
died. Mr. Dibango had lived in France for some time.

Although ``Soul Makossa'' was named after makossa, a Cameroonian style
of music, and its lyrics were in the Douala language of Cameroon, Mr.
Dibango's worldwide hit was an internationalist piece of funk.

With his terse, dryly insistent saxophone lines answering his own
chanted vocals, a tricky stop-start beat and a scrubbing rhythm guitar,
``Soul Makossa'' arrived at the dawn of the disco era and made its way
to dance floors and R\&B radio stations across the United States, Europe
and Africa.

Michael Jackson would later quote its refrain in ``Wanna Be Startin'
Somethin','' a track on his 1983 album, ``Thriller,'' one of the
best-selling records of all time; that song was in turn sampled by
Rihanna for her 2007 ``Don't Stop the Music.'' Mr. Dibango sued them
both in 2009; Mr. Jackson's estate settled out of court. The song has
also been widely sampled in hip-hop.

Although Mr. Dibango was best known for ``Soul Makossa'' and a 1984 hit,
\href{https://www.youtube.com/watch?v=Ph5GB8RArks}{``Abele Dance,''}
there was much more to his career. He recorded and toured prolifically,
appearing worldwide and collaborating with musicians including Herbie
Hancock, Fela Kuti, Peter Gabriel, Angélique Kidjo, Youssou N'Dour, the
Fania All-Stars and Sinead O'Connor. In a 2017
\href{https://www.bbc.com/news/world-europe-52017834}{interview with the
BBC}, Mr. Dibango took pride in the eclecticism of his music.

``You're not a musician because you're African,'' he said. ``You're a
musician because you are musician. Coming from Africa, but first,
musician.''

Emmanuel N'Djoké Dibango was born on Dec. 12, 1933, in Douala, the
largest city in Cameroon. His father was a civil servant; his mother was
a dressmaker. He grew up listening to Protestant church music, local
traditional music and Westernized pop.

At 15 he was sent to Europe to study classical piano and music theory in
Paris and Brussels. But he was drawn to jazz, and he began playing
saxophone in the early l950s.

\includegraphics{https://static01.graylady3jvrrxbe.onion/images/2020/03/24/obituaries/24DIBANGO2/merlin_168800772_b3027088-2b9f-4b31-9cc5-99951109f541-articleLarge.jpg?quality=75\&auto=webp\&disable=upscale}

When he started performing in cabarets and jazz clubs in 1956, his
family cut off his allowance. In Belgium, he began working with
musicians from the Belgian Congo (which would be renamed Zaire after
gaining independence in 1960 and then the Democratic Republic of Congo).
He worked with African Jazz, the group led by Le Grand Kalle (Joseph
Kabasele), in Leopoldville (later renamed Kinshasa) in the early 1960s
before returning to France. By the late 1960s he was leading his own
band in Paris.

``Soul Makossa'' was originally the B-side of a single celebrating
Cameroon's national soccer team. According to ``Turn the Beat Around:
The Secret History of Disco'' (2005), by Peter Shapiro, the New York
City disc jockey David Mancuso found a copy in a West Indian record
store and
\href{https://www.nytimes3xbfgragh.onion/2020/02/13/arts/music/love-saves-the-day-loft-playlist.html}{played
it at the Loft}, a pioneering disco space, and the influential radio
host Frankie Crocker put the song in heavy rotation on WBLS. Soon there
were more than a dozen cover versions, as the imported original disc
sold out. Atlantic Records licensed Mr. Dibango's original, which
reached the American pop Top 40 in 1973.

The song opened a worldwide touring and recording circuit for Mr.
Dibango. He collaborated widely: with the reggae producers Sly Dunbar
and Robbie Shakespeare in Jamaica, with Serge Gainsbourg in Paris, with
the bassist and producer Bill Laswell in the group Deadline in the
United States. In 1992 he recorded ``Wakafrica,'' an album of African
hits with guest appearances by King Sunny Ade, Ladysmith Black Mambazo,
Salif Keita, Papa Wemba, N'dour, Ms. Kidjo and others.

Information on survivors was not immediately available.

Mr. Dibango's extensive catalog includes film scores, jazz standards,
reggae, pop and hip-hop. In 2017 he released ``M \& M,'' a collaboration
with a jazz saxophonist from Mozambique, Moreira Chonquiça, and in 2018
he released ``Cubafrica,'' a collaboration with the Cuban group Cuarteto
Patria. Many of his other albums fused
\href{https://www.youtube.com/watch?\&v=zK69LoCTYus}{jazz, funk, African
instruments, and dance beats} --- electronic or hand-played --- behind
his terse melodic lines.

``Sound is a magma. You have to give it a form. It's never the same,''
Mr. Dibango said in a 1991 interview with
\href{https://en.unesco.org/courier/march-1991/interview-manu-dibango}{UNESCO
Courier magazine}. ``In music there is neither past nor future, only the
present. I must compose the music of my time, not yesterday's music.''

\href{https://www.nytimes3xbfgragh.onion/interactive/2020/obituaries/people-died-coronavirus-obituaries.html?action=click\&pgtype=Article\&state=default\&region=BELOW_MAIN_CONTENT\&context=covid_obits_promo}{}

\hypertarget{those-weve-lost}{%
\section{Those We've Lost}\label{those-weve-lost}}

The coronavirus pandemic has taken an incalculable death toll. This
series is designed to put names and faces to the numbers.

Read more

\includegraphics{https://static01.graylady3jvrrxbe.onion/images/2020/07/30/obituaries/30Pedro/30Pedro-square640.jpg}

\hypertarget{bernaldina-josuxe9-pedro}{%
\section{Bernaldina José Pedro}\label{bernaldina-josuxe9-pedro}}

d. Boa Vista, Brazil

Leader among the Indigenous Macuxi

\includegraphics{https://static01.graylady3jvrrxbe.onion/images/2020/07/31/obituaries/31Swing/merlin_175167783_8913bc90-0d64-43f3-a655-1bb1bf1601c9-square640.jpg}

\hypertarget{john-eric-swing}{%
\section{John Eric Swing}\label{john-eric-swing}}

d. Fountain Valley, Calif.

Champion of Filipino-Americans

\includegraphics{https://static01.graylady3jvrrxbe.onion/images/2020/07/27/obituaries/27Victor/merlin_175001436_38b11f8e-227a-4e2c-9821-7618af9b2524-square640.jpg}

\hypertarget{victor-victor}{%
\section{Victor Victor}\label{victor-victor}}

d. Santo Domingo, Dominican Republic

Beloved musician of the Dominican Republic

\includegraphics{https://static01.graylady3jvrrxbe.onion/images/2020/07/31/obituaries/31Negron/merlin_175160169_516322ae-fd23-4969-b6b2-193ced371105-square640.jpg}

\hypertarget{dr-eddie-negruxf3n}{%
\section{Dr. Eddie Negrón}\label{dr-eddie-negruxf3n}}

d. Fort Walton Beach, Fla.

Internist on Florida's Emerald Coast

\includegraphics{https://static01.graylady3jvrrxbe.onion/images/2020/07/30/obituaries/30Dobson/merlin_175115928_f6b9271c-8f05-4fe1-a38a-5ca4a58f8935-square640.jpg}

\hypertarget{dobby-dobson}{%
\section{Dobby Dobson}\label{dobby-dobson}}

d. Coral Springs, Fla.

Jamaican singer and songwriter

\includegraphics{https://static01.graylady3jvrrxbe.onion/images/2020/08/01/obituaries/28Gonzalez/merlin_175002771_beb57888-3951-409a-ae13-03a94b2e962e-square640.jpg}

\hypertarget{waldemar-gonzalez}{%
\section{Waldemar Gonzalez}\label{waldemar-gonzalez}}

d. White Plains, N.Y.

Teacher and social worker

Advertisement

\protect\hyperlink{after-bottom}{Continue reading the main story}

\hypertarget{site-index}{%
\subsection{Site Index}\label{site-index}}

\hypertarget{site-information-navigation}{%
\subsection{Site Information
Navigation}\label{site-information-navigation}}

\begin{itemize}
\tightlist
\item
  \href{https://help.nytimes3xbfgragh.onion/hc/en-us/articles/115014792127-Copyright-notice}{©~2020~The
  New York Times Company}
\end{itemize}

\begin{itemize}
\tightlist
\item
  \href{https://www.nytco.com/}{NYTCo}
\item
  \href{https://help.nytimes3xbfgragh.onion/hc/en-us/articles/115015385887-Contact-Us}{Contact
  Us}
\item
  \href{https://www.nytco.com/careers/}{Work with us}
\item
  \href{https://nytmediakit.com/}{Advertise}
\item
  \href{http://www.tbrandstudio.com/}{T Brand Studio}
\item
  \href{https://www.nytimes3xbfgragh.onion/privacy/cookie-policy\#how-do-i-manage-trackers}{Your
  Ad Choices}
\item
  \href{https://www.nytimes3xbfgragh.onion/privacy}{Privacy}
\item
  \href{https://help.nytimes3xbfgragh.onion/hc/en-us/articles/115014893428-Terms-of-service}{Terms
  of Service}
\item
  \href{https://help.nytimes3xbfgragh.onion/hc/en-us/articles/115014893968-Terms-of-sale}{Terms
  of Sale}
\item
  \href{https://spiderbites.nytimes3xbfgragh.onion}{Site Map}
\item
  \href{https://help.nytimes3xbfgragh.onion/hc/en-us}{Help}
\item
  \href{https://www.nytimes3xbfgragh.onion/subscription?campaignId=37WXW}{Subscriptions}
\end{itemize}
