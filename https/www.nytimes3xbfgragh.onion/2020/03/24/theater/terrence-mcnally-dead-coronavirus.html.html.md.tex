Sections

SEARCH

\protect\hyperlink{site-content}{Skip to
content}\protect\hyperlink{site-index}{Skip to site index}

\href{https://www.nytimes3xbfgragh.onion/section/theater}{Theater}

\href{https://myaccount.nytimes3xbfgragh.onion/auth/login?response_type=cookie\&client_id=vi}{}

\href{https://www.nytimes3xbfgragh.onion/section/todayspaper}{Today's
Paper}

\href{/section/theater}{Theater}\textbar{}Terrence McNally, Tony-Winning
Playwright of Gay Life, Dies at 81

\url{https://nyti.ms/2WIcph4}

\begin{itemize}
\item
\item
\item
\item
\item
\end{itemize}

\href{https://www.nytimes3xbfgragh.onion/news-event/coronavirus?action=click\&pgtype=Article\&state=default\&region=TOP_BANNER\&context=storylines_menu}{The
Coronavirus Outbreak}

\begin{itemize}
\tightlist
\item
  live\href{https://www.nytimes3xbfgragh.onion/2020/08/04/world/coronavirus-covid-19.html?action=click\&pgtype=Article\&state=default\&region=TOP_BANNER\&context=storylines_menu}{Latest
  Updates}
\item
  \href{https://www.nytimes3xbfgragh.onion/interactive/2020/us/coronavirus-us-cases.html?action=click\&pgtype=Article\&state=default\&region=TOP_BANNER\&context=storylines_menu}{Maps
  and Cases}
\item
  \href{https://www.nytimes3xbfgragh.onion/interactive/2020/science/coronavirus-vaccine-tracker.html?action=click\&pgtype=Article\&state=default\&region=TOP_BANNER\&context=storylines_menu}{Vaccine
  Tracker}
\item
  \href{https://www.nytimes3xbfgragh.onion/2020/08/02/us/covid-college-reopening.html?action=click\&pgtype=Article\&state=default\&region=TOP_BANNER\&context=storylines_menu}{College
  Reopening}
\item
  \href{https://www.nytimes3xbfgragh.onion/live/2020/08/03/business/stock-market-today-coronavirus?action=click\&pgtype=Article\&state=default\&region=TOP_BANNER\&context=storylines_menu}{Economy}
\end{itemize}

Advertisement

\protect\hyperlink{after-top}{Continue reading the main story}

Supported by

\protect\hyperlink{after-sponsor}{Continue reading the main story}

those we've lost

\hypertarget{terrence-mcnally-tony-winning-playwright-of-gay-life-dies-at-81}{%
\section{Terrence McNally, Tony-Winning Playwright of Gay Life, Dies at
81}\label{terrence-mcnally-tony-winning-playwright-of-gay-life-dies-at-81}}

Mr. McNally, who died of coronavirus complications, introduced audiences
to characters and situations that most mainstream theater had previously
shunted into comic asides.

\includegraphics{https://static01.graylady3jvrrxbe.onion/images/2020/03/27/obituaries/24mcnally/merlin_79098622_37b71f4f-a359-46f5-b5be-315fc94220b2-articleLarge.jpg?quality=75\&auto=webp\&disable=upscale}

\href{https://www.nytimes3xbfgragh.onion/by/jesse-green}{\includegraphics{https://static01.graylady3jvrrxbe.onion/images/2018/02/16/multimedia/author-jesse-green/author-jesse-green-thumbLarge.jpg}}\href{https://www.nytimes3xbfgragh.onion/by/neil-genzlinger}{\includegraphics{https://static01.graylady3jvrrxbe.onion/images/2018/06/13/multimedia/author-neil-genzlinger/author-neil-genzlinger-thumbLarge.jpg}}

By \href{https://www.nytimes3xbfgragh.onion/by/jesse-green}{Jesse Green}
and \href{https://www.nytimes3xbfgragh.onion/by/neil-genzlinger}{Neil
Genzlinger}

\begin{itemize}
\item
  Published March 24, 2020Updated April 16, 2020
\item
  \begin{itemize}
  \item
  \item
  \item
  \item
  \item
  \end{itemize}
\end{itemize}

\emph{This obituary is part of a series about people who have died in
the coronavirus pandemic. Read about others}
\href{https://www.nytimes3xbfgragh.onion/series/people-who-have-died-of-the-coronavirus}{\emph{here}}\emph{.}

Terrence McNally, the four-time Tony Award-winning playwright whose
outpouring of work for the theater dramatized and domesticated gay life
across five decades, died on Tuesday in Sarasota, Fla. He was 81.

The cause was complications of the coronavirus, according to his
husband,
\href{https://www.nytimes3xbfgragh.onion/2017/11/15/fashion/weddings/tom-kirdahy-and-terrence-mcnally-an-immediate-and-lasting-need.html}{Tom
Kirdahy}. Mr. McNally had chronic obstructive pulmonary disease, and had
overcome lung cancer. He died at Sarasota Memorial Hospital.

Mr. McNally's Tony Awards attest to his versatility. Two were for books
for musicals, ``Kiss of the Spider Woman'' (1993) and ``Ragtime''
(1998), and two were for plays, and vastly different ones: ``Love!
Valour! Compassion!'' (1995), about gay men who share a vacation house,
and ``Master Class'' (1996), in which the opera diva Maria Callas
reflects on her career.

And those prize winners were only a small part of his oeuvre. With some
three dozen plays to his credit, as well as the books for 10 musicals,
the librettos for four operas and a handful of screenplays for film and
television, Mr. McNally was a remarkably prolific and consistent
dramatist.

His career, which began on Broadway in 1963 with a few lines he
contributed to an adaptation of ``The Lady of the Camellias'' starring
Susan Strasberg, continued without much interruption through last year's
revival of his
``\href{https://www.nytimes3xbfgragh.onion/2019/05/30/theater/frankie-and-johnny-review-audra-mcdonald.html}{Frankie
and Johnny in the Clair de Lune},'' starring Audra McDonald and Michael
Shannon.

In between, in a series of successes including ``The Ritz,'' ``The
Lisbon Traviata,'' ``Lips Together, Teeth Apart'' and ``Love! Valour!
Compassion!,'' Mr. McNally introduced Broadway and Off Broadway
audiences to characters and situations that most mainstream theater had
previously shunted into comic asides.

\includegraphics{https://static01.graylady3jvrrxbe.onion/images/2020/03/25/obituaries/24McNally-valour/merlin_79379944_e644d9c9-f7c4-4c6d-8506-70cc8805ec31-articleLarge.jpg?quality=75\&auto=webp\&disable=upscale}

His first Broadway production, a 1965 bomb called ``And Things That Go
Bump in the Night,'' featured what was, at the time, an almost
unheard-of romance between two men.

Walter Kerr, writing in The New York Herald Tribune, called it an
``infertile cross between Sartre's `No Exit,' Albee's `Tiny Alice,'
Wagner's `Gotterdammerung' and the most portentous high school pageant
you ever saw.''

Mr. McNally told Vogue in 1995, ``I still think that I win, hands down,
the contest for worst first-play reviews --- or any-play reviews.''

\includegraphics{https://static01.graylady3jvrrxbe.onion/images/2019/04/10/t-magazine/entertainment/10tmag-stills-slide-9EC0/10tmag-stills-slide-9EC0-videoSixteenByNineJumbo1600.jpg}

Such reviews did not slow him down, however. Over the next 50 years, his
flagship plays, as well as the teleplay for ``Andre's Mother'' in 1990
and its stage sequel ``Mothers and Sons'' in 2014, traced the same arc
that many gay men were experiencing in their lives over the same period,
from the closet to rebellion, and from disaster to marriage and
parenting.

``The Ritz,'' which opened in 1975, offered Broadway audiences a
farcical snapshot of life in a Manhattan bathhouse, with Mafiosi and
half-naked gay men slamming and opening doors in a replay of
closetedness and liberation. By 1989, when ``The Lisbon Traviata''
opened at the Manhattan Theater Club, that joy had soured considerably,
as the relationship between an obsessive gay opera fanatic and his
younger doctor lover explodes to a Callas recording.

If that play was fueled by unspoken angst about AIDS, the disease moved
to the forefront in ``Lips Together, Teeth Apart'' (1991) and ``Love!
Valour! Compassion!,'' both of which concern groups of friends ---
straight couples in the first, gay men in the second --- confronting
unexpected mortality. In ``Andre's Mother,'' the two groups intersect
when the mother of a man who has died of AIDS cannot bring herself to
acknowledge the grief of his longtime lover.

Image

Mr. McNally, right, at a rehearsal for ``Lips Together, Teeth Apart'' in
1991 with, from left, Nathan Lane, Swoosie Kurtz, Christine Baranski and
Anthony Heald.Credit...Jack Manning/The New York Times

Fourteen years later, in ``Mothers and Sons,'' that lover has remarried
and, with his husband, become the father of a young son --- the first
time this new family reality was depicted on Broadway.

Though the changes Mr. McNally wrote about were epochal for gay men, his
plays were designed not to exclude. However furious, they are also
ingratiating, emphasizing familiar situations, comic personalities and
well-turned put-downs. (``Who are you saving it for?'' Callas bellows at
an unfortunate singer in midsong.) His gay stories never came across as
a narrowing of theater's human focus but as an expansion of it, and by
inviting everyone into them he helped solidify the social change he was
describing.

This was naturally most evident in his books for musicals, several of
which had gay characters and themes and nearly all of which focused on
society's outsiders trying to get in. ``Kiss of the Spider Woman,''
based on the novel by Manuel Puig, daringly puts onstage, in one prison
cell, a macho political prisoner in an unnamed South American
dictatorship and a window-dresser whose greatest passion is for campy
movie musicals. Chita Rivera starred as Aurora, the Spider Woman, a
figure of death and transcendence brought to life by an act of willful
imagination.

And among the out-of-work steelworkers who put together a striptease act
to raise money and morale in ``The Full Monty,'' which opened on
Broadway in 2000, Mr. McNally made sure to expand **** on a gay
relationship only briefly indicated in the British film on which the
musical was based.

His other musicals, which include ``The Rink'' (1984), ``Catch Me if You
Can'' (2011) and ``Anastasia'' (2016), likewise depict Mr. McNally's
gift for sociological specificity within historical sweep, even if most
of them are adaptations. Perhaps that's because, in replicating the arc
of gay life from the 1960s to the 2010s, his plays also replicated the
arc of his own life.

Image

Brian Stokes Mitchell, center, starred on Broadway in 1998 in the
musical ``Ragtime.'' Mr. McNally won a Tony Award for writing the book,
based on the E.L. Doctorow novel.Credit...Sara Krulwich/The New York
Times

Michael Terrence McNally was born on Nov. 3, 1938, in St. Petersburg,
Fla., where his parents, Hubert and Dorothy (Rapp) McNally, had a bar
and grill on the beach. During World War II and just after, the family
lived in Port Chester, N.Y., and his paternal grandfather would take him
to the theater.

One of the first productions he saw was the musical ``Annie Get Your
Gun,'' which opened on Broadway in 1946 and starred Ethel Merman. The
show, he said in an oral history recorded in 2017 for the
\href{http://primarystagesoffcenter.org/interviews/k-o/terrence-mcnally}{Primary
Stages Off Broadway Oral History Project}, ``made a huge impression, so
much so that when she did it on Broadway as a revival when I was a
student at Columbia many years later, I anticipated every moment: `Now
she's going to do this; now that's going to happen.' Suddenly it all
came rushing back.''

The family moved to Corpus Christi, Texas. In high school there Mr.
McNally edited the school newspaper and literary magazine.

``I had a wonderful high school English teacher, Mrs. McElroy,'' he said
in the oral history, ``who loved theater, made me and a few others
really appreciate the English language and the use of it, and she really
got us into Shakespeare.''

He graduated from W.B. Ray High School in Corpus Christi in 1956 and
enrolled at Columbia University. It was a particularly vibrant time for
Broadway. In the oral history, Mr. McNally recalled heading out on his
first night in the city naïvely expecting to walk up to the box office
and buy a ticket to ``My Fair Lady,'' a smash hit that had recently
opened starring Julie Andrews and Rex Harrison. Told that the show was
sold out for months, he walked a few blocks south and saw Gwen Verdon in
``Damn Yankees'' instead.

He earned his bachelor's degree in English in 1960. By then he was in a
relationship with the playwright Edward Albee, whom he had met at a
party in 1959.

``Terrence and I started talking,'' Mr. Albee recalled in an interview
quoted in Mel Gussow's biography, ``Edward Albee: A Singular Journey''
(2012), ``and the next thing I knew, so to speak, we were living
together.''

The relationship lasted five years, but their differing views on how to
deal with their sexuality were a point of tension.

``Edward didn't want to be reviewed as a gay playwright and was never
comfortable coming out,'' Mr. McNally told The San Francisco Chronicle
in 2018. ``That's one of about a million reasons why that relationship
was never going to go anywhere. I became invisible when press was around
or at an opening night. I knew it was wrong. It's so much work to live
that way.''
\href{https://www.nytimes3xbfgragh.onion/2016/09/17/arts/edward-albee-playwright-of-a-desperate-generation-dies-at-88.html}{(Mr.
Albee died in 2016.)}

After the poorly received ``And Things That Go Bump in the Night,'' Mr.
McNally had two other Broadway credits, ``Morning, Noon and Night'' in
1968 and two one-acts under the title ``Bad Habits'' in 1974, before
scoring a success with ``The Ritz,'' which ran for almost a year. His
career culminated in a Tony Award for lifetime achievement in 2019.

Image

Mr. McNally at his home in Manhattan with his husband, Tom Kirdahy, in
2018.Credit...Daniel Dorsa for The New York Times

Mr. McNally and Mr. Kirdahy were joined in a civil union in 2003 and
\href{https://www.nytimes3xbfgragh.onion/2017/11/15/fashion/weddings/tom-kirdahy-and-terrence-mcnally-an-immediate-and-lasting-need.html}{married
in 2010}. He is also survived by a brother, Peter.

In
\href{https://www.nytimes3xbfgragh.onion/2014/03/02/theater/mcnallys-mothers-and-sons-gauges-a-changed-america.html?searchResultPosition=3}{a
2014 interview} with The New York Times, Mr. McNally recalled an
encounter at Stephen Sondheim's 50th-birthday party in 1980 that helped
him shed a personal demon, a turning point in his playwriting. He was
drinking heavily at the time and had been for years.

``Then someone I hardly knew, Angela Lansbury, waved me over to where
she was sitting,'' he said. ``And she said, `I just want to say, I don't
know you very well, but every time I see you, you're drunk, and it
bothers me.' I was so upset. She was someone I revered, and she said
this with such love and concern. I went to an A.A. meeting, and within a
year, I had stopped drinking.''

By 1982, with ``Frankie and Johnny,'' the course of his career had
changed, his vision having deepened and darkened from the zaniness and
absurdity of his earlier work. The play about melancholy lovers ---
Frankie is ``a B.L.T. down sort of person,'' who thinks Johnny is
``looking for someone a little more pheasant under glass'' ---
introduces what would become Mr. McNally's mature theme: that tragedy
and comedy not only coexist but also, like all of us on earth, cohabit.

\href{https://www.nytimes3xbfgragh.onion/interactive/2020/obituaries/people-died-coronavirus-obituaries.html?action=click\&pgtype=Article\&state=default\&region=BELOW_MAIN_CONTENT\&context=covid_obits_promo}{}

\hypertarget{those-weve-lost}{%
\section{Those We've Lost}\label{those-weve-lost}}

The coronavirus pandemic has taken an incalculable death toll. This
series is designed to put names and faces to the numbers.

Read more

\includegraphics{https://static01.graylady3jvrrxbe.onion/images/2020/07/30/obituaries/30Pedro/30Pedro-square640.jpg}

\hypertarget{bernaldina-josuxe9-pedro}{%
\section{Bernaldina José Pedro}\label{bernaldina-josuxe9-pedro}}

d. Boa Vista, Brazil

Leader among the Indigenous Macuxi

\includegraphics{https://static01.graylady3jvrrxbe.onion/images/2020/07/31/obituaries/31Swing/merlin_175167783_8913bc90-0d64-43f3-a655-1bb1bf1601c9-square640.jpg}

\hypertarget{john-eric-swing}{%
\section{John Eric Swing}\label{john-eric-swing}}

d. Fountain Valley, Calif.

Champion of Filipino-Americans

\includegraphics{https://static01.graylady3jvrrxbe.onion/images/2020/07/27/obituaries/27Victor/merlin_175001436_38b11f8e-227a-4e2c-9821-7618af9b2524-square640.jpg}

\hypertarget{victor-victor}{%
\section{Victor Victor}\label{victor-victor}}

d. Santo Domingo, Dominican Republic

Beloved musician of the Dominican Republic

\includegraphics{https://static01.graylady3jvrrxbe.onion/images/2020/07/31/obituaries/31Negron/merlin_175160169_516322ae-fd23-4969-b6b2-193ced371105-square640.jpg}

\hypertarget{dr-eddie-negruxf3n}{%
\section{Dr. Eddie Negrón}\label{dr-eddie-negruxf3n}}

d. Fort Walton Beach, Fla.

Internist on Florida's Emerald Coast

\includegraphics{https://static01.graylady3jvrrxbe.onion/images/2020/07/30/obituaries/30Dobson/merlin_175115928_f6b9271c-8f05-4fe1-a38a-5ca4a58f8935-square640.jpg}

\hypertarget{dobby-dobson}{%
\section{Dobby Dobson}\label{dobby-dobson}}

d. Coral Springs, Fla.

Jamaican singer and songwriter

\includegraphics{https://static01.graylady3jvrrxbe.onion/images/2020/08/01/obituaries/28Gonzalez/merlin_175002771_beb57888-3951-409a-ae13-03a94b2e962e-square640.jpg}

\hypertarget{waldemar-gonzalez}{%
\section{Waldemar Gonzalez}\label{waldemar-gonzalez}}

d. White Plains, N.Y.

Teacher and social worker

Advertisement

\protect\hyperlink{after-bottom}{Continue reading the main story}

\hypertarget{site-index}{%
\subsection{Site Index}\label{site-index}}

\hypertarget{site-information-navigation}{%
\subsection{Site Information
Navigation}\label{site-information-navigation}}

\begin{itemize}
\tightlist
\item
  \href{https://help.nytimes3xbfgragh.onion/hc/en-us/articles/115014792127-Copyright-notice}{©~2020~The
  New York Times Company}
\end{itemize}

\begin{itemize}
\tightlist
\item
  \href{https://www.nytco.com/}{NYTCo}
\item
  \href{https://help.nytimes3xbfgragh.onion/hc/en-us/articles/115015385887-Contact-Us}{Contact
  Us}
\item
  \href{https://www.nytco.com/careers/}{Work with us}
\item
  \href{https://nytmediakit.com/}{Advertise}
\item
  \href{http://www.tbrandstudio.com/}{T Brand Studio}
\item
  \href{https://www.nytimes3xbfgragh.onion/privacy/cookie-policy\#how-do-i-manage-trackers}{Your
  Ad Choices}
\item
  \href{https://www.nytimes3xbfgragh.onion/privacy}{Privacy}
\item
  \href{https://help.nytimes3xbfgragh.onion/hc/en-us/articles/115014893428-Terms-of-service}{Terms
  of Service}
\item
  \href{https://help.nytimes3xbfgragh.onion/hc/en-us/articles/115014893968-Terms-of-sale}{Terms
  of Sale}
\item
  \href{https://spiderbites.nytimes3xbfgragh.onion}{Site Map}
\item
  \href{https://help.nytimes3xbfgragh.onion/hc/en-us}{Help}
\item
  \href{https://www.nytimes3xbfgragh.onion/subscription?campaignId=37WXW}{Subscriptions}
\end{itemize}
