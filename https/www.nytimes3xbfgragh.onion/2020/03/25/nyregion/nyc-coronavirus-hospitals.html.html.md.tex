Sections

SEARCH

\protect\hyperlink{site-content}{Skip to
content}\protect\hyperlink{site-index}{Skip to site index}

\href{https://www.nytimes3xbfgragh.onion/section/nyregion}{New York}

\href{https://myaccount.nytimes3xbfgragh.onion/auth/login?response_type=cookie\&client_id=vi}{}

\href{https://www.nytimes3xbfgragh.onion/section/todayspaper}{Today's
Paper}

\href{/section/nyregion}{New York}\textbar{}13 Deaths in a Day: An
`Apocalyptic' Coronavirus Surge at an N.Y.C. Hospital

\url{https://nyti.ms/3dspIbn}

\begin{itemize}
\item
\item
\item
\item
\item
\item
\end{itemize}

\href{https://www.nytimes3xbfgragh.onion/news-event/coronavirus?action=click\&pgtype=Article\&state=default\&region=TOP_BANNER\&context=storylines_menu}{The
Coronavirus Outbreak}

\begin{itemize}
\tightlist
\item
  live\href{https://www.nytimes3xbfgragh.onion/2020/08/04/world/coronavirus-cases.html?action=click\&pgtype=Article\&state=default\&region=TOP_BANNER\&context=storylines_menu}{Latest
  Updates}
\item
  \href{https://www.nytimes3xbfgragh.onion/interactive/2020/us/coronavirus-us-cases.html?action=click\&pgtype=Article\&state=default\&region=TOP_BANNER\&context=storylines_menu}{Maps
  and Cases}
\item
  \href{https://www.nytimes3xbfgragh.onion/interactive/2020/science/coronavirus-vaccine-tracker.html?action=click\&pgtype=Article\&state=default\&region=TOP_BANNER\&context=storylines_menu}{Vaccine
  Tracker}
\item
  \href{https://www.nytimes3xbfgragh.onion/2020/08/02/us/covid-college-reopening.html?action=click\&pgtype=Article\&state=default\&region=TOP_BANNER\&context=storylines_menu}{College
  Reopening}
\item
  \href{https://www.nytimes3xbfgragh.onion/live/2020/08/04/business/stock-market-today-coronavirus?action=click\&pgtype=Article\&state=default\&region=TOP_BANNER\&context=storylines_menu}{Economy}
\end{itemize}

Advertisement

\protect\hyperlink{after-top}{Continue reading the main story}

Supported by

\protect\hyperlink{after-sponsor}{Continue reading the main story}

\hypertarget{13-deaths-in-a-day-an-apocalyptic-coronavirus-surge-at-an-nyc-hospital}{%
\section{13 Deaths in a Day: An `Apocalyptic' Coronavirus Surge at an
N.Y.C.
Hospital}\label{13-deaths-in-a-day-an-apocalyptic-coronavirus-surge-at-an-nyc-hospital}}

Hospitals in the city are facing the kind of harrowing increases in
cases that overwhelmed health care systems in China and Italy.

\includegraphics{https://static01.graylady3jvrrxbe.onion/images/2020/03/29/video/nyc-hospital-2020/nyc-hospital-2020-videoSixteenByNineJumbo1600.png}

By \href{https://www.nytimes3xbfgragh.onion/by/michael-rothfeld}{Michael
Rothfeld},
\href{https://www.nytimes3xbfgragh.onion/by/somini-sengupta}{Somini
Sengupta},
\href{https://www.nytimes3xbfgragh.onion/by/joseph-goldstein}{Joseph
Goldstein} and
\href{https://www.nytimes3xbfgragh.onion/by/brian-m-rosenthal}{Brian M.
Rosenthal}

\begin{itemize}
\item
  Published March 25, 2020Updated April 14, 2020
\item
  \begin{itemize}
  \item
  \item
  \item
  \item
  \item
  \item
  \end{itemize}
\end{itemize}

In several hours on Tuesday, Dr. Ashley Bray performed chest
compressions at Elmhurst Hospital Center on a woman in her 80s, a man in
his 60s and a 38-year-old who reminded the doctor of her fiancé. All had
tested positive for the
\href{https://www.nytimes3xbfgragh.onion/2020/04/14/us/armen-henderson-arrested-homeless-coronavirus-testing.html}{coronavirus}
and had gone into cardiac arrest. All eventually died.

Elmhurst, a 545-bed public hospital in Queens, has begun transferring
patients not suffering from coronavirus to other hospitals as it moves
toward becoming dedicated entirely to the outbreak.
\href{https://www.nytimes3xbfgragh.onion/2020/04/14/us/armen-henderson-arrested-homeless-coronavirus-testing.html}{Doctors}
and nurses have struggled to make do with a few dozen ventilators. Calls
over a loudspeaker of ``Team 700,'' the code for when a patient is on
the verge of death, come several times a shift. Some have died inside
the emergency room while waiting for a bed.

A refrigerated truck has been stationed outside to hold the bodies of
the dead. Over the past 24 hours, New York City's public hospital system
said in a statement, 13 people at Elmhurst had died.

``It's apocalyptic,'' said Dr. Bray, 27, a general medicine resident at
the hospital.

Across the city, which has become the epicenter of the coronavirus
outbreak in the United States, hospitals are beginning to confront the
kind of harrowing surge in cases that has overwhelmed health care
systems in China, Italy and other countries. On Wednesday evening, New
York City reported 20,011 confirmed cases and 280 deaths.

More than 3,922 coronavirus patients have been hospitalized in the city.
Gov. Andrew M. Cuomo on Wednesday
\href{https://www.nytimes3xbfgragh.onion/2020/03/25/nyregion/coronavirus-new-york-update.html}{offered
a glimmer of hope that social-distancing measures} were starting to slow
the growth in hospitalizations statewide. This week, the state's
hospitalization estimations were down markedly, from a doubling of cases
every two days to every four days.

It is ``almost too good to be true,'' Mr. Cuomo said.

Still, hospitals are under siege. New York City's hospitals run the
gamut from prestigious teaching institutions catering to the elite to
public hospitals providing care for some of the poorest communities in
the nation. Regardless of whom they serve, few have been spared the
impact of the pandemic: A flood of sick and fearful New Yorkers has
besieged emergency rooms across the city.

Working with state and federal officials, hospitals have repeatedly
expanded the portions of their buildings equipped to handle patients who
had stayed home until worsening fevers and difficulty breathing forced
them into emergency rooms. Elmhurst, among the hardest-hit hospitals in
the city, is a prime example of the hardships medical centers and their
staffs are facing across the country.

``Elmhurst is at the center of this crisis, and it's the number one
priority of our public hospital system right now,'' the city's public
hospital system's statement said. ``The front line staff are going above
and beyond in this crisis, and we continue surging supplies and
personnel to this critical facility to keep pace with the crisis.''

Dr. Mitchell Katz, the head of the Health and Hospitals Corporation,
which operates New York City's public hospitals, said plans were
underway to transform many areas of the Elmhurst hospital into intensive
care units for extremely sick patients.

But New York's hospitals may be about to lose their leeway for
creativity in finding spaces.

All of the more than 1,800 intensive care beds in the city are expected
to be full by Friday, according to a Federal Emergency Management Agency
briefing obtained by The New York Times. Patients could stay for weeks,
limiting space for newly sickened people.

\hypertarget{latest-updates-global-coronavirus-outbreak}{%
\section{\texorpdfstring{\href{https://www.nytimes3xbfgragh.onion/2020/08/04/world/coronavirus-cases.html?action=click\&pgtype=Article\&state=default\&region=MAIN_CONTENT_1\&context=storylines_live_updates}{Latest
Updates: Global Coronavirus
Outbreak}}{Latest Updates: Global Coronavirus Outbreak}}\label{latest-updates-global-coronavirus-outbreak}}

Updated 2020-08-04T21:41:55.934Z

\begin{itemize}
\tightlist
\item
  \href{https://www.nytimes3xbfgragh.onion/2020/08/04/world/coronavirus-cases.html?action=click\&pgtype=Article\&state=default\&region=MAIN_CONTENT_1\&context=storylines_live_updates\#link-2daa96b5}{As
  talks drag on, McConnell signals openness to jobless aid extension
  that Republicans have opposed.}
\item
  \href{https://www.nytimes3xbfgragh.onion/2020/08/04/world/coronavirus-cases.html?action=click\&pgtype=Article\&state=default\&region=MAIN_CONTENT_1\&context=storylines_live_updates\#link-1228a480}{Novavax
  sees encouraging results from two studies of its experimental
  vaccine.}
\item
  \href{https://www.nytimes3xbfgragh.onion/2020/08/04/world/coronavirus-cases.html?action=click\&pgtype=Article\&state=default\&region=MAIN_CONTENT_1\&context=storylines_live_updates\#link-4825b93}{Public
  and private schools in Maryland and elsewhere are divided over
  in-person instruction.}
\end{itemize}

\href{https://www.nytimes3xbfgragh.onion/2020/08/04/world/coronavirus-cases.html?action=click\&pgtype=Article\&state=default\&region=MAIN_CONTENT_1\&context=storylines_live_updates}{See
more updates}

More live coverage:
\href{https://www.nytimes3xbfgragh.onion/live/2020/08/04/business/stock-market-today-coronavirus?action=click\&pgtype=Article\&state=default\&region=MAIN_CONTENT_1\&context=storylines_live_updates}{Markets}

Mr. Cuomo said on Wednesday that he had not seen the briefing. He said
he hoped that officials could quickly add units by dipping into a
growing supply of ventilators, the machines that some coronavirus
patients need to breathe.

\includegraphics{https://static01.graylady3jvrrxbe.onion/images/2020/03/25/nyregion/25nyvirus-hospital1/25nyvirus-hospital1-articleLarge.jpg?quality=75\&auto=webp\&disable=upscale}

The federal government is sending a 1,000-bed hospital ship to New York,
although it is not scheduled to arrive until mid-April. Officials have
begun erecting four 250-bed hospitals at the Jacob K. Javits Convention
Center in Midtown Manhattan, which could be ready in a week. President
Trump
\href{https://twitter.com/realDonaldTrump/status/1242826052310794240}{said
on Wednesday on Twitter} that construction was ahead of schedule, but
that could not be independently confirmed.

Officials have also discussed converting hotels and arenas into
temporary medical centers.

At least two city hospitals have filled up their morgues, and city
officials anticipated the rest would reach capacity by the end of this
week, according to the briefing. The state requested 85 refrigerated
trailers from FEMA for mortuary services, along with staff, the briefing
said.

A spokeswoman for the city's office of the chief medical examiner said
the briefing was inaccurate. ``We have significant morgue capacity in
our five citywide sites, and the ability to expand,'' she said.

In interviews, doctors and nurses at hospitals across the city gave
accounts of how they were being stretched.

Workers at several hospitals, including the Jacobi Medical Center in the
Bronx, said employees such as obstetrician-gynecologists and
radiologists have been called to work in emergency wards.

At a branch of the Montefiore Medical Center, also in the Bronx, there
have been one or two coronavirus-related deaths a day, or more, said
Judy Sheridan-Gonzalez, a nurse. There are not always enough gurneys, so
some patients sit in chairs. One patient on Sunday had been without a
bed for 36 hours, she said.

At the Mount Sinai Health System, some hospital workers in Manhattan
have posted photos on social media showing
\href{https://twitter.com/brianmrosenthal/status/1241744193648762882}{nurses
using trash bags as protective gear}. A system spokesman said she was
not aware of that happening and noted the nurses had other gear below
the bags. ``The safety of our staff and patients has never been of
greater importance and we are taking every precaution possible to
protect everyone,'' she said.

With\href{https://www.nytimes3xbfgragh.onion/video/us/politics/100000007051271/cuomo-coronavirus-update.html}{ventilators
in short supply}, NewYork-Presbyterian Hospital, one of the city's
largest systems, has begun using one machine to help multiple patients
at a time, a virtually unheard-of move, a spokeswoman said.

Image

Some people arrive as early as 6 a.m. at Elmhurst to be tested for the
coronavirus.~Credit...Dave Sanders for The New York Times

\href{https://www.nychealthandhospitals.org/wp-content/uploads/2016/07/chna-elmhurst-2013.pdf}{Elmhurst
Hospital Center} opened in 1832 and moved to its current Queens location
in 1957, making it one of the oldest hospitals in New York City.

In the neighborhood it serves, Elmhurst, more than two-thirds of
residents were born outside of the United States, the highest such rate
in the city. It is a safety-net hospital, serving mainly low-income
patients, including many who lack primary care doctors.

Queens accounts for 32 percent of New York City's confirmed coronavirus
cases, more than any other borough and far more than its share of the
city's population. It also has fewer hospitals. Elmhurst is one of three
major hospitals serving a large population and is centrally located,
which in part explains why it is busy in normal times and even busier
now.

Medical workers said they saw the first signs of the virus in early
March --- an increase in patients coming in with flulike symptoms before
the alarm had been fully raised in the city and the country. Tests
results were taking longer then, but they eventually confirmed that many
of these patients had coronavirus.

\href{https://www.nytimes3xbfgragh.onion/news-event/coronavirus?action=click\&pgtype=Article\&state=default\&region=MAIN_CONTENT_3\&context=storylines_faq}{}

\hypertarget{the-coronavirus-outbreak-}{%
\subsubsection{The Coronavirus Outbreak
›}\label{the-coronavirus-outbreak-}}

\hypertarget{frequently-asked-questions}{%
\paragraph{Frequently Asked
Questions}\label{frequently-asked-questions}}

Updated August 4, 2020

\begin{itemize}
\item ~
  \hypertarget{i-have-antibodies-am-i-now-immune}{%
  \paragraph{I have antibodies. Am I now
  immune?}\label{i-have-antibodies-am-i-now-immune}}

  \begin{itemize}
  \tightlist
  \item
    As of right
    now,\href{https://www.nytimes3xbfgragh.onion/2020/07/22/health/covid-antibodies-herd-immunity.html?action=click\&pgtype=Article\&state=default\&region=MAIN_CONTENT_3\&context=storylines_faq}{that
    seems likely, for at least several months.} There have been
    frightening accounts of people suffering what seems to be a second
    bout of Covid-19. But experts say these patients may have a
    drawn-out course of infection, with the virus taking a slow toll
    weeks to months after initial exposure. People infected with the
    coronavirus typically
    \href{https://www.nature.com/articles/s41586-020-2456-9}{produce}
    immune molecules called antibodies, which are
    \href{https://www.nytimes3xbfgragh.onion/2020/05/07/health/coronavirus-antibody-prevalence.html?action=click\&pgtype=Article\&state=default\&region=MAIN_CONTENT_3\&context=storylines_faq}{protective
    proteins made in response to an
    infection}\href{https://www.nytimes3xbfgragh.onion/2020/05/07/health/coronavirus-antibody-prevalence.html?action=click\&pgtype=Article\&state=default\&region=MAIN_CONTENT_3\&context=storylines_faq}{.
    These antibodies may} last in the body
    \href{https://www.nature.com/articles/s41591-020-0965-6}{only two to
    three months}, which may seem worrisome, but that's perfectly normal
    after an acute infection subsides, said Dr. Michael Mina, an
    immunologist at Harvard University. It may be possible to get the
    coronavirus again, but it's highly unlikely that it would be
    possible in a short window of time from initial infection or make
    people sicker the second time.
  \end{itemize}
\item ~
  \hypertarget{im-a-small-business-owner-can-i-get-relief}{%
  \paragraph{I'm a small-business owner. Can I get
  relief?}\label{im-a-small-business-owner-can-i-get-relief}}

  \begin{itemize}
  \tightlist
  \item
    The
    \href{https://www.nytimes3xbfgragh.onion/article/small-business-loans-stimulus-grants-freelancers-coronavirus.html?action=click\&pgtype=Article\&state=default\&region=MAIN_CONTENT_3\&context=storylines_faq}{stimulus
    bills enacted in March} offer help for the millions of American
    small businesses. Those eligible for aid are businesses and
    nonprofit organizations with fewer than 500 workers, including sole
    proprietorships, independent contractors and freelancers. Some
    larger companies in some industries are also eligible. The help
    being offered, which is being managed by the Small Business
    Administration, includes the Paycheck Protection Program and the
    Economic Injury Disaster Loan program. But lots of folks have
    \href{https://www.nytimes3xbfgragh.onion/interactive/2020/05/07/business/small-business-loans-coronavirus.html?action=click\&pgtype=Article\&state=default\&region=MAIN_CONTENT_3\&context=storylines_faq}{not
    yet seen payouts.} Even those who have received help are confused:
    The rules are draconian, and some are stuck sitting on
    \href{https://www.nytimes3xbfgragh.onion/2020/05/02/business/economy/loans-coronavirus-small-business.html?action=click\&pgtype=Article\&state=default\&region=MAIN_CONTENT_3\&context=storylines_faq}{money
    they don't know how to use.} Many small-business owners are getting
    less than they expected or
    \href{https://www.nytimes3xbfgragh.onion/2020/06/10/business/Small-business-loans-ppp.html?action=click\&pgtype=Article\&state=default\&region=MAIN_CONTENT_3\&context=storylines_faq}{not
    hearing anything at all.}
  \end{itemize}
\item ~
  \hypertarget{what-are-my-rights-if-i-am-worried-about-going-back-to-work}{%
  \paragraph{What are my rights if I am worried about going back to
  work?}\label{what-are-my-rights-if-i-am-worried-about-going-back-to-work}}

  \begin{itemize}
  \tightlist
  \item
    Employers have to provide
    \href{https://www.osha.gov/SLTC/covid-19/standards.html}{a safe
    workplace} with policies that protect everyone equally.
    \href{https://www.nytimes3xbfgragh.onion/article/coronavirus-money-unemployment.html?action=click\&pgtype=Article\&state=default\&region=MAIN_CONTENT_3\&context=storylines_faq}{And
    if one of your co-workers tests positive for the coronavirus, the
    C.D.C.} has said that
    \href{https://www.cdc.gov/coronavirus/2019-ncov/community/guidance-business-response.html}{employers
    should tell their employees} -\/- without giving you the sick
    employee's name -\/- that they may have been exposed to the virus.
  \end{itemize}
\item ~
  \hypertarget{should-i-refinance-my-mortgage}{%
  \paragraph{Should I refinance my
  mortgage?}\label{should-i-refinance-my-mortgage}}

  \begin{itemize}
  \tightlist
  \item
    \href{https://www.nytimes3xbfgragh.onion/article/coronavirus-money-unemployment.html?action=click\&pgtype=Article\&state=default\&region=MAIN_CONTENT_3\&context=storylines_faq}{It
    could be a good idea,} because mortgage rates have
    \href{https://www.nytimes3xbfgragh.onion/2020/07/16/business/mortgage-rates-below-3-percent.html?action=click\&pgtype=Article\&state=default\&region=MAIN_CONTENT_3\&context=storylines_faq}{never
    been lower.} Refinancing requests have pushed mortgage applications
    to some of the highest levels since 2008, so be prepared to get in
    line. But defaults are also up, so if you're thinking about buying a
    home, be aware that some lenders have tightened their standards.
  \end{itemize}
\item ~
  \hypertarget{what-is-school-going-to-look-like-in-september}{%
  \paragraph{What is school going to look like in
  September?}\label{what-is-school-going-to-look-like-in-september}}

  \begin{itemize}
  \tightlist
  \item
    It is unlikely that many schools will return to a normal schedule
    this fall, requiring the grind of
    \href{https://www.nytimes3xbfgragh.onion/2020/06/05/us/coronavirus-education-lost-learning.html?action=click\&pgtype=Article\&state=default\&region=MAIN_CONTENT_3\&context=storylines_faq}{online
    learning},
    \href{https://www.nytimes3xbfgragh.onion/2020/05/29/us/coronavirus-child-care-centers.html?action=click\&pgtype=Article\&state=default\&region=MAIN_CONTENT_3\&context=storylines_faq}{makeshift
    child care} and
    \href{https://www.nytimes3xbfgragh.onion/2020/06/03/business/economy/coronavirus-working-women.html?action=click\&pgtype=Article\&state=default\&region=MAIN_CONTENT_3\&context=storylines_faq}{stunted
    workdays} to continue. California's two largest public school
    districts --- Los Angeles and San Diego --- said on July 13, that
    \href{https://www.nytimes3xbfgragh.onion/2020/07/13/us/lausd-san-diego-school-reopening.html?action=click\&pgtype=Article\&state=default\&region=MAIN_CONTENT_3\&context=storylines_faq}{instruction
    will be remote-only in the fall}, citing concerns that surging
    coronavirus infections in their areas pose too dire a risk for
    students and teachers. Together, the two districts enroll some
    825,000 students. They are the largest in the country so far to
    abandon plans for even a partial physical return to classrooms when
    they reopen in August. For other districts, the solution won't be an
    all-or-nothing approach.
    \href{https://bioethics.jhu.edu/research-and-outreach/projects/eschool-initiative/school-policy-tracker/}{Many
    systems}, including the nation's largest, New York City, are
    devising
    \href{https://www.nytimes3xbfgragh.onion/2020/06/26/us/coronavirus-schools-reopen-fall.html?action=click\&pgtype=Article\&state=default\&region=MAIN_CONTENT_3\&context=storylines_faq}{hybrid
    plans} that involve spending some days in classrooms and other days
    online. There's no national policy on this yet, so check with your
    municipal school system regularly to see what is happening in your
    community.
  \end{itemize}
\end{itemize}

In the weeks after, the emergency room began filling up, with more than
200 people at times. Every chair in the waiting room was usually taken.
Patients came in faster than the hospital could add beds; earlier this
week, 60 coronavirus patients had been admitted but were still in the
emergency room. One man waited almost 60 hours for a bed last week, a
doctor said.

The patients coming in now are sicker than before because they were
advised to try to recover at home, doctors said.

Like other hospitals, Elmhurst has come perilously close to running out
of ventilators several times; other hospitals have replenished its
supply.

Despite the more optimistic projections by the state about
hospitalization rates, the crowds outside of Elmhurst have not thinned
out.

The line of people waiting outside of Elmhurst to be tested for the
coronavirus forms as early as 6 a.m., and some stay there until 5 p.m.
Many are told to go home without being tested.

Julio Jimenez, 35, spent six hours in the emergency room on Sunday night
after running a fever while at work in a New Jersey warehouse. He
returned on Monday morning to stand in the testing line in the pouring
rain. On Tuesday, still coughing, eyes puffy, he stood in line for
nearly seven hours and again went home untested.

``I don't know if I have the virus,'' Mr. Jimenez said. ``It's so hard.
It's not just me. It's for many people. It's crazy.''

Rikki Lane, a doctor who has worked at Elmhurst for more than 20 years,
said the hospital had handled ``the first wave of this tsunami.'' She
compared the scene in the emergency department with an overcrowded
parking garage where physicians must move patients in and out of spots
to access other patients blocked by stretchers.

Family members are not permitted inside, she said.

Dr. Lane recalled recently treating a man in his 30s whose breathing
deteriorated quickly and had to be put on a ventilator. ``He was in
distress and panicked, I could see the terror in his eyes,'' she said.
``He was alone.''

Other doctors said they had tried to resuscitate people while drenched
in sweat under their protective gear, face masks fogging up. Some
patients have been found dead in their rooms while doctors were busy
helping others, they said.

Sometimes doctors try to call patients' families when it is clear they
will not recover.

That is what Dr. Bray said she tried to do before the man who reminded
her of her fiancé died on Tuesday. As it turned out, his mother, also
stricken with the coronavirus, was a patient at another hospital.

``We weren't able to get in touch with anybody,'' Dr. Bray said.

Reporting was contributed by Jesse McKinley, Jesse Drucker, Eileen
Sullivan and Michael Schwirtz. Susan C. Beachy contributed research.

Advertisement

\protect\hyperlink{after-bottom}{Continue reading the main story}

\hypertarget{site-index}{%
\subsection{Site Index}\label{site-index}}

\hypertarget{site-information-navigation}{%
\subsection{Site Information
Navigation}\label{site-information-navigation}}

\begin{itemize}
\tightlist
\item
  \href{https://help.nytimes3xbfgragh.onion/hc/en-us/articles/115014792127-Copyright-notice}{©~2020~The
  New York Times Company}
\end{itemize}

\begin{itemize}
\tightlist
\item
  \href{https://www.nytco.com/}{NYTCo}
\item
  \href{https://help.nytimes3xbfgragh.onion/hc/en-us/articles/115015385887-Contact-Us}{Contact
  Us}
\item
  \href{https://www.nytco.com/careers/}{Work with us}
\item
  \href{https://nytmediakit.com/}{Advertise}
\item
  \href{http://www.tbrandstudio.com/}{T Brand Studio}
\item
  \href{https://www.nytimes3xbfgragh.onion/privacy/cookie-policy\#how-do-i-manage-trackers}{Your
  Ad Choices}
\item
  \href{https://www.nytimes3xbfgragh.onion/privacy}{Privacy}
\item
  \href{https://help.nytimes3xbfgragh.onion/hc/en-us/articles/115014893428-Terms-of-service}{Terms
  of Service}
\item
  \href{https://help.nytimes3xbfgragh.onion/hc/en-us/articles/115014893968-Terms-of-sale}{Terms
  of Sale}
\item
  \href{https://spiderbites.nytimes3xbfgragh.onion}{Site Map}
\item
  \href{https://help.nytimes3xbfgragh.onion/hc/en-us}{Help}
\item
  \href{https://www.nytimes3xbfgragh.onion/subscription?campaignId=37WXW}{Subscriptions}
\end{itemize}
