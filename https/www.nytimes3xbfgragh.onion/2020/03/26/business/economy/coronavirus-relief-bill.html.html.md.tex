Sections

SEARCH

\protect\hyperlink{site-content}{Skip to
content}\protect\hyperlink{site-index}{Skip to site index}

\href{https://www.nytimes3xbfgragh.onion/section/business/economy}{Economy}

\href{https://myaccount.nytimes3xbfgragh.onion/auth/login?response_type=cookie\&client_id=vi}{}

\href{https://www.nytimes3xbfgragh.onion/section/todayspaper}{Today's
Paper}

\href{/section/business/economy}{Economy}\textbar{}The Coronavirus
Economy: When Washington Takes Over Business

\url{https://nyti.ms/3dwQPlI}

\begin{itemize}
\item
\item
\item
\item
\item
\item
\end{itemize}

\hypertarget{the-coronavirus-outbreak}{%
\subsubsection{\texorpdfstring{\href{https://www.nytimes3xbfgragh.onion/news-event/coronavirus?name=styln-coronavirus-markets\&region=TOP_BANNER\&variant=undefined\&block=storyline_menu_recirc\&action=click\&pgtype=Article\&impression_id=0f12f5f0-e0fd-11ea-83f5-7d9653a0ee39}{The
Coronavirus
Outbreak}}{The Coronavirus Outbreak}}\label{the-coronavirus-outbreak}}

\begin{itemize}
\tightlist
\item
  live\href{https://www.nytimes3xbfgragh.onion/2020/08/17/world/coronavirus-covid.html?name=styln-coronavirus-markets\&region=TOP_BANNER\&variant=undefined\&block=storyline_menu_recirc\&action=click\&pgtype=Article\&impression_id=0f12f5f1-e0fd-11ea-83f5-7d9653a0ee39}{Latest
  Updates}
\item
  \href{https://www.nytimes3xbfgragh.onion/interactive/2020/us/coronavirus-us-cases.html?name=styln-coronavirus-markets\&region=TOP_BANNER\&variant=undefined\&block=storyline_menu_recirc\&action=click\&pgtype=Article\&impression_id=0f12f5f2-e0fd-11ea-83f5-7d9653a0ee39}{Maps
  and Cases}
\item
  \href{https://www.nytimes3xbfgragh.onion/interactive/2020/science/coronavirus-vaccine-tracker.html?name=styln-coronavirus-markets\&region=TOP_BANNER\&variant=undefined\&block=storyline_menu_recirc\&action=click\&pgtype=Article\&impression_id=0f12f5f3-e0fd-11ea-83f5-7d9653a0ee39}{Vaccine
  Tracker}
\item
  \href{https://www.nytimes3xbfgragh.onion/2020/08/17/us/k-12-schools-reopening.html?name=styln-coronavirus-markets\&region=TOP_BANNER\&variant=undefined\&block=storyline_menu_recirc\&action=click\&pgtype=Article\&impression_id=0f12f5f4-e0fd-11ea-83f5-7d9653a0ee39}{State
  of Play for K-12}
\item
  \href{https://www.nytimes3xbfgragh.onion/live/2020/08/17/business/stock-market-today-coronavirus?name=styln-coronavirus-markets\&region=TOP_BANNER\&variant=undefined\&block=storyline_menu_recirc\&action=click\&pgtype=Article\&impression_id=0f131d00-e0fd-11ea-83f5-7d9653a0ee39}{Markets
  \& Economy}
\end{itemize}

Advertisement

\protect\hyperlink{after-top}{Continue reading the main story}

Supported by

\protect\hyperlink{after-sponsor}{Continue reading the main story}

\hypertarget{the-coronavirus-economy-when-washington-takes-over-business}{%
\section{The Coronavirus Economy: When Washington Takes Over
Business}\label{the-coronavirus-economy-when-washington-takes-over-business}}

The \$2 trillion aid package is expected to save jobs and bail out
companies, but it reorders the relationship between government and
private industry.

\includegraphics{https://static01.graylady3jvrrxbe.onion/images/2020/03/27/business/26virus-govbiz-print/merlin_170977653_b9f99360-056c-4f8a-b2d5-8e508d418402-articleLarge.jpg?quality=75\&auto=webp\&disable=upscale}

By \href{https://www.nytimes3xbfgragh.onion/by/jim-tankersley}{Jim
Tankersley},
\href{https://www.nytimes3xbfgragh.onion/by/alan-rappeport}{Alan
Rappeport},
\href{https://www.nytimes3xbfgragh.onion/by/david-gelles}{David Gelles},
\href{https://www.nytimes3xbfgragh.onion/by/peter-eavis}{Peter Eavis}
and
\href{http://www.nytimes3xbfgragh.onion/by/david-yaffe-bellany}{David
Yaffe-Bellany}

\begin{itemize}
\item
  March 26, 2020
\item
  \begin{itemize}
  \item
  \item
  \item
  \item
  \item
  \item
  \end{itemize}
\end{itemize}

The \$2 trillion
\href{https://www.nytimes3xbfgragh.onion/2020/07/28/us/politics/coronavirus-relief-bills-house-senate.html}{coronavirus
rescue package} that the Senate passed on Wednesday will fundamentally
transform the U.S. government by placing thousands of businesses and
millions of workers on federally funded life support.

The government will pay the wages of some workers who remain on their
companies' payrolls. It will sustain other workers who have lost their
jobs with checks that are as large as --- or even larger than --- what
they were earning before they were laid off. And it will cushion some of
the country's largest corporations from bankruptcy, with taxpayers
taking shares in those companies as collateral.

Rarely before has the government involved itself so deeply in the
business of business. Amid a historic drop-off in economic activity, the
bill temporarily transfers financial responsibility from private
industry to the federal government, allowing the United States to
control the levers of capitalism and potentially decide who wins and who
loses. The level of intervention this week far outstrips the financial
scope and breadth of recovery efforts during the 2008 financial crisis.

That is a controversial proposition in normal times but one lawmakers
deemed necessary now, as companies large and small, from airlines and
big banks to nail salons and brew pubs, face unprecedented hardship. The
ripple effects have already sent
\href{https://www.nytimes3xbfgragh.onion/2020/03/26/business/economy/coronavirus-unemployment-claims.html}{millions
of Americans into unemployment}.

``We went to bed as America and woke up the next morning looking like
social democratic Europe,'' Erik Gordon, professor at the Ross School of
Business at the University of Michigan, said. ``We've made fun of Europe
propping up their failing steel companies and car companies, and when
push comes to shove we're going to outdo them.''

The government's intervention will come in a variety of ways, including
direct payments to individuals and businesses, generous loans in which
the government agrees to backstop losses and equity stakes in companies.
But there are strings attached, such as limits on executive pay and
provisions that require companies receiving assistance to maintain
employment levels at 90 percent of what they were.

Midsize companies, or those with between 500 and 10,000 employees, get
to borrow at an interest rate that is not higher than 2 percent
annually, and don't have to repay principal or interest for six months.
The midsize companies cannot ``outsource or offshore'' jobs from the
start of the loan until two years after it has been repaid.

\hypertarget{latest-updates-the-coronavirus-outbreak-and-the-economy}{%
\section{\texorpdfstring{\href{https://www.nytimes3xbfgragh.onion/live/2020/08/17/business/stock-market-today-coronavirus?action=click\&pgtype=Article\&state=default\&region=MAIN_CONTENT_1\&context=storylines_live_updates}{Latest
Updates: The Coronavirus Outbreak and the
Economy}}{Latest Updates: The Coronavirus Outbreak and the Economy}}\label{latest-updates-the-coronavirus-outbreak-and-the-economy}}

\href{https://www.nytimes3xbfgragh.onion/live/2020/08/17/business/stock-market-today-coronavirus?action=click\&pgtype=Article\&state=default\&region=MAIN_CONTENT_1\&context=storylines_live_updates\#robinhood-a-stock-trading-app-is-valued-at-11-2-billion-in-its-latest-funding-round}{7h
ago}

\href{https://www.nytimes3xbfgragh.onion/live/2020/08/17/business/stock-market-today-coronavirus?action=click\&pgtype=Article\&state=default\&region=MAIN_CONTENT_1\&context=storylines_live_updates\#robinhood-a-stock-trading-app-is-valued-at-11-2-billion-in-its-latest-funding-round}{Robinhood,
a stock trading app, is valued at \$11.2 billion in its latest funding
round.}

\href{https://www.nytimes3xbfgragh.onion/live/2020/08/17/business/stock-market-today-coronavirus?action=click\&pgtype=Article\&state=default\&region=MAIN_CONTENT_1\&context=storylines_live_updates\#new-unemployment-benefits-authorized-by-president-trump-wont-come-until-late-august}{10h
ago}

\href{https://www.nytimes3xbfgragh.onion/live/2020/08/17/business/stock-market-today-coronavirus?action=click\&pgtype=Article\&state=default\&region=MAIN_CONTENT_1\&context=storylines_live_updates\#new-unemployment-benefits-authorized-by-president-trump-wont-come-until-late-august}{New
unemployment benefits authorized by President Trump won't come until
late August.}

\href{https://www.nytimes3xbfgragh.onion/live/2020/08/17/business/stock-market-today-coronavirus?action=click\&pgtype=Article\&state=default\&region=MAIN_CONTENT_1\&context=storylines_live_updates\#heres-what-you-need-to-know-for-the-week-ahead}{15h
ago}

\href{https://www.nytimes3xbfgragh.onion/live/2020/08/17/business/stock-market-today-coronavirus?action=click\&pgtype=Article\&state=default\&region=MAIN_CONTENT_1\&context=storylines_live_updates\#heres-what-you-need-to-know-for-the-week-ahead}{Here's
what you need to know for the week ahead.}

\href{https://www.nytimes3xbfgragh.onion/live/2020/08/17/business/stock-market-today-coronavirus?action=click\&pgtype=Article\&state=default\&region=MAIN_CONTENT_1\&context=storylines_live_updates}{See
more updates}

More live coverage:
\href{https://www.nytimes3xbfgragh.onion/2020/08/17/world/coronavirus-covid.html?action=click\&pgtype=Article\&state=default\&region=MAIN_CONTENT_1\&context=storylines_live_updates}{Global}

Businesses with 500 or fewer employees will get loans directly from
banks to cover more than two months of payrolls and some other operating
expenses, with the government paying off the balance so long as the
companies either do not lay off workers or rehire ones they've already
let go.

The government will inject more than \$60 billion into the airline
industry, including \$25 billion in grants to pay employees of passenger
airlines and \$4 billion for those who work at cargo airlines. About
\$17 billion has been set aside largely for Boeing, which, because of
two deadly crashes, was troubled before the virus brought many
commercial flights to a standstill.

Not all businesses will be eligible for help, and not every eligible
company will agree to the government's terms. And some industries,
including cruise lines and energy companies, were left on the sidelines.

The major cruise companies appear not to qualify for loans because they
are domiciled outside the United States and their employees are spread
across the world.

``We didn't seek or expect a cash bailout, and it doesn't appear anyway
that we would qualify under the terms,'' said Roger Frizzell, a
spokesman for Carnival Corporation. ``We have a significant employee
presence in the U.S., but a majority of our employees are on ships, not
in any location, certainly not based in the U.S.''

The legislation also does not include \$3 billion that the Trump
administration requested to buy crude oil for the Strategic Petroleum
Reserve. Such a purchase could have helped lift demand for oil, and thus
its price, which in the United States has tumbled to less than \$25 a
barrel in recent weeks. Solar and wind businesses were upset that
lawmakers did not make it easier for them to benefit from tax credits
for renewable energy.

The epicenter of the intervention will be the Treasury Department, where
Secretary Steven Mnuchin will oversee nearly a third of the \$2 trillion
in economic relief funds that Congress is approving.

The money will be held in two pots: \$350 billion will be devoted to
loans and loan guarantees for small businesses. And \$500 billion will
be divided among airlines and companies that are critical to national
security, including Boeing, and will prop up the
\href{https://www.nytimes3xbfgragh.onion/2020/03/26/business/economy/fed-coronavirus-stimulus.html}{Federal
Reserve's new emergency lending facilities}, which are intended to
inject nearly \$4 trillion into the economy.

Mr. Mnuchin said on Thursday that the distribution of the money would be
fully transparent. ``When we do take actions, either through our direct
program or throughout programs with the Fed, there will be disclosures
to the American public much faster than they would normally occur,'' he
said on CNBC.

Businesses will also have to cede some control to the federal government
in exchange for lifelines. Companies that borrow money are forbidden to
repurchase their stock or pay dividends during the loan and for a year
after it is repaid. They must not cut staffing by more than 10 percent
through the end of September.

Loans to small businesses, with 500 employees or fewer, are limited to
\$10 million. Loans to cover salaries of over \$100,000 wouldn't qualify
for forgiveness, and businesses must demonstrate that they had not
recently laid off employees, or a smaller amount of the loan would be
subject to forgiveness.

Businesses would not have to repay loans covering up to eight weeks'
worth of payroll expenses. That means that once businesses receive their
loans, a new clock will begin to tick: They'll have to use the money
within two months to avoid repaying it; they also can't pay any employee
more than \$10,000 in those two months if they want that amount to be
forgiven.

Lawmakers also placed restrictions on compensation and pay increases for
executives, moves intended to address one of the criticisms about
bailouts during the 2008 crisis. But pay limits will not necessarily do
away with multimillion-dollar paydays for corporate bosses.

Executives who made more than \$3 million in 2019 could be awarded \$3
million, plus half of any sum in excess of \$3 million. As a result, a
chief executive who earned \$20 million in 2019 would be allowed
compensation of \$11.5 million. The restrictions would apply from the
time the federal support began to one year after it ended.

Even as the government takes on an outsize role in overseeing companies,
Mr. Mnuchin maintained that it should not be in the business of
dictating what private companies did.

``We don't believe in mandating and regulating certain big businesses,''
he said.

And big business, despite its need for help, has seemed unwilling to
cede too much control to the government. On Tuesday, Boeing's chief
executive, David Calhoun, suggested that he wasn't interested in the
government's taking an equity stake in the company, despite the
beleaguered state of the aerospace giant.

``I don't have a need for an equity stake,'' Mr. Calhoun said in an
interview on Fox Business Network. ``If they forced it, we'd just look
at all the other options, and we have got plenty.''

Boeing, which had lobbied for government aid, was not specifically named
in the bill. It nonetheless signaled its approval of the stimulus
package on Wednesday night. ``The bill's access to public and private
liquidity, including loans and loan guarantees, is critical for
airlines, airports, suppliers and manufacturers to bridge to recovery,''
Boeing said in a statement.

The House is now expected to take up the legislation, and President
Trump has signaled that he would sign it quickly into law.

Many of the provisions are intended to offer lifelines to companies and
workers over the coming months, as the country struggles to contain the
pandemic and braces for a recession. But the long-term consequences of a
\$2 trillion bailout of the American economy are unknown.

``This is going to be hard to unravel,'' said Mr. Gordon, the University
of Michigan professor. ``Industries that are propped up stay propped up
for a long time.''

Reporting was contributed by Niraj Chokshi, Jesse Drucker, Emily
Flitter, Clifford Krauss and Ivan Penn.

Advertisement

\protect\hyperlink{after-bottom}{Continue reading the main story}

\hypertarget{site-index}{%
\subsection{Site Index}\label{site-index}}

\hypertarget{site-information-navigation}{%
\subsection{Site Information
Navigation}\label{site-information-navigation}}

\begin{itemize}
\tightlist
\item
  \href{https://help.nytimes3xbfgragh.onion/hc/en-us/articles/115014792127-Copyright-notice}{©~2020~The
  New York Times Company}
\end{itemize}

\begin{itemize}
\tightlist
\item
  \href{https://www.nytco.com/}{NYTCo}
\item
  \href{https://help.nytimes3xbfgragh.onion/hc/en-us/articles/115015385887-Contact-Us}{Contact
  Us}
\item
  \href{https://www.nytco.com/careers/}{Work with us}
\item
  \href{https://nytmediakit.com/}{Advertise}
\item
  \href{http://www.tbrandstudio.com/}{T Brand Studio}
\item
  \href{https://www.nytimes3xbfgragh.onion/privacy/cookie-policy\#how-do-i-manage-trackers}{Your
  Ad Choices}
\item
  \href{https://www.nytimes3xbfgragh.onion/privacy}{Privacy}
\item
  \href{https://help.nytimes3xbfgragh.onion/hc/en-us/articles/115014893428-Terms-of-service}{Terms
  of Service}
\item
  \href{https://help.nytimes3xbfgragh.onion/hc/en-us/articles/115014893968-Terms-of-sale}{Terms
  of Sale}
\item
  \href{https://spiderbites.nytimes3xbfgragh.onion}{Site Map}
\item
  \href{https://help.nytimes3xbfgragh.onion/hc/en-us}{Help}
\item
  \href{https://www.nytimes3xbfgragh.onion/subscription?campaignId=37WXW}{Subscriptions}
\end{itemize}
