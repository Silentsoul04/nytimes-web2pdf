Sections

SEARCH

\protect\hyperlink{site-content}{Skip to
content}\protect\hyperlink{site-index}{Skip to site index}

\href{https://myaccount.nytimes3xbfgragh.onion/auth/login?response_type=cookie\&client_id=vi}{}

\href{https://www.nytimes3xbfgragh.onion/section/todayspaper}{Today's
Paper}

\href{/section/opinion}{Opinion}\textbar{}Dogs, at Least, Love Home
Quarantine

\url{https://nyti.ms/2UmmvTd}

\begin{itemize}
\item
\item
\item
\item
\item
\item
\end{itemize}

\href{https://www.nytimes3xbfgragh.onion/spotlight/at-home?action=click\&pgtype=Article\&state=default\&region=TOP_BANNER\&context=at_home_menu}{At
Home}

\begin{itemize}
\tightlist
\item
  \href{https://www.nytimes3xbfgragh.onion/2020/07/28/books/time-for-a-literary-road-trip.html?action=click\&pgtype=Article\&state=default\&region=TOP_BANNER\&context=at_home_menu}{Take:
  A Literary Road Trip}
\item
  \href{https://www.nytimes3xbfgragh.onion/2020/07/29/magazine/bored-with-your-home-cooking-some-smoky-eggplant-will-fix-that.html?action=click\&pgtype=Article\&state=default\&region=TOP_BANNER\&context=at_home_menu}{Cook:
  Smoky Eggplant}
\item
  \href{https://www.nytimes3xbfgragh.onion/2020/07/27/travel/moose-michigan-isle-royale.html?action=click\&pgtype=Article\&state=default\&region=TOP_BANNER\&context=at_home_menu}{Look
  Out: For Moose}
\item
  \href{https://www.nytimes3xbfgragh.onion/interactive/2020/at-home/even-more-reporters-editors-diaries-lists-recommendations.html?action=click\&pgtype=Article\&state=default\&region=TOP_BANNER\&context=at_home_menu}{Explore:
  Reporters' Obsessions}
\end{itemize}

Advertisement

\protect\hyperlink{after-top}{Continue reading the main story}

\href{/section/opinion}{Opinion}

Supported by

\protect\hyperlink{after-sponsor}{Continue reading the main story}

\hypertarget{dogs-at-least-love-home-quarantine}{%
\section{Dogs, at Least, Love Home
Quarantine}\label{dogs-at-least-love-home-quarantine}}

Mine and I still walk, and we talk a whole lot more.

By Alexandra Horowitz

Dr. Horowitz is a cognitive scientist who studies dogs.

\begin{itemize}
\item
  March 27, 2020
\item
  \begin{itemize}
  \item
  \item
  \item
  \item
  \item
  \item
  \end{itemize}
\end{itemize}

\includegraphics{https://static01.graylady3jvrrxbe.onion/images/2020/03/30/opinion/27horowitz/27horowitz-articleLarge.jpg?quality=75\&auto=webp\&disable=upscale}

Next to me on the couch, one of my dogs twitches his feet and curls his
lower lip in his sleep. His tail thumps on a soft pillow. An ``urph''
--- maybe a stifled bark --- escapes his mouth and he wakes himself up,
looking at me accusingly. ``That was you!'' I tell him.

As a writer and dog-cognition researcher, I can --- and do --- spend the
greatest part of the day observing dogs, talking to dogs and hanging out
beside my dog while working. And now, in this extraordinary time, many
more of us find ourselves working at home with our dogs full time. Good
for us --- and great for the dogs.

Quarantine, and even social distancing, is meant to impose an isolation
that most of us, as a highly social species, work hard throughout our
lives to avoid. Even the most introverted of us need company --- some
touchstone of a shared existence through time. In reflecting on our
changed society, the isolation of the elderly and the sick seems
especially cruel. In many places, older residents quarantined in nursing
homes cannot have visitors, and are suffering not just from potentially
fatal infections but also from
\href{https://www.npr.org/2020/03/07/813113958/opinion-1-unfathomable-toll-of-the-coronavirus-outbreak-dying-alone}{the
loss of the company of their family members} and the comfort of their
presence and touch --- a biological urge.

Our drive to keep animals, dogs in particular, strikes me as similar in
nature: Their simple presence, and their willingness to be touched, is
viscerally satisfying. Time spent reading on the couch is massively
improved by a dog's head resting on my leg; a warm, snuffling muzzle
directed at me is instantly calming. Social media abounds with images of
dogs (some bemused, some
\href{https://twitter.com/Emmasmith77xx/status/1240952373113208832?s=20}{wagging
their tails so hard as to sprain them}) alongside their isolated
persons: Dogs are now our proxy for other humans.

There are some 90 million dogs in the United States, and in some ways,
we have treated dogs as quasi-people all along. Contemporary dogs in
Western households live inside and join us on our walks, in our repose
with a book or in front of the TV --- they even sleep with us.
\href{https://www.nytimes3xbfgragh.onion/2019/08/02/opinion/sunday/talking-dog.html}{We
talk to our dogs} as though they not only understand us, but also care
deeply about what we say.

Still, our current situation highlights that, as indulged, attired and
birthday-party-feted as the average owned dog is in the United States,
many dogs are ordinarily in a daily condition of social isolation.
Unless owners are able to work from home, or their home is their work,
their sociable canids must stay captive and alone for the majority of
their days. Now, the coronavirus quarantine, by imposing similar
hardship on us, is actually giving dogs something that they've deserved
all along: more of our companionship.

And so I was alarmed by the news that after a 17-year-old Pomeranian in
Hong Kong whose owner had Covid-19 tested
``\href{https://www.scmp.com/news/hong-kong/health-environment/article/3075650/first-dog-found-coronavirus-has-died-after}{weak
positive}'' for the coronavirus, the dog was quarantined in a state
facility and
\href{https://coconuts.co/hongkong/news/hong-kong-pomeranian-that-tested-positive-for-covid-19-dies-after-returning-home/}{died
a few days after returning home} (officially virus-free). I worried that
we would suddenly feel we should be distanced from our pets.

Given the dog's lack of symptoms, it appeared that the Pomeranian might
simply have been carrying the virus much like a
``\href{https://www.scmp.com/news/hong-kong/article/3053026/careful-you-can-still-catch-coronavirus-your-dog-even-if-it-not?utm_medium=email\&utm_source=mailchimp\&utm_campaign=enlz-OpinionDaily\&utm_content=20200301\&MCUID=e0a9ed7cd8\&MCCampaignID=5f452d4331\&}{doorknob
or a piece of tissue}'' might, as one reporter explained. On the one
hand, it is a relief if dogs are only carriers, rather than infected
themselves. On the other hand, to relegate them to the status of a
subway pole is to knock them over to the side of objects, not family
members.

Legally,
\href{https://www.theglobeandmail.com/opinion/article-when-it-comes-to-dogs-we-shouldnt-call-ourselves-owners/}{dogs
are considered property}, despite their roles in our lives. What happens
to property when it is found to be dangerous, or no longer useful?
Objects are given up, thrown out, discarded. After
\href{https://www.marketwatch.com/story/second-dog-tests-positive-for-coronavirus-as-owners-warned-not-to-abandon-pets-2020-03-20}{a
second dog tested positive} in Hong Kong (again without symptoms and
again the dog of an owner with Covid-19), city authorities admonished
people not to abandon their pets.

So what should we do with our dogs, now that we find ourselves in their
near-constant company, eyeballed by them as we move through our homes?
For now, at least, we still ought to walk them outdoors: We can all use
the fresh air, and the dogs, at least, can collect the smells of the
day. Some countries have reached the point where all movement outside is
restricted, but even in those places walking the dog is allowed as an
essential exemption.

If you're worried that your dogs --- like doorknobs --- may be touched
by people who are infected, bathe them with soap after the walk. Then,
once home, make the most of the chance to fill their days with
engagements: \href{https://doi.org/10.1007/s10071-016-0976-3}{bouts of
play},
\href{https://doi.org/10.1016/j.applanim.2018.12.009}{treat-finding
games} or \href{https://doi.org/10.3389/fpsyg.2017.01796}{simply being
in contact} --- all of which is health-giving to both dog and person.

In the United States, shelters that recently put out calls seeking
foster care for homeless animals reported being
\href{https://www.nytimes3xbfgragh.onion/2020/03/19/us/coronavirus-foster-pets.html}{inundated
with applicants}: evidence that the value of a pet's company is well
known. When we emerge from this crisis and normalcy is resuscitated, we
will have a chance to reappraise how we want to conduct our lives.

I hope we will maintain some of our current abnormal condition: giving
our dogs the companionship they need. I hope we will come out of this
with a fuller appreciation of the privilege that it is to keep the
company of animals.

Alexandra Horowitz
(\href{https://twitter.com/DogUmwelt?ref_src=twsrc\%5Egoogle\%7Ctwcamp\%5Eserp\%7Ctwgr\%5Eauthor}{@DogUmwelt})
runs the Dog Cognition Lab at Barnard College and is the author of
``\href{https://www.simonandschuster.com/books/Our-Dogs-Ourselves/Alexandra-Horowitz/9781501175008}{Our
Dogs, Ourselves: The Story of a Singular Bond}.''

\emph{The Times is committed to publishing}
\href{https://www.nytimes3xbfgragh.onion/2019/01/31/opinion/letters/letters-to-editor-new-york-times-women.html}{\emph{a
diversity of letters}} \emph{to the editor. We'd like to hear what you
think about this or any of our articles. Here are some}
\href{https://help.nytimes3xbfgragh.onion/hc/en-us/articles/115014925288-How-to-submit-a-letter-to-the-editor}{\emph{tips}}\emph{.
And here's our email:}
\href{mailto:letters@NYTimes.com}{\emph{letters@NYTimes.com}}\emph{.}

\emph{Follow The New York Times Opinion section on}
\href{https://www.facebookcorewwwi.onion/nytopinion}{\emph{Facebook}}\emph{,}
\href{http://twitter.com/NYTOpinion}{\emph{Twitter (@NYTopinion)}}
\emph{and}
\href{https://www.instagram.com/nytopinion/}{\emph{Instagram}}\emph{.}

Advertisement

\protect\hyperlink{after-bottom}{Continue reading the main story}

\hypertarget{site-index}{%
\subsection{Site Index}\label{site-index}}

\hypertarget{site-information-navigation}{%
\subsection{Site Information
Navigation}\label{site-information-navigation}}

\begin{itemize}
\tightlist
\item
  \href{https://help.nytimes3xbfgragh.onion/hc/en-us/articles/115014792127-Copyright-notice}{©~2020~The
  New York Times Company}
\end{itemize}

\begin{itemize}
\tightlist
\item
  \href{https://www.nytco.com/}{NYTCo}
\item
  \href{https://help.nytimes3xbfgragh.onion/hc/en-us/articles/115015385887-Contact-Us}{Contact
  Us}
\item
  \href{https://www.nytco.com/careers/}{Work with us}
\item
  \href{https://nytmediakit.com/}{Advertise}
\item
  \href{http://www.tbrandstudio.com/}{T Brand Studio}
\item
  \href{https://www.nytimes3xbfgragh.onion/privacy/cookie-policy\#how-do-i-manage-trackers}{Your
  Ad Choices}
\item
  \href{https://www.nytimes3xbfgragh.onion/privacy}{Privacy}
\item
  \href{https://help.nytimes3xbfgragh.onion/hc/en-us/articles/115014893428-Terms-of-service}{Terms
  of Service}
\item
  \href{https://help.nytimes3xbfgragh.onion/hc/en-us/articles/115014893968-Terms-of-sale}{Terms
  of Sale}
\item
  \href{https://spiderbites.nytimes3xbfgragh.onion}{Site Map}
\item
  \href{https://help.nytimes3xbfgragh.onion/hc/en-us}{Help}
\item
  \href{https://www.nytimes3xbfgragh.onion/subscription?campaignId=37WXW}{Subscriptions}
\end{itemize}
