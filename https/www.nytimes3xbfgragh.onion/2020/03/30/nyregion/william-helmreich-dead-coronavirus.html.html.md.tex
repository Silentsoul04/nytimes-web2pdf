Sections

SEARCH

\protect\hyperlink{site-content}{Skip to
content}\protect\hyperlink{site-index}{Skip to site index}

\href{https://www.nytimes3xbfgragh.onion/section/nyregion}{New York}

\href{https://myaccount.nytimes3xbfgragh.onion/auth/login?response_type=cookie\&client_id=vi}{}

\href{https://www.nytimes3xbfgragh.onion/section/todayspaper}{Today's
Paper}

\href{/section/nyregion}{New York}\textbar{}William Helmreich,
Sociologist and a Walker in the City, Dies at 74

\url{https://nyti.ms/3dM3Tnx}

\begin{itemize}
\item
\item
\item
\item
\item
\item
\end{itemize}

\href{https://www.nytimes3xbfgragh.onion/news-event/coronavirus?action=click\&pgtype=Article\&state=default\&region=TOP_BANNER\&context=storylines_menu}{The
Coronavirus Outbreak}

\begin{itemize}
\tightlist
\item
  live\href{https://www.nytimes3xbfgragh.onion/2020/08/04/world/coronavirus-covid-19.html?action=click\&pgtype=Article\&state=default\&region=TOP_BANNER\&context=storylines_menu}{Latest
  Updates}
\item
  \href{https://www.nytimes3xbfgragh.onion/interactive/2020/us/coronavirus-us-cases.html?action=click\&pgtype=Article\&state=default\&region=TOP_BANNER\&context=storylines_menu}{Maps
  and Cases}
\item
  \href{https://www.nytimes3xbfgragh.onion/interactive/2020/science/coronavirus-vaccine-tracker.html?action=click\&pgtype=Article\&state=default\&region=TOP_BANNER\&context=storylines_menu}{Vaccine
  Tracker}
\item
  \href{https://www.nytimes3xbfgragh.onion/2020/08/02/us/covid-college-reopening.html?action=click\&pgtype=Article\&state=default\&region=TOP_BANNER\&context=storylines_menu}{College
  Reopening}
\item
  \href{https://www.nytimes3xbfgragh.onion/live/2020/08/03/business/stock-market-today-coronavirus?action=click\&pgtype=Article\&state=default\&region=TOP_BANNER\&context=storylines_menu}{Economy}
\end{itemize}

Advertisement

\protect\hyperlink{after-top}{Continue reading the main story}

Supported by

\protect\hyperlink{after-sponsor}{Continue reading the main story}

Those We've Lost

\hypertarget{william-helmreich-sociologist-and-a-walker-in-the-city-dies-at-74}{%
\section{William Helmreich, Sociologist and a Walker in the City, Dies
at
74}\label{william-helmreich-sociologist-and-a-walker-in-the-city-dies-at-74}}

A scholar of Judaism as well, he walked every block in New York ---
totaling 6,163 miles --- and wrote a book about his odyssey. He died of
the coronavirus.

\includegraphics{https://static01.graylady3jvrrxbe.onion/images/2020/04/01/obituaries/30Helmreich1/30Helmreich1-articleLarge.jpg?quality=75\&auto=webp\&disable=upscale}

By \href{https://www.nytimes3xbfgragh.onion/by/joseph-berger}{Joseph
Berger}

\begin{itemize}
\item
  Published March 30, 2020Updated April 16, 2020
\item
  \begin{itemize}
  \item
  \item
  \item
  \item
  \item
  \item
  \end{itemize}
\end{itemize}

\emph{This obituary is part of a series about people who have died in
the coronavirus pandemic. Read about others}
\href{https://www.nytimes3xbfgragh.onion/series/people-who-have-died-of-the-coronavirus}{\emph{here}}\emph{.}

When William B. Helmreich was 9 years old, his father, a Polish-Jewish
refugee from the Nazis who was curious about his latest haven, New York
City, started taking him on weekend outings that he playfully called
``Last Stop.'' Father and son would choose a subway line at random, take
it to the end and spend a few hours exploring the novelties of
neighborhoods they had never seen.

Those adventures enlivened several years of the 1950s for young Mr.
Helmreich and were in part the germ of two of the 18 books he would
write or edit as a longtime professor of sociology and scholar of
Judaism.

The first of the two, ``Against All Odds: Holocaust Survivors and the
Successful Lives They Made in America'' (1992), was a data-driven study
that highlighted the survivors' resilience and achievements and
contradicted the commonplace image of them as irremediably traumatized.

The second,
\href{https://www.nytimes3xbfgragh.onion/2017/01/12/nyregion/brooklyn-nobody-knows-william-b-helmreich.html}{``The
New York Nobody Knows: Walking 6,000 Miles in the City''} (2013),
chronicled Professor Helmreich's experiences over four years --- and
many pairs of Rockports --- walking virtually every city block, all
121,000, totaling 6,163 miles. Chatting with strangers, he unearthed a
cornucopia of colorful city sidelights; he even once approached members
of the street gang the Bloods outside a Bronx housing project and asked
them where he could buy one of their red jackets.

This is how he explained his disarming technique to the comedian Barry
Mitchell for a YouTube
\href{https://www.youtube.com/watch?v=yQRiZP6jHFQ\&t=1s}{video}:

``I just say, `What's that horse doing in that guy's backyard?', or, `Is
this neighborhood dangerous? Can I get a good apartment for my son?' In
other words, I just start talking to people.''

A distinguished professor of sociology at City College and the City
University of New York's Graduate Center, he died on Saturday at his
home in Great Neck, N.Y. He was 74. His son Jeffrey said the cause was
the coronavirus.

Curious, gregarious and inexhaustibly energetic, Mr. Helmreich was
fearless in his study of human beings. As a graduate student at
Washington University in St. Louis, he chose to do his dissertation on a
group of black-power advocates who were hostile toward white people like
him, even once getting into a tussle with one of its members. In 1973,
the study was turned into his first book, ``The Black Crusaders: A Case
Study of a Black Militant Organization.''

Although for a time he helped organize the annual parade in Manhattan
celebrating Israel, he conducted a two-hour interview in 2003 in Gaza
with a leader of Hamas, Dr. Abdel Aziz Rantisi, who had just survived an
attack by Israeli helicopters.

The book of his that broke important new ground was ``Against All
Odds.'' In writing it he interviewed 380 Holocaust survivors and found
that, far from the pathological stereotypes surrounding them, they had
more stable marriages, equivalent economic status and a lesser need to
seek psychiatric help than other American Jews of the same age.

He argued that traits like adaptability, tenacity and resourcefulness,
which had been needed to endure near starvation, terror and the loss of
so many loved ones, had enabled most survivors to flourish in the
freedom and opportunities that America afforded. The book won an award
from the Jewish Book Council.

What made his ramble through New York so beguiling --- besides the sheer
feat of his feet --- were the serendipitous encounters and discoveries
of offbeat corners of city life. In Bensonhurst, Brooklyn, he met a man
whose ample garage was chock-a-block with old Dodger baseball uniforms,
carousel horses, gaudy amusement arcade machines and vintage cars ---
all as a wistful homage to the Brooklyn of his childhood.

In Gowanus, Brooklyn, he came across a long-dormant grocery on a street
of rowhouses and found that it that had been kept as a shrine by the
descendants of a Neapolitan immigrant who had opened the business a
century before, its Rheingold and Schaefer beer neon signs flashing at
Christmastime in tribute.

``I saw this as a remarkable example of filial piety, something that
today's generation might not understand,'' Professor Helmreich told Mr.
Mitchell. ``Today's generation is much more techie, much more involved
in the present.''

Image

Professor Helmreich's rambles through New York were inspired in part by
a pastime that he and his father called ``Last Stop.'' They would choose
a subway, take it to the end of the line and spend a few hours exploring
the novelties of neighborhoods they had never seen.Credit...Alessandra
Montalto/The New York Times

William Benno Helmreich was born on Aug. 25, 1945 in Zurich. His
parents, Leo and Sally (Finkelstein) Helmreich, had met in Nazi-occupied
Belgium and had spirited their way through France into neutral
Switzerland. In 1946 the family emigrated to the United States, where
his father worked first repairing diamond jewelry and eventually became
a diamond dealer.

Settling on the Upper West Side of Manhattan, his parents sent
red-haired Willie, as he was known, to Manhattan Day School, a modern
Orthodox yeshiva, where teachers noticed his strong tenor voice and had
him star in the annual Purim play. (Among his many adult diversions,
Professor Helmreich sometimes served as a supplemental cantor.)

He reflected on his mixed feelings about his childhood education in a
memoir, ``Wake Up, Wake Up to Do the Work of the Creator'' (1977), and
later studied more advanced yeshivas in ``The World of the Yeshiva: An
Intimate Portrait of Orthodox Judaism'' (1982).

He attended Yeshiva University before doing graduate work at Washington
University. As a professor at City College, he could be a riveting
teacher, known for provocative interchanges with students and a near
photographic memory. Professor Helmreich was the college's longtime
chairman of sociology, writing books on the Jews of Philip Roth's Newark
and the truths and distortions of ethnic stereotypes as well as
follow-up walking guides to, separately, the streets of Brooklyn,
Manhattan, Staten Island and Queens.

In addition to his son Jeffrey, an assistant professor of philosophy and
law at the University of California at Irvine, Mr. Helmreich is survived
by his wife, Helaine Helmreich, a speech therapist who wrote a
well-received novel, ``The Chimney Tree''; another son, Joseph, a
writer; a daughter, Deborah Halpern, a speech pathologist; and four
grandchildren. A third son, Alan, died of a brain aneurysm in 1998 at
the age of 24.

Learning of Professor Helmreich's sudden death, Jonathan Sarna,
professor of American Jewish history at Brandeis University, said: ``He
was in the wrong profession for the coronavirus. Willie loved talking to
people. Social distancing was not in his nature.''

\href{https://www.nytimes3xbfgragh.onion/interactive/2020/obituaries/people-died-coronavirus-obituaries.html?action=click\&pgtype=Article\&state=default\&region=BELOW_MAIN_CONTENT\&context=covid_obits_promo}{}

\hypertarget{those-weve-lost}{%
\section{Those We've Lost}\label{those-weve-lost}}

The coronavirus pandemic has taken an incalculable death toll. This
series is designed to put names and faces to the numbers.

Read more

\includegraphics{https://static01.graylady3jvrrxbe.onion/images/2020/07/30/obituaries/30Pedro/30Pedro-square640.jpg}

\hypertarget{bernaldina-josuxe9-pedro}{%
\section{Bernaldina José Pedro}\label{bernaldina-josuxe9-pedro}}

d. Boa Vista, Brazil

Leader among the Indigenous Macuxi

\includegraphics{https://static01.graylady3jvrrxbe.onion/images/2020/07/31/obituaries/31Swing/merlin_175167783_8913bc90-0d64-43f3-a655-1bb1bf1601c9-square640.jpg}

\hypertarget{john-eric-swing}{%
\section{John Eric Swing}\label{john-eric-swing}}

d. Fountain Valley, Calif.

Champion of Filipino-Americans

\includegraphics{https://static01.graylady3jvrrxbe.onion/images/2020/07/27/obituaries/27Victor/merlin_175001436_38b11f8e-227a-4e2c-9821-7618af9b2524-square640.jpg}

\hypertarget{victor-victor}{%
\section{Victor Victor}\label{victor-victor}}

d. Santo Domingo, Dominican Republic

Beloved musician of the Dominican Republic

\includegraphics{https://static01.graylady3jvrrxbe.onion/images/2020/07/31/obituaries/31Negron/merlin_175160169_516322ae-fd23-4969-b6b2-193ced371105-square640.jpg}

\hypertarget{dr-eddie-negruxf3n}{%
\section{Dr. Eddie Negrón}\label{dr-eddie-negruxf3n}}

d. Fort Walton Beach, Fla.

Internist on Florida's Emerald Coast

\includegraphics{https://static01.graylady3jvrrxbe.onion/images/2020/07/30/obituaries/30Dobson/merlin_175115928_f6b9271c-8f05-4fe1-a38a-5ca4a58f8935-square640.jpg}

\hypertarget{dobby-dobson}{%
\section{Dobby Dobson}\label{dobby-dobson}}

d. Coral Springs, Fla.

Jamaican singer and songwriter

\includegraphics{https://static01.graylady3jvrrxbe.onion/images/2020/08/01/obituaries/28Gonzalez/merlin_175002771_beb57888-3951-409a-ae13-03a94b2e962e-square640.jpg}

\hypertarget{waldemar-gonzalez}{%
\section{Waldemar Gonzalez}\label{waldemar-gonzalez}}

d. White Plains, N.Y.

Teacher and social worker

Advertisement

\protect\hyperlink{after-bottom}{Continue reading the main story}

\hypertarget{site-index}{%
\subsection{Site Index}\label{site-index}}

\hypertarget{site-information-navigation}{%
\subsection{Site Information
Navigation}\label{site-information-navigation}}

\begin{itemize}
\tightlist
\item
  \href{https://help.nytimes3xbfgragh.onion/hc/en-us/articles/115014792127-Copyright-notice}{©~2020~The
  New York Times Company}
\end{itemize}

\begin{itemize}
\tightlist
\item
  \href{https://www.nytco.com/}{NYTCo}
\item
  \href{https://help.nytimes3xbfgragh.onion/hc/en-us/articles/115015385887-Contact-Us}{Contact
  Us}
\item
  \href{https://www.nytco.com/careers/}{Work with us}
\item
  \href{https://nytmediakit.com/}{Advertise}
\item
  \href{http://www.tbrandstudio.com/}{T Brand Studio}
\item
  \href{https://www.nytimes3xbfgragh.onion/privacy/cookie-policy\#how-do-i-manage-trackers}{Your
  Ad Choices}
\item
  \href{https://www.nytimes3xbfgragh.onion/privacy}{Privacy}
\item
  \href{https://help.nytimes3xbfgragh.onion/hc/en-us/articles/115014893428-Terms-of-service}{Terms
  of Service}
\item
  \href{https://help.nytimes3xbfgragh.onion/hc/en-us/articles/115014893968-Terms-of-sale}{Terms
  of Sale}
\item
  \href{https://spiderbites.nytimes3xbfgragh.onion}{Site Map}
\item
  \href{https://help.nytimes3xbfgragh.onion/hc/en-us}{Help}
\item
  \href{https://www.nytimes3xbfgragh.onion/subscription?campaignId=37WXW}{Subscriptions}
\end{itemize}
