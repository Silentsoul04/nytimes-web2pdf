Sections

SEARCH

\protect\hyperlink{site-content}{Skip to
content}\protect\hyperlink{site-index}{Skip to site index}

\href{https://www.nytimes3xbfgragh.onion/section/politics}{Politics}

\href{https://myaccount.nytimes3xbfgragh.onion/auth/login?response_type=cookie\&client_id=vi}{}

\href{https://www.nytimes3xbfgragh.onion/section/todayspaper}{Today's
Paper}

\href{/section/politics}{Politics}\textbar{}Wartime Production Law Has
Been Used Routinely, but Not With Coronavirus

\href{https://nyti.ms/2UP4ige}{https://nyti.ms/2UP4ige}

\begin{itemize}
\item
\item
\item
\item
\item
\end{itemize}

\href{https://www.nytimes3xbfgragh.onion/news-event/coronavirus?action=click\&pgtype=Article\&state=default\&region=TOP_BANNER\&context=storylines_menu}{The
Coronavirus Outbreak}

\begin{itemize}
\tightlist
\item
  live\href{https://www.nytimes3xbfgragh.onion/2020/08/04/world/coronavirus-covid-19.html?action=click\&pgtype=Article\&state=default\&region=TOP_BANNER\&context=storylines_menu}{Latest
  Updates}
\item
  \href{https://www.nytimes3xbfgragh.onion/interactive/2020/us/coronavirus-us-cases.html?action=click\&pgtype=Article\&state=default\&region=TOP_BANNER\&context=storylines_menu}{Maps
  and Cases}
\item
  \href{https://www.nytimes3xbfgragh.onion/interactive/2020/science/coronavirus-vaccine-tracker.html?action=click\&pgtype=Article\&state=default\&region=TOP_BANNER\&context=storylines_menu}{Vaccine
  Tracker}
\item
  \href{https://www.nytimes3xbfgragh.onion/2020/08/02/us/covid-college-reopening.html?action=click\&pgtype=Article\&state=default\&region=TOP_BANNER\&context=storylines_menu}{College
  Reopening}
\item
  \href{https://www.nytimes3xbfgragh.onion/live/2020/08/03/business/stock-market-today-coronavirus?action=click\&pgtype=Article\&state=default\&region=TOP_BANNER\&context=storylines_menu}{Economy}
\end{itemize}

Advertisement

\protect\hyperlink{after-top}{Continue reading the main story}

Supported by

\protect\hyperlink{after-sponsor}{Continue reading the main story}

\hypertarget{wartime-production-law-has-been-used-routinely-but-not-with-coronavirus}{%
\section{Wartime Production Law Has Been Used Routinely, but Not With
Coronavirus}\label{wartime-production-law-has-been-used-routinely-but-not-with-coronavirus}}

The Defense Production Act has been invoked hundreds of thousands of
times in the Trump years. But with the pandemic, the president sees it
as a ``break the glass'' last resort.

\includegraphics{https://static01.graylady3jvrrxbe.onion/images/2020/03/31/us/politics/31dc-virus-dpa1/merlin_171076446_1d17beea-9c0a-4c57-969b-edc268d85d24-articleLarge.jpg?quality=75\&auto=webp\&disable=upscale}

\href{https://www.nytimes3xbfgragh.onion/by/zolan-kanno-youngs}{\includegraphics{https://static01.graylady3jvrrxbe.onion/images/2019/12/13/reader-center/author-zolan-kanno-youngs/author-zolan-kanno-youngs-thumbLarge.png}}\href{https://www.nytimes3xbfgragh.onion/by/ana-swanson}{\includegraphics{https://static01.graylady3jvrrxbe.onion/images/2018/12/10/multimedia/author-ana-swanson/author-ana-swanson-thumbLarge.png}}

By \href{https://www.nytimes3xbfgragh.onion/by/zolan-kanno-youngs}{Zolan
Kanno-Youngs} and
\href{https://www.nytimes3xbfgragh.onion/by/ana-swanson}{Ana Swanson}

\begin{itemize}
\item
  March 31, 2020
\item
  \begin{itemize}
  \item
  \item
  \item
  \item
  \item
  \end{itemize}
\end{itemize}

WASHINGTON --- Chemicals used to construct military missiles. Materials
needed to build drones. Body armor for agents patrolling the southwest
border. Equipment for natural disaster response.

A Korean War-era law called the
\href{https://www.nytimes3xbfgragh.onion/2020/07/22/us/politics/coronavirus-defense-production-act.html}{Defense
Production Act} has been used to place hundreds of thousands of orders
by President Trump and his administration to ensure the procurement of
vital equipment, according to reports submitted to Congress and
interviews with former government officials.

Yet as governors and members of Congress plead with the president to use
the law to force the production of ventilators and other medical
equipment to combat
\href{https://www.nytimes3xbfgragh.onion/news-event/coronavirus}{the
coronavirus pandemic}, he has for weeks treated it like a ``break the
glass'' last resort, to be invoked only when all else fails.

``You know, we're a country not based on nationalizing our business,''
Mr. Trump said earlier this month. ``Call a person over in Venezuela,
ask them how did nationalization of their businesses work out? Not too
well.''

The Defense Production Act includes a range of authorities including
issuing loans to expand a vendor's capacity, controlling the
distribution of a company's products and the more commonly used power of
compelling companies to prioritize the government's order over those of
other clients.

The law's frequent use, especially by the military to give its contract
priority ratings to jump ahead of a vendor's other clients, has prompted
those most familiar with it to question why the administration has been
so hesitant to tap it for a public health emergency that as of Tuesday
has killed more than 3,600 Americans and sickened 181,000.

``What's more important? Building an aircraft carrier or a frigate using
priority ratings or saving a hundred thousand lives using priorities for
ventilators?'' said Larry Hall, who retired in August as the director of
the Defense Production Act program division at the Federal Emergency
Management Agency. ``If we used the president's logic, most of our
economy is already nationalized. But it isn't.''

On Friday, the president said he had finally pressed the law into action
to force General Motors to step up efforts to manufacture ventilators.
Then the federal authorities
\href{https://www.justice.gov/usao-nj/pr/brooklyn-man-arrested-assaulting-fbi-agents-and-making-false-statements-about-his}{raided
the home of a hoarder} in Brooklyn and his warehouse in New Jersey,
invoking the Defense Production Act to recoup tens of thousands of
surgical masks.

Those actions could presage a wider use of the law now that Peter
Navarro, the White House trade adviser, has been appointed the
coordinator of Defense Production Act policy. The law gives the
government the power to subpoena firms and force companies to fulfill
the government's contractual obligations before those of other clients.
Mr. Navarro, who has in the past criticized multinational companies like
General Motors and Walmart for cheating workers and sending jobs abroad,
appears to relish his new role marshaling American industry.

But critics fear that weeks of dithering have already locked in deadly
shortages at American hospitals.

\hypertarget{latest-updates-global-coronavirus-outbreak}{%
\section{\texorpdfstring{\href{https://www.nytimes3xbfgragh.onion/2020/08/04/world/coronavirus-covid-19.html?action=click\&pgtype=Article\&state=default\&region=MAIN_CONTENT_1\&context=storylines_live_updates}{Latest
Updates: Global Coronavirus
Outbreak}}{Latest Updates: Global Coronavirus Outbreak}}\label{latest-updates-global-coronavirus-outbreak}}

Updated 2020-08-04T09:15:14.275Z

\begin{itemize}
\tightlist
\item
  \href{https://www.nytimes3xbfgragh.onion/2020/08/04/world/coronavirus-covid-19.html?action=click\&pgtype=Article\&state=default\&region=MAIN_CONTENT_1\&context=storylines_live_updates\#link-6b644638}{`Long
  days, long nights': Washington prepares for a prolonged fight over
  virus relief.}
\item
  \href{https://www.nytimes3xbfgragh.onion/2020/08/04/world/coronavirus-covid-19.html?action=click\&pgtype=Article\&state=default\&region=MAIN_CONTENT_1\&context=storylines_live_updates\#link-7af9fca0}{Israel's
  rocky reopening of its schools may be a lesson for the U.S.}
\item
  \href{https://www.nytimes3xbfgragh.onion/2020/08/04/world/coronavirus-covid-19.html?action=click\&pgtype=Article\&state=default\&region=MAIN_CONTENT_1\&context=storylines_live_updates\#link-33bf9168}{Hurricane
  Isaias arrives in North Carolina as officials along the East Coast
  scramble.}
\end{itemize}

\href{https://www.nytimes3xbfgragh.onion/2020/08/04/world/coronavirus-covid-19.html?action=click\&pgtype=Article\&state=default\&region=MAIN_CONTENT_1\&context=storylines_live_updates}{See
more updates}

More live coverage:
\href{https://www.nytimes3xbfgragh.onion/live/2020/08/03/business/stock-market-today-coronavirus?action=click\&pgtype=Article\&state=default\&region=MAIN_CONTENT_1\&context=storylines_live_updates}{Markets}

Mr. Trump has ``discretely, just in a discrete way, done something about
ventilators. But there's an overall appeal for him to do much, much
more,'' Speaker Nancy Pelosi said on Tuesday. ``And this is nothing that
can be handled piecemeal. It has to be across the board.''

Invoking the Defense Production Act is hardly a rare occurrence. As
recently as last summer, the Department of Defense used it to
\href{https://www.businessdefense.gov/News/News-Display/Article/1913110/defense-production-act-title-iii-presidential-determinations-to-strengthen-the/}{obtain
rare earth metals} needed to build lasers, jet engines and armored
vehicles.

The Defense Department estimates that it has used the law's powers
300,000 times a year. The Department of Homeland Security --- including
its subsidiary, FEMA ---~placed more than 1,000 so-called rated orders
in 2018, often for hurricane and other disaster response and recovery
efforts, according to
\href{https://www.fema.gov/media-library-data/1582898704576-dc44bbe61cce3cf763cc8a6b92617188/2018_DPAC_Report_to_Congress.pdf}{a
report} submitted to Congress in 2019 by a committee of federal agencies
formed to plan for the effective use of the law.

The law, which was used frequently by previous administrations as well,
does not permit the federal government to assert complete control over a
company. The federal government can, however, use it to jump ahead of
other clients or issue loans so a company can buy all of the supplies it
needs to complete the government's order by a specific date. A rarely
used authority of the law also allows the administration to control the
distribution of a company's products and determine where such materials
go.

The Pentagon has long been aggressive in its use of the law, inserting
language from the wartime act into contracts to ensure delivery of
products by a specific date.

For the coronavirus pandemic, Mr. Trump has elected to rely on the
volunteerism of the private sector to obtain additional personal
protective equipment, virus test kits and hospital equipment. He and his
advisers have argued that using the act has been unnecessary, given the
outpouring of support from large and small American companies --- from
Ford to MyPillow --- that are retooling their factories to make masks,
ventilators and gloves.

People familiar with the president's thinking say he has been skeptical
of using the law, seeing it as anti-American. But politics may have also
influenced Mr. Trump's decision. The president has repeatedly tried to
deflect responsibility for the most significant crisis on American soil
in decades. Using the Defense Production Act would make it clear that
the government is in charge.

Many corporate executives have also lobbied the Trump administration
against using the act, saying they fear that more government
intervention and bureaucracy at a time of intense supply chain
disruptions could do more harm than good.

``The Defense Production Act isn't a magic wand,'' said Neil Bradley,
the executive vice president and chief policy officer at the U.S.
Chamber of Commerce.

``It can't produce highly specialized manufacturing equipment
overnight,'' he added. ``It can't convert a refrigerator factory into a
ventilator factory.''

The only company the administration has targeted with the law is General
Motors. But more actions could come: Mr. Navarro said he anticipated
frequently using the law to crack down on hoarding and price gouging,
and he added that the government might also use the act to ensure
resources are allocated to companies making protective equipment,
medicines and other medical supplies.

``We won't hesitate to use this where we need to,'' he said.

So far, the administration has not been so decisive.

Peter T. Gaynor, the FEMA administrator, said on CNN last week that the
administration would use the law to procure 60,000 coronavirus test
kits, only to back off hours later.

\includegraphics{https://static01.graylady3jvrrxbe.onion/images/2020/03/31/us/politics/31dc-virus-dpa2/merlin_169290744_11e587b4-c14c-4962-b625-dad2648145b5-articleLarge.jpg?quality=75\&auto=webp\&disable=upscale}

With General Motors, it is unclear what the act accomplished. The
company had already announced its intention to collaborate with the
medical device firm Ventec Life Systems to produce ventilators.

After the president threatened on Friday to invoke the Defense
Production Act against G.M., the automaker announced that the venture
would aim to produce up to 10,000 ventilators a month.

\href{https://www.nytimes3xbfgragh.onion/news-event/coronavirus?action=click\&pgtype=Article\&state=default\&region=MAIN_CONTENT_3\&context=storylines_faq}{}

\hypertarget{the-coronavirus-outbreak-}{%
\subsubsection{The Coronavirus Outbreak
›}\label{the-coronavirus-outbreak-}}

\hypertarget{frequently-asked-questions}{%
\paragraph{Frequently Asked
Questions}\label{frequently-asked-questions}}

Updated August 3, 2020

\begin{itemize}
\item ~
  \hypertarget{im-a-small-business-owner-can-i-get-relief}{%
  \paragraph{I'm a small-business owner. Can I get
  relief?}\label{im-a-small-business-owner-can-i-get-relief}}

  \begin{itemize}
  \tightlist
  \item
    The
    \href{https://www.nytimes3xbfgragh.onion/article/small-business-loans-stimulus-grants-freelancers-coronavirus.html?action=click\&pgtype=Article\&state=default\&region=MAIN_CONTENT_3\&context=storylines_faq}{stimulus
    bills enacted in March} offer help for the millions of American
    small businesses. Those eligible for aid are businesses and
    nonprofit organizations with fewer than 500 workers, including sole
    proprietorships, independent contractors and freelancers. Some
    larger companies in some industries are also eligible. The help
    being offered, which is being managed by the Small Business
    Administration, includes the Paycheck Protection Program and the
    Economic Injury Disaster Loan program. But lots of folks have
    \href{https://www.nytimes3xbfgragh.onion/interactive/2020/05/07/business/small-business-loans-coronavirus.html?action=click\&pgtype=Article\&state=default\&region=MAIN_CONTENT_3\&context=storylines_faq}{not
    yet seen payouts.} Even those who have received help are confused:
    The rules are draconian, and some are stuck sitting on
    \href{https://www.nytimes3xbfgragh.onion/2020/05/02/business/economy/loans-coronavirus-small-business.html?action=click\&pgtype=Article\&state=default\&region=MAIN_CONTENT_3\&context=storylines_faq}{money
    they don't know how to use.} Many small-business owners are getting
    less than they expected or
    \href{https://www.nytimes3xbfgragh.onion/2020/06/10/business/Small-business-loans-ppp.html?action=click\&pgtype=Article\&state=default\&region=MAIN_CONTENT_3\&context=storylines_faq}{not
    hearing anything at all.}
  \end{itemize}
\item ~
  \hypertarget{what-are-my-rights-if-i-am-worried-about-going-back-to-work}{%
  \paragraph{What are my rights if I am worried about going back to
  work?}\label{what-are-my-rights-if-i-am-worried-about-going-back-to-work}}

  \begin{itemize}
  \tightlist
  \item
    Employers have to provide
    \href{https://www.osha.gov/SLTC/covid-19/standards.html}{a safe
    workplace} with policies that protect everyone equally.
    \href{https://www.nytimes3xbfgragh.onion/article/coronavirus-money-unemployment.html?action=click\&pgtype=Article\&state=default\&region=MAIN_CONTENT_3\&context=storylines_faq}{And
    if one of your co-workers tests positive for the coronavirus, the
    C.D.C.} has said that
    \href{https://www.cdc.gov/coronavirus/2019-ncov/community/guidance-business-response.html}{employers
    should tell their employees} -\/- without giving you the sick
    employee's name -\/- that they may have been exposed to the virus.
  \end{itemize}
\item ~
  \hypertarget{should-i-refinance-my-mortgage}{%
  \paragraph{Should I refinance my
  mortgage?}\label{should-i-refinance-my-mortgage}}

  \begin{itemize}
  \tightlist
  \item
    \href{https://www.nytimes3xbfgragh.onion/article/coronavirus-money-unemployment.html?action=click\&pgtype=Article\&state=default\&region=MAIN_CONTENT_3\&context=storylines_faq}{It
    could be a good idea,} because mortgage rates have
    \href{https://www.nytimes3xbfgragh.onion/2020/07/16/business/mortgage-rates-below-3-percent.html?action=click\&pgtype=Article\&state=default\&region=MAIN_CONTENT_3\&context=storylines_faq}{never
    been lower.} Refinancing requests have pushed mortgage applications
    to some of the highest levels since 2008, so be prepared to get in
    line. But defaults are also up, so if you're thinking about buying a
    home, be aware that some lenders have tightened their standards.
  \end{itemize}
\item ~
  \hypertarget{what-is-school-going-to-look-like-in-september}{%
  \paragraph{What is school going to look like in
  September?}\label{what-is-school-going-to-look-like-in-september}}

  \begin{itemize}
  \tightlist
  \item
    It is unlikely that many schools will return to a normal schedule
    this fall, requiring the grind of
    \href{https://www.nytimes3xbfgragh.onion/2020/06/05/us/coronavirus-education-lost-learning.html?action=click\&pgtype=Article\&state=default\&region=MAIN_CONTENT_3\&context=storylines_faq}{online
    learning},
    \href{https://www.nytimes3xbfgragh.onion/2020/05/29/us/coronavirus-child-care-centers.html?action=click\&pgtype=Article\&state=default\&region=MAIN_CONTENT_3\&context=storylines_faq}{makeshift
    child care} and
    \href{https://www.nytimes3xbfgragh.onion/2020/06/03/business/economy/coronavirus-working-women.html?action=click\&pgtype=Article\&state=default\&region=MAIN_CONTENT_3\&context=storylines_faq}{stunted
    workdays} to continue. California's two largest public school
    districts --- Los Angeles and San Diego --- said on July 13, that
    \href{https://www.nytimes3xbfgragh.onion/2020/07/13/us/lausd-san-diego-school-reopening.html?action=click\&pgtype=Article\&state=default\&region=MAIN_CONTENT_3\&context=storylines_faq}{instruction
    will be remote-only in the fall}, citing concerns that surging
    coronavirus infections in their areas pose too dire a risk for
    students and teachers. Together, the two districts enroll some
    825,000 students. They are the largest in the country so far to
    abandon plans for even a partial physical return to classrooms when
    they reopen in August. For other districts, the solution won't be an
    all-or-nothing approach.
    \href{https://bioethics.jhu.edu/research-and-outreach/projects/eschool-initiative/school-policy-tracker/}{Many
    systems}, including the nation's largest, New York City, are
    devising
    \href{https://www.nytimes3xbfgragh.onion/2020/06/26/us/coronavirus-schools-reopen-fall.html?action=click\&pgtype=Article\&state=default\&region=MAIN_CONTENT_3\&context=storylines_faq}{hybrid
    plans} that involve spending some days in classrooms and other days
    online. There's no national policy on this yet, so check with your
    municipal school system regularly to see what is happening in your
    community.
  \end{itemize}
\item ~
  \hypertarget{is-the-coronavirus-airborne}{%
  \paragraph{Is the coronavirus
  airborne?}\label{is-the-coronavirus-airborne}}

  \begin{itemize}
  \tightlist
  \item
    The coronavirus
    \href{https://www.nytimes3xbfgragh.onion/2020/07/04/health/239-experts-with-one-big-claim-the-coronavirus-is-airborne.html?action=click\&pgtype=Article\&state=default\&region=MAIN_CONTENT_3\&context=storylines_faq}{can
    stay aloft for hours in tiny droplets in stagnant air}, infecting
    people as they inhale, mounting scientific evidence suggests. This
    risk is highest in crowded indoor spaces with poor ventilation, and
    may help explain super-spreading events reported in meatpacking
    plants, churches and restaurants.
    \href{https://www.nytimes3xbfgragh.onion/2020/07/06/health/coronavirus-airborne-aerosols.html?action=click\&pgtype=Article\&state=default\&region=MAIN_CONTENT_3\&context=storylines_faq}{It's
    unclear how often the virus is spread} via these tiny droplets, or
    aerosols, compared with larger droplets that are expelled when a
    sick person coughs or sneezes, or transmitted through contact with
    contaminated surfaces, said Linsey Marr, an aerosol expert at
    Virginia Tech. Aerosols are released even when a person without
    symptoms exhales, talks or sings, according to Dr. Marr and more
    than 200 other experts, who
    \href{https://academic.oup.com/cid/article/doi/10.1093/cid/ciaa939/5867798}{have
    outlined the evidence in an open letter to the World Health
    Organization}.
  \end{itemize}
\end{itemize}

``We cannot afford to lose hours in this crisis, much less days, in the
production of ventilators,'' Mr. Navarro said, adding that G.M. had been
working too slowly.

While Mr. Trump has generally refrained from using the law, the recently
passed economic stabilization package included \$1 billion to take
action under the Defense Production Act and ramp up purchases of
necessary medical equipment, protective gear and medicines.

Mr. Trump issued an executive order granting the Health and Human
Services Department and other federal agency leaders the authority to
use the act to compel companies to produce medical equipment. But the
agencies have been ambiguous about their plans.

Lizzie Litzow, a spokeswoman for FEMA, said the agency was ``in the
process of reviewing'' the authorities of the production act.

The Department of Health and Human Services did not immediately respond
to a request for comment.

The law permits federal agencies to skip an often bureaucratic
procurement process that can take months and force companies to come to
the table to sign a contract. Given the speed of the crisis, Joshua
Gotbaum, a former assistant secretary of defense for economic security,
said the government did not have the luxury of normal bidding and
contracting.

``Under the process of business as usual, they're not even going to
award contracts until this month or April, then there will be a
protest,'' he said. ``Under the Defense Production Act, they could have
sat down with people in February'' and finalized a contract.

Previous administrations have also been hesitant to invoke the law for
nonmilitary matters.

If the federal government used the law to make itself the priority,
other clients that had worked through the company's procurement process
could have their orders delayed, though under the law, the vendor is
protected from lawsuits.

``My general experience is when you're in the midst of a national
crisis, contractors generally speaking want to help,'' said Ernest B.
Abbott, who was the general counsel of FEMA during the Clinton
administration. ``They want to participate. They want to be able to keep
their people employed to build what's needed for the nation.''

But Mr. Hall, who in the months after Mr. Trump took office assisted
Customs and Border Protection in using the law to secure body armor from
Armor Express, a Michigan-based company, said such caution had allowed
understanding of the law's powers to atrophy.

``You have people on the civil side saying, `What is this thing? I might
get in trouble using it,''' Mr. Hall said.

While FEMA and Health and Human Services have discussed the law in
training situations, Mr. Hall said he often had to press his superiors
to prepare for its use in the event of a national emergency, such as a
pandemic.

``They had the authority to do this years ago,'' Mr. Hall said. ``They
could have filled up the medical stockpile with priority ratings if they
thought it was necessary.''

Maggie Haberman contributed reporting from New York, and Noah Weiland
from Washington.

Advertisement

\protect\hyperlink{after-bottom}{Continue reading the main story}

\hypertarget{site-index}{%
\subsection{Site Index}\label{site-index}}

\hypertarget{site-information-navigation}{%
\subsection{Site Information
Navigation}\label{site-information-navigation}}

\begin{itemize}
\tightlist
\item
  \href{https://help.nytimes3xbfgragh.onion/hc/en-us/articles/115014792127-Copyright-notice}{©~2020~The
  New York Times Company}
\end{itemize}

\begin{itemize}
\tightlist
\item
  \href{https://www.nytco.com/}{NYTCo}
\item
  \href{https://help.nytimes3xbfgragh.onion/hc/en-us/articles/115015385887-Contact-Us}{Contact
  Us}
\item
  \href{https://www.nytco.com/careers/}{Work with us}
\item
  \href{https://nytmediakit.com/}{Advertise}
\item
  \href{http://www.tbrandstudio.com/}{T Brand Studio}
\item
  \href{https://www.nytimes3xbfgragh.onion/privacy/cookie-policy\#how-do-i-manage-trackers}{Your
  Ad Choices}
\item
  \href{https://www.nytimes3xbfgragh.onion/privacy}{Privacy}
\item
  \href{https://help.nytimes3xbfgragh.onion/hc/en-us/articles/115014893428-Terms-of-service}{Terms
  of Service}
\item
  \href{https://help.nytimes3xbfgragh.onion/hc/en-us/articles/115014893968-Terms-of-sale}{Terms
  of Sale}
\item
  \href{https://spiderbites.nytimes3xbfgragh.onion}{Site Map}
\item
  \href{https://help.nytimes3xbfgragh.onion/hc/en-us}{Help}
\item
  \href{https://www.nytimes3xbfgragh.onion/subscription?campaignId=37WXW}{Subscriptions}
\end{itemize}
