Sections

SEARCH

\protect\hyperlink{site-content}{Skip to
content}\protect\hyperlink{site-index}{Skip to site index}

\href{https://www.nytimes3xbfgragh.onion/section/style}{Style}

\href{https://myaccount.nytimes3xbfgragh.onion/auth/login?response_type=cookie\&client_id=vi}{}

\href{https://www.nytimes3xbfgragh.onion/section/todayspaper}{Today's
Paper}

\href{/section/style}{Style}\textbar{}Stop Trying to Be Productive

\url{https://nyti.ms/3dLmkZl}

\begin{itemize}
\item
\item
\item
\item
\item
\item
\end{itemize}

\href{https://www.nytimes3xbfgragh.onion/spotlight/at-home?action=click\&pgtype=Article\&state=default\&region=TOP_BANNER\&context=at_home_menu}{At
Home}

\begin{itemize}
\tightlist
\item
  \href{https://www.nytimes3xbfgragh.onion/interactive/2020/08/27/arts/design/jackson-heights-queens-virtual-walk-tour.html?action=click\&pgtype=Article\&state=default\&region=TOP_BANNER\&context=at_home_menu}{Tour:
  Jackson Heights}
\item
  \href{https://www.nytimes3xbfgragh.onion/interactive/2020/at-home/even-more-reporters-editors-diaries-lists-recommendations.html?action=click\&pgtype=Article\&state=default\&region=TOP_BANNER\&context=at_home_menu}{Explore:
  Reporters' Google Docs}
\item
  \href{https://www.nytimes3xbfgragh.onion/2020/08/31/dining/hand-held-picnic-food.html?action=click\&pgtype=Article\&state=default\&region=TOP_BANNER\&context=at_home_menu}{Rethink:
  Your Picnic Menu}
\item
  \href{https://www.nytimes3xbfgragh.onion/article/choosing-a-pediatrician-children.html?action=click\&pgtype=Article\&state=default\&region=TOP_BANNER\&context=at_home_menu}{Choose:
  A Pediatrician}
\end{itemize}

Advertisement

\protect\hyperlink{after-top}{Continue reading the main story}

Supported by

\protect\hyperlink{after-sponsor}{Continue reading the main story}

\hypertarget{stop-trying-to-be-productive}{%
\section{Stop Trying to Be
Productive}\label{stop-trying-to-be-productive}}

The internet wants you to believe you aren't doing enough with all that
``extra time'' you have now. But staying inside and attending to basic
needs is plenty.

\includegraphics{https://static01.graylady3jvrrxbe.onion/images/2020/04/05/fashion/01VIRUS-PRODUCTIVITY-art/01VIRUS-PRODUCTIVITY-art-articleLarge.jpg?quality=75\&auto=webp\&disable=upscale}

\href{https://www.nytimes3xbfgragh.onion/by/taylor-lorenz}{\includegraphics{https://static01.graylady3jvrrxbe.onion/images/2020/03/18/reader-center/author-taylor-lorenz/author-taylor-lorenz-thumbLarge.png}}

By \href{https://www.nytimes3xbfgragh.onion/by/taylor-lorenz}{Taylor
Lorenz}

\begin{itemize}
\item
  Published April 1, 2020Updated July 13, 2020
\item
  \begin{itemize}
  \item
  \item
  \item
  \item
  \item
  \item
  \end{itemize}
\end{itemize}

When Dave Kyu, 34, an arts administrator in Philadelphia, realized that
he would be working from home for the foreseeable future, he began to
fantasize about the projects he could now complete around the house.

``We went and bought all this paint and cabinet hardware and thought we
were going to do the kitchen cabinet project we had wanted to do
forever,'' he said. Two weeks later, he and his wife haven't touched
their supplies. They have two children and demanding jobs. There's no
extra time.

``We realize now it was a silly thought,'' Mr. Kyu said. ``It's a lot
more stressful than I expected.''

As the coronavirus outbreak has brought life largely indoors, many
people are feeling pressure to
\href{https://www.nytimes3xbfgragh.onion/programs/home-organization/}{organize
every room in their homes},
\href{https://www.vox.com/the-goods/2020/3/27/21195361/quarantine-recipes-cooking-baking-coronavirus-bread}{become
expert home chefs} (or
\href{https://www.nytimes3xbfgragh.onion/2020/03/30/style/bread-baking-coronavirus.html}{bakers}),
\href{https://twitter.com/rosannecash/status/1238700345548627969}{write
the next ``King Lear''} and
\href{https://www.nytimes3xbfgragh.onion/2020/03/20/style/self-care/isolation-exercise-meditation-coronavirus.html}{get
in shape}. The internet --- with its
\href{https://www.elle.com/culture/career-politics/a31671089/coronavirus-staying-at-home-tips/}{constant}
\href{https://www.forbes.com/sites/averyblank/2020/03/23/3-ways-professionals-can-stay-productive-during-the-coronavirus-outbreak/}{stream}
of
\href{https://www.wired.com/story/time-to-do-the-things-you-keep-putting-off/}{how-to}
\href{https://www.inc.com/andrew-thomas/8-ways-to-stay-productive-during-coronavirus.html}{headlines}
and
\href{https://www.nytimes3xbfgragh.onion/2020/03/26/style/viral-challenges-coronavirus.html}{viral
challenges} --- has only reinforced the demand to get things done.

``It's everywhere,'' said Julie Ulstrup, 57, a photographer in Colorado.
``It's in blog posts, it's on social media, it's in emails I get from
people like, `use this time productively!' As if I usually don't.''

But in the midst of a global pandemic that has upended nearly every
facet of modern life, people are finding it harder and harder to get
things done.

``It's tough enough to be productive in the best of times let alone when
we're in a global crisis,'' said Chris Bailey, a productivity consultant
and the author of
``\href{https://www.penguinrandomhouse.com/books/562033/hyperfocus-by-chris-bailey/9780525522256/}{Hyperfocus:
How to Manage Your Attention in a World of Distraction}.'' ``The idea
that we have so much time available during the day now is fantastic, but
these days it's the opposite of a luxury. We're home because we have to
be home, and we have much less attention because we're living through so
much.''

After her office announced that it would be going remote, Sara Johnson,
30, who works in philanthropy, created a detailed schedule of all the
things she'd do with the extra three hours a day that she would no
longer spend commuting. ``I sat down last weekend and just felt like I
hadn't been maximizing this time that I have that I don't usually have
on my hands,'' she said.

``I set an hour on my cal every day for a home workout. Then I'd be on
calls for three hours, then I'd make a homemade breakfast, take a walk
at lunchtime, work on something non-screen-related in the evening, cook
dinner and go on a run,'' she said. So far, she admitted, ``none of this
has stuck.''

This urge to overachieve, even in times of global crisis, is reflective
of America's always-on work culture. In a recent article for The New
Republic, the journalist Nick Martin
\href{https://newrepublic.com/article/156929/work-home-productivity-coronavirus-pandemic}{writes
that} ``this mind-set is the natural endpoint of America's hustle
culture --- the idea that every nanosecond of our lives must be
commodified and pointed toward profit and self-improvement.'' Drew
Millard put it more directly
\href{https://theoutline.com/post/8883/working-from-home-during-the-coronavirus-pandemic-is-not-a-recipe-for-productivity?zd=2\&zi=l5ucyfsj}{in
an essay for The Outline}: If you are lucky enough to be employed, the
only person who cares what you're doing right now is your boss.

Anne Helen Petersen, a journalist and the author of the forthcoming book
``Can't Even: How Millennials Became the Burnout Generation,'' seconded
his assertion. ``We're so used to making every moment of ours productive
in some capacity,'' she said. ``Like, I'm on a walk, I should listen to
this information podcast that makes me more informed or a better
person.''

Dr. Petersen said that the impulse to optimize every minute is
\href{https://www.nytimes3xbfgragh.onion/2019/01/26/business/against-hustle-culture-rise-and-grind-tgim.html}{especially
common in millennials}, many of whom are now balancing work and child
care at home. ``I think for millennials, our brains are particularly
broken in terms of productivity,'' she said. ``Either you give up or
feel bad about it all the time.''

Maggie Schuman, 32, is facing that very quandary now that her family is
taking part in a Peloton challenge through the workout platform's app.

``Every day everyone sends around a green check mark, and for some
reason, now that I have that in my head of this thing I'm supposed to be
doing, I'm not doing it,'' Ms. Schuman, a product specialist in
California, said. ``I feel a bit like a failure.'' She also ignored her
sister when she tagged her in a push-up challenge on Instagram.

Instead, Ms. Schuman has started a gratitude journal and is working on
practicing acceptance. ``You're supposed to be inventing something or
coming up with the next big business idea or doing something great
that's going to be worthy of time spent at home,'' she said. ``I'm
trying to be more OK with just being.''

Noelle Kelso, 38, a scientific consultant in Georgia, said that she's
``trying to find productivity in the small moments'' but that the recent
events have given her perspective.

``For a lot of Americans, everyone's job is at stake right now whether
you thought you were upper middle class, middle or working class,
everyone's livelihood is at stake,'' she said. Right now she is focusing
on not allowing her mind to ``drift to a place of fear, concern, panic
or stress,'' she said, and instead encouraging herself to ``keep the
faith and remain grateful.''

``Putting all this pressure and stress on myself, it's incredibly
counterproductive,'' said Ms. Ulstrup. ``I'm putting stress on myself
during a time that's already stressful.''

Adam Hasham, 40, a product manager in Washington, said that it's only a
matter of time before more people realize that self-optimization in this
time is futile. ``I stopped seeing the light at the end of the tunnel,''
he said, adding that his optimism about the situation had ``gone out the
window.''

``It's like you're underwater,'' Mr. Hasham said.

Dr. Petersen said having compassion during these times is key. ``I think
that everyone is coping with this differently, and there's a real
tendency to shame people who aren't coping with it the way you are or
have different circumstances,'' she said.

Finding small pleasures helps, too. Mr. Bailey offered one suggestion:
``Get yourself some Indian food and drink a bottle of wine with your
spouse. We're going through a lot and we all just need to take it
easy.''

Advertisement

\protect\hyperlink{after-bottom}{Continue reading the main story}

\hypertarget{site-index}{%
\subsection{Site Index}\label{site-index}}

\hypertarget{site-information-navigation}{%
\subsection{Site Information
Navigation}\label{site-information-navigation}}

\begin{itemize}
\tightlist
\item
  \href{https://help.nytimes3xbfgragh.onion/hc/en-us/articles/115014792127-Copyright-notice}{©~2020~The
  New York Times Company}
\end{itemize}

\begin{itemize}
\tightlist
\item
  \href{https://www.nytco.com/}{NYTCo}
\item
  \href{https://help.nytimes3xbfgragh.onion/hc/en-us/articles/115015385887-Contact-Us}{Contact
  Us}
\item
  \href{https://www.nytco.com/careers/}{Work with us}
\item
  \href{https://nytmediakit.com/}{Advertise}
\item
  \href{http://www.tbrandstudio.com/}{T Brand Studio}
\item
  \href{https://www.nytimes3xbfgragh.onion/privacy/cookie-policy\#how-do-i-manage-trackers}{Your
  Ad Choices}
\item
  \href{https://www.nytimes3xbfgragh.onion/privacy}{Privacy}
\item
  \href{https://help.nytimes3xbfgragh.onion/hc/en-us/articles/115014893428-Terms-of-service}{Terms
  of Service}
\item
  \href{https://help.nytimes3xbfgragh.onion/hc/en-us/articles/115014893968-Terms-of-sale}{Terms
  of Sale}
\item
  \href{https://spiderbites.nytimes3xbfgragh.onion}{Site Map}
\item
  \href{https://help.nytimes3xbfgragh.onion/hc/en-us}{Help}
\item
  \href{https://www.nytimes3xbfgragh.onion/subscription?campaignId=37WXW}{Subscriptions}
\end{itemize}
