Sections

SEARCH

\protect\hyperlink{site-content}{Skip to
content}\protect\hyperlink{site-index}{Skip to site index}

\href{https://www.nytimes3xbfgragh.onion/section/technology/personaltech}{Personal
Tech}

\href{https://myaccount.nytimes3xbfgragh.onion/auth/login?response_type=cookie\&client_id=vi}{}

\href{https://www.nytimes3xbfgragh.onion/section/todayspaper}{Today's
Paper}

\href{/section/technology/personaltech}{Personal Tech}\textbar{}How to
Digitize Your Most Important Documents

\href{https://nyti.ms/2yicgGS}{https://nyti.ms/2yicgGS}

\begin{itemize}
\item
\item
\item
\item
\item
\end{itemize}

Advertisement

\protect\hyperlink{after-top}{Continue reading the main story}

Supported by

\protect\hyperlink{after-sponsor}{Continue reading the main story}

Tech Tip

\hypertarget{how-to-digitize-your-most-important-documents}{%
\section{How to Digitize Your Most Important
Documents}\label{how-to-digitize-your-most-important-documents}}

If you have some spare time at home and want a productive project,
consider creating a digital archive of your personal papers.

\includegraphics{https://static01.graylady3jvrrxbe.onion/images/2020/04/02/technology/personaltech/02TECHTIP-LEAD/01TECHTIP_01-articleLarge.jpg?quality=75\&auto=webp\&disable=upscale}

\href{https://www.nytimes3xbfgragh.onion/by/j-d-biersdorfer}{\includegraphics{https://static01.graylady3jvrrxbe.onion/images/2018/06/14/multimedia/author-j-d-biersdorfer/author-j-d-biersdorfer-thumbLarge.png}}

By \href{https://www.nytimes3xbfgragh.onion/by/j-d-biersdorfer}{J. D.
Biersdorfer}

\begin{itemize}
\item
  April 1, 2020
\item
  \begin{itemize}
  \item
  \item
  \item
  \item
  \item
  \end{itemize}
\end{itemize}

While this week's \href{http://www.worldbackupday.com/en/}{World Backup
Day} is typically celebrated most by hard-drive makers and data-storage
services, it's supposed to make people remember to back up their
computers. But even if you're already backing up your digital files, do
you have a backup plan for your one-of-a-kind documents and photos that
you have only on paper --- like birth certificates, marriage licenses
and military-discharge papers?

Scanning copies of your personal papers creates a digital archive that
can also be used as a backup, especially if you have the files
password-protected and stored in a secure location. And even if you
don't have a document scanner, you can create your personal archive with
a smartphone, a few apps and a bit of time. Here's a guide to getting
started.

\hypertarget{step-1-get-organized}{%
\subsection{Step 1: Get Organized}\label{step-1-get-organized}}

Gather all the documents you want to digitize. In addition to vital
records, consider other papers you've saved over the years and might
want to share, like old letters, certificates, diplomas, newspaper
clippings, heirloom family photographs and other sentimental souvenirs
tucked away in albums or boxes. (Keep in mind that while the electronic
copies of some documents may not be suitable for official use, you can
use them yourself for quick reference.)

Image

Digitizing documents is not an inherently exciting task, but it can go
faster if you have your papers stacked and ready to scan.Credit...J.D.
Biersdorfer

Scanning a big pile of documents is repetitive, but it goes faster when
you have the stack organized and are working in a clean, well-lit area.
Spin up your favorite playlist or podcast while you scan, but avoid open
beverages nearby that may spill on your papers.

\hypertarget{step-2-use-a-scanner-if-you-have-one}{%
\subsection{Step 2: Use a Scanner (if You Have
One)}\label{step-2-use-a-scanner-if-you-have-one}}

Do you have a printer that was sold as a ``multifunction'' device but
that you've never used for more than printing? Dig up the manual,
because your device can probably scan and photocopy as well. You may
need to install utility software or find the scanning function in your
computer's system settings.

\includegraphics{https://static01.graylady3jvrrxbe.onion/images/2020/04/02/technology/personaltech/01TECHTIP-02/01TECHTIP_02-articleLarge.jpg?quality=75\&auto=webp\&disable=upscale}

Once you have it set up, open the scanning cover on the top of the
printer and place your original document or photograph face down on the
glass. Close the lid and select the Scan option on the printer or
computer screen to create the digital file.

\hypertarget{step-3-scan-with-a-mobile-app}{%
\subsection{Step 3: Scan With a Mobile
App}\label{step-3-scan-with-a-mobile-app}}

If you don't have a scanner, your smartphone can fill in. Just browse
your app store and download a mobile scanning app. These apps use your
phone's camera to capture an image of the page. That image is saved as a
PDF or JPG file to your phone or an online storage site. Be sure to read
the app's privacy policy first so you know where your data is being
stored and managed by the company.

Image

In addition to capturing standard-size documents, the Adobe Scan app
works with oversized items like certificates and diplomas.Credit...The
New York Times

\href{https://thewirecutter.com}{Wirecutter}, a product-review site
owned by The New York Times,
\href{https://thewirecutter.com/reviews/best-mobile-scanning-apps/}{recommends
a few scanning apps} for Android and iOS devices, including
\href{https://acrobat.adobe.com/us/en/mobile/scanner-app.html}{Adobe
Scan}, which requires a free Adobe Document Cloud account.
\href{https://www.dropbox.com/features/productivity/doc-scanner-app}{Dropbox}
and \href{https://evernote.com/features/document-scanning}{Evernote}
also offer scanning functions in their apps. For identity-related
documents, an app that keeps your files on the phone (like the \$5
Piksoft TurboScan Pro for
\href{https://play.google.com/store/apps/details?id=com.piksoft.turboscan.free\&hl=en_US}{Android}
and
\href{https://apps.apple.com/us/app/turboscan-pro-pdf-scanner/id342548956}{iOS})
can offer more peace of mind.

\hypertarget{step-4-scan-old-photos-with-your-phone}{%
\subsection{Step 4: Scan Old Photos With Your
Phone}\label{step-4-scan-old-photos-with-your-phone}}

Some document-scanner apps can also capture photographs, which is handy
if you're trying to copy old prints firmly stuck in ancient albums. Or
you could just take a picture of the photo with your phone's camera and
\href{https://www.nytimes3xbfgragh.onion/2018/12/19/technology/personaltech/how-to-rescue-repair-and-revive-old-family-photos.html}{use
photo-editing tools} to improve the image.

Image

Apps like PhotoScan by Google Photos capture and enhance an image using
artificial intelligence, but you can make further improvements in a
photo-editing program.Credit...J.D. Biersdorfer

For faded old prints, an app that
\href{https://ai.googleblog.com/2017/04/photoscan-taking-glare-free-pictures-of.html}{uses
artificial intelligence} to capture, clean up and correct color might
help. \href{https://www.google.com/photos/scan/}{PhotoScan} by Google
Photos, free for Android and iOS, is one option. The similarly named
\href{https://photomyne.com/get-photomyne}{Photo Scan by Photomyne} for
Android and iOS also \href{https://photomyne.com/algorithm}{uses
algorithms} to enhance photos and has a limited free version; the full
version costs \$10 a month to get higher image quality and other
features.

\hypertarget{step-5-protect-and-safely-store-your-files}{%
\subsection{Step 5: Protect and Safely Store Your
Files}\label{step-5-protect-and-safely-store-your-files}}

Identity thieves are always on the hunt for personal information, so
password-protecting your files adds a layer of security. Most computer
operating systems include tools for
\href{https://www.digitaltrends.com/computing/how-to-password-protect-folder/}{locking
folders}, and
\href{https://www.pcmag.com/picks/the-best-encryption-software}{commercial
encryption programs} are another option. PDF-editing programs like Adobe
Acrobat DC also include
\href{https://helpx.adobe.com/acrobat/using/securing-pdfs-passwords.html}{a
password feature}.

Image

If you don't have a third-party encryption program to protect your
personal files from prying eyes, there are other methods. For example,
Mac users can save a folder as an encrypted disk image with the Disk
Utility program.Credit...The New York Times

If you want to keep the documents online with Dropbox or a similar
service, encrypt them before uploading. You can also stash your files on
\href{https://us.norton.com/internetsecurity-how-to-encrypt-a-flash-drive.html}{an
encrypted flash drive} in a secure location. Wirecutter has
\href{https://thewirecutter.com/blog/track-encrypt-phone-laptop-traveling/}{a
guide to locking down} your hardware, including
\href{https://www.nytimes3xbfgragh.onion/2018/03/13/smarter-living/how-to-encrypt-your-computers-data.html}{encrypting
a computer's hard drive}.

Hopefully, you'll never have to call up your digital archive for any
serious situations. But by planning ahead when you have the luxury of
time, you can make sure copies of your documents are there when you need
them.

Advertisement

\protect\hyperlink{after-bottom}{Continue reading the main story}

\hypertarget{site-index}{%
\subsection{Site Index}\label{site-index}}

\hypertarget{site-information-navigation}{%
\subsection{Site Information
Navigation}\label{site-information-navigation}}

\begin{itemize}
\tightlist
\item
  \href{https://help.nytimes3xbfgragh.onion/hc/en-us/articles/115014792127-Copyright-notice}{©~2020~The
  New York Times Company}
\end{itemize}

\begin{itemize}
\tightlist
\item
  \href{https://www.nytco.com/}{NYTCo}
\item
  \href{https://help.nytimes3xbfgragh.onion/hc/en-us/articles/115015385887-Contact-Us}{Contact
  Us}
\item
  \href{https://www.nytco.com/careers/}{Work with us}
\item
  \href{https://nytmediakit.com/}{Advertise}
\item
  \href{http://www.tbrandstudio.com/}{T Brand Studio}
\item
  \href{https://www.nytimes3xbfgragh.onion/privacy/cookie-policy\#how-do-i-manage-trackers}{Your
  Ad Choices}
\item
  \href{https://www.nytimes3xbfgragh.onion/privacy}{Privacy}
\item
  \href{https://help.nytimes3xbfgragh.onion/hc/en-us/articles/115014893428-Terms-of-service}{Terms
  of Service}
\item
  \href{https://help.nytimes3xbfgragh.onion/hc/en-us/articles/115014893968-Terms-of-sale}{Terms
  of Sale}
\item
  \href{https://spiderbites.nytimes3xbfgragh.onion}{Site Map}
\item
  \href{https://help.nytimes3xbfgragh.onion/hc/en-us}{Help}
\item
  \href{https://www.nytimes3xbfgragh.onion/subscription?campaignId=37WXW}{Subscriptions}
\end{itemize}
