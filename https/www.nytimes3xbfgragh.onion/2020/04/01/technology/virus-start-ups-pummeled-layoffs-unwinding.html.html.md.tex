Sections

SEARCH

\protect\hyperlink{site-content}{Skip to
content}\protect\hyperlink{site-index}{Skip to site index}

\href{https://www.nytimes3xbfgragh.onion/section/technology}{Technology}

\href{https://myaccount.nytimes3xbfgragh.onion/auth/login?response_type=cookie\&client_id=vi}{}

\href{https://www.nytimes3xbfgragh.onion/section/todayspaper}{Today's
Paper}

\href{/section/technology}{Technology}\textbar{}Start-Ups Are Pummeled
in the `Great Unwinding'

\url{https://nyti.ms/2V64Uyf}

\begin{itemize}
\item
\item
\item
\item
\item
\end{itemize}

\href{https://www.nytimes3xbfgragh.onion/news-event/coronavirus?action=click\&pgtype=Article\&state=default\&region=TOP_BANNER\&context=storylines_menu}{The
Coronavirus Outbreak}

\begin{itemize}
\tightlist
\item
  live\href{https://www.nytimes3xbfgragh.onion/2020/08/03/world/coronavirus-covid-19.html?action=click\&pgtype=Article\&state=default\&region=TOP_BANNER\&context=storylines_menu}{Latest
  Updates}
\item
  \href{https://www.nytimes3xbfgragh.onion/interactive/2020/us/coronavirus-us-cases.html?action=click\&pgtype=Article\&state=default\&region=TOP_BANNER\&context=storylines_menu}{Maps
  and Cases}
\item
  \href{https://www.nytimes3xbfgragh.onion/interactive/2020/science/coronavirus-vaccine-tracker.html?action=click\&pgtype=Article\&state=default\&region=TOP_BANNER\&context=storylines_menu}{Vaccine
  Tracker}
\item
  \href{https://www.nytimes3xbfgragh.onion/2020/08/02/us/covid-college-reopening.html?action=click\&pgtype=Article\&state=default\&region=TOP_BANNER\&context=storylines_menu}{College
  Reopening}
\item
  \href{https://www.nytimes3xbfgragh.onion/live/2020/08/03/business/stock-market-today-coronavirus?action=click\&pgtype=Article\&state=default\&region=TOP_BANNER\&context=storylines_menu}{Economy}
\end{itemize}

Advertisement

\protect\hyperlink{after-top}{Continue reading the main story}

Supported by

\protect\hyperlink{after-sponsor}{Continue reading the main story}

\hypertarget{start-ups-are-pummeled-in-the-great-unwinding}{%
\section{Start-Ups Are Pummeled in the `Great
Unwinding'}\label{start-ups-are-pummeled-in-the-great-unwinding}}

Dozens have laid off thousands, slashed costs and changed their
businesses to try to survive the pandemic. All that may not work.

\includegraphics{https://static01.graylady3jvrrxbe.onion/images/2020/03/31/business/00virus-startup1/00virus-startup1-articleLarge-v3.jpg?quality=75\&auto=webp\&disable=upscale}

\href{https://www.nytimes3xbfgragh.onion/by/erin-griffith}{\includegraphics{https://static01.graylady3jvrrxbe.onion/images/2019/06/18/reader-center/author-erin-griffith/author-erin-griffith-thumbLarge.png}}

By \href{https://www.nytimes3xbfgragh.onion/by/erin-griffith}{Erin
Griffith}

\begin{itemize}
\item
  April 1, 2020
\item
  \begin{itemize}
  \item
  \item
  \item
  \item
  \item
  \end{itemize}
\end{itemize}

SAN FRANCISCO --- After a crush of travel cancellations in March,
WanderJaunt, a short-term home rental start-up in San Francisco, laid
off 56 of its 240 employees last week.

Demand for services from Wonderschool, a start-up that helps people find
day care and preschool providers, dropped by half, leading it to cut
most of its 60-person staff.

And at ClassPass, which offers a membership program for fitness classes,
over 95 percent of revenue evaporated in just 10 days as studios and
gyms around the world shut down. To survive, the start-up slashed
spending, froze hiring and rushed to build a video streaming service for
virtual workouts.

``This is the great unwinding,'' said Martin Pichinson, head of Sherwood
Partners, a Silicon Valley advisory firm that
\href{https://www.nytimes3xbfgragh.onion/2003/05/04/business/a-new-technology-now-that-new-is-old.html?searchResultPosition=4}{restructures
failed start-ups}. In recent weeks, he said, his firm has fielded a
``firestorm'' of calls --- a volume three or four times the highest he
had ever seen.

Start-ups have always been risky, designed to grow fast or die, but the
\href{https://www.nytimes3xbfgragh.onion/news-event/coronavirus?action=click\&pgtype=Article\&state=default\&module=styln-coronavirus\&variant=show\&region=TOP_BANNER\&context=storyline_menu}{coronavirus
pandemic} is turbocharging Silicon Valley's natural selection and
causing a shake-up so sudden it has defied comparison. In just a few
weeks, more than 50 start-ups have cut or furloughed roughly 6,000
employees, according to a tally by The New York Times. Plans for initial
public offerings are on hold. And funding is drying up for many young
tech companies.

\includegraphics{https://static01.graylady3jvrrxbe.onion/images/2020/03/31/business/00virus-startup2/merlin_171077493_0bb5c287-2b73-4d65-8b3b-876284798a18-articleLarge.jpg?quality=75\&auto=webp\&disable=upscale}

The fallout is hitting the highest-profile start-ups as well as the
smaller ones trying to disrupt them.
\href{https://www.nytimes3xbfgragh.onion/2020/03/10/technology/airbnb-hosts-coronavirus.html}{Airbnb},
the home rental start-up valued at \$31 billion, has stopped hiring and
has suspended \$800 million of marketing. Bird, an electric scooter
start-up, laid off 30 percent of its staff last week, while Everlane, an
apparel company, cut or furloughed hundreds of workers.

The real estate start-ups Knotel and Convene have laid off or furloughed
half of their workers. The hiring site ZipRecruiter cut around 40
percent of its staff. OneWeb, a satellite start-up that had raised \$3
billion in venture funding from investors including SoftBank, the
Japanese conglomerate, filed for bankruptcy on Friday and plans to sell
itself. And travel start-ups --- Vacasa, Sonder, Inspirato, Zeus Living
and TripActions, among others --- have been some of the hardest hit.

Daniel Zhao, a senior economist at Glassdoor, a workplace review and job
listings site, said the situation facing start-ups now was worse than in
downturns like the dot-com bust in the early 2000s and the financial
crisis of 2008.

``The coronavirus outbreak is economically akin to a major hurricane
occurring in every state around the country for weeks on end,'' he said.

\hypertarget{latest-updates-economy}{%
\section{\texorpdfstring{\href{https://www.nytimes3xbfgragh.onion/live/2020/08/03/business/stock-market-today-coronavirus?action=click\&pgtype=Article\&state=default\&region=MAIN_CONTENT_1\&context=storylines_live_updates}{Latest
Updates:
Economy}}{Latest Updates: Economy}}\label{latest-updates-economy}}

\href{https://www.nytimes3xbfgragh.onion/live/2020/08/03/business/stock-market-today-coronavirus?action=click\&pgtype=Article\&state=default\&region=MAIN_CONTENT_1\&context=storylines_live_updates\#the-chicago-fed-president-says-its-up-to-congress-to-save-the-economy}{11h
ago}

\href{https://www.nytimes3xbfgragh.onion/live/2020/08/03/business/stock-market-today-coronavirus?action=click\&pgtype=Article\&state=default\&region=MAIN_CONTENT_1\&context=storylines_live_updates\#the-chicago-fed-president-says-its-up-to-congress-to-save-the-economy}{The
Chicago Fed president says it's up to Congress to save the economy.}

\href{https://www.nytimes3xbfgragh.onion/live/2020/08/03/business/stock-market-today-coronavirus?action=click\&pgtype=Article\&state=default\&region=MAIN_CONTENT_1\&context=storylines_live_updates\#faa-says-boeing-has-effectively-mitigated-defects-in-the-737-max}{11h
ago}

\href{https://www.nytimes3xbfgragh.onion/live/2020/08/03/business/stock-market-today-coronavirus?action=click\&pgtype=Article\&state=default\&region=MAIN_CONTENT_1\&context=storylines_live_updates\#faa-says-boeing-has-effectively-mitigated-defects-in-the-737-max}{F.A.A.
says Boeing has `effectively mitigated' defects in the 737 Max.}

\href{https://www.nytimes3xbfgragh.onion/live/2020/08/03/business/stock-market-today-coronavirus?action=click\&pgtype=Article\&state=default\&region=MAIN_CONTENT_1\&context=storylines_live_updates\#small-businesses-got-emergency-loans-but-not-what-they-expected}{14h
ago}

\href{https://www.nytimes3xbfgragh.onion/live/2020/08/03/business/stock-market-today-coronavirus?action=click\&pgtype=Article\&state=default\&region=MAIN_CONTENT_1\&context=storylines_live_updates\#small-businesses-got-emergency-loans-but-not-what-they-expected}{Small
businesses got emergency loans, but not what they expected.}

\href{https://www.nytimes3xbfgragh.onion/live/2020/08/03/business/stock-market-today-coronavirus?action=click\&pgtype=Article\&state=default\&region=MAIN_CONTENT_1\&context=storylines_live_updates}{See
more updates}

More live coverage:
\href{https://www.nytimes3xbfgragh.onion/2020/08/03/world/coronavirus-covid-19.html?action=click\&pgtype=Article\&state=default\&region=MAIN_CONTENT_1\&context=storylines_live_updates}{Global}

The numbers are stark. In March, job listings at the 30 most valuable
start-ups in the United States dropped 19 percent, or an average of 21
jobs each, according to Thinknum Alternative Data, a research company.
Start-up funding in the first three months of 2020 was also on a pace
for its second-steepest quarterly decline in 10 years, said CB Insights,
which tracks start-ups.

Start-ups in some areas --- telemedicine, food delivery, online
learning, remote work, gaming --- are thriving amid the quarantines. And
there were signs that the
\href{https://www.nytimes3xbfgragh.onion/2020/02/24/technology/tech-start-up-boom-deflate.html}{boom
times were shaky} even before the coronavirus brought wide swaths of the
U.S. economy to a halt.

But the pain is now deeper and most likely just beginning, especially as
investors, already bruised by a
\href{https://www.nytimes3xbfgragh.onion/2019/09/26/business/tech-ipo-market.html}{string
of disappointing I.P.O.s last year}, become even more cautious. On March
5, Sequoia Capital, a top venture capital firm in Silicon Valley, issued
a warning to start-ups, calling Covid-19 ``the black swan of 2020.''

Bill Gurley, an investor at the venture capital firm Benchmark, said
that over the past 10 years of the start-up boom, investors had taken on
more and more risk. That has changed, leaving many of the riskiest
start-ups exposed.

```Risk on' happens slowly,'' he said. ```Risk off' happens overnight.''

\hypertarget{let-go-with-little-warning}{%
\subsection{Let Go With Little
Warning}\label{let-go-with-little-warning}}

Image

A Bird executive informed hundreds of workers on a conference call that
they had lost their jobs.Credit...Coley Brown for The New York Times

For start-up workers, the past few weeks have been sobering. Many had
bought into the tech industry's change-the-world ideals, had few
boundaries between their work and personal lives and hoped for big
payouts if their start-ups went public.

Now they were being laid off over video calls.

At Bird, the Los Angeles scooter start-up, which had once been
\href{https://www.nytimes3xbfgragh.onion/2019/07/22/technology/bird-scooters-valuation.html}{valued}
at as much as \$2.5 billion, hundreds of employees were invited to a
video conference call on Friday morning with just an hour's notice. On
the call, the voice of an unidentified executive explained that their
jobs had been eliminated. A slide outlined the terms: a month of
severance pay, three months of medical benefits and one year to exercise
their stock options.

The workers were asked to mail in their laptops, said Jenny Alvauaje, a
Bird data analyst who was on the call. Some workers missed the call but
learned they had been laid off when they lost access to internal systems
shortly after, she said.

In a statement that called the layoffs ``a difficult decision,'' a Bird
spokeswoman added, ``We purposefully and intentionally did not have any
video on to protect privacy as we delivered the news live to
individuals.''

The end was equally abrupt for Nik Buenning, 40, a data scientist at
Panoramic, a marketing software start-up in Los Angeles. He was just
settling into his work-from-home setup on March 23 when a companywide
email said to expect a call from human resources.

Right away, he said, ``people started sending Slack messages like, `I'm
out.' `I'm out.' `I'm out.''' An hour later, he was out, too.

Mr. Buenning signed up for \href{https://upstreamapp.com/}{Upstream}, a
new networking app that unveiled itself earlier than planned to cater to
tech workers affected by coronavirus layoffs. Sites like
\href{https://www.getsilverlining.com/}{Silver Lining}are also helping
people connect with companies that are still hiring.

Many start-up workers have added their names to Google spreadsheets,
which recruiters share in weekly newsletters like
\href{https://www.layoffs.tech/}{Layoff List}, created by a recruiting
company called Drafted. Hiring managers, venture capitalists and
start-up advisers read the newsletter, said Vinayak Ranade, chief
executive of Drafted.

Some laid-off workers said they might flee to the safety of the largest
tech companies, which are sitting on piles of cash and
\href{https://www.nytimes3xbfgragh.onion/2020/03/23/technology/coronavirus-facebook-amazon-youtube.html}{benefiting
from increased use in the quarantines}.

When Kenyon Brown, 24, a product manager, left his job early this year
at Mariana Tek, a software company in Washington, he had eight or nine
promising job leads at start-ups. Those quickly evaporated as the virus
spread. He said he was now more open to big tech companies since they
were still hiring.

The situation ``has definitely forced me to think about my short-term
career in another light,'' he said.

\hypertarget{survival-mode}{%
\subsection{Survival Mode}\label{survival-mode}}

Image

Michael Chen, chief executive of WanderJaunt, a short-term rental
start-up, has slashed its prices.Credit...Whitney Curtis for The New
York Times

The start-up survival guide reads something like this: Cut spending,
lower prices on products, renegotiate fixed costs for things like leases
and ask the government for assistance for the fitness studios, home
rental operators or gig workers they rely on.

Some entrepreneurs said they viewed the coronavirus as a moment to rally
around their company's mission, citing the
``\href{https://a16z.com/2011/04/14/peacetime-ceowartime-ceo-2/}{wartime
C.E.O}.'' idea popularized among start-ups by the venture capitalist Ben
Horowitz. It states that executives facing an ``imminent existential
threat'' do whatever it takes to win.

``There are no distractions now,'' said Michael Chen, 30, chief
executive of WanderJaunt, the short-term home rental start-up.

Mr. Chen's four-year-old company, which had raised \$27 million in
funding, has slashed its prices; a house that rented for \$700 a night
now goes for \$100, for instance. And it has switched its focus from
vacation travelers to those displaced by the virus, like stranded
college students, people seeking a separate work space or medical
workers isolating themselves from family.

At Sonder, a travel start-up in San Francisco that laid off 282 people
and furloughed 135 of its 1,254 workers last week, the speed of
decision-making has increased from a few days to a matter of hours.

``In many ways it's energizing, but it's also quite chaotic,'' said
Francis Davidson, chief executive of Sonder, which raised \$345 million
in funding and was valued at \$1.1 billion. He said his investors had
advised him to cut fast and deep to allow employees to hit the job
market before things got worse and to avoid multiple rounds of layoffs.

``People that are looking for a really coddled environment should not be
in start-up land,'' he said. ``You need to have thick skin and a high
adversity quotient.''

\hypertarget{a-time-of-weeding-out}{%
\subsection{`A Time of Weeding Out'}\label{a-time-of-weeding-out}}

Image

Alexis Ohanian, an investor at Initialized Capital, has encouraged
founders to adapt to the new reality.Credit...Matt Edge for The New York
Times

Many venture capital firms are flush with cash from record-breaking
hauls in recent years. But they may not decide to use the money to keep
struggling start-ups alive.

``There's no doubt that this will be a time of weeding out of start-ups
that can't survive,'' said Mike Jones, an investor at the venture
capital firm Science.

One venture capital firm in San Francisco, Alpha Bridge Ventures, said
it was too small to pour more money into its start-up investments. The
firm has made a promise to the founders it has backed: If their
companies fail because of the coronavirus, it will give them \$25,000
for their next company.

``We can at least take one burden off their shoulders,'' said Jake
Chapman, an investor at Alpha Bridge, adding that two or three of the
firm's 21 investments are at risk of failing.

In lieu of networking events and lavish retreats, venture firms are now
dispensing advice in blog posts, on Twitter and at virtual panels over
Zoom. At a March start-ups event, Alexis Ohanian, an investor at
Initialized Capital in San Francisco, encouraged founders to adapt to
the new reality.

``If what you're doing now is just not a viable solution in this new
world and in a different economy,'' he said, ``then find something that
is.''

Advertisement

\protect\hyperlink{after-bottom}{Continue reading the main story}

\hypertarget{site-index}{%
\subsection{Site Index}\label{site-index}}

\hypertarget{site-information-navigation}{%
\subsection{Site Information
Navigation}\label{site-information-navigation}}

\begin{itemize}
\tightlist
\item
  \href{https://help.nytimes3xbfgragh.onion/hc/en-us/articles/115014792127-Copyright-notice}{©~2020~The
  New York Times Company}
\end{itemize}

\begin{itemize}
\tightlist
\item
  \href{https://www.nytco.com/}{NYTCo}
\item
  \href{https://help.nytimes3xbfgragh.onion/hc/en-us/articles/115015385887-Contact-Us}{Contact
  Us}
\item
  \href{https://www.nytco.com/careers/}{Work with us}
\item
  \href{https://nytmediakit.com/}{Advertise}
\item
  \href{http://www.tbrandstudio.com/}{T Brand Studio}
\item
  \href{https://www.nytimes3xbfgragh.onion/privacy/cookie-policy\#how-do-i-manage-trackers}{Your
  Ad Choices}
\item
  \href{https://www.nytimes3xbfgragh.onion/privacy}{Privacy}
\item
  \href{https://help.nytimes3xbfgragh.onion/hc/en-us/articles/115014893428-Terms-of-service}{Terms
  of Service}
\item
  \href{https://help.nytimes3xbfgragh.onion/hc/en-us/articles/115014893968-Terms-of-sale}{Terms
  of Sale}
\item
  \href{https://spiderbites.nytimes3xbfgragh.onion}{Site Map}
\item
  \href{https://help.nytimes3xbfgragh.onion/hc/en-us}{Help}
\item
  \href{https://www.nytimes3xbfgragh.onion/subscription?campaignId=37WXW}{Subscriptions}
\end{itemize}
