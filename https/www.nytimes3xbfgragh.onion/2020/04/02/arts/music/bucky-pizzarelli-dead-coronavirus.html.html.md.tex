Sections

SEARCH

\protect\hyperlink{site-content}{Skip to
content}\protect\hyperlink{site-index}{Skip to site index}

\href{https://www.nytimes3xbfgragh.onion/section/arts/music}{Music}

\href{https://myaccount.nytimes3xbfgragh.onion/auth/login?response_type=cookie\&client_id=vi}{}

\href{https://www.nytimes3xbfgragh.onion/section/todayspaper}{Today's
Paper}

\href{/section/arts/music}{Music}\textbar{}Bucky Pizzarelli, Master of
the Jazz Guitar, Is Dead at 94

\url{https://nyti.ms/2w8teab}

\begin{itemize}
\item
\item
\item
\item
\item
\item
\end{itemize}

\href{https://www.nytimes3xbfgragh.onion/news-event/coronavirus?action=click\&pgtype=Article\&state=default\&region=TOP_BANNER\&context=storylines_menu}{The
Coronavirus Outbreak}

\begin{itemize}
\tightlist
\item
  live\href{https://www.nytimes3xbfgragh.onion/2020/08/04/world/coronavirus-covid-19.html?action=click\&pgtype=Article\&state=default\&region=TOP_BANNER\&context=storylines_menu}{Latest
  Updates}
\item
  \href{https://www.nytimes3xbfgragh.onion/interactive/2020/us/coronavirus-us-cases.html?action=click\&pgtype=Article\&state=default\&region=TOP_BANNER\&context=storylines_menu}{Maps
  and Cases}
\item
  \href{https://www.nytimes3xbfgragh.onion/interactive/2020/science/coronavirus-vaccine-tracker.html?action=click\&pgtype=Article\&state=default\&region=TOP_BANNER\&context=storylines_menu}{Vaccine
  Tracker}
\item
  \href{https://www.nytimes3xbfgragh.onion/2020/08/02/us/covid-college-reopening.html?action=click\&pgtype=Article\&state=default\&region=TOP_BANNER\&context=storylines_menu}{College
  Reopening}
\item
  \href{https://www.nytimes3xbfgragh.onion/live/2020/08/03/business/stock-market-today-coronavirus?action=click\&pgtype=Article\&state=default\&region=TOP_BANNER\&context=storylines_menu}{Economy}
\end{itemize}

Advertisement

\protect\hyperlink{after-top}{Continue reading the main story}

Supported by

\protect\hyperlink{after-sponsor}{Continue reading the main story}

Those We've Lost

\hypertarget{bucky-pizzarelli-master-of-the-jazz-guitar-is-dead-at-94}{%
\section{Bucky Pizzarelli, Master of the Jazz Guitar, Is Dead at
94}\label{bucky-pizzarelli-master-of-the-jazz-guitar-is-dead-at-94}}

He became a mainstay of the New York jazz scene, often performing with
his celebrated son John. He died of the coronavirus.

\includegraphics{https://static01.graylady3jvrrxbe.onion/images/2020/04/03/obituaries/02Pizzarelli/merlin_47603381_b0f4c640-f3d3-4240-ac6c-dd80d6c5c0eb-articleLarge.jpg?quality=75\&auto=webp\&disable=upscale}

By Peter Keepnews

\begin{itemize}
\item
  Published April 2, 2020Updated April 16, 2020
\item
  \begin{itemize}
  \item
  \item
  \item
  \item
  \item
  \item
  \end{itemize}
\end{itemize}

\emph{This obituary is part of a series about people who have died in
the coronavirus pandemic. Read about others}
\href{https://www.nytimes3xbfgragh.onion/series/people-who-have-died-of-the-coronavirus}{\emph{here}}\emph{.}

\href{https://www.youtube.com/watch?v=goDqtP4MtqQ}{Bucky Pizzarelli},
who after many years as a respected but relatively anonymous session
guitarist became a mainstay of the New York jazz scene in the 1970s,
died on Wednesday in Saddle River, N.J. He was 94.

The guitarist and singer John Pizzarelli, his son and frequent musical
associate, said the cause was the coronavirus.

A master of the subtle art of rhythm guitar as well as a gifted soloist,
Mr. Pizzarelli was sought after for recording sessions in the 1950s and
'60s and can be heard on hundreds of records in various genres, although
he was often uncredited. He also toured with Benny Goodman and was a
longtime member of the ``Tonight Show'' orchestra. But he was little
known to all but the most knowledgeable jazz fans until he was in his
40s.

When Johnny Carson moved ``The Tonight Show'' to California from New
York in 1972, Mr. Pizzarelli stayed behind. He explained at the time
that he did not want to uproot his four school-age children from their
New Jersey home. Freed of the responsibilities of a regular job, he
began performing more frequently in New York nightclubs.

Among those clubs was a Midtown Manhattan spot appropriately named the
Guitar, where he had already attracted attention in a duo with his
fellow guitarist George Barnes in 1970. Reviewing
\href{https://www.nytimes3xbfgragh.onion/1970/10/09/archives/a-jazz-guitar-duo-shows-virtuosity-bucky-pizzarelli-becomes-partner.html}{one
of their first performances}, John S. Wilson of The New York Times
wrote: ``This is a brilliant and unique team. Mr. Barnes and Mr.
Pizzarelli can be dazzling and they can be sensuously brooding. They
sparkle with excitement, leap with joy or relax with a warm romantic
glow.''

After Mr. Pizzarelli and Mr. Barnes parted ways in 1972, Mr. Pizzarelli
began performing and recording in high-profile settings: unaccompanied,
as the leader of small groups, and as a sideman with leading jazz
musicians like the saxophonists Zoot Sims and Bud Freeman and the
violinists Stéphane Grappelli and Joe Venuti.

In 1980 he began performing with a new duo partner:
\href{http://www.johnpizzarelli.com/}{his son John}, 20 years old at the
time, who went on to become a jazz star in his own right. ``That's where
he got his baptism of
fire,''\href{https://www.youtube.com/watch?v=dtwdk8VTyx4\&t=2897s}{Mr.
Pizzarelli told an interviewer in 1997}. ``With me giving him dirty
looks when he played a wrong chord.''

\includegraphics{https://static01.graylady3jvrrxbe.onion/images/2020/04/03/obituaries/02Pizzarelli3/merlin_11898802_f2066bbc-05da-485d-a200-3a05f8e92e8a-articleLarge.jpg?quality=75\&auto=webp\&disable=upscale}

\href{https://www.youtube.com/watch?v=36UBSUodBeg}{The two Pizzarellis
would perform and record together} many times, often joined by Mr.
Pizzarelli's other son, Martin, a bassist, and the vocalist
\href{http://www.jessicamolaskey.com/}{Jessica Molaskey}, John's wife.
John Pizzarelli
\href{https://www.nytimes3xbfgragh.onion/2004/06/08/arts/the-family-that-plays-together-has-an-improbably-good-time.html}{once
described them} as ``the von Trapp family on martinis.'' As John's star
ascended, he frequently employed his father as a sideman.

Mr. Pizzarelli's sons survive him, as do his wife, Ruth (Litchult)
Pizzarelli; two daughters, Anne Hymes and Mary Pizzarelli; and four
grandchildren.

Mr. Pizzarelli was among the few guitarists to play an instrument with
seven strings rather than the customary six. (His son was another;
George van Eps is believed to have been the first.) The extra string,
tuned to a low A, enabled him to provide his own bass line, an important
advantage when he played unaccompanied or in a duo setting.

John Paul Pizzarelli was born on Jan. 9, 1926, in Paterson, N.J., where
his parents, John and Amelia (DiDomenico) Pizzarelli, owned a grocery
store. Two uncles, Pete and Bobby Domenick, played guitar and banjo
professionally, and his uncle Bobby taught him some musical rudiments.

His unlikely nickname was bestowed on him by his father, who as a
teenager had decided to explore the Wild West he knew only from movies
and spent some time as a ranch hand in Odessa, Texas. He returned to New
Jersey with a lot of memories and a lingering love for the West that
would lead him to nickname his young son Buckskin. Shortened to Bucky,
the name stuck.

Mr. Pizzarelli began his professional career in his teens, touring with
the singer Vaughn Monroe, best known for his hit ``Racing With the
Moon.'' After serving two years in the Army, he rejoined the Monroe band
in 1946 and remained until it broke up in 1953.

There followed a brief tenure with the popular instrumental group the
Three Suns, a year with the singer Kate Smith's television show, and a
long stint as a first-call studio musician. In addition to recording
with singers like Frank Sinatra and Sarah Vaughan, Mr. Pizzarelli played
on commercial jingles and numerous pop records, including Ben E. King's
``Stand by Me'' and a string of hits by Dion and the Belmonts.

Image

Mr. Pizzarelli in 1970. He continued to perform into his 90s, even after
a stroke and pneumonia led to several hospitalizations.Credit...David
Redfern/Redferns

He also became a staff musician at NBC, where, starting in 1964, he was
a member of the ``Tonight Show'' ensemble, led at the time by Skitch
Henderson and later by Doc Severinsen. (He also worked for a while in
the band the drummer Bobby Rosengarden led for Johnny Carson's ABC
competitor Dick Cavett.)

Mr. Pizzarelli began his long association with Benny Goodman in 1966,
which lasted until Mr. Goodman's death in 1986. He worked frequently in
New York with small groups led by Mr. Goodman and took part in four
European tours with him in the 1970s.

Mr. Pizzarelli continued to perform into his 90s, even after a stroke
and pneumonia led to hospitalizations in 2015 and 2016 and left him
debilitated. ``I don't remember any of it,'' he said. ``I never knew it
until it was over.''

Friends and family members wondered if he would ever play again. But he
recovered, and by the end of 2016 he was back in action.

\href{http://www.chicagotribune.com/entertainment/music/reich/ct-ent-0624-bucky-pizzarelli-20170623-column.html}{Reviewing
a June 2017 performance at the Jazz Showcase in Chicago}, Howard Reich
of The Chicago Tribune praised Mr. Pizzarelli's ``uncommonly sweet and
delicate tone'' and ``disarmingly straightforward approach to melodic
line.'' ``Even at his exalted age,'' Mr. Reich noted, ``Pizzarelli
brought considerable craft to his solos, dispatching practically every
note with heightened care.''

The guitarist Ed Laub, who studied with Mr. Pizzarelli in the 1960s and
went on to perform with him, summarized Mr. Pizzarelli's philosophy in
\href{https://www.nj.com/inside-jersey/index.ssf/2016/12/nj_guitar_master_bucky_pizzarelli_turns_back_time_as_he_plays_on.html}{an
interview with Inside Jersey magazine} in 2016: ``It's about making
beautiful music. It's not about grandstanding. And that's what his whole
personality is about.''

\href{https://www.nytimes3xbfgragh.onion/interactive/2020/obituaries/people-died-coronavirus-obituaries.html?action=click\&pgtype=Article\&state=default\&region=BELOW_MAIN_CONTENT\&context=covid_obits_promo}{}

\hypertarget{those-weve-lost}{%
\section{Those We've Lost}\label{those-weve-lost}}

The coronavirus pandemic has taken an incalculable death toll. This
series is designed to put names and faces to the numbers.

Read more

\includegraphics{https://static01.graylady3jvrrxbe.onion/images/2020/07/30/obituaries/30Pedro/30Pedro-square640.jpg}

\hypertarget{bernaldina-josuxe9-pedro}{%
\section{Bernaldina José Pedro}\label{bernaldina-josuxe9-pedro}}

d. Boa Vista, Brazil

Leader among the Indigenous Macuxi

\includegraphics{https://static01.graylady3jvrrxbe.onion/images/2020/07/31/obituaries/31Swing/merlin_175167783_8913bc90-0d64-43f3-a655-1bb1bf1601c9-square640.jpg}

\hypertarget{john-eric-swing}{%
\section{John Eric Swing}\label{john-eric-swing}}

d. Fountain Valley, Calif.

Champion of Filipino-Americans

\includegraphics{https://static01.graylady3jvrrxbe.onion/images/2020/07/27/obituaries/27Victor/merlin_175001436_38b11f8e-227a-4e2c-9821-7618af9b2524-square640.jpg}

\hypertarget{victor-victor}{%
\section{Victor Victor}\label{victor-victor}}

d. Santo Domingo, Dominican Republic

Beloved musician of the Dominican Republic

\includegraphics{https://static01.graylady3jvrrxbe.onion/images/2020/07/31/obituaries/31Negron/merlin_175160169_516322ae-fd23-4969-b6b2-193ced371105-square640.jpg}

\hypertarget{dr-eddie-negruxf3n}{%
\section{Dr. Eddie Negrón}\label{dr-eddie-negruxf3n}}

d. Fort Walton Beach, Fla.

Internist on Florida's Emerald Coast

\includegraphics{https://static01.graylady3jvrrxbe.onion/images/2020/07/30/obituaries/30Dobson/merlin_175115928_f6b9271c-8f05-4fe1-a38a-5ca4a58f8935-square640.jpg}

\hypertarget{dobby-dobson}{%
\section{Dobby Dobson}\label{dobby-dobson}}

d. Coral Springs, Fla.

Jamaican singer and songwriter

\includegraphics{https://static01.graylady3jvrrxbe.onion/images/2020/08/01/obituaries/28Gonzalez/merlin_175002771_beb57888-3951-409a-ae13-03a94b2e962e-square640.jpg}

\hypertarget{waldemar-gonzalez}{%
\section{Waldemar Gonzalez}\label{waldemar-gonzalez}}

d. White Plains, N.Y.

Teacher and social worker

Advertisement

\protect\hyperlink{after-bottom}{Continue reading the main story}

\hypertarget{site-index}{%
\subsection{Site Index}\label{site-index}}

\hypertarget{site-information-navigation}{%
\subsection{Site Information
Navigation}\label{site-information-navigation}}

\begin{itemize}
\tightlist
\item
  \href{https://help.nytimes3xbfgragh.onion/hc/en-us/articles/115014792127-Copyright-notice}{©~2020~The
  New York Times Company}
\end{itemize}

\begin{itemize}
\tightlist
\item
  \href{https://www.nytco.com/}{NYTCo}
\item
  \href{https://help.nytimes3xbfgragh.onion/hc/en-us/articles/115015385887-Contact-Us}{Contact
  Us}
\item
  \href{https://www.nytco.com/careers/}{Work with us}
\item
  \href{https://nytmediakit.com/}{Advertise}
\item
  \href{http://www.tbrandstudio.com/}{T Brand Studio}
\item
  \href{https://www.nytimes3xbfgragh.onion/privacy/cookie-policy\#how-do-i-manage-trackers}{Your
  Ad Choices}
\item
  \href{https://www.nytimes3xbfgragh.onion/privacy}{Privacy}
\item
  \href{https://help.nytimes3xbfgragh.onion/hc/en-us/articles/115014893428-Terms-of-service}{Terms
  of Service}
\item
  \href{https://help.nytimes3xbfgragh.onion/hc/en-us/articles/115014893968-Terms-of-sale}{Terms
  of Sale}
\item
  \href{https://spiderbites.nytimes3xbfgragh.onion}{Site Map}
\item
  \href{https://help.nytimes3xbfgragh.onion/hc/en-us}{Help}
\item
  \href{https://www.nytimes3xbfgragh.onion/subscription?campaignId=37WXW}{Subscriptions}
\end{itemize}
