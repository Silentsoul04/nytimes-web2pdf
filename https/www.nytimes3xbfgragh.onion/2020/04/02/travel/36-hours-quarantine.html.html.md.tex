Sections

SEARCH

\protect\hyperlink{site-content}{Skip to
content}\protect\hyperlink{site-index}{Skip to site index}

\href{https://www.nytimes3xbfgragh.onion/section/travel}{Travel}

\href{https://myaccount.nytimes3xbfgragh.onion/auth/login?response_type=cookie\&client_id=vi}{}

\href{https://www.nytimes3xbfgragh.onion/section/todayspaper}{Today's
Paper}

\href{/section/travel}{Travel}\textbar{}36 Hours in \ldots{} Wherever
You Are

\url{https://nyti.ms/2X8MQ9j}

\begin{itemize}
\item
\item
\item
\item
\item
\item
\end{itemize}

\href{https://www.nytimes3xbfgragh.onion/spotlight/at-home?action=click\&pgtype=Article\&state=default\&region=TOP_BANNER\&context=at_home_menu}{At
Home}

\begin{itemize}
\tightlist
\item
  \href{https://www.nytimes3xbfgragh.onion/2020/08/14/dining/lobster-salad-recipe.html?action=click\&pgtype=Article\&state=default\&region=TOP_BANNER\&context=at_home_menu}{Make:
  Lobster Salad}
\item
  \href{https://www.nytimes3xbfgragh.onion/2020/08/15/at-home/coronavirus-at-home-quick-exercises.html?action=click\&pgtype=Article\&state=default\&region=TOP_BANNER\&context=at_home_menu}{Sneak
  In: Exercise}
\item
  \href{https://www.nytimes3xbfgragh.onion/interactive/2020/at-home/even-more-reporters-editors-diaries-lists-recommendations.html?action=click\&pgtype=Article\&state=default\&region=TOP_BANNER\&context=at_home_menu}{See:
  Reporters' Obsessions}
\item
  \href{https://www.nytimes3xbfgragh.onion/2020/08/15/at-home/coronavirus-fall-patio-furniture.html?action=click\&pgtype=Article\&state=default\&region=TOP_BANNER\&context=at_home_menu}{Deck
  Out: Your Porch}
\end{itemize}

Advertisement

\protect\hyperlink{after-top}{Continue reading the main story}

Supported by

\protect\hyperlink{after-sponsor}{Continue reading the main story}

\hypertarget{36-hours-in--wherever-you-are}{%
\section{36 Hours in \ldots{} Wherever You
Are}\label{36-hours-in--wherever-you-are}}

Our first reader-generated itinerary: a weekend of traveling you can do
from home.

\includegraphics{https://static01.graylady3jvrrxbe.onion/images/2020/04/02/travel/02HOURS-READERS-GRID/02HOURS-READERS-GRID-articleLarge.jpg?quality=75\&auto=webp\&disable=upscale}

\href{https://www.nytimes3xbfgragh.onion/by/tacey-rychter}{\includegraphics{https://static01.graylady3jvrrxbe.onion/images/2019/01/29/multimedia/author-tacey-rychter/author-tacey-rychter-thumbLarge.png}}

By \href{https://www.nytimes3xbfgragh.onion/by/tacey-rychter}{Tacey
Rychter}

\begin{itemize}
\item
  Published April 2, 2020Updated April 10, 2020
\item
  \begin{itemize}
  \item
  \item
  \item
  \item
  \item
  \item
  \end{itemize}
\end{itemize}

The New York Times has published its popular
\href{https://www.nytimes3xbfgragh.onion/column/36-hours}{36 Hours
column} for nearly 20 years, helping readers plan weekends in far-flung
destinations all over the world.

For many of us over the past month, our grand plans have shrunk down to
small ones, as have the physical spaces we occupy.

While we can't travel for pleasure right now, the spirit of travel ---
our curiosity, empathy and sense of adventure --- can't be confined.

With all of this in mind, and to continue our 36 Hours column,
\href{https://www.nytimes3xbfgragh.onion/2020/03/19/travel/36-hours-readers.html}{we
called out to our readers} for ideas of what people could do over a
weekend, wherever they are in the world (even if they are homebound),
that embraced the ethos of travel.

We received (and read!) more than 1,400 submissions from all over the
world --- from Guangzhou to Zurich, Sydney to Buenos Aires, and across
the United States.

Below is our first reader-sourced 36 Hours column. We hope it moves you.
The responses have been edited for clarity, style and length.

\begin{center}\rule{0.5\linewidth}{\linethickness}\end{center}

\includegraphics{https://static01.graylady3jvrrxbe.onion/images/2020/04/02/travel/02hours-readers-03/02hours-readers-03-articleLarge.jpg?quality=75\&auto=webp\&disable=upscale}

\hypertarget{friday}{%
\subsection{Friday}\label{friday}}

\hypertarget{1-5-pm-happy-hour}{%
\subsubsection{\texorpdfstring{\textbf{1) 5 p.m. Happy
hour}}{1) 5 p.m. Happy hour}}\label{1-5-pm-happy-hour}}

Make a plan to meet your neighbors at a distance --- each of you
bringing your respective libation --- and yell across the fence, from
your fire escape, across the street or out your window about how much
you'd rather be enjoying that same drink at some chic bar.

--- Kai Romero, San Francisco

Make a Cazuela cocktail (one of Guadalajara's signature drinks), because
it is a great way to get some vitamin C and a little bit of tequila.
Pour grapefruit soda into a bowl, add in some slices of fresh
grapefruit, orange and lime, throw in a shot of reposado tequila and a
pinch of sea salt. The outcome will be the most refreshing drink you'll
ever try.

--- Lorena Kunz Salim, Guadalajara, Mexico

\begin{center}\rule{0.5\linewidth}{\linethickness}\end{center}

\hypertarget{2-7-pm-flavor-country}{%
\subsubsection{\texorpdfstring{\textbf{2) 7 p.m. Flavor
country}}{2) 7 p.m. Flavor country}}\label{2-7-pm-flavor-country}}

Cook your way through a cuisine and share the experience. I am Turkish,
my roommate is Armenian and we're both from Istanbul. We are sharing
recipes from an old cookbook with our friends all over the world and
having dinner with them via Zoom. The recipes are by the
Armenian-American chef George Mardikian, who started his famous San
Francisco restaurant, Omar Khayyam's, during another crisis: the Great
Depression. It's so interesting to see how dishes we both grew up eating
are perceived and translated across cultures.

--- Hande Oynar, Brooklyn, N.Y.

Make
\href{https://cooking.nytimes3xbfgragh.onion/recipes/9814-zuppa-pavese}{Zuppa
Pavese}, a hearty comfort food typical of Lombardy --- the area in Italy
most affected by Covid-19. In individual heated bowls, place a slice of
toasted sourdough and crack one egg; dust with tons of grated
Parmigiano-Reggiano; season with salt and black pepper. Ladle boiling
chicken or bone broth to cover: the heat of the broth soft-cooks the egg
and soaks the bread. Serve the soup immediately, breaking everything
with a spoon and mixing into a fantastic mess. A soothing moment of
closeness at a time when closeness is not an option.

--- Eleonora Baldwin, Rome, Italy

\begin{center}\rule{0.5\linewidth}{\linethickness}\end{center}

Image

Credit...Beverly Dalton

\hypertarget{saturday}{%
\subsection{Saturday}\label{saturday}}

\hypertarget{3-9-am-stretch-and-explore}{%
\subsubsection{\texorpdfstring{\textbf{3) 9 a.m. Stretch and
explore}}{3) 9 a.m. Stretch and explore}}\label{3-9-am-stretch-and-explore}}

Let the sunlight wake you up naturally. My bed faces a window. Ever
since this pandemic exploded, I stopped drawing the curtains. I sit on
my bed, look at the vast city of Manila outside my window and pray for
my people. Find solace in morning meditation. Let your mind travel
through the many small things you can still be grateful for. A litany of
gratitude and a remembrance of the good things that have passed can help
the mind and the soul.

--- Beverly Dalton, Manila, Philippines

Take a walk before your first coffee or tea. Walking, especially early
in the morning, provides a fresh view of the day. It's hard to be
pessimistic when the day is just beginning.

--- Neill Kramer, Bali, Indonesia

Try doing push-ups every day. If you've never done them before, start
with modifications. You can do them on your knees. If you have weak
wrists, try using your forearms or knuckles. Proper form: straight back
(don't arch), keep your butt down, keep your head aligned with your
spine. Your body should be straight. Your shoulders should be positioned
over your hands. Lower yourself until your chest reaches the ground. See
how many you can do, and challenge yourself to do one more each day!

--- Yasmin Assef, Wilton, Conn.

\begin{center}\rule{0.5\linewidth}{\linethickness}\end{center}

\hypertarget{4-11-am-local-wilds}{%
\subsubsection{\texorpdfstring{\textbf{4) 11 a.m. Local
wilds}}{4) 11 a.m. Local wilds}}\label{4-11-am-local-wilds}}

Become an eco-traveler in your own yard. Often we think of nature as
``somewhere else'' --- but it's really all around us, no matter where we
are. What birds are in the trees and bushes, or perched on top of the
light pole? If you don't have a field guide, download the
\href{https://www.audubon.org/}{Audubon identification app} and spend
some time discovering your avian neighbors. (There are apps for
identifying butterflies and native bees, too.)

--- Elaine Stachera Simon, Las Cruces, N.M.

\begin{center}\rule{0.5\linewidth}{\linethickness}\end{center}

Image

Credit...Stacy Kissel

\hypertarget{5-1-pm-go-green}{%
\subsubsection{\texorpdfstring{\textbf{5) 1 p.m. Go
green}}{5) 1 p.m. Go green}}\label{5-1-pm-go-green}}

Wander to the nearest bit of earth, whether that's your backyard, the
tree in front of your house or just a bit of soil in a pot. Plant some
seeds, put water and nutrients into the dirt and wait. Be patient.
Something will grow, and you will be surprised by your ability to enrich
the world around you with life. You may not be traveling far, but you
are actively changing your environment. While we may not witness spring
this year, things will bloom nonetheless.

--- Amanda Reynolds, Brooklyn, N.Y.

Create a green indoor oasis. Collect all your houseplants and arrange
them in the room where you spend the most time. Snip clippings off any
that can spare an offshoot, and place them in water-filled empty jars.
Expand your collection by rooting food scraps such as avocado pits,
pineapple tops and the bottoms of lettuce and green onions.

--- Stacy Kissel, Somerville, Mass.

\begin{center}\rule{0.5\linewidth}{\linethickness}\end{center}

\hypertarget{6-4-pm-real-life-erotic-escapes}{%
\subsubsection{\texorpdfstring{\textbf{6) 4 p.m. Real life, erotic
escapes}}{6) 4 p.m. Real life, erotic escapes}}\label{6-4-pm-real-life-erotic-escapes}}

Start a journal. Use any medium, just do it. Write down your thoughts,
what you're doing in a day, who you're (virtually) talking to, what
you're eating, how you're coping. Think about what's going on in the
world --- how does this make you feel? Journals and diaries are crucial
to historians trying to understand the past --- become part of history
by writing down your experience.

--- Lauren Gray, Lawrence, Kan.

Lean into fantasy! Two days ago I started writing erotica from
quarantine. I offered to send my friends a chapter each week over email.
In the world I'm creating, there's an apocalypse where climaxing is the
only antidote to illness. If you've never tried writing fiction about
desire, I challenge you to it.

--- Annika Berry, Brooklyn, N.Y.

\begin{center}\rule{0.5\linewidth}{\linethickness}\end{center}

\hypertarget{7-7-pm-dinner-and-a-show}{%
\subsubsection{\texorpdfstring{\textbf{7) 7 p.m. Dinner and a
show}}{7) 7 p.m. Dinner and a show}}\label{7-7-pm-dinner-and-a-show}}

Transport yourself to an evening in Italy through the two Franks. First,
lemon pasta inspired by Frank Prisinzano. Enjoyed with a glass of wine
from Francis Ford Coppola's vineyard and, lastly, a showing of his
classic, ``The Godfather.'' Saluti!

--- Taara Sajnani and Abhinav Das, New York, N.Y.

Watch the film ``Before Sunrise'' by Richard Linklater. The movie is
about the spontaneity of travel, the thrill of connecting with someone
from another part of the world, and the deep conversations that are only
had while on the road. It is such a timeless rewatch too.

--- Steven Filie, Chicago, Ill.

Take in the universe. The International Space Station flew over
Portland, Oregon on a clear night and we had six minutes of watching it
stream across the sky. Go to
\href{https://spotthestation.nasa.gov/sightings/}{NASA's Spot the
Station website} to see if there is an observation coming your way.

--- Sue Strater, Portland, Ore.

\begin{center}\rule{0.5\linewidth}{\linethickness}\end{center}

\hypertarget{sunday}{%
\subsection{Sunday}\label{sunday}}

\hypertarget{8-8-am-changing-routines}{%
\subsubsection{\texorpdfstring{\textbf{8) 8 a.m. Changing
routines}}{8) 8 a.m. Changing routines}}\label{8-8-am-changing-routines}}

Treat yourself to a lovely salad for breakfast. One of my favorite
travel memories is a gorgeous
\href{https://cooking.nytimes3xbfgragh.onion/recipes/1018849-classic-caprese-salad}{Caprese
salad} eaten in the sunny dining room of our hotel in Evora, Portugal.
Pair with good coffee and rustic bread. In my experience, people in
Europe wisely see no reason to limit salad to lunch and dinner.

--- Jane Baechle, Albuquerque, N.M.

Make small discoveries. To stretch my legs during the lockdown, I've
been walking around the block every day, and I've started to notice
details that I'd never seen before. Like the fake, painted window on the
building across the road, or the old candle holders that were once used
as part of the street lighting. When the quarantine ends, I hope we
don't forget to appreciate what's been on a doorstep all along.

--- Camilla Capasso, Modena, Italy

\begin{center}\rule{0.5\linewidth}{\linethickness}\end{center}

Image

Credit...Sarah Nisbett

\hypertarget{9-1030-am-use-your-hands}{%
\subsubsection{\texorpdfstring{\textbf{9) 10.30 a.m. Use your
hands}}{9) 10.30 a.m. Use your hands}}\label{9-1030-am-use-your-hands}}

Draw your house. Great monuments and famous buildings have been painted
thousands of times, but how many times have the nooks and crannies of
your apartment been sketched? Let's turn our homes, however small, into
art-worthy inspiration.

--- Josefina Jolly, Buenos Aires, Argentina

Undertake the easiest and most fulfilling origami project of your life
by folding 12 pieces of paper and
\href{https://www.origami-resource-center.com/sonobe-stellated-octahedron.html}{building
this lovely star}. Modular origami has been my absolute favorite
occupational therapy since I was a restless child: the process is
enthralling and soothing.

--- Laila Dib, Berlin, Germany

Draw something that's in front of you. When you make art about the
world, you make the world yours. Doesn't that sound nice?

--- Sarah Nisbett, New York, N.Y.

\begin{center}\rule{0.5\linewidth}{\linethickness}\end{center}

Image

Credit...Elizabeth Yager

\hypertarget{10-12-pm-be-isolated-together}{%
\subsubsection{\texorpdfstring{\textbf{10) 12 p.m. Be isolated,
together}}{10) 12 p.m. Be isolated, together}}\label{10-12-pm-be-isolated-together}}

Check on neighbors on your block or floor with an email, text or phone
call, or leave a card with your name and contact information. Are they
OK? Do they need something from the store? Help with an errand? Food?
Can you bring them a hot dish or home-baked bread? This simple act ---
done carefully and from a safe distance --- palpably reduces our sense
of fear and isolation. I've seen the faces of some neighbors for the
first time. Now they wave.

--- Jim Carrier, Burlington, Vt.

Play a tune on the piano and send the recording to an older friend. I
sent Nonna Augusta, my adopted Italian grandma from the hard-hit town of
Bergamo, a video of me playing ``Ave Maria'' on the piano. Google
Translate helped me let Nonna know in Italian that this was for her
entertainment. Her reaction (translated from Italian): ``Thanks, Mia,
what a thrill. You made me cry. Thank you very much, a beautiful gift
you gave me! Now I listen to it several times.''

--- Mia Gonzales, San Diego, Calif.

Walk to a friend's house and call them from your phone from the street.
They look out their window, and you can see each other in person, using
the phone to hear each other's voices. You are far apart enough to be
safe and it's better than FaceTime.

--- Elizabeth Yager, Brooklyn, N.Y.

Send a postcard. Recently, I signed up with a service called
\href{https://www.postcrossing.com/}{Postcrossing.org}. They match
people around the world with other postcard-loving people. I will write
to someone in Taiwan, and someone from the Netherlands will write to me.
Through the postcards, I get to share my wonderful city and learn about
other people and where they live.

--- Sue Cutsogeorge, Eugene, Ore.

\begin{center}\rule{0.5\linewidth}{\linethickness}\end{center}

\hypertarget{11-3-pm-travel-through-pages}{%
\subsubsection{\texorpdfstring{\textbf{11) 3 p.m. Travel through
pages}}{11) 3 p.m. Travel through pages}}\label{11-3-pm-travel-through-pages}}

Read your way around the world. Helena Attlee's delicious ``The Land
Where Lemons Grow'' is a brilliant example of a
food-meets-history-meets-travel book. Pair with a shot of
\href{https://cooking.nytimes3xbfgragh.onion/recipes/1015304-limoncello-once-removed}{limoncello}.
Salute e forza, Italia!

--- Sergio Pérez Llanos, Santa Cruz de Tenerife, Spain

Explore the United States with William Least Heat-Moon's ``Blue
Highways.'' Published nearly 40 years ago, it retains the power to
transport the reader to the V.F.W. halls, the churches and the
sun-dappled, two-lane roads of the America left behind by interstate
travel and mass commerce. And the people you'll meet along the way are
sufficiently varied, open and engaging to take some of the sting out of
social distancing.

--- Kristin Hawes, Seattle, Wash.

Start a virtual book club. My good friend created a group chat entitled
``Quaranread.'' In a week, we plan to discuss the first chunk of ``The
Poisonwood Bible'' by Barbara Kingsolver over FaceTime. Now is a good
time to travel to new places within ourselves, but a road trip through
one's imagination isn't as exciting if you can't pack other people in
the car.

--- Cameron Edson, Charlottesville, Va.

\begin{center}\rule{0.5\linewidth}{\linethickness}\end{center}

\emph{\textbf{Follow New York Times Travel}}
\emph{on}\href{https://www.instagram.com/nytimestravel/}{\emph{Instagram}}\emph{,}\href{https://twitter.com/nytimestravel}{\emph{Twitter}}
\emph{and}\href{https://www.facebookcorewwwi.onion/nytimestravel/}{\emph{Facebook}}\emph{.
And}\href{https://www.nytimes3xbfgragh.onion/newsletters/traveldispatch}{\emph{sign
up for our weekly Travel Dispatch newsletter}} \emph{to receive expert
tips on traveling smarter and inspiration for your next vacation.
Dreaming up a future getaway or just armchair traveling? Check out
our}\href{https://www.nytimes3xbfgragh.onion/interactive/2020/travel/places-to-visit.html}{\emph{52
Places list}}\emph{.}

Advertisement

\protect\hyperlink{after-bottom}{Continue reading the main story}

\hypertarget{site-index}{%
\subsection{Site Index}\label{site-index}}

\hypertarget{site-information-navigation}{%
\subsection{Site Information
Navigation}\label{site-information-navigation}}

\begin{itemize}
\tightlist
\item
  \href{https://help.nytimes3xbfgragh.onion/hc/en-us/articles/115014792127-Copyright-notice}{©~2020~The
  New York Times Company}
\end{itemize}

\begin{itemize}
\tightlist
\item
  \href{https://www.nytco.com/}{NYTCo}
\item
  \href{https://help.nytimes3xbfgragh.onion/hc/en-us/articles/115015385887-Contact-Us}{Contact
  Us}
\item
  \href{https://www.nytco.com/careers/}{Work with us}
\item
  \href{https://nytmediakit.com/}{Advertise}
\item
  \href{http://www.tbrandstudio.com/}{T Brand Studio}
\item
  \href{https://www.nytimes3xbfgragh.onion/privacy/cookie-policy\#how-do-i-manage-trackers}{Your
  Ad Choices}
\item
  \href{https://www.nytimes3xbfgragh.onion/privacy}{Privacy}
\item
  \href{https://help.nytimes3xbfgragh.onion/hc/en-us/articles/115014893428-Terms-of-service}{Terms
  of Service}
\item
  \href{https://help.nytimes3xbfgragh.onion/hc/en-us/articles/115014893968-Terms-of-sale}{Terms
  of Sale}
\item
  \href{https://spiderbites.nytimes3xbfgragh.onion}{Site Map}
\item
  \href{https://help.nytimes3xbfgragh.onion/hc/en-us}{Help}
\item
  \href{https://www.nytimes3xbfgragh.onion/subscription?campaignId=37WXW}{Subscriptions}
\end{itemize}
