\href{/section/health}{Health}\textbar{}Can an Old Vaccine Stop the New
Coronavirus?

\url{https://nyti.ms/2JCcTxx}

\begin{itemize}
\item
\item
\item
\item
\item
\end{itemize}

\hypertarget{the-coronavirus-outbreak}{%
\subsubsection{\texorpdfstring{\href{https://www.nytimes3xbfgragh.onion/news-event/coronavirus?name=styln-coronavirus-national\&region=TOP_BANNER\&block=storyline_menu_recirc\&action=click\&pgtype=Article\&impression_id=d75aa960-efba-11ea-9838-db61c3659d7e\&variant=undefined}{The
Coronavirus
Outbreak}}{The Coronavirus Outbreak}}\label{the-coronavirus-outbreak}}

\begin{itemize}
\tightlist
\item
  live\href{https://www.nytimes3xbfgragh.onion/2020/09/05/world/coronavirus-covid.html?name=styln-coronavirus-national\&region=TOP_BANNER\&block=storyline_menu_recirc\&action=click\&pgtype=Article\&impression_id=d75ad070-efba-11ea-9838-db61c3659d7e\&variant=undefined}{Latest
  Updates}
\item
  \href{https://www.nytimes3xbfgragh.onion/interactive/2020/us/coronavirus-us-cases.html?name=styln-coronavirus-national\&region=TOP_BANNER\&block=storyline_menu_recirc\&action=click\&pgtype=Article\&impression_id=d75ad071-efba-11ea-9838-db61c3659d7e\&variant=undefined}{Maps
  and Cases}
\item
  \href{https://www.nytimes3xbfgragh.onion/interactive/2020/science/coronavirus-vaccine-tracker.html?name=styln-coronavirus-national\&region=TOP_BANNER\&block=storyline_menu_recirc\&action=click\&pgtype=Article\&impression_id=d75ad072-efba-11ea-9838-db61c3659d7e\&variant=undefined}{Vaccine
  Tracker}
\item
  \href{https://www.nytimes3xbfgragh.onion/2020/09/02/your-money/eviction-moratorium-covid.html?name=styln-coronavirus-national\&region=TOP_BANNER\&block=storyline_menu_recirc\&action=click\&pgtype=Article\&impression_id=d75ad073-efba-11ea-9838-db61c3659d7e\&variant=undefined}{Eviction
  Moratorium}
\item
  \href{https://www.nytimes3xbfgragh.onion/interactive/2020/09/02/magazine/food-insecurity-hunger-us.html?name=styln-coronavirus-national\&region=TOP_BANNER\&block=storyline_menu_recirc\&action=click\&pgtype=Article\&impression_id=d75ad074-efba-11ea-9838-db61c3659d7e\&variant=undefined}{American
  Hunger}
\end{itemize}

\includegraphics{https://static01.graylady3jvrrxbe.onion/images/2020/04/05/science/05Virus-BCG/merlin_171086958_8115b8bf-a794-44a0-8d58-7dc366d60a20-articleLarge.jpg?quality=75\&auto=webp\&disable=upscale}

Sections

\protect\hyperlink{site-content}{Skip to
content}\protect\hyperlink{site-index}{Skip to site index}

\hypertarget{can-an-old-vaccine-stop-the-new-coronavirus}{%
\section{Can an Old Vaccine Stop the New
Coronavirus?}\label{can-an-old-vaccine-stop-the-new-coronavirus}}

A tuberculosis vaccine invented a century ago is cheap and safe, and
seems to bolster the body's immune system.

Dr. Denise Faustman, director of immunobiology at Massachusetts General
Hospital in Boston, has been studying a tuberculosis vaccine, known by
its acronym, B.C.G., and its effect on Type 1 diabetes for many
years.Credit...Tony Luong for The New York Times

Supported by

\protect\hyperlink{after-sponsor}{Continue reading the main story}

\href{https://www.nytimes3xbfgragh.onion/by/roni-caryn-rabin}{\includegraphics{https://static01.graylady3jvrrxbe.onion/images/2018/02/20/multimedia/author-roni-caryn-rabin/author-roni-caryn-rabin-thumbLarge-v3.png}}

By \href{https://www.nytimes3xbfgragh.onion/by/roni-caryn-rabin}{Roni
Caryn Rabin}

\begin{itemize}
\item
  Published April 3, 2020Updated April 5, 2020
\item
  \begin{itemize}
  \item
  \item
  \item
  \item
  \item
  \end{itemize}
\end{itemize}

\href{https://www.nytimes3xbfgragh.onion/es/2020/04/14/espanol/vacuna-coronavirus.html}{Leer
en español}

A vaccine that was developed a hundred years ago to fight the
tuberculosis scourge in Europe is now being tested against the
coronavirus by scientists eager to find a quick way to protect health
care workers, among others.

The Bacillus Calmette-Guerin vaccine is still widely used in the
developing world, where scientists have found that it does more than
prevent TB. The vaccine prevents infant deaths from a variety of causes,
and sharply reduces the incidence of respiratory infections.

The vaccine seems to ``train'' the immune system to recognize and
respond to a variety of infections, including viruses, bacteria and
parasites, experts say. There is little evidence yet that the vaccine
will blunt infection with the coronavirus, but a series of clinical
trials may answer the question in just months.

On Monday, scientists in Melbourne, Australia, started administering the
B.C.G. vaccine or a placebo to thousands of physicians, nurses,
respiratory therapists and other health care workers --- the first of
several randomized controlled trials intended to test the vaccine's
effectiveness against the coronavirus.

``Nobody is saying this is a panacea,'' said Nigel Curtis, an infectious
diseases researcher at the University of Melbourne and Murdoch
Children's Research Institute, who planned the trial. ``What we want to
do is reduce the time an infected health care worker is unwell, so they
recover and can come back to work faster.''

A clinical trial of 1,000 health care workers began 10 days ago in the
Netherlands, said Dr. Mihai G. Netea, an infectious disease specialist
at Radboud University Medical Center in Nijmegen. Eight hundred health
care workers have already signed up. (As in Australia, half of the
participants will receive a placebo.)

Dr. Denise Faustman, director of immunobiology at Massachusetts General
Hospital, is seeking funding to start a clinical trial of the vaccine in
health care workers in Boston as well. Preliminary results could be
available in as little as four months.

``We have really strong data from clinical trials with humans --- not
mice --- that this vaccine protects you from viral and parasitic
infections,'' said Dr. Faustman. ``I'd like to start today.''

\includegraphics{https://static01.graylady3jvrrxbe.onion/images/2020/04/03/science/03VIRUS-BCG2/merlin_171254466_6a54cd11-d3ae-49a2-a170-25c2b885def5-articleLarge.jpg?quality=75\&auto=webp\&disable=upscale}

The B.C.G. vaccine has an unusual history. It was inspired in the 1800s
by the observation that milkmaids did not develop tuberculosis. The
vaccine is named after its inventors, Dr. Albert Calmette and Dr.
Camille Guerin, who developed it in the early 1900s from mycobacterium
bovis, a form of tuberculosis that infects cattle.

The scientists cultured bacterial scrapings from cow udders, and
continued to culture bovine TB for over a decade until it was weak
enough that it no longer caused virulent disease when given to lab
animals.

\hypertarget{latest-updates-the-coronavirus-outbreak}{%
\section{\texorpdfstring{\href{https://www.nytimes3xbfgragh.onion/2020/09/04/world/covid-19-coronavirus.html?action=click\&pgtype=Article\&state=default\&region=MAIN_CONTENT_1\&context=storylines_live_updates}{Latest
Updates: The Coronavirus
Outbreak}}{Latest Updates: The Coronavirus Outbreak}}\label{latest-updates-the-coronavirus-outbreak}}

Updated 2020-09-05T12:05:40.998Z

\begin{itemize}
\tightlist
\item
  \href{https://www.nytimes3xbfgragh.onion/2020/09/04/world/covid-19-coronavirus.html?action=click\&pgtype=Article\&state=default\&region=MAIN_CONTENT_1\&context=storylines_live_updates\#link-1654f6ad}{Research
  connects vaping to a higher chance of catching the virus --- and
  suffering its worst effects.}
\item
  \href{https://www.nytimes3xbfgragh.onion/2020/09/04/world/covid-19-coronavirus.html?action=click\&pgtype=Article\&state=default\&region=MAIN_CONTENT_1\&context=storylines_live_updates\#link-52e4198a}{Another
  college football game won't be played as planned.}
\item
  \href{https://www.nytimes3xbfgragh.onion/2020/09/04/world/covid-19-coronavirus.html?action=click\&pgtype=Article\&state=default\&region=MAIN_CONTENT_1\&context=storylines_live_updates\#link-181cef0}{Pharmaceutical
  companies plan a joint pledge on safety standards as they move
  vaccines to the marketplace.}
\end{itemize}

\href{https://www.nytimes3xbfgragh.onion/2020/09/04/world/covid-19-coronavirus.html?action=click\&pgtype=Article\&state=default\&region=MAIN_CONTENT_1\&context=storylines_live_updates}{See
more updates}

More live coverage:
\href{https://www.nytimes3xbfgragh.onion/live/2020/09/04/business/stock-market-today-coronavirus?action=click\&pgtype=Article\&state=default\&region=MAIN_CONTENT_1\&context=storylines_live_updates}{Markets}

The vaccine was first used in humans in 1921 and was widely adopted
after World War II. Now B.C.G. is primarily used in the developing world
and in countries where TB is still prevalent, where it is given to over
100 million babies a year.

\textbf{\emph{{[}}\href{http://on.fb.me/1paTQ1h}{\emph{Like the Science
Times page on Facebook.}}} ****** \emph{\textbar{} Sign up for the}
\textbf{\href{http://nyti.ms/1MbHaRU}{\emph{Science Times
newsletter.}}\emph{{]}}}

Like other vaccines, B.C.G. has a specific target: TB. But evidence
accumulating over the past decade suggests the vaccine also has
so-called off-target effects, reducing viral illnesses, respiratory
infections and sepsis, and appears to bolster the body's immune system.

The idea is an offshoot of the ``hygiene hypothesis,'' which suggests
that the modern emphasis on cleanliness has deprived children of
exposure to germs. The lack of ``training'' has resulted in weakened
immune systems, less able to resist disease.

One of the earliest studies hinting at the broad benefits of B.C.G.
vaccination was a
\href{https://www.ncbi.nlm.nih.gov/pubmed/21673035}{randomized trial of
2,320 babies in Guinea-Bissau in West Africa,}published in 2011, that
reported that death rates among low-birth-weight babies were
dramatically reduced after vaccination. A follow-up trial reported that
\href{https://www.ncbi.nlm.nih.gov/pubmed/29579158}{infectious-disease
mortality rates in low-birth-weight babies who were vaccinated were cut
by more than 40 percent.}

Other epidemiological studies --- including a 25-year study of over
150,000 children in 33 countries --- have reported a 40 percent
\href{https://pubmed.ncbi.nlm.nih.gov/25725054/?from_single_result=2015\%2C+de+castro+mj\%2C+clinical+infectious+disease\&expanded_search_query=2015\%2C+de+castro+mj\%2C+clinical+infectious+disease}{lower
risk of acute lower respiratory tract infections} in children who
received a B.C.G. vaccine.

A study
\href{https://www.ncbi.nlm.nih.gov/pubmed/?term=wardhana\%2C+BCG\%2C+elderly}{in
the elderly} found that consecutive B.C.G. vaccinations reduced the
incidence of acute upper respiratory tract infections.

\href{https://www.nytimes3xbfgragh.onion/news-event/coronavirus?action=click\&pgtype=Article\&state=default\&region=MAIN_CONTENT_3\&context=storylines_faq}{}

\hypertarget{the-coronavirus-outbreak-}{%
\subsubsection{The Coronavirus Outbreak
›}\label{the-coronavirus-outbreak-}}

\hypertarget{frequently-asked-questions}{%
\paragraph{Frequently Asked
Questions}\label{frequently-asked-questions}}

Updated September 4, 2020

\begin{itemize}
\item ~
  \hypertarget{what-are-the-symptoms-of-coronavirus}{%
  \paragraph{What are the symptoms of
  coronavirus?}\label{what-are-the-symptoms-of-coronavirus}}

  \begin{itemize}
  \tightlist
  \item
    In the beginning, the coronavirus
    \href{https://www.nytimes3xbfgragh.onion/article/coronavirus-facts-history.html?action=click\&pgtype=Article\&state=default\&region=MAIN_CONTENT_3\&context=storylines_faq\#link-6817bab5}{seemed
    like it was primarily a respiratory illness}~--- many patients had
    fever and chills, were weak and tired, and coughed a lot, though
    some people don't show many symptoms at all. Those who seemed
    sickest had pneumonia or acute respiratory distress syndrome and
    received supplemental oxygen. By now, doctors have identified many
    more symptoms and syndromes. In April,
    \href{https://www.nytimes3xbfgragh.onion/2020/04/27/health/coronavirus-symptoms-cdc.html?action=click\&pgtype=Article\&state=default\&region=MAIN_CONTENT_3\&context=storylines_faq}{the
    C.D.C. added to the list of early signs}~sore throat, fever, chills
    and muscle aches. Gastrointestinal upset, such as diarrhea and
    nausea, has also been observed. Another telltale sign of infection
    may be a sudden, profound diminution of one's
    \href{https://www.nytimes3xbfgragh.onion/2020/03/22/health/coronavirus-symptoms-smell-taste.html?action=click\&pgtype=Article\&state=default\&region=MAIN_CONTENT_3\&context=storylines_faq}{sense
    of smell and taste.}~Teenagers and young adults in some cases have
    developed painful red and purple lesions on their fingers and toes
    --- nicknamed ``Covid toe'' --- but few other serious symptoms.
  \end{itemize}
\item ~
  \hypertarget{why-is-it-safer-to-spend-time-together-outside}{%
  \paragraph{Why is it safer to spend time together
  outside?}\label{why-is-it-safer-to-spend-time-together-outside}}

  \begin{itemize}
  \tightlist
  \item
    \href{https://www.nytimes3xbfgragh.onion/2020/05/15/us/coronavirus-what-to-do-outside.html?action=click\&pgtype=Article\&state=default\&region=MAIN_CONTENT_3\&context=storylines_faq}{Outdoor
    gatherings}~lower risk because wind disperses viral droplets, and
    sunlight can kill some of the virus. Open spaces prevent the virus
    from building up in concentrated amounts and being inhaled, which
    can happen when infected people exhale in a confined space for long
    stretches of time, said Dr. Julian W. Tang, a virologist at the
    University of Leicester.
  \end{itemize}
\item ~
  \hypertarget{why-does-standing-six-feet-away-from-others-help}{%
  \paragraph{Why does standing six feet away from others
  help?}\label{why-does-standing-six-feet-away-from-others-help}}

  \begin{itemize}
  \tightlist
  \item
    The coronavirus spreads primarily through droplets from your mouth
    and nose, especially when you cough or sneeze. The C.D.C., one of
    the organizations using that measure,
    \href{https://www.nytimes3xbfgragh.onion/2020/04/14/health/coronavirus-six-feet.html?action=click\&pgtype=Article\&state=default\&region=MAIN_CONTENT_3\&context=storylines_faq}{bases
    its recommendation of six feet}~on the idea that most large droplets
    that people expel when they cough or sneeze will fall to the ground
    within six feet. But six feet has never been a magic number that
    guarantees complete protection. Sneezes, for instance, can launch
    droplets a lot farther than six feet,
    \href{https://jamanetwork.com/journals/jama/fullarticle/2763852}{according
    to a recent study}. It's a rule of thumb: You should be safest
    standing six feet apart outside, especially when it's windy. But
    keep a mask on at all times, even when you think you're far enough
    apart.
  \end{itemize}
\item ~
  \hypertarget{i-have-antibodies-am-i-now-immune}{%
  \paragraph{I have antibodies. Am I now
  immune?}\label{i-have-antibodies-am-i-now-immune}}

  \begin{itemize}
  \tightlist
  \item
    As of right
    now,\href{https://www.nytimes3xbfgragh.onion/2020/07/22/health/covid-antibodies-herd-immunity.html?action=click\&pgtype=Article\&state=default\&region=MAIN_CONTENT_3\&context=storylines_faq}{~that
    seems likely, for at least several months.}~There have been
    frightening accounts of people suffering what seems to be a second
    bout of Covid-19. But experts say these patients may have a
    drawn-out course of infection, with the virus taking a slow toll
    weeks to months after initial exposure.~People infected with the
    coronavirus typically
    \href{https://www.nature.com/articles/s41586-020-2456-9}{produce}~immune
    molecules called antibodies, which are
    \href{https://www.nytimes3xbfgragh.onion/2020/05/07/health/coronavirus-antibody-prevalence.html?action=click\&pgtype=Article\&state=default\&region=MAIN_CONTENT_3\&context=storylines_faq}{protective
    proteins made in response to an
    infection}\href{https://www.nytimes3xbfgragh.onion/2020/05/07/health/coronavirus-antibody-prevalence.html?action=click\&pgtype=Article\&state=default\&region=MAIN_CONTENT_3\&context=storylines_faq}{.
    These antibodies may}~last in the body
    \href{https://www.nature.com/articles/s41591-020-0965-6}{only two to
    three months}, which may seem worrisome, but that's~perfectly normal
    after an acute infection subsides, said Dr. Michael Mina, an
    immunologist at Harvard University. It may be possible to get the
    coronavirus again, but it's highly unlikely that it would be
    possible in a short window of time from initial infection or make
    people sicker the second time.
  \end{itemize}
\item ~
  \hypertarget{what-are-my-rights-if-i-am-worried-about-going-back-to-work}{%
  \paragraph{What are my rights if I am worried about going back to
  work?}\label{what-are-my-rights-if-i-am-worried-about-going-back-to-work}}

  \begin{itemize}
  \tightlist
  \item
    Employers have to provide
    \href{https://www.osha.gov/SLTC/covid-19/standards.html}{a safe
    workplace}~with policies that protect everyone equally.
    \href{https://www.nytimes3xbfgragh.onion/article/coronavirus-money-unemployment.html?action=click\&pgtype=Article\&state=default\&region=MAIN_CONTENT_3\&context=storylines_faq}{And
    if one of your co-workers tests positive for the coronavirus, the
    C.D.C.}~has said that
    \href{https://www.cdc.gov/coronavirus/2019-ncov/community/guidance-business-response.html}{employers
    should tell their employees}~-\/- without giving you the sick
    employee's name -\/- that they may have been exposed to the virus.
  \end{itemize}
\end{itemize}

A recent
\href{https://www.who.int/immunization/sage/meetings/2017/october/1_BCG_report_revised_version_online.pdf}{review
by the World Health Organization concluded that B.C.G. had beneficial
``off-target effects},'' and recommended doing more trials of the
vaccine against a wider range of infections.

``This vaccine has saved as many lives as the polio vaccine --- it's an
amazing story,'' said Dr. Curtis, who designed and launched the B.C.G.
trial in Melbourne in less than a month, hoping to stay one step ahead
of the coronavirus's spread in Australia.

While he described the B.C.G. vaccine as underappreciated, he emphasized
that it was ``not a specific Covid-19 vaccine.'' B.C.G. also cannot be
administered to anyone who has a compromised immune system, because it
is a live-attenuated vaccine --- meaning it contains live but weakened
TB.

Dr. Faustman said it should not be used in hospitalized patients with
active disease, because it may not work fast enough and could interact
poorly with other treatments.

Not everyone is convinced B.C.G. holds much promise. Dr. Domenico
Accili, an endocrinologist at Columbia University, said he thought
efforts to use the vaccine against the coronavirus sound ``a bit like
magical thinking.''

While acknowledging that B.C.G. is ``a non-specific booster of the
immune system,'' he said, ``we should be able to deploy a more tailored
approach.''

One question is what effect the vaccine may have in patients whose
immune systems overreact to the coronavirus, resulting in what are
called cytokine storms. Dr. Randy Cron, an expert on cytokine storms at
the University of Alabama at Birmingham, said it was impossible to know.

A
\href{https://www.medrxiv.org/content/10.1101/2020.03.24.20042937v1}{recent
analysis of the disparate toll the new coronavirus} has taken on middle-
and high-income countries found a correlation with B.C.G. policies,
concluding that countries that did not implement or had abandoned
universal B.C.G. vaccination have had more coronavirus infections per
capita and higher death rates. (Low-income countries were excluded from
the analysis because of unreliable Covid-19 reporting data and generally
poor medical systems.)

``You can make a new vaccine,'' Dr. Faustman said. ``We're really smart,
and we can do that. But it's two years off, and two years is going to be
two years too late.''

``If we've got something generic globally at hand that we can use to
make the human host stronger, this is a win-win for the public right
away.''

Advertisement

\protect\hyperlink{after-bottom}{Continue reading the main story}

\hypertarget{site-index}{%
\subsection{Site Index}\label{site-index}}

\hypertarget{site-information-navigation}{%
\subsection{Site Information
Navigation}\label{site-information-navigation}}

\begin{itemize}
\tightlist
\item
  \href{https://help.nytimes3xbfgragh.onion/hc/en-us/articles/115014792127-Copyright-notice}{©~2020~The
  New York Times Company}
\end{itemize}

\begin{itemize}
\tightlist
\item
  \href{https://www.nytco.com/}{NYTCo}
\item
  \href{https://help.nytimes3xbfgragh.onion/hc/en-us/articles/115015385887-Contact-Us}{Contact
  Us}
\item
  \href{https://www.nytco.com/careers/}{Work with us}
\item
  \href{https://nytmediakit.com/}{Advertise}
\item
  \href{http://www.tbrandstudio.com/}{T Brand Studio}
\item
  \href{https://www.nytimes3xbfgragh.onion/privacy/cookie-policy\#how-do-i-manage-trackers}{Your
  Ad Choices}
\item
  \href{https://www.nytimes3xbfgragh.onion/privacy}{Privacy}
\item
  \href{https://help.nytimes3xbfgragh.onion/hc/en-us/articles/115014893428-Terms-of-service}{Terms
  of Service}
\item
  \href{https://help.nytimes3xbfgragh.onion/hc/en-us/articles/115014893968-Terms-of-sale}{Terms
  of Sale}
\item
  \href{https://spiderbites.nytimes3xbfgragh.onion}{Site Map}
\item
  \href{https://help.nytimes3xbfgragh.onion/hc/en-us}{Help}
\item
  \href{https://www.nytimes3xbfgragh.onion/subscription?campaignId=37WXW}{Subscriptions}
\end{itemize}
