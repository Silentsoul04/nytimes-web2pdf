Sections

SEARCH

\protect\hyperlink{site-content}{Skip to
content}\protect\hyperlink{site-index}{Skip to site index}

\href{https://www.nytimes3xbfgragh.onion/section/opinion/sunday}{Sunday
Review}

\href{https://myaccount.nytimes3xbfgragh.onion/auth/login?response_type=cookie\&client_id=vi}{}

\href{https://www.nytimes3xbfgragh.onion/section/todayspaper}{Today's
Paper}

\href{/section/opinion/sunday}{Sunday Review}\textbar{}Of the Virus and
God, Orange Peels and the Party

\url{https://nyti.ms/2wYtxoj}

\begin{itemize}
\item
\item
\item
\item
\item
\end{itemize}

Advertisement

\protect\hyperlink{after-top}{Continue reading the main story}

\href{/section/opinion}{Opinion}

Supported by

\protect\hyperlink{after-sponsor}{Continue reading the main story}

\hypertarget{of-the-virus-and-god-orange-peels-and-the-party}{%
\section{Of the Virus and God, Orange Peels and the
Party}\label{of-the-virus-and-god-orange-peels-and-the-party}}

A mother and daughter write through distance and disease, searching for
the other's love.

By Yangyang Cheng

Ms. Cheng is a particle physicist.

\begin{itemize}
\item
  April 3, 2020
\item
  \begin{itemize}
  \item
  \item
  \item
  \item
  \item
  \end{itemize}
\end{itemize}

\includegraphics{https://static01.graylady3jvrrxbe.onion/images/2020/04/05/opinion/sunday/05cheng/05cheng-articleLarge.jpg?quality=75\&auto=webp\&disable=upscale}

\href{https://cn.nytimes3xbfgragh.onion/opinion/20200410/coronavirus-china-us/}{阅读简体中文版}\href{https://cn.nytimes3xbfgragh.onion/opinion/20200410/coronavirus-china-us/zh-hant/}{閱讀繁體中文版}

CHICAGO --- My mother believes that God and the Chinese Communist Party
will defeat the novel coronavirus.

``Pray for Wuhan. Pray for China,'' she urges me, referring to the
capital of Hubei Province, where the outbreak started. It is early
February, a week and some since Wuhan was put under lockdown. My mother
lives in our hometown in a neighboring province, and like most places in
China, her city has enacted quarantine measures. But she is relatively
safe there, and knowing that brings me selfish reassurance as I watch
the crisis unfold throughout China: I am her only child and live on the
other side of the planet, which is still barely touched by the
coronavirus.

Every morning since late January, I have woken up in Chicago to a string
of messages from my mother. The emails and texts continue through
lunchtime; occasionally they pop up in the afternoon, and I know it's
been another sleepless night for her.

My mother forwards me reports from Chinese state media about how the
government is taking swift action to combat the epidemic. She sends me
screenshots of conversations with friends, as they discuss life under
quarantine and how to convince unruly family members to stay inside.

She quotes passages from the Bible and shares prayer verses from her
church. With indoor gatherings suspended, the state-sanctioned church my
mother frequents has moved its services online. The pastor preaches via
WeChat, the almighty Chinese messaging app, which the parishioners also
use to check on one another.

My mother is a retired elementary schoolteacher. I was one of her
students. She taught me how to sing the national anthem and how to tie
my red neckerchief after I joined the Young Pioneers, the Party's youth
organization. As a child in the late 1990s, I sat in her classroom day
after day, the lessons about Chinese characters doubling as an
introduction to history and civics. There was never any doubt in my mind
that my mother believed wholeheartedly what she lectured, and for a long
while I believed it as well.

The government-issued textbooks, simple as they were, contained all the
answers about how to tell the good from the bad and what gives life its
meaning. The Party is good; disobeying it is bad. Serving one's country
and its people is the most noble form of living.

But for all its tales of revolutionary martyrs, patriotic education does
not teach how to grieve. After the sudden death of my father 20 years
ago, my mother started going to church. She put copies of the Bible
under both of our pillows, and carried the holy book with her whenever
we took a trip. I was 10 and ambivalent about the existence of any
deity, but I joined her every night for bedtime prayer. It was not
optional.

I left China in the summer of 2009 for graduate school in the United
States, where I continue to live and work. In the solitude of our old
home, my mother has become increasingly devout. Speaking to God eases
her longings.

In recent years, the Chinese government has tightened its authoritarian
grip, cracking down on Muslim minorities and underground churches that
operate outside the purview of the Party. I sometimes wonder if there
will be a time when my mother is forced to choose between her God and
her Party. Has the possibility crossed her mind?

My mother's generation came of age when every belief imported was deemed
a heresy and every practice inherited was banished as superstition. As
China emerged from the abyss of political fanaticism, the Party learned
a valuable lesson: In the face of humanly desires that cannot be
extinguished, be they about the pursuit of material wealth or the need
for spiritual comfort, it is more effective to co-opt than to outlaw.

A communion in the privacy of a home is suspect because such organizing
capability can be used for political purposes. But a grand church in the
heart of a city, like the one my mother attends, knows how to stay in
the government's good graces.

My mother sees no conflict between her religious faith and her political
loyalty. ``Every government wants what's best for its people,'' she
says. ``And every person of a certain age believes in something.''

\begin{center}\rule{0.5\linewidth}{\linethickness}\end{center}

\textbf{The first cases of Covid-19} were reported in Wuhan in late
December, and human-to-human transmission was confirmed in early
January. But for fear of social disruption and political blowback,
Chinese officials censored the information. Revelation of the cover-up
ignited a firestorm. Many people voiced their anger online, demanding
transparency and accountability.

Maybe this will be a teaching moment, I think to myself as I scroll
through Chinese social media: Hashtags like \#Iwantfreespeech were
trending in the early days of the lockdown. I am under no illusion that
the popular discontent of the day will translate into a broader
political awakening, but with a target group of one I might be able to
make some progress, and help my mother realize that the government she
worships is not beyond reproach.

When I list the evidence of deception and mismanagement by the
authorities at the beginning of the outbreak, my mother explains it all
away. The government did not hide anything; the virus has a two-week
incubation period. The lockdown did not come too late; it gave people
time to go back to their hometowns. Residents of Wuhan welcome it;
inside the city, life goes on as usual.

I feel incredulous about how my mother toes the party line. She senses
my exasperation. ``It's not your fault,'' she says, in a voice so gentle
it almost sounds foreign. ``You have been away for too long. You are
misled by Western media.''

My mother only reads Chinese and does not know how to scale the Great
Firewall. But even if she and I consumed the same information, we would
still have opposite reactions.

My mother watches the rapid construction of makeshift hospitals in
Wuhan, and applauds Chinese efficiency; I worry about safety being
compromised to meet an overhyped deadline. My mother sees roadside
checkpoints and neighborhood patrols sprouting up overnight, and praises
the government's thoroughness; I wonder how much the state has
appropriated a public health crisis to expand its surveillance powers.

The mobilization of armed forces to the aid of Wuhan reaffirms my
mother's worship of the military: There was a time in her youth when she
dreamed of putting on a uniform herself. I view any state tool of
violence with deep skepticism and believe the people would be better
served if the resources were spent on education and health care.

``Have you considered the possibility that a government can use its
powers for harm, including against its own people?'' I write to my
mother.

``You think very thoroughly and ask interesting questions,'' she
responds. ``God punishes those who do bad things. If the people listen
to God, they will be protected.''

I can't tell if she is invoking God as the ultimate check on state power
or if she is implying that the state, like God, cannot be questioned.

``Don't you remember what happened outside Tiananmen Square 30 years
ago?'' I feel the words burning at my fingertips, but I refrain from
typing. The subject is forbidden in China. Instead, I raise the example
of Nazi Germany, the dangers of unchecked state power and the complicity
of ordinary people.

``If you have time, read some Hannah Arendt,'' I suggest. ``Her books
are translated and not hard to find.''

My mother writes back to confirm the Chinese names of Arendt's titles.
``If my daughter recommends them, I will definitely read. You have a
doctorate; I have never been to university.''

\begin{center}\rule{0.5\linewidth}{\linethickness}\end{center}

\includegraphics{https://static01.graylady3jvrrxbe.onion/images/2020/04/05/opinion/sunday/05cheng-02/05cheng-02-articleLarge.jpg?quality=75\&auto=webp\&disable=upscale}

\textbf{My mother tells me} she made a stir-fry using only orange peels.
It's mid-February. She has not left her apartment for over two weeks,
and is running low on vegetables.

``I searched online. They are rich in vitamins!'' She congratulates
herself on her resourcefulness: ``I'm so smart. It's too dangerous to go
outside.''

I am gripped with guilt. I realize that I have not asked my mother how
she's doing. I monitor the case count in our hometown and calculate the
likelihood that she will get infected. I deduce from the number and
length of her daily messages her most probable physical state: She
appears rather energetic since she's spending so much time online! I
tell myself that she resides in a safe neighborhood with plenty of
shops, that she has friends and family nearby, that she has her church
and her support groups.

I assuage my conscience by thinking that I'm being rational, and
respecting her autonomy. I have turned our daily communications about
the epidemic into an ad hoc lesson on philosophy and governance*.* But
maybe I'm resorting to logic, math and argument because I dare not
ponder the prospect of my mother in decline, the inevitable curse of
time.

My mother's unwavering faith in the higher powers unsettles me; it
suggests resignation. She believes. She repeats what she's told. She has
packed away her rage and dissatisfaction.

For as far back as I can remember, my mother was always angry. She was
angry at her parents for favoring their sons. She was angry at her
brothers because they were doted on. She was angry at my father when he
was alive and after his death was angry at herself for not cherishing
the time she had had with him. She was angry at the bullies in her
workplace, the rowdy students in her class, the street vendor who
overcharged her for produce. She was angry at me, for every reason and
for no reason at all.

I never gave in to her anger, but I left home as soon as I could. I put
an ocean and two borders between myself and my mother's wrath.

Not long after I moved to the United States, she began asking for my
forgiveness. She credited God for opening her eyes to her sins and
apologized profusely for the ways she had treated me. ``Make me the
outlet for all your displeasure!'' she urged. ``Pour into me all the
dirty water, all the vile words!'' I do not think my mother believes in
revenge as a form of justice, but I recognize the self-hatred in her
plea for punishment: She had to detest herself so much to repeatedly
hurt what she treasures the most, her only child.

For years, my mother has expressed the wish to come live with me. That's
unrealistic, I have told her. She does not speak the language and has no
friends in America. I'm a junior academic, my work is unstable, the
hours are long. None of this matters, my mother has said: As long as
she's with me, she'll be content. She will cook and clean for me. She
has her savings and her pension. She will not be a burden. She only
wants to help. This is her attempt at unconditional love: Anything that
I am and you have use for is yours.

But a mother is not a maid, I have tried to explain. Emotional
dependence is unhealthy. Please, develop a hobby. Please, live for
yourself. I know my mother hears my suggestions as the ultimate
rejection --- that she should find her own life because it is no longer
a part of mine.

For the better part of a year, my mother has been paying
uncharacteristically close attention to world affairs, not so much out
of newly found interest, but as another attempt to connect with me and
correct my wrong-think. I have been writing regularly on Chinese
politics and society for English-language publications, often
criticizing the government's abuses. Knowing how it pressures its
critics and their relatives, I have never mentioned my articles to my
mother: She cannot read them anyway; the language barrier, as well as
our physical distance, should be her protection.

Yet somehow she has found out. My seditious writing has created a giant
negative space between us. We do not speak about it explicitly. But my
mother brings up the topics I write about and presses on with her views,
always aligned with the government's. I push back. Each time I poke
holes in her arguments and challenge her value system, a part of me
craves that if I rub her senses close enough, there will be a new
kindling.

I miss our old fights, not for the wounds they inflicted, but for the
woman I remember and am afraid of losing, the indomitable force who
never settled. I see my mother's submissiveness today as foreboding
decay, like a rock that loses its edges before crumbling to gravel.

``You are a good child with a strong sense of justice,'' my mother
writes when I tell her that I worry about the old, the poor and the
disabled in Wuhan under lockdown. Then she attributes my grievances
about state oppression to the oppressive ways she raised me, and writes
that my political disobedience is little more than a child's rebellion
against a parent.

I am irritated by her suggestion. ``It is not always about you!'' I
type. I look at the words, followed by the flashing cursor, on the
screen and flinch at their cruelty. I hit ``send'' anyway.

\begin{center}\rule{0.5\linewidth}{\linethickness}\end{center}

\textbf{By the beginning of March}, Covid-19 is becoming a global
pandemic. With the outbreak growing in the United States, my mother asks
if I have enough face masks. ``I just learned of this website from a
former student's father. His son is studying in the U.S.'' She types out
its name: A-M-A-Z-O-N.

``You must buy as many masks as you can,'' she emphasizes. ``Gloves too.
If they are sold out, let me know and I will send them to you.''

The Chinese government's draconian methods have halted the virus's
spread in China, but not before it reached foreign shores, where the
authorities and the public have been slow to react.

My mother feels fully vindicated in her steadfast support for the Party.
``Freedom, democracy, human rights: They are all lies! Nothing compares
with the need to stay alive.''

She now spends her days sifting through the internet for the latest
developments outside China, about the mounting number of cases and
Western governments' messy responses. She drills on about her
tried-and-true tips for surviving the epidemic, including what she wore
when she finally went shopping for groceries: two layers of masks, two
layers of gloves, two layers of plastic bags over her shoes, sunglasses
because she does not have goggles and a hooded overcoat with its collar
clasped tight. The security guard at her residential compound
complimented her effort, my mother tells me, giddy: ``He said his family
should learn from me.''

Image

She asks for photos of inside my cabinets, so she can grade my emergency
preparedness. I tell her that I have everything I need, and that
hoarding supplies only makes a community less safe. ``If you cannot stop
worrying, worry about the most vulnerable. The homeless. The uninsured.
The migrants on the border. The people in prison and detention
centers.''

``You are right,'' my mother responds. ``I was only thinking about you,
because you are my daughter.'' She goes on to question why a developed
country like the United States does not provide universal health care:
``In China, everyone gets treatment, and everything for the coronavirus
is free.''

``Why are you saying this? We both know it's not true.'' One can
acknowledge the problems in America without making China into a utopia.

I recall the times when I accompanied my mother to Sunday Mass as a
teenager. The church's corridors would be filled with parents from
nearby villages and their visibly ill children: With slim hope of
finding a doctor, they had come to God. As we walked past them, my
mother would tell me to look away.

I do not know if she is doing the same thing with news about the
coronavirus epidemic in China: Her messages never diverge from the
Party's narrative of resolve, progress and imminent triumph. But the
official narrative is not the whole truth.

I fix my gaze where my mother averts her eyes. I tell myself that if I
cannot offer assistance on the ground, the least I can do is to bear
witness. Sometimes I feel I am obsessed with tragedy and wonder if
that's selfish: I am trying to drown my feelings in an ocean of human
suffering so that my own troubles will seem minuscule by comparison.

``Eat more onions,'' my mother writes. ``The flavor makes you cry, and
that expels the virus.'' I shake my head. If tears were a disinfectant,
I would be invincible.

I accuse my mother of being narrow-minded, tell her that instead of
fretting about me, she should pay more attention to the less fortunate.
I sound noble. I make scientific sense. But really I feel intense
discomfort at being the recipient of her affection: I have so utterly
failed at the most important relationship in my life, that with my
mother. I do not deserve to be loved.

``Everyday/I think about dying/About disease, starvation/violence,
terrorism, war/the end of the world/It helps/keep my mind off things,''
the English poet Roger McGough wrote in
``\href{https://books.google.com/books?id=KaeZAAAAQBAJ\&pg=PT400\&lpg=PT400\&dq=roger+mcgough+survivor\&source=bl\&ots=sqJyACYeji\&sig=ACfU3U3BfNoqNT08dGHtj6_DMhkOtncQPQ\&hl=en\&sa=X\&ved=2ahUKEwipqu6PocPoAhUHVN8KHZI9DBI4FBDoATAAegQIChAB\#v=onepage\&q=roger\%20mcgough\%20survivor\&f=false}{Survivor}''
in 1979.

Mr. McGough \href{https://www.bbc.co.uk/sounds/play/p0860bqq}{updated
the poem} recently. The new version begins: ``Everyday/I think about
coronavirus/about Brexit/about global warming.''

\begin{center}\rule{0.5\linewidth}{\linethickness}\end{center}

Image

\textbf{I have been messaging} with a dear friend in Italy. On March 19,
after the coronavirus death toll there has \href{coronavirus}{surpassed
China's}, he writes to say he and his family are OK. He shares photos of
Italian flags draped over windows in his neighborhood in Rome. He
describes people singing the national anthem from their balconies.

Back in February, my mother had sent me a link to a video clip showing
residents under lockdown in Wuhan singing ``The March of the
Volunteers,'' the Chinese national anthem, from their windows. ``You
must watch this,'' my mother had said. It had moved her to tears.

The Chinese government has now relaxed quarantine measures throughout
the country, and it is trying to construct a narrative of total victory
against the coronavirus. The lockdown of Wuhan is expected to be lifted
on April 8. In my mother's city, people are allowed to go out and about,
but not many do: Some fear a second wave of infections.

Still, with epidemics on just about every continent, for some weeks now,
many overseas Chinese have been returning to the motherland; China today
seems like the safer place to be. But in the face of reduced airline
routes and soaring ticket prices, some students have been unable to find
a flight home. ``Pity the heart of their parents!'' my mother writes.
She sees in the stranded students a reflection of me.

``Will you consider coming back as well?'' my mother had asked over the
phone in mid-March.

Had I been talking to anyone else, I would have quipped that if I
returned to China, I might be put under another kind of quarantine ---
the kind that usually lasts well beyond 14 days. But I bit my tongue.
Sadness swallowed me.

Lately my mother has been sending me prayers she wrote for her church
group, asking for God's help in defeating the virus. ``We are all in a
shared body of humankind,'' reads one, quoting the catchphrase coined by
President Xi Jinping. She mentions my name every time she asks for God's
blessing.

Eleven years ago when I was preparing to leave China, my mother impelled
me to do two things: get baptized and join the Chinese Communist Party.
She was petrified at the thought of me alone in a foreign country. She
wanted me to carry the memberships like talismans so that the two most
powerful entities in this world and the next one would bless my journey.

I fulfilled neither of her wishes. I am not a Communist, and I do not
believe in God. I am a scientist and a writer. It is the responsibility
of my vocations to ask the questions obscured by simplified answers. But
what happens when the questions I ask can never be answered, when a
puzzle has no solution, when every option is wrong?

It is now the beginning of April, and the United States has the most
reported cases of Covid-19 in the world. On the first day that the
people of Illinois were put under a shelter-in-place order, when the
clock struck 7 p.m., thousands of Chicago residents walked to their
balconies to sing Bon Jovi's ``Livin' on a Prayer.''

As dusk sets over the city I call home today, it is a new morning for my
mother in China. I can picture her standing in our old kitchen, her
graying hair tied up in a messy bun. She adds nuts and dried fruit to
her congee. She reads the news from state media. The kettle chirps on
the stove. She fills two thermal flasks with hot water and looks out the
window. She thanks God for her meal and asks him to look after her only
child.

My inbox will soon light up again with messages from my mother. She will
continue to write while I sleep. I imagine a tunnel opening up through
the planet, where our thoughts meet.

Yangyang Cheng
(\href{https://twitter.com/yangyang_cheng?lang=en}{@yangyang\_cheng}) is
a particle physicist and a postdoctoral research associate at Cornell
University.

\emph{The Times is committed to publishing}
\href{https://www.nytimes3xbfgragh.onion/2019/01/31/opinion/letters/letters-to-editor-new-york-times-women.html}{\emph{a
diversity of letters}} \emph{to the editor. We'd like to hear what you
think about this or any of our articles. Here are some}
\href{https://help.nytimes3xbfgragh.onion/hc/en-us/articles/115014925288-How-to-submit-a-letter-to-the-editor}{\emph{tips}}\emph{.
And here's our email:}
\href{mailto:letters@NYTimes.com}{\emph{letters@NYTimes.com}}\emph{.}

\emph{Follow The New York Times Opinion section on}
\href{https://www.facebookcorewwwi.onion/nytopinion}{\emph{Facebook}}\emph{,}
\href{http://twitter.com/NYTOpinion}{\emph{Twitter (@NYTopinion)}}
\emph{and}
\href{https://www.instagram.com/nytopinion/}{\emph{Instagram}}\emph{.}

Advertisement

\protect\hyperlink{after-bottom}{Continue reading the main story}

\hypertarget{site-index}{%
\subsection{Site Index}\label{site-index}}

\hypertarget{site-information-navigation}{%
\subsection{Site Information
Navigation}\label{site-information-navigation}}

\begin{itemize}
\tightlist
\item
  \href{https://help.nytimes3xbfgragh.onion/hc/en-us/articles/115014792127-Copyright-notice}{©~2020~The
  New York Times Company}
\end{itemize}

\begin{itemize}
\tightlist
\item
  \href{https://www.nytco.com/}{NYTCo}
\item
  \href{https://help.nytimes3xbfgragh.onion/hc/en-us/articles/115015385887-Contact-Us}{Contact
  Us}
\item
  \href{https://www.nytco.com/careers/}{Work with us}
\item
  \href{https://nytmediakit.com/}{Advertise}
\item
  \href{http://www.tbrandstudio.com/}{T Brand Studio}
\item
  \href{https://www.nytimes3xbfgragh.onion/privacy/cookie-policy\#how-do-i-manage-trackers}{Your
  Ad Choices}
\item
  \href{https://www.nytimes3xbfgragh.onion/privacy}{Privacy}
\item
  \href{https://help.nytimes3xbfgragh.onion/hc/en-us/articles/115014893428-Terms-of-service}{Terms
  of Service}
\item
  \href{https://help.nytimes3xbfgragh.onion/hc/en-us/articles/115014893968-Terms-of-sale}{Terms
  of Sale}
\item
  \href{https://spiderbites.nytimes3xbfgragh.onion}{Site Map}
\item
  \href{https://help.nytimes3xbfgragh.onion/hc/en-us}{Help}
\item
  \href{https://www.nytimes3xbfgragh.onion/subscription?campaignId=37WXW}{Subscriptions}
\end{itemize}
