Sections

SEARCH

\protect\hyperlink{site-content}{Skip to
content}\protect\hyperlink{site-index}{Skip to site index}

\href{https://www.nytimes3xbfgragh.onion/spotlight/podcasts}{Podcasts}

\href{https://myaccount.nytimes3xbfgragh.onion/auth/login?response_type=cookie\&client_id=vi}{}

\href{https://www.nytimes3xbfgragh.onion/section/todayspaper}{Today's
Paper}

\href{/spotlight/podcasts}{Podcasts}\textbar{}`Everything Is Always Keep
Changing'

\href{https://nyti.ms/2x3320V}{https://nyti.ms/2x3320V}

\begin{itemize}
\item
\item
\item
\item
\item
\end{itemize}

Advertisement

\protect\hyperlink{after-top}{Continue reading the main story}

transcript

Back to Sugar Calling

bars

0:00/41:16

-41:16

transcript

\hypertarget{everything-is-always-keep-changing}{%
\subsection{`Everything Is Always Keep
Changing'}\label{everything-is-always-keep-changing}}

\hypertarget{hosted-by-cheryl-strayed-produced-by-kelly-prime-and-austin-mitchell-and-edited-by-sara-sarasohn}{%
\subsubsection{Hosted by Cheryl Strayed, produced by Kelly Prime and
Austin Mitchell, and edited by Sara
Sarasohn}\label{hosted-by-cheryl-strayed-produced-by-kelly-prime-and-austin-mitchell-and-edited-by-sara-sarasohn}}

\hypertarget{cheryl-strayed-calls-the-author-george-saunders-to-talk-through-the-uncertainties-of-this-moment}{%
\paragraph{Cheryl Strayed calls the author George Saunders to talk
through the uncertainties of this
moment.}\label{cheryl-strayed-calls-the-author-george-saunders-to-talk-through-the-uncertainties-of-this-moment}}

Friday, April 3rd, 2020

\begin{itemize}
\item
  cheryl strayed\\
  So I have this wonderful mentor, the writer George Saunders. He's the
  author of so many beautiful books. ``Lincoln in the Bardo'' is his
  most recent one, ``Tenth of December,'' ``CivilWarLand in Bad
  Decline,'' ``Pastoralia'' and several others. He was my professor when
  I was getting my M.F.A. at Syracuse University. Now it was more than
  20 years ago that I met George. And I didn't really even read his
  writing before I applied there. What's interesting about it is so many
  people were applying there because they wanted to work with George.
  And I didn't. I hadn't read his work. But what happened is he called
  me up and he accepted me into the program. And we had these long
  conversations, and I was so struck by how wonderful he was that I went
  and read his books. And I was like, oh, he is a great writer. I'd love
  to work with him. So I did that. I accepted. And there was this
  graduate school picnic --- this sort of opening picnic, and I met him
  and his wife Paula and their two daughters who were really little kids
  at the time. I have this vivid image of George putting his hand on top
  of one of their heads. And now they're in their 20s and all grown up.
  But I knew that I was meeting somebody who was really, at essence, a
  really wonderful person. I think he's an extraordinary --- a great,
  great writer. But he's an even more extraordinary man. So I was
  thinking about George in this moment, because he has been such a
  guiding light to me and to so many. His humor, his compassion, his
  generosity, and his ability to take the long view, I feel like that is
  the voice I need to hear right now on the phone. So I'm going to give
  him a call.
\item
  {[}phone ringing{]}
\item
  george saunders\\
  Hello?
\item
  cheryl strayed\\
  George.
\item
  george saunders\\
  Cheryl, how you doing?
\item
  cheryl strayed\\
  Hi, it's Cheryl.
\item
  george saunders\\
  I know.
\item
  cheryl strayed\\
  How are you doing?
\item
  george saunders\\
  Good. We're hanging in here.
\item
  cheryl strayed\\
  Oh my gosh. I'm really so excited to get this chance to talk to you in
  the middle of a pandemic.
\item
  george saunders\\
  Yeah. It's a once in a lifetime.
\item
  cheryl strayed\\
  I hope so. How are you? First of all, where are you?
\item
  george saunders\\
  I'm in Corralitos, California. So we're just between Santa Cruz and
  Watsonville --- so kind of up on a hillside. And from the front of our
  house, you can see Watsonville and some of the ocean.
\item
  cheryl strayed\\
  Wow. So you're looking out over --- do you see trees? Do you see
  houses? What do you see?
\item
  george saunders\\
  Well, we're up in Redwoods, and so we see that. And then if you look
  far enough out, there's berry fields. And then beyond that, kind of
  dimly, there's the city of Watsonville. So it's really beautiful. And
  you'd never know there was a problem.
\item
  cheryl strayed\\
  Yeah, isn't that interesting? That's fascinating to me.
\item
  george saunders\\
  Yeah I was staying off media so much, and then I got back on it. And I
  just was kind of stunned by the numbers that they're talking about
  now. And I just can't imagine 100,000 people. That's just literally
  beyond my capacity to imagine it. And it's such a weird disconnect to
  have that in your head and your heart and then look outside and
  everything looks, from this perspective, normal. It's really strange.
\item
  cheryl strayed\\
  Yeah. There is this sense we've been lucky not to be sick, at least
  yet. But beyond us, there are really painful and hard and difficult
  things happening to a lot of people around the globe. And that, to me,
  I have to say, has been terrifying. And it's been distracting. And I'm
  curious how you've been responding to that emotionally.
\item
  george saunders\\
  Well, I think in some ways, I don't know, it's always happening. There
  is always misery. But I also think I've noticed about myself that in
  times like this, my mind wants to have answers for everything. It
  wants to have a take on things to give myself comfort. I think of it
  like when you slip on the ice and in the split second before you're
  about to hit the ground, that's really having no take. You're just out
  of control and the pavement's rushing up. So I think sometimes you
  just go, yeah, we're in that moment. We can pretend we can stop time
  and have a take on hitting the pavement or being mindful as we hit the
  pavement. But in fact, it's really unknown what's happening. And I
  think, especially for writers, us type A people, it's kind of
  difficult to be in a state where you just say I really don't know
  what's going on I have no control over it and no say over it and
  actually, very little influence at this point. That's a strange set of
  thoughts for anybody to have. But I think maybe for those of us who
  interpret for a living, it's especially both confusing and important,
  I think, to say, yeah, we don't know. We have to keep our sensory
  apparatus as open as we can so we don't miss any actual data. And to
  do that in the face of one's own anxiety is kind of difficult.
\item
  cheryl strayed\\
  It is. And as you're talking, you're reminding me, you told me about
  an email you wrote to your graduate students at Syracuse University,
  and I'm wondering if you could read to me what you wrote to them.
\item
  george saunders\\
  Oh. Sure. Sure. No, I'd be happy to. Yeah, it just goes like this.
  Dear, S.U. writers --- jeez, what a hard and depressing and scary
  time, so much suffering and anxiety everywhere. I saw this bee happily
  buzzing around a flower yesterday and felt like, ``Moron! If you only
  knew.'' But it also occurs to me that this is when the world needs our
  eyes and ears and minds. This has never happened before here --- at
  least not since 1918. We are, and especially you are, the generation
  that is going to have to help us make sense of this and recover
  afterwards. What new forms might you invent to fictionalize an event
  like this, where all of the drama is happening in private,
  essentially? Are you keeping records of the emails and texts you're
  getting, the thoughts you're having, the way your hearts and minds are
  reacting to this strange new way of living? It's all important. 50
  years from now, people the age you are now won't believe this ever
  happened or will do the sort of eye roll we all do when someone tells
  us about something crazy that happened in 1960. What will convince
  that future kid is what you are able to write about this. And what
  you're able to write about it will depend on how much sharp attention
  you're paying now and what records you keep, also, I think with how
  open you can keep your heart. I'm trying to practice feeling something
  like, ah, so this is happening now. Or hmm, so this, too, is part of
  life on Earth --- did not know that, universe. Thanks so much,
  stinker. And then I real quick tried to pretend I didn't just call the
  universe a stinker. I did a piece once where I went to live incognito
  in a homeless camp in Fresno for a week. Very intense, but the best
  thing I heard in there was from this older guy from Guatemala, who was
  always saying, ``Everything is always keep changing.'' Truer words
  were never spoken. It's only when we expect solidity, non-change, that
  we get taken by surprise. And we always expect solidity, no matter how
  well we know better. Well, this is all sounding a little preachy, and
  let me confess that I'm not taking my own advice --- at all. It's all
  happening so fast. Paula has what we are hoping is just a bad cold,
  and I'm doing a lot of inept caregiving. Our dogs can feel that
  something weird is going on --- no walk? Again? But I guess what I'm
  trying to say is that the world is like a sleeping tiger, and we tend
  to live our lives there on its back. We're much smaller than the
  tiger, obviously. We're like Barbies and Kens on the back of a tiger.
  Now and then, that tiger wakes up, and that is terrifying. Sometimes
  it wakes up when someone we love dies or someone breaks our heart or
  there's a pandemic. But this is far from the first time that tiger has
  come awake. He she has been doing it since the beginning of time and
  will never stop doing it. And always, there have been writers to
  observe it and later make some sort of sense of it --- or at least
  bear witness to it. It's good for the world for a writer to bear
  witness, and it's good for the writer too, especially if she can bear
  witness with love and humor and, despite it all, some fondness for the
  world, just as it is manifesting --- warts and all. All of this to
  say, there's still work to be done, and now more than ever. There is a
  beautiful story about the Russian poet Anna Akhmatova. Her husband was
  shot and her son arrested during the Stalinist purges. One day, she
  was standing outside the prison with hundreds of other women in
  similar situations. It's Russian cold, and they have to go there every
  day, wait for hours in this big, open yard, then get the answer that
  today and every day, there will be no news. But every day, they keep
  coming back. A woman, recognizing her as the famous poet, says, poet,
  can you write this? And Akhmatova thinks about it for a second and
  goes, yes. I wish you all the best during this crazy period. Someday
  soon, things will be back to some sort of normal, and it will be
  easier to be happy again. I believe this, and I hope it for each one
  of you. I look forward to seeing you all again and working with you,
  and even in time, with sufficient PPE, give you a handshake or a hug.
  Please feel free to email anytime for any reason, George.
\item
  cheryl strayed\\
  George, that's so beautiful. I mean, I know it's been now a couple
  decades, really --- has it been that long since I was your student?
\item
  george saunders\\
  No. Three years. Three years.
\item
  cheryl strayed\\
  But I feel like you wrote that to me. And I love this --- how do you
  say her name --- Akhmotiv ---
\item
  george saunders\\
  Akhmatova.
\item
  cheryl strayed\\
  Akhmatova. I love how Akhmatova says --- or the person standing with
  her says, poet, can you write this? And she says, I can. And I think,
  obviously, that is our job as writers, and it's the job of the artist
  to write that moment. It has been through all time, and it will be
  through all time. And I think that that also translates outside of the
  arts. It translates to, really, every human on the planet who has
  something to contribute about the story of us. And I think that that's
  what's so powerful about that moment of that understanding that, yes,
  we can bear witness, as you say. We don't have to be a novelist. We
  don't have to be a poet. We can do that even as a person.
\item
  george saunders\\
  Yeah. And also you get the feeling that she --- I think you're
  actually right. I think she says, I can. I don't have the text in
  front of me. But the beauty of that is to say, I hear her saying, I
  think I can. I think I can try. And that idea that if you're going to
  describe a moment like that, you'd really have to sort of keep moving
  the fences out. That's a beautiful story, because she includes the
  woman who's interrogating her. If she had just stopped at describing
  the weather in the jail, somehow it's not the full reality. So I think
  that, for me, is a beautiful challenge is to keep saying everything
  that you think or feel or perceive, even if it seems totally trivial,
  might be part of the larger picture. It might be that in 100 years ---
  or not 100 years, but say when we're old people, you look back and it
  might be some totally trivial detail that actually encapsulates the
  whole thing. But you won't know until you get there. And to know, you
  have to record it along the way, I guess.
\item
  cheryl strayed\\
  So this idea --- one thing I'm struck by that both of us keep sort of
  circling around and saying, this control. And I think both of us as
  writers who have invented characters and made them do things, we know
  that, actually, this idea that we can control really anything is just
  an absolute illusion. And yet, it's incredibly hard for us to wrap our
  minds around, right? How are you absorbing that reality that you don't
  know if you're going to be able to go hang out with your friends in a
  month. You don't know if you're going to be able to do that in six
  months. We don't really know when we can all make plans again. So I'm
  wondering, what sense do you make of that?
\item
  george saunders\\
  Well, for me what's interesting is I was really enjoying canceling
  things. Even before the pandemic, I was just like, I'm going to say no
  to everything. And I was getting a little bit ecstatic about that ---
  that I could just be a family person and sit in my room and write. And
  I'm like, that's what I really want to do anyway. So in a way, it's
  weird, I'm still kind of in the place where, for most things, the
  canceling of plans doesn't really make me very anxious. Except the one
  weird thing is I can't see my parents or my kids. And that bothers me,
  because they're both in different cities --- my kids aren't. So that's
  bad. But actually, I think it just caught me at a funny moment where
  I'm happy to stay home. And then I think for me, the control thing has
  to do with it's a personality disorder that I have. I found a way, I
  think, in my work to use it kind of beneficially. In that little
  world, you're absolutely responsible for controlling every line. I
  think that's what I'm finding out. I don't know if that resonates with
  what happens when you write.
\item
  cheryl strayed\\
  Well, and I think too, there is this sense of --- I don't know,
  there's something about paying attention too. When you're writing,
  you're so focused. And that has been one of the primary challenges for
  me is distraction, I find that it's very hard to focus unless I become
  absorbed in something. And writing surely does that. But what about
  other stuff? Are you baking bread and taking up new things? Or are you
  and Paula just laying low?
\item
  george saunders\\
  Well, I mean, Paula has been a little sick. So I've been kind of the
  inept caregiver, as I said in that letter. And that's been going on
  about eight or nine days. And so I really was working pretty well. But
  now I had a story in The New Yorker, so I closed that and did some of
  the peripheral stuff. But I've just been kind of cooking and cleaning
  in my comical American male way. So the thing I'm starting to notice
  --- just micro notice --- is I think I'm a little reliant on that
  state of concentration that you're talking about. That when you go
  into a story, what it feels like to me as I'm just obsessing over the
  small details of the text. I'm starting to think that that's some kind
  of neurological state that's akin to, but not exactly equal to,
  meditation. And it really makes me happy in a kind of not euphoric,
  but it makes me solid. It makes me a little more positive as a person.
  But having found out the writing in that state of mind increases my
  happiness, then it's kind of my responsibility to get into that more
  --- as much as I can.
\item
  cheryl strayed\\
  Do you meditate?
\item
  george saunders\\
  Yeah. Yeah. And we're doing some prayers during this. And yeah, so
  that's been part of our lives for a long time and a very reliable one.
  And that's one of the things about this moment that is interesting is
  that the anxiety that we're talking about is basically people saying,
  I want things to be the way they were. I want to live. I want to not
  be sick. I want to not have to worry about getting somebody else sick.
  I want to go to a bar. But really, what occurs to me is, why? What was
  it that you want to do once you are free again? And for most of us ---
  I think for me, it's just, I want to assume my old habits. Why?
  Because they assuage my anxiety, essentially. And so it does present a
  little bit of a moment to look afresh it what we do with our energy
  when we're healthy. And I know this is a trivial example --- we live
  kind of about half an hour out of town. So it's kind of a big trip to
  go into the store. But we go just about every day, I would say ---
  just out of habit. Well this last period, we haven't been in two
  weeks. And we're eating better probably, you know? So it does make me
  think so many of the ways that we live are kind of just lazy and
  habitual from that small example to the larger things. So I'm trying
  to be optimistic and think that maybe each of us having a little
  enforced spring break and if we're healthy, god willing, it might make
  us look a little differently at the American life that we all live ---
  the innate laziness and violence of it and the habitual part of it ---
  that most of us just accepted and try to make the best of it. So maybe
  a little bit of a breather for the world to go, wait a minute. What
  are we doing? That would be nice. We'll see.
\item
  cheryl strayed\\
  I think you're really right. I think you might remember that I grew up
  in rural Minnesota without running water or electricity or indoor
  plumbing. And my mom was really this incredibly kind of
  self-sufficient person and my stepfather, incredibly self-sufficient,
  grew a lot of our food. And she preserved and canned and pickled
  everything. She baked bread. She knew how to do everything in the
  domestic realm. And I refused to learn any of that, because I wanted
  to. Because I didn't want to be a domestic goddess. I wanted to be a
  great American writer. And I wanted to sort of mark my space in a
  place that wasn't traditionally sort of defined for women. But, boy,
  have I thought about my mom so much in these last couple of weeks
  where I am not getting dinner from the grocery store kind of deli
  counter. And I have these old books of my mom's --- Stocking Up and
  these kind of food preservation books. And I'm opening them for the
  first time. And so that's been kind of interesting. And you're right,
  it's been a little bit of a corrective, because I don't necessarily
  want to be somebody who has to spend two hours making dinner every
  night. But it's kind of been good for me to have to stretch in that
  direction.
\item
  george saunders\\
  Yeah, that's a beautiful story. It's almost like your family kind of
  enacted the last 80 years of American history in one generation, where
  at some point, people could do all that stuff. And then I remember as
  a kid in Chicago, it was so cool to get Jello or to go to Burger King.
  It just seemed like canning your own food was so last century. But
  yeah, no, that's a beautiful --- that's a beautiful idea.
\item
  cheryl strayed\\
  It's true.
\item
  george saunders\\
  I don't know if you're --- are you a Wendell Berry fan?
\item
  cheryl strayed\\
  I am, yeah.
\item
  george saunders\\
  Well, I just was thinking how prescient he is and he had this idea
  that, as his one book is titled, ``It All Turns On Affection.'' And
  it's made me think how many hours of our day, of our working lives are
  spent doing something that we feel connected to, that's really ours,
  you know? And I think that his words, Berry's words, are really
  speaking to me now and saying, well, maybe this is a moment when we
  could at least a little bit investigate moving backwards on that arc
  and taking a little more responsibility for the way we live in the
  world and also minimizing the small violences that he talks about ---
  the kind of inquisitiveness and the idea that what we're going to do
  in this world is exploit whatever's out there to make it easy on
  ourselves. But I thought I'd read his poem. He wrote this for his
  wife, I assume, and it's called ``The Wild Rose.'' Sometimes hidden
  from me in daily custom and in trust so that I live by you unaware, as
  by the beating of my heart. Suddenly, you flare in my sight --- a wild
  rose looming at the edge of thicket --- grace and light where
  yesterday was only shade. And once again, I am blessed, choosing again
  what I chose before.
\item
  cheryl strayed\\
  Wow. Beautiful.
\item
  george saunders\\
  Yeah, just the idea that this --- I was out. I took the dogs on a
  walk. And I mean, it sounds kind of corny, but I was like, oh, trees.
  They really hit me in a different way than they had before. And when I
  was a kid, I read ``Walden'' for the first time on the south side of
  Chicago and then had this kind of cheesy experience of walking through
  a forest. But suddenly, the whole landscape was different because
  Thoreau had written about it. And I think that's happening a bit now.
  You're like, the clouds --- the clouds are still there. They still
  like us.
\item
  cheryl strayed\\
  And we can still commune with it. I mean, I think that that's
  something really powerful about this experience you're talking about
  is that, OK, we have to take a step back from each other, but there
  are these other things that we can step closer to.
\item
  george saunders\\
  Yeah.
\item
  cheryl strayed\\
  Maybe it's the thing we refused to learn in our childhood.
\item
  george saunders\\
  Right.
\item
  cheryl strayed\\
  Like me learning from my mom how to pickle a cucumber or really those
  things I wrote so much about in ``Wild'' --- the way that the natural
  world can feel like our home, that we are not in opposition to it, and
  that we can --- I had a similar experience when I was on a walk the
  other day. And I was looking at the trees and I thought, they don't
  have this virus. They're impervious to this virus. And I felt some
  sort of connection, some sort of sense of comfort emanating from just
  the fact that they were safe from this and that they would stand watch
  over us as they always have. That there is a way that they do that.
\item
  george saunders\\
  This is a funny look at the difference in our minds, because I was
  going to say you felt resentment. Bastards.
\item
  cheryl strayed\\
  Well, no.
\item
  george saunders\\
  No, I think that's exactly right.
\item
  cheryl strayed\\
  For the record, George Saunders, you're the only one calling a bee a
  moron. I would never do that.
\item
  george saunders\\
  Yeah. You have to meet this bee. This particular bee was really bad.
  No, the other thing that I was struck by too is just that maybe it's
  nice to think that this will adjust things in a positive way. But it
  sure is making clear how ridiculous our economy is and the wealth
  differential. And you think about people who have to take a week off
  of work and can't afford it. And I think it's underscoring so many
  pre-existing problems with our country, from the top to the bottom.
  And it's stressing every aspect of what we are, and that's
  disconcerting.
\item
  cheryl strayed\\
  Well, and it's asking us to also then define who we are. And I think
  that the opportunity here is to move in the direction of empathy and
  compassion. And the question is, will we take it? Will those who have
  privilege and power use it for good or use it for, I hate to cast it
  in this way, but for evil? And I do think there's evidence in both
  directions about the path we'll take. What do you think is going to
  happen?
\item
  george saunders\\
  I don't know. For me, what comes to my mind is sort of like, well, in
  a sense, we're asking how did we get in this position in the first
  place? And my fear is that people are going to be so relieved when
  it's quote unquote ``over,'' I'm sure we'll just jump right back into
  the same water. So I don't know. So it could be that maybe something
  good will happen. And I'm certainly rooting for that. But I don't
  know. I don't know.
\item
  cheryl strayed\\
  So, George, I read something on Twitter --- you're not on Twitter, are
  you?
\item
  george saunders\\
  No.
\item
  cheryl strayed\\
  Oh my god.
\item
  george saunders\\
  Or should I say, no. No. I'm not on any social media.
\item
  cheryl strayed\\
  I know. You were for a while, and then you disappeared. And I was glad
  to hear it wasn't you unfollowed me because I'm insufferable. It was
  because you actually ---
\item
  george saunders\\
  Oh, no, no. I had a little Facebook author page, and even that drove
  me up the wall. The joke I always make is I trained myself for years
  to write slowly for a lot of money, so I have no interest in writing
  quickly for free.
\item
  cheryl strayed\\
  Yeah. Well, see, I'm one of those sort of superficial simpletons who
  I'm on social media. I love social media.
\item
  george saunders\\
  Yeah, but again, you do it so authentically that it's lovely. So I
  think everybody has to assess their own power.
\item
  cheryl strayed\\
  Thank you. But let me tell you --- one of the best things, as you
  probably are aware, there are some really cool things on social media.
  And one of my favorite things on Twitter is to follow Carl Reiner.
  He's absolutely brilliant and sweet and hilarious. He turned 98 just a
  few days ago. And he ---
\item
  george saunders\\
  Happy birthday. Mazel tov.
\item
  cheryl strayed\\
  Can you believe this? He tweeted this, and it really stopped my heart.
  This is what he said --- for the first time in memory, I see nothing
  in this world about which I care to joke. And it felt true to me, and
  it felt so sad to me that this 98-year-old man, this funny man, this
  comedian was saying I see nothing about which to joke. And I'm
  wondering what you think of that. Your life is full of so much humor,
  your work is full of so much humor.
\item
  george saunders\\
  I think --- when I was doing all the research on Lincoln, I found
  something interesting. He was majorly depressed. And the only way he
  could get himself out of it is to just joke --- just consciously say
  I'm going to tell some jokes, and that's going to make me feel better.
  So I think we used to think that jokes or humor is a response to being
  happy. You're happy so you make a joke. And that's sometimes true for
  me. But I've also noticed for myself that to allow myself to make a
  joke in a given situation does something to my disposition. It just
  lightens it in a certain way or it activates it, I guess I would say.
  So my hope would be that Carl Reiner, who I adore, was having a bad
  day. And if you got him into a situation, I'm hoping --- I'm hoping
  and praying that he would still find the joke, because it would be
  there for him. I think too, part of it is if you think about the way
  we're processing this thing, most of us are seeing no direct evidence
  of it. I saw it. The other day I was driving out of my house, and I
  saw this beautiful little Rockwellian picture --- these two couples
  were standing across from each other on the road I was driving down,
  shouting out to each other and laughing and having a conversation ---
  with the road being their social distancing. So most of us aren't
  seeing this. It's just sort of theoretical. We're seeing it in the
  media. We're sympathizing from afar, and so on. And so I think for me,
  I'm trying to remember that it's OK to feel whatever you're feeling.
  If you find a joke, go ahead. Or if you find yourself forgetting about
  it for a couple hours and feeling happy, of course. But I hope he
  feels differently. And if he doesn't, that's OK too.
\item
  cheryl strayed\\
  I just want to say, George, rest assured --- of course, two days
  later, I went on Twitter, and there was Carl Reiner, and he made me
  laugh. There was a joke. He made a tweet. He tweeted a joke.
\item
  george saunders\\
  So he was being beautifully honest about what he was feeling at that
  moment. And I love that. That's fantastic. Yeah. I would argue that
  even during the worst time, there are pockets of beauty and pockets of
  pleasure. And it seems to me one thing that is a danger in our time of
  social media and cable news is that we accept the world's, meaning the
  media's, narrative as all pervasive and correct. And that bee that I
  talked about, that was a real bee. And he was having a good day. So in
  a certain way, I think for a writer, the complex chore of imagination
  is to say the world is not a complete fabric of misery right now. It's
  pockets of incredible misery. And so to sort of hold that complexity
  in the brain is difficult. But I think it's a form of moral
  responsibility to say, I don't have to accept an outside
  interpretation of what's a very complex reality, especially if it
  makes me more anxious and less helpful and depressed. But I think to
  be mindful of what pleasures there are is a really good way to make
  sure that when you're needed, you'll be ready --- something like that
  maybe.
\item
  cheryl strayed\\
  And there's beauty in every day. We know that for certain, no matter
  what is happening. And so yeah, I think it is about being able to hold
  many complex ideas that seem sometimes to contradict themselves in one
  hand. And probably this moment simply amplifies, again, what was
  always true. There was always sorrow. There was always violence. There
  was always death and disease and destruction and loss. And there's
  always beauty. And if we choose to be there to see that, we get to be
  the one to see that.
\item
  george saunders\\
  Yeah, we can hold a lot of those ideas at once. I mean, we
  theoretically can --- I can't, but one could. I think Fitzgerald said
  something like that --- that's the mark of an intelligent person is
  the ability to hold two contradictory ideas in mind at the same time.
  So I think that's another thing that reading and writing helps us
  train us. It's like training wheels to almost like a plate-spinner. At
  the beginning, you can't even spin one plate. Well, with training, you
  could do a few. And I think for me, that's what reading and writing
  both do.
\item
  cheryl strayed\\
  Yeah, I really think that story is an essential part of our ability to
  survive and thrive. I really do. We use that phrase a lot --- books
  will save us or story will save us or art saves. And it sounds trite
  or sort of self promotional, especially when we writers are saying,
  no, stories will save us. But the fact is, it's true. This is the
  reason I'm calling you, because, really, when I search my soul and
  think, how have I reckoned with the things that were most difficult?
  How have I learned how to be brave and accept and move on and grieve
  and carry my burdens with maybe a tiny bit more grace? And the answer
  is, no question, it's story. And I do think that's the genuine power.
\item
  george saunders\\
  Yeah. We go through this life so fast, and we don't really have time
  to learn the lessons. But some of them are enshrined in stories. We
  had a thing --- a couple of weeks ago, my dad and I were e-mailing
  about this epidemic. And he said that in Chicago in the 1918 epidemic,
  something like 8,000 people died in that, at that time, smaller city.
  And he said that in the family lore, there was a story that his
  grandmother had had a son who died in the epidemic. And sort of his
  idea was that the body had been left on the street, because things
  were so crazy. So we started talking about that, and my dad and I and
  a cousin and my aunt kind of put our heads together and started doing
  some research. And it turns out, yeah, the kid actually, we always
  heard he was eight. In fact, he was three. And his name was Leroy
  Gendrow. And the story was that he suffered. And so the family could
  hear, and his mother --- my father's grandmother --- screamed and
  wailed at his death. And then because the system was so overwhelmed,
  they had to wrap this little boy up in a sheet in which he died and
  take it out to the street. And one of the details we found out was the
  truck would come by every morning, which gives you some sense of the
  scale of this thing. And so my dad wrote this beautiful email to me
  saying that basically, he'd lived with that story all his life ---
  kind of half heard at a family party. And I think it was painful for
  the family and they suppressed it. And it was during World War I, and
  so people kind of moved on. But the kid never got a grave. It was a
  mass burial, you know? And my dad said how moving it was it now that
  little boy had a name. He had a name --- Leroy Gendrow. He had an age.
  We were able to find out that he was buried at Mount Olivet cemetery
  on the South Side. So I was thinking about why that additional detail
  was important. Now, when I was doing the Lincoln book, I read
  somewhere that the dead actually like to be remembered. They like to
  be verbally recalled, they like their picture to be up. But even
  leaving that aside, why was that satisfying to my family to be able to
  speak this kid's name aloud? And I don't know. I mean, I don't know if
  you have thoughts on that. But it was just, as Carver said, a small
  good thing to be able to speak that kid's name, you know?
\item
  cheryl strayed\\
  Well, and I think that's everything about what we're saying about the
  power of story. Because, of course, what is a story but that we get to
  hear one report of a human existence or a human experience? And I was
  just saying to my kids the other day, some number of people who have
  died of the COVID-19, I was saying OK, that's, the equivalent of this
  entire portion of the city of Portland --- that they're all dead.
  Because it's abstract. Even this family story you'd heard, like, oh,
  my father's grandmother's son --- who is he to you? He's nobody. But
  then he becomes Leroy, and it becomes that family in that apartment
  and that truck that comes every day and that body on the street. This
  is how we build empathy and compassion is that we listen really hard
  to the stories all around us. We see the people around us. So there's
  this grand scale experience, but maybe we live it on the human scale,
  maybe we look again at the people around us with maybe more love.
\item
  george saunders\\
  Yeah. It's funny how the challenge of this moment is similar to the
  fictional challenge, which is can you extrapolate from someone else's
  experience to your own viscerally? And it's kind of hard. What is that
  quote about 100,000 deaths is a statistic, one death is a tragedy. I'm
  probably getting that wrong. Especially in this way that we're
  isolated, now to say, OK, I'm going to actively work on it as part of
  my ethical spiritual life to try to imagine that not everything is
  fine everywhere. And it's the great human struggle, but maybe
  exaggerated at a moment like this. I don't know, it's kind of amazing.
  It's just stunning in a way to think about what's happening, even as
  we speak.
\item
  cheryl strayed\\
  So, George, are you afraid --- one of the things that's come up for me
  --- I'm 51 ---
\item
  george saunders\\
  You're a baby.
\item
  cheryl strayed\\
  I'm a baby. How old are you?
\item
  george saunders\\
  61.
\item
  cheryl strayed\\
  61. OK. So we're not --- yeah, we're just a decade apart here. But my
  mom died when she was 45. I was 22. And I thought, OK, I'm going to
  die young. I'm not going to be 45. I'm not going to reach that age
  without dying. It was kind of like a little superstition that I share
  with a lot of people who lost parents young. They have that same
  thing. When they reach the age that their parent died, it's always a
  thing, right? It's a passage. And here I am, I made it to the other
  side. I'm 51 and here I am. And then this pandemic came along, and one
  of my thoughts when I am in my kind of scaredest place, which I've
  gone to a few times over this last couple of weeks, is I think, see, I
  was right. I am going to die young, and my kids are going to be
  orphans --- or at least motherless, like I was. That they're going to
  suffer like I've suffered. And I have to admit, I don't think of
  myself as somebody who's anxious or much thinking about --- I'm not
  afraid of things very often. And yet suddenly, I felt like I'm afraid
  I'm going to die of this COVID-19. I'm afraid. Are you afraid?
\item
  george saunders\\
  I mean, in my gut, I'm kind of a Pollyanna. And we're being super
  careful. We're basically like Clarks watching the birds as they fly by
  our house, you know? So in my gut, I don't think so. But I think this
  maybe ties in with what we're saying about stories is you look out at
  the world and you read these accounts, even though they're quick and
  they're sort of at this moment kind of surface accounts. And you have
  to start saying it's possible. It's possible. So I think that's part
  of the responsibility is to be careful, but also go yeah, you can read
  his accounts of people who, two weeks ago, were high functioning
  people running around happy and they died. So I think part of it is to
  try to get that into your body and your mind on a daily basis that
  it's temporary.
\item
  cheryl strayed\\
  Hmm. I think the thing is that we keep circling around is everything
  that's scary about this moment has existed all this time. This whole
  51 years I've been alive and the whole 61 you've been alive, this has
  always been true.
\item
  george saunders\\
  Yeah.
\item
  cheryl strayed\\
  Right?
\item
  george saunders\\
  Right.
\item
  cheryl strayed\\
  I think what's happening for me and you and everyone right now is it's
  amplified, and suddenly it's on our doorstep. So in some ways, it
  functions as it's this real thing that's happening. It's also a
  metaphor --- a metaphor, really, for the truth of human existence,
  which is that we're mortal, and that we don't have control, and that
  we have to simply try our best, keep the faith, and maybe pray to the
  divinity in each other and honor the divinity that is within each of
  us.
\item
  george saunders\\
  Yeah. I think it does remind me so much of 9/11 and that feeling --- I
  think it's called sympathetic compassion where you're not in danger
  yourself, but you can imagine the fear and danger someone else is in.
  And that longing to want to do something, or really, in its most
  profound, you're longing that that person not suffer, you're longing
  that that person be happy. And that's very profound. And I think if we
  can cultivate that feeling of wishing the best, that's such a powerful
  thing. And I think that's what we're here to do all the time. But in
  these situations, you feel it. We talk about anxiety --- the anxiety
  of the moment --- and I've been trying to think that some of that is
  useless. Some of it is just neurosis. But part of the anxiety is kind
  of like I read somewhere that there's a meditation you can do where
  you imagine a person that you love very much drowning just beyond your
  reach. And that feeling that comes up when you do that is actually
  compassion. I'm sure another part of that meditation is you imagine
  somebody who isn't so close to you drowning out of your reach, and you
  could actually grow your love in that way, I think. That part of this,
  I think, is I'm trying to think about the usefulness of that --- the
  fact that you could cultivate a feeling of concern for other people,
  and that that concern could get larger. So I suppose if you're looking
  for something that this moment conveys that isn't negative, the
  potential for that might be, as you said, to pray and to try to lure
  out the better parts of ourselves.
\item
  cheryl strayed\\
  Yeah --- to rise to that. OK, well, George, you know what? It has been
  enriching to my soul to hear your voice and to chat with you and to
  catch up. That's what I think has been so striking to many of us is,
  like, how certain it is that we need our friends during this time and
  that human connection. So thank you. Thank you for talking to me and
  being here with me today on the phone.
\item
  george saunders\\
  It's been such a pleasure for me. We miss you so much, and can't wait
  to sit down in person and get within the six-foot radius and all that.
\item
  cheryl strayed\\
  You know, and I'm going to even look out further than that to the day
  that we can actually give each other a hug.
\item
  george saunders\\
  Now, let's not get crazy. Let's not get crazy here. No, and give my
  love to your family. Thank you so much for asking me. I enjoyed every
  minute. I always come away from conversation with you believing that
  much more in writing. So thank you for being out there.
\item
  cheryl strayed\\
  Oh thank you. Love to your family too, George. Bye bye.
\item
  george saunders\\
  All right. Bye bye.
\item
  cheryl strayed\\
  I'm Cheryl Strayed, and this is ``Sugar Calling.'' Next week, Margaret
  Atwood. Thanks for listening.
\end{itemize}

\href{https://www.nytimes3xbfgragh.onion/column/sugar-calling}{\includegraphics{https://static01.graylady3jvrrxbe.onion/images/2020/04/29/podcasts/sugar-calling-album-art/sugar-calling-album-art-square320.jpg}Sugar
Calling}Subscribe:

\begin{itemize}
\tightlist
\item
  \href{https://itunes.apple.com/us/podcast/id1505881384}{Apple
  Podcasts}
\item
  \href{https://podcasts.google.com/?feed=aHR0cHM6Ly9yc3MuYXJ0MTkuY29tL3N1Z2FyLWNhbGxpbmc\&ved=0CAUQrrcFahcKEwjA8Kyn09voAhUAAAAAHQAAAAAQBQ}{Google
  Podcasts}
\end{itemize}

\hypertarget{everything-is-always-keep-changing-1}{%
\section{`Everything Is Always Keep
Changing'}\label{everything-is-always-keep-changing-1}}

\hypertarget{cheryl-strayed-calls-the-author-george-saunders-to-talk-through-the-uncertainties-of-this-moment-1}{%
\subsection{Cheryl Strayed calls the author George Saunders to talk
through the uncertainties of this
moment.}\label{cheryl-strayed-calls-the-author-george-saunders-to-talk-through-the-uncertainties-of-this-moment-1}}

Hosted by Cheryl Strayed, produced by Kelly Prime and Austin Mitchell,
and edited by Sara Sarasohn

Transcript

transcript

Back to Sugar Calling

bars

0:00/41:16

-0:00

transcript

\hypertarget{everything-is-always-keep-changing-2}{%
\subsection{`Everything Is Always Keep
Changing'}\label{everything-is-always-keep-changing-2}}

\hypertarget{hosted-by-cheryl-strayed-produced-by-kelly-prime-and-austin-mitchell-and-edited-by-sara-sarasohn-1}{%
\subsubsection{Hosted by Cheryl Strayed, produced by Kelly Prime and
Austin Mitchell, and edited by Sara
Sarasohn}\label{hosted-by-cheryl-strayed-produced-by-kelly-prime-and-austin-mitchell-and-edited-by-sara-sarasohn-1}}

\hypertarget{cheryl-strayed-calls-the-author-george-saunders-to-talk-through-the-uncertainties-of-this-moment-2}{%
\paragraph{Cheryl Strayed calls the author George Saunders to talk
through the uncertainties of this
moment.}\label{cheryl-strayed-calls-the-author-george-saunders-to-talk-through-the-uncertainties-of-this-moment-2}}

Friday, April 3rd, 2020

\begin{itemize}
\item
  cheryl strayed\\
  So I have this wonderful mentor, the writer George Saunders. He's the
  author of so many beautiful books. ``Lincoln in the Bardo'' is his
  most recent one, ``Tenth of December,'' ``CivilWarLand in Bad
  Decline,'' ``Pastoralia'' and several others. He was my professor when
  I was getting my M.F.A. at Syracuse University. Now it was more than
  20 years ago that I met George. And I didn't really even read his
  writing before I applied there. What's interesting about it is so many
  people were applying there because they wanted to work with George.
  And I didn't. I hadn't read his work. But what happened is he called
  me up and he accepted me into the program. And we had these long
  conversations, and I was so struck by how wonderful he was that I went
  and read his books. And I was like, oh, he is a great writer. I'd love
  to work with him. So I did that. I accepted. And there was this
  graduate school picnic --- this sort of opening picnic, and I met him
  and his wife Paula and their two daughters who were really little kids
  at the time. I have this vivid image of George putting his hand on top
  of one of their heads. And now they're in their 20s and all grown up.
  But I knew that I was meeting somebody who was really, at essence, a
  really wonderful person. I think he's an extraordinary --- a great,
  great writer. But he's an even more extraordinary man. So I was
  thinking about George in this moment, because he has been such a
  guiding light to me and to so many. His humor, his compassion, his
  generosity, and his ability to take the long view, I feel like that is
  the voice I need to hear right now on the phone. So I'm going to give
  him a call.
\item
  {[}phone ringing{]}
\item
  george saunders\\
  Hello?
\item
  cheryl strayed\\
  George.
\item
  george saunders\\
  Cheryl, how you doing?
\item
  cheryl strayed\\
  Hi, it's Cheryl.
\item
  george saunders\\
  I know.
\item
  cheryl strayed\\
  How are you doing?
\item
  george saunders\\
  Good. We're hanging in here.
\item
  cheryl strayed\\
  Oh my gosh. I'm really so excited to get this chance to talk to you in
  the middle of a pandemic.
\item
  george saunders\\
  Yeah. It's a once in a lifetime.
\item
  cheryl strayed\\
  I hope so. How are you? First of all, where are you?
\item
  george saunders\\
  I'm in Corralitos, California. So we're just between Santa Cruz and
  Watsonville --- so kind of up on a hillside. And from the front of our
  house, you can see Watsonville and some of the ocean.
\item
  cheryl strayed\\
  Wow. So you're looking out over --- do you see trees? Do you see
  houses? What do you see?
\item
  george saunders\\
  Well, we're up in Redwoods, and so we see that. And then if you look
  far enough out, there's berry fields. And then beyond that, kind of
  dimly, there's the city of Watsonville. So it's really beautiful. And
  you'd never know there was a problem.
\item
  cheryl strayed\\
  Yeah, isn't that interesting? That's fascinating to me.
\item
  george saunders\\
  Yeah I was staying off media so much, and then I got back on it. And I
  just was kind of stunned by the numbers that they're talking about
  now. And I just can't imagine 100,000 people. That's just literally
  beyond my capacity to imagine it. And it's such a weird disconnect to
  have that in your head and your heart and then look outside and
  everything looks, from this perspective, normal. It's really strange.
\item
  cheryl strayed\\
  Yeah. There is this sense we've been lucky not to be sick, at least
  yet. But beyond us, there are really painful and hard and difficult
  things happening to a lot of people around the globe. And that, to me,
  I have to say, has been terrifying. And it's been distracting. And I'm
  curious how you've been responding to that emotionally.
\item
  george saunders\\
  Well, I think in some ways, I don't know, it's always happening. There
  is always misery. But I also think I've noticed about myself that in
  times like this, my mind wants to have answers for everything. It
  wants to have a take on things to give myself comfort. I think of it
  like when you slip on the ice and in the split second before you're
  about to hit the ground, that's really having no take. You're just out
  of control and the pavement's rushing up. So I think sometimes you
  just go, yeah, we're in that moment. We can pretend we can stop time
  and have a take on hitting the pavement or being mindful as we hit the
  pavement. But in fact, it's really unknown what's happening. And I
  think, especially for writers, us type A people, it's kind of
  difficult to be in a state where you just say I really don't know
  what's going on I have no control over it and no say over it and
  actually, very little influence at this point. That's a strange set of
  thoughts for anybody to have. But I think maybe for those of us who
  interpret for a living, it's especially both confusing and important,
  I think, to say, yeah, we don't know. We have to keep our sensory
  apparatus as open as we can so we don't miss any actual data. And to
  do that in the face of one's own anxiety is kind of difficult.
\item
  cheryl strayed\\
  It is. And as you're talking, you're reminding me, you told me about
  an email you wrote to your graduate students at Syracuse University,
  and I'm wondering if you could read to me what you wrote to them.
\item
  george saunders\\
  Oh. Sure. Sure. No, I'd be happy to. Yeah, it just goes like this.
  Dear, S.U. writers --- jeez, what a hard and depressing and scary
  time, so much suffering and anxiety everywhere. I saw this bee happily
  buzzing around a flower yesterday and felt like, ``Moron! If you only
  knew.'' But it also occurs to me that this is when the world needs our
  eyes and ears and minds. This has never happened before here --- at
  least not since 1918. We are, and especially you are, the generation
  that is going to have to help us make sense of this and recover
  afterwards. What new forms might you invent to fictionalize an event
  like this, where all of the drama is happening in private,
  essentially? Are you keeping records of the emails and texts you're
  getting, the thoughts you're having, the way your hearts and minds are
  reacting to this strange new way of living? It's all important. 50
  years from now, people the age you are now won't believe this ever
  happened or will do the sort of eye roll we all do when someone tells
  us about something crazy that happened in 1960. What will convince
  that future kid is what you are able to write about this. And what
  you're able to write about it will depend on how much sharp attention
  you're paying now and what records you keep, also, I think with how
  open you can keep your heart. I'm trying to practice feeling something
  like, ah, so this is happening now. Or hmm, so this, too, is part of
  life on Earth --- did not know that, universe. Thanks so much,
  stinker. And then I real quick tried to pretend I didn't just call the
  universe a stinker. I did a piece once where I went to live incognito
  in a homeless camp in Fresno for a week. Very intense, but the best
  thing I heard in there was from this older guy from Guatemala, who was
  always saying, ``Everything is always keep changing.'' Truer words
  were never spoken. It's only when we expect solidity, non-change, that
  we get taken by surprise. And we always expect solidity, no matter how
  well we know better. Well, this is all sounding a little preachy, and
  let me confess that I'm not taking my own advice --- at all. It's all
  happening so fast. Paula has what we are hoping is just a bad cold,
  and I'm doing a lot of inept caregiving. Our dogs can feel that
  something weird is going on --- no walk? Again? But I guess what I'm
  trying to say is that the world is like a sleeping tiger, and we tend
  to live our lives there on its back. We're much smaller than the
  tiger, obviously. We're like Barbies and Kens on the back of a tiger.
  Now and then, that tiger wakes up, and that is terrifying. Sometimes
  it wakes up when someone we love dies or someone breaks our heart or
  there's a pandemic. But this is far from the first time that tiger has
  come awake. He she has been doing it since the beginning of time and
  will never stop doing it. And always, there have been writers to
  observe it and later make some sort of sense of it --- or at least
  bear witness to it. It's good for the world for a writer to bear
  witness, and it's good for the writer too, especially if she can bear
  witness with love and humor and, despite it all, some fondness for the
  world, just as it is manifesting --- warts and all. All of this to
  say, there's still work to be done, and now more than ever. There is a
  beautiful story about the Russian poet Anna Akhmatova. Her husband was
  shot and her son arrested during the Stalinist purges. One day, she
  was standing outside the prison with hundreds of other women in
  similar situations. It's Russian cold, and they have to go there every
  day, wait for hours in this big, open yard, then get the answer that
  today and every day, there will be no news. But every day, they keep
  coming back. A woman, recognizing her as the famous poet, says, poet,
  can you write this? And Akhmatova thinks about it for a second and
  goes, yes. I wish you all the best during this crazy period. Someday
  soon, things will be back to some sort of normal, and it will be
  easier to be happy again. I believe this, and I hope it for each one
  of you. I look forward to seeing you all again and working with you,
  and even in time, with sufficient PPE, give you a handshake or a hug.
  Please feel free to email anytime for any reason, George.
\item
  cheryl strayed\\
  George, that's so beautiful. I mean, I know it's been now a couple
  decades, really --- has it been that long since I was your student?
\item
  george saunders\\
  No. Three years. Three years.
\item
  cheryl strayed\\
  But I feel like you wrote that to me. And I love this --- how do you
  say her name --- Akhmotiv ---
\item
  george saunders\\
  Akhmatova.
\item
  cheryl strayed\\
  Akhmatova. I love how Akhmatova says --- or the person standing with
  her says, poet, can you write this? And she says, I can. And I think,
  obviously, that is our job as writers, and it's the job of the artist
  to write that moment. It has been through all time, and it will be
  through all time. And I think that that also translates outside of the
  arts. It translates to, really, every human on the planet who has
  something to contribute about the story of us. And I think that that's
  what's so powerful about that moment of that understanding that, yes,
  we can bear witness, as you say. We don't have to be a novelist. We
  don't have to be a poet. We can do that even as a person.
\item
  george saunders\\
  Yeah. And also you get the feeling that she --- I think you're
  actually right. I think she says, I can. I don't have the text in
  front of me. But the beauty of that is to say, I hear her saying, I
  think I can. I think I can try. And that idea that if you're going to
  describe a moment like that, you'd really have to sort of keep moving
  the fences out. That's a beautiful story, because she includes the
  woman who's interrogating her. If she had just stopped at describing
  the weather in the jail, somehow it's not the full reality. So I think
  that, for me, is a beautiful challenge is to keep saying everything
  that you think or feel or perceive, even if it seems totally trivial,
  might be part of the larger picture. It might be that in 100 years ---
  or not 100 years, but say when we're old people, you look back and it
  might be some totally trivial detail that actually encapsulates the
  whole thing. But you won't know until you get there. And to know, you
  have to record it along the way, I guess.
\item
  cheryl strayed\\
  So this idea --- one thing I'm struck by that both of us keep sort of
  circling around and saying, this control. And I think both of us as
  writers who have invented characters and made them do things, we know
  that, actually, this idea that we can control really anything is just
  an absolute illusion. And yet, it's incredibly hard for us to wrap our
  minds around, right? How are you absorbing that reality that you don't
  know if you're going to be able to go hang out with your friends in a
  month. You don't know if you're going to be able to do that in six
  months. We don't really know when we can all make plans again. So I'm
  wondering, what sense do you make of that?
\item
  george saunders\\
  Well, for me what's interesting is I was really enjoying canceling
  things. Even before the pandemic, I was just like, I'm going to say no
  to everything. And I was getting a little bit ecstatic about that ---
  that I could just be a family person and sit in my room and write. And
  I'm like, that's what I really want to do anyway. So in a way, it's
  weird, I'm still kind of in the place where, for most things, the
  canceling of plans doesn't really make me very anxious. Except the one
  weird thing is I can't see my parents or my kids. And that bothers me,
  because they're both in different cities --- my kids aren't. So that's
  bad. But actually, I think it just caught me at a funny moment where
  I'm happy to stay home. And then I think for me, the control thing has
  to do with it's a personality disorder that I have. I found a way, I
  think, in my work to use it kind of beneficially. In that little
  world, you're absolutely responsible for controlling every line. I
  think that's what I'm finding out. I don't know if that resonates with
  what happens when you write.
\item
  cheryl strayed\\
  Well, and I think too, there is this sense of --- I don't know,
  there's something about paying attention too. When you're writing,
  you're so focused. And that has been one of the primary challenges for
  me is distraction, I find that it's very hard to focus unless I become
  absorbed in something. And writing surely does that. But what about
  other stuff? Are you baking bread and taking up new things? Or are you
  and Paula just laying low?
\item
  george saunders\\
  Well, I mean, Paula has been a little sick. So I've been kind of the
  inept caregiver, as I said in that letter. And that's been going on
  about eight or nine days. And so I really was working pretty well. But
  now I had a story in The New Yorker, so I closed that and did some of
  the peripheral stuff. But I've just been kind of cooking and cleaning
  in my comical American male way. So the thing I'm starting to notice
  --- just micro notice --- is I think I'm a little reliant on that
  state of concentration that you're talking about. That when you go
  into a story, what it feels like to me as I'm just obsessing over the
  small details of the text. I'm starting to think that that's some kind
  of neurological state that's akin to, but not exactly equal to,
  meditation. And it really makes me happy in a kind of not euphoric,
  but it makes me solid. It makes me a little more positive as a person.
  But having found out the writing in that state of mind increases my
  happiness, then it's kind of my responsibility to get into that more
  --- as much as I can.
\item
  cheryl strayed\\
  Do you meditate?
\item
  george saunders\\
  Yeah. Yeah. And we're doing some prayers during this. And yeah, so
  that's been part of our lives for a long time and a very reliable one.
  And that's one of the things about this moment that is interesting is
  that the anxiety that we're talking about is basically people saying,
  I want things to be the way they were. I want to live. I want to not
  be sick. I want to not have to worry about getting somebody else sick.
  I want to go to a bar. But really, what occurs to me is, why? What was
  it that you want to do once you are free again? And for most of us ---
  I think for me, it's just, I want to assume my old habits. Why?
  Because they assuage my anxiety, essentially. And so it does present a
  little bit of a moment to look afresh it what we do with our energy
  when we're healthy. And I know this is a trivial example --- we live
  kind of about half an hour out of town. So it's kind of a big trip to
  go into the store. But we go just about every day, I would say ---
  just out of habit. Well this last period, we haven't been in two
  weeks. And we're eating better probably, you know? So it does make me
  think so many of the ways that we live are kind of just lazy and
  habitual from that small example to the larger things. So I'm trying
  to be optimistic and think that maybe each of us having a little
  enforced spring break and if we're healthy, god willing, it might make
  us look a little differently at the American life that we all live ---
  the innate laziness and violence of it and the habitual part of it ---
  that most of us just accepted and try to make the best of it. So maybe
  a little bit of a breather for the world to go, wait a minute. What
  are we doing? That would be nice. We'll see.
\item
  cheryl strayed\\
  I think you're really right. I think you might remember that I grew up
  in rural Minnesota without running water or electricity or indoor
  plumbing. And my mom was really this incredibly kind of
  self-sufficient person and my stepfather, incredibly self-sufficient,
  grew a lot of our food. And she preserved and canned and pickled
  everything. She baked bread. She knew how to do everything in the
  domestic realm. And I refused to learn any of that, because I wanted
  to. Because I didn't want to be a domestic goddess. I wanted to be a
  great American writer. And I wanted to sort of mark my space in a
  place that wasn't traditionally sort of defined for women. But, boy,
  have I thought about my mom so much in these last couple of weeks
  where I am not getting dinner from the grocery store kind of deli
  counter. And I have these old books of my mom's --- Stocking Up and
  these kind of food preservation books. And I'm opening them for the
  first time. And so that's been kind of interesting. And you're right,
  it's been a little bit of a corrective, because I don't necessarily
  want to be somebody who has to spend two hours making dinner every
  night. But it's kind of been good for me to have to stretch in that
  direction.
\item
  george saunders\\
  Yeah, that's a beautiful story. It's almost like your family kind of
  enacted the last 80 years of American history in one generation, where
  at some point, people could do all that stuff. And then I remember as
  a kid in Chicago, it was so cool to get Jello or to go to Burger King.
  It just seemed like canning your own food was so last century. But
  yeah, no, that's a beautiful --- that's a beautiful idea.
\item
  cheryl strayed\\
  It's true.
\item
  george saunders\\
  I don't know if you're --- are you a Wendell Berry fan?
\item
  cheryl strayed\\
  I am, yeah.
\item
  george saunders\\
  Well, I just was thinking how prescient he is and he had this idea
  that, as his one book is titled, ``It All Turns On Affection.'' And
  it's made me think how many hours of our day, of our working lives are
  spent doing something that we feel connected to, that's really ours,
  you know? And I think that his words, Berry's words, are really
  speaking to me now and saying, well, maybe this is a moment when we
  could at least a little bit investigate moving backwards on that arc
  and taking a little more responsibility for the way we live in the
  world and also minimizing the small violences that he talks about ---
  the kind of inquisitiveness and the idea that what we're going to do
  in this world is exploit whatever's out there to make it easy on
  ourselves. But I thought I'd read his poem. He wrote this for his
  wife, I assume, and it's called ``The Wild Rose.'' Sometimes hidden
  from me in daily custom and in trust so that I live by you unaware, as
  by the beating of my heart. Suddenly, you flare in my sight --- a wild
  rose looming at the edge of thicket --- grace and light where
  yesterday was only shade. And once again, I am blessed, choosing again
  what I chose before.
\item
  cheryl strayed\\
  Wow. Beautiful.
\item
  george saunders\\
  Yeah, just the idea that this --- I was out. I took the dogs on a
  walk. And I mean, it sounds kind of corny, but I was like, oh, trees.
  They really hit me in a different way than they had before. And when I
  was a kid, I read ``Walden'' for the first time on the south side of
  Chicago and then had this kind of cheesy experience of walking through
  a forest. But suddenly, the whole landscape was different because
  Thoreau had written about it. And I think that's happening a bit now.
  You're like, the clouds --- the clouds are still there. They still
  like us.
\item
  cheryl strayed\\
  And we can still commune with it. I mean, I think that that's
  something really powerful about this experience you're talking about
  is that, OK, we have to take a step back from each other, but there
  are these other things that we can step closer to.
\item
  george saunders\\
  Yeah.
\item
  cheryl strayed\\
  Maybe it's the thing we refused to learn in our childhood.
\item
  george saunders\\
  Right.
\item
  cheryl strayed\\
  Like me learning from my mom how to pickle a cucumber or really those
  things I wrote so much about in ``Wild'' --- the way that the natural
  world can feel like our home, that we are not in opposition to it, and
  that we can --- I had a similar experience when I was on a walk the
  other day. And I was looking at the trees and I thought, they don't
  have this virus. They're impervious to this virus. And I felt some
  sort of connection, some sort of sense of comfort emanating from just
  the fact that they were safe from this and that they would stand watch
  over us as they always have. That there is a way that they do that.
\item
  george saunders\\
  This is a funny look at the difference in our minds, because I was
  going to say you felt resentment. Bastards.
\item
  cheryl strayed\\
  Well, no.
\item
  george saunders\\
  No, I think that's exactly right.
\item
  cheryl strayed\\
  For the record, George Saunders, you're the only one calling a bee a
  moron. I would never do that.
\item
  george saunders\\
  Yeah. You have to meet this bee. This particular bee was really bad.
  No, the other thing that I was struck by too is just that maybe it's
  nice to think that this will adjust things in a positive way. But it
  sure is making clear how ridiculous our economy is and the wealth
  differential. And you think about people who have to take a week off
  of work and can't afford it. And I think it's underscoring so many
  pre-existing problems with our country, from the top to the bottom.
  And it's stressing every aspect of what we are, and that's
  disconcerting.
\item
  cheryl strayed\\
  Well, and it's asking us to also then define who we are. And I think
  that the opportunity here is to move in the direction of empathy and
  compassion. And the question is, will we take it? Will those who have
  privilege and power use it for good or use it for, I hate to cast it
  in this way, but for evil? And I do think there's evidence in both
  directions about the path we'll take. What do you think is going to
  happen?
\item
  george saunders\\
  I don't know. For me, what comes to my mind is sort of like, well, in
  a sense, we're asking how did we get in this position in the first
  place? And my fear is that people are going to be so relieved when
  it's quote unquote ``over,'' I'm sure we'll just jump right back into
  the same water. So I don't know. So it could be that maybe something
  good will happen. And I'm certainly rooting for that. But I don't
  know. I don't know.
\item
  cheryl strayed\\
  So, George, I read something on Twitter --- you're not on Twitter, are
  you?
\item
  george saunders\\
  No.
\item
  cheryl strayed\\
  Oh my god.
\item
  george saunders\\
  Or should I say, no. No. I'm not on any social media.
\item
  cheryl strayed\\
  I know. You were for a while, and then you disappeared. And I was glad
  to hear it wasn't you unfollowed me because I'm insufferable. It was
  because you actually ---
\item
  george saunders\\
  Oh, no, no. I had a little Facebook author page, and even that drove
  me up the wall. The joke I always make is I trained myself for years
  to write slowly for a lot of money, so I have no interest in writing
  quickly for free.
\item
  cheryl strayed\\
  Yeah. Well, see, I'm one of those sort of superficial simpletons who
  I'm on social media. I love social media.
\item
  george saunders\\
  Yeah, but again, you do it so authentically that it's lovely. So I
  think everybody has to assess their own power.
\item
  cheryl strayed\\
  Thank you. But let me tell you --- one of the best things, as you
  probably are aware, there are some really cool things on social media.
  And one of my favorite things on Twitter is to follow Carl Reiner.
  He's absolutely brilliant and sweet and hilarious. He turned 98 just a
  few days ago. And he ---
\item
  george saunders\\
  Happy birthday. Mazel tov.
\item
  cheryl strayed\\
  Can you believe this? He tweeted this, and it really stopped my heart.
  This is what he said --- for the first time in memory, I see nothing
  in this world about which I care to joke. And it felt true to me, and
  it felt so sad to me that this 98-year-old man, this funny man, this
  comedian was saying I see nothing about which to joke. And I'm
  wondering what you think of that. Your life is full of so much humor,
  your work is full of so much humor.
\item
  george saunders\\
  I think --- when I was doing all the research on Lincoln, I found
  something interesting. He was majorly depressed. And the only way he
  could get himself out of it is to just joke --- just consciously say
  I'm going to tell some jokes, and that's going to make me feel better.
  So I think we used to think that jokes or humor is a response to being
  happy. You're happy so you make a joke. And that's sometimes true for
  me. But I've also noticed for myself that to allow myself to make a
  joke in a given situation does something to my disposition. It just
  lightens it in a certain way or it activates it, I guess I would say.
  So my hope would be that Carl Reiner, who I adore, was having a bad
  day. And if you got him into a situation, I'm hoping --- I'm hoping
  and praying that he would still find the joke, because it would be
  there for him. I think too, part of it is if you think about the way
  we're processing this thing, most of us are seeing no direct evidence
  of it. I saw it. The other day I was driving out of my house, and I
  saw this beautiful little Rockwellian picture --- these two couples
  were standing across from each other on the road I was driving down,
  shouting out to each other and laughing and having a conversation ---
  with the road being their social distancing. So most of us aren't
  seeing this. It's just sort of theoretical. We're seeing it in the
  media. We're sympathizing from afar, and so on. And so I think for me,
  I'm trying to remember that it's OK to feel whatever you're feeling.
  If you find a joke, go ahead. Or if you find yourself forgetting about
  it for a couple hours and feeling happy, of course. But I hope he
  feels differently. And if he doesn't, that's OK too.
\item
  cheryl strayed\\
  I just want to say, George, rest assured --- of course, two days
  later, I went on Twitter, and there was Carl Reiner, and he made me
  laugh. There was a joke. He made a tweet. He tweeted a joke.
\item
  george saunders\\
  So he was being beautifully honest about what he was feeling at that
  moment. And I love that. That's fantastic. Yeah. I would argue that
  even during the worst time, there are pockets of beauty and pockets of
  pleasure. And it seems to me one thing that is a danger in our time of
  social media and cable news is that we accept the world's, meaning the
  media's, narrative as all pervasive and correct. And that bee that I
  talked about, that was a real bee. And he was having a good day. So in
  a certain way, I think for a writer, the complex chore of imagination
  is to say the world is not a complete fabric of misery right now. It's
  pockets of incredible misery. And so to sort of hold that complexity
  in the brain is difficult. But I think it's a form of moral
  responsibility to say, I don't have to accept an outside
  interpretation of what's a very complex reality, especially if it
  makes me more anxious and less helpful and depressed. But I think to
  be mindful of what pleasures there are is a really good way to make
  sure that when you're needed, you'll be ready --- something like that
  maybe.
\item
  cheryl strayed\\
  And there's beauty in every day. We know that for certain, no matter
  what is happening. And so yeah, I think it is about being able to hold
  many complex ideas that seem sometimes to contradict themselves in one
  hand. And probably this moment simply amplifies, again, what was
  always true. There was always sorrow. There was always violence. There
  was always death and disease and destruction and loss. And there's
  always beauty. And if we choose to be there to see that, we get to be
  the one to see that.
\item
  george saunders\\
  Yeah, we can hold a lot of those ideas at once. I mean, we
  theoretically can --- I can't, but one could. I think Fitzgerald said
  something like that --- that's the mark of an intelligent person is
  the ability to hold two contradictory ideas in mind at the same time.
  So I think that's another thing that reading and writing helps us
  train us. It's like training wheels to almost like a plate-spinner. At
  the beginning, you can't even spin one plate. Well, with training, you
  could do a few. And I think for me, that's what reading and writing
  both do.
\item
  cheryl strayed\\
  Yeah, I really think that story is an essential part of our ability to
  survive and thrive. I really do. We use that phrase a lot --- books
  will save us or story will save us or art saves. And it sounds trite
  or sort of self promotional, especially when we writers are saying,
  no, stories will save us. But the fact is, it's true. This is the
  reason I'm calling you, because, really, when I search my soul and
  think, how have I reckoned with the things that were most difficult?
  How have I learned how to be brave and accept and move on and grieve
  and carry my burdens with maybe a tiny bit more grace? And the answer
  is, no question, it's story. And I do think that's the genuine power.
\item
  george saunders\\
  Yeah. We go through this life so fast, and we don't really have time
  to learn the lessons. But some of them are enshrined in stories. We
  had a thing --- a couple of weeks ago, my dad and I were e-mailing
  about this epidemic. And he said that in Chicago in the 1918 epidemic,
  something like 8,000 people died in that, at that time, smaller city.
  And he said that in the family lore, there was a story that his
  grandmother had had a son who died in the epidemic. And sort of his
  idea was that the body had been left on the street, because things
  were so crazy. So we started talking about that, and my dad and I and
  a cousin and my aunt kind of put our heads together and started doing
  some research. And it turns out, yeah, the kid actually, we always
  heard he was eight. In fact, he was three. And his name was Leroy
  Gendrow. And the story was that he suffered. And so the family could
  hear, and his mother --- my father's grandmother --- screamed and
  wailed at his death. And then because the system was so overwhelmed,
  they had to wrap this little boy up in a sheet in which he died and
  take it out to the street. And one of the details we found out was the
  truck would come by every morning, which gives you some sense of the
  scale of this thing. And so my dad wrote this beautiful email to me
  saying that basically, he'd lived with that story all his life ---
  kind of half heard at a family party. And I think it was painful for
  the family and they suppressed it. And it was during World War I, and
  so people kind of moved on. But the kid never got a grave. It was a
  mass burial, you know? And my dad said how moving it was it now that
  little boy had a name. He had a name --- Leroy Gendrow. He had an age.
  We were able to find out that he was buried at Mount Olivet cemetery
  on the South Side. So I was thinking about why that additional detail
  was important. Now, when I was doing the Lincoln book, I read
  somewhere that the dead actually like to be remembered. They like to
  be verbally recalled, they like their picture to be up. But even
  leaving that aside, why was that satisfying to my family to be able to
  speak this kid's name aloud? And I don't know. I mean, I don't know if
  you have thoughts on that. But it was just, as Carver said, a small
  good thing to be able to speak that kid's name, you know?
\item
  cheryl strayed\\
  Well, and I think that's everything about what we're saying about the
  power of story. Because, of course, what is a story but that we get to
  hear one report of a human existence or a human experience? And I was
  just saying to my kids the other day, some number of people who have
  died of the COVID-19, I was saying OK, that's, the equivalent of this
  entire portion of the city of Portland --- that they're all dead.
  Because it's abstract. Even this family story you'd heard, like, oh,
  my father's grandmother's son --- who is he to you? He's nobody. But
  then he becomes Leroy, and it becomes that family in that apartment
  and that truck that comes every day and that body on the street. This
  is how we build empathy and compassion is that we listen really hard
  to the stories all around us. We see the people around us. So there's
  this grand scale experience, but maybe we live it on the human scale,
  maybe we look again at the people around us with maybe more love.
\item
  george saunders\\
  Yeah. It's funny how the challenge of this moment is similar to the
  fictional challenge, which is can you extrapolate from someone else's
  experience to your own viscerally? And it's kind of hard. What is that
  quote about 100,000 deaths is a statistic, one death is a tragedy. I'm
  probably getting that wrong. Especially in this way that we're
  isolated, now to say, OK, I'm going to actively work on it as part of
  my ethical spiritual life to try to imagine that not everything is
  fine everywhere. And it's the great human struggle, but maybe
  exaggerated at a moment like this. I don't know, it's kind of amazing.
  It's just stunning in a way to think about what's happening, even as
  we speak.
\item
  cheryl strayed\\
  So, George, are you afraid --- one of the things that's come up for me
  --- I'm 51 ---
\item
  george saunders\\
  You're a baby.
\item
  cheryl strayed\\
  I'm a baby. How old are you?
\item
  george saunders\\
  61.
\item
  cheryl strayed\\
  61. OK. So we're not --- yeah, we're just a decade apart here. But my
  mom died when she was 45. I was 22. And I thought, OK, I'm going to
  die young. I'm not going to be 45. I'm not going to reach that age
  without dying. It was kind of like a little superstition that I share
  with a lot of people who lost parents young. They have that same
  thing. When they reach the age that their parent died, it's always a
  thing, right? It's a passage. And here I am, I made it to the other
  side. I'm 51 and here I am. And then this pandemic came along, and one
  of my thoughts when I am in my kind of scaredest place, which I've
  gone to a few times over this last couple of weeks, is I think, see, I
  was right. I am going to die young, and my kids are going to be
  orphans --- or at least motherless, like I was. That they're going to
  suffer like I've suffered. And I have to admit, I don't think of
  myself as somebody who's anxious or much thinking about --- I'm not
  afraid of things very often. And yet suddenly, I felt like I'm afraid
  I'm going to die of this COVID-19. I'm afraid. Are you afraid?
\item
  george saunders\\
  I mean, in my gut, I'm kind of a Pollyanna. And we're being super
  careful. We're basically like Clarks watching the birds as they fly by
  our house, you know? So in my gut, I don't think so. But I think this
  maybe ties in with what we're saying about stories is you look out at
  the world and you read these accounts, even though they're quick and
  they're sort of at this moment kind of surface accounts. And you have
  to start saying it's possible. It's possible. So I think that's part
  of the responsibility is to be careful, but also go yeah, you can read
  his accounts of people who, two weeks ago, were high functioning
  people running around happy and they died. So I think part of it is to
  try to get that into your body and your mind on a daily basis that
  it's temporary.
\item
  cheryl strayed\\
  Hmm. I think the thing is that we keep circling around is everything
  that's scary about this moment has existed all this time. This whole
  51 years I've been alive and the whole 61 you've been alive, this has
  always been true.
\item
  george saunders\\
  Yeah.
\item
  cheryl strayed\\
  Right?
\item
  george saunders\\
  Right.
\item
  cheryl strayed\\
  I think what's happening for me and you and everyone right now is it's
  amplified, and suddenly it's on our doorstep. So in some ways, it
  functions as it's this real thing that's happening. It's also a
  metaphor --- a metaphor, really, for the truth of human existence,
  which is that we're mortal, and that we don't have control, and that
  we have to simply try our best, keep the faith, and maybe pray to the
  divinity in each other and honor the divinity that is within each of
  us.
\item
  george saunders\\
  Yeah. I think it does remind me so much of 9/11 and that feeling --- I
  think it's called sympathetic compassion where you're not in danger
  yourself, but you can imagine the fear and danger someone else is in.
  And that longing to want to do something, or really, in its most
  profound, you're longing that that person not suffer, you're longing
  that that person be happy. And that's very profound. And I think if we
  can cultivate that feeling of wishing the best, that's such a powerful
  thing. And I think that's what we're here to do all the time. But in
  these situations, you feel it. We talk about anxiety --- the anxiety
  of the moment --- and I've been trying to think that some of that is
  useless. Some of it is just neurosis. But part of the anxiety is kind
  of like I read somewhere that there's a meditation you can do where
  you imagine a person that you love very much drowning just beyond your
  reach. And that feeling that comes up when you do that is actually
  compassion. I'm sure another part of that meditation is you imagine
  somebody who isn't so close to you drowning out of your reach, and you
  could actually grow your love in that way, I think. That part of this,
  I think, is I'm trying to think about the usefulness of that --- the
  fact that you could cultivate a feeling of concern for other people,
  and that that concern could get larger. So I suppose if you're looking
  for something that this moment conveys that isn't negative, the
  potential for that might be, as you said, to pray and to try to lure
  out the better parts of ourselves.
\item
  cheryl strayed\\
  Yeah --- to rise to that. OK, well, George, you know what? It has been
  enriching to my soul to hear your voice and to chat with you and to
  catch up. That's what I think has been so striking to many of us is,
  like, how certain it is that we need our friends during this time and
  that human connection. So thank you. Thank you for talking to me and
  being here with me today on the phone.
\item
  george saunders\\
  It's been such a pleasure for me. We miss you so much, and can't wait
  to sit down in person and get within the six-foot radius and all that.
\item
  cheryl strayed\\
  You know, and I'm going to even look out further than that to the day
  that we can actually give each other a hug.
\item
  george saunders\\
  Now, let's not get crazy. Let's not get crazy here. No, and give my
  love to your family. Thank you so much for asking me. I enjoyed every
  minute. I always come away from conversation with you believing that
  much more in writing. So thank you for being out there.
\item
  cheryl strayed\\
  Oh thank you. Love to your family too, George. Bye bye.
\item
  george saunders\\
  All right. Bye bye.
\item
  cheryl strayed\\
  I'm Cheryl Strayed, and this is ``Sugar Calling.'' Next week, Margaret
  Atwood. Thanks for listening.
\end{itemize}

Previous

More episodes ofSugar Calling

\href{https://www.nytimes3xbfgragh.onion/2020/05/20/podcasts/sugar-calling-joy-harjo-poetry-virus.html?action=click\&module=audio-series-bar\&region=header\&pgtype=Article}{\includegraphics{https://static01.graylady3jvrrxbe.onion/images/2020/05/22/podcasts/20sugar-hajo3/20sugar-hajo3-thumbLarge.jpg}}

May 20, 2020~~•~ 35:30`I Release You, Fear'

\href{https://www.nytimes3xbfgragh.onion/2020/05/13/podcasts/sugar-calling-billy-collins-poetry-virus.html?action=click\&module=audio-series-bar\&region=header\&pgtype=Article}{\includegraphics{https://static01.graylady3jvrrxbe.onion/images/2020/05/13/podcasts/13sugar-calling/13sugar-calling-thumbLarge.jpg}}

May 13, 2020`There's a Quiet All Over the World'

\href{https://www.nytimes3xbfgragh.onion/2020/05/06/podcasts/sugar-calling-alice-walker-quarantine-virus.html?action=click\&module=audio-series-bar\&region=header\&pgtype=Article}{\includegraphics{https://static01.graylady3jvrrxbe.onion/images/2020/05/06/podcasts/06sugarcalling/06sugarcalling-thumbLarge.jpg}}

May 6, 2020~~•~ 28:58`Whatever We Have, We Have to Work With It'

\href{https://www.nytimes3xbfgragh.onion/2020/04/29/podcasts/sugar-calling-judy-blume-quarantine-virus.html?action=click\&module=audio-series-bar\&region=header\&pgtype=Article}{\includegraphics{https://static01.graylady3jvrrxbe.onion/images/2020/04/29/podcasts/29sugarcalliing-blume-sub/29sugarcalliing-blume-sub-thumbLarge.jpg}}

April 29, 2020`This Terrible Thing Is Happening, but the World Goes On.'

\href{https://www.nytimes3xbfgragh.onion/2020/04/22/podcasts/sugar-calling-amy-tan-quarantine-virus.html?action=click\&module=audio-series-bar\&region=header\&pgtype=Article}{\includegraphics{https://static01.graylady3jvrrxbe.onion/images/2020/04/27/podcasts/22sugarcalling/22sugarcalling-thumbLarge.jpg}}

April 22, 2020~~•~ 39:19`You Don't Take Dictation. You Find the Truth.'

\href{https://www.nytimes3xbfgragh.onion/2020/04/15/podcasts/sugar-calling-pico-iyer-coronavirus.html?action=click\&module=audio-series-bar\&region=header\&pgtype=Article}{\includegraphics{https://static01.graylady3jvrrxbe.onion/images/2020/04/21/podcasts/15sugarcalling1/15sugarcalling1-thumbLarge.jpg}}

April 15, 2020~~•~ 35:45`Joyful Participation in a World of Sorrows'

\href{https://www.nytimes3xbfgragh.onion/2020/04/08/podcasts/sugar-calling-margaret-atwood-coronavirus.html?action=click\&module=audio-series-bar\&region=header\&pgtype=Article}{\includegraphics{https://static01.graylady3jvrrxbe.onion/images/2020/04/02/books/08sugarcalling1/08sugarcalling1-thumbLarge-v3.jpg}}

April 8, 2020~~•~ 34:32`Roll Up Your Sleeves, Girls'

\href{https://www.nytimes3xbfgragh.onion/2020/04/03/podcasts/sugar-calling-george-saunders-coronavirus.html?action=click\&module=audio-series-bar\&region=header\&pgtype=Article}{\includegraphics{https://static01.graylady3jvrrxbe.onion/images/2020/04/09/podcasts/03sugarcalling-image/merlin_171264408_4ac7fc67-d8cc-45b9-9ec6-bdd20672e694-thumbLarge.jpg}}

April 3, 2020~~•~ 41:16`Everything Is Always Keep Changing'

\href{https://www.nytimes3xbfgragh.onion/column/sugar-calling}{See All
Episodes ofSugar Calling}

Next

Published April 3, 2020Updated April 29, 2020

\begin{itemize}
\item
\item
\item
\item
\item
\end{itemize}

\emph{\textbf{Listen and subscribe to our podcast from your mobile
device:}}
\textbf{\href{https://podcasts.apple.com/us/podcast/sugar-calling/id1505881384}{\emph{Via
Apple Podcasts}}} \emph{\textbf{\textbar{}}}
\textbf{\href{https://open.spotify.com/show/4U8hPiNGIBvTS9zLeiDCN7?si=gRyigD47SPWl-QWgNjgt2w}{\emph{Via
Spotify}}} \emph{\textbf{\textbar{}}}
\textbf{\href{https://www.stitcher.com/podcast/the-new-york-times/sugar-calling}{\emph{Via
Stitcher}}}

\hypertarget{there-was-always-sorrow-there-was-always-violence-there-was-always-death-and-disease-and-destruction-and-loss-and-theres-always-beauty-and-if-we-choose-to-be-there-to-see-that-we-get-to-be-the-ones-to-see-that}{%
\subsection{`There was always sorrow, there was always violence. There
was always death and disease and destruction and loss. And there's
always beauty. And if we choose to be there to see that, we get to be
the ones to see
that.'}\label{there-was-always-sorrow-there-was-always-violence-there-was-always-death-and-disease-and-destruction-and-loss-and-theres-always-beauty-and-if-we-choose-to-be-there-to-see-that-we-get-to-be-the-ones-to-see-that}}

\emph{--- Cheryl Strayed, author}

On our first episode of ``Sugar Calling,'' Cheryl calls her old friend
and mentor from graduate school, the author George Saunders. Today, we
listen in on their conversation.

\includegraphics{https://static01.graylady3jvrrxbe.onion/images/2020/04/09/podcasts/03sugarcalling-image/merlin_171264408_4ac7fc67-d8cc-45b9-9ec6-bdd20672e694-articleLarge.jpg?quality=75\&auto=webp\&disable=upscale}

\begin{center}\rule{0.5\linewidth}{\linethickness}\end{center}

\textbf{On today's episode:}

\begin{itemize}
\item
  \href{http://www.georgesaundersbooks.com/}{George Saunders} is the
  author of nine books, including ``Tenth of December,'' a finalist for
  the National Book Award. He teaches creative writing at Syracuse
  University.
\item
  \href{http://www.cherylstrayed.com/}{Cheryl Strayed} is the author of
  four books, including ``Wild'' and ``Tiny Beautiful Things.'' She was
  the co-host of
  \href{https://www.nytimes3xbfgragh.onion/column/dear-sugars}{The New
  York Times/WBUR podcast ``Dear Sugars,''} and an author of
  \href{https://www.nytimes3xbfgragh.onion/column/the-sweet-spot}{``The
  Sweet Spot,''} an advice column that ran in The Times's Styles
  section.
\end{itemize}

\begin{center}\rule{0.5\linewidth}{\linethickness}\end{center}

\emph{\textbf{Note}}*:* ****** \emph{In this episode, George Saunders
refers incorrectly to the details of a killing. The victim was the
ex-husband of the poet Anna Akhmatova, not her husband, and the killing
occurred in pre-Stalin Russia, not during the Stalinist purges.}

\begin{center}\rule{0.5\linewidth}{\linethickness}\end{center}

Cheryl Strayed is the author of ``Tiny Beautiful Things,'' ``Torch,''
``Brave Enough,'' and the New York Times best seller ``Wild.'' Her books
have been translated into more than 40 languages. She lives in Portland,
Oregon. @CherylStrayed

``Sugar Calling'' is produced by Kelly Prime and Austin Mitchell, and
edited by Sara Sarasohn, with editorial oversight by Wendy Dorr and Lisa
Tobin. This episode was mixed by Brad Fisher and scored by Dan Powell.

Advertisement

\protect\hyperlink{after-bottom}{Continue reading the main story}

\hypertarget{site-index}{%
\subsection{Site Index}\label{site-index}}

\hypertarget{site-information-navigation}{%
\subsection{Site Information
Navigation}\label{site-information-navigation}}

\begin{itemize}
\tightlist
\item
  \href{https://help.nytimes3xbfgragh.onion/hc/en-us/articles/115014792127-Copyright-notice}{©~2020~The
  New York Times Company}
\end{itemize}

\begin{itemize}
\tightlist
\item
  \href{https://www.nytco.com/}{NYTCo}
\item
  \href{https://help.nytimes3xbfgragh.onion/hc/en-us/articles/115015385887-Contact-Us}{Contact
  Us}
\item
  \href{https://www.nytco.com/careers/}{Work with us}
\item
  \href{https://nytmediakit.com/}{Advertise}
\item
  \href{http://www.tbrandstudio.com/}{T Brand Studio}
\item
  \href{https://www.nytimes3xbfgragh.onion/privacy/cookie-policy\#how-do-i-manage-trackers}{Your
  Ad Choices}
\item
  \href{https://www.nytimes3xbfgragh.onion/privacy}{Privacy}
\item
  \href{https://help.nytimes3xbfgragh.onion/hc/en-us/articles/115014893428-Terms-of-service}{Terms
  of Service}
\item
  \href{https://help.nytimes3xbfgragh.onion/hc/en-us/articles/115014893968-Terms-of-sale}{Terms
  of Sale}
\item
  \href{https://spiderbites.nytimes3xbfgragh.onion}{Site Map}
\item
  \href{https://help.nytimes3xbfgragh.onion/hc/en-us}{Help}
\item
  \href{https://www.nytimes3xbfgragh.onion/subscription?campaignId=37WXW}{Subscriptions}
\end{itemize}
