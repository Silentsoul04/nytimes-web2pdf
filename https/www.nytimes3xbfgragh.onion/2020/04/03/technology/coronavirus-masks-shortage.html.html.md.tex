\href{/section/technology}{Technology}\textbar{}It's Bedlam in the Mask
Market, as Profiteers Out-Hustle Good Samaritans

\url{https://nyti.ms/3dRqIGo}

\begin{itemize}
\item
\item
\item
\item
\item
\item
\end{itemize}

\hypertarget{the-coronavirus-outbreak}{%
\subsubsection{\texorpdfstring{\href{https://www.nytimes3xbfgragh.onion/news-event/coronavirus?name=styln-coronavirus-markets\&region=TOP_BANNER\&block=storyline_menu_recirc\&action=click\&pgtype=Article\&impression_id=41c13230-efb9-11ea-b9cb-2d2e06a937fe\&variant=undefined}{The
Coronavirus
Outbreak}}{The Coronavirus Outbreak}}\label{the-coronavirus-outbreak}}

\begin{itemize}
\tightlist
\item
  live\href{https://www.nytimes3xbfgragh.onion/2020/09/05/world/coronavirus-covid.html?name=styln-coronavirus-markets\&region=TOP_BANNER\&block=storyline_menu_recirc\&action=click\&pgtype=Article\&impression_id=41c13231-efb9-11ea-b9cb-2d2e06a937fe\&variant=undefined}{Latest
  Updates}
\item
  \href{https://www.nytimes3xbfgragh.onion/interactive/2020/us/coronavirus-us-cases.html?name=styln-coronavirus-markets\&region=TOP_BANNER\&block=storyline_menu_recirc\&action=click\&pgtype=Article\&impression_id=41c13232-efb9-11ea-b9cb-2d2e06a937fe\&variant=undefined}{Maps
  and Cases}
\item
  \href{https://www.nytimes3xbfgragh.onion/interactive/2020/science/coronavirus-vaccine-tracker.html?name=styln-coronavirus-markets\&region=TOP_BANNER\&block=storyline_menu_recirc\&action=click\&pgtype=Article\&impression_id=41c15940-efb9-11ea-b9cb-2d2e06a937fe\&variant=undefined}{Vaccine
  Tracker}
\item
  \href{https://www.nytimes3xbfgragh.onion/2020/09/02/your-money/eviction-moratorium-covid.html?name=styln-coronavirus-markets\&region=TOP_BANNER\&block=storyline_menu_recirc\&action=click\&pgtype=Article\&impression_id=41c15941-efb9-11ea-b9cb-2d2e06a937fe\&variant=undefined}{Eviction
  Moratorium}
\item
  \href{https://www.nytimes3xbfgragh.onion/interactive/2020/09/02/magazine/food-insecurity-hunger-us.html?name=styln-coronavirus-markets\&region=TOP_BANNER\&block=storyline_menu_recirc\&action=click\&pgtype=Article\&impression_id=41c15942-efb9-11ea-b9cb-2d2e06a937fe\&variant=undefined}{American
  Hunger}
\end{itemize}

\includegraphics{https://static01.graylady3jvrrxbe.onion/images/2020/04/05/business/00virus-mask4/00virus-mask4-articleLarge.jpg?quality=75\&auto=webp\&disable=upscale}

Sections

\protect\hyperlink{site-content}{Skip to
content}\protect\hyperlink{site-index}{Skip to site index}

\hypertarget{its-bedlam-in-the-mask-market-as-profiteers-out-hustle-good-samaritans}{%
\section{It's Bedlam in the Mask Market, as Profiteers Out-Hustle Good
Samaritans}\label{its-bedlam-in-the-mask-market-as-profiteers-out-hustle-good-samaritans}}

Hospitals, governments, do-gooders and hucksters are all competing.
Scams and prices are soaring.

Susan HoughtellingCredit...Shane Lavalette for The New York Times

Supported by

\protect\hyperlink{after-sponsor}{Continue reading the main story}

\href{https://www.nytimes3xbfgragh.onion/by/jack-nicas}{\includegraphics{https://static01.graylady3jvrrxbe.onion/images/2018/11/26/multimedia/author-jack-nicas/author-jack-nicas-thumbLarge.png}}

By \href{https://www.nytimes3xbfgragh.onion/by/jack-nicas}{Jack Nicas}

\begin{itemize}
\item
  Published April 3, 2020Updated May 7, 2020
\item
  \begin{itemize}
  \item
  \item
  \item
  \item
  \item
  \item
  \end{itemize}
\end{itemize}

\href{https://www.nytimes3xbfgragh.onion/es/2020/04/06/espanol/ciencia-y-tecnologia/precio-mascarillas-n95-coronavirus.html}{Leer
en español}

Last month, Susan Houghtelling, a hospital supply-chain manager in
upstate New York, was facing a shortfall of medical supplies when her
inbox suddenly flooded with offers.

There were advertisements for gallons of hand sanitizer, crates of
isolation gowns and, most crucially, pallets of
\href{https://www.nytimes3xbfgragh.onion/2020/05/07/health/masks-banned-n95-coronavirus.html}{N95
masks} --- perhaps the most sought-after product on the planet. All were
for prices that were multiples higher than what she normally paid.

``All of these people are coming out of the woodwork, and all of them
mysteriously now have access to an abundant supply,'' said Ms.
Houghtelling, who works for three hospitals owned by Arnot Health, based
in Elmira. She forwarded dozens of messages to The New York Times from
brand-new vendors. One offered her boxes of 50 surgical masks for \$70
each; she used to pay \$2.28.

One solicitor in particular caught her attention: Blank Industries, a
company that offered N95 masks for nearly \$5 each --- and only if Ms.
Houghtelling ordered a million. She figured it was a scam.

Blank Industries is a real company, but it's an ice-melt manufacturer in
Hudson, Mass. In an interview, Andrew Blank, the founder, said he had
upended his business to sell masks after hearing from a former Chinese
supplier he had once hired to make a new kind of toothbrush. (Mr. Blank
had invented it.) After the coronavirus hit, the supplier turned his
dental-products plant into a mask factory. Mr. Blank told his 12
employees to stop selling rock salt and start selling masks.

Why was he charging \$4.92 for each N95? ``To be honest, I don't even
know what an N95 normally sells for,'' he said.

I told him. ``50 cents?'' he repeated. His supplier was charging him
\$4.75. (His margin would cover shipping costs; he planned to take no
profit.)

The eruption in demand for dwindling amounts of masks has resulted in a
kind of global supply-chain bedlam. In the United States, the federal
government has decided against commandeering American factories to
create a new stream of masks. Instead, federal officials are competing
against states, hospitals and medical suppliers for the same pool of
masks, which come mostly from China.

Yet states and hospitals, whose typical suppliers are overwhelmed and
overextended, have little experience negotiating directly with the
Chinese supply chain. Thousands of middlemen --- entrepreneurs,
do-gooders and profiteers --- have rushed to fill the void.

That frenzy has created a mess of confusion, according to interviews
with hospitals, factories and mask buyers. Production of masks is
soaring, but so are scams, logistical hurdles and, of course, prices.

\hypertarget{latest-updates-the-coronavirus-outbreak-and-the-economy}{%
\section{\texorpdfstring{\href{https://www.nytimes3xbfgragh.onion/live/2020/09/04/business/stock-market-today-coronavirus?action=click\&pgtype=Article\&state=default\&region=MAIN_CONTENT_1\&context=storylines_live_updates}{Latest
Updates: The Coronavirus Outbreak and the
Economy}}{Latest Updates: The Coronavirus Outbreak and the Economy}}\label{latest-updates-the-coronavirus-outbreak-and-the-economy}}

\href{https://www.nytimes3xbfgragh.onion/live/2020/09/04/business/stock-market-today-coronavirus?action=click\&pgtype=Article\&state=default\&region=MAIN_CONTENT_1\&context=storylines_live_updates\#fed-chair-powell-says-interest-rates-are-likely-to-stay-low-for-years}{23h
ago}

\href{https://www.nytimes3xbfgragh.onion/live/2020/09/04/business/stock-market-today-coronavirus?action=click\&pgtype=Article\&state=default\&region=MAIN_CONTENT_1\&context=storylines_live_updates\#fed-chair-powell-says-interest-rates-are-likely-to-stay-low-for-years}{Fed
Chair Powell says interest rates are likely to stay low for years.}

\href{https://www.nytimes3xbfgragh.onion/live/2020/09/04/business/stock-market-today-coronavirus?action=click\&pgtype=Article\&state=default\&region=MAIN_CONTENT_1\&context=storylines_live_updates\#wage-data-remain-distorted-by-big-shifts-in-the-labor-market}{31h
ago}

\href{https://www.nytimes3xbfgragh.onion/live/2020/09/04/business/stock-market-today-coronavirus?action=click\&pgtype=Article\&state=default\&region=MAIN_CONTENT_1\&context=storylines_live_updates\#wage-data-remain-distorted-by-big-shifts-in-the-labor-market}{Wage
data remain distorted by big shifts in the labor market.}

\href{https://www.nytimes3xbfgragh.onion/live/2020/09/04/business/stock-market-today-coronavirus?action=click\&pgtype=Article\&state=default\&region=MAIN_CONTENT_1\&context=storylines_live_updates\#after-taking-the-heaviest-employment-hit-women-make-a-sharp-rebound}{31h
ago}

\href{https://www.nytimes3xbfgragh.onion/live/2020/09/04/business/stock-market-today-coronavirus?action=click\&pgtype=Article\&state=default\&region=MAIN_CONTENT_1\&context=storylines_live_updates\#after-taking-the-heaviest-employment-hit-women-make-a-sharp-rebound}{After
taking the heaviest employment hit, women make a sharp rebound.}

\href{https://www.nytimes3xbfgragh.onion/live/2020/09/04/business/stock-market-today-coronavirus?action=click\&pgtype=Article\&state=default\&region=MAIN_CONTENT_1\&context=storylines_live_updates}{See
more updates}

More live coverage:
\href{https://www.nytimes3xbfgragh.onion/2020/09/04/world/covid-19-coronavirus.html?action=click\&pgtype=Article\&state=default\&region=MAIN_CONTENT_1\&context=storylines_live_updates}{Global}

\hypertarget{were-getting-bombarded}{%
\subsection{`We're getting bombarded'}\label{were-getting-bombarded}}

\includegraphics{https://static01.graylady3jvrrxbe.onion/images/2020/04/02/business/00virus-masks5/merlin_171201603_b7580d91-22b9-496c-a18c-7bbaedafa2d6-articleLarge.jpg?quality=75\&auto=webp\&disable=upscale}

After the coronavirus outbreak began, China
\href{https://www.nytimes3xbfgragh.onion/2020/04/01/business/coronavirus-china-masks.html}{imported
two billion masks}.
\href{https://www.nytimes3xbfgragh.onion/reuters/2020/03/31/world/europe/31reuters-health-coronavirus-france-masks.html}{France
ordered a billion} and vowed to become self-sufficient by year-end. The
U.S. government has done comparatively little to coordinate purchasing
and ensure that American governments and hospitals aren't competing.

Last month, federal officials agreed to buy roughly 600 million N95
masks over the next 18 months. But many states and hospitals are
desperate for supplies right now, and the government has already nearly
exhausted the supply of protective gear in the national stockpile. On
Thursday, the White House said it had
\href{https://www.nytimes3xbfgragh.onion/2020/04/02/world/coronavirus-live-news-updates.html\#link-711303f6}{invoked}
the Defense Production Act, a 1950s law, to ensure the manufacturing
giant 3M sends a certain share of its masks to the United States.

Some of the entrepreneurs stepping up in the government's stead have
succeeded. Operation Masks, a two-week-old nonprofit run by tech
executives, said it had just closed deals for one million N95s for New
York State and 200,000 for Hawaii, charging just over \$3 for each mask,
not including shipping and other costs. On Thursday, Massachusetts
received 1.2 million N95 masks
\href{https://www.wsj.com/articles/a-million-n95-masks-are-coming-from-chinaon-board-the-new-england-patriots-plane-11585821600}{via
the New England Patriots team plane}.

Still, several hospital executives said that while they appreciated the
surge of well-intentioned people, they were overwhelmed with new names
in their inboxes, all offering products they need for prices far higher
than what they typically pay.

``We're getting bombarded,'' said Ed Bonetti, head of supply chain for
the UMass Memorial hospital network in Worcester, Mass.

The hospital is prepared to pay more for masks, but it does not want to
buy counterfeit gear. ``You're in this uncharted territory where you're
struggling to just at least validate,'' Mr. Bonetti said. ``The last
thing we want to do is put product on a clinician that is not going to
protect them.''

\hypertarget{medical-supply-arbitrage}{%
\subsection{Medical-supply arbitrage}\label{medical-supply-arbitrage}}

Image

Yaear Weintroub, a 22-year-old student from Brooklyn, has turned from
the wholesale electronics business to medical supplies.Credit...Chang W.
Lee/The New York Times

Not every new entrant to the market is a good Samaritan. Groups on
Facebook, WhatsApp and Telegram are teeming with posts hawking thousands
of masks at inflated prices.

Some are wholesalers who bought pallets of masks from China or in
liquidation sales and then marked them up. Many more are simply
middlemen who call themselves brokers. They scour the groups for masks
advertised for a relatively low price, and then repost the offer for a
few thousand dollars more. They don't handle the masks or put up their
own money.

Yaear Weintroub is one of those brokers. A 22-year-old community college
student from Brooklyn, he typically sells wholesale electronics to
Amazon sellers. But the online forums he searches for deals became
flooded with listings for masks last month, so he now spends his days
trying to connect buyers and sellers for a bit of medical-supply
arbitrage.

In a recent interview, he said he was working with a partner to close a
deal for 280,000 surgical masks that would increase their price 20
percent and net the pair a roughly \$40,000 profit. He said many of the
brokers sold to other brokers, each one marking up the price, until the
masks presumably make it to a nursing home or a hospital. He said he
would prefer to sell directly to hospitals.

``They're just more serious,'' he said. ``So if I have the goods, I want
a serious buyer for them. And besides, it's a morally good reason.''

To these sellers, medical supplies are simply another hot product to
flip for a profit. Avraham Eisenberg, a New York wholesaler who is
trying to ship masks from China, compared the rush for masks to the fad
several years ago for fidget spinners.

The Justice Department said last month that it would investigate people
manipulating the medical-supply market. Five days later, federal
authorities
\href{https://www.bloomberg.com/news/articles/2020-03-30/suspected-mask-profiteer-arrested-for-coughing-on-fbi-agents}{charged
a Brooklyn man} with lying about price gouging after he tried to sell
1,000 masks and other supplies to a doctor for \$12,000. (He also was
charged with assault after he claimed he had the coronavirus and coughed
on F.B.I. agents.) Federal officials are now distributing
\href{https://www.hhs.gov/about/news/2020/04/02/doj-and-hhs-partner-to-distribute-more-than-half-a-million-medical-supplies-confiscated-from-price-gougers.html}{the
more than half a million supplies} they confiscated from him.

\hypertarget{global-demand-and-stacks-of-cash}{%
\subsection{Global demand and stacks of
cash}\label{global-demand-and-stacks-of-cash}}

In China, the competition is intense. A small number of Chinese
factories are certified by the Food and Drug Administration to make N95
masks, and ``those are the diamonds right now,'' said Lily Liu, a
Chinese hospital executive turned Silicon Valley entrepreneur who now
helps run Operation Masks.

``What's happening at those factories is France shows up in the morning,
and then they get Germany at breakfast, and then Italy after lunch, and
then the U.S. in the afternoon,'' she said. ``In between they get
distributors showing up at their doorstep with stacks of cash.''

That demand has fueled the spike in prices. While some factory owners
are probably making handsome margins, much of the price increase is
likely spread across the supply chain, from the firms that ship and
inspect the masks to those that make the masks' fabric and the machines
that assemble them.

Take Zhou Hua, the owner of a factory in Xuancheng, China, that months
ago made children's clothing. In February, as the coronavirus swept
across his country, he rushed to buy mask machines and spent roughly
\$500,000 transforming his plant. Now his staff has nearly doubled to 75
employees, and they make 1.6 million masks a day.

He said his margins were modest, and blamed higher material costs for
much of the price increases. Most masks use melt-blown fabric to stop
tiny particles. Mr. Zhou said the price of that material had risen 90
percent to about \$53 a ton. He added that the price for machines that
weld straps to the masks had tripled, to roughly \$2,100.

\hypertarget{from-pool-noodles-to-masks}{%
\subsection{From pool noodles to
masks}\label{from-pool-noodles-to-masks}}

Image

Dan Schonfeld, who usually sells pool noodles, tried to secure 100,000
masks from China, at cost, for hospitals.Credit...Joe Carrotta for The
New York Times

The people jumping into the mask market come from across the spectrum.
Dan Schonfeld, for instance, sells pool noodles. He's pretty good at it,
too. He found a reliable supplier in China, slapped sports teams' logos
on them and built a steady business through PoolPartsToGo.com.

When the coronavirus spread last month to his home state, New York, Mr.
Schonfeld thought he could use his connections in China to get masks to
American doctors. He dropped his pool-supply business and began pursuing
masks, vowing not to earn a cent.

``The fast-forward button was pressed at that moment, and it really
hasn't stopped,'' Mr. Schonfeld, 40, said. ``I don't think I slept for
four nights straight.''

He worked his iPhone around the clock, calling American hospitals by day
and Chinese contacts by night. The hospitals were all interested, but
reliable masks were in short supply.

Then, just before midnight on March 19, his pool-noodle supplier in
Ningbo, China, Jensen Jiang, emailed with news. He had secured a deal
with a nearby factory for 100,000 N95 masks at \$2.70 each. But
competing orders were coming in, he said, so Mr. Schonfeld had to decide
quickly.

``Tomorrow is too late,'' Mr. Jiang wrote. Mr. Schonfeld told him to
place the \$35,000 deposit.

The next day, Mr. Schonfeld excitedly called the hospitals. But
executives who had expressed such desperation for masks were suddenly
wary of turning over \$270,000 to a man who was selling pool parts just
days before. One replied ``We just don't know you,'' Mr. Schonfeld said.
``It turned into me needing help.''

Eventually, his lawyers found a new buyer: a network of nonprofits that
care for 35,000 New Yorkers with intellectual disabilities. They wired
the money, and Mr. Schonfeld booked a cargo flight.

Then he awoke to more bad news. ``I am afraid that I made big trouble to
you,'' Mr. Jiang said in a March 26 email. ``All the masks were taken by
government.'' The email included a photo of a closure notice on the
factory's doors, dated 11 days earlier. Mr. Schonfeld didn't know what
to believe.

Image

Mr. Schonfeld received this photo from his Chinese supplier. He doesn't
know if it's authentic. It reads: ``Dear visitors: Upon receiving a
notice from the government department that supervises our work, we will
suspend face masks delivery. We will inform when we would resume.
Apologize for any inconvenience.''

As Mr. Jiang negotiated a refund, which still hasn't arrived, they
decided to find and ship a different mask: the so-called KN95, China's
effective version of the N95. Mr. Schonfeld ordered 150,000 from a new
factory and booked a freight plane for April 3.

But then there was another catch: The F.D.A.'s guidelines for medical
use of KN95 masks in the United States were murky, and Mr. Schonfeld's
lawyers warned that officials could seize them. (On April 2, the agency
\href{https://www.buzzfeednews.com/article/kenbensinger/the-fda-now-says-it-will-allow-imports-of-kn95-masks-an}{said}
it would not block imports of the masks.)

``Every day I wake up, there's a new hurdle,'' he said. ``I just never
thought it would be this hard to help.''

Whatever happens to the shipment, it will end his fling with medical
supplies, he said.

A day later, he mentioned he was looking into ventilator suppliers. ``I
told my wife, `All right, I'm done,''' he said. ``But if I see on the
news that they're begging for ventilators, and I see that there's just
inaction, I don't see how I can just sit back.''

Cao Li and Sheera Frenkel contributed reporting.

Advertisement

\protect\hyperlink{after-bottom}{Continue reading the main story}

\hypertarget{site-index}{%
\subsection{Site Index}\label{site-index}}

\hypertarget{site-information-navigation}{%
\subsection{Site Information
Navigation}\label{site-information-navigation}}

\begin{itemize}
\tightlist
\item
  \href{https://help.nytimes3xbfgragh.onion/hc/en-us/articles/115014792127-Copyright-notice}{©~2020~The
  New York Times Company}
\end{itemize}

\begin{itemize}
\tightlist
\item
  \href{https://www.nytco.com/}{NYTCo}
\item
  \href{https://help.nytimes3xbfgragh.onion/hc/en-us/articles/115015385887-Contact-Us}{Contact
  Us}
\item
  \href{https://www.nytco.com/careers/}{Work with us}
\item
  \href{https://nytmediakit.com/}{Advertise}
\item
  \href{http://www.tbrandstudio.com/}{T Brand Studio}
\item
  \href{https://www.nytimes3xbfgragh.onion/privacy/cookie-policy\#how-do-i-manage-trackers}{Your
  Ad Choices}
\item
  \href{https://www.nytimes3xbfgragh.onion/privacy}{Privacy}
\item
  \href{https://help.nytimes3xbfgragh.onion/hc/en-us/articles/115014893428-Terms-of-service}{Terms
  of Service}
\item
  \href{https://help.nytimes3xbfgragh.onion/hc/en-us/articles/115014893968-Terms-of-sale}{Terms
  of Sale}
\item
  \href{https://spiderbites.nytimes3xbfgragh.onion}{Site Map}
\item
  \href{https://help.nytimes3xbfgragh.onion/hc/en-us}{Help}
\item
  \href{https://www.nytimes3xbfgragh.onion/subscription?campaignId=37WXW}{Subscriptions}
\end{itemize}
