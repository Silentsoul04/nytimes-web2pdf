Sections

SEARCH

\protect\hyperlink{site-content}{Skip to
content}\protect\hyperlink{site-index}{Skip to site index}

\href{https://www.nytimes3xbfgragh.onion/section/business/economy}{Economy}

\href{https://myaccount.nytimes3xbfgragh.onion/auth/login?response_type=cookie\&client_id=vi}{}

\href{https://www.nytimes3xbfgragh.onion/section/todayspaper}{Today's
Paper}

\href{/section/business/economy}{Economy}\textbar{}U.S. Is Nowhere Close
to Reopening the Economy, Experts Say

\url{https://nyti.ms/2XgRamL}

\begin{itemize}
\item
\item
\item
\item
\item
\item
\end{itemize}

\href{https://www.nytimes3xbfgragh.onion/news-event/coronavirus?action=click\&pgtype=Article\&state=default\&region=TOP_BANNER\&context=storylines_menu}{The
Coronavirus Outbreak}

\begin{itemize}
\tightlist
\item
  live\href{https://www.nytimes3xbfgragh.onion/2020/08/04/world/coronavirus-covid-19.html?action=click\&pgtype=Article\&state=default\&region=TOP_BANNER\&context=storylines_menu}{Latest
  Updates}
\item
  \href{https://www.nytimes3xbfgragh.onion/interactive/2020/us/coronavirus-us-cases.html?action=click\&pgtype=Article\&state=default\&region=TOP_BANNER\&context=storylines_menu}{Maps
  and Cases}
\item
  \href{https://www.nytimes3xbfgragh.onion/interactive/2020/science/coronavirus-vaccine-tracker.html?action=click\&pgtype=Article\&state=default\&region=TOP_BANNER\&context=storylines_menu}{Vaccine
  Tracker}
\item
  \href{https://www.nytimes3xbfgragh.onion/2020/08/02/us/covid-college-reopening.html?action=click\&pgtype=Article\&state=default\&region=TOP_BANNER\&context=storylines_menu}{College
  Reopening}
\item
  \href{https://www.nytimes3xbfgragh.onion/live/2020/08/03/business/stock-market-today-coronavirus?action=click\&pgtype=Article\&state=default\&region=TOP_BANNER\&context=storylines_menu}{Economy}
\end{itemize}

Advertisement

\protect\hyperlink{after-top}{Continue reading the main story}

Supported by

\protect\hyperlink{after-sponsor}{Continue reading the main story}

\hypertarget{us-is-nowhere-close-to-reopening-the-economy-experts-say}{%
\section{U.S. Is Nowhere Close to Reopening the Economy, Experts
Say}\label{us-is-nowhere-close-to-reopening-the-economy-experts-say}}

Here's what economists say the United States needs to start returning to
normal amid the coronavirus outbreak --- and how the economy can survive
in the meantime.

\includegraphics{https://static01.graylady3jvrrxbe.onion/images/2020/04/04/business/04DC-VIRUS-ECON-01/merlin_170656812_5ad9987a-b68f-43db-9120-37f671c4dad9-articleLarge.jpg?quality=75\&auto=webp\&disable=upscale}

\href{https://www.nytimes3xbfgragh.onion/by/jim-tankersley}{\includegraphics{https://static01.graylady3jvrrxbe.onion/images/2018/10/19/multimedia/author-jim-tankersley/author-jim-tankersley-thumbLarge.png}}

By \href{https://www.nytimes3xbfgragh.onion/by/jim-tankersley}{Jim
Tankersley}

\begin{itemize}
\item
  Published April 6, 2020Updated April 13, 2020
\item
  \begin{itemize}
  \item
  \item
  \item
  \item
  \item
  \item
  \end{itemize}
\end{itemize}

WASHINGTON --- How long can we keep this up?

It is still very early in the U.S. effort to snuff
\href{https://www.nytimes3xbfgragh.onion/news-event/coronavirus}{a
lethal pandemic} by shutting down much of the
\href{https://www.nytimes3xbfgragh.onion/2020/04/13/business/coronavirus-economy.html}{economy}.
But there is a growing question --- from workers, the White House,
corporate boardrooms and small businesses on the brink --- that hangs
over what is essentially a war effort against a virus that has already
\href{https://coronavirus.jhu.edu/map.html}{killed more than 9,000
Americans}.

There is no good answer yet, in part because we don't even have the data
needed to formulate one.

Essentially, economists say, there won't be a fully functioning economy
again until people are confident that they can go about their business
without a high risk of catching the virus.

``Our ability to reopen the economy ultimately depends on our ability to
better understand the spread and risk of the virus,'' said Betsey
Stevenson, a University of Michigan economist who worked on the White
House Council of Economic Advisers under President Barack Obama. ``It's
also quite likely that we will need to figure out how to reopen the
economy with the virus remaining a threat.''

Public health experts are beginning to make predictions about when
\href{https://www.nytimes3xbfgragh.onion/2020/04/13/business/coronavirus-economy.html}{coronavirus}
infection rates will peak. Economists are calculating when the cost of
continuing to shutter restaurants, shopping malls and other businesses
--- a move that has already pushed some 10 million Americans into
unemployment,
\href{https://www.nytimes3xbfgragh.onion/2020/04/03/upshot/coronavirus-jobless-rate-great-depression.html}{with
millions more on the way} --- will outweigh the savings from further
efforts to slow the virus once the infection curve has flattened out.

Government officials are setting competing targets. President Trump has
pushed his expected date of reopening the economy to the end of April.
``We have to get back to work,'' he said in a briefing on Saturday. ``We
have to open our country again. We don't want to be doing this for
months and months and months. We're going to open our country again.
This country wasn't meant for this.''

Some governors have set much more conservative targets, like Ralph
Northam of Virginia, who canceled the remainder of the school year and
imposed a shelter-at-home order through June 10. Other states, like
Florida, only recently agreed to shut activity down but have set more
aggressive targets --- April 30, in the case of the Sunshine State ---
to restart it.

Those targets are at best mildly informed guesses based on models that
contain variables --- including how many people have the virus and how
effective suppression measures will prove to be. The models cannot yet
give us anything close to a precise answer on the big question looming
over Americans' lives and livelihoods.

To determine when to restart activity, said R. Glenn Hubbard, a former
top economist under President George W. Bush, ``we need more
information.''

Interviews with more than a dozen economists, many of whom are veterans
of past presidential administrations, reveal broad consensus on the
building blocks the economy needs --- but does not yet have --- to begin
the slow process of restoring normalcy in the American economy.

\hypertarget{latest-updates-economy}{%
\section{\texorpdfstring{\href{https://www.nytimes3xbfgragh.onion/live/2020/08/03/business/stock-market-today-coronavirus?action=click\&pgtype=Article\&state=default\&region=MAIN_CONTENT_1\&context=storylines_live_updates}{Latest
Updates:
Economy}}{Latest Updates: Economy}}\label{latest-updates-economy}}

\href{https://www.nytimes3xbfgragh.onion/live/2020/08/03/business/stock-market-today-coronavirus?action=click\&pgtype=Article\&state=default\&region=MAIN_CONTENT_1\&context=storylines_live_updates\#the-chicago-fed-president-says-its-up-to-congress-to-save-the-economy}{13h
ago}

\href{https://www.nytimes3xbfgragh.onion/live/2020/08/03/business/stock-market-today-coronavirus?action=click\&pgtype=Article\&state=default\&region=MAIN_CONTENT_1\&context=storylines_live_updates\#the-chicago-fed-president-says-its-up-to-congress-to-save-the-economy}{The
Chicago Fed president says it's up to Congress to save the economy.}

\href{https://www.nytimes3xbfgragh.onion/live/2020/08/03/business/stock-market-today-coronavirus?action=click\&pgtype=Article\&state=default\&region=MAIN_CONTENT_1\&context=storylines_live_updates\#faa-says-boeing-has-effectively-mitigated-defects-in-the-737-max}{14h
ago}

\href{https://www.nytimes3xbfgragh.onion/live/2020/08/03/business/stock-market-today-coronavirus?action=click\&pgtype=Article\&state=default\&region=MAIN_CONTENT_1\&context=storylines_live_updates\#faa-says-boeing-has-effectively-mitigated-defects-in-the-737-max}{F.A.A.
says Boeing has `effectively mitigated' defects in the 737 Max.}

\href{https://www.nytimes3xbfgragh.onion/live/2020/08/03/business/stock-market-today-coronavirus?action=click\&pgtype=Article\&state=default\&region=MAIN_CONTENT_1\&context=storylines_live_updates\#small-businesses-got-emergency-loans-but-not-what-they-expected}{16h
ago}

\href{https://www.nytimes3xbfgragh.onion/live/2020/08/03/business/stock-market-today-coronavirus?action=click\&pgtype=Article\&state=default\&region=MAIN_CONTENT_1\&context=storylines_live_updates\#small-businesses-got-emergency-loans-but-not-what-they-expected}{Small
businesses got emergency loans, but not what they expected.}

\href{https://www.nytimes3xbfgragh.onion/live/2020/08/03/business/stock-market-today-coronavirus?action=click\&pgtype=Article\&state=default\&region=MAIN_CONTENT_1\&context=storylines_live_updates}{See
more updates}

More live coverage:
\href{https://www.nytimes3xbfgragh.onion/2020/08/04/world/coronavirus-covid-19.html?action=click\&pgtype=Article\&state=default\&region=MAIN_CONTENT_1\&context=storylines_live_updates}{Global}

That includes widespread agreement that the United States desperately
needs more testing for the virus in order to give policymakers the first
key piece of evidence they need to determine how fast the virus is
spreading and when it might be safe for people to return to work.

\includegraphics{https://static01.graylady3jvrrxbe.onion/images/2020/04/04/business/04DC-VIRUS-ECON-02/merlin_171232767_0ed3cc89-70cb-4671-a0b1-c9833d5f8a8c-articleLarge.jpg?quality=75\&auto=webp\&disable=upscale}

Without more testing, ``there's no way that you could set a time limit
on when you could open up the economy,'' said Simon Mongey, a University
of Chicago economist who is among the authors of
\href{https://bfi.uchicago.edu/working-paper/working-paper-2020-25/}{a
new study} that found that rapid deployment of randomized testing for
the virus could reduce its health and economic damage.

``It's going to have to depend on being able to identify people that
have the coronavirus, understanding how readily those people can
transmit the disease to others and then kind of appropriately isolating
people that are contagious,'' Mr. Mongey said.

Policymakers will also need better data on how strained hospitals and
entire regional health care systems are likely to be if the infection
rate flares up and spreads. Ideally, they would sufficiently control the
rate to establish so-called
\href{https://www.nytimes3xbfgragh.onion/2020/03/30/us/politics/trump-governors-coronavirus-testing.html}{contact
tracing} in order to track --- and avoid --- the spread of the virus
across the country.

Once such levels of detection are established, it is possible that
certain workers could begin returning to the job --- for example, in
areas where the chance of infection is low. Some experts have talked
about quickly bringing back workers who contract the virus but recover
with little effect. Testing is the best way to identify such workers,
who may have had the virus with few or no symptoms and possibly not
realized they were ever infected.

While they wait for the infection rate to fall, policymakers will need
to provide more support to workers who have lost jobs or hours and to
businesses teetering on the brink of failure. That could mean trillions
more in small business loans, unemployment benefits and direct payments
to individuals, and it could force the government to get creative in
deploying money to avoid bottlenecks.

Lisa D. Cook, a Michigan State University economist who worked in the
Obama White House, said lawmakers should consider funneling \$1,500 a
month to individuals through mobile apps like Zelle in order to reach
more people, particularly low-income and nonwhite Americans who
disproportionately lack traditional bank accounts. Mobile payments, Ms.
Cook said, would also make it ``easier and faster to make onward
payments to family members and friends in need.''

The government's efforts could prove crucial to maintaining public
support for what amounts to a prolonged economic drought. Adam Ozimek,
the chief economist at Upwork, said additional money for small business
will be crucial throughout the full extent of the crisis --- both to
prevent a crush of business failures and to keep owners and customers
from flouting the national effort to reduce infections.

``I don't think you can force hundreds of thousands of small business
owners to voluntarily shut down and let failure happen to them,'' Mr.
Ozimek said. ``They won't do it, the public won't support it, and
frankly I don't think local authorities would stop them.''

Policymakers will also need to give better support and protection to
Americans who are putting their own health at risk to keep the essential
parts of the economy running, like doctors, nurses, grocery store clerks
and package delivery drivers.

Heather Boushey, the president of the Washington Center for Equitable
Growth, a think tank focused on inequality, said those workers needed to
have paid sick leave, adequate health coverage, access to coronavirus
tests and affordable care for their children while they worked in order
to stay healthy and to protect consumers from further spread of the
virus.

``That is the economy at this point, those workers,'' Ms. Boushey said.
``And their health and safety is imperative to my safety.''

Policymakers will need patience: Restarting activity too quickly could
risk a second spike in infections that could deal more damage than the
first because it would shake people's faith in their ability to engage
in even limited amounts of shopping, dining or other commerce.

``It's important not to lift too early,'' said Emil Verner, a
Massachusetts Institute of Technology economist who is a co-author of
\href{https://papers.ssrn.com/sol3/papers.cfm?abstract_id=3561560}{a new
study} that found that cities that took more aggressive steps to curb
the 1918 flu pandemic in the United States emerged with stronger
economies than cities that did less. ``Because if we lift too early, the
pandemic can take hold again. And that itself is very bad for the
economy.''

Finally, policymakers will need to level with Americans --- and
themselves --- and concede the possibility that the shutdown and its
effects could drag well beyond the end of the month.

Aggressive suppression measures could lead to a gradual resumption of
activity that begins in some places as soon as May, several experts
said. But business as usual might not come back until a vaccine is
developed, which could take more than a year.

``We should certainly be prepared for a meaningful level of deliberate
suppression of economic activity for the rest of the year,'' said Jason
Furman of Harvard University, who was a top economist under Mr. Obama.

The Congressional Budget Office
\href{https://www.cbo.gov/publication/56314}{wrote on Thursday} that it
expected at least a quarter of the current suppression measures to last
through year's end, and that the unemployment rate could still be 9
percent at the end of 2021. Lawmakers need to be ready to keep filling
the void, with support to businesses and workers, said Karl Smith, the
vice president for federal policy at the Tax Foundation in Washington.

``The possibility of an unofficial quarantine for weeks or months after
the official one is lifted is real,'' Mr. Smith said. ``After that, my
guess is that the economy is in major trouble.''

\includegraphics{https://static01.graylady3jvrrxbe.onion/images/2017/01/29/podcasts/the-daily-album-art/the-daily-album-art-articleInline-v2.jpg?quality=75\&auto=webp\&disable=upscale}

\hypertarget{listen-to-the-daily-a-historic-unemployment-crisis}{%
\subsubsection{Listen to `The Daily': A Historic Unemployment
Crisis}\label{listen-to-the-daily-a-historic-unemployment-crisis}}

The U.S. is probably experiencing the highest rates of joblessness since
the Great Depression. But with layoffs mounting daily, the real picture
may be even worse than we think.

transcript

Back to The Daily

bars

0:00/26:24

-26:24

transcript

\hypertarget{listen-to-the-daily-a-historic-unemployment-crisis-1}{%
\subsection{Listen to `The Daily': A Historic Unemployment
Crisis}\label{listen-to-the-daily-a-historic-unemployment-crisis-1}}

\hypertarget{hosted-by-michael-barbaro-produced-by-clare-toeniskoetter-sydney-harper-eric-krupke-and-rachel-quester-with-help-from-stella-tan-and-edited-by-lisa-tobin}{%
\subsubsection{Hosted by Michael Barbaro; produced by Clare
Toeniskoetter, Sydney Harper, Eric Krupke and Rachel Quester; with help
from Stella Tan; and edited by Lisa
Tobin}\label{hosted-by-michael-barbaro-produced-by-clare-toeniskoetter-sydney-harper-eric-krupke-and-rachel-quester-with-help-from-stella-tan-and-edited-by-lisa-tobin}}

\hypertarget{the-us-is-probably-experiencing-the-highest-rates-of-joblessness-since-the-great-depression-but-with-layoffs-mounting-daily-the-real-picture-may-be-even-worse-than-we-think}{%
\paragraph{The U.S. is probably experiencing the highest rates of
joblessness since the Great Depression. But with layoffs mounting daily,
the real picture may be even worse than we
think.}\label{the-us-is-probably-experiencing-the-highest-rates-of-joblessness-since-the-great-depression-but-with-layoffs-mounting-daily-the-real-picture-may-be-even-worse-than-we-think}}

\begin{itemize}
\item
  michael barbaro\\
  From The New York Times, I'm Michael Barbaro. This is ``The Daily.''
\item
  {[}music{]}\\
  Today: To contain the pandemic, the U.S. government has deliberately
  shut down much of the economy. My colleague Jim Tankersley on the
  result --- the worst unemployment crisis in U.S. history.

  It's Monday, April 6.
\item
  archived recording (dan rather)\\
  Good evening. This is the CBS Evening News. Dan Rather reporting. The
  newest government economic figures out today show that the recession
  has not found its bottom.
\end{itemize}

jim tankersley

The last time that the United States set a record for new unemployment
claims was the fall of 1982.

\begin{itemize}
\tightlist
\item
  archived recording (dan rather)\\
  President Reagan persists in remaining optimistic.
\end{itemize}

jim tankersley

Ronald Reagan is president. The country is engaged in a battle against
inflation --- the rapid increase in prices that has just beset the
economy for years. And the Federal Reserve, as part of that battle, has
made the cost of borrowing money really high. And as a result ---

\begin{itemize}
\tightlist
\item
  archived recording (ronald reagan)\\
  My fellow Americans, in recent days, all of us have been swamped by a
  sea of economic statistics --- some good, some bad and some just plain
  confusing. Tonight, in homes across this country, unemployment is the
  problem uppermost on many people's minds.
\end{itemize}

jim tankersley

Almost 700,000 people lose their jobs in the span of one week and file
for unemployment.

michael barbaro

Wow.

jim tankersley

That is a record.

\begin{itemize}
\tightlist
\item
  archived recording (ronald reagan)\\
  We can do it by slowly but surely working our way back to prosperity.
  That will mean jobs for all who are willing to work and fulfillment
  for all who still cherish the American dream.
\end{itemize}

jim tankersley

And it stays a record for a long time. Stays through the next recession,
even through the 2008 financial crisis. That remains the record --- 1982
--- until this March 2020.

\begin{itemize}
\item
  archived recording 1\\
  The weekly unemployment numbers are in, and they are the biggest ever
  by far.
\item
  archived recording 2\\
  3,283,000 Americans filed for unemployment last week.
\item
  archived recording 3\\
  A health pandemic has become a financial pandemic.
\end{itemize}

jim tankersley

The third week of March of 2020, 3.3 million Americans file for
unemployment in one week.

\begin{itemize}
\tightlist
\item
  archived recording\\
  And that is the equivalent of the entire city of Chicago.
\end{itemize}

jim tankersley

That's a new record, and it lasts for exactly one week.

\begin{itemize}
\tightlist
\item
  archived recording\\
  The Labor Department is out with the jaw-dropping new numbers.
\end{itemize}

jim tankersley

In the following week, the fourth week of March, 6.6 million Americans
file for unemployment. Which means in the span of two weeks, nearly 10
million workers in America file for unemployment benefits.

\begin{itemize}
\tightlist
\item
  archived recording\\
  In fact, the past two weeks all but eliminated the jobs created in the
  past five years.
\end{itemize}

jim tankersley

What we hit in a single week in March this year was almost 10 times as
bad as the worst week of 1982. 10 times.

michael barbaro

Jim, take us through the timeline of how we got to this staggering
figure --- 10 million people losing their jobs and needing unemployment
benefits.

jim tankersley

So back in late January, as the virus is spreading rapidly through China
and starting to really play havoc with supply chains there, there start
to be warnings that this could really hurt the United States economy.
But policymakers, particularly inside the White House, are not raising
huge alarm bells.

\begin{itemize}
\item
  archived recording 1\\
  So you have one person on the line. So {[}INAUDIBLE{]}
\item
  archived recording 2\\
  OK. Welcome, everyone, and for those of you on the phone as well.
\end{itemize}

jim tankersley

The White House Council of Economic Advisers summoned a bunch of
reporters for a briefing on February 18.

\begin{itemize}
\tightlist
\item
  archived recording\\
  What's the biggest threat that you see out there?
\end{itemize}

jim tankersley

And the acting chairman of the council, Tomas Philipson, who is a health
economist by the way, told the reporters there, quote ---

\begin{itemize}
\tightlist
\item
  archived recording (tomas philipson)\\
  Let's talk about what I don't think. I don't think corona is a big a
  threat as people make it out to be yet.
\end{itemize}

jim tankersley

``I don't think corona is a big of threat as people make it out to be.''

\begin{itemize}
\tightlist
\item
  archived recording (tomas philipson)\\
  So if you look at flu seasons in the U.S., it's about 40,000 deaths
  every year on average. In 2018, it was 80,000 deaths.
\end{itemize}

jim tankersley

In that same interview at the White House, we asked, well, if you're not
worried about the risk of the virus, what risks are you worried about?

\begin{itemize}
\item
  archived recording\\
  What's the biggest concern that you have?
\item
  archived recording (tomas philipson)\\
  I don't think --- clearly, we don't have a lot of concerns.
\end{itemize}

jim tankersley

And the White House economist basically said, nothing, we don't really
think there's anything that could go wrong.

\begin{itemize}
\tightlist
\item
  archived recording (tomas philipson)\\
  OK. Thank you very much.
\end{itemize}

jim tankersley

And White House advisers would keep that lineup for, I don't know,
almost a month. I was having calls with White House officials who were
saying, it's not going to be that bad. We're not going to need
government intervention. The big risk right now is the stock market is
freaking out, but we think we can get it under control.

\begin{itemize}
\tightlist
\item
  archived recording\\
  Tonight, elected officials taking unprecedented measures to keep
  people safe.
\end{itemize}

michael barbaro

And what happened once it became very clear that this couldn't be kept
under control?

jim tankersley

What happened is basically unprecedented in American history. Industries
started shutting down one by one, sector by sector, county by county,
and that led to all these layoffs.

\begin{itemize}
\tightlist
\item
  archived recording\\
  Life is about to look very different. So let's start with New York
  City, where nightlife is over.
\end{itemize}

jim tankersley

The first workers to lose their jobs are exactly the ones you could see
being hit by the restrictions that governments are putting in place.

\begin{itemize}
\tightlist
\item
  archived recording\\
  Here's the full list of businesses being forced to close at 3 o'clock
  this afternoon. Dine-in restaurants, cafes, coffeehouses, bars,
  taverns, brew pubs, distilleries, clubs.
\end{itemize}

jim tankersley

Restaurants close, and bartenders and waitstaff lose their jobs.
Governors order certain shops to close --- most shops to close --- and
so retail workers lose their jobs.

\begin{itemize}
\tightlist
\item
  archived recording\\
  Movie theaters, indoor and outdoor performance venues, gymnasiums,
  fitness centers, recreation centers, indoor sports facilities, indoor
  exercise facilities, exercise studios, spas and casinos.
\end{itemize}

jim tankersley

Also, it's things like tourism. No one's flying on airplanes anymore. So
suddenly you see airline employees start to lose their jobs. If you work
for a hotel in a destination, a resort, you're starting to lose your
job. If you are someone who relies upon one of those workers who's lost
their jobs --- so say you are a house cleaner in Florida who works for
someone who's an executive at a resort, and that executive loses their
job. Suddenly they don't have money to pay a house cleaner anymore, and
you lose your job. It's these ripple effects. And then they start
getting into parts of the economy we might not necessarily expect: Day
care workers, preschool teachers, law firms. Hospitals are serving
coronavirus patients, but they're not doing a lot of elective surgeries.
So if you work in an elective surgery division, you're at risk or, in
some cases, actually losing your job. All of these sectors are starting
to lay people off. Even the places you might least think are vulnerable
suddenly are.

michael barbaro

So this becomes a kind of wave, where people who are dependent on people
who are dependent on people all end up getting squeezed and eventually
are out of a job.

jim tankersley

Yeah, absolutely. And the wave is just going to keep going. We're going
to see more and more people who can go to work, or at least work from
home, who won't be able to because the revenue source that their
business depends on is drying up. If you're a local newspaper and you
depend on advertising from restaurants that can't be open, they're not
going to advertise. And so you are going to be stuck without any
advertising, and you're probably going to start laying off people, which
we've already started to see.

michael barbaro

You know, Jim, one of the most defining qualities of this whole moment
is how fast everything has moved and changed, how quickly our whole
reality has shifted. And the same is true here. I mean, has there ever
been an example of people's job security collapsing so quickly, not just
the scale of it but the speed of it? I mean, one day, people own
successful businesses. The next day, those businesses are closed. One
day, I have a seemingly secure job, and the next day, I have been laid
off.

jim tankersley

Yeah, we've never seen anything like this in modern America. Usually
recessions spread kind of slowly. They take some time to work their way
through various regions, or various parts of the financial system, or
parts of the economy, to the point where economists often debate whether
or not we're in a recession well into the actual existence of the
recession. But now, I mean, we're a couple of weeks into this one, and
no one disputes we're in a recession. It is so fast. You go to bed one
week thinking the economy works one way, and you wake up two weeks
later, and not only does your industry not have a job for you, your
industry doesn't really work anymore for the time being.

michael barbaro

Wow.

jim tankersley

And then on top of how fast and severe that all is, you are left with
this almost unimaginable before this thought, which is, we're doing this
on purpose. We are doing this because public health officials think the
only way to survive a pandemic virus without losing hundreds of
thousands or even millions of people to it, is to turn the economy off
and then eventually try to turn it all back on again.

michael barbaro

Right. This was the lesser of the awful possibilities that the U.S. had.

jim tankersley

Yes.

{[}music{]}

michael barbaro

We'll be right back.

Jim, now that we have explained that we have 10 million Americans
applying for these unemployment benefits, can you walk us through what
the typical process of applying for that looks like, and what someone
would have gotten before this pandemic?

jim tankersley

So what happens is if you lose your job, you go to your state
unemployment office and file a claim. You can often do that online.
Sometimes you have to do it on a paper form, but you, depending on your
state, get a benefit back that lasts about three months. It's meant to
be a baseline, just something to tide you through. For example, if
you're in California, it ranges between, like, \$40 at a minimum per
week to \$450 at maximum. If you're in Pennsylvania, it's about half
your weekly wage that you were earning before, up to a maximum. And the
way that the system is set up, you don't get as much as you would have
made on your job, but you get something. It's supposed to be there as
sort of an emergency safety net to keep you from getting evicted or
going hungry before you find a new job.

michael barbaro

Got it. And is that still the case during this crisis?

jim tankersley

It's actually better during this crisis, thanks to the law that the
president signed at the end of March --- the CARES Act.

\begin{itemize}
\tightlist
\item
  archived recording (nancy pelosi)\\
  Last night, as you know, the Senate passed important legislation.
  We're very proud of the product.
\end{itemize}

jim tankersley

What that includes --- this was a big point of contention in the law,
but that Democrats fought and got in.

\begin{itemize}
\tightlist
\item
  archived recording (nancy pelosi)\\
  We did jiu-jitsu on it, that it went from a corporate-first proposal
  that the Republicans put forth in the Senate, to a workers-first,
  democratic workers-first legislation.
\end{itemize}

jim tankersley

It includes enhanced unemployment benefits. That means that whatever you
were going to get from the state, you get still, plus \$600 a week.

michael barbaro

Wow.

jim tankersley

Yeah. It's a lot. It's a calculation meant to fill the gap between what
you were earning before you were laid off and what you would've earned
from unemployment.

\begin{itemize}
\tightlist
\item
  archived recording (chuck schumer)\\
  Every American worker who is laid off will have their salary
  remunerated by the federal government so they can pay their bills.
\end{itemize}

jim tankersley

It turned out to be too difficult for states to just calculate, OK, what
was your wage, we're just going to give that to you. So instead, the
lawmakers just figured out the average gap for the average worker
between what they would have gotten at their job and what they would
have gotten in unemployment. That turns out to be about \$600. So if you
are, for example, a bartender and you earn about \$15 an hour on
average, which is a little more than \$30,000 a year, you're going to
end up getting more money out of your new enhanced unemployment check
than you would have just regularly at your job. If you are, say, someone
who works at a small manufacturer who is laid off because there's no
demand now for your products, and you were earning a median salary in
the country --- a little more than \$60,000 a year --- then you are
going to get about the same back in your enhanced unemployment check.
And if you're like a manager of an office or a store that's had to close
and lays everyone off and you earned, you know, \$100,000 a year, you're
not going to be getting really anything close to your full salary back,
but you still are going to be getting back more than you would have
under the unemployment system. The idea is, this does help you get by.
It's just not going to fully fund the lifestyle you had before.

michael barbaro

And how long do these enhanced benefits last?

jim tankersley

Right now Congress has them set up for four months. I think there's
going to be a lot of pressure to continue that if the crisis continues
and this chill in the economy persists past four months.

michael barbaro

Jim, in your reporting, what has been the rationale for providing a
level of unemployment benefit that is significantly greater than what
states have traditionally offered in their unemployment benefits?

jim tankersley

The rationale is that this isn't a case where we are worried about
dissuading people from looking for work. Usually, you don't want to give
people, in the eyes of many economists, what's called a disincentive to
work --- too much money that will just keep them at home on the couch
and not looking for a new job. But in this case, we don't actually want
these people working. We want them to stay home. We want them to
socially distance. We want them to be able to keep paying their rent and
buying groceries and doing the things people need to do. But we have to
make sure somehow that they get money to do that.

michael barbaro

So we've been talking about employees. How has the stimulus bill
impacted or attempted to help employers?

jim tankersley

Well, the government is trying to keep as many of those businesses still
going and not bankrupt as possible.

\begin{itemize}
\tightlist
\item
  archived recording (chuck schumer)\\
  It will keep companies intact.
\end{itemize}

jim tankersley

It's bailing out big corporations, and it's offering loans to small
businesses that can turn into grants if they use the money to keep
employees on their payroll.

\begin{itemize}
\tightlist
\item
  archived recording (chuck schumer)\\
  But with our plan, since they stay on the payroll of their employer,
  as soon as the crisis is over, they all can come back together.
\end{itemize}

jim tankersley

And this is actually an important point. It's probably almost certainly
better for the government to subsidize employees staying on the payroll
of their employer than to pay them unemployment benefits.

michael barbaro

Why?

jim tankersley

Well, we worry about what happens once we're through this crisis and we
got to start the economy back up. What we don't want to see are a bunch
of people left out of work and no easy way to get them back into jobs.
One easy way to make sure they can get back to work is to keep them
tethered to their employer in the first place. So you don't lose your
job. You're basically just getting paid by the government not to work
for your employer who you still work for. It's a little confusing, but
it's really important.

michael barbaro

So it's a little bit like unemployment benefits. But instead of giving
money to someone who loses their job, it kind of goes through the front
door in order to keep that person from losing their job.

jim tankersley

Yeah, it helps the employee not have to look for a new job. It helps the
employer not to have to hire for that job again. And ---

\begin{itemize}
\tightlist
\item
  archived recording (chuck schumer)\\
  And because so many of them will be furloughed rather than fired, if
  they have benefits they continue.
\end{itemize}

jim tankersley

--- if you're on the payroll of someone who offers health insurance, you
keep that too. Which is something that, if you get laid off, you don't
get to keep if you had employer health insurance. So there are a lot of
benefits to it.

\begin{itemize}
\tightlist
\item
  archived recording (chuck schumer)\\
  And that means that company or small business can reassemble once this
  awful plague is over, and our economy can get going quickly.
\end{itemize}

michael barbaro

Jim, when it comes to these unemployment benefits and this stimulus
package that's meant to enhance that benefit and keep people from
needing it in the first place, how much money is there and how many
unemployed people can this system absorb? I mean, when you see something
like 6.6 million people applying for unemployment benefits in a single
week, that makes me and I'm sure many other people wonder how many
people this system can handle?

jim tankersley

It may be a little crazy-sounding, but the money is probably not the
issue for the next couple of months. The issue is how many people can
actually get benefits when they need them. The system is just not built
to handle this amount of demand for new benefits. It's like funneling
all of the traffic of New York onto a country road and trying to figure
out how to get every car through. So some states, we've already seen,
are having to tell people to sign up for benefits on different days of
the week based on the first letter of their last name. What we're going
to see, I think, going forward is a lot more of that. We know it's a
reason why some people who maybe would have signed up for benefits in
the third week of March actually ended up signing up in the fourth week.
And it's a reason the first week in April is probably going to have even
higher numbers of new unemployed than the last week in March. So new
records, for a while maybe. And all of that adds up to tens of millions
of people out of work, which will absolutely strain budgets --- state
budgets, federal budgets. But the issue is going to be how long do
lawmakers have the time and the patience and the money, in particular
from the federal government, to keep funding that while we wait for the
suppression measures here to work and the virus to go away. That is the
great risk here, is that we lose our nerve and lose the strategy and
it's all for naught.

michael barbaro

Jim, right now, everybody of course wants to know when this pandemic
will be over and how we will know it's over. But I'm curious how we will
know when this economic crisis is over.

jim tankersley

I think we're going to know the crisis is over when a critical mass of
Americans feel safe going back to work, going out to eat, going back
shopping. Every economist I talk to says this is a public health
decision first and the economics will follow. Not just because we value
human life so highly, which we do, but because if you don't have the
confidence that if you leave the house you're not very likely to get a
deadly illness, you're not going to keep going through the basic
commerce that you did before. So once the infection rate drops or, I
mean, quite possibly not until we actually have a workable vaccine, then
we will start to see the real ramp up back to the economic activity we
had before this pandemic hit. But the experts that I talked to say is
that we were wrong to think about this as a trade off between human
health and the economy.

michael barbaro

Hmm.

{[}music{]}

jim tankersley

That they go hand in hand. That the more we are able to suppress the
spread of the virus by suppressing the economy, the stronger the economy
will emerge in the end. If we don't stick with this strategy, if we
break out of it too early, if desperate people, the most vulnerable
workers who need to eat, are forced to go back out and try to do their
jobs to earn that living and they start infecting people again and
infections and death rates spike some more, that's actually going to
hurt the economy more than anything. Because it will undermine all of
the confidence that people have left in the system for protecting them.
So it's really this very delicate, probably very slow process of
building that confidence, figuring out exactly who has the virus and
where it's spreading and how it's moving through the population. And
then we can start really, brick by brick, rebuilding this economy that
was just shattered to pieces in the last few weeks.

{[}music{]}

michael barbaro

On that very sobering note, Jim, thank you very much.

jim tankersley

Thank you.

michael barbaro

We'll be right back.

Here's what else you need to know today.

\begin{itemize}
\tightlist
\item
  archived recording (dr. jerome adams)\\
  The next week is going to be our Pearl Harbor moment. It's going to be
  our 9/11 moment. It's going to be the hardest moment for many
  Americans in their entire life.
\end{itemize}

michael barbaro

In interviews on Sunday, top U.S. health officials, including the
surgeon general, Jerome Adams, on NBC prepared Americans for staggering
numbers of infections and deaths in the coming days, comparing it to the
country's darkest days.

\begin{itemize}
\tightlist
\item
  archived recording (dr. anthony fauci)\\
  Well, this is going to be a bad week, unfortunately, if you look at
  the projection of the curves or the kinetics of the curves.
\end{itemize}

michael barbaro

Speaking to CBS, Dr. Anthony Fauci said that at the moment, the crisis
was beyond the government's control.

\begin{itemize}
\tightlist
\item
  archived recording (dr. anthony fauci)\\
  I will not say we have it under control. That would be a false
  statement. We are struggling to get it under control. And that's the
  issue that's at hand right now.
\end{itemize}

michael barbaro

In Britain, Prime Minister Boris Johnson, who has tested positive for
Covid-19, was admitted to the hospital after suffering symptoms for 10
days. And Queen Elizabeth made a rare televised address to the British
people from Windsor Castle.

\begin{itemize}
\tightlist
\item
  archived recording (queen elizabeth ii)\\
  While we have faced challenges before, this one is different. This
  time we join with all nations across the globe in a common endeavor,
  using the great advances of science and our instinctive compassion to
  heal. We will succeed, and that success will belong to every one of
  us. We should take comfort that while we may have more still to
  handle, better days return. We will be with our friends again. We will
  be with our families again. We will meet again.
\end{itemize}

{[}music{]}

michael barbaro

That's it for ``The Daily.'' I'm Michael Barbaro. See you tomorrow.

Advertisement

\protect\hyperlink{after-bottom}{Continue reading the main story}

\hypertarget{site-index}{%
\subsection{Site Index}\label{site-index}}

\hypertarget{site-information-navigation}{%
\subsection{Site Information
Navigation}\label{site-information-navigation}}

\begin{itemize}
\tightlist
\item
  \href{https://help.nytimes3xbfgragh.onion/hc/en-us/articles/115014792127-Copyright-notice}{©~2020~The
  New York Times Company}
\end{itemize}

\begin{itemize}
\tightlist
\item
  \href{https://www.nytco.com/}{NYTCo}
\item
  \href{https://help.nytimes3xbfgragh.onion/hc/en-us/articles/115015385887-Contact-Us}{Contact
  Us}
\item
  \href{https://www.nytco.com/careers/}{Work with us}
\item
  \href{https://nytmediakit.com/}{Advertise}
\item
  \href{http://www.tbrandstudio.com/}{T Brand Studio}
\item
  \href{https://www.nytimes3xbfgragh.onion/privacy/cookie-policy\#how-do-i-manage-trackers}{Your
  Ad Choices}
\item
  \href{https://www.nytimes3xbfgragh.onion/privacy}{Privacy}
\item
  \href{https://help.nytimes3xbfgragh.onion/hc/en-us/articles/115014893428-Terms-of-service}{Terms
  of Service}
\item
  \href{https://help.nytimes3xbfgragh.onion/hc/en-us/articles/115014893968-Terms-of-sale}{Terms
  of Sale}
\item
  \href{https://spiderbites.nytimes3xbfgragh.onion}{Site Map}
\item
  \href{https://help.nytimes3xbfgragh.onion/hc/en-us}{Help}
\item
  \href{https://www.nytimes3xbfgragh.onion/subscription?campaignId=37WXW}{Subscriptions}
\end{itemize}
