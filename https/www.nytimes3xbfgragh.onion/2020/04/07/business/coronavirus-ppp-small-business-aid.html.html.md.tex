Sections

SEARCH

\protect\hyperlink{site-content}{Skip to
content}\protect\hyperlink{site-index}{Skip to site index}

\href{https://www.nytimes3xbfgragh.onion/section/business}{Business}

\href{https://myaccount.nytimes3xbfgragh.onion/auth/login?response_type=cookie\&client_id=vi}{}

\href{https://www.nytimes3xbfgragh.onion/section/todayspaper}{Today's
Paper}

\href{/section/business}{Business}\textbar{}Small Business Aid Program
Stretches Agency to Its Limits

\url{https://nyti.ms/3e35x4a}

\begin{itemize}
\item
\item
\item
\item
\item
\end{itemize}

Advertisement

\protect\hyperlink{after-top}{Continue reading the main story}

Supported by

\protect\hyperlink{after-sponsor}{Continue reading the main story}

\hypertarget{small-business-aid-program-stretches-agency-to-its-limits}{%
\section{Small Business Aid Program Stretches Agency to Its
Limits}\label{small-business-aid-program-stretches-agency-to-its-limits}}

The Small Business Administration is not equipped to handle the demand
for emergency loans. It is trying to fix aging systems, but the delays
are rattling banks and small businesses.

\includegraphics{https://static01.graylady3jvrrxbe.onion/images/2020/04/07/business/07virus-sbafix/merlin_171232413_62d7aee1-fa88-4cce-b244-12425258cf83-articleLarge.jpg?quality=75\&auto=webp\&disable=upscale}

By \href{https://www.nytimes3xbfgragh.onion/by/emily-flitter}{Emily
Flitter},
\href{https://www.nytimes3xbfgragh.onion/by/david-mccabe}{David McCabe}
and \href{https://www.nytimes3xbfgragh.onion/by/stacy-cowley}{Stacy
Cowley}

\begin{itemize}
\item
  April 7, 2020
\item
  \begin{itemize}
  \item
  \item
  \item
  \item
  \item
  \end{itemize}
\end{itemize}

Five days after the start of a \$349 billion emergency effort to get
money into the hands of small businesses, the agency at the heart of the
program is emerging as its biggest bottleneck.

The Small Business Administration, lightly staffed and working with
aging technology, has been caught unprepared for the onrush of demand
from desperate small-business owners who urgently need these loans as
the coronavirus stalls the economy. In a boom year, the agency backs
\$30 billion of small-business loans --- about the same amount that
banks are now seeking on behalf of their customers in a day.

Larry Kudlow, the director of the National Economic Council, said on
Tuesday that 178,000 loans totaling \$50 billion had been approved for
small businesses through the Paycheck Protection Program that was
unveiled Friday by the S.B.A. and the Treasury Department. But bankers,
small-business owners and others participating in the program say very
little of that money has actually reached companies seeking the cash.
The delays are causing confusion and panic among borrowers, especially
those who see Trump administration officials playing up the program's
success. They worry they are being left behind.

``The expectation that this \$2 trillion package would go through
Congress and that the money would be flowing three days later, that was
never a realistic expectation,'' said Patrick Ryan, the chief executive
of First Bank, a lender based in New Jersey. ``But I get why people are
frustrated.''

The wait can be harrowing for small businesses, which are unlikely to
have cash buffers to survive the economic shutdown. Erik Anderson, who
co-owns a chain of high-end hair salons for men in the Midwest, Scissors
and Scotch, said he and his two partners began talking to their bank
last Wednesday about getting a loan under the program.

By Friday, their Omaha-based lender, Union Bank \& Trust, told them that
loans they were seeking for two of their entities had received
authorization from the S.B.A. but that they would have to submit more
paperwork.

``It was a little bit clunky,'' Mr. Anderson said. The partners first
completed one form, then another that the bank said to use instead. The
bank has still not told them exactly when they will receive the money,
although Mr. Anderson said he was hoping it would come at the end of the
week or early next week. A Union Bank \& Trust spokeswoman did not
return calls seeking comment.

Lenders say they are being hamstrung by the S.B.A.'s slow-moving
systems. Some lenders simply can't get access to the agency's online
application system --- which went down for more than an hour on Monday
--- to submit borrower information quickly and at high volumes. They are
frustrated at having to wait for hours to connect to S.B.A. officials by
phone for help. Requests for more logins, which would allow more bank
employees to input borrower data to process loans faster, have been
ignored or denied.

What's more, with no plan in place for banks to sell the loans and
recoup some of the costs of providing cash to borrowers, smaller banks
are simply running out of money. The Federal Reserve said on Monday that
it would
\href{https://www.nytimes3xbfgragh.onion/2020/04/06/business/economy/federal-reserve-small-business-loans.html}{create
a program for buying the loans from banks}, but it has not yet provided
details.

A senior S.B.A. official noted that the agency had added features to
help borrowers and lenders in recent days.

Many banks are worried that the S.B.A. has provided them with paperwork
that fails to give them legal cover to make loans under the program.
They fear that the S.B.A. might later have a technical objection to some
portion of their loans, meaning they could end up being unable to recoup
the sums they're handing out to customers. That problem was a focus of a
web meeting the Georgia Bankers Association held for its members on
Sunday.

The S.B.A.'s instructions tell banks they ``must'' use a specific form
to close their loans, but the form has not yet been updated and has
language incompatible with the paycheck program's terms. The agency is
aware of the problem but has not yet offered a fix, said Jonathan
Hightower, a lawyer on the web call.

Some banks are jury-rigging their own forms for borrowers and hoping
they will pass regulatory muster, he said. ``They expect banks to move
forward with getting these loans closed,'' Mr. Hightower said, referring
to the Treasury Department. ``Now, clearly, that's your decision and
you've got some risks to measure in that.''

The storm of problems means that even borrowers who have successfully
completed a loan application with their bank are stuck in a limbo of
which they may not be aware. Their loans have been approved, but in some
cases the banks offering them are afraid to send out the money.

For instance, of the 75 small businesses that Scott Salmon, a New Jersey
lawyer, has helped with applications, fewer than 10 have been notified
their loans were approved. None have received money.

``If you can't get the loan today or tomorrow, don't worry,'' Treasury
Secretary Steven Mnuchin said Tuesday on Fox Business. ``There will be
money. And if we run out of money, we'll come back for more.'' Later
Tuesday, he said the Treasury had requested an additional \$250 billion
in funding from Congress.

The S.B.A. is trying to increase its capacity, fix its paperwork and
handle more loan requests, but it is struggling to do so quickly. It has
received \$675 million from Congress to lift its operating budget. It
has
\href{https://www.nytimes3xbfgragh.onion/2020/04/03/us/politics/small-business-administration-coronavirus.html}{doled
out millions of dollars} in contracts recently to support the small
business program and its economic disaster loans program, which is also
being used by businesses suffering because of the virus.

In the days before the program opened, there were worries that an S.B.A.
system that processes loan applications, known as E-Tran, would buckle
under the increased demand. The agency's staff worked through the
weekend trying to shore up its technical abilities before the program
went live, according to two people with knowledge of their preparations.

When the program opened on Friday, the system slowed down as loan
applications flowed in, with the issues persisting through the weekend.
According to a Washington bank lobbyist, one of the four biggest banks
reported that on Sunday night it took 72 minutes to enter one loan
application and the E-Tran system crashed 13 times during the process.

Those issues continued. The system slowed down on Monday, to the point
that some lenders were unable to use it. The S.B.A. took steps to try to
smooth out the glitches and increase its capacity for loan applications,
according to two people with knowledge of the matter.

The senior S.B.A. official disputed the idea that there had been a
prolonged outage of the E-Tran system on Monday, saying the agency had
been able to process billions of dollars in loans all day.

The S.B.A. has also been building a portal that will allow lenders that
have not worked with the agency before to begin offering loans. Senator
Marco Rubio, the Florida Republican who is chairman of the Senate's
small business committee, said Monday that the portal was being
developed with Amazon Web Services, the cloud computing giant, and an
unnamed ``systems integrator.''

The official said that the portal for new lenders had gone online on
Tuesday.

Not all banks are waiting for the S.B.A. to iron out the program's
kinks. One of the earliest banks to begin processing loan requests was
ConnectOne Bank, a New Jersey-based lender that has already begun
distributing money under the program, according to its chief executive,
Frank Sorrentino.

Mr. Sorrentino said the bank's officials were satisfied with the
S.B.A.'s paperwork and confident that a market for selling the loans
would soon be up and running. While the agency's online portal was
sometimes slow, he said, it had not stopped the bank from entering
borrowers' data and receiving loan approval from the agency.

``It's amazing what they've been able to accomplish in such a short
time,'' he said. ``I'm really proud of what this administration has been
able to do.''

Alan Rappeport contributed reporting.

Advertisement

\protect\hyperlink{after-bottom}{Continue reading the main story}

\hypertarget{site-index}{%
\subsection{Site Index}\label{site-index}}

\hypertarget{site-information-navigation}{%
\subsection{Site Information
Navigation}\label{site-information-navigation}}

\begin{itemize}
\tightlist
\item
  \href{https://help.nytimes3xbfgragh.onion/hc/en-us/articles/115014792127-Copyright-notice}{©~2020~The
  New York Times Company}
\end{itemize}

\begin{itemize}
\tightlist
\item
  \href{https://www.nytco.com/}{NYTCo}
\item
  \href{https://help.nytimes3xbfgragh.onion/hc/en-us/articles/115015385887-Contact-Us}{Contact
  Us}
\item
  \href{https://www.nytco.com/careers/}{Work with us}
\item
  \href{https://nytmediakit.com/}{Advertise}
\item
  \href{http://www.tbrandstudio.com/}{T Brand Studio}
\item
  \href{https://www.nytimes3xbfgragh.onion/privacy/cookie-policy\#how-do-i-manage-trackers}{Your
  Ad Choices}
\item
  \href{https://www.nytimes3xbfgragh.onion/privacy}{Privacy}
\item
  \href{https://help.nytimes3xbfgragh.onion/hc/en-us/articles/115014893428-Terms-of-service}{Terms
  of Service}
\item
  \href{https://help.nytimes3xbfgragh.onion/hc/en-us/articles/115014893968-Terms-of-sale}{Terms
  of Sale}
\item
  \href{https://spiderbites.nytimes3xbfgragh.onion}{Site Map}
\item
  \href{https://help.nytimes3xbfgragh.onion/hc/en-us}{Help}
\item
  \href{https://www.nytimes3xbfgragh.onion/subscription?campaignId=37WXW}{Subscriptions}
\end{itemize}
