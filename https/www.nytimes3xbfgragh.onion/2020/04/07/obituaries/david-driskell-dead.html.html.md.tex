Sections

SEARCH

\protect\hyperlink{site-content}{Skip to
content}\protect\hyperlink{site-index}{Skip to site index}

\href{https://www.nytimes3xbfgragh.onion/section/obituaries}{Obituaries}

\href{https://myaccount.nytimes3xbfgragh.onion/auth/login?response_type=cookie\&client_id=vi}{}

\href{https://www.nytimes3xbfgragh.onion/section/todayspaper}{Today's
Paper}

\href{/section/obituaries}{Obituaries}\textbar{}David Driskell, 88,
Pivotal Champion of African-American Art, Dies

\url{https://nyti.ms/3aN7BLz}

\begin{itemize}
\item
\item
\item
\item
\item
\end{itemize}

\href{https://www.nytimes3xbfgragh.onion/news-event/coronavirus?action=click\&pgtype=Article\&state=default\&region=TOP_BANNER\&context=storylines_menu}{The
Coronavirus Outbreak}

\begin{itemize}
\tightlist
\item
  live\href{https://www.nytimes3xbfgragh.onion/2020/08/04/world/coronavirus-covid-19.html?action=click\&pgtype=Article\&state=default\&region=TOP_BANNER\&context=storylines_menu}{Latest
  Updates}
\item
  \href{https://www.nytimes3xbfgragh.onion/interactive/2020/us/coronavirus-us-cases.html?action=click\&pgtype=Article\&state=default\&region=TOP_BANNER\&context=storylines_menu}{Maps
  and Cases}
\item
  \href{https://www.nytimes3xbfgragh.onion/interactive/2020/science/coronavirus-vaccine-tracker.html?action=click\&pgtype=Article\&state=default\&region=TOP_BANNER\&context=storylines_menu}{Vaccine
  Tracker}
\item
  \href{https://www.nytimes3xbfgragh.onion/2020/08/02/us/covid-college-reopening.html?action=click\&pgtype=Article\&state=default\&region=TOP_BANNER\&context=storylines_menu}{College
  Reopening}
\item
  \href{https://www.nytimes3xbfgragh.onion/live/2020/08/03/business/stock-market-today-coronavirus?action=click\&pgtype=Article\&state=default\&region=TOP_BANNER\&context=storylines_menu}{Economy}
\end{itemize}

Advertisement

\protect\hyperlink{after-top}{Continue reading the main story}

Supported by

\protect\hyperlink{after-sponsor}{Continue reading the main story}

Those We've Lost

\hypertarget{david-driskell-88-pivotal-champion-of-african-american-art-dies}{%
\section{David Driskell, 88, Pivotal Champion of African-American Art,
Dies}\label{david-driskell-88-pivotal-champion-of-african-american-art-dies}}

An artist himself, Professor Driskell recognized the role of black
artists in the broader story of American art. He died of the
coronavirus.

\includegraphics{https://static01.graylady3jvrrxbe.onion/images/2020/04/09/obituaries/07Driskell1/07Driskell1-articleLarge.jpg?quality=75\&auto=webp\&disable=upscale}

\href{https://www.nytimes3xbfgragh.onion/by/neil-genzlinger}{\includegraphics{https://static01.graylady3jvrrxbe.onion/images/2018/06/13/multimedia/author-neil-genzlinger/author-neil-genzlinger-thumbLarge.jpg}}

By \href{https://www.nytimes3xbfgragh.onion/by/neil-genzlinger}{Neil
Genzlinger}

\begin{itemize}
\item
  Published April 7, 2020Updated April 16, 2020
\item
  \begin{itemize}
  \item
  \item
  \item
  \item
  \item
  \end{itemize}
\end{itemize}

\emph{This obituary is part of a series about}
\href{https://www.nytimes3xbfgragh.onion/series/people-who-have-died-of-the-coronavirus}{\emph{people
who have died in the coronavirus pandemic}}\emph{.}

David C. Driskell, an artist, art historian and curator who was pivotal
in bringing recognition to African-American art and its importance in
the broader story of art in the United States and beyond, died on April
1 in a hospital near his home in Hyattsville, Md. He was 88.

The University of Maryland, where he held the title of distinguished
university professor of art,
\href{https://arhu.umd.edu/news/david-c-driskell-african-american-art-pioneer-dies-88}{said
in a posting} on its website that the cause was the coronavirus.

Professor Driskell was teaching at Fisk University in Nashville in the
mid-1970s when he began putting together
``\href{https://www.lacma.org/art/exhibition/two-centuries-black-american-art}{Two
Centuries of Black American Art}: 1750-1950,'' a landmark exhibition
that was first mounted at the Los Angeles County Museum of Art and that
later traveled to the Dallas Museum of Fine Arts, the High Museum of Art
in Atlanta and the Brooklyn Museum.

It was a sweeping show featuring more than 200 works by 63 named artists
as well as anonymous crafts workers. Some critics found it too
scattershot --- ``as an anthology, it needs serious editing,''
\href{https://www.nytimes3xbfgragh.onion/1977/06/24/archives/2-centuries-of-black-art-at-brooklyn-two-centuries-of-black-art.html?searchResultPosition=1}{Grace
Glueck wrote} in The New York Times --- but Professor Driskell
maintained that that was by design.

``I was not looking for a unified theme,'' he
\href{https://www.nytimes3xbfgragh.onion/1977/06/29/archives/black-art-label-disputed-by-curator.html?searchResultPosition=1}{told
The Times} in 1977. ``And this, of course, usually upsets the critics
because they want to see a continuous kind of thing. I was looking for a
body of work which showed first of all that blacks had been stable
participants in American visual culture for more than 200 years, and by
stable participants I simply mean that in many cases they had been the
backbone.''

\includegraphics{https://static01.graylady3jvrrxbe.onion/images/2020/04/09/obituaries/07driskell5/07driskell5-articleLarge.jpg?quality=75\&auto=webp\&disable=upscale}

Earlier scholars, including Professor Driskell's mentor,
\href{https://americanart.si.edu/artist/james-porter-3843}{James A.
Porter}, had done pioneering work on the history of black art and
artists, but Professor Driskell pushed further.

``Driskell did not so much discover the best known African-American
artists as he did establish African-American art as a legitimate and
distinct field of study,''
\href{http://keithmorrison.com/?page_id=207}{Keith Morrison}, who was
then dean of the Tyler School of Art at Temple University, wrote in the
foreword to ``David C. Driskell: Artist and Scholar,'' a 2006 biography
by Julie L. McGee.

``Very few scholars in the annals of human history can be said to have
established an entire field of study,'' he added, but Professor
Driskell's show ``did just that.''

David Clyde Driskell (pronounced like Driscol) was born on June 7, 1931,
in Eatonton, Ga., southeast of Atlanta. His father, George Washington
Driskell, was a minister, and his mother, Mary Cloud Driskell, was a
homemaker. His mother, he said, passed on to him a washing pot that had
belonged to her grandmother.

``And she said Grandma Leathy would tell her stories about this pot,''
Professor Driskell said in an
\href{https://www.aaa.si.edu/collections/interviews/oral-history-interview-david-driskell-15943\#transcript}{oral
history} recorded in 2009 for the Smithsonian Institution's Archives of
American Art. ``This was the pot that her mother used to cover her head
to pray in slavery so nobody would hear her praying for freedom.''

When he was 5 the family moved to western North Carolina, where he
attended segregated schools. The high school he attended required a
35-mile bus ride each way.

He had received a \$90 scholarship to Shaw University in Raleigh, N.C.,
but at the last minute he decided he wanted to go to Howard University
in Washington. He arrived three weeks after classes had begun, having
not gone through the application process.

``They tried to be firm with me,'' he recalled, ``and said: `School has
been in session for three weeks. You can't just come to college. You
have to make an application.' I insisted on staying. I said, `Well, I'm
here; give me an application.'''

He started out studying history, but in 1951 he took his first art
course, a drawing class. One day a distinguished-looking man dropped in
on the class and began admiring a drawing over Professor Driskell's
shoulder. Realizing that Professor Driskell was not one of the regular
art students, the man asked him what his major was.

``I said history,'' Professor Driskell recalled. ``And he looked at my
drawing and looked at me and said, `You don't belong over there; you
belong here.'''

The man was James Porter, the acclaimed African-American art historian
and a teacher at the university. Professor Driskell studied under him
and other notable scholars on the Howard faculty, and in 1953 he
received a summer scholarship to the Skowhegan School of Painting and
Sculpture in Maine.

He graduated from Howard with a bachelor's degree in art in 1955 and
taught for several years at Talladega College in Alabama. He earned a
master's degree from the Catholic University of America in Washington in
1962, the same year he joined the Howard faculty. He moved to Fisk in
1966.

In 1949, Fisk had received a substantial gift of art from
\href{https://www.nytimes3xbfgragh.onion/1986/03/07/obituaries/georgia-o-keeffe-dead-at-98-shaper-of-modern-art-in-us.html}{Georgia
O'Keeffe}, including several of her own works, and Professor Driskell
took an active role in overseeing and maintaining it.

In 1970 the Whitney Museum of American Art in New York staged an
O'Keeffe retrospective and borrowed one of her paintings from Fisk. That
got Professor Driskell an invitation to a highbrow dinner at a Fifth
Avenue apartment to celebrate the opening.

``Everybody was driving up in big limousines and cars,'' he recalled.
``I arrived in a taxi. The doorman was ushering people in. When I
arrived, the doorman asked me, as I was coming in, `Oh, are you
reporting to work?'''

It was an experience that underscored for him that the art world was
still largely white. Professor Driskell began to change that with the
1976 exhibition, although he acknowledged the disconnect between arguing
that black artists had always been important and then separating out
their work for its own exhibition.

``We don't go around saying `white art,''' he told The Times in 1977,
``but I think it's very important for us to keep saying `black art'
until it becomes recognized as American art.''

Image

Professor Driskell in his studio in 2004. He worked in watercolor,
gouache, collage and more. Credit...Doug Jones/Portland Press Herald,
via Getty Images

As for his own art, Professor Driskell worked in watercolor, gouache,
collage and more. A 1993 exhibition at the Midtown Payson gallery in
Manhattan featured works of his with nature as a central theme,
vermilion and greens dominating.

``Painted strips of canvas and paper are used side by side as collage
material, and both are often further worked with fine calligraphic
strokes,'' Holland Cotter wrote in a review in The New York Times. ``The
sense of hands-on intimacy that results is the work's most engaging
feature.''

Professor Driskell was also an art collector, and not merely a casual
one. He was said to own one of the finest private collections of
African-American art in the world. It numbered some 450 pieces in 1998,
when the University of Maryland, where Professor Driskell had been a
professor since 1977, mounted
\href{http://www.tfaoi.com/aa/1aa/1aa644.htm}{``Narratives of African
American Art and Identity: The David C. Driskell Collection,''} an
exhibition of 100 works that later toured to Dallas, San Francisco and
elsewhere.

In 2001 the University of Maryland established the
\href{http://www.driskellcenter.umd.edu/index.php}{David C. Driskell
Center}, which documents and presents African-American art and holds the
Driskell archive.

He is survived by his wife, Thelma Grace Driskell, whom he married in
1952; two daughters, Daviryne McNeill and Daphne Coles; five
grandchildren; and four great-grandchildren.

As a collector, Professor Driskell was interested in not only black
artists. He liked to scour out-of-the-way shops, and he knew a bargain
when he saw one. In 2003 he told The Baltimore Sun how he had come to
acquire a Rembrandt while strolling through Copenhagen.

He saw it in the window of a quirky-looking shop, he recalled. The owner
thought it was just a facsimile.

``But I had just studied Rembrandt's prints at The Hague,'' Professor
Driskell recalled, ``and I could see this one had ink qualities
different from a reproduction. When he said \$10, I said, `I'll take
it.'''

\href{https://www.nytimes3xbfgragh.onion/interactive/2020/obituaries/people-died-coronavirus-obituaries.html?action=click\&pgtype=Article\&state=default\&region=BELOW_MAIN_CONTENT\&context=covid_obits_promo}{}

\hypertarget{those-weve-lost}{%
\section{Those We've Lost}\label{those-weve-lost}}

The coronavirus pandemic has taken an incalculable death toll. This
series is designed to put names and faces to the numbers.

Read more

\includegraphics{https://static01.graylady3jvrrxbe.onion/images/2020/07/30/obituaries/30Pedro/30Pedro-square640.jpg}

\hypertarget{bernaldina-josuxe9-pedro}{%
\section{Bernaldina José Pedro}\label{bernaldina-josuxe9-pedro}}

d. Boa Vista, Brazil

Leader among the Indigenous Macuxi

\includegraphics{https://static01.graylady3jvrrxbe.onion/images/2020/07/31/obituaries/31Swing/merlin_175167783_8913bc90-0d64-43f3-a655-1bb1bf1601c9-square640.jpg}

\hypertarget{john-eric-swing}{%
\section{John Eric Swing}\label{john-eric-swing}}

d. Fountain Valley, Calif.

Champion of Filipino-Americans

\includegraphics{https://static01.graylady3jvrrxbe.onion/images/2020/07/27/obituaries/27Victor/merlin_175001436_38b11f8e-227a-4e2c-9821-7618af9b2524-square640.jpg}

\hypertarget{victor-victor}{%
\section{Victor Victor}\label{victor-victor}}

d. Santo Domingo, Dominican Republic

Beloved musician of the Dominican Republic

\includegraphics{https://static01.graylady3jvrrxbe.onion/images/2020/07/31/obituaries/31Negron/merlin_175160169_516322ae-fd23-4969-b6b2-193ced371105-square640.jpg}

\hypertarget{dr-eddie-negruxf3n}{%
\section{Dr. Eddie Negrón}\label{dr-eddie-negruxf3n}}

d. Fort Walton Beach, Fla.

Internist on Florida's Emerald Coast

\includegraphics{https://static01.graylady3jvrrxbe.onion/images/2020/07/30/obituaries/30Dobson/merlin_175115928_f6b9271c-8f05-4fe1-a38a-5ca4a58f8935-square640.jpg}

\hypertarget{dobby-dobson}{%
\section{Dobby Dobson}\label{dobby-dobson}}

d. Coral Springs, Fla.

Jamaican singer and songwriter

\includegraphics{https://static01.graylady3jvrrxbe.onion/images/2020/08/01/obituaries/28Gonzalez/merlin_175002771_beb57888-3951-409a-ae13-03a94b2e962e-square640.jpg}

\hypertarget{waldemar-gonzalez}{%
\section{Waldemar Gonzalez}\label{waldemar-gonzalez}}

d. White Plains, N.Y.

Teacher and social worker

Advertisement

\protect\hyperlink{after-bottom}{Continue reading the main story}

\hypertarget{site-index}{%
\subsection{Site Index}\label{site-index}}

\hypertarget{site-information-navigation}{%
\subsection{Site Information
Navigation}\label{site-information-navigation}}

\begin{itemize}
\tightlist
\item
  \href{https://help.nytimes3xbfgragh.onion/hc/en-us/articles/115014792127-Copyright-notice}{©~2020~The
  New York Times Company}
\end{itemize}

\begin{itemize}
\tightlist
\item
  \href{https://www.nytco.com/}{NYTCo}
\item
  \href{https://help.nytimes3xbfgragh.onion/hc/en-us/articles/115015385887-Contact-Us}{Contact
  Us}
\item
  \href{https://www.nytco.com/careers/}{Work with us}
\item
  \href{https://nytmediakit.com/}{Advertise}
\item
  \href{http://www.tbrandstudio.com/}{T Brand Studio}
\item
  \href{https://www.nytimes3xbfgragh.onion/privacy/cookie-policy\#how-do-i-manage-trackers}{Your
  Ad Choices}
\item
  \href{https://www.nytimes3xbfgragh.onion/privacy}{Privacy}
\item
  \href{https://help.nytimes3xbfgragh.onion/hc/en-us/articles/115014893428-Terms-of-service}{Terms
  of Service}
\item
  \href{https://help.nytimes3xbfgragh.onion/hc/en-us/articles/115014893968-Terms-of-sale}{Terms
  of Sale}
\item
  \href{https://spiderbites.nytimes3xbfgragh.onion}{Site Map}
\item
  \href{https://help.nytimes3xbfgragh.onion/hc/en-us}{Help}
\item
  \href{https://www.nytimes3xbfgragh.onion/subscription?campaignId=37WXW}{Subscriptions}
\end{itemize}
