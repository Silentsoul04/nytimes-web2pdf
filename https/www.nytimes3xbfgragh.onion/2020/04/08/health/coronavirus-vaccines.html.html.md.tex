Sections

SEARCH

\protect\hyperlink{site-content}{Skip to
content}\protect\hyperlink{site-index}{Skip to site index}

\href{https://www.nytimes3xbfgragh.onion/section/health}{Health}

\href{https://myaccount.nytimes3xbfgragh.onion/auth/login?response_type=cookie\&client_id=vi}{}

\href{https://www.nytimes3xbfgragh.onion/section/todayspaper}{Today's
Paper}

\href{/section/health}{Health}\textbar{}More Coronavirus Vaccines and
Treatments Move Toward Human Trials

\url{https://nyti.ms/2RnJFqr}

\begin{itemize}
\item
\item
\item
\item
\item
\item
\end{itemize}

\hypertarget{the-coronavirus-outbreak}{%
\subsubsection{\texorpdfstring{\href{https://www.nytimes3xbfgragh.onion/news-event/coronavirus?name=styln-coronavirus-national\&region=TOP_BANNER\&block=storyline_menu_recirc\&action=click\&pgtype=Article\&impression_id=2427c390-efbb-11ea-b4bc-89fe1dc99e22\&variant=undefined}{The
Coronavirus
Outbreak}}{The Coronavirus Outbreak}}\label{the-coronavirus-outbreak}}

\begin{itemize}
\tightlist
\item
  live\href{https://www.nytimes3xbfgragh.onion/2020/09/05/world/coronavirus-covid.html?name=styln-coronavirus-national\&region=TOP_BANNER\&block=storyline_menu_recirc\&action=click\&pgtype=Article\&impression_id=2427eaa0-efbb-11ea-b4bc-89fe1dc99e22\&variant=undefined}{Latest
  Updates}
\item
  \href{https://www.nytimes3xbfgragh.onion/interactive/2020/us/coronavirus-us-cases.html?name=styln-coronavirus-national\&region=TOP_BANNER\&block=storyline_menu_recirc\&action=click\&pgtype=Article\&impression_id=2427eaa1-efbb-11ea-b4bc-89fe1dc99e22\&variant=undefined}{Maps
  and Cases}
\item
  \href{https://www.nytimes3xbfgragh.onion/interactive/2020/science/coronavirus-vaccine-tracker.html?name=styln-coronavirus-national\&region=TOP_BANNER\&block=storyline_menu_recirc\&action=click\&pgtype=Article\&impression_id=2427eaa2-efbb-11ea-b4bc-89fe1dc99e22\&variant=undefined}{Vaccine
  Tracker}
\item
  \href{https://www.nytimes3xbfgragh.onion/2020/09/02/your-money/eviction-moratorium-covid.html?name=styln-coronavirus-national\&region=TOP_BANNER\&block=storyline_menu_recirc\&action=click\&pgtype=Article\&impression_id=2427eaa3-efbb-11ea-b4bc-89fe1dc99e22\&variant=undefined}{Eviction
  Moratorium}
\item
  \href{https://www.nytimes3xbfgragh.onion/interactive/2020/09/02/magazine/food-insecurity-hunger-us.html?name=styln-coronavirus-national\&region=TOP_BANNER\&block=storyline_menu_recirc\&action=click\&pgtype=Article\&impression_id=242811b0-efbb-11ea-b4bc-89fe1dc99e22\&variant=undefined}{American
  Hunger}
\end{itemize}

Advertisement

\protect\hyperlink{after-top}{Continue reading the main story}

Supported by

\protect\hyperlink{after-sponsor}{Continue reading the main story}

\hypertarget{more-coronavirus-vaccines-and-treatments-move-toward-human-trials}{%
\section{More Coronavirus Vaccines and Treatments Move Toward Human
Trials}\label{more-coronavirus-vaccines-and-treatments-move-toward-human-trials}}

Just three months after the start of the coronavirus pandemic, several
biotech companies are beginning trials of promising vaccines and
treatments.

\includegraphics{https://static01.graylady3jvrrxbe.onion/images/2020/04/08/science/08VIRUS-VACCINES1/08VIRUS-VACCINES1-articleLarge.jpg?quality=75\&auto=webp\&disable=upscale}

By \href{https://www.nytimes3xbfgragh.onion/by/knvul-sheikh}{Knvul
Sheikh} and
\href{https://www.nytimes3xbfgragh.onion/by/katie-thomas}{Katie Thomas}

\begin{itemize}
\item
  Published April 8, 2020Updated May 2, 2020
\item
  \begin{itemize}
  \item
  \item
  \item
  \item
  \item
  \item
  \end{itemize}
\end{itemize}

As the
\href{https://www.nytimes3xbfgragh.onion/2020/04/27/world/europe/coronavirus-vaccine-update-oxford.html}{coronavirus}
pandemic spreads at unprecedented rates, invading the lungs of people of
all ages, ethnicities and medical histories, companies are ratcheting up
their efforts to fight the disease with accelerated schedules for
creating new
\href{https://www.nytimes3xbfgragh.onion/2020/04/27/world/europe/coronavirus-vaccine-update-oxford.html}{vaccines},
and beginning clinical trials for potential treatments.

On Wednesday, Novavax, a Maryland-based biotech company, said it would
begin human trials in Australia in mid-May for its vaccine candidate.
Novavax is one of more than two dozen companies that have announced
promising vaccine programs that are speeding through the early stages of
testing unlike ever before.

Also on Wednesday, the stem-cell company Mesoblast said it was starting
a 240-patient clinical trial, supported by the National Institutes of
Health, that would test whether cells derived from bone marrow could
help patients who developed a deadly immune reaction to the
\href{https://www.nytimes3xbfgragh.onion/2020/04/10/health/coronavirus-antibody-test.html}{coronavirus}.

In normal circumstances, development of new
\href{https://www.nytimes3xbfgragh.onion/2020/05/15/us/politics/coronavirus-vaccine-timeline.html}{vaccines}
and treatments would take years. But the pharmaceutical industry is
racing to compress this timeline with the support of nonprofit
organizations, government agencies and regulatory authorities.

Novavax said its vaccine candidate had stimulated a powerful immune
response in lab and animal experiments, producing
\href{https://www.nytimes3xbfgragh.onion/2020/04/10/health/coronavirus-antibody-test.html}{antibodies}
that could fight off the coronavirus.

While a final product that would be widely available is still a year or
more away, the Novavax effort is one of many that is ready to be tested
in people.

A vaccine made by the biotech company Moderna is
\href{https://www.nytimes3xbfgragh.onion/2020/05/18/health/coronavirus-vaccine-moderna.html}{already
in a clinical trial}, which started March 15. Another one, developed by
Inovio Pharmaceuticals, was injected into the first adult volunteers on
Monday. The health care giant
\href{https://www.nytimes3xbfgragh.onion/2020/05/19/business/johnson-baby-powder-sales-stopped.html}{Johnson
\& Johnson} expects to start clinical trials in September, and has
received a nearly \$500 million partnership via a division of the U.S.
Department of Health and Human Services. And experimental vaccines
developed by researchers at the University of Pittsburgh and Baylor
College of Medicine in Houston are also waiting for permission from the
Food and Drug Administration to begin testing in people.

``We're all trying to do something which we have almost no precedents
for, which is accelerating a vaccine in the middle of a pandemic,'' said
Dr. Peter Hotez, who is a co-director of the Texas Children's Hospital
Center for Vaccine Development at Baylor College of Medicine.

There is no approved treatment for Covid-19, the illness caused by the
coronavirus, and researchers and doctors are testing a host of therapies
in a desperate bid to save the lives of people who have few other
options. President Trump has aggressively promoted two old malaria
drugs, which have shown only limited evidence of working as treatments
for the coronavirus. He has pushed for the drugs' broader use in
patients without the more rigorous clinical trials typically used to
evaluate treatments.

\hypertarget{latest-updates-the-coronavirus-outbreak}{%
\section{\texorpdfstring{\href{https://www.nytimes3xbfgragh.onion/2020/09/04/world/covid-19-coronavirus.html?action=click\&pgtype=Article\&state=default\&region=MAIN_CONTENT_1\&context=storylines_live_updates}{Latest
Updates: The Coronavirus
Outbreak}}{Latest Updates: The Coronavirus Outbreak}}\label{latest-updates-the-coronavirus-outbreak}}

Updated 2020-09-05T12:05:40.998Z

\begin{itemize}
\tightlist
\item
  \href{https://www.nytimes3xbfgragh.onion/2020/09/04/world/covid-19-coronavirus.html?action=click\&pgtype=Article\&state=default\&region=MAIN_CONTENT_1\&context=storylines_live_updates\#link-1654f6ad}{Research
  connects vaping to a higher chance of catching the virus --- and
  suffering its worst effects.}
\item
  \href{https://www.nytimes3xbfgragh.onion/2020/09/04/world/covid-19-coronavirus.html?action=click\&pgtype=Article\&state=default\&region=MAIN_CONTENT_1\&context=storylines_live_updates\#link-52e4198a}{Another
  college football game won't be played as planned.}
\item
  \href{https://www.nytimes3xbfgragh.onion/2020/09/04/world/covid-19-coronavirus.html?action=click\&pgtype=Article\&state=default\&region=MAIN_CONTENT_1\&context=storylines_live_updates\#link-181cef0}{Pharmaceutical
  companies plan a joint pledge on safety standards as they move
  vaccines to the marketplace.}
\end{itemize}

\href{https://www.nytimes3xbfgragh.onion/2020/09/04/world/covid-19-coronavirus.html?action=click\&pgtype=Article\&state=default\&region=MAIN_CONTENT_1\&context=storylines_live_updates}{See
more updates}

More live coverage:
\href{https://www.nytimes3xbfgragh.onion/live/2020/09/04/business/stock-market-today-coronavirus?action=click\&pgtype=Article\&state=default\&region=MAIN_CONTENT_1\&context=storylines_live_updates}{Markets}

But Mesoblast is taking a more standard approach, testing the cell
therapy in 240 patients at more than 20 medical centers around the
country, which are part of the Cardiothoracic Surgical Trials Network, a
program created by the N.I.H. Patients will be randomly divided into
groups that will receive the therapy, and those that will get a placebo.
Researchers said the trial could yield initial results within months.

While many people who are infected with the coronavirus experience mild
symptoms, others develop a severe case when their immune systems go into
overdrive and begin attacking the body's organs, which is called a
``cytokine storm.'' This can set off something called acute respiratory
distress syndrome, which damages the lungs and is often deadly. Several
other treatments
\href{https://www.nytimes3xbfgragh.onion/2020/04/01/health/coronavirus-cytokine-storm-immune-system.html}{are
also being tried to calm this storm} and reset the immune system.

Dr. Silviu Itescu, chief executive of Mesoblast, said the company
decided to test its treatment in these Covid-19 patients because its
product had shown good results in children who developed a similar
deadly immune reaction called acute graft versus host disease, in which
the body's immune cells can attack healthy cells after receiving a
bone-marrow transplant. Their treatment is currently being reviewed by
the Food and Drug Administration for use in that disease.

``We put two and two together and said, `We think we've got something
that is safe and could have benefit,''' Dr. Itescu said.

Another stem cell company, Athersys,
\href{https://www.athersys.com/clinical-trials/ards/}{has said it is
also planning a study} of stem cells in coronavirus patients with
advanced respiratory distress syndrome, but is not as far along.

Nine coronavirus patients at Mount Sinai Hospital in New York have
received the Mesoblast treatment on an emergency basis, and doctors
there said the initial response was promising. Six patients were removed
from ventilation and others were being weaned off or had remained stable
--- a welcome development when most patients who need ventilator support
do not survive.

But Dr. Itescu said that even though the treatment was promising, it was
still not clear whether it would work, or whether the early promise was
merely anecdotal. Although many drugs are being tried outside of a
formal study, he said, ``We do think this is the right way, and a
randomized, controlled trial is the only way you are going to know
whether an approach works.''

\href{https://www.pfizer.com/news/press-release/press-release-detail/pfizer_advances_battle_against_covid_19_on_multiple_fronts}{Pfizer
said on Thursday} that it would begin testing an experimental drug, as
well as its rheumatoid arthritis treatment Xeljanz, in clinical trials
against Covid-19. Xeljanz, also known as tofacitinib, is an
anti-inflammatory medicine and could dampen the immune system in
patients with the severe respiratory syndrome. But the company warned
that research was still in its early stages and that Xeljanz should not
currently be used in patients with serious infections.

Still, a vaccine would be the
\href{https://www.nytimes3xbfgragh.onion/2020/01/28/health/coronavirus-vaccine.html}{best
way to stop further spread} of the coronavirus because it enhances the
immune system's natural defenses. Of course, many companies are also
struggling with ways to partner with manufacturing ventures to produce
enough vaccine so that it will be widely available.

\includegraphics{https://static01.graylady3jvrrxbe.onion/images/2020/04/08/science/08VIRUs-VACCINES2/08VIRUs-VACCINES2-articleLarge.jpg?quality=75\&auto=webp\&disable=upscale}

More than one million people around the world have already been sickened
by the coronavirus. For public health experts and those on the front
lines, a vaccine can't come soon enough.

``If you could only have a vaccine, just imagine you could walk out your
door confident that you were not going to get sick,'' said Dr. Gregory
Glenn, the president for research and development at Novavax. ``Because
of that, everyone is very motivated and working to move things
quickly.''

Novavax has worked on experimental vaccines for both SARS and MERS,
which are closely related to the new coronavirus. The company also has
vaccines for the seasonal flu and respiratory syncytial virus, which
causes colds, in the last stages of clinical trials.

When Chinese scientists posted the genetic sequence of the new
coronavirus in January, researchers at Novavax started working on
recombinant technology to make a synthetic version of the virus.
Researchers used a baculovirus to carry bits of genetic material from
the coronavirus into cells. Baculoviruses typically infect insects, so
they cannot replicate and cause illness in humans.

``We never use the real virus,'' Dr. Glenn said. ``But we can fool the
immune system to think it's been attacked.''

\href{https://www.nytimes3xbfgragh.onion/news-event/coronavirus?action=click\&pgtype=Article\&state=default\&region=MAIN_CONTENT_3\&context=storylines_faq}{}

\hypertarget{the-coronavirus-outbreak-}{%
\subsubsection{The Coronavirus Outbreak
›}\label{the-coronavirus-outbreak-}}

\hypertarget{frequently-asked-questions}{%
\paragraph{Frequently Asked
Questions}\label{frequently-asked-questions}}

Updated September 4, 2020

\begin{itemize}
\item ~
  \hypertarget{what-are-the-symptoms-of-coronavirus}{%
  \paragraph{What are the symptoms of
  coronavirus?}\label{what-are-the-symptoms-of-coronavirus}}

  \begin{itemize}
  \tightlist
  \item
    In the beginning, the coronavirus
    \href{https://www.nytimes3xbfgragh.onion/article/coronavirus-facts-history.html?action=click\&pgtype=Article\&state=default\&region=MAIN_CONTENT_3\&context=storylines_faq\#link-6817bab5}{seemed
    like it was primarily a respiratory illness}~--- many patients had
    fever and chills, were weak and tired, and coughed a lot, though
    some people don't show many symptoms at all. Those who seemed
    sickest had pneumonia or acute respiratory distress syndrome and
    received supplemental oxygen. By now, doctors have identified many
    more symptoms and syndromes. In April,
    \href{https://www.nytimes3xbfgragh.onion/2020/04/27/health/coronavirus-symptoms-cdc.html?action=click\&pgtype=Article\&state=default\&region=MAIN_CONTENT_3\&context=storylines_faq}{the
    C.D.C. added to the list of early signs}~sore throat, fever, chills
    and muscle aches. Gastrointestinal upset, such as diarrhea and
    nausea, has also been observed. Another telltale sign of infection
    may be a sudden, profound diminution of one's
    \href{https://www.nytimes3xbfgragh.onion/2020/03/22/health/coronavirus-symptoms-smell-taste.html?action=click\&pgtype=Article\&state=default\&region=MAIN_CONTENT_3\&context=storylines_faq}{sense
    of smell and taste.}~Teenagers and young adults in some cases have
    developed painful red and purple lesions on their fingers and toes
    --- nicknamed ``Covid toe'' --- but few other serious symptoms.
  \end{itemize}
\item ~
  \hypertarget{why-is-it-safer-to-spend-time-together-outside}{%
  \paragraph{Why is it safer to spend time together
  outside?}\label{why-is-it-safer-to-spend-time-together-outside}}

  \begin{itemize}
  \tightlist
  \item
    \href{https://www.nytimes3xbfgragh.onion/2020/05/15/us/coronavirus-what-to-do-outside.html?action=click\&pgtype=Article\&state=default\&region=MAIN_CONTENT_3\&context=storylines_faq}{Outdoor
    gatherings}~lower risk because wind disperses viral droplets, and
    sunlight can kill some of the virus. Open spaces prevent the virus
    from building up in concentrated amounts and being inhaled, which
    can happen when infected people exhale in a confined space for long
    stretches of time, said Dr. Julian W. Tang, a virologist at the
    University of Leicester.
  \end{itemize}
\item ~
  \hypertarget{why-does-standing-six-feet-away-from-others-help}{%
  \paragraph{Why does standing six feet away from others
  help?}\label{why-does-standing-six-feet-away-from-others-help}}

  \begin{itemize}
  \tightlist
  \item
    The coronavirus spreads primarily through droplets from your mouth
    and nose, especially when you cough or sneeze. The C.D.C., one of
    the organizations using that measure,
    \href{https://www.nytimes3xbfgragh.onion/2020/04/14/health/coronavirus-six-feet.html?action=click\&pgtype=Article\&state=default\&region=MAIN_CONTENT_3\&context=storylines_faq}{bases
    its recommendation of six feet}~on the idea that most large droplets
    that people expel when they cough or sneeze will fall to the ground
    within six feet. But six feet has never been a magic number that
    guarantees complete protection. Sneezes, for instance, can launch
    droplets a lot farther than six feet,
    \href{https://jamanetwork.com/journals/jama/fullarticle/2763852}{according
    to a recent study}. It's a rule of thumb: You should be safest
    standing six feet apart outside, especially when it's windy. But
    keep a mask on at all times, even when you think you're far enough
    apart.
  \end{itemize}
\item ~
  \hypertarget{i-have-antibodies-am-i-now-immune}{%
  \paragraph{I have antibodies. Am I now
  immune?}\label{i-have-antibodies-am-i-now-immune}}

  \begin{itemize}
  \tightlist
  \item
    As of right
    now,\href{https://www.nytimes3xbfgragh.onion/2020/07/22/health/covid-antibodies-herd-immunity.html?action=click\&pgtype=Article\&state=default\&region=MAIN_CONTENT_3\&context=storylines_faq}{~that
    seems likely, for at least several months.}~There have been
    frightening accounts of people suffering what seems to be a second
    bout of Covid-19. But experts say these patients may have a
    drawn-out course of infection, with the virus taking a slow toll
    weeks to months after initial exposure.~People infected with the
    coronavirus typically
    \href{https://www.nature.com/articles/s41586-020-2456-9}{produce}~immune
    molecules called antibodies, which are
    \href{https://www.nytimes3xbfgragh.onion/2020/05/07/health/coronavirus-antibody-prevalence.html?action=click\&pgtype=Article\&state=default\&region=MAIN_CONTENT_3\&context=storylines_faq}{protective
    proteins made in response to an
    infection}\href{https://www.nytimes3xbfgragh.onion/2020/05/07/health/coronavirus-antibody-prevalence.html?action=click\&pgtype=Article\&state=default\&region=MAIN_CONTENT_3\&context=storylines_faq}{.
    These antibodies may}~last in the body
    \href{https://www.nature.com/articles/s41591-020-0965-6}{only two to
    three months}, which may seem worrisome, but that's~perfectly normal
    after an acute infection subsides, said Dr. Michael Mina, an
    immunologist at Harvard University. It may be possible to get the
    coronavirus again, but it's highly unlikely that it would be
    possible in a short window of time from initial infection or make
    people sicker the second time.
  \end{itemize}
\item ~
  \hypertarget{what-are-my-rights-if-i-am-worried-about-going-back-to-work}{%
  \paragraph{What are my rights if I am worried about going back to
  work?}\label{what-are-my-rights-if-i-am-worried-about-going-back-to-work}}

  \begin{itemize}
  \tightlist
  \item
    Employers have to provide
    \href{https://www.osha.gov/SLTC/covid-19/standards.html}{a safe
    workplace}~with policies that protect everyone equally.
    \href{https://www.nytimes3xbfgragh.onion/article/coronavirus-money-unemployment.html?action=click\&pgtype=Article\&state=default\&region=MAIN_CONTENT_3\&context=storylines_faq}{And
    if one of your co-workers tests positive for the coronavirus, the
    C.D.C.}~has said that
    \href{https://www.cdc.gov/coronavirus/2019-ncov/community/guidance-business-response.html}{employers
    should tell their employees}~-\/- without giving you the sick
    employee's name -\/- that they may have been exposed to the virus.
  \end{itemize}
\end{itemize}

By combining the recombinant vaccine with an adjuvant, or substance that
increases immune stimulation, Novavax was able to achieve a high
neutralization titer in preclinical tests --- a measure of the
protective antibodies that can block the virus.

The company hopes to see a similar effect after giving more than 130
healthy adults two doses of the vaccine. Results of the trial, which
will be conducted in Australia, are expected around July.

Moderna and Inovio are pioneering a different approach.

Moderna uses RNA technology, while Inovio has developed DNA technology
to package the genetic code of coronavirus spike proteins, which make up
the crown around the virus and help it latch on to cells. This approach
has the advantage of being able to move to trials faster than vaccines
that require the production of viral proteins or a weakened version of
the actual virus to induce an immune response. But the technology is
still unproven. There are no approved RNA or DNA vaccines for any
disease.

Dr. Hotez's team and Johnson \& Johnson, on the other hand, are relying
on technology that is more similar to Novavax's approach because it has
been used successfully to create other vaccines in the past,
\href{https://www.nytimes3xbfgragh.onion/2019/07/26/health/ebola-africa.html}{including
one for Ebola} that has been registered in Europe and is used in the
recent epidemic in the Democratic Republic of Congo.

Some countries already have the manufacturing capabilities that will be
needed to scale up vaccine production, and that will keep costs low if
everything goes well.

``It's not very sexy, but it's a reliable approach,'' Dr. Hotez said.
``We know that it works.''

For now, the first stage of clinical trials for each potential
coronavirus vaccine must focus on how safe or toxic the vaccine may be
at different dose levels. Researchers will collect the medical histories
of volunteers participating in the trials and track their antibody
levels, liver enzymes and other indicators of emerging side effects.

One concern is that the vaccines may inadvertently cause a phenomenon
known as
\href{https://www.ncbi.nlm.nih.gov/pmc/articles/PMC6290032/}{disease
enhancement}, in which vaccinated people develop more severe
inflammation and disease than those who have never been vaccinated.
Studies of early SARS and MERS vaccines noted this troublesome
complication in some animal models.

``If everything looks good and the vaccine appears to be safe, then
we'll go on to trials with much bigger numbers and look at the vaccine
efficacy,'' said Dr. John Ervin, who is leading the Inovio clinical
trial in Kansas City, Mo.

In parallel, companies are planning to continue further animal testing,
as well as investing in manufacturing capacity both in the United States
and abroad. They will need millions of doses for additional clinical
trials and even more if a vaccine eventually goes to market.

Companies also have to be prepared for the possibility that some
candidates will fizzle out or that demand for a vaccine will decrease by
the time one is ready for widespread use. But industry experts are not
waiting for this to happen.

``The virus is racing through crowded urban areas and slums in certain
countries,'' Dr. Hotez said. ``How do you do social distancing in those
places? You don't.''

``We are building out a road map for how we work as a country for the
next two or three years,'' he continued. ``That's roughly the time frame
that we saw for the 1918 flu pandemic and that's probably likely for
Covid-19.''

Advertisement

\protect\hyperlink{after-bottom}{Continue reading the main story}

\hypertarget{site-index}{%
\subsection{Site Index}\label{site-index}}

\hypertarget{site-information-navigation}{%
\subsection{Site Information
Navigation}\label{site-information-navigation}}

\begin{itemize}
\tightlist
\item
  \href{https://help.nytimes3xbfgragh.onion/hc/en-us/articles/115014792127-Copyright-notice}{©~2020~The
  New York Times Company}
\end{itemize}

\begin{itemize}
\tightlist
\item
  \href{https://www.nytco.com/}{NYTCo}
\item
  \href{https://help.nytimes3xbfgragh.onion/hc/en-us/articles/115015385887-Contact-Us}{Contact
  Us}
\item
  \href{https://www.nytco.com/careers/}{Work with us}
\item
  \href{https://nytmediakit.com/}{Advertise}
\item
  \href{http://www.tbrandstudio.com/}{T Brand Studio}
\item
  \href{https://www.nytimes3xbfgragh.onion/privacy/cookie-policy\#how-do-i-manage-trackers}{Your
  Ad Choices}
\item
  \href{https://www.nytimes3xbfgragh.onion/privacy}{Privacy}
\item
  \href{https://help.nytimes3xbfgragh.onion/hc/en-us/articles/115014893428-Terms-of-service}{Terms
  of Service}
\item
  \href{https://help.nytimes3xbfgragh.onion/hc/en-us/articles/115014893968-Terms-of-sale}{Terms
  of Sale}
\item
  \href{https://spiderbites.nytimes3xbfgragh.onion}{Site Map}
\item
  \href{https://help.nytimes3xbfgragh.onion/hc/en-us}{Help}
\item
  \href{https://www.nytimes3xbfgragh.onion/subscription?campaignId=37WXW}{Subscriptions}
\end{itemize}
