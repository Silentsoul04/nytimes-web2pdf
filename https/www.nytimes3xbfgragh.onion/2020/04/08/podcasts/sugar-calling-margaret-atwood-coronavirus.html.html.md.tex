Sections

SEARCH

\protect\hyperlink{site-content}{Skip to
content}\protect\hyperlink{site-index}{Skip to site index}

\href{https://www.nytimes3xbfgragh.onion/spotlight/podcasts}{Podcasts}

\href{https://myaccount.nytimes3xbfgragh.onion/auth/login?response_type=cookie\&client_id=vi}{}

\href{https://www.nytimes3xbfgragh.onion/section/todayspaper}{Today's
Paper}

\href{/spotlight/podcasts}{Podcasts}\textbar{}`Roll Up Your Sleeves,
Girls'

\href{https://nyti.ms/2Xr82aJ}{https://nyti.ms/2Xr82aJ}

\begin{itemize}
\item
\item
\item
\item
\item
\end{itemize}

Advertisement

\protect\hyperlink{after-top}{Continue reading the main story}

transcript

Back to Sugar Calling

bars

0:00/34:32

-34:32

transcript

\hypertarget{roll-up-your-sleeves-girls}{%
\subsection{`Roll Up Your Sleeves,
Girls'}\label{roll-up-your-sleeves-girls}}

\hypertarget{hosted-by-cheryl-strayed-produced-by-kelly-prime-and-edited-by-sara-sarasohn}{%
\subsubsection{Hosted by Cheryl Strayed, produced by Kelly Prime and
edited by Sara
Sarasohn}\label{hosted-by-cheryl-strayed-produced-by-kelly-prime-and-edited-by-sara-sarasohn}}

\hypertarget{cheryl-strayed-calls-the-writer-margaret-atwood-to-talk-about-hard-times-pansies-and-eight-decades-of-getting-it-done}{%
\paragraph{Cheryl Strayed calls the writer Margaret Atwood to talk about
hard times, pansies and eight decades of getting it
done.}\label{cheryl-strayed-calls-the-writer-margaret-atwood-to-talk-about-hard-times-pansies-and-eight-decades-of-getting-it-done}}

Wednesday, April 8th, 2020

\begin{itemize}
\item
  cheryl strayed\\
  Today, I'm going to call Margaret Atwood, the legendary, extraordinary
  Margaret Atwood. Author of so many novels and books of poetry and
  children's books. I was still in my teens when The Handmaid's Tale
  came out, and I read it and was astonished by it. And I've loved her
  ever since. She's written so powerfully about people in incredibly
  difficult circumstances, people who can't even use their own names,
  people whose movement is constricted and whose thought is constricted.
  And I think that there's something, in essence, the way she writes
  about how we persist in the face of extreme difficulty that I felt
  like she might be the person to talk to at this very moment when so
  many of us are wondering, well, how do we live this new way? And what
  is the way forward?
\item
  {[}music{]}\\
  Margaret Atwood has written about that in so many ways and so many
  forms for decades now. I'm going to give her a call.
\item
  {[}beeping{]}
\item
  {[}chime{]}
\item
  margaret atwood\\
  Hi, Cheryl.
\item
  cheryl strayed\\
  Hi, Margaret. I'm so glad to talk to you.
\item
  margaret atwood\\
  Lovely to talk to you.
\item
  cheryl strayed\\
  So where are you, Margaret? And who are you with?
\item
  margaret atwood\\
  I'm right in my study. My sister is here. She has brought her sewing
  machine. I've dredged out my ancient sewing machine. And we're going
  to make facemasks.
\item
  cheryl strayed\\
  Oh, that's so wonderful.
\item
  margaret atwood\\
  And we're going to go get some blue shop towels, because apparently
  they make good linings.
\item
  cheryl strayed\\
  Really?
\item
  margaret atwood\\
  Yeah.
\item
  cheryl strayed\\
  Now, is this sewing machine --- tell me, is this something you've had
  for a long time? And how did you learn to sew?
\item
  margaret atwood\\
  I've had it for a long time. I sewed all my clothes in high school, as
  people who were that way inclined did. And I also sewed clothes for my
  baby sister, who is here with me right now. She's 12 years younger.
  And I also, of course, sewed for my daughter. And I sewed for myself
  for years. And then I just kind of didn't for a while. So I'm going to
  have to learn how to re-thread this thing and oil it and all of those
  things. But I think I can probably get back into the groove.
\item
  cheryl strayed\\
  Yeah. You know, it's so funny that you should say that. Because,
  honestly, just two days ago, my daughter Bobbi, who is named after my
  mother Bobbi, said, do we have a sewing machine? And I said, I do.
  It's in the basement. And it was your grandmother's. The grandmother
  she never met, my mother, was an amazing seamstress and sewed all of
  my clothes when I was growing up. And so we pulled it upstairs. And it
  was the same thing. I was like, OK, I used to know how to thread this
  thing.
\item
  margaret atwood\\
  Yeah. Where does the thread go?
\item
  cheryl strayed\\
  Right. But it's the same thing. And I think it's fascinating. We're
  kind of going back in time in an interesting way. You grew up in rural
  northern Quebec, right?
\item
  margaret atwood\\
  OK, well, it depends what part of growing up you're talking about. So
  one to five, out in the backwoods of Quebec --- except in the winters,
  it was city. It was Ottawa. And then by the time I was 12, I was
  spending a full year in school. And that would be in Toronto. And that
  was where I was doing all of the sewing. So we started in grade 8. And
  guess what we made for our mothers?
\item
  cheryl strayed\\
  Aprons.
\item
  margaret atwood\\
  You got it.
\item
  {[}laughter{]}\\
  They were so appreciative.
\item
  cheryl strayed\\
  I know. You know why I know that? The same thing. I grew up in
  northern Minnesota. Now, I'm a little younger than you. I'm 51. And
  you are ---
\item
  margaret atwood\\
  That isn't just a little younger, Cheryl.
\item
  {[}laughter{]}\\
  That is a lot younger.
\item
  cheryl strayed\\
  OK, how old are you?
\item
  margaret atwood\\
  I'm 80.
\item
  cheryl strayed\\
  But no, the same thing. So this is something you and I share in
  common, this growing up in the sort of wilds of the north country. I
  was northern Minnesota. But because of that, I think of myself as an
  honorary Canadian.
\item
  margaret atwood\\
  Yeah, it's very close.
\item
  cheryl strayed\\
  Very close, yeah. And so I'm curious. Is it true you didn't go to sort
  of formal school until you were about 12?
\item
  margaret atwood\\
  No, in the winters I went. Because my dad was a forest entomologist.
  That means bugs that eat trees. And they are dormant in the winter.
  They don't do anything. So we were always in the city at that time.
  But it was the war. And there was no school, Cheryl. There wasn't any
  pre-school. There wasn't any daycare. There wasn't any schooly school
  school until grade one. So I didn't go to school until I was six. What
  a shock. It was horrible.
\item
  cheryl strayed\\
  Was it? Was it?
\item
  margaret atwood\\
  Well, sit in a desk in a row. It was also the military phase of
  school. So you did a lot of marching here and there. And the girls
  door, the boys door --- you lined up, and then you marched in. And it
  was some kind of eternal curse if you went in the wrong door.
\item
  cheryl strayed\\
  And I'm going to guess that this approach was always contrary to your
  spirit.
\item
  margaret atwood\\
  Well, who knows?
\item
  cheryl strayed\\
  I mean, you seem kind of ---
\item
  margaret atwood\\
  Who knows what your spirit is when you're seven?
\item
  cheryl strayed\\
  Oh, I don't know ---
\item
  margaret atwood\\
  You don't actually know.
\item
  cheryl strayed\\
  I think I disagree. I mean, you have a daughter, right? I mean, wasn't
  she who she is the minute she was born? Didn't you feel her
  personality?
\item
  margaret atwood\\
  Well, sure, but you have to wait and find out what form that is going
  to take.
\item
  cheryl strayed\\
  So when you think about when you became you --- I mean, what were you
  like as a girl?
\item
  margaret atwood\\
  Well, I was --- like all kids, you're in different ways at different
  ages. Are you not?
\item
  cheryl strayed\\
  Yeah.
\item
  margaret atwood\\
  So I would say I was a happy four-year-old. I was a glum
  nine-year-old. I was a shy 12-year-old. And I was the life of the
  party at 16.
\item
  cheryl strayed\\
  And then ever since.
\item
  margaret atwood\\
  But what else do you need to know?
\item
  {[}laughter{]}
\item
  cheryl strayed\\
  Well, Margaret, so when I emailed you and asked if I could call you, I
  told you. I said, this ``Sugar Calling'' thing, I'm really interested
  in talking to writers who are over the age of 60, because I'm seeking
  wisdom. And your reply was, well, I'd love to do it. I don't know if I
  have any wisdom to offer. And of course, I know you're wrong about
  that. But what I'm curious is, are you wiser now than you were at 40,
  or 60 or 20?
\item
  margaret atwood\\
  No, I'm wiser than I was at 17. And if that isn't true for other
  people, they're in trouble.
\item
  {[}laughter{]}
\item
  cheryl strayed\\
  I think we all are.
\item
  margaret atwood\\
  Yeah, so I'd say I'm wiser than I was at 17. And I'm wiser than I was
  at 30. But then you start thinking of wise people you have known. So I
  think my parents were both pretty wise. My mother especially was
  exceptionally wise. My grandmother was very wise. And one of the wise
  things about them was that they instinctively followed the rule of the
  Inuit people of northern Canada, which is, when you're an elder ---
  and it's other people who decide whether you're an elder or not. It's
  not just being old. But when you're an elder, you're considered to be
  wise. But you don't offer any advice unless people ask you for it. And
  I think that's a pretty wise thing to do.
\item
  cheryl strayed\\
  I agree entirely.
\item
  margaret atwood\\
  Yeah, so when my daughter had a child, my friends said to me, just
  remember, you know nothing.
\item
  {[}laughter{]}
\item
  cheryl strayed\\
  I think that's part of wisdom, knowing when to stay silent and to
  simply let somebody make their choices, to sit back and witness.
\item
  margaret atwood\\
  Well, they'll ask you for advice if they want it. And if they don't
  want it, they're not going to listen to it anyway. So why waste your
  time?
\item
  cheryl strayed\\
  Right. Well, I certainly consider you an elder, Margaret. I mean, you
  have ---
\item
  margaret atwood\\
  Well, I'll tell you if I don't know the answer.
\item
  cheryl strayed\\
  Thank you. So I want to ---
\item
  margaret atwood\\
  Or I will say, Cheryl, what do you think? What do you think you should
  do about that?
\item
  cheryl strayed\\
  And I think that that's often the key. In my work as Dear Sugar, where
  I did give people advice when they sought it from me, I would very
  often say, the answer is actually within you. The question that you
  asked me tells me the truth. It tells me the thing that you want to do
  or the thing that you know is the right thing. I think a lot of it is
  very often we're afraid to trust our own instincts. And so when we
  seek advice, we're really just seeking reassurance.
\item
  margaret atwood\\
  Confirmation.
\item
  cheryl strayed\\
  Yeah, confirmation. Exactly. So I want to just back up. Here you are
  sewing masks with your baby sister. Are you guys being socially
  isolated in your place? Are you in Toronto? Or where exactly are you?
\item
  margaret atwood\\
  I'm in Toronto. I socially isolated when I got back from where I had
  been, which was on March the 10th. Of course, I made it out of New
  Zealand, Australia and Ireland in the nick of time.
\item
  cheryl strayed\\
  Wow.
\item
  margaret atwood\\
  And then, of course, I didn't see anybody or breathe on them for the
  required number of days. And of course, now we're social distancing.
  But because she lives in the woods and doesn't see other people, and
  because I've socially isolated for all that time, we're safe together.
\item
  cheryl strayed\\
  Right. Yeah. And is she staying with you until whenever ---
\item
  margaret atwood\\
  She comes in on the weekends and cheers me up.
\item
  cheryl strayed\\
  Yeah. But otherwise, you're alone during the week.
\item
  margaret atwood\\
  It doesn't feel like that, Cheryl, because everybody and their dog
  wants me to do something for them.
\item
  {[}laughter{]}\\
  And we had a squirrel problem just recently. So the squirrel guy came.
  The roofing guy came. They can come, because they're on the outside of
  your house. So you're not near them. They're doing their thing
  outside. So although, yes, I'm, quote, ``alone,'' I'm not alone.
\item
  cheryl strayed\\
  That's true. So tell me about your days.
\item
  margaret atwood\\
  My day.
\item
  cheryl strayed\\
  Well, your days in general, during this time of social isolation.
\item
  margaret atwood\\
  There isn't any in generally, because it's something new every day. So
  today, we went out looking for shop towels colored blue, blue shop
  towels, because they are apparently very good filters for masks. But
  the place we went to we saw from afar that it had a huge, long line-up
  of people waiting to get in. And they were not social distancing. So
  we turned around, and we said, we think we can get some of these
  things online. And we can get other things by ordering them and opting
  for curb pickup. So we're going to try doing that. But we scored
  something that's in quite short supply, Cheryl. And that would be the
  pots of pansies.
\item
  cheryl strayed\\
  Pots of pansies?
\item
  margaret atwood\\
  We actually scored a couple of pots of pansies. The ordinary garden
  shops seem to be closed. However, at our local food depot, we saw two
  pots of pansies. And we grabbed them right away.
\item
  cheryl strayed\\
  And where are you putting these pansies? Are they ---
\item
  margaret atwood\\
  We put them in pots in the back yard. And they look very decorative
  and cheerful.
\item
  cheryl strayed\\
  Beautiful. Wow. So it sounds like you're kind of, so far, doing quite
  well for yourself during this time.
\item
  margaret atwood\\
  You know, Cheryl, you know this yourself, that if you grow up in the
  backwoods, not seeing a lot of people all the time doesn't bother you
  a lot. I think it's hard for people who have been used to going to the
  office, running their business, seeing a lot of people, talking to a
  lot of people in person --- I think it's very hard for them. Because
  they're not used to this.
\item
  cheryl strayed\\
  So OK, so far things are going well. You're thriving in the face of
  this pandemic?
\item
  margaret atwood\\
  What is it that you think might go wrong, Cheryl? I think anybody who
  isn't ill, and who has a place to be and who knows how to get food is
  in a very good place. But there are other people who don't have that.
  They're having a hard time.
\item
  cheryl strayed\\
  And that's my next question. Are you afraid for the world? Are you
  afraid for yourself? What are some of the more --- I mean, what are
  you thinking as you're watching the news?
\item
  margaret atwood\\
  OK, so the older you get, the less afraid you are of those kinds of
  things. You're afraid for other people. But you already know your
  plot, the plot of your life. And the plot of your life is that there's
  a lot more in the past than there's going to be in the future. So time
  is limited anyway. So why waste your time worrying and being afraid?
  But if you're younger, of course, you don't know that plot. And you
  think that this may be my whole future just gone up in smoke.
  Everything that I thought was going to be in my life is in question.
  And particularly if, for instance, you've lost your job, you're
  unemployed, you don't know how to pay for things, you don't know how
  you're going to eat --- these are very worrisome things.

  And what if you have young children? So that can be very, very
  stressful and worrisome. And a lot of people are extremely anxious.
  But if you're asking me, personally --- so day before yesterday, I
  climbed out on the roof, because we had a squirrel problem. I wanted
  to deal with it. And poor Lucia was saying, don't do that! Don't do
  that! I said, what can go wrong? It's a flat roof.
\item
  cheryl strayed\\
  And who is Lucia?
\item
  margaret atwood\\
  She works with me. So she was wringing your hands. Let me do that. I
  said, you're too tall. You wouldn't be able to get out the window.
\item
  {[}laughter{]}
\item
  cheryl strayed\\
  Oh, Margaret. I'm sorry. I'm sorry, I'm with Lucia on this. I'm all
  for pushing the boundaries, but maybe you shouldn't go up on the roof.
  I don't know.
\item
  margaret atwood\\
  Oh, come on. It wasn't dangerous, OK. I don't do deliberately
  dangerous things. Trust me. That's part of my wisdom. I know when it's
  dangerous.
\item
  cheryl strayed\\
  That's, I guess, right. Well, unless the squirrels came after you.
  That could have happened too.
\item
  margaret atwood\\
  They would not come after me. I have an aura of squirrel protection
  around me like an invisible force field.
\item
  {[}laughter{]}
\item
  cheryl strayed\\
  Oh my gosh.
\item
  margaret atwood\\
  I was putting hot pepper down their hole so they wouldn't aggravate
  me.
\item
  cheryl strayed\\
  How does the hot pepper down their hole keep them from aggravating
  you?
\item
  margaret atwood\\
  Well, they're like people, Cheryl. They're mammals. And they have very
  sensitive noses. So the pepper would say to them, this is not where
  you want to be.
\item
  cheryl strayed\\
  Oh, I see.
\item
  margaret atwood\\
  You do not want to be in Margaret's ceiling. Yeah, so the hot pepper
  is to the squirrel like the yellow tape is to the place where you're
  not supposed to go.
\item
  cheryl strayed\\
  Got it. Got it. Because really, I want to come socially isolate with
  you and your sister. Because it sounds like --- and Lucia. You're
  having that sort of barrel of fun there over in Toronto.
\item
  margaret atwood\\
  As much as possible.
\item
  {[}laughter{]}\\
  It's limited parameters. But within those parameters, we're doing the
  best we can to not worry and do things that are positive.
\item
  cheryl strayed\\
  Well, I want to ask you, though, about --- so you mentioned the people
  who are really struggling, who either have family members or loved
  ones who have the virus and are sick or dying, or people whose jobs
  have been lost and whose lives have been deeply stressed. And I
  couldn't help but think that you've written in your books about people
  who are really living in scary and restricted circumstances ---
\item
  margaret atwood\\
  That's true.
\item
  cheryl strayed\\
  --- and dire --- and I thought you might, as an elder, have something
  to say or to sort of tell me about how you crafted those characters,
  what you imagined when you imagined those kind of difficult
  circumstances? And any kind of wisdom you might have to offer to
  people who are having a really difficult time right now.
\item
  margaret atwood\\
  Yeah, well, it would depend what kind of difficult time they're
  having. So a material, physical difficult time, or an emotional
  difficult time, or both. So I was born in 1939, which meant that my
  early childhood was spent during World War II. So when I think of hard
  times I think of people in Europe during those times. Those were very
  hard times. And I've also always been interested in plagues. And I
  think that's partly from hearing about my mom and dad who went through
  the 1919 influenza epidemic, which was very lethal and killed a huge
  number of people around the world, like really a lot. So the whole
  family --- five kids, two grown ups --- went through that. And they
  all survived. So that's rather a hopeful story to have heard. So I
  think for young people who have never been through any of those things
  or lived at a time when they were happening, this seems just
  frightful. Seems like the absolute worst thing. They've never known
  anything like it. It's horrible. But if you've witnessed, heard about,
  and known people who have been through these other things, you think,
  OK, we're going to make it through this. It's going to be bad. And
  things are not going to be the same on the other side. But you know
  they aren't. They change anyway.

  So I've got a little bit of perspective, which is a man called Barry
  Lord, who wrote a book called ``Art \& Energy,'' in which he connects
  the kinds of culture you have with the kind of energy that is
  supporting it. He goes back to the days of fire. He goes through coal,
  which produced a culture of production. So Marx and trade unions and
  everything came out of that. And then oil comes along. And it's very
  cheap, and it doesn't take that many people to produce it. And you get
  a culture of consumption. Lots of cheap stuff. But we're now
  transitioning into renewable energy. And that will produce and is
  producing right now a culture of stewardship. And this time is going
  to give us a bit of a reset button. How are we going to do things
  differently on the other side?
\item
  cheryl strayed\\
  Yeah, I think you're right that we will. And it is interesting. This
  pandemic, I think, is really making people live differently right now,
  today. And I do feel like on the level of really contemplating climate
  change, so much of that conversation has been around all the things we
  can't do. Oh, we're all so tied to our automobiles, and our travel,
  and our jobs and our this and that and the other thing. We can't
  possibly reduce. And yet, here we are. Everything, thanks to COVID-19,
  came to a screeching halt.
\item
  margaret atwood\\
  Yeah. I think we're going to be saying to ourselves, do I really need
  that? Do I really need to be doing that? So I think it is going to be
  --- I've heard a number of people saying, I just wasn't really using
  by my home as a home. I was just using it as a place to sleep. And now
  that I've had to be in it, I'm creating a whole new relationship with
  it. I think you look at stuff you have. And you think, do I really
  need that stuff? Could somebody else be making use of that stuff,
  which for me is sitting in a drawer? So I think a lot of people are
  thinking that way. And they're sorting through their stuff and
  minimizing it. Are you?
\item
  cheryl strayed\\
  I am. Yeah. I've really thought about this in a big way. I was this
  morning and getting dressed and thinking, I'm wearing the same thing
  I've worn for five days.
\item
  margaret atwood\\
  That's a little unhealthy, Cheryl.
\item
  {[}laughter{]}
\item
  cheryl strayed\\
  I know. Well, no. It's been washed in between a time or two.
\item
  margaret atwood\\
  Oh, yes.
\item
  cheryl strayed\\
  I mean, I don't need all the clothes that I think I needed. Now, of
  course, you and I both do public events. And so you need a certain
  kind of wardrobe. But it's been pretty liberating to just adapt a
  uniform, essentially, my stay at home pandemic uniform that, really,
  I'm very happy wearing.
\item
  margaret atwood\\
  And what is that like, Cheryl? Can you describe that for us?
\item
  cheryl strayed\\
  They're comfortable they're basic.
\item
  margaret atwood\\
  What color are they, Cheryl?
\item
  cheryl strayed\\
  They're black. They're black and gray. Why are you asking me these
  questions?
\item
  margaret atwood\\
  Well, because it's a podcast. And people are listening. And they want
  to know what this outfit is like.
\item
  cheryl strayed\\
  OK, so I'm wearing this kind of gray sweater right now over the black
  top I've been wearing all week and the black pants I've been wearing
  all week. What are you wearing right now, now that we're on the
  subject?
\item
  margaret atwood\\
  Red. I'm wearing red, Cheryl. Because I'm old. It picks up the skin
  tones.
\item
  cheryl strayed\\
  You know, I remember you --- I think you said that to me the first
  time we met. You looked beautiful. And you said, yes, it picks up my
  skin tones, this color.
\item
  margaret atwood\\
  Think pink, pack black. You must have heard that.
\item
  cheryl strayed\\
  See, I haven't. And this is why I called you, my dear elder, to give
  me all the best advice. So I'm curious, who do you turn to? It sounds
  like you're not feeling terribly alone and sad and struggling right
  now. But when you do, who do you to turn to or what do you turn to for
  wisdom?

  Or consolation, or enlightenment or comfort.
\item
  margaret atwood\\
  Or just to have a chat. Well, it depends what the subject is.

  I'm not --- let's see, how can I put this.

  I like to talk things over with people whose opinions I value. So I
  have a number of people in my life that I can do that with. Somewhat
  fewer in number than they used to be, Cheryl. Because in my age group,
  people are dropping off the tree.

  So nonetheless, there are younger people who aren't stupid, believe it
  or not.
\item
  cheryl strayed\\
  How does it feel to lose some of your best friends and the people who
  you used to turn to in hard times? How does that feel, to watch people
  around you dying who you love?
\item
  margaret atwood\\
  When people die. OK, people die. I hate to break this to you. But
  that's one of the things that they do. And that has been going on
  pretty much my whole life. Not my first rodeo of people dying. And
  it's always very hard if you value that person. They're not going to
  be in your life anymore in the way that they used to be. They'll still
  be in your life. It's just that there probably won't be any new
  conversations. If there are some new conversations and you're awake
  while you're having them, you're not alone on the planet either. It's
  fairly normal for people to turn up who aren't alive in the usual
  sense anymore.
\item
  cheryl strayed\\
  You feel like the dead talk to you?
\item
  margaret atwood\\
  Well, I think if you're a writer, they've been talking to you all your
  life through their work.
\item
  cheryl strayed\\
  Certainly. So you have a poem --- speaking of writers talking to you
  through their work. You have a poem that I think you wanted to share
  with me. Would you read it to me?
\item
  margaret atwood\\
  Yeah. The question is, what did I do with it? Here it is. Here it is.
  I found it. OK, so this is by a poet called Jay Macpherson. J-A-Y
  M-A-C-P-H-E-R-S-O-N. She was actually a teacher of mine. And then she
  became a friend of mine. She'd had a very peculiar life. She was one
  of those English children that was evacuated during World War II and
  ended up in Newfoundland. And was also a very instinctive poet. So
  this poem is called ``The Beauty of Job's Daughters.'' And it's in a
  book of hers called Poems Twice Told. And what it is about is the
  force of the imagination --- something that we can all think about in
  these times. So the story of Job is that God and the devil decided to
  have a wager about Job. And God allowed Satan to torture Job in
  various different ways, including killing all of his children. So at
  the end of the story, God admits that this hadn't been fair and gives
  Job his wealth back and also gives him other children. So I always
  thought that it was a pretty fishy story. So I kill all your children.
  And these new children are supposed to make up for that. I don't think
  so.
\item
  cheryl strayed\\
  Doesn't work that way.
\item
  margaret atwood\\
  I don't think it works that way. But that's how it works in the story.
  And so this poem is called ``The Beauty of Job's Daughters.'' The old,
  the mad, the blind have fairest daughters. Take Job. The beasts the
  accuser sends at evening shoulder his house and shake it. He's not
  there, attained in age to inwardness of daughters in all the land no
  women found so fair. Angels and sons of God are nearest neighbors. And
  even the accuser may repair to walk with Job in pleasures of his
  daughters. Wide, shining rooms, more warmly lit at evening, gardens
  beyond whose secrets scent the air. Not wiles of men, nor envy of the
  neighbors, riches of earth, nor what heaven holds more rare can take
  from Job the beauty of his daughters, the gardens and the rock, music
  at evening and cup so full that all who come must share. Perhaps we
  passed them? It was late, or evening. And surely, those were desert
  stumps, not daughters. In fact, we doubt that they were ever there.
  The old, the mad, the blind have fairest daughters. In all the land,
  no women found so fair.
\item
  cheryl strayed\\
  You read so beautifully, Margaret.
\item
  margaret atwood\\
  Guess who I learned to read that from?
\item
  cheryl strayed\\
  Who?
\item
  margaret atwood\\
  The person who wrote it.
\item
  cheryl strayed\\
  Really? So Jay was your professor?
\item
  margaret atwood\\
  She was my professor at Victoria College in the University of Toronto
  in 1960, approx.
\item
  cheryl strayed\\
  And did she actually teach you how to read poetry out loud?
\item
  margaret atwood\\
  I heard her read. And one of the things that she said was, some people
  like to read poetry the way actors do. And they call that helping the
  old boy out. But if you read poetry too dramatically, it gets in the
  way of people hearing the words.
\item
  cheryl strayed\\
  She was right. So why did you choose this particular piece to read?
\item
  margaret atwood\\
  It was short.
\item
  {[}laughter{]}\\
  It's effective. If I had said I'm going to read you all Beowulf, that
  takes eight hours. You can get all of Beowulf, by the way, read by an
  Icelandic poet. So he gives it the old what for. And he reads it in
  the original. And what you're hearing is something that's 1,000 years
  old. You're hearing that voice from the deep past. And you won't be
  able to understand it, of course. But you can read along with Seamus
  Heaney's translation for the meaning. And that takes eight hours, the
  whole of the poem.
\item
  cheryl strayed\\
  That's always astounded me, that we can read work that is so, so
  ancient and so old, and it still has relevance and meaning to us
  today.
\item
  margaret atwood\\
  Yeah. The difference is that a lot of the languages that we read, we
  don't know how they were pronounced. But because Icelandic didn't
  change for 1,000 years, you're hearing it. That's how they spoke it.
  So it's not Grendel. It's Gr-r-r-endel!
\item
  cheryl strayed\\
  Can you do that again?
\item
  margaret atwood\\
  Gr-r-r-endel!
\item
  cheryl strayed\\
  That's impressive.
\item
  margaret atwood\\
  Isn't it? Yeah.
\item
  cheryl strayed\\
  How did you get that voice? How can you do that?
\item
  margaret atwood\\
  I got that voice from Svanur Thorkelsson because I heard him read it.
\item
  cheryl strayed\\
  Wow.

  You get it done. From making the facemasks, to scaring the squirrels
  away, to getting ---
\item
  margaret atwood\\
  Well, I get that from my mother. She used to say, roll up your
  sleeves, girls.
\item
  cheryl strayed\\
  Right. And now, look, you and your sister are doing just that. That's
  kind of a beautiful circle, isn't it?
\item
  margaret atwood\\
  That's what we're doing. Her sewing is going to be better than mine.
  I'm just telling you that right now. I'm going to have to get back in
  stride with it.
\item
  cheryl strayed\\
  And that's what you're going to do when we get off the phone. You're
  going to get to your sewed work, is that right?
\item
  margaret atwood\\
  That's what we're going to do. So she brought a lot of fabrics in.
  They're very pretty, actually. And we've washed them. We will iron
  them. And she brought her cutting board. You've got this cutting board
  that measures these things.
\item
  cheryl strayed\\
  Yeah, I know that.
\item
  margaret atwood\\
  You have a little wheel that you can just cut along the line. And so
  that's what we're going to do.
\item
  cheryl strayed\\
  Beautiful. Well, Margaret, it's been an absolute pleasure to speak to
  you today. And thank you so much for taking the time to talk to me. I
  think you're exquisitely wonderful, and brilliant and amazing. So
  happy sewing, and my best to your sister and all the squirrels, as
  well.
\item
  margaret atwood\\
  As long as they stay in their place, right?
\item
  cheryl strayed\\
  All right, bye, Margaret.
\item
  margaret atwood\\
  Bye, Cheryl.
\item
  cheryl strayed\\
  Bye bye.
\item
  {[}music{]}\\
  I'm Cheryl Strayed. And this is ``Sugar Calling.'' Next week, the
  great writer, traveler and speaker, Pico Iyer.

  Thanks for listening.
\end{itemize}

\href{https://www.nytimes3xbfgragh.onion/column/sugar-calling}{\includegraphics{https://static01.graylady3jvrrxbe.onion/images/2020/04/29/podcasts/sugar-calling-album-art/sugar-calling-album-art-square320.jpg}Sugar
Calling}Subscribe:

\begin{itemize}
\tightlist
\item
  \href{https://itunes.apple.com/us/podcast/id1505881384}{Apple
  Podcasts}
\item
  \href{https://podcasts.google.com/?feed=aHR0cHM6Ly9yc3MuYXJ0MTkuY29tL3N1Z2FyLWNhbGxpbmc\&ved=0CAUQrrcFahcKEwjA8Kyn09voAhUAAAAAHQAAAAAQBQ}{Google
  Podcasts}
\end{itemize}

\hypertarget{roll-up-your-sleeves-girls-1}{%
\section{`Roll Up Your Sleeves,
Girls'}\label{roll-up-your-sleeves-girls-1}}

\hypertarget{cheryl-strayed-calls-the-writer-margaret-atwood-to-talk-about-hard-times-pansies-and-eight-decades-of-getting-it-done-1}{%
\subsection{Cheryl Strayed calls the writer Margaret Atwood to talk
about hard times, pansies and eight decades of getting it
done.}\label{cheryl-strayed-calls-the-writer-margaret-atwood-to-talk-about-hard-times-pansies-and-eight-decades-of-getting-it-done-1}}

Hosted by Cheryl Strayed, produced by Kelly Prime and edited by Sara
Sarasohn

Transcript

transcript

Back to Sugar Calling

bars

0:00/34:32

-0:00

transcript

\hypertarget{roll-up-your-sleeves-girls-2}{%
\subsection{`Roll Up Your Sleeves,
Girls'}\label{roll-up-your-sleeves-girls-2}}

\hypertarget{hosted-by-cheryl-strayed-produced-by-kelly-prime-and-edited-by-sara-sarasohn-1}{%
\subsubsection{Hosted by Cheryl Strayed, produced by Kelly Prime and
edited by Sara
Sarasohn}\label{hosted-by-cheryl-strayed-produced-by-kelly-prime-and-edited-by-sara-sarasohn-1}}

\hypertarget{cheryl-strayed-calls-the-writer-margaret-atwood-to-talk-about-hard-times-pansies-and-eight-decades-of-getting-it-done-2}{%
\paragraph{Cheryl Strayed calls the writer Margaret Atwood to talk about
hard times, pansies and eight decades of getting it
done.}\label{cheryl-strayed-calls-the-writer-margaret-atwood-to-talk-about-hard-times-pansies-and-eight-decades-of-getting-it-done-2}}

Wednesday, April 8th, 2020

\begin{itemize}
\item
  cheryl strayed\\
  Today, I'm going to call Margaret Atwood, the legendary, extraordinary
  Margaret Atwood. Author of so many novels and books of poetry and
  children's books. I was still in my teens when The Handmaid's Tale
  came out, and I read it and was astonished by it. And I've loved her
  ever since. She's written so powerfully about people in incredibly
  difficult circumstances, people who can't even use their own names,
  people whose movement is constricted and whose thought is constricted.
  And I think that there's something, in essence, the way she writes
  about how we persist in the face of extreme difficulty that I felt
  like she might be the person to talk to at this very moment when so
  many of us are wondering, well, how do we live this new way? And what
  is the way forward?
\item
  {[}music{]}\\
  Margaret Atwood has written about that in so many ways and so many
  forms for decades now. I'm going to give her a call.
\item
  {[}beeping{]}
\item
  {[}chime{]}
\item
  margaret atwood\\
  Hi, Cheryl.
\item
  cheryl strayed\\
  Hi, Margaret. I'm so glad to talk to you.
\item
  margaret atwood\\
  Lovely to talk to you.
\item
  cheryl strayed\\
  So where are you, Margaret? And who are you with?
\item
  margaret atwood\\
  I'm right in my study. My sister is here. She has brought her sewing
  machine. I've dredged out my ancient sewing machine. And we're going
  to make facemasks.
\item
  cheryl strayed\\
  Oh, that's so wonderful.
\item
  margaret atwood\\
  And we're going to go get some blue shop towels, because apparently
  they make good linings.
\item
  cheryl strayed\\
  Really?
\item
  margaret atwood\\
  Yeah.
\item
  cheryl strayed\\
  Now, is this sewing machine --- tell me, is this something you've had
  for a long time? And how did you learn to sew?
\item
  margaret atwood\\
  I've had it for a long time. I sewed all my clothes in high school, as
  people who were that way inclined did. And I also sewed clothes for my
  baby sister, who is here with me right now. She's 12 years younger.
  And I also, of course, sewed for my daughter. And I sewed for myself
  for years. And then I just kind of didn't for a while. So I'm going to
  have to learn how to re-thread this thing and oil it and all of those
  things. But I think I can probably get back into the groove.
\item
  cheryl strayed\\
  Yeah. You know, it's so funny that you should say that. Because,
  honestly, just two days ago, my daughter Bobbi, who is named after my
  mother Bobbi, said, do we have a sewing machine? And I said, I do.
  It's in the basement. And it was your grandmother's. The grandmother
  she never met, my mother, was an amazing seamstress and sewed all of
  my clothes when I was growing up. And so we pulled it upstairs. And it
  was the same thing. I was like, OK, I used to know how to thread this
  thing.
\item
  margaret atwood\\
  Yeah. Where does the thread go?
\item
  cheryl strayed\\
  Right. But it's the same thing. And I think it's fascinating. We're
  kind of going back in time in an interesting way. You grew up in rural
  northern Quebec, right?
\item
  margaret atwood\\
  OK, well, it depends what part of growing up you're talking about. So
  one to five, out in the backwoods of Quebec --- except in the winters,
  it was city. It was Ottawa. And then by the time I was 12, I was
  spending a full year in school. And that would be in Toronto. And that
  was where I was doing all of the sewing. So we started in grade 8. And
  guess what we made for our mothers?
\item
  cheryl strayed\\
  Aprons.
\item
  margaret atwood\\
  You got it.
\item
  {[}laughter{]}\\
  They were so appreciative.
\item
  cheryl strayed\\
  I know. You know why I know that? The same thing. I grew up in
  northern Minnesota. Now, I'm a little younger than you. I'm 51. And
  you are ---
\item
  margaret atwood\\
  That isn't just a little younger, Cheryl.
\item
  {[}laughter{]}\\
  That is a lot younger.
\item
  cheryl strayed\\
  OK, how old are you?
\item
  margaret atwood\\
  I'm 80.
\item
  cheryl strayed\\
  But no, the same thing. So this is something you and I share in
  common, this growing up in the sort of wilds of the north country. I
  was northern Minnesota. But because of that, I think of myself as an
  honorary Canadian.
\item
  margaret atwood\\
  Yeah, it's very close.
\item
  cheryl strayed\\
  Very close, yeah. And so I'm curious. Is it true you didn't go to sort
  of formal school until you were about 12?
\item
  margaret atwood\\
  No, in the winters I went. Because my dad was a forest entomologist.
  That means bugs that eat trees. And they are dormant in the winter.
  They don't do anything. So we were always in the city at that time.
  But it was the war. And there was no school, Cheryl. There wasn't any
  pre-school. There wasn't any daycare. There wasn't any schooly school
  school until grade one. So I didn't go to school until I was six. What
  a shock. It was horrible.
\item
  cheryl strayed\\
  Was it? Was it?
\item
  margaret atwood\\
  Well, sit in a desk in a row. It was also the military phase of
  school. So you did a lot of marching here and there. And the girls
  door, the boys door --- you lined up, and then you marched in. And it
  was some kind of eternal curse if you went in the wrong door.
\item
  cheryl strayed\\
  And I'm going to guess that this approach was always contrary to your
  spirit.
\item
  margaret atwood\\
  Well, who knows?
\item
  cheryl strayed\\
  I mean, you seem kind of ---
\item
  margaret atwood\\
  Who knows what your spirit is when you're seven?
\item
  cheryl strayed\\
  Oh, I don't know ---
\item
  margaret atwood\\
  You don't actually know.
\item
  cheryl strayed\\
  I think I disagree. I mean, you have a daughter, right? I mean, wasn't
  she who she is the minute she was born? Didn't you feel her
  personality?
\item
  margaret atwood\\
  Well, sure, but you have to wait and find out what form that is going
  to take.
\item
  cheryl strayed\\
  So when you think about when you became you --- I mean, what were you
  like as a girl?
\item
  margaret atwood\\
  Well, I was --- like all kids, you're in different ways at different
  ages. Are you not?
\item
  cheryl strayed\\
  Yeah.
\item
  margaret atwood\\
  So I would say I was a happy four-year-old. I was a glum
  nine-year-old. I was a shy 12-year-old. And I was the life of the
  party at 16.
\item
  cheryl strayed\\
  And then ever since.
\item
  margaret atwood\\
  But what else do you need to know?
\item
  {[}laughter{]}
\item
  cheryl strayed\\
  Well, Margaret, so when I emailed you and asked if I could call you, I
  told you. I said, this ``Sugar Calling'' thing, I'm really interested
  in talking to writers who are over the age of 60, because I'm seeking
  wisdom. And your reply was, well, I'd love to do it. I don't know if I
  have any wisdom to offer. And of course, I know you're wrong about
  that. But what I'm curious is, are you wiser now than you were at 40,
  or 60 or 20?
\item
  margaret atwood\\
  No, I'm wiser than I was at 17. And if that isn't true for other
  people, they're in trouble.
\item
  {[}laughter{]}
\item
  cheryl strayed\\
  I think we all are.
\item
  margaret atwood\\
  Yeah, so I'd say I'm wiser than I was at 17. And I'm wiser than I was
  at 30. But then you start thinking of wise people you have known. So I
  think my parents were both pretty wise. My mother especially was
  exceptionally wise. My grandmother was very wise. And one of the wise
  things about them was that they instinctively followed the rule of the
  Inuit people of northern Canada, which is, when you're an elder ---
  and it's other people who decide whether you're an elder or not. It's
  not just being old. But when you're an elder, you're considered to be
  wise. But you don't offer any advice unless people ask you for it. And
  I think that's a pretty wise thing to do.
\item
  cheryl strayed\\
  I agree entirely.
\item
  margaret atwood\\
  Yeah, so when my daughter had a child, my friends said to me, just
  remember, you know nothing.
\item
  {[}laughter{]}
\item
  cheryl strayed\\
  I think that's part of wisdom, knowing when to stay silent and to
  simply let somebody make their choices, to sit back and witness.
\item
  margaret atwood\\
  Well, they'll ask you for advice if they want it. And if they don't
  want it, they're not going to listen to it anyway. So why waste your
  time?
\item
  cheryl strayed\\
  Right. Well, I certainly consider you an elder, Margaret. I mean, you
  have ---
\item
  margaret atwood\\
  Well, I'll tell you if I don't know the answer.
\item
  cheryl strayed\\
  Thank you. So I want to ---
\item
  margaret atwood\\
  Or I will say, Cheryl, what do you think? What do you think you should
  do about that?
\item
  cheryl strayed\\
  And I think that that's often the key. In my work as Dear Sugar, where
  I did give people advice when they sought it from me, I would very
  often say, the answer is actually within you. The question that you
  asked me tells me the truth. It tells me the thing that you want to do
  or the thing that you know is the right thing. I think a lot of it is
  very often we're afraid to trust our own instincts. And so when we
  seek advice, we're really just seeking reassurance.
\item
  margaret atwood\\
  Confirmation.
\item
  cheryl strayed\\
  Yeah, confirmation. Exactly. So I want to just back up. Here you are
  sewing masks with your baby sister. Are you guys being socially
  isolated in your place? Are you in Toronto? Or where exactly are you?
\item
  margaret atwood\\
  I'm in Toronto. I socially isolated when I got back from where I had
  been, which was on March the 10th. Of course, I made it out of New
  Zealand, Australia and Ireland in the nick of time.
\item
  cheryl strayed\\
  Wow.
\item
  margaret atwood\\
  And then, of course, I didn't see anybody or breathe on them for the
  required number of days. And of course, now we're social distancing.
  But because she lives in the woods and doesn't see other people, and
  because I've socially isolated for all that time, we're safe together.
\item
  cheryl strayed\\
  Right. Yeah. And is she staying with you until whenever ---
\item
  margaret atwood\\
  She comes in on the weekends and cheers me up.
\item
  cheryl strayed\\
  Yeah. But otherwise, you're alone during the week.
\item
  margaret atwood\\
  It doesn't feel like that, Cheryl, because everybody and their dog
  wants me to do something for them.
\item
  {[}laughter{]}\\
  And we had a squirrel problem just recently. So the squirrel guy came.
  The roofing guy came. They can come, because they're on the outside of
  your house. So you're not near them. They're doing their thing
  outside. So although, yes, I'm, quote, ``alone,'' I'm not alone.
\item
  cheryl strayed\\
  That's true. So tell me about your days.
\item
  margaret atwood\\
  My day.
\item
  cheryl strayed\\
  Well, your days in general, during this time of social isolation.
\item
  margaret atwood\\
  There isn't any in generally, because it's something new every day. So
  today, we went out looking for shop towels colored blue, blue shop
  towels, because they are apparently very good filters for masks. But
  the place we went to we saw from afar that it had a huge, long line-up
  of people waiting to get in. And they were not social distancing. So
  we turned around, and we said, we think we can get some of these
  things online. And we can get other things by ordering them and opting
  for curb pickup. So we're going to try doing that. But we scored
  something that's in quite short supply, Cheryl. And that would be the
  pots of pansies.
\item
  cheryl strayed\\
  Pots of pansies?
\item
  margaret atwood\\
  We actually scored a couple of pots of pansies. The ordinary garden
  shops seem to be closed. However, at our local food depot, we saw two
  pots of pansies. And we grabbed them right away.
\item
  cheryl strayed\\
  And where are you putting these pansies? Are they ---
\item
  margaret atwood\\
  We put them in pots in the back yard. And they look very decorative
  and cheerful.
\item
  cheryl strayed\\
  Beautiful. Wow. So it sounds like you're kind of, so far, doing quite
  well for yourself during this time.
\item
  margaret atwood\\
  You know, Cheryl, you know this yourself, that if you grow up in the
  backwoods, not seeing a lot of people all the time doesn't bother you
  a lot. I think it's hard for people who have been used to going to the
  office, running their business, seeing a lot of people, talking to a
  lot of people in person --- I think it's very hard for them. Because
  they're not used to this.
\item
  cheryl strayed\\
  So OK, so far things are going well. You're thriving in the face of
  this pandemic?
\item
  margaret atwood\\
  What is it that you think might go wrong, Cheryl? I think anybody who
  isn't ill, and who has a place to be and who knows how to get food is
  in a very good place. But there are other people who don't have that.
  They're having a hard time.
\item
  cheryl strayed\\
  And that's my next question. Are you afraid for the world? Are you
  afraid for yourself? What are some of the more --- I mean, what are
  you thinking as you're watching the news?
\item
  margaret atwood\\
  OK, so the older you get, the less afraid you are of those kinds of
  things. You're afraid for other people. But you already know your
  plot, the plot of your life. And the plot of your life is that there's
  a lot more in the past than there's going to be in the future. So time
  is limited anyway. So why waste your time worrying and being afraid?
  But if you're younger, of course, you don't know that plot. And you
  think that this may be my whole future just gone up in smoke.
  Everything that I thought was going to be in my life is in question.
  And particularly if, for instance, you've lost your job, you're
  unemployed, you don't know how to pay for things, you don't know how
  you're going to eat --- these are very worrisome things.

  And what if you have young children? So that can be very, very
  stressful and worrisome. And a lot of people are extremely anxious.
  But if you're asking me, personally --- so day before yesterday, I
  climbed out on the roof, because we had a squirrel problem. I wanted
  to deal with it. And poor Lucia was saying, don't do that! Don't do
  that! I said, what can go wrong? It's a flat roof.
\item
  cheryl strayed\\
  And who is Lucia?
\item
  margaret atwood\\
  She works with me. So she was wringing your hands. Let me do that. I
  said, you're too tall. You wouldn't be able to get out the window.
\item
  {[}laughter{]}
\item
  cheryl strayed\\
  Oh, Margaret. I'm sorry. I'm sorry, I'm with Lucia on this. I'm all
  for pushing the boundaries, but maybe you shouldn't go up on the roof.
  I don't know.
\item
  margaret atwood\\
  Oh, come on. It wasn't dangerous, OK. I don't do deliberately
  dangerous things. Trust me. That's part of my wisdom. I know when it's
  dangerous.
\item
  cheryl strayed\\
  That's, I guess, right. Well, unless the squirrels came after you.
  That could have happened too.
\item
  margaret atwood\\
  They would not come after me. I have an aura of squirrel protection
  around me like an invisible force field.
\item
  {[}laughter{]}
\item
  cheryl strayed\\
  Oh my gosh.
\item
  margaret atwood\\
  I was putting hot pepper down their hole so they wouldn't aggravate
  me.
\item
  cheryl strayed\\
  How does the hot pepper down their hole keep them from aggravating
  you?
\item
  margaret atwood\\
  Well, they're like people, Cheryl. They're mammals. And they have very
  sensitive noses. So the pepper would say to them, this is not where
  you want to be.
\item
  cheryl strayed\\
  Oh, I see.
\item
  margaret atwood\\
  You do not want to be in Margaret's ceiling. Yeah, so the hot pepper
  is to the squirrel like the yellow tape is to the place where you're
  not supposed to go.
\item
  cheryl strayed\\
  Got it. Got it. Because really, I want to come socially isolate with
  you and your sister. Because it sounds like --- and Lucia. You're
  having that sort of barrel of fun there over in Toronto.
\item
  margaret atwood\\
  As much as possible.
\item
  {[}laughter{]}\\
  It's limited parameters. But within those parameters, we're doing the
  best we can to not worry and do things that are positive.
\item
  cheryl strayed\\
  Well, I want to ask you, though, about --- so you mentioned the people
  who are really struggling, who either have family members or loved
  ones who have the virus and are sick or dying, or people whose jobs
  have been lost and whose lives have been deeply stressed. And I
  couldn't help but think that you've written in your books about people
  who are really living in scary and restricted circumstances ---
\item
  margaret atwood\\
  That's true.
\item
  cheryl strayed\\
  --- and dire --- and I thought you might, as an elder, have something
  to say or to sort of tell me about how you crafted those characters,
  what you imagined when you imagined those kind of difficult
  circumstances? And any kind of wisdom you might have to offer to
  people who are having a really difficult time right now.
\item
  margaret atwood\\
  Yeah, well, it would depend what kind of difficult time they're
  having. So a material, physical difficult time, or an emotional
  difficult time, or both. So I was born in 1939, which meant that my
  early childhood was spent during World War II. So when I think of hard
  times I think of people in Europe during those times. Those were very
  hard times. And I've also always been interested in plagues. And I
  think that's partly from hearing about my mom and dad who went through
  the 1919 influenza epidemic, which was very lethal and killed a huge
  number of people around the world, like really a lot. So the whole
  family --- five kids, two grown ups --- went through that. And they
  all survived. So that's rather a hopeful story to have heard. So I
  think for young people who have never been through any of those things
  or lived at a time when they were happening, this seems just
  frightful. Seems like the absolute worst thing. They've never known
  anything like it. It's horrible. But if you've witnessed, heard about,
  and known people who have been through these other things, you think,
  OK, we're going to make it through this. It's going to be bad. And
  things are not going to be the same on the other side. But you know
  they aren't. They change anyway.

  So I've got a little bit of perspective, which is a man called Barry
  Lord, who wrote a book called ``Art \& Energy,'' in which he connects
  the kinds of culture you have with the kind of energy that is
  supporting it. He goes back to the days of fire. He goes through coal,
  which produced a culture of production. So Marx and trade unions and
  everything came out of that. And then oil comes along. And it's very
  cheap, and it doesn't take that many people to produce it. And you get
  a culture of consumption. Lots of cheap stuff. But we're now
  transitioning into renewable energy. And that will produce and is
  producing right now a culture of stewardship. And this time is going
  to give us a bit of a reset button. How are we going to do things
  differently on the other side?
\item
  cheryl strayed\\
  Yeah, I think you're right that we will. And it is interesting. This
  pandemic, I think, is really making people live differently right now,
  today. And I do feel like on the level of really contemplating climate
  change, so much of that conversation has been around all the things we
  can't do. Oh, we're all so tied to our automobiles, and our travel,
  and our jobs and our this and that and the other thing. We can't
  possibly reduce. And yet, here we are. Everything, thanks to COVID-19,
  came to a screeching halt.
\item
  margaret atwood\\
  Yeah. I think we're going to be saying to ourselves, do I really need
  that? Do I really need to be doing that? So I think it is going to be
  --- I've heard a number of people saying, I just wasn't really using
  by my home as a home. I was just using it as a place to sleep. And now
  that I've had to be in it, I'm creating a whole new relationship with
  it. I think you look at stuff you have. And you think, do I really
  need that stuff? Could somebody else be making use of that stuff,
  which for me is sitting in a drawer? So I think a lot of people are
  thinking that way. And they're sorting through their stuff and
  minimizing it. Are you?
\item
  cheryl strayed\\
  I am. Yeah. I've really thought about this in a big way. I was this
  morning and getting dressed and thinking, I'm wearing the same thing
  I've worn for five days.
\item
  margaret atwood\\
  That's a little unhealthy, Cheryl.
\item
  {[}laughter{]}
\item
  cheryl strayed\\
  I know. Well, no. It's been washed in between a time or two.
\item
  margaret atwood\\
  Oh, yes.
\item
  cheryl strayed\\
  I mean, I don't need all the clothes that I think I needed. Now, of
  course, you and I both do public events. And so you need a certain
  kind of wardrobe. But it's been pretty liberating to just adapt a
  uniform, essentially, my stay at home pandemic uniform that, really,
  I'm very happy wearing.
\item
  margaret atwood\\
  And what is that like, Cheryl? Can you describe that for us?
\item
  cheryl strayed\\
  They're comfortable they're basic.
\item
  margaret atwood\\
  What color are they, Cheryl?
\item
  cheryl strayed\\
  They're black. They're black and gray. Why are you asking me these
  questions?
\item
  margaret atwood\\
  Well, because it's a podcast. And people are listening. And they want
  to know what this outfit is like.
\item
  cheryl strayed\\
  OK, so I'm wearing this kind of gray sweater right now over the black
  top I've been wearing all week and the black pants I've been wearing
  all week. What are you wearing right now, now that we're on the
  subject?
\item
  margaret atwood\\
  Red. I'm wearing red, Cheryl. Because I'm old. It picks up the skin
  tones.
\item
  cheryl strayed\\
  You know, I remember you --- I think you said that to me the first
  time we met. You looked beautiful. And you said, yes, it picks up my
  skin tones, this color.
\item
  margaret atwood\\
  Think pink, pack black. You must have heard that.
\item
  cheryl strayed\\
  See, I haven't. And this is why I called you, my dear elder, to give
  me all the best advice. So I'm curious, who do you turn to? It sounds
  like you're not feeling terribly alone and sad and struggling right
  now. But when you do, who do you to turn to or what do you turn to for
  wisdom?

  Or consolation, or enlightenment or comfort.
\item
  margaret atwood\\
  Or just to have a chat. Well, it depends what the subject is.

  I'm not --- let's see, how can I put this.

  I like to talk things over with people whose opinions I value. So I
  have a number of people in my life that I can do that with. Somewhat
  fewer in number than they used to be, Cheryl. Because in my age group,
  people are dropping off the tree.

  So nonetheless, there are younger people who aren't stupid, believe it
  or not.
\item
  cheryl strayed\\
  How does it feel to lose some of your best friends and the people who
  you used to turn to in hard times? How does that feel, to watch people
  around you dying who you love?
\item
  margaret atwood\\
  When people die. OK, people die. I hate to break this to you. But
  that's one of the things that they do. And that has been going on
  pretty much my whole life. Not my first rodeo of people dying. And
  it's always very hard if you value that person. They're not going to
  be in your life anymore in the way that they used to be. They'll still
  be in your life. It's just that there probably won't be any new
  conversations. If there are some new conversations and you're awake
  while you're having them, you're not alone on the planet either. It's
  fairly normal for people to turn up who aren't alive in the usual
  sense anymore.
\item
  cheryl strayed\\
  You feel like the dead talk to you?
\item
  margaret atwood\\
  Well, I think if you're a writer, they've been talking to you all your
  life through their work.
\item
  cheryl strayed\\
  Certainly. So you have a poem --- speaking of writers talking to you
  through their work. You have a poem that I think you wanted to share
  with me. Would you read it to me?
\item
  margaret atwood\\
  Yeah. The question is, what did I do with it? Here it is. Here it is.
  I found it. OK, so this is by a poet called Jay Macpherson. J-A-Y
  M-A-C-P-H-E-R-S-O-N. She was actually a teacher of mine. And then she
  became a friend of mine. She'd had a very peculiar life. She was one
  of those English children that was evacuated during World War II and
  ended up in Newfoundland. And was also a very instinctive poet. So
  this poem is called ``The Beauty of Job's Daughters.'' And it's in a
  book of hers called Poems Twice Told. And what it is about is the
  force of the imagination --- something that we can all think about in
  these times. So the story of Job is that God and the devil decided to
  have a wager about Job. And God allowed Satan to torture Job in
  various different ways, including killing all of his children. So at
  the end of the story, God admits that this hadn't been fair and gives
  Job his wealth back and also gives him other children. So I always
  thought that it was a pretty fishy story. So I kill all your children.
  And these new children are supposed to make up for that. I don't think
  so.
\item
  cheryl strayed\\
  Doesn't work that way.
\item
  margaret atwood\\
  I don't think it works that way. But that's how it works in the story.
  And so this poem is called ``The Beauty of Job's Daughters.'' The old,
  the mad, the blind have fairest daughters. Take Job. The beasts the
  accuser sends at evening shoulder his house and shake it. He's not
  there, attained in age to inwardness of daughters in all the land no
  women found so fair. Angels and sons of God are nearest neighbors. And
  even the accuser may repair to walk with Job in pleasures of his
  daughters. Wide, shining rooms, more warmly lit at evening, gardens
  beyond whose secrets scent the air. Not wiles of men, nor envy of the
  neighbors, riches of earth, nor what heaven holds more rare can take
  from Job the beauty of his daughters, the gardens and the rock, music
  at evening and cup so full that all who come must share. Perhaps we
  passed them? It was late, or evening. And surely, those were desert
  stumps, not daughters. In fact, we doubt that they were ever there.
  The old, the mad, the blind have fairest daughters. In all the land,
  no women found so fair.
\item
  cheryl strayed\\
  You read so beautifully, Margaret.
\item
  margaret atwood\\
  Guess who I learned to read that from?
\item
  cheryl strayed\\
  Who?
\item
  margaret atwood\\
  The person who wrote it.
\item
  cheryl strayed\\
  Really? So Jay was your professor?
\item
  margaret atwood\\
  She was my professor at Victoria College in the University of Toronto
  in 1960, approx.
\item
  cheryl strayed\\
  And did she actually teach you how to read poetry out loud?
\item
  margaret atwood\\
  I heard her read. And one of the things that she said was, some people
  like to read poetry the way actors do. And they call that helping the
  old boy out. But if you read poetry too dramatically, it gets in the
  way of people hearing the words.
\item
  cheryl strayed\\
  She was right. So why did you choose this particular piece to read?
\item
  margaret atwood\\
  It was short.
\item
  {[}laughter{]}\\
  It's effective. If I had said I'm going to read you all Beowulf, that
  takes eight hours. You can get all of Beowulf, by the way, read by an
  Icelandic poet. So he gives it the old what for. And he reads it in
  the original. And what you're hearing is something that's 1,000 years
  old. You're hearing that voice from the deep past. And you won't be
  able to understand it, of course. But you can read along with Seamus
  Heaney's translation for the meaning. And that takes eight hours, the
  whole of the poem.
\item
  cheryl strayed\\
  That's always astounded me, that we can read work that is so, so
  ancient and so old, and it still has relevance and meaning to us
  today.
\item
  margaret atwood\\
  Yeah. The difference is that a lot of the languages that we read, we
  don't know how they were pronounced. But because Icelandic didn't
  change for 1,000 years, you're hearing it. That's how they spoke it.
  So it's not Grendel. It's Gr-r-r-endel!
\item
  cheryl strayed\\
  Can you do that again?
\item
  margaret atwood\\
  Gr-r-r-endel!
\item
  cheryl strayed\\
  That's impressive.
\item
  margaret atwood\\
  Isn't it? Yeah.
\item
  cheryl strayed\\
  How did you get that voice? How can you do that?
\item
  margaret atwood\\
  I got that voice from Svanur Thorkelsson because I heard him read it.
\item
  cheryl strayed\\
  Wow.

  You get it done. From making the facemasks, to scaring the squirrels
  away, to getting ---
\item
  margaret atwood\\
  Well, I get that from my mother. She used to say, roll up your
  sleeves, girls.
\item
  cheryl strayed\\
  Right. And now, look, you and your sister are doing just that. That's
  kind of a beautiful circle, isn't it?
\item
  margaret atwood\\
  That's what we're doing. Her sewing is going to be better than mine.
  I'm just telling you that right now. I'm going to have to get back in
  stride with it.
\item
  cheryl strayed\\
  And that's what you're going to do when we get off the phone. You're
  going to get to your sewed work, is that right?
\item
  margaret atwood\\
  That's what we're going to do. So she brought a lot of fabrics in.
  They're very pretty, actually. And we've washed them. We will iron
  them. And she brought her cutting board. You've got this cutting board
  that measures these things.
\item
  cheryl strayed\\
  Yeah, I know that.
\item
  margaret atwood\\
  You have a little wheel that you can just cut along the line. And so
  that's what we're going to do.
\item
  cheryl strayed\\
  Beautiful. Well, Margaret, it's been an absolute pleasure to speak to
  you today. And thank you so much for taking the time to talk to me. I
  think you're exquisitely wonderful, and brilliant and amazing. So
  happy sewing, and my best to your sister and all the squirrels, as
  well.
\item
  margaret atwood\\
  As long as they stay in their place, right?
\item
  cheryl strayed\\
  All right, bye, Margaret.
\item
  margaret atwood\\
  Bye, Cheryl.
\item
  cheryl strayed\\
  Bye bye.
\item
  {[}music{]}\\
  I'm Cheryl Strayed. And this is ``Sugar Calling.'' Next week, the
  great writer, traveler and speaker, Pico Iyer.

  Thanks for listening.
\end{itemize}

Previous

More episodes ofSugar Calling

\href{https://www.nytimes3xbfgragh.onion/2020/05/20/podcasts/sugar-calling-joy-harjo-poetry-virus.html?action=click\&module=audio-series-bar\&region=header\&pgtype=Article}{\includegraphics{https://static01.graylady3jvrrxbe.onion/images/2020/05/22/podcasts/20sugar-hajo3/20sugar-hajo3-thumbLarge.jpg}}

May 20, 2020~~•~ 35:30`I Release You, Fear'

\href{https://www.nytimes3xbfgragh.onion/2020/05/13/podcasts/sugar-calling-billy-collins-poetry-virus.html?action=click\&module=audio-series-bar\&region=header\&pgtype=Article}{\includegraphics{https://static01.graylady3jvrrxbe.onion/images/2020/05/13/podcasts/13sugar-calling/13sugar-calling-thumbLarge.jpg}}

May 13, 2020`There's a Quiet All Over the World'

\href{https://www.nytimes3xbfgragh.onion/2020/05/06/podcasts/sugar-calling-alice-walker-quarantine-virus.html?action=click\&module=audio-series-bar\&region=header\&pgtype=Article}{\includegraphics{https://static01.graylady3jvrrxbe.onion/images/2020/05/06/podcasts/06sugarcalling/06sugarcalling-thumbLarge.jpg}}

May 6, 2020~~•~ 28:58`Whatever We Have, We Have to Work With It'

\href{https://www.nytimes3xbfgragh.onion/2020/04/29/podcasts/sugar-calling-judy-blume-quarantine-virus.html?action=click\&module=audio-series-bar\&region=header\&pgtype=Article}{\includegraphics{https://static01.graylady3jvrrxbe.onion/images/2020/04/29/podcasts/29sugarcalliing-blume-sub/29sugarcalliing-blume-sub-thumbLarge.jpg}}

April 29, 2020`This Terrible Thing Is Happening, but the World Goes On.'

\href{https://www.nytimes3xbfgragh.onion/2020/04/22/podcasts/sugar-calling-amy-tan-quarantine-virus.html?action=click\&module=audio-series-bar\&region=header\&pgtype=Article}{\includegraphics{https://static01.graylady3jvrrxbe.onion/images/2020/04/27/podcasts/22sugarcalling/22sugarcalling-thumbLarge.jpg}}

April 22, 2020~~•~ 39:19`You Don't Take Dictation. You Find the Truth.'

\href{https://www.nytimes3xbfgragh.onion/2020/04/15/podcasts/sugar-calling-pico-iyer-coronavirus.html?action=click\&module=audio-series-bar\&region=header\&pgtype=Article}{\includegraphics{https://static01.graylady3jvrrxbe.onion/images/2020/04/21/podcasts/15sugarcalling1/15sugarcalling1-thumbLarge.jpg}}

April 15, 2020~~•~ 35:45`Joyful Participation in a World of Sorrows'

\href{https://www.nytimes3xbfgragh.onion/2020/04/08/podcasts/sugar-calling-margaret-atwood-coronavirus.html?action=click\&module=audio-series-bar\&region=header\&pgtype=Article}{\includegraphics{https://static01.graylady3jvrrxbe.onion/images/2020/04/02/books/08sugarcalling1/08sugarcalling1-thumbLarge-v3.jpg}}

April 8, 2020~~•~ 34:32`Roll Up Your Sleeves, Girls'

\href{https://www.nytimes3xbfgragh.onion/2020/04/03/podcasts/sugar-calling-george-saunders-coronavirus.html?action=click\&module=audio-series-bar\&region=header\&pgtype=Article}{\includegraphics{https://static01.graylady3jvrrxbe.onion/images/2020/04/09/podcasts/03sugarcalling-image/merlin_171264408_4ac7fc67-d8cc-45b9-9ec6-bdd20672e694-thumbLarge.jpg}}

April 3, 2020~~•~ 41:16`Everything Is Always Keep Changing'

\href{https://www.nytimes3xbfgragh.onion/column/sugar-calling}{See All
Episodes ofSugar Calling}

Next

Published April 8, 2020Updated April 29, 2020

\begin{itemize}
\item
\item
\item
\item
\item
\end{itemize}

\emph{\textbf{Listen and subscribe to our podcast from your mobile
device:}}
\textbf{\href{https://podcasts.apple.com/us/podcast/sugar-calling/id1505881384}{\emph{Via
Apple Podcasts}}} \emph{\textbf{\textbar{}}}
\textbf{\href{https://open.spotify.com/show/4U8hPiNGIBvTS9zLeiDCN7?si=gRyigD47SPWl-QWgNjgt2w}{\emph{Via
Spotify}}} \emph{\textbf{\textbar{}}}
\textbf{\href{https://www.stitcher.com/podcast/the-new-york-times/sugar-calling}{\emph{Via
Stitcher}}}

\hypertarget{this-time-is-going-to-give-us-a-bit-of-a-reset-button-how-are-we-going-to-do-things-differently-on-the-other-side}{%
\subsection{`This time is going to give us a bit of a reset button. How
are we going to do things differently on the other
side?'}\label{this-time-is-going-to-give-us-a-bit-of-a-reset-button-how-are-we-going-to-do-things-differently-on-the-other-side}}

\emph{--- Margaret Atwood, author}

Today, Cheryl calls up one of the most prolific writers of our
generation, the environmentalist and octogenarian Margaret Atwood.
Margaret shares how she's spending her time in isolation, including
sewing masks and fending off squirrels (``Day before yesterday, I
climbed out on the roof because we had a squirrel problem,'' she tells
us).

Cheryl asks Margaret whether she's afraid for herself --- and for the
world --- and the two reflect on how to stay hopeful.

\includegraphics{https://static01.graylady3jvrrxbe.onion/images/2020/04/02/books/08sugarcalling1/merlin_160189194_16177006-165b-4fe9-bf68-cf87d19a5157-articleLarge.jpg?quality=75\&auto=webp\&disable=upscale}

\hypertarget{on-todays-episode}{%
\subsubsection{\texorpdfstring{\textbf{On today's
episode:}}{On today's episode:}}\label{on-todays-episode}}

\href{http://margaretatwood.ca/}{Margaret Atwood} is a Canadian writer
and author of over 50 works of fiction, nonfiction and poetry. Her
\href{https://www.nytimes3xbfgragh.onion/2017/03/10/books/review/margaret-atwood-handmaids-tale-age-of-trump.html}{canonical
1985 novel ``The Handmaid's Tale''} was adapted into a
\href{https://www.nytimes3xbfgragh.onion/2017/04/24/arts/television/review-the-handmaids-tale-creates-a-chilling-mans-world.html?module=inline}{hit
TV series} by Hulu in 2017. She published a sequel in 2019, titled
``\href{https://www.nytimes3xbfgragh.onion/2019/09/03/books/review/testaments-margaret-atwood-handmaids-tale.html}{The
Testaments},'' which was inspired in part by
``\href{https://www.nytimes3xbfgragh.onion/2018/11/28/books/margaret-atwood-sequel-handmaids-tale-testaments.html}{the
world we've been living in}.''

\hypertarget{margarets-quarantine-reading-list-by-mood}{%
\subsubsection{\texorpdfstring{\textbf{Margaret's quarantine reading
list, by
mood:}}{Margaret's quarantine reading list, by mood:}}\label{margarets-quarantine-reading-list-by-mood}}

\begin{itemize}
\tightlist
\item
  \emph{For remembering that it could be worse:}
  ``\href{https://www.amazon.com/Woman-Berlin/dp/1844081117}{A Woman in
  Berlin},'' Anonymous
\end{itemize}

\begin{itemize}
\item
  \emph{What we're going through now feels like:}
  ``\href{https://www.amazon.com/dp/B00NKDOZNM/ref=dp-kindle-redirect?_encoding=UTF8\&btkr=1}{Love
  in the Time of Cholera},'' Gabriel García Márquez
\item
  \emph{It could be better, here's how}:
  ``\href{https://www.amazon.com/Story-More-Climate-Change-Where/dp/0525563385}{The
  Story of More},'' Hope Jahren
\item
  \emph{To distract ourselves}:
  ``\href{https://www.amazon.com/Obsidian-Murders-DreadfulWater-Mystery/dp/1443457086}{The
  Obsidian Murders},'' Thomas King, and
  ``\href{https://www.amazon.com/Bunny-Novel-Mona-Awad/dp/0525559736}{Bunny},''
  Mona Awad
\item
  \emph{For a trip down memory lane}:
  ``\href{https://www.amazon.com/Equivalents-Story-Female-Friendship-Liberation-ebook/dp/B07SSQ4TWH}{The
  Equivalents: A Story of Art, Female Friendship, and Liberation in the
  1960s},'' Maggie Doherty
\end{itemize}

\begin{center}\rule{0.5\linewidth}{\linethickness}\end{center}

Cheryl Strayed is the author of ``Tiny Beautiful Things,'' ``Torch,''
``Brave Enough,'' and the New York Times best seller ``Wild.'' Her books
have been translated into more than 40 languages. She lives in Portland,
Oregon.
\href{https://twitter.com/CherylStrayed?ref_src=twsrc\%5Egoogle\%7Ctwcamp\%5Eserp\%7Ctwgr\%5Eauthor}{@CherylStrayed}

``Sugar Calling'' is produced by Kelly Prime and edited by Sara
Sarasohn, with editorial oversight by Wendy Dorr. This episode was mixed
by Jamie Collazo and scored by Dan Powell.

Advertisement

\protect\hyperlink{after-bottom}{Continue reading the main story}

\hypertarget{site-index}{%
\subsection{Site Index}\label{site-index}}

\hypertarget{site-information-navigation}{%
\subsection{Site Information
Navigation}\label{site-information-navigation}}

\begin{itemize}
\tightlist
\item
  \href{https://help.nytimes3xbfgragh.onion/hc/en-us/articles/115014792127-Copyright-notice}{©~2020~The
  New York Times Company}
\end{itemize}

\begin{itemize}
\tightlist
\item
  \href{https://www.nytco.com/}{NYTCo}
\item
  \href{https://help.nytimes3xbfgragh.onion/hc/en-us/articles/115015385887-Contact-Us}{Contact
  Us}
\item
  \href{https://www.nytco.com/careers/}{Work with us}
\item
  \href{https://nytmediakit.com/}{Advertise}
\item
  \href{http://www.tbrandstudio.com/}{T Brand Studio}
\item
  \href{https://www.nytimes3xbfgragh.onion/privacy/cookie-policy\#how-do-i-manage-trackers}{Your
  Ad Choices}
\item
  \href{https://www.nytimes3xbfgragh.onion/privacy}{Privacy}
\item
  \href{https://help.nytimes3xbfgragh.onion/hc/en-us/articles/115014893428-Terms-of-service}{Terms
  of Service}
\item
  \href{https://help.nytimes3xbfgragh.onion/hc/en-us/articles/115014893968-Terms-of-sale}{Terms
  of Sale}
\item
  \href{https://spiderbites.nytimes3xbfgragh.onion}{Site Map}
\item
  \href{https://help.nytimes3xbfgragh.onion/hc/en-us}{Help}
\item
  \href{https://www.nytimes3xbfgragh.onion/subscription?campaignId=37WXW}{Subscriptions}
\end{itemize}
