Sections

SEARCH

\protect\hyperlink{site-content}{Skip to
content}\protect\hyperlink{site-index}{Skip to site index}

\href{https://www.nytimes3xbfgragh.onion/section/health}{Health}

\href{https://myaccount.nytimes3xbfgragh.onion/auth/login?response_type=cookie\&client_id=vi}{}

\href{https://www.nytimes3xbfgragh.onion/section/todayspaper}{Today's
Paper}

\href{/section/health}{Health}\textbar{}Stay 6 Feet Apart, We're Told.
But How Far Can Air Carry Coronavirus?

\url{https://nyti.ms/2Vzryzg}

\begin{itemize}
\item
\item
\item
\item
\item
\item
\end{itemize}

\hypertarget{the-coronavirus-outbreak}{%
\subsubsection{\texorpdfstring{\href{https://www.nytimes3xbfgragh.onion/news-event/coronavirus?name=styln-coronavirus-national\&region=TOP_BANNER\&variant=undefined\&block=storyline_menu_recirc\&action=click\&pgtype=Article\&impression_id=b8d1f790-e0fc-11ea-9f19-cd4e9f07c8c6}{The
Coronavirus
Outbreak}}{The Coronavirus Outbreak}}\label{the-coronavirus-outbreak}}

\begin{itemize}
\tightlist
\item
  live\href{https://www.nytimes3xbfgragh.onion/2020/08/17/world/coronavirus-covid.html?name=styln-coronavirus-national\&region=TOP_BANNER\&variant=undefined\&block=storyline_menu_recirc\&action=click\&pgtype=Article\&impression_id=b8d1f791-e0fc-11ea-9f19-cd4e9f07c8c6}{Latest
  Updates}
\item
  \href{https://www.nytimes3xbfgragh.onion/interactive/2020/us/coronavirus-us-cases.html?name=styln-coronavirus-national\&region=TOP_BANNER\&variant=undefined\&block=storyline_menu_recirc\&action=click\&pgtype=Article\&impression_id=b8d1f792-e0fc-11ea-9f19-cd4e9f07c8c6}{Maps
  and Cases}
\item
  \href{https://www.nytimes3xbfgragh.onion/interactive/2020/science/coronavirus-vaccine-tracker.html?name=styln-coronavirus-national\&region=TOP_BANNER\&variant=undefined\&block=storyline_menu_recirc\&action=click\&pgtype=Article\&impression_id=b8d21ea0-e0fc-11ea-9f19-cd4e9f07c8c6}{Vaccine
  Tracker}
\item
  \href{https://www.nytimes3xbfgragh.onion/2020/08/17/us/k-12-schools-reopening.html?name=styln-coronavirus-national\&region=TOP_BANNER\&variant=undefined\&block=storyline_menu_recirc\&action=click\&pgtype=Article\&impression_id=b8d21ea1-e0fc-11ea-9f19-cd4e9f07c8c6}{State
  of Play for K-12}
\item
  \href{https://www.nytimes3xbfgragh.onion/live/2020/08/17/business/stock-market-today-coronavirus?name=styln-coronavirus-national\&region=TOP_BANNER\&variant=undefined\&block=storyline_menu_recirc\&action=click\&pgtype=Article\&impression_id=b8d21ea2-e0fc-11ea-9f19-cd4e9f07c8c6}{Markets
  \& Economy}
\end{itemize}

Advertisement

\protect\hyperlink{after-top}{Continue reading the main story}

Supported by

\protect\hyperlink{after-sponsor}{Continue reading the main story}

\hypertarget{stay-6-feet-apart-were-told-but-how-far-can-air-carry-coronavirus}{%
\section{Stay 6 Feet Apart, We're Told. But How Far Can Air Carry
Coronavirus?}\label{stay-6-feet-apart-were-told-but-how-far-can-air-carry-coronavirus}}

Most of the big droplets travel a mere six feet. The role of tiny
aerosols is the ``trillion-dollar question.''

\includegraphics{https://static01.graylady3jvrrxbe.onion/images/2020/04/19/science/19VIRUS-AIRBORNE/14VIRUS-AIRBORNE1-articleLarge.jpg?quality=75\&auto=webp\&disable=upscale}

\href{https://www.nytimes3xbfgragh.onion/by/knvul-sheikh}{\includegraphics{https://static01.graylady3jvrrxbe.onion/images/2020/01/03/reader-center/author-knvul-sheikh/author-knvul-sheikh-thumbLarge.png}}\href{https://www.nytimes3xbfgragh.onion/by/james-gorman}{\includegraphics{https://static01.graylady3jvrrxbe.onion/images/2018/02/16/multimedia/author-james-gorman/author-james-gorman-thumbLarge.jpg}}\href{https://www.nytimes3xbfgragh.onion/by/kenneth-chang}{\includegraphics{https://static01.graylady3jvrrxbe.onion/images/2018/02/16/multimedia/author-kenneth-chang/author-kenneth-chang-thumbLarge.jpg}}

By \href{https://www.nytimes3xbfgragh.onion/by/knvul-sheikh}{Knvul
Sheikh}, \href{https://www.nytimes3xbfgragh.onion/by/james-gorman}{James
Gorman} and
\href{https://www.nytimes3xbfgragh.onion/by/kenneth-chang}{Kenneth
Chang}

\begin{itemize}
\item
  Published April 14, 2020Updated April 20, 2020
\item
  \begin{itemize}
  \item
  \item
  \item
  \item
  \item
  \item
  \end{itemize}
\end{itemize}

The rule of thumb, or rather feet, has been to stand six feet apart in
public. That's supposed to be a safe distance if a person nearby is
coughing or sneezing and is infected with the novel coronavirus,
spreading droplets that may carry virus particles.

And scientists agree that six feet is a sensible and useful minimum
distance, but, some say, farther away would be better.

Six feet has never been a magic number that guarantees complete
protection. The Centers for Disease Control and Prevention, one of the
organizations using that measure, bases its recommendation on the idea
that most large droplets that people expel when they cough or sneeze
will fall to the ground within six feet.

But some scientists, having looked at studies of air flow and being
concerned about smaller particles called aerosols, suggest that people
consider a number of factors, including their own vulnerability and
whether they are outdoors or in an enclosed room, when deciding whether
six feet is enough distance.

Sneezes, for instance, can launch droplets a lot
\href{https://jamanetwork.com/journals/jama/fullarticle/2763852}{farther
than six feet according to a recent study.}

\href{https://www.nytimes3xbfgragh.onion/interactive/2020/04/14/science/coronavirus-transmission-cough-6-feet-ar-ul.html}{}

\includegraphics{https://static01.graylady3jvrrxbe.onion/images/2020/04/13/science/cough-image-still-promo/cough-image-still-promo-articleLarge-v2.jpg}

\hypertarget{this-3-d-simulation-shows-why-social-distancing-is-so-important}{%
\subsection{This 3-D Simulation Shows Why Social Distancing Is So
Important}\label{this-3-d-simulation-shows-why-social-distancing-is-so-important}}

We visualized a cough to show how far respiratory droplets can spread.
If you haven't been keeping your distance to fight the coronavirus, this
may persuade you.

No scientists are suggesting a wholesale change in behavior, or
proposing that some other length for separation from another human, like
seven, or nine feet, is actually the right one.

``Everything is about probability,'' said Dr. Harvey Fineberg, who is
the head of the Standing Committee on Emerging Infectious Diseases and
21st Century Health Threats at the National Academies of Sciences,
Engineering and Medicine. ``Three feet is better than nothing. Six feet
is better than three feet. At that point, the larger drops have pretty
much fallen down. Maybe if you're out of spitting range, that could be
even safer, but six feet is a pretty good number.''

\hypertarget{latest-updates-the-coronavirus-outbreak}{%
\section{\texorpdfstring{\href{https://www.nytimes3xbfgragh.onion/2020/08/17/world/coronavirus-covid.html?action=click\&pgtype=Article\&state=default\&region=MAIN_CONTENT_1\&context=storylines_live_updates}{Latest
Updates: The Coronavirus
Outbreak}}{Latest Updates: The Coronavirus Outbreak}}\label{latest-updates-the-coronavirus-outbreak}}

Updated 2020-08-18T01:58:10.452Z

\begin{itemize}
\tightlist
\item
  \href{https://www.nytimes3xbfgragh.onion/2020/08/17/world/coronavirus-covid.html?action=click\&pgtype=Article\&state=default\&region=MAIN_CONTENT_1\&context=storylines_live_updates\#link-6fdbc8ef}{U.S.
  college campuses grapple with coronavirus fears, outbreaks and
  protests.}
\item
  \href{https://www.nytimes3xbfgragh.onion/2020/08/17/world/coronavirus-covid.html?action=click\&pgtype=Article\&state=default\&region=MAIN_CONTENT_1\&context=storylines_live_updates\#link-12d68713}{For
  primary and secondary students and staff, it's been a difficult
  back-to-school season.}
\item
  \href{https://www.nytimes3xbfgragh.onion/2020/08/17/world/coronavirus-covid.html?action=click\&pgtype=Article\&state=default\&region=MAIN_CONTENT_1\&context=storylines_live_updates\#link-6aa8318c}{Data
  reporting problems are causing confusion in Iowa and elsewhere.}
\end{itemize}

\href{https://www.nytimes3xbfgragh.onion/2020/08/17/world/coronavirus-covid.html?action=click\&pgtype=Article\&state=default\&region=MAIN_CONTENT_1\&context=storylines_live_updates}{See
more updates}

More live coverage:
\href{https://www.nytimes3xbfgragh.onion/live/2020/08/17/business/stock-market-today-coronavirus?action=click\&pgtype=Article\&state=default\&region=MAIN_CONTENT_1\&context=storylines_live_updates}{Markets}

One complicating factor is that aerosols, smaller droplets that can be
emitted when people are breathing and talking, play some role in
spreading the new coronavirus. Studies have shown that
\href{https://www.nejm.org/doi/10.1056/NEJMc2004973}{aerosols can be
created} during certain hospital or laboratory procedures like when
using nebulizers to help patients inhale medication, which makes such
procedures risky for doctors who do them.

If the aerosols that people exhale in other settings are significant in
spreading the disease, the six-foot distance would not be completely
protective because those are carried more easily by air currents.

Aerosols are generally considered to be particles under 5 microns in
diameter, about the size of a red blood cell, and can be spread in the
environment by talking and breathing. But some researchers argue that
this is a false dichotomy. Infectious droplets can't easily be divided
into those that are big enough to fall to the ground quickly and those
that stay aloft because so much depends on environmental conditions and
how deeply they penetrate into the respiratory tract.

``It's really a continuum,'' said Dr. Donald Milton, who studies
bioaerosols at the University of Maryland School of Public Health.

Even without the launching power of a sneeze, air currents could carry a
flow of aerosol sized virus particles exhaled by an infected person 20
feet or more away.

``In any confined geometry like an office room, meeting room, department
store, food store,'' said Eugene Chudnovsky, a physicist at Lehman
College and the City University of New York's Graduate Center. In a
study not yet peer reviewed, he analyzed air flow and showed how, ``the
\href{https://arxiv.org/abs/2003.13689}{vortices in the air are taking
the virus to different places}.''

A preliminary study at the University of Nebraska Medical Center found
\href{https://www.medrxiv.org/content/10.1101/2020.03.23.20039446v2}{evidence
of coronavirus genetic material} on various surfaces in isolation rooms
where infected patients were being treated, including on air vents more
than six feet from the patients. The research, which has not yet been
peer reviewed, indicates that the virus can occasionally travel long
distances.

``The virus is so small, it can hitch a ride even on tiny, tiny
particles,'' Dr. Fineberg said. ``But how important is each size and how
well they can transmit disease is not fully understood.''

\href{https://www.nytimes3xbfgragh.onion/news-event/coronavirus?action=click\&pgtype=Article\&state=default\&region=MAIN_CONTENT_3\&context=storylines_faq}{}

\hypertarget{the-coronavirus-outbreak-}{%
\subsubsection{The Coronavirus Outbreak
›}\label{the-coronavirus-outbreak-}}

\hypertarget{frequently-asked-questions}{%
\paragraph{Frequently Asked
Questions}\label{frequently-asked-questions}}

Updated August 17, 2020

\begin{itemize}
\item ~
  \hypertarget{why-does-standing-six-feet-away-from-others-help}{%
  \paragraph{Why does standing six feet away from others
  help?}\label{why-does-standing-six-feet-away-from-others-help}}

  \begin{itemize}
  \tightlist
  \item
    The coronavirus spreads primarily through droplets from your mouth
    and nose, especially when you cough or sneeze. The C.D.C., one of
    the organizations using that measure,
    \href{https://www.nytimes3xbfgragh.onion/2020/04/14/health/coronavirus-six-feet.html?action=click\&pgtype=Article\&state=default\&region=MAIN_CONTENT_3\&context=storylines_faq}{bases
    its recommendation of six feet} on the idea that most large droplets
    that people expel when they cough or sneeze will fall to the ground
    within six feet. But six feet has never been a magic number that
    guarantees complete protection. Sneezes, for instance, can launch
    droplets a lot farther than six feet,
    \href{https://jamanetwork.com/journals/jama/fullarticle/2763852}{according
    to a recent study}. It's a rule of thumb: You should be safest
    standing six feet apart outside, especially when it's windy. But
    keep a mask on at all times, even when you think you're far enough
    apart.
  \end{itemize}
\item ~
  \hypertarget{i-have-antibodies-am-i-now-immune}{%
  \paragraph{I have antibodies. Am I now
  immune?}\label{i-have-antibodies-am-i-now-immune}}

  \begin{itemize}
  \tightlist
  \item
    As of right
    now,\href{https://www.nytimes3xbfgragh.onion/2020/07/22/health/covid-antibodies-herd-immunity.html?action=click\&pgtype=Article\&state=default\&region=MAIN_CONTENT_3\&context=storylines_faq}{that
    seems likely, for at least several months.} There have been
    frightening accounts of people suffering what seems to be a second
    bout of Covid-19. But experts say these patients may have a
    drawn-out course of infection, with the virus taking a slow toll
    weeks to months after initial exposure. People infected with the
    coronavirus typically
    \href{https://www.nature.com/articles/s41586-020-2456-9}{produce}
    immune molecules called antibodies, which are
    \href{https://www.nytimes3xbfgragh.onion/2020/05/07/health/coronavirus-antibody-prevalence.html?action=click\&pgtype=Article\&state=default\&region=MAIN_CONTENT_3\&context=storylines_faq}{protective
    proteins made in response to an
    infection}\href{https://www.nytimes3xbfgragh.onion/2020/05/07/health/coronavirus-antibody-prevalence.html?action=click\&pgtype=Article\&state=default\&region=MAIN_CONTENT_3\&context=storylines_faq}{.
    These antibodies may} last in the body
    \href{https://www.nature.com/articles/s41591-020-0965-6}{only two to
    three months}, which may seem worrisome, but that's perfectly normal
    after an acute infection subsides, said Dr. Michael Mina, an
    immunologist at Harvard University. It may be possible to get the
    coronavirus again, but it's highly unlikely that it would be
    possible in a short window of time from initial infection or make
    people sicker the second time.
  \end{itemize}
\item ~
  \hypertarget{im-a-small-business-owner-can-i-get-relief}{%
  \paragraph{I'm a small-business owner. Can I get
  relief?}\label{im-a-small-business-owner-can-i-get-relief}}

  \begin{itemize}
  \tightlist
  \item
    The
    \href{https://www.nytimes3xbfgragh.onion/article/small-business-loans-stimulus-grants-freelancers-coronavirus.html?action=click\&pgtype=Article\&state=default\&region=MAIN_CONTENT_3\&context=storylines_faq}{stimulus
    bills enacted in March} offer help for the millions of American
    small businesses. Those eligible for aid are businesses and
    nonprofit organizations with fewer than 500 workers, including sole
    proprietorships, independent contractors and freelancers. Some
    larger companies in some industries are also eligible. The help
    being offered, which is being managed by the Small Business
    Administration, includes the Paycheck Protection Program and the
    Economic Injury Disaster Loan program. But lots of folks have
    \href{https://www.nytimes3xbfgragh.onion/interactive/2020/05/07/business/small-business-loans-coronavirus.html?action=click\&pgtype=Article\&state=default\&region=MAIN_CONTENT_3\&context=storylines_faq}{not
    yet seen payouts.} Even those who have received help are confused:
    The rules are draconian, and some are stuck sitting on
    \href{https://www.nytimes3xbfgragh.onion/2020/05/02/business/economy/loans-coronavirus-small-business.html?action=click\&pgtype=Article\&state=default\&region=MAIN_CONTENT_3\&context=storylines_faq}{money
    they don't know how to use.} Many small-business owners are getting
    less than they expected or
    \href{https://www.nytimes3xbfgragh.onion/2020/06/10/business/Small-business-loans-ppp.html?action=click\&pgtype=Article\&state=default\&region=MAIN_CONTENT_3\&context=storylines_faq}{not
    hearing anything at all.}
  \end{itemize}
\item ~
  \hypertarget{what-are-my-rights-if-i-am-worried-about-going-back-to-work}{%
  \paragraph{What are my rights if I am worried about going back to
  work?}\label{what-are-my-rights-if-i-am-worried-about-going-back-to-work}}

  \begin{itemize}
  \tightlist
  \item
    Employers have to provide
    \href{https://www.osha.gov/SLTC/covid-19/standards.html}{a safe
    workplace} with policies that protect everyone equally.
    \href{https://www.nytimes3xbfgragh.onion/article/coronavirus-money-unemployment.html?action=click\&pgtype=Article\&state=default\&region=MAIN_CONTENT_3\&context=storylines_faq}{And
    if one of your co-workers tests positive for the coronavirus, the
    C.D.C.} has said that
    \href{https://www.cdc.gov/coronavirus/2019-ncov/community/guidance-business-response.html}{employers
    should tell their employees} -\/- without giving you the sick
    employee's name -\/- that they may have been exposed to the virus.
  \end{itemize}
\item ~
  \hypertarget{what-is-school-going-to-look-like-in-september}{%
  \paragraph{What is school going to look like in
  September?}\label{what-is-school-going-to-look-like-in-september}}

  \begin{itemize}
  \tightlist
  \item
    It is unlikely that many schools will return to a normal schedule
    this fall, requiring the grind of
    \href{https://www.nytimes3xbfgragh.onion/2020/06/05/us/coronavirus-education-lost-learning.html?action=click\&pgtype=Article\&state=default\&region=MAIN_CONTENT_3\&context=storylines_faq}{online
    learning},
    \href{https://www.nytimes3xbfgragh.onion/2020/05/29/us/coronavirus-child-care-centers.html?action=click\&pgtype=Article\&state=default\&region=MAIN_CONTENT_3\&context=storylines_faq}{makeshift
    child care} and
    \href{https://www.nytimes3xbfgragh.onion/2020/06/03/business/economy/coronavirus-working-women.html?action=click\&pgtype=Article\&state=default\&region=MAIN_CONTENT_3\&context=storylines_faq}{stunted
    workdays} to continue. California's two largest public school
    districts --- Los Angeles and San Diego --- said on July 13, that
    \href{https://www.nytimes3xbfgragh.onion/2020/07/13/us/lausd-san-diego-school-reopening.html?action=click\&pgtype=Article\&state=default\&region=MAIN_CONTENT_3\&context=storylines_faq}{instruction
    will be remote-only in the fall}, citing concerns that surging
    coronavirus infections in their areas pose too dire a risk for
    students and teachers. Together, the two districts enroll some
    825,000 students. They are the largest in the country so far to
    abandon plans for even a partial physical return to classrooms when
    they reopen in August. For other districts, the solution won't be an
    all-or-nothing approach.
    \href{https://bioethics.jhu.edu/research-and-outreach/projects/eschool-initiative/school-policy-tracker/}{Many
    systems}, including the nation's largest, New York City, are
    devising
    \href{https://www.nytimes3xbfgragh.onion/2020/06/26/us/coronavirus-schools-reopen-fall.html?action=click\&pgtype=Article\&state=default\&region=MAIN_CONTENT_3\&context=storylines_faq}{hybrid
    plans} that involve spending some days in classrooms and other days
    online. There's no national policy on this yet, so check with your
    municipal school system regularly to see what is happening in your
    community.
  \end{itemize}
\end{itemize}

It is also unclear how many virus particles it takes to start an
infection, how long the viral particles remain viable or if studies like
the one in Nebraska simply detected the genetic calling card the virus
left behind.

Spacing is an effective solution because it also reduces the number of
people in a confined space. That reduces the likelihood of an infected
person being in the group. And if there is one, fewer other people might
be infected.

Dr. Michael Osterholm, director of the Center for Infectious Disease
Research and Policy at the University of Minnesota, said the six-foot
distance would clearly ``reduce the number of droplets you come in
contact with. I have no doubt about that.''

But, he said: ``The question is what does it take for you to get
infected? And that I think is the trillion-dollar question we have.''

He said, ``Maybe all it takes is an aerosol. You don't need any droplets
at all.'' If that's the case, he said, then someone who is at high risk
would not want to be in the same room with someone who is infected or
might be infected.

Current guidelines already suggest that anyone at high risk should stay
home and not be out in public in the first place. And they seem to be
working. Places where people
\href{https://www.nytimes3xbfgragh.onion/interactive/2020/04/02/us/coronavirus-social-distancing.html}{reduced
travel and started social distancing weeks ago}, especially in
California, New York and Washington, are starting to show a reduction in
the number of new coronavirus cases.

People still need to shop and take care of necessities, Dr. Osterholm
said, but reducing the risk of exposure to all possible modes of
transmission --- infected surfaces, droplets and smaller aerosols --- is
important.

``Your job is to limit it as much as you can.''

Advertisement

\protect\hyperlink{after-bottom}{Continue reading the main story}

\hypertarget{site-index}{%
\subsection{Site Index}\label{site-index}}

\hypertarget{site-information-navigation}{%
\subsection{Site Information
Navigation}\label{site-information-navigation}}

\begin{itemize}
\tightlist
\item
  \href{https://help.nytimes3xbfgragh.onion/hc/en-us/articles/115014792127-Copyright-notice}{©~2020~The
  New York Times Company}
\end{itemize}

\begin{itemize}
\tightlist
\item
  \href{https://www.nytco.com/}{NYTCo}
\item
  \href{https://help.nytimes3xbfgragh.onion/hc/en-us/articles/115015385887-Contact-Us}{Contact
  Us}
\item
  \href{https://www.nytco.com/careers/}{Work with us}
\item
  \href{https://nytmediakit.com/}{Advertise}
\item
  \href{http://www.tbrandstudio.com/}{T Brand Studio}
\item
  \href{https://www.nytimes3xbfgragh.onion/privacy/cookie-policy\#how-do-i-manage-trackers}{Your
  Ad Choices}
\item
  \href{https://www.nytimes3xbfgragh.onion/privacy}{Privacy}
\item
  \href{https://help.nytimes3xbfgragh.onion/hc/en-us/articles/115014893428-Terms-of-service}{Terms
  of Service}
\item
  \href{https://help.nytimes3xbfgragh.onion/hc/en-us/articles/115014893968-Terms-of-sale}{Terms
  of Sale}
\item
  \href{https://spiderbites.nytimes3xbfgragh.onion}{Site Map}
\item
  \href{https://help.nytimes3xbfgragh.onion/hc/en-us}{Help}
\item
  \href{https://www.nytimes3xbfgragh.onion/subscription?campaignId=37WXW}{Subscriptions}
\end{itemize}
