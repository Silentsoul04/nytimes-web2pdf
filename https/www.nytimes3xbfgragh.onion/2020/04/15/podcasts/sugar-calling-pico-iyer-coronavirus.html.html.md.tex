Sections

SEARCH

\protect\hyperlink{site-content}{Skip to
content}\protect\hyperlink{site-index}{Skip to site index}

\href{https://www.nytimes3xbfgragh.onion/spotlight/podcasts}{Podcasts}

\href{https://myaccount.nytimes3xbfgragh.onion/auth/login?response_type=cookie\&client_id=vi}{}

\href{https://www.nytimes3xbfgragh.onion/section/todayspaper}{Today's
Paper}

\href{/spotlight/podcasts}{Podcasts}\textbar{}`Joyful Participation in a
World of Sorrows'

\url{https://nyti.ms/3aaJENb}

\begin{itemize}
\item
\item
\item
\item
\item
\end{itemize}

Advertisement

\protect\hyperlink{after-top}{Continue reading the main story}

transcript

Back to Sugar Calling

bars

0:00/35:45

-35:45

transcript

\hypertarget{joyful-participation-in-a-world-of-sorrows}{%
\subsection{`Joyful Participation in a World of
Sorrows'}\label{joyful-participation-in-a-world-of-sorrows}}

\hypertarget{hosted-by-cheryl-strayed-produced-by-kelly-prime-and-edited-by-sara-sarasohn-editorial-oversight-by-wendy-dorr}{%
\subsubsection{Hosted by Cheryl Strayed, produced by Kelly Prime and
edited by Sara Sarasohn. Editorial oversight by Wendy
Dorr.}\label{hosted-by-cheryl-strayed-produced-by-kelly-prime-and-edited-by-sara-sarasohn-editorial-oversight-by-wendy-dorr}}

\hypertarget{cheryl-strayed-calls-the-writer-pico-iyer-to-talk-about-forest-fires-the-dalai-lama-and-losing-what-you-love}{%
\paragraph{Cheryl Strayed calls the writer Pico Iyer to talk about
forest fires, the Dalai Lama and losing what you
love.}\label{cheryl-strayed-calls-the-writer-pico-iyer-to-talk-about-forest-fires-the-dalai-lama-and-losing-what-you-love}}

Wednesday, April 15th, 2020

\begin{itemize}
\item
  cheryl strayed\\
  Today, I'm going to call Pico Iyer. I first met him on the page when I
  was 20 and I read his book ``Video Night in Kathmandu.'' And right
  then and there, I knew, this is a writer I'll follow anywhere. He's
  published more than a dozen books, many of them about his travels
  around the world, but more importantly, about the journey inward. He
  very beautifully captures the miraculous ordinary moments and
  exchanges that we have on our travels and in life. And he's always
  looking for the deeper meaning in that. His most recent book is called
  ``Autumn Light.'' It's about his life in Japan. He's there right now.
  And I'm going to give him a call.
\item
  {[}music{]}
\item
  {[}line ringing{]}
\item
  cheryl strayed\\
  Hello?
\item
  pico iyer\\
  Hello. Is that Cheryl?
\item
  cheryl strayed\\
  It's Cheryl. Is that Pico?
\item
  pico iyer\\
  Yes. I'm so happy to hear your voice. We get to talk at last.
\item
  cheryl strayed\\
  I'm so excited. At last we get to speak, Pico.
\item
  pico iyer\\
  The feeling's mutual.
\item
  cheryl strayed\\
  I mean, we've corresponded over email, we've had a couple of almost
  meetings, but it's just never happened. So here we are.
\item
  pico iyer\\
  One of the beautiful things this curious moment has brought us, that
  you and I can speak more easily across 6,000 miles than perhaps if we
  were on the same west coast.
\item
  cheryl strayed\\
  Exactly. So tell me, where are you right now?
\item
  pico iyer\\
  I'm in a tiny apartment in a boring suburb in Japan. And there are
  cherry blossoms flowering over every last stream and path, bright blue
  spring morning. The nightingales have been teaching their young how to
  sing. Old folk are taking themselves and their dogs for a walk around
  the neighborhood. And as you probably know, things have tightened up
  quite a bit in Japan, so everybody is braced for maybe a spike in
  coronavirus cases. But for now, things really look very much as they
  always do.
\item
  cheryl strayed\\
  And what city are you a suburb of?
\item
  pico iyer\\
  We're a suburb of the great ancient capital of Nara. So it was the
  capital of Japan in the 8th century, but where we live is this little
  town that looks like it was modeled on the San Fernando Valley. It's a
  suburb younger than my kids are. We've been here 27 years now, kind of
  living almost as if we were students. We don't have a car, we don't
  have a bicycle. I've actually never used a cell phone until this
  conversation, when I'm borrowing my wife's. So we live in quite an
  isolated place. So I really feel blessed here. And of course, everyone
  here is really, really worried about the sick around the world and the
  people who don't have a home and the people who don't have resources.
  But Japan is relatively cushioned from that. And so I think people
  here are very good at remaining calm and remaining cheerful, and
  stoical, too, and realizing that things are a little hard right now,
  but they're much harder everywhere else.
\item
  cheryl strayed\\
  Right. So there hasn't yet been that sense of emergency in Japan. How
  long have you all been socially isolating?
\item
  pico iyer\\
  Well, that really only came in this week. And nobody is blase, but for
  the moment, we feel very lucky, and especially this springtime sense,
  because again, you know this, I'm sure, that everyone loves the cherry
  blossoms in Japan precisely because they don't last. And I think this
  is a culture that's always known how to live with impermanence because
  they've been living for 1,400 years very close to earthquake and
  tsunami and fire and warfare. And so they know a little bit about how
  to live with the fact that everything is constantly uncertain,
  changing, fragile and mortal. And the cherry blossoms are such a good
  example of that because it's the fact they only last for 10 days that
  makes them so beautiful. And if they lasted forever, we would take
  them for granted. When I came here, I heard this wonderful phrase
  about how life is about a joyful participation in a world of sorrows.
  In other words, every life ends in death, every meeting ends in a
  separation, but that's not a reason to grieve. It's actually a reason
  to find our beauty and joy right now.
\item
  cheryl strayed\\
  It's beautiful. I love those cherry blossoms. It is so clearly a
  metaphor for what I think the world is experiencing right now, being
  reminded of the fleeting beauty that also ends in a kind of demise or
  a loss. And I'm wondering if you can tell me more about what you know
  about impermanence.
\item
  pico iyer\\
  Well, that's a beautiful question, and I think I don't know any more
  than anybody else, except over my 63 years now, I've suffered my share
  of losses. And I actually remember very vividly as I'm talking to you,
  there was this one day when I was in my family home in the hills of
  California, and I saw this distant line of orange cutting through a
  hillside. So I went downstairs to call the fire department. And when I
  came upstairs again, I saw that our house was encircled by 70-foot
  flames, a forest fire that was being whipped along by 70-mile-per-hour
  winds. And by the end of that evening, three hours later, I had lost
  my house, every last thing I owned in the world, pretty much even my
  dreams of being a writer because all my next three books and all my
  notes were handwritten, and they were gone forever.
\item
  cheryl strayed\\
  And when was this, Pico?
\item
  pico iyer\\
  This was actually in 1990. I remember now, it was a shock to
  everybody, our whole town, but after about a year, I would say that
  there were some people who were shocked and scarred, but thought,
  well, maybe this is a chance to live a little differently and more
  closely to what really calls to me and my deeper self. When it came to
  replacing my things, for example, I realized I didn't need 90 percent
  of my books, my clothes, my furniture. The insurance company was ready
  to replace them, but I could live more lightly. Then I thought, well,
  actually since I don't have a family home in the U.S., maybe I can
  spend more time in Japan, a country that I really feel at home at. And
  then, since I'd lost all my notes, but I still was possessed by the
  wish to write and haunted by certain subjects, I thought, well, now's
  the time to try fiction, which I think I would have been too scared or
  shy to try otherwise. So at this great remove, 30 years on, I don't
  see that fire as entirely a bad thing in my life. And I see it as, in
  many ways, liberating me from the habits that I was kind of
  sleepwalking with through my life. I still get scared every time I see
  walls of flame rise up above the mountains above our rebuilt house,
  which has happened maybe 11 times since. But, I feel that that fire
  reduced me to nothing and forced me to think about what I care about,
  and actually sharpened my sense of priorities and made possible my
  life now.
\item
  cheryl strayed\\
  Yeah, right. I think that's one of the most fascinating things about
  human existence to me. You and I both teach writing. We almost taught
  together at our beloved Esalen Institute in Big Sur- -
\item
  pico iyer\\
  Yes.
\item
  cheryl strayed\\
  --- a couple years ago. But one of the things that I'm always struck
  by in my own life, and also in the lives of others --- and because I
  teach writing, I get to read so many other people's stories when we
  ask them to write --- so often, those stories are about the hardest
  thing. About the deepest, most painful wound and the time where they
  lost everything. And what's always true about that is we wouldn't have
  wished it upon ourselves, we wouldn't have wished it often upon our
  worst enemy, if we have enemies --- I don't, I hope you don't, either
  --- but they are always the things that teach us the most. I always
  say this to my students. If we went around the circle and I said you
  have to tell everyone how you learned the most important lessons of
  your life, they would almost always be painful experiences or big
  losses. And yet, so we know this to be true, right? You know this to
  be true, I know this to be true. Why do we resist it, then? I mean, we
  still say, but please, universe, don't let the bad thing happen to me.
\item
  pico iyer\\
  I think we resist it the way we resist going to the dentist. We all
  know we'll be better if that dentist fills our cavity and cleanses our
  infection and straightens our teeth, but nonetheless, we know it's
  going to be painful to get to that end result. I mean, it's funny what
  you were just saying, it's so true. And I think one of the things that
  the fire taught me, I feel, is that it was a reminder, as you said, of
  what we always know, which is that we have much less power over the
  world than we imagine, but I feel we have much more power over how we
  respond to the world than we suspect. And really, it's our responses
  to circumstances that define us much more than the circumstances
  themselves. And every writer knows --- well, actually that night of
  the fire, as soon as I was able to escape and drive downtown after
  three hours, the first thing I did was go to a friend's computer and
  write a piece about having an eyewitness view on this moment of
  history. And I suppose it was my small way of saying --- every writer
  knows this, but I think every person can be reminded of it --- the
  fire hasn't stripped me of everything. I still have my feelings, my
  thoughts, my words. I'm not powerless. It's a wonderful sharing that,
  when people go into those dark places, they're coming back with gifts
  to present to their readers or their friends that they never would if
  they were just saying it's a bright day and I'm having a wonderful
  time.
\item
  cheryl strayed\\
  And I think maybe that's those extremes that in both of your examples,
  really, that are given to us by the natural world. There's the
  outrageous beauty of those cherry blossoms, and then when they fade,
  there's still beauty. We still have beauty here every day. We might
  have to look a little harder for it, but it's there. And that fire
  scorches --- takes everything from you, and the first thing you think
  after reckoning with that loss is I still have the most important
  things.
\item
  pico iyer\\
  Exactly.
\item
  cheryl strayed\\
  Right? I still have everything I need.
\item
  pico iyer\\
  Exactly. That's right. You know, my parents come from India, and the
  great Shiva, the god of India, is the god of creation and destruction.
  And as you say, fire, water, they're all about those same things. And
  I think all of us, our first concern is with the people who are
  undefended and on the streets and really sick or in bad shape right
  now. But I think many of us who might have a job to return to and a
  roof over our heads and remain healthy through this virus, I think
  it's giving us really a chance to remember what sustains us and what
  we care about, which has much more to do with what you were just
  describing --- the cherry blossoms, as they flower, as they die, the
  fire --- than with what we were probably talking about two months ago,
  which is Meghan and Harry or what's the latest tweet. It's suddenly
  cut through all that. And for the fortunate, at least, this is a
  chance suddenly to cut that vicious cycle. Because I think for many
  years, I and many people I know, have felt we're in such a hurry we
  can't see what a hurry we're in. And more and more stuff is coming in
  on us, and we don't have time to process it. And suddenly, nature has
  intervened and said, OK, come to a halt. Your life is out of control.
  Stop. Now, where do you want your priorities to be? What do you want
  to spend the rest of your life doing? You don't know how long that
  rest of the life is, but this is your chance to get your life in
  balance a little bit. And I certainly felt that in this moment.
\item
  cheryl strayed\\
  It sounds like to me, though, and what I know of you, you already are
  a deeply reflective person. You don't really do social media or really
  absorb much news. And so how has this experience shifted your
  consciousness? Has it?
\item
  pico iyer\\
  So it's interesting because I'm not a religious person, but, for
  example, I love tennis, so I spend a lot of time watching Roger
  Federer. And I love --- I wish I were exaggerating, I'm not --- I love
  writing, so I spend a lot of time reading Alice Munro. And I love
  life, and ---
\item
  cheryl strayed\\
  Just so you know, Alice Munro is my favorite writer.
\item
  pico iyer\\
  Actually, I was driving along Highway 1 near Esalen when the news of
  her Nobel came on the radio. And I was so excited, I literally drove
  to the side of the road and jumped up and shouted, ``There's justice
  in the world.''
\item
  cheryl strayed\\
  Oh, that's so beautiful. I love her.
\item
  pico iyer\\
  So I feel I learned so much from reading Alice Munro, so much from
  watching Roger Federer, and so I wanted to spend time trying to turn
  to those people who I think are professionals of life. And in this
  instance, professionals of solitude and what to do with it. So I spend
  a lot of my life, I suppose, hanging out with monks, and, when I'm
  here in our apartment, reading Thomas Merton and Thoreau and listening
  to Leonard Cohen, because I think they're essentially professionals at
  the art of living. And I want to learn from them. I want to find out
  from them what matters.
\item
  cheryl strayed\\
  Yeah. So do I. So you mentioned Leonard Cohen, who I also love. You
  knew him, and you wrote the liner notes for four of his albums,
  correct?
\item
  pico iyer\\
  Yes. Yes.

  I actually, I first met him when he was living as an ordained Zen
  monk. And of course, it was eye-opening and humbling. And I drove up
  to these high, cold, dark mountains behind Los Angeles. There was a
  man in a tattered black robe who I didn't even recognize as my
  lifelong {[}INAUDIBLE{]}. And so Leonard Cohen, at the age of 61,
  spent five years --- he could have been doing anything in the world
  --- he spent five years just scrubbing floors and shoveling snow,
  cooking for his fellow monks, driving his aged abbot to the doctor.
  And he, more than anyone I've met, would tell me this was the great,
  as he would put it, voluptuous and delicious adventure he had found in
  his life. And I thought, when someone in his 60s who really has tasted
  everything --- sex, drugs, rock and roll, you name it --- says that, I
  should think about it. Because there was a man who'd been famous for
  40 years, he's a globally adored heartthrob, but I would say he was
  the wisest, kindest, and least impressed by himself person I've ever
  met, other than perhaps the Dalai Lama. And I thought, my heavens,
  this is really an instruction about what matters in life. Probably
  when I was a kid, I dreamed of being Leonard Cohen, the wandering
  minstrel and the cool poet with the beautiful girlfriend and the
  Armani suit. But meeting him in that monastery, I thought, no,
  actually what he's telling me is about selflessness and modesty and
  becoming an anonymous grunt, which I think is how he would refer to
  himself in that period.
\item
  cheryl strayed\\
  Yeah. And you look to him, is he one of your sources of wisdom when
  you --- you were talking about the writers who are amongst you, who
  offer those words of consolation and recognition or insight when
  you're looking for it ---
\item
  pico iyer\\
  Absol ---
\item
  cheryl strayed\\
  Where do you look?
\item
  pico iyer\\
  You know, I woke up and was listening to him over breakfast this
  morning, some of his earliest songs, and I was noticing something
  interesting, which is, when he was a young man, he was always writing
  about spiritual life. Jesus in his lonely wooden tower and Joseph in a
  manger, Sisters of Mercy. Once he became a monk, he didn't write about
  the spiritual life, he wrote, just as you were saying earlier, about
  pain and panic and confusion and loss, and the fact that mortality,
  death was coming closer and closer to him. And it struck me that, as a
  young man, he was all about yearning, as so many of us are, but once
  he really gave himself to this unflinching discipline, he was just
  going deep, deep, deep, deep into those dark places that you were
  talking about that you share with fellow writers in Esalen and
  elsewhere. And I think his two great gifts were intimacy and honesty.
  And that's why, even though he's left the world now, so many people I
  think feel him to be a very close friend and a real guide in difficult
  times.
\item
  cheryl strayed\\
  Yeah. And so Pico, I want to ask you, I know you've spent time with
  the Dalai Lama. And you traveled around Japan with him, is that
  correct?
\item
  pico iyer\\
  Yes. I mean, I've been lucky, I've known him since I was a teenager,
  and I wrote a book on him. And so since my book, I travel every year
  across Japan with him. And we traveled up to the tsunami area in Japan
  just a few months after 18,500 people had been killed in the tsunami.
  And we went to this little fishing village called Ishinomaki. And when
  we got there, it was a wasteland. It looked like pictures of Hiroshima
  after the war. And the only thing that survived, miraculously, was
  this little temple in the shelter of a mountain. And there were a few
  little schoolboys in school uniform, maybe five years old, who were
  waiting at the front and shaking his hand. And my wife told me that
  they were chosen because they were orphans. They'd just lost their
  parents. Now the whole community was looking after them. But they'd
  been given the honor of saying hello to the Dalai Lama. And then as he
  was sitting in the temple, he said, of course there's nothing I can
  give you except my heart and my presence. But he suddenly remembered a
  moment in March 1959 when he was in his hometown of Lhasa, Tibet, when
  he was told he had to leave his hometown that night. So no time to say
  goodbye to his friends, he had to leave then and there, and walk for
  two weeks across the highest mountains on the planet to get to safety
  in exile. And of course, as he said that, everyone there knew that,
  for 52 years at that point, he'd never been back to his homeland. It's
  now 60 years he's never been back to Tibet. And I think it was his way
  of saying, I can't imagine what you've been through, but loss is an
  equal opportunity employer. All of us have lost something, and we're
  all in this together. And that suffering is a fact of life, and
  actually the grace of it is that it helps us reach towards others. I
  have a great friend who's a Benedictine monk, and he sent this line to
  his friends last week, which is the best cure for anxiety is thinking
  of others. And I thought, thank heavens for monks. Doesn't matter if
  you're a Benedictine or not, that's really good advice. And maybe
  that's another of the possibilities that this moment holds for us,
  that we're not caught up in our own agendas, racing off to write this
  article, give this speech, do that meeting. We're suddenly alone with
  our thoughts, and our thoughts can turn to other people better.
\item
  cheryl strayed\\
  You said in one of your Ted Talks --- and I'm not remembering the
  story, so I want you to tell it to me again --- it was when you were
  traveling with the Dalai Lama, and you said to him each morning, what
  are we going to do, or what are we going to find, and he said, ``I
  don't know.'' Am I getting that right?
\item
  pico iyer\\
  Yes. And I think even more impressively, when he's in an auditorium
  with 30,000 people and, for example, somebody will put up a hand and
  ask a very serious question about either a marriage or children and
  child rearing, and he'll fool around laughing and say, ``You've asked
  the worst person in the world. I'm a celibate monk. I don't know a
  thing about marriage and relationships with a wife and kids. You know
  much more about that.'' And I think the first thing I'd say is, he has
  the courage to say that there's so many areas in the world he doesn't
  know about, and the humility to point out that we know many things
  about the world that the Dalai Lama doesn't. Well, you mentioned a Ted
  Talk. And one of the most moving I've ever seen was given by Pope
  Francis, who said --- the gist of it that I took away was, the more
  powerful you are, the more you have to be humble, and the more you
  have to essentially be kneeling before reality, before the future,
  before all the things that we know that we can't control. And so I
  love the fact that the Dalai Lama, who so many people seek out for
  wisdom, of which he has a huge amount, still will often say I don't
  know. Because he doesn't know what's happening tomorrow. He doesn't
  know what's happening tonight. He's a human being, as he would always
  stress. And so whenever people talk to him about his magical powers,
  he says, oh, I wish I had magical powers. If I did, I'd bring peace to
  Tibet and peace to the Middle East, and I wouldn't have a sore throat
  right now. And so don't look to me, the Dalai Lama, as somebody who's
  above it all. I, too, am surrendering. And he probably surrenders more
  than most. I think that's what distinguishes people like, probably,
  Pope Francis and Dalai Lama and the ones that we look to for wisdom
  --- Emily Dickinson, one of my great heroines --- because they're
  living even more than most of us in that state of surrender. And we
  think of them as all-knowing people, but the beauty of what they're
  carrying back to us is, I don't know. None of us knows.
\item
  cheryl strayed\\
  So listen, I love --- you and I have a lot of the same taste --- I
  love Emily Dickinson, too. But guess what I got to do last year? I got
  to go and write for an hour in her room where she wrote, where she
  lived and where she died. They told me not to touch her bed, but I was
  like a foot away from the bed where she died. And ---
\item
  pico iyer\\
  Can I ask what came out of that writing?
\item
  cheryl strayed\\
  You know, it was a radio show in England, I think it was the BBC was
  doing a thing on the life of Emily Dickinson and her legacy. And so
  they invited three writers. It was me, Sharon Olds and Ocean Vuong.
  And we each came at our separately appointed time slots and were left
  alone with the door closed in Emily Dickinson's bedroom. And I sat at
  her little desk, and I just looked --- I spent a lot of the time
  looking out the window, just trying to inhabit her, to see what she
  was seeing. And so I wrote a really kind of little piece about what I
  saw gazing out of Emily Dickinson's window. It's never left the pages
  of my journal, and probably never will, but it was a moving
  experience. It reminded me of how much we owe the people who came
  before us and did things that are still inspiring us today, and that
  we're part of that story, too. Part of impermanence isn't looking at
  other people and saying, well, they lived and they died and life goes
  on, it's also seeing that we, too, are going to live and die and life
  goes on.
\item
  pico iyer\\
  I am envious. If you can believe it, just this morning, before our
  conversation began, I was writing about her letters. I've always loved
  her poems, but as you know, I think when you read her letters, which
  are the most passionate love letters to her sister-in-law. And she
  would stand at the window you were describing and watch her
  sister-in-law walk to her brother's house and also get betrayed by the
  brother, and that --- just the image of the woman in her room with
  this huge passion that came out in words rather than touch. What
  haunting thing. I should tell you, I've been a couple of times to
  Thomas Merton's monastery, Gethsemane, in Kentucky. And the last time
  I was there, I met the one remaining monk there who studied under
  Thomas Merton and kind of carries the torch for him. And he walked me
  across the fields to Thomas Merton's hermitage, which is about 10
  minutes away from the monastery. And as he did, he said, I've got to
  tell you, for the last three years, I've been having a passionate love
  affair, and I've just completely lost my heart to this person. So of
  course, my ears pricked up, who would a monk be in love with? And you
  won't be surprised to hear that he said Emily Dickinson. She's the one
  I live for. I read her every day. And when we got to Thomas Merton's
  hermitage, this monk, who was in about late 70s, I would say, sat down
  and he started reciting Emily Dickinson poems, knowing that that was
  the truest way of keeping the legacy of literature. The legacy of
  solitude, the legacy of introspection. Everything that Thomas Merton
  represented, Emily Dickinson represents, too. And when we talk of
  impermanence, I earlier in this conversation said Leonard Cohen died,
  but everyone is listening to his songs and reading his poems still.
  And so what a beautiful permanence is all around us in art and music
  and poetry and nature. The blue skies.
\item
  cheryl strayed\\
  Right. That is the permanence, art and beauty, and loss and sorrow.
  Those are the permanent universal things.
\item
  pico iyer\\
  Yes.
\item
  cheryl strayed\\
  So one thing I want to ask you, Pico, I want to know if I have this
  right. You were born in England to parents who were born in India, and
  then you all moved to California when you were a kid. Is that correct?
\item
  pico iyer\\
  You have it absolutely right, Cheryl. Yes, I moved to California when
  I was about seven. But because of the exchange rates then, and also
  the Californian educational system was very different from the English
  one, I worked out it was cheaper to continue going to school in
  England and flying back three times a year to see my parents than to
  go to the local private school in California. So that's one of the
  reasons why I got into this traveling madness. I was commuting to
  school by plane alone over the North Pole from the age of nine, so I
  had to define home inwardly. And so I just have to sit here in my
  apartment in Japan talking to you in Portland, and I have in my mind's
  eye the view through the monastery window in Big Sur. And that's my
  home that's inside me and that grounds me, and that actually really
  consoles and inspires me at this moment of uncertainty.
\item
  cheryl strayed\\
  And your mother is in California, is that right?
\item
  pico iyer\\
  She is in California. She's been in hospital for a couple of weeks. I
  was all packed, I was set to fly over to see her, and then I found
  that visitors aren't allowed in hospitals in California right now, for
  very good reasons. And so again, I would be no closer to her if I was
  in her empty house 10 minutes by car from the hospital than I am right
  now, when I can still talk to her by phone and communicate.
\item
  cheryl strayed\\
  Right. And your mom, just to be clear, doesn't have the coronavirus.
  She's simply unwell. Is that right?
\item
  pico iyer\\
  That's right. She's safe from the virus. She's really just suffering
  from being 88 years old. And all being well, this time next week, I
  will be with her in her home because she's getting better, and I am
  planning to fly over to offer her moral support and company. So I'm
  hoping the end of that story is in sight and it's a happy ending.
\item
  cheryl strayed\\
  Oh, that's nice. I know so many people are really, really afraid for
  their elderly parents right now because we keep hearing over and over,
  right, that this virus is especially dangerous for people the older
  they are. Have you been particularly afraid for your mom during this
  time?
\item
  pico iyer\\
  Because her health is always uncertain and I know how frail she is, I
  was worried when she went to the hospital. And before the ambulance
  took her to the hospital, the paramedic very kindly called me in Japan
  and said, do you allow us to take your mother to the hospital, even
  though we're taking her into a place of maximal risk? And I thought I
  had to take that risk on on her behalf. And yes, she suffers from
  asthma. I suffer from asthma, my wife suffers from asthma. My wife and
  I are both in our 60s. We're all really in a state of risk. But I
  think the broader thing is, I'm sure you've found, it's one of the
  things that growing older brings us. I remember when I was in my 20s,
  I was really scared of dying. I thought, what will happen to my
  brilliant career, all the books I could write, all the places I could
  see? You know, silly, silly, silly, silly stuff. And as I got older, I
  realized what I'm most scared of is seeing my loved ones die or, if I
  died, what would happen to them. They would be, financially and in
  other ways, undefended. I'm much more concerned for my mother, and
  actually for my wife and for my kids, than I am for me, because that
  would be the real heartbreak, if my wife was left alone or if my
  mother were to see her son go before her. That's always one of the
  hardest things. And I'm grateful for the passing years for teaching me
  that it's really other people's deaths that we're most scared about,
  not our own.
\item
  cheryl strayed\\
  Right.
\item
  pico iyer\\
  And the other thing that had come to my mind --- again, we spoke about
  the Dalai Lama --- and in the last year or two, he's been talking a
  lot about what he calls emotional disarmament. And I take that to mean
  he thinks, of course, because he embodies it, the most important thing
  is to keep our hearts open, to feel for everything, and for one
  another. But he thinks that anger, panic, and confusion keep us away
  from reality, prevent us from making clear decisions. And some people
  are surprised that I, as a journalist, make sure I never take in more
  than five minutes of the news every day, but ideally two minutes of
  the news every day, or less. And I think that's because I don't want
  to be in a state of panic or confusion. It's really important to want
  to know what's going on in the world, and especially this moment, it's
  very important to know what we can and cannot do, what precautions we
  have to take. But I think you can do all of that in a couple of
  minutes in the day. And then for the rest of the day, the poems we've
  been discussing, the wise people we've been thinking about, our
  family, our friends, I think that's going to make us much better
  prepared for what's going to happen than keeping up with every last
  second of speculation and opinion and science we don't know what to do
  with.
\item
  cheryl strayed\\
  I think that that's such a fine distinction. Absorbing too much news
  can easily make us feel panicked, right? Absorbing enough that you
  have the information you need for clarity is what the goal is, but so
  many of us have a hard time shutting it off. And I know that, in my
  own life, that there is that tendency to think that, if I pay
  attention to everything that's being said about this virus or about
  these times or the economic fallout, that in some way I can maybe stop
  it from happening, or stop it from happening as bad for me. And yet,
  there is panic at the center of that, and a kind of lie at the center
  of that because, of course, we can't control it. When you were
  talking, and we've been talking so much about mortality, I was
  remembering this time, it was actually the last day of the year. It
  was December 31, 1991, so I was I was 23. And I was driving with the
  man who was then my husband. We were driving from Minneapolis up to
  this place called Madeline Island in Lake Superior, off the sort of
  shoreline of Wisconsin. And it was a snowy, icy day, as it often is.
  My ex-husband was driving, and I was in the passenger seat, and we
  lost control of the car. It just started to skid on the ice. We were
  going 55 miles an hour, and suddenly we were just sailing all over the
  highway. And it was terrifying. It was the only time in my life that I
  honestly thought I was going to die. And I was saying to my
  ex-husband, get control of the car, get control of the car, get
  control of the car over and over again, as we were just spinning out
  all over. And then there was this moment where the car just left the
  highway, like we just went airborne off the side of the road. And the
  ditch was far below, so we were flying. And we were going to crash.
  And I stopped saying get control of the car, and both my ex-husband
  and I at the same moment, at the very same moment when we hit the air,
  looked at each other and grabbed each other. And we said I love you at
  the same time. And then the car crashed. And we survived, but what
  shook me most deeply is that moment that we went, both of us,
  simultaneously, from panic to clarity. And inside that clarity was the
  only thing that matters: We just wanted to say I love you one more
  time to each other. And it wasn't scary. It wasn't panic. It was just
  the truth. And I've always thought that there's something about that
  those two things can sit so closely beside each other, can exist even
  within the same second, that there was something revealing about that.
  And maybe there's something in that that's relevant for now, too.
\item
  pico iyer\\
  I'm silenced by that. That is such a haunting, beautiful story. All of
  life is right in that moment. How many miracles in every day actually
  keep us going? We often notice only the terrible things that happen,
  and don't realize how many wonderful things are taken for granted. At
  the time of our forest fire, I thought, ``Oh, if only I'd had five
  more minutes, I'd have been able to save everything that's important
  to me.'' And then I thought, ``Well, if only I'd had five less
  minutes, I'd have been dead.'' I'm so glad you survived that accident,
  and even more, that you can share that story, because I think just
  having that story in our heads can turn our life in a better
  direction.
\item
  cheryl strayed\\
  Yeah. I got to see that. I got to live that. And I hope it's what I do
  the next time around, too. I hope so.
\item
  pico iyer\\
  Yeah, I mean it's kind of like a dress rehearsal. I often think it's
  one of the graces that we have as a writer, that we can use our, so to
  speak, job to help prepare us for these things, and to think through
  the inevitable stuff of life, which otherwise I find I'm much too
  ready to ignore.
\item
  cheryl strayed\\
  Indeed. Pico, it has been an absolute pleasure to talk to you. You've
  made me feel comforted and inspired and enlightened. So thank you.
  Thank you for speaking to me from all the way across these miles.
\item
  pico iyer\\
  Thank you for inviting me to be a part of this.
\item
  cheryl strayed\\
  Bye, Pico.
\item
  pico iyer\\
  Bye bye.
\item
  {[}music{]}
\item
  cheryl strayed\\
  I'm Cheryl Strayed. This is ``Sugar Calling.'' Next week, Amy Tan.
\end{itemize}

\href{https://www.nytimes3xbfgragh.onion/column/sugar-calling}{\includegraphics{https://static01.graylady3jvrrxbe.onion/images/2020/04/29/podcasts/sugar-calling-album-art/sugar-calling-album-art-square320.jpg}Sugar
Calling}Subscribe:

\begin{itemize}
\tightlist
\item
  \href{https://itunes.apple.com/us/podcast/id1505881384}{Apple
  Podcasts}
\item
  \href{https://podcasts.google.com/?feed=aHR0cHM6Ly9yc3MuYXJ0MTkuY29tL3N1Z2FyLWNhbGxpbmc\&ved=0CAUQrrcFahcKEwjA8Kyn09voAhUAAAAAHQAAAAAQBQ}{Google
  Podcasts}
\end{itemize}

\hypertarget{joyful-participation-in-a-world-of-sorrows-1}{%
\section{`Joyful Participation in a World of
Sorrows'}\label{joyful-participation-in-a-world-of-sorrows-1}}

\hypertarget{cheryl-strayed-calls-the-writer-pico-iyer-to-talk-about-forest-fires-the-dalai-lama-and-losing-what-you-love-1}{%
\subsection{Cheryl Strayed calls the writer Pico Iyer to talk about
forest fires, the Dalai Lama and losing what you
love.}\label{cheryl-strayed-calls-the-writer-pico-iyer-to-talk-about-forest-fires-the-dalai-lama-and-losing-what-you-love-1}}

Hosted by Cheryl Strayed, produced by Kelly Prime and edited by Sara
Sarasohn. Editorial oversight by Wendy Dorr.

Transcript

transcript

Back to Sugar Calling

bars

0:00/35:45

-0:00

transcript

\hypertarget{joyful-participation-in-a-world-of-sorrows-2}{%
\subsection{`Joyful Participation in a World of
Sorrows'}\label{joyful-participation-in-a-world-of-sorrows-2}}

\hypertarget{hosted-by-cheryl-strayed-produced-by-kelly-prime-and-edited-by-sara-sarasohn-editorial-oversight-by-wendy-dorr-1}{%
\subsubsection{Hosted by Cheryl Strayed, produced by Kelly Prime and
edited by Sara Sarasohn. Editorial oversight by Wendy
Dorr.}\label{hosted-by-cheryl-strayed-produced-by-kelly-prime-and-edited-by-sara-sarasohn-editorial-oversight-by-wendy-dorr-1}}

\hypertarget{cheryl-strayed-calls-the-writer-pico-iyer-to-talk-about-forest-fires-the-dalai-lama-and-losing-what-you-love-2}{%
\paragraph{Cheryl Strayed calls the writer Pico Iyer to talk about
forest fires, the Dalai Lama and losing what you
love.}\label{cheryl-strayed-calls-the-writer-pico-iyer-to-talk-about-forest-fires-the-dalai-lama-and-losing-what-you-love-2}}

Wednesday, April 15th, 2020

\begin{itemize}
\item
  cheryl strayed\\
  Today, I'm going to call Pico Iyer. I first met him on the page when I
  was 20 and I read his book ``Video Night in Kathmandu.'' And right
  then and there, I knew, this is a writer I'll follow anywhere. He's
  published more than a dozen books, many of them about his travels
  around the world, but more importantly, about the journey inward. He
  very beautifully captures the miraculous ordinary moments and
  exchanges that we have on our travels and in life. And he's always
  looking for the deeper meaning in that. His most recent book is called
  ``Autumn Light.'' It's about his life in Japan. He's there right now.
  And I'm going to give him a call.
\item
  {[}music{]}
\item
  {[}line ringing{]}
\item
  cheryl strayed\\
  Hello?
\item
  pico iyer\\
  Hello. Is that Cheryl?
\item
  cheryl strayed\\
  It's Cheryl. Is that Pico?
\item
  pico iyer\\
  Yes. I'm so happy to hear your voice. We get to talk at last.
\item
  cheryl strayed\\
  I'm so excited. At last we get to speak, Pico.
\item
  pico iyer\\
  The feeling's mutual.
\item
  cheryl strayed\\
  I mean, we've corresponded over email, we've had a couple of almost
  meetings, but it's just never happened. So here we are.
\item
  pico iyer\\
  One of the beautiful things this curious moment has brought us, that
  you and I can speak more easily across 6,000 miles than perhaps if we
  were on the same west coast.
\item
  cheryl strayed\\
  Exactly. So tell me, where are you right now?
\item
  pico iyer\\
  I'm in a tiny apartment in a boring suburb in Japan. And there are
  cherry blossoms flowering over every last stream and path, bright blue
  spring morning. The nightingales have been teaching their young how to
  sing. Old folk are taking themselves and their dogs for a walk around
  the neighborhood. And as you probably know, things have tightened up
  quite a bit in Japan, so everybody is braced for maybe a spike in
  coronavirus cases. But for now, things really look very much as they
  always do.
\item
  cheryl strayed\\
  And what city are you a suburb of?
\item
  pico iyer\\
  We're a suburb of the great ancient capital of Nara. So it was the
  capital of Japan in the 8th century, but where we live is this little
  town that looks like it was modeled on the San Fernando Valley. It's a
  suburb younger than my kids are. We've been here 27 years now, kind of
  living almost as if we were students. We don't have a car, we don't
  have a bicycle. I've actually never used a cell phone until this
  conversation, when I'm borrowing my wife's. So we live in quite an
  isolated place. So I really feel blessed here. And of course, everyone
  here is really, really worried about the sick around the world and the
  people who don't have a home and the people who don't have resources.
  But Japan is relatively cushioned from that. And so I think people
  here are very good at remaining calm and remaining cheerful, and
  stoical, too, and realizing that things are a little hard right now,
  but they're much harder everywhere else.
\item
  cheryl strayed\\
  Right. So there hasn't yet been that sense of emergency in Japan. How
  long have you all been socially isolating?
\item
  pico iyer\\
  Well, that really only came in this week. And nobody is blase, but for
  the moment, we feel very lucky, and especially this springtime sense,
  because again, you know this, I'm sure, that everyone loves the cherry
  blossoms in Japan precisely because they don't last. And I think this
  is a culture that's always known how to live with impermanence because
  they've been living for 1,400 years very close to earthquake and
  tsunami and fire and warfare. And so they know a little bit about how
  to live with the fact that everything is constantly uncertain,
  changing, fragile and mortal. And the cherry blossoms are such a good
  example of that because it's the fact they only last for 10 days that
  makes them so beautiful. And if they lasted forever, we would take
  them for granted. When I came here, I heard this wonderful phrase
  about how life is about a joyful participation in a world of sorrows.
  In other words, every life ends in death, every meeting ends in a
  separation, but that's not a reason to grieve. It's actually a reason
  to find our beauty and joy right now.
\item
  cheryl strayed\\
  It's beautiful. I love those cherry blossoms. It is so clearly a
  metaphor for what I think the world is experiencing right now, being
  reminded of the fleeting beauty that also ends in a kind of demise or
  a loss. And I'm wondering if you can tell me more about what you know
  about impermanence.
\item
  pico iyer\\
  Well, that's a beautiful question, and I think I don't know any more
  than anybody else, except over my 63 years now, I've suffered my share
  of losses. And I actually remember very vividly as I'm talking to you,
  there was this one day when I was in my family home in the hills of
  California, and I saw this distant line of orange cutting through a
  hillside. So I went downstairs to call the fire department. And when I
  came upstairs again, I saw that our house was encircled by 70-foot
  flames, a forest fire that was being whipped along by 70-mile-per-hour
  winds. And by the end of that evening, three hours later, I had lost
  my house, every last thing I owned in the world, pretty much even my
  dreams of being a writer because all my next three books and all my
  notes were handwritten, and they were gone forever.
\item
  cheryl strayed\\
  And when was this, Pico?
\item
  pico iyer\\
  This was actually in 1990. I remember now, it was a shock to
  everybody, our whole town, but after about a year, I would say that
  there were some people who were shocked and scarred, but thought,
  well, maybe this is a chance to live a little differently and more
  closely to what really calls to me and my deeper self. When it came to
  replacing my things, for example, I realized I didn't need 90 percent
  of my books, my clothes, my furniture. The insurance company was ready
  to replace them, but I could live more lightly. Then I thought, well,
  actually since I don't have a family home in the U.S., maybe I can
  spend more time in Japan, a country that I really feel at home at. And
  then, since I'd lost all my notes, but I still was possessed by the
  wish to write and haunted by certain subjects, I thought, well, now's
  the time to try fiction, which I think I would have been too scared or
  shy to try otherwise. So at this great remove, 30 years on, I don't
  see that fire as entirely a bad thing in my life. And I see it as, in
  many ways, liberating me from the habits that I was kind of
  sleepwalking with through my life. I still get scared every time I see
  walls of flame rise up above the mountains above our rebuilt house,
  which has happened maybe 11 times since. But, I feel that that fire
  reduced me to nothing and forced me to think about what I care about,
  and actually sharpened my sense of priorities and made possible my
  life now.
\item
  cheryl strayed\\
  Yeah, right. I think that's one of the most fascinating things about
  human existence to me. You and I both teach writing. We almost taught
  together at our beloved Esalen Institute in Big Sur- -
\item
  pico iyer\\
  Yes.
\item
  cheryl strayed\\
  --- a couple years ago. But one of the things that I'm always struck
  by in my own life, and also in the lives of others --- and because I
  teach writing, I get to read so many other people's stories when we
  ask them to write --- so often, those stories are about the hardest
  thing. About the deepest, most painful wound and the time where they
  lost everything. And what's always true about that is we wouldn't have
  wished it upon ourselves, we wouldn't have wished it often upon our
  worst enemy, if we have enemies --- I don't, I hope you don't, either
  --- but they are always the things that teach us the most. I always
  say this to my students. If we went around the circle and I said you
  have to tell everyone how you learned the most important lessons of
  your life, they would almost always be painful experiences or big
  losses. And yet, so we know this to be true, right? You know this to
  be true, I know this to be true. Why do we resist it, then? I mean, we
  still say, but please, universe, don't let the bad thing happen to me.
\item
  pico iyer\\
  I think we resist it the way we resist going to the dentist. We all
  know we'll be better if that dentist fills our cavity and cleanses our
  infection and straightens our teeth, but nonetheless, we know it's
  going to be painful to get to that end result. I mean, it's funny what
  you were just saying, it's so true. And I think one of the things that
  the fire taught me, I feel, is that it was a reminder, as you said, of
  what we always know, which is that we have much less power over the
  world than we imagine, but I feel we have much more power over how we
  respond to the world than we suspect. And really, it's our responses
  to circumstances that define us much more than the circumstances
  themselves. And every writer knows --- well, actually that night of
  the fire, as soon as I was able to escape and drive downtown after
  three hours, the first thing I did was go to a friend's computer and
  write a piece about having an eyewitness view on this moment of
  history. And I suppose it was my small way of saying --- every writer
  knows this, but I think every person can be reminded of it --- the
  fire hasn't stripped me of everything. I still have my feelings, my
  thoughts, my words. I'm not powerless. It's a wonderful sharing that,
  when people go into those dark places, they're coming back with gifts
  to present to their readers or their friends that they never would if
  they were just saying it's a bright day and I'm having a wonderful
  time.
\item
  cheryl strayed\\
  And I think maybe that's those extremes that in both of your examples,
  really, that are given to us by the natural world. There's the
  outrageous beauty of those cherry blossoms, and then when they fade,
  there's still beauty. We still have beauty here every day. We might
  have to look a little harder for it, but it's there. And that fire
  scorches --- takes everything from you, and the first thing you think
  after reckoning with that loss is I still have the most important
  things.
\item
  pico iyer\\
  Exactly.
\item
  cheryl strayed\\
  Right? I still have everything I need.
\item
  pico iyer\\
  Exactly. That's right. You know, my parents come from India, and the
  great Shiva, the god of India, is the god of creation and destruction.
  And as you say, fire, water, they're all about those same things. And
  I think all of us, our first concern is with the people who are
  undefended and on the streets and really sick or in bad shape right
  now. But I think many of us who might have a job to return to and a
  roof over our heads and remain healthy through this virus, I think
  it's giving us really a chance to remember what sustains us and what
  we care about, which has much more to do with what you were just
  describing --- the cherry blossoms, as they flower, as they die, the
  fire --- than with what we were probably talking about two months ago,
  which is Meghan and Harry or what's the latest tweet. It's suddenly
  cut through all that. And for the fortunate, at least, this is a
  chance suddenly to cut that vicious cycle. Because I think for many
  years, I and many people I know, have felt we're in such a hurry we
  can't see what a hurry we're in. And more and more stuff is coming in
  on us, and we don't have time to process it. And suddenly, nature has
  intervened and said, OK, come to a halt. Your life is out of control.
  Stop. Now, where do you want your priorities to be? What do you want
  to spend the rest of your life doing? You don't know how long that
  rest of the life is, but this is your chance to get your life in
  balance a little bit. And I certainly felt that in this moment.
\item
  cheryl strayed\\
  It sounds like to me, though, and what I know of you, you already are
  a deeply reflective person. You don't really do social media or really
  absorb much news. And so how has this experience shifted your
  consciousness? Has it?
\item
  pico iyer\\
  So it's interesting because I'm not a religious person, but, for
  example, I love tennis, so I spend a lot of time watching Roger
  Federer. And I love --- I wish I were exaggerating, I'm not --- I love
  writing, so I spend a lot of time reading Alice Munro. And I love
  life, and ---
\item
  cheryl strayed\\
  Just so you know, Alice Munro is my favorite writer.
\item
  pico iyer\\
  Actually, I was driving along Highway 1 near Esalen when the news of
  her Nobel came on the radio. And I was so excited, I literally drove
  to the side of the road and jumped up and shouted, ``There's justice
  in the world.''
\item
  cheryl strayed\\
  Oh, that's so beautiful. I love her.
\item
  pico iyer\\
  So I feel I learned so much from reading Alice Munro, so much from
  watching Roger Federer, and so I wanted to spend time trying to turn
  to those people who I think are professionals of life. And in this
  instance, professionals of solitude and what to do with it. So I spend
  a lot of my life, I suppose, hanging out with monks, and, when I'm
  here in our apartment, reading Thomas Merton and Thoreau and listening
  to Leonard Cohen, because I think they're essentially professionals at
  the art of living. And I want to learn from them. I want to find out
  from them what matters.
\item
  cheryl strayed\\
  Yeah. So do I. So you mentioned Leonard Cohen, who I also love. You
  knew him, and you wrote the liner notes for four of his albums,
  correct?
\item
  pico iyer\\
  Yes. Yes.

  I actually, I first met him when he was living as an ordained Zen
  monk. And of course, it was eye-opening and humbling. And I drove up
  to these high, cold, dark mountains behind Los Angeles. There was a
  man in a tattered black robe who I didn't even recognize as my
  lifelong {[}INAUDIBLE{]}. And so Leonard Cohen, at the age of 61,
  spent five years --- he could have been doing anything in the world
  --- he spent five years just scrubbing floors and shoveling snow,
  cooking for his fellow monks, driving his aged abbot to the doctor.
  And he, more than anyone I've met, would tell me this was the great,
  as he would put it, voluptuous and delicious adventure he had found in
  his life. And I thought, when someone in his 60s who really has tasted
  everything --- sex, drugs, rock and roll, you name it --- says that, I
  should think about it. Because there was a man who'd been famous for
  40 years, he's a globally adored heartthrob, but I would say he was
  the wisest, kindest, and least impressed by himself person I've ever
  met, other than perhaps the Dalai Lama. And I thought, my heavens,
  this is really an instruction about what matters in life. Probably
  when I was a kid, I dreamed of being Leonard Cohen, the wandering
  minstrel and the cool poet with the beautiful girlfriend and the
  Armani suit. But meeting him in that monastery, I thought, no,
  actually what he's telling me is about selflessness and modesty and
  becoming an anonymous grunt, which I think is how he would refer to
  himself in that period.
\item
  cheryl strayed\\
  Yeah. And you look to him, is he one of your sources of wisdom when
  you --- you were talking about the writers who are amongst you, who
  offer those words of consolation and recognition or insight when
  you're looking for it ---
\item
  pico iyer\\
  Absol ---
\item
  cheryl strayed\\
  Where do you look?
\item
  pico iyer\\
  You know, I woke up and was listening to him over breakfast this
  morning, some of his earliest songs, and I was noticing something
  interesting, which is, when he was a young man, he was always writing
  about spiritual life. Jesus in his lonely wooden tower and Joseph in a
  manger, Sisters of Mercy. Once he became a monk, he didn't write about
  the spiritual life, he wrote, just as you were saying earlier, about
  pain and panic and confusion and loss, and the fact that mortality,
  death was coming closer and closer to him. And it struck me that, as a
  young man, he was all about yearning, as so many of us are, but once
  he really gave himself to this unflinching discipline, he was just
  going deep, deep, deep, deep into those dark places that you were
  talking about that you share with fellow writers in Esalen and
  elsewhere. And I think his two great gifts were intimacy and honesty.
  And that's why, even though he's left the world now, so many people I
  think feel him to be a very close friend and a real guide in difficult
  times.
\item
  cheryl strayed\\
  Yeah. And so Pico, I want to ask you, I know you've spent time with
  the Dalai Lama. And you traveled around Japan with him, is that
  correct?
\item
  pico iyer\\
  Yes. I mean, I've been lucky, I've known him since I was a teenager,
  and I wrote a book on him. And so since my book, I travel every year
  across Japan with him. And we traveled up to the tsunami area in Japan
  just a few months after 18,500 people had been killed in the tsunami.
  And we went to this little fishing village called Ishinomaki. And when
  we got there, it was a wasteland. It looked like pictures of Hiroshima
  after the war. And the only thing that survived, miraculously, was
  this little temple in the shelter of a mountain. And there were a few
  little schoolboys in school uniform, maybe five years old, who were
  waiting at the front and shaking his hand. And my wife told me that
  they were chosen because they were orphans. They'd just lost their
  parents. Now the whole community was looking after them. But they'd
  been given the honor of saying hello to the Dalai Lama. And then as he
  was sitting in the temple, he said, of course there's nothing I can
  give you except my heart and my presence. But he suddenly remembered a
  moment in March 1959 when he was in his hometown of Lhasa, Tibet, when
  he was told he had to leave his hometown that night. So no time to say
  goodbye to his friends, he had to leave then and there, and walk for
  two weeks across the highest mountains on the planet to get to safety
  in exile. And of course, as he said that, everyone there knew that,
  for 52 years at that point, he'd never been back to his homeland. It's
  now 60 years he's never been back to Tibet. And I think it was his way
  of saying, I can't imagine what you've been through, but loss is an
  equal opportunity employer. All of us have lost something, and we're
  all in this together. And that suffering is a fact of life, and
  actually the grace of it is that it helps us reach towards others. I
  have a great friend who's a Benedictine monk, and he sent this line to
  his friends last week, which is the best cure for anxiety is thinking
  of others. And I thought, thank heavens for monks. Doesn't matter if
  you're a Benedictine or not, that's really good advice. And maybe
  that's another of the possibilities that this moment holds for us,
  that we're not caught up in our own agendas, racing off to write this
  article, give this speech, do that meeting. We're suddenly alone with
  our thoughts, and our thoughts can turn to other people better.
\item
  cheryl strayed\\
  You said in one of your Ted Talks --- and I'm not remembering the
  story, so I want you to tell it to me again --- it was when you were
  traveling with the Dalai Lama, and you said to him each morning, what
  are we going to do, or what are we going to find, and he said, ``I
  don't know.'' Am I getting that right?
\item
  pico iyer\\
  Yes. And I think even more impressively, when he's in an auditorium
  with 30,000 people and, for example, somebody will put up a hand and
  ask a very serious question about either a marriage or children and
  child rearing, and he'll fool around laughing and say, ``You've asked
  the worst person in the world. I'm a celibate monk. I don't know a
  thing about marriage and relationships with a wife and kids. You know
  much more about that.'' And I think the first thing I'd say is, he has
  the courage to say that there's so many areas in the world he doesn't
  know about, and the humility to point out that we know many things
  about the world that the Dalai Lama doesn't. Well, you mentioned a Ted
  Talk. And one of the most moving I've ever seen was given by Pope
  Francis, who said --- the gist of it that I took away was, the more
  powerful you are, the more you have to be humble, and the more you
  have to essentially be kneeling before reality, before the future,
  before all the things that we know that we can't control. And so I
  love the fact that the Dalai Lama, who so many people seek out for
  wisdom, of which he has a huge amount, still will often say I don't
  know. Because he doesn't know what's happening tomorrow. He doesn't
  know what's happening tonight. He's a human being, as he would always
  stress. And so whenever people talk to him about his magical powers,
  he says, oh, I wish I had magical powers. If I did, I'd bring peace to
  Tibet and peace to the Middle East, and I wouldn't have a sore throat
  right now. And so don't look to me, the Dalai Lama, as somebody who's
  above it all. I, too, am surrendering. And he probably surrenders more
  than most. I think that's what distinguishes people like, probably,
  Pope Francis and Dalai Lama and the ones that we look to for wisdom
  --- Emily Dickinson, one of my great heroines --- because they're
  living even more than most of us in that state of surrender. And we
  think of them as all-knowing people, but the beauty of what they're
  carrying back to us is, I don't know. None of us knows.
\item
  cheryl strayed\\
  So listen, I love --- you and I have a lot of the same taste --- I
  love Emily Dickinson, too. But guess what I got to do last year? I got
  to go and write for an hour in her room where she wrote, where she
  lived and where she died. They told me not to touch her bed, but I was
  like a foot away from the bed where she died. And ---
\item
  pico iyer\\
  Can I ask what came out of that writing?
\item
  cheryl strayed\\
  You know, it was a radio show in England, I think it was the BBC was
  doing a thing on the life of Emily Dickinson and her legacy. And so
  they invited three writers. It was me, Sharon Olds and Ocean Vuong.
  And we each came at our separately appointed time slots and were left
  alone with the door closed in Emily Dickinson's bedroom. And I sat at
  her little desk, and I just looked --- I spent a lot of the time
  looking out the window, just trying to inhabit her, to see what she
  was seeing. And so I wrote a really kind of little piece about what I
  saw gazing out of Emily Dickinson's window. It's never left the pages
  of my journal, and probably never will, but it was a moving
  experience. It reminded me of how much we owe the people who came
  before us and did things that are still inspiring us today, and that
  we're part of that story, too. Part of impermanence isn't looking at
  other people and saying, well, they lived and they died and life goes
  on, it's also seeing that we, too, are going to live and die and life
  goes on.
\item
  pico iyer\\
  I am envious. If you can believe it, just this morning, before our
  conversation began, I was writing about her letters. I've always loved
  her poems, but as you know, I think when you read her letters, which
  are the most passionate love letters to her sister-in-law. And she
  would stand at the window you were describing and watch her
  sister-in-law walk to her brother's house and also get betrayed by the
  brother, and that --- just the image of the woman in her room with
  this huge passion that came out in words rather than touch. What
  haunting thing. I should tell you, I've been a couple of times to
  Thomas Merton's monastery, Gethsemane, in Kentucky. And the last time
  I was there, I met the one remaining monk there who studied under
  Thomas Merton and kind of carries the torch for him. And he walked me
  across the fields to Thomas Merton's hermitage, which is about 10
  minutes away from the monastery. And as he did, he said, I've got to
  tell you, for the last three years, I've been having a passionate love
  affair, and I've just completely lost my heart to this person. So of
  course, my ears pricked up, who would a monk be in love with? And you
  won't be surprised to hear that he said Emily Dickinson. She's the one
  I live for. I read her every day. And when we got to Thomas Merton's
  hermitage, this monk, who was in about late 70s, I would say, sat down
  and he started reciting Emily Dickinson poems, knowing that that was
  the truest way of keeping the legacy of literature. The legacy of
  solitude, the legacy of introspection. Everything that Thomas Merton
  represented, Emily Dickinson represents, too. And when we talk of
  impermanence, I earlier in this conversation said Leonard Cohen died,
  but everyone is listening to his songs and reading his poems still.
  And so what a beautiful permanence is all around us in art and music
  and poetry and nature. The blue skies.
\item
  cheryl strayed\\
  Right. That is the permanence, art and beauty, and loss and sorrow.
  Those are the permanent universal things.
\item
  pico iyer\\
  Yes.
\item
  cheryl strayed\\
  So one thing I want to ask you, Pico, I want to know if I have this
  right. You were born in England to parents who were born in India, and
  then you all moved to California when you were a kid. Is that correct?
\item
  pico iyer\\
  You have it absolutely right, Cheryl. Yes, I moved to California when
  I was about seven. But because of the exchange rates then, and also
  the Californian educational system was very different from the English
  one, I worked out it was cheaper to continue going to school in
  England and flying back three times a year to see my parents than to
  go to the local private school in California. So that's one of the
  reasons why I got into this traveling madness. I was commuting to
  school by plane alone over the North Pole from the age of nine, so I
  had to define home inwardly. And so I just have to sit here in my
  apartment in Japan talking to you in Portland, and I have in my mind's
  eye the view through the monastery window in Big Sur. And that's my
  home that's inside me and that grounds me, and that actually really
  consoles and inspires me at this moment of uncertainty.
\item
  cheryl strayed\\
  And your mother is in California, is that right?
\item
  pico iyer\\
  She is in California. She's been in hospital for a couple of weeks. I
  was all packed, I was set to fly over to see her, and then I found
  that visitors aren't allowed in hospitals in California right now, for
  very good reasons. And so again, I would be no closer to her if I was
  in her empty house 10 minutes by car from the hospital than I am right
  now, when I can still talk to her by phone and communicate.
\item
  cheryl strayed\\
  Right. And your mom, just to be clear, doesn't have the coronavirus.
  She's simply unwell. Is that right?
\item
  pico iyer\\
  That's right. She's safe from the virus. She's really just suffering
  from being 88 years old. And all being well, this time next week, I
  will be with her in her home because she's getting better, and I am
  planning to fly over to offer her moral support and company. So I'm
  hoping the end of that story is in sight and it's a happy ending.
\item
  cheryl strayed\\
  Oh, that's nice. I know so many people are really, really afraid for
  their elderly parents right now because we keep hearing over and over,
  right, that this virus is especially dangerous for people the older
  they are. Have you been particularly afraid for your mom during this
  time?
\item
  pico iyer\\
  Because her health is always uncertain and I know how frail she is, I
  was worried when she went to the hospital. And before the ambulance
  took her to the hospital, the paramedic very kindly called me in Japan
  and said, do you allow us to take your mother to the hospital, even
  though we're taking her into a place of maximal risk? And I thought I
  had to take that risk on on her behalf. And yes, she suffers from
  asthma. I suffer from asthma, my wife suffers from asthma. My wife and
  I are both in our 60s. We're all really in a state of risk. But I
  think the broader thing is, I'm sure you've found, it's one of the
  things that growing older brings us. I remember when I was in my 20s,
  I was really scared of dying. I thought, what will happen to my
  brilliant career, all the books I could write, all the places I could
  see? You know, silly, silly, silly, silly stuff. And as I got older, I
  realized what I'm most scared of is seeing my loved ones die or, if I
  died, what would happen to them. They would be, financially and in
  other ways, undefended. I'm much more concerned for my mother, and
  actually for my wife and for my kids, than I am for me, because that
  would be the real heartbreak, if my wife was left alone or if my
  mother were to see her son go before her. That's always one of the
  hardest things. And I'm grateful for the passing years for teaching me
  that it's really other people's deaths that we're most scared about,
  not our own.
\item
  cheryl strayed\\
  Right.
\item
  pico iyer\\
  And the other thing that had come to my mind --- again, we spoke about
  the Dalai Lama --- and in the last year or two, he's been talking a
  lot about what he calls emotional disarmament. And I take that to mean
  he thinks, of course, because he embodies it, the most important thing
  is to keep our hearts open, to feel for everything, and for one
  another. But he thinks that anger, panic, and confusion keep us away
  from reality, prevent us from making clear decisions. And some people
  are surprised that I, as a journalist, make sure I never take in more
  than five minutes of the news every day, but ideally two minutes of
  the news every day, or less. And I think that's because I don't want
  to be in a state of panic or confusion. It's really important to want
  to know what's going on in the world, and especially this moment, it's
  very important to know what we can and cannot do, what precautions we
  have to take. But I think you can do all of that in a couple of
  minutes in the day. And then for the rest of the day, the poems we've
  been discussing, the wise people we've been thinking about, our
  family, our friends, I think that's going to make us much better
  prepared for what's going to happen than keeping up with every last
  second of speculation and opinion and science we don't know what to do
  with.
\item
  cheryl strayed\\
  I think that that's such a fine distinction. Absorbing too much news
  can easily make us feel panicked, right? Absorbing enough that you
  have the information you need for clarity is what the goal is, but so
  many of us have a hard time shutting it off. And I know that, in my
  own life, that there is that tendency to think that, if I pay
  attention to everything that's being said about this virus or about
  these times or the economic fallout, that in some way I can maybe stop
  it from happening, or stop it from happening as bad for me. And yet,
  there is panic at the center of that, and a kind of lie at the center
  of that because, of course, we can't control it. When you were
  talking, and we've been talking so much about mortality, I was
  remembering this time, it was actually the last day of the year. It
  was December 31, 1991, so I was I was 23. And I was driving with the
  man who was then my husband. We were driving from Minneapolis up to
  this place called Madeline Island in Lake Superior, off the sort of
  shoreline of Wisconsin. And it was a snowy, icy day, as it often is.
  My ex-husband was driving, and I was in the passenger seat, and we
  lost control of the car. It just started to skid on the ice. We were
  going 55 miles an hour, and suddenly we were just sailing all over the
  highway. And it was terrifying. It was the only time in my life that I
  honestly thought I was going to die. And I was saying to my
  ex-husband, get control of the car, get control of the car, get
  control of the car over and over again, as we were just spinning out
  all over. And then there was this moment where the car just left the
  highway, like we just went airborne off the side of the road. And the
  ditch was far below, so we were flying. And we were going to crash.
  And I stopped saying get control of the car, and both my ex-husband
  and I at the same moment, at the very same moment when we hit the air,
  looked at each other and grabbed each other. And we said I love you at
  the same time. And then the car crashed. And we survived, but what
  shook me most deeply is that moment that we went, both of us,
  simultaneously, from panic to clarity. And inside that clarity was the
  only thing that matters: We just wanted to say I love you one more
  time to each other. And it wasn't scary. It wasn't panic. It was just
  the truth. And I've always thought that there's something about that
  those two things can sit so closely beside each other, can exist even
  within the same second, that there was something revealing about that.
  And maybe there's something in that that's relevant for now, too.
\item
  pico iyer\\
  I'm silenced by that. That is such a haunting, beautiful story. All of
  life is right in that moment. How many miracles in every day actually
  keep us going? We often notice only the terrible things that happen,
  and don't realize how many wonderful things are taken for granted. At
  the time of our forest fire, I thought, ``Oh, if only I'd had five
  more minutes, I'd have been able to save everything that's important
  to me.'' And then I thought, ``Well, if only I'd had five less
  minutes, I'd have been dead.'' I'm so glad you survived that accident,
  and even more, that you can share that story, because I think just
  having that story in our heads can turn our life in a better
  direction.
\item
  cheryl strayed\\
  Yeah. I got to see that. I got to live that. And I hope it's what I do
  the next time around, too. I hope so.
\item
  pico iyer\\
  Yeah, I mean it's kind of like a dress rehearsal. I often think it's
  one of the graces that we have as a writer, that we can use our, so to
  speak, job to help prepare us for these things, and to think through
  the inevitable stuff of life, which otherwise I find I'm much too
  ready to ignore.
\item
  cheryl strayed\\
  Indeed. Pico, it has been an absolute pleasure to talk to you. You've
  made me feel comforted and inspired and enlightened. So thank you.
  Thank you for speaking to me from all the way across these miles.
\item
  pico iyer\\
  Thank you for inviting me to be a part of this.
\item
  cheryl strayed\\
  Bye, Pico.
\item
  pico iyer\\
  Bye bye.
\item
  {[}music{]}
\item
  cheryl strayed\\
  I'm Cheryl Strayed. This is ``Sugar Calling.'' Next week, Amy Tan.
\end{itemize}

Previous

More episodes ofSugar Calling

\href{https://www.nytimes3xbfgragh.onion/2020/05/20/podcasts/sugar-calling-joy-harjo-poetry-virus.html?action=click\&module=audio-series-bar\&region=header\&pgtype=Article}{\includegraphics{https://static01.graylady3jvrrxbe.onion/images/2020/05/22/podcasts/20sugar-hajo3/20sugar-hajo3-thumbLarge.jpg}}

May 20, 2020~~•~ 35:30`I Release You, Fear'

\href{https://www.nytimes3xbfgragh.onion/2020/05/13/podcasts/sugar-calling-billy-collins-poetry-virus.html?action=click\&module=audio-series-bar\&region=header\&pgtype=Article}{\includegraphics{https://static01.graylady3jvrrxbe.onion/images/2020/05/13/podcasts/13sugar-calling/13sugar-calling-thumbLarge.jpg}}

May 13, 2020`There's a Quiet All Over the World'

\href{https://www.nytimes3xbfgragh.onion/2020/05/06/podcasts/sugar-calling-alice-walker-quarantine-virus.html?action=click\&module=audio-series-bar\&region=header\&pgtype=Article}{\includegraphics{https://static01.graylady3jvrrxbe.onion/images/2020/05/06/podcasts/06sugarcalling/06sugarcalling-thumbLarge.jpg}}

May 6, 2020~~•~ 28:58`Whatever We Have, We Have to Work With It'

\href{https://www.nytimes3xbfgragh.onion/2020/04/29/podcasts/sugar-calling-judy-blume-quarantine-virus.html?action=click\&module=audio-series-bar\&region=header\&pgtype=Article}{\includegraphics{https://static01.graylady3jvrrxbe.onion/images/2020/04/29/podcasts/29sugarcalliing-blume-sub/29sugarcalliing-blume-sub-thumbLarge.jpg}}

April 29, 2020`This Terrible Thing Is Happening, but the World Goes On.'

\href{https://www.nytimes3xbfgragh.onion/2020/04/22/podcasts/sugar-calling-amy-tan-quarantine-virus.html?action=click\&module=audio-series-bar\&region=header\&pgtype=Article}{\includegraphics{https://static01.graylady3jvrrxbe.onion/images/2020/04/27/podcasts/22sugarcalling/22sugarcalling-thumbLarge.jpg}}

April 22, 2020~~•~ 39:19`You Don't Take Dictation. You Find the Truth.'

\href{https://www.nytimes3xbfgragh.onion/2020/04/15/podcasts/sugar-calling-pico-iyer-coronavirus.html?action=click\&module=audio-series-bar\&region=header\&pgtype=Article}{\includegraphics{https://static01.graylady3jvrrxbe.onion/images/2020/04/21/podcasts/15sugarcalling1/15sugarcalling1-thumbLarge.jpg}}

April 15, 2020~~•~ 35:45`Joyful Participation in a World of Sorrows'

\href{https://www.nytimes3xbfgragh.onion/2020/04/08/podcasts/sugar-calling-margaret-atwood-coronavirus.html?action=click\&module=audio-series-bar\&region=header\&pgtype=Article}{\includegraphics{https://static01.graylady3jvrrxbe.onion/images/2020/04/02/books/08sugarcalling1/08sugarcalling1-thumbLarge-v3.jpg}}

April 8, 2020~~•~ 34:32`Roll Up Your Sleeves, Girls'

\href{https://www.nytimes3xbfgragh.onion/2020/04/03/podcasts/sugar-calling-george-saunders-coronavirus.html?action=click\&module=audio-series-bar\&region=header\&pgtype=Article}{\includegraphics{https://static01.graylady3jvrrxbe.onion/images/2020/04/09/podcasts/03sugarcalling-image/merlin_171264408_4ac7fc67-d8cc-45b9-9ec6-bdd20672e694-thumbLarge.jpg}}

April 3, 2020~~•~ 41:16`Everything Is Always Keep Changing'

\href{https://www.nytimes3xbfgragh.onion/column/sugar-calling}{See All
Episodes ofSugar Calling}

Next

Published April 15, 2020Updated April 29, 2020

\begin{itemize}
\item
\item
\item
\item
\item
\end{itemize}

\emph{\textbf{Listen and subscribe to our podcast from your mobile
device:}}\\
\textbf{\href{https://podcasts.apple.com/us/podcast/sugar-calling/id1505881384}{\emph{Via
Apple Podcasts}}} \emph{\textbf{\textbar{}}}
\textbf{\href{https://open.spotify.com/show/4U8hPiNGIBvTS9zLeiDCN7?si=gRyigD47SPWl-QWgNjgt2w}{\emph{Via
Spotify}}} \emph{\textbf{\textbar{}}}
\textbf{\href{https://www.stitcher.com/podcast/the-new-york-times/sugar-calling}{\emph{Via
Stitcher}}}

\hypertarget{the-most-important-thing-is-to-keep-our-hearts-open-to-feel-for-everything-and-for-one-another}{%
\subsection{`The most important thing is to keep our hearts open, to
feel for everything, and for one
another.'}\label{the-most-important-thing-is-to-keep-our-hearts-open-to-feel-for-everything-and-for-one-another}}

\emph{--- Pico Iyer, author}

Today, Cheryl calls Pico Iyer, the travel writer and novelist, at his
apartment in Japan. Pico tells the story of losing his home to a forest
fire in the early '90s, and the unexpected lessons in that loss. ``I
feel we have much more power over how we respond to the world than we
suspect,'' he tells Cheryl.

Cheryl asks Pico about his travels with the Dalai Lama, and the two
discuss coming to grips with impermanence --- via the fleeting beauty of
a cherry blossom.

\includegraphics{https://static01.graylady3jvrrxbe.onion/images/2020/04/21/podcasts/15sugarcalling1/15sugarcalling1-articleLarge.jpg?quality=75\&auto=webp\&disable=upscale}

\hypertarget{on-todays-episode}{%
\subsubsection{\texorpdfstring{\textbf{On today's
episode:}}{On today's episode:}}\label{on-todays-episode}}

\href{https://picoiyerjourneys.com/}{Pico Iyer} is a
\href{https://www.nytimes3xbfgragh.onion/2019/04/22/books/review/pico-iyer-autumn-light-memoir-japan.html}{journalist,
memoirist and travel writer extraordinaire}. Born in England and raised
in California, Pico wrote for Time magazine in New York before moving to
Japan in his 20s. His essays have been featured in the
``\href{https://www.amazon.com/Best-American-Travel-Writing-2004/dp/0618341269}{Best
American Travel Writing}'' and
``\href{https://www.amazon.com/dp/B0036Q4M7W/ref=dp-kindle-redirect?_encoding=UTF8\&btkr=1}{Best
American Spiritual Writing}'' anthologies, and he is a
\href{https://www.nytimes3xbfgragh.onion/2019/09/20/opinion/aging-marriage-autumn.html}{regular}
\href{https://www.nytimes3xbfgragh.onion/2012/01/01/opinion/sunday/the-joy-of-quiet.html}{contributor}
to
\href{https://www.nytimes3xbfgragh.onion/2016/12/31/opinion/sunday/what-do-we-know.html}{The
New York Times's Opinion} section.

\hypertarget{picos-five-books-to-keep-you-company-in-uncertain-times}{%
\subsubsection{\texorpdfstring{\textbf{Pico's five books to keep you
company in uncertain
times:}}{Pico's five books to keep you company in uncertain times:}}\label{picos-five-books-to-keep-you-company-in-uncertain-times}}

\begin{itemize}
\tightlist
\item
  \emph{``}\href{https://www.amazon.com/Etty-Hillesum-Interrupted-1941-1943-Westerbork/dp/0805050876}{An
  Interrupted Life},'' Etty Hillesum
\end{itemize}

\begin{itemize}
\item
  ``\href{https://www.amazon.com/Too-Much-Happiness-Vintage-International-ebook/dp/B002VT6C0K/ref=sr_1_1?dchild=1\&keywords=too+much+happiness\&qid=1586895086\&s=books\&sr=1-1}{Too
  Much Happiness},'' Alice Munro
\item
  ``\href{https://www.amazon.com/Alone-Classic-Adventure-Richard-Byrd-ebook/dp/B004HD4A1Q/ref=sr_1_1?dchild=1\&keywords=alone+admiral+byrd\&qid=1586895161\&s=books\&sr=1-1}{Alone},''
  Admiral Richard E. Byrd
\item
  ``\href{https://www.amazon.com/Olive-Again-Novel-Elizabeth-Strout-ebook/dp/B07NCPLS2R/ref=sr_1_1?dchild=1\&keywords=olive\%2C+again\&qid=1586895199\&s=books\&sr=1-1}{Olive,
  Again},'' Elizabeth Strout
\item
  ``\href{https://www.amazon.com/Zen-Mind-Beginners-Informal-Meditation-ebook/dp/B00I8USOM0/ref=sr_1_2?crid=ZC4QSZUMV9AO\&dchild=1\&keywords=zen+mind\%2C+beginner\%27s+mind\&qid=1586895238\&s=books\&sprefix=zen+mind\%2C+be\%2Cstripbooks\%2C149\&sr=1-2}{Zen
  Mind, Beginner's Mind},'' Shunryu Suzuki
\end{itemize}

\begin{center}\rule{0.5\linewidth}{\linethickness}\end{center}

\begin{center}\rule{0.5\linewidth}{\linethickness}\end{center}

Cheryl Strayed is the author of ``Tiny Beautiful Things,'' ``Torch,''
``Brave Enough,'' and the New York Times best seller ``Wild.'' Her books
have been translated into more than 40 languages. She lives in Portland,
Ore.
\href{https://twitter.com/CherylStrayed?ref_src=twsrc\%5Egoogle\%7Ctwcamp\%5Eserp\%7Ctwgr\%5Eauthor}{@CherylStrayed}

``Sugar Calling'' is produced by Kelly Prime and edited by Sara
Sarasohn, with editorial oversight by Wendy Dorr. This episode was mixed
by Jamie Collazo. Our theme music is by Dan Powell.

Advertisement

\protect\hyperlink{after-bottom}{Continue reading the main story}

\hypertarget{site-index}{%
\subsection{Site Index}\label{site-index}}

\hypertarget{site-information-navigation}{%
\subsection{Site Information
Navigation}\label{site-information-navigation}}

\begin{itemize}
\tightlist
\item
  \href{https://help.nytimes3xbfgragh.onion/hc/en-us/articles/115014792127-Copyright-notice}{©~2020~The
  New York Times Company}
\end{itemize}

\begin{itemize}
\tightlist
\item
  \href{https://www.nytco.com/}{NYTCo}
\item
  \href{https://help.nytimes3xbfgragh.onion/hc/en-us/articles/115015385887-Contact-Us}{Contact
  Us}
\item
  \href{https://www.nytco.com/careers/}{Work with us}
\item
  \href{https://nytmediakit.com/}{Advertise}
\item
  \href{http://www.tbrandstudio.com/}{T Brand Studio}
\item
  \href{https://www.nytimes3xbfgragh.onion/privacy/cookie-policy\#how-do-i-manage-trackers}{Your
  Ad Choices}
\item
  \href{https://www.nytimes3xbfgragh.onion/privacy}{Privacy}
\item
  \href{https://help.nytimes3xbfgragh.onion/hc/en-us/articles/115014893428-Terms-of-service}{Terms
  of Service}
\item
  \href{https://help.nytimes3xbfgragh.onion/hc/en-us/articles/115014893968-Terms-of-sale}{Terms
  of Sale}
\item
  \href{https://spiderbites.nytimes3xbfgragh.onion}{Site Map}
\item
  \href{https://help.nytimes3xbfgragh.onion/hc/en-us}{Help}
\item
  \href{https://www.nytimes3xbfgragh.onion/subscription?campaignId=37WXW}{Subscriptions}
\end{itemize}
