Sections

SEARCH

\protect\hyperlink{site-content}{Skip to
content}\protect\hyperlink{site-index}{Skip to site index}

\href{https://www.nytimes3xbfgragh.onion/section/technology}{Technology}

\href{https://myaccount.nytimes3xbfgragh.onion/auth/login?response_type=cookie\&client_id=vi}{}

\href{https://www.nytimes3xbfgragh.onion/section/todayspaper}{Today's
Paper}

\href{/section/technology}{Technology}\textbar{}John Horton Conway, a
`Magical Genius' in Math, Dies at 82

\url{https://nyti.ms/3epJ7dz}

\begin{itemize}
\item
\item
\item
\item
\item
\item
\end{itemize}

\href{https://www.nytimes3xbfgragh.onion/news-event/coronavirus?action=click\&pgtype=Article\&state=default\&region=TOP_BANNER\&context=storylines_menu}{The
Coronavirus Outbreak}

\begin{itemize}
\tightlist
\item
  live\href{https://www.nytimes3xbfgragh.onion/2020/08/04/world/coronavirus-covid-19.html?action=click\&pgtype=Article\&state=default\&region=TOP_BANNER\&context=storylines_menu}{Latest
  Updates}
\item
  \href{https://www.nytimes3xbfgragh.onion/interactive/2020/us/coronavirus-us-cases.html?action=click\&pgtype=Article\&state=default\&region=TOP_BANNER\&context=storylines_menu}{Maps
  and Cases}
\item
  \href{https://www.nytimes3xbfgragh.onion/interactive/2020/science/coronavirus-vaccine-tracker.html?action=click\&pgtype=Article\&state=default\&region=TOP_BANNER\&context=storylines_menu}{Vaccine
  Tracker}
\item
  \href{https://www.nytimes3xbfgragh.onion/2020/08/02/us/covid-college-reopening.html?action=click\&pgtype=Article\&state=default\&region=TOP_BANNER\&context=storylines_menu}{College
  Reopening}
\item
  \href{https://www.nytimes3xbfgragh.onion/live/2020/08/03/business/stock-market-today-coronavirus?action=click\&pgtype=Article\&state=default\&region=TOP_BANNER\&context=storylines_menu}{Economy}
\end{itemize}

Advertisement

\protect\hyperlink{after-top}{Continue reading the main story}

Supported by

\protect\hyperlink{after-sponsor}{Continue reading the main story}

those we've lost

\hypertarget{john-horton-conway-a-magical-genius-in-math-dies-at-82}{%
\section{John Horton Conway, a `Magical Genius' in Math, Dies at
82}\label{john-horton-conway-a-magical-genius-in-math-dies-at-82}}

He made profound contributions to number theory, coding theory,
probability theory, topology, algebra and more --- and created games
from it all. He died of the coronavirus.

\includegraphics{https://static01.graylady3jvrrxbe.onion/images/2020/04/18/obituaries/14Conway1/merlin_171560415_a372e14d-1b58-4837-bc68-e474efc320a0-articleLarge.jpg?quality=75\&auto=webp\&disable=upscale}

By Siobhan Roberts

\begin{itemize}
\item
  Published April 15, 2020Updated April 17, 2020
\item
  \begin{itemize}
  \item
  \item
  \item
  \item
  \item
  \item
  \end{itemize}
\end{itemize}

\emph{This obituary is part of a series about people who have died in
the coronavirus pandemic. Read about others}
\href{https://www.nytimes3xbfgragh.onion/series/people-who-have-died-of-the-coronavirus}{\emph{here}}\emph{.}

John Horton Conway, the English-born Princeton mathematician whose body
of work ranged from the rigorously highbrow to the frivolously fun,
earning him prizes and a reputation as a creative, iconoclastic and even
magical genius, died on Saturday in New Brunswick, N.J. He was 82.

His wife, Diana Conway, said his death, at a nursing home, was caused by
Covid-19.

Dr. Conway's boundless curiosity produced profound contributions to
number theory, game theory, coding theory, group theory, knot theory,
topology, probability theory, algebra, analysis, combinatorics and more.
Foremost, he considered himself a classical geometer.

``His swath was probably broader than anyone who ever lived,'' said the
mathematician Neil Sloane, a collaborator with Dr. Conway and the
founder of the \href{https://oeis.org/}{On-Line Encyclopedia of Integer
Sequences}. ``I've worked with a lot of people, and he was the fastest
at solving a problem and would pursue a topic as far as it would go.''
(The two were co-authors of 50 papers and published the 706-page book
``Sphere Packings, Lattices and Groups.'')

During what Dr. Conway called his ``annus mirabilis,'' roughly 1969 to
1970, he discovered what's known as the Conway group, an entity in the
realm of mathematical symmetry that inhabits 24-dimensional space. He
discovered a new type of number, ``surreal numbers.'' And he invented
the cellular automaton Game of Life, which is among the most beautiful
mathematical models of computation. He described it as a ``no-player
never-ending'' game.

\includegraphics{https://static01.graylady3jvrrxbe.onion/images/2020/04/15/obituaries/15conway-gif/15conway-gif-articleLarge-v3.gif?quality=75\&auto=webp\&disable=upscale}

His friend
\href{https://www.nytimes3xbfgragh.onion/2010/05/24/us/24gardner.html}{Martin
Gardner}, the longtime mathematical games columnist for Scientific
American, called the Game of Life Dr. Conway's ``most famous
brainchild.'' He reckoned that at the game's peak of popularity --- with
users programming it at home and at work --- one quarter of the world's
computers were playing it.

``Conway's LIFE changed mine,'' the musician Brian Eno said in an email.
``I think Conway himself thought it rather trivial, but for a
nonmathematician like me, it was a shock to the intuition, a shattering
revelation --- to watch glorious complexity emerging from staid
simplicity.''

Dr. Conway was proudest of his discovery of surreal numbers. (The
Stanford computer scientist Donald Knuth had come up with the name while
writing the novelette ``Surreal Numbers: How Two Ex-Students Turned on
to Pure Mathematics and Found Total Happiness.'')

Described by Mr. Gardner as ``an astonishing feat of legerdemain,'' the
surreals are a super-continuum of numbers, including all the
old-fashioned real ones (integers, fractions and irrationals like pi) as
well as those that go above, beyond, below and within, embracing both
the infinites and the infinitesimals.

Dr. Conway always hoped that surreal numbers might find practical
applications, perhaps in helping to illuminate the universe on the
cosmic and quantum scales.

One of Dr. Conway's favorite accomplishments was the Free Will Theorem,
conceptualized casually over the course of a decade with his friend and
fellow Princeton mathematician Simon Kochen and first published in 2006
(and later revised).

The theorem, simply put, is this: If physicists have free will while
performing experiments, then elementary particles possess free will as
well. And this, Dr. Conway and Dr. Kochen surmised, probably explains
why and how humans have free will in the first place.

``In mathematics and physics there are two kinds of geniuses,'' Dr.
Kochen said by phone from his home in Princeton, echoing something once
said about the physicist Richard Feynman. ``There are the ordinary
geniuses --- they are just like you and me but they are better at it; if
we'd worked hard enough, maybe we could get some of the same results.

``But then there are the magical geniuses,'' he added. ``Richard Feynman
was a magical genius. And the same always struck me about John --- he
was a magical mathematician. He was a magical genius rather than an
ordinary genius.''

Image

Credit...Denise Applewhite/Princeton University

John Horton Conway was born on Dec. 26, 1937, in Liverpool, England, the
third child and only son of Cyril and Agnes (Boyce) Conway. His father,
an autodidact, had left school at age 14 and, with his photographic
memory, made a living playing cards. Later he was a technician in the
chemistry lab at the Liverpool Institute High School for Boys, setting
up experiments for students, among them George Harrison and Paul
McCartney.

Dr. Conway's mother, a great reader, especially of Dickens, had worked
from age 11. Family lore has it that she boasted about finding her son
at age of 4 reciting the powers of two. At 18, in 1956, John left home
for the University of Cambridge, where he earned his Ph.D. His adviser,
Harold Davenport, a number theorist, once said that when he would give
Dr. Conway a problem to solve, ``he would return with a very good
solution to another problem.''

As a student, Dr. Conway cultivated his acknowledged lifelong preference
for being lazy, playing games and doing no work. He could be easily
distracted by what he called ``nerdish delights.'' He once went on a
flexagon binge, courtesy of Mr. Gardner, who described flexagons as
``polygons, folded from straight or crooked strips of paper, which have
the fascinating property of changing their faces when they are flexed.''

He built a water-powered computer, which he called Winnie (Water
Initiated Nonchalantly Numerical Integrating Engine). He read and
annotated H.S.M. Coxeter's edition of W.W. Rouse Ball's classic work,
``Mathematical Recreations and Essays,'' and wrote Coxeter a lengthy
letter that started a lifelong friendship between these two classical
geometers.

Hired at Cambridge as an assistant lecturer, Dr. Conway gained a
reputation for his high jinks (not to mention his disheveled
appearance). Lecturing on symmetry and the Platonic solids, he might
bring in a turnip as a prop, carving it one slice at a time into, say,
an \href{https://mathworld.wolfram.com/Icosahedron.html}{icosahedron},
with its 20 triangular faces, eating the scraps as he went. ``He was by
far the most charismatic lecturer in the faculty,'' his Cambridge
colleague
\href{https://www.theguardian.com/science/2019/jan/09/sir-peter-swinnerton-dyer-obituary}{Peter
Swinnerton-Dyer} once said.

Dr. Conway invented a profusion of games --- like Phutball (short for
Philosopher's Football, which is a little like checkers on a Go board)
and collected them in the book ``Winning Ways for Your Mathematical
Plays,'' in collaboration with Elwyn Berlekamp and Richard Guy.

All the gaming was supported by a loyal following of graduate students,
among them Simon Norton, with whom Dr. Conway published the Monstrous
Moonshine conjecture, investigating an elusive symmetry group that lives
in 196,883 dimensions. (His Ph.D. student Richard Borcherds received the
prestigious Fields Medal in 1998 for his proof of the conjecture.)

At the University of Cambridge Dr. Conway rose to become a professor in
mathematics as well as a supernumerary fellow at Gonville and Caius
College, his alma mater there. He was named a fellow of the Royal
Society in 1981.

In 1985, with four co-authors, he published ``The ATLAS of Finite
Groups,'' one of the most important books in group theory.

That same year, he was invited to give a talk at Princeton, and a job
offer followed: In 1987, he took up the position of the John von Neumann
professor of applied and computational mathematics. In announcing the
hire, Princeton's president called Dr. Conway ``one of the most eminent
mathematicians of the century.''

At Princeton Dr. Conway, with his mischievous and seductive aura, drew
news media attention. Asked by a reporter for The New York Times about
his life of the mind, he replied: ``What happens most of the time is
nothing. You just can't have ideas often.''

He became a fellow of the American Academy of Arts and Sciences in 1992.
A fellow inductee, the mathematician Robert MacPherson, recalled that at
the ceremony Dr. Conway accepted his honor in what appeared to be green
running shorts.

His first two marriages, to Eileen Howe and Larissa Queen, ended in
divorce.

In addition to his wife, he is survived by four daughters from his first
marriage, Annie, Ellie and Susie Conway and Rosie Wayman; two sons from
his second marriage, Oliver and Alex; a son with Ms. Conway, Gareth;
three grandchildren; and six great-grandchildren.

At Princeton he was almost invariably recruited to give the first-year
course intended to persuade students to become math majors. And he
offered extracurricular content, like a campus tour titled ``How to
Stare at a Brick Wall.''

He gave over his summers --- prime research time --- to teaching at math
camps. He was a star attraction, despite the fact that his talks were
advertised vaguely as ``John Conway Hour, NTBA'' (Not to Be Announced).
He would take topic requests from students and deliver an extemporaneous
lecture.

Math, Dr. Conway believed, should be fun. ``He often thought that the
math we were teaching was too serious,'' said Mira Bernstein, a
mathematician and a former executive director of Canada/USA Mathcamp, an
international summer program for high-school students. ``And he didn't
mean that we should be teaching them silly math --- to him, fun was
deep. But he wanted to make sure that the playfulness was always, always
there.''

Dr. Conway persevered in finding the fun through triple bypass surgery,
a suicide attempt and a number of strokes. Sometimes he would regale
anyone willing to listen on the science of rainbows or on his Doomsday
rule for calculating the day of the week for any given date.

And there were ever more games of Phutball, which Dr. Conway was not
very good at. Sometimes, when all seemed lost --- when he was almost
certainly beaten at his own game, though he might yet magically prevail
--- he'd delight in borrowing from Mark Twain, admonishing his
opponents, ``Reports of my death have been greatly exaggerated!''

Siobhan Roberts is the author of ``Genius at Play: The Curious Mind of
John Horton Conway'' (2015).

\href{https://www.nytimes3xbfgragh.onion/interactive/2020/obituaries/people-died-coronavirus-obituaries.html?action=click\&pgtype=Article\&state=default\&region=BELOW_MAIN_CONTENT\&context=covid_obits_promo}{}

\hypertarget{those-weve-lost}{%
\section{Those We've Lost}\label{those-weve-lost}}

The coronavirus pandemic has taken an incalculable death toll. This
series is designed to put names and faces to the numbers.

Read more

\includegraphics{https://static01.graylady3jvrrxbe.onion/images/2020/07/30/obituaries/30Pedro/30Pedro-square640.jpg}

\hypertarget{bernaldina-josuxe9-pedro}{%
\section{Bernaldina José Pedro}\label{bernaldina-josuxe9-pedro}}

d. Boa Vista, Brazil

Leader among the Indigenous Macuxi

\includegraphics{https://static01.graylady3jvrrxbe.onion/images/2020/07/31/obituaries/31Swing/merlin_175167783_8913bc90-0d64-43f3-a655-1bb1bf1601c9-square640.jpg}

\hypertarget{john-eric-swing}{%
\section{John Eric Swing}\label{john-eric-swing}}

d. Fountain Valley, Calif.

Champion of Filipino-Americans

\includegraphics{https://static01.graylady3jvrrxbe.onion/images/2020/07/27/obituaries/27Victor/merlin_175001436_38b11f8e-227a-4e2c-9821-7618af9b2524-square640.jpg}

\hypertarget{victor-victor}{%
\section{Victor Victor}\label{victor-victor}}

d. Santo Domingo, Dominican Republic

Beloved musician of the Dominican Republic

\includegraphics{https://static01.graylady3jvrrxbe.onion/images/2020/07/31/obituaries/31Negron/merlin_175160169_516322ae-fd23-4969-b6b2-193ced371105-square640.jpg}

\hypertarget{dr-eddie-negruxf3n}{%
\section{Dr. Eddie Negrón}\label{dr-eddie-negruxf3n}}

d. Fort Walton Beach, Fla.

Internist on Florida's Emerald Coast

\includegraphics{https://static01.graylady3jvrrxbe.onion/images/2020/07/30/obituaries/30Dobson/merlin_175115928_f6b9271c-8f05-4fe1-a38a-5ca4a58f8935-square640.jpg}

\hypertarget{dobby-dobson}{%
\section{Dobby Dobson}\label{dobby-dobson}}

d. Coral Springs, Fla.

Jamaican singer and songwriter

\includegraphics{https://static01.graylady3jvrrxbe.onion/images/2020/08/01/obituaries/28Gonzalez/merlin_175002771_beb57888-3951-409a-ae13-03a94b2e962e-square640.jpg}

\hypertarget{waldemar-gonzalez}{%
\section{Waldemar Gonzalez}\label{waldemar-gonzalez}}

d. White Plains, N.Y.

Teacher and social worker

Advertisement

\protect\hyperlink{after-bottom}{Continue reading the main story}

\hypertarget{site-index}{%
\subsection{Site Index}\label{site-index}}

\hypertarget{site-information-navigation}{%
\subsection{Site Information
Navigation}\label{site-information-navigation}}

\begin{itemize}
\tightlist
\item
  \href{https://help.nytimes3xbfgragh.onion/hc/en-us/articles/115014792127-Copyright-notice}{©~2020~The
  New York Times Company}
\end{itemize}

\begin{itemize}
\tightlist
\item
  \href{https://www.nytco.com/}{NYTCo}
\item
  \href{https://help.nytimes3xbfgragh.onion/hc/en-us/articles/115015385887-Contact-Us}{Contact
  Us}
\item
  \href{https://www.nytco.com/careers/}{Work with us}
\item
  \href{https://nytmediakit.com/}{Advertise}
\item
  \href{http://www.tbrandstudio.com/}{T Brand Studio}
\item
  \href{https://www.nytimes3xbfgragh.onion/privacy/cookie-policy\#how-do-i-manage-trackers}{Your
  Ad Choices}
\item
  \href{https://www.nytimes3xbfgragh.onion/privacy}{Privacy}
\item
  \href{https://help.nytimes3xbfgragh.onion/hc/en-us/articles/115014893428-Terms-of-service}{Terms
  of Service}
\item
  \href{https://help.nytimes3xbfgragh.onion/hc/en-us/articles/115014893968-Terms-of-sale}{Terms
  of Sale}
\item
  \href{https://spiderbites.nytimes3xbfgragh.onion}{Site Map}
\item
  \href{https://help.nytimes3xbfgragh.onion/hc/en-us}{Help}
\item
  \href{https://www.nytimes3xbfgragh.onion/subscription?campaignId=37WXW}{Subscriptions}
\end{itemize}
