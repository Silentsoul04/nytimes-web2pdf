Sections

SEARCH

\protect\hyperlink{site-content}{Skip to
content}\protect\hyperlink{site-index}{Skip to site index}

\href{https://www.nytimes3xbfgragh.onion/section/climate}{Climate}

\href{https://myaccount.nytimes3xbfgragh.onion/auth/login?response_type=cookie\&client_id=vi}{}

\href{https://www.nytimes3xbfgragh.onion/section/todayspaper}{Today's
Paper}

\href{/section/climate}{Climate}\textbar{}E.P.A. Weakens Controls on
Mercury

\url{https://nyti.ms/2z7lkPt}

\begin{itemize}
\item
\item
\item
\item
\item
\item
\end{itemize}

\hypertarget{climate-and-environment}{%
\subsubsection{\texorpdfstring{\href{https://www.nytimes3xbfgragh.onion/section/climate?name=styln-climate\&region=TOP_BANNER\&variant=undefined\&block=storyline_menu_recirc\&action=click\&pgtype=Article\&impression_id=f7149e50-e394-11ea-af8a-cf9df229f915}{Climate
and
Environment}}{Climate and Environment}}\label{climate-and-environment}}

\begin{itemize}
\tightlist
\item
  \href{https://www.nytimes3xbfgragh.onion/2020/08/17/climate/alaska-oil-drilling-anwr.html?name=styln-climate\&region=TOP_BANNER\&variant=undefined\&block=storyline_menu_recirc\&action=click\&pgtype=Article\&impression_id=f7149e51-e394-11ea-af8a-cf9df229f915}{Arctic
  Refuge}
\item
  \href{https://www.nytimes3xbfgragh.onion/interactive/2020/climate/trump-environment-rollbacks.html?name=styln-climate\&region=TOP_BANNER\&variant=undefined\&block=storyline_menu_recirc\&action=click\&pgtype=Article\&impression_id=f7149e52-e394-11ea-af8a-cf9df229f915}{Trump's
  Changes}
\item
  \href{https://www.nytimes3xbfgragh.onion/interactive/2020/04/19/climate/climate-crash-course-1.html?name=styln-climate\&region=TOP_BANNER\&variant=undefined\&block=storyline_menu_recirc\&action=click\&pgtype=Article\&impression_id=f7149e53-e394-11ea-af8a-cf9df229f915}{Climate
  101}
\item
  \href{https://www.nytimes3xbfgragh.onion/interactive/2018/08/30/climate/how-much-hotter-is-your-hometown.html?name=styln-climate\&region=TOP_BANNER\&variant=undefined\&block=storyline_menu_recirc\&action=click\&pgtype=Article\&impression_id=f7149e54-e394-11ea-af8a-cf9df229f915}{Is
  Your Hometown Hotter?}
\end{itemize}

Advertisement

\protect\hyperlink{after-top}{Continue reading the main story}

Supported by

\protect\hyperlink{after-sponsor}{Continue reading the main story}

\hypertarget{epa-weakens-controls-on-mercury}{%
\section{E.P.A. Weakens Controls on
Mercury}\label{epa-weakens-controls-on-mercury}}

The agency is changing the way it calculates the benefits of mercury
controls, a move that would effectively loosen the rules on other toxic
pollutants.

\includegraphics{https://static01.graylady3jvrrxbe.onion/images/2020/01/22/climate/00CLI-MERCURY-HFO1/00CLI-MERCURY-HFO1-articleLarge-v2.jpg?quality=75\&auto=webp\&disable=upscale}

\href{https://www.nytimes3xbfgragh.onion/by/lisa-friedman}{\includegraphics{https://static01.graylady3jvrrxbe.onion/images/2018/07/18/multimedia/author-lisa-friedman/author-lisa-friedman-thumbLarge.png}}\href{https://www.nytimes3xbfgragh.onion/by/coral-davenport}{\includegraphics{https://static01.graylady3jvrrxbe.onion/images/2018/10/03/multimedia/author-coral-davenport/author-coral-davenport-thumbLarge-v2.png}}

By \href{https://www.nytimes3xbfgragh.onion/by/lisa-friedman}{Lisa
Friedman} and
\href{https://www.nytimes3xbfgragh.onion/by/coral-davenport}{Coral
Davenport}

\begin{itemize}
\item
  April 16, 2020
\item
  \begin{itemize}
  \item
  \item
  \item
  \item
  \item
  \item
  \end{itemize}
\end{itemize}

WASHINGTON --- The Trump administration on Thursday weakened regulations
on the release of mercury and other toxic metals from oil and coal-fired
power plants, another step toward rolling back health protections in the
middle of a pandemic.

The new Environmental Protection Agency rule does not eliminate
restrictions on the release of mercury, a heavy metal linked to brain
damage. Instead, it creates a new method of calculating the costs and
benefits of curbing mercury pollution that environmental lawyers said
would fundamentally undermine the legal underpinnings of controls on
mercury and many other pollutants.

By reducing the positive health effects of regulations on paper and
raising their economic costs, the new method could be used to justify
loosening restrictions on any pollutant that the fossil fuel industry
has deemed too costly to control.

``That is the big unstated goal,'' said David Konisky, a professor of
public and environmental affairs at Indiana University. ``This is less
about mercury than about potentially constraining or handcuffing future
efforts by the E.P.A. to regulate air pollution.''

The proposed change is the latest in the Trump administration's
long-running effort to roll back environmental regulations and reduce
regulatory burdens, particularly on the coal, oil and gas industries.
Over the past three years the administration has weakened rules to cut
planet-warming carbon dioxide emissions from coal-fired power plants,
restrict coal companies from dumping debris in streams and claimed
falsely that that President Trump has revived the dying coal industry.

Over the past few weeks as the nation struggled with the coronavirus,
the administration has also rushed to loosen
curbs\href{https://www.nytimes3xbfgragh.onion/2020/03/30/climate/trump-fuel-economy.html}{on
automobile tailpipe emissions}, opted not to strengthen a regulation
on\href{https://www.nytimes3xbfgragh.onion/2020/04/14/climate/coronavirus-soot-clean-air-regulations.html}{industrial
soot emissions} and moved to drop the threat of punishment to
\href{https://www.nytimes3xbfgragh.onion/2020/01/30/climate/trump-bird-deaths.html}{companies
that kill birds ``incidentally.''}

The deregulatory push appears designed to secure less restrictive rules
quickly, in case Republicans lose control of Congress and the White
House in November. A new government could move quickly under the
Congressional Review Act to overturn any regulation or federal rule
within 60 days of it being finalized --- making any rule completed after
late May or early June vulnerable.

The mercury rollback is a particular victory for Robert E. Murray, the
former chief executive of Murray Energy Corporation and a top
fund-raiser for Mr. Trump. Mr. Murray personally requested the rollback
in a
written\href{https://www.nytimes3xbfgragh.onion/2018/01/09/climate/coal-murray-trump-memo.html}{``wish
list''} to top officials shortly after the president took office. The
company has since declared bankruptcy and is undergoing a
reorganization.

``Under this action, no more mercury will be emitted into the air than
before,'' Andrew R. Wheeler, the E.P.A. administrator said in announcing
the rule.

\href{https://www.nytimes3xbfgragh.onion/section/climate?action=click\&pgtype=Article\&state=default\&region=MAIN_CONTENT_1\&context=storylines_keepup}{}

\hypertarget{climate-and-environment-}{%
\subsubsection{Climate and Environment
›}\label{climate-and-environment-}}

\hypertarget{keep-up-on-the-latest-climate-news}{%
\paragraph{Keep Up on the Latest Climate
News}\label{keep-up-on-the-latest-climate-news}}

Updated Aug. 18, 2020

Here's what you need to know this week:

\begin{itemize}
\item
  \begin{itemize}
  \tightlist
  \item
    Five automakers
    \href{https://www.nytimes3xbfgragh.onion/2020/08/17/climate/california-automakers-pollution.html?action=click\&pgtype=Article\&state=default\&region=MAIN_CONTENT_1\&context=storylines_keepup}{sealed
    a binding agreement} with California to follow the state's stricter
    tailpipe emissions rules.
  \item
    The Trump
    administration\href{https://www.nytimes3xbfgragh.onion/2020/08/13/climate/trump-methane.html?action=click\&pgtype=Article\&state=default\&region=MAIN_CONTENT_1\&context=storylines_keepup}{eliminated
    a major methane rule}, even as leaks are worsening, in a decision
    that researchers warned ignored science.
  \item
    Climate change leaders said
    \href{https://www.nytimes3xbfgragh.onion/2020/08/12/climate/kamala-harris-environmental-justice.html?action=click\&pgtype=Article\&state=default\&region=MAIN_CONTENT_1\&context=storylines_keepup}{the
    vice-presidential choice of Kamala Harris} signaled that Democrats
    will have a focus on environmental justice.
  \end{itemize}
\end{itemize}

Environmental lawyers and public health leaders called the timing of the
final mercury rule, as well as its substance, an attack on air quality.

``What is most disconcerting to me is this administration's lack of
interest in science and, frankly, their lack of concern for our nation's
children,'' said Aaron Bernstein, interim director of the Center for
Climate, Health and the Global Environment at the Harvard T.H. Chan
School of Public Health. ``Mercury pollution in the United States
damages our children's brains before they even come into the world, and
estimates are that that cost is in the billions of dollars.''

Patrick Parenteau, a professor at the Vermont Law School, noted that in
virtually every environmental rollback, Mr. Trump's E.P.A. has
acknowledged in the fine print that enormous increases in health
problems and deaths will occur because of increased pollution.

A plan to weaken carbon dioxide emissions at power plants, for example,
predicted as many as
\href{https://www.nytimes3xbfgragh.onion/2018/08/21/climate/epa-coal-pollution-deaths.html}{1,400
additional premature deaths} a year. A draft
\href{https://www.epa.gov/sites/production/files/2019-09/documents/draft_policy_assessment_for_pm_naaqs_09-05-2019.pdf}{analysis}
of the soot policy put forward this week showed that tightening the
existing standard by 25 percent could save as many as 12,150 lives a
year.

Two people close to the administration said the White House was
concerned enough about the public perception of loosening environmental
rules during the outbreak that it held the mercury plan for several
weeks after it passed a review from the White House Office of
Information and Regulatory Affairs. E.P.A. officials assured the White
House that the agency was merely responding as required to a 2015
Supreme Court ruling that found it must justify the economic impact of
the mercury standards.

The weakening of the mercury rule would be one of the most significant
regulatory rollbacks engineered by the Trump administration.
\href{https://www.nytimes3xbfgragh.onion/2011/03/17/science/earth/17epa.html}{The
existing federal regulation on mercury pollution, completed in 2012}, is
the most expensive clean air regulation ever written by the E.P.A.

\href{https://www.nytimes3xbfgragh.onion/2018/12/28/climate/mercury-coal-pollution-regulations.html}{When
it published its draft cost-benefit changes last year,} the Trump
administration's E.P.A. said the cost of cutting mercury from power
plant emissions ``dwarfs'' the economic benefits and argued that the
Obama rule could not be justified as ``appropriate and necessary.''

The E.P.A.'s 2012 regulation was the first federal standard to require
power plants to install pricey pollution controls to limit mercury
emissions --- at a cost to industry of \$9.6 billion a year. The Obama
administration justified that cost with some mathematical legerdemain of
its own. The Obama E.P.A. tallied not just the benefits of reducing
mercury but also ``co-benefits'' like reducing sulfur dioxide, fine
particulate matter and other pollutants that were also curbed by the
equipment.

In his announcement on Thursday, Mr. Wheeler called the Obama
administration's accounting of health benefits ``dishonest.''

Driving down mercury emissions alone, the studies at the time found,
would yield a \$6 million annual benefit, a fraction of the cost of the
controls. But by adding in co-benefits like projected gains in avoided
heart disease, asthma attacks and other health problems, the total
benefits reached \$80 billion over five years. Overall, the Obama
administration estimated that the rule would prevent 4,700 heart
attacks, 130,000 asthma attacks and 11,000 premature deaths each year.

Under the Trump administration's new rule, such co-benefits will no
longer be calculated with cost, only direct benefits.

While coal producers urged Mr. Trump to roll back the rule, the vast
majority of electric utility companies have agreed the cost-benefit
changes may be of little help to them, because they have already spent
the billions of dollars needed to come into compliance. Many of those
companies urged the Trump administration to leave the mercury measure in
place.

\includegraphics{https://static01.graylady3jvrrxbe.onion/images/2020/01/22/climate/00CLI-MERCURY-HFO2/00CLI-MERCURY-HFO2-articleLarge.jpg?quality=75\&auto=webp\&disable=upscale}

Coal plants subject to the rule ``have already spent millions of dollars
to install mercury equipment to reduce mercury emissions,'' wrote Scott
A. Weaver, the director of air quality services for American Electric
Power, an Ohio-based electric utility company that operates power plants
in 11 states, in a public comment on a draft of the rule.

``Rescinding the standards at this point will create new problems'' Mr.
Weaver wrote, noting that companies that have sought to recoup the cost
of installing mercury control equipment through bills to customers may
no longer legally be able to do so. That means the new rule could
actually cost companies more money.

Yet in comments to the E.P.A, major coal companies and the United States
Chamber of Commerce also supported changing the underlying cost-benefit
analysis to prevent future rules that they might consider onerous.

Matthew Davis, a former E.P.A. scientist who worked with the agency's
office of children's health protection to develop the original rule,
said weakening the rule still represented a threat to children's health.

``The reason we did this rule is because children and developing fetuses
are harmed by mercury,'' said Mr. Davis, who now works at the
\href{https://www.lcv.org/}{League of Conservation Voters}.

Jeffrey R. Holmstead, an attorney for the electric utility industry who
served in the E.P.A. under the second President George Bush, rejected
the idea that the decision was a cynical calculation to set the stage
for other regulatory rollbacks.

``It's not as if the administration has been shy about rolling back
other things,'' Mr. Holmstead said, dismissing environmental concerns.
``I don't think there's any impact whatsoever from this.''

In Congress, the Obama-era mercury rule has bipartisan support. Senator
Thomas R. Carper of Delaware, the top Democrat on the Environment and
Public Works committee, and Senator Lamar Alexander of Tennessee, the
chairman of an energy and water appropriations subcommittee, wrote an
opinion piece in
\href{https://www.usatoday.com/story/opinion/2019/11/12/trump-epa-keep-mercury-pollution-rule-keep-america-healthy-column/4156901002/}{USA
Today} in November urging the Trump administration to leave the
regulation unchanged.

``Changing the rule after billions of dollars have already been spent
means that utilities will have less certainty about federal
regulations,'' they wrote. ``The gains we have made over the past decade
to protect children and families from dangerous mercury pollution should
not be lost.''

\emph{For more climate news sign up for}
\href{https://www.nytimes3xbfgragh.onion/newsletters/climate-change}{\emph{the
Climate Fwd: newsletter}} \emph{or follow}
\href{https://twitter.com/nytclimate}{\emph{@NYTClimate on
Twitter}}\emph{.}

Advertisement

\protect\hyperlink{after-bottom}{Continue reading the main story}

\hypertarget{site-index}{%
\subsection{Site Index}\label{site-index}}

\hypertarget{site-information-navigation}{%
\subsection{Site Information
Navigation}\label{site-information-navigation}}

\begin{itemize}
\tightlist
\item
  \href{https://help.nytimes3xbfgragh.onion/hc/en-us/articles/115014792127-Copyright-notice}{©~2020~The
  New York Times Company}
\end{itemize}

\begin{itemize}
\tightlist
\item
  \href{https://www.nytco.com/}{NYTCo}
\item
  \href{https://help.nytimes3xbfgragh.onion/hc/en-us/articles/115015385887-Contact-Us}{Contact
  Us}
\item
  \href{https://www.nytco.com/careers/}{Work with us}
\item
  \href{https://nytmediakit.com/}{Advertise}
\item
  \href{http://www.tbrandstudio.com/}{T Brand Studio}
\item
  \href{https://www.nytimes3xbfgragh.onion/privacy/cookie-policy\#how-do-i-manage-trackers}{Your
  Ad Choices}
\item
  \href{https://www.nytimes3xbfgragh.onion/privacy}{Privacy}
\item
  \href{https://help.nytimes3xbfgragh.onion/hc/en-us/articles/115014893428-Terms-of-service}{Terms
  of Service}
\item
  \href{https://help.nytimes3xbfgragh.onion/hc/en-us/articles/115014893968-Terms-of-sale}{Terms
  of Sale}
\item
  \href{https://spiderbites.nytimes3xbfgragh.onion}{Site Map}
\item
  \href{https://help.nytimes3xbfgragh.onion/hc/en-us}{Help}
\item
  \href{https://www.nytimes3xbfgragh.onion/subscription?campaignId=37WXW}{Subscriptions}
\end{itemize}
