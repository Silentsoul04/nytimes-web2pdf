Sections

SEARCH

\protect\hyperlink{site-content}{Skip to
content}\protect\hyperlink{site-index}{Skip to site index}

\href{https://www.nytimes3xbfgragh.onion/section/technology}{Technology}

\href{https://myaccount.nytimes3xbfgragh.onion/auth/login?response_type=cookie\&client_id=vi}{}

\href{https://www.nytimes3xbfgragh.onion/section/todayspaper}{Today's
Paper}

\href{/section/technology}{Technology}\textbar{}Welcome to the `Rabbit
Hole'

\url{https://nyti.ms/2XGa5HZ}

\begin{itemize}
\item
\item
\item
\item
\item
\end{itemize}

\href{https://www.nytimes3xbfgragh.onion/spotlight/at-home?action=click\&pgtype=Article\&state=default\&region=TOP_BANNER\&context=at_home_menu}{At
Home}

\begin{itemize}
\tightlist
\item
  \href{https://www.nytimes3xbfgragh.onion/2020/07/28/books/time-for-a-literary-road-trip.html?action=click\&pgtype=Article\&state=default\&region=TOP_BANNER\&context=at_home_menu}{Take:
  A Literary Road Trip}
\item
  \href{https://www.nytimes3xbfgragh.onion/2020/07/29/magazine/bored-with-your-home-cooking-some-smoky-eggplant-will-fix-that.html?action=click\&pgtype=Article\&state=default\&region=TOP_BANNER\&context=at_home_menu}{Cook:
  Smoky Eggplant}
\item
  \href{https://www.nytimes3xbfgragh.onion/2020/07/27/travel/moose-michigan-isle-royale.html?action=click\&pgtype=Article\&state=default\&region=TOP_BANNER\&context=at_home_menu}{Look
  Out: For Moose}
\item
  \href{https://www.nytimes3xbfgragh.onion/interactive/2020/at-home/even-more-reporters-editors-diaries-lists-recommendations.html?action=click\&pgtype=Article\&state=default\&region=TOP_BANNER\&context=at_home_menu}{Explore:
  Reporters' Obsessions}
\end{itemize}

Advertisement

\protect\hyperlink{after-top}{Continue reading the main story}

Supported by

\protect\hyperlink{after-sponsor}{Continue reading the main story}

The Shift

\hypertarget{welcome-to-the-rabbit-hole}{%
\section{Welcome to the `Rabbit
Hole'}\label{welcome-to-the-rabbit-hole}}

Introducing an audio series about how the internet is changing, and how
it's changing us.

\includegraphics{https://static01.graylady3jvrrxbe.onion/images/2020/04/22/business/16roose-podcast-sub/merlin_171447873_f59f5a45-3a67-4223-a72e-b181bc2f546f-articleLarge.jpg?quality=75\&auto=webp\&disable=upscale}

\href{https://www.nytimes3xbfgragh.onion/by/kevin-roose}{\includegraphics{https://static01.graylady3jvrrxbe.onion/images/2018/02/20/multimedia/author-kevin-roose/author-kevin-roose-thumbLarge.jpg}}

By \href{https://www.nytimes3xbfgragh.onion/by/kevin-roose}{Kevin Roose}

\begin{itemize}
\item
  April 16, 2020
\item
  \begin{itemize}
  \item
  \item
  \item
  \item
  \item
  \end{itemize}
\end{itemize}

Recently, I saw something on Twitter that annoyed me. It was an
\href{https://www.clevescene.com/scene-and-heard/archives/2020/04/14/were-not-afraid-of-any-virus-crowds-gathered-outside-of-ohio-statehouse-protesting-coronavirus-shutdowns}{article}
from a local Ohio news site, describing a small crowd that had gathered
outside the statehouse to protest the state's coronavirus shutdowns.
They were urging the governor to ignore social distancing guidelines and
open the state for business, a move that could endanger public health
and potentially put thousands of Ohioans' lives at risk.

After fuming silently for a few minutes, I texted the link to my mother,
who lives in Ohio, with the caption ``Idiots!''

Then I caught myself. I wondered: Did a few dozen protesters doing
something reckless and irresponsible thousands of miles away from me
really deserve my outrage? Or had I allowed myself to be whipped into a
frenzy by people I'd never met, on a platform engineered to capture my
attention and convert it into advertising dollars? Was I ignoring the
larger forces at work --- like the professional partisans who have made
a living exploiting the paranoid and fearful, or the long, slow erosion
of trust in mainstream health authorities --- in favor of dunking on an
easy target? Was I actually mad, or did I just feel like I \emph{should}
be mad?

This is not a new thought pattern. I have it a few times a day, whenever
I find myself behaving in a way that feels less like genuine
self-expression and more like giving in to the invisible tug of a
faceless internet platform. Is the salad I'm making for dinner actually
what I want to eat, or do I just think the photo will make me look
healthy and responsible when I post it on Instagram later? Did I really
want to watch ``Schitt's Creek,'' or did I just trust Netflix's
recommendations more than my own taste? Which of my tastes, thoughts,
and habits are really mine, and which were put there by an algorithm?

I've started calling this sensation ``machine drift,'' and I feel it
every day. Especially now, when millions of Americans are stuck inside
during a pandemic, the question of how our personalities and preferences
are being shaped by our digital surroundings seems more pressing than
ever.

After all, for many of us, the world has been reduced to what we
experience on screens. And the things on those screens are not neutral
or inert. They've been put there on purpose and arranged, often by a
combination of humans and sophisticated forms of artificial
intelligence, to accomplish some goal. Maybe that goal is to make us
click, buy or share. Maybe it's to persuade us, or harden some part of
our identity. Most of the time, what these machines want from us appears
harmless. Once in a while, it actually is.

\includegraphics{https://static01.graylady3jvrrxbe.onion/images/2020/04/16/business/16roose-podcast/merlin_171620523_52d9c32c-d2c8-496b-bad5-95cd2732f04c-articleLarge.jpg?quality=75\&auto=webp\&disable=upscale}

Marshall McLuhan, the 20th century media theorist, is often credited
with saying, ``We shape our tools, and thereafter our tools shape us.''

We know, by now, how we've shaped our tools. The origin stories of our
internet mega-platforms have been told over and over again, in glossy
business magazines and cable news specials. We know that Twitter ignited
the Arab Spring, and that Facebook ads helped elect President Trump. We
know that Google knows everything about us, and that Instagram's effect
on our culture has been immeasurable.

Less clear, though, is the answer to the question raised by the other
side of McLuhan's aphorism. How are our tools shaping us?

I've been obsessed with the internet since I was a preteen, and I've
been fascinated as things I used to consider niche phenomena --- things
like ad microtargeting, meme culture and online conspiracy theories ---
have wormed their way into the center of the national discourse. In the
past few years, internet culture has emerged not just as a nerdy
subculture, but as mass culture itself. And our reliance on the internet
as a primary source of information, entertainment and community seems to
be changing the ways we talk, the things we enjoy and the sources we
trust.

As a tech columnist, I'm inundated with both the good and bad parts of
the internet every day. And often, I have a hard time telling where the
internet stops and my personality starts. These days, more of my
thoughts arrive as trite, Twitter-size observations, and my more offbeat
tastes have been transformed into smoother, more mainstream ones by the
centripetal force of algorithmic recommendations. I've heard stories of
people I know struggling with their own versions of machine drift --- a
friend whose political views were upended by a Facebook group, a friend
who reoriented her life around chasing Instagram fame, an acquaintance
who dropped into a dark corner of 4chan and never came back.

Last year, after the mosque shootings in Christchurch, New Zealand --- a
heinous mass murder committed by a white nationalist with ties to
far-right internet groups, whose
\href{https://www.nytimes3xbfgragh.onion/2019/03/15/technology/facebook-youtube-christchurch-shooting.html}{actions
seemed} to be calculated to produce maximum internet virality --- my
colleague Andy Mills and I decided to try to figure out where the
internet was leading us, and what made its pull so powerful.

With the help of a small team of talented colleagues, we set out on a
yearlong project to make a narrative audio series that would tell the
stories of people who were shaping, and being shaped by, the internet.

You might have read versions of
\href{https://www.nytimes3xbfgragh.onion/interactive/2019/06/08/technology/youtube-radical.html}{some
of these stories} in text form already. But by telling them in audio
form, they've come alive in a new way. And although we didn't plan to
release the series during a national lockdown, it might turn out to be
the right time to explore the contours of the internet --- this place
I've spent my entire adult life, this strange ecosystem we all now
inhabit --- and figure out how, and why, it's changing who we are.

The series comes out on Thursday. It's called
``\href{https://www.nytimes3xbfgragh.onion/2020/04/16/podcasts/rabbit-hole-internet-youtube-virus.html}{Rabbit
Hole}.'' I hope you'll listen.

\includegraphics{https://static01.graylady3jvrrxbe.onion/images/2020/04/18/podcasts/18insider1/rabbit-hole-album-art-articleInline.jpg?quality=75\&auto=webp\&disable=upscale}

\hypertarget{listen-to-the-rabbit-hole-prologue}{%
\subsubsection{Listen to the `Rabbit Hole'
prologue.}\label{listen-to-the-rabbit-hole-prologue}}

START HERE.

\includegraphics{https://static01.graylady3jvrrxbe.onion/images/2020/04/18/podcasts/18insider1/rabbit-hole-album-art-articleInline.jpg?quality=75\&auto=webp\&disable=upscale}

\hypertarget{listen-to-the-first-episode-of-rabbit-hole}{%
\subsubsection{Listen to the first episode of `Rabbit
Hole.'}\label{listen-to-the-first-episode-of-rabbit-hole}}

A young man finds escape on the internet. He doesn't realize that on the
other side of the screen, a force is pulling him in.

transcript

Back to Rabbit Hole

bars

0:00/26:49

-26:49

transcript

\hypertarget{listen-to-the-first-episode-of-rabbit-hole-1}{%
\subsection{Listen to the first episode of `Rabbit
Hole.'}\label{listen-to-the-first-episode-of-rabbit-hole-1}}

\hypertarget{with-kevin-roose-produced-by-andy-mills-julia-longoria-and-sindhu-gnanasambandan-and-edited-by-larissa-anderson-and-wendy-dorr}{%
\subsubsection{With Kevin Roose, produced by Andy Mills, Julia Longoria
and Sindhu Gnanasambandan, and edited by Larissa Anderson and Wendy
Dorr}\label{with-kevin-roose-produced-by-andy-mills-julia-longoria-and-sindhu-gnanasambandan-and-edited-by-larissa-anderson-and-wendy-dorr}}

\hypertarget{a-young-man-finds-escape-on-the-internet-he-doesnt-realize-that-on-the-other-side-of-the-screen-a-force-is-pulling-him-in}{%
\paragraph{A young man finds escape on the internet. He doesn't realize
that on the other side of the screen, a force is pulling him
in.}\label{a-young-man-finds-escape-on-the-internet-he-doesnt-realize-that-on-the-other-side-of-the-screen-a-force-is-pulling-him-in}}

\begin{itemize}
\item
  archived recording 1\\
  Take the spin, now you're in with the techno set. You're going surfing
  on the internet.
\item
  archived recording 2\\
  It's another day in the life of the Jamesons, maybe a family a little
  bit like yours. Except it's not really just another day. Today's the
  day I'm taking my family surfing around the world on the internet.
\item
  archived recording 3\\
  It's cool. Dad's finally installed the internet on our home computer.
  Now I can surf the net any time I want.
\end{itemize}

{[}music{]}

\begin{itemize}
\item
  andy mills\\
  Test, test, test. Is this the state line?
\item
  kevin roose\\
  I just saw the West Virginia Welcome Center.
\item
  andy mills\\
  Wild, Wonderful, West Virginia with Jim Justice.
\item
  kevin roose\\
  Oh, the governor's name is Jim Justice. I feel like the campaign
  signs, like, make themselves.
\item
  andy mills\\
  Vote Justice.
\item
  keving roose\\
  You know?
\end{itemize}

andy mills

OK, so, just like, tell me the story and I'll poke you on the way.

kevin roose

OK.

\begin{itemize}
\tightlist
\item
  kevin roose\\
  This is our exit here.
\end{itemize}

kevin roose

So last year, you and I hopped in a car ---

\begin{itemize}
\tightlist
\item
  andy mills\\
  Can you just --- can you just tell me, like, what are you up to today
  and what are you hoping to find out?
\end{itemize}

kevin roose

--- to meet this guy.

\begin{itemize}
\tightlist
\item
  kevin roose\\
  His story is really interesting. So he says that he was radicalized
  through YouTube videos and spent, you know, several years becoming
  progressively more extreme in his politics.
\end{itemize}

kevin roose

After this shooting in New Zealand last March, there was a lot of talk
about online radicalization. The shooter was a white nationalist who
clearly spent a lot of time in far right internet communities.

\begin{itemize}
\tightlist
\item
  kevin roose\\
  I've been looking for a good case study of how it happens to just one
  person, what that path looks like.
\end{itemize}

kevin roose

And I kept hearing over and over again from sources I was talking to in
this world, like, you have to look at YouTube.

\begin{itemize}
\tightlist
\item
  kevin roose\\
  There are elements of it that seem very familiar to me and some that
  don't seem so familiar. So we're going to have him walk us through it.
  We missed our turn. I was so excited about this that I missed my turn.
\end{itemize}

kevin roose

So we drove down to West Virginia, where he lives.

\begin{itemize}
\tightlist
\item
  kevin roose\\
  Hey, how's it going? All right, great.
\end{itemize}

kevin roose

He's standing right there. He's on his headphones. Big smile.

\begin{itemize}
\item
  caleb cain\\
  Nice to meet you guys.
\item
  andy mills\\
  Hi, Andy.
\item
  kevin roose\\
  This is Andy.
\item
  andy mills\\
  I'm going to stick a microphone around you.
\item
  caleb cain\\
  That's fine.
\end{itemize}

kevin roose

He's got a Gorillaz t-shirt on. He says hi and brings us into his
friend's house.

\begin{itemize}
\item
  kevin roose\\
  Hi, I'm Kevin. Nice to meet you.
\item
  caleb's friend\\
  Nice to meet you.
\item
  kevin roose\\
  Thank you for letting us crash your ---
\item
  caleb's friend\\
  Oh yeah.
\item
  kevin roose\\
  --- your house here.
\end{itemize}

kevin roose

And the first thing he does is ---

\begin{itemize}
\item
  caleb cain\\
  Now I'm gonna just sit this over here.
\item
  andy mills\\
  Can you describe what it is that you're setting down?
\item
  caleb cain\\
  A Glock-43.
\end{itemize}

kevin roose

--- pull out his gun.

\begin{itemize}
\item
  kevin roose\\
  Is that the first time you've had a gun?
\item
  caleb cain\\
  I've liked firearms my whole life. I'm not against firearms. But,
  yeah, that's --- I've always been like, well, I don't really need one.
  But then the day after I got death threats, went out and bought one.
\item
  kevin roose\\
  Oh, man.
\item
  caleb cain\\
  I'll probably never have to use it. These guys usually just like send
  SWAT teams to your house and {[}EXPLETIVE{]} like that. But, you know,
  it's just to be safe.
\item
  kevin roose\\
  Well, where should we sit? Where's the best place for us to ---
\item
  caleb's friend\\
  You can sit here.
\end{itemize}

kevin roose

Eventually we go over, we sit on the couch.

\begin{itemize}
\item
  andy mills\\
  Give me a little test here.
\item
  kevin roose\\
  Oh, sure. Testing, one, two, three. Do-re-mi.
\end{itemize}

kevin roose

And he starts telling what begins like a pretty relatable and familiar
story.

\begin{itemize}
\item
  kevin roose\\
  So can you just start, like, tell us your name and how old you are?
\item
  caleb cain\\
  Yeah. My name's Caleb Cain. I'm 26 years old. I was born in Florida
  but I grew up here in West Virginia.
\item
  kevin roose\\
  When did you move to West Virginia?
\item
  caleb cain\\
  My mother had me with some man that I never met. And then she, like,
  immediately left Florida, I guess. I don't know all the details there.
\end{itemize}

kevin roose

He had kind of a rough childhood, as he describes it.

\begin{itemize}
\tightlist
\item
  caleb cain\\
  Got raised by my grandparents.
\end{itemize}

kevin roose

Didn't really have a lot of friends.

\begin{itemize}
\item
  kevin roose\\
  What were you like as a kid?
\item
  caleb cain\\
  Um, really shy and nerdy, picked on on the bus.
\end{itemize}

kevin roose

Didn't really feel like he fit in.

\begin{itemize}
\tightlist
\item
  caleb cain\\
  Going to high school was hell. I hated it. I saw everybody as
  conformists.
\end{itemize}

kevin roose

Like a lot of teenagers, he got really into video games.

\begin{itemize}
\tightlist
\item
  caleb cain\\
  So Zelda was a big one, Donkey Kong.
\end{itemize}

kevin roose

And then he discovered ---

\begin{itemize}
\tightlist
\item
  caleb cain\\
  Freshman year of high school was whenever I got high speed internet.
\end{itemize}

kevin roose

--- the internet. I don't know what I would have done without the
internet.

\begin{itemize}
\item
  archived recording (video gamer)\\
  This thing is awesome! {[}LAUGHING{]}
\item
  caleb cain\\
  It was like a escape.
\end{itemize}

kevin roose

And the internet is a revolution for him, because finally ---

\begin{itemize}
\item
  caleb cain\\
  That's when I started playing a lot of online video games.
\item
  archived recording (video gamer 1)\\
  Oh {[}EXPLETIVE{]}, in the window of the house!
\item
  archived recording (video gamer 2)\\
  Oh, I'm reloading.
\end{itemize}

kevin roose

He finds people who are like him.

\begin{itemize}
\item
  caleb cain\\
  People over the country, all over the world.
\item
  archived recording (video gamer 1)\\
  Oh, that was so good, man.
\item
  archived recording (video gamer 2)\\
  Oh, nice. {[}LAUGHTER{]}
\item
  caleb cain\\
  I met a lot of friends.
\item
  archived recording (video gamer)\\
  They grow up so quickly. {[}LAUGHTER{]}
\end{itemize}

kevin roose

And he develops this new routine where, like, every day he comes home
from school, gets on his computer ---

\begin{itemize}
\tightlist
\item
  archived recording (video gamer)\\
  All right, this is Call of Duty 2.
\end{itemize}

kevin roose

--- plays a bunch of video games. And then, later at night ---

\begin{itemize}
\tightlist
\item
  caleb cain\\
  I'd turn on YouTube.
\end{itemize}

kevin roose

--- he goes on YouTube.

\begin{itemize}
\item
  archived recording (youtube clip 1)\\
  (SINGING) Chocolate rain.
\item
  archived recording (youtube clip 2)\\
  I'm on crack.
\end{itemize}

kevin roose

And YouTube at the time, it was ---

\begin{itemize}
\tightlist
\item
  archived recording (youtube clip 3)\\
  Is this supposed to be forever?
\end{itemize}

kevin roose

--- mostly ---

\begin{itemize}
\tightlist
\item
  archived recording (youtube clip 4)\\
  Charlie! That really hurt.
\end{itemize}

kevin roose

--- viral videos and comedy sketches. This was like, you know, early,
early YouTube.

\begin{itemize}
\item
  archived recording (youtube clip 1)\\
  I want my money!
\item
  archived recording (youtube clip 2)\\
  You need to relax.
\item
  caleb cain\\
  Stay up all night and watch, like ---
\item
  archived recording (youtube clip)\\
  (SINGING) It's my dick in a box, dick in a box, girl.
\item
  caleb cain\\
  --- and bust up laughing.
\end{itemize}

kevin roose

If Caleb didn't feel like he fit in in high school, in his sort of
physical surroundings ---

\begin{itemize}
\item
  archived recording (youtube clip 1)\\
  And the categories are Potent Potables.
\item
  archived recording (youtube clip 2)\\
  You wouldn't know it, but I like people.
\end{itemize}

kevin roose

YouTube was the place that he felt most at home.

\begin{itemize}
\item
  archived recording (youtube clip)\\
  I like people, but I like them in short bursts.
\item
  kevin roose\\
  You're speaking to me very deeply.
\item
  caleb cain\\
  Yeah, exactly.
\item
  kevin roose\\
  I experienced this too. Like, I was not cool in high school. And I
  remember the internet kind of being a place that you would go to,
  like, escape.
\item
  caleb cain\\
  Yeah.
\item
  kevin roose\\
  And it was sort of nicer than your real life, in some cases.
\item
  caleb cain\\
  You're right. It was nicer back then.
\item
  kevin roose\\
  Were you political at the time?
\item
  caleb cain\\
  Political in like a very surface level sense, right? Like
  anti-authority and like, you know. Most of my politics as a teenager
  came from, like, Dead Kennedys, and ---
\item
  archived recording (dead kennedys)\\
  (SINGING) California! Uber Alles!
\item
  archived recording (michael moore)\\
  This is Michael Moore. I am here to make a citizen's arrest.
\item
  caleb cain\\
  --- Michael Moore documentaries. And that influenced me a lot.
\item
  archived recording (michael moore)\\
  I mean, I wish that CNN and the other mainstream media would just for
  once tell the truth about what's going on in this country.
\item
  caleb cain\\
  So there was very much that punk rock influence inside of me.
\end{itemize}

kevin roose

He also got really into these new atheist videos.

\begin{itemize}
\item
  archived recording (cristopher hitchens)\\
  When I say that I think religion poisons everything ---
\item
  caleb cain\\
  I remember watching, like, Christopher Hitchens on YouTube.
\item
  archived recording (cristopher hitchens)\\
  I mean to say it infects us in our most basic integrity.
\item
  caleb cain\\
  You'd get old uploads ---
\item
  archived recording (richard dawkins)\\
  For me, what matters is the truth.
\item
  caleb cain\\
  --- of like a Richard Dawkins speech.
\item
  archived recording (richard dawkins)\\
  There is nothing special about the Bible.
\end{itemize}

andy mills

Oh, I remember these videos. They felt like, kind of scandalous at the
time.

kevin roose

Right. They felt subversive. They felt like watching people say the
uncomfortable thing.

\begin{itemize}
\tightlist
\item
  archived recording (sam harris)\\
  If God is good and loving and just, and he wanted to guide us morally
  with a book, why give us a book that supports slavery.
\end{itemize}

kevin roose

By today's standards, obviously, like, this is extremely tame. But at
the time, like, this was pretty edgy stuff.

\begin{itemize}
\item
  kevin roose\\
  So this would have been, like, early Obama years, right?
\item
  caleb cain\\
  Yeah. Yeah, early Obama years.
\item
  kevin roose\\
  Did you like Obama? Or What did you feel about him?
\item
  caleb cain\\
  I liked him. I didn't know much about him, because I didn't look into
  actual politics. So I didn't know what he was doing. But, yeah, I
  liked him. I thought, yeah, we have a black president. That's cool.
  Like, you know, first black president. Making progress.
\item
  andy mills\\
  What year was it that you graduated from school?
\item
  caleb cain\\
  I graduated 2011, went to college.
\end{itemize}

kevin roose

And then he goes off to college. And college, like, just doesn't really
take for him.

\begin{itemize}
\tightlist
\item
  caleb cain\\
  I wanted to go and do an environmental major. And I didn't have a good
  time. Most the time I'd stay in my room. Even on nice days when people
  were out on the quad throwing frisbees, I'd sit in my room and play
  video games. And then there was also this embarrassing moment where I
  got into a fight with this kid on campus and kind of got laughed off
  of campus. And that night, I freaking left. Because I wasn't going to
  class anyway. And I just withdrew from my classes and I left. And I
  came back to West Virginia.
\end{itemize}

kevin roose

So he moves back in with his grandparents. And there, like, he doesn't
have much to do. He doesn't have a job. He doesn't really have a
direction.

\begin{itemize}
\tightlist
\item
  caleb cain\\
  It's just me in a room and a bed.
\end{itemize}

kevin roose

He spent a lot of time ---

\begin{itemize}
\tightlist
\item
  caleb cain\\
  Slept a lot.
\end{itemize}

kevin roose

--- feeling depressed.

\begin{itemize}
\tightlist
\item
  caleb cain\\
  I'd sit in my basement and just be, like, so freaking down.
\end{itemize}

kevin roose

At one point he even loses his gaming computer.

\begin{itemize}
\item
  caleb cain\\
  My gaming computer got stolen, by the way, which really
  {[}EXPLETIVE{]} drove me up a wall. And so not I didn't even have
  video games. Now I have a crappy little computer that can hardly run
  anything. But it can run YouTube.
\item
  archived recording (youtube clip)\\
  Whoa, that's a cool rainbow.
\end{itemize}

kevin roose

And so now he's in his early 20s. He's living at his grandparents'
house.

\begin{itemize}
\item
  archived recording (youtube clip)\\
  It's not so much that it's bad.
\item
  kevin roose\\
  And he feels like, at this time in his life, when he should be, like,
  finding his way, he should be starting a career and thinking about
  having a family ---
\item
  archived recording (youtube clip)\\
  So hide ya kids, hide ya wife.
\end{itemize}

kevin roose

Instead, he's just watching YouTube.

\begin{itemize}
\item
  archived recording (youtube clip 1)\\
  Look at that horse.
\item
  archived recording (youtube clip 2)\\
  All we know is how little we know.
\item
  archived recording (youtube clip 3)\\
  (SINGING) What does the fox say?
\end{itemize}

kevin roose

And then ---

\begin{itemize}
\item
  archived recording (athene)\\
  The human brain is a network of approximately 100 billion neurons.
\item
  caleb cain\\
  I found a documentary called, ``God is in the Neurons.''
\item
  archived recording (athene)\\
  When we grow up, our moral and ethical compass is almost entirely
  forged by our environments.
\item
  caleb cain\\
  It's about like, cognitive dissonance, and like how you can fall into
  patterns of behavior. And since you have neuroplasticity, that you can
  get out of those patterns of behavior.
\item
  archived recording (athene)\\
  When we are self-aware, we can alter misplaced emotions, because we
  control the thoughts that cause them.
\item
  caleb cain\\
  So I got in this mindset of, oh, my brain is just like this tool that
  I can shape into whatever I want.
\end{itemize}

kevin roose

And then he stumbles into this emerging wing of YouTube.

\begin{itemize}
\tightlist
\item
  caleb cain\\
  And so I started, just going through self-help content.
\end{itemize}

kevin roose

Self-help videos.

\begin{itemize}
\item
  archived recording (self-help video 1)\\
  What dream or vision do you want to turn into reality?
\item
  archived recording (tony robbins)\\
  All of our success and failure in life comes from little decisions.
\item
  caleb cain\\
  Cheesy stuff, to be honest with you.
\item
  archived recording (tony robbins)\\
  The process of conditioning ourselves actually feels incredible.
\item
  caleb cain\\
  Like Tony Robbins and stuff.
\item
  archived recording (self-help video)\\
  Every mind possesses the potential to be utterly free of all ---
\item
  caleb cain\\
  And Zen Buddhism stuff.
\item
  archived recording (alan watts)\\
  We want to change our consciousness.
\end{itemize}

kevin roose

They're like, people with advice specifically for ---

\begin{itemize}
\tightlist
\item
  archived recording (self-help video)\\
  You continue to do those same behaviors that keep you from making the
  change.
\end{itemize}

kevin roose

--- guys like Caleb. And then ---

\begin{itemize}
\item
  archived recording (stefan molyneux)\\
  To be truly free is both very easy and very hard.
\item
  caleb cain\\
  --- I found Stef.
\item
  archived recording (stefan molyneux)\\
  But we can only be kept in the cages we refuse to see.
\end{itemize}

kevin roose

So Stefan Molyneux is this Canadian libertarian, formerly a historian
and an entrepreneur. And then he sort of became like a podcaster guy.

\begin{itemize}
\item
  archived recording (stefan molyneux)\\
  Good morning, everybody. It's Stefan Molyneux from Freedomain Radio. I
  hope that you're doing very well.
\item
  caleb cain\\
  Stef just was in the sidebar one day, and I clicked on it.
\item
  archived recording (stefan molyneux)\\
  You really have to open the often iron-bound doors of your heart.
\end{itemize}

kevin roose

When YouTube, sort of early in its life, removed its 15-minute limit on
how long videos could be, he just started pumping out hour, two-hour
long shows called Freedomain Radio.

\begin{itemize}
\tightlist
\item
  archived recording (stefan molyneux)\\
  The approach that we take at Freedomain Radio, the sort of
  philosophical Socratic approach that we take, can be very, very
  helpful for you.
\end{itemize}

kevin roose

Where he would expound on philosophical ideas.

\begin{itemize}
\tightlist
\item
  archived recording (stefan molyneux)\\
  Philosophy is the all-discipline. It covers everything. And that's
  why, to me, it is the most exciting and fundamental ---
\end{itemize}

kevin roose

And Stefan Molyneux is telling him things that, you know, make him feel
better. He's saying, this depression that you're going through, it's not
permanent. Things will get better. And that a lot of the disillusionment
and pain that young men like Caleb are facing is not actually their
fault.

\begin{itemize}
\tightlist
\item
  archived recording (stefan molyneux)\\
  From the perspective of a young man, just take a brief look at society
  ---
\end{itemize}

kevin roose

It's the fault of society.

\begin{itemize}
\tightlist
\item
  archived recording (stefan molyneux)\\
  I mean, you get of course on-demand pornography. You get video games
  that are unbelievably realistic, absorbing and addictive. And what
  else do they have to look forward to? Well, they can get themselves
  involved in higher education and graduate an average of \$25,000 in
  debt to a job market that is pretty stagnant or declining. You've seen
  real wages ---
\end{itemize}

kevin roose

He finds a lot of what he's saying pretty sensible.

\begin{itemize}
\item
  archived recording (stefan molyneux)\\
  College students are damn right to be depressed. Their society is
  unsustainable, because nobody's asking the fundamental questions about
  why the society is the way it is, why things are so bad.
\item
  caleb cain\\
  I was like yeah, yeah, that's true.
\end{itemize}

kevin roose

And he's not just talking at people. He invites people to send him
questions.

\begin{itemize}
\tightlist
\item
  archived recording (stefan molyneux)\\
  I have had a number of requests to do a podcast on how to meet a nice
  girl, because there are a lot of ---
\end{itemize}

kevin roose

He invites them into his life.

\begin{itemize}
\item
  caleb cain\\
  He would talk about how he grew up.
\item
  archived recording (stefan molyneux)\\
  I was interested in morality from a very early age. I was physically,
  emotionally and mentally abused.
\item
  caleb cain\\
  And he's talked about how he was so much better now.
\item
  archived recording (stefan molyneux)\\
  I have consistently said, if you have problems with your parents, talk
  to a therapist.
\item
  caleb cain\\
  And he had went to therapy and his life had improved.
\item
  archived recording (stefan molyneux)\\
  Before we met, Christina had spent quite some time working on herself.
\item
  caleb cain\\
  And he had a wife who would come on stream with him.
\item
  archived recording (christina papadopoulus\\
  I had struggled with a previous relationship.
\item
  kevin roose\\
  He has his wife on this channel and they talk about their life
  together.
\item
  archived recording (stefan molyneux)\\
  I'm watching my daughter learn how to make jokes. Finding out what is
  funny for her, what is not funny for her and why. Like, I just --- I
  want to kiss her hair all day. Just because, you know, I must kiss
  your brain.
\item
  caleb cain\\
  He talks about how much he loves his daughter. It's like, I want all
  that stuff. I want a family like that. Because that's what I wanted my
  whole life. I just wanted a stable family. And I thought, well, if I
  just keep watching more and more, I'll be like Stef.
\item
  archived recording 1 (stefan molyneux)\\
  Hey, everybody, Stefan Molyneux from Freedomain Radio.
\item
  archived recording 2 (stefan molyneux)\\
  Hi, everybody, this is Stefan Molyneux.
\item
  archived recording 3 (stefan molyneux)\\
  Hi, everybody, it's Stefan Molyneux's Freedomain Radio. I hope you're
  doing very well.
\end{itemize}

kevin roose

And for Caleb ---

\begin{itemize}
\item
  archived recording (stefan molyneux)\\
  You got happiness without responsibility.
\item
  kevin roose\\
  --- Stefan Molyneux, his voice and his videos, become a source of
  stability for him.
\item
  archived recording (stefan molyneux)\\
  There are very specific things that people need to do to be happy.
\end{itemize}

kevin roose

And Caleb says that all the stuff that he's watching on YouTube, like,
it's actually helping.

\begin{itemize}
\tightlist
\item
  caleb cain\\
  I started working at Dairy Queen.
\end{itemize}

kevin roose

He gets a new job.

\begin{itemize}
\tightlist
\item
  caleb cain\\
  And then I started, like, getting over a lot of my social anxiety,
  because now I'm forced to interact with people and I'm also hanging
  out with all these high school kids.
\end{itemize}

kevin roose

He starts to feel like things are picking up for him, finally.

\begin{itemize}
\item
  caleb cain\\
  And after that, I mean, it was just more and more of that.
\item
  archived recording (stefan molyneux)\\
  So the problem is not that we don't know how to
\item
  archived recording (foster the people)\\
  {[}Intro to ``Pumped up Kicks''{]}
\item
  caleb cain\\
  I was pretty much always on YouTube.
\item
  archived recording (lorde)\\
  (SINGING) Every song's like gold teeth, Grey Goose, tripping in the
  bathroom.
\end{itemize}

kevin roose

He's watching not only the self-help stuff from Stefan Molyneux.

\begin{itemize}
\item
  archived recording (gyote)\\
  (SINGING) But you didn't have to cut me off.
\item
  archived recording (history channel documentary)\\
  The Great Pyramid at Giza.
\end{itemize}

kevin roose

He starts getting pretty into Joe Rogan ---

\begin{itemize}
\tightlist
\item
  archived recording\\
  Joe Rogan is the man.
\end{itemize}

kevin roose

--- the biggest podcast, YouTube talk show guy that there is on the
internet.

\begin{itemize}
\tightlist
\item
  archived recording (joe rogan)\\
  Thank you very much for coming by, man. This is cool as
  {[}EXPLETIVE{]}.
\end{itemize}

kevin roose

But back then, was just starting to experiment with uploading his
interviews to YouTube.

\begin{itemize}
\item
  archived recording (joe rogan)\\
  Anthony Bourdain is with us, ladies and gentlemen.
\item
  archived recording (stefan molyneux)\\
  Hi everybody, it's Stefan Molyneux.
\end{itemize}

kevin roose

He just keeps watching and watching ---

\begin{itemize}
\tightlist
\item
  archived recording\\
  You're almost like an accidental ---
\end{itemize}

kevin roose

--- and watching.

\begin{itemize}
\item
  archived recording (interposing youtube clip)\\
  Take ownership of what they have done in their life.
\item
  caleb cain\\
  If I wasn't at work, any single moment that I had, I was watching
  YouTube videos.
\item
  archived recording (miley cyrus)\\
  (SINGING) We can't stop, we won't stop.
\item
  archived recording (joe rogan)\\
  This is episode 500.
\item
  archived recording\\
  Probably, it's at that point, 10, 12, 13, 14 hours a day.
\item
  archived recording (stefan molyneux)\\
  --- judge individuals as bad people.
\item
  caleb cain\\
  I sound like a crazy person, but that's what I would do.
\end{itemize}

kevin roose

And when Caleb talks about watching YouTube videos during this period in
his life, he talks about experiencing this as the sensation of falling.
But the thing that he doesn't even really know to think about is that on
the other side of his screen there's a force that's pulling him in. And
that force has to do with a French guy named Guillaume ---

\begin{itemize}
\item
  kevin roose\\
  Guillaume, hello. This is Kevin.
\item
  guillaume chaslot\\
  Hey, how are you?
\item
  kevin roose\\
  Doing well. How are you?
\item
  guillaume chaslot\\
  Great.
\end{itemize}

kevin roose

--- who, even though they've never met ---

\begin{itemize}
\tightlist
\item
  guillaume chaslot\\
  So, yeah, I think everything is ready here.
\end{itemize}

kevin roose

--- is a really important part of Caleb's story.

{[}music{]}

\begin{itemize}
\tightlist
\item
  guillaume chaslot\\
  I did a PhD in artificial intelligence, and then I worked at
  Microsoft.
\end{itemize}

kevin roose

Guilllaume Chaslot is a pretty smart guy.

\begin{itemize}
\item
  kevin roose\\
  You studied how to make robots, essentially.
\item
  guillaume chaslot\\
  Exactly.
\end{itemize}

kevin roose

Got his doctorate, studying machine learning. And then in 2010 he got
his dream job, working at Google.

\begin{itemize}
\tightlist
\item
  guillaume chaslot\\
  And so when I joined Google, I actually didn't know which project I
  would be put on.
\end{itemize}

kevin roose

And when he gets to Google, he gets this really interesting and exciting
assignment.

\begin{itemize}
\tightlist
\item
  guillaume chaslot\\
  Yeah, it turned out they needed someone on the AI of YouTube. So I was
  the perfect fit.
\end{itemize}

kevin roose

And the project he's assigned to work on is ultimately what
distinguishes YouTube from every other website on the internet. And that
has to do with the recommendations sidebar, and the artificial
intelligence that makes the whole thing work.

\begin{itemize}
\item
  kevin roose\\
  And did this seem to you like a big deal, to be given a job at Google,
  working on YouTube recommendations?
\item
  guillaume chaslot\\
  Yeah. That was really amazing at first to realize that my work was
  going to be affecting so many people. So I thought that was going to
  be a good thing. I thought, OK, we can make artificial intelligence to
  make the world a better place.
\item
  kevin roose\\
  And when you got there, I assume there was some algorithm that was
  selecting videos for people. How did that algorithm work?
\item
  guillaume chaslot\\
  Yeah. So initially, when YouTube started, what's best was clicks.
  Like, the more people clicked on videos, the better they thought it
  was. And then they realized that it led to too many clickbaits. So
  people would click on the title, then realize that the video was not
  at all but what was said in the title. And then they would leave the
  platform immediately. So that would be actually bad for YouTube. So
  then they switched their measure to total watch time.
\item
  kevin roose\\
  And how was the goal of this algorithm explained to you? Like, what
  did you understand about what YouTube's executives wanted?
\item
  guillaume chaslot\\
  The idea was to maximize watch time at all cost. To just make it grow
  as big as possible.
\end{itemize}

kevin roose

It seems like a simple shift, but that shift has radical consequences.

\begin{itemize}
\item
  archived recording 1\\
  YouTube's watch time has gone up by 50 percent.
\item
  archived recording 2\\
  Seeing accelerating usage growth.
\end{itemize}

kevin roose

It produces these numbers that no one has ever seen before.

\begin{itemize}
\item
  archived recording 1\\
  YouTube now pulls in more than a billion dollars a quarter.
\item
  archived recording 2\\
  It's just an incredible size of audience who are consuming ever more
  video content.
\item
  archived recording 3\\
  It's not so much if you're watching YouTube, it's how much.
\item
  kevin roose\\
  How are you feeling about your work at this point?
\item
  guillaume chaslot\\
  I think we were so excited on working on this project that we didn't
  really question too much that watch time was a good metric. We were
  thinking, yeah, I mean, if people are watching longer, they might be
  happier about what they're watching. So at the time I felt pretty good
  about this. Yeah.
\item
  caleb cain\\
  If I had a moment to stick earbuds in my ear ---
\item
  archived recording (interposing youtube clips)\\
  To reject how we are human by other ---
\item
  caleb cain\\
  Any single moment that I wasn't talking to someone, I was consuming
  content.
\item
  archived recording (joe rogan)\\
  Hey, everybody. This episode of the podcast is brought to you by ---
\item
  guillaume chaslot\\
  But then I realized there were some issues. People were noticing that
  you had a problem with maximizing just watch time that it creates this
  filter bubbles.
\item
  kevin roose\\
  Say more about that.
\item
  guillaume chaslot\\
  The way I explained it when I was at YouTube at the time was ---
\item
  archived recording (youtube clip)\\
  Say hi to everybody.
\item
  guillaume chaslot\\
  When you watch a cat video then the recommendation engine can say, oh,
  you watched a cat video so we're going to give you ---
\item
  archived recording (youtube clip)\\
  Meow.
\item
  guillaume chaslot\\
  --- another cat video, and then another cat video.
\item
  archived recording (youtube clip)\\
  Meow, meow, meow, meow, meow, meow.
\item
  guillaume chaslot\\
  And then another cat video --- more of the same, more of the same,
  more of the same.

  At the time I was really worried about wasting human potential. If you
  could on all of YouTube, but then the thing that's going to keep you
  watching the most is cats, is it the right thing to do to give you,
  again, cats on cats on cats?
\end{itemize}

kevin roose

And over time Guillaume realizes that this filter bubble problem he's
been noticing, it's actually worse than everyone just watching the same
cat videos over and over.

\^{}archived recording\^{} (demonstraters)

{[}CHANTING{]}

\begin{itemize}
\item
  guillaume chaslot\\
  So at the time it was a demonstration in Cairo.
\item
  kevin roose\\
  In Cairo, Egypt?
\item
  guillaume chaslot\\
  Yes.
\item
  archived recording\\
  Violence has erupted in the Egyptian capital Cairo, as those ---
\end{itemize}

kevin roose

He sees these news videos, these political videos, like this conflict in
Egypt that's going on at the time. And he sees that the algorithm is
showing people in different groups, the same thing, over and over.

\begin{itemize}
\item
  guillaume chaslot\\
  You would see a video from the site of the protesters, and then it
  will recommend another video from the site of protesters. So you would
  only see the site of protesters. If you start with the side of the
  police, you would only see the side of the police.
\item
  archived recording\\
  {[}SPEAKING ARABIC{]}
\item
  guillaume chaslot\\
  Then you had only one side of reality. You couldn't see both sides.
\item
  archived recording\\
  {[}SPEAKING ARABIC{]}
\item
  guillaume chaslot\\
  So these two different realities were created.
\item
  kevin roose\\
  Once you recognized that there were these filter bubbles, these sort
  of algorithmic echo chambers, what did you do about it?
\item
  guillaume chaslot\\
  So the first thing I didn't want to do is complain about it and try to
  find, like, the bad example that shows what's wrong with it. Because I
  didn't want to be the grumpy French guy who complains. So what I did
  is side projects. I created, with another engineer, who's still at
  YouTube, we created an algorithm that did the exact opposite. It got
  out of the filter bubble.
\item
  kevin roose\\
  And did any of these side projects have any impact at YouTube? Like,
  did they move into testing, were they implemented, did managers like
  them?
\item
  guillaume chaslot\\
  No, there were always just prototype. But then they were never even
  test on real users.
\item
  kevin roose\\
  And why do you think that is?
\item
  guillaume chaslot\\
  I mean, the way they were saying it is that, OK, it's now our
  objective. And our objective was to increase watch time.
\item
  kevin roose\\
  So, the problem of political polarization --- of giving people only
  one side of a story --- you're noticing this problem while you're at
  YouTube, and it sounds like you are trying to address it through these
  side projects. But you know, your bosses are not saying, Guillaume,
  that's the best idea we've ever heard, let's put it live on the site
  right now. Like, how did things unfold for you at YouTube from there?
\item
  guillaume chaslot\\
  Yeah. So when I proposed the third project to my manager, he told me,
  ``If I were you, I wouldn't work on it too much.'' And then for a few
  months I didn't work on it. Then when I started working on it again,
  then I got fired for a bad performance review. Which is true, because
  then they spend so much time on this project that I spend less time on
  my main project.
\end{itemize}

kevin roose

So Guillaume left Google --- actually left Silicon Valley altogether and
moved back to France. And he says that he kind of stopped thinking about
the YouTube algorithm altogether. Until one day, like an old French
romance, the two meet again.

{[}music{]}

\begin{itemize}
\tightlist
\item
  guillaume chaslot\\
  I was in a bus ride from Lyon to Paris, which was a six-hour bus ride.
  I was working in Paris, but I had family in Lyon, so I was visiting my
  family and then coming back to work in Paris. And there were these new
  buses with Wi-Fi. So I thought, OK, let's give it a try.
\end{itemize}

kevin roose

So he's sitting there on the bus. He's on his laptop. He's doing some
work. And he notices that on the screen next to him ---

\begin{itemize}
\tightlist
\item
  guillaume chaslot\\
  My neighbor was watching YouTube videos for a very long time.
\end{itemize}

kevin roose

The guy was just going from one recommended video to the next
recommended video, to the next to the next and the next.

Advertisement

\protect\hyperlink{after-bottom}{Continue reading the main story}

\hypertarget{site-index}{%
\subsection{Site Index}\label{site-index}}

\hypertarget{site-information-navigation}{%
\subsection{Site Information
Navigation}\label{site-information-navigation}}

\begin{itemize}
\tightlist
\item
  \href{https://help.nytimes3xbfgragh.onion/hc/en-us/articles/115014792127-Copyright-notice}{©~2020~The
  New York Times Company}
\end{itemize}

\begin{itemize}
\tightlist
\item
  \href{https://www.nytco.com/}{NYTCo}
\item
  \href{https://help.nytimes3xbfgragh.onion/hc/en-us/articles/115015385887-Contact-Us}{Contact
  Us}
\item
  \href{https://www.nytco.com/careers/}{Work with us}
\item
  \href{https://nytmediakit.com/}{Advertise}
\item
  \href{http://www.tbrandstudio.com/}{T Brand Studio}
\item
  \href{https://www.nytimes3xbfgragh.onion/privacy/cookie-policy\#how-do-i-manage-trackers}{Your
  Ad Choices}
\item
  \href{https://www.nytimes3xbfgragh.onion/privacy}{Privacy}
\item
  \href{https://help.nytimes3xbfgragh.onion/hc/en-us/articles/115014893428-Terms-of-service}{Terms
  of Service}
\item
  \href{https://help.nytimes3xbfgragh.onion/hc/en-us/articles/115014893968-Terms-of-sale}{Terms
  of Sale}
\item
  \href{https://spiderbites.nytimes3xbfgragh.onion}{Site Map}
\item
  \href{https://help.nytimes3xbfgragh.onion/hc/en-us}{Help}
\item
  \href{https://www.nytimes3xbfgragh.onion/subscription?campaignId=37WXW}{Subscriptions}
\end{itemize}
