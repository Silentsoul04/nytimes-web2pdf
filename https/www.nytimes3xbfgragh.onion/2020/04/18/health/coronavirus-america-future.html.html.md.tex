Sections

SEARCH

\protect\hyperlink{site-content}{Skip to
content}\protect\hyperlink{site-index}{Skip to site index}

\href{/section/health}{Health}\textbar{}The Coronavirus in America: The
Year Ahead

\url{https://nyti.ms/3eyhHCk}

\begin{itemize}
\item
\item
\item
\item
\item
\item
\end{itemize}

\href{https://www.nytimes3xbfgragh.onion/news-event/coronavirus?action=click\&pgtype=Article\&state=default\&region=TOP_BANNER\&context=storylines_menu}{The
Coronavirus Outbreak}

\begin{itemize}
\tightlist
\item
  live\href{https://www.nytimes3xbfgragh.onion/2020/08/04/world/coronavirus-cases.html?action=click\&pgtype=Article\&state=default\&region=TOP_BANNER\&context=storylines_menu}{Latest
  Updates}
\item
  \href{https://www.nytimes3xbfgragh.onion/interactive/2020/us/coronavirus-us-cases.html?action=click\&pgtype=Article\&state=default\&region=TOP_BANNER\&context=storylines_menu}{Maps
  and Cases}
\item
  \href{https://www.nytimes3xbfgragh.onion/interactive/2020/science/coronavirus-vaccine-tracker.html?action=click\&pgtype=Article\&state=default\&region=TOP_BANNER\&context=storylines_menu}{Vaccine
  Tracker}
\item
  \href{https://www.nytimes3xbfgragh.onion/2020/08/02/us/covid-college-reopening.html?action=click\&pgtype=Article\&state=default\&region=TOP_BANNER\&context=storylines_menu}{College
  Reopening}
\item
  \href{https://www.nytimes3xbfgragh.onion/live/2020/08/04/business/stock-market-today-coronavirus?action=click\&pgtype=Article\&state=default\&region=TOP_BANNER\&context=storylines_menu}{Economy}
\end{itemize}

\includegraphics{https://static01.graylady3jvrrxbe.onion/images/2020/04/19/science/18VIRUS-FUTURE1/00VIRUS-FUTURE1-articleLarge.jpg?quality=75\&auto=webp\&disable=upscale}

\hypertarget{the-coronavirus-in-america-the-year-ahead}{%
\section{The Coronavirus in America: The Year
Ahead}\label{the-coronavirus-in-america-the-year-ahead}}

There will be no quick return to our previous lives, according to nearly
two dozen experts. But there is hope for managing the scourge now and in
the long term.

A tent in an outdoor I.C.U. hospital in Central Park.Credit...Misha
Friedman for The New York Times

Supported by

\protect\hyperlink{after-sponsor}{Continue reading the main story}

\href{https://www.nytimes3xbfgragh.onion/by/donald-g-mcneil-jr}{\includegraphics{https://static01.graylady3jvrrxbe.onion/images/2018/06/13/multimedia/author-donald-g-mcneil-jr/author-donald-g-mcneil-jr-thumbLarge-v4.png}}

By
\href{https://www.nytimes3xbfgragh.onion/by/donald-g-mcneil-jr}{Donald
G. McNeil Jr.}

\begin{itemize}
\item
  Published April 18, 2020Updated June 1, 2020
\item
  \begin{itemize}
  \item
  \item
  \item
  \item
  \item
  \item
  \end{itemize}
\end{itemize}

\href{https://www.nytimes3xbfgragh.onion/es/2020/04/21/espanol/ciencia-y-tecnologia/coronavirus-futuro.html}{Leer
en español}

\hypertarget{listen-to-this-article}{%
\subsubsection{Listen to This Article}\label{listen-to-this-article}}

Audio Recording by Audm

\emph{To hear more audio stories from publishers, like The New York
Times, download}
\href{https://www.audm.com/?utm_source=nyt\&utm_medium=embed\&utm_campaign=coronavirus_year_ahead}{\emph{Audm
for iPhone or Android}}\emph{.}

The
\href{https://www.nytimes3xbfgragh.onion/2020/06/01/nyregion/nyc-coronavirus-flatbush-brooklyn.html}{coronavirus}
is spreading from America's biggest cities to its suburbs, and has begun
encroaching on the nation's rural regions. The virus is believed to have
infected millions of citizens and has killed more than 34,000.

Yet President Trump this week proposed guidelines for reopening the
economy and suggested that a swath of the United States would soon
resume something resembling normalcy. For weeks now, the
administration's view of the crisis and our future has been rosier than
that of its own medical advisers, and of scientists generally.

In truth, it is not clear to anyone where this crisis is leading us.
More than 20 experts in public health, medicine, epidemiology and
history shared their thoughts on the future during in-depth interviews.
When can we emerge from our homes? How long, realistically, before we
have a treatment or vaccine? How will we keep the virus at bay?

Some felt that American ingenuity, once fully engaged, might well
produce advances to ease the burdens. The path forward depends on
factors that are certainly difficult but doable, they said: a carefully
staggered approach to reopening, widespread testing and surveillance, a
treatment that works, adequate resources for health care providers ---
and eventually an effective vaccine.

Still, it was impossible to avoid gloomy forecasts for the next year.
The scenario that Mr. Trump has been unrolling at his daily press
briefings --- that the lockdowns will end soon, that a protective pill
is almost at hand, that football stadiums and restaurants will soon be
full --- is a fantasy, most experts said.

``We face a doleful future,'' said Dr. Harvey V. Fineberg, a former
president of the National Academy of Medicine.

He and others foresaw an unhappy population trapped indoors for months,
with the most vulnerable possibly quarantined for far longer. They
worried that a vaccine would initially elude scientists, that weary
citizens would abandon restrictions despite the risks, that the virus
would be with us from now on.

``My optimistic side says the virus will ease off in the summer and a
vaccine will arrive like the cavalry,'' said Dr. William Schaffner, a
preventive medicine specialist at Vanderbilt University medical school.
``But I'm learning to guard against my essentially optimistic nature.''

Most experts believed that once the crisis was over, the nation and its
economy would revive quickly. But there would be no escaping a period of
intense pain.

Exactly how the pandemic will end depends in part on medical advances
still to come. It will also depend on how individual Americans behave in
the interim. If we scrupulously protect ourselves and our loved ones,
more of us will live. If
\href{https://www.nytimes3xbfgragh.onion/2020/04/14/us/bishop-gerald-glenn-coronavirus.html}{we
underestimate the virus}, it will find us.

\hypertarget{more-americans-may-die-than-the-white-house-admits}{%
\subsubsection{More Americans may die than the White House
admits.}\label{more-americans-may-die-than-the-white-house-admits}}

\includegraphics{https://static01.graylady3jvrrxbe.onion/images/2020/04/19/science/18VIRUS-FUTURE2/merlin_171637683_2e5b6d03-b9c0-4671-bcc6-8f847fb7618d-articleLarge.jpg?quality=75\&auto=webp\&disable=upscale}

\includegraphics{https://static01.graylady3jvrrxbe.onion/images/2017/01/29/podcasts/the-daily-album-art/the-daily-album-art-articleInline-v2.jpg?quality=75\&auto=webp\&disable=upscale}

\hypertarget{listen-to-the-daily-the-next-year-or-two-of-the-pandemic}{%
\subsubsection{Listen to `The Daily': The Next Year (or Two) of the
Pandemic}\label{listen-to-the-daily-the-next-year-or-two-of-the-pandemic}}

Immunity passports. Stop-start lockdowns. Strategic self-infection. This
is what our new normal might look like.

transcript

Back to The Daily

bars

0:00/27:22

-27:22

transcript

\hypertarget{listen-to-the-daily-the-next-year-or-two-of-the-pandemic-1}{%
\subsection{Listen to `The Daily': The Next Year (or Two) of the
Pandemic}\label{listen-to-the-daily-the-next-year-or-two-of-the-pandemic-1}}

\hypertarget{hosted-by-michael-barbaro-produced-by-alexandra-leigh-young-and-clare-toeniskoetter-with-help-from-daniel-guillemette-and-edited-by-mj-davis-lin-and-theo-balcomb}{%
\subsubsection{Hosted by Michael Barbaro; produced by Alexandra Leigh
Young and Clare Toeniskoetter; with help from Daniel Guillemette; and
edited by M.J. Davis Lin and Theo
Balcomb}\label{hosted-by-michael-barbaro-produced-by-alexandra-leigh-young-and-clare-toeniskoetter-with-help-from-daniel-guillemette-and-edited-by-mj-davis-lin-and-theo-balcomb}}

\hypertarget{immunity-passports-stop-start-lockdowns-strategic-self-infection-this-is-what-our-new-normal-might-look-like}{%
\paragraph{Immunity passports. Stop-start lockdowns. Strategic
self-infection. This is what our new normal might look
like.}\label{immunity-passports-stop-start-lockdowns-strategic-self-infection-this-is-what-our-new-normal-might-look-like}}

\begin{itemize}
\item
  michael barbaro\\
  From The New York Times, I'm Michael Barbaro. This is ``The Daily.''

  Today: As President Trump urges states to begin reopening their
  economies, a debate begins over when and how to end the lockdowns.
  Science reporter Donald G. McNeil Jr. on what that might look like.

  It's Monday, April 20.

  So Donald, we have come to you at just about every turn in this
  pandemic to understand what's next, and the portraits of the future
  that you have painted for us each time we talk have been strikingly
  accurate. You told us all the way back in February to prepare for
  lockdowns. Those happened. You told us to prepare for high death
  rates. You said that people we know would die, and that, sadly, has
  happened. You warned us of shortages of medical supplies. That too has
  happened. Just about everything you said would happen has more or less
  happened. So I want to turn to the next installment of this rolling
  conversation we've been having with you.
\item
  donald g. mcneil jr.\\
  OK.
\item
  michael barbaro\\
  And start with a question that I think is on everyone's mind right
  now, which is when and how we start to reopen our society and what
  that would look like.
\item
  donald g. mcneil jr.\\
  OK. Well, look, I'm not some dark angel who's simply looking into the
  future.
\item
  michael barbaro\\
  Understood.
\item
  donald g. mcneil jr.\\
  I'm talking to experts. I'm looking at the other pandemics I've
  covered. I'm talking to medical historians and making predictions
  based on data. And so I think a lot of people think that by May or
  June or August, we're going to return to something like normalcy. But
  all the experts I talk to say, no, that's a fantasy.
\item
  michael barbaro\\
  Why is it a fantasy?
\item
  donald g. mcneil jr.\\
  Well, because if we all tried to come out at once, everything would
  look cool for about three weeks. And a week or two after that the
  emergency rooms would start to fill again, and people would start to
  die again. Flattening the curve is a notion that people love, but when
  we say we're flattening the curve, no, we're plateauing at a very high
  level of the curve. That means a steady rate of deaths. So what we
  want is to see the lockdown last until we get back down to close to
  what the normal baseline rate of deaths is. And it's going to be piece
  by piece. We might be able to let a certain number of people ride the
  subway each day. People still staying apart from each other. We're not
  going to be able to let people sit next to each other in football
  stadiums. Maybe the team will play on the field, and there will be
  some cameramen in the stands, but it's not going to be the way it was
  before.
\item
  michael barbaro\\
  OK. So if there's not going to be any kind of single moment where the
  curtain goes up and life resumes, what does a lifting of these
  lockdowns start to look like?
\item
  donald g. mcneil jr.\\
  Well, the best description I heard of it was in an article. It was on
  March 19, by Tomas Pueyo, a writer in San Francisco. And it was called
  ``The Hammer and the Dance.'' And the hammer is the lockdown. There
  was no question of social distancing light. It was more like, bam,
  everybody has to go into their houses. But then, once you get the
  deaths down to a minimum, then you begin the dance. And the dance is
  dancing a little bit out, and opening up some of the restrictions and
  seeing, you know, how far can you let people go. And then you see what
  that does to death rates. And once the death rates go up, you have to
  go back and leave the dance floor and go into lockdown again. And it
  repeats again and again. And it's dance in, dance out, dance in, dance
  out. And basically that's what the epidemiological models show we have
  to do.
\item
  michael barbaro\\
  So in this scenario, tiptoeing back to normal happens, and if there is
  a resurgence in infections, then the hammer comes back down.
\item
  donald g. mcneil jr.\\
  Exactly.
\item
  michael barbaro\\
  So let's talk about what the opening starts to look like. Describe the
  dance.
\item
  donald g. mcneil jr.\\
  OK. So the dance is you have this whole list of things you can ask
  people to do. One of them being wear masks. Another one is sit six
  feet apart from your dining companions at a restaurant. So a
  restaurant that had 100 customers before now has about 10 customers in
  it. You have to cut down travel between cities, because people from a
  hot zone, like New York recently, could go to a cool zone and start
  infections. International travelers will have to be quarantined for 14
  days so that we're sure they're not infected when they come in. So we
  will have to find restrictions on people's movements that keeps them
  from getting too close and letting the virus spread again. Because
  when the virus spreads, deaths follow.
\item
  michael barbaro\\
  OK. Let's get a little more specific here. What does the dance mean
  for, let's say, schools?
\item
  donald g. mcneil jr.\\
  Well, schools is a big mystery, because we don't know much about the
  virus in children. Very, very few children get hospitalized and die.
  It happens, but it's very uncommon. But the question is, how much do
  the kids get the virus and transmit the virus? Because if you open
  schools, the children may be fine, but the teachers won't necessarily
  be, and the children's parents and grandparents won't be. So mayors
  and governors are going to have to decide, maybe we're going to open
  school for two weeks and see what happens? Or maybe we're going to let
  half the kids go to school this week but have them sitting six feet
  apart, and then next week the other half of the kids get to come to
  school.
\item
  michael barbaro\\
  And Donald, what about the typical workplace. From the way you're
  describing schools, it sounds like, at the very least, people at an
  office or a factory would need to be six feet apart. What else?
\item
  donald g. mcneil jr.\\
  Well, it depends on what you mean by the average workplace. Are you a
  construction crew? Are you an office, like The New York Times? Are you
  a factory? Are you a food processing plant? Everyone is different. And
  for those of us who work most of the day on computers, there's not a
  whole lot of reason to go back to the workplace. It may be more of a
  source of danger than any advantage you get from being there. Whereas
  if you work in a chicken processing plant, you've got to be near the
  chickens. So somehow they're going to have to stretch out those
  chicken disassembly lines, so that people can do it safely without
  infecting each other. And they'll have to wear masks, and maybe
  there'll be plexiglass between the workers. There's lots of measures
  you can take, and they're going to be different for every single
  workplace.
\item
  michael barbaro\\
  I want to talk for just a moment about geography, because I wonder if
  the scenarios you're describing would apply as rigorously in one part
  of, for example, the United States as another, based on issues like
  density or the cycle of this virus. There are many places in the
  United States that have been largely untouched by this virus. Or there
  are places like Washington state, where it seems the virus has turned
  a corner. So would this hammer and dance scenario apply in the same
  way to every corner of the country? Or could it be applied more
  lightly or more severely, based on density, based on number of
  infections?
\item
  donald g. mcneil jr.\\
  Yes and no. It's a fantasy to believe that there are any parts of the
  country that are untouched by this virus. It is much more intense in
  some places, like New York and New Orleans and Detroit, but it's
  everywhere. If you look at any map of cases, it's certainly in every
  state. It's in multiple counties in every state. So it's going to have
  to be different techniques developed, not just for every state in
  every county, but for every work situation, so that you are not
  transmitting virus to each other.
\item
  michael barbaro\\
  Donald, is the point of the hammer and the dance to avoid having the
  rest of the country get infected, or does it simply slow a inevitable
  version of mass infection but not prevent it?
\item
  donald g. mcneil jr.\\
  It slows it down. A certain amount of infection is inevitable, because
  emergency workers are going to have to be at their jobs. Medical
  workers are going to get infected. There's going to be a certain level
  of infection at all times, but you want to damp it down to the
  minimum. But all the models say that maybe 3 percent of America is
  infected by now. The other epidemiologists think maybe it's 10
  percent. That 10 percent is a pretty high guess. So that still leaves
  300 million Americans uninfected. So we're a country that's somewhere
  north of 90 percent susceptible. And each time we do the dance out and
  dance in, we chop a few more people off that susceptible list. A few
  of them are dead. A bunch of them become immune. But it only proceeds
  in fits and starts. Maybe we go three million at a time, three million
  at a time, three million at a time. It's not a fast process.
\item
  michael barbaro\\
  And how long do the experts you talked to say that this dance will
  need to occur? At some point, do we stop dancing and we just get to go
  back to normal?
\item
  donald g. mcneil jr.\\
  The dance will end when we have a vaccine that we can all take, or if
  we have a prophylactic pill that will protect us all, like PrEP
  protects you from HIV. Or when so many of us would become infected ---
  70-plus percent have become infected, and the virus dies out or slows
  down just because it can't find any new victims to infect.
\item
  michael barbaro\\
  Mm-hmm.
\item
  donald g. mcneil jr.\\
  But making a vaccine within 18 months is extremely optimistic. The
  record we've ever had for producing a vaccine is four years. That was
  a mumps vaccine produced back in the `50s. We've got new techniques
  that speed it up, but some things can't be sped up. And then after we
  design the vaccine, we have to think about producing the vaccine. If
  we need 300 million doses of vaccine, or if we need two shots --- if
  we need 600 million doses of vaccine --- that's a gigantic
  undertaking. I talked to some vaccine production experts, and they
  said, the average vaccine plant in America makes 5 to 10 million
  doses. So we've got to find some way to make 300 to 600 million doses
  of vaccine, and we have to start planning now, even before the vaccine
  exists.
\item
  michael barbaro\\
  Donald, you just said that the record for making a vaccine is four
  years. Are you saying that there is a version of this where this
  dance, in the hammer and dance metaphor, goes on not for one year but
  something like four years?
\item
  donald g. mcneil jr.\\
  That's a kind of worst case scenario, as far as timing is concerned.
  Everything may get lowered in this. We may test vaccines in a way that
  we would think of as ethically unthinkable in normal times. We may go
  with somewhat lower safety standards for vaccines, because we're
  facing a situation where life is dangerous. But yeah, the fastest
  human vaccine ever made was mumps. Four years, from start to finish.
\item
  michael barbaro\\
  So I just want to be clear. Is there a world, according to the experts
  you're talking to, who say the world of tiptoeing out and coming back
  and realizing that there's a resurgence, and then tiptoeing back out
  again, that that could last years?
\item
  donald g. mcneil jr.\\
  Yes.
\item
  michael barbaro\\
  And during that, the virus lasts? It doesn't die off in the hot
  summers?
\item
  donald g. mcneil jr.\\
  There is zero evidence that this virus dies off in hot weather. The
  virus is circulating now in India, in Brazil, in Australia.
\item
  michael barbaro\\
  Very, very hot environments.
\item
  donald g. mcneil jr.\\
  Exactly. There's no evidence that it's going to die out in the summer.
\item
  michael barbaro\\
  Donald, I'm willing to speak for many listeners when I say that the
  idea of living in masks, with everyone sitting six feet apart, and the
  expectation that a viral resurgence might mean that even that becomes
  too risky and so everyone goes back under lockdown --- that that could
  go on for years is really pretty depressing and kind of hard to wrap
  one's head around.
\item
  donald g. mcneil jr.\\
  I know. But if we don't go into a dance, means there's going to be
  more infections and more deaths.
\item
  {[}music{]}
\item
  michael barbaro\\
  We'll be right back.

  Don, during this period, it feels like not everyone is going to be on
  equal footing, in the sense that some of us will have been infected
  and will possess some kind of immunity to Covid-19, right? And many of
  us, of course, will not have been infected and will not possess that
  immunity. So how does that factor into this, give and take, this dance
  you're describing?
\item
  donald g. mcneil jr.\\
  Well, we're already seeing this happen. There are a very small number
  of people now who are immune. And they're very much in demand. They're
  in demand for their blood, because you can harvest antibodies from
  their blood. They're in demand for doing medical jobs that are
  dangerous, like intubating patients, because they can do it without
  fear of infection. They'll be in demand for all sorts of other jobs,
  because they can travel anywhere. They can do face-to-face
  interactions with customers without any fear. And they're going to be
  a special class in society. There'll be different standards, different
  ways of living for two different classes in society. So say you're an
  immune and I'm not. I basically have to stay inside here in hiding,
  and I'm depressed as hell about it. I have a granddaughter who's going
  to be born in June, and I sit here in my apartment thinking, I may not
  see her until she's two years old. And that's horrible to contemplate,
  but it may be a fact. Meanwhile, you, as an immune, could go visit my
  granddaughter or go out and take my job, or ---
\item
  michael barbaro\\
  No one can take your job, Donald, just to be clear.
\item
  donald g. mcneil jr.\\
  Not a heck of a lot of people want my job. There'll be different
  standards, different ways of living for two different classes in
  society for a while. It's quite science fiction-y. It's kind of
  dystopian.
\item
  michael barbaro\\
  Right. That kind of a societal schism, right, between the immune and
  the still susceptible, it would seem to require a few pieces of firm
  knowledge. One is that there's a widespread ability to tell who is
  immune versus who is not, and the other piece of knowledge is that we
  know for sure that being infected confers immunity. So how do you
  resolve either of those?
\item
  donald g. mcneil jr.\\
  We don't know those things yet. We don't know them for dead sure, but
  as Tony Fauci says, usually when you recover from disease, you're
  immune for a while. But we don't know how long the immunity lasts,
  because this disease has only been around since November at the
  earliest.
\item
  michael barbaro\\
  Mm-hmm. Are there plans? Do you envision a universe in which people
  literally wear some sort of sign that they are immune and that that
  can be rigorously proven?
\item
  donald g. mcneil jr.\\
  Yes. That is already beginning to take place. Germany is talking about
  issuing certificates, but you don't want something you can pass from
  person to person. So China takes people's cell phones and has a QR
  code that you read as you come into the subway, you come into a
  restaurant, you come into any place where somebody can check your
  phone. And that readout tells the person looking at your phone that
  you're immune. So yeah, people are already thinking about this.
\item
  michael barbaro\\
  In the scenario that you're laying out, immunity and the ability to
  show immunity would seem to be incredibly desirable and might create
  some very complicated incentives.
\item
  donald g. mcneil jr.\\
  Yes. People are going to be tempted to become immunes and to do it the
  chickenpox party way.
\item
  michael barbaro\\
  What do you mean?
\item
  donald g. mcneil jr.\\
  To deliberately infect themselves, hoping for the best, but that they
  emerge at the end of it as an immune. That temptation is going to be
  enormous.
\item
  michael barbaro\\
  You're describing deliberate and perhaps large scale acts of
  self-infection.
\item
  donald g. mcneil jr.\\
  Yes. I am. There is a lot of historical precedent for that stuff. In
  the years before smallpox vaccine, people used to take pus from the
  blisters of smallpox victims, or dried up scabs from smallpox victims,
  and put the pus into --- stab a little hole in their child's arm and
  put some of the pus in there, or blow the scabs into your nose. And
  that would give you a usually mild case of smallpox. And I talked to
  an immunologist who said, look, the child had about a 1 percent chance
  of dying, and the parents recognized that. And that was better,
  because if you had a smallpox epidemic come through, usually, about a
  third of the susceptibles died. So if you had six kids, you might
  think hard about giving them all a little bit of smallpox in their
  arm.
\item
  michael barbaro\\
  Don, what you're describing is a very calculated risk that some people
  may take to self-infect, and I wonder just how dangerous it may be if
  people miscalculate.
\item
  donald g. mcneil jr.\\
  With the virus acting the way it is acting now, it's very dangerous.
  We know that there are people who are more at risk, but we also know
  that it's unpredictable --- that young, athletic, healthy people,
  totally unpredictably, end up on a ventilator, and a few of them die.
\item
  michael barbaro\\
  Right.
\item
  donald g. mcneil jr.\\
  So it's people making life or death decisions for freedom.
\item
  michael barbaro\\
  And a paycheck.
\item
  donald g. mcneil jr.\\
  And a paycheck, yes.
\item
  michael barbaro\\
  Everything that you're talking about here seems to assume the absence
  of a successful treatment and a world where we're pretty much waiting
  for a vaccine. Could that change? When we've asked you this question
  in the past, you have said that there wasn't a viable treatment. Do
  you have any sense that that may be changing and that a treatment
  might come before a vaccine?
\item
  donald g. mcneil jr.\\
  I don't have any sense that it's changing in an important way. There
  is something that doctors place some hope in, and that is taking the
  blood from people who've recovered from the disease, and removing the
  red blood cells and the white blood cells and everything else and
  leaving just the antibodies. Those antibodies glob onto the virus and
  neutralize it, and so you can inject them into somebody who's sick.
  But it's a technology that is not fast-moving, so it's not in the
  immediate future. But scientists are hopeful that it's in the mid-term
  future.
\item
  michael barbaro\\
  So in other words, for the time being, we are left pretty much with
  the hammer and the dance as our most practical solution.
\item
  donald g. mcneil jr.\\
  Exactly.
\item
  michael barbaro\\
  So Donald, with everything that you just said in mind, I want to
  return to that first time we ever spoke. Because when we had that
  conversation, you were well ahead of anybody else in preparing us for
  what the pandemic would mean on a personal level. You told us about
  your stockpile of food, your month worth of medicine, and you told us
  that we should be prepared not to see our friends and our family for a
  while. And all of that proved prescient. But there's one area where I
  recall you, perhaps, underestimated something. You said be prepared
  for a couple of months of something that might look like a lockdown.
  And now, it's feeling like it's going to last a lot longer. So with
  that in mind, what preparations are you making now that the rest of us
  may end up making for ourselves in the coming days, weeks, months, to
  prepare for that new reality?
\item
  donald g. mcneil jr.\\
  I'm busy. I'm working harder than I've ever worked hard before, and
  that's given me a sense of purpose. That gives me some psychological
  stability that a lot of other people don't have the luck to have. I'm
  thinking about what am I going to do to entertain myself? I'm thinking
  of buying a fishing rod and becoming one of those old guys on the
  piers who's fishing for bluefish. I like trout fishing, but it seems
  unlikely I'm going to get away to do that for a while. So it's just,
  you have to mentally prepare yourself for these things. And now be it
  said, I have visions for the future that are rosy, and that keeps me
  going too.
\item
  michael barbaro\\
  You do?
\item
  donald g. mcneil jr.\\
  Yeah. If you look at the periods after World War I and after World War
  II, not only did the economy come back after both those periods.
  Remember, after the one-two punch of World War I plus the Spanish
  influenza, we had the roaring `20s. After World War II, we had a
  period in which, in Europe, the war widows and pensions fund became
  the European welfare state. War profiteers suddenly faced higher taxes
  and a lot of scrutiny. People had pretty much had it with the rich by
  that time, and they wanted more. They wanted the G.I. Bill. They
  wanted federally-sponsored mortgages for housing. They wanted better
  health care. And to some extent, they got it. I was born in 1954, and
  my parents were of the generation that had made it through the Great
  Depression and through the Second World War. And they had a different
  attitude on life. They had a sense of, hey, we did this incredible
  thing. We did it all together. We got through it. I hope that sort of
  era comes again, that people will take more pleasure in small things
  in life. And that we will do more to make sure that we take care of
  each other, and that that'll go through all levels of society. And
  like I said, I try not to predict the future, but maybe something like
  that will emerge from this.
\item
  michael barbaro\\
  Well, of all the visions that you have given us, I like that one the
  best.
\item
  donald g. mcneil jr.\\
  Me too.
\item
  {[}music{]}
\item
  michael barbaro\\
  Donald, thank you very much.
\item
  donald g. mcneil jr.\\
  Thank you.
\item
  michael barbaro\\
  On Friday, Vice President Mike Pence declared that states have
  sufficient amount of testing to begin the process of reopening their
  economies, a claim that governors in multiple states immediately
  rejected as untrue. It's not accurate to say there's plenty of testing
  out there, and the governor should just get it done. That's just not
  being straightforward. In interviews on CNN on Sunday, the governor of
  Maryland, Larry Hogan, and the governor of Virginia, Ralph Northam,
  said they would need significantly higher levels of testing before
  they would consider lifting their lockdowns.
\item
  archived recording (ralph northam)\\
  That's just delusional to be making statements like that. For the
  national level to say that we have what we need and really to have no
  guidance to the state levels is just irresponsible, because we're not
  there yet.
\end{itemize}

{[}music{]}

michael barbaro

We'll be right back.

Here's what else you need to today.

\begin{itemize}
\tightlist
\item
  archived recording (protestors)\\
  {[}CHANTING{]}
\end{itemize}

michael barbaro

Demonstrators took to the streets in several states across the country
over the weekend to protest stay-at-home orders that they said were
destroying the economy.

\begin{itemize}
\tightlist
\item
  archived recording (protestor)\\
  I hope Whitmer understands that she's not just going to be able to
  rule Michigan like other politicians rule China or something like
  that. We're going to take our liberties, and we're going to keep them,
  and we're going to fight for them.
\end{itemize}

michael barbaro

The protests, which drew hundreds of people in Michigan, Texas and North
Carolina, among other states, were largely organized by anti-government
activists and libertarian groups, and were encouraged by President
Trump, who took to Twitter to urge residents to liberate their states.
But polling shows that most Americans support the lockdowns. A recent
survey by Pew Research found that 66 percent of respondents feared that
states would lift restrictions too quickly.

{[}music{]}

That's it for ``The Daily.'' I'm Michael Barbaro. See you tomorrow.

Covid-19, the illness caused by the coronavirus, is arguably
\href{https://public.flourish.studio/visualisation/1830480/}{the leading
cause of death in the United States} right now. The virus has killed
more than 1,800 Americans almost every day since April 7, and the
official toll
\href{https://www.nytimes3xbfgragh.onion/2020/04/10/nyregion/new-york-coronavirus-death-count.html}{may
be an undercount}.

By comparison, heart disease typically kills 1,774 Americans a day, and
cancer kills 1,641.

Yes, the coronavirus curves are plateauing. There are fewer hospital
admissions in New York, the center of the epidemic, and fewer Covid-19
patients in I.C.U.s. The daily death toll is still grim, but no longer
rising.

An epidemiological model produced by the White House originally
predicted 100,000 to 240,000 deaths by midsummer. Another model, by the
University of Washington's Institute for Health Metrics and Evaluation,
now puts that figure at 60,000.

While this is encouraging news, it masks some significant concerns. The
institute's projection runs through Aug. 4, describing only the first
wave of this epidemic. Without a vaccine, the virus is expected to
circulate for years, and the death tally will rise over time.

The gains to date were achieved only by shutting down the country, a
situation that cannot continue indefinitely. The White House's
``phased'' plan for reopening will surely raise the death toll no matter
how carefully it is executed. The best hope is that fatalities can be
held to a minimum.

Reputable longer-term projections for how many Americans will die vary,
but they are all grim. Various experts
\href{https://www.nytimes3xbfgragh.onion/2020/03/13/us/coronavirus-deaths-estimate.html}{consulted
by the Centers for Disease Control and Prevention in March} predicted
that the virus eventually could reach 48 percent to 65 percent of all
Americans, with a fatality rate just under 1 percent, and would kill up
to 1.7 million of them if nothing were done to stop the spread.

A model by researchers at Imperial College London
\href{https://www.whitehouse.gov/briefings-statements/remarks-president-trump-vice-president-pence-members-coronavirus-task-force-press-briefing-14/}{cited
by the president} on March 30
\href{https://www.imperial.ac.uk/media/imperial-college/medicine/sph/ide/gida-fellowships/Imperial-College-COVID19-NPI-modelling-16-03-2020.pdf}{predicted
2.2 million deaths} in the United States by September under the same
circumstances.

By comparison, about 420,000 Americans died in World War II.

The limited data from China are even more discouraging. Its epidemic has
been halted --- for the moment --- and virtually everyone infected in
its first wave has died or recovered.

China has officially reported
\href{https://www.worldometers.info/coronavirus/}{about 83,000 cases and
4,632 deaths}, which is a fatality rate of over 5 percent. The Trump
administration
\href{https://www.nytimes3xbfgragh.onion/2020/04/02/us/politics/cia-coronavirus-china.html}{has
questioned the figures} but has not produced more accurate ones.

Fatality rates
\href{https://www.nytimes3xbfgragh.onion/2020/04/10/nyregion/new-york-coronavirus-death-count.html}{depend
heavily on how overwhelmed} hospitals get and what percentage of cases
are tested. China's estimated death rate was 17 percent in the first
week of January, when Wuhan was in chaos, according to
\href{https://www.cebm.net/covid-19/global-covid-19-case-fatality-rates/}{a
Center for Evidence-Based Medicine report}, but only 0.7 percent by late
February.

In this country, hospitals in several cities, including New York, came
to the brink of chaos. Officials in both
\href{https://www.nytimes3xbfgragh.onion/2020/04/17/world/coronavirus-news.html\#link-f029419}{Wuhan}
and
\href{https://www.nytimes3xbfgragh.onion/2020/04/14/nyregion/new-york-coronavirus-deaths.html}{New
York} had to revise their death counts upward this week when they
realized that many people had died at home of Covid-19, strokes, heart
attacks or other causes, or because ambulances never came for them.

In fast-moving epidemics, far more victims pour into hospitals or die at
home than doctors can test; at the same time, the mildly ill or
asymptomatic never get tested. Those two factors distort the true
fatality rate in opposite ways. If you don't know how many people are
infected, you
\href{https://www.nytimes3xbfgragh.onion/2020/04/17/us/coronavirus-death-rate.html}{don't
know how deadly a virus is}.

Only when tens of thousands of antibody tests are done will we know how
many silent carriers there may be in the United States. The C.D.C.
\href{https://www.nytimes3xbfgragh.onion/2020/03/31/health/coronavirus-asymptomatic-transmission.html}{has
suggested it might be 25 percent} of those who test positive.
Researchers in Iceland
\href{https://www.cnn.com/2020/04/01/europe/iceland-testing-coronavirus-intl/index.html}{said
it might be double that}.

China is also revising its own estimates. In February, a
\href{https://jamanetwork.com/journals/jama/fullarticle/2762130}{major
study} concluded that only 1 percent of cases in Wuhan were
asymptomatic.
\href{https://www.scmp.com/news/china/society/article/3080058/coronavirus-china-launches-study-asymptomatic-cases-and-shared}{New
research} says
\href{https://www.scmp.com/news/china/society/article/3079012/coronavirus-68-cent-cases-confirmed-china-past-eight-days-had-no}{perhaps
60 percent} were. Our knowledge gaps are still wide enough to make
epidemiologists weep.

``All models are just models,'' Dr. Anthony S. Fauci, science adviser to
the White House coronavirus task force, has said. ``When you get new
data, you change them.''

There may be good news buried in this inconsistency: The virus may also
be mutating to cause fewer symptoms. In the movies, viruses become more
deadly. In reality, they usually become less so, because asymptomatic
strains reach more hosts. Even the 1918 Spanish flu virus eventually
\href{https://www.ncbi.nlm.nih.gov/pmc/articles/PMC3180813/}{faded into
the seasonal H1N1 flu}.

\hypertarget{latest-updates-global-coronavirus-outbreak}{%
\section{\texorpdfstring{\href{https://www.nytimes3xbfgragh.onion/2020/08/04/world/coronavirus-cases.html?action=click\&pgtype=Article\&state=default\&region=MAIN_CONTENT_1\&context=storylines_live_updates}{Latest
Updates: Global Coronavirus
Outbreak}}{Latest Updates: Global Coronavirus Outbreak}}\label{latest-updates-global-coronavirus-outbreak}}

Updated 2020-08-04T21:57:55.984Z

\begin{itemize}
\tightlist
\item
  \href{https://www.nytimes3xbfgragh.onion/2020/08/04/world/coronavirus-cases.html?action=click\&pgtype=Article\&state=default\&region=MAIN_CONTENT_1\&context=storylines_live_updates\#link-2daa96b5}{As
  talks drag on, McConnell signals openness to jobless aid extension
  that Republicans have opposed.}
\item
  \href{https://www.nytimes3xbfgragh.onion/2020/08/04/world/coronavirus-cases.html?action=click\&pgtype=Article\&state=default\&region=MAIN_CONTENT_1\&context=storylines_live_updates\#link-1228a480}{Novavax
  sees encouraging results from two studies of its experimental
  vaccine.}
\item
  \href{https://www.nytimes3xbfgragh.onion/2020/08/04/world/coronavirus-cases.html?action=click\&pgtype=Article\&state=default\&region=MAIN_CONTENT_1\&context=storylines_live_updates\#link-4825b93}{Public
  and private schools in Maryland and elsewhere are divided over
  in-person instruction.}
\end{itemize}

\href{https://www.nytimes3xbfgragh.onion/2020/08/04/world/coronavirus-cases.html?action=click\&pgtype=Article\&state=default\&region=MAIN_CONTENT_1\&context=storylines_live_updates}{See
more updates}

More live coverage:
\href{https://www.nytimes3xbfgragh.onion/live/2020/08/04/business/stock-market-today-coronavirus?action=click\&pgtype=Article\&state=default\&region=MAIN_CONTENT_1\&context=storylines_live_updates}{Markets}

At the moment, however, we do not know
\href{https://www.scmp.com/news/china/science/article/3079879/chinas-initial-coronavirus-outbreak-wuhan-spread-twice-fast-we}{exactly
how transmissible}
\href{https://www.nytimes3xbfgragh.onion/2020/04/17/us/coronavirus-death-rate.html}{or
lethal the virus is}. But refrigerated trucks parked outside hospitals
tell us all we need to know: It is far worse than a bad flu season.

\hypertarget{the-lockdowns-will-end-but-haltingly}{%
\subsubsection{The lockdowns will end, but
haltingly.}\label{the-lockdowns-will-end-but-haltingly}}

Image

Commuters on the Staten Island Ferry.Credit...Misha Friedman for The New
York Times

No one knows exactly what percentage of Americans have been infected so
far --- estimates have ranged from 3 percent to 10 percent --- but it is
likely a safe bet that at least 300 million of us are still vulnerable.

Until a vaccine or another protective measure emerges, there is no
scenario, epidemiologists agreed, in which it is safe for that many
people to suddenly come out of hiding. If Americans pour back out in
force, all will appear quiet for perhaps three weeks.

Then the emergency rooms will get busy again.

``There's this magical thinking saying, `We're all going to hunker down
for a while and then the vaccine we need will be available,''' said Dr.
Peter J. Hotez, dean of the National School of Tropical Medicine at
Baylor College of Medicine.

In his wildly popular March 19 article in Medium,
``\href{https://medium.com/@tomaspueyo/coronavirus-the-hammer-and-the-dance-be9337092b56}{Coronavirus:
The Hammer and the Dance},'' Tomas Pueyo correctly predicted the
national lockdown, which he called the hammer, and said it would lead to
a new phase, which he called the dance, in which essential parts of the
economy could reopen, including some schools and some factories with
skeleton crews.

Every epidemiological model envisions something like the dance. Each
assumes the virus will blossom every time too many hosts emerge and
force another lockdown. Then the cycle repeats. On the models, the
\href{https://www.imperial.ac.uk/media/imperial-college/medicine/sph/ide/gida-fellowships/Imperial-College-COVID19-NPI-modelling-16-03-2020.pdf}{curves
of rising and falling deaths resemble a row of shark teeth}.

Surges are inevitable, the models predict, even when stadiums, churches,
theaters, bars and restaurants remain closed, all travelers from abroad
are quarantined for 14 days, and domestic travel is tightly restricted
to prevent high-intensity areas from reinfecting low-intensity ones.

The tighter the restrictions, experts say, the fewer the deaths and the
longer the periods between lockdowns. Most models assume states will
eventually do widespread temperature checks, rapid testing and contact
tracing, as is routine in Asia.

Even the \href{https://www.whitehouse.gov/openingamerica/}{``Opening Up
America Again'' guidelines}
\href{https://www.nytimes3xbfgragh.onion/2020/04/16/us/politics/coronavirus-trump-guidelines.html}{Mr.
Trump issued on Thursday} have three levels of social distancing, and
recommend that vulnerable Americans stay hidden. The plan endorses
testing, isolation and contact tracing --- but does not specify how
these measures will be paid for, or how long it will take to put them in
place.

On Friday, none of that stopped the president from contradicting his own
message by
\href{https://www.nytimes3xbfgragh.onion/2020/04/17/us/coronavirus-cases-news-update.html\#link-23cd1c40}{sending
out tweets encouraging protesters} in Michigan, Minnesota and Virginia
to fight their states' shutdowns.

China did not allow Wuhan, Nanjing or other cities to reopen until
intensive surveillance found zero new cases
\href{https://www.cnn.com/2020/03/24/asia/coronavirus-wuhan-lockdown-lifted-intl-hnk/index.html}{for
14 straight days, the virus's incubation period}. Compared with China or
Italy, the United States is still a playground.

Americans can take domestic flights, drive where they want, and roam
streets and parks. Despite restrictions, everyone seems to know someone
discreetly arranging play dates for children, holding backyard barbecues
or meeting people on dating apps.

Partly as a result, the country has
\href{https://www.worldometers.info/coronavirus/country/us/}{seen up to
30,000 new case infections each day}. ``People need to realize that it's
not safe to play poker wearing bandannas,'' Dr. Schaffner said.

Even with rigorous measures, Asian countries have had trouble keeping
the virus under control.

China, which has reported about 100 new infections per day, recently
\href{https://www.businessinsider.com/coronavirus-china-orders-all-movie-theaters-to-close-again-2020-3}{closed
all the country's movie theaters} again. Singapore has
\href{https://www.cnbc.com/2020/04/03/singapore-shuts-schools-temporarily-closes-workplaces-to-curb-coronavirus.html}{closed
all schools and nonessential workplaces}. Japan recently
\href{https://www.nytimes3xbfgragh.onion/2020/04/07/world/asia/japan-coronavirus-emergency.html}{declared
a state of emergency}. (South Korea has
\href{https://www.nbcnews.com/news/world/south-korea-s-return-normal-interrupted-uptick-coronavirus-cases-n1176021}{struggled
at times}, too, but on Sunday reported only eight new cases, the first
single-digit increase in two months.)

Resolve to Save Lives, a public health advocacy group run by
\href{https://resolvetosavelives.org/about/team/tom-frieden}{Dr. Thomas
R. Frieden}, the former director of the C.D.C., has
\href{https://vitalstrategies.zoom.us/rec/play/uJUkIe2t-m83SYeSsgSDA6UsW426e6qs1iBIq6Zeyk7kUHkCZlL0YLcSZOrv8sWG1OUPpKL9uf0WUssz?continueMode=true}{published
detailed and strict criteria} for when the economy
\href{https://preventepidemics.org/wp-content/uploads/2020/04/COV020_WhenHowLoosenFaucet_v4.pdf}{can
reopen} and when it must
\href{https://preventepidemics.org/wp-content/uploads/2020/04/COV020_WhenHowTightenFaucet_v3.pdf}{be
closed}.

Reopening requires declining cases for 14 days, the tracing of 90
percent of contacts, an end to health care worker infections,
recuperation places for mild cases and many other hard-to-reach goals.

``We need to reopen the faucet gradually, not allow the floodgates to
reopen,'' Dr. Frieden said. ``This is a time to work to make that day
come sooner.''

\hypertarget{immunity-will-become-a-societal-advantage}{%
\subsubsection{Immunity will become a societal
advantage.}\label{immunity-will-become-a-societal-advantage}}

Image

Staff members of the Nido de Esperanza, a non-profit in the Washington
Heights neighborhood of Manhattan, handed out food to families with
small children.Credit...Misha Friedman for The New York Times

Imagine an America divided into two classes:
\href{https://www.nytimes3xbfgragh.onion/2020/04/11/us/coronavirus-survivors.html}{those
who have recovered from infection with the coronavirus} and presumably
have some immunity to it; and those who are still vulnerable.

``It will be a frightening schism,'' Dr. David Nabarro, a World Health
Organization
\href{https://www.who.int/dg/speeches/detail/who-director-general-s-opening-remarks-at-the-media-briefing-on-covid-19-on-21-february-2020}{special
envoy on Covid-19}, predicted. ``Those with antibodies will be able to
travel and work, and the rest will be discriminated against.''

Already, people with presumed immunity are very much in demand,
\href{https://www.medicinenet.com/script/main/art.asp?articlekey=229963}{asked
to donate their blood for antibodies} and doing risky medical jobs
fearlessly.

Soon the government will have to invent a way to certify who is truly
immune. A test for IgG antibodies, which are produced once immunity is
established, would make sense, said Dr. Daniel R. Lucey, an expert on
pandemics at Georgetown University's law school. Many companies are
working on them.

Dr. Fauci has said the
\href{https://www.politico.com/news/2020/04/10/fauci-coronavirus-immunity-cards-for-americans-are-being-discussed-178784}{White
House was discussing certificates} like those
\href{https://www.newsweek.com/germany-antibodies-tests-general-public-immunity-certificates-1494934}{proposed
in Germany}. China
\href{https://www.nytimes3xbfgragh.onion/2020/03/01/business/china-coronavirus-surveillance.html}{uses
cellphone QR codes} linked to the owner's personal details so others
cannot borrow them.

The California adult-film industry
\href{https://www.nytimes3xbfgragh.onion/2012/11/06/health/unlikely-model-for-hiv-prevention-porn-industry.html}{pioneered
a similar idea a decade ago}. Actors use a cellphone app to prove they
have tested H.I.V. negative in the last 14 days, and producers can
verify the information on a password-protected website.

As Americans stuck in lockdown see their immune neighbors resuming their
lives and perhaps even taking the jobs they lost, it is not hard to
imagine the enormous temptation to join them through self-infection,
experts predicted. Younger citizens in particular will calculate that
risking a serious illness may still be better than impoverishment and
isolation.

``My daughter, who is a Harvard economist, keeps telling me her age
group needs to have Covid-19 parties to develop immunity and keep the
economy going,'' said
\href{https://kingcenter.stanford.edu/people/michele-barry}{Dr. Michele
Barry}, who directs the Center for Innovation in Global Health at
Stanford University.

The comment, she explained later, was meant in jest. Still, it has
happened before.

In the 1980s, Cuba
\href{https://www.nytimes3xbfgragh.onion/2012/05/08/health/a-regimes-tight-grip-lessons-from-cuba-in-aids-control.html}{successfully
contained its small AIDS epidemic by brutally forcing} everyone who
tested positive into isolation camps. Inside, however, the residents had
their own
\href{https://www.nytimes3xbfgragh.onion/2012/05/08/health/cubas-aids-sanitariums-fortresses-against-a-viral-foe.html}{bungalows,
food, medical care, salaries, theater troupes and art classes}.

Dozens of Cuba's homeless youths
\href{https://www.nytimes3xbfgragh.onion/2012/05/08/health/cubas-aids-sanitariums-fortresses-against-a-viral-foe.html}{infected
themselves} through sex or blood injections to get in, said Dr. Jorge
Pérez Ávila, an AIDS specialist who is Cuba's version of Dr. Fauci. Many
died before antiretroviral therapy was introduced.

It would be a gamble for American youth, too. The obese and
immunocompromised are clearly at risk, but even slim, healthy young
Americans have died of Covid-19.

\hypertarget{the-virus-can-be-kept-in-check-but-only-with-expanded-resources}{%
\subsubsection{The virus can be kept in check, but only with expanded
resources.}\label{the-virus-can-be-kept-in-check-but-only-with-expanded-resources}}

Image

A health worker checking her protective equipment in a mirror in Central
Park in New York.Credit...Misha Friedman for The New York Times

The next two years will proceed in fits and starts, experts said. As
more immune people get back to work, more of the economy will recover.

But if too many people get infected at once, new lockdowns will become
inevitable. To avoid that, widespread testing will be imperative.

Dr. Fauci has said ``the virus will tell us'' when it's safe. He means
that once a national baseline of hundreds of thousands of daily tests is
established across the nation, any viral spread can be spotted when the
percentage of positive results rises.

Detecting rising fevers as they
\href{https://www.nytimes3xbfgragh.onion/2020/03/18/health/coronavirus-fever-thermometers.html}{are
mapped by Kinsa's smart thermometers} may give an earlier signal, Dr.
Schaffner said.

But diagnostic testing has been
\href{https://www.nytimes3xbfgragh.onion/2020/03/10/us/coronavirus-testing-delays.html}{troubled
from the beginning}. Despite assurances from the White House, doctors
and patients
\href{https://www.nytimes3xbfgragh.onion/2020/04/06/health/coronavirus-testing-us.html}{continue
to complain of delays and shortages}.

To keep the virus in check, several experts insisted, the country also
must start isolating all the ill --- including mild cases.

In this country, patients who test positive are asked to stay in their
homes but keep away from their families.

Television news has been filled with recuperating personalities like
CNN's Chris Cuomo, sweating alone in his basement while his wife left
food atop the stairs, his children waved and the dogs hung back.

But even Mr. Cuomo ended up illustrating why the W.H.O. strongly opposes
home isolation. On Wednesday, he
\href{https://www.cnn.com/2020/04/15/us/chris-cuomo-wife-coronavirus/index.html}{revealed
that his wife} had the virus.

``If I was forced to select only one intervention, it would be the rapid
isolation of all cases,'' said Dr. Bruce Aylward, who led the
\href{https://www.who.int/docs/default-source/coronaviruse/who-china-joint-mission-on-covid-19-final-report.pdf}{W.H.O.
observer team to China}.

In China, anyone testing positive, no matter how mild their symptoms,
was required to immediately enter an infirmary-style hospital --- often
set up in a gymnasium or community center outfitted with oxygen tanks
and CT scanners.

There, they recuperated under the eyes of nurses. That reduced the risk
to families, and being with other victims relieved some patients' fears.
Nurses even \href{https://www.youtube.com/watch?v=M7fqUHdoOyc}{led dance
and exercise classes} to raise spirits, and help victims clear their
lungs and keep their muscle tone.

Still, experts were divided on the idea of such wards. Dr. Fineberg
co-wrote
\href{https://www.nytimes3xbfgragh.onion/2020/04/07/opinion/coronavirus-smart-quarantine.html}{a
New York Times Op-Ed article} calling for mandatory but ``humane
quarantine processes.''

By contrast, \href{https://www.hsph.harvard.edu/marc-lipsitch/}{Marc
Lipsitch}, an epidemiologist at the Harvard T.H. Chan School of Public
Health, opposed the idea, saying: ``I don't trust our government to
remove people from their families by force.''

Ultimately, suppressing a virus requires
\href{https://www.centerforhealthsecurity.org/our-work/pubs_archive/pubs-pdfs/2020/a-national-plan-to-enable-comprehensive-COVID-19-case-finding-and-contact-tracing-in-the-US.pdf}{testing
all the contacts} of every known case. But the United States is far
short of that goal.

Someone working in a restaurant or factory may have dozens or even
hundreds of contacts. In China's Sichuan Province, for example, each
known case
\href{https://www.who.int/docs/default-source/coronaviruse/who-china-joint-mission-on-covid-19-final-report.pdf}{had
an average of 45} contacts.

The C.D.C.
\href{https://www.npr.org/sections/health-shots/2020/04/10/831200054/cdc-director-very-aggressive-contact-tracing-needed-for-u-s-to-return-to-normal}{has
about 600 contact tracers} and, until recently, state and local health
departments
\href{https://www.ncsddc.org/an-army-of-contact-tracers-will-bend-the-covid-19-curve/}{employed
about 1,600,} mostly for tracing syphilis and tuberculosis cases.

China hired and trained
\href{https://www.who.int/docs/default-source/coronaviruse/who-china-joint-mission-on-covid-19-final-report.pdf}{9,000
in Wuhan alone}. Dr. Frieden
\href{https://www.nytimes3xbfgragh.onion/2020/04/12/opinion/cdc-coronavirus.html}{recently
estimated that the United States will need} at least 300,000.

\hypertarget{there-will-not-be-a-vaccine-soon}{%
\subsubsection{There will not be a vaccine
soon.}\label{there-will-not-be-a-vaccine-soon}}

Image

A patient arrived at NewYork-Presbyterian Hospital in
Manhattan.Credit...Misha Friedman for The New York Times

Even though limited human trials of three candidates --- two here and
one in China --- have already begun, Dr. Fauci has repeatedly said that
any effort to make a vaccine will take at least a year to 18 months.

All the experts familiar with vaccine production agreed that even that
timeline was optimistic. Dr. Paul Offit, a vaccinologist at the
Children's Hospital of Philadelphia, noted that the
\href{https://www.cdc.gov/vaccines/pubs/pinkbook/mumps.html}{record is
four years}, for the mumps vaccine.

Researchers differed sharply over what should be done to speed the
process.
\href{https://www.nytimes3xbfgragh.onion/2020/04/08/health/coronavirus-vaccines.html}{Modern
biotechnology} techniques using RNA or DNA platforms make it possible to
develop candidate vaccines faster than ever before.

But clinical trials take time, in part because there is no way to rush
the production of antibodies in the human body.

Also, for unclear reasons, some previous vaccine candidates against
coronaviruses like SARS have triggered ``antibody-dependent
enhancement,'' which makes recipients more susceptible to infection,
rather than less. In the past, vaccines
\href{https://www.ncbi.nlm.nih.gov/pmc/articles/PMC3490694/}{against
H.I.V}\href{https://www.ncbi.nlm.nih.gov/pmc/articles/PMC5443089/}{.}and
\href{https://www.thelancet.com/journals/laninf/article/PIIS1473-3099(18)30023-9/fulltext}{dengue}
have unexpectedly done the same.

A new vaccine is usually first tested in fewer than 100 young, healthy
volunteers. If it appears safe and produces antibodies, thousands more
volunteers --- in this case, probably front-line workers at the highest
risk --- will get either it or a placebo in what is called a Phase 3
trial.

\href{https://www.nytimes3xbfgragh.onion/news-event/coronavirus?action=click\&pgtype=Article\&state=default\&region=MAIN_CONTENT_3\&context=storylines_faq}{}

\hypertarget{the-coronavirus-outbreak-}{%
\subsubsection{The Coronavirus Outbreak
›}\label{the-coronavirus-outbreak-}}

\hypertarget{frequently-asked-questions}{%
\paragraph{Frequently Asked
Questions}\label{frequently-asked-questions}}

Updated August 4, 2020

\begin{itemize}
\item ~
  \hypertarget{i-have-antibodies-am-i-now-immune}{%
  \paragraph{I have antibodies. Am I now
  immune?}\label{i-have-antibodies-am-i-now-immune}}

  \begin{itemize}
  \tightlist
  \item
    As of right
    now,\href{https://www.nytimes3xbfgragh.onion/2020/07/22/health/covid-antibodies-herd-immunity.html?action=click\&pgtype=Article\&state=default\&region=MAIN_CONTENT_3\&context=storylines_faq}{that
    seems likely, for at least several months.} There have been
    frightening accounts of people suffering what seems to be a second
    bout of Covid-19. But experts say these patients may have a
    drawn-out course of infection, with the virus taking a slow toll
    weeks to months after initial exposure. People infected with the
    coronavirus typically
    \href{https://www.nature.com/articles/s41586-020-2456-9}{produce}
    immune molecules called antibodies, which are
    \href{https://www.nytimes3xbfgragh.onion/2020/05/07/health/coronavirus-antibody-prevalence.html?action=click\&pgtype=Article\&state=default\&region=MAIN_CONTENT_3\&context=storylines_faq}{protective
    proteins made in response to an
    infection}\href{https://www.nytimes3xbfgragh.onion/2020/05/07/health/coronavirus-antibody-prevalence.html?action=click\&pgtype=Article\&state=default\&region=MAIN_CONTENT_3\&context=storylines_faq}{.
    These antibodies may} last in the body
    \href{https://www.nature.com/articles/s41591-020-0965-6}{only two to
    three months}, which may seem worrisome, but that's perfectly normal
    after an acute infection subsides, said Dr. Michael Mina, an
    immunologist at Harvard University. It may be possible to get the
    coronavirus again, but it's highly unlikely that it would be
    possible in a short window of time from initial infection or make
    people sicker the second time.
  \end{itemize}
\item ~
  \hypertarget{im-a-small-business-owner-can-i-get-relief}{%
  \paragraph{I'm a small-business owner. Can I get
  relief?}\label{im-a-small-business-owner-can-i-get-relief}}

  \begin{itemize}
  \tightlist
  \item
    The
    \href{https://www.nytimes3xbfgragh.onion/article/small-business-loans-stimulus-grants-freelancers-coronavirus.html?action=click\&pgtype=Article\&state=default\&region=MAIN_CONTENT_3\&context=storylines_faq}{stimulus
    bills enacted in March} offer help for the millions of American
    small businesses. Those eligible for aid are businesses and
    nonprofit organizations with fewer than 500 workers, including sole
    proprietorships, independent contractors and freelancers. Some
    larger companies in some industries are also eligible. The help
    being offered, which is being managed by the Small Business
    Administration, includes the Paycheck Protection Program and the
    Economic Injury Disaster Loan program. But lots of folks have
    \href{https://www.nytimes3xbfgragh.onion/interactive/2020/05/07/business/small-business-loans-coronavirus.html?action=click\&pgtype=Article\&state=default\&region=MAIN_CONTENT_3\&context=storylines_faq}{not
    yet seen payouts.} Even those who have received help are confused:
    The rules are draconian, and some are stuck sitting on
    \href{https://www.nytimes3xbfgragh.onion/2020/05/02/business/economy/loans-coronavirus-small-business.html?action=click\&pgtype=Article\&state=default\&region=MAIN_CONTENT_3\&context=storylines_faq}{money
    they don't know how to use.} Many small-business owners are getting
    less than they expected or
    \href{https://www.nytimes3xbfgragh.onion/2020/06/10/business/Small-business-loans-ppp.html?action=click\&pgtype=Article\&state=default\&region=MAIN_CONTENT_3\&context=storylines_faq}{not
    hearing anything at all.}
  \end{itemize}
\item ~
  \hypertarget{what-are-my-rights-if-i-am-worried-about-going-back-to-work}{%
  \paragraph{What are my rights if I am worried about going back to
  work?}\label{what-are-my-rights-if-i-am-worried-about-going-back-to-work}}

  \begin{itemize}
  \tightlist
  \item
    Employers have to provide
    \href{https://www.osha.gov/SLTC/covid-19/standards.html}{a safe
    workplace} with policies that protect everyone equally.
    \href{https://www.nytimes3xbfgragh.onion/article/coronavirus-money-unemployment.html?action=click\&pgtype=Article\&state=default\&region=MAIN_CONTENT_3\&context=storylines_faq}{And
    if one of your co-workers tests positive for the coronavirus, the
    C.D.C.} has said that
    \href{https://www.cdc.gov/coronavirus/2019-ncov/community/guidance-business-response.html}{employers
    should tell their employees} -\/- without giving you the sick
    employee's name -\/- that they may have been exposed to the virus.
  \end{itemize}
\item ~
  \hypertarget{should-i-refinance-my-mortgage}{%
  \paragraph{Should I refinance my
  mortgage?}\label{should-i-refinance-my-mortgage}}

  \begin{itemize}
  \tightlist
  \item
    \href{https://www.nytimes3xbfgragh.onion/article/coronavirus-money-unemployment.html?action=click\&pgtype=Article\&state=default\&region=MAIN_CONTENT_3\&context=storylines_faq}{It
    could be a good idea,} because mortgage rates have
    \href{https://www.nytimes3xbfgragh.onion/2020/07/16/business/mortgage-rates-below-3-percent.html?action=click\&pgtype=Article\&state=default\&region=MAIN_CONTENT_3\&context=storylines_faq}{never
    been lower.} Refinancing requests have pushed mortgage applications
    to some of the highest levels since 2008, so be prepared to get in
    line. But defaults are also up, so if you're thinking about buying a
    home, be aware that some lenders have tightened their standards.
  \end{itemize}
\item ~
  \hypertarget{what-is-school-going-to-look-like-in-september}{%
  \paragraph{What is school going to look like in
  September?}\label{what-is-school-going-to-look-like-in-september}}

  \begin{itemize}
  \tightlist
  \item
    It is unlikely that many schools will return to a normal schedule
    this fall, requiring the grind of
    \href{https://www.nytimes3xbfgragh.onion/2020/06/05/us/coronavirus-education-lost-learning.html?action=click\&pgtype=Article\&state=default\&region=MAIN_CONTENT_3\&context=storylines_faq}{online
    learning},
    \href{https://www.nytimes3xbfgragh.onion/2020/05/29/us/coronavirus-child-care-centers.html?action=click\&pgtype=Article\&state=default\&region=MAIN_CONTENT_3\&context=storylines_faq}{makeshift
    child care} and
    \href{https://www.nytimes3xbfgragh.onion/2020/06/03/business/economy/coronavirus-working-women.html?action=click\&pgtype=Article\&state=default\&region=MAIN_CONTENT_3\&context=storylines_faq}{stunted
    workdays} to continue. California's two largest public school
    districts --- Los Angeles and San Diego --- said on July 13, that
    \href{https://www.nytimes3xbfgragh.onion/2020/07/13/us/lausd-san-diego-school-reopening.html?action=click\&pgtype=Article\&state=default\&region=MAIN_CONTENT_3\&context=storylines_faq}{instruction
    will be remote-only in the fall}, citing concerns that surging
    coronavirus infections in their areas pose too dire a risk for
    students and teachers. Together, the two districts enroll some
    825,000 students. They are the largest in the country so far to
    abandon plans for even a partial physical return to classrooms when
    they reopen in August. For other districts, the solution won't be an
    all-or-nothing approach.
    \href{https://bioethics.jhu.edu/research-and-outreach/projects/eschool-initiative/school-policy-tracker/}{Many
    systems}, including the nation's largest, New York City, are
    devising
    \href{https://www.nytimes3xbfgragh.onion/2020/06/26/us/coronavirus-schools-reopen-fall.html?action=click\&pgtype=Article\&state=default\&region=MAIN_CONTENT_3\&context=storylines_faq}{hybrid
    plans} that involve spending some days in classrooms and other days
    online. There's no national policy on this yet, so check with your
    municipal school system regularly to see what is happening in your
    community.
  \end{itemize}
\end{itemize}

It is possible to speed up that process with ``challenge trials.''
Scientists vaccinate small numbers of volunteers, wait until they
develop antibodies, and then ``challenge'' them with a deliberate
infection to see if the vaccine protects them.

Challenge trials are used only when a disease is completely curable,
such as malaria
\href{https://www.nytimes3xbfgragh.onion/2017/09/28/health/typhoid-vaccine-trial.html}{or
typhoid fever}. Normally, it is ethically unthinkable to challenge
subjects with a disease with no cure, such as Covid-19.

But in these abnormal times, several experts argued that putting a few
Americans at high risk for fast results could be more ethical than
leaving millions at risk for years.

``Fewer get harmed if you do a challenge trial in a few people than if
you do a Phase 3 trial in thousands,'' said Dr. Lipsitch, who recently
published
\href{https://academic.oup.com/jid/advance-article/doi/10.1093/infdis/jiaa152/5814216}{a
paper advocating challenge trials} in the Journal of Infectious
Diseases. Almost immediately, he said, he heard from volunteers.

Others were deeply uncomfortable with that idea. ``I think it's very
unethical --- but I can see how we might do it,'' said Dr. Lucey.

The hidden danger of challenge trials, vaccinologists explained, is that
they recruit too few volunteers to show whether a vaccine creates
enhancement, since it may be a rare but dangerous problem.

``Challenge trials won't give you an answer on safety,'' said Michael T.
Osterholm, director of the University of Minnesota's Center for
Infectious Disease Research and Policy. ``It may be a big problem.''

Dr. W. Ian Lipkin, a virologist at Columbia University's Mailman School
of Public Health, suggested an alternative strategy. Pick at least two
vaccine candidates, briefly test them in humans and do challenge trials
in monkeys. Start making the winner immediately, even while widening the
human testing to look for hidden problems.

As arduous as testing a vaccine is, producing hundreds of millions of
doses is even tougher, experts said.

Most American vaccine plants produce only about 5 million to 10 million
doses a year, needed largely by the 4 million babies born and 4 million
people who reach age 65 annually, said Dr. R. Gordon Douglas Jr., a
former president of Merck's vaccine division.

But if a vaccine is invented, the United States could need 300 million
doses --- or 600 million if two shots are required. And just as many
syringes.

``People have to start thinking big,'' Dr. Douglas said. ``With that
volume, you've got to start cranking it out pretty soon.''

Flu vaccine plants are large, but those that grow the vaccines in
chicken eggs are not suitable for modern vaccines, which grow in cell
broths, he said.

European countries have plants but will need them for their own
citizens. China has a large vaccine industry, and may be able to expand
it over the coming months. It might be able to make vaccines for the
United States, experts said. But captive customers must pay whatever
price the seller asks, and the safety and efficacy standards
\href{https://www.nytimes3xbfgragh.onion/2020/04/16/world/europe/coronavirus-antibody-test-uk.html}{of
some Chinese companies are imperfect}.

India and Brazil also have large vaccine industries. If the virus moves
rapidly through their crowded populations, they may lose millions of
citizens but achieve widespread herd immunity well before the United
States does. In that case, they might have spare vaccine plant capacity.

Alternatively, suggested Arthur M. Silverstein, a retired medical
historian at the Johns Hopkins School of Medicine, the government might
take over and sterilize existing liquor or beer plants, which have large
fermentation vats.

``Any distillery could be converted,'' he said.

\hypertarget{treatments-are-likely-to-arrive-first}{%
\subsubsection{Treatments are likely to arrive
first.}\label{treatments-are-likely-to-arrive-first}}

Image

Coronavirus samples for testing at Mirimus, a medical research company
in Brooklyn.Credit...Misha Friedman for The New York Times

In the short term, experts were more optimistic about treatments than
vaccines. Several felt that so-called convalescent serum could work.

The basic technique has been used for over a century: Blood is drawn
from people who have recovered from a disease, then filtered to remove
everything but the antibodies. The antibody-rich immunoglobulin is
injected into patients.

The obstacle is that there are now relatively few survivors to harvest
blood from.

In the pre-vaccine era, antibodies were ``farmed'' in horses and sheep.
But that process was hard to keep sterile, and animal proteins sometimes
triggered allergic reactions.

The modern alternative is monoclonal antibodies. These treatment
\href{https://www.nytimes3xbfgragh.onion/2019/08/12/health/ebola-outbreak-cure.html}{regimens},
which recently
\href{https://www.aljazeera.com/news/2020/04/relief-ebola-outbreak-drc-declared-200410121800377.html}{came
very close to conquering} the Ebola epidemic in eastern Congo, are the
most likely short-term game changer, experts said.

The most effective antibodies are chosen, and the genes that produce
them are spliced into a benign virus that will grow in a cellular broth.

But, as with vaccines, growing and purifying monoclonal antibodies takes
time. In theory, with enough production, they could be used not just to
save lives but to protect front-line workers.

Antibodies can last for weeks before breaking down --- how long depends
on many factors, Dr. Silverstein noted --- and they cannot kill virus
that is already hidden inside cells.

Having a daily preventive pill would be an even better solution, because
pills can be synthesized in factories far faster than vaccines or
antibodies can be grown and purified.

But even if one were invented, production would have to ramp up until it
was as ubiquitous as aspirin, so 300 million Americans could take it
daily.

Mr. Trump has mentioned hydroxychloroquine and azithromycin so often
that
\href{https://www.nytimes3xbfgragh.onion/2020/04/04/health/coronavirus-drug-trump-hydroxycholoroquine.html}{his
news conferences sound like infomercials}. But all the experts agreed
with Dr. Fauci that no decision should be made
\href{https://www.nytimes3xbfgragh.onion/2020/04/17/health/trump-hydroxychloroquine-coronavirus.html}{until
clinical trials are completed.}

Some recalled that in the 1950s inadequate testing of thalidomide caused
thousands of children
\href{http://broughttolife.sciencemuseum.org.uk/broughttolife/themes/controversies/thalidomide}{to
be born with malformed limbs}. More than one hydroxychloroquine study
has been halted after patients who got high doses
\href{https://www.nytimes3xbfgragh.onion/2020/04/12/health/chloroquine-coronavirus-trump.html}{developed
abnormal heart} rhythms.

``I doubt anyone will tolerate high doses, and there are vision issues
if it accumulates,'' Dr. Barry said. Others were just as harsh,
especially about Mr. Trump's idea of combining a chloroquine with
azithromycin.

``It's total nonsense,'' said Dr. Luciana Borio, a former director of
medical and biodefense preparedness at the National Security Council.
``I told my family, if I get Covid, do not give me this combo.''

Chloroquine might protect patients hospitalized with pneumonia against
lethal cytokine storms because it damps down immune reactions, several
doctors said.

That does not, however, make it useful for preventing infections,
\href{http://transcripts.cnn.com/TRANSCRIPTS/2004/05/sitroom.02.html}{as
Mr. Trump has implied it would be}, because it has no known antiviral
properties.

Several antivirals, including remdesivir,
\href{https://www.scmp.com/tech/science-research/article/3051820/chinese-trials-gileads-antiviral-drug-treat-coronavirus-still}{favipiravir}
and baloxavir, are being tested against the coronavirus; the latter two
are flu drugs.

Trials of various combinations in China are set to issue results by next
month, but they will be small and possibly inconclusive because doctors
there ran out of patients to test. End dates for most trials in the
United States are not yet set.

\hypertarget{goodbye-america-first}{%
\subsubsection{Goodbye, `America First.'}\label{goodbye-america-first}}

Image

A church in Washington Heights took donated food and other goods for
Nido de Esperanza, the non-profit.Credit...Misha Friedman for The New
York Times

Previously unthinkable societal changes have taken place already.
Schools and business have closed in every state, and tens of millions
have applied for unemployment. Taxes and mortgage payments are delayed,
and foreclosures forbidden.

Stimulus checks, intended to offset the crisis, began landing in
checking accounts this week, making much of America, temporarily, a
welfare state.
\href{https://www.nytimes3xbfgragh.onion/2020/04/08/business/economy/coronavirus-food-banks.html}{Food
banks are opening} across the country, and huge lines have formed.

A public health crisis of this magnitude requires international
cooperation on a scale not seen in decades. Yet Mr. Trump is
\href{https://www.nytimes3xbfgragh.onion/2020/04/14/us/coronavirus-updates.html}{moving
to defund the W.H.O.}, the only organization capable of coordinating
such a response.

And he spent most of this year antagonizing China, which now has the
world's most powerful functioning economy and may become the dominant
supplier of drugs and vaccines. China has used the pandemic to extend
its global influence, and says it has sent medical gear and equipment to
nearly 120 countries.

A major recipient is the United States, through Project Airbridge, an
air-cargo operation overseen by Mr. Trump's son-in-law, Jared Kushner.

This is not a world in which ``America First'' is a viable strategy,
several experts noted.

``If President Trump cares about stepping up the public health efforts
here, he should look for avenues to collaborate with China and stop the
insults,'' said Nicholas Mulder, an economic historian at Cornell
University. He has called Mr. Kushner's project ``Lend-Lease in
reverse,'' a reference to American military aid to other countries
during World War II.

Dr. Osterholm was even blunter. ``If we alienate the Chinese with our
rhetoric, I think it will come back to bite us,'' he said.

``What if they come up with the first vaccine? They have a choice about
who they sell it to. Are we top of the list? Why would we be?''

Once the pandemic has passed, the national recovery may be swift. The
economy rebounded after both world wars, Dr. Mulder noted.

The psychological fallout will be harder to gauge. The isolation and
poverty caused by a long shutdown may drive up rates of
\href{https://www.nytimes3xbfgragh.onion/2020/04/17/nyregion/new-york-city-domestic-violence-coronavirus.html}{domestic
abuse},
\href{https://www.nytimes3xbfgragh.onion/2020/04/09/opinion/sunday/coronavirus-depression-anxiety.html}{depression}
and
\href{https://jamanetwork.com/journals/jamapsychiatry/fullarticle/2764584}{suicide}.

Even political perspectives
\href{https://time.com/5674850/2020-election-demographics/}{may} shift.
Initially, the virus heavily hit Democratic cities like Seattle, New
York and Detroit. But as it spreads through the country, it will spare
no one.

Even voters in Republican-leaning states who do not blame Mr. Trump for
America's
\href{https://www.nytimes3xbfgragh.onion/2020/03/28/us/testing-coronavirus-pandemic.html}{lack
of preparedness} or for
\href{https://www.politico.com/news/2020/04/03/trump-obamacare-coronavirus-164285}{limiting
access to health insurance} may change their minds if they see friends
and relatives die.

In one of the most provocative analyses in his follow-up article,
``\href{https://medium.com/@tomaspueyo/coronavirus-out-of-many-one-36b886af37e9}{Coronavirus:
Out of Many, One},'' Mr. Pueyo analyzed Medicare and census data on age
and
\href{https://www.inverse.com/article/33016-trump-voters-more-likely-to-be-fat-no-college-education}{obesity}
in
\href{https://www.nytimes3xbfgragh.onion/2020/04/03/us/coronavirus-states-without-stay-home.html}{states
that recently resisted shutdowns} and counties that voted Republican in
2016.

He calculated that those voters could be 30 percent more likely to die
of the virus.

In the periods after both wars, Dr. Mulder noted, society and incomes
became more equal. Funds created for veterans' and widows' pensions led
to social safety nets, measures like the G.I. Bill and V.A. home loans
were adopted, unions grew stronger, and tax benefits for the wealthy
withered.

If a vaccine saves lives, many Americans may become less suspicious of
conventional medicine and more accepting of science in general ---
including climate change, experts said.

The blue skies that have shone above American cities during this
lockdown era could even become permanent.

Image

The view from Weehawken, N.J.Credit...Misha Friedman for The New York
Times

Advertisement

\protect\hyperlink{after-bottom}{Continue reading the main story}

\hypertarget{site-index}{%
\subsection{Site Index}\label{site-index}}

\hypertarget{site-information-navigation}{%
\subsection{Site Information
Navigation}\label{site-information-navigation}}

\begin{itemize}
\tightlist
\item
  \href{https://help.nytimes3xbfgragh.onion/hc/en-us/articles/115014792127-Copyright-notice}{©~2020~The
  New York Times Company}
\end{itemize}

\begin{itemize}
\tightlist
\item
  \href{https://www.nytco.com/}{NYTCo}
\item
  \href{https://help.nytimes3xbfgragh.onion/hc/en-us/articles/115015385887-Contact-Us}{Contact
  Us}
\item
  \href{https://www.nytco.com/careers/}{Work with us}
\item
  \href{https://nytmediakit.com/}{Advertise}
\item
  \href{http://www.tbrandstudio.com/}{T Brand Studio}
\item
  \href{https://www.nytimes3xbfgragh.onion/privacy/cookie-policy\#how-do-i-manage-trackers}{Your
  Ad Choices}
\item
  \href{https://www.nytimes3xbfgragh.onion/privacy}{Privacy}
\item
  \href{https://help.nytimes3xbfgragh.onion/hc/en-us/articles/115014893428-Terms-of-service}{Terms
  of Service}
\item
  \href{https://help.nytimes3xbfgragh.onion/hc/en-us/articles/115014893968-Terms-of-sale}{Terms
  of Sale}
\item
  \href{https://spiderbites.nytimes3xbfgragh.onion}{Site Map}
\item
  \href{https://help.nytimes3xbfgragh.onion/hc/en-us}{Help}
\item
  \href{https://www.nytimes3xbfgragh.onion/subscription?campaignId=37WXW}{Subscriptions}
\end{itemize}
