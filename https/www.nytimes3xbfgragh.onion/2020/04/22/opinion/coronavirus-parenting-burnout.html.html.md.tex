Sections

SEARCH

\protect\hyperlink{site-content}{Skip to
content}\protect\hyperlink{site-index}{Skip to site index}

\href{https://myaccount.nytimes3xbfgragh.onion/auth/login?response_type=cookie\&client_id=vi}{}

\href{https://www.nytimes3xbfgragh.onion/section/todayspaper}{Today's
Paper}

\href{/section/opinion}{Opinion}\textbar{}Two Parents. Two Kids. Two
Jobs. No Child Care.

\url{https://nyti.ms/2VSD4Wi}

\begin{itemize}
\item
\item
\item
\item
\item
\item
\end{itemize}

Advertisement

\protect\hyperlink{after-top}{Continue reading the main story}

\href{/section/opinion}{Opinion}

Supported by

\protect\hyperlink{after-sponsor}{Continue reading the main story}

\hypertarget{two-parents-two-kids-two-jobs-no-child-care}{%
\section{Two Parents. Two Kids. Two Jobs. No Child
Care.}\label{two-parents-two-kids-two-jobs-no-child-care}}

And no end in sight.

\href{https://www.nytimes3xbfgragh.onion/by/farhad-manjoo}{\includegraphics{https://static01.graylady3jvrrxbe.onion/images/2019/01/08/opinion/farhad-manjoo-opinion/farhad-manjoo-opinion-thumbLarge.png}}

By \href{https://www.nytimes3xbfgragh.onion/by/farhad-manjoo}{Farhad
Manjoo}

Opinion Columnist

\begin{itemize}
\item
  April 22, 2020
\item
  \begin{itemize}
  \item
  \item
  \item
  \item
  \item
  \item
  \end{itemize}
\end{itemize}

\includegraphics{https://static01.graylady3jvrrxbe.onion/images/2020/04/21/opinion/21manjoo/21manjoo-articleLarge.jpg?quality=75\&auto=webp\&disable=upscale}

Things are starting to break down. I don't just mean psyche and soul and
social bonds and sense of self. I mean ordinary, overworked things
around the house: the dishwasher, the garage-door opener, the fridge
door, the faucet in the kids' bathroom, the towel hanger in the kids'
bathroom, the toilet in the kids' bathroom. (Don't ask.)

Bodies, too. So far we've had a sprain (our 9-year-old apparently
attempting the land speed record on a scooter meant for a toddler), a
possible fracture (him again, attempting same on a bicycle, after dark)
and enough stress-induced gastrointestinal distress to burn out the poor
plunger. For a school diary entry, our 7-year-old recounted how, during
an overly raucous, parentally imposed game of
\href{https://youtu.be/h2gaHOLu4tI}{Chicken Charades}, she bit brutally
into her lip, resulting in The World's Biggest Canker Sore.

``This is the worst day of my life,'' she reported.

There is much ambient love in my household under lockdown, but I'm sorry
to say that at any particular moment there is likely to be a state of
war, and the alliances shift more capriciously than those in the Trump
administration. Over the weekend, the kids, often fiercely battling each
other over increasingly diminished parental attention, unexpectedly
banded together and presented us with a written list of demands. We
refused to sign; my wife was taken prisoner of war, and the children do
not appear to think highly of the laws of conflict. (They let her nap,
but only with the lights on.)

\begin{quote}
Kids presented us with a written declaration of war.
\href{https://t.co/DprEYqj02H}{pic.twitter.com/DprEYqj02H}

--- Farhad Manjoo (@fmanjoo)
\href{https://twitter.com/fmanjoo/status/1252685070806048768?ref_src=twsrc\%5Etfw}{April
21, 2020}
\end{quote}

Does this sound fun?

Maybe one day we'll get a kick out of it. For now, we do not.

Yes, I am blessed beyond all measure. My family and I are so far healthy
and financially secure. We live in the San Francisco Bay Area, where
government has been surprisingly effective and the death toll from the
\href{https://www.nytimes3xbfgragh.onion/2020/06/17/parenting/public-toilets-virus-kids.html}{coronavirus}
minimal. My wife and I both have jobs that can be done mostly from home,
and I've worked primarily from home for much of my career, so I didn't
even have to adjust. Also, as my parents keep telling everyone they
know, my kids happen to be the world's most perfect children.

And yet, in this fifth week of sheltering in place, I am really starting
to wonder how anyone could think this is sustainable.

Attempting to work full time while rooming with, feeding and educating
one or more children during the pandemic is not going well --- not for
me, and not for most people I know. Though we are embarrassingly
\href{https://www.nytimes3xbfgragh.onion/2019/01/09/opinion/meditation-internet.html}{indulgent
of self-care}, neither of us feels as if we are doing anything other
than failing at everything, every day.

There is a largely meaningless debate going on right now about when to
``reopen'' the economy, as if you could just roll up the clanky gate on
society and take us back to the way we were on New Year's Day.

The whole argument is moot, because across vast parts of the country,
school has closed for good for the academic year, and many of the
already meager, expensive and inconvenient options for child care during
the summer may still be deemed unsafe. The economy will not really
``open up,'' and life will not really return to normal, as long as
parents don't have any place to send our children during the day.

Something will have to give. This is not just about my personal
annoyance and incapacity. Much of the future depends on parents coping.
\href{https://www.theatlantic.com/politics/archive/2015/11/how-working-moms-are-changing-american-households/433332/}{In
about half of America's two-parent households}, both parents work full
time. Even in ordinary times, American parents report
\href{https://www.nytimes3xbfgragh.onion/2015/11/05/upshot/stressed-tired-rushed-a-portrait-of-the-modern-family.html}{feeling
deep stress about raising children while working} (the stress is, of
course,
\href{https://english.emmaclit.com/2017/05/20/you-shouldve-asked/}{borne
largely by women}).

But across demographics and income levels, the pandemic has undone many
of the supports parents usually rely on to manage raising children while
working. If even fancy me is faring so poorly, I can't imagine how
others --- the single parents, the front-line-worker parents, the newly
unemployed parents --- are coping.

Actually, I don't have to imagine it. Parental burnout under lockdown
has been a hot topic these last few weeks. ``The parents are not all
right,''
\href{https://gen.medium.com/parents-are-not-ok-66ab2a3e42d9}{writes
Chloe I. Cooney in the online magazine Gen}. The
\href{https://parenting.nytimes3xbfgragh.onion/?type=roundup\&link=intro}{New
York Times's parenting} section
\href{https://www.nytimes3xbfgragh.onion/2020/04/08/parenting/coronavirus-self-care.html}{abounds}
with warnings of and
\href{https://www.nytimes3xbfgragh.onion/2020/03/30/parenting/coronavirus-comfort-food-kids.html}{solutions
to} burnout. And when I asked my Twitter followers how they're faring as
working parents, I was bombarded with dozens of tales of woe.

``I feel like I'm doing everything poorly and have no energy to focus on
anything,'' said Nilesh Shah, an assistant professor at the University
of Pittsburgh who is raising a 2-year-old with his wife, who also works
full time for the university.

``I know I am not working to the best of my capabilities,'' said Llyra
De La Mere, a market-research analyst who lives with her husband and
their 12-year-old autistic son near Seattle. ``I'm forgetting things
constantly. I've lost every semblance of organization. It's embarrassing
and troubling and I'm not proud of the work I'm doing,'' she said.

Another mom, Erika Brown Ekiel, who lives near me in the Bay Area, told
me she feels on the brink of burnout. She described ``such a level of
emotional and mental stress that it's hard to see how much longer this
can go on without creating really dangerous consequences all throughout
society.''

She tried to continue, beginning to explain how her kids' interruptions
were limiting her capacity to get in any state of workplace ``flow.''
But just then she was interrupted by her nearly naked son. ``Put your
pants and shirt back on,'' she scolded.

Here is where, ordinarily, I would conclude with a grand thought about
America: I might venture that cross-society parental stress under
pandemic could forge a new parental voting bloc. That perhaps now
universal child care will be regarded as a necessity, not some kind of
indulgence. But the kids are asking for lunch, and I have to break it to
them that all the hot dogs are gone. Pray for me. I don't want to be
taken prisoner of war.

\emph{The Times is committed to publishing}
\href{https://www.nytimes3xbfgragh.onion/2019/01/31/opinion/letters/letters-to-editor-new-york-times-women.html}{\emph{a
diversity of letters}} \emph{to the editor. We'd like to hear what you
think about this or any of our articles. Here are some}
\href{https://help.nytimes3xbfgragh.onion/hc/en-us/articles/115014925288-How-to-submit-a-letter-to-the-editor}{\emph{tips}}\emph{.
And here's our email:}
\href{mailto:letters@NYTimes.com}{\emph{letters@NYTimes.com}}\emph{.}

\emph{Follow The New York Times Opinion section on}
\href{https://www.facebookcorewwwi.onion/nytopinion}{\emph{Facebook}}\emph{,}
\href{http://twitter.com/NYTOpinion}{\emph{Twitter (@NYTopinion)}}
\emph{and}
\href{https://www.instagram.com/nytopinion/}{\emph{Instagram}}\emph{.}

Advertisement

\protect\hyperlink{after-bottom}{Continue reading the main story}

\hypertarget{site-index}{%
\subsection{Site Index}\label{site-index}}

\hypertarget{site-information-navigation}{%
\subsection{Site Information
Navigation}\label{site-information-navigation}}

\begin{itemize}
\tightlist
\item
  \href{https://help.nytimes3xbfgragh.onion/hc/en-us/articles/115014792127-Copyright-notice}{©~2020~The
  New York Times Company}
\end{itemize}

\begin{itemize}
\tightlist
\item
  \href{https://www.nytco.com/}{NYTCo}
\item
  \href{https://help.nytimes3xbfgragh.onion/hc/en-us/articles/115015385887-Contact-Us}{Contact
  Us}
\item
  \href{https://www.nytco.com/careers/}{Work with us}
\item
  \href{https://nytmediakit.com/}{Advertise}
\item
  \href{http://www.tbrandstudio.com/}{T Brand Studio}
\item
  \href{https://www.nytimes3xbfgragh.onion/privacy/cookie-policy\#how-do-i-manage-trackers}{Your
  Ad Choices}
\item
  \href{https://www.nytimes3xbfgragh.onion/privacy}{Privacy}
\item
  \href{https://help.nytimes3xbfgragh.onion/hc/en-us/articles/115014893428-Terms-of-service}{Terms
  of Service}
\item
  \href{https://help.nytimes3xbfgragh.onion/hc/en-us/articles/115014893968-Terms-of-sale}{Terms
  of Sale}
\item
  \href{https://spiderbites.nytimes3xbfgragh.onion}{Site Map}
\item
  \href{https://help.nytimes3xbfgragh.onion/hc/en-us}{Help}
\item
  \href{https://www.nytimes3xbfgragh.onion/subscription?campaignId=37WXW}{Subscriptions}
\end{itemize}
