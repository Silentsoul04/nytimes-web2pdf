Sections

SEARCH

\protect\hyperlink{site-content}{Skip to
content}\protect\hyperlink{site-index}{Skip to site index}

\href{https://www.nytimes3xbfgragh.onion/spotlight/podcasts}{Podcasts}

\href{https://myaccount.nytimes3xbfgragh.onion/auth/login?response_type=cookie\&client_id=vi}{}

\href{https://www.nytimes3xbfgragh.onion/section/todayspaper}{Today's
Paper}

\href{/spotlight/podcasts}{Podcasts}\textbar{}`You Don't Take Dictation.
You Find the Truth.'

\url{https://nyti.ms/2zhECl3}

\begin{itemize}
\item
\item
\item
\item
\item
\item
\end{itemize}

Advertisement

\protect\hyperlink{after-top}{Continue reading the main story}

transcript

Back to Sugar Calling

bars

0:00/39:19

-39:19

transcript

\hypertarget{you-dont-take-dictation-you-find-the-truth}{%
\subsection{`You Don't Take Dictation. You Find the
Truth.'}\label{you-dont-take-dictation-you-find-the-truth}}

\hypertarget{hosted-by-cheryl-strayed-produced-by-kelly-prime-and-edited-by-sara-sarasohn-editorial-oversight-by-wendy-dorr}{%
\subsubsection{Hosted by Cheryl Strayed, produced by Kelly Prime and
edited by Sara Sarasohn. Editorial oversight by Wendy
Dorr.}\label{hosted-by-cheryl-strayed-produced-by-kelly-prime-and-edited-by-sara-sarasohn-editorial-oversight-by-wendy-dorr}}

\hypertarget{cheryl-strayed-talks-with-the-writer-amy-tan-about-family-and-the-long-arc-of-resilience}{%
\paragraph{Cheryl Strayed talks with the writer Amy Tan about family and
the long arc of
resilience.}\label{cheryl-strayed-talks-with-the-writer-amy-tan-about-family-and-the-long-arc-of-resilience}}

Wednesday, April 22nd, 2020

\begin{itemize}
\item
  cheryl strayed\\
  Today, I'm going to call Amy Tan. I fell in love with her a long time
  ago, when her first book ``The Joy Luck Club'' came out. I read it
  right away, in 1989, and I was utterly astonished and astounded by the
  lives she wrote about in those pages. And especially the dynamics
  between the mother and the daughter in that book really moved me
  deeply. Amy Tan is also the author of ``The Valley of Amazement,''
  ``The Kitchen God's Wife,'' and other books. Most recently, she wrote
  a memoir called ``Where the Past Begins.'' It's about her life as a
  writer and also about her family history and how it has been brought
  to bear on her life. I'm going to give her a call.
\item
  {[}music{]}
\item
  {[}line ringing{]}
\item
  amy tan\\
  Hello?
\item
  cheryl strayed\\
  Amy?
\item
  amy tan\\
  Yes?
\item
  cheryl strayed\\
  Hi, this is Cheryl.
\item
  amy tan\\
  Hi. Hi, Cheryl.
\item
  cheryl strayed\\
  Oh, it's so nice to hear your voice.
\item
  amy tan\\
  Good to hear you too.
\item
  cheryl strayed\\
  So where are you right now? Where am I calling you?
\item
  amy tan\\
  You are calling me in my home in Sausalito.
\item
  cheryl strayed\\
  So you're socially isolating with your husband, is that right, your
  husband Lou?
\item
  amy tan\\
  My husband, yes. My husband and two dogs, we are in this house. I feel
  very, very lucky actually to be here in this particular house, which I
  actually built, with the idea that one day we'd be old and not be able
  to go out, and what would I need to feel like it was OK to be
  sequestered in a way from the world.
\item
  cheryl strayed\\
  And what did that turn out to be? What is your house like? And what
  can you see from your windows?
\item
  amy tan\\
  It's like a treehouse, and it's surrounded by oak trees, and windows
  all on three sides of this great room that I'm sitting in. Everything
  is ADA compliant, so if I would be in a wheelchair, I could go
  anywhere. The bathrooms are all sliding doors. The entries outside. So
  it's all set up for getting old, and I hope I get old. But what I see,
  I see oak trees. But the main thing I see is Angel Island. Which is, I
  see the bay. I see San Francisco Bay, Raccoon Strait, Sausalito
  Harbor, but in the middle of all that is Angel Island. Do you know
  what Angel Island is? Do you know its history?
\item
  cheryl strayed\\
  I want to say that there --- was it a place where immigrants came. Is
  that right?
\item
  amy tan\\
  Yes, in particular Chinese people. It was a quarantine station. I just
  think it's so ironic in a terrible way that this is the symbol that
  I'm looking at daily is the quarantine where people did not know when
  they would get out of this place and what the future would be and if
  they would be sent back. So great uncertainty and unhappiness there.
\item
  cheryl strayed\\
  And when was that in existence? What years?
\item
  amy tan\\
  That was, I think it was like 1910 until 1940 or so. The dates are
  approximate. But I used to live in the house that's next to the
  current one that we have, and that was built in 1908. And I would look
  at the house and think, the woman who lived in this house in 1908
  would look at that island, and that's where all these Chinese people
  were imprisoned really, and that I was the person now looking at this
  island, and I was the Chinese woman looking at it. So I had a great
  deal of connection between me and what I was seeing, a great deal of
  gratitude that I was on the other side of that.
\item
  cheryl strayed\\
  Your parents both immigrated from China in the `40s, is that correct?
\item
  amy tan\\
  Yes, `47 for my father and `49 for my mother.
\item
  cheryl strayed\\
  And where did they come when --- how did they arrive to the United
  States?
\item
  amy tan\\
  My father came by boat, and my mother flew on Philippine Airlines. She
  had to take a boat from Shanghai to Hong Kong, and then eventually she
  flew. She couldn't wait for the boat. She wanted to be reunited with
  my father. And they were in San Francisco, and by then there was no
  quarantine station. But they came on student visas, which then lapsed,
  and then they became illegal. So they faced deportation for about 10
  years.
\item
  cheryl strayed\\
  And then finally were able to become U.S. citizens?
\item
  amy tan\\
  Yes. Yes.
\item
  cheryl strayed\\
  And how were they able to become citizens?
\item
  amy tan\\
  Eventually, they got, I think, a kind of residency. And I just
  remember that when I was about --- it was around 1962 or so, and I
  remember being at home and my parents were crying, and they had just
  come back from being sworn in as American citizens. And I couldn't
  appreciate that. It was like, oh, big deal, they went and took a test,
  and now they're citizens. So what, you know? I had no idea that for
  all those years, they had been facing this terrible uncertainty.
  Because we were kids, we were born here, we're Americans. We didn't
  know any different. We didn't know anything about China.
\item
  cheryl strayed\\
  Right. Obviously, your work has been so deeply about your past and
  your family's history. And I guess I'm curious. I know you became a
  writer, with ``The Joy Luck Club'' being published when you were in
  your 30s, you were kind of surprised essentially to become a novelist?
\item
  amy tan\\
  Yeah. Yeah. Well, I was a short story writer. Somebody else called it
  a novel. It was a series of short stories really.
\item
  cheryl strayed\\
  You're right. I always forget that. It's that the stories in ``The Joy
  Luck Club'' sit together so beautifully, it reads to me like a novel.
  What I was so moved by is really at the heart of that book is that
  mother-daughter bond. And I'm curious, how did you come to write about
  that mother and daughter, and what did your real mother make of it
  when she read the book when the book came out?
\item
  amy tan\\
  I was writing this at a time when I was not really that close to my
  mother. It was more of a detente or more of an avoidance of situations
  that might inflame both of us. And when I started to write, these
  stories just naturally came out when I began writing about a girl,
  then suddenly her mother injected herself into the story. And I began
  to see how much of who I was had been formed by my --- whatever I had
  absorbed from my mother, but also how I had reacted, rebelled against
  her. So I started writing this, and ironically, the voices that were
  the easiest to me were the mothers. And I realized how much I really
  had paid attention, and I was trying to figure out what these mothers
  --- my mother's intention had been, especially in telling these
  stories about the past that I always thought were so boring. Or you
  know I'd say, you've told me that already, which was the wrong thing
  to say to her. If I said that to her, she'd just go into a fury and
  then threaten to kill herself. So I had to be very careful and just
  kind of tune out and just, yeah, yeah, mm-hm.
\item
  cheryl strayed\\
  If you said, mom, you already told me that story, she would threaten
  to kill herself?
\item
  amy tan\\
  Often, if there was anything that made her feel that she was
  unappreciated or that I'd rather she not be there at that particular
  moment, yes, it would lead to a suicide threat and occasionally
  suicide attempts. And so that was something I grew up with, which
  shaped me a great deal. Not that I'm suicidal, but in looking at what
  it was that made my mother feel suicidal, that she had a kind of
  helplessness, a loneliness, an inability to control what was happening
  in her life. She got that from her mother, who did kill herself, and
  her mother killed herself because she was forced to become a concubine
  after she was widowed. And she was widowed when her husband, my
  grandfather, died during the pandemic in 1919.
\item
  cheryl strayed\\
  Right, the Spanish influenza pandemic.
\item
  amy tan\\
  Yeah. Yeah.
\item
  cheryl strayed\\
  And this was in Shanghai? They were in Shanghai when ---
\item
  amy tan\\
  Yes. Yes, in Shanghai. For the reason that I'm writing, which is to
  understand this transference, this inheritance of emotion over
  generations. Your reactions, the way you think about the world, your
  philosophy, your cosmology of how you view things, how the universe
  operates and how you operate within that world, that came from, I
  think in many ways, from the pandemic, which changed the direction of
  our family's life. And so that suicide came out of that, and my
  mother, being the suicidal child who watched her mother die at age 9,
  was the mother I ended up with, who guided me to who I am today and
  how I react to things.
\item
  cheryl strayed\\
  I love thinking about the ways that our mothers, the ways they shaped
  us not just as humans, but as writers, and I'm really struck by what
  you said about your mom, about her helplessness. And I do think that
  when you're angry and you threaten somebody with your suicide, that is
  a way of really expressing a profound helplessness. And yet, I think
  of your mom and everything you've said about her or written about her,
  she was such a resilient person. She was such a survivor, right?
\item
  amy tan\\
  Yeah. Yeah, it was a great contradiction that she could go through
  anything, and yet there would be the smallest thing that would make
  her feel annihilated. But you're right, she had such fortitude, such
  resilience, but it came across in ways that really made me want to not
  be like her. I mean, one of her things is she constantly worried, and
  she would try to find the reasons why something happened. And if she
  could find the reason, then she couldn't do it. But at the same time,
  she could handle the reality of what was going on and just be strong
  and push through and tell people she wasn't taking this. And she had
  very strong opinions, very strong ethics. She would become livid if
  anybody was condescending to her, and she couldn't always fight back
  and say, don't you dare treat me that way, because she was a Chinese
  woman who didn't speak English very well. And people would treat her
  in this way, patronizing, condescending way, but that would make her
  livid. And that's something she passed on to me. I get a just
  immediate reaction where I explode if somebody is treating me in a
  patronizing way.
\item
  cheryl strayed\\
  Right. It's so amazing to me the ways that we carry our mothers within
  us, positive and negative. My mom died when she was 45.
\item
  amy tan\\
  So young.
\item
  cheryl strayed\\
  Yes. You know, Amy, I feel connected to you. I've never met you, but
  part of you is embedded in part of me because --- I'm going to cry ---
  because ``The Joy Luck Club'' was one of the last books that my mom
  and I read together.
\item
  amy tan\\
  Oh ---
\item
  cheryl strayed\\
  And ---
\item
  amy tan\\
  --- God.
\item
  cheryl strayed\\
  --- we both loved it so much. And it was --- I had a really different
  kind of relationship with my mom than the one you had with yours, but
  there was a language, there was something that you gave us that
  allowed us to talk even more fully about our relationship. And it was
  you, your book, ``The Joy Luck Club'' opened up that dialogue in the
  last year of my mom's life. I remember ---
\item
  amy tan\\
  Wow.
\item
  cheryl strayed\\
  --- so your book came out two years before my mom died, and then the
  movie of it came out two years after she died. And I remember sitting
  in the theater watching it alone, and I remember just crying and
  thinking, I wish --- my mother should be here with me.
\item
  amy tan\\
  Oh. Oh, well, that makes me cry to hear that.
\item
  cheryl strayed\\
  I'm sorry, I'm not supposed to ---
\item
  amy tan\\
  Thank you for telling me. No, no, it's a good thing. It's good when
  you're moved to tears.
\item
  cheryl strayed\\
  I'm not supposed to cry, but.
\item
  amy tan\\
  No.
\item
  cheryl strayed\\
  I know you have also experienced deep grief. Your brother Peter and
  your father John both died in 1968, when you were 15, and they died of
  the same thing.
\item
  amy tan\\
  Yes.

  It was --- it actually started in `67 and ended --- it was exactly one
  year, February to February. My brother first became ill. My older
  brother, who I adored.
\item
  cheryl strayed\\
  And how old was he at the time?
\item
  amy tan\\
  He died just about a week before he turned 17, and he had a brain
  tumor, a kind of brain tumor that would today be curable. And two
  weeks before he died, my father was diagnosed with glioblastoma, and
  then he died six months later. So it was one solid year. There was no
  break whatsoever, February to February, of this specter of doom, but
  also a need to pretend that everything was OK and a miracle would
  happen. And the pretense is what I hated, the, what I consider to be
  the false hope at the time, and that is something I write a lot about.
  I want to understand this notion, and is there anything in the way of
  hopelessness or futile hope? And I've come away thinking, no, hope is
  a very, very important part of this that we have to really safeguard.
  We have to keep that because that is part of what gives us fortitude.
  And it could be the most impossible situation, but for you to see a
  future, you have to cling to that. And I feel that very strongly.
\item
  cheryl strayed\\
  And yet, when you were a kid and your family was facing this horrific
  couple of years, you resented that sense that there should be hope?
\item
  amy tan\\
  Oh, yeah. It made me angry because I felt that it was a betrayal,
  especially after my brother died, that the miracle didn't happen and
  they were still going through this. My father was a Baptist minister.
  He was an engineer, but he was also a Baptist minister. So this
  religious component was part of it. If we were good enough, if we
  prayed hard enough, the miracle would happen. And it didn't happen, so
  it was either that I didn't get the reward or I was not good enough.
  And it made me cynical, which is a terrible thing to have happen when
  you're a young person. I don't know when cynicism comes into a child's
  life, when they first recognize it in themselves, but I certainly felt
  it when I was 14 and 15.
\item
  cheryl strayed\\
  Mm-hm. And when your family, really almost half of your family died.
\item
  amy tan\\
  Yeah, and what I thought was the best half of the family because I
  loved my brother and my father. I was a daddy's girl, and he was
  supposed to protect me from my mother, and he left. And then I was
  angry at my father because he abandoned me by dying.
\item
  cheryl strayed\\
  Right. I'm curious, that experience that you had --- we're seeing the
  numbers every day climb of the people around the globe who have died
  of Covid-19. And I'm curious if knowing that so many people are
  grieving the loss of loved ones, if that's made you at all remember or
  think about what you experienced as a young woman when suddenly you
  lost people in your life?
\item
  amy tan\\
  Yeah. I think more than anything, it's not simply the grief, it's not
  knowing. Not knowing when this is going to end. During the time my
  father and brother were dying, my mother made us feel it was a curse
  and we would all die. So we were waiting for something worse, even
  worse to happen, and the worst that happened was my mother went crazy,
  to use a very non-psychiatric term. So I think it's that uncertainty
  and the feeling of something impending that is worse than what is
  presently happening. And each day, we are overwhelmed by news, the
  numbers. It's a crescendo of these numbers, and it just doesn't stop.
  And it starts with this incredulity. People wake up in the morning and
  say, it was just a bad dream. And then you're incredulous to find, no,
  it's real. And each day, that momentary sense of disbelief, I think,
  overtakes a lot of us, and then we settle into it and say, oh my God,
  when is this going to stop? And all these people are giving us
  information, and who's telling the truth or who knows the truth?
  Nobody could possibly know. This has never happened before with this
  particular disease. So I think that uncertainty combined with losses,
  and not for me --- it's not so much that I fear dying for myself, I
  fear it for my husband. We've been together for 50 years, and I don't
  know what I would do if something happened to him.

  My fear for friends who have certain conditions that would make them
  very vulnerable, besides age. I have lost one friend who, ironically,
  tragically, was the person who wrote the textbook on pulmonary
  disease, and he died from Covid-19.
\item
  cheryl strayed\\
  Oh, I'm so sorry.
\item
  amy tan\\
  Yeah.
\item
  cheryl strayed\\
  What was his name?
\item
  amy tan\\
  John Murray. His wife is Diane Johnson, the novelist. I don't know if
  you've heard of her.
\item
  cheryl strayed\\
  Yeah.
\item
  amy tan\\
  So we've been friends for about 30-some years, and so it was very,
  very sad.
\item
  cheryl strayed\\
  So we're afraid of our friends dying. You also mentioned you're afraid
  of your husband Lou dying. I'm wondering what you're doing to manage
  that feeling that you describe so well, which I do think is very much
  like grief, that you wake up, you forget for a moment what's true, and
  then you remember the truth. And I remember this over and over in the
  first year or so after my mom died, that I would wake up --- there was
  something kind of safe about the dream life, and I'd wake up and
  think, oh, she's dead. And you're right, this is exactly what I've
  thought during the pandemic. And it's because the world --- I always
  say, when you lose somebody who is essential to you, the world as you
  know it ends, and you have to figure out how to live in the new world
  without that person. And here we are in the pandemic, that the world
  as we knew it has ended. And so how are you navigating in the new
  world?
\item
  amy tan\\
  I have anxiety like everybody else, and it comes as a sense of
  breathlessness. I don't really know it in my head until I feel it in
  my body, and I say, why do I feel I can't breathe? And I know I don't
  have Covid-19, but I just think, what's going on? And then I recognize
  the anxiety, and I know that what it would be is suddenly this
  affecting us and Lou getting sick and what would I do and losing him.
  So what I do that is different from my mother is not worry about ---
  not focus on, how did this happen, where does this come from? Is to be
  super prepared. I am known among my friends as Debbie Downer. I'm the
  person who gets prepared for every single disaster. I have a shed full
  of stuff for earthquakes. I have things for ---
\item
  cheryl strayed\\
  Really?
\item
  amy tan\\
  I have shelves with grab bags to go, with everything you need for a
  wildfire. So when this pandemic first started, I went and gathered a
  lot of the stuff I already had, which included masks. I have goggles.
  I have sterilizing things, UV water purifiers, all of that kind of
  stuff, gloves. I already had that stuff, and then I added to that. And
  then I know that something could happen, but I would have control,
  some degree of control.
\item
  cheryl strayed\\
  Some degree of control.
\item
  amy tan\\
  Yeah, some degree of control. I can't control everything, but I won't
  fall to pieces. I have a little box that has everything, and I know I
  go there, I have procedure, so I can be methodical.
\item
  cheryl strayed\\
  It is such a fine line, isn't it, between your friends kind of making
  fun of you because you have all this stuff, and yet it's actually kind
  of reasonable. I too --- I live in Portland, Oregon, where we're all
  just waiting for this major earthquake that they keep saying is
  overdue. And for years now, I kept thinking, I have to buy those
  earthquake preparedness packs. And the first week that it got real in
  my household, I bought these 20-year bags for seven people. My
  family's only four, but I figure we'll help out a few neighbors too if
  we can. And I think when my kids saw this, they're like, mom, you're
  kind of weird. But the thing is, I wasn't trying to alarm anyone. I'm
  trying to be prepared. But ---
\item
  amy tan\\
  How can --- I don't know why people would deny that. It's this
  superstitious mentality of, if you do it, it's going to happen.
\item
  cheryl strayed\\
  Well, and that's, I want to go back to --- this is something that I
  really am curious about, because when your brother and father died in
  the late `60s, we just had such a different culture of the ways that
  we talk about death, a very different culture of grief. I mean, that
  it was this topic that was to be avoided. And I'm curious, here you
  are, my heart breaks for you, Amy, this teenage girl. You've lost your
  big brother, your father. I'm sure you were also maybe afraid that you
  too might get sick and die, all of those things. And I'm curious, so
  many people out there right now are having to face these big losses
  and having to grieve, and I know a lot of people around them are
  wondering, what do I say? How do I address this sorrow, this
  monumental sorrow? And I'm curious, what do you wish people did or
  said to you during your time of greatest need and loss and sorrow?
\item
  amy tan\\
  Well, you have to keep in mind I was very young. I mean, I was a
  teenager, and I wanted to have a normal life. I wanted to be asked out
  on a date. I'd never been asked out on a date. I wanted to have that
  life and to escape. And I think about that with a lot of the kids
  today, and that I want to say, I understand if you want to escape,
  that you want to leave the house, or you just want to take your mind
  out of this. You don't want to be with family. You don't want to be
  supportive. You want somebody to say to you, that's fine the way that
  you feel, that's natural. That's what I would have wanted. I'm sure
  there are a lot of people out there who do that for their kids. But I
  think parents, adults are so overwhelmed by what's happening, they may
  be demanding more things from their kids, that they behave better,
  that they --- and you know what happens is you behave worse. Instead
  of making things easier for your family, you make it harder because
  you resent it. You're scared, and you can't express that because
  you're a teenager, and nothing scares you, so to speak. And I wanted
  to just have somebody say, I know why you're angry. I needed some
  understanding. There was a complication on top of that is that a few
  days or a week or two before my father died, I was caught reading a
  dirty book, ``Catcher in the Rye,'' and ---
\item
  cheryl strayed\\
  Oh, my goodness.
\item
  amy tan\\
  --- a banned book. So the youth minister came to counsel me, and he
  sexually molested me. He threw me down on my bed and sexually molested
  me.
\item
  cheryl strayed\\
  Oh, Amy.
\item
  amy tan\\
  So that day, it was --- before the miracle busted up, exploded in our
  faces, that was already gone. The whole idea that religion and people
  and praying together and community, that was gone too. I came out of
  that room a different person.
\item
  cheryl strayed\\
  Oh, Amy.
\item
  amy tan\\
  And ultimately --- no, no --- ultimately, I'm glad because what it
  taught me is that I have to ask my own questions. I have to say, what
  is it that I really feel? What's hypocritical? What's truly not
  kindness and being a good person?

  And so I ask those questions myself. It doesn't get dictated to me in
  the way that it used to because the spokespeople for what is good and
  evil are not necessarily coming from the best sources. So I learned
  that. You don't take dictation, you find the truth.
\item
  cheryl strayed\\
  Well, and you have to be awake to it. I think we share that experience
  too. I've been sexually abused. And I think that, for me, the most
  meaningful act probably in my life has been that move that you just
  described. Where you take an awful thing that happened to you, the
  death of an essential person in your life who you love or a sexual
  violation, you take that really, really ugly thing and make it, in
  some ways, beautiful. You make it not disempower you, but be the
  place, the source of your power. Adrienne Rich talks about our wounds
  being the source of our greatest power. I really believe that.
\item
  amy tan\\
  I want to just interject something here. I was talking to a friend, a
  psychiatrist, about people who go through terrible things and they
  have that with them the rest of their lives, and it makes them a
  little dysfunctional or greatly dysfunctional. And I wouldn't want
  people to feel guilty that they weren't able to find their strength
  when a terrible thing happened to them or a series of terrible things.
  So I just say that because there are a lot of people going through
  terrible stuff right now, and maybe they're not going to be able to
  come out and say, I'm resilient. Maybe they're just going to fall
  apart. And I just want to say, that's good, lift your hand and ask for
  help.
\item
  cheryl strayed\\
  And I think, too, one of the things about resilience that I've
  observed in my own life and so many people who I admire is that
  resilience doesn't necessarily happen the next day, right? Sometimes
  you have to wander around in the darkness for a decade, for two
  decades. I mean, anyone who's ever recovered from a long addiction or
  changed their lives in these radically essential ways knows this to be
  true. So it's also never too late to ---
\item
  amy tan\\
  Exactly, yeah.
\item
  cheryl strayed\\
  --- find that strength.
\item
  amy tan\\
  Mm-hm. It's there, it's rooted, and it can come out, and you know, it
  can surprise you. I wonder about so many people and how they're going
  to deal with this, and that becomes part of the thing that makes me
  anxious, that what's going to happen to this world? Because I'm going
  to come out OK. My life now is not a lot different from what it has
  been as a writer. But what about these people in these tiny places who
  are going to lose those tiny places? What's going to happen to them?
  So I think part of it is to feel you can't do anything, but I think
  it's important to feel compassion, to empathize as much as you can,
  which is what I think we do as fiction writers. It's our practice to
  do that. Imagination I think, is the closest thing we have to
  compassion.
\item
  cheryl strayed\\
  Oh, yeah. You wrote something really beautiful that I wrote down in my
  notebook in your memoir, ``Where the Past Begins.'' You said that Jane
  Eyre taught you that loneliness had more to do with being
  misunderstood than being alone. And I think it's a wonderful example
  of what you're saying, that as memoirists, fiction writers, as
  writers, it's not about delivering the message to everyone that would
  be the one wise thing, but making people feel seen and recognized. I'm
  curious, I asked you to read something to me, and can I ask you to
  share the poem?
\item
  amy tan\\
  Now, the poem that I've chosen has to do with the fact that as
  writers, I think that, for many of us, what draws us to writing is to
  come up with a way to express the exact feeling at the exact moment,
  and that when somebody reads that, they say, that's how it is for me.
  And that's what this poem is about. This poem was written by my
  editor, who's also my publisher and is also a renowned poet, Dan
  Halpern. He wrote this five years ago. He didn't remember writing it,
  but he knew that he must've written it because it was published in
  Poetry about five years ago.
\item
  cheryl strayed\\
  He doesn't remember writing it?
\item
  amy tan\\
  He doesn't remember, and it is so prescient.
\item
  cheryl strayed\\
  Oh, that's funny. So, yeah, it's prescient. It's called
  ``Pandemania,'' right?
\item
  amy tan\\
  ``Pandemania.''
\item
  cheryl strayed\\
  By Daniel Halpern.
\item
  amy tan\\
  Daniel Halpern. Here it is.

  ``There are fewer introductions in plague years, hands held back,
  jocularity no longer bellicose, even among men. Breathing's generally
  wary, labored, as they say, when the end is at hand. But this is the
  everyday intake of the imperceptible life force willed now, slow.
  Well, just cautious in inhabited air. As for ongoing dialogue, no
  longer an exuberant plosive to make a point, but a new squirreling of
  airspace, a new sense of boundary.

  ``Genghis Khan said the hand is the first thing one man gives to
  another. Not in this war. A gesture of limited distance now suffices,
  a nod, a minor smile or a hand slightly raised, not in search of its
  counterpart, just a warning within the acknowledgment to stand back.
  Each beautiful stranger a barbarian breathing on the other side of the
  gate.''
\item
  cheryl strayed\\
  Amy, what a perfect poem for this moment.
\item
  amy tan\\
  I know.
\item
  cheryl strayed\\
  This is honestly my life. I walk around Northeast Portland and I greet
  people with a seeming sense of pleasantry from a distance, but I
  think, barbarian, stay on the other side of that gate.
\item
  amy tan\\
  You're infected. Are you infected? Or, why aren't you wearing a mask?
  Or why are you too close to me?
\item
  cheryl strayed\\
  Do you walk around your neighborhood? Are you having this same
  experience?
\item
  amy tan\\
  I've done it twice. But we see people running by, no masks, too close,
  coming by too close to us, and I think it's impolite. I think it's
  hostile. So there's a new separateness that has come up among
  strangers.
\item
  cheryl strayed\\
  Right. And that actually scares me, that dynamic, because I sense in
  myself sometimes that building sense of not just fear, but almost this
  --- Daniel in this poem, he even says, he mentions this term war. It's
  you against me, and you stay in your breath space and I'll stay in
  mine. And I can't help but fear what's already gone on so much over
  the last many years and probably through all of history, but
  especially in the last few years this anti-immigrant sentiment ---
\item
  amy tan\\
  Oh my God, yeah.
\item
  cheryl strayed\\
  --- the attacks against many Asian-Americans in the U.S.
\item
  amy tan\\
  Yeah, especially now. Yeah.
\item
  cheryl strayed\\
  And, right, even our president calling it the Chinese virus. And I'm
  afraid about that dynamic of this.
\item
  amy tan\\
  And I just want to say this thing about Chinese virus. Some people
  say, well, we called it Ebola. Ebola River is where Ebola was.
  Ebolese, they're not people. You're not saying the ``Ebolese.''
  Chinese is a race. Chinese is not a location. Chinese live throughout
  the world, including the United States, people who were born in this
  country. And he has made it a race card, and that's what angers me so
  much in how they have renamed this virus. They have made it race, not
  location. So the thing about with people who go running by without a
  mask, it's not ignorance because I have read online the hostile
  remarks. It's like, you don't tell us what to do, we don't have to
  wear that.
\item
  cheryl strayed\\
  Yeah. Yeah, I've seen it. I just try to assume --- my nature is just
  to assume goodwill. I don't, I never believe that the barbarians are
  at the gate, and maybe that's just because I'm an idiot. But I always
  want to assume the good in people, so ---
\item
  amy tan\\
  Well, the barbarians are within. I mean ---
\item
  cheryl strayed\\
  We are the barbarians at the gate.
\item
  amy tan\\
  --- we are barbarians, all of us are barbarians.
\item
  cheryl strayed\\
  Yeah, one runner without a mask to screw it all up. Yeah, no, I'm with
  you, sister.
\item
  amy tan\\
  Just temporarily.
\item
  cheryl strayed\\
  I'm with you.
\item
  amy tan\\
  Yeah.
\item
  cheryl strayed\\
  So, Amy, what an absolute pleasure it has been to talk to you.
\item
  amy tan\\
  Oh, really nice talking to you too. It's such a pleasure to talk about
  these things that we're feeling, that we're sharing, just to have the
  companionship of feeling. We're not together, but we are. They're
  thoughts or feelings that have been in me, but when somebody gives you
  a vehicle for letting it come out, it's always a wonderful outlet, a
  purge.
\item
  cheryl strayed\\
  Yeah. Thank you, my dear.
\item
  amy tan\\
  OK.
\item
  cheryl strayed\\
  Lovely talking to you.
\item
  amy tan\\
  All right.
\item
  cheryl strayed\\
  Bye, bye.
\item
  amy tan\\
  Bye, bye.
\item
  {[}music{]}
\item
  cheryl strayed\\
  I'm Cheryl Strayed. This is ``Sugar Calling.'' Next week, Judy Blume.
\end{itemize}

\href{https://www.nytimes3xbfgragh.onion/column/sugar-calling}{\includegraphics{https://static01.graylady3jvrrxbe.onion/images/2020/04/29/podcasts/sugar-calling-album-art/sugar-calling-album-art-square320.jpg}Sugar
Calling}Subscribe:

\begin{itemize}
\tightlist
\item
  \href{https://itunes.apple.com/us/podcast/id1505881384}{Apple
  Podcasts}
\item
  \href{https://podcasts.google.com/?feed=aHR0cHM6Ly9yc3MuYXJ0MTkuY29tL3N1Z2FyLWNhbGxpbmc\&ved=0CAUQrrcFahcKEwjA8Kyn09voAhUAAAAAHQAAAAAQBQ}{Google
  Podcasts}
\end{itemize}

\hypertarget{you-dont-take-dictation-you-find-the-truth-1}{%
\section{`You Don't Take Dictation. You Find the
Truth.'}\label{you-dont-take-dictation-you-find-the-truth-1}}

\hypertarget{cheryl-strayed-talks-with-the-writer-amy-tan-about-family-and-the-long-arc-of-resilience-1}{%
\subsection{Cheryl Strayed talks with the writer Amy Tan about family
and the long arc of
resilience.}\label{cheryl-strayed-talks-with-the-writer-amy-tan-about-family-and-the-long-arc-of-resilience-1}}

Hosted by Cheryl Strayed, produced by Kelly Prime and edited by Sara
Sarasohn. Editorial oversight by Wendy Dorr.

Transcript

transcript

Back to Sugar Calling

bars

0:00/39:19

-0:00

transcript

\hypertarget{you-dont-take-dictation-you-find-the-truth-2}{%
\subsection{`You Don't Take Dictation. You Find the
Truth.'}\label{you-dont-take-dictation-you-find-the-truth-2}}

\hypertarget{hosted-by-cheryl-strayed-produced-by-kelly-prime-and-edited-by-sara-sarasohn-editorial-oversight-by-wendy-dorr-1}{%
\subsubsection{Hosted by Cheryl Strayed, produced by Kelly Prime and
edited by Sara Sarasohn. Editorial oversight by Wendy
Dorr.}\label{hosted-by-cheryl-strayed-produced-by-kelly-prime-and-edited-by-sara-sarasohn-editorial-oversight-by-wendy-dorr-1}}

\hypertarget{cheryl-strayed-talks-with-the-writer-amy-tan-about-family-and-the-long-arc-of-resilience-2}{%
\paragraph{Cheryl Strayed talks with the writer Amy Tan about family and
the long arc of
resilience.}\label{cheryl-strayed-talks-with-the-writer-amy-tan-about-family-and-the-long-arc-of-resilience-2}}

Wednesday, April 22nd, 2020

\begin{itemize}
\item
  cheryl strayed\\
  Today, I'm going to call Amy Tan. I fell in love with her a long time
  ago, when her first book ``The Joy Luck Club'' came out. I read it
  right away, in 1989, and I was utterly astonished and astounded by the
  lives she wrote about in those pages. And especially the dynamics
  between the mother and the daughter in that book really moved me
  deeply. Amy Tan is also the author of ``The Valley of Amazement,''
  ``The Kitchen God's Wife,'' and other books. Most recently, she wrote
  a memoir called ``Where the Past Begins.'' It's about her life as a
  writer and also about her family history and how it has been brought
  to bear on her life. I'm going to give her a call.
\item
  {[}music{]}
\item
  {[}line ringing{]}
\item
  amy tan\\
  Hello?
\item
  cheryl strayed\\
  Amy?
\item
  amy tan\\
  Yes?
\item
  cheryl strayed\\
  Hi, this is Cheryl.
\item
  amy tan\\
  Hi. Hi, Cheryl.
\item
  cheryl strayed\\
  Oh, it's so nice to hear your voice.
\item
  amy tan\\
  Good to hear you too.
\item
  cheryl strayed\\
  So where are you right now? Where am I calling you?
\item
  amy tan\\
  You are calling me in my home in Sausalito.
\item
  cheryl strayed\\
  So you're socially isolating with your husband, is that right, your
  husband Lou?
\item
  amy tan\\
  My husband, yes. My husband and two dogs, we are in this house. I feel
  very, very lucky actually to be here in this particular house, which I
  actually built, with the idea that one day we'd be old and not be able
  to go out, and what would I need to feel like it was OK to be
  sequestered in a way from the world.
\item
  cheryl strayed\\
  And what did that turn out to be? What is your house like? And what
  can you see from your windows?
\item
  amy tan\\
  It's like a treehouse, and it's surrounded by oak trees, and windows
  all on three sides of this great room that I'm sitting in. Everything
  is ADA compliant, so if I would be in a wheelchair, I could go
  anywhere. The bathrooms are all sliding doors. The entries outside. So
  it's all set up for getting old, and I hope I get old. But what I see,
  I see oak trees. But the main thing I see is Angel Island. Which is, I
  see the bay. I see San Francisco Bay, Raccoon Strait, Sausalito
  Harbor, but in the middle of all that is Angel Island. Do you know
  what Angel Island is? Do you know its history?
\item
  cheryl strayed\\
  I want to say that there --- was it a place where immigrants came. Is
  that right?
\item
  amy tan\\
  Yes, in particular Chinese people. It was a quarantine station. I just
  think it's so ironic in a terrible way that this is the symbol that
  I'm looking at daily is the quarantine where people did not know when
  they would get out of this place and what the future would be and if
  they would be sent back. So great uncertainty and unhappiness there.
\item
  cheryl strayed\\
  And when was that in existence? What years?
\item
  amy tan\\
  That was, I think it was like 1910 until 1940 or so. The dates are
  approximate. But I used to live in the house that's next to the
  current one that we have, and that was built in 1908. And I would look
  at the house and think, the woman who lived in this house in 1908
  would look at that island, and that's where all these Chinese people
  were imprisoned really, and that I was the person now looking at this
  island, and I was the Chinese woman looking at it. So I had a great
  deal of connection between me and what I was seeing, a great deal of
  gratitude that I was on the other side of that.
\item
  cheryl strayed\\
  Your parents both immigrated from China in the `40s, is that correct?
\item
  amy tan\\
  Yes, `47 for my father and `49 for my mother.
\item
  cheryl strayed\\
  And where did they come when --- how did they arrive to the United
  States?
\item
  amy tan\\
  My father came by boat, and my mother flew on Philippine Airlines. She
  had to take a boat from Shanghai to Hong Kong, and then eventually she
  flew. She couldn't wait for the boat. She wanted to be reunited with
  my father. And they were in San Francisco, and by then there was no
  quarantine station. But they came on student visas, which then lapsed,
  and then they became illegal. So they faced deportation for about 10
  years.
\item
  cheryl strayed\\
  And then finally were able to become U.S. citizens?
\item
  amy tan\\
  Yes. Yes.
\item
  cheryl strayed\\
  And how were they able to become citizens?
\item
  amy tan\\
  Eventually, they got, I think, a kind of residency. And I just
  remember that when I was about --- it was around 1962 or so, and I
  remember being at home and my parents were crying, and they had just
  come back from being sworn in as American citizens. And I couldn't
  appreciate that. It was like, oh, big deal, they went and took a test,
  and now they're citizens. So what, you know? I had no idea that for
  all those years, they had been facing this terrible uncertainty.
  Because we were kids, we were born here, we're Americans. We didn't
  know any different. We didn't know anything about China.
\item
  cheryl strayed\\
  Right. Obviously, your work has been so deeply about your past and
  your family's history. And I guess I'm curious. I know you became a
  writer, with ``The Joy Luck Club'' being published when you were in
  your 30s, you were kind of surprised essentially to become a novelist?
\item
  amy tan\\
  Yeah. Yeah. Well, I was a short story writer. Somebody else called it
  a novel. It was a series of short stories really.
\item
  cheryl strayed\\
  You're right. I always forget that. It's that the stories in ``The Joy
  Luck Club'' sit together so beautifully, it reads to me like a novel.
  What I was so moved by is really at the heart of that book is that
  mother-daughter bond. And I'm curious, how did you come to write about
  that mother and daughter, and what did your real mother make of it
  when she read the book when the book came out?
\item
  amy tan\\
  I was writing this at a time when I was not really that close to my
  mother. It was more of a detente or more of an avoidance of situations
  that might inflame both of us. And when I started to write, these
  stories just naturally came out when I began writing about a girl,
  then suddenly her mother injected herself into the story. And I began
  to see how much of who I was had been formed by my --- whatever I had
  absorbed from my mother, but also how I had reacted, rebelled against
  her. So I started writing this, and ironically, the voices that were
  the easiest to me were the mothers. And I realized how much I really
  had paid attention, and I was trying to figure out what these mothers
  --- my mother's intention had been, especially in telling these
  stories about the past that I always thought were so boring. Or you
  know I'd say, you've told me that already, which was the wrong thing
  to say to her. If I said that to her, she'd just go into a fury and
  then threaten to kill herself. So I had to be very careful and just
  kind of tune out and just, yeah, yeah, mm-hm.
\item
  cheryl strayed\\
  If you said, mom, you already told me that story, she would threaten
  to kill herself?
\item
  amy tan\\
  Often, if there was anything that made her feel that she was
  unappreciated or that I'd rather she not be there at that particular
  moment, yes, it would lead to a suicide threat and occasionally
  suicide attempts. And so that was something I grew up with, which
  shaped me a great deal. Not that I'm suicidal, but in looking at what
  it was that made my mother feel suicidal, that she had a kind of
  helplessness, a loneliness, an inability to control what was happening
  in her life. She got that from her mother, who did kill herself, and
  her mother killed herself because she was forced to become a concubine
  after she was widowed. And she was widowed when her husband, my
  grandfather, died during the pandemic in 1919.
\item
  cheryl strayed\\
  Right, the Spanish influenza pandemic.
\item
  amy tan\\
  Yeah. Yeah.
\item
  cheryl strayed\\
  And this was in Shanghai? They were in Shanghai when ---
\item
  amy tan\\
  Yes. Yes, in Shanghai. For the reason that I'm writing, which is to
  understand this transference, this inheritance of emotion over
  generations. Your reactions, the way you think about the world, your
  philosophy, your cosmology of how you view things, how the universe
  operates and how you operate within that world, that came from, I
  think in many ways, from the pandemic, which changed the direction of
  our family's life. And so that suicide came out of that, and my
  mother, being the suicidal child who watched her mother die at age 9,
  was the mother I ended up with, who guided me to who I am today and
  how I react to things.
\item
  cheryl strayed\\
  I love thinking about the ways that our mothers, the ways they shaped
  us not just as humans, but as writers, and I'm really struck by what
  you said about your mom, about her helplessness. And I do think that
  when you're angry and you threaten somebody with your suicide, that is
  a way of really expressing a profound helplessness. And yet, I think
  of your mom and everything you've said about her or written about her,
  she was such a resilient person. She was such a survivor, right?
\item
  amy tan\\
  Yeah. Yeah, it was a great contradiction that she could go through
  anything, and yet there would be the smallest thing that would make
  her feel annihilated. But you're right, she had such fortitude, such
  resilience, but it came across in ways that really made me want to not
  be like her. I mean, one of her things is she constantly worried, and
  she would try to find the reasons why something happened. And if she
  could find the reason, then she couldn't do it. But at the same time,
  she could handle the reality of what was going on and just be strong
  and push through and tell people she wasn't taking this. And she had
  very strong opinions, very strong ethics. She would become livid if
  anybody was condescending to her, and she couldn't always fight back
  and say, don't you dare treat me that way, because she was a Chinese
  woman who didn't speak English very well. And people would treat her
  in this way, patronizing, condescending way, but that would make her
  livid. And that's something she passed on to me. I get a just
  immediate reaction where I explode if somebody is treating me in a
  patronizing way.
\item
  cheryl strayed\\
  Right. It's so amazing to me the ways that we carry our mothers within
  us, positive and negative. My mom died when she was 45.
\item
  amy tan\\
  So young.
\item
  cheryl strayed\\
  Yes. You know, Amy, I feel connected to you. I've never met you, but
  part of you is embedded in part of me because --- I'm going to cry ---
  because ``The Joy Luck Club'' was one of the last books that my mom
  and I read together.
\item
  amy tan\\
  Oh ---
\item
  cheryl strayed\\
  And ---
\item
  amy tan\\
  --- God.
\item
  cheryl strayed\\
  --- we both loved it so much. And it was --- I had a really different
  kind of relationship with my mom than the one you had with yours, but
  there was a language, there was something that you gave us that
  allowed us to talk even more fully about our relationship. And it was
  you, your book, ``The Joy Luck Club'' opened up that dialogue in the
  last year of my mom's life. I remember ---
\item
  amy tan\\
  Wow.
\item
  cheryl strayed\\
  --- so your book came out two years before my mom died, and then the
  movie of it came out two years after she died. And I remember sitting
  in the theater watching it alone, and I remember just crying and
  thinking, I wish --- my mother should be here with me.
\item
  amy tan\\
  Oh. Oh, well, that makes me cry to hear that.
\item
  cheryl strayed\\
  I'm sorry, I'm not supposed to ---
\item
  amy tan\\
  Thank you for telling me. No, no, it's a good thing. It's good when
  you're moved to tears.
\item
  cheryl strayed\\
  I'm not supposed to cry, but.
\item
  amy tan\\
  No.
\item
  cheryl strayed\\
  I know you have also experienced deep grief. Your brother Peter and
  your father John both died in 1968, when you were 15, and they died of
  the same thing.
\item
  amy tan\\
  Yes.

  It was --- it actually started in `67 and ended --- it was exactly one
  year, February to February. My brother first became ill. My older
  brother, who I adored.
\item
  cheryl strayed\\
  And how old was he at the time?
\item
  amy tan\\
  He died just about a week before he turned 17, and he had a brain
  tumor, a kind of brain tumor that would today be curable. And two
  weeks before he died, my father was diagnosed with glioblastoma, and
  then he died six months later. So it was one solid year. There was no
  break whatsoever, February to February, of this specter of doom, but
  also a need to pretend that everything was OK and a miracle would
  happen. And the pretense is what I hated, the, what I consider to be
  the false hope at the time, and that is something I write a lot about.
  I want to understand this notion, and is there anything in the way of
  hopelessness or futile hope? And I've come away thinking, no, hope is
  a very, very important part of this that we have to really safeguard.
  We have to keep that because that is part of what gives us fortitude.
  And it could be the most impossible situation, but for you to see a
  future, you have to cling to that. And I feel that very strongly.
\item
  cheryl strayed\\
  And yet, when you were a kid and your family was facing this horrific
  couple of years, you resented that sense that there should be hope?
\item
  amy tan\\
  Oh, yeah. It made me angry because I felt that it was a betrayal,
  especially after my brother died, that the miracle didn't happen and
  they were still going through this. My father was a Baptist minister.
  He was an engineer, but he was also a Baptist minister. So this
  religious component was part of it. If we were good enough, if we
  prayed hard enough, the miracle would happen. And it didn't happen, so
  it was either that I didn't get the reward or I was not good enough.
  And it made me cynical, which is a terrible thing to have happen when
  you're a young person. I don't know when cynicism comes into a child's
  life, when they first recognize it in themselves, but I certainly felt
  it when I was 14 and 15.
\item
  cheryl strayed\\
  Mm-hm. And when your family, really almost half of your family died.
\item
  amy tan\\
  Yeah, and what I thought was the best half of the family because I
  loved my brother and my father. I was a daddy's girl, and he was
  supposed to protect me from my mother, and he left. And then I was
  angry at my father because he abandoned me by dying.
\item
  cheryl strayed\\
  Right. I'm curious, that experience that you had --- we're seeing the
  numbers every day climb of the people around the globe who have died
  of Covid-19. And I'm curious if knowing that so many people are
  grieving the loss of loved ones, if that's made you at all remember or
  think about what you experienced as a young woman when suddenly you
  lost people in your life?
\item
  amy tan\\
  Yeah. I think more than anything, it's not simply the grief, it's not
  knowing. Not knowing when this is going to end. During the time my
  father and brother were dying, my mother made us feel it was a curse
  and we would all die. So we were waiting for something worse, even
  worse to happen, and the worst that happened was my mother went crazy,
  to use a very non-psychiatric term. So I think it's that uncertainty
  and the feeling of something impending that is worse than what is
  presently happening. And each day, we are overwhelmed by news, the
  numbers. It's a crescendo of these numbers, and it just doesn't stop.
  And it starts with this incredulity. People wake up in the morning and
  say, it was just a bad dream. And then you're incredulous to find, no,
  it's real. And each day, that momentary sense of disbelief, I think,
  overtakes a lot of us, and then we settle into it and say, oh my God,
  when is this going to stop? And all these people are giving us
  information, and who's telling the truth or who knows the truth?
  Nobody could possibly know. This has never happened before with this
  particular disease. So I think that uncertainty combined with losses,
  and not for me --- it's not so much that I fear dying for myself, I
  fear it for my husband. We've been together for 50 years, and I don't
  know what I would do if something happened to him.

  My fear for friends who have certain conditions that would make them
  very vulnerable, besides age. I have lost one friend who, ironically,
  tragically, was the person who wrote the textbook on pulmonary
  disease, and he died from Covid-19.
\item
  cheryl strayed\\
  Oh, I'm so sorry.
\item
  amy tan\\
  Yeah.
\item
  cheryl strayed\\
  What was his name?
\item
  amy tan\\
  John Murray. His wife is Diane Johnson, the novelist. I don't know if
  you've heard of her.
\item
  cheryl strayed\\
  Yeah.
\item
  amy tan\\
  So we've been friends for about 30-some years, and so it was very,
  very sad.
\item
  cheryl strayed\\
  So we're afraid of our friends dying. You also mentioned you're afraid
  of your husband Lou dying. I'm wondering what you're doing to manage
  that feeling that you describe so well, which I do think is very much
  like grief, that you wake up, you forget for a moment what's true, and
  then you remember the truth. And I remember this over and over in the
  first year or so after my mom died, that I would wake up --- there was
  something kind of safe about the dream life, and I'd wake up and
  think, oh, she's dead. And you're right, this is exactly what I've
  thought during the pandemic. And it's because the world --- I always
  say, when you lose somebody who is essential to you, the world as you
  know it ends, and you have to figure out how to live in the new world
  without that person. And here we are in the pandemic, that the world
  as we knew it has ended. And so how are you navigating in the new
  world?
\item
  amy tan\\
  I have anxiety like everybody else, and it comes as a sense of
  breathlessness. I don't really know it in my head until I feel it in
  my body, and I say, why do I feel I can't breathe? And I know I don't
  have Covid-19, but I just think, what's going on? And then I recognize
  the anxiety, and I know that what it would be is suddenly this
  affecting us and Lou getting sick and what would I do and losing him.
  So what I do that is different from my mother is not worry about ---
  not focus on, how did this happen, where does this come from? Is to be
  super prepared. I am known among my friends as Debbie Downer. I'm the
  person who gets prepared for every single disaster. I have a shed full
  of stuff for earthquakes. I have things for ---
\item
  cheryl strayed\\
  Really?
\item
  amy tan\\
  I have shelves with grab bags to go, with everything you need for a
  wildfire. So when this pandemic first started, I went and gathered a
  lot of the stuff I already had, which included masks. I have goggles.
  I have sterilizing things, UV water purifiers, all of that kind of
  stuff, gloves. I already had that stuff, and then I added to that. And
  then I know that something could happen, but I would have control,
  some degree of control.
\item
  cheryl strayed\\
  Some degree of control.
\item
  amy tan\\
  Yeah, some degree of control. I can't control everything, but I won't
  fall to pieces. I have a little box that has everything, and I know I
  go there, I have procedure, so I can be methodical.
\item
  cheryl strayed\\
  It is such a fine line, isn't it, between your friends kind of making
  fun of you because you have all this stuff, and yet it's actually kind
  of reasonable. I too --- I live in Portland, Oregon, where we're all
  just waiting for this major earthquake that they keep saying is
  overdue. And for years now, I kept thinking, I have to buy those
  earthquake preparedness packs. And the first week that it got real in
  my household, I bought these 20-year bags for seven people. My
  family's only four, but I figure we'll help out a few neighbors too if
  we can. And I think when my kids saw this, they're like, mom, you're
  kind of weird. But the thing is, I wasn't trying to alarm anyone. I'm
  trying to be prepared. But ---
\item
  amy tan\\
  How can --- I don't know why people would deny that. It's this
  superstitious mentality of, if you do it, it's going to happen.
\item
  cheryl strayed\\
  Well, and that's, I want to go back to --- this is something that I
  really am curious about, because when your brother and father died in
  the late `60s, we just had such a different culture of the ways that
  we talk about death, a very different culture of grief. I mean, that
  it was this topic that was to be avoided. And I'm curious, here you
  are, my heart breaks for you, Amy, this teenage girl. You've lost your
  big brother, your father. I'm sure you were also maybe afraid that you
  too might get sick and die, all of those things. And I'm curious, so
  many people out there right now are having to face these big losses
  and having to grieve, and I know a lot of people around them are
  wondering, what do I say? How do I address this sorrow, this
  monumental sorrow? And I'm curious, what do you wish people did or
  said to you during your time of greatest need and loss and sorrow?
\item
  amy tan\\
  Well, you have to keep in mind I was very young. I mean, I was a
  teenager, and I wanted to have a normal life. I wanted to be asked out
  on a date. I'd never been asked out on a date. I wanted to have that
  life and to escape. And I think about that with a lot of the kids
  today, and that I want to say, I understand if you want to escape,
  that you want to leave the house, or you just want to take your mind
  out of this. You don't want to be with family. You don't want to be
  supportive. You want somebody to say to you, that's fine the way that
  you feel, that's natural. That's what I would have wanted. I'm sure
  there are a lot of people out there who do that for their kids. But I
  think parents, adults are so overwhelmed by what's happening, they may
  be demanding more things from their kids, that they behave better,
  that they --- and you know what happens is you behave worse. Instead
  of making things easier for your family, you make it harder because
  you resent it. You're scared, and you can't express that because
  you're a teenager, and nothing scares you, so to speak. And I wanted
  to just have somebody say, I know why you're angry. I needed some
  understanding. There was a complication on top of that is that a few
  days or a week or two before my father died, I was caught reading a
  dirty book, ``Catcher in the Rye,'' and ---
\item
  cheryl strayed\\
  Oh, my goodness.
\item
  amy tan\\
  --- a banned book. So the youth minister came to counsel me, and he
  sexually molested me. He threw me down on my bed and sexually molested
  me.
\item
  cheryl strayed\\
  Oh, Amy.
\item
  amy tan\\
  So that day, it was --- before the miracle busted up, exploded in our
  faces, that was already gone. The whole idea that religion and people
  and praying together and community, that was gone too. I came out of
  that room a different person.
\item
  cheryl strayed\\
  Oh, Amy.
\item
  amy tan\\
  And ultimately --- no, no --- ultimately, I'm glad because what it
  taught me is that I have to ask my own questions. I have to say, what
  is it that I really feel? What's hypocritical? What's truly not
  kindness and being a good person?

  And so I ask those questions myself. It doesn't get dictated to me in
  the way that it used to because the spokespeople for what is good and
  evil are not necessarily coming from the best sources. So I learned
  that. You don't take dictation, you find the truth.
\item
  cheryl strayed\\
  Well, and you have to be awake to it. I think we share that experience
  too. I've been sexually abused. And I think that, for me, the most
  meaningful act probably in my life has been that move that you just
  described. Where you take an awful thing that happened to you, the
  death of an essential person in your life who you love or a sexual
  violation, you take that really, really ugly thing and make it, in
  some ways, beautiful. You make it not disempower you, but be the
  place, the source of your power. Adrienne Rich talks about our wounds
  being the source of our greatest power. I really believe that.
\item
  amy tan\\
  I want to just interject something here. I was talking to a friend, a
  psychiatrist, about people who go through terrible things and they
  have that with them the rest of their lives, and it makes them a
  little dysfunctional or greatly dysfunctional. And I wouldn't want
  people to feel guilty that they weren't able to find their strength
  when a terrible thing happened to them or a series of terrible things.
  So I just say that because there are a lot of people going through
  terrible stuff right now, and maybe they're not going to be able to
  come out and say, I'm resilient. Maybe they're just going to fall
  apart. And I just want to say, that's good, lift your hand and ask for
  help.
\item
  cheryl strayed\\
  And I think, too, one of the things about resilience that I've
  observed in my own life and so many people who I admire is that
  resilience doesn't necessarily happen the next day, right? Sometimes
  you have to wander around in the darkness for a decade, for two
  decades. I mean, anyone who's ever recovered from a long addiction or
  changed their lives in these radically essential ways knows this to be
  true. So it's also never too late to ---
\item
  amy tan\\
  Exactly, yeah.
\item
  cheryl strayed\\
  --- find that strength.
\item
  amy tan\\
  Mm-hm. It's there, it's rooted, and it can come out, and you know, it
  can surprise you. I wonder about so many people and how they're going
  to deal with this, and that becomes part of the thing that makes me
  anxious, that what's going to happen to this world? Because I'm going
  to come out OK. My life now is not a lot different from what it has
  been as a writer. But what about these people in these tiny places who
  are going to lose those tiny places? What's going to happen to them?
  So I think part of it is to feel you can't do anything, but I think
  it's important to feel compassion, to empathize as much as you can,
  which is what I think we do as fiction writers. It's our practice to
  do that. Imagination I think, is the closest thing we have to
  compassion.
\item
  cheryl strayed\\
  Oh, yeah. You wrote something really beautiful that I wrote down in my
  notebook in your memoir, ``Where the Past Begins.'' You said that Jane
  Eyre taught you that loneliness had more to do with being
  misunderstood than being alone. And I think it's a wonderful example
  of what you're saying, that as memoirists, fiction writers, as
  writers, it's not about delivering the message to everyone that would
  be the one wise thing, but making people feel seen and recognized. I'm
  curious, I asked you to read something to me, and can I ask you to
  share the poem?
\item
  amy tan\\
  Now, the poem that I've chosen has to do with the fact that as
  writers, I think that, for many of us, what draws us to writing is to
  come up with a way to express the exact feeling at the exact moment,
  and that when somebody reads that, they say, that's how it is for me.
  And that's what this poem is about. This poem was written by my
  editor, who's also my publisher and is also a renowned poet, Dan
  Halpern. He wrote this five years ago. He didn't remember writing it,
  but he knew that he must've written it because it was published in
  Poetry about five years ago.
\item
  cheryl strayed\\
  He doesn't remember writing it?
\item
  amy tan\\
  He doesn't remember, and it is so prescient.
\item
  cheryl strayed\\
  Oh, that's funny. So, yeah, it's prescient. It's called
  ``Pandemania,'' right?
\item
  amy tan\\
  ``Pandemania.''
\item
  cheryl strayed\\
  By Daniel Halpern.
\item
  amy tan\\
  Daniel Halpern. Here it is.

  ``There are fewer introductions in plague years, hands held back,
  jocularity no longer bellicose, even among men. Breathing's generally
  wary, labored, as they say, when the end is at hand. But this is the
  everyday intake of the imperceptible life force willed now, slow.
  Well, just cautious in inhabited air. As for ongoing dialogue, no
  longer an exuberant plosive to make a point, but a new squirreling of
  airspace, a new sense of boundary.

  ``Genghis Khan said the hand is the first thing one man gives to
  another. Not in this war. A gesture of limited distance now suffices,
  a nod, a minor smile or a hand slightly raised, not in search of its
  counterpart, just a warning within the acknowledgment to stand back.
  Each beautiful stranger a barbarian breathing on the other side of the
  gate.''
\item
  cheryl strayed\\
  Amy, what a perfect poem for this moment.
\item
  amy tan\\
  I know.
\item
  cheryl strayed\\
  This is honestly my life. I walk around Northeast Portland and I greet
  people with a seeming sense of pleasantry from a distance, but I
  think, barbarian, stay on the other side of that gate.
\item
  amy tan\\
  You're infected. Are you infected? Or, why aren't you wearing a mask?
  Or why are you too close to me?
\item
  cheryl strayed\\
  Do you walk around your neighborhood? Are you having this same
  experience?
\item
  amy tan\\
  I've done it twice. But we see people running by, no masks, too close,
  coming by too close to us, and I think it's impolite. I think it's
  hostile. So there's a new separateness that has come up among
  strangers.
\item
  cheryl strayed\\
  Right. And that actually scares me, that dynamic, because I sense in
  myself sometimes that building sense of not just fear, but almost this
  --- Daniel in this poem, he even says, he mentions this term war. It's
  you against me, and you stay in your breath space and I'll stay in
  mine. And I can't help but fear what's already gone on so much over
  the last many years and probably through all of history, but
  especially in the last few years this anti-immigrant sentiment ---
\item
  amy tan\\
  Oh my God, yeah.
\item
  cheryl strayed\\
  --- the attacks against many Asian-Americans in the U.S.
\item
  amy tan\\
  Yeah, especially now. Yeah.
\item
  cheryl strayed\\
  And, right, even our president calling it the Chinese virus. And I'm
  afraid about that dynamic of this.
\item
  amy tan\\
  And I just want to say this thing about Chinese virus. Some people
  say, well, we called it Ebola. Ebola River is where Ebola was.
  Ebolese, they're not people. You're not saying the ``Ebolese.''
  Chinese is a race. Chinese is not a location. Chinese live throughout
  the world, including the United States, people who were born in this
  country. And he has made it a race card, and that's what angers me so
  much in how they have renamed this virus. They have made it race, not
  location. So the thing about with people who go running by without a
  mask, it's not ignorance because I have read online the hostile
  remarks. It's like, you don't tell us what to do, we don't have to
  wear that.
\item
  cheryl strayed\\
  Yeah. Yeah, I've seen it. I just try to assume --- my nature is just
  to assume goodwill. I don't, I never believe that the barbarians are
  at the gate, and maybe that's just because I'm an idiot. But I always
  want to assume the good in people, so ---
\item
  amy tan\\
  Well, the barbarians are within. I mean ---
\item
  cheryl strayed\\
  We are the barbarians at the gate.
\item
  amy tan\\
  --- we are barbarians, all of us are barbarians.
\item
  cheryl strayed\\
  Yeah, one runner without a mask to screw it all up. Yeah, no, I'm with
  you, sister.
\item
  amy tan\\
  Just temporarily.
\item
  cheryl strayed\\
  I'm with you.
\item
  amy tan\\
  Yeah.
\item
  cheryl strayed\\
  So, Amy, what an absolute pleasure it has been to talk to you.
\item
  amy tan\\
  Oh, really nice talking to you too. It's such a pleasure to talk about
  these things that we're feeling, that we're sharing, just to have the
  companionship of feeling. We're not together, but we are. They're
  thoughts or feelings that have been in me, but when somebody gives you
  a vehicle for letting it come out, it's always a wonderful outlet, a
  purge.
\item
  cheryl strayed\\
  Yeah. Thank you, my dear.
\item
  amy tan\\
  OK.
\item
  cheryl strayed\\
  Lovely talking to you.
\item
  amy tan\\
  All right.
\item
  cheryl strayed\\
  Bye, bye.
\item
  amy tan\\
  Bye, bye.
\item
  {[}music{]}
\item
  cheryl strayed\\
  I'm Cheryl Strayed. This is ``Sugar Calling.'' Next week, Judy Blume.
\end{itemize}

Previous

More episodes ofSugar Calling

\href{https://www.nytimes3xbfgragh.onion/2020/05/20/podcasts/sugar-calling-joy-harjo-poetry-virus.html?action=click\&module=audio-series-bar\&region=header\&pgtype=Article}{\includegraphics{https://static01.graylady3jvrrxbe.onion/images/2020/05/22/podcasts/20sugar-hajo3/20sugar-hajo3-thumbLarge.jpg}}

May 20, 2020~~•~ 35:30`I Release You, Fear'

\href{https://www.nytimes3xbfgragh.onion/2020/05/13/podcasts/sugar-calling-billy-collins-poetry-virus.html?action=click\&module=audio-series-bar\&region=header\&pgtype=Article}{\includegraphics{https://static01.graylady3jvrrxbe.onion/images/2020/05/13/podcasts/13sugar-calling/13sugar-calling-thumbLarge.jpg}}

May 13, 2020`There's a Quiet All Over the World'

\href{https://www.nytimes3xbfgragh.onion/2020/05/06/podcasts/sugar-calling-alice-walker-quarantine-virus.html?action=click\&module=audio-series-bar\&region=header\&pgtype=Article}{\includegraphics{https://static01.graylady3jvrrxbe.onion/images/2020/05/06/podcasts/06sugarcalling/06sugarcalling-thumbLarge.jpg}}

May 6, 2020~~•~ 28:58`Whatever We Have, We Have to Work With It'

\href{https://www.nytimes3xbfgragh.onion/2020/04/29/podcasts/sugar-calling-judy-blume-quarantine-virus.html?action=click\&module=audio-series-bar\&region=header\&pgtype=Article}{\includegraphics{https://static01.graylady3jvrrxbe.onion/images/2020/04/29/podcasts/29sugarcalliing-blume-sub/29sugarcalliing-blume-sub-thumbLarge.jpg}}

April 29, 2020`This Terrible Thing Is Happening, but the World Goes On.'

\href{https://www.nytimes3xbfgragh.onion/2020/04/22/podcasts/sugar-calling-amy-tan-quarantine-virus.html?action=click\&module=audio-series-bar\&region=header\&pgtype=Article}{\includegraphics{https://static01.graylady3jvrrxbe.onion/images/2020/04/27/podcasts/22sugarcalling/22sugarcalling-thumbLarge.jpg}}

April 22, 2020~~•~ 39:19`You Don't Take Dictation. You Find the Truth.'

\href{https://www.nytimes3xbfgragh.onion/2020/04/15/podcasts/sugar-calling-pico-iyer-coronavirus.html?action=click\&module=audio-series-bar\&region=header\&pgtype=Article}{\includegraphics{https://static01.graylady3jvrrxbe.onion/images/2020/04/21/podcasts/15sugarcalling1/15sugarcalling1-thumbLarge.jpg}}

April 15, 2020~~•~ 35:45`Joyful Participation in a World of Sorrows'

\href{https://www.nytimes3xbfgragh.onion/2020/04/08/podcasts/sugar-calling-margaret-atwood-coronavirus.html?action=click\&module=audio-series-bar\&region=header\&pgtype=Article}{\includegraphics{https://static01.graylady3jvrrxbe.onion/images/2020/04/02/books/08sugarcalling1/08sugarcalling1-thumbLarge-v3.jpg}}

April 8, 2020~~•~ 34:32`Roll Up Your Sleeves, Girls'

\href{https://www.nytimes3xbfgragh.onion/2020/04/03/podcasts/sugar-calling-george-saunders-coronavirus.html?action=click\&module=audio-series-bar\&region=header\&pgtype=Article}{\includegraphics{https://static01.graylady3jvrrxbe.onion/images/2020/04/09/podcasts/03sugarcalling-image/merlin_171264408_4ac7fc67-d8cc-45b9-9ec6-bdd20672e694-thumbLarge.jpg}}

April 3, 2020~~•~ 41:16`Everything Is Always Keep Changing'

\href{https://www.nytimes3xbfgragh.onion/column/sugar-calling}{See All
Episodes ofSugar Calling}

Next

April 22, 2020

\begin{itemize}
\item
\item
\item
\item
\item
\item
\end{itemize}

\emph{\textbf{Listen and subscribe to our podcast from your mobile
device:}}\\
\textbf{\href{https://podcasts.apple.com/us/podcast/sugar-calling/id1505881384}{\emph{Via
Apple Podcasts}}} \emph{\textbf{\textbar{}}}
\textbf{\href{https://open.spotify.com/show/4U8hPiNGIBvTS9zLeiDCN7?si=gRyigD47SPWl-QWgNjgt2w}{\emph{Via
Spotify}}} \emph{\textbf{\textbar{}}}
\textbf{\href{https://www.stitcher.com/podcast/the-new-york-times/sugar-calling}{\emph{Via
Stitcher}}}

\hypertarget{it-could-be-the-most-impossible-situation-but-for-you-to-see-a-future-you-have-to-cling-to-that}{%
\subsection{`It could be the most impossible situation. But for you to
see a future, you have to cling to
that.'}\label{it-could-be-the-most-impossible-situation-but-for-you-to-see-a-future-you-have-to-cling-to-that}}

\emph{--- Amy Tan, author, speaking about hope}

\emph{Note: This episode contains descriptions of suicide and sexual
violence.}

Today, Cheryl phones Amy Tan at her home overlooking Sausalito Bay in
California. The two discuss managing anxiety. ``I'm the person who gets
prepared for every single disaster,'' Amy says.

They compare notes on grief, hope and the importance of asking for help.

\includegraphics{https://static01.graylady3jvrrxbe.onion/images/2020/04/27/podcasts/22sugarcalling/22sugarcalling-articleLarge.jpg?quality=75\&auto=webp\&disable=upscale}

\hypertarget{on-todays-episode}{%
\subsubsection{\texorpdfstring{\textbf{On today's
episode:}}{On today's episode:}}\label{on-todays-episode}}

\href{http://www.amytan.net/}{Amy Tan} is the author of six novels, two
\href{https://www.nytimes3xbfgragh.onion/1992/11/08/books/childrens-books-girl-overboard.html}{children's
books} and a
\href{https://www.nytimes3xbfgragh.onion/2017/10/16/books/amy-tan-memoir.html}{memoir}.
The film adaptation of her
\href{https://www.nytimes3xbfgragh.onion/1989/03/19/books/the-joy-luck-club.html?searchResultPosition=2}{best-selling
first novel, ``The Joy Luck Club},'' saw
\href{https://www.nytimes3xbfgragh.onion/2018/09/09/movies/joy-luck-club-crazy-rich-asians.html?searchResultPosition=3}{lines
winding around the block} in cities including New York and Los Angeles.
In 2008, her novel ``The Bonesetter's Daughter'' was
\href{https://www.nytimes3xbfgragh.onion/2008/08/31/arts/music/31melv.html}{adapted
into an opera}.

\hypertarget{amy-tans-book-recommendations-for-surviving-confinement}{%
\subsubsection{\texorpdfstring{\textbf{Amy Tan's book recommendations
for surviving
confinement:}}{Amy Tan's book recommendations for surviving confinement:}}\label{amy-tans-book-recommendations-for-surviving-confinement}}

\begin{itemize}
\tightlist
\item
  \emph{For parents of restless children:
  ``}\href{https://heydaybooks.com/book/how-to-teach-nature-journaling/}{How
  to Teach Nature Journaling: Curiosity, Wonder, Attention},'' John Muir
  Laws and Emilie Lygren
\end{itemize}

\begin{itemize}
\item
  \emph{For the curious:
  ``}\href{https://www.nytimes3xbfgragh.onion/2009/05/31/books/review/Royte-t.html}{Summer
  World},'' Bernd Heinrich
\item
  \emph{For remembering the natural world:
  ``}\href{https://www.penguinrandomhouse.com/books/536247/devotions-by-mary-oliver/}{Devotions},''
  Mary Oliver
\item
  \emph{For thinking about how we got into this mess:
  ``}\href{https://www.basicbooks.com/titles/mark-w-moffett/the-human-swarm/9781541617292/}{The
  Human Swarm: How Our Societies Arise, Thrive, and Fall},'' Mark
  Moffett
\item
  \emph{For a good time}:
  ``\href{https://www.harpercollins.com/9780062913463/nothing-to-see-here/}{Nothing
  to See Here},'' Kevin Wilson
\item
  \emph{For the immune system:}
  ``\href{https://www.penguinrandomhouse.com/books/572963/undo-it-by-dean-ornish-md-and-anne-ornish/}{Undo
  It!},'' Dean Ornish, M.D. and Anne Ornish
\end{itemize}

\begin{center}\rule{0.5\linewidth}{\linethickness}\end{center}

Cheryl Strayed is the author of ``Tiny Beautiful Things,'' ``Torch,''
``Brave Enough,'' and the New York Times best seller ``Wild.'' Her books
have been translated into more than 40 languages. She lives in Portland,
Ore.
\href{https://twitter.com/CherylStrayed?ref_src=twsrc\%5Egoogle\%7Ctwcamp\%5Eserp\%7Ctwgr\%5Eauthor}{@CherylStrayed}

``Sugar Calling'' is produced by Kelly Prime and edited by Sara
Sarasohn, with editorial oversight by Wendy Dorr. This episode was mixed
by Jamie Collazo. Our theme music is by Dan Powell.

Advertisement

\protect\hyperlink{after-bottom}{Continue reading the main story}

\hypertarget{site-index}{%
\subsection{Site Index}\label{site-index}}

\hypertarget{site-information-navigation}{%
\subsection{Site Information
Navigation}\label{site-information-navigation}}

\begin{itemize}
\tightlist
\item
  \href{https://help.nytimes3xbfgragh.onion/hc/en-us/articles/115014792127-Copyright-notice}{©~2020~The
  New York Times Company}
\end{itemize}

\begin{itemize}
\tightlist
\item
  \href{https://www.nytco.com/}{NYTCo}
\item
  \href{https://help.nytimes3xbfgragh.onion/hc/en-us/articles/115015385887-Contact-Us}{Contact
  Us}
\item
  \href{https://www.nytco.com/careers/}{Work with us}
\item
  \href{https://nytmediakit.com/}{Advertise}
\item
  \href{http://www.tbrandstudio.com/}{T Brand Studio}
\item
  \href{https://www.nytimes3xbfgragh.onion/privacy/cookie-policy\#how-do-i-manage-trackers}{Your
  Ad Choices}
\item
  \href{https://www.nytimes3xbfgragh.onion/privacy}{Privacy}
\item
  \href{https://help.nytimes3xbfgragh.onion/hc/en-us/articles/115014893428-Terms-of-service}{Terms
  of Service}
\item
  \href{https://help.nytimes3xbfgragh.onion/hc/en-us/articles/115014893968-Terms-of-sale}{Terms
  of Sale}
\item
  \href{https://spiderbites.nytimes3xbfgragh.onion}{Site Map}
\item
  \href{https://help.nytimes3xbfgragh.onion/hc/en-us}{Help}
\item
  \href{https://www.nytimes3xbfgragh.onion/subscription?campaignId=37WXW}{Subscriptions}
\end{itemize}
